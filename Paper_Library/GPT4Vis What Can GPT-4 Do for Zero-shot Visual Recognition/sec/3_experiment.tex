\section{Experiments}

\subsection{Datasets}

\begin{table}[t]
    \centering
    \scalebox{0.83}{
    \setlength{\tabcolsep}{2.0pt}
    \begin{tabular}{llccc} 
    \toprule
     Modality &  Dataset & \#Classes & Res. (Ave.) & \#Samples \\ \midrule
     \multirow{11}{*}{Image} &  DTD~\cite{dtd} & 47 & 300\x300 &  1,692  \\
       &  EuroSAT~\cite{helber2019eurosat} & 10 & 64\x64 & 8,100 \\
       & SUN397~\cite{xiao2010sun} &  397 & 969\x776   & 19,850 \\
       & RAF-DB~\cite{li2017reliable} & 7 & 100\x100 &  3,068 \\
       & Caltech101~\cite{caltech101} & 101 & 300\x200  & 2,465 \\
       & ImageNet~\cite{deng2009imagenet} &  1000 & 469\x387  & 50,000 \\ 
       & FGVC-Aircraft~\cite{aircraft} & 100 & 1098\x747 & 3,333 \\
       & Flower102~\cite{flower} & 102 & 667\x500 & 2,463 \\
       & Stanford Cars~\cite{Stanfordcar} & 196 & 360\x240 & 8,041 \\
       & Food101~\cite{bossard2014food} &  101 & 500\x350 & 30,300 \\
       & Oxford Pets~\cite{parkhi2012cats} & 37 & 500\x350 & 3,669 \\ \midrule
     \multirow{4}{*}{Video}  & UCF-101~\cite{ucf101} (Split 1) & 101 & 320\x240 & 3,783 \\
       & HMDB-51~\cite{hmdb} (Split 1) & 51 & 340\x256 & 1,530 \\
       & Kinetics-400~\cite{i3d} & 400 & 420\x320 & 19,796 \\
       & Sth-Sth V1~\cite{sth-sth} & 174 & 176\x100 & 11,522 \\ \midrule
    Point Cloud   & ModelNet10~\cite{modelnet40} & 10 & 224\x224 & 908 \\
    \bottomrule
    \end{tabular}
    }
    \caption{The statistics of these evaluated datasets.}
    \label{tab:datasets}
\end{table}


This study evaluates 16 visual datasets across images, videos, and point clouds. The evaluation employs the widely recognized validation sets for these benchmarks, with Table~\ref{tab:datasets} providing detailed statistics for each dataset.



\subsection{Implementation Details}



\begin{figure}[!t]
  \centering
   \includegraphics[width=0.985\linewidth]{sec/figures/sen.pdf}
   \caption{Sentences generated by GPT-4 for ``British Shorthairs''.}
   \label{fig:gpt4_sentences}
\end{figure}

\begin{figure}[!t]
  \centering
   \includegraphics[width=0.985\linewidth]{sec/figures/gpt4v_prompt.pdf}
   \caption{Prompts for image, video, and point cloud datasets: (a) An example from RAF-DB~\cite{li2017reliable} illustrates 7-class facial expression recognition. (b) A video example from HMDB-51~\cite{hmdb} demonstrates 51-class action recognition, where ellipses indicate category names omitted due to space constraints. (c) An example from ModelNet10~\cite{modelnet40} for point cloud classification across 10 categories, where ellipses again indicate the truncation of category names owing to space constraints. Please zoom in for best view.} 
   \label{fig:gpt4v}
\end{figure}

\noindent\textbf{GPT-4 Generated Descriptions.}
Using the GPT-4 API (version gpt-4-1106-preview), we generate $K$ descriptive sentences for each category, with $K$ defaulting to be 20. As an example, for the ``British Shorthairs'' category from the Oxford Pets dataset~\cite{parkhi2012cats}, we present our prompts alongside GPT-4's responses in Figure~\ref{fig:gpt4_sentences}.

\vspace{1mm}
\noindent\textbf{Utilizing GPT-4 with Vision.}
In our study, we employ the GPT-4V API (specifically, gpt-4-vision-preview) to evaluate 16 different benchmarks. For videos, we select 8 frames via uniform sampling for API processing, and for point clouds, we provide images from six perspectives. Figure~\ref{fig:gpt4v} showcases the interaction with GPT-4V, highlighting both the prompts used and the subsequent responses for evaluations across images, videos, and point clouds.






\begin{table*}[t]
    \centering
    \resizebox{\textwidth}{!}{
    \begin{NiceTabular}{l*{9}c}
        \toprule
        Dataset & \Block{1-3}{Describable Textures (DTD)} & & & \Block{1-3}{EuroSAT} & & & \Block{1-3}{SUN397} & & \\
        Backbone (\#Param) & Baseline & \textbf{\whblue{GPT Prompts}} & Top-1 \(\Delta\) & Baseline & \textbf{\whblue{GPT Prompts}} & Top-1 \(\Delta\) & Baseline & \textbf{\whblue{GPT Prompts}} & Top-1 \(\Delta\) \\ \cmidrule(lr){1-1} \cmidrule(lr){2-4} \cmidrule(lr){5-7} \cmidrule(lr){8-10}
        CLIP ViT-B/32 (88M) & 42.1 / 69.3 & \gpt{46.9 / 78.0} & \textcolor{teal}{+4.8} & 40.2 / 87.4 & \gpt{49.4 / 83.1} & \textcolor{teal}{+9.2} & 59.2 / 88.1 & \gpt{63.3 / 91.0} & \textcolor{teal}{+4.1} \\ 
        CLIP ViT-B/16 (86M) & 46.0 / 72.6 & \gpt{48.5 / 78.4} & \textcolor{teal}{+2.5} & 45.8 / 81.0 & \gpt{48.8 / 83.7} & \textcolor{teal}{+3.0} & 60.9 / 89.1 & \gpt{65.5 / 91.9} & \textcolor{teal}{+4.6} \\
        CLIP ViT-L/14 (304M) & 51.8 / 77.4 & \gpt{54.8 / 81.6} & \textcolor{teal}{+3.0} & 44.0 / 95.2 & \gpt{54.1 / 95.0} & \textcolor{teal}{+10.1} & 65.2 / 91.4 & \gpt{70.3 / 94.2} & \textcolor{teal}{+5.1} \\
        EVA ViT-E/14 (4.4B) & 61.5 / 86.5 & \gpt{66.5 / 94.7} & \textcolor{teal}{+5.0} & 55.1 / 96.9 & \gpt{70.3 / 93.7} & \textcolor{teal}{+15.2} & 71.6 / 92.3 & \gpt{75.6 / 95.8} & \textcolor{teal}{+4.0} \\ \midrule
        \rpink 
        \textbf{\whred{GPT-4V}} & \Block{1-3}{ 57.7 / 83.3} & & & \Block{1-3}{46.8 / 86.2}  & &  & \Block{1-3}{59.2 / 78.1} & & \\
        \bottomrule
        \toprule
        Dataset & \Block{1-3}{Real-world Affective Faces (RAF-DB)} & & & \Block{1-3}{Caltech101} & & & \Block{1-3}{ImageNet-1K} & & \\
        Backbone (\#Param)  & Baseline & \textbf{\whblue{GPT Prompts}} & Top-1 \(\Delta\) & Baseline & \textbf{\whblue{GPT Prompts}} & Top-1 \(\Delta\)  & Baseline & \textbf{\whblue{GPT Prompts}} & Top-1 \(\Delta\)  \\ \cmidrule(lr){1-1} \cmidrule(lr){2-4} \cmidrule(lr){5-7} \cmidrule(lr){8-10}
        CLIP ViT-B/32 (88M) & 22.4 / 76.6 & \gpt{45.8 / 90.6} & \textcolor{teal}{+23.4} & 86.8 / 99.1 & \gpt{92.8 / 99.6} & \textcolor{teal}{+6.0} & 59.0 / 85.6 & \gpt{63.7 / 88.7} & \textcolor{teal}{+4.7} \\ 
        CLIP ViT-B/16 (86M) & 27.5 / 69.1 & \gpt{54.4 / 94.4} & \textcolor{teal}{+26.9} & 87.9 / 98.7 & \gpt{94.6 / 99.6} & \textcolor{teal}{+6.7} & 64.1 / 89.3 & \gpt{68.7 / 91.6} & \textcolor{teal}{+4.6} \\
        CLIP ViT-L/14 (304M) & 26.1 / 72.1 & \gpt{47.2 / 92.0} & \textcolor{teal}{+21.1} & 86.7 / 99.3 & \gpt{96.2 / 100.0} & \textcolor{teal}{+9.5} & 71.6 / 92.2 & \gpt{75.5 / 94.5} & \textcolor{teal}{+4.9} \\
        EVA ViT-E/14 (4.4B) & 31.0 / 90.9 & \gpt{54.9 / 93.7} & \textcolor{teal}{+23.9} & 94.0 / 99.7 & \gpt{97.9 / 100.0} & \textcolor{teal}{+3.9} & 78.4 / 93.5 & \gpt{81.6 / 96.1} & \textcolor{teal}{+3.2} \\ \midrule
        \rpink
        \textbf{\whred{GPT-4V}} & \Block{1-3}{68.7 / 93.8} & & & \Block{1-3}{93.7 / 98.2}  & & & \Block{1-3}{63.1 / 78.2}  & &  \\
        \bottomrule
        \toprule
        Dataset & \Block{1-3}{FGVC-Aircraft} & & & \Block{1-3}{Flower102} & & & \Block{1-3}{Stanford Cars} & & \\
        Backbone (\#Param)  & Baseline & \textbf{\whblue{GPT Prompts}} & Top-1 \(\Delta\) & Baseline & \textbf{\whblue{GPT Prompts}} & Top-1 \(\Delta\)  & Baseline & \textbf{\whblue{GPT Prompts}} & Top-1 \(\Delta\)  \\ \cmidrule(lr){1-1} \cmidrule(lr){2-4} \cmidrule(lr){5-7} \cmidrule(lr){8-10}
        CLIP ViT-B/32 (88M) & 16.6 / 44.3 & \gpt{21.9 / 54.3} & \textcolor{teal}{+5.3} & 61.6 / 77.6 & \gpt{71.8 / 89.6} & \textcolor{teal}{+10.2} & 58.9 / 90.8 & \gpt{61.2 / 92.5} & \textcolor{teal}{+2.3} \\ 
        CLIP ViT-B/16 (86M) & 21.1 / 55.0 & \gpt{28.0 / 65.4} & \textcolor{teal}{+6.9} & 64.8 / 80.2 & \gpt{74.5 / 91.4} & \textcolor{teal}{+9.7} & 63.6 / 93.7 & \gpt{66.8 / 95.6} & \textcolor{teal}{+3.2} \\
        CLIP ViT-L/14 (304M) & 27.3 / 69.5 & \gpt{36.3 / 82.2} & \textcolor{teal}{+9.0} & 73.1 / 86.3 & \gpt{81.5 / 94.0} & \textcolor{teal}{+8.4} & 76.2 / 97.9 & \gpt{77.9 / 98.6} & \textcolor{teal}{+1.7} \\
        EVA ViT-E/14 (4.4B) & 50.6 / 86.9 & \gpt{58.4 / 96.6} & \textcolor{teal}{+7.8} & 82.1 / 90.8 & \gpt{87.0 / 95.0} & \textcolor{teal}{+4.9} & 94.2 / 99.8 & \gpt{94.5 / 99.9} & \textcolor{teal}{+0.3} \\ \midrule
        \rpink 
        \textbf{\whred{GPT-4V}} & \Block{1-3}{56.6 / 80.8} & & & \Block{1-3}{69.1 / 77.3}  & & & \Block{1-3}{62.7 / 81.8}  & &  \\
        \bottomrule
        \toprule
       Dataset & \Block{1-3}{Food101} & & & \Block{1-3}{Oxford Pets} & & & \Block{1-3}{UCF-101} & & \\
        Backbone (\#Param)  & Baseline & \textbf{\whblue{GPT Prompts}} & Top-1 \(\Delta\) & Baseline & \textbf{\whblue{GPT Prompts}} & Top-1 \(\Delta\)  & Baseline & \textbf{\whblue{GPT Prompts}} & Top-1 \(\Delta\)  \\ \cmidrule(lr){1-1} \cmidrule(lr){2-4} \cmidrule(lr){5-7} \cmidrule(lr){8-10}
        CLIP ViT-B/32 (88M) & 78.0 / 95.1 & \gpt{80.8 / 96.1} & \textcolor{teal}{+2.8} & 79.9 / 96.3 & \gpt{89.3 / 99.7} & \textcolor{teal}{+9.4} & 59.9 / 83.1 & \gpt{69.9 / 93.1} & \textcolor{teal}{+10.0} \\ 
        CLIP ViT-B/16 (86M) & 84.0 / 97.1 & \gpt{86.3 / 97.8} & \textcolor{teal}{+2.3} & 81.8 / 96.4 & \gpt{91.0 / 99.8} & \textcolor{teal}{+9.2} & 64.4 / 84.8 & \gpt{72.0 / 93.4} & \textcolor{teal}{+7.6} \\
        CLIP ViT-L/14 (304M) & 89.9 / 98.5 & \gpt{91.4 / 98.7} & \textcolor{teal}{+1.5} & 88.2 / 97.7 & \gpt{94.1 / 99.9} & \textcolor{teal}{+5.9} & 72.3 / 92.4 & \gpt{80.6 / 97.0} & \textcolor{teal}{+8.3} \\
        EVA ViT-E/14 (4.4B) & 93.4 / 99.0 & \gpt{93.2 / 99.1} & \textcolor{blue}{-0.2} & 93.0 / 98.7 & \gpt{95.8 / 99.9} & \textcolor{teal}{+2.8} & 74.8 / 93.3 & \gpt{86.5 / 99.0} & \textcolor{teal}{+11.7} \\ \midrule
        \rpink
        \textbf{\whred{GPT-4V}} & \Block{1-3}{86.2 / 93.8} & & & \Block{1-3}{90.8 / 98.6}  & & & \Block{1-3}{83.7 / 94.9} & &  \\
        \bottomrule
        \toprule
       Dataset & \Block{1-3}{HMDB-51} & & & \Block{1-3}{Kinetics-400} & & & \Block{1-3}{Something-Something V1} & & \\
        Backbone (\#Param)  & Baseline & \textbf{\whblue{GPT Prompts}} & Top-1 \(\Delta\) & Baseline & \textbf{\whblue{GPT Prompts}} & Top-1 \(\Delta\)  & Baseline & \textbf{\whblue{GPT Prompts}} & Top-1 \(\Delta\)  \\ \cmidrule(lr){1-1} \cmidrule(lr){2-4} \cmidrule(lr){5-7} \cmidrule(lr){8-10}
        CLIP ViT-B/32 (88M) & 38.4 / 64.9 & \gpt{47.2 / 78.4} & \textcolor{teal}{+8.8} & 47.9 / 75.2 & \gpt{52.7 / 79.5} & \textcolor{teal}{+4.8} & 2.2 / 7.9 & \gpt{3.0 / 10.6} & \textcolor{teal}{+0.8} \\ 
        CLIP ViT-B/16 (86M) & 41.9 / 69.9 & \gpt{51.1 / 80.5} & \textcolor{teal}{+9.2} & 52.8 / 78.4 & \gpt{55.2 / 82.1} & \textcolor{teal}{+2.4} & 2.7 / 9.2 & \gpt{3.4 / 11.2} & \textcolor{teal}{+0.7} \\
        CLIP ViT-L/14 (304M) & 46.8 / 75.1 & \gpt{55.6 / 86.0} & \textcolor{teal}{+8.8} & 60.6 / 83.9 & \gpt{63.3 / 87.2} & \textcolor{teal}{+2.7} & 3.7 / 11.9 & \gpt{3.9 / 13.8} & \textcolor{teal}{+0.2} \\
        EVA ViT-E/14 (4.4B) & 41.5 / 70.3 & \gpt{56.4 / 86.7} & \textcolor{teal}{+14.9} & 61.5 / 84.9 & \gpt{67.6 / 87.7} & \textcolor{teal}{+6.1} & 3.8 / 12.2 & \gpt{5.3 / 16.2} & \textcolor{teal}{+1.5} \\ \midrule
        \rpink
        \textbf{\whred{GPT-4V}} & \Block{1-3}{63.2 / 86.9} & & & \Block{1-3}{58.8 / 75.7}  & & & \Block{1-3}{4.6 / 11.3}  & &  \\
        \bottomrule
        % \toprule
        \ctoprule{1-7}
       Dataset & \Block{1-3}{ModelNet10} & & & \Block{1-3}{\textbf{Average over 16 datasets}} \\
        Backbone (\#Param)  & Baseline & \textbf{\whblue{GPT Prompts}} & Top-1 \(\Delta\)  & Baseline & \textbf{\whblue{GPT Prompts}} & Top-1 \(\Delta\)  \\ \cmidrule(lr){1-1} \cmidrule(lr){2-4} \cmidrule(lr){5-7}
        CLIP ViT-B/32 (88M) & 42.9 / 84.8 & \gpt{47.8 / 86.1} & \textcolor{teal}{+4.9} & 49.7 / 76.6 & \gpt{56.7 / 81.9} & \textcolor{teal}{+7.0} \\ 
        CLIP ViT-B/16 (86M) & 39.1 / 82.2 & \gpt{48.7 / 93.3} & \textcolor{teal}{+9.6}  & 53.0 / 77.9 & \gpt{59.8 / 84.4} & \textcolor{teal}{+6.8} \\
        CLIP ViT-L/14 (304M) & 59.5 / 88.1 & \gpt{60.2 / 92.2} & \textcolor{teal}{+0.7} & 58.9 / 83.1 & \gpt{65.2 / 87.9} & \textcolor{teal}{+6.3} \\
        EVA ViT-E/14 (4.4B) & 70.0 / 94.2 & \gpt{80.3 / 99.7} & \textcolor{teal}{+10.3} & 66.0 / 86.9 & \gpt{73.2 / 90.9} & \textcolor{teal}{+7.2} \\ 
        \cmidrule{1-7}
        \rpink
        \textbf{\whred{GPT-4V}} & \Block{1-3}{66.8 / 90.9} & & & \Block{1-3}{64.5 / 81.9}  & & \\
        \cbottomrule{1-7}
        % \bottomrule
    \end{NiceTabular}
    }
    \caption{Main results in zero-shot visual recognition across the 16 datasets, reporting Top-1 and Top-5 accuracy (\%). We also report the parameter count of CLIP's image backbone for reference. ``Baseline'' denotes the direct use of category names. ``\textbf{\whblue{GPT Prompts}}'' refers to the utilization of multi-sentence descriptions generated by GPT-4 Turbo API for category names. ``\textbf{\whred{GPT-4V}}'' indicates the use of the GPT-4 Turbo with vision API for visual content recognition.}
    \label{tab: gain} % TODO: label this table later
\end{table*}




% Moreover, by the end of our experimental phase (November 17, 2023), due to OpenAI's ongoing limitation of 100 RPD (requests per day) per account for this model, we've adopted batch testing to ensure we fully utilize each request. For image datasets, we submit sets of images (\eg, 10) in a single query, prompting the API to return results for the entire batch simultaneously.
% For video and point cloud data, we involve inputting 10 videos (totaling 30 frames) or 5 point cloud instances (also totaling 30 images) at once, coupled with specific prompts such as sample names to facilitate separation and identification.
% Further details about our batch testing prompts is available in our \href{https://github.com/whwu95/GPT4Vis}{Code Repo}.
 







% \begin{figure*}
%   \centering
%   \begin{subfigure}{0.3\linewidth}
%     \fbox{\rule{0pt}{2in} \rule{.9\linewidth}{0pt}}
%     \caption{An example of a subfigure.}
%     \label{fig:short-a}
%   \end{subfigure}
%   \hfill
%   \begin{subfigure}{0.3\linewidth}
%     \fbox{\rule{0pt}{2in} \rule{.9\linewidth}{0pt}}
%     \caption{An example of a subfigure.}
%     \label{fig:short-b}
%   \end{subfigure}
%   \hfill  
%   \begin{subfigure}{0.3\linewidth}
%     \fbox{\rule{0pt}{2in} \rule{.9\linewidth}{0pt}}
%     \caption{Another example of a subfigure.}
%     \label{fig:short-c}
%   \end{subfigure}
%   \caption{Example of a short caption, which should be centered.}
%   \label{fig:short}
% \end{figure*}


\subsection{Gains from \whblue{GPT Prompts}}
% 16 3模态 4backbone
% GPT Prompts: RAF-DB, EuroSAT, Aircraft, flower, pets的提升巨大,xx负向;
% SSV1值得一提,CLIP, GPT-4V都很差
% GPT4V:大部分超L/14 CLIP
Table~\ref{tab: gain} showcases our evaluation results on 16 datasets and their average performance. 
For each dataset, we've detailed results using four different CLIP backbones, including OpenAI CLIP~\cite{clip}'s configurations of ViT-B/32, ViT-B/16, and ViT-L/14, each pre-trained with 400 million image-text pairs, and the EVA CLIP~\cite{sun2023eva}'s ViT-E/14, which is notable for its 4.4 billion parameters (14$\times$ that of ViT-L/14) and training on 2 billion image-text pairs. We will delve into an analysis of these results next.

Descriptions generated by GPT-4 distinctly surpass the CLIP baseline in a majority of datasets, boasting an average top-1 accuracy improvement of 7\% across 16 datasets. This consistent enhancement across all three modalities—images, videos, and point clouds—highlights the method's potent generalizability. More specifically:  

1) For image datasets, with RAF-DB~\cite{li2017reliable} as a focal point, GPT Prompts enable an over 20\% increment in accuracy across various backbones. For other datasets like EuroSAT~\cite{helber2019eurosat} satellite image classification, Flower~\cite{flower} fine-grained recognition, Pets~\cite{parkhi2012cats} fine-grained recognition, Aircraft~\cite{aircraft} fine-grained classification, and Caltech101~\cite{caltech101} object classification, we observe boosts of approximately 9-15\%. Smaller gains in Stanford Cars~\cite{Stanfordcar} and Food101~\cite{bossard2014food} suggest that a high density of similar categories may lead to ambiguous descriptions, confusing the CLIP model. In general, larger CLIP models achieve better zero-shot recognition performance on image tasks, and GPT-generated prompts reliably offer additional enhancements.


2) On video datasets, especially HMDB-51~\cite{hmdb} and UCF101~\cite{ucf101}, we observe astonishing gains of up to 11-15\%, indicating that rich descriptions of human actions align better with video content than simpler phrases. The Something-Something V1 (SSV1)~\cite{sth-sth} dataset, however, exhibits poor performance with the CLIP baseline (\textless~4\% Top-1) due to the lack of temporal modeling. Unlike Kinetics, UCF, and HMDB datasets, which can be recognized through scenes and object appearances as shown in Figure~\ref{fig:k400}, SSV1 demands the understanding of complex object-object and human-object interactions, requiring robust temporal and motion modeling for correct recognition. Hence, activities cannot be inferred merely from individual frames (\eg, Pushing something so it spins), as demonstrated in Figure~\ref{fig:SSV1}. 
In essence, with scene-based video recognition datasets, the larger the CLIP model, the greater the zero-shot performance, a trend consistent with image tasks where GPT Prompts lead to additional gains. Yet, in datasets where temporal modeling is crucial, CLIP's simple frame averaging strategy falls short, and GPT prompts cannot compensate for this deficiency.
% However, for video datasets that depend heavily on temporal relationships, simply averaging image embeddings from multiple frames via CLIP fails to achieve decent performance due to due to its inability to model temporal dynamics, making GPT prompts ineffective in such cases.

3) For point cloud datasets, employing multiple rendered viewpoints for zero-shot recognition with CLIP achieves noteworthy accuracy, mirroring the positive effects seen with image and scene-based video datasets. The integration of GPT Prompts further amplifies these positive results.





\begin{figure}[t]
  \centering
  \begin{subfigure}{1\linewidth}
    % \fbox{\rule{0pt}{2in} \rule{.9\linewidth}{0pt}}
    \includegraphics[width=\textwidth]{sec/figures/k400_case1.pdf}
    \caption{Ground Truth: Biking through snow. }
    \label{fig:k400_case1}
  \end{subfigure}
  \begin{subfigure}{1\linewidth}
    % \fbox{\rule{0pt}{2in} \rule{.9\linewidth}{0pt}}
    \includegraphics[width=\textwidth]{sec/figures/k400_case2.pdf}
    \caption{Ground Truth: Grinding meat.}
    \label{fig:k400_case2}
  \end{subfigure}
   \caption{Two video examples from the Kinetics dataset~\cite{i3d} accurately predicted by GPT-4V.}
  \label{fig:k400}
\end{figure}


\begin{figure}[t]
  \centering
  \begin{subfigure}{1\linewidth}
    % \fbox{\rule{0pt}{2in} \rule{.9\linewidth}{0pt}}
    \includegraphics[width=\textwidth]{sec/figures/ss_case1.pdf}
    \caption{Ground Truth: Pushing something so it spins (\faCheck). GPT-4V Prediction: Pretending to pick something up (\faTimes). }
    \label{fig:case1}
  \end{subfigure}
  \begin{subfigure}{1\linewidth}
    % \fbox{\rule{0pt}{2in} \rule{.9\linewidth}{0pt}}
    \includegraphics[width=\textwidth]{sec/figures/ss_case2.pdf}
    \caption{Ground Truth: Putting something into something (\faCheck). GPT-4V Prediction: Opening something (\faTimes).}
    \label{fig:case2}
  \end{subfigure}
   \caption{Two video examples from the Something-Something dataset~\cite{sth-sth} incorrectly predicted by GPT-4V.}
  \label{fig:SSV1}
\end{figure}

\begin{table*}[t]
  \centering
  \scalebox{0.82}{
  \setlength{\tabcolsep}{2.5pt}
    \begin{tabular}{lccccccccccc|ccc|c}
    \toprule
    Prompt for CLIP ViT-B/32 & DTD & SAT & SUN & RAF & Caltech & ImageNet & Aircraft & Flower & Cars & Food & Pets & K400 & UCF & HMDB & MNet10\\
    \midrule
    Baseline: Category name                      & 42.1 & 40.2 & 59.2 & 22.4 & 86.8 & 59.0 & 16.6 & 61.6 & 58.9 & 78.0 & 79.9 & 47.9 & 59.9 & 38.4 & {42.9} \\
    Hand-crafted Prompt                & 43.7 & 45.3 & {62.0} & 24.2 & {91.1} & {62.0} & 19.5 & 67.0 & {60.4} & {80.5} & 87.4 & {49.8} & 61.5 & 43.0 & 40.1 \\
    \whblue{\textbf{GPT Prompts}}                & {44.6} & \gpt{\textbf{49.4}} & 57.7 & \gpt{\textbf{45.8}} & 90.8 & 59.6 & {21.5} & \gpt{\textbf{71.8}} & 53.0 &  80.0 & {87.4} & 47.9 & {69.4} & {44.5} & \gpt{\textbf{47.8}} \\
    Hand-crafted Prompt + \whblue{\textbf{GPT Prompts}} & \gpt{\textbf{46.9}} & {48.0} & \gpt{\textbf{63.3}} & {34.5} & 
    \gpt{\textbf{92.8}} & \gpt{\textbf{63.7}} & \gpt{\textbf{21.9}} & {70.1} & \gpt{\textbf{61.2}} & \gpt{\textbf{80.8}} & \gpt{\textbf{89.3}} & \gpt{\textbf{52.7}} & \gpt{\textbf{69.9}} & \gpt{\textbf{47.2}} & 42.0 \\
    \bottomrule
    \end{tabular}
    }
  \caption{Evaluating the impact of different prompts on CLIP-based zero-shot visual recognition in image, video, and point cloud datasets. ``Hand-crafted Prompt'' denotes a fixed template, such as ``A photo of a \{category name\}." for image datasets, ``A video of a person \{category name\}." for video datasets, and ``A point cloud depth map of a \{category name\}." for point cloud datasets. ``\whblue{\textbf{GPT Prompts}}'' refers to descriptive sentences generated by GPT-4. ``Hand-crafted Prompt + \whblue{\textbf{GPT Prompts}}'' refers to a concatenation of a template with each descriptive sentence generated by GPT-4, such as ``A photo of a \{Category\}. \{GPT-generated sentence\}". 
  % \textbf{Bold} indicates the best and {underline} indicates the second best.
}
  \label{tab:clip_prompt}
\end{table*}

\subsection{Zero-shot Visual Performance of \whred{GPT-4V}}
To evaluate the visual capabilities of GPT-4V, as shown in Table~\ref{tab: gain}, we conduct quantitative evaluation across 16 datasets. Utilizing straightforward prompts (depicted in Figure~\ref{fig:gpt4v}), we obtain predictions from GPT-4V. 
Analyzing the average results from these 16 datasets, GPT-4V's top-1 accuracy approaches that of EVA's ViT-E. Specifically:

1) On image datasets, GPT-4V significantly outstrips the largest CLIP model EVA ViT-E on the RAF-DB dataset~\cite{li2017reliable} (68.7\% \vs 31.0\%), demonstrating a strong capability in facial expression recognition. Additionally, it outperforms EVA ViT-E in the fine-grained task of aircraft recognition~\cite{aircraft} (56.6\% vs. 50.6\%) and achieves comparable results in Caltech101 object recognition~\cite{caltech101}.
GPT-4V's ability to classify textures~\cite{dtd}, satellite images~\cite{helber2019eurosat}, and recognize pets~\cite{parkhi2012cats} situates it between the performance levels of CLIP ViT-L and EVA ViT-E. However, it slightly lags behind ViT-L in the more specialized areas of flower~\cite{flower} and food~\cite{bossard2014food} recognition. In the broad-spectrum challenge of ImageNet~\cite{deng2009imagenet} 1k-class recognition and car type identification~\cite{Stanfordcar}, GPT-4V's accuracy falls between that of ViT-B/32 and ViT-B/16.
In scenarios such as scene recognition~\cite{xiao2010sun}, GPT-4V's efficacy is close to ViT-B/32, illustrating its competitive yet varying performance across a spectrum of visual tasks.
It's noteworthy that, as per the GPT-4V documentation\footnote{https://platform.openai.com/docs/guides/vision}, the low-resolution version of the model scales images to 512$\times$512, while the high-resolution version scales to 2048$\times$2048. As Table~\ref{tab:datasets} illustrates, many of the datasets feature relatively lower resolutions, with the majority significantly below 512$\times$512. This discrepancy may impact GPT-4V's recognition accuracy, as seen in the case of the EuroSAT dataset, which has a resolution of 64$\times$64.

2) For video datasets, it’s important to highlight that Something-Something V1~\cite{sth-sth} focuses on modeling temporal relationships, whereas UCF101\cite{ucf101}, HMDB51\cite{hmdb}, and Kinetics~\cite{i3d} are less dependent on such temporal relationships, meaning actions can often be inferred from individual frames, as shown in Figure~\ref{fig:k400}. GPT-4V performs well on Kinetics, UCF101, and HMDB51, significantly surpassing EVA ViT-E's performance on UCF101 and HMDB51: achieving 83.7\% \vs 74.8\% on UCF, and an even more significant 63.2\% \vs 41.5\% on HMDB. This superior performance may be due to GPT-4V's adeptness at drawing inferences from the context of adjacent frames.
Notably, GPT-4V's performance on the SSV1 dataset is also markedly poor, at just 4.6\% top-1 accuracy, which aligns with the CLIP baseline. 
This is exemplified in Figure~\ref{fig:SSV1}, where isolating each frame does not provide enough context to ascertain the person's activity; only through the analysis of motion information across a sequence of frames can we make a prediction. Such results highlight the limitations of GPT-4V in temporal modeling due to the absence of a video encoder capable of processing temporal dynamics and motions.

3) For point cloud datasets, GPT-4V demonstrates excellent performance with just six rendered images, on par with EVA ViT-E. It stands to reason that adding more views would likely enhance the recognition accuracy even further.

In this section, all the above results are intended to provide baseline data and experience for future research, encouraging the development of more effective prompts to guide GPT-4V towards more accurate recognition. 
 



\begin{figure}[t]
  \centering
   \includegraphics[width=1\linewidth]{sec/figures/clip_vs_gpt4prompt.png}
   \vspace{-5mm}
   \caption{GPT Prompt \vs Category Name in some classes.} 
   \label{fig:clip_gpt4prompt}
\end{figure}



\subsection{Ablation Studies on \whblue{GPT Prompts}}
Here we present several ablation studies demonstrating the impact of prompts on CLIP's zero-shot performance.

\vspace{1mm}
\noindent\textbf{Impact of different prompts.}
Table~\ref{tab:clip_prompt} comprehensively exhibits the results of different prompts on the zero-shot visual recognition performance of CLIP across various datasets. Augmenting this with Hand-crafted Prompt combined with the category names leads to further improvements in most datasets, showcasing the method's robustness.
We then explore the effectiveness of employing multiple GPT-generated descriptive sentences related to category names. We find that GPT Prompt outperforms the baseline in 14 datasets. Figure~\ref{fig:clip_gpt4prompt} showcases the performance enhancement of GPT Prompts over category names in certain categories. 
Also, GPT Prompt can achieve better performance than Hand-crafted Prompts in 10 datasets. 
Our conjecture is that single category names may convey more global concepts, while the fine-grained details in generated descriptions are likely to align more closely with the visual content, thus amplifying inter-class distinctiveness. The strategy of generating multiple descriptive sentences may potentially further augment this effect. 
% For instance, in the image dataset RAF-DB (45.8\% \vs 24.2\%), video dataset UCF101 (69.4\% \vs 61.5), and point cloud dataset ModelNet10 (47.8\% \vs 40.1\%).


However, it's noteworthy that GPT Prompts are either below or roughly on par with Hand-crafted Prompt in 6 datasets, particularly in SUN397~\cite{xiao2010sun}, ImageNet-1k~\cite{deng2009imagenet}, Oxford Cars~\cite{Stanfordcar}, and Kinetics-400~\cite{i3d}. These datasets generally have a large number of categories with an emphasis on highly fine-grained classification. For such closely similar categories (like similar cars or scenes), richer descriptions generated may not be as distinctive as simply using the category name. Therefore, we consider combining ``Hand-crafted Prompt + GPT Prompts" to amalgamate the advantages of both, which has led to improved results in 11 datasets. For the 4 datasets (\ie, EuroSAT~\cite{helber2019eurosat}, RAF-DB~\cite{li2017reliable}, Flower102~\cite{flower} and ModelNet10~\cite{modelnet40}) where GPT Prompts demonstrate a clear advantage, the integration of Hand-crafted Prompt has been deemed unnecessary.



\begin{table}[t]
    \centering
    \scalebox{0.9}{
    \begin{tabular}{lcc}
    \toprule
     Method & \# Sentences   &  Top-1(\%) \\ \midrule
      Baseline: Category name   & 1 & 40.2  \\ \midrule
      \multirow{6}{*}{\whblue{\textbf{GPT Prompts}}}  & 1 & 38.4  \\
       & 3 & 42.9   \\
       & 5 & 47.4  \\
       & 10 & 49.1  \\
       & 20 & 49.4   \\
       & 30 & 49.9   \\
      \bottomrule
    \end{tabular}
    }
    \caption{The impact of different numbers of sentences generated by GPT on EuroSAT dataset. Backbone: CLIP ViT-B/32.}
    \label{tab:n_sen}
\end{table}

\vspace{1mm}
\noindent\textbf{Impact of sentence quantity generated by GPT.}
Our exploration also delved into the effect of the number of descriptive sentences generated by GPT-4 on zero-shot performance. 
Taking the EuroSAT~\cite{helber2019eurosat} dataset as an example, as shown in Table~\ref{tab:n_sen}, performance with only one generated sentence was lower than using the category name alone. However, an increase to three sentences led to a noticeable improvement and surpassed the baseline (42.9\% \vs 40.2\%). With five sentences, there was a substantial performance boost. In pursuit of identifying a saturation point for this improvement, we observed that increasing to 20 sentences brought about minimal additional benefits. Consequently, we adopt the generation of 20 sentences as the default setting for our experiments.





