\section{Problem Statement}

\vspace{-.5em}

Consider a MA-STL specification $\Psi$ on $N$ agents $\mathcal{N}= \{1,2,\ldots,N\}$, where each agent is at a position $\position[t]{i} \in \positionspace \subset \R^n$ with $n$ being 2 or 3 for 2D or 3D environments respectively.
Assume that each state $\state[t]{i}$ of agent $i$ can be directly mapped to its position $\position[t]{i}$, say the first $n$ elements of $\state[t]{i}$ by a function $\filterstate: \statespace \rightarrow \positionspace$.
Similar to \citet{zhang_gcbf_2024}, we include a LiDAR based observation of $\nrays > 0$ for each agent measuring the distance to the nearest obstacle in the environment with a sensing radius $R > 0$.
The $j$-th ray of agent $i$ is denoted as $\yray[j]{i}(t)$, where $\yray[j]{i}(t) \in \mathbb{R}^+$ is the distance to the nearest obstacle in the direction of the $j$-th ray at time $t$.

% Along the lines of \citep{Sun2022}, assume that we have a controller to follow these paths within a $\epsilon$-bound, 
% ensuring that any trajectory with these plans satisfies $\Psi$ within our error-bound 
% and focus on creating these plans for use by the controller.

\paragraph{The MA-STL motion planning problem}
\label{sec:stl-mamp}
We now establish the problem of motion planning for MA-STL in multi-agent systems. 
Essentially, the objective is to identify a set of reference goals that when followed satisfy a given MA-STL specification, 
while ensuring that there are no collisions between the agents.
Suppose there are $N$ agents involved, and the time bound is denoted by $T_h$. 
The planner $\plannn{i}$ generates a sequence of goals $\traj_{\goal{i}} = (\goal[0]{i}, \goal[1]{i}, \ldots, \goal[T]{i})$
 for agent $i$ with a given plan length $T < T_h$.
Each agent has a size radius represented by $r$, where $r > 0$. 
This means that when an agent is at position $p \in \mathcal{W}$, it is entirely contained within a ball of radius $r$ centered at $p$, 
denoted as $B_{r}(p)$. 
With these considerations, we can define the planning problem as follows:

\begin{definition}[Motion Planning in MA-STL]
	For a given MA-STL specification $\Psi = \bigwedge_{i=1}^N \phi_i$ and a set of $N$ agents $\agents$, the motion planning problem is finding a distributed
	control policy $\pi_i$ and a planner $\plannn{i}$ for each agent $i$ such that the following conditions are satisfied for closed-loop trajectories of agents in $\mathcal{N}$ with length $T_h$:
\begin{itemize}
	\item (Safety - Agents) For all $t \in [0, T_h]$, and for all $i, j \in \mathcal{N}$ where $i \neq j$, $||p_i(t) - p_j(t)|| \geq 2r$.
	\item (Safety - Obstacles) For all $t \in [0, T_h]$, and for all $i \in \mathcal{N}$, $\yray[j]{i} (t) \geq 2r$ for all $j \in [n_{rays}]$.
	\item (STL Satisfaction) 
	% Async STL satisfaction
	There exists $t_0, t_1, \ldots, t_{T}$ such that $t_i \in \{0, \ldots, T_h\}$ and $t_0 < t_1 < \ldots < t_{T}$ 
	% Sync STL satisfaction (choose one)
	% TODO: Explain the difference between the two
	such that the closed-loop trajectories $\tau = (s(t_0), s(t_1), \ldots, s(t_{T}))$
	of the agents satisfy the MA-STL specification $\Psi$ i.e. $\rho(\Psi, \tau) \geq 0$.

 \item (Achievability) For all $i\in \agents$, given the goal trajectory $\traj_{\goal{i}}$  of length $T$ from $\plannn{i}$, the gap  $\trajdist{\traj_i}{\traj_{\goal{i}}} = \sum_{t'=0}^{T} \norm{\filterstate(\state[t_{t'}]{i}) - \filterstate(\goal[t']{i})}_2 < \epsilon$  for a small %N $\epsilon>0$, 
 $\epsilon \in \R^+$. %Say is low? Quantify? < \eps?
\end{itemize}
\label{def:ps-STL-MA}
\end{definition}
\vspace{-1.5em}

%\begin{definition}[Motion Planning in MA-STL]
%	A motion planning problem in MA-STL is defined by a tuple
%	\begin{equation}
%		\langle p^\text{init}_1, \cdots, p^\text{init}_N, \Psi \rangle,
%	\end{equation}
%	where $p^\text{init}_i \in \mathcal{W}$ represents the initial position of agent $i$, and $\Psi$ is an MA-STL formula. The objective is to find a group of reference paths $(p_1, \cdots, p_N)$ that satisfy the following conditions:
%	\begin{enumerate}
%		\item (Initial conditions) $p_i(0) = p^\text{init}_i$, for all $i \in [N]$.
%		\item (No inter-agent collisions) For all $t \in [0, T]$, and for all $i, j \in [N]$ where $i \neq j$, $B_{s_i+\epsilon}(p_i(t)) \cap B_{s_j+\epsilon}(p_j(t)) = \emptyset$.
%		\item (STL Satisfaction) The reference paths $(p_1, \cdots, p_N)$ are $\epsilon$-robust with respect to $\Psi$.
%	\end{enumerate}
%	\label{def:STL-MAMP}
%\end{definition}
%
%Instead of searching for the reference paths among all possible time-based functions, the proposed approach restricts the search space to piece-wise linear (PWL) paths. Throughout the paper, we refer to the reference PWL path for agent $i$ as $\pwl_i$. 
%
%
%\begin{definition}[Piece-wise Linear Path]
%	A piece-wise linear path $\pwl_i$ in the workspace $\mathcal{W}$ is a function $\pwl_i: \nnreals \rightarrow \mathcal{W}$ that maps a time instant $t$ to a position $\pwl_i(t) \in \mathcal{W}$. It is constructed from a sequence of waypoints ${(t_{i,k}, p_{i,k})}_{k=0}^{K_i}$, where each waypoint is associated with a specific time stamp and position.
%	
%	The path $\pwl_i(t)$ is defined as follows within each time interval $[t_{i,k-1}, t_{i,k}]$:
%	\begin{equation}
%		\pwl_i(t) = p_{i,k-1} + \frac{p_{i,k} - p_{i,k-1}}{t_{i,k} - t_{i,k-1}} (t - t_{i,k-1}),
%	\end{equation}
%	where $p_{i,k-1}$ and $p_{i,k}$ are the positions at the previous and current waypoints, respectively.
%	
%	The sequence of waypoints has associated time stamps $t_{i,k}$ and positions $p_{i,k}$. The waypoints are arranged in increasing order of time stamps, denoted as $0=t_{i,0} \leq t_{i,1} \leq \cdots \leq t_{i,K}$, where $K_i$ is the total number of waypoints for path $S_i$. Each time interval $[t_{i,k-1},t_{i,k}]$ corresponds to a segment of $\pwl_i$, denoted as $\pwl^{(k)}_i$.
%	
%\end{definition}

\paragraph{Scaling STL for Multi-agent Systems}
\label{sec:ps-stl}
% \joe{Do we need this motivation? Piece-wise linear planners are only mentioned in this section.}

\begin{wraptable}{t}{0.5\textwidth}
	\centering
	\begin{tabular}{c|c|c}
		N & Spec. 1 / Spec. 2 & Planning Time (s)  \\
		\hline
		3 & 1 seq / 2 seq & 11 / 292 \\
		5 & 1 seq / 2 seq & 211 / - \\
	\end{tabular}
	\caption{Planning when considering disjoint time or space \citep{Sun2022}, a PWL plan with $K=6$  segments (1 \seq~) / $K=10$ segments (2 \seq). The $X$ \seq~ spec. has $X$ sequential waypoints.}
	\label{tab:disj_time_or_space}
\end{wraptable}

Given an MA-STL specification $\Psi$ on a system of $N$-agents we would like to provide a decentralized algorithm 
to execute a policy satisfying the specification with high probability. While we might assume a plan-then-execute technique~\citep{Sun2022} that finds a Piece-Wise Linear (PWL) path for each agent with $K$ segments,
such an approach quickly fails to scale with specification complexity and number of agents 
 when considering collisions between agents at planning time.
We posit this is primarily due to its collision avoidance mechanism that introduces $\mathcal{O}({C^N_2}*K^2)$ new variables, which quickly blows up (where $C^N_2 = N(N-1)/2$).
Consider two goal regions $A$ and $B$ and a sequential STL specification requiring agents to visit $A$ (viz. 1 \seq) or to visit $A$ then $B$ (viz. 2 \seq) while avoiding collisions.
Table \ref{tab:disj_time_or_space} demonstrates this by timing out (over 50 minutes) for a simple STL specification with $N=5$ agents in a 2-D environment with Single Integrator dynamics as well as all specifications and number of agents considered in this work (Sec. \ref{sec:exp-setup}, App. \ref{app:stl-specs}) .  %\joe{claim tested on a few}

% \begin{table}[h!]
% 	\centering
	
% 	\begin{tabular}{c|c|c}
% 		N & spec1 / spec2 & time to plan (s)  \\
% 		\hline
% 		3 & 1 seq / 2 seq& 11 / 292\\
% 		5 & 1 seq / 2 seq& 211 / -\\
% 		% instead of / use latex
% 	\end{tabular}
% 	\caption{Planning when considering disjoint time or space \citep{Sun2022}, a PWL plan w/ $K=6$  segments (1 seq) / $K=10$ segments (2 seq).
% 	The $X$ seq spec. has $X$ sequential waypoints.
% 	% Notably the planner times out for $N=5$ agents with 2 sequential waypoints.
% 	}
% 	\label{tab:disj_time_or_space}
% \end{table}
% \vspace{-2em}

Accounting for collision-avoidance independent of the objective is not novel  \citep{qin2021learning, yu2022learning}, but, as we argue in this paper,  in order to satisfy 
an STL specification, one must account for the temporal nature of the specification simultaneously with performing any collision-avoidance maneuvers.
An alternative, as we propose, is to plan for the objectives while adjusting for collision avoidance by means of an iterative training procedure involving the safety controller (such as \gcbfp) and the planner.
% Substantiate this with results on these methods failing for STL specs