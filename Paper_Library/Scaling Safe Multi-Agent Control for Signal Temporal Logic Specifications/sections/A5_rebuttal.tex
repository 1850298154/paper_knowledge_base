\section{Additional Baselines}
We focus on the most complex non-linear environment as presented in the main paper (DubinsCar) and have presented the results over 30 initial seeds in Table  \ref{tab:rebuttal-perf} and \ref{tab:rebuttal-finish-safe} and included bar charts with error bars for the main experiments in Figures \ref{fig:bar plots} and \ref{fig:bar plots2}. We chose this format as including all values in the tables created excessive clutter.

\subsection{Following a single common plan}
Since the tasks are homogenous, one might attempt to follow a common plan among agents with a formation. To compare this, we include results on a common shared plan generated by the STLPY solver \cite{kurtz2022mixed} (marked STLPY (S)) for the specifications considered in Table  \ref{tab:rebuttal-perf} and \ref{tab:rebuttal-finish-safe}. The initial starting state of this plan was near the first predicate in each specification (viz. goal $A$).

An obvious benefit of a single plan generated by the STLPY solver is a reduced planning time. Using a single common plan also enforced a rudimentary level of coordination among agents as evident in the \branch~task reducing the number of crossing paths (present in the earlier individually generated STLPY plans). However, in long horizon specifications such as \loopspec~and \signalspec~this effect is detrimental causing increased safety violations (leading to a low overall success rate) due to a lack of a more informed coordination among agents. We posit the fixed predicate sizes further play a role in preventing this coordination scheme to succeed.

\subsection{Centralized Planner}

We found that the challenges in scalable collision avoidance modeling with the PWL MA-STL planner \cite{Sun2022} (discussed in Section \ref{sec:ps-stl}) were also present in other centralized STL planning strategies \cite{dawson2022robust}. Namely, directly adding predicates for collision avoidance into the STL specification proved intractable for optimization over the long planning horizons considered due to an $\mathcal{O}({C^N_2}*K^2)$ variable blowup for $N$ agents and planning horizon $K$. To address this following our proposed method, we attempt planning for the joint multi-agent task without considering collisions during planning, while introducing run-time collision avoidance schemes like \gcbfp. As a result, in the simulation step of \cite{dawson2022robust} (Alg. 1, line 4), the \gcbfp policy would be incorporated into the rollout. However, in our experiments, we found that the gradient vanishes too quickly (within 50 time steps) to meaningfully differentiate through the trajectory, which spans over 200 time steps for all our tasks (Table \ref{tab:stl-specs-formal}, Appendix \ref{app:stl-specs}). Consequently, the algorithm in \cite{dawson2022robust} cannot directly handle the longer-range STL tasks considered in our work. Additionally, we noted that the proposed approach was not robust to many initial starting positions, often leading to unfruitful plans.
    
To build a centralized planner with \cite{dawson2022robust}, we opted to run the proposed method without incorporating collision avoidance during planning time simulations. This choice allowed us to achieve longer optimization horizons within a limited initial starting position range. The resulting plans were executed using the trained \gcbfp~controller. The results, as shown marked CE in \ref{tab:rebuttal-perf} and \ref{tab:rebuttal-finish-safe} for the DubinsCar Environment, indicate that the proposed approach performed demonstrably worse than ours and the other compared methods (note Success Rate). The planning time of this centralized approach scaled linearly with the number of agents since agent-agent interactions were not considered. These results further point to the importance of an achievability loss ($\mathcal{L}_{ach}$, Sec. \ref{sec:app-diff-stl-planning}) when planning in the presence of run-time collision avoidance. This outcome reinforces the notion that current methods for centralized planning in multi-agent STL requires further advancements to effectively handle scalable collision avoidance.

\subsection{Signal Specification (demonstrating Until, $U$)}

In Table \ref{tab:rebuttal-perf} we include a \signalspec~specification demonstrating the Until operator. This is a variant of our \loopspec~specification (Table \ref{tab:stl-specs-formal}, Appendix \ref{app:stl-specs}) called \signalspec~with an additional predicate $\Psi_1$ representing reaching goal $A$ twice. If \loopspec~is represented as $\Phi_1$, and $\Psi_2$ represents reaching a new goal $D$, the \signalspec~specification is ($\Phi_1 U_{[0,1]}\Psi_1)\land\Psi_2$ i.e. \loopspec~ $A$, $B$, and $C$ until $A$ is reached twice, then reach $D$. 


\begin{table*}[t!]
	\small
	\centering


\scalebox{0.55}{

\begin{tabular}{ll|ccccc||ccccc||ccccc}
\toprule
 &  & \multicolumn{5}{c|}{Planning Time (s) $\downarrow$} & \multicolumn{5}{c|}{Success Rate $\uparrow$} & \multicolumn{5}{c}{TtR $\downarrow$} \\
			\midrule
 \multicolumn{2}{c|}{Planner}  & \multirowcell{2}{\shortstack{CE \cite{dawson2022robust}}} & \multirowcell{2}{\shortstack{GNN-ODE}} &\ \multirowcell{2}{\shortstack{ODE}} & \multirowcell{2}{\shortstack{STLPY}} & \multirowcell{2}{\shortstack{STLPY (S)}} & \multirowcell{2}{\shortstack{CE \cite{dawson2022robust}}} & \multirowcell{2}{\shortstack{GNN-ODE}} &\ \multirowcell{2}{\shortstack{ODE}} & \multirowcell{2}{\shortstack{STLPY}} & \multirowcell{2}{\shortstack{STLPY (S)}} & \multirowcell{2}{\shortstack{CE \cite{dawson2022robust}}} & \multirowcell{2}{\shortstack{GNN-ODE}} &\ \multirowcell{2}{\shortstack{ODE}} & \multirowcell{2}{\shortstack{STLPY}} & \multirowcell{2}{\shortstack{STLPY (S)}} \\
\cline{1-2}Spec & N &  &  &  &  &  &  &  &  &  &  &  &  &  &  &  \\
\midrule
\multirow[t]{6}{*}[-1em]{\STAB{\rotatebox[origin=c]{90}{\footnotesize Branch}}} & 8 & 8.29 & \textbf{0.01} & \textbf{0.01} & 22.48 & 1.53 & 87.50 & \textbf{100.00} & 98.75 & 85.00 & \textbf{100.00} & 598.21 & 708.25 & 518.89 & \textbf{357.00} & 520.00 \\
 & 16 & 16.73 & \textbf{0.02} & \textbf{0.02} & 43.80 & 1.55 & 80.00 & \textbf{100.00} & 96.67 & 52.50 & 99.17 & 638.81 & 740.77 & 556.64 & \textbf{429.81} & 582.44 \\
 & 32 & 35.62 & \textbf{0.03} & \textbf{0.03} & 87.92 & 1.52 & 65.62 & \textbf{85.10} & 81.35 & 18.12 & 83.23 & 757.52 & 823.10 & 651.32 & \textbf{572.39} & 695.18 \\
\cmidrule{1-17}
\multirow[t]{6}{*}[-1em]{\STAB{\rotatebox[origin=c]{90}{\footnotesize Cover}}} & 8 & 16.20 & 0.02 & \textbf{0.01} & 10.40 & 2.56 & 95.00 & \textbf{100.00} & 98.75 & 95.00 & \textbf{100.00} & 1044.00 & 1056.25 & 758.04 & \textbf{429.76} & 845.00 \\
 & 16 & 30.73 & \textbf{0.02} & \textbf{0.02} & 20.14 & 2.58 & 76.25 & \textbf{97.50} & 97.08 & 76.25 & 87.71 & 1159.90 & 1113.79 & 795.49 & \textbf{536.33} & 989.38 \\
 & 32 & 63.98 & 0.03 & \textbf{0.02} & 40.19 & 2.61 & 52.50 & 81.77 & \textbf{89.38} & 53.75 & 54.90 & 1319.63 & 1241.27 & 893.52 & \textbf{708.22} & 1218.61 \\
\cmidrule{1-17}
\multirow[t]{6}{*}[-1em]{\STAB{\rotatebox[origin=c]{90}{\footnotesize Loop}}} & 8 & 23.41 & 0.02 & \textbf{0.01} & 26.16 & 4.37 & 12.50 & \textbf{99.58} & 97.50 & 80.00 & 85.83 & 1757.25 & 1762.04 & 4192.00 & \textbf{1095.29} & 1179.33 \\
 & 16 & 52.27 & \textbf{0.02} & \textbf{0.02} & 52.79 & 4.38 & 7.50 & \textbf{98.96} & 96.25 & 66.25 & 59.38 & 2884.33 & 1819.61 & 4139.50 & \textbf{1301.54} & 1379.57 \\
 & 32 & 107.17 & \textbf{0.03} & \textbf{0.03} & 111.62 & 4.41 & 6.25 & \textbf{97.60} & 72.50 & 38.75 & 26.35 & 3111.50 & 1951.11 & 4478.88 & \textbf{1598.19} & 1940.58 \\
\cmidrule{1-17}
\multirow[t]{6}{*}[-1em]{\STAB{\rotatebox[origin=c]{90}{\footnotesize Sequence}}} & 8 & 16.35 & 0.02 & \textbf{0.01} & 3.46 & 0.66 & 95.00 & 97.92 & 77.50 & 97.50 & \textbf{100.00} & 1044.00 & 1069.44 & 1296.13 & \textbf{637.11} & 850.75 \\
 & 16 & 29.80 & \textbf{0.02} & \textbf{0.02} & 6.96 & 0.68 & 80.00 & 89.38 & 72.50 & 85.00 & \textbf{94.58} & 1156.38 & 1169.49 & 1337.42 & \textbf{785.97} & 957.92 \\
 & 32 & 63.99 & \textbf{0.03} & \textbf{0.03} & 13.62 & 0.68 & 48.12 & 61.88 & 49.58 & 59.38 & \textbf{63.23} & 1307.78 & 1226.14 & 1376.86 & \textbf{1013.90} & 1234.37 \\
\cmidrule{1-17}
\multirow[t]{6}{*}[-1em]{\STAB{\rotatebox[origin=c]{90}{\footnotesize Signal}}} & 8 & 56.59 & 0.02 & \textbf{0.01} & 13.60 & 17.87 & 0.00 & \textbf{100.00} & \textbf{100.00} & 97.92 & \textbf{100.00} & - & 2338.50 & 4179.25 & \textbf{976.53} & 1137.25 \\
 & 16 & 110.95 & \textbf{0.02} & \textbf{0.02} & 23.97 & 18.21 & 0.00 & \textbf{100.00} & 91.25 & 91.88 & 93.33 & - & 2352.67 & 4314.25 & \textbf{1116.07} & 1331.53 \\
 & 32 & 228.49 & \textbf{0.03} & \textbf{0.03} & 47.99 & 18.25 & 0.62 & \textbf{76.25} & 67.50 & 50.31 & 47.81 & 1761.00 & 2479.90 & 4664.87 & \textbf{1290.72} & 1831.21 \\
\cmidrule{1-17}
\bottomrule
\end{tabular}


}

	\caption{Performance of \textbf{STLPY} \cite{kurtz2022mixed}: (multi-agent), \textbf{STLPY(S)} \cite{kurtz2022mixed} : STLPY with a single common plan, \textbf{CE} \cite{dawson2022robust} : Centralized Counterexample guided planner and our approaches (\textbf{ODE}, \textbf{GNN-ODE}). 
 % the scalability in 
 vs the number of agents ($N$) and specification complexity for the DubinsCar Environment. 
 The results are averaged over 30 seeds and we highlight the best result in \textbf{bold}.
 \label{tab:rebuttal-perf}
		%	\SJ{The cells with the dark blue colors are not readable.  Define units: seconds for planning and TtR and for success rate}
  % We note an average 65\% improved success rate and highlight the best result in \textbf{bold}.
% \zk{TODO: add a TL;DR here.}	
}

\end{table*}

\begin{table*}[t!]
	\small
	\centering
\scalebox{0.65}{


\begin{tabular}{ll|ccccc|ccccc}
\toprule
 &  & \multicolumn{5}{c|}{Finish Rate $\uparrow$} & \multicolumn{5}{c}{Safety Rate $\uparrow$} \\
 \midrule
\multicolumn{2}{c|}{Planner}& \multirowcell{2}{\shortstack{CE \cite{dawson2022robust}}} & \multirowcell{2}{\shortstack{GNN-ODE}} &\ \multirowcell{2}{\shortstack{ODE}} & \multirowcell{2}{\shortstack{STLPY}} & \multirowcell{2}{\shortstack{STLPY (S)}} & \multirowcell{2}{\shortstack{CE \cite{dawson2022robust}}} & \multirowcell{2}{\shortstack{GNN-ODE}} &\ \multirowcell{2}{\shortstack{ODE}} & \multirowcell{2}{\shortstack{STLPY}} & \multirowcell{2}{\shortstack{STLPY (S)}} \\
\cline{1-2} Spec & N &  &  &  &  &  &  &  &  &  &  \\
\midrule
\multirow[t]{6}{*}[-1em]{\STAB{\rotatebox[origin=c]{90}{\footnotesize Branch}}} & 8 & 88.00 & \textbf{100.00} & 99.00 & \textbf{100.00} & \textbf{100.00} & \textbf{100.00} & \textbf{100.00} & \textbf{100.00} & 85.00 & \textbf{100.00} \\
 & 16 & 80.00 & \textbf{100.00} & 98.00 & 99.00 & \textbf{100.00} & 97.50 & \textbf{100.00} & 98.54 & 53.75 & 99.58 \\
 & 32 & 76.00 & \textbf{98.00} & 94.00 & 90.00 & 95.00 & \textbf{88.75} & 87.19 & 85.21 & 32.50 & 86.88 \\
\cmidrule{1-12}
\multirow[t]{6}{*}[-1em]{\STAB{\rotatebox[origin=c]{90}{\footnotesize Cover}}} & 8 & \textbf{100.00} & \textbf{100.00} & 99.00 & 95.00 & \textbf{100.00} & 95.00 & \textbf{100.00} & \textbf{100.00} & 97.50 & \textbf{100.00} \\
 & 16 & 96.00 & \textbf{100.00} & 99.00 & 78.00 & \textbf{100.00} & 80.00 & 97.50 & \textbf{98.33} & 87.50 & 87.71 \\
 & 32 & 98.00 & \textbf{99.00} & 98.00 & 80.00 & \textbf{99.00} & 53.75 & 81.98 & \textbf{91.04} & 56.88 & 55.21 \\
\cmidrule{1-12}
\multirow[t]{6}{*}[-1em]{\STAB{\rotatebox[origin=c]{90}{\footnotesize Loop}}} & 8 & 12.00 & \textbf{100.00} & \textbf{100.00} & 98.00 & \textbf{100.00} & \textbf{100.00} & \textbf{100.00} & 97.50 & 82.50 & 85.83 \\
 & 16 & 10.00 & 99.00 & \textbf{100.00} & 99.00 & 99.00 & 53.75 & \textbf{100.00} & 96.25 & 67.50 & 59.79 \\
 & 32 & 11.00 & 99.00 & \textbf{100.00} & \textbf{100.00} & 99.00 & 28.12 & \textbf{98.85} & 72.50 & 38.75 & 27.29 \\
\cmidrule{1-12}
\multirow[t]{6}{*}[-1em]{\STAB{\rotatebox[origin=c]{90}{\footnotesize Sequence}}} & 8 & \textbf{100.00} & 99.00 & 78.00 & 98.00 & \textbf{100.00} & 95.00 & 99.17 & 99.17 & \textbf{100.00} & \textbf{100.00} \\
 & 16 & 96.00 & 95.00 & 73.00 & 89.00 & \textbf{100.00} & 83.75 & 93.54 & \textbf{97.50} & 90.00 & 94.58 \\
 & 32 & \textbf{98.00} & 68.00 & 51.00 & 84.00 & \textbf{98.00} & 48.12 & 86.15 & \textbf{93.75} & 64.38 & 64.27 \\
\cmidrule{1-12}
\multirow[t]{6}{*}[-1em]{\STAB{\rotatebox[origin=c]{90}{\footnotesize Signal}}} & 8 & 0.00 & \textbf{100.00} & \textbf{100.00} & 99.00 & \textbf{100.00} & \textbf{100.00} & \textbf{100.00} & \textbf{100.00} & 98.75 & \textbf{100.00} \\
 & 16 & 0.00 & \textbf{100.00} & \textbf{100.00} & 97.00 & \textbf{100.00} & 57.50 & \textbf{100.00} & 91.25 & 95.00 & 93.54 \\
 & 32 & 1.00 & 90.00 & \textbf{97.00} & 64.00 & \textbf{97.00} & 17.50 & \textbf{83.12} & 68.12 & 75.94 & 48.85 \\
\cmidrule{1-12}
\bottomrule
\end{tabular}


}

	\caption{Finish Rate and Safety Rate of \textbf{STLPY} \cite{kurtz2022mixed}: (multi-agent), \textbf{STLPY(S)} \cite{kurtz2022mixed} : STLPY with a single common plan, \textbf{CE} \cite{dawson2022robust} : Centralized Counterexample guided planner and our approaches (\textbf{ODE}, \textbf{GNN-ODE}). 
 % the scalability in 
 vs the number of agents ($N$) and specification complexity for the DubinsCar Environment. Note that a successful trajectory is both finishing the task \textit{and} being safe. While STLPY and other single agent planning schemes may reach the goal quickly, safety is violated (shown in the Safety Rate).
 The results are averaged over 30 seeds  and we highlighted the best result in \textbf{bold}.
 \label{tab:rebuttal-finish-safe}
		%	\SJ{The cells with the dark blue colors are not readable.  Define units: seconds for planning and TtR and for success rate}
  % We note an average 65\% improved success rate and highlight the best result in \textbf{bold}.
% \zk{TODO: add a TL;DR here.}	
}
\end{table*}


% \subsection*{Centralized MA-STL Planner}

% We observed that the challenges in scalable collision avoidance modeling with the PWL MA-STL planner [16] (Section 3) were also present in other centralized STL planning strategies [3a]. Specifically, adding collision avoidance predicates directly into the STL specification was intractable for optimization over long planning horizons.

% To address this, we proposed planning for the joint multi-agent task without considering collisions during planning, while applying run-time collision avoidance schemes like GCBF+. In the simulation step of [3a] (Alg. 1, line 4), GCBF+ was incorporated into the rollout. However, our experiments revealed that the gradient vanished too quickly (within 50 time steps) to effectively differentiate through the trajectory, which spans over 200 time steps for all tasks (Table 4, Appendix Sec. B). Consequently, the algorithm in \cite{dawson} cannot handle the longer-range STL tasks in our work and was also not robust to many initial starting positions, often leading to unfruitful plans.

\begin{figure}[t!] % [t!] forces the figure to the top
	\centering 
	\begin{minipage}{\textwidth}
		\centering
		% \import{img/}{planner.pdf_tex}
		\includegraphics[width=\linewidth]{img/barplots/Single_Plan_DubinsCar_Success_Rate.png} % Replace with your image
	\end{minipage}%
 \\
	\begin{minipage}{\textwidth}
		\centering
		% \import{img/}{planner.pdf_tex}
		\includegraphics[width=\linewidth]{img/barplots/Single_Plan_DubinsCar_Safety_Rate.png} % Replace with your image
	\end{minipage}
 \\
 \begin{minipage}{\textwidth}
		\centering
		% \import{img/}{planner.pdf_tex}
		\includegraphics[width=\linewidth]{img/barplots/Single_Plan_DubinsCar_Planning_Time_s.png} % Replace with your image
	\end{minipage}%

 
	\caption{
		We provide bar plots with error bars notating the standard deviations of the metrics considered. These are complementary to Table \ref{tab:rebuttal-perf} and Table \ref{tab:rebuttal-finish-safe}.
	}
	\label{fig:bar plots}
\end{figure}

\begin{figure}[t!]
 \begin{minipage}{\textwidth}
		\centering
		% \import{img/}{planner.pdf_tex}
		\includegraphics[width=\linewidth]{img/barplots/Single_Plan_DubinsCar_TtR.png} % Replace with your image
	\end{minipage}%
 \\
 \begin{minipage}{\textwidth}
		\centering
		% \import{img/}{planner.pdf_tex}
		\includegraphics[width=\linewidth]{img/barplots/Single_Plan_DubinsCar_Finish_Rate.png} % Replace with your image
	\end{minipage}%
	\caption{
		We provide bar plots with error bars notating the standard deviations of the metrics considered. These are complementary to Table \ref{tab:rebuttal-perf} and Table \ref{tab:rebuttal-finish-safe}.
	}
	\label{fig:bar plots2}
\end{figure}