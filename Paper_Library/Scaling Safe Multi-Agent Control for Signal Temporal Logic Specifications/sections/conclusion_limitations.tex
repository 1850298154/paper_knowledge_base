\vspace{-0.5em}
\section{Limitations}

\paragraph{Model-based learning}
While a model-free approach to collision-avoidance \citep{wang2023acorl,Xiong2022} would be more amenable to handle
unknown environment dynamics, our approach is inherently model-based (as is \gcbf~\citep{zhang_neural_2023,zhang_gcbf_2024}, MACBF \citep{qin2021learning} and CAM \citep{yu2022learning}).
This is primarily due to the underlying controller and \gcbf~(akin to a barrier certificate), using the next state of the system 
while calculating the derivative for use in the loss function. 
%However even in a setting such as ours, we note a deficiency in the objective 
%achieving capabilities of the controllers for these complex temporal specifications rather than simple single goal tasks.
% point2
\paragraph{Map Complexity, Homogeneity} Additionally, since the approach is decentralized, complex maps requiring communication and coordination between agents
may cause safety issues. As mentioned in \citep{zhang_gcbf_2024}, it may be hard in dense regions to act in a decentralized
manner thus necessitating the use of inter-agent communication. 
% \joe{refer to results highlighting this.}
% point3
We have considered the homogeneous case in this work, where all agents have the same dynamics and STL specifications. 
However, in the heterogeneous case, agents may have different dynamics and STL specifications thus needing a more complex controller
and a planner capable of generalizing to multiple goal positions or STL specifications.
% point4
% Our experiments are limited to the 2D plane and we leave a 3D study to future work.
Our planner does not consider obstacles directly, although as demonstrated in the Appendix (Tables \ref{tab:single-spec-v-n-results}, \ref{tab:dubins-obs-spec-v-n-results}, \ref{tab:double-spec-v-n-results}), the \gcbfp~ controller to an extent provides inherent collision avoidance capabilities.
% point6
Finally, the approach is limited by the complexity of the environment and the number of agents. While we have shown that the approach scales
well with the number of agents, the complexity of the environment and the number of obstacles may cause the planner to fail to find a achievable plan.
\vspace{-0.5em}
\section{Conclusion}
In this work, we have presented a novel approach to planning for multi-agent systems with Signal Temporal Logic specifications.
Primarily we have shown that by using a differentiable STL robustness metric, we can optimize for the satisfaction of complex
temporal specifications given a controller with MA collision avoidance capabilities. 
We demonstrate that by training a GNN-ODE planner with a carefully constructed loss function
% in conjunction with a collision avoidance controller
we can overcome the limitations of the plan-then-execute approach and scale to complex specifications and large numbers of agents.
% We have also shown that our approach is able to handle complex environments with obstacles in  
% different configurations by means of the \gcbfp~controller.