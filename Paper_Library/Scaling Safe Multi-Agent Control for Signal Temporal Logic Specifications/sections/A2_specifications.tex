\section{STL Specifications}
\label{app:stl-specs}

We formally define the Signal Temporal Logic (STL) specifications used in the experiments in Table \ref{tab:stl-specs-formal}. The specifications include a sequential waypoint task (\seq), a coverage task (\cover), a loop task (\loopspec), and a branching task (\branch). The specifications are defined over a time horizon $T$ and are satisfied if the agents satisfy the corresponding STL formula.
We use four markers $A$, $B$, $C$, and $D$ to represent rectangular predicates centered around x-y coordinates $[0,0]$, $[2,2]$, $[2,0]$, and $[0,2]$, respectively. The predicates are defined as $p_i = \text{dist}(s_i, p_i) \leq 1.0$ where $\text{dist}(s_i, p_i)$ is the L1-norm ($|\cdot|_1$) distance between the agent $i$'s state $s_i$ and the predicate $p_i$.


\begin{table}[h]
    \centering
    \begin{tabular}{c|c|c|cc}
        \hline
        \textbf{Spec.} & \textbf{Description} & \textbf{Formula} & T & k \\
        \hline
        \seq & Sequence of goals
        &  $\E{0}{T/3}(A) \land \E{T/3}{2T/3}(B) \land \E{2T/3}{T}(C)$ & 15 & 20\\ 
        \cover & Coverage over goals & $\E{0}{T}(A) \land \E{0}{T}(B) \land \E{0}{T}(C)$ & 15 & 20 \\
        \loopspec & Loop over goals
        & $\G{0}{T/2}\left(\E{0}{T/2}(A) \land \E{0}{T/2}(B) \land \E{0}{T/2}(C)\right)$ & 30 & 20 \\
        \signalspec & Loop then move to final
        & $(\loopspec U_{[0,1]}\Psi_1)\land\E{0}{T}(D)$ & 30 & 20 \\
        \branch & Branching & $\left( \E{0}{T}(A) \land \E{0}{T}(B) \right) \lor \left( \E{0}{T}(C) \land \E{0}{T}(D) \right)$
        & 20 & 10\\
        \hline
    \end{tabular}
    \caption{STL specifications used in the experiments.
    $T$ and $k$ are the specification lengths and goal sample intervals respectively.}
    \label{tab:stl-specs-formal}
\end{table}