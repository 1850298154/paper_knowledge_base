\begin{abstract}
The pandemic by COVID-19 is causing a devastating effect on the health of the global population. Currently, there are several efforts to prevent the spread of the virus. Among those efforts, cleaning and disinfecting public areas have become important tasks and they should be automated in future smart cities. To contribute in this direction, this paper proposes a coverage path planning algorithm for a spraying drone, an unmanned aerial vehicle that has mounted a sprayer/sprinkler system, to disinfect areas. \textcolor{black}{State-of-the-art planners consider a camera instead of a sprinkler, in consequence, the expected coverage will differ at running time because the liquid dispersion is different from a camera's projection model. In addition, current planners assume that the vehicles can fly outside the target region; this assumption can not be satisfied in our problem, because disinfections are performed at low altitudes. Our method presents i) a new sprayer/sprinkler model that fits a more realistic coverage volume to the drop dispersion and ii) a planning algorithm that efficiently restricts the flight to the region of interest avoiding potential collisions in bounded scenes. The algorithm has been tested in several simulation scenes, showing that \textcolor{black}{it} is effective and covers more areas with respect to other two approaches in the literature.} Note that the proposal is not limited to disinfection applications, but can be applied to other ones, such as painting or precision agriculture.
\end{abstract}

