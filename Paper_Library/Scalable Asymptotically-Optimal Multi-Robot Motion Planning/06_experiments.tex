\label{sec:experiments}


This section provides an experimental evaluation of \drrtstar\ by
demonstrating practical convergence, scalability, and applicability to
dual-arm manipulation. The approach and alternatives are executed on a
cluster with Intel(R) Xeon(R) CPU E5-4650 @ 2.70GHz processors, and
128GB of RAM.  \footnote{Additional data are provided in Appendix
\ref{apx:Experiments}.}

\begin{wrapfigure}{r}{0.255\textwidth}
  \centering
  \includegraphics[width=0.250\textwidth]{polygonal_benchmark}
  \caption{The 2D environment where the 2 disk robots 
    operate.}
  \label{fig:poly_enviro}
\end{wrapfigure}

\noindent \textbf{2 Disk Robots among 2D Polygons:} This
base-case test involves $ 2 $ disks ($\cspace_i := \reals^2$) of
radius $0.2$, in a $10.2 \times 10.2$ region, as in
Figure~\ref{fig:poly_enviro}. The disks have to swap positions between
$(0,0)$ and $(9,9)$. This is a setup where it is possible to compute
the explicit roadmap, which is not practical in more involved
scenarios. In particular, \drrtstar\ is tested against: a) running
$\astar$ on the implicit tensor roadmap $\mmgimp$ (referred to as
``Implicit \astar'') defined over the same individual roadmaps with
$N$ nodes each as those used by \drrtstar; and b) an explicitly
constructed \prmstar\ roadmap with $N^2$ nodes in the composite space.

Results are shown in Figure~\ref{fig:polygonal_benchmark}.  \drrtstar\
converges to the optimal path over $\mmgimp$, similar to the one
discovered by Implicit \astar, while quickly finding an initial
solution of high quality. Furthermore, the implicit tensor product
roadmap $\mmgimp$ is of comparable quality to the explicitly
constructed roadmap.

\begin{figure}[h]
\centering
\includegraphics[width=0.48\textwidth]{convergence_without_label}
\caption{ Average solution cost over iterations. Data averaged over 
$10$ roadmap pairs.  $\drrtstar$ (solid line) converges to the optimal 
path through $\mmgimp$ (dashed line).}
\label{fig:polygonal_benchmark}
\end{figure}

Table~\ref{tab:2_robot} presents running times.
\drrtstar\ and implicit \astar\ construct $2$ $N$-sized roadmaps 
(row~3), which are faster to construct than the \prmstar\ roadmap in
$\cspace$ (row~1).  \prmstar\ becomes very costly as $N$ increases.
For $N=500$, the explicit roadmap contains $250,000$ vertices, taking
$1.7$GB of RAM to store, which was the upper limit for the machine
used. When the roadmap can be constructed, it is quicker to query
(row~2). \drrtstar\ quickly returns an initial solution (row~5), and
converges within $5\%$ of the optimum length (row~6) well before
Implicit \astar\ returns a solution as $N$ increases (row~4). The next
benchmark further emphasizes this point.

\begin{table}[h]
\centering
\caption{Construction and query times (SECs) for 2 disk robots.}
\label{tab:2_robot}
\small
\begin{tabular}{|l|c|c|c|}
\hline
\multicolumn{1}{|r|}{\textbf{Number of nodes: $ N $ =}} & \textbf{50} & \textbf{100} & \textbf{200} \\ \hline
{$N^2$-PRM* construction}                     & 3.427        & 13.293        & 69.551        \\ \hline
{$N^2$-PRM* query}                            & 0.002       & 0.004        & 0.023        \\ \hline
{2 $N$-size PRM* construction}              & 0.1351        & 0.274        & 0.558       \\ \hline
{Implicit A* search over $\mmgimp$}                     & 0.684       & 2.497        & 10.184        \\ \hline
{\drrtstar\ over $\mmgimp$ (initial) }                    & 0.343       & 0.257        & 0.358        \\ \hline
{\drrtstar\ over $\mmgimp$ (converged) }                    & 3.497       & 4.418        & 5.429        \\ \hline
\end{tabular}
\end{table}

\textbf{Many Disk Robots among 2D Polygons:} In the same environment as
above, the number of robots $R$ is increased to evaluate scalability.
Each robot starts on the perimeter of the environment and is tasked
with reaching the opposite side. An $N=50$ roadmap is constructed for
every robot. It quickly becomes intractable to construct a \prmstar\
roadmap in the composite space of many robots.

\begin{figure}[h]
\centering
\includegraphics[width=0.475\textwidth]{random_graphception}
\includegraphics[width=0.475\textwidth]{random_multi_cost_over_time}
\caption{  Data  averaged over $10$ runs.
(\textit{Top}): Relative solution cost and success ratio
of \drrtstar, \drrt\ and \rrtstar\ for increasing $R$. \drrtstar:
average iteration and variance for initial solution (top of box), and
solution cost and variance after $100,000$ iterations
(bottom). Similar results for \rrtstar.  Single data point for \drrt\
(no quality improvement after first solution). (\textit{Bottom}):
Solution costs over time.} 
\label{fig:scalability}
\end{figure}

Figure~\ref{fig:scalability} shows the inability of alternatives to
compete with \drrtstar\ in scalability. Solution costs are normalized
by an optimistic estimate of the path cost for each case, which is the
sum of the optimal solutions for each robot, disregarding robot-robot
interactions.  Implicit \astar\ fails to return solutions even for 3
robots. Directly executing \rrtstar\ in the composite space fails to
do so for $R \geq 6$.  The original \drrt\ method (without the
informed search component) starts suffering in success ratio for
$R \geq 5$ and returns worse solutions than \drrtstar.  The average
solution times for \drrt\ may decrease as $R$ increases but this is
due to the decreasing success ratio, i.e., \drrt\ begins to only
succeed at easy problems.

\begin{figure}[h]
\includegraphics[height=1.25in]{vertical_moto_query}
\includegraphics[height=1.25in]{success_rate_motoman}
\centering
\includegraphics[width=0.48\textwidth]{motoman_cost_over_time}
\caption{
(\textit{Top}): \drrtstar\ is run for a dual-arm 
manipulator to go from its home position (above) to a reaching 
configuration (below) and achieves perfect success ratio as $n$ 
increases.
(\textit{Bottom}): \drrtstar\ solution quality over time.  Here,
larger roadmaps provide benefits in terms of running time and solution
quality.}
\label{fig:motoman_convergence}
\end{figure}

\textbf{Dual-arm manipulator:} This test shows the benefits
of \drrtstar\ when planing for two $7$-dimensional arms.
Figure~\ref{fig:motoman_convergence} shows that
\rrtstar\ fails to return solutions within $100K$
iterations. Using small roadmaps is also insufficient for this
problem.  Both \drrtstar\ and Implicit \astar\ require larger roadmaps
to begin succeeding. But with $N \geq 500$, Implicit \astar\ always
fails, while \drrtstar\ maintains a $100\%$ success ratio. As
expected, roadmaps of increasing size result in higher quality
path. The informed nature of \drrtstar\ also allows to find initial
solutions fast, which together with the branch-and-bound primitive
allows for good convergence.

%Future work will be directed toward using \drrtstar\ when a kinematic
%linkage between parts of a system is unavoidable.

