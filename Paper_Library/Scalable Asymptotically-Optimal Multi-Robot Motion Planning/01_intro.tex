%% We need more references to multi-robot planning work

Many multi-robot planning applications \cite{Wagner:2015bd,
Gravot:2003kh, Gharbi:2009fu} require high-dimensional platforms to
simultaneously move in a shared workspace, where high-quality paths
must be computed quickly as sensing input is updated, as in
Fig. \ref{fig:setup}.  Preprocessing given knowledge of the static
scene can help the online computation of high-quality paths.
Sampling-based roadmaps can help with such high-dimensional challenges
and provide primitives for preprocessing a static
scene \cite{Kavraki1996Probabilistic-R, LaValle2001}.  These methods
converge to optimal solutions given sufficient
density \cite{Karaman2011Sampling-based-}, i.e., at least
$O(n\log(n))$ edges are needed for a roadmap with $n$ vertices, while
near-optimal solutions are achieved after finite computation
time \cite{Dobson:2015_Finite, Pavone:2015fmt}.


%==================

%Such methods quickly produce solutions by leveraging the
%configuration space ($\cspace$-space) abstraction.  While motion
%planning complexity depends exponentially on the problem's
%dimensionality $d$ \cite{kolountzakis1998analysis, HsuKav98},
%sampling-based approaches scale better than alternative methods.
%Primarily, their performance suffers from the existence of narrow
%passages in $\cfree$ \cite{Shi_Amato2014Narrow}.


%==================

\begin{figure}[t]
\centering
\includegraphics[height=1.3in]{setup_start}
\includegraphics[height=1.3in]{setup_goal}
\caption{Simultaneous planning for multiple high-dimensional systems is a
difficult, motivating challenge for this work.}
\label{fig:setup}
\end{figure}

%==================

Na\"ively constructing a sampling-based roadmap or tree in the robots'
composite configuration space provides asymptotic optimality but does
not scale well. In particular, memory requirements depend
exponentially on the problem's
dimension \cite{Schwartz1983Coordinated-Piano}.  The alternative is
decoupled planing, where paths for robots are computed independently
and then coordinated \cite{Leroy1999Multipath-Coordination}.  These
methods, however, typically lack completeness and optimality
guarantees. Hybrid approaches can achieve optimal decoupling to retain
guarantees
\cite{Berg:2009ve}.  The problem is more complex when the robots
exhibit non-trivial dynamics \cite{Peng2005Kinodynamic-Coord}.
Collision avoidance or control methods can scale to large numbers of
robots, but typically lack global path quality
guarantees \cite{vandenberg2011Reciprocal-Collision,
Tang2015Complete-Multi}.

%==================

The previously proposed \drrt\ approach \cite{SoloveySH16:ijrr} is a
scalable sampling-based approach, which is probabilistically complete.
It searches an implicit tensor product roadmap of roadmaps
constructued for each robot individually \cite{Svestka:1998ud}.  The
current work proposes \drrtstar and shows that it is an efficient
asymptotically optimal variant of the prior method.  Simulations show
the method practically generates high-quality paths while scaling to
complex, high-dimensional problems, where alternatives
fail. \footnote{Additional material is provided in the appendices of 
this paper as well as the accompanying video of the original MRS
submission.}


