\subsection{More Experimental Data}
\label{apx:Experiments}

This appendix presents additional experimental data omitted from 
Section~\ref{sec:experiments}.

%%% First
\subsubsection{2-Robot Benchmark}

\begin{figure}[ht]
    \centering
    \includegraphics[width=0.43\textwidth]{two_robot_cost_over_time}
    \caption{2-Robot convergence data over time.}
    \label{fig:2r_time}
\end{figure}

For the two-robot benchmark, additional data is presented in 
Figure~\ref{fig:2r_time}.  Here, the data presented in 
Figure~\ref{fig:polygonal_benchmark} is shown again over time instead 
of over iterations.


\begin{figure}[ht]
    \centering
    %\includegraphics[width=0.2\textwidth]{polygonal_multi_robot_roadmap}
    \includegraphics[width=0.43\textwidth]{random_scalability_timings}
    \caption{$R$-robot solution times for varying $R$.}
    \label{fig:rr_time}
\end{figure}


%%% Second
\subsubsection{R-Robot Benchmark}

For the $R$-robot benchmark, additional data is presented in 
Figure~\ref{fig:rr_time}, showing query resolution times for the 
various methods.

\begin{figure}[ht]
    \centering
    \includegraphics[width=0.43\textwidth]{graphception}
    \includegraphics[width=0.43\textwidth]{multi_cost_over_time}
    \caption{(\textit{Top}): Convergence rate and success ratio over 
    the minimal $9$-node roadmap (\textit{Bottom}): Solution cost over
    time when using the minimal roadmap.}
    \label{fig:synth_results}
\end{figure}

To emphasize the lack of scalability for alternate methods, additional
experiments were run in this setup using a minimal roadmap.  The tests
use a $9$-node roadmap for each robot as illustrated in 
Figure~\ref{fig:nine_grid}.  Each roadmap is constructed with slight
perturbations to the nodes within the shaded regions indicated in the
figure.

\begin{wrapfigure}{r}{0.168\textwidth}
  \centering
  \includegraphics[width=0.166\textwidth]{polygonal_multi_robot_roadmap}
  \caption{Minimal graph for the $R$-robot case.}
  \label{fig:nine_grid}
\end{wrapfigure}

The data for this modified benchmark (shown in 
Figure~\ref{fig:synth_results}) indicate that even using a very small
roadmap does not allow alternate methods to scale.  While the methods
scale better, Implict \astar\ does time out for $R \geq 5$, and 
\rrtstar\ times out for $R \geq 6$.

% Using the same environment as
% previous benchmark, a $9$-node roadmap is constructed for each disk robot, 
% as seen in Figure~\ref{fig:scalability}(\textit{bottom}). In this 
% test, the number of robots $R$ is increased to evaluate the 
% scalability of \drrtstar, and this emphasizes the intractability of 
% the alternative methods employed in the first benchmark. Each robot 
% starts in its corresponding region as shown in 
% Figure~\ref{fig:scalability} (\textit{bottom}) and must reach the 
% opposite region (i.e. robot 1's goal is region 2). As the number of 
% robots increases it is no longer tractable to explicitly build 
% roadmaps in the composite space.

% Figure~\ref{fig:scalability} outlines the data for this benchmark,
% and a few key points are observed.  For $R \geq 6$, the number of
% verticies in $\mmgimp$ increases to $>530,000$.  This means the
% explicit composite \prmstar\ roadmap is infeasible to construct, and
% as the data indicate, the implicit \astar\ search no longer returns
% a solution within a reasonable time frame, beyond $ R=4 $. \rrtstar\ 
% is not constrained to the roadmap, and discovers better solutions for 
% $ R=3 $, but it fails to return solutions within $ 100,000 $ 
% iterations, for $ R \geq 5 $. \drrtstar\ however is able to quickly 
% produce an initial solution for an increasing number of robots and can 
% solve the highly constrained case of $R = 8$ robots for this 
% environment.  Furthermore, the data indicate that \drrtstar\ also 
% eventually converges to near the optimal solution, though for the sake
% of this experiment, an optimistic estimate of the optimal solution 
% cost employed, which is simply the sum of the optimal solutions for 
% each robot, disregarding robot-robot collisions.

%%% Third
\subsubsection{Manipulator Benchmark}

For the dual-arm manipulator benchmark, additional data is presented 
in Figure~\ref{fig:moto_iter}.  Here, the data of 
Figure~\ref{fig:motoman_convergence} is shown over iterations 
instead of over time.

\begin{figure}[h]
    \centering
    \includegraphics[width=0.43\textwidth]{motoman_convergence}
    \caption{Motoman benchmark solution quality over iterations.}
    \label{fig:moto_iter}
\end{figure}

