\label{sec:analysis}

In this section, the theoretical properties of \drrtstar\ are 
examined, beginning with a study of the asymptotic convergence of the 
implicit roadmap $\mmgimp$ to containing a path in $\cfree$ whose cost 
converges to the optimum.  Then, it is shown \drrtstar\ eventually 
discovers the shortest path in $\mmgimp$, and that the combination of 
these two facts proves the asymptotic optimality of \drrtstar.

For simplicity, the analysis is restricted to the setting of robots
operating in Euclidean space, i.e. $\cspace_i$ is a $d$-dimensional
Euclidean hypercube $[0,1]^d$ for fixed $d \geq 2$. \footnote{For 
simplicity, it is assumed that all the robots have the same number  
of degrees of freedom $d$.}  Additionally, the analysis is restricted 
to the specific cost function of \emph{total distance}, i.e., 
$|\Sigma|:=\sum_{i=1}^R|\sigma_i|$. Discussion on lifting these 
restrictions is provided in Section \ref{sec:discuss}.

\subsection{Optimal Convergence of $\mmgimp$}

For each robot, an asymptotically optimal \prmstar\ roadmap 
$\mmgraph_i$ is constructed having $n$ samples and using a connection
radius $r(n)$ necessary for asymptotic convergence to the optimum
\cite{Karaman2011Sampling-based-}.  By the nature of sampling-based
algorithms, each graph cannot converge to the true optimum with finite
computation, as such a solution may have clearance of exactly $0$.  
Instead, this work focuses on the notion of a robust optimum
\footnote{Note that the given definition of robust optimum is similar 
    to that in previous work~\cite{Solovey_2016_rss}.}, showing that 
the tensor product roamdap $\mmgimp$ converges to this value.

\begin{definition}
  A trajectory $\Sigma:[0,1]\rightarrow \cfree$ is \emph{robust} if
  there exists a fixed $\delta>0$ such that for every
  $\tau\in [0,1],X\in \cinv$ it holds that
  $\|\Sigma(\tau)-X\|_2\geq \delta$, where $\|\cdot\|_2$ denotes the
  standard Euclidean distance.
\end{definition}

\begin{definition}
  A value $c > 0$ which denotes a path cost is robust if for every
  fixed $\epsilon > 0$ there exists a robust path $\Sigma$ such that
  $|\Sigma| \leq (1+\epsilon)c$. The \emph{robust optimum},
  denoted by $c^*$, is the infimum over all such values.
\end{definition}

For any fixed $n\in \mathbb{N}^+$, and a specific instance of
$\mmgimp$ constructed from $R$ roadmaps, having $n$ samples each,
denote by $\Sigma^{(n)}$ the shortest path from $S$ to $T$ 
over~$\mmgimp$.

\begin{definition}
  $\mmgimp$ is asymptotically optimal (AO) if for every fixed
  $\epsilon > 0$ it holds that
  $|\Sigma^{(n)}| \leq (1+\epsilon)c^*$ a.a.s.\footnote{Let
      $A_1,A_2,\ldots$ be random variables in some probability space
      and let $B$ be an event depending on~$A_n$. We say that $B$
      occurs \emph{asymptotically almost surely} (a.a.s.) if
      $\lim_{n\rightarrow \infty}\Pr[B(A_n)]=1$.}, where the
    probability is over all the instantiations of $\mmgimp$ with $n$
    samples for each PRM.
\end{definition}

Using this definition, the following theorem is proven.  Recall that 
$d$ denotes the dimension of a single-robot configuration space. 
 
\begin{theorem}
$\mmgimp$ is AO when $$r(n)\geq \radstar=(1+\eta)2 \left(\frac{1}{d} 
\right)^{\frac{1}{d}} \left( \frac{\log n}{n} \right)^{\frac{1}{d}},$$
where $\eta$ is any constant larger than $0$.
\label{thm:opt_graph}
\end{theorem}

{\bf Remark.} Note that \radstar\ was developed in
\cite[Theorem~4.1]{Pavone:2015fmt}, and guarantees AO of \prmstar for 
a single robot. The proof technique described in that work will be one
of the ingredients used to prove Theorem~\ref{thm:opt_graph}. 
\footnote{Note that \radstar\ can be refined to incorporate the 
proportion of $\cfree_i$, which would reduce this expression.} 

By definition of $c^*$, for any given $\epsilon>0$ there exists
robust trajectory $\Sigma:[0,1]\rightarrow \cfree$, and fixed
$\delta>0$, such that the cost of $\Sigma$ is at most 
$(1+1/2\cdot \epsilon)c^*$ and for every
$X\in \cinv, \tau\in [0,1]$ it holds that
$\|\Sigma(\tau)-X\|\geq \delta$.  Next, it is shown that $\mmgimp$
contains a trajectory $\Sigma^{(n)}$ such that
\begin{equation}
|\Sigma^{(n)}|\leq (1+o(1))\cdot |\Sigma|, \label{eq:eps_approx}
\end{equation}
a.a.s.. This immediately implies that
$|\Sigma^{(n)}| \leq (1+\epsilon)c^*$, which will finish the proof of
Theorem~\ref{thm:opt_graph}.

Thus, it remains to show that there exists a trajectory on $\mmgimp$ 
which satisfies Equation~\ref{eq:eps_approx} a.a.s..  As a first
step, it will be shown that the robustness of
$\Sigma = (\sigma_1,\ldots,\sigma_R)$ in the composite space implies
robustness in the single-robot setting, i.e., robustness along
$\sigma_i$.

For $\tau \in [0,1]$ define the forbidden space parameterized by
$\tau$ as \vspace{-0.1in} $$\cinv_i(\tau) = \cinv_i \cup 
\bigcup_{j=1, j \neq i}^R I_i^j( \sigma_j (\tau) ).$$ 

\begin{claim}
For every robot $i$, $\tau \in [0,1]$, and $q_i \in \cinv_i(\tau)$, 
$\| \sigma_i(\tau) - q_i \|_2 \geq \delta$.
\label{claim:robust}
\end{claim}
%
\begin{proof}
  Fix a robot $i$, and fix some $\tau \in [0,1]$ and a 
  configuration $q_i \in \cinv_i(\tau)$.  Next, define the
  following composite configuration
$$Q = (\sigma^1(\tau), \dots, q_i, \dots , \sigma^R(\tau)).$$
Note that it differs from $\Sigma(\tau)$ only in the $i$-th robot's
configuration. By the robustness of $\Sigma$ it follows that
%
\begin{align*}
\delta & \leq \| \Sigma(\tau) - Q \|_2\\ 
       & = \left( \| \sigma_i(\tau) - q_i \|_2^2 + \sum_{j=1,j \neq i}^R \| \sigma_j(\tau) -
           \sigma_j(\tau) \|_2^2 \right)^{\frac{1}{2}} \\
       &\leq \| \sigma_i(\tau) - q_i \|_2.
\end{align*}
\end{proof}

The result of claim \ref{claim:robust} is that the paths
$\sigma_1, \dots, \sigma_R$ are robust in the sense that there is
sufficient clearance for the individual robots to not collide with
each other given a fixed location of a single robot.  A Lemma is 
derived using proof techniques from the 
literature~\cite{Pavone:2015fmt}, and it implies every $\graph_i$ 
contains a single-robot path $\sigma_i^{(n)}$ that converges to 
$\sigma_i$
%
\begin{lemma}
For every robot $i$, $\graph_i$ constructed with $n$ samples and a
connection radius $r(n)\geq \radstar$ contains a 
path $\sigma_i^{(n)}$ with the following attributes a.a.s.: 
\begin{itemize}
\item[(i)] $\sigma_i^{(n)}(0) = s_i$, $\sigma_i^{(n)}(1) = t_i$; 
\item[(ii)] $|\sigma_i^{(n)}| \leq (1 + o(1)) |\sigma_i|$; 
\item[(iii)] $\forall q \in \textup{Im}(\sigma_i^{(n)})$, $\exists 
\tau \in [0,1]$ s.t. $\|q-\sigma_i(\tau)\|_2 \leq \radstar$.
\end{itemize}
\label{lem:prm}
\end{lemma}
\begin{proof}
  The first property (i) follows from the fact that $s_i,t_i$
  are directly added to $\graph_i$. The rest follows from the proof of
  Theorem~4.1 in~\cite{Pavone:2015fmt}, which is applicable here since
  $r(n)\geq \radstar$.
\end{proof}

Lemma~\ref{lem:prm} also implies that $\mmgimp$ contains a path in
$\cspace$, that represents robot-to-obstacle collision-free motions,
and minimizes the multi-robot metric cost. In particular, define
$\Sigma^{(n)}=(\sigma_1^{(n)},\ldots, \sigma_R^{(n)})$, where
$\sigma_i^{(n)}$ are obtained from Lemma~\ref{lem:prm}. Then
$$|\Sigma^{(n)}|=\sum_{i=1}^R|\sigma_i^{(n)}|\leq  (1 +
o(1))\sum_{i=1}^R|\sigma_i|\leq (1+o(1))|\Sigma|.$$
However, it is not clear whether this ensures the existence of a path
where robot-robot collisions are avoided.  That is, although
$\textup{Im}(\sigma^{(n)}_i)\subset \cfree_i$, it might be the case
that $\textup{Im}(\Sigma^{(n)})\cap \cinv \neq \emptyset$.  Next it is
shown that $\sigma_1^{(n)},\ldots, \sigma_R^{(n)}$ can be
reparametrized to induce a composite-space path whose image is fully
contained in $\cfree$, with length equivalent to $\Sigma^{(n)}$.

For each robot $i$, denote by $V_i=(v_i^1,\ldots,v_i^{\ell_i})$ the
chain of $\graph_i$ vertices traversed by $\sigma^{(n)}_i$. For every
$v_i^j\in V_i$ assign a timestamp $\tau_i^j$ of the closest
configuration along $\sigma^i$, i.e., \vspace{-0.1in}
$$\tau_i^j=\argmin_{\tau\in [0,1]}\|v_i^j-\sigma_i(\tau)\|_2.$$
Also, define $\T_i=(\tau_i^1,\ldots,\tau_i^{\ell_i})$ and denote by
$\T$ the ordered list of $\bigcup_{i=1}^R\T_i$, according to the
timestamp values. Now, for every $i$, define a global timestamp
function $T\!S_i:\T \rightarrow V_i$ which assigns to each global 
timestamp in $\T$ a single-robot configuration from $V_i$.  It thus
specifies in which vertex robot $i$ resides at time $\tau \in \T$.
For $\tau\in \T$, let $j$ be the largest index such that
$\tau_i^j \leq \tau$. Then simply assign $T\!S_i(\tau)= \tau_i^j$. 
From property (iii) in Lemma~\ref{lem:prm} and 
Claim~\ref{claim:robust} it follows that no robot-robot collisions are 
induced by the reparametrization, concluding the proof of
Theorem~\ref{thm:opt_graph}.

% For each robot $i$ denote by $V_i=(v_i^1,\ldots,v_i^{\ell_i})$ the
% chain of $\graph_i$ vertices traversed by $\sigma^{(n)}_i$.
% Hypothetically, one could assign to each $v_i^j\in V_i$ a timestamp
% $\tau_i^j$ of the closest configuration along $\sigma^i$, i.e.,
% $$\tau_i^j=\argmin_{\tau\in [0,1]}\|v_i^j-\sigma_i(\tau)\|_2.$$
% This would ensure that there is a joint reparametrization of
% $\sigma_1^{(n)},\ldots, \sigma_R^{(n)}$ in which the robots remain
% collision free, due to property (iii) in Lemma~\ref{lem:prm} and
% Claim~\ref{claim:robust}. Unfortunately, this is insufficient as
% $\mmgimp$ does not permit the robots to proceed independently along
% the edges of their individual PRMs. That is, given two adjacent
% vertices $v=(v_1,\ldots,v_R),v'=(v'_1,\ldots,v'_R)$ of $\mmgimp$ the
% robots must move along the edge $(v,v')$ in a gradual and timed
% manner, so that they cover an $\alpha$-fraction of their edge, for any
% $\alpha\in [0,1]$ at the same time.

% This problem can be resolved by assigning timestamps that are shared 
% among the robots. Define $\T_i=(\tau_i^1,\ldots,\tau_i^{\ell_i})$ and 
% denote by $\T$ the ordered list of $\bigcup_{i=1}^R\T_i$, according to 
% the timestamp values. Now, for every $i$, define a timestamp function
% $T\!S_i:\T \rightarrow V_i$ which assigns to each timestamp in $\T$ a
% single-robot configuration from $V_i$. In other words, it specifies in
% which vertex robot $i$ must reside in time $\tau \in \T$.  For a given
% $\tau\in \T$ let $j$ be the largest index such that $\tau_i^j \leq 
% \tau$. Then simply assign $T\!S_i(\tau)= \tau_i^j$.  Here again the 
% proof relies on property (iii) in Lemma~\ref{lem:prm} and 
% Claim~\ref{claim:robust} to show that no robot-robot collisions are
% induced by the reparametrization.  This concludes the proof of
% Theorem~\ref{thm:opt_graph}. \chups{Is the reparameterization still
% an issue?  We have modified the implicit graph to include edges where
% one or more robots can stay still.}

\subsection{Asymptotic Optimality of $\drrtstar$}

Finally, \drrtstar\ is shown to be AO.  Denote by $m$ the time budget 
in Algorithm~\ref{algo:drrtstar},  i.e., the number of iterations of 
the loop. Denote by $\Sigma^{(n,m)}$ the solution returned by 
$\drrtstar$ for $n$ and $m$.

\begin{theorem}
  If $r(n)>\radstar$ then for every fixed $\epsilon>0$ it holds that
  $$\lim_{n,m\rightarrow \infty}\Pr\left[|\Sigma^{(n,m)}|\leq
    (1+\epsilon)c^*\right]=1.$$
  \label{thm:ao_drrt}
\end{theorem}

Since $\mmgimp$ is AO (Theorem~\ref{thm:opt_graph}), it suffices to
show that for any fixed $n$, and a fixed instance of $\mmgimp$,
defined over $R$ PRMs with $n$ samples each, $\drrtstar$ eventually 
(as $m$ tends to infinity), finds the optimal trajectory over 
$\mmgimp$.  This can be shown using the properties of a Markov chain 
with absorbing states \cite[Theorem~11.3]{Snell2012:intro_prob}.
While a full proof is omitted here, the high-level idea is similar to
what is presented in previous work \cite[Theorem~3]{SoloveySH16:ijrr},
and expanded upon in Appendix~A.
By restricting the states of the Markov chain to being the graph
vertices along the optimal path, setting the target vertex to be an
absorbing vertex, and showing that the probability of transitioning
along any edge in this path is nonzero (i.e. the probability is
proportional to $\frac{\mu(\textup{Vol}(V_{k}))}{\mu(\cfree)} > 0$), 
then the probability that this process does not reach the target state 
along the optimal path converges to $0$ as the number of \drrtstar\
iterations tends to infinity.  The final step is to show that the
above statements hold when both $m$ and $n$ tend to $\infty$. A proof
for this phenomenon can be found
in~\cite[Theorem~6]{SoloveySH16:ijrr}.


%%% Local Variables:
%%% mode: latex
%%% TeX-master: "../isrr"
%%% End:
