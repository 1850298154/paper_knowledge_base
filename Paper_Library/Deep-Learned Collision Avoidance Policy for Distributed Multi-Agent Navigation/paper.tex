%%%%%%%%%%%%%%%%%%%%%%%%%%%%%%%%%%%%%%%%%%%%%%%%%%%%%%%%%%%%%%%%%%%%%%%%%%%%%%%%
%2345678901234567890123456789012345678901234567890123456789012345678901234567890
%        1         2         3         4         5         6         7         8

\documentclass[letterpaper, 10 pt, journal, twoside]{IEEEtran}  
% Comment this line out if you need a4paper

%\documentclass[a4paper, 10pt, conference]{ieeeconf}      
% Use this line for a4 paper

%\IEEEoverridecommandlockouts                              
% This command is only needed if you want to use the \thanks command

%\overrideIEEEmargins                                      
% Needed to meet printer requirements.

% See the \addtolength command later in the file to balance the column lengths
% on the last page of the document

% Paper headers 
\markboth{IEEE Robotics and Automation Letters. Preprint Version. Accepted December, 2016}
{Long \MakeLowercase{\textit{et al.}}: Deep-Learned Collision Avoidance Policy for Distributed Multi-Agent Navigation}  
% Use only for final RAL version

\usepackage{color}
\usepackage{graphicx}
\usepackage{amsmath}
\usepackage{amssymb}
\usepackage{tabularx}
\usepackage{subcaption}
\captionsetup{font=small}
\captionsetup[sub]{font=small}
\usepackage{url}
\usepackage{cite}

\DeclareMathOperator*{\argmin}{\arg\!\min}
\DeclareMathOperator{\sigmoid}{sigmoid}


\title{Deep-Learned Collision Avoidance Policy for Distributed Multi-Agent Navigation} %Use for final RAL version

% Make room for more info lines in the \author command  
\author{Pinxin Long, Wenxi Liu*, and Jia Pan*%
\thanks{Manuscript received: September, 10, 2016; Revised December, 6, 2016; Accepted December, 29, 2016. This paper was recommended for publication by Editor Nancy Amato upon evaluation of the Associate Editor and Reviewers' comments.
This work was supported by HKSAR
Research  Grants  Council  (RGC)  General  Research  Fund
(GRF), CityU 17204115, 21203216, and NSFC/RGC Joint Research Scheme CityU103/16 . Asterisk
indicates the corresponding author.}
\thanks{Pinxin Long is with Dorabot Inc., Shenzhen, China. This work was done while the author was at the City University of Hong Kong (e-mail: pinxinlong@gmail.com)}%
\thanks{*Wenxi Liu is with the Department of Computer Science, Fuzhou University, Fuzhou, China (e-mail: wenxi.liu@hotmail.com)}%
\thanks{*Jia Pan is with the Department of Mechanical and Biomedical Engineering, City University of Hong Kong, Hong Kong, China (e-mail: jiapan@cityu.edu.hk)}%
\thanks{Digital Object Identifier (DOI): see top of this page.}
}
%Use only for final RAL version. 


% \author{Pinxin Long \and Wenxi Liu \and Jia Pan % <-this % stops a space
% %\thanks{*This work was not supported by any organization}% <-this % stops a space
% \thanks{The authors are with the Department of Mechanical and Biomedical Engineering, the City University of Hong Kong.
%         %{\tt\small b.d.researcher@ieee.org}
%         }%
% }

\begin{document}
\maketitle
%\thispagestyle{empty}
%\pagestyle{empty}

%%%%%%%%%%%%%%%%%%%%%%%%%%%%%%%%%%%%%%%%%%%%%%%%%%%%%%%%%%%%%%%%%%%%%%%%%%%%%%%%
\begin{abstract}
High-speed, low-latency obstacle avoidance that is insensitive to sensor noise is essential for enabling multiple decentralized robots to function reliably in cluttered and dynamic environments. While other distributed multi-agent collision avoidance systems exist, these systems require online geometric optimization where tedious parameter tuning and perfect sensing are necessary. 

We present a novel end-to-end framework to generate reactive collision avoidance policy for efficient distributed multi-agent navigation. Our method formulates an agent's navigation strategy as a deep neural network mapping from the observed noisy sensor measurements to the agent's steering commands in terms of movement velocity. We train the network on a large number of frames of collision avoidance data collected by repeatedly running a multi-agent simulator with different parameter settings. We validate the learned deep neural network policy in a set of simulated and real scenarios with noisy measurements and demonstrate that our method is able to generate a robust navigation strategy that is insensitive to imperfect sensing and works reliably in all situations. We also show that our method can be well generalized to scenarios that do not appear in our training data, including scenes with static obstacles and agents with different sizes. Videos are available at \url{https://sites.google.com/view/deepmaca}.

\end{abstract}

% Keywords appear just beneath the abstract. Use only for final RAL version.  
\begin{IEEEkeywords}
Collision Avoidance; Distributed Robot Systems; Deep Learning; Multi-Agent Navigation %Path Planning for Multiple Mobile Robots or Agents; Autonomous Vehicle Navigation; Learning and Adaptive Systems

\end{IEEEkeywords}

\section{Introduction}

One of the most fundamental problems in combinatorial optimization is the traveling salesperson problem (TSP), formalized as early as 1832 (c.f. \cite[Ch 1]{ABCC07}).
In an instance of  TSP we are given a set of $n$ cities $V$ along with their pairwise symmetric distances, $c:V\times V \to\R_{\geq 0}$. The goal is to find a Hamiltonian cycle of minimum cost. In the metric TSP problem, which we study here, the distances satisfy the triangle inequality. Therefore, the problem is equivalent to finding a closed Eulerian connected walk of minimum cost.%\footnote{Given such an Eulerian cycle, we can use the triangle inequality to shortcut vertices visited more than once to get a Hamiltonian cycle.}

It is NP-hard to approximate TSP within a factor of $\frac{123}{122}$ \cite{KLS15}.  An algorithm of Christofides-Serdyukov~\cite{Chr76,Ser78} from four decades ago gives a $\frac32$-approximation for TSP.
Over the years there have been numerous attempts to improve the Christofides-Serdyukov algorithm and exciting progress has been made for various special cases of metric TSP, e.g., \cite{OSS11,MS11,Muc12,SV12,HNR21, KKO20, HN19, GLLM21}.
 Recently, ~\cite{KKO21} gave the first improvement for the general case by demonstrating that the so-called ``max entropy" algorithm of \cite{OSS11} gives a randomized $\frac{3}{2}-\epsilon$ approximation for some $\epsilon > 10^{-36}$.% (see \cite{VS20} for a historical note about TSP)

%After a long line of work %~\cite{Wol80,SW90,BP91,Goe95,CV00,GLS05,BM10,BC11,SWV12, HNR17,HN19, KKO20a} 
	%the best known approximation algorithm for the general case of the problem is $\frac{3}{2}-\epsilon$ for some $\epsilon > 10^{-36}$ due to ~\cite{KKO21}, a result that built upon the work of the third author, Saberi, and Singh ~\cite{OSS11}. 
	The method introduced in \cite{KKO21} exploits the optimum solution to the following linear programming relaxation of metric TSP studied by \cite{DFJ59,HK70,BG93}, also known as the subtour elimination LP:
\begin{equation}\label{eq:tsplp}
\begin{aligned}
	\min \quad& \sum_{u,v} x_{\{u,v\}} c(u,v)& \\
	\text{s.t.,} \quad &  \sum_{u} x_{\{u,v\}} = 2&\forall v\in V,\\
	& \sum_{u\in S, v\notin S} x_{\{u,v\}}\geq 2,&\forall S \subsetneq V, S\not= \emptyset\\
	& x_{\{u,v\}}\geq 0 &\forall u,v\in V.
\end{aligned}	
\end{equation} 
	
	 However, ~\cite{KKO21} did not show that the integrality gap of the subtour elimination polytope is bounded below $\frac{3}{2}$, and therefore did not make progress towards the ``4/3 conjecture" which posits that the integrality gap of LP \eqref{eq:tsplp} is $\frac{4}{3}$. In this work we remedy this discrepancy by proving the following theorem, improving upon the bound of $\frac{3}{2}$ from Wolsey~\cite{Wol80} in 1980:

\begin{theorem}\label{thm:main}
	Let $x$ be a solution to LP \eqref{eq:tsplp} for a TSP instance. For some absolute constant $\epsilon > 10^{-36}$, the \hyperlink{tar:alg}{max entropy algorithm} outputs a TSP tour with expected cost at most $\frac{3}{2}-\epsilon$ times the cost of $x$. Therefore the integrality gap of the subtour elimination LP is at most $\frac{3}{2} - \epsilon$. 
\end{theorem} 

To prove \cref{thm:main}, we amend Section 4 of \cite{KKO21} but keep the remainder of the analysis essentially the same. Unlike \cite{KKO21}, this argument now preserves the integrality gap by avoiding the use of the optimum solution in bounding the cost of the matching. See \cref{sec:overview} for a discussion of our new approach.
%We note that the analysis in this paper is not specialized to the max entropy algorithm (although we rely on many results from \cite{KKO21} to obtain \cref{thm:main} itself); instead, it is valid for any algorithm which samples a spanning tree from the support of a solution to LP \eqref{eq:tsplp} and then adds the minimum cost matching on the odd degree vertices of the tree.  
%Instead, we use the polygon representation of near minimum cuts \cite{Ben95,BG08} to bound  the cost of the matching (see the following section for an overview of our new findings). %An added benefit of avoiding the use of OPT in the analysis is  %We remark this makes the analysis constructive 
%We remark that this allows future analyses to explicitly compute and possibly utilize the relevant laminar family of near minimum cuts (whereas previously one needed to know OPT to find the laminar family used in the analysis in \cite{KKO21}).
%In particular, we show that to get a bound better than $\frac{3}{2}$ for this class of algorithm it is (essentially) sufficient to handle the case in which the near minimum cuts of $x$ are a laminar family.

\subsection{Other Consequences}
\paragraph{Path TSP} In recent exciting work, Traub, Vygen, Zenklusen \cite{TVZ20} showed that an $\alpha$-approximation algorithm for metric TSP can be used as a black box to get a $\alpha(1+\eps)$ approximation algorithm for Path TSP. This together with \cite{KKO21} implies that there is a $3/2-\eps$ approximation algorithm for Path TSP (for $\eps>10^{-36}$). On the other hand, it is a folklore result that the integrality gap of the natural LP relaxation of Path TSP is at least $3/2$.  Therefore, a consequence of the above theorem is that although the best possible approximation factors of the two problem are the same (up to polynomial reductions), the natural LP relaxation of metric TSP has a strictly smaller integrality gap.


\paragraph{2-ECSM} In the 2-edge-connected multi-subgraph problem, or 2-ECSM for short, we are given a weighted graph $G$ and we want to find a minimum cost 2-edge-connected spanning subgraph, where an edge can be chosen multiple times.
The classical Christofides-Serdyukov algorithm gives a 3/2-approximation for 2-ECSM and despite significant attempts \cite{CR98,BFS16,SV14,BCCGISW20} improved algorithms were designed only for special cases of the problem.
Since in \cite{BG93} it is shown that LP \eqref{eq:tsplp} is a valid relaxation for 2-ECSM, we obtain:

\begin{corollary}	
For some absolute constant $\epsilon > 10^{-36}$ the \hyperlink{tar:alg}{max entropy algorithm} is a randomized $\frac{3}{2}-\epsilon$ approximation for the 2-edge-connected multi-subgraph problem.
\end{corollary}
%Beyond these theorems, we believe the analysis in this paper will open new avenues to improve the arguments in ~\cite{KKO21}. The analysis in that work is by nature non-constructive because it uses information about the optimal solution. Here we remove this weakness and could in principle construct the proposed fractional matching in polynomial time. Although of course this has no practical benefit since the algorithm always finds the minimum cost matching, this may allow future works to manipulate the algorithm to better serve the analysis.

%We analyze the max-entropy rounding algorithm introduced in \cite{OSS11} and slightly modified in \cite{KKO20, KKO21}. 

%In other words, we design a feasible vector for the $O$-join polytope to ``satisfy'' all near min cuts ``crossed on both  sides'' 


%Whereas Section 4 of ~\cite{KKO21} only deals with the near minimum cuts of $x$ (where $x$ is a solution to LP \eqref{eq:tsplp}) which lie along the optimal Hamiltonian cycle, we deal with all near minimum cuts of $x$ using the so-called polygon representation of near minimum cuts ~\cite{Ben97,BG08}. %The results give new intuition for the structure of cuts that are within $\frac{6}{5}$ or less of the edge connectivity of the graph.

 %: we show that we can incur a cost of $O(\eta^2) \cdot c(x)$ to ensure that the set of cuts with $x(\delta(S)) \le 2+\eta$ is a laminar family.


\subsection{New techniques and contributions}\label{sub:newtechniques}

This paper can be seen as a case study on how to reason about and deal with {\em near} minimum cuts. One can deduce from the classical cactus representation of a graph $G$ \cite{DKL76} (i) the structure of {\em all} min cuts of $G$ and (ii) the structure of the edges of $G$ in the sense that every edge $\{u,v\}$ maps to a unique {\em path} in the cactus between the images of $u$ and $v$. Furthermore, such a path intersects every cycle of the cactus on at most one cactus edge. The theory has found many application from designing fast algorithms
\cite{Kar00,KP09} to the analysis of approximation algorithms for TSP \cite{KKO20} and connectivity augmentation \cite{BGJ20,CTZ21}.

Two decades later, the theory of min cuts was extended to near min cuts in works of Bencz\'ur and Goemans \cite{Ben95, BG08} where they introduced the polygon representation which represents all cuts of a graph with at most $\frac{6}{5}k$ edges, where $k$ is its edge connectivity. Although these works completely classify the structure of all near min cuts of a given graph $G$, they do not characterize the structure of the \textit{edges} of $G$ with respect to these cuts, which can be important in applications (for example, in many of the recent applications of min cuts,
 one also needs to exploit the structure of the edges in relation to the cactus).
The structure on the edges turns out to be highly relevant in this work as well, and as a byproduct of our analysis we make progress towards classifying the way in which the edges of $G$ relate to the structure of the polygon representation.
 
 % and (to some extent) a classification of the set of edges of $G$ with respect to the polygon representation of Bencz\'ur and Goemans.
 
  %i
 %s to give a better understanding of the structure of edges of $G$ with respect to its near min cuts.

  %One can partition the edges of $G$ into sets $F_1\dots,F_m$ such that the set of edges in every min cut $(S,\overline{S})$ of $G$ is the union of edges in a pair $F_i,F_j$ for $i\ neq j$.
%\Nathan{Shayan can add something} For example...

For motivation, consider a generic family of network design problems in which we want to construct a network such that every pair $u,v$ of vertices has connectivity at least $c_{u,v}$. A natural approach is to write an LP relaxation to find a (minimum cost) vector $x: E \to \R_{\ge 0}$ such that for every cut $S$ separating $u$ and $v$, $x(\delta(S))\geq c_{u,v}$. We can round this LP using independent rounding or a dependent rounding scheme such as sampling from max entropy distributions. Using classical concentration bounds one can show that if $x(\delta(S))\gg c_{u,v}$ then with high probability the rounded solution has at least $c_{u,v}$ edges across this cut. So the main challenge is to ``fix'' near tight cuts, i.e., cuts where $x(\delta(S))\approx c_{u,v}$.  For an explicit instantiation of this scheme see \cite{KKOZ22}. A better understanding of the global structure of the family of near tight cuts has the potential to significantly simplify or even improve the approximation factor of such rounding algorithms. A classical technique to design algorithms for such network design problems is to apply uncrossing to extreme point solutions of the LP. One can view our contribution as an approximate uncrossing technique that deals with all near tight cuts (instead of just tight cuts) as we explain next.
%Next, we explain how our main theorem can be used to give global structure for near tight cuts in the case that $c_{u,v}=2$ for all $u,v$ and we contrast it with the classical uncrossing technique which only deals with tight/min cuts. 


\paragraph{An Approximate Uncrossing Technique.} A fundamental technique in the field of approximation algorithms is the uncrossing technique\footnote{See e.g. \cite{LRS11} for a number of applications of this technique.} of Jain \cite{Jai01}. Given a graph $G=(V,E)$,  a weight vector $x:E\to\R_{\geq 0}$, and a  function $f:V\to\R$, suppose that $x(\delta(S))\geq f(S)$ for all $S\subseteq V$. Let $\cN$ be the family of sets $S$ such that $x(\delta(S)) = f(S)$, i.e., the family of {\em tight} sets with respect to $f$. The uncrossing technique says that if $f$ is (weakly) supermodular then we can refine $\cN$ to a laminar family of sets, $\cH$, such that if all sets of $\cH$ are tight, then all sets of $\cN$ are tight as well. For a concrete example, suppose $f$ is a constant function, say $f(S)=2$ for all $\emptyset\subsetneq S\subsetneq V$. Then, sets of $\cH$ can be constructed using the cactus representation \cite{DKL76} of cuts in $\cN$. The significance of this method is that if $x$ is a basic feasible solution to a LP with constraints $x(\delta(S))\geq f(S)$ for all $S$, one can use this machinery to argue that the support of $x$ has size $O(|V|)$.

Informally, we prove the following, which 
can be seen as  an {\em approximate uncrossing technique}: 
\begin{theorem}[Informal]\label{thm:uncrossing}Suppose we have a vector $x:E\to\R_{\geq 0}$ such that $x(\delta(S))\geq f(S)$ for all $S$; define $\cN$ to be sets $S$ where $x(\delta(S))\leq f(S)(1+\eps)$ for some fixed $\eps>0$. If $f(.)$ is constant, say $f(S)=2$ for all $S$, then there is a set $\cN^*\subseteq \cN$ and a collection of edge sets $F_1,\dots,F_m\subseteq E$ such that the following hold:
\begin{itemize}
	\item $|\cN^*|= O(|V|)$, $m= O(|V|)$.
	\item $x(F_i)\geq 1-\eps/2$ for all $1\leq i\leq m$.
	\item Every edge $e$ is in at most $O(1)$ of the $F_i$'s.
	\item For every set $S\in \cN\smallsetminus \cN^*$ there exists $1\leq i<j\leq m$ such that $F_i\cap F_j=\emptyset$ and $F_i\cup F_j\subseteq \delta(S)$ and for every $S\in \cN^*$, there exists $1\leq i\leq m$ such that $F_i\subseteq \delta(S)$. 
\end{itemize}
\end{theorem}
In words, although we cannot simply refine $\cN$ to a linear number of sets, we can refine the edges in cuts of $\cN$ to a linear number of sets $F_1,\dots, F_m$ such  that we can essentially capture the edges of $\delta(S)$ for any $S\in \cN\smallsetminus \cN^*$ by a pair of disjoint $F_i$'s. We give a slightly weaker condition for cuts in $\cN^*$; namely we only capture half of their edges by $F_i$'s.

\begin{example}For a simple example of the above theorem, suppose $\eps=0$, i.e. $\cN$ is the set of min cuts of a graph $G$. Furthermore, suppose that every proper  cut in $\cN$ is \hyperlink{tar:crossing}{crossed} (recall that $S$ is proper if $1<|S|<|V|-1$) and that $\cN$ has at least one proper cut. 
Then, one can use an uncrossing technique, namely that if $A,B\in \cN$ then $A\cap B\in \cN$, to prove that $G$ must be cycle, namely we can order vertices of $G$, $v_0,\dots,v_{n-1}$ such that $x_{\{v_i,v_{i+1\text{ mod n}}\}}=1$.
In such a case we let $\cN^*=\emptyset$ and $F_i=E(v_i,v_{i+1\text{ mod }n})$.
%partition $V$ into sets $a_0,\dots,a_{m-1}$ such that 
%Let $\C$ be a connected component of crossing cuts of $\cN$, namely, for any pair of sets $A,B\in \C$ there is a path of crossing cuts all from $\C$ that goes from $A$ to $B$.
% and further suppose that $\cN$ can be represented by a cycle $C$ in the sense every min cut of $\cN$ corresponds to a min cut of $C$ and vice versa. Here we assume $a_0,\dots,a_{m-1}$ are the nodes of $C$ where each $a_i$ is identified with a disjoint set of vertices where $V=\uplus_{i=1}^m a_i$. In such a case, we can simply let $\cN^*=\emptyset$ and $F_i=E(a_i,a_{i+1\text{ mod }m})$. 
\label{eg:cycle}\end{example}

\begin{example}\label{eg:laminar}
For a second example, suppose again $\eps=0$ and $\cN$ is the set of mincuts of a graph $G$ where $\cN$ forms a laminar family (no two cuts cross). It turns out that we cannot decompose edges of cuts of $\cN$ into a linear sized collection of sets where every edge appears only a constant number of times. The main reason is that some edges may appear in an unbounded number of cuts. In this case we let $\cN^*=\cN$ and for every $A\in \cN$ (with immediate parent $B\in \cN$ in the laminar family) we add a set $F_A=\delta(A)\smallsetminus \delta(B)$  to our collection.  It is straightforward to show, using the structure of min cuts, that $x(F_A)\geq 1$; furthermore, since the size of a laminar family is linear in $V$, this gives a valid decomposition in the sense of above theorem.
\end{example}
Lastly, if $\eps=0$ and $\cN$ is the set of min cuts of an arbitrary graph, one can represent all min cuts of $\cN$ by a cactus \cite{DKL76} which can be seen as a tree of cycles. In such a case, one can use a construction similar to \cref{eg:cycle} for each cycle where instead of a vertex $v_i$ we have a set $a_i \subseteq V$ and one similar to \cref{eg:laminar} for the tree part of the cactus. For a concrete application of such a decomposition of min cuts see \cite{KKO20}.
%More generally, if $\cN$ corresponds to the set of min cuts of an arbitrary graph, the cuts of $\cN$ can be represented by a {\em cactus graph}. In such a case we add one $F_i$ for every edge of a cycle of the cactus. 


%and further for simplicity assume that there is a single connected component of crossing cuts in $\cN$, namely we can go from any $A$ to $B$ for $A,B\in\cN$ simply following crossing cuts of $\cN$. Then, one can represent cuts in $\cN$ by the set of min cuts of a cycle, namely we can contract vertices of $G$ 

%For a concrete application , suppose we need at least two edges in every set in $\cN^*$, say in a network optimization problem. Then, if we make sure that we have at least one edge in each $F_i$, all typical cuts constraints, $\cN\smallsetminus \cN^*$,  are satisfied, so we  reduce the problem to cuts in $\cN^*$. 


One of the main challenges in dealing with near min cuts relative to min cuts is that if $x(\delta(A)),x(\delta(B))\leq 2+\eps$ then $x(\delta(A\cap B))\leq 2+2\eps$. Therefore, if $\eps=0$, then min cuts are closed under intersection, set difference and union, but this is no longer true when $\eps>0$. So, to employ the classical uncrossing machinery one should be very careful to "uncross" only a constant number of times (independent of $\eps$) to make sure that every cut remains within $2+O(\eps)$. This is the main reason that the polygon representation of near min cuts (see below) is more sophisticated, e.g., we can no longer argue $x(E(a_i, a_{i+1}))\approx 1$, see \cref{fig:nearmincutbadexample}.

Although we don't study it here, we believe it may be worthwhile to find generalizations of \cref{thm:uncrossing} which hold for any (weakly) supermodular function.% That could be helpful in many questions based on the network optimization framework of Jain \cite{Jai01}.

\begin{remark} 
 We do not explicitly prove \cref{thm:uncrossing} in this extended abstract, as it is not used to prove \cref{thm:main}. However it can be deduced from arguments in \cref{sec:twoside} and \cref{app:oneside}. 
%In \cref{sec:overview} we discuss the main ideas of the proof of \cref{thm:uncrossing}. Here, let us explain the main challenge: In principal one might try to simply extend the above decomposition for the case $\eps=0$. The main challenge is that if $x(\delta(A)),x(\delta(B))\leq 2+\eps$ then $x(\delta(A\cap B))\leq 2+2\eps$. Therefore, if $\eps=0$, then min cuts are closed under intersection, set difference and union, but this is no longer true when $\eps>0$. So, to employ the classical uncrossing machinery one should be very careful to "uncross" only a constant number of times (independent of $\eps$) to make sure that every cut remains within $2+O(\eps)$. This is the main reason that the polygon representation of near min cuts (see below) is more sophisticated, e.g., we can no longer argue $x(E(a_i, a_{i+1}))\approx 1$, see \cref{fig:nearmincutbadexample}.
\end{remark}





\paragraph{Extensions to the Polygon Representation} To obtain our uncrossing framework we prove new properties of the polygon representation.
Given a graph $G=(V,E)$, let $k$ be the edge-connectivity of $G$, i.e. the number of edges in a minimum cut of $G$. For $\eps>0$, consider the set of $(1+\eps)$-near minimum cuts of $G$: cuts $(S,\overline{S})$ where $|E(S,\overline{S})| < (1+\eps)k$.
Bencz\'ur \cite{Ben95} and Bencz\'ur, Goemans \cite{BG08} proved that if $\eps \le 1/5$ then the near minimum cuts of $G$ admit a {\em polygon representation}. Namely, every connected component $\cC$ of \hyperlink{tar:crossing}{crossing} $(1+\eps)$ near min cuts can be represented by the diagonals of a convex polygon. In this polygon, the vertices of $G$ are partitioned into sets called \textit{atoms}, and every atom is mapped to a cell of this polygon defined by the diagonals and the boundary of the polygon itself (see \cref{sec:polyrep} for more details). 

The polygon representation can be seen as a generalization of the well-known cactus representation \cite{DKL76} of minimum cuts where a cycle of the cactus is replaced by a convex polygon. Unlike a cycle, some vertices/atoms map to the interior of the polygon, which are called ``inside'' atoms. The inside atoms at first look like a mystery and one can ask many questions about them such as how many can exist and what structures they can exhibit.



 Here, we explain two lemmas we proved which might find further applications beyond TSP in the future. 
%Our results give new intuition and understanding about the structure of polygon representations. These guide our analysis of the integrality gap of the subtour LP.
 %For example, one of our new observations is a 
 First, we give a necessary condition for a cell of a polygon to contain an inside atom:
\begin{lemma}[Informal, see \cref{thm:halfplanes}]
	Consider a polygon $P$ for a connected component $\C$ of a family of $1+\eps$ near min cuts for $\eps \le 1/5$ (where representing diagonals correspond to cuts in $\C$). Any cell of $P$ that has an inside atom must have at least $\Omega(1/\eps)$ many sides. 
\end{lemma}
This can be seen as a generalization of \cite[Lem 22]{BG08} to the case in which the cell is allowed to be adjacent to vertices of the polygon $P$.

Now, we explain our second extension: it follows from the cactus representation of minimum cuts that for a graph $G$ and a min cut $S$ one can partition the set of all min cuts that cross $S$ into two groups ${\cal A}=\{A_1,\dots,A_k\}$ and ${\cal B}=\{B_1,\dots,B_l\}$ for some $k,l\geq 0$ such that $S\cap A_1\subseteq S\cap A_2 \subseteq \dots S\cap A_k$ and, similarly, $S\cap B_1\subseteq \dots\subseteq S\cap B_l$. We prove a generalization of this fact for near min cuts:
\begin{lemma}[Informal, see \cref{lem:crosschain}]
Consider the set of $1+\eps$ near min cuts of a graph $G$ for $\eps\leq 1/10$; for any such near min cut $S$, one can partition the $1+\eps$ near min cuts crossing $S$ into two groups ${\cal A}=\{A_1,\dots,A_k\}$ and ${\cal B}=\{B_1,\dots,B_l\}$ such that $S\cap A_1 \subseteq S\cap A_2\subseteq \dots \subseteq S\cap A_k$ and similarly for cuts in ${\cal B}$.
\end{lemma}

\subsection{Outline of rest of paper} After reviewing preliminaries in \cref{sec:prelims}, we give a high-level overview of our proof technique in \cref{sec:overview}. The main new technical contributions of this paper are in \cref{sec:polyrep} and  \cref{sec:twoside}. The remaining content of the paper essentially follows from ~\cite{KKO21}. %Therefore, the reader may want to skip \cref{sec:proof-of-main}. 



\section{Other Related Work}
\label{sec:related}
There has been a renewed interest in integrated task and motion planning algorithms. Most research in this direction has been focused on deterministic environments~\citep{cambon09_asymov,plaku10_sampling,dornhege12_semantic,kaelbling11_hierarchical,garrett15_ffrob,dantam16_incremental}. \cite{kaelbling13_hpnPOMDP} consider  a partially observable formulation of the problem. Their approach utilizes regression modules on belief fluents to develop a regression-based solution algorithm. \cite{sucan12_tmp_mdp} use an explicit multigraph to represent the plan or policy for which motion planning refinements are desired.  \cite{hadfield15_modular} address problems where the high-level formulation is deterministic and the low-level is determinized using most likely observations. In contrast, our approach employs abstraction to bridge MDP solvers and motion planners to solve problems where the high-level model is stochastic. In addition, the transitions in our MDP formulation depend on properties of the refined motion planning trajectories (e.g., battery usage). 

Principles of abstraction in MDPs have been well studied~\citep{hostetler14_state,bai16_markovian,li06_abstractMDP,singh95_abstractRL}. However, these directions of work assume that the full, unabstracted MDP can be efficiently expressed as a discrete MDP. \cite{marecki06_cmdp} consider continuous time MDPs with finite sets of states and actions. In contrast, our focus is on MDPs with high-dimensional, uncountable state and action spaces. Recent work on deep reinforcement learning  (e.g., \citep{hausknecht16_iclr,mnih15_drl}) presents  approaches for using deep neural networks in conjunction  with reinforcement learning to solve MDPs with continuous state spaces. We believe that these approaches can be used in a complementary fashion with our proposed approach. They could be used to learn maneuvers spanning shorter-time horizons, while our approach could be used to efficiently abstract their representations and to use them as actions or macros in longer-horizon tasks. 

Efforts towards improved representation languages are orthogonal to our contributions~\citep{fox02_pddl+}. The fundamental computational complexity results indicating growth in complexity with increasing sizes of state spaces, branching factors, and time horizons remain true regardless of the solution approach taken. It is unlikely that a uniformly precise model, a simulator at the level of precision of individual atoms, or even circuit diagrams of every component used by the agent will help it solve the kind of complex tasks on which humans would appreciate assistance. On the other hand, not using any model at all would result in dangerous agents that would not be able to safely evaluate the possible outcomes of their actions. Our results show that these divides can be bridged using hierarchical modeling and solution approaches that simplify the representational requirements and offer computational advantages that could make autonomous robots feasible in the real world. 


\begin{figure*}
  \centering
    \includegraphics[width=\textwidth]{images/overview.png}
  \caption{
  Overview of our approach. (a) We aim to find a collision-free path for a rigid body from its current configuration $q_t$ to the goal configuration $q_g$. (b) We assume no prior knowledge about the scene and represent obstacles by points and normals sampled on object surfaces. (c) Our neural network learns the PointNet encoding of observed points and normals together with the motion policy. (d) The learned network generates actions that move the body towards the goal configuration along a collision-free path.   
 }
  \label{fig:overview}
\vspace{-0.4cm}
\end{figure*}

\begin{figure*}[t]%% placement specifier
%% Use \includegraphics command to insert graphic files. Place graphics files in 
%% working directory.
\centering%% For centre alignment of image.
\includegraphics[width=0.8\textwidth]{Paper_Figures/Figure1.png}
%% Use \caption command for figure caption and label.
\caption{LLM and digital twin enhanced dynamic robot task allocation}\label{fig:framework}
%% https://en.wikibooks.org/wiki/LaTeX/Importing_Graphics#Importing_external_graphics
\end{figure*}


\section{Methodology}
The methodology developed in this study is illustrated in Fig. \ref{fig:framework}. It introduces a scalable framework for situation-aware decision-making in multi-robot task allocation, enabled by an LLM. The system is capable to operate with or without an initial digital model (e.g., BIM) and has a closed-loop feedback mechanism that synchronizes the physical site with its digital representation.

The Digital Twin serves as a bridge for synchronization between the physical site and the digital model. The digital model integrates: (1) a list of construction tasks and their precedence relationship that can either be presented in the form of a structured list or be automatically extracted from the BIM; (2) site information (i.e., material supplies and locations), which can either be manually created or detected by robots on the physical site and transmitted to the digital model; and, (3) building information that includes object geometries, types, and poses. It includes the completed structures as well as placeholders for components that have not yet been constructed. Each component is associated with attribute information, including its name (reference ID), layer (unbuilt, as-built, materials, inactive), material type, installation pose, and gripping pose. Based on this information, a comprehensive task list that contains the completion status can be extracted. It covers all active tasks (i.e., tasks that are scheduled to be performed in this project stage), including both the ones that are completed and the pending ones waiting to be assigned and executed.

When an update event is triggered, the robots extract the list of tasks to complete and their logic relationships from the digital model. Updates to this list may originate from the physical site, where task completion status and site changes are detected and transmitted via ROS. Conversely, new or modified task instructions are synchronized back to the digital twin. Human supervision plays a critical role in interpreting construction progress and intervening when unforeseen uncertainties occur, such as weather disruptions, equipment failures, or site access limitations. These decisions and environmental conditions are reflected in the digital twin and influence task allocation strategies.

The central processor coordinates the intelligent decision-making layer. Here, an LLM interprets contextual inputs (e.g., natural language updates from human operators), supported by a repository of domain knowledge. Rather than solving the optimization problem directly, the LLM identifies relevant parameters and updates constraints within a pre-defined IP model used for task allocation. The optimized task allocation plan is then sent to the robot controllers, which execute the tasks accordingly. Completion data flows back to the digital twin and task model, ensuring continuous updates and alignment between the virtual and physical environments.




\subsection{Digital twin system}

As a critical component of the proposed framework, the digital twin system integrates the digital model and the system central processor to continuously monitor and manage the project's status.  It also functions as the user interface, connecting human users with  system backends and enabling them to visualize, supervise, and intervene in the project using natural language via keyboard or voice input. The digital twin state at time $t$ is formalized as:
\begin{align}
    S_t = D_0+\mathcal{F}(S_{t-1}, R_t, SI_t, \Delta_{task}, \Delta_{robot})
    \end{align}
Here, $S_t$ represents the current state of the digital twin at time $t$, $D_0$ denote the initial information from the digital model $D$, $\mathcal{F}$ is the state update function, $R_t$ and $SI_t$ denote the robot states (i.e., arm joint states and base locations and orientations) and site information at time $t$,  respectively, and $\Delta_{task}$ and $\Delta_{robot}$  capture updates to task statuses and robot statuses (i.e., high-level task each robot in the team is performing) at time $t$, respectively.  Fig. \ref{fig:dt} illustrates the system architecture and information flow of the technical implementation. The three core capabilities, visualization, supervision, and intervention, are supported by four key modules, including the synchronized visualization module, task status tracker, robot status tracker, and user command receiver. 



\textbf{Visualization} ($D_0, R_t, SI_t \rightarrow S_t$): The synchronized visualization module, adapted from the authors' previous work \cite{wang2024enabling}, is responsible for continuously updating the visual representation of the site and robots in the digital twin based on $SI_t$ and $R_t$. Before the process starts, robot emulators are generated by transmitting the URDF models of the robots and associated mesh files from ROS to the digital twin. In the initial state $S_0$, the basic digital twin scene is generated from $D_0$, which includes both geometric and attribute information. In this step, component geometries received from $D$ are instantiated at the corresponding locations in the digital twin as the environment. Attribute information embedded in $D$ is transmitted to the digital twin to adjust the properties of the corresponding components, such as its name, layer, attachment offset, and visualization (e.g., color).  During the runtime, the robot emulators mirror on-site robots' movements by subscribing to their state data $R_t$, thereby reflecting synchronized robot states to human users. $SI_t$ is similarly integrated through subscription to corresponding ROS topics.

\textbf{Supervision} $(D_0,\Delta_{task}, \Delta_{robot}\rightarrow S_t)$: The supervision function allows the users to quickly get an oversight of construction progress and multi-robot operations to facilitate intervention decision making, which is enabled by the robot status tracker and task status tracker. At $t=0$, the task status tracker extracts from $D$ a comprehensive list of tasks involved. According to the progress, each task is marked as uninitiated, ongoing, or completed. As construction progresses, these labels are updated according to $\Delta_{task}$. Meanwhile, the robot status tracker provides the operation status of each robot in the teams in the form of high-level task name or description. If no task is being performed, the status of the corresponding robot will be marked as "idle".

\textbf{Intervention}: The user command receiver module supports user interventions when users make intervention decisions during the visualization and supervision process or uncertainties occur. In these cases, users can submit high-level instructions in natural language by keyboard typing or directly through voice. The commands are processed by the embedded LLM into actionable modifications. These modifications are then forwarded as valid inputs to the task allocation module, updating $\Delta_{task}$ and triggering state updates via $\mathcal{F}$.

After the completion of tasks or upon reaching a project milestone, the updated task list and building information are transmitted to update the digital model. This information includes components installed during construction and changes in site conditions, such as leftover construction materials on-site. As a result, the digital model maintains an up-to-date record and accurate representation of the project. This closed-loop feedback mechanism ensures that the digital model accurately reflects the project's as-built condition. It also enhances the model's utility in subsequent phases of work, such as facility management, progress auditing, and future renovations.

\begin{figure}[t]

\centering
\includegraphics[width=0.48\textwidth]{Paper_Figures/DT_Framework.png}

\caption{Digital twin system framework}\label{fig:dt}

\end{figure}


\subsection{Multi-robot task allocation: plan creation}
% \bofutodo{ Integer Programming algorithm theory) Done}

This section formally defines the task allocation and scheduling problem, formulates it as an integer program, and describes the algorithms used to solve it.

Suppose there are \(n_\Acal\) capabilities, denoted by \(\Acal = \{1, \ldots, n_\Acal\}\), \(n_\Rcal\) heterogeneous robots, denoted by \(\Rcal = \{1, \ldots, n_\Rcal\}\), and \(n_\Tcal\) tasks, denoted by \(\Tcal = \{1, \ldots, n_\Tcal\}\). Each robot has a subset of capabilities, and each task requires a team of robots with the required capabilities to serve it. When a team of qualified robots \(\Rcal_i \subset \Rcal\) arrive, the task \(i \in \Tcal\) can be completed after time duration \(T^d_{i}\). A robot may only serve one task at a time. Additionally, each task may have a set of dependencies \(\Tcal_i \subset \Tcal\), i.e., other tasks that must be completed beforehand.
The goal is to schedule tasks on a set of available robots in a way that minimizes the makespan, the time by which all tasks have been completed.

We formulate the above-mentioned task allocation and scheduling problem as an integer program. Here, we provide common notations in Table \ref{tab:variable_definition}. The decision variables are \(x_{ir}\) for task assignment and \(t^s_{ir}\), \(t^e_{ir}\), \(t^s_i\), and \(t^e_i\) for scheduling.

\begin{table}[t]
  \caption{Definition of the notation.}
  \label{tab:variable_definition}%
    \begin{tabular}{p{0.06\linewidth}|p{0.82\linewidth}} 
    \toprule
     & Meaning
    \\
    \midrule
    \(x_{ir}\) & = 1 if task \(i \in \Tcal\) is assigned to robot \(r \in \Rcal\).
    \\
    \(t^s_{ir}\) & The start time if robot \(r \in \Rcal\) works on task \(i \in \Tcal\).
    \\
    \(t^e_{ir}\) & The end time if robot \(r \in \Rcal\) works on task \(i \in \Tcal\).
    \\
    \(t^s_i\) & The start time of task \(i \in \Tcal\).
    \\
    \(t^e_i\) & The end time of task \(i \in \Tcal\).
    \\
    \(y_{ijr}\) & Auxiliary variables for robot scheduling.
    \\
    \(y_{ij}\) & Auxiliary variables for task dependencies.
    \\
    \(T^D_{i}\) & The time duration to complete task \(i \in \Tcal\).
    \\
    \(T^s_i\) & The earliest start time for task \(i \in \Tcal\).
    \\
    \(T^e_i\) & The latest end time for task \(i \in \Tcal\).
    \\
    \(T_{\text{large}}\) & A large time constant.
    \\
    \(a_{kr}\) & The amount of capability \(k \in \Acal\) available on robot \(r \in \Rcal\).
    \\
    \(b_{ki}\) & The amount of capability \(k \in \Acal\) required to execute task \(i \in \Tcal\)
    \\
    \bottomrule
    \end{tabular}
\end{table}

\textbf{Objective function:}
The objective function jointly minimizes the makespan and the individual task completion times. The third piece in the objective function penalizes the number of robots assigned to tasks. The makespan is defined as the maximum end time among all tasks. We set \(C_m \gg C_s \approx C_r\) to ensure that the makespan is the primary objective.
\begin{align}
    % \min_{x_{ir}, \ t^s_{ir}, \ t^e_{ir}, \ t^s_i, \ t^e_i} 
    \min_{\substack{x_{ir}, \ t^s_i, \ t^e_i, \\ t^s_{ir}, \ t^e_{ir}}}\ ( C_m \max_{i \in \Tcal} \ t^e_i + C_s \sum_{i \in \Tcal} \ t^e_i + C_r \sum_{i \in \Tcal} \sum_{r \in \Rcal} x_{ir} ) \label{eqn:objective}
\end{align}

\textbf{Variable bound constraints:}
\(x_{ir}\) is a binary task allocation variable, equal to 1 if task \(i \in \Tcal\) is assigned to a team that contains robot \(r \in \Rcal\), and 0 otherwise. The scheduling variables are continuous, with the requirement that the end time is always larger the start time.
\begin{align}
    x_{ir} \in \{0, 1\}, \quad &\forall i \in \Tcal, \forall r \in \Rcal \label{eqn:assignment_var} \\
    0 \leq t^s_i \leq t^d_i, \quad &\forall i \in \Tcal \label{eqn:task_time_var} \\
    0 \leq t^s_{ir} \leq t^d_{ir}, \quad &\forall i \in \Tcal, \forall r \in \Rcal \label{eqn:robot_time_var}
\end{align}

\textbf{Task dependency constraints:}
The task dependency is specified using the time variables. If task \(i\) depends on the completion of task \(j\), then task \(i\) can start only after task \(j\) is completed.
\begin{align}
    t^s_i \geq t^e_j, \quad &\forall j \in \Tcal_i, \forall i \in \Tcal \label{eqn:task_dependency}
\end{align}

\textbf{Task requirement constraints:}
A capability model is introduced to define the task requirement constraints.
Let \(a_{kr} \in \nonnegativerealset\) denote the amount of capability \(k \in \Acal\) available on robot \(r \in \Rcal\), and let \(b_{ki} \in \nonnegativerealset\) denote the amount of capability \(k \in \Acal\) required to execute task \(i \in \Tcal\).
% Each task must be assigned to exactly one robot, as specified in \eqref{eqn:task_assign_to_one_robot}.
The robot team assigned to a task must collectively possess all the capabilities required by that task, as enforced by \eqref{eqn:task_capability_requirement}.
\begin{align}
    % \sum_{r \in R} x_{ir} = 1, \quad & \forall i \in \Tcal \label{eqn:task_assign_to_one_robot} \\
    \sum_{r \in \Rcal} {a_{kr} x_{ir}} \geq b_{ki}, \quad &\forall k \in \Acal, \forall i \in \Tcal \label{eqn:task_capability_requirement}
\end{align}

\textbf{Task schedule constraints:}
Equation \eqref{eqn:task_duration} specifies the relation between the task start and end time.
The constraints \eqref{eqn:task_start_leq_robot_start}-\eqref{eqn:task_end_geq_robot_end} are organized into two groups. The first group, \eqref{eqn:task_start_leq_robot_start}-\eqref{eqn:task_start_geq_robot_start}, ensures that when robot \(r \in \Rcal\) is in the team assigned to task \(i \in \Tcal\), the task start time \(t^s_i\) equals the robot-specific start time \(t^s_{ir}\). Similarly, the second group, \eqref{eqn:task_end_leq_robot_end}-\eqref{eqn:task_end_geq_robot_end}, guarantees that the task end time \(t^e_i\) matches \(t^e_{ir}\) under the same assignment.
% The third group, \eqref{eqn:task_duration_leq}-\eqref{eqn:task_duration_geq}, enforces that the task duration is given by \(t^e_{ir} = t^s_{ir} + T^D_{i}\) when robot \(r\) is in the team for task \(i\).
Note that \(T_{\text{large}}\) is a large time constant.
\begin{align}
    t^e_i = t^s_i + T^D, \quad & \forall i \in \Tcal \label{eqn:task_duration} \\
    t^s_i \leq t^s_{ir} + T_{\text{large}} (1 - x_{ir}), \quad & \forall r \in \Rcal, \forall i \in \Tcal \label{eqn:task_start_leq_robot_start} \\
    t^s_i \geq t^s_{ir} - T_{\text{large}} (1 - x_{ir}), \quad & \forall r \in \Rcal, \forall i \in \Tcal \label{eqn:task_start_geq_robot_start} \\
    t^e_i \leq t^e_{ir} + T_{\text{large}} (1 - x_{ir}), \quad & \forall r \in \Rcal, \forall i \in \Tcal \label{eqn:task_end_leq_robot_end} \\
    t^e_i \geq t^e_{ir} - T_{\text{large}} (1 - x_{ir}), \quad & \forall r \in \Rcal, \forall i \in \Tcal \label{eqn:task_end_geq_robot_end}
    % t^e_{ir} \leq t^s_{ir} + T^D_{i} + T_{\text{large}} (1 - x_{ir}), \quad & \forall r \in \Rcal, \forall i \in \Tcal \label{eqn:task_duration_leq}  \\
    % t^e_{ir} \geq t^s_{ir} + T^D_{i} - T_{\text{large}} (1 - x_{ir}), \quad & \forall r \in \Rcal, \forall i \in \Tcal \label{eqn:task_duration_geq}
\end{align}

\textbf{Robot schedule (no-overlap) constraints:}
Each robot can perform at most one task at a time, so their task schedules must not overlap, as specified in \eqref{eqn:robot_schedule1}-\eqref{eqn:robot_schedule2}. When the auxiliary variable \(y_{ijr} = 1\), task \(i\) is scheduled before task \(j\); otherwise, task \(i\) is scheduled after task \(j\).
\begin{align}
    t^e_{ir} &\leq t^s_{jr} + T_{\text{large}} (1 - y_{ijr}), &\quad \forall i < j \in \Tcal, \forall r \in \Rcal \label{eqn:robot_schedule1} \\
    t^e_{jr} &\leq t^s_{ir} + T_{\text{large}} \ y_{ijr}, &\quad \forall i < j \in \Tcal, \forall r \in \Rcal \label{eqn:robot_schedule2} \\
    y_{ijr} &\in \{0, 1\}, &\quad \forall i < j \in \Tcal, \forall r \in \Rcal 
\end{align}

% \begin{align}
%     t^e_{ir} \leq t^s_{jr} \text{ or }  t^e_{jr} \leq t^s_{ir}, \forall i, j \in \Tcal, \ \text{s.t.} \ i \neq j, \forall r \in \Rcal \label{eqn:robot_schedue}
% \end{align}

% \begin{align}
%     \phi(t^s_{ir}, t^e_{ir}, t^s_{jr}, t^e_{jr}) \neq 0, \quad \forall i, j \in \Tcal, \ \text{s.t.} \ i \neq j, \forall r \in \Rcal \label{eqn:robot_schedue} \\
%     \phi(t^s_{ir}, t^e_{ir}, t^s_{jr}, t^e_{jr}) = \begin{cases}
%          \ \ \ 1, \quad t^s_{ir} \leq t^e_{ir} \leq t^s_{jr} \leq t^e_{jr} \\
%         -1, \quad t^s_{jr} \leq t^e_{jr} \leq t^s_{ir} \leq t^e_{ir} \\
%          \ \ \ 0, \quad \text{Otherwise} \label{eqn:robot_schedue_phi}
%     \end{cases}
% \end{align}

\textbf{Time window constraints (optional):}
The time window constraint is an optional condition that requires a task to be completed within a specified time interval.
\begin{align}
    T^s_i \leq t^s_i \leq t^e_i \leq T^e_i, \quad \forall i \in \Tcal \text{ with a time constraint} \label{eqn:task_time_window}
\end{align}

\textbf{Task conflict constraints (optional):}
The task conflict constraints are optional and ensure that two tasks \(i\) and \(j\) with conflicting resource requirements are not executed concurrently. \(y_{ij}\) is a binary auxiliary variable.
\begin{align}
    t^e_{i} &\leq t^s_{j} + T_{\text{large}} (1 - y_{ij}), &\forall i < j \in \Tcal \ \text{with conflicts} \label{eqn:task_conflict1}\\
    t^e_{j} &\leq t^s_{i} + T_{\text{large}} \ y_{ij}, &\forall i < j \in \Tcal \ \text{with conflicts} \label{eqn:task_conflict2} \\
    y_{ij} &\in \{0, 1\}, &\forall i < j \in \Tcal \ \text{with conflicts}\label{eqn:task_conflict3}
\end{align}


% \begin{align}
%     \phi(t^s_i, t^e_i, t^s_j, t^e_j) \neq 0, \quad \forall i, j \in \Tcal, \ \text{s.t.} \ i \text{ conflicts with } j \label{eqn:task_conflict} \\
%     \phi(t^s_i, t^e_i, t^s_j, t^e_j) = \begin{cases}
%          \ \ \ 1, \quad t^s_i \leq t^e_i \leq t^s_j \leq t^e_j \\
%         -1, \quad t^s_j \leq t^e_j \leq t^s_i \leq t^e_i \\
%          \ \ \ 0, \quad \text{Otherwise}
%     \end{cases}
% \end{align}

The above integer program can be solved using a CP-SAT solver. A CP-SAT solver is an optimization engine that combines constraint programming (CP) techniques with Boolean satisfiability (SAT) solving methods. It is designed to efficiently tackle combinatorial optimization problems - such as scheduling, planning, and assignment tasks - by systematically exploring potential solutions while adhering to a set of constraints. In our work, we employ the CP-SAT solver provided by Google OR-Tools to efficiently solve the integer program.

\renewcommand{\arraystretch}{1.3} % Adjust row height

\begin{table*}[h!]
\footnotesize
\setlength{\tabcolsep}{5pt}
\centering
\caption{Types of constraint and parameter changes}
\begin{tabular}{p{3cm}|p{6cm}|p{0.7cm}|p{6.8cm}}
    \toprule % Top line added
    \textbf{Constraint Type} & \textbf{Definition} & \textbf{Label} & \textbf{Parameter} \\
    \midrule
    Task Dependency & Adjustments in the sequence or prerequisite relationships among tasks & \textbf{1} & \texttt{[task\_id, successors, +/-]} \\

    Task Duration & Variations in the estimated time to complete tasks & \textbf{2} & \texttt{[task\_id, new\_duration]} \\

    Task Starting Time & Changes to tasks' earliest or planned start times & \textbf{3} & \texttt{[task\_id, start\_time\_change]} \\

    Number of Robot & Variations in robot availability & \textbf{4} & \texttt{[new\_robot\_type\_id, robot\_number\_change]} \\

    Task Conflict Constraints & Some tasks cannot be performed at the same time & \textbf{5} & \texttt{[task\_id1, task\_id2]} \\
    \bottomrule % Bottom line added
\end{tabular}
\label{tab:constraint_changes}
\end{table*}

\subsection{LLM-driven adaptive decision-making}

To enable adaptive task reallocation in response to dynamic site conditions, we introduce a formal mapping that captures how the LLM interprets natural language narratives and transforms them into actionable modifications to the task allocation problem. This process serves as the core of the narrative-driven adaptation mechanism, facilitating human-in-the-loop flexibility without requiring direct manipulation of optimization code.

Fig. \ref{fig:modular_system} illustrates the pipeline for enabling adaptive multi-robot task allocation driven by natural language inputs. The process begins with a narrative containing dynamic project updates, such as task sequencing preferences, resource delays, or timing adjustments. An LLM processes this narrative to extract actionable information, identifying relevant task entities (e.g., painting, window installation, wall-drilling) and interpreting relationships such as precedence constraints or temporal shifts. To ensure accurate task reference, a structured database is used to map textual descriptions to corresponding task IDs (e.g., wall-drilling → T6).

The extracted instructions are then categorized into discrete flag types, each representing a specific kind of modification as demonstrated in Table \ref{tab:constraint_changes}. Each flag is paired with a structured parameter representation, which encapsulates the necessary information to modify the optimization code. These flags are designed to align with designated insertion points in a standalone optimization codebase.

Let the input to the LLM be a narrative description
$\mathcal{N}$, such as a sentence or paragraph provided by a site supervisor or worker, which describes changes in site conditions, task statuses, or scheduling preferences. The LLM acts as a function:


\begin{align}
\mathcal{M}: \mathcal{N} \rightarrow\left\{\left(C_k, \theta_k\right)\right\}_{k=1}^K
\end{align}

Where:

$\mathcal{M}$ is the mapping function learned or encoded by the LLM


$C_k$ is the type of constraint (e.g., dependency, time shift)

$\theta_k$ is the parameter set for the update
\\

Each $\left(C_k, \theta_k\right)$ pair defines a flagged constraint to be dynamically injected into the optimization model. These flags are associated with predefined templates in the integer programming codebase, allowing seamless integration with the solver without altering the base model structure. For example:

\begin{itemize}
    \item A narrative like "Our skilled wall-drilling worker will be arriving an hour late" may yield $\left(C= 3, \theta=\{T6, 1\}\right)$.

    
    \item A command like "The owner has requested that painting be completed before window installation" maps to $\left(C= 1, \theta=\{T13 \prec T 9\}\right)$.
\end{itemize}





To ensure consistency, the LLM references a structured task knowledge base $\mathcal{T}=$ $\left\{\left(T_i\right.\right.$, Description$\left.\left._i\right)\right\}$ that maps textual task names to task IDs. This helps avoid ambiguity and ensures each extracted constraint can be accurately linked to the model.





The LLM output is then parsed into a structured JSON format and passed to the optimization backend, which updates the constraint set accordingly:

\begin{align}
\mathcal{C}_{\text {new }}=\mathcal{C}_{\text {original }} \cup\left\{C_k\left(\theta_k\right)\right\}
\end{align}


This interaction between the LLM and the optimizer ensures the system can continuously adapt to evolving conditions while maintaining model transparency and interpretability.

Please note that the optimization algorithm operates independently from the LLM. The LLM's role is limited to interpreting the narrative and modifying only the flagged portions of the code, leaving the optimization logic itself untouched. This separation ensures that the optimization process remains robust and interpretable while gaining the flexibility through natural language instructions. The result is a hybrid, human-in-the-loop system that bridges narrative reasoning and formal optimization in a modular, scalable manner.


\begin{figure}[t]

\centering
\includegraphics[width=0.48\textwidth]{Paper_Figures/Figure3.png}

\caption{Pipeline for narrative-driven adaptive schedule optimization}\label{fig:modular_system}
\end{figure}





\subsection{LLM implementation and prompt design}
Prompt design is critical for LLMs to perform high-quality information extraction, particularly given the intricate nature of construction scheduling tasks \cite{white2023prompt, atreja2024prompt}. To create effective prompts tailored for construction project scheduling, we utilize prompt engineering strategies articulated by White et al. (2023) \cite{white2023prompt}. These strategies emphasize the necessity of integrating context descriptions, structured templates, and explicit output formatting into prompts. In our scenario, context descriptions detail construction tasks, their interdependencies, durations, and associated robotic capabilities, and we further employ JSON as the structured output format to ensure clarity.

In addition, we incorporate two techniques into our prompt design. The first technique is chain-of-thought (CoT) prompting, which guides the LLM through step-by-step reasoning \cite{wei2022chain}. This approach can facilitate accurate task identification and constraint recognition by prompting LLM to engage in deliberate, intermediate reasoning stages before concluding. Specifically, our CoT prompts instruct the LLM to: (1) carefully read the task description, (2) identify the specific task and associated robotic system, (3) determine the constraint type, (4) extract relevant parameters, and (5) format the extracted information. The second technique is few-shot learning. Given that prior research indicates LLMs excel in few-shot learning scenarios \cite{brown2020language, hegselmann2023tabllm}, we provide multiple examples within our prompt. These examples serve as explicit references, allowing LLM to capture task requirements more effectively and thereby enhancing the accuracy of its outputs. By leveraging these prompt engineering techniques, we develop a structured prompt template tailored for prompting LLMs to analyze task descriptions of construction project scheduling (see Appendix \ref{app1}). 

For model selection, we refer to the widely recognized Multi-task Language Understanding (MMLU) benchmark \cite{hendrycks2020measuring}. We utilize two popular LLM families, OpenAI’s GPT series and Anthropic’s Claude series, because of their demonstrated capabilities in complex reasoning and strong performance across diverse NLP tasks \cite{fan2023nphardeval, sonoda2024diagnostic}. Specifically, from OpenAI, we select GPT-4o-mini, GPT-4o, GPT-4.1-mini and GPT-4.1. From Anthropic, we include Claude-Haiku and Claude-Sonnet.

 



\subsection{Multi-robot task allocation: replanning}\label{sec:method-replanning}

Suppose the task schedule and robot assignments have already been determined through the task allocation optimization described in the previous section, and the plan has been partially executed.
At time \(T^R\), updated information becomes available reflecting changes in task conditions, such as modifications to robot capabilities, availability, expected task durations, additional task dependencies, or time window constraints.
These changes may render the current plan infeasible or suboptimal, necessitating the generation of a new plan. At this point, some tasks have been completed, others are in progress, and the rest have not yet started. For the completed and ongoing tasks, their assigned robots and schedules must remain unchanged during the replanning. Furthermore, in many real-world scenarios, modifying the original plan may incur additional operational costs. To account for this, penalties can be introduced into the optimization to discourage unnecessary deviations from the original plan.

Given an original task allocation and scheduling plan, represented by the solution to the variables $^0x_{ir}$, $^0t^{s}_{ir}$, $^0t^e_{ir}$, $^0t^s_i$, and $^0t^e_i$. Let $\Tcal_{-} = \{i \in \Tcal \mid {^0t^s_i} \leq T^R\}$ denote the set of tasks that are ongoing or completed by the replanning time $T^R$, and let $\Tcal_{+} = \{i \in \Tcal \mid {^0t^s_i} > T^R\}$ represent the set of tasks that have not yet started. The replanning optimization is formulated as follows.


\textbf{Replanning optimization:}
The replanning objective jointly minimizes the makespan and individual task completion times while also penalizing deviations from the original plan. We set \(C_m \gg C_s \approx C_x \approx C_t \) to ensure that the makespan remains the primary objective, with the other terms balanced relative to one another. The constraints include the original constraints \eqref{eqn:assignment_var}-\eqref{eqn:task_conflict3}, along with additional constraints \eqref{eqn:schedule_fixed}-\eqref{eqn:assignment_fixed} to ensure plan consistency.
\begin{align}
    \min_{\substack{x_{ir}, \ t^s_i, \ t^e_i, \\ t^s_{ir}, \ t^e_{ir}}} \ ( & C_m \max_{i \in \Tcal} \ t^e_i + C_s \sum_{i \in \Tcal} \ t^e_i + C_r \sum_{i \in \Tcal} \sum_{r \in \Rcal} x_{ir} \nonumber \\
    & + C_x \Delta x + C_t \Delta t) \\
    \text{subject to} & \ \eqref{eqn:assignment_var}-\eqref{eqn:task_conflict3} \text{ and } \eqref{eqn:schedule_fixed}-\eqref{eqn:assignment_fixed} \nonumber
\end{align}

\textbf{Plan change penalty for:}
Changes to the original task assignments and schedules are penalized during replanning to ensure that the new plan accommodates updated task conditions while minimizing deviations from the original plan.
\begin{align}
    \Delta x &= \sum_{\forall i \in \Tcal_{+}} \sum_{\forall r \in \Rcal} |x_{ir} - {^0x_{ir}}| \label{eqn:schedule_change_penalty} \\
    \Delta t &= \sum_{\forall i \in \Tcal_{+}} \left( |t^s_i - {^0t^s_i}| + |t^e_i - {^0t^e_i}| \label{eqn:assignment_change_penalty} \right)
\end{align}

\textbf{Historical plan constraints for completed tasks:} The robot assignments and task schedules for ongoing or completed tasks must remain unchanged to ensure consistency between the updated plan and the actual execution.
\begin{align}
    t^s_i = {^0t^s_i}, \ t^e_i = {^0t^e_i}, \quad \forall i \in \Tcal_{-} \label{eqn:schedule_fixed} \\
    x_{ir} = {^0x_{ir}} \quad \forall r \in \Rcal, \forall i \in \Tcal_{-} \label{eqn:assignment_fixed}
\end{align}









\section{Experiments and Results}
\label{sec:exp}

This section presents experiments and results of the proposed framework. We have evaluated this framework in various simulated scenarios and compared it to ORCA. We have also tested our method on a real multi-robot system. 

\subsection{Experiment Setup}
\subsubsection{\textbf{Scenarios}} We evaluated our learned policy on six different scenarios with different number of agents (as shown in Figure~\ref{fig:scenarios}). Note that
the test scenarios \textit{3 Obstacles} and \textit{1 Obstacle} have static obstacles, which never appear in any data collection scenarios for training the CANet. In addition, since the trained policy outputs the collision avoidance velocity in a random manner, its performance is averaged over 20 simulations. 

We compared the performance of learned policy with the ORCA policy. Most parameters of the ORCA policy are set to be the same as the values used in the data generation for the learned policy (as stated in Section~\ref{sec:data:data-gen}), but we tuned some parameters to optimize ORCA's performance. In particular, to obtain the best performance of ORCA, we 
change $\textsc{timeHorizonObs}$ to $10.0$s for scenarios with static obstacles and tune $\textsc{timeHorizon}$ for different scenarios. 
%$\textsc{protectRadius}$ is difficult to be tuned for ORCA, so we attempt two different values $0.2$m and $0.5$m in our experiments.
In each simulation, the performance of ORCA is evaluated with two different $\textsc{protectRadius}$ values  $0.2$m and $0.5$m.
The time step size $\tau$ of the sensing-acting cycle is set to $0.1$s. The detailed description for each test scenario is as follows:
\renewcommand{\labelitemi}{\textbullet}
\begin{itemize}
\item \textit{Crossing}: agents are separated in two groups, and their path will intersect in the bottom left corner;
\item \textit{Circle}: agents are initially located along a circle and each agent's goal is to reach its antipodal position;
\item \textit{Swap}: two groups of agents moving in opposite directions swap their positions;
\item \textit{Random}: agents are randomly initialized in a cluttered environment and are assigned random goals; 
\item \textit{3 Obstacles}: six agents move across three obstacles;
\item \textit{1 Obstacle}: four agents initialized on a circle move towards their antipodal positions, and an obstacle is located at the center. 
\end{itemize}
The trajectories generated using the learned navigation policy are shown in Figure~\ref{fig:circle} and Figure~\ref{fig:paths}.


\subsubsection{\textbf{Performance Metrics}} To compare the performance of our framework and ORCA quantitatively, we use the following performance metrics: 
\renewcommand{\labelitemi}{\textbullet}
\begin{itemize}
\item \textit{Total travel time}: the time taken by the last agent to reach its goal;
\item \textit{Total distance traveled}: the total distance traveled by all agents to reach their goals.
\item \textit{Safety margin}: the agent's closest distance to other agents and static obstacles;
\item \textit{Completion}: if all agents reach their goals within a time limit without any collisions, the scenario is successfully completed. 
\end{itemize}

\begin{figure} 
\centering
\includegraphics[width=0.8\linewidth]{fig/long7.pdf}
\caption{Six scenarios used to compare the navigation performance of our learned policy and the ORCA policy. 
}
\label{fig:scenarios}
\vspace*{-0.15in}
\end{figure}


\subsection{Quantitative Comparisons}
In Figure~\ref{fig:time} and~\ref{fig:distance}, we measure two metrics -- \textit{total travel time} and \textit{total traveled distance} -- to evaluate the performance of our approach and ORCA.
We can observe that when comparing with the ORCA policy with $\textsc{protectRadius}=0.5$, the learned policy provides better or comparable performance in terms of navigation duration and trajectory length. In most scenarios, the ORCA policy with $\textsc{protectRadius}=0.2$ has shorter navigation time and trajectory length than the learned policy. This is because the ORCA policy with $\textsc{protectRadius}=0.2$ is very aggressive and allows a small safe margin during the navigation, as shown in Table~\ref{tab:safety-margin}. Both our learned policy and the ORCA policy with $\textsc{ProtectRadius}=0.5$ try to keep a large enough margin with nearby agents/obstacles. 
The difference is that the ORCA policy with $\textsc{ProtectRadius}=0.5$ uses the protect radius parameter to keep a hard margin: no obstacles/agents should get closer to the agent than $\textsc{ProtectRadius} - \textsc{radius} = 0.5 - 0.2 = 0.3$m, and this constraint may be too conservative in a cluttered scene. 

Instead, our method learns the preference for margin implicitly from the data and is able to keep the clearance in an adaptive manner: in cluttered situations, the agents can endure a small safety-margin while in an open space, the agents will tend to keep a large safety-margin. For instance, the safety margin in the \emph{3 Obstacles} scenario is smaller than in the \emph{1 Obstacle} scene as shown in Table~\ref{tab:safety-margin}, because the former is more cluttered.

We also set up a more challenging scenario with an L-shape static obstacle at the center (shown in Figure~\ref{fig:stuck}) and measure the \textit{Completion} metric. We randomly generate $100$ initial states where all agents are randomly placed and they are assigned with appropriate random goals. We then compare our method and ORCA by counting the number of failures, i.e., some agents do not reach their goals or severely collide with other agents/obstacles during the runtime. ORCA has a failure rate of $15\%$ while our learned policy only has $2\%$. Figure~\ref{fig:stuck} shows a \textit{stuck} case for ORCA while our learned policy can complete it successfully.

\begin{figure}[!h]
\centering
\includegraphics[width=0.8\linewidth]{fig/long8.pdf}
\caption{Trajectories of five scenarios using the learned policy.} 
\label{fig:paths}
\vspace*{-0.2in}
\end{figure}

\begin{figure} 
\centering
\includegraphics[width=1\linewidth]{fig/long9.pdf}
\caption{Total time of our method and ORCA in all scenarios.}
\label{fig:time}
%\vspace*{-0.2in}
\end{figure}

\begin{figure} 
\centering
\includegraphics[width=1\linewidth]{fig/long10.pdf}
\caption{Distance traveled of our method and ORCA in all scenarios.}
\label{fig:distance}
\end{figure}

\begin{table*}
 \begin{tabularx}{1\textwidth}{l|X|X|X|X|X|X}
  	Safety Margin (min/ave) & Crossing & Circle & Swap & Random & 3 Obstacles & 1 Obstacle  \\
   \hline
%   \hline
   Our method &  0.028 / 0.197 & 0.281 / 0.281 & 0.209 / 0.365 & 0.171 / 0.334 & 0.012 / 0.188 &0.108 / 0.154\\
   \hline
   ORCA - \textsc{protectRadius}=0.5 & 0.300 / 0.375 &	0.262 / 0.297 &	0.276 / 0.295 &	0.297 / 0.299 &	0.299 / 0.301 &	0.300 / 0.365
 \\
   \hline
   ORCA - \textsc{protectRadius}=0.2 & 0.000 / 0.003 & -0.016 / -0.004	& 0.000 / 0.098 &	0.000 /  0.136  &	0.000 / 0.000 &	0.004 / 0.004 \\
 \end{tabularx}
\caption{The minimum and average safety margins for agents when using our learned policy and ORCA policy with $\textsc{protectRadius}=0.5$ and $\textsc{protectRadius}=0.2$ in all six scenarios.}
\label{tab:safety-margin}
\vspace*{-0.2in}
\end{table*}


\begin{figure}[h] 
\begin{subfigure}{0.2\textwidth}
\includegraphics[width=1.0\linewidth]{fig/long11a.pdf}
\caption{ORCA}
\label{fig:stuck-rvo}
\end{subfigure}
\begin{subfigure}{0.2\textwidth}
\includegraphics[width=1.0\linewidth]{fig/long11b.pdf}
\caption{Our method}
\label{fig:stuck-dnn}
\end{subfigure}

\caption{For the given scenario with static obstacles,
one agent gets stuck when using the ORCA policy while all agents reach their goals with the learned policy.}
\label{fig:stuck}
\vspace*{-0.2in}
\end{figure}


\subsection{Generalization}

An interesting phenomenon while using the learned policy is that in a highly symmetrical scenario like \textit{Circle}, the agents will present certain cooperative behaviors since all agents are using the same learned policy. For instance, a cooperative rotation behavior is shown in Figure~\ref{fig:circle-dnn} where agents starts to rotate at the same pace when they are close to each other. While for ORCA (as shown in Figure~\ref{fig:circle-rvo}), each agent passes the central area by itself without any collective behaviors and some agents yields jerky motions. 

The good generalization capability is another notable feature of our method. The learned policy's performance  in scenarios with static obstacles demonstrates that it generalizes well to handle to previously unseen situations. In addition, we also evaluate the learned policy in the \textit{Circle} scenario with four different-sized agents. In Figure~\ref{fig:ours-same}, agents with the same size have identical paths as they take the same strategy to avoid collision with each other. When one agent gets bigger (as shown in Figure~\ref{fig:ours-diff}), it will deviate more from the original path to generate safe movements and this causes other agents on its path to adjust navigation behaviors accordingly. Please note that this experiment (Figure~\ref{fig:rvo-same} and~\ref{fig:rvo-diff}) does not reveal the generalization of ORCA since ORCA always knows all agents' radii before computing the collision-free velocity.

We have also demonstrated the proposed method on real robots where each robot is mounted with a Hokuyo URG-04LX-UG01 2D lidar sensor. In Figure~\ref{fig:real-robot}, four robots, three on the right side and one on the left side, are moving to their antipodal positions. As we observe, each robot can effectively avoid collisions with other robots during the navigation in a complete distributed manner. In addition, our system does not require any AR tags and/or additional motion capture systems to offer each agent with the position and/or velocity information about the other agents. In this way, our system can achieve real decentralized multi-agent navigation without any centralized components. 

\begin{figure}[h] 
\centering
\begin{subfigure}{0.22\textwidth}
\includegraphics[width=3.5cm,height=3.5cm]{fig/long12a.pdf}
\caption{Ours -- same $\textsc{radius}$}
\label{fig:ours-same}
\end{subfigure}
\begin{subfigure}{0.22\textwidth}
\includegraphics[width=3.5cm,height=3.5cm]{fig/long12b.pdf}
\caption{Ours -- different $\textsc{radius}$}
\label{fig:ours-diff}
\end{subfigure} \\
\centering
\begin{subfigure}{0.215\textwidth}
\includegraphics[width=3.325cm,height=3cm]{fig/long12c.pdf}
\caption{ORCA -- same $\textsc{radius}$}
\label{fig:rvo-same}
\end{subfigure}
\begin{subfigure}{0.215\textwidth}
\includegraphics[width=3.325cm,height=3cm]{fig/long12d.pdf}
\caption{ORCA -- different $\textsc{radius}$}
\label{fig:rvo-diff}
\end{subfigure}
\caption{Our method generalizes well to situations where agents have different physical sizes. Four agents have the same size in (a) and have different $\textsc{radius}$ in (b). In (b), one of four agents has a larger size than others. For ORCA (c) and (d), the paths of different-sized agents and same-sized agents are similar, since ORCA explicitly knows all agents' radii and velocities before calculating a steering command.}
\label{fig:size}
\vspace*{-0.1in}
\end{figure}

\begin{figure} 
\centering
\includegraphics[width=1\linewidth,height=4.5cm]{fig/long13.pdf}
\caption{A real-robot experiment in 1 vs. 3 scenario. We use different colors to distinguish the trajectories of different agents.}%More videos and experiments are available in https://sites.google.com/view/deepmaca.
\label{fig:real-robot}
\vspace*{-0.2in}
\end{figure}
\section{Conclusions}
In this paper, we set out to address the problem of multi-tasking robots in multi-robot tasks. 
%A fundamental limitation of existing multi-robot systems was addressed by the removal of a restrictive assumption that was often made--robots are single-tasking.
%Our method allowed coalitions to overlap thus enabling multi-tasking robots. 
We observed that the key underlying challenge was to reason about the physical constraints that could be synergistically satisfied.
%which directly affected the feasibility of multi-tasking.
To address the challenge, we developed our method based on the information invariant theory and modeled constraints as information instances. 
%This allowed us to reason about the relationships between constraints by reasoning about those between information requirements. 
Thereby, a formal and general framework to achieve multi-tasking robots was developed. 
We showed that our algorithm was sound and complete under our problem settings. 
%Our method was integrated with a simple greedy heuristic for task allocation.
Simulation  results  were  provided  to  show  the  effectiveness  of  our approach under resource-constrained situations and in handling challenging situations. % in a multi-UAV simulator. 

% The idea of multi-tasking is attractive in many ways. 
% Humans are living in multi-tasking environments--at any point of time, 
% we are optimizing for more than one task. 
% Multi-task often leads to more efficient task performance since it allows us to exploit task synergies. 
% The work presented in this paper takes us one step forward in realizing multi-tasking robots. 
% In particular, we started looking at the feasibility of multi-tasking. 
% There are many potential directions to pursue along this direction. First, several limitations are present in the current approach. 
% For example, although our method guarantees that there exists a physical configuration that satisfies all the constraints, it does not explicitly take the environmental influence into account. For example, a narrow corridor may prevent a robot formation from passing through, even though all the constraints for the formation do not introduce any conflicts. In this sense, our work should better be characterized as establishing a necessary condition for multi-tasking. Also, our method is mainly focused on the ``{\it planning}'' phase and hence does not address how the robots reach the desired configuration and maintain the constraints. These issues are assumed to be handled by the execution layer.

% More generally, the question of how to execute the tasks with overlapping coalitions is not addressed in this work. 
% As we already discussed, executing individual tasks with non-overlapping coalitions is straightforward but task synergies impose additional requirements on the task execution: how should the robots that are assigned multiple tasks execute them? Should they consider them in a prioritized strategy~\cite{van2005prioritized}? Or should they combine the different tasks in a way that is similar to motor schemas~\cite{arkin2}. 
% Communication requirements for maintaining the constraints must also be taken into account. How should the robots optimize their communication to improve the task performance? 

% The stringency of the physical constraints is another interesting question. It may be desirable to relax the constraints in certain situations (e.g., due to environmental influences). In such cases, it may be important to consider the problem where the constraints are least violated~\cite{kim2012revision}, or specify task constraints in different ways to increase the diversity of the configurations~\cite{srivastava2007domain} so as to make it robust to different environments. 


{\small
\bibliographystyle{IEEEtran}
\bibliography{references}
}


\end{document}












