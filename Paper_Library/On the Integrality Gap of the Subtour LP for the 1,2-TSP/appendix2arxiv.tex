\section{Proof of Lemma~\ref{lem:pure-nonpure}} \label{app:purenonpure}
We now prove Lemma~\ref{lem:pure-nonpure} from Section~\ref{sec:better}.
%\setcounter{lemma}{2}
\begin{lemma}
If $OPT(\SUBT) < n+1$, then the difference between the cost of the 2M used and the tour constructed by the Papadimitriou-Yannakakis algorithm can be upper bounded by $\alpha n_\text{pure} + \beta (n_\text{non-pure}-\ell)$, where
$n_\text{pure}$ is the number of nodes in pure cycles in the 2M, $n_\text{non-pure}$ is the number of nodes in the non-pure cycle, and $\ell$ is the number of edges of cost 2 in the non-pure cycle, 
for any values of $\alpha, \beta$ so that $9\alpha \ge 2$ and $3\alpha + 2\beta \ge 1$.
\end{lemma}

\begin{proof}
Recall from Section~\ref{sec:12gap} that the Papadimitriou-Yannakakis algorithm starts by finding a maximum cardinality bipartite matching in a graph which has a node for each pure cycle on one side, and a node for each node in the instance on the other side. There is an edge $(C,i)$ if $i\not\in C$, and there exists some node $j$ in $C$ such that $(i,j)$ is an edge of cost 1.

In Lemma~\ref{lem:PYsubt}, we show that $OPT(\SUBT)\geq n+r$, where $r$ is the number of pure cycles that are not matched in the maximum cardinality bipartite matching.
Hence, the assumption that $OPT(\SUBT)<n+1$ implies that all the pure cycles are matched.
In order to show that this implies the lemma, we will repeat some key parts of the algorithm and analysis of Papadimitriou and Yannakakis.

Consider the directed graph $F=({\cal C},A)$ which has a node for every cycle in the 2M, and an arc $(C,C')$ if the maximum cardinality bipartite matching contains an edge from cycle $C$ to a node $i$ in cycle $C'$. 
Each node in $F$ that corresponds to a pure cycle has outdegree 1, and the non-pure cycle (if it exists) has outdegree 0.
Papadimitriou and Yannakakis show how to find a spanning subgraph of $F'$ of $F$ such that each nontrivial component is an in-tree of depth one or a path of length two. The only possible trivial component is the node that corresponds to the non-pure cycle.
Since the non-pure cycle has outdegree 0, it can only occur in a nontrivial component as the root of an in-tree, or as the endpoint of a path of length two. It turns out that the latter does not happen in the construction described by Papadimitriou and Yannakakis, but even if it did, we could just remove the last edge in the length-two path to obtain one in-tree of depth one and one trivial component containing the non-pure cycle. Hence, we may assume the non-pure cycle only occurs in a nontrivial component as the root of an in-tree.



Papadimitriou and Yannakakis now merge the cycles in one component of $F'$ into a single cycle containing at least one edge of cost 2 as follows:
If the component is an in-tree of depth one, let $C$ be the cycle corresponding to the root, let $C_1, \ldots, C_m$ be the remaining cycles in the component, and let $v_i$ be the node in $C$ such that $(C_i,v_i)$ was in the bipartite matching. 
We consider the nodes in $C$ in clockwise order, starting from a node $v\neq v_i$ for $i=1, \ldots, m$ if such a node exists, and an arbitrary node $v$ otherwise. If we encounter two adjacent nodes $v_i,v_j$ in $\{v_1, \ldots, v_m\}$, then we merge the corresponding cycles $C_i$ and $C_j$ with $C$ according to (a) in Figure~\ref{fig:PY}. Otherwise, if the current node is $v_j$ but its clockwise neighbor is not or if its clockwise neighbor is the first node $v$, then we merge $C_j$ with $C$ as in (b) in Figure~\ref{fig:PY}. Finally, if the component is a path of length two, we merge the three cycles as in (c) in Figure~\ref{fig:PY}. Note that each cycle in the resulting graph contains at least one edge of cost 2, and hence we can find a tour of the same cost by removing the edges of cost 2, and arbitrarily connecting the resulting paths into a tour.




\newlength{\arrowlength}
\settowidth{\arrowlength}{$\rightarrow$}
\newlength{\tw}
\setlength{\tw}{.83\textwidth}
\begin{figure}
\begin{center}
\subfloat[ ]{
\begin{tabular}{cp{\arrowlength}c}
\begin{minipage}[c]{.207\tw}
\includegraphics[width=.207\tw]{figures/PY2a1.pdf}
\end{minipage}
&
\begin{minipage}[c]{\arrowlength} $\rightarrow$ \end{minipage}
&
\begin{minipage}[c]{.207\tw}
\includegraphics[width=.207\tw]{figures/PY2a2.pdf}
\end{minipage}
\end{tabular}
}
\qquad
\subfloat[ ]{
\begin{tabular}{cp{\arrowlength}c}
\begin{minipage}[c]{.19\tw}
\includegraphics[width=.19\tw]{figures/PY2b1.pdf}
\end{minipage}
&
\begin{minipage}[c]{\arrowlength} $\rightarrow$ \end{minipage}
&
\begin{minipage}[c]{.19\tw}
\includegraphics[width=.19\tw]{figures/PY2b2.pdf}
\end{minipage}
\end{tabular}
}
\\
\subfloat[ ]{
\begin{tabular}{cp{\arrowlength}c}
\begin{minipage}[c]{.26\tw}
\includegraphics[width=.26\tw]{figures/PY2c1.pdf}
\end{minipage}
&
\begin{minipage}[c]{\arrowlength} $\rightarrow$ \end{minipage}
&
\begin{minipage}[c]{.26\tw}
\includegraphics[width=.26\tw]{figures/PY2c2.pdf}
\end{minipage}
\end{tabular}
}
\end{center}
\caption{The three cases for merging cycles in \cite{PapadimitriouY93} (Figure 2 in \cite{PapadimitriouY93}). The black edges indicate edges of cost 1, and the grey edges (potentially) have cost 2.}\label{fig:PY}
\end{figure}


We now show that the number of edges of cost 2 that are added by merging cycles according to Figure~\ref{fig:PY} can be upper bounded by $\alpha n_\text{pure}+ \beta (n_\text{non-pure}-\ell)$, provided that $\alpha$ and $\beta$ are so that $9\alpha \ge 2$ and $3\alpha + 2\beta \ge 1$.


We say a node is involved in a merging if it is either a node in one of the cycles that are fully drawn in Figure~\ref{fig:PY}, or if it is node $v$ or $v_i$ in subfigure (b). Note that each node is involved in at most one merging. 
Recall that the non-pure cycle can only occur as the root of a 1-tree of depth one or as a trivial component in $F'$, and hence, only the partially drawn cycle in (a) and (b) is (potentially) a non-pure cycle.

We now examine each of the cases (a), (b) and (c) in Figure~\ref{fig:PY} in turn.
In Figure~\ref{fig:PY} (a) one edge of cost 2 is added and we can charge this edge to the (at least) 6 nodes from pure cycles involved in this merging, as long as $6\alpha \geq 1$. This is indeed the case, because we have the stronger requirement that $9\alpha \geq 2$. In (b), again, one edge of cost 2 is added, and we can charge the edge to the (at least) three nodes of the pure cycle involved in the merging and the 2 nodes of the (potentially) non-pure cycle involved in the merging, as long as $3\alpha + 2\beta \ge 1$ (in case the 2 nodes were part of the non-pure cycle), and $5\alpha \geq 1$ (in case the 2 nodes were part of a pure cycle). Finally, in Figure~\ref{fig:PY} (c), two edges of cost 2 are added; we can charge the two edges to the (at least) nine nodes from pure cycles involved in the merging as long as $9 \alpha \geq 2$.

Hence, we have shown that difference in cost between the tour and the 2M can be charged to the nodes, in such a way that each node is charged at most once, and a node in a pure cycle is charged at most $\alpha$ and a node in a non-pure cycle is charged at most $\beta$.

Finally, we remark that a node in a non-pure cycle is charged only in case (b). Now, if $(v_i,v)$ in Figure~\ref{fig:PY}(b) is an edge of cost 2, then there is no need to charge any nodes, since the cost after merging is the same as before the merge. Hence, if we direct all edges of the non-pure cycle in clockwise direction, then the head of the edges of cost 2 is never charged. The total chage to the nodes in the non-pure cycle is therefore at most $\beta(n_\text{non-pure}-\ell)$.
\mqed\end{proof}

\section{Proof of Corollary~\ref{cor:12biconn}} \label{app:twolemmas}
We now show that the worst-case integrality gap for the subtour LP for the 1,2-TSP can be found on graphs of cost 1 edges that are biconnected, as stated in Corollary~\ref{cor:12biconn} in Section~\ref{sec:comp}.  Let $OPT(G)$ and $\SUBT(G)$ be the cost of the optimal tour and the value of the subtour LP (respectively) when $G$ is the graph of cost 1 edges.
We start by proving that the worst case is obtained on a connected graph.
%\addtocounter{lemma}{1}
\begin{lemma} \label{lem:12connect}
Let $G=(V,E)$ be the graph of cost 1 edges in a 1,2-TSP instance.  Then if $G=(V,E)$ is not connected, there exists a connected graph $G'=(V',E')$ with $|V'|=|V|+1$ such that $OPT(G)/\SUBT(G) \leq OPT(G')/\SUBT(G')$.
\end{lemma}

\begin{proof}
Suppose $G$ has more than one connected component. We create $G'=(V',E')$ by adding a new vertex $i^*$ to the graph, and adding edges from all $j \in V$ to $i^*$ so that $V' = V \cup \set{i^*}$ and $E' = E \cup \set{(i^*,j):  j \in V}$.  Given a tour of $G'$, we can easily produce a tour of $G$ of no greater cost by shortcutting $i^*$, so that $OPT(G) \leq OPT(G')$.  Let $x$ be an optimal solution to the subtour LP for the graph $G$.  We now define a solution $x'$ for $G'$, where $x'_{ij} = x_{ij}$ if $i$ and $j$ are in the same connected component of $G$, while if $i$ and $j$ are in different connected components of $G$, then we set $x'_{ij} = 0$, $x'_{i^*i}=x_{ij}$, and $x'_{i^*j} = x_{ij}$.  It is easy to see that the cost of $x'$ is the same as that of $x$.  We now argue that there is some solution $x''$ feasible for the subtour LP on $G'$ such that its cost is no greater, so that $\SUBT(G') \leq \SUBT(G)$.  It is clear that the bounds constraints (\ref{boundscons}) are satisfied for $x'$ and the degree constraints (\ref{degreecons}) are satisfied for $x'$ for all $i \in V$; however, the degree constraint for $i^*$ may not be satisfied.  Since for any component $C \subseteq V$ of $G$, $x(\delta(C)) \geq 2$, it is clear that $x'(\delta(i^*)) \geq 2$, but it may be the case that $x'(\delta(i^*)) > 2$.  For the subtour constraints (\ref{subtourcons}), consider any $S \subset V'$, $S \neq \emptyset$, such that $i^* \notin S$.  Then $x'(\delta(S)) \geq x(\delta(S)) \geq 2$, and for any $S \subseteq V'$ with $i^* \in S$, $S \neq \{i^*\}$, $x'(\delta(S)) = x'(\delta(V'-S)) \geq 2$ by the previous argument.  Finally, Goemans and Bertsimas~\cite{GoemansB90} have shown (see also Williamson~\cite{Williamson90}) that if edge costs obey the triangle inequality, and there is some solution $x'$ to the subtour LP in which degree constraints are exceeded but all other constraints are met, then there is another feasible solution $x''$ of no greater cost in which all constraints are satisfied.  Hence we have that $\SUBT(G') \leq \SUBT(G)$. Thus we have that $OPT(G)/\SUBT(G) \leq OPT(G')/\SUBT(G')$.
\mqed\end{proof}

\begin{lemma}\label{lem:12biconn}
Let $G=(V,E)$ be the graph of cost 1 edges in a 1,2-TSP instance.  Then if $G=(V,E)$ is connected but not biconnected, there exists a biconnected $G'=(V',E')$ with $|V'|=|V|+1$ such that $OPT(G)/\SUBT(G) \leq OPT(G')/\SUBT(G')$.
\end{lemma}

\begin{proof}
By hypothesis we assume that the graph $G=(V,E)$ is connected.  Let $i_1, \ldots, i_k$ be all the cut vertices of $G$, and let $C_1,\ldots, C_\ell$ be all the connected components formed when these vertices are removed, so that $C_1,\ldots,C_\ell, \set{i_1},\ldots, \set{i_k}$ form a partition of $V$.  We create a new graph $G'=(V',E')$ by adding a new vertex $i^*$, and adding edges from $i^*$ to each vertex in $C_1 \cup \cdots \cup C_\ell$, so that $V' = V \cup \set{i^*}$ and $E' = E \cup \set{(i^*,j): j \in C_p \mbox{ for some } p}$. We note that $G'$ is biconnected. As before, we have $OPT(G) \leq OPT(G')$ since given a tour of $G'$ we can shortcut $i^*$ to get a tour of $G$.  Let $x$ be an optimal subtour LP solution for graph $G$.  We now argue, as we did in the proof of Lemma~\ref{lem:12connect}, that we can create an $x'$ that costs no more than $x$ such that all the subtour and bounds constraints are obeyed, and all degree constraints are either met or exceeded; this will imply that $\SUBT(G') \leq \SUBT(G)$, and complete the proof.  Suppose without loss of generality that removing cut vertex $i_1$ creates components $C_1$ and $C = C_2 \cup \cdots \cup C_\ell \cup \set{i_2} \cup \cdots \cup \set{i_k}$, so that $C_1$, $\set{i_1}$, and $C$ partition $V$.  We set $x'_{ij} = 0$ and $x'_{i^*i}=x'_{i^*j}=x_{ij}$ if $i \in C_1$ and $j \in C$; $x'_{ij} = x_{ij}$ otherwise.  If $i \in C_1$ and $j \in C$, then $(i,j) \notin E$ since $i_1$ is a cut vertex, so the cost of $x'$ is no more than that of $x$.  The arguments that all constraints are satisfied except for the degree constraint on $i^*$ follow as in the proof of Lemma~\ref{lem:12connect}.  We now must argue that $x'(\delta(i^*)) \geq 2$.  To do this, we show that $\sum_{i \in C_1, j \in C} x_{ij} \geq 1$.  Since $x(\delta(i_1)) = 2$, it must be the case that either $\sum_{j \in C} x_{i_1j} \leq 1$ or $\sum_{j \in C_1} x_{i_1j} \leq 1$; without loss of generality we assume the former is true.  Then since $x(\delta(C_1 \cup \set{i_1})) \geq 2$, and $x(\delta(C_1 \cup \set{i_1})) = \sum_{j \in C} x_{i_1j} + \sum_{i \in C_1, j \in C} x_{ij}$, it follows that $\sum_{i \in C_1, j \in C} x_{ij} \geq 1$, and the proof is complete.
\mqed\end{proof}


