\section{Introduction}
The Traveling Salesman Problem (TSP) is one of the most well studied problems in combinatorial optimization. Given a set of cities $\{1, 2, \ldots, n\}$, and distances $c( i, j )$ for traveling from city $i$ to $j$, the goal is to find a tour of minimum length that visits each city exactly once.
An important special case of the TSP is the case when the distance forms a metric, i.e., $c(i,j)\le c(i,k) + c(k,j)$ for all $i,j,k$, and all distances are symmetric, i.e., $c(i,j)=c(j,i)$ for all $i,j$. The symmetric TSP is known to be NP-hard, even if $c(i,j)\in \{1,2\}$ for all $i,j$~\cite{PapadimitriouY93}; note that such instances trivially obey the triangle inequality.  Such instances are also known to be APX-hard; that is, there is no $\alpha$-approximation algorithm for the problem for some $\alpha > 1$ unless $P = NP$.

The metric TSP can be approximated to within a factor of $\tfrac 32$ using an algorithm by Christofides~\cite{Christofides76} from 1976.  The algorithm combines a minimum spanning tree with a matching on the odd-degree nodes to get an Eulerian graph that can be shortcut to a tour; the analysis shows that the minimum spanning tree and the matching cost no more than the optimal tour and half the optimal tour respectively.
Better results are known for several special cases, but, surprisingly, no progress has been made on approximating the general symmetric TSP in more than thirty years. A natural direction for trying to obtain better approximation algorithms is to use linear programming. The following linear programming relaxation of the traveling salesman problem was used by Dantzig, Fulkerson, and Johnson~\cite{DantzigFJ54} in 1954.  For simplicity of notation, we let $G=(V,E)$ be a complete undirected graph on $n$ nodes.  In the LP relaxation, we have a variable $x(e)$ for all $e = (i,j)$ that denotes whether we travel directly between cities $i$ and $j$ on our tour.  Let $c(e) = c(i,j)$, and let $\delta(S)$ denote the set of all edges with exactly one endpoint in $S \subseteq V$.
Then the relaxation is \lps & &
& \mbox{Min} & \sum_{e \in E} c(e) x(e) \\
(\SUBT) & \mbox{subject to:} & & & \sum_{e \in \delta(i)} x(e) = 2, & \forall i \in V, \numb{degreecons} \\
& & & & \sum_{e \in \delta(S)}x(e) \geq 2, & \forall S\subset V,\, 3 \leq |S| \leq |V|-3, \numb{subtourcons}\\
& & & & 0 \leq x(e) \leq 1, & \forall e \in E. \numb{boundscons} \elps The first
set of constraints (\ref{degreecons}) are called the {\em degree constraints}.  The second set of constraints (\ref{subtourcons}) are sometimes called {\em subtour elimination constraints} or sometimes just {\em subtour constraints}, since they prevent solutions in which there is a subtour of just the nodes in $S$. As a result, the linear program is sometimes called the {\em subtour LP}.
It has been shown by Wolsey~\cite{Wolsey80} (and later Shmoys and Williamson~\cite{ShmoysW90}) that Christofides' algorithm finds a tour of length at most~$\tfrac 32$ times the optimal value of the subtour LP; these proofs show that the minimum spanning tree and the matching on odd-degree nodes can be bounded above by the value of the subtour LP, and half the value of the subtour LP, respectively.  This implies that the integrality gap, the worst case ratio of the length of an optimal tour divided by the optimal value of the LP, is at most~$\tfrac 32$. However, no examples are known that show that the integrality gap can be as large as~$\tfrac 32$; in fact, no examples are known for which the integrality gap is greater than $\tfrac 43$. A well known conjecture states that the integrality gap is indeed $\tfrac 43$; see (for example) Goemans~\cite{Goemans95}.

Recently, progress has been made in several directions, both in improving the best approximation guarantee and in determining the exact integrality gap of the subtour LP for certain special cases of the symmetric TSP.
In the {\em graph}-TSP, the costs $c(i,j)$ are equal to the shortest path distance in an underlying unweighted graph.  If the graph is cubic and 3-connected, Gamarnik, Lewenstein and Sviridenko~\cite{GamarnikLS05} show an approximation algorithm with guarantee slightly better than $\tfrac 32$.
Oveis Gharan, Saberi, and Singh~\cite{OveisGharanSS10} show that the graph-TSP can be approximated to within $\tfrac 32-\epsilon$ for a small constant $\epsilon>0$ for all graphs.
Boyd, Sitters, van der Ster and Stougie~\cite{BoydSSS11}, and Aggarwal, Garg and Gupta~\cite{AggarwalGG11} independently give a $\tfrac 43$-approximation algorithm if the underlying graph is cubic.
M\"omke and Svensson~\cite{MomkeS11} improve these results by giving a 1.461-approximation for the graph-TSP and an $\tfrac 43$-approximation algorithm if the underlying graph is subcubic. Mucha~\cite{Mucha11} improves the analysis of the M\"omke-Svensson algorithm to a  $\tfrac{13}{9}$-approximation algorithm, and Seb\H o and Vygen~\cite{SeboV12} combine the ideas of M\"omke and Svensson~\cite{MomkeS11} with an algorithm based on a carefully chosen ear decomposition of the graph to obtain a $\tfrac75$-approximation algorithm.  All of these $\alpha$-approximation algorithms imply a corresponding upper bound of $\alpha$ on the integrality gap for the corresponding graph-TSP instances.

In Schalekamp, Williamson and van Zuylen~\cite{SchalekampWvZ11}, three of the authors of this paper resolve a related conjecture. A 2-matching of a graph is a set of edges such that no edge appears twice and each node has degree two, i.e., it is an integer solution to the LP $(\SUBT)$  with only constraints (\ref{degreecons}) and (\ref{boundscons}). Note that a minimum-cost 2-matching thus provides a lower bound on the length of the optimal TSP tour. A minimum-cost 2-matching can be found in polynomial time using a reduction to a certain minimum-cost matching problem. Boyd and Carr~\cite{BoydC11} conjecture that the worst case ratio of the cost of a minimum-cost 2-matching and the optimal value of the subtour LP is at most $\tfrac {10}9$. This conjecture was proved to be true by Schalekamp et al.\ and examples are known that show this result is tight.

Unlike the techniques used to obtain better results for the graph-TSP, the techniques of Schalekamp et al.\ work on general weighted instances that are symmetric and obey the triangle inequality. However, their results only apply to 2-matchings and it is not clear how to enforce global connectivity on the solution obtained by their method. A potential direction for progress on resolving the integrality gap for the subtour LP is a conjecture by Schalekamp et al.\ that the worst-case integrality gap is attained for instances for which the optimal subtour LP solution is a basic solution to the linear program obtained by dropping the subtour elimination constraints.


In this paper, we turn our attention to the 1,2-TSP, where $c(i,j)\in\{1,2\}$ for all $i,j$. Note that bounding the cost of enforcing connectivity is relatively easy in this class of problems, since we may connect any two components for an increase in cost of at most 2. Papadimitriou and Yannakakis~\cite{PapadimitriouY93} show how to approximate 1,2-TSP within a factor of $\tfrac{11}9$ by computing a minimum-cost 2-matching and merging its cycles into a tour. In addition, they show a ratio of $\tfrac 76$ if they start with a minimum-cost 2-matching that has no cycles of length 3. Bl\"aser and Ram~\cite{BlaeserR05} improve this ratio and the best known approximation factor of $\tfrac 87$ is given by Berman and Karpinski~\cite{BermanK06}.

We do not know a tight bound on the integrality gap of the subtour LP even in the case of the 1,2-TSP.  As an upper bound, we appear to know only that the gap is at most $\tfrac 32$ via Wolsey's result.   There is an easy 9 city example showing that the gap must be at least $\tfrac{10}9$; see Figure~\ref{fig:12tsp}.  This example has been extended to a class of instances on $9k$ nodes for any positive integer $k$ by Williamson~\cite{Williamson90}.  The contribution of this paper is to  begin a study of the integrality gap of the 1,2-TSP, and to improve our state of knowledge for the subtour LP in this case. We prove an upper bound on the integrality gap for the subtour LP of $\tfrac 54$, which is the first bound on the integrality gap with value less than $\tfrac 43$ for a natural class of TSP instances. Under an analog of a conjecture of Schalekamp et al.~\cite{SchalekampWvZ11}, we show that the integrality gap is at most $\tfrac 76$, and with an additional assumption on the structure of the solution, we can improve this bound to $\tfrac{10}9$. We describe these results in more detail below.



\begin{figure}[t]
\begin{center}
\includegraphics[height=.75in]{figures/bad910}
\end{center}
\caption{Illustration of the worst example known for the integrality gap for
the 1,2-TSP.  The figure on the
left shows all edges of cost 1.  The figure in the center gives the subtour LP
solution, in which the dotted edges have value $\tfrac 12$, and the solid
edges have value 1; this is also an optimal fractional 2-matching.  The figure on the left gives the optimal
tour and the optimal 2-matching.} \label{fig:12tsp}
\end{figure}


%We start by giving a bound on the subtour LP in the general case of 1,2-TSP.
All the known approximation algorithms since the initial work of Papadimitriou and Yannakakis~\cite{PapadimitriouY93} on the problem start by computing a minimum-cost 2-matching.
However, the example of Figure~\ref{fig:12tsp} shows that an optimal 2-matching can be as much as $\tfrac{10}9$ times the value of the subtour LP for the 1,2-TSP, so we cannot directly replace the bound on the optimal solution in these approximation algorithms with the subtour LP in the same way that Wolsey did with Christofides' algorithm in the general case. Using the result of Schalekamp, Williamson, and van Zuylen~\cite{SchalekampWvZ11} and a new lemma that relates part of the analysis of Papadimitriou and Yannakakis~\cite{PapadimitriouY93} to the subtour LP bound, we obtain a preliminary upper bound on the integrality gap of the subtour LP for the 1,2-TSP of $\tfrac{7}{9}\cdot\tfrac{10}{9}+\tfrac{4}{9} = \tfrac{106}{81} \approx 1.3086.$ 

To improve this upper bound to $\tfrac{5}{4}$, we first show stronger results in some cases.
A fractional 2-matching is a basic optimal solution to the LP $(\SUBT)$ with only constraints (\ref{degreecons}) and (\ref{boundscons}).
Schalekamp et al.~\cite{SchalekampWvZ11} have conjectured that the worst-case integrality gap for the subtour LP is obtained when the optimal solution to the subtour LP is an extreme point of the fractional 2-matching polytope.
We show that if this is the case for 1,2-TSP
%, i.e. in the worst case there is an optimal solution to the subtour LP that is a basic solution to the fractional 2-matching problem,
then we can find a tour of cost at most $\tfrac 76$ the cost of the fractional 2-matching, implying that the integrality gap is at most $\tfrac 76$ in these cases.
Next, we show that if this optimal solution to the fractional 2-matching problem has a certain structure, then we can find a tour of cost at most $\tfrac{10}9$ times the cost of the fractional 2-matching, implying an upper bound on the integrality gap of $\tfrac{10}9$ for these cases.  Figure~\ref{fig:12tsp} shows that this result is tight.  

We then use the previous arguments to show that one can construct a tour of cost at most $\tfrac{5}{4}$ times the subtour LP value. To do this, we prove that we can assume without loss of generality that the optimal value of the subtour LP is less than $n+1$, where $n$ denotes the number of nodes. Combined with a more careful analysis based on the results obtained before, we obtain our main result.
The results above all lead to polynomial-time algorithms, though we do not state the exact running times.


Finally, we perform computational experiments to show that the integrality gap is at most $\tfrac{10}9$ for $n \leq 12$.  We conjecture that the integrality gap is in fact exactly $\tfrac{10}9$.

We note that the upper bound on the integrality gap for general 1,2-TSP instances presented in this paper is stronger than the bound that appeared in a preliminary version of this paper~\cite{QianSWvZ11} of $\tfrac{19}{15}$. In the time between publication of the preliminary version and the current revision, Mnich and M\"omke~\cite{MnichM13} obtained an upper bound of $\tfrac{5}{4}$ on the integrality gap for 1,2-TSP instances that have the additional property of being ``fractionally Hamiltonian'', which means that the optimal objective value of the subtour LP is equal to the number of nodes in the instance. In this version, using the same techniques as in the preliminary version, we show an unconditional upper bound on the integrality gap of $\tfrac{5}{4}$, and a bound of $\tfrac{26}{21}$ for fractionally Hamiltonian instances.


The remainder of this paper is structured as follows.  Section~\ref{sec:12gap} contains preliminaries and a first general bound on the integrality gap for the 1,2-TSP.  We show how to obtain stronger bounds if the optimal subtour LP solution is a fractional 2-matching in Section~\ref{sec:12-2m}. In Section~\ref{sec:better}, we combine the arguments from the previous sections and show that the integrality gap without any assumptions on the structure of the subtour LP solution is at most $\tfrac{5}{4}$. We describe our computational experiments in Section~\ref{sec:comp}.  Finally, we close with a conjecture on the integrality gap of the subtour LP for the 1,2-TSP in Section~\ref{sec:conc}.
\iftoggle{abs}{Some proofs are omitted due to space reasons and can be found in the full version. of the paper.}{}



