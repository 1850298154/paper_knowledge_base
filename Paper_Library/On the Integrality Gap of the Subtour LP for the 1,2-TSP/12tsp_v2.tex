\section{Preliminaries and a first bound on the integrality gap}
\label{sec:12gap}
We will work extensively with 2-matchings and fractional 2-matchings; that is, extreme points $x$ of the LP $(\SUBT)$ with only constraints (\ref{degreecons}) and (\ref{boundscons}), where in the first case the solutions are required to be integer. For convenience we will abbreviate ``fractional 2-matching'' by F2M and ``2-matching'' by 2M.  The basic solutions of the F2M polytope have the following well-known structure (attributed to Balinski~\cite{Balinski65}).  Each connected component of the support graph (that is, the edges $e$ for which $x(e) > 0$) is either a cycle on at least three nodes with $x(e)=1$ for all edges $e$ in the cycle, or consists of odd-sized cycles with $x(e) = \tfrac 12$ for all edges $e$ in the cycle connected by paths of edges $e$ with $x(e) = 1$ for each edge $e$ in the path (the center figure in Figure~\ref{fig:12tsp} is an example).  We call the former components {\em integer components} and the latter {\em fractional components}.  In a fractional component, we call a path of edges $e$ with $x(e)=1$ a {\em 1-path}.  The edges $e$ with $x(e)=\tfrac 12$ in cycles are called {\em cycle edges}.  An F2M with a single component is called {\em connected}, and we call a component {\em 2-connected} if the sum of the $x$-values on the edges crossing any cut is at least $2$. We let $n$ denote the number of nodes in an instance.



As mentioned in the introduction, Schalekamp, Williamson, and van Zuylen~\cite{SchalekampWvZ11} have  shown the following.

\begin{theorem}[Schalekamp et al.~\cite{SchalekampWvZ11}]
\label{thm:109}
If edge costs obey the triangle inequality, then the cost of an optimal 2-matching is at most $\tfrac{10}9$ times the value of the subtour LP.
\end{theorem}
It is not hard to show that this immediately implies an upper bound of $\tfrac 43 \times \tfrac{10}9$ on the integrality gap of the subtour LP for the 1,2-TSP: we can just compute a minimum cost 2-matching at a cost of $\tfrac{10}9$ the value of the subtour LP, remove the most expensive edge from each cycle, which gives a collection of node-disjoint paths, and add edges of cost 2 to combine these paths into a tour.  Each cycle has at least 3 edges; at worst, we remove an edge of cost 1 from each cycle and then need an edge of cost 2 to patch the paths into a tour.  Thus the overall cost increases by a factor of $\tfrac43$, giving a tour of cost at most $\tfrac43 \times \tfrac{10}9$ times the value of the subtour LP.

The algorithm of Papadimitriou and Yannakakis~\cite{PapadimitriouY93} improves on this idea, by cleverly merging the cycles of the optimal 2M solution.
We summarize the properties of their algorithm that we will use. 
First, observe that we can assume without loss of generality that the optimal 2M solution consists of a number of cycles with only edges of cost 1 (``pure'' cycles) and at most one cycle which has one or more edges of cost 2 (the ``non-pure'' cycle), by deleting the edges of cost 2 and combining the resulting disjoint paths into a single cycle.  Moreover, if $i$ is a node in the non-pure cycle which is incident on an edge of cost 2 in the cycle, then there can be no edge of cost 1 connecting $i$ to a node in a pure cycle (since otherwise, we can merge the non-pure cycle with a pure cycle without increasing the cost).

The Papadimitriou-Yannakakis algorithm solves the following bipartite matching problem: On one side we have a node for every pure cycle, and on the other side, we have a node for every node in the instance. There is an edge from pure cycle $C$ to node $i$, if $i\not\in C$ and there is an edge of cost 1 from $i$ to some node in $C$. Let $r$ be the number of pure cycles that are unmatched in a maximum cardinality bipartite matching. Papadimitriou and Yannakakis show how to ``patch together'' the matched cycles. We refer the reader to their original paper~\cite{PapadimitriouY93} for more details.
The resulting cycles are then combined into a tour of cost at most 
\begin{equation}\tfrac {7}9 OPT(2M) + \tfrac 49 n + \tfrac 13 r,\label{eq:PY}\end{equation}
where $OPT(2M)$ is the cost of an optimal 2M solution.\footnote{In~\cite{PapadimitriouY93}, $OPT(2M)$ is expressed as $n+k$, where $k$ is the number of edges of cost 2 in the optimal 2M solution. The number of unmatched pure cycles is denoted by $r_2$. The bound given by~\cite{PapadimitriouY93} is $n+k +\tfrac29(n-n_2-k) + r_2$, where $n_2$ is a quantity that is lower bounded by $3r_2$. Therefore, the bound in~\cite{PapadimitriouY93} can be upper bounded by $\tfrac79(n+k)+\tfrac 49n + \tfrac13 r_2$.}


We now show how to convert this bound into a bound in terms of the optimal value to \SUBT.
\begin{lemma}\label{lem:PYsubt}
Let $r$ be the number of pure cycles that are unmatched in a maximum cardinality bipartite matching instance defined by Papadimitriou and Yannakakis.
Then 
\[OPT(\SUBT) \ge n + r.\]
\end{lemma}

% \begin{theorem}\label{thm:intgap12}
% The integrality gap of the subtour LP is at most $\tfrac{106}{81}$ for 1,2-TSP.
% \end{theorem}
\begin{proof}
We note that for a bipartite matching instance, the size of the minimum cardinality vertex cover is equal to the size of the maximum matching. We use this fact to construct a feasible dual solution to the subtour LP that has value $n+r$.
Let ${\cal C}_M, V_M$ be the pure cycles and nodes (in the original graph), for which the corresponding nodes in the bipartite matching instance are in the minimum cardinality vertex cover.
The dual of the subtour LP $(\SUBT)$ is \lps & &
& \mbox{Max} & 2\sum_{S \subset V} y(S) + 2\sum_{i \in V} y(i) - \sum_{e \in E} z(e)\\
(D) &\mbox{subject to:} %\hspace*{-1.5cm}
& & & \sum_{S \subset V: e \in \delta(S)} y(S) + y(i) + y(j) - z(e) \leq c(e), & \forall e = (i,j), \\
& & & & y(S) \geq 0, &\hspace*{-2.25cm} \forall S\subset V,\, 3 \leq |S| \leq n-3,\\
& & & & z(e) \geq 0, &\hspace*{-2.25cm} \forall e \in E. \elps
We set $z(e) = 0$ for each $e \in E$, and we set $y(i) =\tfrac 12$ for each $i\in V\backslash V_M$. For a pure cycle on a set of nodes $C$, we set $y(C)=\tfrac12$, if the cycle is not in ${\cal C}_M$. The dual objective for this solution is exactly $n+r$: its value is $n$ plus the number of pure cycles minus the size of the vertex cover, or $n$ plus the number of pure cycles minus the size of the matching, since the vertex cover has the same size as the matching.  Thus it is the same as $n$ plus the number of pure cycles not in the matching, or $n+r$.

It remains to show that the dual constructed is feasible.  Define the {\em load} on an edge $e = (i,j)$ of solution $(y,z)$ to be $\sum_{S \subset V: e \in \delta(S)} y(S) + y(i) + y(j) - z(e)$.  For any edge $e=(i,j)$  of cost 1 inside a cycle of the 2M, the load on the edge is at most 1, since the only potentially non-zero dual variables loading the edge are the dual variables $y(i)$ and $y(j)$. For an edge $(i,j)$ where $i\in C$ and $j\in C'\neq C$, the load is $y(i)+y(j)+y(C)+y(C')\le 2$.
Suppose $(i,j)$ has cost 1, and the cycles $C$ and $C'$ are both pure cycles. Then the edge occurs twice in the bipartite matching instance (namely, once going from $i$ to $C$ and once going from $j$ to $C'$) and hence the dual of at least two of the four objects $i,j, C$ and $C'$ has been reduced to 0. The total load on edge $(i,j)$ is thus at most 1.
Now, suppose $C'$ is the non-pure cycle, then $y_{C'}=0$, since we only increased the dual variables for the pure cycles. Moreover, at least one endpoint of the $(j,C)$ edge in the bipartite matching instance must be in the vertex cover, so the load on edge $(i,j)$ is again at most 1.
%\qed
\end{proof}

Note that, combined with (\ref{eq:PY}) and Theorem \ref{thm:109}, Lemma~\ref{lem:PYsubt} implies that the cost of the tour is at most $ \tfrac{7}{9} \cdot \tfrac{10}{9} OPT(\SUBT) + \tfrac{4}{9} OPT(\SUBT)=\tfrac {106}{81}OPT(\SUBT).$
This bound obtained on the integrality gap seems rather weak, as the best known lower bound on the integrality gap is only $\tfrac{10}9$.
Schalekamp, Williamson, and van Zuylen~\cite{SchalekampWvZ11} have conjectured that the integrality gap (or worst-case ratio) of the subtour LP occurs when the solution to the subtour LP is a fractional 2-matching.

\begin{conj}[Schalekamp et al.~\cite{SchalekampWvZ11}] \label{conj:tsp}
Let $\alpha_n$ be the integrality gap of the subtour LP on $n$ vertices.  Then there exists an instance which has an optimal subtour LP solution that is an F2M and for which the optimal tour has cost at least $\alpha_n$ times the subtour LP cost.
\end{conj}


In the next section, we show that we can obtain better bounds on the integrality gap of the subtour LP in the case that the optimal solution is a fractional 2-matching.
In Section~\ref{sec:better}, we then show how to combine Lemma~\ref{lem:PYsubt} with the bounds in the next section to obtain an upper bound of $\tfrac{5}{4}$ on the integrality gap.

\section{Better bounds if the optimal solution is an F2M}
\label{sec:12-2m}
If the optimal solution to the subtour LP is a fractional 2-matching, then a natural approach to obtaining a good tour is to start with the edges with $x$-value 1, and add as many edges of cost 1 and $x$-value $\tfrac 12$ as possible, without creating a cycle on a subset of the nodes. In other words, we will propose an algorithm that creates an acyclic spanning subgraph $(V,T)$ where all nodes have degree one or two.
We will refer to an acyclic spanning subgraph in which all nodes have degree one or two as a partial tour.
A partial tour can be extended to a tour by adding $d/2$ edges of cost $2$, where $d$ is the number of degree $1$ nodes. The cost of the tour is $c( T ) + d$, where $c( T ) = \sum_{e \in T} c(e)$.




\iftoggle{abs}{}{We will use the following lemma.}
\begin{lemma}\label{lem:third}
Let $G=(V,T)$ be a partial tour.
Let $A$ be a set of edges not in $T$ that form an odd cycle or a path on $V'\subset V$, where the nodes in $V'$ have degree one in $T$.
We can find $A'\subset A$ such that $(V,T\cup A')$ is a partial tour, and
\begin{itemize}
\item
 $|A'|\ge \tfrac 13 |A|$ if $A$ is a cycle,
 \item
 $|A'|\ge \tfrac 13 (|A|-1)$ if $A$ is a path,
 \end{itemize}
\end{lemma}

\iftoggle{abs}{We will now use the lemma above to show a bound of $\tfrac 76$ on the integrality gap if the optimal subtour LP solution is a fractional 2-matching.}{We postpone the proof of the lemma and first prove the implication for the bound on the integrality gap if the optimal subtour LP solution is a fractional 2-matching.}

\begin{theorem}\label{thm:tour76}
There exists a tour of cost at most $\tfrac76$ times the cost of a connected F2M solution if $c(i,j)\in \{1,2\}$ for all $i,j$.
\end{theorem}
\begin{proof}
Let $P = \{ e \in E : x(e) = 1 \}$ (the edges in the 1-paths of $x$).
We will start the algorithm with $T = P$. Let $R = \{ e \in E : x(e) = \tfrac 12 \mbox{ and } c(e) = 1 \}$ (the edges of cost $1$ in the cycles of $x$). 
Note that the connected components of the graph $(V,R)$ consist of paths and odd cycles. The main idea is that we consider these components one by one, and use Lemma~\ref{lem:third} to show that we can add a large number of the edges of each path and cycle, where we keep as an invariant that $T$ is a partial tour.
Note that by Lemma~\ref{lem:third}, the number of edges added from each path or cycle $A$ is at least $|A|/3$, except for the paths for which $|A| \equiv 1 \pmod 3$. Let ${\cal P}_1$ be this set of paths.
We would like to claim that we add a third of the edges on average from each component, and we therefore preprocess the paths in ${\cal P}_1$, where we add one edge (either the first or last edge from each path in ${\cal P}_1$) to $T$ if this is possible without creating a cycle in $T$, and if so, we remove this edge and its neighboring edge in $R$ (if any) from $R$. After the preprocessing, we use Lemma~\ref{lem:third} to process each of the components in $(V,R)$.

We call a path $A$ in ${\cal P}_1$ ``eared'' if the 1-paths that are incident on the first and last node of the path are such that they go between two neighboring nodes of $A$. \iftoggle{abs}{It is not hard to show that we can add an edge from at least half of the paths in ${\cal P}_1$ that are not eared.}{It is not hard to see that we can add an edge from at least half of the paths in ${\cal P}_1$ that are not eared:
If we cannot add either the first or the last edge from a path $A$ in ${\cal P}_1$, and $A$ is not eared, then it must be the case that either the first or the last edge forms a cycle with an edge that was added to $T$ from another path in ${\cal P}_1$. 
}

\iftoggle{abs}{}{After preprocessing the paths in ${\cal P}_1$, we iterate through the connected components in $(V,R)$ and add edges to $T$ while maintaining that $T$ is a partial tour.
By Lemma~\ref{lem:third}, the number of edges added from each path or cycle $A$
is at least $|A|/3$, except for the paths in ${\cal P}_1$. }
We now consider two cases for the paths in ${\cal P}_1$, depending on whether we added an edge from the path to $T$ in the preprocessing step or not. 
Note that for a path $A$ in ${\cal P}_1$% that is not eared, and
for which we added an edge to $T$ in the preprocessing step, $R$ contains a path of $|A|-2$ edges after the preprocessing step, and by Lemma~\ref{lem:third}, we add at least $(|A|-2-1)/3$ of these to $T$. Together with the edge added in the preprocessing step, we thus add at least $1+(|A|-2-1)/3 = |A|/3$ edges. 
For a path in ${\cal P}_1$ for which we did not add an edge to $T$ in the preprocessing stage, we add at least $ (|A|-1)/3$ edges.
Now, recall that a path $A$ in ${\cal P}_1$ has $|A|\equiv 1\pmod 3$, and that the number of edges added is an integer, so in the first case, the number of edges added is at least $|A|/3 + \tfrac 23$ and in the second case it is $|A|/3-\tfrac 13$.
Let $z$ be the number of eared paths in ${\cal P}_1$. Then, the number of paths in ${\cal P}_1$ that are in the second case is at most $z$ plus the number of paths in ${\cal P}_1$ that fall in the first case. Hence, the total number of edges from $R$ that were added to $T$ can be lower bounded by $\tfrac 13 |R|-\tfrac 13 z$. We now give an upper bound on the number of nodes of degree one in $T$.

Let $k$ be the number of cycle nodes in $x$, i.e. $k = \# \{ i \in V : x(i,j) = \tfrac 12\mbox{ for some $j\in V$} \}$, and let $p$
be the number of cycle edges of cost 2 in $x$, i.e. $p = \# \{ e \in E : x(e) = \tfrac 12 \mbox{ and } c(e) = 2 \} $.
Note that $(V,R)$ contains $p$ paths (which may have zero edges) on the cycle nodes, and hence $p\ge z$.
Initially, when $T$ contains only the edges in the 1-paths, all $k$ nodes have degree one, and there are $k-p$ edges in $R$. We argued that we added at least $\tfrac 13 |R|-\tfrac 13 z = \tfrac 13 k - \tfrac 13 p-\tfrac 13 z$ edges to $T$. Each edge reduces the number of nodes of degree one by two, and hence, the number of nodes of degree one at the end of the algorithm is at most $k - 2(\tfrac 13 k - \tfrac 13 p - \tfrac 13 z) =\tfrac 13 k + \tfrac 23 p + \tfrac 23 z$.
Recall that $c( P )$ denotes the cost of the 1-paths, and the total cost of $T$ at the end of the algorithm is at most 
$c(P)+\tfrac 13 k -\tfrac 13p-\tfrac 13z$. Since at most $\tfrac 13 k + \tfrac 23 p + \tfrac 23 z$ nodes have degree one in $T$, we can extend $T$ into a tour of cost at most $c(P) + \tfrac 23 k +\tfrac 13 p +\tfrac 13 z$.

The cost of the solution $x$ can be expressed as $c(P) + \tfrac 12 k + \tfrac 12 p$.
Note that each 1-path connects two cycle nodes, hence $c(P)\ge \tfrac 12 k$.
Moreover, an eared path $A$ is incident to one (if $|A|=1$) or two (if $|A|>1$) 1-paths of length two, since the support graph of $x$ is simple.
Therefore we can lower bound $c( P )$ by $\tfrac 12 k + z$.
Therefore, $\tfrac 76\left(c(P) + \tfrac 12 k + \tfrac 12 p\right) \ge c(P) + \tfrac 1{12}k + \tfrac 16 z+\tfrac 7{12}k +\tfrac 7{12} p
\ge c(P)+\tfrac 23k +\tfrac 13 z + \tfrac 13 p$, where $p \geq z$ is used in the last inequality.%\qed
\end{proof}

\begin{proofof}{Lemma~\ref{lem:third}}
The basic idea behind the proof of the lemma is the following: We go through the edges of $A$ in order, and try to add them to $T$ if this does not create a cycle or node of degree three in $T$. If we cannot add an edge, we simply skip the edge and continue to the next edge.  Since the edges in $T$ form a collection of disjoint paths and each node in $A$ has degree one in $T$, we can always add either the first edge or the second edge of $A$: if the first edge cannot be added, then adding it to $T$ must create a cycle, and since the edges in $T$ form a collection of node disjoint paths, adding the second edge of the path or cycle to $T$ cannot create a cycle.
Similarly, we need to skip at most two edges between two edges that are successfully added to $T$: first, an edge is skipped because otherwise we create a node of degree three in $T$, and if a second edge is skipped, then this must be because adding that edge to $T$ would create a cycle. But then, adding the next edge on the path cannot create a cycle in $T$.

To lower bound the number of edges from we can add from each path or cycle $A$, we partition the edges into groups of two or three consecutive edges.
For a path $A$, the first group contains the first two edges, and each subsequent group contains the next three edges. The final group contains the last zero, one or two edges of the path. For each group except the last group, at least one edge is added to $T$. 
Hence, we can conclude that we can add at least $(|A|-4)/3$ from the groups of size three, and 1 for the first group, for a total of $(|A|-1)/3$ edges, where $|A|$ denotes the number of edges in $A$.
For a cycle $A$, we need to be slightly more careful, since the argument that we can add at least one edge from the last group of size three does not hold if the very first edge was added to $T$ (since it may be the case that the first and third edge of the group cannot be added without creating a node of degree three, and the second edge of the group cannot be added without creating a cycle). Therefore, we let the first group contain two consecutive edges, where the {\it second} edge is the edge that was the first to be added to $T$. By the same argument as for the path, we can thus conclude that we can add at least $(|A|-1)/3$ edges.


We now show that by being a little more careful, we can in fact add $|A|/3$ edges if $A$ is a cycle.
Note that the number of nodes in $A$ is odd, and hence there must be some $j$ such that the path in $T$ that starts in $u_j$ ends in some node $v\not\in A$.
We claim that if we consider the edges in $A$ starting with either edge $\{u_{j-1}, u_j\}$ or edge $\{u_j, u_{j+1}\}$, we are guaranteed that for at least one of these starting points, we can add both the first and the third edge to $T$.

Clearly, neither $\{u_{j-1}, u_j\}$ nor $\{u_j, u_{j+1}\}$ can create a cycle if we add it to $T$.
So suppose that $T\cup \{u_{j-1},u_{j}\} \cup \{u_{j+1}, u_{j+2}\}$ contains a cycle. This cycle does not contain the node $u_j$, because the path in $T$ that starts in $u_j$ ends in some node $v\not\in C$. Hence $T$ contains a path that starts in $u_{j+1}$ and ends in $u_{j+2}$. But then $T \cup \{u_{j},u_{j+1}\} \cup \{u_{j+2}, u_{j+3}\}$ does not have a cycle, since if it did, $T$ must have a path starting in $u_{j+2}$ and ending in $u_{j+3}$ which is only possible if $u_{j+1}=u_{j+3}$. Since the number of nodes in $A$ is at least three, this is not possible.
%\qed
\end{proofof}




We remark that the ratio of $\tfrac76$ in Theorem~\ref{thm:tour76} is achieved if every 1-path contains just one edge of cost 1, and all cycle edges have cost 1. However, in such a case, we could find another optimal F2M solution of the same cost, which has fewer cycle edges:
 If we have a 1-path of cost 1 with endpoints in two different odd cycles of edges with $x(e)=\tfrac 12$, we can obtain the alternative solution by removing the 1-path, and increasing the $x$-value on the four cycle edges incident on its endpoints to 1, and then alternating between setting the $x$-value to 0 and 1 around the cycles. Now, since the cycles are odd, the degree constraints are again satisfied. The objective value does not increase because we only change the $x$-value on edges of cost 1. For a 1-path of cost 1 with endpoints in the same odd cycle, the cycle gives us two paths between the endpoints, one of odd length and one of even length. We can alternate increasing and decreasing the $x$-value by $\tfrac12$ on the odd-length path and finally decrease the $x$-value for the 1-path to $\tfrac12$, to obtain a new F2M solution of the same cost with fewer cycle edges. We note that these modifications may increase the number of components of the F2M solution.

This motivates the following definition. We call an F2M solution {\it canonical}, if all edges in the support have cost 1 and all 1-paths contain at least two edges. If a canonical F2M solution is connected, we can improve the analysis in Theorem~\ref{thm:tour76} to show the following.
\begin{theorem}\label{thm:tour109}
There exists a tour of cost at most $\tfrac {10}9$ times the cost of a  connected canonical F2M solution if $c(i,j)\in \{1,2\}$ for all $i,j$.
\end{theorem}
\begin{proof}
We adapt the final paragraph of the proof of Theorem~\ref{thm:tour76}. As before, the cost of the tour is at most $c(P)+\tfrac23k + \tfrac13p+\tfrac13z$. However, since all cycle edges have cost 1, $p=0$ and $z=0$. The cost of the tour is thus at most $c(P)+\tfrac23k$.

The cost of the F2M solution is $c(P)+\tfrac12k$. Since each cycle node is the endpoint of a 1-path and vice versa, the number of 1-paths is $k/2$. By the fact that $x$ is canonical, each of these 1-paths has cost at least two, so we get that $c( P ) \geq k$.
The proof is concluded by noting that then $\tfrac{10}9\left(c(P)+\tfrac 12k\right) \geq c(P)+ \tfrac 19 k + \tfrac{10}9\cdot\tfrac 12 k  = c(P)+\tfrac23k$.
%\qed
\end{proof}



\section{An upper bound of $\tfrac{5}{4}$ on the integrality gap}\label{sec:better}
We now show how to use the results in the previous two sections to obtain an upper bound of $\tfrac{5}{4}$ on the integrality gap for the general case.
In addition, we show that if all edges in the support of the optimal subtour LP solution have cost 1, then the integrality gap is at most $\tfrac{26}{21}$.

We will bound the integrality gap of the solution obtained by the Papadimitriou-Yannakakis algorithm, by (i) bounding the difference between the cost of the 2M and the subtour LP, and (ii) bounding the difference between the 2M solution and the tour constructed from it by the Papadimitriou-Yannakakis algorithm.

As in the Papadimitriou-Yannakakis algorithm described in Section~\ref{sec:12gap}, we call a cycle in a 2M a ``pure'' cycle if all its edges have cost 1, and a ``non-pure'' cycle otherwise. 
The idea behind this section is to show that the quantity in (i) can be ``charged'' to the nodes in the non-pure cycle only, and that the quantity in (ii) can be ``charged'' mainly to the nodes in the pure cycles.

We first state the following lemma, which formalizes the second statement. %\iftoggle{abs}{Its proof is deferred to the full version of the paper.}{}
\begin{lemma}\label{lem:pure-nonpure}
If $OPT(\SUBT) < n+1$, then the difference between the cost of the 2M used and the tour constructed by the Papadimitriou-Yannakakis algorithm can be upper bounded by $\alpha n_\text{pure} + \beta (n_\text{non-pure}-\ell)$, where
$n_\text{pure}$ is the number of nodes in pure cycles in the 2M, $n_\text{non-pure}$ is the number of nodes in the non-pure cycle, and $\ell$ is the number of edges of cost 2 in the non-pure cycle, 
for any values of $\alpha, \beta$ so that $9\alpha \ge 2$ and $3\alpha + 2\beta \ge 1$.
\end{lemma}
Note that Lemma~\ref{lem:PYsubt} and the assumption that $OPT(\SUBT)<n+1$ imply that the Papadimitriou-Yannakakis algorithm finds a bipartite matching that matches all the pure cycles.
A careful look at the analysis of Papadimitriou and Yannakakis~\cite{PapadimitriouY93} then shows that their algorithm finds a tour which satisfies the lemma. The details basically follow the analysis of Papadimitriou and Yannakakis, and are therefore postponed to Appendix~\ref{app:purenonpure}.

The key observation in this section is that we can indeed restrict our attention to instances with $OPT(\SUBT) < n+1$, the requirement of Lemma~\ref{lem:pure-nonpure}.
\begin{lemma}\label{lem:worstcase}
The worst-case integrality gap is attained on an instance with subtour LP value less than $n+1$, where $n$ is the number of nodes in the instance.
\end{lemma}
\iftoggle{abs}{The idea behind the proof %of the lemma
is that, if $\lfloor OPT(\SUBT)\rfloor= n+k$, then the total $x$-value on edges with cost 2 is at least $k$. We can add $k$ nodes and for each new node, add edges of cost 1 to each existing node. We obtain a feasible subtour solution for the new instance with the same cost as the solution for the original instance, by rerouting one unit of flow from edges with cost 2 to go through each new node. Also, the cost of the optimal tour on the new instance is at least the cost of the optimal tour on the original instance, and hence, the integrality gap of the new instance is at least the integrality gap of the original instance.}{
\begin{proof}
Consider an instance $I$ on $n$ nodes for which the ratio between the length of the optimal tour and the subtour LP value is $\gamma$, and suppose $OPT(SUBT) = n+k$ for some $k\ge 1$.
We construct an instance $I'$ with $n'=n+1$ nodes, for which the ratio between the length of the optimal tour and the subtour LP value is at least $\gamma$, and the optimal value of the subtour LP is at most $n+k=n'+k-1$. Repeatedly applying this procedure proves the lemma.

If $OPT(SUBT)=n+k$, then the subtour LP solution on $I$ has a total $x$-value of $k$ on edges of cost 2, since the objective value is equal to $\sum_{e\in E} x(e) + \sum_{e\in E: c(e)=2} x(e)$, and $\sum_{e\in E} x(e)=\tfrac 12 \sum_{v\in V}\sum_{e\in \delta(v)} x(e)=n$.
We fix an optimal subtour solution $x$, and we construct $I'$ from $I$, by adding one node $i$, and adding edges $(i,j)$ of cost 1, for every $j$ in $I$ that is incident on an edge $e$ with $c(e)=2$ and $x(e)>0$.
All other edges incident on $i$ get cost 2.
Note that the optimal tour on $I'$ has length at least the length of the optimal tour on $I$, since we can take a tour on $I'$ and shortcut $i$ to obtain a tour on $I$. On the other hand, we can use $x$ to define a feasible solution on $I'$, by ``rerouting'' one unit in total from edges $e=(j,k)$ with $c(e)=2$ to the edges $(j,i)$ and $(i,k)$. Since the cost of this solution on $I'$ is the same as the cost of $x$, the ratio between the length of the optimal tour and the subtour LP value has not decreased.\mqed\end{proof}}
\begin{remark}
We note that the proof of Lemma~\ref{lem:worstcase} implies that to compute integrality gaps or approximation guarantees, we may assume without loss of generality that an instance has an optimal subtour LP value of at most $n+1$, where $n$ is the number of nodes in the instance. If this does not hold, we may add nodes as in the proof of Lemma~\ref{lem:worstcase} without increasing $OPT(\SUBT)$, and a tour of cost $C$ on the extended instance can be shortcut to a tour on the original instance of cost at most $C$.
\end{remark}

\begin{theorem}\label{thm:final}
The integrality gap of the subtour LP is at most $\tfrac{5}{4}$ for the 1,2-TSP, and it is at most $\tfrac{26}{21}$ for 1,2-TSP instances for which $OPT(\SUBT)<n+\tfrac12$, where $n$ is the number of nodes in the instance.
\end{theorem}
\begin{proof}
By Lemma~\ref{lem:worstcase}, we can assume without loss of generality that $OPT(\SUBT)$ $< n+1$.
To compute a tour, we first drop the subtour elimination constraints and find an optimal F2M solution. Since the F2M problem is a relaxation of the subtour LP, and it is half-integral, its objective value is either $n+\tfrac 12$ or $n$. 

We first consider the case $OPT(\SUBT)<n+\tfrac12$, in which case the optimal F2M solution has objective value $n$. Since all edges in the support of the F2M solution have cost 1, we may assume by the arguments preceding Theorem~\ref{thm:tour109} that all 1-paths contain at least two edges of cost 1; in other words, we may assume the components of the F2M solution are canonical. By applying Theorem~\ref{thm:tour109} we convert each fractional component of the F2M solution into a cycle on the nodes in the component.

Note that each cycle that is the result of applying Theorem~\ref{thm:tour109} contains at least one edge of cost 2. By the observation of Papadimitriou and Yannakakis~\cite{PapadimitriouY93}, we may merge these into a single non-pure cycle. The integer components of the F2M solution are pure cycles, since the support of the F2M solution only contains edges of cost $1$. We let $n_\text{pure}$ be the number of nodes in the pure cycles (or, equivalently, in the integer components of the F2M solution), and let $n_\text{non-pure}$ be the number of nodes in the non-pure cycle (or, equivalently, the number of nodes in the fractional components of the F2M solution).
Let $\ell$ be the number of cost 2 edges in the computed 2-matching. By Theorem~\ref{thm:tour109}, $\ell \le \tfrac{1}9n_\text{non-pure}$. 


If we apply the Papadimitriou-Yannakakis algorithm to this 2-matching, this increases the cost by at most $\alpha n_\text{pure} + \beta (n_\text{non-pure}-\ell)$, 
provided that $9\alpha\ge 2$ and $3\alpha+2\beta\ge1$ by Lemma~\ref{lem:pure-nonpure}. 
Choosing $\alpha=\tfrac 5{21}, \beta=\tfrac17$, we thus find that the total cost of the tour is at most $n+\ell+\tfrac5{21}n_\text{pure}+\tfrac 17 n_\text{non-pure} - \tfrac 17 \ell\le n+\tfrac 5{21}n_\text{pure} + (\tfrac 17+\tfrac 67\cdot \tfrac 19) n_\text{non-pure} = (1+\tfrac 5{21})n$, where we used the fact that $\ell\le \tfrac19n_\text{non-pure}$.

If $n+\tfrac12 \le OPT(\SUBT)<n+1$, the optimal F2M solution has cost at most $n+\tfrac12$. We temporarily decrease the cost of the unique cost-2 edge in the F2M to 1, and follow the same procedure as above, to find a 2-matching. Let $n_\text{non-pure}$ be the number of nodes in the non-pure cycle, and note that $n_\text{non-pure}$ is at least 9, since a fractional component of a canonical F2M solution contains at least two odd cycles, containing at least six nodes, and at least three 1-paths, containing at least one additional node each.

Let the cost of this 2-matching (with respect to the true costs) be $n+\ell$, where by Theorem~\ref{thm:tour109}, $\ell-1\le \tfrac 19 n_\text{non-pure}$.
As in the case when $OPT(\SUBT)<n+\tfrac12$, we apply the Papadimitriou-Yannakakis algorithm to this 2-matching, and by Lemma~\ref{lem:pure-nonpure} this increases the cost by at most $\alpha n_\text{pure} + \beta (n_\text{non-pure}-\ell)$. We now choose $\alpha=\tfrac14, \beta=\tfrac18$, to get that the total cost of the tour is at most
$n+\ell+\tfrac14n_\text{pure}+\tfrac18n_\text{non-pure}-\tfrac18\ell= n+\tfrac14n_\text{pure}+\tfrac98n_\text{non-pure}+\tfrac78(\ell-1)+\tfrac78$.
Now, recall that $\ell-1\le \tfrac19n_\text{non-pure}$ and that $n_\text{non-pure}\ge9$, and thus $\tfrac78\le \tfrac58 + \tfrac14\cdot\tfrac19 n_\text{non-pure}$.
Hence, we can upper bound the cost of the tour by $n+\tfrac14n_\text{pure}+ (\tfrac 98 + \tfrac78\cdot \tfrac19 +\tfrac14 \cdot \tfrac19)n_\text{non-pure}+\tfrac 58
=\tfrac54(n+\tfrac12) \le \tfrac54 OPT(\SUBT)$.
\mqed\end{proof}

\begin{remark}
The bound of $\tfrac54$ in Theorem~\ref{thm:final} may be marginally improved by a more careful analysis of small instances. It appears that in order to decrease the bound to $\tfrac{10}9$, or even $\tfrac{11}9$, more substantial new ideas are needed, however. 
\end{remark}