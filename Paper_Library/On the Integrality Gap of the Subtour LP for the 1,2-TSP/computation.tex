\section{Computational results}
\label{sec:comp}

In the case of the 1,2-TSP, for a fixed $n$ we can generate all instances as follows. For each value of $n$, we first generate all nonisomorphic graphs on $n$ nodes using the software package NAUTY~\cite{McKay81}. We let the cost of edges be one for all edges in $G$ and let the cost of all other edges be two. Then each of the generated graph $G$ gives us an instance of 1,2-TSP problem with $n$ nodes, and this covers all instances of the 1,2-TSP for size $n$ up to isomorphism.

In fact, we can do slightly better by only generating biconnected graphs. We say that a graph $G=(V,E)$ is {\em biconnected} if it is connected and there is no vertex $v \in V$ such that removing $v$ disconnects the graph; such a vertex $v$ is a {\em cut vertex}. It is possible to show that the subtour LP value is at least $n+1$ if $G$ is not biconnected, hence, by Lemma~\ref{lem:worstcase} it suffices to consider biconnected graphs.  However, the proof of Lemma \ref{lem:worstcase} involves adding additional new nodes (perhaps many of them).  Using a similar technique to the one in the proof of Lemma~\ref{lem:worstcase}, one can show that given a graph on $n$ vertices, there is a biconnected graph on at most $n+2$ vertices that has no better ratio of optimal tour to subtour LP value. 
In Appendix~\ref{app:twolemmas} we prove two lemmas that imply the following corollary.
%The following corollary follows readily from two lemmas that are proved in the appendix.
%We will show below that given a graph on $n$ vertices, there is a biconnected graph on at most $n+2$ vertices that has no better ratio of optimal tour to subtour LP value.  We will then show that our overall methodology allows us to obtain a bound on the integrality gap for $n \leq 12$.


\begin{corollary}\label{cor:12biconn}
Let $G=(V,E)$ be the graph of cost 1 edges in a 1,2-TSP instance.  Then if $G=(V,E)$ is not biconnected, there exists a biconnected $G'=(V',E')$ with $|V'|\le |V|+2$ such that $OPT(G)/\SUBT(G) \leq OPT(G')/\SUBT(G')$.
\end{corollary}



For each instance of size $n$, we solve the subtour LP and the corresponding integer program using CPLEX 12.1~\cite{cplex121} and a Macintosh laptop computer with dual core 2GHz processor and 1GB of memory.
%%% DPW: I don't think this sentence adds any crucial information
%We construct a mathematical programming model for CPLEX for each value of $n$, and for each instance we only changed the objective function and whether the solution is restricted to be integral or not for each instance of $G$.
It is known that the integrality gap is 1 for $n \leq 5$, so we only consider problems of size $n \geq 6$.   The results are summarized in Table~\ref{tab:12tab}.
\begin{table}[t]
\begin{center}
\begin{tabular}{|c|c|c|c|c|c|c|c|c|}
\hline
$n$ & 6 & 7 & 8 & 9 & 10 & 11& 12 \\
\hline
Subtour IP/LP ratio & $8/7.5$ &  $8/7.5$ &  $9/8.5$ & $10/9$ & $11/10$  & 12/11 & 13/12 \\
\# graphs & 56 & 468 & 7,123 & 194,066 & 9,743,542 & 900,969,091 & $-$ \\
\hline
\end{tabular}
\end{center}
\caption{The subtour LP integrality gap for 1,2-TSP for $6 \leq n \leq 12$, where the ratio for $6 \leq n \leq 10$ is only on biconnected graphs.  The second row shows the number of nonisomorphic biconnected graphs for $6 \leq n \leq 11$.}
\label{tab:12tab}
\iftoggle{abs}{\vspace*{-\baselineskip}}{}
\end{table}
For $n = 11$, the number of nonisomorphic biconnected graphs is nearly a billion and thus too large to consider, so we turn to another approach.
\iftoggle{abs}{For $n = 11$ and $n=12$, we use the fact that we know a lower bound on the integrality gap of $\alpha_n = \frac{n+1}n$, namely for the instances we obtain by adding two or three additional nodes to one of the 1-paths in the example in Figure~\ref{fig:12tsp}. }
{
\begin{figure}
\begin{center}
\includegraphics[height=.75in]{figures/badexlarger}
\end{center}
\caption{Illustration of the instances with integrality gap at least $\frac{12}{11}$ for $n=11$ (without the grey node) and $\frac{13}{12}$ for $n=12$ (with the grey node) for
the 1,2-TSP.  All edges of cost 1 are shown.}\label{fig:badexlarger}
\end{figure}
For $n = 11$ and $n=12$, we use the fact that we know a lower bound on the integrality gap of $\frac{n+1}n$, namely for the instances depicted in Figure~\ref{fig:badexlarger}. The claimed lower bounds on the integrality gap for these instances follow readily from the integrality gap for the example in Figure~\ref{fig:12tsp}.
}
We then check whether this is the worst integrality gap for each vertex of subtour LP. A list of non-isomorphic vertices of the subtour LP is available for $n=6$ to $12$ at Sylvia Boyd's website \url{http://www.site.uottawa.ca/~sylvia/subtourvertices}. In order to check whether the lower bound on the integrality gap is tight, we solve the following integer programming problem for each vertex $x$ of the polytope for $n=11$ and $n=12$, where now the costs $c(e)$ are the decision variables, and $x$ is fixed:
\iftoggle{abs}{
\[\max\{ z - \alpha_n \sum_{e \in E} c(e) x(e): \sum_{e \in T} c(e) \geq z\  \forall \mbox{ tours } T;  c(e) \in  \{1,2\} \  \forall e \in E. \}\]}
{
\lps
& & & \mbox{\sf Max} &z - \alpha_n \sum_{e \in E} c(e) x(e) \\
& \mbox{subject to:} \\
& & & & \sum_{e \in T} c(e) \geq z, & \forall \mbox{ tours } T,\\
& & & & c(e) \in  \{1,2\}, & \forall e \in E.  \elps
}
Note that $\alpha_n$ is the lower bound on the integrality gap for instances of $n$ nodes. If the objective is nonpositive for all of the vertices of the subtour LP, then we know that $\alpha_n$ is the integrality gap for a particular value of $n$.% exists a bad example wiour model finds the instance, i.e. set of edge costs, that gives the largest difference between the cost of shortest tour and the cost on current vertex's support graph. Then the ratio of $z$ and $\sum_{e \in E} c_e x_e$ is worst case integrality gap for current vertex. Over all vertices, we can find the worst cast integrality gap for $n = 11$ or $12$.



Since the number of non-isomorphic tours of $n$ nodes is $(n-1)!/2$, the number of constraints is too large for CPLEX for $n =11$  or $12$. We overcome this difficulty by first solving the problem with only tours that have at least $n-1$ edges in the support graph of the vertex $x$, and repeatedly adding additional violated tours. %Our results shows
We find that the worst case integrality gap for $n=11$ is $\frac{12}{11}$ and for $n=12$ is $\frac{13}{12}$.



%We then considered only subcubic nonisomorphic graphs, and continued the experiment up through $n=13$.  The results are summarized in Table~\ref{tab:12tab}.

%\begin{table}[t]
%\begin{center}
%\begin{tabular}{|c|c|c|c|c|c|c|c|c|}
%\hline
%$n$ & 6 & 7 & 8 & 9 & 10 & 11$^*$ & 12$^*$ & 13$^*$ \\
%\hline
%Subtour IP/LP ratio & $8/7.5$ &  $8/7.5$ &  $9/8.5$ & $10/9$ & $11/10$  & $12/11$ & $13/12$ & $14/13$ \\
%\# graphs & 56 & 468 & 7,123 & 194,066 & 9,743,542 & & & \\
%\hline
%\end{tabular}
%\end{center}
%\caption{The subtour LP integrality gap for 1,2-TSP for $6 \leq n \leq 10$, along with the number of nonisomorphic biconnected graphs.  For $n=11, 12, 13$, %we only consider subcubic instances in the underlying graph of cost 1 edges.}
%\label{tab:12tab}
%\end{table}

\iftoggle{abs}{}{
% We now show that the worst-case integrality gap for the subtour LP for the 1,2-TSP can be found on graphs of cost 1 edges that are biconnected.  Let $OPT(G)$ and $\SUBT(G)$ be the cost of the optimal tour and the value of the subtour LP (respectively) when $G$ is the graph of cost 1 edges.
% We start by proving that the worst case is obtained on a connected graph.

% \begin{lemma} \label{lem:12connect}
% Let $G=(V,E)$ be the graph of cost 1 edges in a 1,2-TSP instance.  Then if $G=(V,E)$ is not connected, there exists a connected graph $G'=(V',E')$ with $|V'|=|V|+1$ such that $OPT(G)/\SUBT(G) \leq OPT(G')/\SUBT(G')$.
% \end{lemma}

% \begin{proof}
% Suppose $G$ has more than one connected component. We create $G'=(V',E')$ by adding a new vertex $i^*$ to the graph, and adding edges from all $j \in V$ to $i^*$ so that $V' = V \cup \set{i^*}$ and $E' = E \cup \set{(i^*,j):  j \in V}$.  Given a tour of $G'$, we can easily produce a tour of $G$ of no greater cost by shortcutting $i^*$, so that $OPT(G) \leq OPT(G')$.  Let $x$ be an optimal solution to the subtour LP for the graph $G$.  We now define a solution $x'$ for $G'$, where $x'_{ij} = x_{ij}$ if $i$ and $j$ are in the same connected component of $G$, while if $i$ and $j$ are in different connected components of $G$, then we set $x'_{ij} = 0$, $x'_{i^*i}=x_{ij}$, and $x'_{i^*j} = x_{ij}$.  It is easy to see that the cost of $x'$ is the same as that of $x$.  We now argue that there is some solution $x''$ feasible for the subtour LP on $G'$ such that its cost is no greater, so that $\SUBT(G') \leq \SUBT(G)$.  It is clear that the bounds constraints (\ref{boundscons}) are satisfied for $x'$ and the degree constraints (\ref{degreecons}) are satisfied for $x'$ for all $i \in V$; however, the degree constraint for $i^*$ may not be satisfied.  Since for any component $C \subseteq V$ of $G$, $x(\delta(C)) \geq 2$, it is clear that $x'(\delta(i^*)) \geq 2$, but it may be the case that $x'(\delta(i^*)) > 2$.  For the subtour constraints (\ref{subtourcons}), consider any $S \subset V'$, $S \neq \emptyset$, such that $i^* \notin S$.  Then $x'(\delta(S)) \geq x(\delta(S)) \geq 2$, and for any $S \subseteq V'$ with $i^* \in S$, $S \neq \{i^*\}$, $x'(\delta(S)) = x'(\delta(V'-S)) \geq 2$ by the previous argument.  Finally, Goemans and Bertsimas~\cite{GoemansB90} have shown (see also Williamson~\cite{Williamson90}) that if edge costs obey the triangle inequality, and there is some solution $x'$ to the subtour LP in which degree constraints are exceeded but all other constraints are met, then there is another feasible solution $x''$ of no greater cost in which all constraints are satisfied.  Hence we have that $\SUBT(G') \leq \SUBT(G)$. Thus we have that $OPT(G)/\SUBT(G) \leq OPT(G')/\SUBT(G')$.
% \mqed\end{proof}

% \begin{lemma}\label{lem:12biconn}
% Let $G=(V,E)$ be the graph of cost 1 edges in a 1,2-TSP instance.  Then if $G=(V,E)$ is connected but not biconnected, there exists a biconnected $G'=(V',E')$ with $|V'|=|V|+1$ such that $OPT(G)/\SUBT(G) \leq OPT(G')/\SUBT(G')$.
% \end{lemma}

% \begin{proof}
% By hypothesis we assume that the graph $G=(V,E)$ is connected.  Let $i_1, \ldots, i_k$ be all the cut vertices of $G$, and let $C_1,\ldots, C_\ell$ be all the connected components formed when these vertices are removed, so that $C_1,\ldots,C_\ell, \set{i_1},\ldots, \set{i_k}$ form a partition of $V$.  We create a new graph $G'=(V',E')$ by adding a new vertex $i^*$, and adding edges from $i^*$ to each vertex in $C_1 \cup \cdots \cup C_\ell$, so that $V' = V \cup \set{i^*}$ and $E' = E \cup \set{(i^*,j): j \in C_p \mbox{ for some } p}$. We note that $G'$ is biconnected. As before, we have $OPT(G) \leq OPT(G')$ since given a tour of $G'$ we can shortcut $i^*$ to get a tour of $G$.  Let $x$ be an optimal subtour LP solution for graph $G$.  We now argue, as we did in the proof of Lemma~\ref{lem:12connect}, that we can create an $x'$ that costs no more than $x$ such that all the subtour and bounds constraints are obeyed, and all degree constraints are either met or exceeded; this will imply that $\SUBT(G') \leq \SUBT(G)$, and complete the proof.  Suppose without loss of generality that removing cut vertex $i_1$ creates components $C_1$ and $C = C_2 \cup \cdots \cup C_\ell \cup \set{i_2} \cup \cdots \cup \set{i_k}$, so that $C_1$, $\set{i_1}$, and $C$ partition $V$.  We set $x'_{ij} = 0$ and $x'_{i^*i}=x'_{i^*j}=x_{ij}$ if $i \in C_1$ and $j \in C$; $x'_{ij} = x_{ij}$ otherwise.  If $i \in C_1$ and $j \in C$, then $(i,j) \notin E$ since $i_1$ is a cut vertex, so the cost of $x'$ is no more than that of $x$.  The arguments that all constraints are satisfied except for the degree constraint on $i^*$ follow as in the proof of Lemma~\ref{lem:12connect}.  We now must argue that $x'(\delta(i^*)) \geq 2$.  To do this, we show that $\sum_{i \in C_1, j \in C} x_{ij} \geq 1$.  Since $x(\delta(i_1)) = 2$, it must be the case that either $\sum_{j \in C} x_{i_1j} \leq 1$ or $\sum_{j \in C_1} x_{i_1j} \leq 1$; without loss of generality we assume the former is true.  Then since $x(\delta(C_1 \cup \set{i_1})) \geq 2$, and $x(\delta(C_1 \cup \set{i_1})) = \sum_{j \in C} x_{i_1j} + \sum_{i \in C_1, j \in C} x_{ij}$, it follows that $\sum_{i \in C_1, j \in C} x_{ij} \geq 1$, and the proof is complete.
% \mqed\end{proof}

We can now observe that our overall computation leads to a bound of $\frac{10}9$ on the integrality gap for instances of the 1,2-TSP with $n \leq 12$.  Suppose the worst-case integrality gap for these instances is attained for an instance with $k$ vertices.  If $k \leq 8$, then we know that there is a biconnected graph on at most 10 vertices with no better integrality gap, and we have determined the worst-case ratio for all biconnected graphs on at most 10 vertices.  If $k=9$ or $k=10$, then we know there is a biconnected graph on at most 12 vertices with no better integrality gap, and we have determined the worst-case ratio for all biconnected graphs with at most 10 vertices and all instances with 11 or 12 vertices.  If $k=11$ or $k=12$, then we have determined the worst-case ratio for all instances with 11 or 12 vertices.  Thus for any instance with $n\leq 12$, we have determined that the integrality gap is at most $\frac{10}9$.
}



