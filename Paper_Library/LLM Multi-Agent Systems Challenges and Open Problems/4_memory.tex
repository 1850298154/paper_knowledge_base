
\section{Agent Memory and Information Retrieval}\label{sec: memory}
The memory in single-LLM agent systems refers to the agent's ability to record, manage, and utilize data, such as past historical queries and some external data sources, to help inference and enhance decision-making and reasoning~\citep{yao2023tree,park2023generative,li2023mot,wang2023augmenting,guo2023prompt}.
While the memory in a single-LLM agent system primarily focuses on internal data management and utilization, a multi-agent system requires agents to work collaboratively to complete some tasks, necessitating the individual memory capabilities of each agent as well as a sophisticated mechanism for sharing, integrating, and managing information across agents, thus poses challenges to memory and information retrieval. 


\subsection{Classifications of Memory in Multi-agent Systems}\label{sec: memory_classification}
Based on the work flow of a multi-agent system, we categorize memory in multi-agent system as follows.


\begin{itemize}[leftmargin=2em, itemsep=-0.3em, topsep=0em]
    \item \textit{Short-term memory:} This is the immediate, transient memory used by a Large Language Model (LLM) during a conversation or interaction, \textit{e}.\textit{g}., working memory in ~\cite{jinxin2023cgmi}. It is ephemeral, existing only for the duration of the ongoing interaction and does not persist once the conversation ends.
    \item \textit{Long-term Memory}: This type of memory stores historical queries and responses, essentially chat histories from earlier sessions, to support inferences for future interactions. Typically, this memory is stored in external data storage, such as a vector database, to facilitate recall of past interactions.

    \item \textit{External data storage}: This is an emerging area in LLM research where models are integrated with external data storage like vector databases, such that the agents can access additional knowledge from these databases, enhancing their ability to ground and enrich their responses~\cite{rag}. This allows the LLM to produce responses that are more informative, accurate, and highly relevant to the specific context of the query. 

    \item \textit{Episodic Memory}: This type of memory encompasses a collection of interactions within multi-agent systems. It plays a crucial role when agents are confronted with new tasks or queries. By referencing past interactions that have contextual similarities to the current query, agents can significantly enhance the relevance and accuracy of their responses. Episodic Memory allows for a more informed approach to reasoning and problem-solving, enabling a more adaptive and intelligent response mechanism, thus serves as a valuable asset in the multi-agent system, 

    \item \textit{Consensus Memory}: In a multi-agent system where agents work on a task collaboratively, consensus memory acts as a unified source of shared information, such as  common sense, some domain-specific knowledge, etc, \textit{e}.\textit{g}., skill library in~\cite{jinxin2023cgmi}. Agents utilize consensus memory to align their understanding and strategies with the tasks, thus enhancing an effective and cohesive collaboration among agents.
\end{itemize}


While both single-agent and multi-agent systems handle short-term memory and long-term memory, multi-agent systems introduce additional complexities due to the need for inter-agent communication, information sharing, and adaptive memory management. 




\subsection{Challenges in Multi-agent Memory Management}\label{sec: memory_challenges}
Managing memory in multi-agent systems is fraught with challenges and open problems, especially in the realms of safety, security, and privacy. We outline these as follows:

\textbf{Hierarchical Memory Storage:} In a multi-agent system, different agents often have varied functionalities and access needs. Some agents may have to query their sensitive data, but they don't want such data to be accessed by other parties.
While ensuring the consensus memory to be accessible to all clients, implementing robust access control mechanisms is crucial to ensure sensitive information of an agent is not accessible to all agents. Additionally, as the agents in a system collaborative on one task, and their functionalities share same contexts, their external data storage and memories may overlap. If the data and functionalities of these agents are not sensitive, adopting an unified data storage  can effectively manage redundancy among the data, and furthermore, ensure consistency across the multi-agent system, leading to more efficient and precise maintenance of memory.


\textbf{Maintenance of Consensus Memory:} As consensus memory is obtained by all agents when collaborating on a task,
ensuring the integrity of shared knowledge is critical to ensure the correct execution of the tasks in the multi-agent systems. 
Any tampering or unauthorized modification of consensus memory can lead to systemic failures of the execution. 
Thus, a rigorous access control is important to mitigate risks of data breaches.



\textbf{Communication and information exchange:}
Ensuring effective communication and information exchange between agents is essential in multi-agent systems. Each agent may hold critical pieces of information, and seamless integration of these is vital for the overall system performance. 

\textbf{Management of Episodic Memory. } Leveraging past interactions within the multi-agent system to enhance responses to new queries is challenging in multi-agent systems. Determining how to effectively recall and utilize contextually relevant past interactions among agents for current problem-solving scenarios is important.




These challenges underscore the need for continuous research and development in the field of multi-agent systems, focusing on creating robust, secure, and efficient memory management methodologies.
