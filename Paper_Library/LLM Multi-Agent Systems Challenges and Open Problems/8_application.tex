
\section{Applications in Blockchain}\label{sec: blockchain_app}

Multi-agent systems offer significant advantages to blockchain systems by augmenting their capabilities and efficiency. 
Essentially, these multi-agent systems serve as sophisticated tools for various tasks on blockchain and Web3 systems. 
Also, 
blockchain nodes can be viewed as agents with specific roles and capabilities~\cite{ankile2023see}.
Given that both Blockchain systems and multi-agent systems are inherently distributed, the blockchain networks can be integrated with multi-agent systems seamlessly. 
By assigning a dedicated agent to each blockchain node, it's possible to enhance data analyzing and processing while bolstering security and privacy in the chain. 



\subsection{Multi-Agent Systems As a Tool}\label{sec: mas_tool_in_blockchain}
To cast a brick to attract jade, we give some potential directions that multi-agents systems can act as tools to benefit blockchain systems. 


\textbf{Smart Contract Analysis. }
Smart contracts are programs stored on a blockchain that run when predetermined conditions are met. 
Multi-agents work together to analyze and audit smart contracts. The agents can have different specializations, such as identifying security vulnerabilities, legal compliance, and optimizing contract efficiency. Their collaborative analysis can provide a more comprehensive review than a single agent could achieve alone.

\textbf{Consensus Mechanism Enhancement.} Consensus mechanisms like Proof of Work (PoW)~\cite{gervais2016security} or Proof of Stake (PoS)~\cite{saleh2021blockchain} are critical for validating transactions and maintaining network integrity. Multi-agent systems can collaborate to monitor network activities, analyze transaction patterns, and identify potential security threats. By working together, these agents can propose enhancements to the consensus mechanism, making the blockchain more secure and efficient.



\textbf{Fraud Detection. } 
Fraud detection is one of the most important task in financial monitoring. As an example, \cite{ankile2023see} studies fraud detection through the perspective of an external observer who detects price manipulation by analyzing the transaction sequences or the price movements of a specific asset.
Multi-agent systems can benefit fraud detection in blockchain as well. Agents can be deployed with different roles, such as monitoring transactions for fraudulent activities and analyzing user behaviors. Each agent could also focus on different behavior patterns to improve the accuracy and efficiency of the fraud detection process. 



\subsection{Blockchain Nodes as Agents}\label{sec: blockchain_nodes_as_agents}
\cite{ankile2023see} identifies  blockchain nodes as agents, and studies fraud detection in the chain from the perspective an external observer. However, as powerful LLM agents with analyzing and reasoning capabilities, there are much that the agents can do, especially when combined with game theory and enable the agents to negotiate and debate. Below we provide some perspectives. 


\textbf{Smart Contract Management and Optimization.}
Smart contracts are programs that execute the terms of a contract between a buyer and a seller in a blockchain system. The codes are fixed, and are self-executed when predetermined conditions are met. 
Multi-agent systems can automate and optimize the execution of smart contracts with more flexible terms and even dynamic external information from users. Agents can negotiate contract terms on behalf of their users, manage contract execution, and even optimize gas fees (in the context of Ethereum~\cite{wood2014ethereum}. The agents can analyze context information , such as past actions and pre-defined criteria, and utilize the information with flexibility.
Such negotiations can also utilize game theory, such as Stackelberg Equilibrium~\citep{von2010market,conitzer2006computing} when there is a leader negotiator and Nash Equilibrium~\cite{kreps1989nash} when no leader exists. 

