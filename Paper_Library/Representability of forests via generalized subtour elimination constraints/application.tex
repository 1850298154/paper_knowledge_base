\section{Subadditive functions and problem applications}
\label{section:applications}

Let~$\mc{F}$ be a family of forests. While the results in Section~\ref{subsection:rhs} establish that the strongest GSEC-based relaxation for~$\mc{F}$ is~$\P{\ub{\mc{F}}}$, computing~$\ub{\mc{F}}{S}$ may be too expensive, as it requires optimizing over~$\mc{F}$. To address this, we introduce here an approach that, under suitable conditions, allows one to easily obtain a RHS function~$\f$ such that~$\P{f} \subseteq \P{\lb{\mc{F}}}$ represents~$\mc{F}$.

Let~$g : 2^{V} \to \R_+$ be such that~$g(S) \leq |S|$, for every~$S \subseteq V$. In this section, we focus on families of forests of the form~$\mc{F}(\Theta(g))$, where
\begin{equation}
    \label{def:theta}
    \Theta(g) \coloneqq \{\text{tree~$T \subseteq G$} :  g(V(T')) \leq 1,~\forall \text{subtree~$T' \subseteq T$} \}.
\end{equation}

\noindent
It follows from Theorem~\ref{thm:characterization1} that, if~$\mc{F}(\mc{T})$ is representable and~$\mc{T}$ contains all trees with no edges, then we can assume without loss of generality that~$\mc{T} = \Thetaset{g}$.
\begin{proposition}
    \label{proposition:representable_f_g}
    Let~$g : 2^{V} \to \R_+$ be such that~$g(S) \leq |S|$, for every~$S \subseteq V$. Then~$\Ft{\Thetaset{g}}$ is representable. Moreover, if~$\mc{T}$ contains all trees with no edges and~$\Ft{\mc{T}}$ is representable, then~$\mc{T} = \Thetaset{\ell_{\Ft{\mc{T}}}}$.
\end{proposition}
\begin{proof}
    To ease notation, let~$\mc{F} = \Ft{\Thetaset{g}}$.
    To prove the first part of the statement, it suffices to show that that~$\mc{F} = \Phiset{\lb{\mc{F}}}$. By Claim~\ref{claim:ft_lower_bound}, we may write
    \begin{align*}
        \Phiset{\lb{\mc{F}}} & = \{F \in \allforests : |F'| \geq \lb{\mc{F}}{V(F')},~\forall F' \subseteq F \} \\
        & = \{F \in \allforests : 1 \geq 1 + \mb{I}(V(T) \in \B{\mc{F}}), ~\forall \text{tree~$T \subseteq F, T \neq \emptyset$}\} \\
        & = \{F \in \allforests : V(T) \notin \B{\mc{F}}, ~\forall \text{tree~$T \subseteq F, T \neq \emptyset$}\}
    \end{align*}

    \noindent
    Now, suppose that~$F$ is a forest that does not belong to~$\mc{F}$. Since~$\mc{F}$ contains the empty graph, there exists a subforest~$T \subseteq F$ such that~$T \in \M{\mc{F}}$. As~$V(T) \in \B{\mc{F}}$ and~$T$ is a tree (by minimal infeasibility), it follows that~$F \notin \Phiset{\lb{\mc{F}}}$. 
    
    Conversely, suppose that~$F \in \mc{F}$ and let~$T$ be a nonempty subtree of~$F$. Since~$\mc{F}$ is downward closed, we know that~$T \in \Thetaset{g}$. Therefore,~$g(V(T)) \leq 1$, which implies that~$V(T) \notin \B{\mc{F}}$. Indeed, suppose that there exists a tree~$T' \in \M{\mc{F}}$ such that~$V(T') = V(T)$. Then~$T' \notin \Theta(g)$ and every subtree~$T'' \subsetneq T'$ belongs to~$\Theta(g)$, meaning that~$g(V(T')) > 1$, a contradiction. This shows that~$F \in \Phiset{\lb{\mc{F}}}$.
    
    To close the proof, suppose that~$\Ft{\mc{T}}$ is representable, so~$\Ft{\mc{T}} = \Phiset{\lb{\Ft{\mc{T}}}}$. Then~$$\mc{T} = \{\text{tree~$T \in \Phiset{\lb{\Ft{\mc{T}}}}$}\} = \{\text{tree~$T \subseteq G$} : \lb{\Ft{\mc{T}}}{V(T')} \leq 1,~\forall \text{subtree~$T' \subsetneq T$}\}.$$
\end{proof}

Let~$\f$ be a RHS function and observe that~$\P{f}$ does not necessarily represent~$\Ft{\Thetaset{f}}$. For example, we might have~$\f(V) = |V|$ (so~$\P{f} = \{0\}$) while~$\Thetaset{f}$ contains a tree~$T$ in~$G$ with~$E(T) \neq \emptyset$. One can show, however, that if~$\f$ is a \emph{subadditive set function} --- that is,~$\f{A \cup B} \leq \f{A} + \f{B}$ for every~$A, B \subseteq V$ with~$A \cap B = \emptyset$\,%
\footnote{The standard definition of a subadditive set function requires this inequality to hold for all~$A, B \subseteq V$. In our setting, however, it suffices to consider only disjoint sets.}
--- then~$\P{\f}$ does represent~$\Ft{\Thetaset{\f}}$. Extending this reasoning, we obtain the following result.

\begin{proposition}
    \label{prop:subadditive_rhs_function}
    Let~$g : 2^{V} \to \R_+$ be a subadditive set function such that~$g(S) \leq |S|$, for every~$S \subseteq V$. Let~$\f$ be the RHS function given by~$\f(S) \coloneqq \max\{1, \lceil g(S) \rceil\}$, for all~$\emptyset \subsetneq S \subseteq V$. Then~$\P{\f}$ represents~$\Ft{\Thetaset{g}}$.
\end{proposition}
\begin{proof}
    Let~$F \subseteq G$ be a forest such that~$\mathbbm{1}_F \in \P{f_g} \cap \Z^E$. For every subtree~$T \in F$ and subtree~$T' \subseteq T$ with~$|V(T')| \geq 2$, we have that~$\mathbbm{1}_F(V(T')) = |V(T')| - 1 \leq |V(T')| - \f{V(T')}$, which implies that~$g(V(T')) \leq \f{V(T')} \leq 1$. This shows that~$T \in \Thetaset{g}$, and consequently,~$F \in \Ft{\Thetaset{g}}$.
    
    For the converse, suppose that~$F \in \Ft{\Thetaset{g}}$. To show that~$\mathbbm{1}_F \in \P{f}$ satisfy the GSECs, take an arbitrary set~$\emptyset \subsetneq S \subseteq V$ and let~$F'$ be the subgraph of~$F$ induced by~$S$. Since every tree~$T \in F'$ belongs to~$\Thetaset{g}$, we know that~$g(V(T)) \leq 1$. Hence, by subadditivity of~$\lceil g(\,\cdot\,) \rceil$,
    $$\mathbbm{1}_F(S) = |S| - |F'| \leq |S| - \sum_{T \in F'} \lceil g(V(T)) \rceil \leq |S| - \lceil g(S) \rceil \leq |S| - \f{S}.$$
\end{proof}

A particular subclass of subadditive set functions that will be convenient for us are the \emph{XOS} functions~\cite{feige2006maximizing}:
\begin{definition}
    \label{def:xos}
    We say that~$g : 2^{V} \to \R$ is \emph{XOS} with respect to a set~$\mc{W} \subseteq \R^{V}$ if~$g(S) = \max_{w \in \mc{W}} \{w(S)\}$, for all~$\emptyset \subsetneq S \subseteq V$. For convenience, we assume that~$g(\emptyset) = 0$.
\end{definition}
\begin{fact}
    \label{fact:xos}
    Every XOS set function~$g : 2^{V} \to \R$ is subadditive.
\end{fact}
\begin{proof}
    Let~$\mc{W}$ be such that~$g(S) = \max_{w \in \mc{W}} \{w(S)\}$, for all~$\emptyset \subsetneq S \subseteq V$.
    Let~$A, B \subseteq V$ be such that~$A \cap B = \emptyset$. Let~$\bar{w} \in \mc{W}$ be such that~$g(A \cup B) = \bar{w}(A \cup B)$. Then~$$g(A \cup B) = \bar{w}(A) + \bar{w}(B) \leq \max_{w \in \mc{W}}\{w(A)\} + \max_{w \in \mc{W}}\{w(B)\} = g(A) + g(B).$$
\end{proof}

In conclusion, to formulate a family of forests~$\Ft{\mc{T}}$ using GSECs, it suffices to find an XOS function~$g$ such that~$\mc{T} = \Thetaset{g}$ and~$g(S) \leq |S|$, for all~$S \subseteq V$. In what follows, we apply this approach to the bike sharing rebalancing problem and a robust capacitated minimum spanning tree problem. We again emphasize that these formulations cannot be obtained using the framework of Ghosal et al.~\cite{ghosal2024}.

\subsection{Bike sharing rebalancing problem}

Recall the definition of~$\routesbrp$ and~$\fbrp$ from Section~\ref{section:setup}, and let~$\Cpath$ and~$\Qpath$ be set as in Example~\ref{example:vrp}. To show that~$\vrpprob{\routesbrp}$ can be expressed as~$\vrpform{\fbrp}$, we begin with a simple lemma. Although this result was already discussed somewhat informally in~\cite{dell2014bike}, we include the proof for completeness.
\begin{lemma}
    \label{lemma:brp}
    Let~$P = (v_1, \ldots, v_\ell)$ be a path in~$G$. For each~$i \in [\ell]$, define~$D(i) \coloneqq \sum_{j \in [i]} d(v_i)$ (and~$D(0) = 0$). Moreover, define~$D_{\max}(i) \coloneqq \max_{j \in \{0, \ldots, i\}} \{D(j)\}$ and~$D_{\min}(i) \coloneqq \min_{j \in \{0, \ldots, i\}} \{D(j)\}$. The path~$P$ belongs to~$\routesbrp$ if and only if~$D_{\max}(i) - D_{\min}(i) \leq Q$, for all~$i \in [\ell]$. 
\end{lemma}
\begin{proof}
    Suppose that~$P \in \routesbrp$, meaning that there exists~$q$ such that~$0 \leq q + D(i) \leq Q$, for all~$i \in [\ell]$. Fix~$i \in [\ell]$ and note that~$q + D_{\max}(i) \leq Q$ and~$q + D_{\min}(i) \geq 0$. Therefore,~$q \geq - D_{\min}(i)$ and~$D_{\max}(i) - D_{\min}(i) \leq Q$.

    For the converse, assume that~$D_{\max}(i) - D_{\min}(i) \leq Q$, for all~$i \in [\ell]$. Set~$q = - D_{\min}(\ell)$ and observe that~$q + D(i) \geq 0$, for all~$i \in [\ell]$. Moreover,
    $$q + D(i) = D(i) - D_{\min}(\ell) \leq D_{\max}(\ell) - D_{\min}(\ell) \leq Q.$$
\end{proof}

Using Lemma~\ref{lemma:brp}, we obtain the following characterization of BRP-feasible paths.
\begin{proposition}
    \label{prop:brp}
    Let~$P = (v_1, \ldots, v_\ell)$ be a path in~$G$. Then~$P \in \routesbrp$ if and only if~$|\sum_{p = i}^j d_{v_p}| \leq Q$, for every~$0 < i \leq j \leq \ell$.
\end{proposition}
\begin{proof}
    Let~$D, D_{\max}$ and~$D_{\min}$ be as in the statement of Lemma~\ref{lemma:brp}.
    To prove the forward direction, assume that~$0 \leq q \leq Q$ is such that~$0 \leq q + D(j) \leq Q$, for all~$j \in [\ell]$. Then
    \begin{align*}
        & 0 \leq q + D(j) \leq Q \\
        \iff & 0 \leq q + D(i - 1) + (D(j) - D(i - 1)) \leq Q \\
        \iff & -q - D(i - 1) \leq D(j) - D(i - 1) \leq Q - q - D(i - 1) \\
        \implies & -Q \leq D(j) - D(i - 1) \leq Q,
    \end{align*}
    where the last line follows from~$0 \leq q + D(i - 1) \leq Q$ (recall that~$D(0) = 0$). We are thus done by the fact that~$D(j) - D(i - 1) = \sum_{p = i}^j d_{v_p}$.

    For the converse, we fix~$i \in [\ell]$ and we show that~$D_{\max}(i) - D_{\min}(i) \leq Q$ (by Lemma~\ref{lemma:brp}). Let~$j_{\max}$ and~$j_{\min}$ be such that~$D_{\max}(i) = D(j_{\max})$ and~$D_{\min}(i) = D(j_{\min})$. Suppose first that~$j_{\max} > j_{\min}$. Since~$|d(\{v_{j_{\min} + 1}, \ldots, v_{j_{\max}}\}) | \leq Q$, it follows that~$$D_{\max}(i) - D_{\min}(i) = D(j_{\max}) -  D(j_{\min}) = d((v_{j_{\min} + 1}, \ldots, v_{j_{\max}})) \leq Q.$$ On the other hand, if~$j_{\max} < j_{\min}$, we know that~$|d(\{v_{j_{\max} + 1}, \ldots, v_{j_{\min}}\}) | \leq Q$, meaning that~$$D_{\min}(i) - D_{\max}(i) = D(j_{\min}) -  D(j_{\max}) = d((v_{j_{\max} + 1}, \ldots, v_{j_{\min}})) \geq -Q.$$
\end{proof}

Now, for each~$\emptyset \subsetneq S \subseteq V$, define the XOS function~$$g_{\tsc{brp}}(S) \coloneqq \max \{d(S) / Q, -d(S) / Q \}.$$ By Proposition~\ref{prop:brp}, we have that~$\routesbrp = \Thetaset{g_{\tsc{brp}}} \cap \Cpath$. Moreover, Proposition~\ref{prop:subadditive_rhs_function} implies that~$\P{\fbrp}$ represents~$\Ft{\Thetaset{g_{\tsc{brp}}}}$, meaning that~$\P{\fbrp} \cap \Qpath$ represents~$$\Ft{\Thetaset{g_{\tsc{brp}}}} \cap \Cpath = \Ft{\Thetaset{g_{\tsc{brp}}} \cap \Cpath} = \Ft{\routesbrp}.$$ 

Therefore, every feasible solution~$\bar{x}$ to formulation~$\vrpform{\fbrp}$ corresponds to a feasible solution~$\mc{S} = \{C_1, \ldots, C_k\}$ for problem~$\vrpprob{\routesbrp}$ (where the ``correspondence'' is in the sense of Equation~\eqref{eq:correspondence_vrp}).

\subsection{Robust capacitated minimum spanning tree problem}

As in Section~\ref{section:setup}, let~$G_0 = (V_0, E_0)$ be a connected undirected graph with~$V_0 = \{0\} \dot\cup V$ and~$E_0 = \{0v : v \in V\} \dot\cup E$. Set~$G = G_0 - \{0\}$ and let $Q \in \Q_+$ be a capacity value. In the CMST~\cite{hall1992polyhedral, hall1996, uchoa2008robust}, each vertex~$v \in V$ has a demand~$d_v \in [0, Q] \cap \Q_+$, and the goal is to find a spanning tree~$T$ of~$G$, rooted at~$0$, and such that the total demand of each subtree~$T'$ hanging from~$0$ does not exceed $Q$. 

Inspired by previous work on the robust CVRP~\cite{gounaris2013robust, subramanyam2020robust, pessoa2021branch}, we now introduce the robust CMST (RCMST), where, instead of assuming that~$d \in \Q^V_+$ is deterministic, we only know that~$d$ belongs to a given \emph{uncertainty set} \hypertarget{def:uncertainty_set}{$\mc{U} \subseteq \R^V_+$}. The subtrees~$T'$ rooted at a child of~$0$ must then satisfy the robust capacity constraint~$\max_{d \in \U}\{d(V(T')\} \leq Q$.

For each~$\emptyset \subsetneq S \subseteq V$, define the XOS function~$$g_{\tsc{rcmst}}(S) \coloneqq \max_{d \in \U}\left\{d(S) / Q\right\}.$$ The set of trees in~$G$ satisfying the robust capacity constraints is given by~$\Thetaset{g_{\tsc{rcmst}}}$. Therefore, defining~$f_{\tsc{rcmst}}(\,\cdot\,) \coloneqq \max\{1, \lceil g_{\tsc{rcmst}}(\,\cdot\,) \rceil\}$, we have that~$\P{f_{\tsc{rcmst}}}$ represents~$\mc{F}_{\tsc{rcmst}} = \Ft{\Thetaset{g_{\tsc{rcmst}}}}$. In this way, we can formulate the RCMST as
\begin{subequations}
\label{form:rcmstp}
\begin{align}
    \min~~ & c^\T x \\
    \text{s.t.~~} & x(V_0) = |V|, \\
    & x(\{0\} \cup S) \leq |S|, & \forall \emptyset \subsetneq S \subseteq V, \label{ineq:rcmst_depot} \\
    & x|_E \in \P{f_{\tsc{rcmst}} ; G}, \\
    & x \in \Z^E,
\end{align}
\end{subequations}
where constraints~\eqref{ineq:rcmst_depot} enforce the subtour elimination constraints for subsets of vertices containing the depot.

When~$\U$ is a singleton, Formulation~\eqref{form:rcmstp} reduces to the CMST formulation of Hall~\cite{hall1996}. For budgeted and factor model uncertainty sets,~$g_{\tsc{rcmst}}(S)$ can be computed efficiently using the analytical solutions of Gounaris et al.~\cite{gounaris2013robust}.
