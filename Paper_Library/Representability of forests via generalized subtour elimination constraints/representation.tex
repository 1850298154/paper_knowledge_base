\section{Representable families of forests}
\label{section:representation}

From now on, we fix~$G = (V, E)$ to be an arbitrary undirected graph (which may not be the same as the graph~$G_0 - \{0\}$ from Section~\ref{section:setup}). For any RHS function~$\f$, we define the polytope
\begin{equation}
    \label{def:polytope}
    \tag{$\mc{P}(f ; G)$}
    \mc{P}(f ; G) \coloneqq \{x \in [0, 1]^{E} : x(S) \leq |S| - \f{S},~~\forall \emptyset \subsetneq S \subseteq V\}.
\end{equation}
Since the graph~$G$ is fixed, we sometimes omit the dependence on~$G$ from the notation. In particular, we write~$\P{f}$ instead of~$\P{f; G}$. Additionally, to avoid repeating ourselves, whenever we write~$\P{f}$, it is implicitly assumed that~$\f$ is a RHS function.

The GSECs imply the SECs, so the integer vectors inside~$\P{f}$ correspond to forests in~$G$ (for any RHS function~$\f$). We use \hypertarget{def:omega}{$\allforests$} to denote the family of all forests in~$G$. The notion of \emph{representability} is formalized as follows.
\begin{definition}
    \label{def:representation}
    Let~$\mc{F}$ be a family of forests in~$G$ and let~$\mc{P} \subseteq \R^E$.
    We say that~$\mc{P}$ \emph{represents}~$\mc{F}$ if~$\mc{P} \cap \Z^{E} = \{\mathbbm{1}_{F} : F \in \mc{F}\}$. Furthermore,~$\mc{F}$ is \emph{representable} if there exists a RHS function~$\f$ such that~$\P{f}$ represents~$\mc{F}$.
\end{definition}

Definition~\ref{def:representation} identifies each forest in~$\mc{F}$ with its edge set. However, different forests may share the same edge set, as they might differ only by a set of isolated vertices. Consequently, the same set~$\P{f}$ may represent two distinct families of forests~$\mc{F}$ and~$\mc{F}'$, as long as their incidence vectors coincide. In this way, we often focus on \emph{edge-consistent} families of forests:
\begin{definition}
    \label{def:edge-consistency}
    A family of forests~$\mc{F}$ is \emph{edge-consistent} if, for every pair of forests~$F$ and~$F'$ in~$G$ with~$E(F) = E(F')$, we have that~$F \in \mc{F}$ if and only if~$F' \in \mc{F}$.
\end{definition}

The assumption of edge-consistency is without loss of generality: given any family of forests~$\mc{F}'$, one can always construct a unique edge-consistent family~$\mc{F}$ such that~$\{\mathbbm{1}_{F} : F \in \mc{F}\} = \{\mathbbm{1}_{F} : F \in \mc{F}'\}$. Furthermore, Definition~\ref{def:edge-consistency} is convenient for how we express our characterization, since we associate each forest~$F \in \mc{F}$ with both its subgraphs (see Fact~\ref{fact:downward}) and its vertex set~$V(F)$ (see Definition~\ref{def:lower_bound}). Definition~\ref{def:edge-consistency} thus ensures that forests with identical edge sets but different vertex sets are treated in the same way.


\subsection{The main characterization}
\label{subsection:blocking_property}

We start by deriving necessary conditions for an edge-consistent family of forests~$\mc{F}$ to be representable. Thus, let us assume for the moment that~$\P{f}$ represents~$\mc{F}$.

By the definition of RHS functions (Definition~\ref{def:rhs_function}),~$\P{f}$ contains the origin, so we immediately obtain the following fact.
\begin{fact}
    \label{fact:trivial_forest}
    If~$\mc{F}$ is an edge-consistent representable family of forests, then~$\mc{F}$ contains all the forests with no edges (including the empty graph).
\end{fact}

\noindent
Another simple observation is that if~$x \in \P{f}$, then every~$y \leq x$ (componentwise) also belongs to~$\P{f}$, so~$\mc{F}$ must be downward closed (Definition~\ref{def:downward_closed}):
\begin{fact}
    \label{fact:downward}
    If~$\mc{F}$ is an edge-consistent representable family of forests, then~$\mc{F}$ is downward closed.
\end{fact}

\noindent
Note that if~$\mc{F}$ is representable but not edge-consistent, we may have that~$F \in \mc{F}$ while~$F' \subsetneq F$ does not belong to~$\mc{F}$ (but there exists~$F'' \in \mc{F}$ such that~$\mathbbm{1}_{F''} = \mathbbm{1}_{F'} \leq \mathbbm{1}_F$).

Fact~\ref{fact:downward} implies that feasible forests cannot contain \emph{minimal infeasible forests}, defined as follows. 

\begin{definition}
    \label{def:minimal_infeasible}
    Let~$\mc{F}$ be a family of forests. A forest~$F \subseteq G$ is \emph{minimal infeasible} with respect to~$\mc{F}$ if~$F \notin \mc{F}$ and every proper subgraph~$F' \subsetneq F$ belongs to~$\mc{F}$. The notation~$\mc{M}(\mc{F})$ denotes the set of all such minimal infeasible forests.
\end{definition}

Definition~\ref{def:minimal_infeasible} is key for our characterization of (edge-consistent) representable families of forests. For an intuition of why this is the case, consider Example~\ref{example:brp}: path~$(v_1, v_2, v_3)$ is infeasible but not minimal; on the other hand, path~$(v_1, v_2)$ is minimal infeasible, and no feasible solution can cover customers~$\{v_1, v_2\}$ using a single path (or route). This simple example illustrates that the forests in~$\M{\mc{F}}$ may impose lower bounds on the number of components that some feasible forests can have. Motivated by this observation, we introduce the following definitions. 

We remark that Definition~\ref{def:lower_bound} is stated with respect to a generic family of forests~$\mc{C} \subseteq \allforests$, as this will be convenient in later sections. However, for now, we always assume that~$\mc{C} = \allforests$, and in this case, we omit the~$\allforests$ for simplicity (so we write~$\ell_{\mc{F}}$ and~$\mc{B}(\mc{F})$ instead of~$\ell_{\mc{F}, \allforests}$ and~$\mc{B}(\mc{F}, \allforests)$).
\begin{definition}
    \label{def:lower_bound}
    Let~$\mc{F}$ and~$\mc{C}$ be two families of forests. Define~$\mc{B}(\mc{F}, \mc{C}) \coloneqq \{V(F) : F \in \M{\mc{F}} \cap \mc{C}\}$ and
    \begin{equation}
        \tag{$\ell_{\mc{F}, \mc{C}}$}
        \ell_{\mc{F}, \mc{C}}(S) \coloneqq 1 +
        \begin{cases}
            \max\{|F| : F \in \M{\mc{F}} \cap \mc{C}, V(F) = S\}, & \text{if~$S \in \mc{B}(\mc{F}, \mc{C})$,}\\
            0, & \text{otherwise.}
        \end{cases}
    \end{equation}
    for each~$\emptyset \subsetneq S \subseteq V$. For convenience,~$\ell_{\mc{F}, \mc{C}}(\emptyset) = 0$.
\end{definition}

\begin{definition}
\label{def:blocking_property}
    A family of forests~$\mc{F}$ has the \emph{minimal infeasibility property} if every~$F \in \mc{F}$ satisfies~$|F| \geq \lb{\mc{F}}{V(F)}$.
\end{definition}

\begin{lemma}
    \label{lemma:minimal_necessary}
    If~$\mc{F}$ is an edge-consistent representable family of forests, then~$\mc{F}$ has the minimal infeasibility property.
\end{lemma}
\begin{proof}
    Let~$\P{f}$ represent~$\mc{F}$ and suppose by contradiction that~$F \in \mc{F}$ is such that~$|F| \leq \lb{\mc{F}}{V(F)} - 1$. Since~$\lb{\mc{F}}{\emptyset} = 0$, we know that~$|F| \geq 1$, meaning that~$\lb{\mc{F}}{V(F)} \geq 2$. Definition~\ref{def:lower_bound} implies that there exists a forest~$H \in \M{\mc{F}}$ with~$V(H) = V(F)$ such that~$|H| = \lb{\mc{F}}{V(F)} - 1$. 
    
    By edge-consistency of~$\mc{F}$, no forest~$F' \in \mc{F}$ has the same edge set as~$H \notin \mc{F}$, so~$\mathbbm{1}_{H} \notin \{\mathbbm{1}_{F'} : F' \in \mc{F}\}$. We thus show the desired contradiction by proving that~$\mathbbm{1}_{H} \in \P{f} \cap \Z^E$. To do this, we use case analysis to verify that~$\mathbbm{1}_{H}$ satisfies the GSEC~$x(S) \leq |S| - \f{S}$, for every~$\emptyset \subsetneq S \subseteq V$.\\
    \noindent
    \textbf{Case~$|V(F) \cap S| = 0$:} Since~$V(H) = V(F)$, it follows from the definition of RHS functions  that~$0 = \mathbbm{1}_{H}(S) \leq |S| - \f{S}$. 

    \noindent
    \textbf{Case~$|V(F) \cap S| = |V(F)|$:} Since~$\P{f}$ represents~$\mc{F}$ and~$F \in \mc{F}$, we know that~$\mathbbm{1}_F \in \P{f}$. Hence,~$$\mathbbm{1}_{H}(S) = |V(H)| - (\lb{\mc{F}}{V(F)} - 1) \leq |V(F)| - |F| = \mathbbm{1}_{F}(S) \leq |S| - \f{S}.$$ 
    
    \noindent
    \textbf{Case~$0 < |V(F) \cap S| < |V(F)|$:} Let~$H'$ be the forest obtained by deleting from~$H$ all the vertices that are not in~$S$. By minimality of~$H$,~$H'$ belongs to~$\mc{F}$ and~$\mathbbm{1}_{H'} \in \P{f}$ (by representability). Hence,~$\mathbbm{1}_{H}(S) = \mathbbm{1}_{H'}(S) \leq |S| - \f{S}$.
\end{proof}

Although not used in our development, it is worth noting that, if~$\mc{F}$ satisfies the minimal infeasibility property, then all minimal infeasible forests spanning a given set of vertices have the same number of components.
\begin{claim}
    \label{claim:minimal_same_size}
    Let~$\mc{F}$ be a family of forests satisfying the minimal infeasibility property. Then, for every~$F, F' \in \M{\mc{F}}$ with~$V(F) = V(F')$, we have that~$|F| = |F'|$. 
\end{claim}
\begin{proof}
    Suppose by contradiction that~$F, F' \in \M{\mc{F}}$ are such that~$V(F) = V(F')$ and~$|F| < |F'|$. Since~$F$ and~$F'$ are minimal infeasible,~$F \subsetneq F'$ and~$F' \subsetneq F$, which implies that there exists~$e \in E(F) \setminus E(F')$. Let~$H$ be obtained from~$F$ by deleting edge~$e$, i.e.,~$V(H) = V(F)$ and~$E(H) = E(F) \setminus \{e\}$. By minimal infeasibility of~$F$,~$H$ is feasible. Moreover,~$|H| = |F| + 1 \leq |F'| \leq \lb{\mc{F}}{V(F)} - 1$, contradicting the minimal infeasibility property.
\end{proof}

Next, for any set function~$f : 2^V \to \R$, we define
\begin{equation}
    \label{def:phi_f}
    \tag{$\Phi(f)$}
    \Phi(f) \coloneqq \{F \in \allforests : |F'| \geq f(V(F')),~\forall F' \subseteq F \}. 
\end{equation}

\noindent
Using this notation, Definitions~\ref{def:downward_closed} and~\ref{def:blocking_property} can be concisely expressed as follows.
\begin{lemma}
    \label{lemma:phi_lower_bound}
    Let~$\mc{F}$ be a family of forests. Then~$\mc{F}$ is nonempty, downward closed and has the minimal infeasibility property if and only if~$\mc{F} = \Phiset{\lb{\mc{F}}}$.
\end{lemma}
\begin{proof}
    By the definition of~$\Phiset$, it is clear that if~$\mc{F} = \Phiset{\lb{\mc{F}}}$, then~$\mc{F}$ is nonempty, downward closed and has the minimal infeasibility property (note that~$\Phiset{\lb{\mc{F}}}$ always contains the empty graph). To show the other direction, let~$F$ be an arbitrary forest in~$G$.

    Since~$\mc{F}$ is nonempty and downward closed, it follows that~$\emptyset \in \mc{F}$. Hence, whenever~$F \notin \mc{F}$ there exists~$F' \subseteq F$ such that~$F' \in \M{\mc{F}}$. By the definition of~$\lb{\mc{F}}$,~$|F'| \leq \lb{\mc{F}}{V(F')} - 1$, meaning that both~$F'$ and~$F$ do not belong to~$\Phiset{\lb{\mc{F}}}$. To show the other direction of the inclusion, suppose that~$F \in \mc{F}$. Downward closedness implies that every~$F' \subseteq F$ belongs to~$\mc{F}$, while the minimal infeasibility property gives~$|F'| \geq \lb{\mc{F}}{V(F')}$. This proves that~$F \in \Phiset{\lb{\mc{F}}}$.
\end{proof}

Moreover,~$\Phiset{f}$ is representable whenever it is edge-consistent (and~$f$ is a RHS function).
\begin{lemma}
    \label{lemma:phi_representable}
    Let~$\f$ be a RHS function and suppose that~$\Phiset{f}$ is edge-consistent. Then~$\P{f}$ represents~$\Phiset{f}$.
\end{lemma}
\begin{proof}
    By the definition of representability (Definition~\ref{def:representation}), we need to prove that~$\P{f} \cap \Z^E = \{\mathbbm{1}_F : F \in \Phiset{f}\}$. Let~$F$ be an arbitrary forest in~$G$ and suppose first that~$F \notin \Phiset{f}$, so there exists~$F' \subseteq F$ such that~$|F'| < \f{V(F')}$. Then
    $$\mathbbm{1}_F(V(F')) \geq \mathbbm{1}_{F'}(V(F')) = |V(F')| - |F'| > |V(F')| - \f{V(F')},$$
    meaning that~$\mathbbm{1}_F \notin \P{f}$. 
    
    For the converse, suppose that~$F \in \Phiset{f}$. Our goal is to show that~$\mathbbm{1}_F \in \P{f}$. Take an arbitrary set~$\emptyset \subsetneq S \subseteq V$. By edge-consistency, we can add singletons to~$F$ to obtain~$H \supseteq F$ such that~$H \in \Phiset{f}$,~$S \subseteq V(H)$ and~$\mathbbm{1}_F = \mathbbm{1}_H$. Let~$F'$ be obtained from~$H$ by deleting the vertices that are not in~$S$. Since~$\Phiset{f}$ is downward closed, we know that~$|F'| \geq \f{V(F')}$, therefore,~$$\mathbbm{1}_F(S) = \mathbbm{1}_H(S) = \mathbbm{1}_{F'}(S) = |S| - |F'| \leq |S| - \f{S},$$ as desired.
\end{proof}

Combining Lemmas~\ref{lemma:phi_lower_bound} and~\ref{lemma:phi_representable} we obtain the following characterization of an edge-consistent representable family of forests.

\begin{theorem}
    \label{thm:characterization1}
    Let~$\mc{F}$ be an edge-consistent family of forests. Then~$\mc{F}$ is representable if and only if~$\mc{F} = \Phiset{\lb{\mc{F}}}$.
\end{theorem}
\begin{proof}
    Since~$\mc{F}$ is representable, it follows from Facts~\ref{fact:trivial_forest} and~\ref{fact:downward} and Lemma~\ref{lemma:minimal_necessary}, that~$\mc{F}$ is nonempty, downward closed and has the minimal infeasibility property. Lemma~\ref{lemma:phi_lower_bound} then yields~$\mc{F} = \Phiset{\lb{\mc{F}}}$. Conversely, Lemma~\ref{lemma:phi_representable} implies that~$\P{\lb{\mc{F}}}$ represents~$\mc{F}$.
\end{proof}

Lastly, before continuing our discussion, we offer two simple examples illustrating how Theorem~\ref{thm:characterization1} applies to families of forests that can and cannot be represented with GSECs.

\begin{example}
\label{example:cmst}
Let~$Q \in \Q_+$ and $d \in \Q^V_{+}$ be such that~$d_v \leq Q$ for all~$v \in V$. Consider the family of forests~$\mc{F}_{\tsc{cmst}} = \{ F \in \allforests : d(V(T)) \leq Q,~\forall \, \text{tree } T \in F \}$.
Clearly,~$\mc{F}_{\tsc{cmst}}$ is downward closed and contains all the forests with no edges. Moreover, any minimal infeasible forest with respect to~$\mc{F}_{\tsc{cmst}}$ is a tree~$T \subseteq G$ such that~$d(V(T)) > Q$. Therefore, for every~$\emptyset \subsetneq S \subseteq V$,~$\lb{\mc{F}_{\tsc{cmst}}}{S} \leq 2$, and equality implies that~$d(S) > Q$, so the vertices in~$S$ cannot be covered with a single tree. We thus conclude that~$\mc{F}_{\tsc{cmst}}$ satisfies the minimal infeasibility property, and by Theorem~\ref{thm:characterization1},~$\mc{F}_{\tsc{cmst}}$ is representable. This example is consistent with previous work showing that GSECs can be used to formulate the CMST~\cite{hall1996}.~\qed
\end{example}

\begin{example}
\label{example:degree}
Let~$b \in \Z^V_+$ be a vector of upper bounds on the degree of each vertex, and consider the family of forests~$\mc{F}_{\tsc{deg}} = \{ F \in \allforests : |\delta_F(v)| \leq b_v,~\forall v \in V(F) \}$. The family~$\mc{F}_{\tsc{deg}}$ is downward closed and contains all the forests with no edges. However, perhaps not surprisingly,~$\mc{F}_{\tsc{deg}}$ cannot be represented with GSECs. To see this, suppose that~$G$ is the complete graph,~$V = \{v_1, v_2, v_3, v_4\}$, and~$b_v = 2$, for all~$v \in V$. Consider the spanning trees~$T_1$ and~$T_2$ with~$E(T_1) = \{v_1v_2, v_1v_3, v_1v_4\}$ and~$E(T_2) = \{v_1v_2, v_2v_3, v_3v_4\}$. Since~$T_1$ is minimally infeasible, we have~$\lb{\mc{F}_{\tsc{deg}}}{V} \ge 2$. However,~$\mathbbm{1}_{T_2}(V) = 3 > |V| - 2 = 2$, which shows that the minimal infeasibility property fails.~\qed
\end{example}

\subsection{Different RHS functions}
\label{subsection:rhs}

Although Theorem~\ref{thm:characterization1} precisely identifies the conditions on a family of forests that guarantee its representability via GSECs, the set~$\P{\lb{\mc{F}}}$ may provide a weak polyhedral relaxation of the convex hull of~$\{\mathbbm{1}_F : F \in \mc{F}\}$. This weakness can be particularly undesirable when using the relaxation~$\P{\lb{\mc{F}}}$ in a branch-and-cut algorithm. In this sense, we now assume that~$\mc{F}$ is representable, and we ask which choices of RHS functions~$\f$ ensure that~$\P{f}$ represents~$\mc{F}$. 

Our first result shows that~$\P{\lb{\mc{F}}}$ is the weakest relaxation of this type. We remark that the following statements are presented in a somewhat general form, as this will be useful to prove the results in Section~\ref{section:extension}.
\begin{lemma}
    \label{lemma:weakest_relaxation}
    Let~$\mc{F}$ be a family of forests and let~$f$ be a RHS function such that~$\{\mathbbm{1}_F : F \in \mc{F}\} \subseteq \P{f}$. Then
    \begin{enumerate}[(a), leftmargin=*, align=left]
        \item for any forest~$F \in \M{\mc{F}}$,~$\mathbbm{1}_F \notin \P{f}$ implies that~$x(V(F)) \leq |V(F)| - |F| - 1$ is valid for~$\P{f}$; and \label{item:weakest_relaxation1}
        \item if~$\P{f}$ represents~$\mc{F}$, then~$\P{f} \subseteq \P{\lb{\mc{F}}}$. \label{item:weakest_relaxation2}
    \end{enumerate}
\end{lemma}
\begin{proof}
    To show item~\ref{item:weakest_relaxation1}, let~$F \in \M{\mc{F}}$ be such that~$\mathbbm{1}_F \notin \P{f}$. Let~$U = V(F)$ and suppose by contradiction that~$x(U) \leq |U| - |F| - 1$ is not valid for~$\P{f}$. We show that this implies that~$\mathbbm{1}_F$ satisfy all the GSECs~$x(S) \leq |S| - \f{S}$ defining~$\P{f}$, contradicting the choice of~$F$. By the definition of RHS functions, we assume without loss of generality that~$U \cap S \neq \emptyset$.

    \noindent
    \textbf{Case~$|U \cap S| = |U|$:} We first claim that~$\f{S} \leq |S \setminus U| + |F|$. To see this, suppose by contradiction that~$\f{S} \geq |S \setminus U| + |F| + 1$. For any~$\bar{x} \in \P{f}$,~$$\bar{x}(U) \leq \bar{x}(S) \leq |S| - \f{S} \leq |U| - |F| - 1,$$ contradicting the assumption that~$x(U) \leq |U| - |F| - 1$ is not valid for~$\P{f}$. Hence,~$|F| \geq \f{S} - |S \setminus U|$, which yields~$$\mathbbm{1}_F(S) = \mathbbm{1}_F(U) = |U| - |F| \leq |U| - (\f{S} -|S \setminus U|) = |S| - \f{S}.$$
    
    \noindent
    \textbf{Case~$0 < |V(F) \cap S| < |V(F)|$:} Let~$H$ be the forest obtained by deleting from~$F$ all the vertices that are not in~$S$. By minimality of~$F$, we know that~$H \in \mc{F}$. Since~$\mathbbm{1}_H \in \{\mathbbm{1}_{F'} : F' \in \mc{F}\} \subseteq \P{f}$, it follows that~$\mathbbm{1}_F(S) = \mathbbm{1}_H(S) \leq |S| - \f{S}$.\\

    To prove item~\ref{item:weakest_relaxation2}, we can just apply item~\ref{item:weakest_relaxation1} for each minimal infeasible forest defining~$\lb{\mc{F}}$. Specifically, let~$\emptyset \subsetneq S \subseteq V$ be such that~$\lb{\mc{F}}{S} \geq 2$, and let~$F \in \M{\mc{F}}$ be such that~$V(F) = S$ and~$|F| = \lb{\mc{F}}{V(F)} - 1$. Then, item~\ref{item:weakest_relaxation1} implies that~$x(S) \leq |S| - \lb{\mc{F}}{S}$ is valid for~$\P{f}$.
\end{proof}

On the other hand, perhaps not surprisingly, the strongest possible set~$\P{f}$ that represents~$\mc{F}$ is given by the following RHS function.
\begin{definition}
    \label{def:upper_bound}
    Let~$\mc{F}$ be a nonempty family of forests. Define the RHS function
    \begin{equation*}
        u_{\mc{F}}(S) \coloneqq \min \{|S| - |E(F) \cap E(S)| : F \in \mc{F}\},
    \end{equation*}
    for each~$\emptyset \subsetneq S \subseteq V$.
\end{definition}

\begin{lemma}
    \label{lemma:strongest_relaxation}
    Let~$\mc{F}$ be a family of forests. Then:
    \begin{enumerate}[(A), leftmargin=*, align=left]
        \item $\{\mathbbm{1}_F : F \in \mc{F}\} \subseteq \P{\ub{\mc{F}}}$; \label{item:strongest_relaxation1}
        \item if~$\P{f}$ contains~$\{\mathbbm{1}_F : F \in \mc{F}\}$, then~$\P{f}$ also contains~$\P{\ub{\mc{F}}}$; and \label{item:strongest_relaxation2}
        \item if~$\mc{F}$ is edge-consistent and representable, then~$\P{\ub{\mc{F}}}$ represents~$\mc{F}$.\label{item:strongest_relaxation3}
    \end{enumerate}
\end{lemma}
\begin{proof}
    We prove items~\ref{item:strongest_relaxation1} and~\ref{item:strongest_relaxation2} jointly. Suppose that~$\{\mathbbm{1}_F : F \in \mc{F}\} \subseteq \P{f}$ and take an arbitrary set~$\emptyset \subsetneq S \subseteq V$. Since~$x(S) \leq |S| - \f{S}$ is valid for~$\{\mathbbm{1}_F : F \in \mc{F}\}$, we have that
    \begin{align}
        \f{S} & \leq \min \{|S| - \mathbbm{1}_F(S) : F \in \mc{F}\} \nonumber \\
        & = \min\{|S| - |E(F) \cap E(S)| : F \in \mc{F} \} \nonumber \\
        & = \ub{\mc{F}}{S}. \nonumber
    \end{align}
    The inequality above shows that any point~$\bar{x}$ in~$\{\mathbbm{1}_F : F \in \mc{F}\}$ satisfies~$\bar{x}(S) \leq |S| - \ub{\mc{F}}{S}$, proving~\ref{item:strongest_relaxation1}. We have also shown that, for any~$\bar{x} \in \P{\ub{\mc{F}}}$, we have~$\bar{x}(S) \leq |S| - \ub{\mc{F}}{S} \leq |S| - \f{S}$, meaning that~\ref{item:strongest_relaxation2} also holds.

    Using Theorem~\ref{thm:characterization1}, we prove~\ref{item:strongest_relaxation3} by showing that~$\P{\ub{\mc{F}}} \cap \Z^E = \P{\lb{\mc{F}}} \cap \Z^E$. Since~$\P{\lb{\mc{F}}} \cap \Z^E = \{\mathbbm{1}_F : F \in \mc{F}\}$, item~\ref{item:strongest_relaxation1} gives~$\P{\ub{\mc{F}}} \cap \Z^E \supseteq \P{\lb{\mc{F}}} \cap \Z^E$. For the other side of the inclusion, apply item~\ref{item:strongest_relaxation2} with~$\f = \lb{\mc{F}}$ to obtain that~$\P{\ub{\mc{F}}} \subseteq \P{\lb{\mc{F}}}$.
\end{proof}

Applying Lemmas~\ref{lemma:weakest_relaxation} and~\ref{lemma:strongest_relaxation} with Theorem~\ref{thm:characterization1}, we close the section with the next characterization.

\begin{theorem}
    \label{thm:characterization2}
    Let~$\mc{F}$ be an edge-consistent family of forests. Then~$\P{f}$ represents~$\mc{F}$ if and only if~$\mc{F} = \Phiset{\lb{\mc{F}}}$ and~$\P{\ub{\mc{F}}} \subseteq \P{f} \subseteq \P{\lb{\mc{F}}}$.
\end{theorem}
\begin{proof}
    By Theorem~\ref{thm:characterization1} and Lemmas~\ref{lemma:weakest_relaxation} and~\ref{lemma:strongest_relaxation}, it suffices to show that, under the assumption that~$\mc{F} = \Phiset{\lb{\mc{F}}}$, we have that~$\P{\ub{\mc{F}}} \subseteq \P{f} \subseteq \P{\lb{\mc{F}}}$ implies that~$\P{f}$ represents~$\mc{F}$. Indeed,~$\P{\ub{\mc{F}}} \cap \Z^E \subseteq \P{f} \cap \Z^E \subseteq \P{\lb{\mc{F}}} \cap \Z^E$, and from Lemmas~\ref{lemma:phi_representable} and~\ref{lemma:strongest_relaxation} we know that both~$\P{\lb{\mc{F}}}$ and~$\P{\ub{\mc{F}}}$ represents~$\mc{F}$. Hence,~$\{\mathbbm{1}_F : F \in \mc{F}\} = \P{\ub{\mc{F}}} \cap \Z^E = \P{\lb{\mc{F}}} \cap \Z^E = \P{f} \cap \Z^E$.
\end{proof}
