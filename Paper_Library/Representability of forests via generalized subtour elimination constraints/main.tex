\documentclass[a4paper,11pt]{article}

\usepackage{anyfontsize}

\usepackage[margin=1.2in]{geometry}
\usepackage{amsmath,amsthm, amssymb, amsfonts, amsbsy}
\usepackage{thmtools} %
\usepackage{thm-restate} %
\usepackage{authblk}
\usepackage{booktabs}
\usepackage{colortbl}
\usepackage[pdftex,dvipsnames,table]{xcolor}
\usepackage{xfrac}
\usepackage{xargs}
\usepackage[utf8]{inputenc}
\usepackage[mathscr]{euscript}
\usepackage{xspace}
\usepackage{graphicx}
\usepackage{color}
\usepackage{enumerate}
\usepackage[english]{babel}
\usepackage{verbatim}
\usepackage{float}
\usepackage{soul}
\usepackage{rotating}
\usepackage{caption}
\usepackage[draft]{fixme}
\usepackage{longtable}
\usepackage{mathtools}
\usepackage[shortlabels]{enumitem}
\usepackage{tabularx}
\usepackage{bbm}
\usepackage[inkscapelatex=false]{svg}
\usepackage{subfig}
\usepackage{makecell}
\usepackage{xparse}

\definecolor{lightgray}{gray}{0.9}
\restylefloat{table}

\makeatletter
\newcommand{\multiline}[1]{%
  \begin{tabularx}{\dimexpr\linewidth-\ALG@thistlm}[t]{@{}X@{}}
    #1
  \end{tabularx}
}
\makeatother

\usepackage{algorithm}%
\usepackage[noend]{algpseudocode}%




\usepackage{hyperref}
\usepackage{cleveref}

\makeatletter
\newcounter{HALG@line}
\renewcommand{\theHALG@line}{\thealgorithm.\arabic{ALG@line}}
\makeatother

\hypersetup{
    colorlinks,
    linkcolor={red!50!black},
    citecolor={blue!50!black},
    urlcolor={blue!80!black}
}

\pdfstringdefDisableCommands{%
  \let\mytitle\@firstofone
}

\newenvironment{tightquote}
  {\begingroup\leftskip=2em \rightskip=2em \parindent=0pt \parskip=0pt}
  {\par\endgroup}

\newcommand{\ota}{\textcolor{red}}
\newcommand{\rfchg}[1]{\textcolor{blue}{#1}}

\providecommand{\keywords}[1]{\textit{Keywords:} #1}
\usepackage{natbib}

\newcommand{\T}{\mathsf{\scriptscriptstyle T}} %
\newcommand{\overbar}[1]{\mkern 1.5mu\overline{\mkern-1.5mu#1\mkern-1.5mu}\mkern 1.5mu} %

\DeclareMathOperator*{\argmax}{arg\,max}
\DeclareMathOperator*{\argmin}{arg\,min}

\newcommand{\R}{\mathbb{R}}
\newcommand{\Z}{\mathbb{Z}}
\newcommand{\N}{\mathbb{N}}
\newcommand{\B}{\mathbb{B}}
\newcommand{\Q}{\mathbb{Q}}
\newcommand{\true}{\textsc{true}}
\newcommand{\false}{\textsc{false}}
\newcommand{\NP}{\mathscr{NP}}
\newcommand{\Pclass}{\mathscr{P}}
\newcommand{\bigO}{\mathcal{O}}
\newcommand{\polytope}{\mathcal{P}}
\newcommand{\polytopeQ}{\mathcal{Q}}
\newcommand{\mb}[1]{\mathbb{#1}}
\newcommand{\mbbm}[1]{\mathbbm{#1}}
\newcommand{\tsc}[1]{\textsc{#1}}
\newcommand{\oa}[1]{\vec{#1}}
\newcommand{\mc}[1]{\mathcal{#1}}
\newcommand{\ttt}[1]{\texttt{#1}}
\newcommand{\allones}{\mathbbm{1}}

\newcommand{\conv}{\textsc{conv}}
\newcommand{\proj}{\textsc{proj}}
\newcommand{\lp}{\textsc{LP}}

\makeatletter
\newtheoremstyle{mystyle}%
{\topsep}{\topsep}
{\itshape}{}
{\bfseries}{}
{0.5em}
{\thmname{\@ifempty{#3}{#1}\@ifnotempty{#3}{#3}}}
\makeatother

\theoremstyle{mystyle}
\newtheorem*{mytheorem}{NAME}%

\theoremstyle{plain}
\newtheorem{theorem}{Theorem}
\newtheorem{proposition}{Proposition}
\newtheorem{corollary}{Corollary}
\newtheorem{lemma}{Lemma}
\newtheorem{claim}{Claim}
\newtheorem{fact}{Fact}

\theoremstyle{definition}
\newtheorem{problem}{Problem}
\newtheorem{question}{Question}
\newtheorem{definition}{Definition}
\newtheorem{example}{Example}
\newtheorem{remark}{Remark}
\newtheorem{assumption}{Assumption}

\newcommand{\routes}{\hyperlink{def:routes}{\mc{R}}}
\newcommand{\vrpprob}[1]{\hyperlink{problem:vrp}{\tsc{vrp-prob}(#1)}}
\NewDocumentCommand{\fcvrp}{g}{
  \hyperlink{def:fcvrp}{
    f_{\tsc{cvrp}}\IfValueT{#1}{(#1)}
  }
}
\NewDocumentCommand{\f}{g}{
  \hyperref[def:rhs_function]{
    f\IfValueT{#1}{(#1)}
  }
}
\newcommand{\vrpform}[1]{\hyperref[form:cvrp]{\tsc{vrp-form}(#1)}}
\NewDocumentCommand{\fbrp}{g}{
  \hyperlink{def:fbrp}{
    f_{\tsc{brp}}\IfValueT{#1}{(#1)}
  }
}
\newcommand{\routesbrp}{\hyperlink{def:routes_brp}{\mc{R}_{\tsc{brp}}}}
\renewcommand{\P}[1]{\hyperref[def:polytope]{\mc{P}(#1)}}
\newcommand{\allforests}{\hyperlink{def:omega}{\Omega}}
\newcommand{\M}[1]{\hyperref[def:minimal_infeasible]{\mc{M}(#1)}}
\NewDocumentCommand{\lb}{m g}{
  \hyperref[def:lower_bound]{
    \ell_{#1}\IfValueT{#2}{(#2)}
  }
}
\renewcommand{\B}[1]{\hyperref[def:lower_bound]{\mc{B}(#1)}}
\NewDocumentCommand{\Phiset}{g}{
  \hyperref[def:phi_f]{
    \Phi\IfValueT{#1}{(#1)}
  }%
}
\NewDocumentCommand{\ub}{m g}{
  \hyperref[def:upper_bound]{
    u_{#1}\IfValueT{#2}{(#2)}
  }
}
\newcommand{\Qpath}{\hyperref[example:vrp]{\mc{Q}_{\tsc{path}}}}
\newcommand{\Cpath}{\hyperref[example:vrp]{\mc{C}_{\tsc{path}}}}
\newcommand{\Ft}[1]{\hyperref[example:vrp]{\mc{F}(#1)}}
\newcommand{\propstar}{\hyperref[item:path_star]{$(\star)$}}
\NewDocumentCommand{\Thetaset}{g}{
  \hyperref[def:theta]{
    \Theta\IfValueT{#1}{(#1)}
  }%
}
\NewDocumentCommand{\fg}{m g}{
  \hyperref[proposition:subadditive_rhs_function]{
    f_{#1}\IfValueT{#2}{(#2)}
  }%
}
\newcommand{\U}{\hyperlink{def:uncertainty_set}{\mc{U}}}

\newtheoremstyle{named}{}{}{}{}{\bfseries}{.}{.5em}{#1 #3}
\theoremstyle{named}
\newtheorem*{mydefinition}{Definition}

\usepackage{silence}
\begin{document}
\title{Representability of forests via generalized subtour elimination constraints}

\author[1]{Matheus J. Ota\thanks{(mjota@uwaterloo.ca)}}
\affil[1]{University of Waterloo, Waterloo, Ontario, Canada}

\maketitle

\begin{abstract}
Generalized subtour elimination constraints (GSECs) are widely used in state-of-the-art exact algorithms for vehicle routing and network design problems, as their right-hand sides often capture problem-specific feasibility conditions of each solution component. In this work, we present the first characterization of the families of forests that can be represented as the integer points inside a polytope defined by GSECs. This result generalizes a recent framework developed for vehicle routing problems under uncertainty and broadens the applicability of GSEC-based formulations to a wider class of combinatorial problems. In particular, using our characterization, we recover vehicle routing formulations that could not be obtained with previous results. Additionally, we show that GSECs can naturally model a robust variant of the capacitated minimum spanning tree problem.
\end{abstract}

\noindent
\keywords{representability, generalized subtour elimination constraints, network design, vehicle routing, branch-and-cut}

\section{Introduction}

One of the most fundamental problems in combinatorial optimization is the traveling salesperson problem (TSP), formalized as early as 1832 (c.f. \cite[Ch 1]{ABCC07}).
In an instance of  TSP we are given a set of $n$ cities $V$ along with their pairwise symmetric distances, $c:V\times V \to\R_{\geq 0}$. The goal is to find a Hamiltonian cycle of minimum cost. In the metric TSP problem, which we study here, the distances satisfy the triangle inequality. Therefore, the problem is equivalent to finding a closed Eulerian connected walk of minimum cost.%\footnote{Given such an Eulerian cycle, we can use the triangle inequality to shortcut vertices visited more than once to get a Hamiltonian cycle.}

It is NP-hard to approximate TSP within a factor of $\frac{123}{122}$ \cite{KLS15}.  An algorithm of Christofides-Serdyukov~\cite{Chr76,Ser78} from four decades ago gives a $\frac32$-approximation for TSP.
Over the years there have been numerous attempts to improve the Christofides-Serdyukov algorithm and exciting progress has been made for various special cases of metric TSP, e.g., \cite{OSS11,MS11,Muc12,SV12,HNR21, KKO20, HN19, GLLM21}.
 Recently, ~\cite{KKO21} gave the first improvement for the general case by demonstrating that the so-called ``max entropy" algorithm of \cite{OSS11} gives a randomized $\frac{3}{2}-\epsilon$ approximation for some $\epsilon > 10^{-36}$.% (see \cite{VS20} for a historical note about TSP)

%After a long line of work %~\cite{Wol80,SW90,BP91,Goe95,CV00,GLS05,BM10,BC11,SWV12, HNR17,HN19, KKO20a} 
	%the best known approximation algorithm for the general case of the problem is $\frac{3}{2}-\epsilon$ for some $\epsilon > 10^{-36}$ due to ~\cite{KKO21}, a result that built upon the work of the third author, Saberi, and Singh ~\cite{OSS11}. 
	The method introduced in \cite{KKO21} exploits the optimum solution to the following linear programming relaxation of metric TSP studied by \cite{DFJ59,HK70,BG93}, also known as the subtour elimination LP:
\begin{equation}\label{eq:tsplp}
\begin{aligned}
	\min \quad& \sum_{u,v} x_{\{u,v\}} c(u,v)& \\
	\text{s.t.,} \quad &  \sum_{u} x_{\{u,v\}} = 2&\forall v\in V,\\
	& \sum_{u\in S, v\notin S} x_{\{u,v\}}\geq 2,&\forall S \subsetneq V, S\not= \emptyset\\
	& x_{\{u,v\}}\geq 0 &\forall u,v\in V.
\end{aligned}	
\end{equation} 
	
	 However, ~\cite{KKO21} did not show that the integrality gap of the subtour elimination polytope is bounded below $\frac{3}{2}$, and therefore did not make progress towards the ``4/3 conjecture" which posits that the integrality gap of LP \eqref{eq:tsplp} is $\frac{4}{3}$. In this work we remedy this discrepancy by proving the following theorem, improving upon the bound of $\frac{3}{2}$ from Wolsey~\cite{Wol80} in 1980:

\begin{theorem}\label{thm:main}
	Let $x$ be a solution to LP \eqref{eq:tsplp} for a TSP instance. For some absolute constant $\epsilon > 10^{-36}$, the \hyperlink{tar:alg}{max entropy algorithm} outputs a TSP tour with expected cost at most $\frac{3}{2}-\epsilon$ times the cost of $x$. Therefore the integrality gap of the subtour elimination LP is at most $\frac{3}{2} - \epsilon$. 
\end{theorem} 

To prove \cref{thm:main}, we amend Section 4 of \cite{KKO21} but keep the remainder of the analysis essentially the same. Unlike \cite{KKO21}, this argument now preserves the integrality gap by avoiding the use of the optimum solution in bounding the cost of the matching. See \cref{sec:overview} for a discussion of our new approach.
%We note that the analysis in this paper is not specialized to the max entropy algorithm (although we rely on many results from \cite{KKO21} to obtain \cref{thm:main} itself); instead, it is valid for any algorithm which samples a spanning tree from the support of a solution to LP \eqref{eq:tsplp} and then adds the minimum cost matching on the odd degree vertices of the tree.  
%Instead, we use the polygon representation of near minimum cuts \cite{Ben95,BG08} to bound  the cost of the matching (see the following section for an overview of our new findings). %An added benefit of avoiding the use of OPT in the analysis is  %We remark this makes the analysis constructive 
%We remark that this allows future analyses to explicitly compute and possibly utilize the relevant laminar family of near minimum cuts (whereas previously one needed to know OPT to find the laminar family used in the analysis in \cite{KKO21}).
%In particular, we show that to get a bound better than $\frac{3}{2}$ for this class of algorithm it is (essentially) sufficient to handle the case in which the near minimum cuts of $x$ are a laminar family.

\subsection{Other Consequences}
\paragraph{Path TSP} In recent exciting work, Traub, Vygen, Zenklusen \cite{TVZ20} showed that an $\alpha$-approximation algorithm for metric TSP can be used as a black box to get a $\alpha(1+\eps)$ approximation algorithm for Path TSP. This together with \cite{KKO21} implies that there is a $3/2-\eps$ approximation algorithm for Path TSP (for $\eps>10^{-36}$). On the other hand, it is a folklore result that the integrality gap of the natural LP relaxation of Path TSP is at least $3/2$.  Therefore, a consequence of the above theorem is that although the best possible approximation factors of the two problem are the same (up to polynomial reductions), the natural LP relaxation of metric TSP has a strictly smaller integrality gap.


\paragraph{2-ECSM} In the 2-edge-connected multi-subgraph problem, or 2-ECSM for short, we are given a weighted graph $G$ and we want to find a minimum cost 2-edge-connected spanning subgraph, where an edge can be chosen multiple times.
The classical Christofides-Serdyukov algorithm gives a 3/2-approximation for 2-ECSM and despite significant attempts \cite{CR98,BFS16,SV14,BCCGISW20} improved algorithms were designed only for special cases of the problem.
Since in \cite{BG93} it is shown that LP \eqref{eq:tsplp} is a valid relaxation for 2-ECSM, we obtain:

\begin{corollary}	
For some absolute constant $\epsilon > 10^{-36}$ the \hyperlink{tar:alg}{max entropy algorithm} is a randomized $\frac{3}{2}-\epsilon$ approximation for the 2-edge-connected multi-subgraph problem.
\end{corollary}
%Beyond these theorems, we believe the analysis in this paper will open new avenues to improve the arguments in ~\cite{KKO21}. The analysis in that work is by nature non-constructive because it uses information about the optimal solution. Here we remove this weakness and could in principle construct the proposed fractional matching in polynomial time. Although of course this has no practical benefit since the algorithm always finds the minimum cost matching, this may allow future works to manipulate the algorithm to better serve the analysis.

%We analyze the max-entropy rounding algorithm introduced in \cite{OSS11} and slightly modified in \cite{KKO20, KKO21}. 

%In other words, we design a feasible vector for the $O$-join polytope to ``satisfy'' all near min cuts ``crossed on both  sides'' 


%Whereas Section 4 of ~\cite{KKO21} only deals with the near minimum cuts of $x$ (where $x$ is a solution to LP \eqref{eq:tsplp}) which lie along the optimal Hamiltonian cycle, we deal with all near minimum cuts of $x$ using the so-called polygon representation of near minimum cuts ~\cite{Ben97,BG08}. %The results give new intuition for the structure of cuts that are within $\frac{6}{5}$ or less of the edge connectivity of the graph.

 %: we show that we can incur a cost of $O(\eta^2) \cdot c(x)$ to ensure that the set of cuts with $x(\delta(S)) \le 2+\eta$ is a laminar family.


\subsection{New techniques and contributions}\label{sub:newtechniques}

This paper can be seen as a case study on how to reason about and deal with {\em near} minimum cuts. One can deduce from the classical cactus representation of a graph $G$ \cite{DKL76} (i) the structure of {\em all} min cuts of $G$ and (ii) the structure of the edges of $G$ in the sense that every edge $\{u,v\}$ maps to a unique {\em path} in the cactus between the images of $u$ and $v$. Furthermore, such a path intersects every cycle of the cactus on at most one cactus edge. The theory has found many application from designing fast algorithms
\cite{Kar00,KP09} to the analysis of approximation algorithms for TSP \cite{KKO20} and connectivity augmentation \cite{BGJ20,CTZ21}.

Two decades later, the theory of min cuts was extended to near min cuts in works of Bencz\'ur and Goemans \cite{Ben95, BG08} where they introduced the polygon representation which represents all cuts of a graph with at most $\frac{6}{5}k$ edges, where $k$ is its edge connectivity. Although these works completely classify the structure of all near min cuts of a given graph $G$, they do not characterize the structure of the \textit{edges} of $G$ with respect to these cuts, which can be important in applications (for example, in many of the recent applications of min cuts,
 one also needs to exploit the structure of the edges in relation to the cactus).
The structure on the edges turns out to be highly relevant in this work as well, and as a byproduct of our analysis we make progress towards classifying the way in which the edges of $G$ relate to the structure of the polygon representation.
 
 % and (to some extent) a classification of the set of edges of $G$ with respect to the polygon representation of Bencz\'ur and Goemans.
 
  %i
 %s to give a better understanding of the structure of edges of $G$ with respect to its near min cuts.

  %One can partition the edges of $G$ into sets $F_1\dots,F_m$ such that the set of edges in every min cut $(S,\overline{S})$ of $G$ is the union of edges in a pair $F_i,F_j$ for $i\ neq j$.
%\Nathan{Shayan can add something} For example...

For motivation, consider a generic family of network design problems in which we want to construct a network such that every pair $u,v$ of vertices has connectivity at least $c_{u,v}$. A natural approach is to write an LP relaxation to find a (minimum cost) vector $x: E \to \R_{\ge 0}$ such that for every cut $S$ separating $u$ and $v$, $x(\delta(S))\geq c_{u,v}$. We can round this LP using independent rounding or a dependent rounding scheme such as sampling from max entropy distributions. Using classical concentration bounds one can show that if $x(\delta(S))\gg c_{u,v}$ then with high probability the rounded solution has at least $c_{u,v}$ edges across this cut. So the main challenge is to ``fix'' near tight cuts, i.e., cuts where $x(\delta(S))\approx c_{u,v}$.  For an explicit instantiation of this scheme see \cite{KKOZ22}. A better understanding of the global structure of the family of near tight cuts has the potential to significantly simplify or even improve the approximation factor of such rounding algorithms. A classical technique to design algorithms for such network design problems is to apply uncrossing to extreme point solutions of the LP. One can view our contribution as an approximate uncrossing technique that deals with all near tight cuts (instead of just tight cuts) as we explain next.
%Next, we explain how our main theorem can be used to give global structure for near tight cuts in the case that $c_{u,v}=2$ for all $u,v$ and we contrast it with the classical uncrossing technique which only deals with tight/min cuts. 


\paragraph{An Approximate Uncrossing Technique.} A fundamental technique in the field of approximation algorithms is the uncrossing technique\footnote{See e.g. \cite{LRS11} for a number of applications of this technique.} of Jain \cite{Jai01}. Given a graph $G=(V,E)$,  a weight vector $x:E\to\R_{\geq 0}$, and a  function $f:V\to\R$, suppose that $x(\delta(S))\geq f(S)$ for all $S\subseteq V$. Let $\cN$ be the family of sets $S$ such that $x(\delta(S)) = f(S)$, i.e., the family of {\em tight} sets with respect to $f$. The uncrossing technique says that if $f$ is (weakly) supermodular then we can refine $\cN$ to a laminar family of sets, $\cH$, such that if all sets of $\cH$ are tight, then all sets of $\cN$ are tight as well. For a concrete example, suppose $f$ is a constant function, say $f(S)=2$ for all $\emptyset\subsetneq S\subsetneq V$. Then, sets of $\cH$ can be constructed using the cactus representation \cite{DKL76} of cuts in $\cN$. The significance of this method is that if $x$ is a basic feasible solution to a LP with constraints $x(\delta(S))\geq f(S)$ for all $S$, one can use this machinery to argue that the support of $x$ has size $O(|V|)$.

Informally, we prove the following, which 
can be seen as  an {\em approximate uncrossing technique}: 
\begin{theorem}[Informal]\label{thm:uncrossing}Suppose we have a vector $x:E\to\R_{\geq 0}$ such that $x(\delta(S))\geq f(S)$ for all $S$; define $\cN$ to be sets $S$ where $x(\delta(S))\leq f(S)(1+\eps)$ for some fixed $\eps>0$. If $f(.)$ is constant, say $f(S)=2$ for all $S$, then there is a set $\cN^*\subseteq \cN$ and a collection of edge sets $F_1,\dots,F_m\subseteq E$ such that the following hold:
\begin{itemize}
	\item $|\cN^*|= O(|V|)$, $m= O(|V|)$.
	\item $x(F_i)\geq 1-\eps/2$ for all $1\leq i\leq m$.
	\item Every edge $e$ is in at most $O(1)$ of the $F_i$'s.
	\item For every set $S\in \cN\smallsetminus \cN^*$ there exists $1\leq i<j\leq m$ such that $F_i\cap F_j=\emptyset$ and $F_i\cup F_j\subseteq \delta(S)$ and for every $S\in \cN^*$, there exists $1\leq i\leq m$ such that $F_i\subseteq \delta(S)$. 
\end{itemize}
\end{theorem}
In words, although we cannot simply refine $\cN$ to a linear number of sets, we can refine the edges in cuts of $\cN$ to a linear number of sets $F_1,\dots, F_m$ such  that we can essentially capture the edges of $\delta(S)$ for any $S\in \cN\smallsetminus \cN^*$ by a pair of disjoint $F_i$'s. We give a slightly weaker condition for cuts in $\cN^*$; namely we only capture half of their edges by $F_i$'s.

\begin{example}For a simple example of the above theorem, suppose $\eps=0$, i.e. $\cN$ is the set of min cuts of a graph $G$. Furthermore, suppose that every proper  cut in $\cN$ is \hyperlink{tar:crossing}{crossed} (recall that $S$ is proper if $1<|S|<|V|-1$) and that $\cN$ has at least one proper cut. 
Then, one can use an uncrossing technique, namely that if $A,B\in \cN$ then $A\cap B\in \cN$, to prove that $G$ must be cycle, namely we can order vertices of $G$, $v_0,\dots,v_{n-1}$ such that $x_{\{v_i,v_{i+1\text{ mod n}}\}}=1$.
In such a case we let $\cN^*=\emptyset$ and $F_i=E(v_i,v_{i+1\text{ mod }n})$.
%partition $V$ into sets $a_0,\dots,a_{m-1}$ such that 
%Let $\C$ be a connected component of crossing cuts of $\cN$, namely, for any pair of sets $A,B\in \C$ there is a path of crossing cuts all from $\C$ that goes from $A$ to $B$.
% and further suppose that $\cN$ can be represented by a cycle $C$ in the sense every min cut of $\cN$ corresponds to a min cut of $C$ and vice versa. Here we assume $a_0,\dots,a_{m-1}$ are the nodes of $C$ where each $a_i$ is identified with a disjoint set of vertices where $V=\uplus_{i=1}^m a_i$. In such a case, we can simply let $\cN^*=\emptyset$ and $F_i=E(a_i,a_{i+1\text{ mod }m})$. 
\label{eg:cycle}\end{example}

\begin{example}\label{eg:laminar}
For a second example, suppose again $\eps=0$ and $\cN$ is the set of mincuts of a graph $G$ where $\cN$ forms a laminar family (no two cuts cross). It turns out that we cannot decompose edges of cuts of $\cN$ into a linear sized collection of sets where every edge appears only a constant number of times. The main reason is that some edges may appear in an unbounded number of cuts. In this case we let $\cN^*=\cN$ and for every $A\in \cN$ (with immediate parent $B\in \cN$ in the laminar family) we add a set $F_A=\delta(A)\smallsetminus \delta(B)$  to our collection.  It is straightforward to show, using the structure of min cuts, that $x(F_A)\geq 1$; furthermore, since the size of a laminar family is linear in $V$, this gives a valid decomposition in the sense of above theorem.
\end{example}
Lastly, if $\eps=0$ and $\cN$ is the set of min cuts of an arbitrary graph, one can represent all min cuts of $\cN$ by a cactus \cite{DKL76} which can be seen as a tree of cycles. In such a case, one can use a construction similar to \cref{eg:cycle} for each cycle where instead of a vertex $v_i$ we have a set $a_i \subseteq V$ and one similar to \cref{eg:laminar} for the tree part of the cactus. For a concrete application of such a decomposition of min cuts see \cite{KKO20}.
%More generally, if $\cN$ corresponds to the set of min cuts of an arbitrary graph, the cuts of $\cN$ can be represented by a {\em cactus graph}. In such a case we add one $F_i$ for every edge of a cycle of the cactus. 


%and further for simplicity assume that there is a single connected component of crossing cuts in $\cN$, namely we can go from any $A$ to $B$ for $A,B\in\cN$ simply following crossing cuts of $\cN$. Then, one can represent cuts in $\cN$ by the set of min cuts of a cycle, namely we can contract vertices of $G$ 

%For a concrete application , suppose we need at least two edges in every set in $\cN^*$, say in a network optimization problem. Then, if we make sure that we have at least one edge in each $F_i$, all typical cuts constraints, $\cN\smallsetminus \cN^*$,  are satisfied, so we  reduce the problem to cuts in $\cN^*$. 


One of the main challenges in dealing with near min cuts relative to min cuts is that if $x(\delta(A)),x(\delta(B))\leq 2+\eps$ then $x(\delta(A\cap B))\leq 2+2\eps$. Therefore, if $\eps=0$, then min cuts are closed under intersection, set difference and union, but this is no longer true when $\eps>0$. So, to employ the classical uncrossing machinery one should be very careful to "uncross" only a constant number of times (independent of $\eps$) to make sure that every cut remains within $2+O(\eps)$. This is the main reason that the polygon representation of near min cuts (see below) is more sophisticated, e.g., we can no longer argue $x(E(a_i, a_{i+1}))\approx 1$, see \cref{fig:nearmincutbadexample}.

Although we don't study it here, we believe it may be worthwhile to find generalizations of \cref{thm:uncrossing} which hold for any (weakly) supermodular function.% That could be helpful in many questions based on the network optimization framework of Jain \cite{Jai01}.

\begin{remark} 
 We do not explicitly prove \cref{thm:uncrossing} in this extended abstract, as it is not used to prove \cref{thm:main}. However it can be deduced from arguments in \cref{sec:twoside} and \cref{app:oneside}. 
%In \cref{sec:overview} we discuss the main ideas of the proof of \cref{thm:uncrossing}. Here, let us explain the main challenge: In principal one might try to simply extend the above decomposition for the case $\eps=0$. The main challenge is that if $x(\delta(A)),x(\delta(B))\leq 2+\eps$ then $x(\delta(A\cap B))\leq 2+2\eps$. Therefore, if $\eps=0$, then min cuts are closed under intersection, set difference and union, but this is no longer true when $\eps>0$. So, to employ the classical uncrossing machinery one should be very careful to "uncross" only a constant number of times (independent of $\eps$) to make sure that every cut remains within $2+O(\eps)$. This is the main reason that the polygon representation of near min cuts (see below) is more sophisticated, e.g., we can no longer argue $x(E(a_i, a_{i+1}))\approx 1$, see \cref{fig:nearmincutbadexample}.
\end{remark}





\paragraph{Extensions to the Polygon Representation} To obtain our uncrossing framework we prove new properties of the polygon representation.
Given a graph $G=(V,E)$, let $k$ be the edge-connectivity of $G$, i.e. the number of edges in a minimum cut of $G$. For $\eps>0$, consider the set of $(1+\eps)$-near minimum cuts of $G$: cuts $(S,\overline{S})$ where $|E(S,\overline{S})| < (1+\eps)k$.
Bencz\'ur \cite{Ben95} and Bencz\'ur, Goemans \cite{BG08} proved that if $\eps \le 1/5$ then the near minimum cuts of $G$ admit a {\em polygon representation}. Namely, every connected component $\cC$ of \hyperlink{tar:crossing}{crossing} $(1+\eps)$ near min cuts can be represented by the diagonals of a convex polygon. In this polygon, the vertices of $G$ are partitioned into sets called \textit{atoms}, and every atom is mapped to a cell of this polygon defined by the diagonals and the boundary of the polygon itself (see \cref{sec:polyrep} for more details). 

The polygon representation can be seen as a generalization of the well-known cactus representation \cite{DKL76} of minimum cuts where a cycle of the cactus is replaced by a convex polygon. Unlike a cycle, some vertices/atoms map to the interior of the polygon, which are called ``inside'' atoms. The inside atoms at first look like a mystery and one can ask many questions about them such as how many can exist and what structures they can exhibit.



 Here, we explain two lemmas we proved which might find further applications beyond TSP in the future. 
%Our results give new intuition and understanding about the structure of polygon representations. These guide our analysis of the integrality gap of the subtour LP.
 %For example, one of our new observations is a 
 First, we give a necessary condition for a cell of a polygon to contain an inside atom:
\begin{lemma}[Informal, see \cref{thm:halfplanes}]
	Consider a polygon $P$ for a connected component $\C$ of a family of $1+\eps$ near min cuts for $\eps \le 1/5$ (where representing diagonals correspond to cuts in $\C$). Any cell of $P$ that has an inside atom must have at least $\Omega(1/\eps)$ many sides. 
\end{lemma}
This can be seen as a generalization of \cite[Lem 22]{BG08} to the case in which the cell is allowed to be adjacent to vertices of the polygon $P$.

Now, we explain our second extension: it follows from the cactus representation of minimum cuts that for a graph $G$ and a min cut $S$ one can partition the set of all min cuts that cross $S$ into two groups ${\cal A}=\{A_1,\dots,A_k\}$ and ${\cal B}=\{B_1,\dots,B_l\}$ for some $k,l\geq 0$ such that $S\cap A_1\subseteq S\cap A_2 \subseteq \dots S\cap A_k$ and, similarly, $S\cap B_1\subseteq \dots\subseteq S\cap B_l$. We prove a generalization of this fact for near min cuts:
\begin{lemma}[Informal, see \cref{lem:crosschain}]
Consider the set of $1+\eps$ near min cuts of a graph $G$ for $\eps\leq 1/10$; for any such near min cut $S$, one can partition the $1+\eps$ near min cuts crossing $S$ into two groups ${\cal A}=\{A_1,\dots,A_k\}$ and ${\cal B}=\{B_1,\dots,B_l\}$ such that $S\cap A_1 \subseteq S\cap A_2\subseteq \dots \subseteq S\cap A_k$ and similarly for cuts in ${\cal B}$.
\end{lemma}

\subsection{Outline of rest of paper} After reviewing preliminaries in \cref{sec:prelims}, we give a high-level overview of our proof technique in \cref{sec:overview}. The main new technical contributions of this paper are in \cref{sec:polyrep} and  \cref{sec:twoside}. The remaining content of the paper essentially follows from ~\cite{KKO21}. %Therefore, the reader may want to skip \cref{sec:proof-of-main}. 



\section{Generalized subtour elimination constraints and edge-based formulations for vehicle routing problems}
\label{section:setup}

We begin our discussion with the definition of a generic class of VRPs. We are given a complete undirected graph~$G_0 = (V_0, E_0)$ with edge costs~$c \in \Q^{E_0}_+$. The set of vertices~$V_0 = \{0\} \dot\cup V$ contains a special vertex denoted~$0 \in V_0$ representing the \emph{depot}, while the remaining vertices in~$V$ represent the \emph{customers}. We have~$k \in \Z_{++}$ available vehicles, all located initially at the depot. For convenience, we also set~$G = G_0 - \{0\} = (V, E)$, that is,~$G$ is the graph obtained from~$G_0$ by removing the depot.

Let \hypertarget{def:routes}{$\mc{R}$} be a family of feasible paths (or routes) in~$G$. We define problem \hypertarget{problem:vrp}{$\tsc{vrp-prob}(\mc{R})$} as follows. Let~$\mc{S} = \{C_1, \ldots, C_k\}$ be a set of~$k$ (simple) cycles, where each~$C_i \subseteq G_0$ contains the depot. Let~$P_i = C_i - \{0\} \subseteq G$ be the path obtained from~$C_i$ by deleting the depot. We say that~$\mc{S}$ is \emph{feasible} for~$\vrpprob{\mc{R}}$ if each~$P_i$ belongs to~$\routes$ and~$\{V(P_i)\}_{i \in [k]}$ forms a partition of~$V$. The objective of~$\vrpprob{\mc{R}}$ is to find a feasible solution~$\mc{S} = \{C_1, \ldots, C_k\}$ that minimizes the edge costs~$\sum_{i \in [k]} c(E_0(C_i))$.

All the VRPs mentioned in the introduction~\cite{laporte1983branch, gounaris2013robust, Dinh2018, ghosal2020, ghosal2024, dell2014bike} can be expressed as problems in the form of~$\vrpprob{\mc{R}}$.%
\footnote{
Some of these VRPs may allow solutions that use at most~$k$ cycles, and our reasoning extends naturally to this case. On the other hand, we do not address here VRPs that allow solutions visiting customers more than once, such as the VRPs with \emph{splittable demands}~\cite{archetti2008split}.}
For instance, the classical capacitated vehicle routing problem (CVRP) corresponds to problem~$\vrpprob{\mc{R}_{\tsc{cvrp}}}$, where~$\mc{R}_{\tsc{cvrp}} = \{\text{path~$P \subseteq G$} : d(V(P)) \leq C\}$,~$d \in \Q^V_+$ is a vector of customer demands, and~$Q \in \Q_+$ is the vehicle capacity. 

For any subset~$\emptyset \subsetneq S \subseteq V$, define \hypertarget{def:fcvrp}{$f_{\textsc{cvrp}}(S) \coloneqq \max\{1, \lceil d(S)/Q \rceil\}$}. A well-known CVRP formulation due to Laporte and Nobert~\cite{laporte1983branch} is as follows:
\begin{subequations}
\label{form:cvrp}
\begin{align}
    \tsc{vrp-form}(\fcvrp) \quad \quad \quad \min~~ & c^\T x \\
    \text{s.t.~~} & x(\delta_{G_0}(0)) = 2k, \label{eq:cvrp_depot} \\
    & x(\delta_{G_0}(v)) = 2, & \forall v \in V \label{eq:cvrp_customers} \\
    & x(S) \leq |S| - \fcvrp{S}, & \forall \emptyset \subsetneq S \subseteq V, \label{ineq:cvrp_gsec} \\
    & x_e \leq 1 + \mb{I}(0 \in e), & \forall e \in E_0, \\
    & x \in \Z^{E_0}_+,
\end{align}
\end{subequations}
where~$\mb{I}(\,\cdot\,)$ denotes the indicator function. Inequalities~\eqref{ineq:cvrp_gsec} are named GSECs, since they reduce to the standard \emph{subtour elimination constraints} (SECs) whenever~$\fcvrp{S} = 1$, for all~$\emptyset \subsetneq S \subseteq V$ (see~\cite{TothV02, laporte1986generalized}).

To associate Formulation~\eqref{form:cvrp} with different VRPs, we introduce the following definition.
\begin{definition}
    \label{def:rhs_function}
    A function~$f : 2^{V} \to \Z_+$ is a \emph{RHS function} if~$f(S) \in \{1, \ldots, |S|\}$, for every~$\emptyset \subsetneq S \subseteq V$. For convenience, the function~$f$ also satisfies~$f(\emptyset) = 0$.
\end{definition}
IP formulations for many VRPSD variants~\cite{gounaris2013robust, Dinh2018, ghosal2020, ghosal2024} can be obtained from Formulation~\eqref{form:cvrp} by just replacing the RHS function~$\fcvrp$.%
\footnote{Certain formulations instead use inequalities~$x(\delta_{G_0}(S)) \geq 2 \f{S}$. These inequalities can be shown to be equivalent to the GSECs by summing the degree constraints~\eqref{eq:cvrp_customers} over all~$v \in S$. When the formulation is instead defined on a complete directed graph~$D = (\{0\}\cup V, A)$ (with~$x \in \R^A$), the analogous inequalities~$x(\delta_D^+(S)) \geq \f{S}$ are also equivalent to the GSECs~$x(S) \le |S| - \f{S}$.}
The framework of Ghosal et al.~\cite{ghosal2024} partially explains this phenomenon, as they establish that~$\vrpprob{\routes}$ can be solved with~$\vrpform{f}$ (for some RHS function~$\f$) whenever the family of feasible paths~$\routes$ contains every path with at most one vertex and is both \emph{downward closed} and \emph{permutation invariant}, which we formally define next. 

For convenience in the following sections, we present these properties with respect to a general family of subgraphs~$\mc{H}$, rather than the family of feasible paths~$\routes$. Since general graphs cannot be written as tuples (as in the case of paths), we replace the term permutation invariance with \emph{vertex-consistency}.

\begin{definition}
    \label{def:downward_closed}
    A family of subgraphs~$\mc{H}$ of~$G$ is \emph{downward closed} if, for every~$F \in \mc{H}$ and~$F' \subseteq F$, we have that~$F' \in \mc{H}$.
\end{definition}

\begin{definition}
    \label{def:vertex_consistency}
    A family of subgraphs~$\mc{H}$ of~$G$ is \emph{vertex-consistent} if, for every~$F, F' \subseteq G$ with~$V(F) = V(F')$, we have that~$F \in \mc{H}$ if and only if~$F' \in \mc{H}$.
\end{definition}

\noindent
In fact, when applied to the family of feasible paths~$\routes$, the downward closedness property of Ghosal et al.~\cite{ghosal2024} is stronger than Definition~\ref{def:downward_closed}, as it states that if~$P = (v_1, \ldots, v_\ell) \in \routes$, then~$P' = (v_{i_1}, \ldots, v_{i_t}) \in \routes$, for every~$1 \leq i_1 < \ldots < i_t \leq \ell$. One can verify, however, that if~$\routes$ satisfies Definition~\ref{def:vertex_consistency}, then the two properties are equivalent.

As pointed out in Section~\ref{section:intro}, although the framework of Ghosal et al.~\cite{ghosal2024} unifies several VRPSD variants, it does not capture the GSEC-based formulation of Dell’Amico et al.~\cite{dell2014bike} for the BRP (or the 1-commodity pickup-and-delivery TSP formulation of~\cite{hernandez2003one}). In this problem, denoted~$\vrpprob{\mc{R}_{\tsc{brp}}}$, the vehicle load represents bikes, while the demands correspond to the number of bikes that must be picked up or delivered at each station. Vehicles are located at a central depot and start their routes with an initial load between~$0$ and~$Q \in \Q_+$. Customer demands~$d_v$ can be positive or negative (with~$|d_v| \leq Q$), and the accumulated load along a route must always remain within the interval~$[0, Q]$. Dell'Amico et al.~\cite{dell2014bike} show that the BRP can be formulated as~$\vrpform{f_{\tsc{brp}}}$, where, for every~$\emptyset \subsetneq S \subseteq V$, \hypertarget{def:fbrp}{$f_{\tsc{brp}}(S) \coloneqq \max\{1, \lceil |d(S)| / Q \rceil\}$}.

Formally, a path~$P = (v_1, \ldots, v_\ell)$ belongs to \hypertarget{def:routes_brp}{$\mc{R}_{\tsc{brp}}$} if and only if there exists an initial load~$q \in [0, Q] \cap \Q$ such that~$0 \leq q + \sum_{j \in [i]} d_{v_i} \leq Q$, for all~$i \in [\ell]$. Or equivalently (see Proposition~\ref{prop:brp}), if there exists~$q'$ such that~$0 \leq q' + \sum_{j = i}^\ell d_{v_i} \leq Q$, for all~$i \in [\ell]$. The following simple example shows that~$\routesbrp$ may not be vertex-consistent.

\begin{example}
\label{example:brp}
Suppose that~$Q = 1$,~$k = 1$ and we only have three customers~$v_1$,~$v_2$ and~$v_3$, with demands~$d_{v_1} = 1$,~$d_{v_2} = 1$ and~$d_{v_3} = -1$. Consider a route that starts at the depot and visits customers~$v_1, v_2, v_3$, in this order. This route corresponds to path~$P = (v_1, v_2, v_3)$, which does not belong to~$\routesbrp$, since~$d_{v_1} + d_{v_2} = 2 > 1 = Q$. On the other hand,~$P' = (v_1, v_3, v_2) \in \routesbrp$, as the vehicle can leave the depot with zero initial load and the accumulated load stays within~$[0, Q]$ at all times.~\qed
\end{example}

Example~\ref{example:brp} raises the question of which problems of the form~$\vrpprob{\mc{R}}$ can be modeled using GSECs but are not captured by the sufficient conditions of Ghosal et al.~\cite{ghosal2024}. To investigate this further, in Section~\ref{section:representation}, we drop the degree constraints~\eqref{eq:cvrp_depot} and~\eqref{eq:cvrp_customers} from Formulation~\eqref{form:cvrp}, and we characterize the forests that can be represented solely with the GSECs and the edge upper-bound constraints. Once these tools are developed, we reintroduce the degree constraints in Section~\ref{section:extension}.

\section{Representable families of forests}
\label{section:representation}

From now on, we fix~$G = (V, E)$ to be an arbitrary undirected graph (which may not be the same as the graph~$G_0 - \{0\}$ from Section~\ref{section:setup}). For any RHS function~$\f$, we define the polytope
\begin{equation}
    \label{def:polytope}
    \tag{$\mc{P}(f ; G)$}
    \mc{P}(f ; G) \coloneqq \{x \in [0, 1]^{E} : x(S) \leq |S| - \f{S},~~\forall \emptyset \subsetneq S \subseteq V\}.
\end{equation}
Since the graph~$G$ is fixed, we sometimes omit the dependence on~$G$ from the notation. In particular, we write~$\P{f}$ instead of~$\P{f; G}$. Additionally, to avoid repeating ourselves, whenever we write~$\P{f}$, it is implicitly assumed that~$\f$ is a RHS function.

The GSECs imply the SECs, so the integer vectors inside~$\P{f}$ correspond to forests in~$G$ (for any RHS function~$\f$). We use \hypertarget{def:omega}{$\allforests$} to denote the family of all forests in~$G$. The notion of \emph{representability} is formalized as follows.
\begin{definition}
    \label{def:representation}
    Let~$\mc{F}$ be a family of forests in~$G$ and let~$\mc{P} \subseteq \R^E$.
    We say that~$\mc{P}$ \emph{represents}~$\mc{F}$ if~$\mc{P} \cap \Z^{E} = \{\mathbbm{1}_{F} : F \in \mc{F}\}$. Furthermore,~$\mc{F}$ is \emph{representable} if there exists a RHS function~$\f$ such that~$\P{f}$ represents~$\mc{F}$.
\end{definition}

Definition~\ref{def:representation} identifies each forest in~$\mc{F}$ with its edge set. However, different forests may share the same edge set, as they might differ only by a set of isolated vertices. Consequently, the same set~$\P{f}$ may represent two distinct families of forests~$\mc{F}$ and~$\mc{F}'$, as long as their incidence vectors coincide. In this way, we often focus on \emph{edge-consistent} families of forests:
\begin{definition}
    \label{def:edge-consistency}
    A family of forests~$\mc{F}$ is \emph{edge-consistent} if, for every pair of forests~$F$ and~$F'$ in~$G$ with~$E(F) = E(F')$, we have that~$F \in \mc{F}$ if and only if~$F' \in \mc{F}$.
\end{definition}

The assumption of edge-consistency is without loss of generality: given any family of forests~$\mc{F}'$, one can always construct a unique edge-consistent family~$\mc{F}$ such that~$\{\mathbbm{1}_{F} : F \in \mc{F}\} = \{\mathbbm{1}_{F} : F \in \mc{F}'\}$. Furthermore, Definition~\ref{def:edge-consistency} is convenient for how we express our characterization, since we associate each forest~$F \in \mc{F}$ with both its subgraphs (see Fact~\ref{fact:downward}) and its vertex set~$V(F)$ (see Definition~\ref{def:lower_bound}). Definition~\ref{def:edge-consistency} thus ensures that forests with identical edge sets but different vertex sets are treated in the same way.


\subsection{The main characterization}
\label{subsection:blocking_property}

We start by deriving necessary conditions for an edge-consistent family of forests~$\mc{F}$ to be representable. Thus, let us assume for the moment that~$\P{f}$ represents~$\mc{F}$.

By the definition of RHS functions (Definition~\ref{def:rhs_function}),~$\P{f}$ contains the origin, so we immediately obtain the following fact.
\begin{fact}
    \label{fact:trivial_forest}
    If~$\mc{F}$ is an edge-consistent representable family of forests, then~$\mc{F}$ contains all the forests with no edges (including the empty graph).
\end{fact}

\noindent
Another simple observation is that if~$x \in \P{f}$, then every~$y \leq x$ (componentwise) also belongs to~$\P{f}$, so~$\mc{F}$ must be downward closed (Definition~\ref{def:downward_closed}):
\begin{fact}
    \label{fact:downward}
    If~$\mc{F}$ is an edge-consistent representable family of forests, then~$\mc{F}$ is downward closed.
\end{fact}

\noindent
Note that if~$\mc{F}$ is representable but not edge-consistent, we may have that~$F \in \mc{F}$ while~$F' \subsetneq F$ does not belong to~$\mc{F}$ (but there exists~$F'' \in \mc{F}$ such that~$\mathbbm{1}_{F''} = \mathbbm{1}_{F'} \leq \mathbbm{1}_F$).

Fact~\ref{fact:downward} implies that feasible forests cannot contain \emph{minimal infeasible forests}, defined as follows. 

\begin{definition}
    \label{def:minimal_infeasible}
    Let~$\mc{F}$ be a family of forests. A forest~$F \subseteq G$ is \emph{minimal infeasible} with respect to~$\mc{F}$ if~$F \notin \mc{F}$ and every proper subgraph~$F' \subsetneq F$ belongs to~$\mc{F}$. The notation~$\mc{M}(\mc{F})$ denotes the set of all such minimal infeasible forests.
\end{definition}

Definition~\ref{def:minimal_infeasible} is key for our characterization of (edge-consistent) representable families of forests. For an intuition of why this is the case, consider Example~\ref{example:brp}: path~$(v_1, v_2, v_3)$ is infeasible but not minimal; on the other hand, path~$(v_1, v_2)$ is minimal infeasible, and no feasible solution can cover customers~$\{v_1, v_2\}$ using a single path (or route). This simple example illustrates that the forests in~$\M{\mc{F}}$ may impose lower bounds on the number of components that some feasible forests can have. Motivated by this observation, we introduce the following definitions. 

We remark that Definition~\ref{def:lower_bound} is stated with respect to a generic family of forests~$\mc{C} \subseteq \allforests$, as this will be convenient in later sections. However, for now, we always assume that~$\mc{C} = \allforests$, and in this case, we omit the~$\allforests$ for simplicity (so we write~$\ell_{\mc{F}}$ and~$\mc{B}(\mc{F})$ instead of~$\ell_{\mc{F}, \allforests}$ and~$\mc{B}(\mc{F}, \allforests)$).
\begin{definition}
    \label{def:lower_bound}
    Let~$\mc{F}$ and~$\mc{C}$ be two families of forests. Define~$\mc{B}(\mc{F}, \mc{C}) \coloneqq \{V(F) : F \in \M{\mc{F}} \cap \mc{C}\}$ and
    \begin{equation}
        \tag{$\ell_{\mc{F}, \mc{C}}$}
        \ell_{\mc{F}, \mc{C}}(S) \coloneqq 1 +
        \begin{cases}
            \max\{|F| : F \in \M{\mc{F}} \cap \mc{C}, V(F) = S\}, & \text{if~$S \in \mc{B}(\mc{F}, \mc{C})$,}\\
            0, & \text{otherwise.}
        \end{cases}
    \end{equation}
    for each~$\emptyset \subsetneq S \subseteq V$. For convenience,~$\ell_{\mc{F}, \mc{C}}(\emptyset) = 0$.
\end{definition}

\begin{definition}
\label{def:blocking_property}
    A family of forests~$\mc{F}$ has the \emph{minimal infeasibility property} if every~$F \in \mc{F}$ satisfies~$|F| \geq \lb{\mc{F}}{V(F)}$.
\end{definition}

\begin{lemma}
    \label{lemma:minimal_necessary}
    If~$\mc{F}$ is an edge-consistent representable family of forests, then~$\mc{F}$ has the minimal infeasibility property.
\end{lemma}
\begin{proof}
    Let~$\P{f}$ represent~$\mc{F}$ and suppose by contradiction that~$F \in \mc{F}$ is such that~$|F| \leq \lb{\mc{F}}{V(F)} - 1$. Since~$\lb{\mc{F}}{\emptyset} = 0$, we know that~$|F| \geq 1$, meaning that~$\lb{\mc{F}}{V(F)} \geq 2$. Definition~\ref{def:lower_bound} implies that there exists a forest~$H \in \M{\mc{F}}$ with~$V(H) = V(F)$ such that~$|H| = \lb{\mc{F}}{V(F)} - 1$. 
    
    By edge-consistency of~$\mc{F}$, no forest~$F' \in \mc{F}$ has the same edge set as~$H \notin \mc{F}$, so~$\mathbbm{1}_{H} \notin \{\mathbbm{1}_{F'} : F' \in \mc{F}\}$. We thus show the desired contradiction by proving that~$\mathbbm{1}_{H} \in \P{f} \cap \Z^E$. To do this, we use case analysis to verify that~$\mathbbm{1}_{H}$ satisfies the GSEC~$x(S) \leq |S| - \f{S}$, for every~$\emptyset \subsetneq S \subseteq V$.\\
    \noindent
    \textbf{Case~$|V(F) \cap S| = 0$:} Since~$V(H) = V(F)$, it follows from the definition of RHS functions  that~$0 = \mathbbm{1}_{H}(S) \leq |S| - \f{S}$. 

    \noindent
    \textbf{Case~$|V(F) \cap S| = |V(F)|$:} Since~$\P{f}$ represents~$\mc{F}$ and~$F \in \mc{F}$, we know that~$\mathbbm{1}_F \in \P{f}$. Hence,~$$\mathbbm{1}_{H}(S) = |V(H)| - (\lb{\mc{F}}{V(F)} - 1) \leq |V(F)| - |F| = \mathbbm{1}_{F}(S) \leq |S| - \f{S}.$$ 
    
    \noindent
    \textbf{Case~$0 < |V(F) \cap S| < |V(F)|$:} Let~$H'$ be the forest obtained by deleting from~$H$ all the vertices that are not in~$S$. By minimality of~$H$,~$H'$ belongs to~$\mc{F}$ and~$\mathbbm{1}_{H'} \in \P{f}$ (by representability). Hence,~$\mathbbm{1}_{H}(S) = \mathbbm{1}_{H'}(S) \leq |S| - \f{S}$.
\end{proof}

Although not used in our development, it is worth noting that, if~$\mc{F}$ satisfies the minimal infeasibility property, then all minimal infeasible forests spanning a given set of vertices have the same number of components.
\begin{claim}
    \label{claim:minimal_same_size}
    Let~$\mc{F}$ be a family of forests satisfying the minimal infeasibility property. Then, for every~$F, F' \in \M{\mc{F}}$ with~$V(F) = V(F')$, we have that~$|F| = |F'|$. 
\end{claim}
\begin{proof}
    Suppose by contradiction that~$F, F' \in \M{\mc{F}}$ are such that~$V(F) = V(F')$ and~$|F| < |F'|$. Since~$F$ and~$F'$ are minimal infeasible,~$F \subsetneq F'$ and~$F' \subsetneq F$, which implies that there exists~$e \in E(F) \setminus E(F')$. Let~$H$ be obtained from~$F$ by deleting edge~$e$, i.e.,~$V(H) = V(F)$ and~$E(H) = E(F) \setminus \{e\}$. By minimal infeasibility of~$F$,~$H$ is feasible. Moreover,~$|H| = |F| + 1 \leq |F'| \leq \lb{\mc{F}}{V(F)} - 1$, contradicting the minimal infeasibility property.
\end{proof}

Next, for any set function~$f : 2^V \to \R$, we define
\begin{equation}
    \label{def:phi_f}
    \tag{$\Phi(f)$}
    \Phi(f) \coloneqq \{F \in \allforests : |F'| \geq f(V(F')),~\forall F' \subseteq F \}. 
\end{equation}

\noindent
Using this notation, Definitions~\ref{def:downward_closed} and~\ref{def:blocking_property} can be concisely expressed as follows.
\begin{lemma}
    \label{lemma:phi_lower_bound}
    Let~$\mc{F}$ be a family of forests. Then~$\mc{F}$ is nonempty, downward closed and has the minimal infeasibility property if and only if~$\mc{F} = \Phiset{\lb{\mc{F}}}$.
\end{lemma}
\begin{proof}
    By the definition of~$\Phiset$, it is clear that if~$\mc{F} = \Phiset{\lb{\mc{F}}}$, then~$\mc{F}$ is nonempty, downward closed and has the minimal infeasibility property (note that~$\Phiset{\lb{\mc{F}}}$ always contains the empty graph). To show the other direction, let~$F$ be an arbitrary forest in~$G$.

    Since~$\mc{F}$ is nonempty and downward closed, it follows that~$\emptyset \in \mc{F}$. Hence, whenever~$F \notin \mc{F}$ there exists~$F' \subseteq F$ such that~$F' \in \M{\mc{F}}$. By the definition of~$\lb{\mc{F}}$,~$|F'| \leq \lb{\mc{F}}{V(F')} - 1$, meaning that both~$F'$ and~$F$ do not belong to~$\Phiset{\lb{\mc{F}}}$. To show the other direction of the inclusion, suppose that~$F \in \mc{F}$. Downward closedness implies that every~$F' \subseteq F$ belongs to~$\mc{F}$, while the minimal infeasibility property gives~$|F'| \geq \lb{\mc{F}}{V(F')}$. This proves that~$F \in \Phiset{\lb{\mc{F}}}$.
\end{proof}

Moreover,~$\Phiset{f}$ is representable whenever it is edge-consistent (and~$f$ is a RHS function).
\begin{lemma}
    \label{lemma:phi_representable}
    Let~$\f$ be a RHS function and suppose that~$\Phiset{f}$ is edge-consistent. Then~$\P{f}$ represents~$\Phiset{f}$.
\end{lemma}
\begin{proof}
    By the definition of representability (Definition~\ref{def:representation}), we need to prove that~$\P{f} \cap \Z^E = \{\mathbbm{1}_F : F \in \Phiset{f}\}$. Let~$F$ be an arbitrary forest in~$G$ and suppose first that~$F \notin \Phiset{f}$, so there exists~$F' \subseteq F$ such that~$|F'| < \f{V(F')}$. Then
    $$\mathbbm{1}_F(V(F')) \geq \mathbbm{1}_{F'}(V(F')) = |V(F')| - |F'| > |V(F')| - \f{V(F')},$$
    meaning that~$\mathbbm{1}_F \notin \P{f}$. 
    
    For the converse, suppose that~$F \in \Phiset{f}$. Our goal is to show that~$\mathbbm{1}_F \in \P{f}$. Take an arbitrary set~$\emptyset \subsetneq S \subseteq V$. By edge-consistency, we can add singletons to~$F$ to obtain~$H \supseteq F$ such that~$H \in \Phiset{f}$,~$S \subseteq V(H)$ and~$\mathbbm{1}_F = \mathbbm{1}_H$. Let~$F'$ be obtained from~$H$ by deleting the vertices that are not in~$S$. Since~$\Phiset{f}$ is downward closed, we know that~$|F'| \geq \f{V(F')}$, therefore,~$$\mathbbm{1}_F(S) = \mathbbm{1}_H(S) = \mathbbm{1}_{F'}(S) = |S| - |F'| \leq |S| - \f{S},$$ as desired.
\end{proof}

Combining Lemmas~\ref{lemma:phi_lower_bound} and~\ref{lemma:phi_representable} we obtain the following characterization of an edge-consistent representable family of forests.

\begin{theorem}
    \label{thm:characterization1}
    Let~$\mc{F}$ be an edge-consistent family of forests. Then~$\mc{F}$ is representable if and only if~$\mc{F} = \Phiset{\lb{\mc{F}}}$.
\end{theorem}
\begin{proof}
    Since~$\mc{F}$ is representable, it follows from Facts~\ref{fact:trivial_forest} and~\ref{fact:downward} and Lemma~\ref{lemma:minimal_necessary}, that~$\mc{F}$ is nonempty, downward closed and has the minimal infeasibility property. Lemma~\ref{lemma:phi_lower_bound} then yields~$\mc{F} = \Phiset{\lb{\mc{F}}}$. Conversely, Lemma~\ref{lemma:phi_representable} implies that~$\P{\lb{\mc{F}}}$ represents~$\mc{F}$.
\end{proof}

Lastly, before continuing our discussion, we offer two simple examples illustrating how Theorem~\ref{thm:characterization1} applies to families of forests that can and cannot be represented with GSECs.

\begin{example}
\label{example:cmst}
Let~$Q \in \Q_+$ and $d \in \Q^V_{+}$ be such that~$d_v \leq Q$ for all~$v \in V$. Consider the family of forests~$\mc{F}_{\tsc{cmst}} = \{ F \in \allforests : d(V(T)) \leq Q,~\forall \, \text{tree } T \in F \}$.
Clearly,~$\mc{F}_{\tsc{cmst}}$ is downward closed and contains all the forests with no edges. Moreover, any minimal infeasible forest with respect to~$\mc{F}_{\tsc{cmst}}$ is a tree~$T \subseteq G$ such that~$d(V(T)) > Q$. Therefore, for every~$\emptyset \subsetneq S \subseteq V$,~$\lb{\mc{F}_{\tsc{cmst}}}{S} \leq 2$, and equality implies that~$d(S) > Q$, so the vertices in~$S$ cannot be covered with a single tree. We thus conclude that~$\mc{F}_{\tsc{cmst}}$ satisfies the minimal infeasibility property, and by Theorem~\ref{thm:characterization1},~$\mc{F}_{\tsc{cmst}}$ is representable. This example is consistent with previous work showing that GSECs can be used to formulate the CMST~\cite{hall1996}.~\qed
\end{example}

\begin{example}
\label{example:degree}
Let~$b \in \Z^V_+$ be a vector of upper bounds on the degree of each vertex, and consider the family of forests~$\mc{F}_{\tsc{deg}} = \{ F \in \allforests : |\delta_F(v)| \leq b_v,~\forall v \in V(F) \}$. The family~$\mc{F}_{\tsc{deg}}$ is downward closed and contains all the forests with no edges. However, perhaps not surprisingly,~$\mc{F}_{\tsc{deg}}$ cannot be represented with GSECs. To see this, suppose that~$G$ is the complete graph,~$V = \{v_1, v_2, v_3, v_4\}$, and~$b_v = 2$, for all~$v \in V$. Consider the spanning trees~$T_1$ and~$T_2$ with~$E(T_1) = \{v_1v_2, v_1v_3, v_1v_4\}$ and~$E(T_2) = \{v_1v_2, v_2v_3, v_3v_4\}$. Since~$T_1$ is minimally infeasible, we have~$\lb{\mc{F}_{\tsc{deg}}}{V} \ge 2$. However,~$\mathbbm{1}_{T_2}(V) = 3 > |V| - 2 = 2$, which shows that the minimal infeasibility property fails.~\qed
\end{example}

\subsection{Different RHS functions}
\label{subsection:rhs}

Although Theorem~\ref{thm:characterization1} precisely identifies the conditions on a family of forests that guarantee its representability via GSECs, the set~$\P{\lb{\mc{F}}}$ may provide a weak polyhedral relaxation of the convex hull of~$\{\mathbbm{1}_F : F \in \mc{F}\}$. This weakness can be particularly undesirable when using the relaxation~$\P{\lb{\mc{F}}}$ in a branch-and-cut algorithm. In this sense, we now assume that~$\mc{F}$ is representable, and we ask which choices of RHS functions~$\f$ ensure that~$\P{f}$ represents~$\mc{F}$. 

Our first result shows that~$\P{\lb{\mc{F}}}$ is the weakest relaxation of this type. We remark that the following statements are presented in a somewhat general form, as this will be useful to prove the results in Section~\ref{section:extension}.
\begin{lemma}
    \label{lemma:weakest_relaxation}
    Let~$\mc{F}$ be a family of forests and let~$f$ be a RHS function such that~$\{\mathbbm{1}_F : F \in \mc{F}\} \subseteq \P{f}$. Then
    \begin{enumerate}[(a), leftmargin=*, align=left]
        \item for any forest~$F \in \M{\mc{F}}$,~$\mathbbm{1}_F \notin \P{f}$ implies that~$x(V(F)) \leq |V(F)| - |F| - 1$ is valid for~$\P{f}$; and \label{item:weakest_relaxation1}
        \item if~$\P{f}$ represents~$\mc{F}$, then~$\P{f} \subseteq \P{\lb{\mc{F}}}$. \label{item:weakest_relaxation2}
    \end{enumerate}
\end{lemma}
\begin{proof}
    To show item~\ref{item:weakest_relaxation1}, let~$F \in \M{\mc{F}}$ be such that~$\mathbbm{1}_F \notin \P{f}$. Let~$U = V(F)$ and suppose by contradiction that~$x(U) \leq |U| - |F| - 1$ is not valid for~$\P{f}$. We show that this implies that~$\mathbbm{1}_F$ satisfy all the GSECs~$x(S) \leq |S| - \f{S}$ defining~$\P{f}$, contradicting the choice of~$F$. By the definition of RHS functions, we assume without loss of generality that~$U \cap S \neq \emptyset$.

    \noindent
    \textbf{Case~$|U \cap S| = |U|$:} We first claim that~$\f{S} \leq |S \setminus U| + |F|$. To see this, suppose by contradiction that~$\f{S} \geq |S \setminus U| + |F| + 1$. For any~$\bar{x} \in \P{f}$,~$$\bar{x}(U) \leq \bar{x}(S) \leq |S| - \f{S} \leq |U| - |F| - 1,$$ contradicting the assumption that~$x(U) \leq |U| - |F| - 1$ is not valid for~$\P{f}$. Hence,~$|F| \geq \f{S} - |S \setminus U|$, which yields~$$\mathbbm{1}_F(S) = \mathbbm{1}_F(U) = |U| - |F| \leq |U| - (\f{S} -|S \setminus U|) = |S| - \f{S}.$$
    
    \noindent
    \textbf{Case~$0 < |V(F) \cap S| < |V(F)|$:} Let~$H$ be the forest obtained by deleting from~$F$ all the vertices that are not in~$S$. By minimality of~$F$, we know that~$H \in \mc{F}$. Since~$\mathbbm{1}_H \in \{\mathbbm{1}_{F'} : F' \in \mc{F}\} \subseteq \P{f}$, it follows that~$\mathbbm{1}_F(S) = \mathbbm{1}_H(S) \leq |S| - \f{S}$.\\

    To prove item~\ref{item:weakest_relaxation2}, we can just apply item~\ref{item:weakest_relaxation1} for each minimal infeasible forest defining~$\lb{\mc{F}}$. Specifically, let~$\emptyset \subsetneq S \subseteq V$ be such that~$\lb{\mc{F}}{S} \geq 2$, and let~$F \in \M{\mc{F}}$ be such that~$V(F) = S$ and~$|F| = \lb{\mc{F}}{V(F)} - 1$. Then, item~\ref{item:weakest_relaxation1} implies that~$x(S) \leq |S| - \lb{\mc{F}}{S}$ is valid for~$\P{f}$.
\end{proof}

On the other hand, perhaps not surprisingly, the strongest possible set~$\P{f}$ that represents~$\mc{F}$ is given by the following RHS function.
\begin{definition}
    \label{def:upper_bound}
    Let~$\mc{F}$ be a nonempty family of forests. Define the RHS function
    \begin{equation*}
        u_{\mc{F}}(S) \coloneqq \min \{|S| - |E(F) \cap E(S)| : F \in \mc{F}\},
    \end{equation*}
    for each~$\emptyset \subsetneq S \subseteq V$.
\end{definition}

\begin{lemma}
    \label{lemma:strongest_relaxation}
    Let~$\mc{F}$ be a family of forests. Then:
    \begin{enumerate}[(A), leftmargin=*, align=left]
        \item $\{\mathbbm{1}_F : F \in \mc{F}\} \subseteq \P{\ub{\mc{F}}}$; \label{item:strongest_relaxation1}
        \item if~$\P{f}$ contains~$\{\mathbbm{1}_F : F \in \mc{F}\}$, then~$\P{f}$ also contains~$\P{\ub{\mc{F}}}$; and \label{item:strongest_relaxation2}
        \item if~$\mc{F}$ is edge-consistent and representable, then~$\P{\ub{\mc{F}}}$ represents~$\mc{F}$.\label{item:strongest_relaxation3}
    \end{enumerate}
\end{lemma}
\begin{proof}
    We prove items~\ref{item:strongest_relaxation1} and~\ref{item:strongest_relaxation2} jointly. Suppose that~$\{\mathbbm{1}_F : F \in \mc{F}\} \subseteq \P{f}$ and take an arbitrary set~$\emptyset \subsetneq S \subseteq V$. Since~$x(S) \leq |S| - \f{S}$ is valid for~$\{\mathbbm{1}_F : F \in \mc{F}\}$, we have that
    \begin{align}
        \f{S} & \leq \min \{|S| - \mathbbm{1}_F(S) : F \in \mc{F}\} \nonumber \\
        & = \min\{|S| - |E(F) \cap E(S)| : F \in \mc{F} \} \nonumber \\
        & = \ub{\mc{F}}{S}. \nonumber
    \end{align}
    The inequality above shows that any point~$\bar{x}$ in~$\{\mathbbm{1}_F : F \in \mc{F}\}$ satisfies~$\bar{x}(S) \leq |S| - \ub{\mc{F}}{S}$, proving~\ref{item:strongest_relaxation1}. We have also shown that, for any~$\bar{x} \in \P{\ub{\mc{F}}}$, we have~$\bar{x}(S) \leq |S| - \ub{\mc{F}}{S} \leq |S| - \f{S}$, meaning that~\ref{item:strongest_relaxation2} also holds.

    Using Theorem~\ref{thm:characterization1}, we prove~\ref{item:strongest_relaxation3} by showing that~$\P{\ub{\mc{F}}} \cap \Z^E = \P{\lb{\mc{F}}} \cap \Z^E$. Since~$\P{\lb{\mc{F}}} \cap \Z^E = \{\mathbbm{1}_F : F \in \mc{F}\}$, item~\ref{item:strongest_relaxation1} gives~$\P{\ub{\mc{F}}} \cap \Z^E \supseteq \P{\lb{\mc{F}}} \cap \Z^E$. For the other side of the inclusion, apply item~\ref{item:strongest_relaxation2} with~$\f = \lb{\mc{F}}$ to obtain that~$\P{\ub{\mc{F}}} \subseteq \P{\lb{\mc{F}}}$.
\end{proof}

Applying Lemmas~\ref{lemma:weakest_relaxation} and~\ref{lemma:strongest_relaxation} with Theorem~\ref{thm:characterization1}, we close the section with the next characterization.

\begin{theorem}
    \label{thm:characterization2}
    Let~$\mc{F}$ be an edge-consistent family of forests. Then~$\P{f}$ represents~$\mc{F}$ if and only if~$\mc{F} = \Phiset{\lb{\mc{F}}}$ and~$\P{\ub{\mc{F}}} \subseteq \P{f} \subseteq \P{\lb{\mc{F}}}$.
\end{theorem}
\begin{proof}
    By Theorem~\ref{thm:characterization1} and Lemmas~\ref{lemma:weakest_relaxation} and~\ref{lemma:strongest_relaxation}, it suffices to show that, under the assumption that~$\mc{F} = \Phiset{\lb{\mc{F}}}$, we have that~$\P{\ub{\mc{F}}} \subseteq \P{f} \subseteq \P{\lb{\mc{F}}}$ implies that~$\P{f}$ represents~$\mc{F}$. Indeed,~$\P{\ub{\mc{F}}} \cap \Z^E \subseteq \P{f} \cap \Z^E \subseteq \P{\lb{\mc{F}}} \cap \Z^E$, and from Lemmas~\ref{lemma:phi_representable} and~\ref{lemma:strongest_relaxation} we know that both~$\P{\lb{\mc{F}}}$ and~$\P{\ub{\mc{F}}}$ represents~$\mc{F}$. Hence,~$\{\mathbbm{1}_F : F \in \mc{F}\} = \P{\ub{\mc{F}}} \cap \Z^E = \P{\lb{\mc{F}}} \cap \Z^E = \P{f} \cap \Z^E$.
\end{proof}

\section{Extension and the case of vehicle routing problems}
\label{section:extension}

Building on Theorem~\ref{thm:characterization2}, we now extend our GSEC-based characterization to more general MIP formulations. Specifically, we consider formulations whose feasible regions can be represented as~$\P{f} \cap \mc{Q} \cap \Z^E$, where~$\mc{Q} \subseteq \R^E$ is associated with additional constraints that are not necessarily GSECs (note that such constraints may arise from the projection of a higher-dimensional polyhedron). 

Since the subtour elimination constraints are always valid for polytope~$\P{f}$, we assume without loss of generality that these inequalities are all valid for~$\mc{Q}$, so~$\mc{Q}$ represents a family of forests~$\mc{C}$. We further assume that the family~$\mc{C}$ is known, and our goal is to study which choices of RHS functions (if any) allow us to represent a target family of forests~$\mc{H} \subseteq \mc{C}$ as the integer vectors inside~$\P{f} \cap \mc{Q}$. The following example illustrates how this abstraction may apply in the context of Section~\ref{section:setup}.

\begin{example}
\label{example:vrp}
Consider the setup in Section~\ref{section:setup}, where~$G_0 = (V_0, E_0)$ is a complete undirected graph with~$V_0 = \{0\} \dot\cup V$ and~$E_0 = \{0v : v \in V\} \dot\cup E$. Recall that, in this case, we set~$G = (V, E) = G_0 - \{0\}$. Let~$\mc{C}_{\tsc{path}}$ be the family of all subgraphs in~$G$ whose components are paths, and observe that~$\mc{C}_{\tsc{path}}$ is represented by the polytope
\begin{equation*}
\mc{Q}_{\tsc{path}} =
\left\{x|_{E} \in \R^{E}:~
\begin{aligned}
& x(\delta_{G_0}(v)) = 2, && \forall v \in V \\
& x(S) \leq |S| - 1, && \forall \emptyset \subsetneq S \subseteq V \\
& 0 \leq x_e \leq 1 + \mb{I}(0 \in e), && \forall e \in E_0
\end{aligned}
\right\}.
\end{equation*}

\noindent
Hence, by characterizing the family of forests~$\mc{H} \subseteq \mc{C}_{\tsc{path}}$ that can be represented by~$\P{f} \cap \mc{Q}_{\tsc{path}}$, we consequently determine the types of VRPs that can be modeled as in formulation~$\vrpform{f}$ without constraint~\eqref{eq:cvrp_depot} (i.e.,~$x(\delta_{G_0}(0)) = 2k$).~\qed
\end{example}

The reason that we exclude the depot constraint~\eqref{eq:cvrp_depot} in Example~\ref{example:vrp} is that its inclusion could cause the set of forests represented by~$\Qpath$ to lose its downward-closedness property, which is essential for the following result.

\begin{proposition}
    \label{prop:characterization3}
    Let~$\mc{C}$ be a nonempty, edge-consistent and downward closed family of forests in~$G$, and let~$\mc{Q} \subseteq \R^E_+$ represent~$\mc{C}$. Let~$\mc{H} \subseteq \mc{C}$ be an edge-consistent family of forests. Then~$\P{f} \cap \mc{Q}$ represents~$\mc{H}$ if and only if
    \begin{enumerate}[(i), leftmargin=*, align=left]
        \item $\mc{H} = \Phiset{\ell_{\mc{H}, \mc{C}}} \cap \mc{C}$; and \label{item:downward_representation1}
        \item $\P{u_{\mc{H}}} \subseteq \P{f} \subseteq \P{\ell_{\mc{H}, \mc{C}}}$. \label{item:downward_representation2}
    \end{enumerate}
\end{proposition}
\begin{proof}
    Suppose that~$\P{f} \cap \mc{Q}$ represents~$\mc{H}$ and let~$\mc{F}$ be the edge-consistent family of forests represented by~$\P{f}$. Since~$\{\mathbbm{1}_F : F \in \mc{F} \cap \mc{C}\} = \{\mathbbm{1}_F : F \in \mc{H}\}$ and~$\mc{F}$,~$\mc{C}$ and~$\mc{H}$ are all edge-consistent, it follows that~$\mc{H} = \mc{F} \cap \mc{C}$.

    We first prove that~$\Phiset{\ell_{\mc{H}, \mc{C}}} \cap \mc{C} \subseteq \mc{H}$, let~$F$ be a forest in~$\mc{C} \setminus \mc{H}$. Since both~$\mc{C}$ and~$\mc{F}$ are nonempty and downward closed, we have that the empty graph belongs to~$\mc{C} \cap \mc{F} = \mc{H}$. Hence, by downward-closedness of~$\mc{C}$, there exists~$F' \in \M{\mc{H}} \cap \mc{C}$. By the definition of~$\lb{\mc{F}, \mc{C}}$,~$|F'| \leq \lb{\mc{F}, \mc{C}}{V(F')} - 1$, so~$F \notin \Phiset{\ell_{\mc{H}, \mc{C}}}$. To show the other direction of the inclusion, we use the following simple claim.\\
    
    \begin{minipage}{0.9\linewidth}
    \begin{claim}
    \label{claim:proof_minimal_subset}
    $\M{\mc{H}} \cap \mc{C} \subseteq \M{\mc{F}}$.
    \end{claim}
    \begin{proof}
    Let~$F \in \M{\mc{H}} \cap \mc{C}$. Since~$\mc{H} = \mc{F} \cap \mc{C}$, we have~$F \notin \mc{F}$. 
    Moreover, for every proper subgraph~$F' \subsetneq F$, we have~$F' \in \mc{H} \subseteq \mc{F}$. 
    Hence,~$F$ is minimal (with respect to inclusion) among the elements of~$\mc{C}$ that are not in~$\mc{F}$, 
    that is,~$F \in \M{\mc{F}}$.
    \end{proof}
    \end{minipage}\\[0.3cm]

    \noindent
    Claim~\ref{claim:proof_minimal_subset} implies that, for every~$\emptyset \subsetneq S \subseteq V$,~$\lb{\mc{H}, \mc{C}}{S} \leq \lb{\mc{F}}{S}$. Hence, by the definition of~$\Phiset$,~$\mc{H} = \mc{F} \cap \mc{C} = \Phiset{\lb{\mc{F}}} \cap \mc{C} \subseteq \Phiset{\lb{\mc{H}, \mc{C}}} \cap \mc{C}$.

    To show item~\ref{item:downward_representation2}, we observe that Claim~\ref{claim:proof_minimal_subset} also implies that, if~$F \in \M{\mc{H}} \cap \mc{C}$, then~$\mathbbm{1}_F \notin \P{f}$. By item~\ref{item:weakest_relaxation1}, inequality~$x(V(F)) \leq |V(F)| - |F| - 1$ is valid for~$\P{f}$, and therefore,~$\P{f} \subseteq \P{\lb{\mc{H}, \mc{C}}}$. Proving~$\P{\ub{\mc{H}}} \subseteq \P{f}$ is thus immediate from item~\ref{item:strongest_relaxation2} of Lemma~\ref{lemma:strongest_relaxation}.

    Let us now assume that both items~\ref{item:downward_representation1} and~\ref{item:downward_representation2} hold and, again, let~$\mc{F}$ be the edge-consistent forest represented by~$\P{f}$. Our goal is to show that~$\mc{F} \cap \mc{C} = \mc{H}$. Item~\ref{item:strongest_relaxation1} of Lemma~\ref{lemma:strongest_relaxation} gives~$\{\mathbbm{1}_F : F \in \mc{H}\} \subseteq \P{\ub{\mc{H}}} \subseteq \P{f}$, meaning that~$\{\mathbbm{1}_F : F \in \mc{H}\} \subseteq \P{f} \cap \Z^E = \{\mathbbm{1}_F : F \in \mc{F}\}$. Since~$\mc{F}$ is edge-consistent and~$\mc{H} \subseteq \mc{C}$, this implies that~$\mc{H} \subseteq \mc{F} \cap \mc{C}$. To prove the reverse inclusion, let~$F \in \mc{F} \cap \mc{C}$. Since~$\mathbbm{1}_F \in \P{f} \subseteq \P{\lb{\mc{H}, \mc{C}}}$, for every~$\emptyset \subsetneq S \subseteq V$,
    \begin{equation*}
        \mathbbm{1}_F(S) \leq |S| - \lb{\mc{H}, \mc{C}}{S}
        \iff |S| - |E(S) \cap E(F)| \geq \lb{\mc{H}, \mc{C}}{S}.
    \end{equation*}
    In particular, for every~$F' \subseteq F$ and~$S = V(F')$, we have that~$|F'| \geq \lb{\mc{H}, \mc{C}}{V(F')}$. Combining this with item~\ref{item:downward_representation1} we conclude that~$F \in \mc{H} = \Phiset{\lb{\mc{H}, \mc{C}}} \cap \mc{C}$.
\end{proof}

\paragraph{\textbf{Vehicle routing problems and componentwise feasibility.}}

In order to connect Proposition~\ref{prop:characterization3} with the VRPs discussed in Section~\ref{section:setup}, we recall that feasible solutions for these VRPs are composed of routes whose corresponding paths belong to a given family of feasible paths~$\routes$. In this sense, we introduce the following definition.
\begin{definition}
    \label{def:forest_tree}
    For any family of trees~$\mc{T}$ in~$G$, we define
    \begin{equation*}
        \label{def:f_t}
        \mc{F}(\mc{T}) \coloneqq \{F \in \Omega : T \in \mc{T},~\text{for every tree~$T \in F$} \}.
    \end{equation*}
\end{definition}

\noindent
Note that not every representable family of forests can be expressed as in Definition~\ref{def:f_t}. For example, the family of forests in~$G$ containing at most~$t \in \Z_{++}$ edges is not of the form~$\Ft{\mc{T}}$ but it can be represented with the GSEC~$x(V) \leq t$.

It follows directly from Definition~\ref{def:f_t} that we can simplify the formula for~$\lb{\mc{F}, \mc{C}}$ whenever~$\mc{F} = \Ft{\mc{T}}$.
\begin{claim}
    \label{claim:ft_lower_bound}
    Let~$\mc{C}$ be a family of forests and let~$\mc{T}$ be a family of trees in~$G$. Then, for every~$\emptyset \subsetneq S \subseteq V$,
    $$\lb{\Ft{\mc{T}}, \mc{C}}{S} = 1 + \mb{I}(S \in \B{\Ft{\mc{T}}, \mc{C}}).$$
\end{claim}
\begin{proof}
    It suffices to show that, for every~$S \in \B{\Ft{\mc{T}}, \mc{C}}$,~$\lb{\Ft{\mc{T}}, \mc{C}}{S} = 2$. Fix such a set~$S$ and let~$F \in \M{\Ft{\mc{T}}} \cap \mc{C}$ be such that~$V(F) = S$. Since~$F \notin \Ft{\mc{T}}$, there exists a tree~$T \in F$ such that~$T \notin \mc{T}$. Hence, by minimality of~$F$, we know that~$F = T$ and~$|F| = 1$, as desired.
\end{proof}

As in Ghosal et al.~\cite{ghosal2024}, let us assume that~$\routes$ contains all paths with at most one vertex. Hence, since~$\Ft{\routes}$ contains the empty graph, the minimal infeasible forests with respect to~$\Ft{\routes}$ have exactly one component (as otherwise they would not be minimal). Consider the sets~$\Cpath$ and~$\Qpath$ from Example~\ref{example:vrp}. 
Setting the target family of forests~$\mc{H}$ to~$\Ft{\routes}$ and substituting the definition of the set~$\Phiset{\lb{\mc{H}, \Cpath}}$ into Proposition~\ref{prop:characterization3}, we learn that there exists a RHS function~$\f$ such that~$\P{f} \cap \mc{Q}$ represents~$\Ft{\routes}$ if and only if
\begin{align}
    \Ft{\routes} & = \{F \in \Cpath : |F'| \geq \lb{\mc{F}(\routes), \Cpath}{V(F')},~\forall F' \subseteq F \} \nonumber \\
    & = \{F \in \Cpath : |F'| \geq 1 + \mb{I}(V(F') \in \B{\Ft{\routes}, \Cpath}),~\forall F' \subseteq F, F' \neq \emptyset \} \nonumber \\
    & = \{F \in \Cpath : 1 \geq 1 + \mb{I}(V(T) \in \B{\Ft{\routes}, \Cpath}),~\forall \text{tree~$T \subseteq F, T \neq \emptyset$} \} \nonumber \\
    & = \{F \in \Cpath : V(T) \notin \B{\Ft{\routes}, \Cpath},~\forall \text{tree~$T \subseteq F, T \neq \emptyset$}\}, \label{eq:routes_equivalence1}
\end{align}

\noindent
where the second equality follows from Claim~\ref{claim:ft_lower_bound}.

Therefore,
\begin{align}
    \routes & = \{F \in \Ft{\routes} : |F| \leq 1\} \nonumber \\
    & = \{\text{path~$P \subseteq G$} : V(P') \notin \B{\Ft{\routes}, \Cpath},~\forall \text{subpath~$P' \subseteq P, P' \neq \emptyset$}\},\label{eq:routes_equivalence2}
\end{align}
and note that~$\routes$ satisfies~\eqref{eq:routes_equivalence2} if and only if~$\Ft{\routes}$ satisfies~\eqref{eq:routes_equivalence1}.

Using essentially the same reasoning as that used to prove Lemma~\ref{lemma:phi_lower_bound},
it follows that~$\Ft{\routes}$ satisfies the above equation if and only if~$\routes$ is downward closed and it satisfies the following variant of the minimal infeasibility property:
\begin{itemize}[(I),leftmargin=*, align=left]
    \item[($\star$)] If~$P$ and~$P'$ are two paths in~$G$ with the same set of vertices, then~$P \in \M{\Ft{\routes}}$ implies~$P' \notin \routes$. \label{item:path_star}
\end{itemize}

\begin{claim}
    \label{claim:minimal_infeasible_path}
    Let~$\routes$ be a family of paths in~$G$ containing all paths with at most one vertex. Then~$\routes$ satisfies~\eqref{eq:routes_equivalence2} if and only if~$\routes$ is downward closed and satisfies property \propstar.
\end{claim}
\begin{proof}
    It is clear that if~\eqref{eq:routes_equivalence2} holds, then~$\routes$ is downward closed. To show that~$\routes$ satisfies \propstar, let~$P$ and~$P'$ be two paths in~$G$ with~$P \in \M{\Ft{\routes}}$ and~$V(P) = V(P')$. Since~$P \in \Cpath$, we know that~$V(P) \in \B{\Ft{\routes}, \Cpath}$, meaning that~$P' \notin \routes$.
    
    To prove the converse, let~$\mc{R}'$ be the set in the RHS of~\eqref{eq:routes_equivalence2}. Suppose that~$P \subseteq G$ is a path that does not belong to~$\routes$ (and thus, to~$\Ft{\routes}$). Since~$\Ft{\routes}$ contains the empty graph and~$\Cpath$ is downward closed, there exists a forest~$P' \subseteq P$ such that~$P' \in \M{\Ft{\routes}} \cap \Cpath$. Moreover, by Claim~\ref{claim:ft_lower_bound},~$P'$ is a path. Hence,~$V(P') \in \B{\Ft{\routes}, \Cpath}$ and~$P$ does not belong to~$\mc{R}'$. 
    
    To show the other side of the inclusion, assume that~$P \in \routes$. By property~\propstar, there exists no path~$P' \in \M{\Ft{\routes}}$ such that~$V(P) = V(P')$. Combining this observation with Claim~\ref{claim:ft_lower_bound} we learn that~$V(P) \notin \B{\Ft{\routes}, \Cpath}$. By downward closedness of~$\Ft{\routes}$, we can repeat the same argument for every subpath~$P''$ of~$P$, proving that~$P \in \mc{R}'$.
\end{proof}

Consequently, we obtain the following result.
\begin{corollary}
\label{corollary:ghosal}
Let~$\routes$ be a family of paths in~$G$ that contains all paths with at most one vertex, is downward closed, and satisfies \propstar. Then there exists a RHS function~$\f$ such that~$\bar{x} \in \R^{E_0}$ is feasible for~$\vrpform{\f}$ if and only if~$\bar{x}$ is the incidence vector of a solution to~$\vrpprob{\routes}$, i.e., there exists a feasible solution~$\mc{S} = \{C_1, \ldots, C_k\}$ for~$\vrpprob{\routes}$ such that, for every~$e \in E_0$,
\begin{equation}
    \label{eq:correspondence_vrp}
    \bar{x}_e = \sum_{i = 1}^k \mb{I}(e \in E(C_i)).
\end{equation}
\end{corollary}
\begin{proof}
    Suppose that~$\routes$ satisfies the conditions in the statement. By Claim~\ref{claim:minimal_infeasible_path}, $\routes$ satisfies~\eqref{eq:routes_equivalence2}. As~\eqref{eq:routes_equivalence2} is equivalent to~\eqref{eq:routes_equivalence1}, this is also equivalent to~$\Ft{\routes} = \Phiset{\lb{\Ft{\routes}, \Cpath}} \cap \Cpath$. Applying Proposition~\ref{prop:characterization3}, we learn that, for any RHS function~$\f$ such that~$\P{\ub{\Ft{\routes}}} \subseteq \P{\f} \subseteq \P{\lb{\Ft{\routes}, \Cpath}}$, the set~$\P{\f} \cap \Qpath$ represents~$\Ft{\routes}$. Therefore,
    \begin{equation*}
    \left\{x|_{E} \in \R^E :~
    \begin{aligned}
    & x(\delta_{G_0}(0)) = 2k, && \\
    & x|_E \in \P{\f} \cap \Qpath, && \\
    & 0 \leq x_e \leq 1 + \mb{I}(0 \in e), && \forall e \in E_0
    \end{aligned}
    \right\}
    \end{equation*}
    represents~$\{F \in \Ft{\routes} : |F| = k\}$, proving the statement.
\end{proof}

Since property~\propstar~is weaker than vertex-consistency (or permutation invariance, see Definition~\ref{def:vertex_consistency}), Corollary~\ref{corollary:ghosal} concretely establishes that, even when specialized for VRPs, Proposition~\ref{prop:characterization3} generalizes the result of Ghosal et al.~\cite{ghosal2024}.

\section{Subadditive functions and problem applications}
\label{section:applications}

Let~$\mc{F}$ be a family of forests. While the results in Section~\ref{subsection:rhs} establish that the strongest GSEC-based relaxation for~$\mc{F}$ is~$\P{\ub{\mc{F}}}$, computing~$\ub{\mc{F}}{S}$ may be too expensive, as it requires optimizing over~$\mc{F}$. To address this, we introduce here an approach that, under suitable conditions, allows one to easily obtain a RHS function~$\f$ such that~$\P{f} \subseteq \P{\lb{\mc{F}}}$ represents~$\mc{F}$.

Let~$g : 2^{V} \to \R_+$ be such that~$g(S) \leq |S|$, for every~$S \subseteq V$. In this section, we focus on families of forests of the form~$\mc{F}(\Theta(g))$, where
\begin{equation}
    \label{def:theta}
    \Theta(g) \coloneqq \{\text{tree~$T \subseteq G$} :  g(V(T')) \leq 1,~\forall \text{subtree~$T' \subseteq T$} \}.
\end{equation}

\noindent
It follows from Theorem~\ref{thm:characterization1} that, if~$\mc{F}(\mc{T})$ is representable and~$\mc{T}$ contains all trees with no edges, then we can assume without loss of generality that~$\mc{T} = \Thetaset{g}$.
\begin{proposition}
    \label{proposition:representable_f_g}
    Let~$g : 2^{V} \to \R_+$ be such that~$g(S) \leq |S|$, for every~$S \subseteq V$. Then~$\Ft{\Thetaset{g}}$ is representable. Moreover, if~$\mc{T}$ contains all trees with no edges and~$\Ft{\mc{T}}$ is representable, then~$\mc{T} = \Thetaset{\ell_{\Ft{\mc{T}}}}$.
\end{proposition}
\begin{proof}
    To ease notation, let~$\mc{F} = \Ft{\Thetaset{g}}$.
    To prove the first part of the statement, it suffices to show that that~$\mc{F} = \Phiset{\lb{\mc{F}}}$. By Claim~\ref{claim:ft_lower_bound}, we may write
    \begin{align*}
        \Phiset{\lb{\mc{F}}} & = \{F \in \allforests : |F'| \geq \lb{\mc{F}}{V(F')},~\forall F' \subseteq F \} \\
        & = \{F \in \allforests : 1 \geq 1 + \mb{I}(V(T) \in \B{\mc{F}}), ~\forall \text{tree~$T \subseteq F, T \neq \emptyset$}\} \\
        & = \{F \in \allforests : V(T) \notin \B{\mc{F}}, ~\forall \text{tree~$T \subseteq F, T \neq \emptyset$}\}
    \end{align*}

    \noindent
    Now, suppose that~$F$ is a forest that does not belong to~$\mc{F}$. Since~$\mc{F}$ contains the empty graph, there exists a subforest~$T \subseteq F$ such that~$T \in \M{\mc{F}}$. As~$V(T) \in \B{\mc{F}}$ and~$T$ is a tree (by minimal infeasibility), it follows that~$F \notin \Phiset{\lb{\mc{F}}}$. 
    
    Conversely, suppose that~$F \in \mc{F}$ and let~$T$ be a nonempty subtree of~$F$. Since~$\mc{F}$ is downward closed, we know that~$T \in \Thetaset{g}$. Therefore,~$g(V(T)) \leq 1$, which implies that~$V(T) \notin \B{\mc{F}}$. Indeed, suppose that there exists a tree~$T' \in \M{\mc{F}}$ such that~$V(T') = V(T)$. Then~$T' \notin \Theta(g)$ and every subtree~$T'' \subsetneq T'$ belongs to~$\Theta(g)$, meaning that~$g(V(T')) > 1$, a contradiction. This shows that~$F \in \Phiset{\lb{\mc{F}}}$.
    
    To close the proof, suppose that~$\Ft{\mc{T}}$ is representable, so~$\Ft{\mc{T}} = \Phiset{\lb{\Ft{\mc{T}}}}$. Then~$$\mc{T} = \{\text{tree~$T \in \Phiset{\lb{\Ft{\mc{T}}}}$}\} = \{\text{tree~$T \subseteq G$} : \lb{\Ft{\mc{T}}}{V(T')} \leq 1,~\forall \text{subtree~$T' \subsetneq T$}\}.$$
\end{proof}

Let~$\f$ be a RHS function and observe that~$\P{f}$ does not necessarily represent~$\Ft{\Thetaset{f}}$. For example, we might have~$\f(V) = |V|$ (so~$\P{f} = \{0\}$) while~$\Thetaset{f}$ contains a tree~$T$ in~$G$ with~$E(T) \neq \emptyset$. One can show, however, that if~$\f$ is a \emph{subadditive set function} --- that is,~$\f{A \cup B} \leq \f{A} + \f{B}$ for every~$A, B \subseteq V$ with~$A \cap B = \emptyset$\,%
\footnote{The standard definition of a subadditive set function requires this inequality to hold for all~$A, B \subseteq V$. In our setting, however, it suffices to consider only disjoint sets.}
--- then~$\P{\f}$ does represent~$\Ft{\Thetaset{\f}}$. Extending this reasoning, we obtain the following result.

\begin{proposition}
    \label{prop:subadditive_rhs_function}
    Let~$g : 2^{V} \to \R_+$ be a subadditive set function such that~$g(S) \leq |S|$, for every~$S \subseteq V$. Let~$\f$ be the RHS function given by~$\f(S) \coloneqq \max\{1, \lceil g(S) \rceil\}$, for all~$\emptyset \subsetneq S \subseteq V$. Then~$\P{\f}$ represents~$\Ft{\Thetaset{g}}$.
\end{proposition}
\begin{proof}
    Let~$F \subseteq G$ be a forest such that~$\mathbbm{1}_F \in \P{f_g} \cap \Z^E$. For every subtree~$T \in F$ and subtree~$T' \subseteq T$ with~$|V(T')| \geq 2$, we have that~$\mathbbm{1}_F(V(T')) = |V(T')| - 1 \leq |V(T')| - \f{V(T')}$, which implies that~$g(V(T')) \leq \f{V(T')} \leq 1$. This shows that~$T \in \Thetaset{g}$, and consequently,~$F \in \Ft{\Thetaset{g}}$.
    
    For the converse, suppose that~$F \in \Ft{\Thetaset{g}}$. To show that~$\mathbbm{1}_F \in \P{f}$ satisfy the GSECs, take an arbitrary set~$\emptyset \subsetneq S \subseteq V$ and let~$F'$ be the subgraph of~$F$ induced by~$S$. Since every tree~$T \in F'$ belongs to~$\Thetaset{g}$, we know that~$g(V(T)) \leq 1$. Hence, by subadditivity of~$\lceil g(\,\cdot\,) \rceil$,
    $$\mathbbm{1}_F(S) = |S| - |F'| \leq |S| - \sum_{T \in F'} \lceil g(V(T)) \rceil \leq |S| - \lceil g(S) \rceil \leq |S| - \f{S}.$$
\end{proof}

A particular subclass of subadditive set functions that will be convenient for us are the \emph{XOS} functions~\cite{feige2006maximizing}:
\begin{definition}
    \label{def:xos}
    We say that~$g : 2^{V} \to \R$ is \emph{XOS} with respect to a set~$\mc{W} \subseteq \R^{V}$ if~$g(S) = \max_{w \in \mc{W}} \{w(S)\}$, for all~$\emptyset \subsetneq S \subseteq V$. For convenience, we assume that~$g(\emptyset) = 0$.
\end{definition}
\begin{fact}
    \label{fact:xos}
    Every XOS set function~$g : 2^{V} \to \R$ is subadditive.
\end{fact}
\begin{proof}
    Let~$\mc{W}$ be such that~$g(S) = \max_{w \in \mc{W}} \{w(S)\}$, for all~$\emptyset \subsetneq S \subseteq V$.
    Let~$A, B \subseteq V$ be such that~$A \cap B = \emptyset$. Let~$\bar{w} \in \mc{W}$ be such that~$g(A \cup B) = \bar{w}(A \cup B)$. Then~$$g(A \cup B) = \bar{w}(A) + \bar{w}(B) \leq \max_{w \in \mc{W}}\{w(A)\} + \max_{w \in \mc{W}}\{w(B)\} = g(A) + g(B).$$
\end{proof}

In conclusion, to formulate a family of forests~$\Ft{\mc{T}}$ using GSECs, it suffices to find an XOS function~$g$ such that~$\mc{T} = \Thetaset{g}$ and~$g(S) \leq |S|$, for all~$S \subseteq V$. In what follows, we apply this approach to the bike sharing rebalancing problem and a robust capacitated minimum spanning tree problem. We again emphasize that these formulations cannot be obtained using the framework of Ghosal et al.~\cite{ghosal2024}.

\subsection{Bike sharing rebalancing problem}

Recall the definition of~$\routesbrp$ and~$\fbrp$ from Section~\ref{section:setup}, and let~$\Cpath$ and~$\Qpath$ be set as in Example~\ref{example:vrp}. To show that~$\vrpprob{\routesbrp}$ can be expressed as~$\vrpform{\fbrp}$, we begin with a simple lemma. Although this result was already discussed somewhat informally in~\cite{dell2014bike}, we include the proof for completeness.
\begin{lemma}
    \label{lemma:brp}
    Let~$P = (v_1, \ldots, v_\ell)$ be a path in~$G$. For each~$i \in [\ell]$, define~$D(i) \coloneqq \sum_{j \in [i]} d(v_i)$ (and~$D(0) = 0$). Moreover, define~$D_{\max}(i) \coloneqq \max_{j \in \{0, \ldots, i\}} \{D(j)\}$ and~$D_{\min}(i) \coloneqq \min_{j \in \{0, \ldots, i\}} \{D(j)\}$. The path~$P$ belongs to~$\routesbrp$ if and only if~$D_{\max}(i) - D_{\min}(i) \leq Q$, for all~$i \in [\ell]$. 
\end{lemma}
\begin{proof}
    Suppose that~$P \in \routesbrp$, meaning that there exists~$q$ such that~$0 \leq q + D(i) \leq Q$, for all~$i \in [\ell]$. Fix~$i \in [\ell]$ and note that~$q + D_{\max}(i) \leq Q$ and~$q + D_{\min}(i) \geq 0$. Therefore,~$q \geq - D_{\min}(i)$ and~$D_{\max}(i) - D_{\min}(i) \leq Q$.

    For the converse, assume that~$D_{\max}(i) - D_{\min}(i) \leq Q$, for all~$i \in [\ell]$. Set~$q = - D_{\min}(\ell)$ and observe that~$q + D(i) \geq 0$, for all~$i \in [\ell]$. Moreover,
    $$q + D(i) = D(i) - D_{\min}(\ell) \leq D_{\max}(\ell) - D_{\min}(\ell) \leq Q.$$
\end{proof}

Using Lemma~\ref{lemma:brp}, we obtain the following characterization of BRP-feasible paths.
\begin{proposition}
    \label{prop:brp}
    Let~$P = (v_1, \ldots, v_\ell)$ be a path in~$G$. Then~$P \in \routesbrp$ if and only if~$|\sum_{p = i}^j d_{v_p}| \leq Q$, for every~$0 < i \leq j \leq \ell$.
\end{proposition}
\begin{proof}
    Let~$D, D_{\max}$ and~$D_{\min}$ be as in the statement of Lemma~\ref{lemma:brp}.
    To prove the forward direction, assume that~$0 \leq q \leq Q$ is such that~$0 \leq q + D(j) \leq Q$, for all~$j \in [\ell]$. Then
    \begin{align*}
        & 0 \leq q + D(j) \leq Q \\
        \iff & 0 \leq q + D(i - 1) + (D(j) - D(i - 1)) \leq Q \\
        \iff & -q - D(i - 1) \leq D(j) - D(i - 1) \leq Q - q - D(i - 1) \\
        \implies & -Q \leq D(j) - D(i - 1) \leq Q,
    \end{align*}
    where the last line follows from~$0 \leq q + D(i - 1) \leq Q$ (recall that~$D(0) = 0$). We are thus done by the fact that~$D(j) - D(i - 1) = \sum_{p = i}^j d_{v_p}$.

    For the converse, we fix~$i \in [\ell]$ and we show that~$D_{\max}(i) - D_{\min}(i) \leq Q$ (by Lemma~\ref{lemma:brp}). Let~$j_{\max}$ and~$j_{\min}$ be such that~$D_{\max}(i) = D(j_{\max})$ and~$D_{\min}(i) = D(j_{\min})$. Suppose first that~$j_{\max} > j_{\min}$. Since~$|d(\{v_{j_{\min} + 1}, \ldots, v_{j_{\max}}\}) | \leq Q$, it follows that~$$D_{\max}(i) - D_{\min}(i) = D(j_{\max}) -  D(j_{\min}) = d((v_{j_{\min} + 1}, \ldots, v_{j_{\max}})) \leq Q.$$ On the other hand, if~$j_{\max} < j_{\min}$, we know that~$|d(\{v_{j_{\max} + 1}, \ldots, v_{j_{\min}}\}) | \leq Q$, meaning that~$$D_{\min}(i) - D_{\max}(i) = D(j_{\min}) -  D(j_{\max}) = d((v_{j_{\max} + 1}, \ldots, v_{j_{\min}})) \geq -Q.$$
\end{proof}

Now, for each~$\emptyset \subsetneq S \subseteq V$, define the XOS function~$$g_{\tsc{brp}}(S) \coloneqq \max \{d(S) / Q, -d(S) / Q \}.$$ By Proposition~\ref{prop:brp}, we have that~$\routesbrp = \Thetaset{g_{\tsc{brp}}} \cap \Cpath$. Moreover, Proposition~\ref{prop:subadditive_rhs_function} implies that~$\P{\fbrp}$ represents~$\Ft{\Thetaset{g_{\tsc{brp}}}}$, meaning that~$\P{\fbrp} \cap \Qpath$ represents~$$\Ft{\Thetaset{g_{\tsc{brp}}}} \cap \Cpath = \Ft{\Thetaset{g_{\tsc{brp}}} \cap \Cpath} = \Ft{\routesbrp}.$$ 

Therefore, every feasible solution~$\bar{x}$ to formulation~$\vrpform{\fbrp}$ corresponds to a feasible solution~$\mc{S} = \{C_1, \ldots, C_k\}$ for problem~$\vrpprob{\routesbrp}$ (where the ``correspondence'' is in the sense of Equation~\eqref{eq:correspondence_vrp}).

\subsection{Robust capacitated minimum spanning tree problem}

As in Section~\ref{section:setup}, let~$G_0 = (V_0, E_0)$ be a connected undirected graph with~$V_0 = \{0\} \dot\cup V$ and~$E_0 = \{0v : v \in V\} \dot\cup E$. Set~$G = G_0 - \{0\}$ and let $Q \in \Q_+$ be a capacity value. In the CMST~\cite{hall1992polyhedral, hall1996, uchoa2008robust}, each vertex~$v \in V$ has a demand~$d_v \in [0, Q] \cap \Q_+$, and the goal is to find a spanning tree~$T$ of~$G$, rooted at~$0$, and such that the total demand of each subtree~$T'$ hanging from~$0$ does not exceed $Q$. 

Inspired by previous work on the robust CVRP~\cite{gounaris2013robust, subramanyam2020robust, pessoa2021branch}, we now introduce the robust CMST (RCMST), where, instead of assuming that~$d \in \Q^V_+$ is deterministic, we only know that~$d$ belongs to a given \emph{uncertainty set} \hypertarget{def:uncertainty_set}{$\mc{U} \subseteq \R^V_+$}. The subtrees~$T'$ rooted at a child of~$0$ must then satisfy the robust capacity constraint~$\max_{d \in \U}\{d(V(T')\} \leq Q$.

For each~$\emptyset \subsetneq S \subseteq V$, define the XOS function~$$g_{\tsc{rcmst}}(S) \coloneqq \max_{d \in \U}\left\{d(S) / Q\right\}.$$ The set of trees in~$G$ satisfying the robust capacity constraints is given by~$\Thetaset{g_{\tsc{rcmst}}}$. Therefore, defining~$f_{\tsc{rcmst}}(\,\cdot\,) \coloneqq \max\{1, \lceil g_{\tsc{rcmst}}(\,\cdot\,) \rceil\}$, we have that~$\P{f_{\tsc{rcmst}}}$ represents~$\mc{F}_{\tsc{rcmst}} = \Ft{\Thetaset{g_{\tsc{rcmst}}}}$. In this way, we can formulate the RCMST as
\begin{subequations}
\label{form:rcmstp}
\begin{align}
    \min~~ & c^\T x \\
    \text{s.t.~~} & x(V_0) = |V|, \\
    & x(\{0\} \cup S) \leq |S|, & \forall \emptyset \subsetneq S \subseteq V, \label{ineq:rcmst_depot} \\
    & x|_E \in \P{f_{\tsc{rcmst}} ; G}, \\
    & x \in \Z^E,
\end{align}
\end{subequations}
where constraints~\eqref{ineq:rcmst_depot} enforce the subtour elimination constraints for subsets of vertices containing the depot.

When~$\U$ is a singleton, Formulation~\eqref{form:rcmstp} reduces to the CMST formulation of Hall~\cite{hall1996}. For budgeted and factor model uncertainty sets,~$g_{\tsc{rcmst}}(S)$ can be computed efficiently using the analytical solutions of Gounaris et al.~\cite{gounaris2013robust}.

\section{Conclusions}
In this paper, we set out to address the problem of multi-tasking robots in multi-robot tasks. 
%A fundamental limitation of existing multi-robot systems was addressed by the removal of a restrictive assumption that was often made--robots are single-tasking.
%Our method allowed coalitions to overlap thus enabling multi-tasking robots. 
We observed that the key underlying challenge was to reason about the physical constraints that could be synergistically satisfied.
%which directly affected the feasibility of multi-tasking.
To address the challenge, we developed our method based on the information invariant theory and modeled constraints as information instances. 
%This allowed us to reason about the relationships between constraints by reasoning about those between information requirements. 
Thereby, a formal and general framework to achieve multi-tasking robots was developed. 
We showed that our algorithm was sound and complete under our problem settings. 
%Our method was integrated with a simple greedy heuristic for task allocation.
Simulation  results  were  provided  to  show  the  effectiveness  of  our approach under resource-constrained situations and in handling challenging situations. % in a multi-UAV simulator. 

% The idea of multi-tasking is attractive in many ways. 
% Humans are living in multi-tasking environments--at any point of time, 
% we are optimizing for more than one task. 
% Multi-task often leads to more efficient task performance since it allows us to exploit task synergies. 
% The work presented in this paper takes us one step forward in realizing multi-tasking robots. 
% In particular, we started looking at the feasibility of multi-tasking. 
% There are many potential directions to pursue along this direction. First, several limitations are present in the current approach. 
% For example, although our method guarantees that there exists a physical configuration that satisfies all the constraints, it does not explicitly take the environmental influence into account. For example, a narrow corridor may prevent a robot formation from passing through, even though all the constraints for the formation do not introduce any conflicts. In this sense, our work should better be characterized as establishing a necessary condition for multi-tasking. Also, our method is mainly focused on the ``{\it planning}'' phase and hence does not address how the robots reach the desired configuration and maintain the constraints. These issues are assumed to be handled by the execution layer.

% More generally, the question of how to execute the tasks with overlapping coalitions is not addressed in this work. 
% As we already discussed, executing individual tasks with non-overlapping coalitions is straightforward but task synergies impose additional requirements on the task execution: how should the robots that are assigned multiple tasks execute them? Should they consider them in a prioritized strategy~\cite{van2005prioritized}? Or should they combine the different tasks in a way that is similar to motor schemas~\cite{arkin2}. 
% Communication requirements for maintaining the constraints must also be taken into account. How should the robots optimize their communication to improve the task performance? 

% The stringency of the physical constraints is another interesting question. It may be desirable to relax the constraints in certain situations (e.g., due to environmental influences). In such cases, it may be important to consider the problem where the constraints are least violated~\cite{kim2012revision}, or specify task constraints in different ways to increase the diversity of the configurations~\cite{srivastava2007domain} so as to make it robust to different environments. 

\bibliographystyle{abbrv}
\bibliography{bibliography}


\end{document}
