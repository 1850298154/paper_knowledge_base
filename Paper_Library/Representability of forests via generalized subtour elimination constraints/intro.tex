\section{Introduction}
\label{section:intro}

Several exact algorithms for network design and vehicle routing problems (VRPs) use edge-based integer programming (IP) formulations with \emph{generalized subtour elimination constraints} (GSECs)~\cite{TothV02, hall1996, dell2014bike, gounaris2013robust, Dinh2018, ghosal2020, ghosal2024, laporte1986generalized, laporte1983branch}. In these formulations, the right-hand side (RHS) values of the GSECs encode the problem-specific feasibility conditions of each component of a solution. Several of these problems are then solved using roughly the same branch-and-cut algorithm, with the only modification being in the RHS coefficients of the separated GSECs~\cite{TothV02, ghosal2024, Dinh2018, gounaris2013robust, dell2014bike}.

In the context of vehicle routing problems with stochastic demands (VRPSDs), Ghosal et al.~\cite{ghosal2024} recently proposed a framework that provides \emph{sufficient conditions} under which a set of feasible routes can be modeled using GSECs. More precisely, they have shown that a branch-and-cut approach based on GSECs can be applied whenever the set of feasible routes is both \emph{downward closed} and \emph{permutation invariant}. Using these conditions, they presented a unified argument for several existing exact algorithms for VRPSDs, including the robust VRPSD~\cite{gounaris2013robust}, the chance-constrained VRPSD~\cite{Dinh2018}, and the distributionally robust VRPSD~\cite{ghosal2020}. However, their framework does not provide a full characterization, as they also prove that these conditions are \emph{not necessary}. Indeed, as we show later in Section~\ref{section:setup}, the set of feasible routes for the bike sharing rebalancing problem (BRP) is not permutation invariant, but Dell'Amico et al.~\cite{dell2014bike} still successfully model this problem using GSECs.

To better understand the modeling power of GSECs, we abstract away the degree constraints in VRP formulations and we focus on identifying the families of forests that can be represented solely with GSECs. As a result of this study, we generalize and extend the framework of Ghosal et al.~\cite{ghosal2024} in several ways. Specifically, we make the following contributions.
\begin{itemize}
    \item We establish the first characterization (sufficient and necessary conditions) for determining whether a given family of forests can be represented using only GSECs (Section~\ref{subsection:blocking_property}). Additionally, when the conditions of this characterization are satisfied, we precisely identify the set of RHS values that can be used in the GSECs to represent the given family of forests (Section~\ref{subsection:rhs}).
    \item Based on this result, we also characterize the families of forests that can be represented using GSEC together with some additional constraints (Section~\ref{section:extension}). This result generalizes the conditions of Ghosal et al.~\cite{ghosal2024}, since it implies that a certain \emph{minimal infeasibility property} is a weaker condition than permutation invariance, yet it is still sufficient to model VRPs using GSECs.
    \item We provide a simplified approach to verify whether certain families of forests satisfy our conditions, and applying this method, we obtain GSEC-based formulations that could not be recovered with the framework of Ghosal et al.~\cite{ghosal2024} (Section~\ref{section:applications}). In particular, we recover the BRP formulation of Dell'Amico et al.~\cite{dell2014bike}, and we derive a GSEC-based formulation for a robust variant of the capacitated minimum spanning tree (CMST) problem~\cite{hall1992polyhedral, hall1996, uchoa2008robust}.
\end{itemize}

\noindent
Overall, these results highlight the potential of GSECs as a versatile modeling tool for a broader class of combinatorial optimization problems, extending the framework of Ghosal et al.~\cite{ghosal2024} beyond VRPSDs and even VRPs. \\

\noindent
\textbf{Notation:} Let~$G = (V, E)$ be an undirected graph. For each~$v \in V$,~$\delta(v)$ denotes the set of edges incident to~$v$ (or~$\delta_G(v)$ when the graph must be specified). For each~$S \subseteq V$,~$E(S)$ refers to the set of edges with both endpoints in~$S$. The notation~$H \subseteq G$ indicates that~$H$ is a subgraph of~$G$. If~$C_1, \ldots, C_t \subseteq G$ are the (connected) components of~$H$, then we may write~$H = \{C_1, \ldots, C_t\}$. In particular, if~$H$ is a forest,~$t = |H| = |V(H)| - |E(H)|$ (if~$H$ is the empty graph, then~$|H| = 0$). We represent any path~$P \subseteq G$ by a tuple of vertices~$(v_1, \ldots, v_t)$, that is, if~$P = (v_1, \ldots, v_t) \subseteq G$, then~$E(P) = \{v_i v_{i+1} : i \in [t-1]\}$ (we assume~$[0] = \emptyset$). Paths are always assumed to be simple, and~$P = (v_1, \ldots, v_t) = (v_t, \ldots, v_1)$.

For any vector (respectively, function)~$g$, we use~$g_i$ and~$g(i)$ interchangeably, and for any subset~$U$ of its indices (respectively, domain), we define~$g(U) \coloneqq \sum_{i \in U} g(i)$. For any vector~$x \in \R^E$ and~$S \subseteq V$, we use the notation~$x(S) \coloneqq x(E(S)) = \sum_{e \in E(S)} x_e$. Moreover, for each~$E' \subseteq E$, ~$x|_{E'}$ refers to the restriction (or projection) of~$x$ onto~$\R^{E'}$. Finally, for any~$H \subseteq G$, the vector~$\mathbbm{1}_H \in \R^E$ denotes the characteristic vector of~$E(H)$.
