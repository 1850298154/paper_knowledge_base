\section{Extension and the case of vehicle routing problems}
\label{section:extension}

Building on Theorem~\ref{thm:characterization2}, we now extend our GSEC-based characterization to more general MIP formulations. Specifically, we consider formulations whose feasible regions can be represented as~$\P{f} \cap \mc{Q} \cap \Z^E$, where~$\mc{Q} \subseteq \R^E$ is associated with additional constraints that are not necessarily GSECs (note that such constraints may arise from the projection of a higher-dimensional polyhedron). 

Since the subtour elimination constraints are always valid for polytope~$\P{f}$, we assume without loss of generality that these inequalities are all valid for~$\mc{Q}$, so~$\mc{Q}$ represents a family of forests~$\mc{C}$. We further assume that the family~$\mc{C}$ is known, and our goal is to study which choices of RHS functions (if any) allow us to represent a target family of forests~$\mc{H} \subseteq \mc{C}$ as the integer vectors inside~$\P{f} \cap \mc{Q}$. The following example illustrates how this abstraction may apply in the context of Section~\ref{section:setup}.

\begin{example}
\label{example:vrp}
Consider the setup in Section~\ref{section:setup}, where~$G_0 = (V_0, E_0)$ is a complete undirected graph with~$V_0 = \{0\} \dot\cup V$ and~$E_0 = \{0v : v \in V\} \dot\cup E$. Recall that, in this case, we set~$G = (V, E) = G_0 - \{0\}$. Let~$\mc{C}_{\tsc{path}}$ be the family of all subgraphs in~$G$ whose components are paths, and observe that~$\mc{C}_{\tsc{path}}$ is represented by the polytope
\begin{equation*}
\mc{Q}_{\tsc{path}} =
\left\{x|_{E} \in \R^{E}:~
\begin{aligned}
& x(\delta_{G_0}(v)) = 2, && \forall v \in V \\
& x(S) \leq |S| - 1, && \forall \emptyset \subsetneq S \subseteq V \\
& 0 \leq x_e \leq 1 + \mb{I}(0 \in e), && \forall e \in E_0
\end{aligned}
\right\}.
\end{equation*}

\noindent
Hence, by characterizing the family of forests~$\mc{H} \subseteq \mc{C}_{\tsc{path}}$ that can be represented by~$\P{f} \cap \mc{Q}_{\tsc{path}}$, we consequently determine the types of VRPs that can be modeled as in formulation~$\vrpform{f}$ without constraint~\eqref{eq:cvrp_depot} (i.e.,~$x(\delta_{G_0}(0)) = 2k$).~\qed
\end{example}

The reason that we exclude the depot constraint~\eqref{eq:cvrp_depot} in Example~\ref{example:vrp} is that its inclusion could cause the set of forests represented by~$\Qpath$ to lose its downward-closedness property, which is essential for the following result.

\begin{proposition}
    \label{prop:characterization3}
    Let~$\mc{C}$ be a nonempty, edge-consistent and downward closed family of forests in~$G$, and let~$\mc{Q} \subseteq \R^E_+$ represent~$\mc{C}$. Let~$\mc{H} \subseteq \mc{C}$ be an edge-consistent family of forests. Then~$\P{f} \cap \mc{Q}$ represents~$\mc{H}$ if and only if
    \begin{enumerate}[(i), leftmargin=*, align=left]
        \item $\mc{H} = \Phiset{\ell_{\mc{H}, \mc{C}}} \cap \mc{C}$; and \label{item:downward_representation1}
        \item $\P{u_{\mc{H}}} \subseteq \P{f} \subseteq \P{\ell_{\mc{H}, \mc{C}}}$. \label{item:downward_representation2}
    \end{enumerate}
\end{proposition}
\begin{proof}
    Suppose that~$\P{f} \cap \mc{Q}$ represents~$\mc{H}$ and let~$\mc{F}$ be the edge-consistent family of forests represented by~$\P{f}$. Since~$\{\mathbbm{1}_F : F \in \mc{F} \cap \mc{C}\} = \{\mathbbm{1}_F : F \in \mc{H}\}$ and~$\mc{F}$,~$\mc{C}$ and~$\mc{H}$ are all edge-consistent, it follows that~$\mc{H} = \mc{F} \cap \mc{C}$.

    We first prove that~$\Phiset{\ell_{\mc{H}, \mc{C}}} \cap \mc{C} \subseteq \mc{H}$, let~$F$ be a forest in~$\mc{C} \setminus \mc{H}$. Since both~$\mc{C}$ and~$\mc{F}$ are nonempty and downward closed, we have that the empty graph belongs to~$\mc{C} \cap \mc{F} = \mc{H}$. Hence, by downward-closedness of~$\mc{C}$, there exists~$F' \in \M{\mc{H}} \cap \mc{C}$. By the definition of~$\lb{\mc{F}, \mc{C}}$,~$|F'| \leq \lb{\mc{F}, \mc{C}}{V(F')} - 1$, so~$F \notin \Phiset{\ell_{\mc{H}, \mc{C}}}$. To show the other direction of the inclusion, we use the following simple claim.\\
    
    \begin{minipage}{0.9\linewidth}
    \begin{claim}
    \label{claim:proof_minimal_subset}
    $\M{\mc{H}} \cap \mc{C} \subseteq \M{\mc{F}}$.
    \end{claim}
    \begin{proof}
    Let~$F \in \M{\mc{H}} \cap \mc{C}$. Since~$\mc{H} = \mc{F} \cap \mc{C}$, we have~$F \notin \mc{F}$. 
    Moreover, for every proper subgraph~$F' \subsetneq F$, we have~$F' \in \mc{H} \subseteq \mc{F}$. 
    Hence,~$F$ is minimal (with respect to inclusion) among the elements of~$\mc{C}$ that are not in~$\mc{F}$, 
    that is,~$F \in \M{\mc{F}}$.
    \end{proof}
    \end{minipage}\\[0.3cm]

    \noindent
    Claim~\ref{claim:proof_minimal_subset} implies that, for every~$\emptyset \subsetneq S \subseteq V$,~$\lb{\mc{H}, \mc{C}}{S} \leq \lb{\mc{F}}{S}$. Hence, by the definition of~$\Phiset$,~$\mc{H} = \mc{F} \cap \mc{C} = \Phiset{\lb{\mc{F}}} \cap \mc{C} \subseteq \Phiset{\lb{\mc{H}, \mc{C}}} \cap \mc{C}$.

    To show item~\ref{item:downward_representation2}, we observe that Claim~\ref{claim:proof_minimal_subset} also implies that, if~$F \in \M{\mc{H}} \cap \mc{C}$, then~$\mathbbm{1}_F \notin \P{f}$. By item~\ref{item:weakest_relaxation1}, inequality~$x(V(F)) \leq |V(F)| - |F| - 1$ is valid for~$\P{f}$, and therefore,~$\P{f} \subseteq \P{\lb{\mc{H}, \mc{C}}}$. Proving~$\P{\ub{\mc{H}}} \subseteq \P{f}$ is thus immediate from item~\ref{item:strongest_relaxation2} of Lemma~\ref{lemma:strongest_relaxation}.

    Let us now assume that both items~\ref{item:downward_representation1} and~\ref{item:downward_representation2} hold and, again, let~$\mc{F}$ be the edge-consistent forest represented by~$\P{f}$. Our goal is to show that~$\mc{F} \cap \mc{C} = \mc{H}$. Item~\ref{item:strongest_relaxation1} of Lemma~\ref{lemma:strongest_relaxation} gives~$\{\mathbbm{1}_F : F \in \mc{H}\} \subseteq \P{\ub{\mc{H}}} \subseteq \P{f}$, meaning that~$\{\mathbbm{1}_F : F \in \mc{H}\} \subseteq \P{f} \cap \Z^E = \{\mathbbm{1}_F : F \in \mc{F}\}$. Since~$\mc{F}$ is edge-consistent and~$\mc{H} \subseteq \mc{C}$, this implies that~$\mc{H} \subseteq \mc{F} \cap \mc{C}$. To prove the reverse inclusion, let~$F \in \mc{F} \cap \mc{C}$. Since~$\mathbbm{1}_F \in \P{f} \subseteq \P{\lb{\mc{H}, \mc{C}}}$, for every~$\emptyset \subsetneq S \subseteq V$,
    \begin{equation*}
        \mathbbm{1}_F(S) \leq |S| - \lb{\mc{H}, \mc{C}}{S}
        \iff |S| - |E(S) \cap E(F)| \geq \lb{\mc{H}, \mc{C}}{S}.
    \end{equation*}
    In particular, for every~$F' \subseteq F$ and~$S = V(F')$, we have that~$|F'| \geq \lb{\mc{H}, \mc{C}}{V(F')}$. Combining this with item~\ref{item:downward_representation1} we conclude that~$F \in \mc{H} = \Phiset{\lb{\mc{H}, \mc{C}}} \cap \mc{C}$.
\end{proof}

\paragraph{\textbf{Vehicle routing problems and componentwise feasibility.}}

In order to connect Proposition~\ref{prop:characterization3} with the VRPs discussed in Section~\ref{section:setup}, we recall that feasible solutions for these VRPs are composed of routes whose corresponding paths belong to a given family of feasible paths~$\routes$. In this sense, we introduce the following definition.
\begin{definition}
    \label{def:forest_tree}
    For any family of trees~$\mc{T}$ in~$G$, we define
    \begin{equation*}
        \label{def:f_t}
        \mc{F}(\mc{T}) \coloneqq \{F \in \Omega : T \in \mc{T},~\text{for every tree~$T \in F$} \}.
    \end{equation*}
\end{definition}

\noindent
Note that not every representable family of forests can be expressed as in Definition~\ref{def:f_t}. For example, the family of forests in~$G$ containing at most~$t \in \Z_{++}$ edges is not of the form~$\Ft{\mc{T}}$ but it can be represented with the GSEC~$x(V) \leq t$.

It follows directly from Definition~\ref{def:f_t} that we can simplify the formula for~$\lb{\mc{F}, \mc{C}}$ whenever~$\mc{F} = \Ft{\mc{T}}$.
\begin{claim}
    \label{claim:ft_lower_bound}
    Let~$\mc{C}$ be a family of forests and let~$\mc{T}$ be a family of trees in~$G$. Then, for every~$\emptyset \subsetneq S \subseteq V$,
    $$\lb{\Ft{\mc{T}}, \mc{C}}{S} = 1 + \mb{I}(S \in \B{\Ft{\mc{T}}, \mc{C}}).$$
\end{claim}
\begin{proof}
    It suffices to show that, for every~$S \in \B{\Ft{\mc{T}}, \mc{C}}$,~$\lb{\Ft{\mc{T}}, \mc{C}}{S} = 2$. Fix such a set~$S$ and let~$F \in \M{\Ft{\mc{T}}} \cap \mc{C}$ be such that~$V(F) = S$. Since~$F \notin \Ft{\mc{T}}$, there exists a tree~$T \in F$ such that~$T \notin \mc{T}$. Hence, by minimality of~$F$, we know that~$F = T$ and~$|F| = 1$, as desired.
\end{proof}

As in Ghosal et al.~\cite{ghosal2024}, let us assume that~$\routes$ contains all paths with at most one vertex. Hence, since~$\Ft{\routes}$ contains the empty graph, the minimal infeasible forests with respect to~$\Ft{\routes}$ have exactly one component (as otherwise they would not be minimal). Consider the sets~$\Cpath$ and~$\Qpath$ from Example~\ref{example:vrp}. 
Setting the target family of forests~$\mc{H}$ to~$\Ft{\routes}$ and substituting the definition of the set~$\Phiset{\lb{\mc{H}, \Cpath}}$ into Proposition~\ref{prop:characterization3}, we learn that there exists a RHS function~$\f$ such that~$\P{f} \cap \mc{Q}$ represents~$\Ft{\routes}$ if and only if
\begin{align}
    \Ft{\routes} & = \{F \in \Cpath : |F'| \geq \lb{\mc{F}(\routes), \Cpath}{V(F')},~\forall F' \subseteq F \} \nonumber \\
    & = \{F \in \Cpath : |F'| \geq 1 + \mb{I}(V(F') \in \B{\Ft{\routes}, \Cpath}),~\forall F' \subseteq F, F' \neq \emptyset \} \nonumber \\
    & = \{F \in \Cpath : 1 \geq 1 + \mb{I}(V(T) \in \B{\Ft{\routes}, \Cpath}),~\forall \text{tree~$T \subseteq F, T \neq \emptyset$} \} \nonumber \\
    & = \{F \in \Cpath : V(T) \notin \B{\Ft{\routes}, \Cpath},~\forall \text{tree~$T \subseteq F, T \neq \emptyset$}\}, \label{eq:routes_equivalence1}
\end{align}

\noindent
where the second equality follows from Claim~\ref{claim:ft_lower_bound}.

Therefore,
\begin{align}
    \routes & = \{F \in \Ft{\routes} : |F| \leq 1\} \nonumber \\
    & = \{\text{path~$P \subseteq G$} : V(P') \notin \B{\Ft{\routes}, \Cpath},~\forall \text{subpath~$P' \subseteq P, P' \neq \emptyset$}\},\label{eq:routes_equivalence2}
\end{align}
and note that~$\routes$ satisfies~\eqref{eq:routes_equivalence2} if and only if~$\Ft{\routes}$ satisfies~\eqref{eq:routes_equivalence1}.

Using essentially the same reasoning as that used to prove Lemma~\ref{lemma:phi_lower_bound},
it follows that~$\Ft{\routes}$ satisfies the above equation if and only if~$\routes$ is downward closed and it satisfies the following variant of the minimal infeasibility property:
\begin{itemize}[(I),leftmargin=*, align=left]
    \item[($\star$)] If~$P$ and~$P'$ are two paths in~$G$ with the same set of vertices, then~$P \in \M{\Ft{\routes}}$ implies~$P' \notin \routes$. \label{item:path_star}
\end{itemize}

\begin{claim}
    \label{claim:minimal_infeasible_path}
    Let~$\routes$ be a family of paths in~$G$ containing all paths with at most one vertex. Then~$\routes$ satisfies~\eqref{eq:routes_equivalence2} if and only if~$\routes$ is downward closed and satisfies property \propstar.
\end{claim}
\begin{proof}
    It is clear that if~\eqref{eq:routes_equivalence2} holds, then~$\routes$ is downward closed. To show that~$\routes$ satisfies \propstar, let~$P$ and~$P'$ be two paths in~$G$ with~$P \in \M{\Ft{\routes}}$ and~$V(P) = V(P')$. Since~$P \in \Cpath$, we know that~$V(P) \in \B{\Ft{\routes}, \Cpath}$, meaning that~$P' \notin \routes$.
    
    To prove the converse, let~$\mc{R}'$ be the set in the RHS of~\eqref{eq:routes_equivalence2}. Suppose that~$P \subseteq G$ is a path that does not belong to~$\routes$ (and thus, to~$\Ft{\routes}$). Since~$\Ft{\routes}$ contains the empty graph and~$\Cpath$ is downward closed, there exists a forest~$P' \subseteq P$ such that~$P' \in \M{\Ft{\routes}} \cap \Cpath$. Moreover, by Claim~\ref{claim:ft_lower_bound},~$P'$ is a path. Hence,~$V(P') \in \B{\Ft{\routes}, \Cpath}$ and~$P$ does not belong to~$\mc{R}'$. 
    
    To show the other side of the inclusion, assume that~$P \in \routes$. By property~\propstar, there exists no path~$P' \in \M{\Ft{\routes}}$ such that~$V(P) = V(P')$. Combining this observation with Claim~\ref{claim:ft_lower_bound} we learn that~$V(P) \notin \B{\Ft{\routes}, \Cpath}$. By downward closedness of~$\Ft{\routes}$, we can repeat the same argument for every subpath~$P''$ of~$P$, proving that~$P \in \mc{R}'$.
\end{proof}

Consequently, we obtain the following result.
\begin{corollary}
\label{corollary:ghosal}
Let~$\routes$ be a family of paths in~$G$ that contains all paths with at most one vertex, is downward closed, and satisfies \propstar. Then there exists a RHS function~$\f$ such that~$\bar{x} \in \R^{E_0}$ is feasible for~$\vrpform{\f}$ if and only if~$\bar{x}$ is the incidence vector of a solution to~$\vrpprob{\routes}$, i.e., there exists a feasible solution~$\mc{S} = \{C_1, \ldots, C_k\}$ for~$\vrpprob{\routes}$ such that, for every~$e \in E_0$,
\begin{equation}
    \label{eq:correspondence_vrp}
    \bar{x}_e = \sum_{i = 1}^k \mb{I}(e \in E(C_i)).
\end{equation}
\end{corollary}
\begin{proof}
    Suppose that~$\routes$ satisfies the conditions in the statement. By Claim~\ref{claim:minimal_infeasible_path}, $\routes$ satisfies~\eqref{eq:routes_equivalence2}. As~\eqref{eq:routes_equivalence2} is equivalent to~\eqref{eq:routes_equivalence1}, this is also equivalent to~$\Ft{\routes} = \Phiset{\lb{\Ft{\routes}, \Cpath}} \cap \Cpath$. Applying Proposition~\ref{prop:characterization3}, we learn that, for any RHS function~$\f$ such that~$\P{\ub{\Ft{\routes}}} \subseteq \P{\f} \subseteq \P{\lb{\Ft{\routes}, \Cpath}}$, the set~$\P{\f} \cap \Qpath$ represents~$\Ft{\routes}$. Therefore,
    \begin{equation*}
    \left\{x|_{E} \in \R^E :~
    \begin{aligned}
    & x(\delta_{G_0}(0)) = 2k, && \\
    & x|_E \in \P{\f} \cap \Qpath, && \\
    & 0 \leq x_e \leq 1 + \mb{I}(0 \in e), && \forall e \in E_0
    \end{aligned}
    \right\}
    \end{equation*}
    represents~$\{F \in \Ft{\routes} : |F| = k\}$, proving the statement.
\end{proof}

Since property~\propstar~is weaker than vertex-consistency (or permutation invariance, see Definition~\ref{def:vertex_consistency}), Corollary~\ref{corollary:ghosal} concretely establishes that, even when specialized for VRPs, Proposition~\ref{prop:characterization3} generalizes the result of Ghosal et al.~\cite{ghosal2024}.
