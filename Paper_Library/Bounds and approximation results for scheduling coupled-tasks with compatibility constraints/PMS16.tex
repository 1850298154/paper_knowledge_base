%\documentclass\{10pt,journal,cspaper,compsoc\}{IEEEtran}
\documentclass[runningheads]{PMS2016}


\usepackage[ansinew]{inputenc}
\usepackage[T1]{fontenc}
\usepackage[top=3cm,bottom=3.5cm,left=4cm,right=3.8cm,dvips]{geometry}
\usepackage{nopageno}

%\usepackage{index}
\usepackage{fancyhdr}

\usepackage{indentfirst}
\usepackage{sectsty}
\allsectionsfont{\usefont{OT1}{phv}{bc}{n}\selectfont \normalsize \bf }
\usepackage{lmodern} % so that fonts are not bitmnaps in the PDF after pdftolatex

%graphics
\usepackage[dvips]{graphicx}
%sub-figures
\usepackage{subfigure}
%graphics directory
%\graphicspath{{figures/}}
%hyperlinks
\usepackage{url}
\usepackage{harvard}
\usepackage{amssymb}
\usepackage{epsfig}

%%%%%%%%%%%
% Packages
%%%%%%%%%%%%
\usepackage{amssymb,amsmath,graphicx}
%\usepackage\{ruled\}{algorithm2e}
\usepackage{tabularx}
\usepackage{psfrag,graphics,caption}
\usepackage{pgf}
\usepackage{epsfig}
\usepackage{color}
\usepackage{wasysym}
\usepackage{tikz}
\usepackage{dot2texi}
\usepackage{multirow}
\usetikzlibrary{decorations,arrows,shapes}





%%%%%%%%%%%
% Commandes
%%%%%%%%%%%
\newtheorem{transformation}{Transformation}
\newenvironment{bulletitemize}{%
\renewcommand{\labelitemi}{$\bullet$}%
\begin{itemize}}{\end{itemize}}
\newcommand{\rod}[1]{\par\textcolor{red}{Rod : \textsc{#1}}\par}%\vspace{0.3cm}}
\newcommand{\nini}[1]{\par\textcolor{blue}{Nini : \textsc{#1}}\par}
% Figures avec psfrag
\def \afig#1#2#3 {\begin{figure}\{htbp\}
\begin{center} 
\mbox{\psfig{file=#1.eps,width=#3}}
\end{center}
\caption{#2}
\label{fig:#1}
\end{figure}}


\def \bfig#1#2 {\begin{figure}\{hhh\}
\begin{center} 
\mbox{\psfig{file=#1.eps}}
\end{center}
\caption{#2}
\label{fig:#1}
\end{figure}}


\hyphenation{appro-xi-ma-ble}

\def\bbbn{\rm I\!N}

\makeindex

\begin{document}

\pagenumbering{empty}
\pagestyle{headings}
\mainmatter

%%%%%%%%%%%%
% En-tete
%%%%%%%%%%%%
%\title{New complexity results}
%\author{\small \textbf{B. Darties ${^{1}}$, G. Simonin ${^{2}}$, R. Giroudeau${^{3}}$ and J.C König${^{3}}$}}

\title{Bounds and approximation results for scheduling coupled-tasks with compatibility constraints}

\author{R. Giroudeau\inst{1},  J.C K\"{o}nig \inst{1}, B. Darties\inst{2} and G. Simonin \inst{3}}

\institute{
LIRMM UMR 5506, 161 rue Ada 34392, Montpellier France\\
\email{$\{$rgirou,konig$\}$@lirmm.fr}
\and
LE2I UMR6306, Univ. Bourgogne Franche-Comt\'e, F-21000 Dijon, France\\
\email{benoit.darties@u-bourgogne.fr}
\and
Insight Centre for Data Analytics, University College Cork, Ireland\\
\email{gilles.simonin@insight-centre.org}
}

\maketitle

\index{B. Darties}\index{R. Giroudeau}\index{J.C K\"{o}nig}\index{G. Simonin}

\noindent

%\tableofcontents

\begin{abstract}
This article is devoted to propose some lower and upper bounds for the coupled-tasks scheduling problem in presence of compatibility constraints  according to classical complexity hypothesis  ($\mathcal{P} \neq \mathcal{NP}$, $\mathcal{ETH}$). Moreover, we develop  an efficient polynomial-time approximation algorithm for the specific case for which the topology describing the compatibility constraints is a quasi split-graph.\newline
\textbf{Keywords:} coupled-task, compatibility graph, complexity, approximation.

 \end{abstract}
 \noindent

 
\section{Introduction, motivations, model}

We consider in this paper the coupled-task scheduling problem subject to compatibility constraints. The motivation of this model is related to data acquisition processes using radar sensors: a sensor emits a radio pulse (first sub-task $a_i$), and listen for an echo reply (second sub-task $b_i$). To make the notation less cluttered, the processing time of a sub-task will be denoted by $a_i$ instead of $p_{a_i}$ used in the theory of scheduling. Between these two instants (emission and reception), clearly there is an idle time $L_i$ due to the propagation, in both sides, of the radio pulse. A coupled-task $(a_i, L_i, b_i)$, introduced by \citeasnoun{Shapiro}, is a natural way to model such data acquisition. This model has been widely studied in several works,\texttt{ i.e.} \citeasnoun {BEKPTW09}. Other works proposed a generalization of this model by including compatibility constraints: scheduling a sub-task  during the idle time of another requires that both tasks are compatible.  The relations of compatibility are modeled by a compatibility graph $G$, linking pair of compatible tasks only. This model is detailed in \citeasnoun{sdgk11journal}. In previous works, we studied the complexity of scheduling coupled-tasks with compatible constraints under several parameters like the size of the sub-tasks or the class  of the compatibility graph~\cite{sgk13}. 

In this work, we propose original complexity and approximation results for the problem of scheduling \textit{stretched} coupled-task with compatibility constraints. A \textit{stretched} coupled-tasks $i$ is a coupled-task having both sub-tasks processing time and idle time equal to a triplet $(\alpha(i),\alpha(i),\alpha(i))$, where  $\alpha(i)$ is the $\textit{stretch factor}$ of the task $i$ - one can apply a stretch factor $\alpha(i)$ to a reference task $(1,1,1)$ to obtain $i$ -. 

The objective is to minimize the makespan $C_{max}$. 
The input of the problem is a collection of coupled-tasks $\mathcal{T}=\{t_1, t_2, \dots t_n$\}, a stretch factor function $\alpha : \mathcal{T}\rightarrow \bbbn$, and a compatibility graph $G_c=(\mathcal{T},E)$ where edge from $E$ link pairs of compatible tasks only.  When dealing with stretched coupled-tasks only,  a edge $\{x,y\} \in E$ exists if $\alpha(x) = \alpha(y)$ (then $x$ and $y$ can be scheduled together without idle time as the idle time of one task is employed to schedule the sub-task of the other, thus we can schedule sequentially $a_x, a_y, b_x, b_y$ - or $a_y, a_x, b_y, b_x$ - in $\frac{4\alpha(x)}{3}$ time units), or if $3\alpha(x) \leq \alpha(y)$ (then $x$ can be entirely executed during the idle time of $y$ \textit{i.e.} $a_y, a_x, b_x, b_y$ and scheduling both tasks requires  $3\alpha(y)$ time units). We note $\#(X)$ the number of different stretch factors in a set of tasks $X$, and we note $d_{G}(X)$ the maximum degree of any vertex $x\in X$ in a graph $G_c$. 

We use the well-known Graham notation \cite{GLLRK79} to define the problems presented in this paper. In this work, we propose new complexity and inapproximability results when the compatibility graph is a restricted  $1-stage~bipartite$ graph $G=(X,Y,E)$, \texttt{i.e.} a bipartite graph where edges are oriented from $X$ to $Y$ only. Then we show the problem is $\mathcal{NP}$-complete on a quasi-split graph $G=(G_X,G_Y,E)$\footnote{A quasi split graph is a connected graph $G=(G_X,G_Y,E)$, with $G_X$ a connected non-oriented graph (not complete)  and $G_Y$ a independent set. The other arcs are oriented from $X$ to $Y$ only.} even if $\#(V(G_X))=1$ and $\#(V(G_Y))=1$, but is $5/4$-approximable. 

\section{Complexity and approximation results}
%\subsection{Non-approximability results}

%This subsection is devoted to a non-approximability result which improved the $\mathcal{NP}$-completeness result given in \cite{dsgk15}.
\begin{theorem}
\label{bipartiinapprox}
  Deciding whether an instance of
   $1|\alpha, G_c =$  $1-stage-bipartite,$ $ \#(X)=2,\#(Y)=1,d_{G_c}(X) \in \{1,2\}, d_{G_c}(Y)\in \{3,4\} | C_{max}$ is a problem hard to approximate within $\frac{21-\rho^{\textsc{Max-3DM-2}}}{20} \leq \rho$, where $\rho^{\textsc{Max-3DM}}$ gives the upper bound for the \textsc{Max-3DM}. Since $\rho^{\textsc{Max-3DM-2}} \leq \frac{140}{141}$, we obtain $1+\frac{1}{2820}$.
   \end{theorem}

\begin{proof}

We prove first that the problem is $\mathcal{NP}$-complete via a polynomial-time reduction. Based on this reduction, we apply the gap-preserving reduction.

The proof is based on a reduction from the maximum \textsc{$3$ Dimensional Matching} (\textsc{Max-3DM}) \cite{np}: let $A$, $B$, and $C$ be three disjoint sets of equal size, with $n=|A|=|B|=|C|$, and a set $T \subseteq A \times B \times C$ of triplet, with $|T|=m$. The aim is to find a matching (set of mutually disjoint triplets) $T^* \subseteq T$ of maximum size. This problem is well known to be $\mathcal{NP}$-complete.
The restricted version of this problem in which each element of $A\cup B \cup C$ appears exactly twice is denoted \textsc{Max-3DM-2} and remains $\mathcal{NP}$-complete \cite{Chlebik}. In this restricted version, we have $m=2n$. %Figure \ref{fig:3DMreduc} presents an instance of \textsc{Max-3DM-2} with three disjoint subsets $A=\{a_1, a_2, a_3\}$, $B=\{b_1, b_2, b_3\}$, $C=\{c_1, c_2, c_3\}$, each with $n=3$ elements, and a set $T=\{t_1, t_2, t_3, t_4, t_5, t_6\}$ of $m=6$ triplets.

%\begin{figure}[!h]
%\begin{center}
%\includegraphics[width=.6\linewidth]{./figures/3DMreduc}
%\caption{An instance of \textsc{Max-3DM-2}}
%\label{fig:3DMreduc}
%\end{center}
%\end{figure}

We transform the instance of \textsc{Max-3DM-2} to an instance of $1|\alpha, G_c =  1-stage~bi$ $ partite,$ $\#(X)=2,\#(Y)=1,d_{G_c}(X) \in \{1,2\}, d_{G_c}(Y)\in \{3,4\} | C_{max}=63n- 3k (1-\epsilon)$ as follows: we define a set of tasks $X\cup Y$ and model the compatibility constraint with a graph $G_c=(X,Y,E)$. For each element $x_i \in A\cup B \cup C$, we add an \textit{item} coupled-task $x_i$ into $X$ with $\alpha(x_i)=1$. For each triplet $t_i \in T$, we add a \textit{box} coupled-task $t_i$ to $Y$ with $\alpha(t_i)=9$, and an \textit{item} coupled-task $t'_i$ with  $\alpha(t'_i)=2+\epsilon$. For each $t_i\in T$ and each $x_i \in t_i$,  we add the compatibility arc $(x_i, t_i)$ to $E$. We also add the  compatibility arc $(t'_i, t_i)$ to $E$. So, the set of $X$-tasks (resp. $Y$-tasks) are constituted by \textit{item} coupled-task $x_i$ and $t'_i$ (resp. \textit{box} coupled-task).


Clearly we have $m$ \textit{box} coupled-tasks (each with an idle time of $9$ units) of degree $4$ in $G_c$, $m$ \textit{item} coupled-tasks with stretch factor $2+\epsilon$ of degree $1$ in $G_c$, and $3n$ \textit{item} coupled-tasks with stretch factor $1$  of degree $2$ in $G_c$. Moreover $G_c$ is a bipartite graph. %Figure \ref{fig:3DMreduc_graph} presents a part of the construction obtained through this reduction, limited to items tasks $a_1$, $a_2$, and $a_3$ and their compatible \textit{box} coupled-tasks. 
The reduction  is constructed in polynomial time. 

%\begin{figure}[!h]
%\begin{center}
%\includegraphics[width=.6\linewidth]{./figures/3DMreduc_graph}
%\caption{Transformation from a $3DM-2$ instance}
%\label{fig:3DMreduc_graph}
%\end{center}
%\end{figure}


 It exists a schedule of length $63n- 3k (1-\epsilon)$ iff it exists a matching of size $k$ for \textsc{Max-3DM-2} instance.

%The problem is clearly in $\mathcal{NP}$.

\end{proof}

%\subsection{Lower bound for subexponential-time algorithms}
\enlargethispage{0.5cm}
Hereafter, we propose some negative results  concerning the existence of subexponential-time algorithms under the following complexity-theoretic hypothesis that is known as the  Exponential-Time Hypothesis (see \cite{Woeginger01} for a survey on exact algorithms for $\mathcal{NP}$-hard problems) for stretched coupled-tasks, and other ones previously studied.
 
 Recall first the {\sc Exponential-Time Hypothesis} (\cite{ImpagliazzoP01}, and \cite{ImpagliazzoPZ01}): there exists a constant $c >1$ such that there exists no algorithm for $3-$Satisfiability  that uses only $O(c^l)$ time where $l$  denotes the number of variables.


\begin{corollary}\label{ETH}
Assuming the Exponential-Time Hypothesis, there exists no algorithm with a worst-case running time that is subexponential in $n$ (the number of vertices),  \textit{i.e.}:
  
  \begin{enumerate}
  \item For the $1|a_i=b_i=p, L_i=2p,G_c | C_{max}$ problem   in $O(2^{o(n)})$ time
  \item For $1|a_i=a, b_i=b, L_i=a+b,G_c | C_{max}$ in $O(2^{o(n)})$ time
 % \item $1|\alpha(a_i),G_c=star | C_{max}$ (subset sum)
\item  $1|\alpha,G_c=1-bipartite| C_{max}$ in $O(2^{O(n)})$-time algorithm.
\end{enumerate}
\end{corollary}

\begin{proof}

\begin{enumerate}
  \item  For $1|a_i=b_i=p, L_i=2p,G_c| C_{max}$:  In  \cite{RooijNB13}, the authors proved that for {\sc Partition into triangles} on graphs of maximum degree four, there is no algorithm with a worst-case running  time  $O(2^{o(n)})$ that is subexponential in $n$.
 
  Therefore, we transform a {\sc Partition into triangles} instance with $n$ vertices and $m$ edges into an equivalent instance $G_c$  for bounded degree at most four. Since the transformation is linear (see \cite{sdgk11journal}) the result holds.
  
  \item For the problem $1|a_i=a, b_i=b, L_i=a+b,G_c | C_{max}$: In \cite{LokshtanovMS11} the authors proved that for {\sc Hamiltonian path} there is no   $O(2^{o(n)})$-time algorithm. As the same way as previously the transformation is linear (see \cite{sdgk11journal}).
  %\item $1|\alpha(a_i),G_c=star | C_{max}$ (subset sum)
\item  $1|\alpha,G_c=1-bipartite | C_{max}$: In \cite{ChenJZ14}, the authors proved that for {\sc Max 3DM}, there is no $O(2^{O(n)})$-time algorithm, therefore this result is transposed to the scheduling problem using the first part of the proof of Theorem \ref{bipartiinapprox}.
\end{enumerate}
\end{proof}


\begin{theorem}
Scheduling stretched coupled task in presence of a quasi split graph is a $\mathcal{NP}-$complete problem even if $\#(V(G_X)) = 1$ and $\#(V(G_Y)) = 1$
\end{theorem}

\begin{proof}
The proof is based on a reduction from a variant of the well-know $\mathcal{NP}$-complete {\sc Partition into triangles}. This problem consists to ask if  the vertices of a graph $G=(V,E)$, with $|V|=3q, q \in \bbbn$, can be partitioned into $q$ disjoints sets $T_1, T_2, \ldots, T_q$, each containing exactly three vertices, such that for 
each $T_i=\{u_i,v_i,w_i\}, 1 \leq i \leq q$, all three of the edges $\{u_i,v_i\}, \{u_i,w_i\}, \{w_i,v_i\}$ belong to $E$.

The problem {\sc Partition into triangles} remains $\mathcal{NP}$-complete even if the graph $G$ can be partitioned into three sets with the same size, $A,~B$ et $C$ such that each set is an independent  set \cite{morandini}. The polynomial-time transformation is based on this variant.
Let $G = (A \cup B \cup C,E)$ be an instance of the variant of {\sc Partition into Triangles}. We consider the  split-graph $G'=(A \cup B,C,E')$ obtained as follows:

%\begin{itemize}

%\item 
$\forall v \in A$ (resp. $B$), we create a vertex $A_v$ (resp. $B_v$) with processing time $(1,1,1)$. Moreover, $\forall v \in C$ we create a task $C_v$ with  processing time $(4,4,4)$.
%\item 
The edges between $A$ and $B$ remained the same as the $G'$ whereas the edge between $A \cup B$ and $C$ are oriented.
%\item 
Finally in order to have a connected graph, we add two news vertices (resp. one) $z_0$ and  $z_1$ (resp. $z_2$ with processing time equal to $(1,1,1)$ (resp. $(4,4,4)$). We add  edges between $z_0$ to $A_v$ (resp. $z_1$ to $B_v$). Lastly, we add the three edges $(z_0,z_2)$, $(z_1, z_2)$ and $(z_0, z_1)$.
%\end{itemize}

Notice that the graph $A_v \cup B_v$ form a bipartite graph. The problem is clearly in $\mathcal{NP}$. It exists a positive solution for the variant of  {\sc Partition into triangles} iff a valid schedule of length $12 \times (|C|+1)$  exists. It is sufficient to execute the two tasks $A_v$ and $B_{v'}$ in four units of time into a task $C_u$.





%The following algorithm seems to be optimal: the sequantial-time of the algorithm is equal to the sum of the $Y$-tasks plus the largest processing-time for a $X$-task executed outside of the $Y$-tasks. It is sufficient to find the $X$-task with the largest processing time non covered by one of $Y$-task.

\end{proof}


%\subsubsection{Approximation results on some split-graph}


\begin{theorem}
\label{approxsplit}
The problem is  $5/4$-approximable on quasi split-graph where $\#(V(G_Y)) = 1$.
\end{theorem}
\enlargethispage{0.5cm}
\begin{proof}
W.l.o.g., we suppose that the processing time of $X$-tasks (resp. $Y$-tasks) is $(1,1,1)$ (resp. $\alpha(y_i)$).  Indeed, if $\alpha(x) >1$, we put $\alpha(y_i)=\lfloor \frac{\alpha(y_i)}{\alpha(x)} \rfloor$ and $\alpha(x)=1$ .

\textbf{Algorithm:} we  transform the problem into an oriented maximum flow-problem between $G_X$ and $G_Y$ with two sources $s$ and $t$, with $\omega(s,x)=\omega(x,y)=1$ and $\omega(y,t)=\lfloor \frac{\alpha(y_i)}{3\alpha(x)}\rfloor, \forall y_i \in Y, \forall x \in XG_Y$ where $\omega(i,j)$ is the capacity of an arc $(i,j)$ . After the computation of a maximum flow $F$ of value $f$,  for the uncovered remaining $X$-tasks a maximum $M$-matching ($|M|=m$) is applied. The schedule consists in processing first,  the $Y$-tasks with $X$-tasks inside. The $M$-tasks are executed after. Lastly, we schedule $s$ isolated-tasks. The length of schedule given by the algorithm is $C_{max} \leq  \sum_{y_i \in Y} 3\alpha(y_i)+4m+3s$ with $2m+s+f=n=|X|$ and $ \sum_{y_i \in Y} 3\alpha(y_i) \geq 9f$. In similar way, the optimal length is $C^*_{max} \geq \sum_{y_i \in Y} \alpha(y_i)+4m^*+3s^*$. We suppose that in $Y$-tasks where are $p^*$-edges processed and $r^*$ isolated-tasks, then we obtain $2(p^*+m^*)+r^*+s^*=n$, $p^*+r^* \leq f$, and $\sum_{y_i \in Y} \alpha(y_i) \geq 12p^*+9r^*$. In the worst-case, the $p^*$-edges are split  into two tasks (so $p^*$ news tasks are added to $s^*$), and also the matched-edges  are split (for each $m^*$ edges one task is executed into the $Y$-task, instead of  one of $r^*$-tasks). Therefore, $2m^*$ tasks are added to the $s$-value.
In the worst case, we have $m^*=r^*$, $s=s^*+p^*+2r^*$ and $m=0$. In such case,  $C_{max} \leq 12p^*+9r^*+3s^*+3p^*+6r^*$ and $C^*_{max}=12p^*+9r^*+4r^*+3s^*$. Thus $\rho \leq \frac{15p^*+15r^*+3s^*}{12p^*+13r^*+3s^*} \leq \max(5/4,15/13,1)=5/4$.

%Clearly, the optimal length is at least $C^*_{max} \geq \sum_{y_i \in Y} \alpha(y_i)+req$ with $req$ the length of the tasks from $G_x$ which are not executed on $G_Y$, whereas the algorithm lead a length at most $C_{max} \leq  \sum_{y_i \in Y} \alpha(y_i)+req' +3p$ where $p$ is the number  of $Y$-tasks executed into a optimal solution ($\sum_{y_i \in Y} \alpha(y_i)\geq 12p$.) 




% Therefore, if $p$ is the edge-number executed into the $Y$tasks. In this algorithm, there are at most $p$ isolated tasks  which are processed. Thus the gap between $T$ and $T^*$ is at most $3p$. Since $T^*$ is greater than $12p$ (a edge is processed into four units of time and there are two units of time for the considered task). So,  $\rho=15p/12p$  \textit{i.e.} $5/4$.

\textbf{Tightness:} it exists an example for the $C^*_{max}=36$, and for the heuristic $C_{max}=45$. Consider the graph: three triangles $(x_1,x_2,y_1)$, $(x_3,x_4,y_2)$, and  $(x_5,x_6,y_1)$. We add the edges $(x_2,y_3$, $(x_3,y_1)$ and $(x_5,y_2)$. %The processing time of the $X$-tasks (resp. $Y$-tasks) is $(1,1,1)$ (resp. $(4,4,4)$). 
The optimal solution consists in executing the $X$-tasks into the $Y$-tasks; whereas the heuristic leads the solution in which three $X$-tasks are processed after the $Y$-tasks.
\end{proof}

%\section{Conclusion}
%In this article, we propose some lower and upper bounds for coupled-tasks scheduling problem  in presence of compatibility constraint given by a specified graph (bipartite, quasi split graph). 

\bibliographystyle{agsm}
\begin{thebibliography}{xx}

\harvarditem[Bla\.{z}ewicz et~al.]{Bla\.{z}ewicz et~al.}{2009}{BEKPTW09}
Bla\.{z}ewicz, J., Ecker, K., Kis, T., Potts, C., Tanas, M. \harvardand\
  Whitehead, J.  \harvardyearleft 2009\harvardyearright , Scheduling of
  coupled tasks with unit processing times, {\em Technical report, Poznan
  University of Technology}.
 
  
  \harvarditem{Chen, Jansen \harvardand\ Zhang}{2014}{ChenJZ14}
 Chen L.,  Jansen K. \harvardand\ Zhang G. \harvardyearleft
  2014\harvardyearright , {On the optimality of approximation schemes for the classical scheduling problem}, {\em {Proceedings of the Twenty-Fifth Annual {ACM-SIAM} Symposium on Discrete Algorithms, {SODA} 2014, Portland, Oregon, USA, January 5-7, 2014}}

\harvarditem{Chlebik}{2003}{Chlebik}
Chleb{\'i}k M. \harvardand\ Chleb{\'i}kov{\'a} J.  \harvardyearleft 1979\harvardyearright , {Inapproximability results for bounded variants of optimization problems}, {\em Electronic Colloquium on Computational Complexity} {\bf 10} (26),
  pp.~1--26.

\harvarditem{Garey \harvardand\ Johnson}{1979}{np}
Garey, M.~R. \harvardand\ Johnson, D.~S.  \harvardyearleft
  1979\harvardyearright , {\em {Computers and Intractability: A Guide to the
  Theory of NP-Completeness}}, W. H. Freeman \& Co., New York, NY, USA.

 \harvarditem{Graham et~al.}{1979}{GLLRK79}
Graham, R. L., Lawler, E. L., Lenstra, J. K. \harvardand\ Kan, A. H. G. Rinnooy \harvardyearleft 1979\harvardyearright , {Optimization and Approximation in Deterministic Sequencing and Scheduling: a Survey}, {\em Annals of Discrete Mathematics} {\bf 5},
  pp.~287--326.
  
\harvarditem{Impagliazzo \harvardand\ Paturi}{2001}{ImpagliazzoP01}
Impagliazzo, R. \harvardand\ Paturi, R.  \harvardyearleft 2001\harvardyearright
  , {On the Complexity of $k$-SAT}, {\em Journal of Computer and System
  Sciences} {\bf 62}(2),~367--375.

\harvarditem[Impagliazzo et~al.]{Impagliazzo et~al.}{2001}{ImpagliazzoPZ01}
Impagliazzo, R., Paturi, R. \harvardand\ Zane, F.  \harvardyearleft
  2001\harvardyearright , {Which Problems Have Strongly Exponential
  Complexity?}, {\em Journal of Computer and System Sciences} {\bf
  63}(4),~512--530.

\harvarditem[Lokshtanov et~al.]{Lokshtanov, Marx \harvardand\
  Saurabh}{2011}{LokshtanovMS11}
Lokshtanov, D., Marx, D. \harvardand\ Saurabh, S.  \harvardyearleft
  2011\harvardyearright , {Lower bounds based on the Exponential Time
  Hypothesis}, {\em Bulletin of the EATCS} {\bf 105},~41--72.


	\harvarditem[Morandini]{Morandini, M.}{2004}{morandini}
Morandini, M. \harvardyearleft 2004\harvardyearright , NP-complete problem: partition into triangles, {\em Technical report, Universita` di Udini}.
  
  
\harvarditem{Shapiro}{1980}{Shapiro}
Shapiro, R.~D.  \harvardyearleft 1980\harvardyearright , {Scheduling coupled
  tasks}, {\em Naval Research Logistics Quarterly} {\bf 27},~477--481.

\harvarditem[Simonin et~al.]{Simonin, Darties, Giroudeau \harvardand\
  K\"{o}nig}{2011}{sdgk10journal}
Simonin, G., Darties, B., Giroudeau, R. \harvardand\ K\"{o}nig, J.-C.
  \harvardyearleft 2011\harvardyearright , Isomorphic coupled-task scheduling
  problem with compatibility constraints on a single processor, {\em J.
  of Scheduling} {\bf 14}(5),~501--509.

\harvarditem[Simonin et~al.]{Simonin et~al.}{2013}{sgk13}
Simonin, G., Giroudeau, R. \harvardand\ K{\"o}nig, J.-C.  \harvardyearleft
  2013\harvardyearright , Approximating a coupled-task scheduling problem in
  the presence of compatibility graph and additional tasks', {\em International
  Journal of Planning and Scheduling} {\bf 1}(4),pp.~285--300.

\harvarditem[Simonin et~al.]{Simonin et~al.}{2012}{sdgk11journal}
Simonin, G., Giroudeau, R., K\"{o}nig, J.-C. \harvardand\ {B. Darties}
  \harvardyearleft 2012\harvardyearright , {Theoretical Aspects of Scheduling
  Coupled-Tasks in the Presence of Compatibility Graph}', {\em Algorithmic in
  Operations Research} {\bf 7}(1),~1----12.

  \harvarditem{van Rooij et~al.}{2013}{RooijNB13}
van Rooij J. M. M.,  van Kooten Niekerk, M. E. \harvardand\ Bodlaender  H.   L.  \harvardyearleft 2013\harvardyearright , {Partition Into Triangles on Bounded Degree Graphs}, {\em Theory Computing Systems} {\bf 52}(4),
  pp.~687-718.
  
\harvarditem{Woeginger}{2001}{Woeginger01}
Woeginger, G.~J.  \harvardyearleft 2001\harvardyearright , {Exact Algorithms
  for {NP}-Hard Problems: A Survey}, {\em in} Combinatorial Optimization,
  pp.~185--208.


\end{thebibliography}


\end{document}

%%%%%%%%%%%%%
%%%%%%%%%%%%%
%%%%%%%%%%%%%
%%%%%%%%%%%%%





