%%%%%%%%%%%%%%%%%%%%%%%%%%%%%%%%%%%%%%%%%%%%%%%%%%%%%%%%%%%%%%%%%%%%%%%%%%%%%%%%
%2345678901234567890123456789012345678901234567890123456789012345678901234567890
%        1         2         3         4         5         6         7         8

\documentclass[letterpaper, 10 pt, conference]{ieeeconf}  % Comment this line out if you need a4paper
%\documentclass[a4paper, 10pt, conference]{ieeeconf}      % Use this line for a4 paper
\IEEEoverridecommandlockouts                              % This command is only needed if 
                                                          % you want to use the \thanks command
\overrideIEEEmargins                                      % Needed to meet printer requirements.

%In case you encounter the following error:
%Error 1010 The PDF file may be corrupt (unable to open PDF file) OR
%Error 1000 An error occurred while parsing a contents stream. Unable to analyze the PDF file.
%This is a known problem with pdfLaTeX conversion filter. The file cannot be opened with acrobat reader
%Please use one of the alternatives below to circumvent this error by uncommenting one or the other
%\pdfobjcompresslevel=0
%\pdfminorversion=4

% See the \addtolength command later in the file to balance the column lengths
% on the last page of the document

% The following packages can be found on http:\\www.ctan.org
%\usepackage{graphics} % for pdf, bitmapped graphics files
%\usepackage{epsfig} % for postscript graphics files
%\usepackage{mathptmx} % assumes new font selection scheme installed
%\usepackage{times} % assumes new font selection scheme installed
%\usepackage{amsmath} % assumes amsmath package installed
%\usepackage{amssymb}  % assumes amsmath package installed

\title{\LARGE \bf
Safe and Efficient Model Predictive Control Using Neural Networks: An Interior Point Approach
}


\author{Daniel Tabas and Baosen Zhang% <-this % stops a space
\thanks{This work is partially supported by the National Science Foundation Graduate Research Fellowship Program under Grant No. DGE-1762114, NSF grants ECCS-1930605 and ECCS-2023531. Any opinions, findings, conclusions, or recommendations expressed in this material are those of the authors and do not necessarily reflect the views of the National Science Foundation.}% <-this % stops a space
\thanks{Authors are with the department of Electrical and Computer Engineering, University of Washington, Seattle, WA, United States. $\{$dtabas, zhangbao$\}$@uw.edu.
}}

\usepackage{amsmath,amssymb,amsfonts,graphicx,comment}
\usepackage{xcolor}
\usepackage{cite}


\begin{document}

\newcommand{\N}{\mathbb{N}}
\newcommand{\Z}{\mathbb{Z}}
\newcommand{\R}{\mathbb{R}}
\newcommand{\any}{\text{ $\forall$ }}
\newcommand{\e}{\text{e}}
\newcommand{\E}{\mathcal{E}}
\newcommand\m[1]{\begin{bmatrix}#1\end{bmatrix}}

\newcommand{\X}{\mathcal{X}}
\newcommand{\U}{\mathcal{U}}
\newcommand{\D}{\mathcal{D}}
\renewcommand{\int}{\textbf{int }}
\newcommand{\B}{\mathbb{B}}
\newcommand{\F}{\mathcal{F}}
\newcommand{\T}{\mathcal{T}}
\newcommand{\V}{\mathcal{V}}
\renewcommand{\P}{\mathcal{P}}
\newcommand{\Q}{\mathcal{Q}}
\newcommand{\Y}{\mathcal{Y}}
\renewcommand{\S}{\mathcal{S}}
\newcommand{\M}{\mathcal{M}}

\renewcommand{\u}{\textbf{u}}
\newcommand{\x}{\textbf{x}}

\newcommand{\todo}[1]{\textcolor{red}{#1}}

\newtheorem{proposition}{Proposition}
\newtheorem{definition}{Definition}

\newcommand{\revision}[1]{\textcolor{black}{#1}}

\newcommand{\newrevision}[1]{\textcolor{black}{#1}}

\maketitle
\thispagestyle{empty}
\pagestyle{empty}


%%%%%%%%%%%%%%%%%%%%%%%%%%%%%%%%%%%%%%%%%%%%%%%%%%%%%%%%%%%%%%%%%%%%%%%%%%%%%%%%
\begin{abstract}

Model predictive control (MPC) provides a useful means for controlling systems with constraints, but suffers from the computational burden of repeatedly solving an optimization problem in real time. Offline (explicit) solutions for MPC attempt to alleviate real time computational challenges using either multiparametric programming or machine learning. The multiparametric approaches are typically applied to linear or quadratic MPC problems, while learning-based approaches can be more flexible and are less memory-intensive. Existing learning-based approaches offer significant speedups, but the challenge becomes ensuring constraint satisfaction while maintaining good performance. In this paper, we provide a neural network parameterization of MPC policies that explicitly encodes the constraints of the problem. By exploring the interior of the MPC feasible set in an unsupervised learning paradigm, the neural network finds better policies faster than projection-based methods and exhibits substantially shorter solve times. We use the proposed policy to solve a robust MPC problem, and demonstrate the performance and computational gains on a standard test system. 

\end{abstract}


%%%%%%%%%%%%%%%%%%%%%%%%%%%%%%%%%%%%%%%%%%%%%%%%%%%%%%%%%%%%%%%%%%%%%%%%%%%%%%%%


%{\bf \large{Some references}}\\
%\begin{itemize}
%    \item\cite{Wang2015} Optimal control for epidemic routing of two files with different priorities in Delay Tolerant Networks
%  A Multi-Layer Swarm Control Model for Information Propagation and Multi-Tasking 2019 ACC Fuming Zhang
%\end{itemize}
%Outline - 
%The problem is defined as controlling the performance of a network composed of mobile robots by controlling the the individual performance (define performance) of each of the node.

%The multi-robot system has an architecture that can be described by a graph determined by their positions and interactions. The topology and the features (define features) of the interactions together delivers the performance (of information propagation, routing, etc.) of a network.

%Uncertainty brings in randomness. Exsting work only considers certain stochasticity - whether a link is there or not. There are other issues to be consider. e.g. - The distances between different pairs of robots only only defined the topology, which specifies who can talk with who, but also how good they are communicating. The transmission quality can be described by a transmission rate, which converged the completion of transmission on one link into a Poisson process (that the completion is seen as an event).

%Existing graph theory tools can handle either a static graph or a time-varying graph with known dynamics (or random graphs). When the uncertainty disrupts the dynamics, existing tools all fail. We need new tools analyzing graphs that :1) with stochasticity in diffrent ~ , 2) with structues / patterns that we can leverage.

%So that we can deliver control schemes that can guarantee/optimize the performance of the network in a decentralized fashion.


%\subsection{Existing results}
%Li found a way of determining the most important links considering the chronological order.

%\subsection{Challenge}
%(maybe future work - If we want to improve the transmission quality, we essentially want to move that pair of robots close to each other.

%However, moving robots will affect the relative position to other robots. 

%Improving transmission rate all around may cause conglomeration.

%We may want to find a way of moving robots to improve the topology / transmission but not to cause conglomeration.)

%\subsection{Approach}

%\paragraph{Stage I:}
%The topology of the communication network is fixed. Any robot improving its communication quality with one of its neighbours will suffer from a decreased communication quality with other neighbors. 

%Model - Fixed topology, robot $i$ has $d_1$ neighbors, the delivery of a piece of message along a communication link is modeled as a Poisson process, each communication channel is associated with a parameter $\lambda_{ij}$ determining the Poisson process. We have $\sum_{\forall j} \lambda_{ij} = 1$.

%Aim - control each node's distribution of $\lambda$ to optimize the performance of the system.

%To be added - the model of information resources?

%\paragraph{Stage II:}
%The robots are moving on a 1-D or 2-D space that moving closer towards one neighbor with risk distancing itself with other neighbors.

%The 'perimeter' nodes are fixed.

%Communication qualify is related to the distance between the pair of robots. 

We are interested in leveraging spatio-temporal sensing capabilities of robotic teams to explore or monitor large-scale complex environments
(\textit{e.g.,} forests, oceans, and underground caves and tunnels \cite{breitenmoser2010voronoi,cassandras2016smart,gusrialdi2008voronoi,wei2018,yu2019synchronous}).
%\xyucomment{I mean 'deliver the tasks'. I edit the sentence accordingly. Thanks!}
While individual robots can each monitor or explore limited regions, a team of them can efficiently cover larger areas.  Naturally, individual robots in swarms are expected
to exchange information and make certain decisions based on the information gathered by itself and received from other robots \cite{hsieh2008decentralized,moarref2020automated}.
It is, therefore, essential for any robot to be able to
transmit its gathered data to others in the team. 

As the demand of information transmission can emerge between any pair of nodes, we use a propagation model to study the efficiency of the connectivity of the whole team. 
In such models, robots are represented as nodes of a graph,
and interactions 
%(\textit{i.e.} communication)
between nodes are represented by edges.
%The nodes and the edges together form a network.
At the beginning of a propagation, the nodes 
%(\textit{i.e.} robots) 
are considered as `non-informed', and will be `informed’ by the information propagated through the graph. A faster convergence of all nodes into the status of `informed' indicates a more \textit{efficiently connected} network. 
Existing works abstract the propagation of information from any individual agent to the rest of the team as compartmental models \cite{huang2006information,khelil2002epidemic,zanette2002dynamics}, which are also broadly used in modeling the spread of an infectious diseases \cite{brauer2008compartmental}.
%In such models, robots are represented as nodes of a graph,
%and interactions 
%(\textit{i.e.} communication)
%between nodes are represented by edges.
%The nodes and the edges together form a network.
% The nodes 
% %(\textit{i.e.} robots) 
% are labeled according to their status of whether the robots are `informed’ by the information propagated through the swarm.
The flow of the nodes from one compartment to another
%(\textit{i.e.} uninformed robots get `informed’ with the information.)
occurs when active links exist between robots that are
`informed’ and `non-informed’ %the information 
(\textit{i.e.,} 
only `informed' robots %carrying the information 
can transmit the information to `non-informed' ones).%robots that are not yet informed). 


%\thales{However, epidemic models often operate under the assumption that the networks are static and deterministic, so that the flow between compartments can be analyzed with ordinary differential equations \cite{}. In reality, neither human contacts nor robots working in communication challenging environments can guarantee such consistent connectivity in large-scale networks. The classic compartmental-models-based control strategies for such networks also focus on the statistical behavior of the nodes, ignoring the topology of the networks. The analysis and solutions are, therefore, generally valid with giant node numbers reaching thermodynamic limits \cite{}. } 


However, compartmental models often focus on the statistical behavior of the nodes, overlooking the capabilities of the \textit{individual-level} decision making that may impact the network's performance.
%\xyucomment{I commented out this paragraph and edit it as a couple of bridge sentences. }
Recent works in robotics have noticed and embraced the potential of controlling individual robots so that the performance of the whole network (\textit{i.e.,} the transmission rates, if a propagation model is considered,) can be impacted or improved. %swarm can jointly achieve the desired status. 
The topology of networks was analyzed in \cite{Preciado2014} and control strategies were proposed to identify 
and isolate the nodes that have a higher impact on
the propagation process. The design of {\it time-varying} networks where robots leverage their mobility to move into each other's communication ranges to form and maintain temporal communication links has been addressed in  \cite{hollinger2012multirobot,khodayi2019distributed,yu2020synthesis}. Information can be propagated in such networks by relying on time-respect routing paths formed by a sequence chronologically ordered temporal links \cite{yu2020synthesis}. 
Nonetheless, such approaches still require thorough and perfect knowledge of the network's topology and the robots' ability to maintain active links and requires robots to execute their motion plans precisely both in time and space. As such, the resulting networks lack robustness to uncertainties in link formation timings that may result from errors in motion plan execution due to localization errors and/or actuation uncertainties.

%\thales{Stochastic models, such as networks based on percolation theory concepts \cite{}, have also been introduced in works in epidemiology in recent years to carry out a more realistic analysis. Percolation theory assumes a growing collection of edges between nodes. When the addition of edges fits a stochastic process, the growing graphs yielded are modeled as Erd\"os–R\'enyi networks, and have been well studied in the field of random graphs \cite{}. Noticed that such stochastic models focus on a random \textit{existence} of edges on a graph, echoing the assumption of Canadian Traveler Problems, in which the graphs are partially visible, and the edges are randomly blocked \cite{}. }

%Real-world communication networks synthesized in robot 
%swarms face uncertainties causing stochastic performance
%of the networks.
%that can not always be directly addressed by existing graph theory tools. 
Existing stochastic graph models focus on random changes in a network's topology centered around random creation and removal of edges within a graph. For example, the Canadian Traveller Problem assumes partially visible graphs where edges randomly appear and disappear \cite{bar1991canadian,nikolova2008route}.
For time-varying networks,
\cite{knizhnik2022flow,shen2022topology} assumed that any temporal link
may suffer from deviations from
its scheduled timing due to the uncertain arrival time of one robot into another robot’s communication range.
%\xyucomment{I commented out the original paragraph and edit the following graph as shown in blue. }
%Consider robot swarms moving in challenging environments
%with network topology changing over time.
%Any temporal link that occurs and disappears in such a network may suffer from deviations in its scheduled timing \cite{ACC2022} due to the uncertain arrival time of one robot into another robot’s communication range \cite{ICRA 2022}.
Information routing or propagation planned for such networks may experience severe delays if subsequent links along a routing path appear out of chronological order in the presence of link formation uncertainties resulting from uncertainties in robot motions \cite{yu2020synthesis}.  These challenges are addressed in \cite{shen2022topology} for networks with periodically time-varying interconnection topologies.  Control strategies to `fix’ nodes with 
higher impact on the whole network’s performance were also proposed in \cite{shen2022topology} similar in spirit to those presented in \cite{Preciado2014}.

In the Canadian Traveler Problems and \cite{shen2022topology}, messages are assumed to be transmitted instantaneously between nodes whenever a link appears. As such, the resulting strategies are solely based on the \textit{existence} of an available link between a certain pair of robots at a certain time point. When the time to transmit a message is non-trivial, the question goes beyond whether the information can be transmitted or not and must consider the quality of the transmission (\textit{e.g.,} safe, fast, confident). Consider a pair of robots flying through a Poisson forest maintaining a communication link that requires line-of-sight (LOS). The randomly distributed obstacles may intermittently disrupt the LOS between the robots causing randomized delay in the completion of the information transmission task. In these situations, it becomes difficult to determine whether a robot \textit{is} informed or not at a given time point which then impacts the robots ability to plan subsequent actions.

In this paper, we aim to develop receding horizon control schemes that allows individual robots to re-direct their transmission resources 
(\textit{e.g.,} time, power) to different neighboring robots on the communication network with stochastic
links to guarantee an exponentially fast
convergence status in the information propagation. We model the completion of the message transmission across a link as a Poisson Point Process. The density parameter of this process is determined by the transmission resources invested by the nodes.  Each node carries limited transmission resources that is distributed among all the links connecting to it.
The completion of the transition of a 
message from one node to another is then modeled as a Markov process. 
All robots would then be able to update the distribution of their own transmission resources according to the neighboring current 
states and follow the control 
directions. Therefore, the control
strategy takes into account the 
transmission resource capabilities of the
robots and acts on them to fulfill
performance requirements.
Such a set of resources changes 
according to the application, for
example, in the Poisson forest we
might require that the robots stay close
together at the cost of decreasing the
covered area; in a situation in which
the robots need to move back and forth
to carry information ({\it e.g.,} 
surveillance
around buildings or in tunnels), we
might require the robots to increase their
frequency of visits among each other.

The paper is organized as follows. Sec.~\ref{sec:graph_theory} and Sec.~\ref{sec:SI_model} provides theoretical backgrounds of the graph structure and the propagation model. The problem is formally stated at the end of Sec.~\ref{ProbForm}. Sec.~\ref{section3} introduces our approach of developing the receding horizon control scheme for the stochastic network. Sec.~\ref{simulation} validates our proposed control schemes with numerical examples. Sec.~\ref{conclusion} concludes the paper and proposes future directions.
\section{Problem Formulation} \label{sec:pf}

% Describe the system:

In this paper, we consider the problem of regulating discrete-time dynamical systems of the form \begin{align}
    x_{t+1} = Ax_t + Bu_t + d_t \label{eqn:2-26-5}
\end{align} where $x_t \in \R^n$ is the system state at time $t$, $u_t \in \R^m$ is the control input, and $d_t \in \R^n$ is an uncertain input that captures exogenous disturbances and/or linearization error (if the true system dynamics are nonlinear) \cite{Boyd1994}. We assume the pair $(A,B)$ is stabilizable.
%Let $\X \subset \R^n$ and $\U \subset \R^m$ be sets representing the constraints of the system. The set $\X$ represents the region of safe operation within the state space, defined by engineering constraints. The set $\U$ represents actuation limits. We assume $\X$ and $\U$ are polytopes given by $\X = \{x \in \R^n \mid F_x x \leq g_x\}$ and $\U = \{u \in \R^m \mid F_u u \leq g_u\}.$
\revision{The input constraints (actuation limits) are $\U = \{u \in \R^m \mid F_u u \leq g_u\}$ while the state constraints arising from safety-critical engineering considerations are $\X = \{x \in \R^n \mid F_x x \leq g_x\}$.}

% State the MPC problem:

We consider the problem of operating the system \eqref{eqn:2-26-5} using finite-horizon model predictive control. The goal is to choose, given initial condition $x_0 \in \X$, a sequence of inputs $\textbf{u}$ of length $\tau$ that minimizes the cost of operating the system while respecting the operational constraints.

% Once $\hat{\u}^*_t$ is chosen, the first action is implemented, i.e. we take $u_t = \hat{u}_t^*$. After a state evolution step, a new state is observed, and $\hat{\u}^*_{t+1}$ is computed given the observation of $x_{t+1}$.

However, since the disturbances $d_t$ are unknown ahead of time, the designer must carefully consider how to achieve both optimality and constraint satisfaction. 
% we cannot compute the cost \eqref{eqn:2-26-1} or guarantee satisfaction of the state constraints \eqref{eqn:2-26-3}. However,
Robust MPC literature contains many ways to handle the presence of disturbances in both the cost and constraints \cite{Bemporad1999}. For example, the \textit{certainty-equivalent}  approach \cite{Alessio2009} considers only the nominal system trajectory, while the \textit{min-max} approach \cite{Grancharova2009} considers the worst-case disturbance. Interpolating between these two extremes, the \textit{tube-based} approach \cite{Langson2004} considers the cost of a nominal trajectory while guaranteeing that the true trajectory satisfies constraints. A \textit{stochastic} point of view in \cite{Farina2016} considers the disturbance as a random variable and minimizes the expected cost while providing probabilistic guarantees for constraint satisfaction. 

\revision{In most robust MPC formulations, the set of possible disturbances is modeled as either a finite set, a bounded set, or a probability distribution \cite{Saltk2018}.} In this paper, we assume the disturbances lie in a closed and bounded set $\D := \{d \in \R^n \mid F_d d \leq g_d\}$. \revision{In order to ensure constraint satisfaction, we operate the system within a \textit{robust control invariant set} (RCI) $\S \subseteq \X$, defined as a set of initial conditions for which there exists a feedback policy in $\U$ keeping all system trajectories in $\S$, under any disturbance sequence in $\D$ \cite{Blanchini2015}. In our simulations, we used approximately-maximal RCIs computed with the semidefinite program from \cite{Liu2015}.} 

\revision{With $\S := \{x \in \R^n \mid F_s x \leq g_s\}$, we define the \textit{target set} $\T$ as $\{x \in \R^n \mid x + d \in \S, \ \forall\ d \in \D\} = \{x \in \R^n \mid F_s x \leq \tilde{g}_s\}$ where for each row $i$, $\tilde{g}_s^{(i)} = g_s^{(i)} - \max_{d \in \D} F_s^{(i)T}d$ \cite{Blanchini2015}}. \newrevision{Any policy that maps $\S$ to $\T$ under the nominal dynamics will map $\S$ to itself under the true dynamics, rendering $\S$ robustly invariant.} \revision{By constraining the nominal state to the target set, robust constraint satisfaction is guaranteed for the first time step. Since $\S$ is RCI, this is sufficient for keeping closed-loop trajectories inside $\S$. Under this formulation, the MPC problem is posed as follows, given initial state $x_0$:
\begin{subequations} \label{eqn:2-26-7} \begin{gather}
    \min_{\u} \sum_{k=0}^{\tau-1} l(x_k,u_k) + l_F(x_{\tau}) \label{eqn:2-26-8}\\
    \text{subject to $\forall\ k$: } x_{k+1} = Ax_k + Bu_k \label{eqn:2-26-9}\\
    x_{k+1} \in \T \label{eqn:2-26-10}\\
    u_k \in \U \label{eqn:2-27-9}
\end{gather} \end{subequations} where $l$ and $l_F$ are stage and terminal costs that are differentiable but possibly nonlinear or even non-convex.} \newrevision{Although \eqref{eqn:2-26-7} differs from the standard tube-based approach, the techniques introduced in this paper can be applied to a variety of MPC formulations.}

%Due to the assumption that $\X$ is an RCI, \eqref{eqn:2-26-10} is feasible for any $x_t \in \X$, giving rise to the property of \textit{recursive feasibility}. 
%This in turn implies that \eqref{eqn:2-26-7} is feasible when the optimal solution is implemented in a receding horizon fashion. 
%Since $\X$ and $\D$ are polytopes, the constraint \eqref{eqn:2-26-10} is itself a polytopic constraint, hence tractable.

\revision{In this paper, we seek to derive a safe feedback policy $\pi_\theta: \R^n \rightarrow \R^m$ that approximates the explicit solution to \eqref{eqn:2-26-7} by first approximating the optimal control sequence with a function $\mu_\theta: \R^n \rightarrow \R^{m\tau}$ and then implementing the first action of the sequence in the closed loop. In practice, any MPC policy implemented in closed loop must be stabilizing and recursively feasible. Recursive feasibility is the property that closed-loop trajectories generated by the MPC controller will not lead to states in which the MPC problem is infeasible. This property is guaranteed when $\S$ is RCI \cite{Blanchini2015}. If recursive feasibility is not guaranteed, then a backup controller must be developed or a control sequence that is feasible for the most immediate time steps can be used. There is suggestion in the literature that the latter approach performs quite well in practice~\cite{Wang2010}, but the theoretical aspects remain open. In terms of stability, recursive feasibility guarantees that trajectories will remain within a bounded set. Since this work focuses on constraint satisfaction, we do not consider stricter notions of stability.}

%Two important questions are the closed-loop stability and recursive feasibility of the system \eqref{eqn:2-26-5} when the controls are given by a finite horizon MPC. Roughly speaking, the system is stable and recursively feasible if \eqref{eqn:2-26-10} is satisfied at every step. The present focus is on constraint satisfaction, and for more discussion on stability and recursive feasibility, we refer the reader to \cite{Scokaert1999,Pannocchia2011,Mayne2005,Lofberg2012}.

% for relevant results on stability. Recursive feasibility is the property of \eqref{eqn:2-26-7} being feasible for all $x \in \X$ \cite{Lofberg2012}, but situations in which \eqref{eqn:2-26-7} is infeasible are outside the scope of this paper.

%However, in this paper we focus on systems with persistent disturbances, so asymptotic stability of the origin is not of chief concern. We focus on robustness/recursive feasibility and refer the reader to previous results on stability of MPC \todo{cite}. 


% Restate as parametric MPC problem

% In this paper, we are interested in parametric solutions to \eqref{eqn:2-26-7}. Specifically, 

%\subsection*{Proposed Approach}

 %The first goal of this paper is to derive a class of functions $\M := \{\mu_\theta(x_0): \S \rightarrow \F(x_0) \mid \theta \in \R^d\}$ consisting of functions that return feasible solutions to \eqref{eqn:2-26-7} for any initial condition $x_0 \in \S$ without using an iterative procedure, projection algorithm, or penalty function. The second goal is to find a set of parameters $\theta$ allowing $\mu_\theta$ to approximate the optimal solution to \eqref{eqn:2-26-7} as a function of $x_0$.
    
%In this paper, we propose a novel function class $\Pi := \{\pi_\theta: \R^n \rightarrow \R^{mT} \mid \theta \in \R^d\}$ consisting of functions that return feasible solutions to \eqref{eqn:2-26-7}, and we find a set of parameters $\theta$ allowing $\pi_\theta(x_t)$ to approximate the optimal solution.
%The selection of $\theta$ is posed as follows: \begin{subequations} \label{eqn:2-26-11} \begin{gather}
%    \min_\theta \sum_{t=0}^{T-1} l_t(x_t,u_t) + l_T(x_T) \label{eqn:2-27-2}\\
%    \text{subject to} \ \forall\ t \geq 0: x_{t+1} = Ax_t + Bu_t \\
%    x_{t+1} + d_t \in \X, \ \forall\ d_t \in \D \label{eqn:2-27-3}\\
%    u_t \in \U \label{eqn:2-27-4}\\
%    \u = \pi_\theta(x_0)
%\end{gather} \end{subequations}
%\todo{The cost \eqref{eqn:2-27-2} needs to be summarized over the unknown initial conditions, for example by taking the expectation or maximum with respect to $x_0 \in \X$. What is the correct thing to write? In other words, what function are we actually minimizing when we sample $x_0$ and run SGD?}

%A feedforward function $\mu_\theta$ would be much more computationally efficient than solving \eqref{eqn:2-26-7} using a standard solver~(which may require a large number of iterations).  The challenge comes in enforcing the constraints of \eqref{eqn:2-26-7} on the parameter $\theta$. The main contribution of this paper is to provide a function class $\M$ based on a neural network architecture that encodes these constraints explicitly through the structure of the network.

%The construction of $\M$ bears similarity to standard interior-point algorithms~\cite{Boyd2009}, in that there is a Phase I (find a strictly feasible starting point) and a Phase II (find the optimal solution). However, unlike interior-point methods, neither Phase I nor Phase II in our algorithm requires iterative steps, resulting in a feasible approximation to the solution of \eqref{eqn:2-26-7} that is efficient in terms of both computational cost and memory requirements. Phase I is solved by exploiting the structure of the MPC problem, and Phase II is solved using the proposed NN architecture. 

% We denote  as the feasible set of \eqref{eqn:2-26-7}, parameterized by initial condition $x_t$. Below, we introduce a class of policies  that return feasible solutions to \eqref{eqn:2-26-7}, and we describe the process for choosing the optimal $\theta$.


\section{Phase I: Finding a Feasible Point} \label{sec:p1}

\revision{The feasible set of \eqref{eqn:2-26-7} is a polytope $\F(x_0) \subseteq \R^{m\tau}$, defined by the following inequalities in $\u$: \begin{subequations} \begin{align}
        H_s(M_0 x_0 + M_u \textbf{u}) &\leq \tilde{h}_s, \label{eqn:5-22-22-4}\\
        H_u \textbf{u} &\leq h_u \label{eqn:5-22-22-5}
    \end{align} \end{subequations} where $H_s,H_u,M_0,M_u,\tilde{h}_s,$ and $h_u$ are block matrices and vectors derived from the system dynamics and constraints}.
\revision{In this paper, we assume that $\F(x_0)$ \newrevision{has nonempty interior} for all $x_0 \in \S$. Since the state constraints $\S$ form an RCI, $\F(x_0)$ is already guaranteed to be nonempty, and the assumption of nonempty interior is only marginally more restrictive.} 

\revision{The gauge map technique introduced in \cite{Tabas2021a} provides a way to constrain the outputs of a neural network $\mu_\theta: \R^n \rightarrow \R^{m\tau}$ to $\F(x_0)$ without a projection or penalty function, but $\F(x_0)$ must contain the origin in its interior. If this is not the case, then we must temporarily ``shift'' $\F(x_0)$ by subtracting any one of its interior points. In this section, we discuss several ways to reduce the complexity of finding an interior point.}

\revision{We begin by considering the feasibility problem for the one-step safe action set defined as $\V(x_0) = \{u \in \R^m \mid u \in \U, Ax_0 + Bu \in \T\},$ which is guaranteed to have an interior point by the assumption on $\F(x_0).$ \newrevision{One way to find an interior point of $\V(x_0)$ is to minimize the maximum constraint violation:}} \begin{subequations} \label{eqn:3-17-3} \begin{gather}
    \min_{u,s} s \label{eqn:8-23-22-1}\\
    \text{subject to: } F_s(Ax_0 + Bu) \leq \tilde{g}_s + s\textbf{1} \label{eqn:5-24-22-1}\\
    F_u u \leq g_u + s\textbf{1} \label{eqn:5-24-22-2}
\end{gather} \end{subequations} \revision{which has an optimal cost $s^* \leq 0$ if \newrevision{$\V(x_0)$ is nonempty}, and $s^* < 0$ if \newrevision{$\V(x_0)$ has nonempty interior} \cite{Boyd2009}. To avoid solving a linear program online during closed-loop implementation, the solution to \eqref{eqn:3-17-3} can be stored as a piecewise affine (PWA) function $\pi_0(x_0):\R^n \rightarrow \R^m$ \cite{Jones2007}. Although solutions to multiparametric LPs can be demanding on computer memory, we take advantage of the fact that feasibility problems have low accuracy requirements: any \newrevision{suboptimal} solution to \eqref{eqn:3-17-3} that achieves a cost $s < 0$ for all $x_0 \in \S$ is acceptable.
}
\newrevision{ \begin{definition}
A function $\pi_0: \R^n \rightarrow \R^m$ is said to solve \eqref{eqn:3-17-3} if,  for all $x_0 \in \S$, the optimal cost of \eqref{eqn:3-17-3} is negative when the decision variable $u$ is fixed at $\pi_0(x_0)$. 
\end{definition}}

\revision{Existing techniques for approximate multiparametric linear programming \cite{Filippi2004}, especially those that generate continuous solutions \cite{Spjotvold2005}, can be used to reduce the memory requirements of offline solutions to \eqref{eqn:3-17-3}.} 

\ifx
\newrevision{To characterize the set of allowable approximations to the optimal solution of \eqref{eqn:3-17-3}, we pose the following feasibility problem: \begin{align}
    \min_{u,s} 0
    \text{ s.t. } s < 0, \eqref{eqn:5-24-22-1}, \eqref{eqn:5-24-22-2}. \label{eqn:8-24-22-1}
\end{align} We will say that a function $\pi_0$ solves \eqref{eqn:8-24-22-1} if $\pi_0(x_0)$ solves \eqref{eqn:8-24-22-1} for all $x_0 \in \S$.}
\fi

\revision{To show just how far one can go with reducing complexity, we will construct an affine (rather than PWA)} 
%\newrevision{approximate} solution to \eqref{eqn:3-17-3} \newrevision{achieving negative cost \eqref{eqn:8-23-22-1} 
\newrevision{function that solves \eqref{eqn:3-17-3},} 
\revision{for the system studied in Section \ref{sec:sims}.} 
Let $\pi_0(x_0) = Wx_0 + w$. If $W \in \R^{m \times n}$ and $w \in \R^m$ satisfy
\begin{subequations} \label{eqn:3-17-1}
\begin{align}
    F_x(Ax_0 + B(Wx_0+w)) &< \tilde{g}_x \\
    F_u (Wx_0+w) &< g_u 
\end{align} \end{subequations} for all $x_0 \in \S$, then $\pi_0(x_0) = Wx_0 + w$ \newrevision{solves \eqref{eqn:3-17-3}}. %returns an interior point of $\V(x_0)$ for all $x_0 \in \S$. 
\newrevision{The following optimization problem} can be solved to find $W$ and $w$ or certify that none exists. Let $\Y(s) = \{x_0 \in \R^n \mid F_s(Ax_0 + B(Wx_0+w)) \leq \tilde{g}_s + s \textbf{1}, F_u (Wx_0 + w) \leq g_u + s \textbf{1}\}.$ \newrevision{If the optimal cost of} \begin{gather}
    \min_{W,w,s} s 
    \text{ subject to } \newrevision{\S} \subseteq \Y(s) \label{eqn:3-17-4}
\end{gather} \newrevision{is negative, then %$x_0 \in \Y(s^*)$ for all $x_0 \in \S$, thus
\eqref{eqn:3-17-1} holds for all $x_0 \in \S$, thus $\pi_0$ solves \eqref{eqn:3-17-3}. This happens to be the case for the example in Section \ref{sec:sims}, taken from \cite{Zeilinger2011}}. The constraint in \eqref{eqn:3-17-4} is a polytope containment constraint in halfspace representation, \newrevision{thus \eqref{eqn:3-17-4} can be solved as} a linear program \cite{Sadraddini2019}.

\revision{Now consider the feasibility problem for $\F(x_0)$, which is obtained by replacing \eqref{eqn:5-24-22-1} and \eqref{eqn:5-24-22-2} with \eqref{eqn:5-22-22-4} and \eqref{eqn:5-22-22-5}, and changing the optimization variable from $u \in \R^m$ to $\u \in \R^{m\tau}$. One would naturally expect the complexity of the PWA solution to this feasibility problem to increase rapidly with the time horizon $\tau$, as more decision variables and constraints are added. However, the next proposition shows that the cardinality of the stored partition can be made constant in $\tau$.}
\revision{\begin{proposition}[Phase I solution] \label{prop:2-27-1} If $\pi_0$ \newrevision{solves \eqref{eqn:3-17-3},} then the \newrevision{vector $\mu_0(x_0) := \m{\pi_0(x_0)^T,\ldots,\pi_0(x_{\tau-1})^T}^T$}, where $x_{k+1} = Ax_k + B \pi_0(x_k)$, is an interior point of $\F(x_0)$ for any $x_0 \in$ \newrevision{$\S$}.
\end{proposition}
\begin{proof} \newrevision{If $\pi_0$ solves \eqref{eqn:3-17-3}, then $\pi_0(x) \in \int \V(x)$ for all $x \in \S$. Applying the definition of $\V$ in an inductive argument,} it is straightforward to show that the \newrevision{state trajectory associated with $\mu_0(x_0)$ is entirely contained } %policy $\pi_0$ generates trajectories
in $\S$. Fix any such trajectory $\{x_1,\ldots,x_\tau\} \subset \S$ originating from $x_0 \in \S$ under policy $\pi_0$. For any $k \in \{1,\ldots,\tau\}$, \newrevision{$x_k \in \S$ implies $\pi_0(x_k) \in \int \V(x_k)$, which} implies $\pi_0(x_k) \in \int \U$ and $Ax_k + B\pi_0(x_k) \in \int \T$. Since this holds for all $k$, the constraints defining $\F(x_0)$ hold strictly at $\mu_0(x_0)$. %We conclude $\mu_0(x_0) \in \int \F(x_0)$.
%Using induction and the definition of $\V$, it is straightforward to show that if $x_0 \in \S$ then $\pi_0$ generates a trajectory in $\S$. For any trajectory $\{x_0,\ldots,x_\tau\} \subset \S,$ the Cartesian product $\V(x_0) \times \cdots \V(x_{\tau-1})$ is nonempty and contains $\mu_0(x_0)$ in its interior. By the definition of $\V$, this product set is a subset of $\F(x_0),$ thus $\mu_0(x_0) \in \int \F(x_0).$
%For any $x_k \in \S, k \in \{0,\ldots,\tau-1\},$ we have $\pi_0(x_k) \in \int \V(x_k),$ implying $x_{k+1} := Ax_k + B \pi_0(x_k) \in \int \T \subset \S$. Since $x_0 \in \S$, we conclude from induction that $\pi_0$ generates trajectories in $\S$. Therefore, $\pi_0(x_k) \in \int \U$ and $Ax_k + B\pi_0(x_k) \in \int \T$ for all $k$. We conclude that $\mu_0(x_0) \in \int \F(x_0)$. 
\end{proof}}

%Proposition \ref{prop:2-27-1} removes dependence on the length of the time horizon $\tau$ but leaves unanswered the question of how to find an interior point of $\V(\cdot)$. Next, we show how to construct an affine or PWA function $\phi$, which returns such an interior point. Unsurprisingly, the approximation $\phi$ is of substantially lower complexity than a PWA solution to \eqref{eqn:2-26-7}, since it is often significantly easier to find a feasible control action compared to the optimal action. 

% in the sense of Proposition \ref{prop:2-27-1}.

%Let $\phi_{\text{affine}}(x) = Wx + w$ where $W \in \R^{m \times n}$ and $w \in \R^m$, and let $\Y(s) = \{x \in \R^n \mid F_x(Ax + B \phi_{\text{affine}}(x)) \leq \tilde{g}_x + s \textbf{1},\ F_u \phi_{\text{affine}}(x) \leq g_u + s\textbf{1}\}.$

%If $s^* \geq 0,$ then $\phi$ cannot be expressed as an affine function. In that case, we can increase its complexity by expressing it as a PWA function generated by solving the following multiparametric linear programming (mpLP) problem over the parameter set $x \in \X$: 

%Solving the mpLP \eqref{eqn:3-17-3} is substantially more efficient than solving \eqref{eqn:2-26-7} because \eqref{eqn:3-17-3} has a linear cost, lower accuracy requirement, and a factor of $\tau$ fewer decision variables and constraints. 
In our simulations on the example from \cite{Zeilinger2011},  $\eqref{eqn:3-17-4}$ was feasible with negative optimal cost, \newrevision{meaning that a polyhedral partition of the state space was not needed}
%i.e only one region was needed
(see Section \ref{sec:sims}). This indicates that the minimum number of regions \newrevision{in a polyhedral state space partition associated with a PWA
%to find an admissible 
solution to \eqref{eqn:3-17-3}} is in general very small relative to the number of regions in an explicit solution to \eqref{eqn:2-26-7}. %\todo{Say something about in our simulation how many pieces was needed. E.g., linear function was successful or only one piece.} 
%Any technique for approximate mpLP (e.g. \cite{Filippi2004,Spjotvold2005} and the references therein) can solve \eqref{eqn:3-17-3}.

\section{Phase II: Optimizing Performance} \label{sec:p2}

In this section, we construct a class of policies from $x_0 \in \S$ to $\F(x_0)$, that can be trained using standard machine learning packages. Although it is difficult to constrain the output of a neural network to an arbitrary polytope such as $\F(x_0)$, it is easy to constrain the output to the hypercube $\B_\infty$ by applying a clamping function elementwise in the output layer. We apply a mapping between polytopes that is closed-form, differentiable, and bijective. This mapping establishes an equivalence between $\B_\infty$ and $\F(x_0)$, allowing one to constrain the outputs of the policy to $\F(x_0)$. The mapping from $\B_\infty$ to $\F(x_0)$ is called the \textit{gauge map}. The concept is illustrated in Figure \ref{fig:nn_diagram}. 

\begin{figure}
    \centering
    \includegraphics[width=7cm]{Figures/nn_diagram.png}
    \caption{The proposed control policy uses a neural network combined with the Phase I solution and a \textit{gauge map} to constrain the decision $\u$ to the MPC feasible set $\F(x_0)$. The first action from the sequence $\u$ is extracted and implemented. On the right, the action of the gauge map is illustrated.}
    \label{fig:nn_diagram}
\end{figure}

We begin constructing the gauge map by introducing some preliminary concepts. A \textit{C-set} is a convex, compact set that contains the origin as an interior point. The \newrevision{\textit{gauge function} with respect to C-set $\P \subset \R^n$, denoted $\gamma_{\P}: \R^n \rightarrow \R_+$, is the function whose sublevel sets are scaled versions of $\P$. Specifically,} the gauge of a vector $v$ with respect to $\P$ is given by $\gamma_\P(v) = \inf\{\lambda \geq 0 \mid v \in \lambda \P\}.$ If $\P$ is a polytopic C-set given by $\{v \in \R^k \mid Fv \leq g\}$, then \newrevision{$\gamma_{\P}$ is the pointwise maximum over a finite set of affine functions \cite{Tabas2021a}:} \begin{align} \gamma_\P(v) = \max_i \frac{F^{(i)T}v}{g^{(i)}}. \label{eqn:5-25-22-2} \end{align} %\cite{Tabas2021a}. 
%The gauge function of a point in the $\infty$-norm ball $\B_\infty$ is the $\infty$-norm of the point, i.e. $\gamma_{\B_\infty}(\cdot) = \|\cdot\|_\infty$. 
Given two C-sets $\P$ and $\Q$, the \textit{gauge map} $G: \P \rightarrow \Q$ is %defined as 
\begin{align} G(v \mid \P,\Q) = \frac{\gamma_\P(v)}{\gamma_\Q(v)} \cdot v. \label{eqn:3-21-1} \end{align} 
\newrevision{This function maps level sets of $\gamma_{\P}$ to level sets of $\gamma_{\Q}$.}
%\revision{The gauge map from $\P$ to $\Q$ applies a scaling factor to points in $\P$ such that the image of any level set of $\gamma_{\P}$ under the gauge map is equal to the corresponding level set of $\gamma_{\Q}$. }

\begin{proposition} \label{prop:2-27-2}
Given two polytopic C-sets $\P$ and $\Q$, the gauge map $G: \P \rightarrow \Q$ is subdifferentiable and bijective. Further, given a function $\pi_0$ from Proposition \ref{prop:2-27-1}, the set $\tilde{\F}(x) := [\F(x)-\pi_0(x)]$ is a C-set for all $x \in \S$.
\end{proposition}

\begin{proof}
The properties of subdifferentiability and bijectivity come from \cite{Tabas2021a}. For the C-set property, fix $x \in \S$. Since $\S,$ $\U,$ and $\D$ are convex and compact, so is $\F(x)$. Since $\mu_0(x)$ is an interior point of $\F(x),$ the set $\tilde{\F}(x)$ contains the origin as an interior point and is therefore a C-set.
\end{proof}

We now use the gauge map in conjunction with the Phase I solution to construct a neural network whose output is confined to $\F(x_0).$
%present a novel class of functions with superior flexibility to approximate the solution to \eqref{eqn:2-26-7}. 
%From there, the gauge map and Phase I solution will map the output to $\F(x_t)$.
Let $\psi_\theta: \S \rightarrow \B_\infty$ be a neural network parameterized by $\theta$. \revision{A safe policy is constructed by composing the gauge map $G: \B_\infty \rightarrow \tilde{\F}(x_0)$ with $\psi_\theta$, then adding $\mu_0(x_0)$ to map the solution into $\F(x_0)$:}%, and define $\mu_\theta: \S \rightarrow \F(x_0)$ as 
\begin{align}
    \mu_\theta(x_0) = G(\cdot \mid \B_\infty, \tilde{\F}(x_0)) \circ \psi_\theta(x_0)  + \mu_0(x_0). \label{eqn:3-24-1}
\end{align} \revision{Computing the gauge map online simply requires evaluating $H_s M_0 x_0$ from \eqref{eqn:5-22-22-4} as well as the operations in \eqref{eqn:5-25-22-2}.}

The function $\mu_\theta$ has several important properties for approximating the optimal solution to \eqref{eqn:2-26-7}. First, it leverages the universal function approximation properties of neural networks \cite{Hornik1989} along with the bijectivity of the gauge map (Proposition \ref{prop:2-27-2}) to explore all interior points of $\F(x_0).$ This is an advantage over projection-based methods \cite{Chen2018d} which may be biased towards the boundary of $\F(x_0)$ when the optimal solution may lie on the interior. Second, $\mu_\theta$ is evaluated in closed form, and its outputs are constrained to $\F(x_0)$ without the use of an optimization layer \cite{Maddalena2020} that may have high computational overhead. Finally, the subdifferentiability of the gauge map (Proposition \ref{prop:2-27-2}) enables selection of parameter $\theta$ using standard automatic differentiation techniques.

%The class $\Pi$ inherits the property of universal function approximation from standard feedforward neural networks with ReLU activation functions \cite{Hornik1989}, and since the range of all functions in $\Pi$ is constructed to be exactly the feasible set of \eqref{eqn:2-26-7} for any initial condition $x_t \in \X$, the class $\Pi$ can approximate any feasible policy.

%and, by Proposition \ref{prop:2-27-2}, inherits the approximation capabilities of $\psi_\theta$. 
\subsection*{Optimizing the parameter $\theta$}

Similar to the approach taken in \cite{Akesson2006}, we optimize $\theta$ by sampling $x \in \S$ and applying stochastic gradient descent. At each iteration, a new batch of initial conditions $\{x_0^j\}_{j=1}^M$ is sampled from $\S$ and the loss is computed as \begin{align}
    J(\theta) = \frac{1}{M} \sum_{j=1}^M \sum_{k=0}^{\tau-1} l(x_k^j,u_k^j) + l_F(x_{\tau}^j) \label{eqn:2-27-10}
\end{align} with the control sequences $\u^j$ given by $\mu_\theta(x_0^j)$ and state trajectories $\x^j$ generated according to the nominal dynamics. \revision{The parameters $\theta$ are updated in the direction of $\nabla_\theta J$, which is easily computed using automatic differentiation~\cite{Gune2018}.}

%where $\hat{\u}^j_t = \pi_\theta(x^j)$, $\hat{x}_{k+1}^j = A\hat{x}_k^j + B\hat{u}_k^j,$ and $\hat{x}_0^j = x^j$.
\section{Simulations} \label{sec:sims}

\subsection{Test systems}

%We simulate the proposed policy on two standard test systems used for studying parameteric MPC (see, e.g.,  \cite{Domahidi2011,Chen2018d,Zeilinger2011}). 
%The first is a double integrator system with $n = 2$ and $m = 1$. The system matrices and constraints are \begin{gather}
%    A = \m{1&1\\0&1},\ B = \m{0 \\1}, \\
%    \|x\|_\infty \leq 1,\ \|u\|_\infty \leq 1,\ \|d\|_\infty \leq 0.1 \label{eqn:2-27-7}.
%\end{gather} 

%A time horizon of $\tau=5$ is used. The interior point function $\pi_0$ is synthesized using \eqref{eqn:3-17-4}:
%\begin{gather*}
%    W = \m{-0.520 & -1.37},\ w = 0.0108.
%\end{gather*}

We simulate the proposed policy using a modified example from \cite{Zeilinger2011} with $n=3$, $m=2,$ and $\tau=5$. The system matrices, constraints, costs, and Phase I solution (found using~\eqref{eqn:3-17-4}) are given below:
\begin{gather}
    A = \m{-.5&.3&-1 \\.2&-.5&.6\\1&.6&-.6},\ B = \m{-.601&-.890\\.955&-.715\\.246&-.184}, \\
    \|x\|_\infty \leq 5,\ \|u\|_\infty \leq 1,\ \|d\|_\infty \leq 0.1 \label{eqn:2-27-8}, \\
    l(x,u) = \|x\|_2^2 + c_1\|u\|_2^2, \ l_F(x) = c_2\|x\|_2^2 \\
    W = \m{0.116 &  0.210 & -0.370 \\
       -0.320 & -0.104 & -0.122}, w = \m{-0.157 \\ -0.0533} \nonumber
\end{gather} where $c_1$ and $c_2$ are positive constants. Although quadratic costs are used in the simulations, the proposed method can work with any differentiable cost. 

\revision{We evaluate the performance of a given policy in both open- and closed-loop experiments. In the open-loop experiments, we evaluate the MPC cost \eqref{eqn:2-26-8} and compare it to the optimal cost. The fraction suboptimality is \begin{align}
    \delta = \frac{c_{nn} - c_{mpc}}{c_{mpc}} \label{eqn:5-25-22-1}
\end{align} where $c_{nn}$ is the average cost \eqref{eqn:2-26-8} incurred by the control sequence $\mu_\theta$ on a validation set $\{x_0^j\}_{j=1}^{N_{val}} \subset \S$ and $c_{mpc}$ is the optimal cost.}

\revision{In the closed-loop experiments, we evaluate the performance of a policy $\pi_\theta(x_t): \R^n \rightarrow \R^m, t \geq 0$ which is derived from $\mu_\theta(x_t)$ by taking the first action in the sequence. We simulate \eqref{eqn:2-26-5} for $T \gg \tau$ time steps.}
The trajectory cost in the closed-loop experiments is computed as 
    $\sum_{t = 0}^{T-1} l(x_t,u_t) + l_F(x_T)$ %where the stage and terminal costs are given by $
    %l(x_t,u_t) = \|x_t\|_2^2 + c_1\|u_t\|_2^2,\ l_F(x_T) = c_2 \|x_T\|_2^2
%$ with $c_1, c_2 > 0$. 
and the disturbance is modeled as an autoregressive sequence \cite{Srinath1995}, 
$d_{t+1} = \alpha d_t + (1-\alpha) \hat{d}$ %\label{eqn:3-19-1}$ 
where $\alpha \in (0,1)$ and $\hat{d}$ is drawn uniformly over $\D$.

\subsection{Benchmarks}

We compare the proposed method to two of the most common approaches for learning a solution to \eqref{eqn:2-26-7}. The first benchmark is a penalty-based approach \cite{Drgona2020} which enforces the constraints \eqref{eqn:2-26-10} and \eqref{eqn:2-27-9} by augmenting the cost \eqref{eqn:2-27-10} with a linear penalty term on constraint violations given by $\beta \cdot \max\{0,F_x x_t-\tilde{g}_x\}$ where the $\max$ is evaluated elementwise and $\beta > 0$. Since the penalty-based approach does not encode state constraints in the policy, the policy is constrained to the Cartesian product $\U^\tau = \prod_{k=0}^{\tau-1} \U$ using scaled $\tanh$ functions elementwise.

The second benchmark is a projection-based approach \cite{Chen2018d} which constrains the policy to the set $\F(x_0)$ by solving a convex quadratic program in the output layer of a neural network \cite{Agrawal2019}. % Instead of using $\tanh$ activations to clamp the output of a neural network to $\B_\infty$ and then using the gauge map to constrain the policy to $\F(x_t)$, the projection-based approach uses scaled $\tanh$ activations to clamp the neural network output $\hat{\textbf{v}}_t$ to the product set $\U^\tau$ and then projects the result onto $\F(x_t) \subseteq \U^\tau$ by solving the following projection:
The optimization layer $\textbf{v} \rightarrow \textbf{u}$ returns
\begin{gather*}
    \underset{\u}{\arg \min} \|\textbf{v} - \u\|_2^2
    \text{ subject to } \u \in \F(x_0).
\end{gather*}

Another class of approaches to learning-based MPC seeks to learn the optimal solution to \eqref{eqn:2-26-7} using regression \cite{Parisini1995, Domahidi2011,Maddalena2020}. Specifically, data-label pairs $(x_0,u_0^*)$ are generated by sampling $x_0$ from $\S$, solving \eqref{eqn:2-26-7} for each sample, and extracting $u_0^*$ from the optimal solution $\u^*$. Then, a neural network or other function approximator is trained to learn the relationship between $x_0$ and $u_0^*$. Performance and constraint satisfaction are handled e.g. by bounding the approximation error with respect to the MPC oracle. We do not compare against this type of approach since it requires a large number of trained samples, making it difficult to compare with our and the other unsupervised examples. 

\subsection{Neural network design}

\revision{The neural networks were designed with $n$ inputs, $m\tau$ outputs, and two hidden layers with rectified linear unit (ReLU) activation functions. The width of the networks was chosen during hyperparameter tuning. In particular, we performed 30 iterations of random search over the width of the network (number of neurons per hidden layer) $\in \{64,\ldots,1024\}$, the batch size (number of initial conditions, $M$) $\in \{100,\ldots,3000\}$ and the learning rate (LR, the step size for gradient descent) $\in [10^{-5},10^{-3}]$. For each set of hyperparameters under consideration, we computed the validation score using \eqref{eqn:5-25-22-1} with $N_{val}=100$. The hyperparameters after tuning are reported in Table \ref{table:hparams}.}

\begin{table}
\centering
\caption{Hyperparameters for the three neural networks.}
\begin{tabular}{ |c|c|c|c| } 
 \hline
 Type & Width & LR & $M$\\ 
 \hline
 Gauge & $859$ & $4.7 \times 10^{-4}$ & $1655$ \\%& 0.0070 & .0015\\ 
 Penalty & $318$ & $8.7 \times 10^{-4}$ & 133 \\%&0.0083 \\ 
 Projection & $956$ & $9.0 \times 10^{-5}$ & $813$ \\%& .010 & .024\\
 \hline
\end{tabular}
\label{table:hparams}
\end{table}

\subsection{Simulation results}
Here we compare our proposed approach (Gauge NN), the penalty-based approach (Penalty NN), the projection-based approach (Projection NN) and the ``ground truth'' obtained by solving  \eqref{eqn:2-26-7} online in \texttt{cvxpy}. \revision{The results of the open-loop experiments are shown in Table \ref{table:open_loop}, with performance computed relative to the optimal MPC solution using \eqref{eqn:5-25-22-1} with $N_{val}= 100$ trials. The proposed Gauge NN achieves lower cost compared to the projection-based method, and has a much lower computational complexity (solve time is only 6\% of projection).  Table \ref{table:open_loop} only compares the NNs with safety guarantees because constraint violations are not accounted for in \eqref{eqn:5-25-22-1}.} 

\begin{table}
\centering
\caption{Open-loop test results.}
\begin{tabular}{ |c|c|c| } 
 \hline
 Type & $\delta$ \eqref{eqn:5-25-22-1} & Solve time (sec)\\ 
 \hline
 Gauge & 0.007 & .0015\\ 
 Projection & 0.010 & .024\\
 \hline
\end{tabular}
\label{table:open_loop}
\end{table}

Figure \ref{fig:train} shows the training curves for each type of network. The lower training cost achieved by the Gauge NN illustrates that it can be more efficient to explore the interior of the feasible set than the boundary. \revision{Since the MPC cost in the simulations is strictly convex, solutions with lower cost are closer to the optimal solution.}
%Thus, a lower training cost directly corresponds to a shorter distance between $\mu_\theta$ and the optimal solution.} %The large initial costs incurred by the Penalty NN are due to constraint violations, which the network learns to avoid (albeit without guarantees). 

%\begin{figure}[b]
%    \centering
%    \includegraphics[height=4cm]{Figures/loss_2n1m_False_copy.png}
%    \hfill
%    \includegraphics[height=4cm]{Figures/loss_3n2m_False_copy.png}
%    \caption{Training trajectories for the 2-state system (left) and 3-state system (right). Our proposed Gauge-based approach achieves lower cost at a much faster rate.}
%    \label{fig:train}
%\end{figure}

\begin{figure}
    \centering
    \includegraphics[width=8cm]{Figures/loss_3n2m_False_no_zoom.png}
    \caption{Training trajectories for the three types of neural netwokrs. Our proposed Gauge-based approach achieves lower cost at a much faster rate.}
    \label{fig:train}
\end{figure}

Figure \ref{fig:pareto_3} compares the policies in terms of computation time and test performance. The box-and-whisker plots indicate the range of performance over 100 test trajectories of length $T=50$, while the vertical position of each box indicates the average time to compute a control action. Of the policies with safety guarantees (Gauge NN, Projection NN, and online MPC), \revision{the Gauge NN achieves Pareto efficiency in terms of average solve time and median trajectory cost.} 
%Interestingly, the Gauge NN can achieve a lower median cost than the online MPC. 
Our intuition behind the high performance of the neural networks is that \eqref{eqn:2-26-7} is a heuristic and the unsupervised learning approach can lead to better closed-loop policies.
%there is no reason why solving an MPC problem on horizon $\tau$ should serve as a lower bound on trajectory cost when testing on the longer horizon $T$ with disturbances. With the right architecture and training algorithm, NNs can learn better policies. 

\begin{figure}
    \centering
    \includegraphics[width=8cm]{Figures/pareto.png}
    \caption{Solve time vs. trajectory cost for the networks under consideration applied to the 3-state system. The Gauge NN is Pareto-efficient in terms of cost and computation time compared to the other techniques with safety guarantees (Online MPC and Projection NN).}
    \label{fig:pareto_3}
\end{figure}


%\todo{Add a figure showing solve time vs. horizon length for NN and MPC and a table of solve times for $\tau=5$. Alternatively, I could try to put the box plot y-axis on a log scale. Thoughts?}

\ifx

\subsection{Extension to scenario-based MPC}
A drawback to both standard online MPC and neural network based methods is that they minimize the cost of a nominal trajectory, where the disturbance is not explicitly taken into account. 
% A possible explanation for the differences in performance shown in Figures \ref{fig:pareto_2} and \ref{fig:pareto_3} is that both the online MPC and the neural networks seek to minimize the cost of a nominal trajectory of length $\tau$, but are tested on a length-$T$ trajectory with disturbances. 
One way to address the issue of disturbances is via the \emph{scenario approach} \cite{Lucia2018}, in which the cost \eqref{eqn:2-26-8} is averaged over $N$ sequences of disturbances sampled from $\D$ according to \eqref{eqn:3-19-1}. The MPC \eqref{eqn:2-26-7} is replaced by \begin{subequations}
\begin{gather}
    \min_{\hat{\u}_t} \frac{1}{N} \sum_{i=1}^N \sum_{k=t}^{t+\tau-1} l(\hat{x}_k^i,\hat{u}_k) + l_F(\hat{x}_{t+\tau}^i) \\
    \text{s.t. }
    \hat{x}_{k+1}^i = A\hat{x}_k^i + B\hat{u}_k + d_k^i \ \forall\ i,k;\ \hat{x}_t^i = x_t  \label{eqn:3-20-1}\\
    \tilde{x}_{k+1} = A\tilde{x}_k + B\hat{u}_k \ \forall\ k;\ \tilde{x}_t = x_t  \label{eqn:3-21-2}\\
    \tilde{x}_{k+1} \in \T,\ \hat{u}_k \in \U \ \forall\ i,k. \label{eqn:3-21-3}
\end{gather} \end{subequations} For feasibility under any disturbance sequence, \eqref{eqn:3-21-3} constrains the nominal trajectory in \eqref{eqn:3-21-2} to the target set.
For the NN, the policy class \eqref{eqn:3-24-1} remains unchanged but the loss function \eqref{eqn:2-27-10} is averaged over scenarios, becoming \begin{align}
    J(\theta) = \frac{1}{MN} \sum_{i=1}^N \sum_{j=1}^M \sum_{k=0}^{\tau-1} l(\hat{x}_k^{i,j},\hat{u}_k^{j}) + l_F(\hat{x}_{\tau}^{i,j})
\end{align} with the controls $\hat{\u}_0^{j}$ given by $\pi_\theta(x_0^j)$ and trajectories by \eqref{eqn:3-20-1} for each scenario $i$ and each initial condition $j$.

Figure \ref{fig:pareto_s} shows that when noise is considered using the scenario approach, the MPC computation times increase sharply while the Gauge NN remains competitive in terms of median trajectory cost. The efficiency gained by taking an interior point perspective is multiplied when the designer wishes to account for disturbances using a scenario approach.

\begin{figure}
    \centering
    \includegraphics[width=8cm]{Figures/pareto_3n2m_True.png}
    \caption{Solve time vs. trajectory cost in the scenario-based setting, for the 3-state system. The computation time for online MPC increases dramatically, while Gauge NN achieves both lowest computation time and trajectory cost.}
    \label{fig:pareto_s}
\end{figure}

\fi
\section{Conclusion}
\label{sec:conclusion}
We propose \ours{}, a novel video understanding framework for effective domain adaptation of MLLMs.
We propose to automatically collect skill-specific CoT annotations from video QA datasets and construct a skill-based reasoning pipeline that combines a lightweight skill assigner with a collection of LoRA-based expert adapters.
Empirical results on three diverse benchmarks demonstrate consistent gains of \ours{} over strong baselines, highlighting the enhanced quality of our reasoning traces.


\bibliography{references2}
\bibliographystyle{ieeetr}

\end{document}
