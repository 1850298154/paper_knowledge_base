
In this work, we designed efficient optimization algorithms that either optimize trajectories directly or learn a sampling density and an associated sampling pattern in MRI. Overall, the main highlights of this work are:
\begin{itemize}
 	\item The compressed sensing theories designed for the Fourier-Wavelet system with $\ell^1$ reconstruction (e.g. \cite{adcock2021compressive}) seem nearly optimal from an experimental point of view. Sampling schemes can be designed based on a density that is close to Shannon's rate at the k-space center and that decays towards the high frequencies. The precise shape of the density depends on the images structure. 
 	\item In that context, the Bayesian optimization of densities is an attractive method to design sampling schemes. It works with small datasets, ensures the convergence to a global minimizer. Its performance is close to much heavier trajectory optimizers and is from one to two orders of magnitude faster. 
 	\item In the case of unrolled neural network reconstructions, the proposed Bayesian optimization framework is still interesting with gains of up to $1$dB in average on the fastMRI knee validation set. However, the gain can be nearly doubled with a direct optimization of the trajectories. A possible explanation for this fact is that the family of densities is too poor to describe the best convoluted trajectories. \rev{Another one is that the theoretical bases of the compressed sensing theory break down when using neural network reconstruction methods. This calls for a renewed theory for this fast expanding field}.
 	\item As points of minor importance, we improved the Sparkling trajectories \cite{lazarus2019sparkling}, by changing the discrepancies and provided various improvements to the direct optimization of trajectories by using the Extra-Adam algorithm to handle hard constraints and by training reconstruction networks on families of operators.
    \item \rev{As a prospect, the extensions of these ideas to 3D multi-coil imaging is particularly relevant. There is no a priori obstacle to apply this rather lightweight formalism and to obtain dataset tailored sampling distributions.}
 \end{itemize} 