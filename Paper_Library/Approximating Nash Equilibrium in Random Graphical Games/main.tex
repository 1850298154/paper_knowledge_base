\documentclass[11pt, letterpaper]{article}

\usepackage{games}
\usepackage{verbatim}
%\input{content/glossary}

% changing the following can be used to toggle author notes between on and off

\usepackage{tikz}
\thispagestyle{empty}

\begin{document}

\title{Approximating Nash Equilibrium in Random Graphical Games}
\author{ Morris Yau \thanks{ \texttt{morrisy@mit.edu}}}
\affil{Massachusetts Institute of Technology}
\date{\today}
\maketitle

\begin{abstract}
       Computing Nash equilibrium in multi-agent games is a longstanding challenge at the interface of game theory and computer science.  It is well known that a general normal form game in $N$ players and $k$ strategies requires exponential space simply to write down.  This \emph{Curse of Multi-Agents} prompts the study of \emph{succinct} games which can be written down efficiently.  A canonical example of a \emph{succinct} game is the \emph{graphical game} which models players as nodes in a graph interacting with only their neighbors in direct analogy with markov random fields.  Graphical games have found applications in wireless, financial, and social networks.  However, computing the nash equilbrium of graphical games has proven challenging. Even for \emph{polymatrix} games, a model where payoffs to an agent can be written as the sum of payoffs of interactions with the agent's neighbors, it has been shown that computing an $\epsilon$ approximate nash equilibrium is PPAD hard for $\epsilon$ smaller than a constant.  The focus of this work is to circumvent this computational hardness by considering average case graph models i.e random graphs.  We provide a quasipolynomial time approximation scheme (QPTAS) for computing an $\epsilon$ approximate nash equilibrium of polymatrix games on random graphs with edge density greater than $\poly(k, \frac{1}{\epsilon}, \ln(N))$ with high probability.  Furthermore, with the same runtime we can compute an $\epsilon$-approximate Nash equilibrium that $\epsilon$-approximates the maximum social welfare of any nash equilibrium of the game.  Our primary technical innovation is an "accelerated rounding" of a novel hierarchical convex program for the nash equilibrium problem.  Our accelerated rounding also yields faster algorithms for Max-2CSP on the same family of random graphs, which may be of independent interest.  
       
\end{abstract}



\thispagestyle{empty}
\setcounter{page}{0}
\newpage

\tableofcontents

\thispagestyle{empty}
\setcounter{page}{0}
\newpage



\section{Introduction}

Beginning with the seminal works of Nash \cite{Nash48} and Von Neumann \cite{vonneumann1947}, game theory has been concerned with the study of strategic interactions amongst rational agents. In his landmark 1951 paper, Nash defined his namesake equilibrium and proved its existence for every game in mixed strategies.  Since the 1950's, the rise of the internet heralded a new brand of commerce for computational ecosystems comprised of many agents.  Today, the applications of game theory span a remarkable range from cryptocurrency and blockchain technologies to the deployment and design of 5G networks and edge devices.  These developments brought computation to the forefront of game theory, wherein the problem of computing the Nash equilibrium of two player games came to be of paramount importance.  

Towards resolving this problem in the negative, Daskalakis, Goldberg, and Papadimitriou \cite{DP06} in their oft celebrated work proved that computing two player nash equilibrium is PPAD-hard.  As is standard in theoretical computer science, the subsequent hope is to approximate the nash equilibrium to additive error $\epsilon$.  These efforts have proven fruitful, as a quasipolynomial time approximation scheme was achieved by Lipton, Mehta, and Markakis \cite{LMM03}.  Despite the good news, extending these techniques to larger numbers of players is met with an immediate obstacle.  

Quite simply, the normal form game between $N$ agents taking $k$ actions requires $k^N$ space to write down. Even linear time algorithms would have runtimes exponential in $N$.  Thus, an algorithmic theory of multi-agent games would have to be married with succinct models of multi-agent interactions.  A particularly attractive modeling choice is that of the \emph{graphical game} introduced by Kearns, Littman, and Singh \cite{KLS13}.  \emph{Graphical games} are a class of games where agents are represented as nodes in a graph interacting with their neighbors.  At one extreme, every game can be modeled by interactions on the complete graph which offers no savings in space. At the other end of the spectrum, games played on sparse graphs can be succinctly represented and capture settings where agents are influenced by a comparatively small set of neighbors.  This is especially relevant in network settings where devices compete for bandwidth resources with their proximal neighbors; in financial networks where institutions transact with peer institutions; in social networks where agents can influence a small set of trusted associates.  As such, \emph{graphical games} are practically well motivated models for mitigating what is colloquially known as the \emph{Curse of Multi-agents}.         

Of course, the elephant in the room is whether Nash equilibrium can be efficiently computed in graphical games.  The answer to this question ought to depend on the structure of the game.  However, even for polymatrix games, which are graphical games where the payoff to each agent is the sum of the payoffs of games played with the agent's neighbors in a graph, it is PPAD hard to compute a nash equilibrium.  Even worse for the algorithmist, Rubinstein \cite{Rubinstein14a} showed that it is PPAD hard to $\epsilon$-approximate the nash equilibrium of a polymatrix game for small constant $\epsilon > 0$.  Moreoever, the hard instance is a simple $3$-regular graph.  This astounding negative result stands in stark contrast with the good news enjoyed by approximation algorithms in two player games.  In light of strong negative results, the next natural step in the theoretical computer science toolbox is to consider average case models for games.  In this work, we aim to answer the question\\

\textit{Does there exist a large class of average case games on graphs, for which it is possible to design efficient}
\centerline{\textit{algorithms for approximating nash equilibrium?}}
\\
\\
At a high level, our answer to this question is that for polymatrix games with smooth payoffs on random graphs that are mildly dense, approximating a nash equilibrium is as easy as approximating the nash equilibrium of a two player game.  That is to say, there exists a quasipolynomial time approximation scheme.  The intuitive reason for this is that for a random graph with smooth payoffs, the effect of an agents neighbors can be captured by succinct aggregate statistics of their actions reminiscent of both mean field theory in statistical physics and the study of maximum constraint satisfaction problems in approximation algorithms.  It is this latter family of problems from which we draw technical inspiration.  

\section{Preliminaries}
We begin by introducing some notation and defining the polymatrix game.  We will also define what it means for a payoff function to be $L$-smooth.
\paragraph{Notation:}  We work with $N \in \Z^+$ player games where each player can adopt one of $k \in \Z^+$ strategies.  For a graph $G$ let $V$ and $E$ be the vertices and edges of the graph respectively.  For a node $i \in V$ let $N(i)$ denote the set of vertices that are neighbors of $i$.  Let $a_i \in [k]$ denote the strategy of player $i$.  Let $\vec{a} \in [k]^N$ denote the vector of strategies $(a_1,a_2,...,a_N)$ adopted by all $N$ players.  Let $\calP(k,V)$ be the space of probability distributions over $[k]^N$.  Let $\Gamma \defeq \{\{\mu_{ip}\}_{p \in [k]}\}_{i \in [N]}$ denote a product probability distribution over $[k]^N$ where $\mu_{ip} = \mathbb{P}(a_i = p)$.  The focus of this work will be on random graphs drawn from $G(N,p)$ where $p$ is the probability of an edge existing between any pair of vertices.  We denote the average degree $d \defeq pN$.  
\begin{definition} (Polymatrix Game)
A polymatrix game is specified by a graph $G = (V,E)$ over $N$ vertices denoted $V$ and edges $E$. Furthermore, a set of payoff games $\{p_{ij}\}_{(i,j) \in E}$ is defined on each edge of $G$.  Each payoff game $p_{ij}: [k]^2 \rightarrow \R$ takes as input a pair of actions $(a_i,a_j)$ each selected from a space of $k$ discrete actions and outputs a real valued payoff.  The payoff function to player $i$ is $p_i(a_1,a_2,...,a_N) = \sum_{j \in N(i)} p_{ij}(a_i,a_j)$.  Note that  $p_{ij}(a_i,a_j)$ is the payoff to node $i$ and $p_{ji}(a_j,a_i)$ is the payoff to node $j$.            
\end{definition}
To enable a fair comparison for $\epsilon$ approximate nash equilibrium, we will rescale the payoffs so that the payoff to each agent is in the interval $[0,1]$.  See appendix for rescaling.  Henceforth we will be using $\{f_{ij}\}_{(i,j) \in E}$ as the rescaled payoff functions. Next we define $L$-smooth payoff functions.    

\begin{definition} \label{def:smoothness} (L-Smoothness)
We say a polymatrix game is $L$-smooth for a constant $L > 0$ if for all nodes $i$ and $j \in [N]$, the payoff to node $i$ for the game $f_{ij}$ is upper bounded  
\[\max_{p,q \in [k]^2} |f_{ij}(p,q)| \leq \frac{L}{d}\]
\end{definition}
For example, if each payoff $p_{ij}(\cdot, \cdot) \in [0,1]$ and $\min_{\vec{a} \in [k]^N} p_i(a_1,...,a_N) = 0$ and $\max_{\vec{a} \in [k]^N} p_i(a_1,...,a_N) = d$  then the rescaled payoff $f_{ij}(a_i,a_j) = \frac{p_{ij}(a_i,a_j)}{d}$ is $L$-smooth for $L = 1$.  Next we define the mixed nash equilibrium.  

\begin{definition} (Mixed Nash Equilibrium)
Let $\Gamma \defeq \{\{\mu_{ip}\}_{p \in [k]}\}_{i \in [N]}$ define a probability distribution over $[k]$ for all agents $i \in [N]$ where $\mathbb{P}_\Gamma(a_i = p) = \mu_{ip}$.  A mixed Nash equilibrium is distribution over mixed strategies $\Gamma$ satisfying the following equilibrium condition.

$$\mathbb{E}_{\Gamma}[f_i(a_1,...,a_n)] \geq \E_{\Gamma}[f_i(a_1,...,a_{i-1},w,...,a_n)]$$  
For all $w \in [k]$ and for all $i \in [n]$.
\end{definition}

\begin{definition} (Approximate Mixed Nash Equilibrium)
  We define an $\epsilon$-approximate Nash equilibrium as a product distribution over mixed strategies $\Gamma \defeq \{\{\mu_{ip}\}_{p \in [k]}\}_{i \in [N]}$ satisfying
$$\mathbb{E}_{\Gamma}[f_i(a_1,...,a_n)] \geq \E_{\Gamma}[f_i(a_1,...,a_{i-1},w,...,a_n)] - \epsilon$$  
For all $w \in [k]$ and for all $i \in [n]$.  
\end{definition}
 
Without the smoothness assumption and for general graphs $G$, it is known that computing an $\epsilon$-approximate nash equilibrium is PPAD hard for a small constant $\epsilon$.  In fact that hard instance is a three regular graph.  This paints a bleak picture for efficient equilbrium computation in possibly the most amenable class of graphical games.  In this work we find that for polymatrix games with $L$-smooth payoffs defined on random graphs with density greater than $\frac{k^8}{\epsilon^4} \ln^3(N)$, there exists a $(Nk)^{O(\frac{L^4\ln(k)}{\epsilon^4})}$ time algorithm for finding an $\epsilon$-approximate nash equilibrium with high probability $1 - N^{-\ln^2(N)}$ over the randomness in the graph.  

\begin{restatable} [Polymatrix Nash Equilibrium]{thm}{maintheorem} 
\label{thm:main-theorem}
Let $(G, \{p_{ij}\}_{i,j \in [N]})$ be a polymatrix game defined on a $G$ with rescaled and normalized payoffs $\{f_{ij}\}_{i,j \in [N]}$ that are $L$-smooth.  If $G$ is a random graph with edge density $d = \frac{L^8k^8}{\epsilon^8} \ln^3(N)$.  Then there is a $(Nk)^{O(\frac{L^4\ln(k)}{\epsilon^4})}$ time deterministic algorithm for finding an $\epsilon$-approximate nash equilibrium with high probability $1 - N^{-\ln^2(N)}$ over the randomness in the graph.     
\end{restatable}

\section{Related Work}
There are a variety of works on approximating polymatrix nash equilibria.  Polymatrix zero sum games are polynomial time solvable \cite{CCDP16}.  Polymatrix nash equilibrium can be approximated to a large additive constant \cite{DFSS14}.  Work on max-2CSP and combinatorial optimization via sum of squares include \cite{BRS11} \cite{RT12} \cite{GS13}. Finally, there is work on linear programming relaxations for computing nash equilibrium \cite{WHN16}.       

\section{A Convex Hierarchy for Nash Equilibrium}

\paragraph{Idealized Program: } Consider the following polynomial system $\calP$ comprised of polynomial inequalities in indeterminates $\{a_{ir}\}_{i \in [N], r \in [k]}$ where $a_{ir}$ is the indicator of the $r$'th action for player $i$.  A pure nash equilibrium is then a solution to the following polynomial system 
\[\calP =  \begin{cases} 
      \sum_{j \in N(i)} \sum_{p,q \in [k]^2}f_{ij}(p,q)a_{ip}a_{jq} - \sum_{j \in N(i)} \sum_{q \in [k]}f_{ij}(w,q)a_{jq} \geq 0 & \forall w \in [k] \\
      a_{ip}^2 - a_{ip} = 0 &  \forall i\in [N], p \in [k]\\
      \sum_{p=1}^k a_{ip} - 1 = 0 & \forall i \in [N] 
   \end{cases}
\]
Here the second constraint ensures $a_{ip} \in \{0,1\}$, which we refer to as the "booleanity" constraint.  The third constraint ensures each agent can adopt only one strategy, which we refer to as the "coloring" constraint.  The first constraint is the pure nash equilibrium constraint.  A pure nash equilibrium is not guaranteed to exist so $\calP$ is not necessarily feasible.  However, a mixed nash equilibrium is guaranteed to exist.  Therefore we know there exists a distribution over $[k]^N$ such that      
\[\calQ =  \begin{cases} 
      \E_\zeta[\sum_{j \in N(i)} \sum_{p,q \in [k]^2}f_{ij}(p,q)a_{ip}a_{jq} - \sum_{j \in N(i)} \sum_{q \in [k]}f_{ij}(w,q)a_{jq}] \geq 0 & \forall w \in [k] \\
      \E_\zeta[a_{ip}^2 - a_{ip}] = 0 &  \forall i \in [N],\forall p \in [k]\\
      \E_\zeta[\sum_{p=1}^k a_{ip} - 1] = 0 & \forall i \in [N]\\ 
      \E_\zeta[a_{ip}a_{jq}] = \E_\zeta[a_{ip}]\E_\zeta[a_{jq}] & \forall i \neq j \in [N], \forall p,q \in [k]^2 
   \end{cases}
\]

Here, $\calQ$ is a set of constraints over a distribution $\zeta$ over elements of $[k]^N$.  The first constraint it the mixed nash equilibrium constraint.  The second and third are the booleanity and coloring constraints.  The final constraint ensures that $\zeta$ is a product distribution which gives us a mixed strategy nash equilibrium. Indeed, a solution $\zeta$ to constraints in $\calQ$ would constitute a mixed Nash equilibrium.  However, finding $\zeta$ is evidently intractable.  For starters we are searching over the space of distributions over $[k]^N$, and the constraints are desperately non-convex.  Fortunately, there is a polynomial optimization toolbox for satisfying constraints that take on the form of $\calQ$.  Although an overview of polynomial optimization is beyond the scope of this paper, we will introduce some of the fundamentals and discuss how the case of Nash equilibrium requires conceptual modifications of the standard dogma.      

\paragraph{The Relaxation Toolbox and the Problem of Feasibility: }

In combinatorial optimization, one is often faced with maximizing a polynomial subject to polynomial constraints e.g maximum cut.  This reduces to solving systems of polynomial inequalities and equalities.  However, solving a polynomial system for an "integral" solution is NP-hard.  Therefore, we must settle for solving a convex relaxation of the original nonconvex polynomial optimization.  The convex relaxation solves for a distribution, or rather a pseudo-distribution, over integral solutions to the original polynomial optimization.  After designing and solving the convex relaxation, a rounding algorithm takes as input the pseudo-distribution (a distribution over large cuts) and aims to produce an approximate solution to the original problem e.g finding a large cut.  The key problem with designing a convex relaxation for Nash equilibrium is that the integral solution, the pure nash equilibrium, is not guaranteed to exist. This is more than a technicality, as both the Sherali Adams and Sum of Squares hierarchies of $\calP$ are infeasible.  

Alternatively, one may be motivated to relax the mixed nash equilibrium as it's guaranteed to exist and is also the object we are searching for.  However, this introduces a layer of misdirection that is hard to conceptually reason about.  If we're searching for a distribution over $[k]^N$, then the relaxation is attempting to find a pseudo-distribution over distributions over $[k]^N$.  Such an approach is hard to reason about and it is not the one taken in this work.  Rather, we search for mixed nash equilibrium by identifying a simple fact about correlated equilibria.  


\paragraph{Conditioning Preserving Correlated Equilibrium}

A correlated equilibrium is a distribution $\zeta$ over $[k]^N$. An action vector $\vec{a} = (a_1,a_2,...,a_N) \in [k]^N$ is drawn from a distribution $\zeta$ such that each agent $i \in [N]$ plays action $a_i$. The distribution $\zeta$ satisfies that each agent $i$ has no incentive to deviate from its prescribed action $a_i \in [k]$ given that all other agents take actions drawn from $\zeta$ conditioned on $a_i$ denoted $\zeta|_{a_i}$.
The key insight is the following fact.   

In an $L$-smooth polymatrix game on a $G(N,p)$ random graph for $p \geq \frac{L^8k^8}{\epsilon^8} \ln^3(N)$, any correlated equilibrium that remains a correlated equilibrium after conditioning on all possible actions of all possible small subsets of agents is approximately a mixed nash equilibrium over the agents whose actions are not conditioned upon.

At a high level, a correlated equilibrium is not a Nash equilibrium because the distribution $\zeta$ over $[k]^N$ is not a product distribtution.  However, if we take a correlated equilibrium and condition $\zeta$ on the action of a random  agent $a_i$, then the total entropy of $\zeta|_{a_i}$ decreases which brings it closer to being a product distribution.  It stands to reason that conditioning on the actions of a few agents $S = \{a_{i_1}, a_{i_2},..., a_{i_y}\}$ can remove most of the correlations in $\zeta|_{S}$ thus bringing it to an approximate product distribution.  If $\zeta$ is a correlated equilibrium that continues to be a correlated equilibrium after conditioning on the actions of a small number of agents, then the resulting conditional distribution $\zeta|_{S}$ would be close to a mixed nash equilibrium.   One caveat is that the agents who's actions have been conditioned upon don't necessarily participate in the resulting Nash equilibrium so some small modifications will have to be made to the resulting equilibrium.  

Our goal is then to reformulate this high level idea into a convex program and design a rounding algorithm.  This is the subject of the next few sections, which we start with an overview of the fundamentals of polynomial optimization.  For a more thorough overview, see \cite{Nesterov00} \cite{Lasserre06} \cite{Shor87} \cite{Parrilo03}.  

\paragraph{Polynomial Optimization Fundamentals:}
First we define the pseudoexpectation functional $\pE[\cdot]$.  

\begin{definition}
Let $\bold{a} \defeq \{a_{ir}\}_{i \in [N], r \in [k]}$.  
A degree $\ell$ pseudoexpectation $\pE: \R[\bold{a}]^{\leq \ell} \rightarrow \R$ is a linear functional from the space of degree up to $\ell$ polynomials in variables $\bold{a}$ to the reals. Pseudoexpectations satisfy the following desirable properties. 
\begin{enumerate}
    \item Firstly, $\pE[1] = 1$ just like an expectation.
    \item Secondly, pseudoexpectations are linear functionals so that for any collection of degree less than $\ell$ polynomials $\{P_i(\bold{a})\}_{i \in [t]}$, we have  $\pE[\sum_{i=1}^t P_i(\bold{a})] = \sum_{i=1}^t \pE[P_i(\bold{a})]$. 
    \item Finally, for sum of squares pseudoexpectation functionals, we additionally require that $\pE[SoS_{\ell}(\bold{a})] \geq 0$ where $SoS_{\ell}(\bold{a})$ is the space of polynomials of degree less than $\ell$ that can be written as the sum of squares of polynomials.  
\end{enumerate}
\end{definition}

    

We are often searching for a pseudoexpectation functional satisfying certain constraints.  To distinguish between different pseudoexpectation functionals, we index $\pE_{\zeta}$ by an index $\zeta$.  Occasionally, we will refer to $\zeta$ as a pseudodistribution, but this is only for indexing purposes. Pseudoexpectations for degree $\ell$ polynomials over $N$ variables can be found in $N^{O(\ell)}$ time with some additional care required for bit precision issues.  This follows from the duality of the sum of squares proof system and semidefinite programming see \cite{Nesterov00} \cite{Lasserre06} \cite{Shor87} \cite{Parrilo03}.      

\begin{algorithm}
\caption{Convex Hierarchy for Polymatrix Nash Equilibrium}\label{alg:convex}
\textbf{Input: } Polymatrix game $(G,\{f_{ij}\}_{i,j \in [N]})$ \\
Let $\ell = \frac{L^4\ln(k)}{\epsilon^4}$, and search for a degree $\ell$ pseudoexpectation $\pE_\zeta[\cdot]$ satisfying the following  
\begin{enumerate}
    \item \textbf{Conditioning Preserving Correlated Equilibrium:} For all $i \in [N]$, for all $w \in [k]$, and for all $S_i \subset \{\{a_{jr}\}_{j \in [N]/i}\}_{r \in [k]}$ of size $|S_i| \leq \ell-2$.  
    \[\pE_\zeta[(\sum_{j \in N(i)} \sum_{p,q \in [k]^2}f_{ij}(p,q)a_{ip}a_{jq} - \sum_{j \in N(i)} \sum _{q \in [k]}f_{ij}(w,q)a_{jq} +\epsilon)\prod_{a_{uv} \in S_i}a_{uv}] \geq 0\] 
    \item \textbf{Booleanity:} $\forall i \in [N]$, $\forall p \in [k]$, and  for all $S \subset \{a_{ir}\}_{i \in [N], r \in [k]}$ satisfying $|S| \leq \ell - 2$ we have
    \[\pE_\zeta[(a_{ip}^2 - a_{ip})\prod_{a_{uv} \in S}a_{uv}] = 0\]
    \item \textbf{Coloring:} $\forall i \in [N]$ and  for all $S \subset \{a_{ir}\}_{i \in [N], r \in [k]}$ of size $|S| \leq \ell - 1$ we have  
    \[\pE_\zeta[(\sum_{p=1}^k a_{ip} - 1)\prod_{ a_{uv} \in S}a_{uv}] = 0\]
    \item \textbf{Normalization:} $$\pE_\zeta[1] = 1$$
    \item \textbf{Sum of Squares:} $\forall p(\bold{a}) \in SoS_{\ell}(\{a_{ir}\}_{i \in [N], r \in [k]})$ we constrain  
    \[\pE_\zeta[p(\bold{a})] \geq 0\] 
\end{enumerate}
\end{algorithm}

\paragraph{Formulating a Semidefinite Program: }
\pref{alg:convex} searches for a pseudoexpectation $\pE_\zeta$ satisfying the following constraints.  We have the normalization and sum of squares constraints, which are standard.  Furthermore we have the booleanity and coloring constraints which are also standard.  Collectively the booleanity and coloring constraints enforce that $a_{ip} \in \{0,1\}$ and that each player can adopt only one action at a time in expectation.  These constraints are also relaxed through the sherali adams hierarchy.  This implies that the constraints still hold for any set of fixings of the actions of up to $\ell-2$ players.  Finally, we have the conditioning preserving correlated equilibrium constraint.  For all $i \in [N]$, for all $w \in [k]$, and for all $S_i \subset \{\{a_{jr}\}_{j \in [N]/i}\}_{r \in [k]}$ of size $|S_i| \leq \ell-2$.  
    \[\pE_\zeta[(\sum_{j \in N(i)} \sum_{p,q \in [k]^2}f_{ij}(p,q)a_{ip}a_{jq} - \sum_{j \in N(i)} \sum _{q \in [k]}f_{ij}(w,q)a_{jq} +\epsilon)\prod_{a_{uv} \in S_i}a_{uv}] \geq 0\]           
This constraint ensures that the equilibrium (possibly correlated) we are searching for continues to be an equilibrium even if any small subset of $\ell-2$ players fix their actions.  Here the equilibrium condition is constrained to hold for every player who's actions are not fixed.  Ultimately, the conditioning preserving correlated equilibrium constraint is our replacement for both the Nash equilibrium constraint of $\calQ$ and the product distribution constraint   $\E_\zeta[a_{ip}a_{jq}] = \E_\zeta[a_{ip}]\E_\zeta[a_{jq}]$ of $\calQ$.  It turns out that the conditioning preserving correlated equilibria are close to mixed nash equilibria in a very precise sense which we'll exploit in the rounding algorithms detailed in \pref{sec:rounding}.          

\paragraph{Hybridized Hierarchies and Proof Systems: }
Note that we are neither working with the Sherali-Adams nor Sum of Squares relaxation.  We are also not working with a standard hybridized hierarchy because the Nash constraint for player $i$ only holds for multilinear polynomials that do not include $\{a_{ir}\}_{r \in [k]}$. As such, it will be important to design a rounding algorithm which is oblivious to the weakenings in the relaxation. 

Although it is possible that the sum of squares relaxation of pure nash equilibrium is feasible, it's not immediately obvious how to prove such a statement as the pure nash equilibrium is not guaranteed to exist.  We circumvent this problem, by explicitly designing the relaxation hierarchy for the polymatrix mixed nash equilibrium.  Note, that the relaxation is infeasible without the L-smooth payoffs assumption.

\paragraph{Explicit Constraints:} As we're not working with a standard relaxation, it's important to explicitly list the constraints of our convex program. These details are left for \pref{sec:explicit} 

\paragraph{Runtime: } The runtime of our algorithm is dominated by the convex program solve of \pref{alg:convex}.  The subsequent rounding analysis succeeds with small additive error in the constraints, which allows us to apply the ellipsoid algorithm \cite{GLS81}.  The subsequent rounding procedures require no more than $(Nk)^{O(\frac{L^4 \ln(k)}{\epsilon^4})}$ time upper bounded by an exhaustive search over the variable space. 

\begin{lemma} \label{lem:runtime}
\pref{alg:convex} can be solved via the ellipsoid algorithm \cite{GLS81} to $2^{-\poly(N,k)}$ additive accuracy in $(Nk)^{O(\frac{L^4 \ln(k)}{\epsilon^4})}$ time.      
\end{lemma}

\begin{proof}
\pref{alg:convex} is a convex program in $(Nk)^{O(\frac{L^4\ln(k)}{\epsilon^4})}$ variables and $(Nk)^{O(\frac{L^4\ln(k)}{\epsilon^4})}$ convex and explicitly bounded constraints with a feasible region of size greater than $2^{-\poly(N,k)}$.  Furthermore, both \pref{lem:rounding-main} and \pref{alg:bestrespond} hold for any $2^{-\poly(N,k)}$ additive error in the constraints. Therefore, the ellipsoid algorithm \cite{GLS81} produces a solution to \pref{alg:convex} in $(Nk)^{O(\frac{L^4 \ln(k)}{\epsilon^4})}$ time.         
\end{proof}

\section{Rounding Algorithms} \label{sec:rounding}
In this section we present our rounding algorithm.  Our rounding proceeds in two phases.  In the first phase we adopt the conditioning procedure of \cite{BRS11} to bring $\pE_{\zeta}$ close to a product distribution which we denote $\pE_{\hat{\zeta}}$.  Here, to achieve quasipolynomial running time we must improve the $k$ dependence of the \cite{BRS11} analysis. Towards these ends, we develop a novel polynomial approximation argument of the pseudo-covariance matrix, which accelerates the \cite{BRS11} rounding at the expense of an additional $\poly(k)$ in the edge density of the graph.  This argument is the key to our quasipolynomial time algorithm.   

In the second phase of our rounding, we must take the approximate product distribution $\pE_{\hat{\zeta}}$ and turn it into a mixed nash equilibrium.  This is done via the  BestResponsePursuit \pref{alg:brp}.  At a high level, \pref{alg:brp} takes all players which do not have the mixed nash equilibrium constraint satisfied, which we denote $K_1$, and attempts to set them to best respond to the mixed strategies of the players in $V/K_1$.  Of course this does not work because $K_1$ could be a clique.  This would only work, if the following condition is satisfied.  For all $i \in V$, the fraction of edges between $i$ and $K_1$ is small i.e $\frac{N_{K_1}(i)}{d} \leq \frac{\epsilon}{L}$.  If this is the case, then setting every player in $K_1$ to best respond to the mixed strategies of player in $V/K_1$ would result in an $2\epsilon$ approximate mixed nash equilibrium.  This is a consequence of the $L$-smoothness assumption.  

A priori, $K_1$ does not satisfy this condition.  However, given $K_1$ it is possible to show that there exists a set $W$ such that $K_1 \subset W$, $|W| \leq \frac{4}{3}|K_1|$, and for all $i \in V/W$, we have $\frac{N_W(i)}{d} \leq \frac{\epsilon}{L}$. Then we can hope to form a set $U \subseteq W$ such that $\frac{|U|}{|W|} \geq \frac{9}{10}$ and for all $i \in U$, $\frac{N_W(i)}{d} \leq \frac{\epsilon}{L}$.  This allows us to set the players in $U$ to best respond to the mixed strategies of players in $V/W$.  After this is done, we are left with a set of players $K_2 = W/U$, where $|K_2| \leq \frac{|K_1|}{5}$.  Now iterating the entire procedure on $K_2$ instead of $K_1$ we can hope to iteratively contract the set of players not participating in the mixed nash equilibrium until only a small set of players are left.  This small set of players can then all be set to best respond.

Each iteration of the above argument comes at a cost to the approximation of the mixed nash equilibrium so it is important to keep track of the sources of error as they add up over multiple iterations.  The details of this argument is the subject of \pref{sec:best-response-pursuit}. Next we state the guarantees of the conditioning rounding procedure.  To start we define a notion of distance quantifying how far a probability distribution is from a product distribution.  

\begin{definition}
Let $\Delta(i)$ be the local correlation at node $i \in [N]$ defined to be.   
\[\Delta(i) \defeq \frac{1}{|N(i)|}\sum_{j \in N(i)} \sum_{(p,q) \in [k]^2} |\pE[a_{ip} a_{jq}] - \pE[a_{ip}]\pE[a_{jq}]|]\]
Let $\Delta$ be the average local correlation of the nodes in $G$.  
\[\Delta \defeq \frac{1}{2|E|}\sum_{(i,j) \in E} \sum_{(p,q) \in [k]^2} |\pE[a_{ip} a_{jq}] - \pE[a_{ip}]\pE[a_{jq}]|]\]
\end{definition}

Intuitively, correlation/covariance capture how far a pair of random variables $a_i$ and $a_j$ are from a product distribution.  The average covariance over all pairs of random variables over all edges of $G$ is then a measure of how far a distribution in $\calP(k,V)$ is from a product distribution.

\begin{fact} [Conditioning] For a degree $\ell$ pseudoexpectation $\pE_{\zeta}$ over the polynomial system $\{x_i^2 = x_i\}_{i=1}^N$, the following conditioning procedure is well defined.  

\[\pE_{\zeta}[p(x_1,...,x_N) | x_i = 1] \defeq \frac{\pE_{\zeta}[p(x_1,...,x_N)x_i]}{\pE_{\zeta}[x_i]}\]
for any $p(x_1,...,x_N) \in \R[x_1,...,x_N]^{\leq \ell}$
\end{fact}

Note that the conditioning preserving correlated equilibrium constraint for player $i$ is preserved after conditioning on the actions of any player $j \neq i$.  Furthermore, this holds for any set of $\ell-1$ such conditionings.  Next, we will prove our conditioning rounding result for the class of low threshold rank graphs, which are a generalization of expander graphs and include the $G(N,p)$ graphs we study in this work.    

\begin{definition}
We say a graph $G$ has $(\rho, \tau)$ threshold rank if 
the normalized adjacency matrix $\frac{1}{d}A$ of $G$ with eigenvalues $\{\lambda_i\}_{i=1}^N$ satisfies $|\{i \in [N]| \lambda_i \geq \rho\}| = \tau$. 
\end{definition}
In particular, the $G(N,p)$ random graphs considered in this work have $(\frac{1}{\sqrt{d}},1)$ threshold rank. The following lemma is the main rounding guarantee of the conditioning procedure.    
\begin{lemma} \label{lem:rounding-main}
Consider the following procedure, 
\begin{enumerate}
    \item Initialize seed set $S = \emptyset$, and assignments $\cal{R} = \emptyset$
    \item Let $a_1,a_2,...,a_N$ be random variables such that $\mathbb{P}(a_i = p) = \pE_\zeta[a_{ip}]$ for all $p \in [k]$ and for all $i \in [N]$.
    \item Let $i \in V/S$ be drawn uniformly at random, and append $i$ to $S$
    \item draw $p_i \sim a_i$ and append $p_i$ to $\cal{R}$
    \item Update $\pE_\zeta[\cdot] \gets \pE_\zeta[\cdot|a_i = p_i]$
    \item While $|S| \leq \frac{L^4\ln(k)}{\epsilon^4}-2$ go to step (2.)
\end{enumerate}
Let $\pE_{\hat{\zeta}}$ be the pseudoexpectation that is the output of the algorithm above, and let $\Delta_{\hat{\zeta}}$ be its associated local correlation.  When the algorithm terminates we have $S \defeq \{i_1,i_2,...,i_{r}\}$ and $R \defeq \{p_{i_1}, p_{i_2}, ..., p_{i_r}\}$. Let $\cal{U}$ be the distribution over $\{(i_y,p_y)\}_{y \in [r]}$ induced by the algorithm.  Then 
$$\E_{\{(i_y,p_y)\}_{y \in [r].  } \sim \cal{U}}[\Delta_{\hat{\zeta}}|a_{i_1} = p_{i_1}, a_{i_2} = p_{i_2} ,...,a_{i_r} = p_{i_r}] \leq \frac{\epsilon^2}{L^2}$$ 
\end{lemma}

The conditioning algorithm selects a player at random and fixes/conditions their actions according to the local probability distribution prescribed by $\pE_{\zeta}$.  Iterating this algorithm reduces the local correlation.  The proof of \pref{lem:rounding-main} is left for \pref{sec:mutual-information}.  Additionally, it turns out there's no need for the procedure to be executed iteratively.  One could simply enumerate all possible conditionings of the actions of $\ell -2$ agents.  One such conditioning is guaranteed to bring the local correlation down to $\epsilon$.  Next, we move onto the guarantees of \pref{alg:brp}.  

\begin{restatable}[Best Response Pursuit]{lem}{bestresponsepursuit}
\label{lem:best-response-pursuit}
Let $\pE_{\zeta}$ be a pseudoexpectation satisfying the constraints in \pref{alg:convex}. Let $\Delta_\zeta \leq \epsilon^2$ be the local correlation of $\pE_{\zeta}$.  Let $K_1 \subset V$ be the set of nodes satisfying $\Delta_\zeta(i) \geq 2\epsilon$. Let $\mu_{ip} = \pE[a_{ip}] \text{   for all }  i \in [N] \text{ and for all } p \in [k]$.  Let $\Gamma_1 \gets \{\{\mu_{ip}\}_{p=1}^k\}_{i=1}^N$.  
We define $K_2,K_3,...,K_T$ and $\Gamma_2,\Gamma_3,...,\Gamma_T$ iteratively as follows. Let $W = Flag(K_t,G)$ and let $(K_{t+1}, \Gamma_{t+1}) = BestRespond(W,\Gamma_t)$.  Here we iterate until $K_T = \emptyset$.  Then we have for all $i \in V$ with probability $1 - N^{-\ln^2(N)}$ over the randomness of the graph,    
\[\sum_{j \in N(i)} \sum_{(p,q) \in [k]^2} f_{ij}(p,q) \pE_{\Gamma_T}[a_{ip}]\pE_{\Gamma_T}[a_{jq}] - \sum_{q \in [k]} f_{ij}(w,q) \pE_{\Gamma_T}[a_{jq}]  \leq 120L\epsilon\] 
\end{restatable} 
Given \pref{lem:best-response-pursuit} we can conclude the following guarantee holds for our rounding algorithm.  
\begin{lemma} [Rounding] \label{lem:best-response-main}
Let $\pE_{\zeta}$ be the pseudoexpectation output by \pref{alg:convex}, which is then passed to the conditioning procedure in \pref{lem:rounding-main}.  Let $\pE_{\hat{\zeta}}$ be the pseudoexpectation output by the conditioning procedure in \pref{lem:rounding-main}. Then \pref{alg:brp}, BestResponsePursuit($\pE_{\hat{\zeta}}$) outputs a product distribution $\Gamma \defeq \{\mu_{ip}\}_{i \in [N], p \in [k]} \in \calP(k,V)$ satisfying that for all $i \in [N]$ and for all $w \in [k]$

\[\sum_{j \in N(i)} \sum_{p,q \in [k]^2}f_{ij}(p,q)\E_\Gamma[a_{ip}]\E_\Gamma[a_{jq}] - \sum_{j \in N(i)} \sum_{q \in [k]}f_{ij}(w,q)\E_\Gamma[a_{jq}] \geq -\epsilon\]
That is to say $\Gamma$ is an $\epsilon$-approximate mixed nash equilibrium
\end{lemma}

\begin{proof}
Using \pref{lem:rounding-main} we obtain $\pE_{\hat{\zeta}}$ satisfying local correlation $\Delta \leq \frac{\epsilon^2}{L^2}$.  Then using \pref{lem:best-response-pursuit} we obtain a product distribution $\Gamma$ satisfying  

\[\sum_{j \in N(i)} \sum_{p,q \in [k]^2}f_{ij}(p,q)\E_\Gamma[a_{ip}]\E_\Gamma[a_{jq}] - \sum_{j \in N(i)} \sum_{q \in [k]}f_{ij}(w,q)\E_\Gamma[a_{jq}] \geq -\epsilon\]
as desired.  
\end{proof}
We now have all the ingredients to prove \pref{thm:main-theorem}
\maintheorem*

\begin{proof}
We have by \pref{lem:runtime} that \pref{alg:convex} can be solved in $(Nk)^{O(\frac{L^4 \ln(k)}{\epsilon^4})}$ time.  We have by \pref{lem:best-response-main} that the conditioning procedure and \pref{alg:brp} produce an $\epsilon$-approximate Nash equilibrium with high probability $1 - N^{-\ln^2(N)}$.  The runtime of both procedures is upper bounded by the time it takes to exhaustively search over all $\ell-2$ size subsets of conditionings of $\bold{a}$.  This can be done in time $(Nk)^{O(\frac{L^4 \ln(k)}{\epsilon^4})}$.  
\end{proof}

\begin{algorithm}
\caption{Best Response Pursuit}\label{alg:brp}
\textbf{Input: } $\pE_{\zeta}$\;
\KwResult{$\Gamma = \{\{\mu_{ip}\}_{p=1}^k\}_{i=1}^N$}
$\mu_{ip} \gets \pE[a_{ip}] \text{   for all }  i \in [N] \text{ and for all } p \in [k]$\;
$K_1 \gets \emptyset$\;
\For{$i \in [N]$}
{
\If{
$\Delta(i) \geq 2\epsilon$}{$K_1 \gets K_1 \cup \{i\}$}
}
$\Gamma_1 \gets \{\{\mu_{ip}\}_{p=1}^k\}_{i \in V/K_1}$ \;
\For{$t \in [T]$}{
$W = Flag(K_t,G)$\;
$(\Gamma_{t+1}, K_{t+1}) \gets BestRespond(W, \Gamma_t)$\;
$\Gamma \gets \Gamma_{t+1}$\;
\If{$K_{t+1} = \emptyset$}{ \textbf{break}} 
}

$\textbf{Return: } \Gamma$
\end{algorithm}

\begin{algorithm}
\caption{Flag}\label{alg:flag}
\KwData{$K \subseteq V$}
\KwResult{$W \subseteq V$}
$W \gets K$\;
$H_{old} \gets K$\;
\While{$|H_{old}| \geq \ln^3(N)$}{
$H_{new} \gets \emptyset$\;
\For{$i \in \overline{W}$}{
\If{$\frac{|N_{H_{old}}(i)|}{d} \geq \max(10 \frac{|H_{old}|}{N}, 10\frac{\ln(n)}{d})$}{
$H_{new} \gets H_{new} \cup \{i\}$\;
}
}
$W \gets W \cup H_{new}$\;
$H_{old} \gets H_{new}$
}

$\textbf{Return: } W$

\end{algorithm}

\begin{algorithm}
\caption{BestResponse($\Gamma,W$)}\label{alg:bestrespond}
\textbf{Input: }$\Gamma = \{\{\mu_{ip}\}_{p=1}^k\}_{i \in V/W}, W$\;
\KwResult{$\hat{\Gamma} = \{\{\hat{\mu}_{ip}\}_{p=1}^k\}_{i \in V/Q}, Q$}
$\{\hat{\mu}_{ip}\}_{p=1}^k\}_{i \in V/W} \gets  \{\{\mu_{ip}\}_{p=1}^k\}_{i \in V/W}$ \;
\If{$|W| \leq \ln^3(N)$}{
$Q \gets \emptyset$
} \Else{
$Q \gets \emptyset$\;
\For{$i \in W$}{
\If{$\frac{|N_{W}(i)|}{|N(i)|} \geq \epsilon$}{
$Q \gets \{i\}$\;
} 
}
}

\For{$i \in W/Q$}{
\For{$p \in [k]$}{
\If{$\E_{\Gamma}[\sum_{j \in N_{\overline{W}}(i)}\sum_{q\in [k]} f_{i,j}(p,q)a_{jq}] \geq \max_{p' \in [k]} \E[\sum_{j \in N_{\overline{W}}(i)}\sum_{q\in [k]} f_{i,j}(p',q)a_{jq}]$}{
$r_{ip} \gets 1$\;
} \Else{
$r_{ip} \gets 0$
}
}
}
$\hat{\mu}_{ip} \gets \frac{r_{ip}}{\sum_{p \in [k]}r_{ip}} $ \text{ for all } $i \in W/Q$ \text{ and for all } $p \in [k]$\\
$\textbf{return: } (\{\{\hat{\mu}_{ip}\}_{p=1}^k\}_{i \in V/Q},Q)$\;
\end{algorithm}

\section{Propagation of Mutual Information}\label{sec:mutual-information}
In the next two sections we work towards a proof of \pref{lem:rounding-main}.  First we present an upper bound on the local correlation $\Delta_{\zeta}$ in terms of the global mutual information defined to be $\frac{1}{N^2}\sum_{i,j \in [N]^2}I(a_i,a_j)$.  Here $\{a_i\}_{i \in [N]}$ are the $\ell$-local random variables over the agent's actions provided by the level $\ell$ sherali-adams hierarchy on the booleanity constraints.  The pairwise mutual information $I(a_i,a_j)$ is therefore well defined.  Once we relate $\Delta_{\zeta}$ to  $\frac{1}{N^2}\sum_{i,j \in [N]^2}I(a_i,a_j)$ we can use a standard conditioning argument to show that after a small number of rounds of conditioning $\frac{1}{N^2}\sum_{i,j \in [N]^2}I(a_i,a_j)$ is small.  This would in turn show that $\Delta_{\zeta}$ is small.   

\begin{lemma} \label{lem:loc_to_glob}
Let $\Delta_{\zeta}$ denote the local correlation of pseudodistribution $\pE_\zeta$ on a $(\rho,\tau)$ threshold rank graph $G$ defined as
\[\Delta_{\zeta} \defeq \frac{1}{2|E|}\sum_{(i,j) \in E} \sum_{(p,q) \in [k]^2} |\pE[a_{ip} a_{jq}] - \pE[a_{ip}]\pE[a_{jq}]|]\]
Let $f(x) = \sum_{r=1}^p \alpha_r x^r$ be a univariate polynomial satisfying $|x| \leq f(x) \leq |x| + \epsilon'$ for $x \in [-1,1]$.  Let $H \in \R^{kN \times kN}$ be a matrix row indexed by $i \in [N]$ and $p \in [k]$ and column indexed by $j \in [N]$ and $q \in [k]$.  We define $H_{\{(i,p),(j,q)\}} = f(\pE[a_{ip}a_{jq}] - \pE[a_{ip}]\pE[a_{jq}])$.  
Then we have that 
\[\Delta_{\zeta}  \leq
\frac{\rho k}{N} \|H\|_* + \tau \sqrt{2(\epsilon'k^2)^2 + \frac{2}{N^2}\sum_{i,j \in [N]^2}I(a_i,a_j)} \]
\end{lemma}


\begin{proof}
We begin with the local correlation 
\[\Delta_\zeta \defeq \frac{1}{2|E|}\sum_{(i,j) \in E} \sum_{(p,q) \in [k]^2} |\pE[a_{ip}a_{jq}] - \pE[a_{ip}]\pE[a_{jq}]|] = \frac{1}{|E|}\langle A, K\rangle  \]
where $K_{ij} := \sum_{(p,q) \in [k]^2} |\pE[a_{ip} a_{jq}] - \pE[a_{ip}]\pE[a_{jq}]|$.  Let $M$ be a matrix such that $M_{ij} = \sum_{(p,q) \in [k]^2}f(\pE[a_{ip}a_{jq}] - \pE[a_{ip}]\pE[a_{jq}])$ for univariate polynomial $f$ satisfying $|x| \leq f(x) \leq |x| + \epsilon'$.  Here $f(x)$ is the polynomial defined in \pref{lem:jackson}.  We use the univariate polynomial entrywise upper bound to obtain 
\[\frac{1}{|E|}\langle A, K \rangle  \leq \frac{1}{|E|}\langle A, M\rangle\]
Now we split the eigenvalues of $G$ into those larger than $\rho$ and those less than $\rho$.  

\[ \frac{1}{|E|}\langle A, M \rangle  = \frac{1}{N}\langle \frac{N}{|E|}A, M \rangle = \frac{1}{N}\sum_{i=1}^N \lambda_i \langle v_iv_i^T, M \rangle = \frac{1}{N}(\sum_{|\lambda_i| \leq \rho} \lambda_i \langle v_iv_i^T, M \rangle + \sum_{|\lambda_i| \geq \rho} \lambda_i \langle v_iv_i^T, M \rangle) \]
Taking absolute values in the second term we upper bound by 
\[ \leq \frac{1}{N}\Big(\sum_{|\lambda_i| \leq \rho} \lambda_i \langle v_iv_i^T, M \rangle + \sum_{|\lambda_i| \geq \rho} |\lambda_i| |\langle v_iv_i^T, M \rangle|\Big) \]
Using the fact that the eigenvalues of the normalized adjacency matrix is upper bounded by $1$ and taking cauchy schwarz we obtain 
\[ \leq \frac{1}{N}\Big(\sum_{|\lambda_i| \leq \rho} \lambda_i \langle v_iv_i^T, M \rangle + \sum_{|\lambda_i| \geq \rho} (\sum_{(r,s) \in [N]^2} v_{ir}^2v_{is}^2)^{1/2} \| M\|_F\Big) \]

\[ = \frac{1}{N}\Big(\sum_{|\lambda_i| \leq \rho} \lambda_i \langle v_iv_i^T, M \rangle + \sum_{|\lambda_i| \geq \rho} (\sum_{r=1}^{N} v_{ir}^2) \| M\|_F\Big) \]

Using the fact that $\|v_i\|^2 = 1$ we obtain 
\[ = \frac{1}{N}\Big(\sum_{|\lambda_i| \leq \rho} \lambda_i \langle v_iv_i^T, M \rangle + \sum_{|\lambda_i| \geq \rho} \| M\|_F\Big) \]
Using the $(\rho,\tau)$ threshold rank condition we obtain.  
\[ = \frac{1}{N}\Big(\sum_{|\lambda_i| \leq \rho} \lambda_i \langle v_iv_i^T, M \rangle + \tau \sqrt{\sum_{i,j \in [N]^2}(M_{ij})^2}\Big) \]
Let $v^{\oplus k} = (v,v,...,v) \in \R^k$.  Let $\overline{v}_i := (v_{i1}^{\oplus k}, v_{i2}^{\oplus k}, ... , v_{iN}^{\oplus k}) \in \R^{kN}$.  Let $H \in \R^{kN \times kN}$ be a matrix row indexed by $i \in [N]$ and $p \in [k]$ and column indexed by $j \in [N]$ and $q \in [k]$.  We define $H_{\{(i,p),(j,q)\}} = f(\pE[a_{ip}a_{jq}] - \pE[a_{ip}]\pE[a_{jq}])$.  Let $\phi_1,...,\phi_{kN}$ be the eigenvalues of $H$ with corresponding eigenvectors $q_1,...,q_{kN}$.  Then we have 

\[=  \frac{1}{N}\Big(\sum_{|\lambda_i| \leq \rho} \lambda_i \langle \overline{v}_i\overline{v}_i^T, H \rangle + \tau \sqrt{\sum_{i,j \in [N]^2}(M_{ij})^2}\Big) \]



\[=  \frac{1}{N}\Big(\sum_{|\lambda_i| \leq \rho} \lambda_i \langle \overline{v}_i\overline{v}_i^T, \sum_{j=1}^{kN} \phi_j q_jq_j^T \rangle + \tau \sqrt{\sum_{i,j \in [N]^2}(M_{ij})^2}\Big) \]

\[=  \frac{1}{N}\Big(\sum_{j=1}^{kN} \phi_j \sum_{|\lambda_i| \leq \rho} \lambda_i \langle \overline{v}_i\overline{v}_i^T, q_jq_j^T \rangle + \tau \sqrt{\sum_{i,j \in [N]^2}(M_{ij})^2}\Big) \]
Applying absolute values and noting that $\langle \overline{v}_i\overline{v}_i^T, q_jq_j^T \rangle$ is a square we upper bound by 
\[\leq  \frac{1}{N}\Big(\sum_{j=1}^{kN} |\phi_j| \sum_{|\lambda_i| \leq \rho} |\lambda_i| \langle \overline{v}_i\overline{v}_i^T, q_jq_j^T \rangle + \tau \sqrt{\sum_{i,j \in [N]^2}(M_{ij})^2}\Big) \]
Upper bounding $\lambda_i \leq \rho$ we obtain 
\[\leq  \frac{1}{N}\Big(\sum_{j=1}^{kN} |\phi_j| \sum_{|\lambda_i| \leq \rho} \rho k \langle \frac{1}{\sqrt{k}}\overline{v}_i \frac{1}{\sqrt{k}}\overline{v}_i^T, q_jq_j^T \rangle + \tau \sqrt{\sum_{i,j \in [N]^2}(M_{ij})^2}\Big) \]

Using the fact that $\langle \frac{1}{\sqrt{k}}\overline{v}_i \frac{1}{\sqrt{k}}\overline{v}_i^T, q_jq_j^T \rangle$ is a square for all $i \in [N]$ we upper bound by 

\[\leq  \frac{1}{N}\sum_{j=1}^{kN} |\phi_j|  \rho k\sum_{i \in [N]} \langle \frac{1}{\sqrt{k}}\overline{v}_i \frac{1}{\sqrt{k}}\overline{v}_i^T, q_jq_j^T \rangle + \frac{\tau}{N} \sqrt{\sum_{i,j \in [N]^2}(M_{ij})^2} \]

\[= \frac{1}{N}\sum_{j=1}^{kN} |\phi_j|  \rho k\sum_{i \in [N]} \langle \frac{1}{\sqrt{k}} \overline{v_i},q_j\rangle^2 + \frac{\tau}{N} \sqrt{\sum_{i,j \in [N]^2}(M_{ij})^2} \]

We know that $\frac{1}{\sqrt{k}}\overline{v}_i$ for $i \in [N]$ forms an orthonormal basis, and $\|q_j\|^2 = 1$.  Therefore, we upper bound the first term by   

\[\leq \frac{1}{N}\sum_{j=1}^{kN} |\phi_j|  \rho k + \frac{\tau}{N} \sqrt{\sum_{i,j \in [N]^2}(M_{ij})^2} \]
 
 
\[=  \frac{1}{N}\sum_{j=1}^{kN} |\phi_j|\rho k + \frac{\tau}{N} \sqrt{\sum_{i,j \in [N]^2}(M_{ij} - \sum_{(p,q) \in [k]^2}|cov(a_{ip}, a_{jq})| + \sum_{(p,q) \in [k]^2}|cov(a_{ip}, a_{jq})|)^2}\]

Applying $(a+b)^2 \leq 2a^2 + 2b^2$ to the second term we obtain

\[\leq  \frac{1}{N}\sum_{j=1}^{kN} |\phi_j|\rho k + \tau \sqrt{\frac{1}{N^2}\sum_{i,j \in [N]^2}\big(2(M_{ij} - \sum_{(p,q) \in [k]^2}|cov(a_{ip}, a_{jq})|)^2 + 2(\sum_{(p,q) \in [k]^2}|cov(a_{ip}, a_{jq})|)^2}) \]

Using the fact that $M$ is entrywise $\epsilon'k^2$ close to $K$ we have  

\[\leq  \frac{\rho k}{N} \|H\|_* + \tau \sqrt{2(\epsilon'k^2)^2 + \frac{2}{N^2}\sum_{i,j \in [N]^2}(\sum_{(a,b) \in [k]^2}|cov(a_{ip}, a_{jq})|)^2} \]
Applying pinsker's inequality we obtain 
\[\leq  \frac{\rho k}{N} \|H\|_* + \tau \sqrt{2(\epsilon'k^2)^2 + \frac{4}{N^2}\sum_{i,j \in [N]^2}I(a_i,a_j)} \]
which is the desired inequality.  
\end{proof}

Now we have upper bounded $\Delta_{\zeta}$ by a relation involving $\frac{4}{N^2}\sum_{i,j \in [N]^2}I(a_i,a_j)$ and $\|H\|_*$.  In the next section we bound $\|H\|_*$.  
\section{Accelerated Rounding via Polynomial Approximation}
The following lemma is a self contained fact about the nuclear norm of matrices that are written as the sum of PSD matrices.    

\begin{lemma} \label{lem:nuclear_norm}
Let $f(x) = \sum_{r=1}^p \alpha_r x^r$.  Let $H \in \R^{kN \times kN}$ be a matrix row indexed by $i \in [N]$ and $p \in [k]$ and column indexed by $j \in [N]$ and $q \in [k]$.  We define $H_{\{(i,p),(j,q)\}} = f(\pE[a_{ip}a_{jq}] - \pE[a_{ip}]\pE[a_{jq}])$.  Then 
\[\|H\|_* \leq Nk\sum_{r=1}^p |\alpha_r|\]
In particular, for $f(x)$ defined in \pref{lem:jackson} where $|x| \leq f(x) \leq |x| + \epsilon'$
\[\|H\|_* \leq \frac{18Nk}{\sqrt{\pi}\epsilon'}\]
\end{lemma}

\begin{proof}
Let $Q_{\{(ip),(jq)\}} = cov(a_{ip},a_{jq}) = \langle v_{ip}, v_{jq} \rangle$.  Let $Q^{\odot r}$ be the entrywise $r$'th power of $Q$.  We know 

\[Q^{\odot r}_{\{(ip),(jq)\}} \defeq cov(a_{ip},a_{jq})^r = \langle v_{ip}, v_{jq} \rangle^r =  \langle v_{ip}^{\otimes r}, v_{jq}^{\otimes r}\rangle\]

Thus we have $Q^{\odot r}$ is positive semidefinite, which can also be seen by applying the Schur product theorem.  Thus we can upper bound the nuclear norm 
\[\|Q^{\odot r}\|_* = Tr(Q^{\odot r}) =  \sum_{i \in [N], p \in [k]} \|v_{ip}\|^{2r} \leq Nk\] 

where the inequality follows from the fact that $\langle v_{ip}, v_{jq} \rangle \leq 1$. This implies we can upper bound the nuclear norm of $H$ as follows 
 \[\| H \|_* = \| \sum_{r=1}^p \alpha_r Q^{\odot r} \|_* \leq \sum_{r=1}^p \|\alpha_r Q^{\odot r} \|_* = \sum_{r=1}^p |\alpha_r|\| Q^{\odot r} \|_* \leq Nk\sum_{r=1}^p |\alpha_r|\]
 Where the first inequality follows from triangle inequality, and the second inequality follows from the nonnegativity of the nuclear norm. Using \pref{lem:jackson} we upper bound by $\sum_{r=1}^p |\alpha_r| \leq \frac{18}{\sqrt{\pi}}p \leq \frac{18}{\sqrt{\pi}\epsilon'}$ which allows us to conclude
\[\| H \|_* \leq \frac{18Nk}{\sqrt{\pi}\epsilon'} \] 
as desired.  
\end{proof}

Finally we are ready to prove \pref{lem:rounding-main}
\begin{lemma} \label{lem:rounding-main}
Let $a_1,a_2,...,a_N$ be random variables such that $\mathbb{P}(a_i = p) = \pE[a_{ip}]$ for all $p \in [k]$ and for all $i \in [N]$.  Consider the following procedure, 
\begin{enumerate}
    \item Initialize seed set $S = \emptyset$, and assignments $\cal{R} = \emptyset$
    \item Let $i \in V/S$ be drawn uniformly at random, and append $i$ to $S$ 
    \item draw $p_i \sim a_i$ and append $p_i$ to $\cal{R}$
    \item While $|S| \leq \frac{\ln(k)}{L^4\epsilon^4}$ go to step (2.)
\end{enumerate}
When the procedure terminates we have $S \defeq \{i_1,i_2,...,i_{r}\}$ and $R \defeq \{p_{i_1}, p_{i_2}, ..., p_{i_r}\}$. Let $\cal{U}$ be the distribution over $\{(i_y,p_y)\}_{y \in [r]}$ induced by the procedure.  Then 
$$\E_{\{(i_y,p_y)\}_{y \in [r].  } \sim \cal{U}}[\Delta|a_{i_1} = p_{i_1}, a_{i_2} = p_{i_2} ,...,a_{i_r} = p_{i_r}] \leq \frac{\epsilon^2}{L^2}$$  

\end{lemma}
\begin{proof}
Let $\epsilon' = \frac{\epsilon^2}{k^2}$.  Then our bound on $\Delta$ is as follows. By \pref{lem:loc_to_glob} we have 
\[\Delta \leq  \frac{\rho k}{N} \|H\|_* + \tau \sqrt{2(\epsilon'k^2)^2 + \frac{2}{N^2}\sum_{i,j \in [N]^2}I(a_i,a_j)} \]
For $G(N,p)$ random graphs, with $d = pN$ we have $\rho = \frac{1}{\sqrt{d}}$ and $\tau = 1$ so we obtain 
\[=  \frac{ k}{N\sqrt{d}} \|H\|_* + \sqrt{2(\epsilon'k^2)^2 + \frac{2}{N^2}\sum_{i,j \in [N]^2}I(a_i,a_j)} \]
Setting $\epsilon' = \frac{\epsilon}{k^2}$ we obtain 
\[=  \frac{k}{\sqrt{d}N} \|H\|_* + \sqrt{2\epsilon^2 + \frac{2}{N^2}\sum_{i,j \in [N]^2}I(a_i,a_j)} \]
Applying \pref{lem:nuclear_norm} we have  $\|H\|_* \leq \frac{Nk}{\epsilon'}= \frac{Nk^3}{\epsilon}$ which yields 
\[\leq  \frac{ k}{N\sqrt{d}} N \frac{k^3}{\epsilon} + \sqrt{2\epsilon^4 + \frac{2}{N^2}\sum_{i,j \in [N]^2}I(a_i,a_j)} \]
Setting $d = \frac{k^8}{\epsilon^4}$ we obtain 
\[\leq  \epsilon + \sqrt{2\epsilon^2 +  \frac{2}{N^2}\sum_{i,j \in [N]^2}I(a_i,a_j)}\]

At this point we have established 
\begin{multline}
\E_{\{(i_y,p_y)\}_{y \in [r]  } \sim \cal{U}}[\Delta|a_{i_1} = p_{i_1}, a_{i_2} = p_{i_2} ,...,a_{i_r} = p_{i_r}]\\ \leq  \epsilon + \E_{\{(i_y,p_y)\}_{y \in [r]  } \sim \cal{U}}\left[\sqrt{2\epsilon^2 +  \frac{2}{N^2}\sum_{i,j \in [N]^2}I(a_i,a_j|a_{i_1}=p_{i_1}, a_{i_2}=p_{i_2}, ..., a_{i_r}=p_{i_r})}\right]    
\end{multline}

We have by Jensen's inequality 
\[\leq  \epsilon + \sqrt{2\epsilon^2 + \E_{\{(i_y,p_y)\}_{y \in [r]  } \sim \cal{U}}\left[ \frac{2}{N^2}\sum_{i,j \in [N]^2}I(a_i,a_j|a_{i_1}=p_{i_1}, a_{i_2}=p_{i_2}, ..., a_{i_r}=p_{i_r})\right]} \]


Finally via \pref{lem:global_correlation} we have $\frac{1}{N^2}\sum_{i,j \in [N]^2}I(a_i,a_j|a_{i_1}, a_{i_2}, ..., a_{i_r}) \leq \epsilon^2$ after $\frac{\ln(k)}{\epsilon^2}$ rounds of conditioning.  
Thus we conclude

$$\E_{\{(i_y,p_y)\}_{y \in [r]} \sim \cal{U}}[\Delta|a_{i_1} = p_{i_1}, a_{i_2} = p_{i_2} ,...,a_{i_r} = p_{i_r}] \leq \epsilon$$
as desired. If we used $\frac{L^4\ln(k)}{\epsilon^4}$  
\end{proof}

In the next section we present the guarantees of the Best Response Pursuit \pref{alg:brp}.  
\section{Best Response Pursuit}  \label{sec:best-response-pursuit}
When local correlation is smaller than $\epsilon^2$,  this implies via markov that the number of nodes $i$ with $\Delta(i) \geq \epsilon$ is less than $ \epsilon N$.  Let these 'bad' nodes be $K_1$, and let $\overline{K_1}$ be the set of 'good' nodes.  A naive strategy is to set all the nodes in $K_1$ to best respond to the nodes in $\overline{K}_1$.  This would succeed if all the nodes in $V$ had less than a $10\epsilon$ fraction of edges incident to $K_1$.  Setting all the nodes in $K_1$ to best respond would satisfy the Nash equilibrium constraints to error $10L\epsilon$.  Thus, a preliminary goal is to assemble a set $K$ such that $K_1 \subseteq W$ satisfying that every node in $\overline{W}$ satisfies that no more than a $10\epsilon$ fraction of edges are incident to $W$. Given such a set $W$, we could set all the nodes in $W$ to best response and the nodes $i \in \overline{W}$ will  continue to satisfy $\Phi(i) \leq \epsilon + 10L\epsilon$.  Unfortunately, this does not guarantee that all the nodes $i \in W$ will satisfy $\Phi(i) \leq 10L\epsilon$ as their may be nodes in $W$ with too many edges in $W$.  Indeed, we will not be able to set all the nodes in $W$ to best response.  However, we can set most of the nodes in $W$ to best respond. This leaves us with a set $K_2 \subset W$ which has not been set to best response with $|K_2| \leq \frac{|K_1|}{2}$.  Now we can iterate the same procedure on $K_2$ as had been performed on $K_1$ which contracts the set of bad nodes geometrically whilst controlling the error in the constraints of the good nodes.  

\subsection{Best response preliminaries: }
\paragraph{Notation:} 
let $Q \subseteq V$ and let $\overline{Q}$ be the complement of $Q$ in $V$. Let $G_{Q} \defeq (Q, E_{Q})$ be a subgraph of $G$ where $E_{Q} \defeq \{(i,j) \in E| (i,j) \in Q\}$. Let $N_Q(i) \defeq \{j \in Q| (i,j) \in Q\}$.   
Let $\Gamma$ denote the product distribution that sets $\mathbb{P}_{\Gamma}(a_{ip}) = \pE[a_{ip}]$.  Let $\Gamma_Q$ be the product distribution defined over the nodes in $Q$ such that $\mathbb{P}_{\Gamma_Q}(a_{ip}) = \mathbb{P}_{\Gamma}(a_{ip})$ for all $i \in Q$. 

Let $\Psi_{\Gamma_Q}(\cdot): \overline{Q} \rightarrow \R^k$ denote a function taking as input a node $\overline{Q}$ and outputting a vector $v = (v_1,...,v_k) \in \R^k$.
\[
v_p \defeq 
\begin{cases} 
1 & \E_{\Gamma_Q}[\sum_{j \in N_{Q}(i)} \sum_{q \in [k]} f_{i,j}(p,q)a_{jq}] \geq \E_{\Gamma_Q}[\sum_{j \in N_{Q}(i)} \sum_{q \in [k]} f_{i,j}(p',q)a_{jq}] \text{ }\forall p' \in [k]\\
0 & \text{otherwise}
\end{cases}
\]
Furthermore let $\Lambda_{\Gamma_{Q}}(i) \defeq \frac{\Psi_{\Gamma_Q}(i)}{\| \Psi_{\Gamma_Q}(i)\|_1}$.  With a slight abuse of notation, we will sometimes interpret $\Lambda_{\Gamma_{Q}}(i)$ as a probability distribution over $[k]$ for all $i \in \overline{Q}$.  We also use $\Gamma_Q \bigcup \Lambda_{\Gamma_Q}(i)$ for $i \in \overline{Q}$ to denote appending the best response distribution over actions of node $i$ onto the distribution $\Gamma_Q$ defined over the nodes $Q$.  Similarly we use the notation $\Gamma_Q \bigcup_{i \in Y} \Lambda_{\Gamma_Q}(i)$ to denote appending the best response distribution of all nodes $i \in Y$ to $\Gamma_Q$ where $Y$ is any set of nodes disjoint from $Q$.  


We begin our discussion with a few definitions.           
First we define $\Delta_\zeta(i)$ to be the average covariance of player $i$ with players that are neighbors of $i$.  
\begin{definition}
Let $\Delta_\zeta(i)$ for $i \in [N]$ be defined as 
\[\Delta_\zeta(i) \defeq \sum_{j \in N(i)}\sum_{(p,q) \in [k]^2}|\pE_\zeta[a_{ip}a_{jq}] - \pE_\zeta[a_{ip}]\pE_\zeta[a_{jq}]|\]
\end{definition}
Next we define $\Phi_\Gamma(i)$ for a product distribution $\Gamma \in \calP(k,V)$ to be the value of the Nash equilibrium constraint.
\begin{definition}
Let $\Gamma \in \calP(k,V)$ be a product distribution defined to be $\Gamma \defeq \{\pE_\zeta[a_{ip}]\}_{i \in [N], p \in [k]}$.  Then we define 
\[\Phi_{\Gamma}(i) \defeq \max_{w \in [k]} \Big(\sum_{j \in N(i)} \sum_{q \in [k]} f_{ij}(w,q) \pE_\zeta[a_{jq}] -\sum_{j \in N(i)} \sum_{(p,q) \in [k]^2} f_{ij}(p,q) \pE_\zeta[a_{ip}]\pE_\zeta[a_{jq}]\Big) \]
\end{definition}

Clearly we are trying to find a pseudoexpectation $\pE_\zeta$ such that $\Phi_\Gamma(i) \leq \epsilon$ for all $i \in [N]$.  The next lemma shows that after the rounding procedure in \pref{lem:rounding-main} we satisfy the constraints of $\Phi_\Gamma(i)$ up to an error that scales with the average pairwise covariance  
\begin{restatable}[]{lem}{constraintcorr}
\label{lem:constraintcorr}
Let $\Gamma \defeq \{\pE_\zeta[a_{ip}]\}_{i \in [N], p \in [k]}$ be a product distribution where $\pE_\zeta$ is the output of \pref{lem:rounding-main}.  Then we have that for all $i \in [N]$  
\[\Phi_{\Gamma}(i) \leq \epsilon + L\Delta_\zeta(i)\]
\end{restatable}
We defer the proof of \pref{lem:constraintcorr} to \pref{sec:brplemmas}.  With\pref{lem:constraintcorr}, it suffices to show that our rounding \pref{alg:brp} produces a product distribution such that $\Delta(i) \leq \frac{\epsilon}{L}$ for all $i \in [N]$.  

\begin{lemma}
If global correlation $\Delta \leq \epsilon^2$ then no more than $\epsilon N$ nodes satisfy $\Delta(i) \geq 2\epsilon$ 
\end{lemma}

\begin{proof}
Assume the contrary, if more than $\epsilon N$ nodes had $\Delta(i) \geq 2\epsilon$ then 
\[\Delta \defeq \frac{1}{|E|} \sum_{i=1}^N |N(i)| \Delta(i) \geq \frac{pN}{2p N^2} \sum_{i=1}^N  \Delta(i) \geq \epsilon^2\] 
which is a contradiction.  
\end{proof}
Finally, we need a lemma that says the Nash constraint is still $2\epsilon$ approximately satisfied if an $\frac{\epsilon}{L}$ fraction of the neighbors of $i$ play best response or in fact any response.
\begin{restatable}{lem}{smooth} 
\label{lem:smooth}
Let $Q$ be a subset of the vertices $V$ in graph $G$.  Let 
Let $\calP(k,Q)$ denote the space of product distributions over $[k]$ actions in $|Q|$ dimensions for agents in the vertices $Q$.  Let $\Gamma_Q$ be a product distribution in $\calP(k,Q)$.  Similarly, let $\calP(k,\overline{Q})$ denote the space of product distributions over $[k]$ actions in $|\overline{Q}|$ dimensions for agents in the vertices $\overline{Q}$.  
Then for all $i \in Q$ we have  
\[\max_{\Gamma_{\overline{Q}} \in \calP(k,\overline{Q})} \Phi_{\Gamma_Q \bigcup \Gamma_{\overline{Q}}}(i) \leq \Phi_{\Gamma_Q}(i) + L\frac{|N_{\overline{Q}}(i)|}{|N(i)|}\]
Here $\Phi_{\Gamma_Q \bigcup \Gamma_{\overline{Q}}}(i)$ is the value of the nash equilibrium constraint on agent $i$ over the vertices in $Q \bigcup \overline{Q} = V$ whereas $\Phi_{\Gamma_Q}(i)$ is only evaluated over with respect to the vertices $Q$.  
\end{restatable}
The proof is deferred to \pref{sec:brplemmas}.   The next subsection is devoted to the proof of  \pref{lem:best-response-pursuit}.  

\subsection{Best Response Rounding Proof}
In this section we prove \pref{lem:best-response-pursuit}, which provides guarantees for \pref{alg:brp}.  First we present the guarantees for \pref{alg:flag} 

\begin{restatable}{lem}{appendlemma} 
\label{lem:append-lemma} Let $H$ be a subset of $V$ with $|H| \geq ln^3(N)$.  Then $W = Flag(H,G)$ satisfies the following   

\begin{enumerate}
    \item $|W| \leq \frac{4}{3}|H|$
    \item $\frac{|N_{W}(i)|}{d} \leq \frac{40 |H|}{3N}+ 10\frac{\ln^2(N)}{d}$ for all $i \in \overline{W}$
\end{enumerate}

with high probability $1 - N^{-\ln^3(N)}\ln(N)$ over the randomness of $G$.  
\end{restatable} 

Next we present guarantees for \pref{alg:bestrespond}. 
\begin{restatable}{lem}{bestresponselemma} 
\label{lem:best-response}
Let $W \subseteq V$ and let $\Gamma \in \calP(k,N)$ be a product distribution over $k$ elements in $N$ dimensions.  We represent the product distribution $\Gamma \defeq \{\{\mu_{ip}\}_{p \in [k]}\}_{i \in V}$ where $\mu_{ip} = \mathbb{P}[a_{ip}=1]$.  Let $\Gamma_{\overline{W}} = \{\{\mu_{ip}\}_{p \in [k]}\}_{i \in \overline{W}}$ .  Let $\zeta \defeq \max_{i \in \overline{W}} \Phi_{\Gamma_{\overline{W}}}(i)$.  Let $\gamma \defeq \max_{i \in \overline{W}} \frac{|N_W(i)|}{d} $.  Then $BestRespond(W, \Gamma)$ outputs $(B, \hat{\Gamma})$ satisfying $|B| \leq \frac{|W|}{10}$ and $\max_{i \in \overline{B}} \Phi_{\hat{\Gamma}}(i) \leq \max(\zeta + L\gamma, 10 L\epsilon )$    
\end{restatable}
Finally we move on to prove \pref{lem:best-response-pursuit}
\bestresponsepursuit*

\begin{proof}
  Let $\zeta_t \defeq \max_{i \in \overline{W}} \Phi_{\Gamma_t}(i)$, and let $\gamma_t \defeq \max_{i \in \overline{W}} \frac{|N_W(i)|}{d}$.  For any $t \in [T-1]$ we have, 
\[\zeta_{t+1}\leq \max(\zeta_{t} + L\gamma_{t+1}, 100L\epsilon)
\leq \max(\zeta_{t} + L(\frac{4c}{3}|K_{t+1}| + c\frac{\ln^2(N)}{d}), 100L\epsilon)\]
Here the first inequality follows from \pref{lem:best-response}.  The second inequality follows from \pref{cor:append-lemma} which establishes that $\gamma_t \leq \frac{4c}{3}|K_t| + c \frac{\ln^2(N)}{d}$. Next, we continue to upper bound by   

\[\leq \max(\zeta_{t} + L (\frac{4c}{3} \frac{1}{4^{t}}\epsilon + c \frac{\ln^2(N)}{d}), 100L\epsilon) \leq \max(\zeta_0, 100L\epsilon) + L\frac{4c}{3} \sum_{i=1}^t \frac{1}{4^i} \epsilon + Lc \frac{t\ln^2(N)}{d}\]
 Here the first inequality follows from \pref{lem:best-response} where we have $|K_{t+1}| \leq \frac{|K_t|}{4}$.  The second inequality follows from the generic observation that for a sequence $a_1,a_2,...,a_{T} \in \R$ satisfying $a_{t+1} = \max(a_t + c_{t+1},b)$, we can conclude $a_{t+1} \leq \max(a_0,b) + \sum_{i=1}^{t+1} c_i$. 
 
 \[\leq 100L\epsilon + L\frac{16c}{9}\epsilon + Lc \frac{t\ln^2(N)}{d} \leq 100L\epsilon + L\frac{16c}{9}\epsilon + Lc \frac{\ln^3(N)}{d} \leq 100L\epsilon + L\frac{16c}{9}\epsilon + Lc\epsilon\] 
 
 The first inequality follows from the base case $\zeta_0 \leq 2\epsilon$.  The second inequality follows from $t \leq \ln(N)$, which holds because $K_t$ contracts exponentially until $K_{T-1} \leq \ln^3(N)$.  Note that the above bound is independent of $t$ because we have carried out all the geometric sums.  Therefore, $\zeta_T \leq 100L\epsilon + L\frac{16c}{9} \epsilon$ which guarantees for all $i \in V$ that 

\[\Phi_{\Gamma_T}(i) \leq 100L\epsilon + 20L\epsilon\] 
\end{proof}












%

\IEEEPARstart{T}{wo} %
main challenges in the deployment of large-scale swarms are the localization and coordination of vehicles.
Localization methods that rely on external infrastructure 
(e.g., GPS) 
are prone to systematic errors (e.g., multipath effect)
and may not always be available.
Coordination strategies that are centralized can deconflict motion plans to prevent collisions and gridlock, but introduce a single point of failure and are difficult to scale in swarm size due to communication bandwidth limitations.

This paper presents a unified formation flying pipeline for unmanned aerial vehicles (UAVs).
Our pipeline uses \textit{onboard} sensors for localization, which eliminate the need for external positioning systems, and \textit{distributed} techniques for coordination, which enable each vehicle to make decisions independently while communicating their state to a subset of the team.
For \textit{localization}, we use an off-the-shelf commercial visual inertial odometry (VIO) package \cite{VIO}
that fuses inertial measurement unit (IMU) and downward-facing monocular camera measurements to estimate changes in the vehicle pose.
\edit{For \textit{coordination}, we present distributed formation control and task assignment strategies that run onboard the vehicles, do not rely on a common reference frame, and use vehicle-to-vehicle communication.} 
Key features of our formation control strategy include scalability to a large number of vehicles and robustness to disturbances.
The latter is crucial for reaching the desired formations with sensing imperfections.
Our task assignment strategy uses an auction-based algorithm to guarantee conflict-free assignments.
This algorithm can deconflict vehicle gridlocks resulting from distributed collision avoidance (type 3 deadlock~\cite{Wang2017}) and is well-suited for vehicles with limited computational capability and low-bandwidth communication. 


\begin{figure}[t!]
	\begin{center}
		\includegraphics[trim =0mm 10mm 0mm 0mm, clip, width=\columnwidth]{Figs/slanted_plane.png}	
		\caption{
		Six multirotors in a slanted plane formation.
		Vehicles communicate with each other, make distributed decisions onboard, and use VIO for localization.}
		\label{fig:slantedplane}
	\end{center}
\end{figure}


\subsection{Contributions}

This research extends our previous work on UAV formations~\cite{Fathian2019} and presents a unified pipeline consisting of \textit{onboard localization} and \textit{distributed coordination}.
The three main contributions of this work are:
\begin{enumerate}
    \item \edit{scalable formulation of control design suitable for
    onboard sensing without a common reference frame;}
    \item algorithms for deconfliction via \edit{distributed} task assignment of vehicles to desired formation points;    
    \item simulation- and hardware-ready open-source pipeline.
\end{enumerate}
\edit{Our pipeline is tested in hardware with six multirotors (see Fig.~\ref{fig:slantedplane}), and 
to our knowledge is the first demonstration of formation flying that does not rely on external sensing, fiducial markers for localization, a common reference frame, or a centralized base station for coordination.}
The only requirements for the presented pipeline are that the vehicles can communicate, can find the transformation between their VIO start frames, and the environment is sufficiently textured---a standard assumption for VIO systems.
As such, this framework paves the way for future, real-world deployments of aerial vehicle swarms in large numbers and without requiring external localization infrastructure.


\begin{figure} [t!]
\centering
	\begin{subfigure}[b]{0.32\columnwidth}
	   %
	    \includegraphics[width=0.8\textwidth,left]{Figs/Frames2_full.pdf}
	    \caption{\scriptsize full alignment}
	    \label{fig:frame-a}
	\end{subfigure}
	\begin{subfigure}[b]{0.32\columnwidth}
	    \includegraphics[width=0.8\textwidth,center]{Figs/Frames2_orientation.pdf}
	    \caption{\scriptsize orientation alignment}
	    \label{fig:frame-b}
	\end{subfigure}
	\begin{subfigure}[b]{0.32\columnwidth}
	    \includegraphics[width=0.8\textwidth,right]{Figs/Frames2_none.pdf}
	    \caption{\scriptsize no alignment}
        \label{fig:frame-c}
	\end{subfigure}
\caption{\edit{Required alignment of UAV frames in existing swarm strategies: (a) the most restrictive case requiring a common reference frame, i.e., orientation and origin of the frames must be aligned; (b) only the orientation of the frames must be aligned; (c) no alignment restrictions (this work).}}
	\label{fig:Frames}
\end{figure}




\subsection{Related Work}

Existing aerial swarms can be grouped based on the coordination (centralized vs.\ distributed) and localization (external vs.\ onboard) methods used. 
\edit{It is further crucial to distinguish these methods based on the level of alignment required for the vehicle coordinate frames; see Fig.~\ref{fig:Frames}.} 
 
\edit{
Works with \textit{centralized} coordination and \textit{external} localization include~\cite{Preiss2017, Honig2018, Du2019}, which are based on lightweight UAVs with limited onboard computational capability and therefore rely on an external motion capture system and a base station.
Works with \textit{distributed} coordination and \textit{external} localization include \cite{wilson2020robotarium}, \cite{enright2004spheres}, where robots execute distributed controls  based on external localization by motion capture and ultrasonic beacons, respectively.
Works with \textit{centralized} coordination and \textit{onboard} localization include~\cite{Forster2013}, \cite{Loianno2016}, which use a ground station for task assignment among vehicles.
In \cite{Weinstein2018}, formation flying based on VIO is demonstrated, where motion planning and assignment are run on a base station to ensure collision-free trajectories.
The coordination strategies used in aforementioned works require a \textit{common reference frame} (Fig.~\ref{fig:frame-a}).
}


\edit{
Despite the large body of work on formation control~\cite{Oh2015}, and the variety of onboard sensing solutions for localization (e.g., VIO~\cite{Delmerico2018}), few frameworks demonstrated formation flying with \textit{distributed} coordination and \textit{onboard} localization.
A key reason is reliance of many distributed control and assignment algorithms on aligned frames (Fig.~\ref{fig:frame-a}, \ref{fig:frame-b}), which require computation-expensive and/or communication-intensive synchronization/consensus steps for frame alignment.
Equally important, dependence on alignment in existing methods \cite{Wang2017,Turpin2014, van2011reciprocal, morgan2016swarm} diminishes robustness to inherent noise and unobservable errors that cannot be corrected (e.g., disparities between the actual and estimated body frame \textit{orientation} caused by VIO drift).
Leveraging coordination methods that are \textit{robust to misaligned frames} is hence crucial and a focus of this work. 
}






\edit{
Examples of other pipelines with distributed coordination and onboard localization include \cite{Montijano2016,Tron2016}.
Both works demonstrated formation flying on three UAVs, required information from an external motion capture system due to hardware limitations, did not incorporate collision avoidance, and required frame alignment.
}
\edittwo{Note that while~\cite{Montijano2016,Tron2016} can achieve formations with arbitrary headings as illustrated in Fig.~\ref{fig:frame-c}, knowledge of relative orientations is still required; therefore, they belong to the category of Fig.~\ref{fig:frame-b}.}






\if 0

\r{
decentralized coordination setting combined with VIO:
D-CAPT [26]~\cite{}:
ORCA ~\cite{}: 
CBF [2]~\cite{} :
[A]
}

\r{Robusteness in coordination,  with compounded noise/latency, which would eventually break (b).\\


some existing algorithm might as well
work in a similar fully decentralized setting, when combined with VIO
as proposed here. For example, D-CAPT [26], ORCA, CBF [2] might also be
useful for such a task and are computationally even more efficient than
the proposed approach. \\

R2:  onboard sensing for localization ->
 Finally, the related work section only
focuses on this aspect of the pipeline, discussing how many formation papers include
onboard localization but barely sells the advantages of the coordination module (the actual
proposal of the paper) against other competitors such as [26] or [A] or to mention similar
coordination pipelines. \\


Given a solution to this problem, the controller in Section III seems unnecessary, each drone
has a target position and can use a local controller with collision avoidance that drives it to
that position. Note that such controllers exists in the literature (e.g., RVO in any of its
multi-agent variantes), they are distributed in nature and only require local sensing.


}

\fi
%\input{content/techniques-final}
%%%  LaTeX support: latex@mdpi.com 
%  For support, please attach all files needed for compiling as well as the log file, and specify your operating system, LaTeX version, and LaTeX editor.

%=================================================================
\documentclass[drones,article,submit,moreauthors,pdftex]{mdpi}
% For posting an early version of this manuscript as a preprint, you may use "preprints" as the journal and change "submit" to "accept". The document class line would be, e.g., \documentclass[preprints,article,accept,moreauthors,pdftex]{mdpi}. This is especially recommended for submission to arXiv, where line numbers should be removed before posting. For preprints.org, the editorial staff will make this change immediately prior to posting.

%--------------------
% Class Options:
%--------------------
%----------
% journal
%----------
% Choose between the following MDPI journals:
% acoustics, actuators, addictions, admsci, adolescents, aerospace, agriculture, agriengineering, agronomy, ai, algorithms, allergies, alloys, analytica, animals, antibiotics, antibodies, antioxidants, applbiosci, appliedchem, appliedmath, applmech, applmicrobiol, applnano, applsci, aquacj, architecture, arts, asc, asi, astronomy, atmosphere, atoms, audiolres, automation, axioms, bacteria, batteries, bdcc, behavsci, beverages, biochem, bioengineering, biologics, biology, biomass, biomechanics, biomed, biomedicines, biomedinformatics, biomimetics, biomolecules, biophysica, biosensors, biotech, birds, bloods, blsf, brainsci, breath, buildings, businesses, cancers, carbon, cardiogenetics, catalysts, cells, ceramics, challenges, chemengineering, chemistry, chemosensors, chemproc, children, chips, cimb, civileng, cleantechnol, climate, clinpract, clockssleep, cmd, coasts, coatings, colloids, colorants, commodities, compounds, computation, computers, condensedmatter, conservation, constrmater, cosmetics, covid, crops, cryptography, crystals, csmf, ctn, curroncol, currophthalmol, cyber, dairy, data, dentistry, dermato, dermatopathology, designs, diabetology, diagnostics, dietetics, digital, disabilities, diseases, diversity, dna, drones, dynamics, earth, ebj, ecologies, econometrics, economies, education, ejihpe, electricity, electrochem, electronicmat, electronics, encyclopedia, endocrines, energies, eng, engproc, ent, entomology, entropy, environments, environsciproc, epidemiologia, epigenomes, est, fermentation, fibers, fintech, fire, fishes, fluids, foods, forecasting, forensicsci, forests, foundations, fractalfract, fuels, futureinternet, futureparasites, futurepharmacol, futurephys, futuretransp, galaxies, games, gases, gastroent, gastrointestdisord, gels, genealogy, genes, geographies, geohazards, geomatics, geosciences, geotechnics, geriatrics, hazardousmatters, healthcare, hearts, hemato, heritage, highthroughput, histories, horticulturae, humanities, humans, hydrobiology, hydrogen, hydrology, hygiene, idr, ijerph, ijfs, ijgi, ijms, ijns, ijtm, ijtpp, immuno, informatics, information, infrastructures, inorganics, insects, instruments, inventions, iot, j, jal, jcdd, jcm, jcp, jcs, jdb, jeta, jfb, jfmk, jimaging, jintelligence, jlpea, jmmp, jmp, jmse, jne, jnt, jof, joitmc, jor, journalmedia, jox, jpm, jrfm, jsan, jtaer, jzbg, kidney, kidneydial, knowledge, land, languages, laws, life, liquids, literature, livers, logics, logistics, lubricants, lymphatics, machines, macromol, magnetism, magnetochemistry, make, marinedrugs, materials, materproc, mathematics, mca, measurements, medicina, medicines, medsci, membranes, merits, metabolites, metals, meteorology, methane, metrology, micro, microarrays, microbiolres, micromachines, microorganisms, microplastics, minerals, mining, modelling, molbank, molecules, mps, msf, mti, muscles, nanoenergyadv, nanomanufacturing, nanomaterials, ncrna, network, neuroglia, neurolint, neurosci, nitrogen, notspecified, nri, nursrep, nutraceuticals, nutrients, obesities, oceans, ohbm, onco, oncopathology, optics, oral, organics, organoids, osteology, oxygen, parasites, parasitologia, particles, pathogens, pathophysiology, pediatrrep, pharmaceuticals, pharmaceutics, pharmacoepidemiology, pharmacy, philosophies, photochem, photonics, phycology, physchem, physics, physiologia, plants, plasma, pollutants, polymers, polysaccharides, poultry, powders, preprints, proceedings, processes, prosthesis, proteomes, psf, psych, psychiatryint, psychoactives, publications, quantumrep, quaternary, qubs, radiation, reactions, recycling, regeneration, religions, remotesensing, reports, reprodmed, resources, rheumato, risks, robotics, ruminants, safety, sci, scipharm, seeds, sensors, separations, sexes, signals, sinusitis, skins, smartcities, sna, societies, socsci, software, soilsystems, solar, solids, sports, standards, stats, stresses, surfaces, surgeries, suschem, sustainability, symmetry, synbio, systems, taxonomy, technologies, telecom, test, textiles, thalassrep, thermo, tomography, tourismhosp, toxics, toxins, transplantology, transportation, traumacare, traumas, tropicalmed, universe, urbansci, uro, vaccines, vehicles, venereology, vetsci, vibration, viruses, vision, waste, water, wem, wevj, wind, women, world, youth, zoonoticdis 

%---------
% article
%---------
% The default type of manuscript is "article", but can be replaced by: 
% abstract, addendum, article, book, bookreview, briefreport, casereport, comment, commentary, communication, conferenceproceedings, correction, conferencereport, entry, expressionofconcern, extendedabstract, datadescriptor, editorial, essay, erratum, hypothesis, interestingimage, obituary, opinion, projectreport, reply, retraction, review, perspective, protocol, shortnote, studyprotocol, systematicreview, supfile, technicalnote, viewpoint, guidelines, registeredreport, tutorial
% supfile = supplementary materials

%----------
% submit
%----------
% The class option "submit" will be changed to "accept" by the Editorial Office when the paper is accepted. This will only make changes to the frontpage (e.g., the logo of the journal will get visible), the headings, and the copyright information. Also, line numbering will be removed. Journal info and pagination for accepted papers will also be assigned by the Editorial Office.

%------------------
% moreauthors
%------------------
% If there is only one author the class option oneauthor should be used. Otherwise use the class option moreauthors.

%---------
% pdftex
%---------
% The option pdftex is for use with pdfLaTeX. If eps figures are used, remove the option pdftex and use LaTeX and dvi2pdf.

%=================================================================
% MDPI internal commands
\firstpage{1} 
\makeatletter 
\setcounter{page}{\@firstpage} 
\makeatother
\pubvolume{1}
\issuenum{1}
\articlenumber{0}
\pubyear{2022}
\copyrightyear{2022}
%\externaleditor{Academic Editor: Firstname Lastname}
\datereceived{} 
%\daterevised{} % Only for the journal Acoustics
\dateaccepted{} 
\datepublished{} 
%\datecorrected{} % Corrected papers include a "Corrected: XXX" date in the original paper.
%\dateretracted{} % Corrected papers include a "Retracted: XXX" date in the original paper.
\hreflink{https://doi.org/} % If needed use \linebreak
%\doinum{}
%------------------------------------------------------------------
% The following line should be uncommented if the LaTeX file is uploaded to arXiv.org
%\pdfoutput=1

%=================================================================
% Add packages and commands here. The following packages are loaded in our class file: fontenc, inputenc, calc, indentfirst, fancyhdr, graphicx, epstopdf, lastpage, ifthen, lineno, float, amsmath, setspace, enumitem, mathpazo, booktabs, titlesec, etoolbox, tabto, xcolor, soul, multirow, microtype, tikz, totcount, changepage, attrib, upgreek, cleveref, amsthm, hyphenat, natbib, hyperref, footmisc, url, geometry, newfloat, caption

\usepackage{xcolor}
\usepackage{subcaption}

%=================================================================
%% Please use the following mathematics environments: Theorem, Lemma, Corollary, Proposition, Characterization, Property, Problem, Example, ExamplesandDefinitions, Hypothesis, Remark, Definition, Notation, Assumption
%% For proofs, please use the proof environment (the amsthm package is loaded by the MDPI class).

%=================================================================
% Full title of the paper (Capitalized)
\Title{Drone Detection and Tracking in Real-Time by Fusion of Different Sensing Modalities}

% MDPI internal command: Title for citation in the left column
\TitleCitation{Title}

% Author Orchid ID: enter ID or remove command
\newcommand{\orcidauthorA}{0000-0000-0000-000X} % Add \orcidA{} behind the author's name
%\newcommand{\orcidauthorB}{0000-0000-0000-000X} % Add \orcidB{} behind the author's name

% Authors, for the paper (add full first names)
\Author{Fredrik Svanström $^{1,\dagger}$, Fernando Alonso-Fernandez $^{2,}$*\orcidA{0000-0002-1400-346X} and Cristofer Englund $^{2,3,\ddagger}$}

%\longauthorlist{yes}

% MDPI internal command: Authors, for metadata in PDF
\AuthorNames{Fredrik Svanström, Fernando Alonso-Fernandez and Cristofer Englund}

% MDPI internal command: Authors, for citation in the left column
\AuthorCitation{Lastname, F.; Lastname, F.; Lastname, F.}
% If this is a Chicago style journal: Lastname, Firstname, Firstname Lastname, and Firstname Lastname.

% Affiliations / Addresses (Add [1] after \address if there is only one affiliation.)
\address{%
$^{1}$ \quad Air Defence Regiment. Swedish Armed Forces\\
$^{2}$ \quad School of Information Technology, Halmstad University, SE 301 18 Halmstad, Sweden\\
$^{3}$ \quad RISE, Lindholmspiren 3A, SE 417 56 Gothenburg, Sweden}

% Contact information of the corresponding author
\corres{Correspondence: feralo@hh.se}

% Current address and/or shared authorship
\firstnote{Email: DroneDetectionThesis@gmail.com} 
\secondnote{Email: cristofer.englund@hh.se}
% The commands \thirdnote{} till \eighthnote{} are available for further notes

%\simplesumm{} % Simple summary

%\conference{} % An extended version of a conference paper

% Abstract (Do not insert blank lines, i.e. \\) 
\abstract{
Automatic detection of flying drones is a key issue where its presence, especially if unauthorized, can create risky situations or compromise security.
%
Here, we design and evaluate a multi-sensor drone detection system.
%
In conjunction with standard video cameras and microphone sensors, we explore the use of thermal infrared cameras, pointed out as a feasible and promising solution that is scarcely addressed in the related literature.
%
Our solution integrates a fish-eye camera as well to monitor a wider part of the sky and steer the other cameras towards objects of interest.
%
The sensing solutions are complemented with an ADS-B receiver, a GPS receiver, and a radar module. However, our final deployment has not included the latter due to its limited detection range.
%
The thermal camera is shown to be a feasible solution as good as the video camera, even if the camera employed here has a lower resolution. 
%
Two other novelties of our work are the creation of a new public dataset of multi-sensor annotated data that expands the number of classes compared to existing ones, as well as the study of the detector performance as a function of the sensor-to-target distance.
%
Sensor fusion is also explored, showing that the system can be made more robust in this way, mitigating false detections of the individual sensors.
%
}

% Keywords
\keyword{Drone detection, UAV detection, Anti-drone systems} 

% The fields PACS, MSC, and JEL may be left empty or commented out if not applicable
%\PACS{J0101}
%\MSC{}
%\JEL{}

%%%%%%%%%%%%%%%%%%%%%%%%%%%%%%%%%%%%%%%%%%
% Only for the journal Diversity
%\LSID{\url{http://}}

%%%%%%%%%%%%%%%%%%%%%%%%%%%%%%%%%%%%%%%%%%
% Only for the journal Applied Sciences
%\featuredapplication{Authors are encouraged to provide a concise description of the specific application or a potential application of the work. This section is not mandatory.}
%%%%%%%%%%%%%%%%%%%%%%%%%%%%%%%%%%%%%%%%%%

%%%%%%%%%%%%%%%%%%%%%%%%%%%%%%%%%%%%%%%%%%
% Only for the journal Data
%\dataset{DOI number or link to the deposited data set if the data set is published separately. If the data set shall be published as a supplement to this paper, this field will be filled by the journal editors. In this case, please submit the data set as a supplement.}
%\datasetlicense{License under which the data set is made available (CC0, CC-BY, CC-BY-SA, CC-BY-NC, etc.)}

%%%%%%%%%%%%%%%%%%%%%%%%%%%%%%%%%%%%%%%%%%
% Only for the journal Toxins
%\keycontribution{The breakthroughs or highlights of the manuscript. Authors can write one or two sentences to describe the most important part of the paper.}

%%%%%%%%%%%%%%%%%%%%%%%%%%%%%%%%%%%%%%%%%%
% Only for the journal Encyclopedia
%\encyclopediadef{For entry manuscripts only: please provide a brief overview of the entry title instead of an abstract.}

%%%%%%%%%%%%%%%%%%%%%%%%%%%%%%%%%%%%%%%%%%
\begin{document}

%%%%%%%%%%%%%%%%%%%%%%%%%%%%%%%%%%%%%%%%%%

% The order of the section titles is: Introduction, Materials and Methods, Results, Discussion, Conclusions for these journals: aerospace,algorithms,antibodies,antioxidants,atmosphere,axioms,biomedicines,carbon,crystals,designs,diagnostics,environments,fermentation,fluids,forests,fractalfract,informatics,information,inventions,jfmk,jrfm,lubricants,neonatalscreening,neuroglia,particles,pharmaceutics,polymers,processes,technologies,viruses,vision


\section{Introduction}

Drones, also referred to small and remotely controlled unmanned aerial vehicles (UAVs), can fulfil valuable societal roles such as law enforcement, medical, construction, search and rescue, parcel delivery, remove area exploration, topographic mapping, forest/water management, or inspection of big infrastructures like power grids \cite{drones6060147}. 
%
%Examples include delivery of defibrillators \cite{[Sanfridsson19]}, fire fighting \cite{INNOCENTE201980}, law enforcement, construction, search and rescue, parcel delivery, etc. 
%
Their low cost and ease of operation have caused drones to find their way into consumer use just for recreation and entertainment as well \cite{[FAI]}.
%
Unfortunately, they can also be intentionally or unintentionally misused, threatening the safety of others. %In the worst case,
For example, an aircraft can be severely damaged if it collides with a consumer-sized drone, even at moderate speeds %, as shown by researchers at the University of Dayton
\cite{[Dayton]}, and an ingested drone can rapidly disable an aircraft engine.
%
An increasingly common risk is the report of drone sightings in restricted airport areas, which ultimately has led to total closure of the airport and cancellation of hundreds of flights \cite{UAVincidents}. 
%
Several near-misses and verified collisions with UAVs have involved hobbyist drone operators violating aviation safety regulations, sometimes without knowledge.
%
This rapid development in the use and potential misuse of UAVs has consequently produced an increase in research on drone detection \cite{Taha19,[GoogleTrends]} to counteract potential risks due to intrusion in restricted areas, either intentional or unintentional. 
%

In this work, we address the design and evaluation of an automatic multi-sensor drone detection and tracking system.
%
Our developments are built on state-of-the-art machine learning techniques, extending methods from conclusions and related literature recommendations \cite{Guvenc18,Taha19}.
%
In addition to effective detection, classification and tracking methods, the existing literature also points out sensor fusion as a critical open area to achieve more accuracy and robustness compared to a single sensor.
%
Despite this, research in sensor fusion for drone detection is scarce \cite{Samaras_2019,Guvenc18,Diamantidou19,Shi18}.
%
Our work also encompasses collecting and annotating a public dataset to accomplish training and evaluation of the system.
%
Another fundamental challenge is the lack of public reference databases that serve as a benchmark for researchers \cite{Taha19}.
%
Three different consumer-grade drones are included in the dataset together with birds, airplanes and helicopters, in which constitutes the published dataset with the biggest number of target classes (drone, bird, airplane and helicopter, in comparison to others which only contain three, two or one of these classes only).
%
To achieve effective detection of drones, in building the classes, we have considered including other flying objects that are likely to be mistaken for a drone \cite{Saqib17,Aker17}.
%
Another missing piece in previous studies that we address here is the system's classification performance as a function of the distance to the target, with annotations of the database including such information as well.

A preliminary version of this article appeared at a conference~\cite{drone20icpr}.
%
In the present contribution, we substantially increase the amount of reported results, e.g. the previous paper only reported the precision, recall and F1-score of the individual sensors provided in Tables~\ref{tab:results-IRcam}, \ref{tab:results-Vcam} and \ref{tab:results-audio}, as well as the fusion results of Figure~\ref{fig:fusion-results1}.
%
Here, we extensively analyze the effect of internal parameters of the different detectors on their performance for the various sensors.
%
We also report results with a radar module and provide comments about the fish-eye camera motion detector, all of them missing in the previous publication.
%
Additional results on the fusion of sensors are also provided, including an Appendix with complementary observations and visual examples.
%
New detailed information about the system architecture, hardware and software employed is also provided, including details about implementation and design choices not included in the previous publication.
%
The related work is also described in more detail.

The rest of the paper is organized as follows.
%
Section~\ref{sect:soa} describes related work.
%
Section ~\ref{sect:materialsmethods} extensively describes the proposed system, including the architecture, the hardware components, the involved software, the Graphical User Interface, and the dataset. 
%
The experimental results are presented and discussed in Section~\ref{sect:results}. 
%
Finally, the conclusions are given in Section~\ref{sect:conclusions}.


%%%%%%%%%%%%%%%%%%%%%%%%%%%%%%%%%%%%%%%%%%
\section{Related Work}
\label{sect:soa}

%To automatically detect flying drones, some form of sensor system is needed. As shown in \cite{Samaras_2019}, the fusion of data from multiple sensors %, i.e. using several sensors in combination 
%to achieve more accurate results than derived from single sensors, while compensating for their individual weaknesses, is well-founded when it comes to the drone detection task.

Fusing data from multiple sensors allows for more accurate results than a single sensor, while compensating for their individual weaknesses \cite{Samaras_2019}.
%
The sensors used for drone detection include:
%that may be considered, %for drone detection and classification tasks, 
%and hence can be found in the related scientific literature, are:
$i$) radar (on several different frequency bands, both active and passive), $ii$) cameras in the visible spectrum, $iii$) cameras detecting thermal infrared emission (IR), $iv$) microphones to detect acoustic vibrations, i.e. sound, $v$) sensors to detect radio frequency signals to and from the drone and the controller (RF), and $vi$) scanning lasers (Lidar). As mentioned in \cite{Samaras_2019} and explored further in \cite{Boddhu13}, even humans are employed for the task. It has also been successfully demonstrated that animals can be trained for this role \cite{[GuardFromAboveBV]}.
%
Systems for drone detection utilizing one or more of the sensors mentioned above may also be combined with some effector to try to bring the flying drone down or take control of it in some other way. An effector component, however, is not part of this work.

An introduction to the subject and a comparison of drone detection and tracking techniques is given in the 2018 paper \cite{Guvenc18}. It highlights as open research the use of fusion techniques to exploit data from multiple sensors, and the development of effective machine learning techniques for detection, classification and tracking in various scenarios of interest. 
%
%An introduction to the subject of drone detection can be found in \cite{Guvenc18}, published in 2018. It contains a comparison of advantages and disadvantages of different drone detection and tracking techniques. Regarding sensor fusion, the authors state that ``For accurate and quick detection/tracking of UAVs, data fusion techniques that can simultaneously use information from multiple types of sensors carry critical importance, and this constitutes an open research area''.
%
%One of the conclusions of \cite{Guvenc18} is also that ``Future research directions include developing effective machine learning methods for detecting, classifying, and tracking drones in various scenarios of interest''. 
The paper also briefly provides an overview of ways to interdict unauthorized drones.

A comprehensive review of about 50 references is given in the 2019 paper \cite{Taha19}, which comments the different machine learning techniques based on the type of sensor, including its limitations.
%
The lack of public reference datasets is identified as an essential issue. Furthermore, no study analyzes the classification performance in relation to the distance to the drone. %, or the of the regression models. 
In addition, the sensing device, drone type, detection range, or dataset used is usually not specified, all key aspects to make works reproducible. 
%
%The paper \cite{Taha19} presents a comprehensive review of the literature on machine learning for drone detection and classification. Published in 2019, it summarizes the results of %the related work found in 
%about 50 references %into several tables, 
%based on the type of sensor, describing the specifications and limitations of the different techniques.
%
%The authors stress on more than one occassion the 
%
%%One of the take-home messages is that the authors, on more than one occasion, stresses the current 
%lack of publicly available reference datasets. Furthermore, they also write that ``No study was presented which investigated the classification performance as a function of drone distance not to speak of determining the range using regression models'', a direction which is pinpointed as ``very interesting''.
%
%They also highlight that most of the research in drone detection using visual sensors do not specify the sensing device, drone type, detection range, or dataset used, all being key aspects to make works reproducible. 
%
%%This can be a very interesting research area in the future'', and that ``Most of  the research in visual drone detection fails to specify the type of the acquisition device, the drone type, the detection range, and the dataset used in their research. These details are key to validate the work and make it comparable with related literature''.
%
%%A weakness of \cite{Taha19} is that it does not include any papers that make use of thermal infrared cameras and, as we will see next, this is something that will be of importance to this work.
%

Also from 2019, the paper \cite{Samaras_2019} has an exhaustive 178 references, not specific to drone detection and classification but also regarding the general techniques employed. 
%
%Within these, there are not only articles specific to drone detection and classification, but also regarding the general techniques employed, e.g. useful aspects of sensor fusion and machine learning.
%
It %directly influences the present work, 
emphasizes as well the lack of works that use %tackling UAV detection and classification using 
thermal cameras, despite the successful use of such sensors together with deep learning-based methods for general object detection. %As such, the authors motivate researchers to ``turn their attention to this novel subject''.
%
%%such as "For the case of thermal cameras, we present deep learning based methods for general object detection and clas- sification because, to the best of authors’ knowledge, there is currently no work tackling the UAV detection and classification problem", and "Deep learning based object detection and classification architectures have been suc- cessfully utilized on thermal imagery for generic targets yet not for UAVs. This could be the motivation for researchers to turn their attention to this novel subject".

\subsection{Sensors Detecting Thermal Infrared Emission}

The work \cite{ANDRASI2017183}, from 2017, does not utilize machine learning, but %. Instead, detection and classification is done by 
a human looking at the output video stream.
%
The sensor is a low-cost FLIR Lepton 80$\times$60 pixels thermal camera. Connected to a Raspberry Pi, the authors are able to detect three different drone types up to %a distance of 
100m. One conclusion of the paper is that the drone battery, and not the motors (as one may presume), is the most significant source of heat radiation.
%
With the background from this paper and the ones above, the present work will try to extend these findings using a higher resolution sensor (FLIR Boson with 320$\times$256 pixels) in combination with machine learning methods. % to build a system that automatically detects, classifies and tracks drones. 
The IR camera will also be combined with at least one additional sensor.

Thermal cameras combined with deep-learning detection and tracking are explored in the 2019 paper \cite{wang_chen_choi_kuo_2019}. The IR videos are of 1920$\times$1080, but the sensor is not specified.
%
Detection is done with a Faster-RCNN. Given the difficulties of acquiring enough data, it uses a modified Cycle-GAN (General Adversarial Network) to produce synthetic thermal training data.
%
Via precision-recall curves, the thermal detector is shown to achieve better performance than a visible sensor used for comparison.
%
Sadly, no details about the sensor-to-target distance are given.
%
The database is said to be public as ``USC drone detection and tracking dataset'', but without a link. The dataset is also mentioned in \cite{Wu18}, but the link in that paper is not working.
%
Compared to \cite{wang_chen_choi_kuo_2019}, the present paper uses three different drone types instead of one. We also expand the number of target classes to four and, additionally, we explore the detection performance as a function of sensor-to-target distance.

A thermal infrared camera is also used as one of the sensors in \cite{Diamantidou19}, but %just as pointed out in \cite{Taha19}, 
the paper fails to specify the type, field of view or even the resolution of the sensors used, so even if there are useful tables of the results, any comparison is unfortunately hard to achieve.

\subsection{Sensors in the Visible Range}

A widespread method to detect drones %, as described in recent scientific papers, 
is to combine a video camera with a detector based on a convolutional neural network (CNN).
%
The paper \cite{Park17}, from 2017, studies six different CNN models, providing metrics for training time, speed performance (frames per second) and precision-recall curves. The comparison shows that considering the speed and accuracy trade-off, YOLOv2 \cite{Redmon17} seems to be the most appropriate detection model.

As it turns out from other studies on drone detection in the visible range, the YOLOv2 architecture is prevalent \cite{Wu18}, \cite{Liu_2018}, \cite{Saqib17} \cite{Aker17}. A lightweight version of the more recent YOLOv3 is utilized in \cite{Unlu19}.
%
The success of the YOLOv2 motivates its use in the present paper. This choice will also enable comparison to the papers mentioned above. %To be able to match the results from the thermal infrared sensor, the YOLOv2 is also used in that part.


The use of pan/tilt platforms to steer cameras in the direction of suspicious objects has also led to the use of wide-angle sensors. %, as in the present paper.
%
%The fact that the two primary sensors (IR and visible) of the drone detector system in the present paper are placed on a pan/tilt platform also arises the need for a wide-angle sensor to steer the sensors in directions where suspicious objects appear. 
%
In \cite{Unlu19}, a static camera with 110$^{\circ}$ field of view (FoV) is used together with a YOLOv3 detector to align a rotating narrow-field camera. To find objects of interest with the wide-angle camera, the paper \cite{Unlu19} employs a Gaussian Mixture Model (GMM) foreground detector \cite{Stauffer99}, a strategy also followed in the present paper. However, our wide-angle sensor has an even wider FoV (180$^{\circ}$). As pointed out in \cite{Unlu19}, this setup is prone to produce false alarms in some situations, but as described later, %in Section 3.1.3, 
this can be mitigated by tuning the associated detection and tracking algorithms.

Of all papers found using versions of the YOLO architecture for detection, \cite{Liu_2018} has the most output target classes with three (drone, airplane and helicopter), followed by \cite{Aker17} with two classes (drone and bird). However, none of the YOLO papers reports the detection accuracy as a function of the sensor-to-target distance.

\subsection{Acoustic Sensors}

Numerous papers have also explored the use of acoustic sensors. Some like \cite{Kim17}, \cite{Siriphun18}, \cite{Park15} utilize the Fast Fourier Transform (FFT) to extract features from the audio signals.
%
However, the Mel Frequency Cepstrum Coefficients (MFCC) seems to be the most popular technique, as used in \cite{Anwar19}, \cite{Liu17}, \cite{Jeon17}, \cite{Bernardini17}. The MFCC consists of a non-linear mapping of the original frequency according to the auditory mechanism of the human ear, and it is the most commonly used audio feature in current drone recognition tasks \cite{Liu_2018}.
%

When comparing %three different 
classification models, %for the drone detection task, 
the authors of \cite{Jeon17} conclude that Long Short-Term Memory (LSTM) \cite{Hochreiter97} achieves the best performance and F1-score. In that paper, the classification is binary (drone or background). The present work expands the range of output classes of \cite{Jeon17} by adding a helicopter class.
%
Also, the maximum acoustic detection range in the reviewed papers is 290 m, using a 120-element microphone array and a DJI Phantom 2 drone \cite{Busset15}. It is worth noting that the DJI Flame Wheel F450, one of the drones used in this work, is detected at a distance of 160 m by the microphone array.

\subsection{Radar}

Since radar is the most common technology to detect flying objects, it is not far-fetched to apply it to drones. 
%
However, a system designed to detect aircrafts often has features to reduce unwanted echoes from small, slow and low-flying objects, which is precisely what characterises UAVs. 
%
%Doing this with a radar system designed to detect aircrafts is however not as straight forward as it might seem, since those systems often have features to reduce unwanted echoes from small, slow and low-flying objects, which is precisely what characterise the drones in question.
%
The small Radar Cross Sections (RCS) of medium-sized consumer drones are described in \cite{Patel18}, and from \cite{Herschfelt17} we have that the RCS of the DJI Flame Wheel F450 is -17 dBsm (0.02 m$^2$). The paper \cite{Gong19} points out that flying birds have similar RCS, which can lead to false targets. % when detecting drones.
%
The F450 drone is also used in \cite{Fuhrmann17}, where the micro-doppler characteristics of drones are investigated. These are typically echoes from the moving blades of the propellers, and they can be detected on top of the bulk motion doppler signal of the drone. Since the propellers are generally made from plastic, the RCS of these parts are even smaller, and in \cite{Patel18}, it is stated that the echoes from the propellers are 20 to 25 dB weaker than the drone body itself. Nevertheless, papers like \cite{Bjorklund18}, \cite{Drozdowicz16} and \cite{Rahman18} accompany \cite{Fuhrmann17} in exploring the possibility of classifying drones using the micro-doppler signature.

\subsection{Other Drone Detection Techniques}

Very few drones are autonomous in the flight phase. Generally, they are controlled by ground equipment, and often send information on some radio frequency (RF), which can be used to detect them as well. The three drones used here are all controlled in real-time. The information sent out range from just simple telemetry such as battery level (DJI Flame wheel F450), a live video stream (Hubsan H107D+), to both a video stream and extensive position and status information (DJI Phantom 4 Pro).
%
Utilizing the RF fingerprint is described in \cite{Birnbach17}, and in \cite{Shorten18}, a CNN is used with data from an antenna array so that the direction to the drone controller can be calculated within a few degrees. In \cite{Ezuma20}, signals from 15 different drone controllers are classified with an accuracy of 98.13\% using only three RF features (shape factor, kurtosis and variance) with a K-Nearest Neighbour (KNN) classifier.
%
The use of LiDAR (Light Detection And Ranging) and LADAR (LAser Detection And Ranging) has also been explored \cite{Kim_2018}, 
%in combination with background subtraction and a nearest neighbour classifier is shown to 
successfully detecting drones up to 2 km. % in \cite{Kim_2018}.

\subsection{Sensor Fusion}

The paper \cite{Samaras_2019} mentioned data fusion from multiple sensors as a way to improve accuracy, since sensor combination helps to compensate for individual weaknesses. 
%
As mentioned above, the benefits of sensor fusion are also pointed out in \cite{Guvenc18}.
%
Considering that missed detection of an intruding drone will bring more security threats than false alarms, the authors of \cite{Shi18} conduct audio, video and RF detection in parallel, using a logical OR operation to fuse the results of the three detectors.
%
They also highlight that combining such heterogeneous data sources is an open research question since the simple combination of the results separately obtained by audio, video, and RF surveillance can induce significant information loss. In contrast, it would be of greater significance to develop reliable fusion techniques at the feature level as well. 
%
%Furthermore under the section Challenges and Open Research Is- sues they conclude that "Although we have realized basic functions for drone surveillance in our anti-drone system, new challenges have been raised, and there are still some open research issues to be addressed: Heterogeneous Information Fusion: The result of drone detection should be more than a sim- ple combination of the results separately obtained by audio, video, and RF surveillance, which will induce great information loss. It is of great signif- icance to develop reliable techniques for the fusion of audio, video, and RF information from the aspect of feature fusion and decision fusion".
%
Without any further specifications of the sensors used besides that they are visible, thermal and 2D-radar, the work \cite{Diamantidou19} presents promising results from experiments using a multilayer perceptron (MLP) to perform sensor fusion in a drone detector/classifier with just one output class.
%
Just as in \cite{Samaras_2019}, the present paper also considers early and late sensor fusion and differentiates these two principles based on whether the sensor data is fused before or after the detection element.

\subsection{Drone Detection Datasets}

As mentioned, the work \cite{Taha19} points out the lack of publicly available datasets. This is also highlighted in \cite{Samaras_2019}, especially with thermal infrared cameras.
%
%They conclude that creating a dataset for UAV detection with thermal images might be budget-wise out of reach of many universities and research centers.
%
The latter paper also states the strong need for real-world UAV audio datasets. % that could serve as research benchmarks.

In the references of \cite{Samaras_2019}, there are two useful links to datasets for visible video detectors. One of these is \cite{Reiser20}, where 500 annotated drone images can be found. 
%
This is far away from the 203k images of our database (plus audio clips).
%
The other link leads to the dataset \cite{SafeShore} of the drone-vs-bird challenge held by the Horizon2020 SafeShore project consortium. However, the dataset is only available upon request and with restrictions to the usage and sharing of the data. The drone-vs- bird challenge is also mentioned in \cite{Saqib17}, \cite{Aker17} and by the winning team of the 2017 challenge \cite{Schumann17}.

The dataset used in \cite{Unlu19} is not available due to confidentiality. Since the French Ministry of Defence funded the work, one can presume that the dataset, in one way or another, is a property of the French Government or the French Armed Forces.



%%%%%%%%%%%%%%%%%%%%%%%%%%%%%%%%%%%%%%%%%%
\section{Materials and Methods}
\label{sect:materialsmethods}

This section describes the proposed methodology and outlines the automatic drone detection system, first on a system level and then in deeper detail. We detail the hardware components, how they are connected to the main computational resource, and the involved software running in the drone detection system, including the graphical user interface.
%
The methods used for the composition of our dataset are also described, including a description of the dataset, its division per sensor type, target class and sensor-to-target distance.
%
Our descriptive approach is motivated by \cite{Taha19}, which highlights that most works on visual drone detection do not specify the acquisition device, drone type, detection range, or dataset, all being key details that allow replication and comparison with other works.


\begin{figure} [htb]
\centering
\includegraphics[width=0.7\textwidth]{system.png}
\caption{System architecture, hardware and software parts. 
%Top: system architecture with hardware and software parts. Bottom left: main hardware parts of the detection system. Bottom right: 
%
The deployed system is shown just north of the runway at Halmstad airport (IATA/ICAO code: HAD/ESMT). Pictures were originally appearing in \cite{drone20thesis} and published in \cite{drone20icpr}. Reprinted with permission.
\label{fig:system}}
\end{figure}


\subsection{Architecture of the System}

A drone detection system must be able to both cover a large volume of airspace and have sufficient resolution to distinguish small objects like drones and tell them apart from other types of objects.
%
Combining wide and narrow field of view (FoV) cameras is one way to accomplish this \cite{Unlu19}. Another way, shown in \cite{Liu_2018}, is to use an array of high-resolution cameras. 
%
Here, we follow the first approach since our solution has been designed with portability in mind (Figure~\ref{fig:system}c).
%
Since the present paper uses only one infrared sensor with a fixed FoV, there is no possibility of having neither a wide-angle infrared sensor nor an array of such sensors. The proposed way to achieve the desired volume coverage with the IR-sensor is to have it on a moving platform, as shown in Figure~\ref{fig:system}b. This platform can either have objects assigned to it or search by itself at moments when the sensors are not busy detecting and classifying objects.
%
The overall system architecture, detailing the hardware and software components employed, is shown in Figure~\ref{fig:system}a. The layout of the Graphical User Interface (GUI) elements is shown in the bottom right part (boxes in blue). 


To be able to react to moving objects and also to have the ability to track those, the combined time constraints of the detection cycle and the control loop of the moving platform means that the system must work in close to real-time. Hence, all the detection and classification processes must be done efficiently and with as little delay as possible. The feedback loop of the moving platform must run at a sub-second speed.
%
In putting together such a system involving several sensors and mechanical parts, choosing the proper methods is critical. 
%
All these constraints, in turn, impose demands on the efficiency of the software as well.
%
Another aspect is that to detect the drones with high efficiency, the system must also recognize and track other flying objects that are likely to be mistaken for drones. For some of these drone-like objects, this is indigenous hard, e.g. birds. For others, it is technically possible since some of them announce their presence and location via radio, e.g. ADS-B, over which most aircrafts regularly transmit messages with varied content.
%
Combining the data from several sensors under the time constraints described above must be kept simple and streamlined too. This, together with the fact that very few papers have explored sensor fusion techniques, is the motivation to have a system where the inclusion and weights of the sensors can be altered at runtime to find a feasible setting.
%


\subsection{Hardware}

As primary electro-optical sensors, we use a thermal infrared camera (denoted as IRcam in Figure~\ref{fig:system}a) and a video camera (Vcam). 
%
Our system can keep track of cooperative aircrafts via ADS-B information that is made available with an antenna, which collects the aircraft’s position, velocity vectors and identification information broadcasted by aircrafts equipped with such a system.
%
We also include audio information through a microphone, which is employed to distinguish drones from other objects in the vicinity, such as helicopters.
%
All computations are made in a standard laptop, also used to present the results to the user via the designed GUI.
%

Since the primary cameras have a limited FoV, a fish-eye lens camera (Fcam) covering 180$^{\circ}$ horizontally and 90$^{\circ}$ vertically is also used. The fish-eye camera is used to detect moving objects in its FoV and then steer the IRcam and Vcam towards the detected objects using the pan/tilt platform. 
%
If the Fcam detects nothing, the platform can be set to move in two different search patterns to scan the sky.
%
As additional sensors, our system also includes a GPS receiver.
%
The hardware components are mounted on a standard surveyor tripod to provide stability to the system. This solution also facilitates the deployment of the system outdoors, as shown in Figure~\ref{fig:system}c. 
%
Due to the nature of the system, it must also quickly be transported to and from any deployment. Hence, % a transport solution is available, where 
the system can be disassembled into a few large parts and placed in a transport box.



\subsubsection{Thermal Infrared Camera (IRcam)}

We employ a FLIR Breach PTQ-136 using the Boson 320$\times$256 pixels detector (Y16 with 16-bit grey-scale). The FoV of the IRcam s 24$^{\circ}$ horizontally and 19$^{\circ}$ vertically.
%
%An example image taken from the IRcam video stream can be seen in Figure~\ref{fig:system_GUI}, bottom right.
%
It is worth noting that this sensor has a higher resolution than the FLIR Lepton sensor with 80$\times$60 pixels used in \cite{ANDRASI2017183}. 
%
In that paper, the authors detected three drone types up to a distance of 100m. However, it was done manually by a person looking at the live video stream. 
%
In contrast, the present paper employs an automatic detection solution.
%
The signal of the IRcam is sent to the laptop at 60 frames per second (FPS) using the USB-C port, which also powers the IRcam. 
%

\subsubsection{Video Camera (Vcam)}

To capture video in the visible range, we employ a Sony HDR-CX405 video camera.
%
The feed is taken from the HDMI port, which is captured with an Elgato Cam Link 4K frame grabber that provides a 1280$\times$720 video stream in YUY2-format (16 bits per pixel) at 50 FPS.
%
The FoV of the Vcam can be made wider or narrower using its adjustable zoom lens.
%
In this work, it is adjusted to have about the same FoV as the IRcam. 
%

\subsubsection{Fish-eye Lens Camera (Fcam)}

To counteract the limited FoV of the IRcam and Vcam, a fish-eye lens camera is used to monitor a wider area of the sky and then steer and focus these two towards the detected objects to ascertain if they are drones or something else. 
%
The fish-eye lens camera employed is an ELP 8 Megapixel with a FoV of 180$^{\circ}$ degrees, which provides a 30FPS 1024$\times$768 video stream in Mjpg-format at 30 FPS via USB.

\subsubsection{Microphone}

The microphone is used to distinguish drones from other objects emitting sounds, such as, for example, helicopters. Here, we use a Boya BY-MM1 mini cardioid directional microphone connected directly to the laptop.
%
Data is stored in .wav format, with a sampling frequency of 44100 Hz.


\subsubsection{ADS-B Receiver}

To track aircraft equipped with transponders, an ADS-B receiver is also used. This consists of an antenna and a NooElec Nano 2+ Software Defined Radio receiver (SDR). This is tuned to 1090 MHz so that the identification and positional data sent out as a part of the 1 Hz squitter message can be decoded and displayed. The Nano 2+ SDR receiver is connected to the laptop via USB.

\subsubsection{GPS receiver}

To correctly present the decoded ADS-B data, the system is equipped with a G-STAR IV BU-353S4 GPS receiver connected via USB. The receiver outputs messages following the National Marine Electronics Association (NMEA) standard.


\begin{figure} [htb]
\centering
\includegraphics[width=0.6\textwidth]{data_collection.jpg}
\caption{The data collection setup using a lighter version of the system. Picture originally appearing in \cite{drone20thesis}. \label{fig:data_collection}}
\end{figure}


\subsubsection{Pan/tilt platform and servo controller}

The pan/tilt platform is a Servocity DDT-560H direct drive tilt platform together with the DDP-125 Pan assembly, also from Servocity. To achieve the pan/tilt motion, two Hitec HS-7955TG servos are used.
%
A Pololu Mini Maestro 12-Channel USB servo controller is included so that the respective position of the servos can be controlled from the laptop. Since the servos have shown a tendency to vibrate when holding the platform in specific directions, a third channel of the servo controller is also used to give the possibility to switch on and off the power to the servos using a small optoisolated relay board.
%

To supply the servos with the necessary voltage and power, both a net adapter and a DC-DC converter are available. The DC-DC solution is used when the system is deployed outdoors, and, for simplicity, it uses the same battery type as one of the available drones.
%
Some other parts from Actobotics are also used in mounting the system, and the following have been designed and 3D-printed: adapters for the IR, video and fish-eye lens cameras, a radar module mounting plate and a case for the servo controller and power relay boards.

A lighter version of the IRcam and Vcam mounting without the pan/tilt platform has also been prepared. This is used on a lightweight camera tripod when collecting the dataset, simplifying transportation and giving the possibility to set its direction manually. The data collection setup is shown in Figure~\ref{fig:data_collection}.

An unforeseen problem occurring when designing the system was actually of mechanical nature. Even though the system uses a pan/tilt platform with ball-bearings and very high-end titanium gear digital servos, the platform was observed to oscillate in some situations. This phenomenon was mitigated by carefully balancing the tilt platform and introducing some friction in the pivot point of the pan segment. It might also be the case that such problems could be overcome using a servo programmer. Changing the internal settings of the servos could also increase their maximum ranges from 90$^{\circ}$ to 180$^{\circ}$. This would extend the volume covered by the thermal infrared and video cameras so that all targets tracked by the fish-eye camera could be investigated, not just a portion of them, as now.

\subsubsection{Computer}

A Dell Latitude 5401 laptop handles the computational part of the system. It is equipped with an Intel i7-9850H CPU and an Nvidia MX150 GPU. The computer is connected to all the sensors mentioned above and the servo controller using the built-in ports and an additional USB hub, as shown in Figure~\ref{fig:system}c.

It is observed, regarding the sensors, that there is a lower limit of around 5 FPS, where the system becomes so slow that the ability to track flying objects is lost. All actions taken by the system must be well balanced, and just such a simple thing as plotting the ADS-B panel with a higher frequency than necessary can cause a drop in the FPS rate. Such problems can be overcome by using more than one computational resource.

\subsubsection{Radar module}

As indicated earlier, our solution does not include a radar module in its final deployment. 
%
However, since one was available, it was included in our preliminary tests. 
%
It is a radar module from K-MD2, whose specifications are shown in Figure~\ref{fig:radar} \cite{RFbeam}.
%
Its exclusion was motivated by its short practical detection range.
%
Interestingly, the K-MD2 radar module is also used in another research project connected to drones \cite{Mostafa18}, not to detect drones, but instead mounted on board one as part of the navigation aids in GNSS2 denied environments.


\begin{figure} [htb]
\centering
\includegraphics[width=0.6\textwidth]{radar.jpg}
\caption{The radar module K-MD2. Picture taken from\\ https://www.rfbeam.ch/files/products/21/downloads/Datasheet\_K-MD2.pdf \label{fig:radar}}
\end{figure}


\subsection{Software}

Matlab is the primary development environment in which the drone detection system has been developed.
%
The software for drone detection consists of the main script and five separate `workers', one per sensor, as shown in Figure~\ref{fig:system}a.
%
The main script and the workers can run asynchronously and in parallel, enabled by the Matlab parallel computing toolbox. 
%
This allows each detector to run independently of the others. This also allows the different sensors to run asynchronously, handling as many frames per second as possible without inter-sensor delays and waiting time.
%
The main script communicates with the workers using pollable data queues.
%
The \textit{Fcam worker} utilizes a foreground/background detector via GMM background subtraction \cite{Stauffer99,gmmmatlab} and a multi-object Kalman filter tracker \cite{kalmanmatlab}. After calculating the position of the best-tracked target (defined as the one with the longest track history), it sends the azimuth and elevation angles to the main script, which then controls the pan/tilt platform, so that the moving object can be analysed further by the IR and video cameras.
%
The \textit{IRcam} and \textit{Vcam} workers are similar in their basic structure, and both import and run a trained YOLOv2 detector, fine-tuned with annotated ground truth to work with data from each camera.
%
The information sent to the main script is the region of the image where the object has been detected, the class of the detected target found in the ground truth of the training dataset, the confidence, and the horizontal and vertical offsets in degrees from the centre point of the image. The latter is used to calculate servo commands and track the object.
%
%
The \textit{Audio} worker sends information about the class and confidence to the main script. 
%
It uses a classification function built on an LSTM architecture, which is applied to MFCC features extracted from audio signals captured with the microphone. 
%
Unlike the others, the \textit{ADS-B} worker has two output queues, one consisting of current tracks and the other of the history tracks. 
%
The ``current'' queue contains the current positions and additional information (Id, position, altitude, calculated distance from the system, azimuth and elevation in degrees as calculated relative to the system, time since the message was received and the target category).
%
The ``history'' tracks queue is just a list of the old positions and altitudes.
%
This partition in two queues saves computational resources by reducing the amount of data from the worker drastically, so that only the information needed/used by the main process is sent. 
%
It also makes easier to control the data flow, since the length of the history track queue is easily set if it is separated.
%
%This is done so that the presentation clearly shows the heading and altitude changes of the targets.
%
All of the above workers also send a confirmation of the command from the main script to run the detector/classifier or to be idle. The number of frames per second currently processed is also sent to the main script.

Table~\ref{tab:output-classes}
shows the different classes that each worker can provide to the main script.
%
Note that not all sensors can output all the target classes. 
%
The audio worker has an additional ``background'' class, and the ADS-B will output a ``no data'' class if the vehicle category field of the received message is empty (since it is not a mandatory field of ADS-B messages).



\begin{table}[htb]
\caption{Output classes of the sensors and their corresponding class colours. Table originally appearing in \cite{drone20thesis}. \label{tab:output-classes}}

%\small
%\footnotesize
%\scriptsize
\begin{center}
\begin{tabular}{c}

\includegraphics[width=0.85\textwidth]{output_classes.png}  \\ 

\end{tabular}

\end{center}

\end{table}
\normalsize



%We use three machine learning techniques, two of them supervised, and one unsupervised.
%
%
%The second supervised machine learning technique is a classification function built on an LSTM-architecture \cite{Hochreiter97}, which is applied to features extracted from audio signals captured with a microphone. 
%
%Finally, an unsupervised technique is used to find moving objects in the FoV of the wide-angle camera (Fcam) via GMM background subtraction \cite{Stauffer99}.


\subsubsection{Main Script}

This is the core of the system. Besides starting the five workers (threads) and setting up the queues to communicate with these, it also sends commands to and reads data from the servo controller and the GPS receiver.
%
After the start-up sequence, the script goes into a loop that runs until the program is stopped by the user via the graphical user interface (GUI).

Updating the GUI and reading user inputs are the most frequent tasks on every loop iteration. The main script interacts with the workers and the servo controller at regular intervals. Servo positions are read, and queues are polled ten times a second. The system results, i.e. the system output label and confidence, are also calculated using the most recent results from the workers. Furthermore, at a rate of 5 Hz, new commands are sent to the servo controller for execution. Every two seconds the ADS-B plot is updated. Different intervals for various tasks make the script more efficient since, for example, an aircraft sends out its position via ADS-B every second. Hence, updating the ADS-B plots too often would only be a waste of computational resources. 

\subsubsection{IRcam worker}

Raw images are processed with Matlab function \texttt{imflatfield}\footnote{2-D image flat-field correction using Gaussian smoothing with a standard deviation of $\sigma$ to approximate the shading component of the input image. It cancels the artefacts caused by variations in the pixel-to-pixel sensitivity of the detector and by distortions in the optical path. As a result, the corrected image has more uniform brightness.} for 2-D image flat-field correction (with $\sigma$=30) followed by \texttt{imadjust}\footnote{Increases the contrast of the image by mapping intensity values so that 1\% of the data is saturated at low and high intensities} to increase image contrast.
%
Flat-field correction uses Gaussian smoothing to approximate the shading component of the input image.
%
Next, the input image is processed by the YOLOv2-detector, with a given detection threshold and the execution environment set to GPU. The output from the detector consists of an array of class labels, confidence scores and bounding boxes for all objects detected and classified. The detector output may be no data at all, or just as well, data for several detected objects. In this implementation, only the detection with the highest confidence score is sent to the main script.
%
Images from the IR camera are 320$\times$256 pixels.
%
To present the result in the GUI at the same size as the Vcam output, the image is resized to 640$\times$512.
%
Then, the bounding box, class label, and confidence score are inserted into the image. To clearly indicate the detected class, the inserted annotation uses the same colour scheme as in Table~\ref{tab:output-classes}.
%
The current state of the detector and its performance (frames per second) is also inserted in the top left corner of the image. Such information is also sent to the main script with the detection results.

The YOLOv2 detector is formed by modifying a pretrained MobileNetv2 following \cite{yolo2matlab} so that the first 12 layers out of 53 are used for feature extraction. 
%
The input layer is changed to 256$\times$256$\times$3. 
%
Six detection layers and three final layers are also added to the network. Besides setting the number of output classes of the final layers, the anchor boxes used are also specified.
%
To estimate the numbers of anchor boxes to use and the size of these, the training data is processed using the \texttt{estimateAnchorBoxes} Matlab function. This function uses a $k$-means clustering algorithm to find suitable anchor boxes sizes given the number of boxes to be used, returning as well the mean intersection-over-union (IoU) value of the anchor boxes in each cluster. We test the number of anchor boxes from one to nine to provide a plot over the IoU as a function of the number of boxes, as shown in Figure~\ref{fig:IRcam-anchors-yolo}.
%

\begin{figure} [htb]
\centering
\includegraphics[width=0.7\textwidth]{IRcam-anchors-yolo.png}
\caption{Plot used to assess the number of anchor boxes to be implemented in the IRcam worker YOLOv2 detector. Picture originally appearing in \cite{drone20thesis}.}
\label{fig:IRcam-anchors-yolo}
\end{figure}

When choosing the number of anchor boxes to use, the trade-off to consider is that a high IoU ensures that the anchor boxes overlap well with the bounding boxes of the training data, but, on the other hand, using more anchor boxes will also increase the computational cost and may lead to over-fitting. After assessing the plot, the number of anchor boxes is chosen to be three, and the sizes of these (with the scaling factor of 0.8 in width to match the downsize of the input layer from 320 to 256 pixels) are taken from the output of the \texttt{estimateAnchorBoxes} function.

The detector is trained with \texttt{trainYOLOv2ObjectDetector}\footnote{Returns an object detector trained using YOLO v2 with the specified architecture, in our case a pretrained MobileNetv2 \cite{yolo2matlab} with the modifications mentioned in the main text.} using data picked from the available dataset (Section~\ref{sect:results}).
%
%after the evaluation data have been selected and put aside (Section~\ref{sect:database}). The training data for the IRcam YOLOv2 detector consists of 120 video clips, each one just over 10 seconds and evenly distributed among all classes and all distance bins, making the total number of annotated images in the training set 37428. 
The detector is trained for five epochs using the stochastic gradient descent with momentum (SGDM) optimizer and an initial learning rate of 0.001.
%
Using a computer with an Nvidia GeForce RTX2070 8GB GPU, the time for one epoch is 39 min. 
%
The training function includes pre-processing augmentation consisting of horizontal flipping (50\% probability), scaling (zooming) by a factor randomly picked from a continuous uniform distribution in the range [1, 1.1], and random colour jittering for brightness, hue, saturation, and contrast.

\subsubsection{Vcam worker}

To be able to compare their results, the methods and settings of the Vcam worker are very similar to the IRcam worker above, with some exceptions. 
%
Images are of 1280$\times$720 pixels, which are then resized to 640$\times$512 without further pre-processing. 
%
Given the bigger image size, the input layer of the YOLOv2 detector here has a size of 416$\times$416$\times$3. 
%
%The training set here consists of 37519 images, with the detector trained for five epochs as well.
%
Due to the increased image size, the training time is also extended compared to the IR case. When using a computer with an Nvidia GeForce RTX2070 8GB GPU, the time for one epoch is 2 h 25 min. 
%
%The training set consists of 37519 images, and 
The detector is trained for five epochs, just like the IRcam detector.

\subsubsection{Fcam worker}

Initially, the fish-eye lens camera was mounted upwards, but this caused the image distortion to be significant in the area just above the horizon where the interesting targets usually appear. After turning the camera so that it faces forward (as seen in Figure \ref{fig:system}b), the motion detector is less affected by image distortion.
%

The images from the camera are of 1280$\times$768 pixels, but the lower half (showing the area below the sky horizon) is cropped so that 1024$\times$384 pixels remain to be processed. 
%
Images are then analysed using the Matlab \texttt{ForegroundDetector} function \cite{gmmmatlab}, which compares an input video frame to a background model to determine whether individual pixels are part of the background or the foreground. 
%
The function uses a background subtraction algorithm based on Gaussian Mixture Models (GMM) \cite{Stauffer99}, producing a binary mask with pixels of foreground objects set to one.
%
The mask is next processed with the \texttt{imopen} function, which performs a morphological opening (erosion followed by dilation) to eliminate noise. The structural element is set to 3$\times$3, so that very small objects are deleted.
%
Then, the \texttt{BlobAnalysis} function is applied, which outputs the centroids and bounding boxes of all objects in the binary mask provided by the \texttt{ForegroundDetector} function. 
%
All centroids and bounding boxes are sent finally to a multi-object tracker based on Kalman filters \cite{kalmanmatlab}, created with the \texttt{configureKalmanFilter} function, which tracks objects across multiple frames. 
%
A Kalman filter is used to predict the centroid and bounding box of each track in a new frame based on their previous motion history. Then, the predicted tracks are assigned to the detections given by the foreground/background detector by minimizing a cost function that takes into account the Euclidean distance between predictions and detections.
%
%
The object with the longest track history is picked and retained as the best. 
%

In the Fcam presentation window, all tracks (both updated and predicted) are visualised, and the track considered to be the best is marked with red (Figure~\ref{fig:Fcam_red}).
%
The output from the Fcam worker is the FPS status, together with the elevation and azimuth angles of the best track, if such track exists at the moment. 
%
Of all the workers, the Fcam is the one with the most tuning parameters, as will be seen in Section~\ref{sect:results}. This involves choosing and tuning the image processing operations, foreground detector and blob analysis settings, and the multi-object Kalman filter tracker parameters.


\begin{figure} [htb]
\centering
\includegraphics[width=0.85\textwidth]{Fcam_red.png}
\caption{Fish-eye camera image with a detected track in red. Picture originally appearing in \cite{drone20thesis}.}
\label{fig:Fcam_red}
\end{figure}

\subsubsection{Audio worker}

The audio worker collects acoustic data in a one-second-long buffer (44100 samples), set to be updated 20 times per second. 
%
To classify the sound source in the buffer, it is first processed with the Matlab \texttt{mfcc} function, which extracts MFCC features. 
%
We employ the default parameters (a Hamming window with a length of 3\%  and an overlap of 2\% of the sampling frequency, and the number of coefficients per window equal to 13). 
%
Based on empirical trails, the parameter \texttt{LogEnergy} is set to \texttt{Ignore}, meaning that the log-energy is not calculated. Then the extracted features are sent to the classifier.
%
The extracted features are then sent to an LSTM classifier consisting of an input layer, two bidirectional LSTM layers with a dropout layer in-between, a fully connected layer, a softmax layer and a classification layer. The classifier builds on \cite{[LSTM]} but increasing the number of classes from two to three and an additional dropout layer between the bidirectional LSTM layers.

%The audio database has 30 ten-second clips of each of the output classes. Five clips from each class are set aside to be used for evaluation. The remaining audio clips are used for the training process. Out of each ten-second clip, the last second is used for validation and the rest for training. 
%
The classifier is trained from scratch for 120 epochs, after which it shows signs of over-fitting. 
%
%Since the audio worker is essentially only a classifier and not a detector and classifier (like YOLOv2), it 
%
It includes a class of general background sounds (Table~\ref{tab:output-classes}) recorded outdoors in the typical deployment environment of the system. Also, it has some clips of the sounds from the servos moving the pan/tilt platform. Like the other workers, the output is a class label with a confidence score.
%
A graphical presentation of the audio input and extracted features is also available, as shown in Figure~\ref{fig:audio-features}, but this is not included in the final GUI layout.


\begin{figure} [htb]
\centering
\includegraphics[width=0.85\textwidth]{audio-features.png}
\caption{Two examples of the audio worker plots with the audio  amplitudes and the extracted MFCC features. Pictures originally appearing in \cite{drone20thesis}.}
\label{fig:audio-features}
\end{figure}


\subsubsection{ADS-B worker}

We implement the ADS-B decoding in Matlab using functions from the Communications toolbox ~\cite{adsbmatlab}.
%
All vehicle categories that can be seen as subclasses to the airplane target label are clustered together. These are all the classes ``Light'', ``Medium'', ``Heavy'', ``HighVortex'', ``VeryHeavy'' and ``HighPerformanceHighSpeed''. The class ``Rotorcraft'' is translated into helicopter. Interestingly, there is also a ``UAV'' category label. This is also implemented in the ADS-B worker, translated into drone.

One might wonder if there are any such aircraft sending out that belong to the UAV vehicle category. Examples are found by looking at the Flightradar24 service (Figure~\ref{fig:flightradar}). Here we can find one such drone flying at Gothenburg City Airport, one of the locations used when collecting the dataset. The drone is operated by the company Everdrone AB, involved in the automated external defibrillators delivery trails of \cite{[Sanfridsson19]}.
%
Another example is the UK Coastguard/Border Force surveillance drone that is regularly flying at night over the straight of Dover since December of 2019. This is naturally a large drone with a wingspan of 7.3 m.

\begin{figure} [htb]
\centering
\includegraphics[width=0.95\textwidth]{flightradar.png}
\caption{Example of drones sending ADS-B information.
%Left: A drone flying at Gothenburg City Airport. Right: Surveillance drone over the straight of Dover. 
Images from www.flightradar24.com}
\label{fig:flightradar}
\end{figure}

As mentioned above, not all aircrafts will send out their vehicle category as part of the ADS-B squitter message. 
%
However, in our experience, about half of the aircrafts send out their category. This justifies its inclusion in this work, as one of our principles is to detect and keep track of other flying objects that are likely to be mistaken by a drone.
%
In the output message of the ADS-B worker, the confidence of the classification is set to 1 if the vehicle category message has been received. If not, the label is set to airplane (the most common aircraft type) with the confidence to 0.75 so that there is a possibility for any of the other sensors to influence the final classification.



\subsection{Graphical User Interface (GUI)}
\label{sect:GUI}

The Graphical User Interface of the system is shown in Figure~\ref{fig:system_GUI}.
%
The GUI is part of the main script but is described separately. 
%
It shows the output of the different workers, including video streams of the cameras, providing various options to control the system configuration.
%
The Matlab command window is made visible in the bottom centre so that messages (e.g. exceptions) can be monitored during the development and use of the system.


\begin{figure} [htb]
\centering
\includegraphics[width=0.95\textwidth]{system_GUI1_labelled.jpg}
\caption{Graphical User Interface of the system with labels of the different elements. 
%See Section~\ref{sect:GUI} for further details. 
Picture originally appearing in \cite{drone20thesis} and published in \cite{drone20icpr}. Reprinted with permission.}
\label{fig:system_GUI}
\end{figure}

The ADS-B presentation area (left) consists of a PPI-type (Plan Position Indicator) display and an altitude display. Besides presenting the ADS-B targets, the PPI display also shows system information. The solid green line is the main direction of the system relative to the north. The other green lines present the field of motion of the pan/tilt platform (dashed) and the field of view of the fish-eye lens camera (dotted).
%
The actual direction of the pan/tilt platform is presented with a solid red line, and the field of view of the thermal infrared and video cameras are represented using dashed red lines. If the fish-eye lens camera worker tracks any object, its direction is indicated by a solid cyan line.
%
ADS-B targets are presented using the class label colours of Table~\ref{tab:output-classes}, together with the track history plots. The altitude information is presented in a logarithmic plot to make the lower altitude portion more prominent.

The area directly below the ADS-B presentation is the control panel (seen in better detail in Figure~\ref{fig:gui-fig18-fig19}a).
%
Starting from the top left corner, we have radio buttons for the range settings of the ADS-B PPI and altitude presentations. Next is the number of ADS-B targets currently received, and below that, the set orientation angle relative to the north. The ``Close GUI'' button is used to shut down the main script and the workers.
%
The GPS-position presentation changes colour to green when the GPS receiver has received a correct position after pressing the ``Get GPS-pos'' button. Pressing the ``Set ori-angle'' button opens an input dialogue box so that the orientation angle of the system can be entered. Below that two buttons for switching the detectors between running and idle mode and a choice to display the raw Fcam image or the moving object mask only (not shown).
%
The servo settings can be controlled with the buttons of the middle column.
%
To aid the Fcam in finding targets, the pan/tilt can be set to move in two different search patterns. One where the search is done from side to side using a static elevation of 10$^{\circ}$, so that the area from the horizon up to 20$^{\circ}$ is covered, and another one where the search is done with two elevation angles to increase the coverage.
%
The pan/tilt platform can also be controlled by the elevation and azimuth angles from one of the workers. This is set by the ``assign'' buttons of the fourth column, placed in priority order from top to down. The ``IR\&V assign'' setting means that a target has to be detected by both the IRcam and Vcam workers, and if so, the pan/tilt platform is controlled by the angular values from the IRcam worker.
%
The rightmost column of the control panel shows the status information regarding the performance in FPS of the workers and the elevation and azimuth angles of the Fcam worker target (if such a target exists). The status displays are red if the worker is not connected, yellow if the detector is idle, and green if it is running.

The results panel (seen in better detail in Figure~\ref{fig:gui-fig18-fig19}b) features settings for the sensor fusion and presents the workers and system results.
%
The first column (servo controller) indicates the source of information currently controlling the servos of the pan/tilt platform.
%
In the bottom left corner, the angles of the servos are presented. 
%
The settings for the sensor fusion and the detection results presentation are found in the middle of the panel. 
%
The information in the right part of the panel is the current time and the position of the system. The system elevation and azimuth relative to the north are also presented here. Note the difference in azimuth angle compared to the bottom left corner where the system internal angle of the pan/tilt platform is presented.
%
The last part of the results panel (bottom right corner) presents offset angles for the ADS-B target if one is in the field of view of the thermal infrared and video cameras. These values are used to detect systematic errors in the orientation of the system. The sloping distance to the ADS-B target is also presented. 


\begin{figure} [htb]
\centering
\includegraphics[width=0.95\textwidth]{gui-fig18-fig19.jpg}
\caption{Detail of the control panel and results panel of the Graphical User Interface. In the control panel, note a bird detected and tracked by the IRcam worker. Pictures originally appearing in \cite{drone20thesis}.}
\label{fig:gui-fig18-fig19}
\end{figure}

\subsection{Database}
\label{sect:database}

A dataset has also been collected to accomplish the necessary training of the detectors and classifiers. The annotation has been done so that others can inspect, edit, and use the dataset. The fact that the datasets for the thermal infrared and the visible video sensors are collected under the same conditions and using the same targets ensures that a comparison between the two sensor types is well-founded.
%
%During this research, we have also captured a dataset to train and evaluate our system. 
%
The dataset is fully described in \cite{DiBdataset} and available at \cite{svanstrom_fredrik_2020_5500576}.
%
The videos and audios are recorded at locations in and around Halmstad Airport (IATA/ICAO code: HAD/ESMT), Gothenburg City Airport (GSE/ESGP) and Malmö Airport (MMX/ESMS).
%
Three different drones are used to collect and compose the dataset. These are of the following types: Hubsan H107D+, a small-sized first-person-view (FPV) drone; the high-performance DJI Phantom 4 Pro; and finally, the medium-sized kit drone DJI Flame Wheel. This can be built as a quadcopter (F450) or a hexacopter configuration (F550). The version used in this work is an F450 quadcopter. All three types can be seen in Figure~\ref{fig:db_drones}.
%
These drones differ a bit in size, with Hubsan H107D+ being the smallest, having a side length from motor to motor of 0.1 m. The Phantom 4 Pro and the DJI Flame Wheel F450 are larger with 0.3 and 0.4	m motor-to-motor side lengths, respectively.
%
To comply with regulations (drones must be flown within visual range), the dataset is recorded in daylight, even if the thermal infrared or acoustic sensors could be effective at night. 


\begin{figure} [htb]
\centering
\includegraphics[width=0.85\textwidth]{db_drones.png}
\caption{The three drones of our dataset. %Left: Hubsan H107D+. Middle: DJI Phantom 4 Pro. Right: DJI Flame Wheel F450. 
Pictures were originally appearing in \cite{drone20thesis} and published in \cite{drone20icpr} and \cite{DiBdataset}. Reprinted with permission.}
\label{fig:db_drones}
\end{figure}


The videos and audio files are cut into ten-second segments to facilitate annotation. 
%
The acquisition was performed during the drastic reduction in air traffic due to the COVID19 pandemic. Therefore, to get a more comprehensive dataset, both in terms of aircraft types and sensor-to-target distances, the data is complemented with non-copyrighted material of \cite{[VIRTUALAIRFIELD]}. %, in particular 11 plus 38 video clips in the airplane and helicopter categories, respectively.
%
Overall, the dataset contains 90 audio clips and 650 videos (365 IR and 285 visible, of ten seconds each), with a total of 203328 annotated images (publicly available).
%
The thermal infrared videos have a resolution of 320$\times$256 pixels, and the visible videos have 640$\times$512.
%
%
The greatest sensor-to-target distance for a drone in the dataset is 200 m.
%
Audio files consist of 30 ten-second clips of each of the three output audio classes (Table~\ref{tab:output-classes}), and the amount of videos among the four output video classes is shown in Table~\ref{tab:db-stats}.
%
The background sound class contains general background sounds recorded outdoor in the typical deployment environment of the system, and also includes some clips of the sounds from the servos moving the pan/tilt platform.


\begin{table}[htb]
\caption{Distribution of the thermal infrared and visible videos. Each video has approximately 10 seconds. Reprinted with permission from \cite{drone20icpr}. \label{tab:db-stats}}

%\small
%\footnotesize
%\scriptsize
\begin{center}
\begin{tabular}{cccccccccc}

%\multicolumn{5}{c}{} \\

\multicolumn{1}{c}{} & \multicolumn{4}{c}{thermal infrared (365)}  & & \multicolumn{4}{c}{visible (285)} \\  \cline{2-5} \cline{7-10}

bin & airpl. & bird & drone & helicop.  & &  airpl. & bird & drone & helicop. \\ \hline
Close & 9 & 10 & 24  & 15  & &  17 & 10 & 21  & 27 \\
Medium & 25 & 23 & 94 & 20  & & 17 & 21 & 68 & 24 \\
Distant & 40 & 46 & 39 & 20  & & 25 & 20 & 25 & 10 \\ \hline

\end{tabular}

\end{center}

\end{table}
\normalsize

The video part of the database is divided into three category bins: Close, Medium and Distant.
%
This is because one of our aims is to explore performance as a function of the sensor-to-target distance.
%
The borders between these bins are chosen to follow the industry-standard Detect, Recognize and Identify (DRI) requirements \cite{[DRI]}, building on the Johnson criteria \cite{[Chevalier16]}, as shown in Figure~\ref{fig:DRI}.
%
The Close bin is from 0 m to a distance where the target is 15 pixels wide (the requirement for `identification', e.g. explicitly telling the specific drone model, aircraft, helicopter, bird, etc.). The Medium bin is from where the target is from 15 down to 5 pixels (`recognition' of the target, e.g. a drone, an aircraft, a helicopter... albeit without the possibility of identifying the model), and the Distant bin is beyond that (`detection', e.g. there is something).
%


\begin{figure} [htb]
\centering
\includegraphics[width=0.7\textwidth]{DRI-fig24.jpg}
\caption{The DRI requirements (from \cite{[DRI]}).}
\label{fig:DRI}
\end{figure}








\begin{table}[htb]
\caption{Results with the thermal infrared sensor (confidence threshold and IoU requirement of 0.5). The average of the three F1-scores is 0.7601. Reprinted with permission from \cite{drone20icpr}. \label{tab:results-IRcam}}

%\small
%\footnotesize
%\scriptsize
\begin{center}
\begin{tabular}{cccccc}

%\multicolumn{6}{c}{} \\

\multicolumn{1}{c}{} & \multicolumn{5}{c}{distance bin: \textbf{CLOSE}}   \\ \cline{2-6}

 & airplane & bird & drone  & helicopter  &  average \\ \hline

\textbf{Precision} & 0.9197 & 0.7591 & 0.9159 & 0.9993& 0.8985 \\
\textbf{Recall} & 0.87367 & 0.85087 & 0.87907 & 0.87927 & 0.8706   \\ \hline
\textbf{F1-score} & & & & & 0.88447   \\
\hline


\multicolumn{6}{c}{} \\

\multicolumn{1}{c}{} & \multicolumn{5}{c}{distance bin: \textbf{MEDIUM}}   \\ \cline{2-6}

 & airplane & bird & drone  & helicopter  &  average \\ \hline

\textbf{Precision} &  0.82817 & 0.50637 & 0.89517 & 0.95547 & 0.7962  \\
\textbf{Recall} &  0.70397 & 0.70337 & 0.80347 & 0.83557 & 0.7615 \\ \hline
\textbf{F1-score} & & & & & 0.77857   \\ \hline


\multicolumn{6}{c}{} \\

\multicolumn{1}{c}{} & \multicolumn{5}{c}{distance bin: \textbf{DISTANT}}   \\ \cline{2-6}

 & airplane & bird & drone  & helicopter  &  average \\ \hline

\textbf{Precision} &  0.78227 & 0.61617 & 0.82787 & 0.79827 & 0.7561  \\
\textbf{Recall} &  0.40437 & 0.74317 & 0.48367 & 0.45647 & 0.5218 \\ \hline
\textbf{F1-score} & & & & & 0.61757   \\ \hline

\end{tabular}

\end{center}

\end{table}
\normalsize




\begin{table}[htb]
\caption{Results with the thermal infrared sensor (confidence threshold 0.8, IoU requirement 0.5). Table originally appearing in \cite{drone20thesis}. \label{tab:results-IRcam-th0.8}}

%\small
%\footnotesize
%\scriptsize
\begin{center}
\begin{tabular}{ccccc}

\multicolumn{1}{c}{} & \multicolumn{3}{c}{distance bin} & \multicolumn{1}{c}{Average}  \\ \cline{2-4}

 & close & medium & distant &  \\ \hline

\textbf{Precision} & 0.9985 & 0.9981 & 1.0000 & 0.9987\\
\textbf{Recall} & 0.2233 & 0.1120 & 0.0019 & 0.1124 \\ \hline

\end{tabular}

\end{center}

\end{table}
\normalsize









%%%%%%%%%%%%%%%%%%%%%%%%%%%%%%%%%%%%%%%%%%
\section{Results}
\label{sect:results}

We provide evaluation results of the individual sensors in terms of precision, recall and F1-score.
%
Precision is the fraction of the total number of true positive detections. In other words, it measures how many of the detected objects are relevant.
%
Recall is the fraction of the total number of labelled samples in the positive class that are true positive. In other words, how many of the relevant objects are detected.
%
With the video detectors, these metrics can be obtained using the Matlab function \texttt{bboxPrecisionRecall}, which measures the accuracy of bounding box overlap between detected and ground truth boxes.
%
Since we also have confidence scores, the \texttt{evaluateDetectionPrecision} function can be used to plot precision curves as a function of the recall value.
%
%
After that, we perform sensor fusion experiments, a direction barely followed in the related literature \cite{Diamantidou19,Shi18,Unlu19}.
%
To carry out experiments, two disjoint sets of videos were created for training and testing (i.e. a video selected for the training set is not used in the testing set). 
%
The training set comprises 120 infrared and 120 visible clips (10 for each class and target bin per spectrum), resulting in 37428 infrared and 37519 visible frames.
%
The evaluation set comprises 60 infrared and 60 visible clips (5 for each class and target bin per spectrum), resulting in 18691 infrared and 18773 visible frames.
%
Since the duration of each clip is roughly the same ($\sim$10 seconds), the amount of samples per class and target bin is approximately balanced.
%
%a part of the video dataset is set aside for evaluation (120 infrared and 120 visible clips, 5 for each class and target bin per spectrum). 
%
%Since the videos do not have all exactly the same duration (10.4 seconds on average), the total number of images per class varies slightly, as shown in Table XX.
%
%Among the remaining videos not set aside for evaluation, 240 are selected as evenly distributed as possible to create the training set.
%
For the audio classifier, five 10-second clips from each output category are selected for evaluation, and the remaining clips for training. 
%
Since the audio classifier processes a one-second input buffer, the clips cut into that length, with an overlap of 0.5 seconds, resulting in 20 samples per clip.
%
This results in 297 clips in the evaluation set, 99 from each class.

In evaluating the YOLOv2 detector, it provides an array of class labels, detection confidence, and bounding boxes of detected objects.
%
Here, not only the classification label but also the placement of the bounding box must be taken under consideration.
%
When it produces multiple bounding boxes for the same object, only the strongest one is used, chosen as the box with the highest IoU (Intersection Over Union) with the annotations in the training data.
%
To assign a box to a class, it must also have a minimum IoU with the training object it is supposed to detect.
%
In related works, an IoU of 0.5 is usually employed \cite{Park17,Saqib17,Aker17}, although a lower IoU of 0.2 is used in \cite{Schumann17}.
%
In this work, we will stick to 0.5.
%
A threshold to the detection confidence can also be imposed, so bounding boxes with small confidence can be rejected, even if their IoU is above 0.5.




\begin{figure} [htb]
\centering
\includegraphics[width=0.6\textwidth]{F1score-IRcam.png}
\caption{F1-score with the thermal infrared sensor as a function of detection threshold, using all the 18691 images in the evaluation dataset. Picture originally appearing in \cite{drone20thesis}.}
\label{fig:results-IRcam-F1}
\end{figure}


\begin{figure} [htb]
\centering
\includegraphics[width=0.85\textwidth]{precision-recall-IRcam.png}
\caption{Precision and recall curves with the thermal infrared sensor. The achieved values with a detection threshold of 0.5 are marked by stars. Pictures originally appearing in \cite{drone20thesis}.}
\label{fig:results-IRcam-PR}
\end{figure}


\subsection{Thermal Infrared Camera (IRcam)}

The precision, recall and F1-score of the IRcam worker detector with a confidence threshold set to 0.5 and an IoU requirement of 0.5 are shown in Table~\ref{tab:results-IRcam}.
%
We can observe that the precision and recall values are well balanced.
%
Altering the setting to a higher decision threshold of 0.8 leads to higher precision at the cost of a lower recall value, as shown in Table~\ref{tab:results-IRcam-th0.8}. The drop in recall with increasing sensor-to-target distance is also prominent.




To further explore the detection threshold setting, we run the evaluation with values from 0.1 up to 1.0 in steps of 0.1. The results in the form of an F1-score are shown in Figure~\ref{fig:results-IRcam-F1}.
%
Using not only the bounding boxes and class labels but also the confidence scores, the detector can be evaluated in the form of precision vs recall curves as well (Figure~\ref{fig:results-IRcam-PR}).
%
Note that the average precision results output from the Matlab \texttt{evaluateDetectionPrecision} function is defined as the area under the precision vs recall curve. Hence it is not the same as the actual average precision values of Table~\ref{tab:results-IRcam}.
%
To distinguish that we mean the area under the curve, we denote this as AP in Figure~\ref{fig:results-IRcam-PR}, just as in the original YOLO paper \cite{Redmon17}.
%
This is also  the definition used in \cite{Saqib17}, and to further adopt the notation of these papers, we denote the mean AP taken over all classes as the mAP, which is given in Table~\ref{tab:results-IRcam-mAP}.
%



\begin{table}[htb]
\caption{Mean values over all classes of the area under the precision vs recall curve (mAP) of the thermal infrared sensor. Table originally appearing in \cite{drone20thesis}. \label{tab:results-IRcam-mAP}}

%\small
%\footnotesize
%\scriptsize
\begin{center}
\begin{tabular}{ccccc}

\multicolumn{1}{c}{} & \multicolumn{3}{c}{distance bin} & \multicolumn{1}{c}{Average}  \\ \cline{2-4}

 & close & medium & distant &  \\ \hline

\textbf{mAP} & 0.8704 & 0.7150 & 0.5086 & 0.7097 \\ \hline

\end{tabular}

\end{center}

\end{table}
\normalsize


The choice of the detection threshold will affect the achieved precision and recall values. The stars in Figure~\ref{fig:results-IRcam-PR} indicate the results with a detection threshold of 0.5 (as reported in Table~\ref{tab:results-IRcam}). We can conclude that such a threshold results in a balanced precision-recall combination near the top right edge of the respective curves. 
%
Compare this balanced behaviour to the  precision and recall values obtained with a detection threshold of 0.8 (Table~\ref{tab:results-IRcam-th0.8}).
%
From observations of the behaviour when running the drone detection system, we can also point out that a common source of false alarms of the thermal infrared sensor is small clouds and edges of large clouds lit up by the sun. An example of this can be seen in Figure~\ref{fig:results-IRcam-false-target}.

\begin{figure} [htb]
\centering
\includegraphics[width=0.85\textwidth]{IRcam-false-fig29.jpg}
\caption{A false target of the thermal infrared sensor caused by a small cloud lit by the sun. Pictures originally appearing in \cite{drone20thesis}.}
\label{fig:results-IRcam-false-target}
\end{figure}

In comparing our work with other research using thermal infrared sensors, results can be found in \cite{ANDRASI2017183}. 
%
However, the sensor used in that article to detect drones (up to a distance of 100 m) has a resolution of just 80$\times$60 pixels, and the system does not involve any form of machine learning feature (the task is done by a human looking at the output video stream). 
%
%In that paper, three different drones were used, and a DJI Phantom 4 was detected by the setup up to a distance of 51 m.
%
In \cite{wang_chen_choi_kuo_2019}, curves for the precision and recall of a machine learning-based thermal detector are presented. It is stated that the video clips used for training and evaluation have a frame resolution of 1920$\times$1080. Unfortunately, the paper fails to mention if this is also the resolution of the sensor. Neither is the input size of the detection network specified in detail, other than the images are re-scaled so that the shorter side has 600 pixels.
%
The most substantial results to relate to in \cite{wang_chen_choi_kuo_2019} is that since the system also contains a video camera with the same image size as the thermal one, the authors can conclude that the thermal drone detector performs over 8\% better than the video detector.


\subsection{Video Camera (Vcam)}

The results of the previous sub-section will be compared to the results of the video detector in what follows.
%
To enable comparison, the same methods and settings as the infrared camera are used here, leading to the precision, recall and F1-score results of Table~\ref{tab:results-Vcam} with a confidence threshold and IoU requirement of 0.5.
%
These results differ no more than 3\% from the results of the thermal infrared detector. Recall that the input layers of the YOLOv2 detectors are different, and hence, the resolution of the visible sensor is 1.625 higher than the thermal infrared detector (416$\times$416 vs 256$\times$256).
%
So even with a lower resolution and using images in grey-scale and not in colour, the thermal infrared sensor performs as well as the visible one. 
%
This is in line with \cite{ANDRASI2017183}, where the infrared detector outperforms the visible one when the image size is the same, although in our case, this happens even with a smaller image size.



\begin{table}[htb]

\caption{Results with the visible camera (confidence threshold and IoU requirement of 0.5). The average of the three F1-scores is 0.7849. Reprinted with permission from \cite{drone20icpr}. \label{tab:results-Vcam}}
%\small
%\footnotesize
%\scriptsize
\begin{center}
\begin{tabular}{cccccc}

%\multicolumn{6}{c}{} \\

\multicolumn{1}{c}{} & \multicolumn{5}{c}{distance bin: CLOSE}   \\ \cline{2-6}

 & airplane & bird & drone  & helicopter  &  average \\ \hline

\textbf{Precision} & 0.8989 & 0.8284 & 0.8283 & 0.9225 & 0.8695 \\
\textbf{Recall} & 0.7355 & 0.7949 & 0.9536 & 0.9832 & 0.8668 \\ \hline
\textbf{F1-score} & & & & & 0.8682   \\
\hline


\multicolumn{6}{c}{} \\

\multicolumn{1}{c}{} & \multicolumn{5}{c}{distance bin: MEDIUM}   \\ \cline{2-6}

 & airplane & bird & drone  & helicopter  &  average \\ \hline

\textbf{Precision} & 0.8391 & 0.7186 & 0.7710 & 0.9680 & 0.8242 \\
\textbf{Recall} & 0.7306 & 0.7830 & 0.7987 & 0.7526 & 0.7662 \\ \hline
\textbf{F1-score} & & & & & 0.7942   \\ \hline


\multicolumn{6}{c}{} \\

\multicolumn{1}{c}{} & \multicolumn{5}{c}{distance bin: DISTANT}   \\ \cline{2-6}

 & airplane & bird & drone  & helicopter  &  average \\ \hline

\textbf{Precision} & 0.7726 & 0.6479 & 0.8378 & 0.6631 & 0.7303 \\
\textbf{Recall} & 0.7785 & 0.7841 & 0.5519 & 0.5171 & 0.6579 \\ \hline
\textbf{F1-score} & & & & & 0.6922   \\ \hline

\end{tabular}

\end{center}

\end{table}
\normalsize


Just as for the thermal infrared camera, we can also explore the effects of the detection threshold setting. This can be seen in Figure~\ref{fig:results-Vcam-F1}.
%
The precision vs recall curves of the video camera images for the different target classes and distance bins are shown in Figure~\ref{fig:results-Vcam-PR}.
%
In a similar vein, a detection threshold of 0.5 results in a balanced precision-recall combination near the top right edge of the respective curves.
%
Notably, when inspecting the precision-recall curves, the video camera detector performs outstandingly when it comes to distant airplanes. This has its explanation in that such targets often presents a very large signature consisting not only of the airplane itself but also contrails behind it.
%
Also, calculating the mAP from these results, we obtain Table~\ref{tab:results-Vcam-mAP}, with an average of 0.7261.
%
Once again, it is not far from the 0.7097 mAP of the thermal infrared detector of the previous sub-section. 



\begin{figure} [htb]
\centering
\includegraphics[width=0.6\textwidth]{F1score-Vcam.png}
\caption{F1-score with the visible camera as a function of detection threshold, using all the 18773 images in the evaluation dataset. Picture originally appearing in \cite{drone20thesis}.}
\label{fig:results-Vcam-F1}
\end{figure}


\begin{figure} [htb]
\centering
\includegraphics[width=0.85\textwidth]{precision-recall-Vcam.png}
\caption{Precision and recall curves with the visible camera. The achieved values with a detection threshold of 0.5 are marked by stars. Pictures originally appearing in \cite{drone20thesis}.}
\label{fig:results-Vcam-PR}
\end{figure}


\begin{table}[htb]
\caption{Mean values over all classes of the area under the precision vs recall curve (mAP) of the visible camera. Table originally appearing in \cite{drone20thesis}. \label{tab:results-Vcam-mAP}}

%\small
%\footnotesize
%\scriptsize
\begin{center}
\begin{tabular}{ccccc}

\multicolumn{1}{c}{} & \multicolumn{3}{c}{distance bin} & \multicolumn{1}{c}{Average}  \\ \cline{2-4}

 & close & medium & distant &  \\ \hline

\textbf{mAP} & 0.8474 & 0.7477 & 0.5883 & 0.7261 \\ \hline

\end{tabular}

\end{center}

\end{table}
\normalsize




The most frequent problem with the video detector is the auto-focus feature of the video camera. For this type of sensor, clear skies are not the ideal weather, but rather scenery with objects that can help the camera to set the focus correctly. However, note that this fact is not heavily affecting the evaluation results of the visible camera detector performance, as presented above, since only videos where the objects are seen clearly, and hence are possible to annotate, are used.
%
Figure~\ref{fig:results-Vcam-autofocus-fail} shows an example of when the focus is set wrongly so that only the thermal infrared worker detects the drone.
%
This issue justifies the multi-sensor approach followed here. Also, cameras not affected by this issue (such as the thermal infrared or the fish-eye) could be used to aid the focus of the video camera. % towards the area where the object is appearing.


%The result is also ahead to what is presented in \cite{Saqib17}, where a mAP of 0.66 achieved, albeit using a detector with drones as the only output class and giving no information about the sensor-to-target distances.
%
Comparing the results to other papers, our results are ahead of what is presented in \cite{Saqib17}, where a mAP of 0.66 was achieved, albeit using a detector with drones as the only output class and giving no information about the sensor-to-target distances.
%
Also, we see that the YOLOv2 detector in \cite{Park17} achieves an F1-score of 0.728 with the same detection threshold and IoU-requirement. This F1-score is just below the results of the thermal infrared and video camera workers.
%
However, one notable difference lies in that the detector in \cite{Park17} has only one output class. This fact could confirm the doctrine of this work, i.e. that the detectors should also be trained to recognize objects easily confused for being drones. Unfortunately, there is no notation of the sensor-to-target distance other than that ``75\% of the drones have widths smaller than 100 pixels''. Since the authors implement an original YOLOv2 model from darknet, it is assumed that the input size of the detector is 416$\times$416 pixels.
%
A YOLOv2 architecture with an input size of 480$\times$480 pixels is implemented in \cite{Aker17}. The detector has two output classes, birds and drones. Based on the presented precision-recall curve, a precision and recall of 0.9 can be achieved simultaneously.
%
%
To summarize this comparison, we provide the performance of the thermal infrared and video camera detectors together with the reported comparable results in Table~\ref{tab:results-IRcam-Vcam-SOA}. The table also shows the output classes used.




\begin{table}[htb]
\caption{Results from the related works and the thermal infrared and video camera detectors. A = Airplane, B = Bird, D = Drone and H = Helicopter. Table originally appearing in \cite{drone20thesis}. \label{tab:results-IRcam-Vcam-SOA}}

%\small
%\footnotesize
%\scriptsize
\begin{center}
\begin{tabular}{ccccccc}

\multicolumn{1}{c}{Work} & \multicolumn{1}{c}{Sensor} & \multicolumn{4}{c}{Results} & \multicolumn{1}{c}{Classes}   \\ \cline{3-6}

 & & Precision & Recall & F1-score & mAP &  \\ \hline

%Wang et al. \cite{wang_chen_choi_kuo_2019} & Thermal &  &  &  &  &   \\ 

%Wang et al. \cite{wang_chen_choi_kuo_2019} & Visible &  &  &  &  &   \\ 
 
Park et al. \cite{Park17} & Visible &  &  & 0.73 & & D  \\ 

Liu  et al. \cite{Liu_2018} & Visible & 0.99 & 0.80  &  & & A, D, H   \\ 

Saqib et al. \cite{Saqib17} & Visible &  &  &  & 0.66 & D   \\ 

Aker and Kalkan. \cite{Aker17} & Visible & $\sim$0.9 & $\sim$0.9 &  &  & B, D  \\ 

\textbf{thermal infrared} & Thermal & 0.82 & 0.72 & 0.76 & 0.71 & A, B, D, H  \\ 

\textbf{video camera} & Visible & 0.81 & 0.76 & 0.78 & 0.73 & A, B, D, H   \\  

\hline

\end{tabular}

\end{center}

\end{table}
\normalsize



\begin{figure} [htb]
\centering
\includegraphics[width=0.85\textwidth]{Vcam-autofocus-fail-fig33.jpg}
\caption{A drone is detected only by the thermal infrared sensor since the auto-focus of the video camera is wrongly set. Pictures originally appearing in \cite{drone20thesis}.}
\label{fig:results-Vcam-autofocus-fail}
\end{figure}


%\begin{figure} [htb]
%\centering
%\includegraphics[width=0.7\textwidth]{Vcam-standalone-fig34.jpg}
%\caption{Airplanes, a helicopter and birds detected when running the visible camera detector as a stand-alone 5application on an aircraft video from \cite{[VIRTUALAIRFIELDhttps://www.overleaf.com/project]}. Note the FPS-performance.}
%\label{fig:results-Vcam-standalone}
%\end{figure}


\subsection{Fish-Eye Camera Motion Detector (Fcam)}

The fish-eye lens camera is included to be able to cover a larger airspace volume than covered by just the field of view of the thermal infrared and video camera. However, the drone detection system does not totally rely on the fish-eye worker to detect objects of interest since the system also includes the search programs moving the pan/tilt platform when no other targets are detected. The search programs can easily be turned on or off using the control panel of the GUI.

It was initially observed that the foreground/background-based motion detector of the fish-eye worker was sensitive to objects moving with the wind, as shown in Figure~\ref{fig:results-Fcam-false-targets}.
%
The false targets have been mitigated by extensive tuning of the image processing operations, the GMM foreground detector \cite{gmmmatlab}, the blob analysis settings and the parameters of the multi-object Kalman filter tracker from \cite{kalmanmatlab} as follows. All other parameters or computation methods not mentioned here are left at the default values described in \cite{gmmmatlab},\cite{kalmanmatlab}.

\begin{itemize}


\item Some of the most critical points in this were first to remove the \texttt{imclose}\footnote{Morphological closing.} and \texttt{imfill}\footnote{Flood-fill of holes, i.e. background pixels that cannot be reached by filling in the background from the edge of the image.} functions that were initially implemented after the \texttt{imopen}\footnote{Morphological opening.} function in the image processing operations.


\item To only detect small targets, in the blob analysis settings of the \texttt{BlobAnalysis} function, a maximum blob area of 1000 pixels was implemented.

%In the blob analysis settings, the minimum blob area was reduced, so that smaller objects are detectable, and a maximum blob area of 1000 pixels was also implemented.

%To only detect small targets the parameter 'MaximumBlobArea' of the function 'BlobAnalysis' was set to 1000.


\item The parameters of the \texttt{ForegroundDetector} function were changed so that the model would adapt faster to changing conditions, and hence react quicker. 
%
The number of initial video frames for training the background model was reduced from 150 (default) to 10, and the learning rate for parameter updates increased from 0.005 (default) to 0.05. 
%
Furthermore, to reduce false alarms, the threshold to determine the background model (i.e. minimum possibility for pixels to be considered background values) was increased from 0.7 (default) to 0.85.
%
The number of Gaussian modes in the mixture model is left to the default value of 5.

%So the parameters 'NumTrainingFrames' and 'LearningRate'  were set to 10 and 0.05 respectively (default is 150 and 0.005). To reduce false alarms the parameter 'MinimumBackgroundRatio' was set to 0.85 instead of the default value of 0.7. 
%
%Furthermore, the minimum background ratio setting of the foreground detector was increased together with the learning rate, so that the GMM more quickly adapts to changing conditions.


\item Tuning the parameters of the Kalman filter multi-object tracker was also important. These were altered to make the tracker slower to start new tracks and quicker to terminate them if no moving objects were present at the predicted positions.
%
The motion model in the \texttt{configureKalmanFilter} function was chosen to be of the type ``constant velocity''.
%
The initial location of unassigned objects is given by the centroid computed by the foreground detector.
%
The initial estimate error was changed from [200, 50] (default) to [200, 200]. This specifies the variance of the initial estimates of location and velocity of the tracked object, with the initial state estimation error covariance matrix built as 2$\times$2 diagonal matrix with these two values on the main diagonal.
%
Larger values here help the filter to adapt to the detection results faster. %
This affect the first few detections, after which the estimate error is obtained from the noise and input data. The function assumes a zero initial velocity at the position of the initial location.
%
The motion noise was changed from [100, 25] (default) to [50, 50], with the process noise covariance matrix built as a 2$\times$2 diagonal matrix with these two values on the main diagonal. These values specify the tolerance (variance) for the deviation on location and velocity, compensating for the difference between the actual motion and that of the constant velocity model.
%
Here the changes were made since we must assume that the initial estimate error and the motion noise are uniform in x- and y-dimensions. 
%
The measurement noise covariance was increased from 10 (default) to 100, specifying the variance inaccuracy of the detected location.
%
Increasing this value enables the Kalman filter to remove more noise from the detections. This parameter was changed by trial and error.
%
Both the values of the motion noise and measurement noise stay constant.


%Regarding the parameters of the Kalman-filter: The motion model was chosen to be of the type 'ConstantVelocity' and the parameters 'InitialEstimateError, MotionNoise, MeasurementNoise' were set to [200, 200], [50, 50], 10. In the example described in the matlab help-page this was [200, 50], [100, 25], 100. Here the changes were made since we must assume that the initial estimate error and the motion noise are uniform in x- and x'-dimensions. The measurement noise parameter was changed by trial and error.

% Create a Kalman filter object.
%kalmanFilter = configureKalmanFilter('ConstantVelocity', ...
 %    centroid,       [200, 50],           [100, 25],  100);
 %    InitialLocation,InitialEstimateError,MotionNoise,MeasurementNoise
 %                    [200, 200],          [50, 50],   10

\end{itemize}


%Some of the most critical points in this was first to remove the \texttt{imclose}\footnote{Morphological closing.} and \texttt{imfill}\footnote{Flood-fill of holes, i.e. background pixels that cannot be reached by filling in the background from the edge of the image.} functions that were initially implemented after the \texttt{imopen}\footnote{Morphological opening.} function in the image processing operations.
%
%Furthermore, the minimum background ratio setting of the foreground detector was increased together with the learning rate, so that the GMM more quickly adapts to changing conditions. In the blob analysis settings, the minimum blob area was reduced, so that smaller objects are detectable and a maximum blob area was also implemented.
%
%Tuning the parameters of the Kalman filter multi-object tracker was also important. These were altered to make the tracker slower to start new tracks and quicker to terminate them if no moving objects were present at the predicted positions.


%
%\textbf{We employ the default parameters of the function (Gaussians in the mixture model=5, 
%threshold to determine background model=0.7, 
%initial video frames for training the background model=150, 
%learning rate for parameter updates across time=0.005).
%}


\begin{figure} [htb]
\centering
\includegraphics[width=0.85\textwidth]{Fcam-false-targets-fig35.jpg}
\caption{False targets in a fish-eye camera image. Picture originally appearing in \cite{drone20thesis}.}
\label{fig:results-Fcam-false-targets}
\end{figure}


With an image size of 1024$\times$384, the fish-eye worker moving object detector and tracker has been found, during the evaluation sessions, to be an effective way to assign objects to the pan/tilt platform up to a distance of 50 m against drones. 
%
Beyond this distance, the drone appears 
%falls into the ``distant'' category \cite{DiBdataset}, being theoretically 
so small (in pixels) that it is deleted by the \texttt{imopen} function.
%
The maximum resolution of the fish-eye camera employed is 3264$\times$2448, so a greater detection range should theoretically be achievable. However, using a higher resolution is also more demanding in computational resources, leading to a reduction in the FPS performance for the fish-eye worker and the other workers. Since the fish-eye lens camera is also complemented by the search program, where the pan/tilt platform can be controlled by the output of the other workers, the choice has been made to settle with this resolution and the limited detection range that follows from this.
%
Figure~\ref{fig:fig36-37}a shows a drone tracked by the fish-eye worker. At this moment, the pan/tilt platform is controlled by the output of the fish-eye worker. Just a moment later, as seen in Figure~\ref{fig:fig36-37}b, the drone is detected by the thermal infrared and video camera workers, and the thermal infrared worker output therefore controls the pan/tilt platform.

\begin{figure} [htb]
\centering
\includegraphics[width=0.95\textwidth]{fig36-37.png}
\caption{Tracking of a drone with different workers. Pictures originally appearing in \cite{drone20thesis}.
%Left: a drone tracked by the fish-eye worker. Right: the drone tracked by the thermal infrared worker just a second later (note the timestamp).
}
\label{fig:fig36-37}
\end{figure}

\subsection{Microphone}

The precision and recall results of the different classes with this detector are shown in Table~\ref{tab:results-audio}. 
%
Against a drone, the practical range of the acoustic sensor is 35-45 m, depending on how the drone is flying. This is in parity with the 50 m of \cite{Park15}, but far from the 160 m against a F450 drone reported in
\cite{Busset15} with its much more complex microphone configuration (a 120 elements microphone array).


\begin{table}[htb]
\caption{Results with the audio detector. Reprinted with permission from \cite{drone20icpr}. \label{tab:results-audio}}

%\small
%\footnotesize
%\scriptsize
\begin{center}
\begin{tabular}{ccccc}

%\multicolumn{6}{c}{} \\

\multicolumn{5}{c}{}    \\ \cline{2-5}

 & drone & helicopter & background  & average \\ \hline

\textbf{Precision} & 0.9694 & 0.8482 & 0.9885 & 0.9354 \\
\textbf{Recall} & 0.9596 & 0.9596 & 0.8687 & 0.9293 \\ \hline
\textbf{F1-score} & & & & 0.9323   \\
\hline

\end{tabular}

\end{center}

\end{table}
\normalsize



Our average F1-score is 0.9323, which is higher compared to \cite{Jeon17}, which also uses MFCC features. Out of the three classifier types tested in \cite{Jeon17}, the one comprising a LSTM-RNN performs the best, with a F1-score of 0.6984. The classification problem in that paper is binary (drone or background).
%
Another paper applying MFCC-features and Support Vector Machines (SVM) as classifier is \cite{Bernardini17}, with a precision of 0.983. Five output classes are used (drone, nature daytime, crowd, train passing and street with traffic), and the classification is based on a one-against-one strategy. Hence ten binary SVM classifiers are implemented. 
%
The final output label is then computed using the max-wins voting principle. 
%
The paper \cite{Kim17}, on the other hand, applies the Fast Fourier Transform (FFT) to extract features from the audio signals.
%
The results of the audio detector, together with results reported in other studies, are summarized in Table~\ref{tab:results-audio-SOA}.


\begin{table}[htb]
\caption{Results from the related works and the audio detector. D = Drone, H = Helicopter, BG = Background. The work \cite{Bernardini17} defines four background classes: nature daytime, crowd, train passing and street with traffic. Table originally appearing in \cite{drone20thesis}. \label{tab:results-audio-SOA}}

%\small
%\footnotesize
%\scriptsize
\begin{center}
\begin{tabular}{ccccc}

\multicolumn{1}{c}{Work} & \multicolumn{3}{c}{Results} & \multicolumn{1}{c}{Classes}   \\ \cline{2-4}

 & Precision & Recall & F1-score &   \\ \hline

Kim et al. \cite{Kim17} & 0.88  & 0.83 & 0.85  &  D, BG  \\ 

Jeon et al. \cite{Jeon17} & 0.55 & 0.96 & 0.70  &  D, BG  \\ 

Bernardi et al. \cite{Bernardini17} & 0.98  &  &   &  D, BG  \\ 

\textbf{audio worker} & 0.94 & 0.93 & 0.93 &  D, H, BG  \\ 

\hline

\end{tabular}

\end{center}

\end{table}
\normalsize



\subsection{Radar Module}

From the datasheet of the K-MD2 \cite{RFbeam}, we have that it can detect a person with a Radar Cross Section (RCS) of 1 m$^2$ up to a distance of 100 m. Since we have from \cite{Patel18} that the RCS of the F450 drone is 0.02 m$^2$, it is straightforward to calculate that, theoretically, the F450 should be possible to detect up to a distance of $100 \times \sqrt[4]{{0.02/1}} = 37.6$ m. 
%
Furthermore, given that the micro-doppler echoes from the rotors are 20 dB below that of the drone body, these should be detectable up to a distance of $100 \times \sqrt[4]{{0.02/1 \times 100}} = 11.9$ m.
%

We have observed that the F450 drone is detected and tracked by the K-MD2 up to a maximum distance of 24 m. This is, however, the maximum recorded distance, and it is observed that the drone is generally detected up to a distance of 18 m.
%
Due to the short practical detection range of the radar module, it is not included in the drone detection system of this work.
%
The micro-doppler signature can also be detected at short distances, as shown in Figure~\ref{fig:fig40-41}a. Corresponding range-doppler plots showing the micro-doppler of flying drones can be found in \cite{Drozdowicz16} and \cite{Rahman18} as well.
%
Compared to the echo of a person walking in front of the radar, as we can see in Figure~\ref{fig:fig40-41}b, no such micro-doppler signature is present in that case.


\begin{figure} [htb]
\centering
\includegraphics[width=0.6\textwidth]{fig40-41.jpg}
\caption{Micro-doppler signature of different elements in front of the radar module. Pictures originally appearing in \cite{drone20thesis}.
%Top: micro-doppler signature of the F450 drone. Bottom: the echo of a person walking in front of the radar module.
}
\label{fig:fig40-41}
\end{figure}


\subsection{Sensor Fusion}

We investigated the early or late sensor fusion choice based on whether the sensor data is fused before or after the detection element.
%
By early sensor fusion, we mean to fuse the images from the thermal infrared and video cameras before classification, either the raw images (sensor level fusion), or features extracted from them (feature level). 
%
%before processing them in a detector and classifier. 
%
Late sensor fusion will, in this case, be to combine the output decision from the separate detectors running on each camera stream, weighted by the confidence score of each sensor (decision level fusion). Other existing late fusion approaches entail combining the confidence scores (score level fusion) or the ranked class output of each system (rank level) \cite{FIERREZ201857}.
%

When investigating early sensor fusion at the raw image level in this research, the pixel-to-pixel matching of the images was the biggest issue. Even if this is possible in a static scenario, it turned out to be an unfeasible solution against moving objects with the available equipment due to the small but still noticeable difference in latency between the cameras. 
%
An interesting early sensor fusion method is also found in \cite{Unlu19} where the image from the narrow-angle camera is inserted into the image from the wide-angle camera and then processed by a single YOLO detector. It is unclear how they avoid situations when the inserted image obscures the object found in the wide-angle image.
%
In \cite{Diamantidou19}, they implemented early sensor fusion by concatenating feature vectors extracted from three different sensors (visible, thermal and 2D-radar), which are fed to a trained multilayer perceptron (MLP) classifier.
%
To do so here would likely require much more training data, not only on an individual sensor level but especially on a system level. Such amounts of data have not been possible to achieve within the scope of this work. 

The mentioned issues are the motives for implementing non-trained late sensor fusion in this work.
%
The sensor fusion implemented consists of utilizing the class outputs and the confidence scores of the available sensors in a weighed manner, smoothing their result over time.
%
A weighted fusion approach is shown to be more robust compared to other non-trained techniques such as voting, majority rule, the arithmetic combination of confidences (via, e.g. mean, median or product), or taking the most confident classifier (max rule) \cite{FIERREZ201857}.
%
To carry out the fusion, every time the main script polls the worker’s queues, it puts the results in a 4$\times$4 matrix, organized so that each class is a column and each sensor is a row. 
%
The matrix values depend not only on the class label and the confidence but also on the setting of which sensors to include and the weight of the specific sensor, i.e. how much we trust it at the moment.
%
A new 1$\times$4 matrix is then formed by column-wise sum.
%
This array is in turn placed as a new row in a 10$\times$4 first-in-first-out time-smoothing matrix. Since we have close to ten FPS from the workers, this will be the results of approximately the last second. Once again, the 10 columns are summarized into a 1$\times$4 matrix, and the column with the highest value will be the output system class.
%
The system output confidence is calculated by normalizing the value found to be highest over the number of sensors included at the moment. An additional condition before presenting the system result is that the minimum number of sensors that detects any object must fulfil the GUI setting, as shown in Figure~\ref{fig:fusion-options}.
%
The figure also shows an example of how sensor fusion is enabled in the GUI, including the weight choice for each sensor and the minimum number required.
%
With such a dynamical setting, it is possible to use not only the OR-function, as in \cite{Shi18}, but more sophisticated variants by varying the number of sensors included and required for detection, including their weights.


\begin{figure} [htb]
\centering
\includegraphics[width=0.45\textwidth]{fusion1.png}
\includegraphics[width=0.45\textwidth]{fusion2.png}
\caption{Two examples of sensor fusion results. Pictures were originally appearing in \cite{drone20thesis} and published in \cite{drone20icpr}. Reprinted with permission.}
\label{fig:fusion-options}
\end{figure}


Evaluating the fusion in operational conditions was more challenging than expected since this research was mostly done during the COVID19 pandemic. 
%
Regular flights at the airports considered in this work were cancelled, hence the possibility for a thorough system evaluation against airplanes.
%
Using the screen recorder, it was possible to do a frame-by-frame analysis of a typical drone detection, as shown in Figure~\ref{fig:fusion-results1}a.
%
The ``servo'' column indicates the current servo controlling sensor and the ``Fcam'' column specifies if the fish-eye camera motion detector is tracking the drone. 
%
The respective class output labels of each worker are shown in the rest of the columns. Note that the system output is more stable and lasts for more frames than the thermal infrared and video camera individually, indicating the benefit of the sensor fusion. 
%
Since there is no information from the ADS-B receiver in this case, that column has been omitted from the table.
%
Figure~\ref{fig:system_GUI} above is the third frame from 14:46:18. As it can be seen, the thermal infrared, video camera, and audio workers detect and classify the drone correctly. The fish-eye camera worker also tracks the drone, and the thermal infrared worker controls the pan/tilt platform.


\begin{figure} [htb]
\centering
%\includegraphics[width=0.12\textwidth]{blank.png}
%\includegraphics[width=0.12\textwidth]{blank.png}
%\includegraphics[width=0.12\textwidth]{blank.png}
%\includegraphics[width=0.45\textwidth]{fusion_results1.png}
%\includegraphics[width=0.45\textwidth]{fusion_results2.png}
\includegraphics[width=0.95\textwidth]{fusion_results1-2.png}
\caption{Evaluating the efficiency the sensor fusion.
%Left: frame-by-frame analysis of drone detection during one evaluation session. Right: false detections appearing in a ten minutes long section of screen recording from an evaluation session, including the type of object causing the false detection. 
Pictures were originally appearing in \cite{drone20thesis} and published in \cite{drone20icpr}. Reprinted with permission.
}
\label{fig:fusion-results1}
\end{figure}


To measure the efficiency of the sensor fusion, we can consider occasions such as the one described in Figure~\ref{fig:fusion-results1}a as a \textit{detection opportunity}. If we define this to be when the drone is continuously observable in the field of view of the thermal infrared and video cameras, and hence possible for the system (including the audio classifier) to analyse and track, we have carried out 73 such opportunities in the screen recordings from the evaluation sessions.
%
The duration of the individual detection opportunities is from just fractions of a second up to 28 seconds. This is highly dependent on how the drone is flying and whether the system can track the drone. We can see that Figure~\ref{fig:fusion-results1}a describes the frame-by-frame analysis of a detection opportunity lasting for three seconds.
%
Comparing the system results after the sensor fusion with the output from the respective sensors, we can observe that the system outputs a drone classification at some time in 78\% of the detection opportunities. Closest to this is the performance of the video camera detector that outputs a drone classification in 67\% of the opportunities.


We have also looked at the system behaviour without a drone flying in front of it to analyse false detections.
%
To do this, a ten-minute section of videos from the evaluation sessions was reviewed frame-by-frame.
%
Figure~\ref{fig:fusion-results1}b shows the timestamps, sensor types, causes of false detection, and resulting output labels. 
%
Setting the minimum number of sensors option to two prevents all the false detections in the figure from becoming false detections on a system level.
%
The false detections caused by insects flying just in front of the sensors are very short-lived, while the ones caused by clouds can last longer, up to several seconds.
%
Figure~\ref{fig:fig43} shows the false detections of the thermal infrared at 02:34 and the video camera at 02:58.
%
As described earlier, the individual weaknesses observed for the primary sensors are  sensitivity to clouds (thermal infrared) and autofocus problems (video camera). However, the fusion detection pipeline shows that such individual shortcomings can be overcome with a multi-sensor solution. 

Some other screenshots from the system evaluation sessions and interesting complementary observations are also pointed out in Appendix~\ref{sect:appendix}.


\begin{figure} [htb]
\centering
\includegraphics[width=0.7\textwidth]{fig43.jpg}
\caption{False detections of the thermal infrared at 02:34 and the video camera at 02:58 indicated in Figure~\ref{fig:fusion-results1}b. Pictures originally appearing in \cite{drone20thesis}.}
\label{fig:fig43}
\end{figure}



%%%%%%%%%%%%%%%%%%%%%%%%%%%%%%%%%%%%%%%%%%
%\section{Discussion}
%
%Authors should discuss the results and how they can be interpreted from the perspective of previous studies and of the working hypotheses. The findings and their implications should be discussed in the broadest context possible. Future research directions may also be highlighted.
%


%%%%%%%%%%%%%%%%%%%%%%%%%%%%%%%%%%%%%%%%%%
\section{Conclusions}
\label{sect:conclusions}

The increased use of drones for many recreational and professional purposes is creating concerns about the safety and security of premises where such vehicles may create risky situations or compromise security. 
%
Here, we explore the design of a multi-sensor drone detection system that employs state-of-the-art feature extraction and machine learning techniques, e.g. YOLOv2 detector \cite{Redmon17}, GMM background subtraction \cite{Stauffer99,gmmmatlab}, Kalman filters \cite{kalmanmatlab}, MFCC audio features or LSTM classifiers \cite{Hochreiter97}.
%
We employ a standard video camera and audio microphone complemented with a thermal infrared camera.
%
These two cameras are steered towards potential objects of interest with a fish-eye lens camera with a wider field of view that is used to detect moving objects at once over a bigger portion of the sky. 
%
We also include an ADS-B receiver, which allows for tracking cooperative aircrafts that broadcast their vehicle type, and a GPS receiver.
%
We have also evaluated a radar module, which has been finally discarded due to its very short practical range.
%
The sensors are all placed on a pan/tilt platform mounted on a standard surveyor's tripod.
%
This allows easy deployment of the solution outdoors and portability since the system can be disassembled into a few large parts and placed in a transport box.


The performance of the individual sensors is evaluated in terms of precision, recall, F1-score and mean average precision (mAP).
%
We observe that thermal infrared sensors are suitable for the drone detection task with machine learning techniques. 
%
The infrared detector achieves an F1-score of 0.7601, showing similar performance as the visible video detector with an F1-score of 0.7849. The audio classifier achieves an F1-score of 0.9323.
%
By sensor fusion, we also make the detection and classification more robust than any of the sensors individually, showing the efficiency of sensor fusion as a means to mitigate false detections. 
%
Another novelty is the investigation of detection performance as a function of the sensor-to-target distance.
%
We also contribute with a multi-sensor dataset, which overcomes the lack of publicly available data for this task. %
Apart from the feasibility of the sensors employed, especially the thermal infrared one, our dataset also uses an expanded number of target classes compared to related papers.
%
Our dataset is also especially suited for the comparison of infrared and visible video detectors due to the similarities in conditions and target types in the set.
%
To the best of our knowledge, we are also the first to explore the benefits of including ADS-B data to better separate targets prone to be mistaken for drones.

Future research could implement a distance estimation function based on the output from the detectors. Such research could also include investigating distributed detection and tracking and further using the temporal dimension to separate drones from other targets based on their behaviour over time as the system tracks them.
%
In a distributed scenario \cite{Yin20ojsp_cooperative}, several agents (detection stations in our case) could cooperate, each having access to only a few sensing modalities or varying computing capabilities or battery power.
%
Replicating one collection station with many different sensors may be cost-prohibitive or difficult if transport and deployment need to be done quickly (e.g. we used a lighter version with only two sensors and manual steering for dataset collection, Figure~\ref{fig:data_collection}). 
%
However, cooperative operation entails challenges, such as handling orchestration and task distribution between units, connectivity via 4G/5G networks, or cloud/edge computing if the deployed hardware lacks such capability. 
%
Another challenge is the optimal fusion of information from sources that may have different data quality (e.g. sensors with different features), different modalities (e.g. visible, NIR, audio...), or different availability at a given time (e.g. a specific sensor is not deployed or the object is not visible from that location). In the case of visual sensors, there is also a need to match targets observed by sensors in different physical locations since the object is seen from a different point of view.  On the other hand, the first station that detects a target can notify it and provide helpful information to aid in the detection by the others.  
%

%
Another direction is to implement the YOLO detector in the fish-eye camera. However, this would demand a training dataset or skewing video camera images to match the fish-eye lens distortion.
%
It would also be interesting to use YOLO v3 instead since it is more efficient in detecting small objects according to \cite{Unlu19}, or even more recent versions, which currently go up to YOLO v7.
%
In this paper, we kept using YOLO v2 since it is the predominant choice in the literature (as seen in recent surveys \cite{9765451}), enabling a fairer comparison of our results with previous works.
%
The performance of the audio classifier with respect to the sensor-to-target distance could also be explored in the same way as the video sensors. 
%
Furthermore, the influence of covariates given by different weather conditions could also be investigated.

A radar with a good range would have contributed significantly to the system results since it is the only available sensor that can measure the distance to the target efficiently. 
%
%Another way to increase the efficiency of the system could also be to exploit the temporal dimension better, i.e. to use the flight paths and the behaviour of the object to better classify them.
%
Likewise, it would be interesting to test the system against a drone equipped with a transponder to see the performance of the ADS-B worker at a short distance. Such transponders weigh as little as 20 grams \cite{uAvioni}. However, the price of such equipment ($\sim$3k€) is still an issue for non-professional drone users.
%
Since ADS-B information is included in the system, this could also be implemented as ground truth for an online learning feature. Images of ADS-B targets could be saved, and the detectors would be retrained at regular intervals using these additional images. To further enlarge the training data set, all images of detected objects could be saved and annotated manually at a later stage.

This work can also be helpful in other areas.
%
One example is road traffic surveillance, since most parts and scripts (except the ADS-B receiver) are applicable after appropriate retraining to detect and track vulnerable road users (pedestrians), vehicles, etc. and even specific vehicle types such as light ones (bikes, motorcycles) or heavy ones (trucks).
%
Another application involving the detection of people is surveillance of large areas where access is restricted, and operator monitoring can be very resource-consuming if not automated, such as outdoors.
%



%%%%%%%%%%%%%%%%%%%%%%%%%%%%%%%%%%%%%%%%%%
%\section{Patents}
%
%This section is not mandatory, but may be added if there are patents resulting from the work reported in this manuscript.

%%%%%%%%%%%%%%%%%%%%%%%%%%%%%%%%%%%%%%%%%%
\vspace{6pt} 

%%%%%%%%%%%%%%%%%%%%%%%%%%%%%%%%%%%%%%%%%%
%% optional
%\supplementary{The following supporting information can be downloaded at:  \linksupplementary{s1}, Figure S1: title; Table S1: title; Video S1: title.}

% Only for the journal Methods and Protocols:
% If you wish to submit a video article, please do so with any other supplementary material.
% \supplementary{The following supporting information can be downloaded at: \linksupplementary{s1}, Figure S1: title; Table S1: title; Video S1: title. A supporting video article is available at doi: link.}

%%%%%%%%%%%%%%%%%%%%%%%%%%%%%%%%%%%%%%%%%%
\authorcontributions{
%For research articles with several authors, a short paragraph specifying their individual contributions must be provided. The following statements should be used ``Conceptualization, X.X. and Y.Y.; methodology, X.X.; software, X.X.; validation, X.X., Y.Y. and Z.Z.; formal analysis, X.X.; investigation, X.X.; resources, X.X.; data curation, X.X.; writing---original draft preparation, X.X.; writing---review and editing, X.X.; visualization, X.X.; supervision, X.X.; project administration, X.X.; funding acquisition, Y.Y. All authors have read and agreed to the published version of the manuscript.'', please turn to the  \href{http://img.mdpi.org/data/contributor-role-instruction.pdf}{CRediT taxonomy} for the term explanation. Authorship must be limited to those who have contributed substantially to the work~reported.
%
This work has been carried out by Fredrik Svanström under the supervision of Fernando Alonso-Fernandez and Cristofer Englund in the context of his Master's Thesis at Halmstad University (Master's Programme in Embedded and Intelligent Systems). The thesis is available at \cite{drone20thesis}.  
%
Fredrik Svanström: Conceptualization, Methodology, Investigation, Data curation, Writing –review \& editing; Fernando Alonso-Fernandez: Conceptualization, Supervision, Funding acquisition, Writing – original draft; Cristofer Englund: Conceptualization, Supervision, Writing – review \& editing.
}

\funding{
%Please add: ``This research received no external funding'' or ``This research was funded by NAME OF FUNDER grant number XXX.'' and  and ``The APC was funded by XXX''. Check carefully that the details given are accurate and use the standard spelling of funding agency names at \url{https://search.crossref.org/funding}, any errors may affect your future funding.
%
Author F. A.-F. thanks the Swedish Research Council (VR) for funding his research.
%
Authors F. A.-F. and C. E. thank the Swedish Innovation Agency (VINNOVA) for funding their research.
}

%\institutionalreview{In this section, you should add the Institutional Review Board Statement and approval number, if relevant to your study. You might choose to exclude this statement if the study did not require ethical approval. Please note that the Editorial Office might ask you for further information. Please add “The study was conducted in accordance with the Declaration of Helsinki, and approved by the Institutional Review Board (or Ethics Committee) of NAME OF INSTITUTE (protocol code XXX and date of approval).” for studies involving humans. OR “The animal study protocol was approved by the Institutional Review Board (or Ethics Committee) of NAME OF INSTITUTE (protocol code XXX and date of approval).” for studies involving animals. OR “Ethical review and approval were waived for this study due to REASON (please provide a detailed justification).” OR “Not applicable” for studies not involving humans or animals.}

%\informedconsent{Any research article describing a study involving humans should contain this statement. Please add ``Informed consent was obtained from all subjects involved in the study.'' OR ``Patient consent was waived due to REASON (please provide a detailed justification).'' OR ``Not applicable'' for studies not involving humans. You might also choose to exclude this statement if the study did not involve humans.
%
%Written informed consent for publication must be obtained from participating patients who can be identified (including by the patients themselves). Please state ``Written informed consent has been obtained from the patient(s) to publish this paper'' if applicable.}

\dataavailability{
%In this section, please provide details regarding where data supporting reported results can be found, including links to publicly archived datasets analyzed or generated during the study. Please refer to suggested Data Availability Statements in section ``MDPI Research Data Policies'' at \url{https://www.mdpi.com/ethics}. If the study did not report any data, you might add ``Not applicable'' here.
%
The data used in this paper is fully described in \cite{DiBdataset} and publicly available at \cite{svanstrom_fredrik_2020_5500576}.
} 

%\acknowledgments{
%In this section you can acknowledge any support given which is not covered by the author contribution or funding sections. This may include administrative and technical support, or donations in kind (e.g., materials used for experiments).
%}

\conflictsofinterest{The authors declare no conflict of interest.}
%{Declare conflicts of interest or state ``The authors declare no conflict of interest.'' Authors must identify and declare any personal circumstances or interest that may be perceived as inappropriately influencing the representation or interpretation of reported research results. Any role of the funders in the design of the study; in the collection, analyses or interpretation of data; in the writing of the manuscript; or in the decision to publish the results must be declared in this section. If there is no role, please state ``The funders had no role in the design of the study; in the collection, analyses, or interpretation of data; in the writing of the manuscript; or in the decision to publish the~results''.} 

%%%%%%%%%%%%%%%%%%%%%%%%%%%%%%%%%%%%%%%%%%
%% Optional
%\sampleavailability{Samples of the compounds ... are available from the authors.}

%% Only for journal Encyclopedia
%\entrylink{The Link to this entry published on the encyclopedia platform.}

\abbreviations{Abbreviations}{
The following abbreviations are used in this manuscript:\\

\noindent 
\begin{tabular}{@{}ll}
MDPI & Multidisciplinary Digital Publishing Institute\\
DOAJ & Directory of open access journals\\
TLA & Three letters acronym\\
LD & Linear dichroism
\end{tabular}
}

%%%%%%%%%%%%%%%%%%%%%%%%%%%%%%%%%%%%%%%%%%
%% Optional
\appendixtitles{yes} % Leave argument "no" if all appendix headings stay EMPTY (then no dot is printed after "Appendix A"). If the appendix sections contain a heading then change the argument to "yes".
\appendixstart
\appendix
\section[\appendixname~\thesection]{Complementary screenshots and observations to the sensor fusion evaluation sessions}
\label{sect:appendix}

%\subsection[\appendixname~\thesubsection]{}
%The appendix is an optional section that can contain details and data supplemental to the main text---for example, explanations of experimental details that would disrupt the flow of the main text but nonetheless remain crucial to understanding and reproducing the research shown; figures of replicates for experiments of which representative data are shown in the main text can be added here if brief, or as Supplementary Data. Mathematical proofs of results not central to the paper can be added as an appendix.

%\section[\appendixname~\thesection]{}
%All appendix sections must be cited in the main text. In the appendices, Figures, Tables, etc. should be labeled, starting with ``A''---e.g., Figure A1, Figure A2, etc.

In this section, some screenshots from the system evaluation sessions and some interesting observations are pointed out.
%
The images also indicate the FPS performance. 
%
The system can process 6 FPS or more from all cameras and over 10 processing cycles per second for the input audio stream. %Note that the ADS-B solution chosen is the more computationally demanding out of the two alternatives evaluated. 
%
To be able to evaluate the system performance, a screen recording software has also been running on the computer at the same time as the system software. 
%
%This setup was a necessity since the drone flying took all the computational power of the thesis author during the evaluation sessions.
%

The ideal situation is that all the sensors output the correct classification of the detected target and that the fish-eye camera tracks the object. This is, however, far from the case at all times.
%
Nevertheless, after the sensor fusion, the system output class is observed to be robust, as shown in Figure~\ref{fig:fig44} and \ref{fig:fig45}, where the video and thermal infrared cameras classify the drone incorrectly, but still with a correct system output.

\begin{figure} [htb]
\centering
\includegraphics[width=0.9\textwidth]{fig44.jpg}
\caption{A correct system output even if the video camera classifies the drone as a bird. Picture originally appearing in \cite{drone20thesis}.}
\label{fig:fig44}
\end{figure}

\begin{figure} [htb]
\centering
\includegraphics[width=0.9\textwidth]{fig45.jpg}
\caption{A correct system output even if the thermal infrared camera classifies the drone as a helicopter. Picture originally appearing in \cite{drone20thesis}.}
\label{fig:fig45}
\end{figure}

Since the output class depends on the confidence score, the result is sometimes the opposite, as shown in Figure~\ref{fig:fig46}, so a very confident sensor (audio) causes the system output to be wrong. 
%
If this turns out to be frequent, the weight of the sensor can easily be adjusted, or the sensor can be excluded entirely from the system result. 
%
The time smoothing procedure of the sensor fusion will also reduce the effect of an occasional misclassification so that the system output stays correct, as seen in Figure~\ref{fig:fig47}. Naturally, there are also times when all sensors are wrong, as evident in Figure~\ref{fig:fig48}.

\begin{figure} [htb]
\centering
\includegraphics[width=0.9\textwidth]{fig46.jpg}
\caption{The high confidence score of the audio classifier causes the system output to be incorrect, just like the audio output. Picture originally appearing in \cite{drone20thesis}.}
\label{fig:fig46}
\end{figure}

\begin{figure} [htb]
\centering
\includegraphics[width=0.9\textwidth]{fig47.jpg}
\caption{The time smoothing part of the sensor fusion reduces the effect of an occasional misclassification, even if that has a high confidence score. Picture originally appearing in \cite{drone20thesis}.}
\label{fig:fig47}
\end{figure}

\begin{figure} [htb]
\centering
\includegraphics[width=0.9\textwidth]{fig48.jpg}
\caption{Several sensors misclassify the drone at the same time. Picture originally appearing in \cite{drone20thesis}.}
\label{fig:fig48}
\end{figure}


To the best of our knowledge, the inclusion of an ADS-B receiver in a drone detection system has not yet been described in the scientific literature, so it is interesting to see how this information is utilized.
%
In Figure~\ref{fig:fig32}, an airplane is detected and classified at a sloping distance of more than 35000 m. The ADS-B information guides the pan/tilt platform in the direction of the airplane so that the video camera can detect it.
%
At this distance, even a large commercial airplane is only about 1.4 pixels, hence well below the detection limit of the DRI criteria.
%
The reason behind the fact that the ADS-B FoV-target distance display is empty is that the main script will not present that information until the target is within a 30000 m horizontal distance. This limit is set based on the assumption that no target beyond 30000 m should be detectable, which turned out to be wrong.
%

\begin{figure} [htb]
\centering
\includegraphics[width=0.9\textwidth]{fig32.jpg}
\caption{An airplane detected and classified correctly by the video camera worker at a sloping distance of more than 35000 m. See the text for details. Picture originally appearing in \cite{drone20thesis}.}
\label{fig:fig32}
\end{figure}


Looking at Figure~\ref{fig:fig49}, we can see that the ADS-B information will appear in the results panel when the airplane comes within 30000 m horizontal distance from the system. At this moment, the sloping distance is 32000 m, and the offset between the camera direction and the calculated one is zero. Moreover, since the system has not yet received the vehicle category information, the target is marked with a square in the ADS-B presentation area, and the confidence score of the ADS-B result is 0.75.
%
The next interesting event, shown in Figure~\ref{fig:fig50}, is when the system receives the vehicle category message. To indicate this, the symbol in the ADS-B presentation is changed to a circle, and the confidence is set to one since we are sure that it is an airplane. At a distance of 20800 m, it is also detected and classified correctly by the thermal infrared worker, as shown in Figure~\ref{fig:fig51}.

\begin{figure} [htb]
\centering
\includegraphics[width=0.9\textwidth]{fig49.jpg}
\caption{When the airplane is within 30000 m horizontal distance the ADS-B information is presented in the results panel. Picture originally appearing in \cite{drone20thesis}.}
\label{fig:fig49}
\end{figure}

\begin{figure} [htb]
\centering
\includegraphics[width=0.9\textwidth]{fig50.jpg}
\caption{The system has received the vehicle category message, so the confidence for the airplane classification is set to 1. The airplane is now at a distance of 23900m. Picture originally appearing in \cite{drone20thesis}.}
\label{fig:fig50}
\end{figure}


\begin{figure} [htb]
\centering
\includegraphics[width=0.9\textwidth]{fig51.jpg}
\caption{The thermal infrared worker detects the airplane at a distance of 20800 m. Picture originally appearing in \cite{drone20thesis}.}
\label{fig:fig51}
\end{figure}

\clearpage

%%%%%%%%%%%%%%%%%%%%%%%%%%%%%%%%%%%%%%%%%%
\begin{adjustwidth}{-\extralength}{0cm}
%\printendnotes[custom] % Un-comment to print a list of endnotes

\reftitle{References}

% Please provide either the correct journal abbreviation (e.g. according to the “List of Title Word Abbreviations” http://www.issn.org/services/online-services/access-to-the-ltwa/) or the full name of the journal.
% Citations and References in Supplementary files are permitted provided that they also appear in the reference list here. 

%=====================================
% References, variant A: external bibliography
%=====================================
%\bibliography{your_external_BibTeX_file}

\bibliography{bibliography}

%=====================================
% References, variant B: internal bibliography
%=====================================
%\begin{thebibliography}{999}
%% Reference 1
%\bibitem[Author1(year)]{ref-journal}
%Author~1, T. The title of the cited article. {\em Journal Abbreviation} {\bf 2008}, {\em 10}, 142--149.
%% Reference 2
%\bibitem[Author2(year)]{ref-book1}
%Author~2, L. The title of the cited contribution. In {\em The Book Title}; Editor 1, F., Editor 2, A., Eds.; Publishing House: City, Country, 2007; pp. 32--58.
%% Reference 3
%\bibitem[Author3(year)]{ref-book2}
%Author 1, A.; Author 2, B. \textit{Book Title}, 3rd ed.; Publisher: Publisher Location, Country, 2008; pp. 154--196.
%% Reference 4
%\bibitem[Author4(year)]{ref-unpublish}
%Author 1, A.B.; Author 2, C. Title of Unpublished Work. \textit{Abbreviated Journal Name} year, \textit{phrase indicating stage of publication (submitted; accepted; in press)}.
%% Reference 5
%\bibitem[Author5(year)]{ref-communication}
%Author 1, A.B. (University, City, State, Country); Author 2, C. (Institute, City, State, Country). Personal communication, 2012.
%% Reference 6
%\bibitem[Author6(year)]{ref-proceeding}
%Author 1, A.B.; Author 2, C.D.; Author 3, E.F. Title of presentation. In Proceedings of the Name of the Conference, Location of Conference, Country, Date of Conference (Day Month Year); Abstract Number (optional), Pagination (optional).
%% Reference 7
%\bibitem[Author7(year)]{ref-thesis}
%Author 1, A.B. Title of Thesis. Level of Thesis, Degree-Granting University, Location of University, Date of Completion.
%% Reference 8
%\bibitem[Author8(year)]{ref-url}
%Title of Site. Available online: URL (accessed on Day Month Year).
%\end{thebibliography} 

% If authors have biography, please use the format below
%\section*{Short Biography of Authors}
%\bio
%{\raisebox{-0.35cm}{\includegraphics[width=3.5cm,height=5.3cm,clip,keepaspectratio]{Definitions/author1.pdf}}}
%{\textbf{Firstname Lastname} Biography of first author}
%
%\bio
%{\raisebox{-0.35cm}{\includegraphics[width=3.5cm,height=5.3cm,clip,keepaspectratio]{Definitions/author2.jpg}}}
%{\textbf{Firstname Lastname} Biography of second author}

% For the MDPI journals use author-date citation, please follow the formatting guidelines on http://www.mdpi.com/authors/references
% To cite two works by the same author: \citeauthor{ref-journal-1a} (\citeyear{ref-journal-1a}, \citeyear{ref-journal-1b}). This produces: Whittaker (1967, 1975)
% To cite two works by the same author with specific pages: \citeauthor{ref-journal-3a} (\citeyear{ref-journal-3a}, p. 328; \citeyear{ref-journal-3b}, p.475). This produces: Wong (1999, p. 328; 2000, p. 475)

%%%%%%%%%%%%%%%%%%%%%%%%%%%%%%%%%%%%%%%%%%
%% for journal Sci
%\reviewreports{\\
%Reviewer 1 comments and authors’ response\\
%Reviewer 2 comments and authors’ response\\
%Reviewer 3 comments and authors’ response
%}
%%%%%%%%%%%%%%%%%%%%%%%%%%%%%%%%%%%%%%%%%%
\end{adjustwidth}
\end{document}



%\newcommand{\E}{\mathbf{E}}

%\newcommand{\Dist}{\mathcal{D}}
%\newcommand{\Inter}[1]{\widetilde{#1}}
%\newcommand{\Poly}[1]{poly(#1)}
%\newcommand{\CC}[1]{\widehat{#1}}
%\newcommand{\Src}{\overline{src}}
%\newcommand{\Snk}{\overline{snk}}
%\newcommand{\Ol}[1]{\overline{#1}}
%\newcommand{\Ul}[1]{\underline{#1}}



\section{Preliminaries}
\label{sec:prelim}

\paragraph{Single-agent Mechanisms}
%%
%% services
%%

%A service provider faces $n$ agents and may offer service to any
%subset of size at most $k$ of them.

We consider the provisioning of an abstract service.  This service may
be parameterized by an {\em attribute}, e.g., quality of service, and
may be accompanied by a required payment.  We denote the outcome
obtained by an agent as $\outcome \in \outcomespace$.  We view this
outcome as giving an indicator for whether or not an agent is served and
as describing attributes of the service such as quality of service and
monetary payments.  Let $\ALLOC(\outcome) \in \{0,1\}$ be an indicator
for whether the agent is served or not; let $\PAYMENT(\outcome)
\in \reals$ denote any payment the agent is required to make.  In a
randomized environment (e.g., randomness from a randomized mechanism or Bayesian
environment) the outcome an agent receives is a random variable from a
distribution over $\outcomespace$.  The space of all such
distributions is denoted $\outcomedistspace$.

%%
%% types, utilities, prior
%%
The agent has a type $\type$ from a finite type space $\typespace$.
This type is drawn from distribution $\dens \in \distover{\typespace}$
and we equivalently denote by $\dens$ the probability mass function.
I.e., for every $\type \in \typespace$, $\dens(\type)$ is the
probability that the type is~$\type$.  The utility function $\util
\,:\, \typespace \times \outcomespace \to \reals$ maps the agent's
type and the outcome to real valued utility.  The agent is a von
Neumann--Morgenstern expected utility maximizer and we extend $\util$
to $\distoutcomespace$ linearly, i.e., for $\outcome
\in \distoutcomespace$, $\util(\type, \outcome)$ is the expectation of $\util$ where the outcome is drawn according
to~$\outcome$.
%$\util(\type,\outcome) = \expect[v \sim \outcome]{\util(\type,v)}$.
We do not require the usual assumption of quasi-linearity.



%%
%% incentives, service rule
%%
%By the revelation principle, we will restrict attention to Bayesian
%incentive compatible mechanisms, i.e., ones in which truthtelling is a
%Bayes-Nash equilibrium.  For an implicit mechanism we will denote the
%{\em outcome rule} by $\outcomes(\types)$ where
%$\outcome\agind(\types)$ is the service to agent $\agent$.  Similarly
%we will denote by $\allocs(\types)$ and $\prices(\types)$ the {\em
%  allocation} and {\em payment rule}.  These may be random variables
%and we denote their expectations by $\expallocs(\types) =
%\expect{\allocs(\type)}$ and $\expprices(\types) =
%\expect{\prices(\types)}$.
%When the agents' types are drawn from the
%product distribution $\dists = (\dist\agind[1] \times \cdots \times
%\dist\agind[n])$, agent $\agent$'s {\em interim outcome rule} is
%$\outcome\agind(\type\agind) =
%\evalat{\outcome\agind(\types)}{\type\agind}$ for $\types \sim \dists$ denoting the
%distribution over the service that agent $\agent$ receives when other
%types are drawn from the distribution.

%%
%% outcome rule
%%
A single-agent mechanism, without loss of generality by the revelation
principle, is just an {\em outcome rule}, a mapping from
the agent's type to a distribution over outcomes.  We denote an {\em
  outcome rule} by $\toutcome \,:\, \typespace
\to \outcomedistspace$.  We say that an outcome rule $\toutcome$ is
{\em incentive compatible} (IC) and {\em individually rational} (IR) if for
all $\type,\type' \in \typespace$, respectively,
\begin{align}
\tag{IC}\label{eq:IC}
\util(\type,\toutcome(\type)) &\geq \util(\type,\toutcome(\type')),\\
\tag{IR}\label{eq:IR}
\util(\type,\toutcome(\type)) &\geq 0.
\end{align}


%%
%% allocation rule
%%
%As the mechanism designer faces a constraint only on the allocation
%portion of the service it will be useful to separate out that part.
We refer to restriction of the outcome rule to the indicator for
service as the {\em allocation rule}.  As the allocation to
each agent is a binary random variable, distributions over allocations
are fully described by their expected value.  Therefore the allocation
rule $\talloc \,:\, \typespace \to [0,1]$ for a given outcome rule
$\outcome$ is $\talloc(\type) = \expect{\ALLOC(\toutcome(\type))}$.


%%
%% Tiger: added the following Examples
%%
We give two examples to illustrate the abstract model described above.
The first example is the standard quasi-linear risk-neutral preference
which is prevalent in auction theory.  Here the agent's type space is
$\typespace \subset \posreals$ where $\type \in \typespace$ represents the
agent's valuation for the item.  The outcome space is $\outcomespace =
\{0, 1\} \times \posreals$ where an outcome $\outcome$ in this space
indicates whether or not the item is sold to the agent, by
$\ALLOC(\outcome)$, and at what price, by $\PAYMENT(\outcome)$.  The
agent's quasi-linear utility function is $\util(\type, \outcome) =
\type \cdot \ALLOC(\outcome) - \PAYMENT(\outcome)$.
%
% example 2
%
The second example is that of an $\numservice$-item unit-demand (also
quasi-linear and risk-neutral) preference.  Here the type space is
$\typespace \subset \posreals^\numservice$ and a type $\type
\in \typespace$ indicates the agent's valuation for each of the items
when the agent's value for no service is normalized to zero.  An
outcome space is $\outcomespace = \{0,\ldots,\numservice\} \times
\posreals$.  The first coordinate of $\outcome$ specifies which item
the agent receives or none and $\ALLOC(\outcome) = 1$ if it is
non-zero; the second coordinate of $\outcome$ specifies the required
payment $\PAYMENT(\outcome)$.  The agent's utility for $\outcome$ is
the value the agent attains for the item received less her payment.
Beyond these two examples, our framework can easily incorporate more
general agent preferences exhibiting, e.g., risk aversion or a budget
limit.


\Xcomment{
%%
%% Tiger: added the following Examples
%%
We give two examples to illustrate the set of notations above.  The
first example is the standard quasi-linear risk-neutral preference
which is prevalent in auction theory.  Here the type of the agent is
simply her valuation for the item, an outcome $\outcome$ is a tuple
$(\sdalloc, \sdprice) \in \{0, 1\} \times \posreals$ that indicates
(by $\sdalloc$) whether or not the item is sold to the agent and (by
$\sdprice$) at what price.  $\ALLOC(\sdalloc,\sdprice) = \sdalloc$ and
$\PAYMENT(\sdalloc,\sdprice)$ is $\sdprice$.  The agent's quasi-linear
utility function is $\util(\type, (\sdalloc,\sdprice)) = \type
\sdalloc - \sdprice$.
%
% example 2
%
The second example is that of a $\numservice$-item unit-demand (also
quasi-linear and risk-neutral) preference.  Here the type $\type :
\{0,\ldots, \numservice\} \to \posreals$ indicates the agent's
valuation for each of the items or $\mdtype{0} = 0$ for no service.
An outcome $\outcome = (\mdalloc, \mdprice) \in
\{0,\ldots,\numservice\} \times \posreals$.  Here, $\mdalloc$
specifies which item the agent receives or none if $\mdalloc = 0$,
therefore, $\ALLOC(\mdalloc,\mdprice) = 1$ if $\mdalloc > 0$ and zero
otherwise, $\PAYMENT(\mdalloc,\mdprice) = \mdprice$.  The agent's
utility is $\util(\type,(\mdalloc,\mdprice)) = \mdtype{\mdalloc} -
\mdprice$.  Of course, our framework can easily incorporate more
general settings such as those involving a risk averse agent with a
budget limit.
}


%%
%% quantile
%%
% An allocation rule $\talloc$ induces a partial ordering on the type
% space.  Given two types $\type$ and $\type'$ we can order them by
% which is more likely to be allocated and refer to the one that is more
% likely to be allocated as the stronger one.  Furthermore, given the
% distribution $\dist$ over type space, we can associate with each type
% a quantile which refers to the strength of a type relative to the
% distribution.  The quantile $\quant \in [0,1]$ of a type is the
% probability that a random type from the distribution is stronger.  Let
% $\quantl$ be the probability that a random type is strictly stronger
% and let $\quante$ be the probability that a type is equally strong,
% i.e., $\quantl = \prob[\type'\sim\dist]{\talloc(\type') >
%   \talloc(\type)}$ and $\quante = \prob[\type'\sim\dist]{\talloc(\type')
%   = \talloc(\type)}$.  The quantile for a type $\type$ and type
% allocation rule $\alloc$, denoted $\QUANT(\type,\talloc)$, is a random
% draw from the distribution $\quantl + U[0,\quante]$.  Importantly,
% the distribution over quantiles of a random type $\type \sim \dist$ is
% $U[0,1]$.  (The randomization of $\quante$ mass is simply a random
% tie-breaking rule; we encourage the reader to assume no ties, i.e.,
% that $\quante = 0$, and that $\QUANT(\type,\talloc)$ is deterministic.)
%

%%
%% single-agent quantile-allocation-constrained mechanism design
%%
% The transformation from type to quantile and a type allocation rule
% induces a {\em (quantile) allocation rule} $\qalloc \,:\, [0,1] \to
% [0,1]$ that maps quantiles to allocation probabilities; for a given
% type-allocation rule $\alloc$ and the mapping of type to quantile that
% it induces, the quantile allocation rule $\qalloc$ is defined as
% $\qalloc(\QUANT(\type,\talloc)) = \talloc(\type)$.  Notice that there
% may be multiple type allocation rules with the same quantile
% allocation rule.  A single-agent mechanism design problem is the
% following: {\em Given a desired quantile allocation rule, design the
%   optimal type outcome rule that induces this quantile allocation
%   rule.}
%
%%
%% single-agent downward-closed quantile-allocation-constrained mechanism design
%%
% A more permissive requirement which is consistent with a standard
% downward-closure assumption is to require the outcome rule to induce a
% quantile allocation rule that dominated by the desired quantile
% allocation rule.  The {\em cumulative allocation rule} is the integral
% of the quantile allocation rule, i.e., $\cumalloc(\quant) =
% \int_0^\quant \qalloc(\dquant)\, d \dquant$ for quantile allocation
% rule $\qalloc$.  A cumulative allocation rule $\cumalloc$ {\em
%   dominates} $\cumalloc'$ if for all $\quant$, $\cumalloc(\quant) \geq
% \cumalloc'(\quant)$.  An outome or allocation rule dominates another
% outcome or allocation rule of their induced cumulative allocation
% rules satisfy the dominance condition.  Our single-agent mechanism
% design problem is the following: {\em Given a desired quantile
%   allocation rule, design the optimal type outcome rule that is
%   dominated by the desired quantile allocation rule.}
%
%%
%% single-agent ex ante constrained mechanism design
%%
% A special case of quantile allocation rule is the one that places an
% ex ante constraint on the probability that the agent is allocated.
% For a given constraint $\aquant$ this allocation rule is a step
% function, denoted $\aqalloc[\aquant]$, which steps from 0 to 1 at $\quant =
% \aquant$.  Such a rule says to allocate to the agent with ex ante
% probability $\aquant$.
%

%%
%% single-agent optimal mechanisms
%%
% Given a quantile allocation rule $\qalloc$ for a type space
% $\typespace$ and a distribution over types $\dist$, a ``solution'' to
% the single-agent problem constructs an optimal type outcome rule
% $\RULE(\qalloc)$ with expected performance which we denote
% $\REV(\qalloc)$.  For the special case above of an ex ante allocation
% probability constraint $\aquant$, i.e., the quantile allocation rule
% $\aqalloc[\aquant]$, we overload notation and write $\REV(\aquant)$
% and $\RULE(\aquant)$ for $\REV(\aqalloc[\aquant])$ and
% $\RULE({\aqalloc[\aquant]})$, respectively.
%

%%
%% single-agent problem
%%

Consider the following single-agent mechanism design problem.  A
feasibility constraint is given by an upper bound $\talloc(\type)$ on
the probability that the agent is served as a function of her type
$\type$; the distribution on types in $\typespace$ is given by
$\dens$.  The single-agent problem is to find the outcome rule
$\toutcome^*$ that satisfies the allocation constraint of $\talloc$
and maximizes the performance, e.g., revenue.  This problem is
described by the following program:
\begin{align*}
\max_{\toutcome}:  & \quad \expect[\type \sim \dens, \toutcome(\type)]{\PAYMENT(\toutcome(\type))}
\tag{SP}\label{eq:rule-rev-def} \\
\hbox{s.t. } & \quad \expect[\toutcome(\type)]{\ALLOC(\toutcome(\type))} \leq \talloc(\type), \qquad \forall \type \in \typespace \\
 & \quad \toutcome \text{ is IC and IR}.
\end{align*}
We denote the outcome rule $\toutcome^*$ that optimizes this program
by $\RULE(\talloc)$ and its revenue by $\REV(\talloc) = \expect[\type
  \sim \dens, \toutcome^*(\type)]{\PAYMENT(\toutcome^*(\type))}$.  We note
that, although this paper focuses on revenue maximization, the same
techniques presented can be applied to maximize (or minimize) general
separable objectives such as social welfare.

\paragraph{Multi-agent Mechanisms}


%%
%% feasibility
%%
There are $n$ independent agents.  Agents need not be identical,
i.e., agent $\agent$'s type space is~$\typespace\agind$, the
probability mass function for her type is~$\dens\agind$,
her outcome space is $\outcomespace\agind$, and her utility function
is $\util\agind$. The profile of agent types is denoted by $\types =
(\type\agind[1],\ldots,\type\agind[n]) \in \typespace\agind[1] \times
\cdots \times \typespace\agind[n] = \typespaces$, the joint
distribution on types is $\denss \in \distover{\typespace_1}\times \cdots \times \distover{\typespace_n}$, a vector of outcomes is
%% Tiger:  removed "\outcomes =" and "\allocs = ".  we never use them in the paper for deterministic outcomes, and the
%%     current notation coincides with the randomized outcomes, which may cause confusion.  Also, replaced all
%%     semi-colons by commas in this sentence
%$\outcomes =
$(\outcome\agind[1], \cdots, \outcome\agind[n]) \in \outcomespaces$, and an allocation is
%$\allocs =
$(\alloc\agind[1],\ldots,\alloc\agind[n]) \in \{0,1\}^n$.  The
mechanism has an inter-agent feasibility constraint that permits
serving at most $k$~agents, i.e., $\sum_\agent \alloc\agind \leq
k$.\footnote{Furthremore, in Section~\ref{sec:border_matroid}, we
  review the theory of {\em matroids} and extend our basic results
  environments with feasibility constraint derived from a matroid set
  system.}  A mechanism that obeys this constraint is {\em
  feasible}. The mechanism has no inter-agent constraint on attributes
or payments.


%%
%% induced allocation rules
%%
A mechanism maps type profiles to a (distribution over) outcome
vectors via an {\em ex post outcome rule}, denoted $\epoutcomes \,:\,
\typespaces \to \outcomedistspaces$ where $\epoutcome\agind(\types)$
is the outcome obtained by agent $\agent$. We will similarly define
$\epallocs \,:\, \typespaces \to [0,1]^n$ as the {\em ex post
  allocation rule} (where $[0,1] \equiv \distover{\{0,1\}}$).  The ex
post allocation rule~$\epallocs$ and the probability mass
function~$\denss$ on types induce {\em interim} outcome and allocation
rules.  For agent~$\agent$ with type~$\type\agind$ and $\types \sim
\distribution[\types]{\types \given \type\agind}$ the interim outcome
and allocation rules are $\intoutcome\agind(\type\agind) =
\distribution[\types]{\epoutcome\agind(\types) \given \type\agind }$
and $\intalloc\agind(\type\agind) =
\distribution[\types]{\epalloc\agind(\types) \given \type\agind}
\equiv \expect[\types]{\epalloc\agind(\types) \given
  \type\agind}$.\footnote{We use notation $\distribution{X \given E}$
  to denote the distribution of random variable $X$ conditioned on the
  event $E$.}  A profile of interim allocation rules is feasible if it
is derived from an ex post allocation rule as described above; the set
of all feasible interim allocation rules is denoted by $\InAllocSpace$.
A mechanism is Bayesian incentive compatible and interim individually
rational if equations~\eqref{eq:IC} and~\eqref{eq:IR}, respectively,
hold for all $\agent$ and all~$\type\agind$.



%%
%% examples revisited
%%
Consider again the examples described previously of quasi-linear
single-dimensional and unit-demand preferences.  For the
single-dimensional example, the multi-agent mechanism design problem
is the standard single-item $k$-unit auction problem.  For the
unit-demand example, the multi-agent mechanism design problem is an
{\em attribute auction}.  In this problem there are $k$-units
available and each unit can be configured in one of $\numservice$
ways.  Importantly, the designer's feasibility constraint restricts
the number of units sold to be $k$ but places no restrictions on how
the units can be configured.  E.g., a restaurant has $k$ tables but
each diner can order any of the $m$ entrees on the menu.

% The interim type allocation rule of an agent, as described above, induces a mapping from type space to quantile space, and this mapping induces an interim quantile allocation rule.  We call the resulting profile $\qallocs = (\qalloc\agind[1],\ldots,\qalloc\agind[n])$ of quantile allocation rules {\em feasible} as there exists a ex post feasible outcome rule that, for the given distribution, induces them.

%% Tiger:  added \agind for all variables (since it makes more sense for talking about interim allocations).  added a pointer to Section single-agent
A reduction from multi-agent mechanism design to single-agent
mechanism design as we have described above would assume that for any
types pace $\typespace\agind$, any probability mass
function~$\dens\agind$, and interim allocation rule $\intalloc\agind$,
the optimal outcome rule $\RULE(\intalloc\agind)$ and its performance
$\REV(\intalloc\agind)$ can be found efficiently (see
Section~\ref{sec:single-agent} for examples).  The goal then is to
construct an optimal multi-agent auction from these single-agent
mechanisms.  Our approach to such a reduction is
%% Tiger:  replaced "the following" by "as follows".
%the following.
as follows.
\begin{enumerate}
\item \label{step:optimize}
Optimize, over all feasible profiles of interim allocation rules
$\intallocs = (\intalloc\agind[1],\ldots,\intalloc\agind[n]) \in \InAllocSpace$, the
%% Tiger:  replaced "total performance of the optimal outcome rules" by "sum of performances of the allocation rules"
%total performance of the optimal outcome
sum of performances of the allocation
rules $\sum_\agent \REV(\intalloc\agind)$.
\item \label{step:expostize} Implement the profile of interim outcome
  rules $\intoutcomes$ given by $\intoutcome\agind =
  \RULE(\intalloc\agind)$ with a feasible ex post outcome rule $\epoutcomes$.
\end{enumerate}

Two issues should be noted.  First, Step~\ref{step:expostize} requires
an argument that the existence of a feasible ex post outcome rule for
a given profile of interim allocation rules implies the existence of
one that combines the optimal interim outcome rules
from~$\RULE(\cdot)$.  We address this issue in Section~\ref{sec:alg}.
Second, Step~\ref{step:optimize} requires that we optimize over
jointly feasible interim allocation rules, and after solving for
$\intallocs$, its implementation by an ex post allocation rule is
needed to guide Step~\ref{step:expostize}.  We address this issue in
Section~\ref{sec:ssa}.  For single-unit (i.e., $k=1$) auctions a
characterization of the necessary and sufficient condition for interim
feasibility was provided by Kim Border.

%% Jason: i am not sure why the discussion below is in the
%% preliminaries section.

%% When $k$, the
%% number of agents we are allowed to serve, is~$1$, we give a general
%% structural theorem in Section~\ref{sec:int2expost} showing, by an
%% efficient algorithm, that every set of feasible interim allocation
%% rules can be implemented by a \emph{stochastic sequential allocation}
%% procedure. %azarakhsh changed the name
%% This allows us to describe all
%% feasibility constraints on interim allocation rules by polynomially
%% many linear inequalities, with some additional variables.  In
%% geometric terms, we identify the (generally exponentially-faceted)
%% polytope of all jointly feasible interim allocation rules as a
%% projection of a higher-dimensional polytope that has only polynomially
%% many facets. This gives, as a corollary, a generalization of Border's
%% inequality, which was known only for single-item auctions before this
%% work:

% For instance, single-item auction environments, i.e., $k=1$ and $\outcomespace = \{0,1\} \times \reals$,\footnote{In a single-item auction the outcome for an agent is the allocation and payment.} \citet{MR84} and \citet{M84} independently provided a condition necessary condition for feasibility.  The following theorem, due to \citet{B91} states that in these environments the same condition is also sufficient.

%\begin{reltheorem}[\citealp{B91}]
\begin{theorem}[\citealp{B91}]
\label{thm:border}%
In a single-item auction environment, interim allocation rules
$\intallocs$ are feasible (i.e., $\intallocs \in \InAllocSpace$) if
and only if the following holds:
\begin{align}
%% Tiger:  removed \type\agind\in
%\type\agind \in
\forall S\agind[1] \subseteq \typespace\agind[1],
\cdots, \forall S\agind[n] \subseteq \typespace\agind[n]: &\qquad
\sum_{\agent=1}^n \expect{\intalloc\agind(t_i) \given \type\agind \in
    S\agind}\cdot \prob{\type\agind \in S\agind}
        \le \prob[\types \sim \denss]{\exists i \in[n]: \type\agind \in S\agind
    } \tag{MRMB} \label{eq:MRM}
\end{align}
%\end{reltheorem}
\end{theorem}


%% \begin{example}[single-item, common budget]
%% Supply is $k = 1$, budget is $B \in \posreals$, type space is
%% $\typespace = \posreals$, service space is $\outcomespace =
%% \{0,1\}\times \emptyset \times \posreals$, and agent utility
%% $\util(\type,(\alloc,\emptyset,\price)$ is $\type \alloc - \price$ if $\price
%% \leq B$ and $-\infty$, otherwise.  The designer's objective is welfare
%% \citep{M00} or revenue \citep{LR96}.
%% \end{example}

%% \begin{example}[single-item, concave utility] Supply is $k = 1$, concave
%% utility function is $U : \posreals \to \posreals$, type space is
%% $\typespace = \posreals$, service space is $\outcomespace =
%% \{0,1\}\times \emptyset \times \posreals$, and agent utility
%% $\util(\type,(\alloc,\emptyset,\price)) = U(\type \alloc - \price)$.  The
%% designer's objective is revenue \citep{M84,MR84}.
%% \end{example}

%% \begin{example}[multi-unit, attribute auctions]
%% Supply is $k$, number of attributes is $m$, service space is
%% $\outcomespace = \{0,1\} \times \{1,\ldots,m\} \times \posreals$, type
%% space is $\typespace = \posreals^m$, and agent utility is quasi-linear
%% $\util(\type,(\alloc,\attrib,\price)) = \alloc (\type \cdot \attrib) -
%% \price$ (where $\type \cdot \attrib$ is the vector inner product). The
%% designer's objective is revenue.
%% \end{example}



\Xcomment{
\paragraph{Old Preliminary Section}



Consider a simple auction with $n$ participating agents for getting
%%  Tiger Nov 25: rephrased
%some type
one of $m$~services. In this
paper, we assume that we can serve at most $k$ of these agents. Each
agent $i \in [n]$ has a private type that defines his
%%  Tiger Nov 25: changed to plural forms to emphasize multi-dimensionality
values for the services.  The type of agent $i$ is denoted by $t_i \in T_i$,
%%  Tiger Nov 25: added the following explanation for valuation
and we use $v_i(t_i) = (v_i(t_i))_{j=1}^m \in \mathbb R^m$ to represent the vector of bidder~$i$'s valuations for the
$m$ services when his type is~$t_i$.
$T_i$ is
called the types pace of agent $i$.  Suppose that the agents' types
are drawn from independent but not
%%  Tiger Nov 25: added "necessarily"
necessarily identical distributions,
%%  Tiger Nov 25: added the following
publicly known as $\Dist = \Dist_1 \times \Dist_2 \times \cdots \times \Dist_n$.
$f_i(t_i)$ denotes the probability that agent $i$ has type $t_i$.
Furthermore, for every $S_i \subset T_i$, we define $f_i(S_i) =
\sum_{t_i \in S_i} f_i(t_i)$.
%%  Tiger Nov 25: deleted the following sentence on BNE
%A Bayes-Nash equilibrium (BNE) is a profile of strategies for mapping
%%  Tiger Nov 25: changed "values" to "types"
%values
%types to bid to obtain best response assuming all other agents are using the best response strategy and their values are coming from %%  Tiger Nov 25: added "the known"
%the known distribution.
A mechanism solicits bids from the agents and determines the outcome which consists
of an allocation $x = (x_1(t_1),\ldots, x_n(t_n))$ and payments
$p = (p_1(t_1), p_2(t_2), \ldots, p_n(t_n))$,
%%  Tiger Nov 25: added the following explanation
where $x_i(t_i) = (x_i(t_i))_{j=1}^m \in \mathbb [0, 1]^m$ represents the probabilities with which bidder~$i$ receives service~$j$.
%%  Tiger Nov 25: deleted the following sentence
%When agents are bidding in the mechanism, they know their own type and other agents' distributions.


%%  Tiger Nov 25: deleted the following sentences
%Interim allocation is the expected allocation assigned to an agent with a given type where the expectation is over all other agents' types.
%Suppose the profile of types $t=(t_1,\cdots, t_n)$ is distributed according to a publicly known distribution $\Dist$.
Given any allocation rule $x$
%%  Tiger Nov 25: added "and payment rule p"
and payment rule $p$, we can define a corresponding
\emph{interim allocation rule}
%%  Tiger Nov 25: changed [0,1] to [0, 1]^m; otherwise I don't see how v_i(t_i) \cdot x_i(t_i) is defined.
$\Inter{x}_i : T_i \to [0, 1]^m$
%%  Tiger Nov 25: added interim payment rule
\emph{interim allocation rule} $\Inter{p}_i: T_i \to \mathbb R_+$
for each agent $i$ that specifies the expected
%%  Tiger Nov 25: replaced the phrase
%probability of winning
allocation and payment
for each type of agent $i$, i.e., $\Inter{x}_i(t_i) = \E_{t_{-i}
\sim \Dist_{-i}}[x_i(t_i, t_{-i})]$,
%%  Tiger Nov 25: added definition of interim payment
and $\Inter{p}_i(t_i) = \E_{t_{-i} \sim \Dist_{-i}} [p_i(t_i, t_{-i})]$.
The allocation $x$ that is
assigned to a profile of types by the mechanism is also called
ex-post allocation.
%%  Tiger Nov 25: rewrote the definition of Bayesian truthfulness
% To design BNE mechanisms, the mechanism should satisfy the following constraint:
If for each agent~$i$ and all types $t_i, t_j \in T_i$, a mechanism satisfies
\[
v_i(t_i)\cdot \Inter{x}_i(t_i) - \Inter{p}_i(t_i) \ge v_i(t_i)\cdot
\Inter{x}_i(t_j) - \Inter{p}_i(t_j)
\]
then the mechanism is said to be \emph{Bayesian truthful}.
%%  Tiger Nov 25: changed BNE to Bayesian truthful
Note that the constraint for Bayesian truthfulness is in terms only of interim allocation
and payments.
%%  Tiger Nov 25: rewrote the following before the enumeration
%As a result, to design a BNE mechanism we can formulate by one constraint per type of an agent.  However, we should be able to map the interim allocation to feasible ex post allocations to maintain the interim allocation probability for all agents and also to satisfy the feasibility constraint for the available service(items).
%To be able to build our mechanism based on computed interim allocations, we need to find the answer to the following questions:
As a result, when designing a Bayesian truthful mechanism, one may check all incentive constraints by inspecting only
the interim allocation and payment rules for each bidder, instead of going through the exponentially large description
of ex-post rules.  However, not every interim allocation rule $(\Inter{x}_1, \Inter{x}_2, \ldots, \Inter{x}_n)$ can be implemented by a feasible ex-post allocation; even if it
can, it is still challenging to find the right ex-post allocation implementing it.  In this paper, we give efficient algorithmic
solutions to the following two problems, which enable us to design optimal mechanisms in terms of reduced forms
described by interim allocations, for domains which would otherwise seem intractable:

\begin{enumerate}
\item
Are $\Inter{x}_1, \cdots, \Inter{x}_n$ feasible?  In other words, is
there a corresponding feasible ex-post allocation rule~$x$ such that
%%  Tiger Nov 25: rid "the following holds?" and added "for all i, for all t_i$
%the following holds?
$\forall i \in [n], \forall t_i \in T_i$, $\Inter{x}_i(t_i) = \E_{t_{-i} \sim \Dist_{-i}}[x_i(t_i, t_{-i})]$?

\item
%%  Tiger Nov 25: added the following, and shorten the question
If the answer to~\label{prob:A} is affirmative,
how can such an $x$ be computed?
\end{enumerate}

%%  Tiger Nov 25: refined the description
%For the case of $k=1$,
For the domain where one service ($m=1$) is to be provided to at most one agent $(k=1)$,
Maskin and Riley~\cite{MR84} and
Matthews~\cite{M84} independently
%%  Tiger Nov 25: rewrote the sentence.  "came up with" sounds too informal
provided a condition for the first problem, which they proved to be necessary.
%came up with necessary conditions for \eqref{prob:A}.
The next theorem
%which is
due to Border~\cite{B91} states that the same condition is in fact both
necessary and sufficient for that domain.
%$k=1$.


\begin{theorem}[\citealp{B91}]
\label{thm:border}%
In an $n$-agent
%% Tiger:  replaced "single-item" by "single-service"
%single-item
single-service
auction environment, interim allocation
rules $(\Inter{x}_1, \cdots, \Inter{x}_n)$
%%  Tiger Nov 25: restructured the sentence
%, there exists a feasible ex-post allocation rule $x$ that implements the interim allocation rules
can be implemented by a feasible ex-post allocation rule~$\epallocs$
if and only if the following holds:
\begin{align}
    \forall S_1 \subseteq T_1, \cdots, \forall S_n \subseteq T_n: &\qquad \sum_{i=1}^n \sum_{t_i\in S_i} \Inter{x}_i(t_i) f_i(t_i)
        \le \Pr_{t \sim \denss}\left[\exists i \in[n]: t_i \in S_i
    \right] \tag{MRM} \label{eq:MRM}
\end{align}
\end{theorem}
}

%% Remove vspace after blocks in algorithms
\makeatletter
\patchcmd{\ALG@doentity}{\item[]\nointerlineskip}{}{}{}
\makeatother

% Fix line number references for multiple algorithms
\makeatletter
\newcounter{algorithmicH}
\let\oldalgorithmic\algorithmic
\renewcommand{\algorithmic}{%
  \stepcounter{algorithmicH}%
  \oldalgorithmic}
\renewcommand{\theHALG@line}{ALG@line.\thealgorithmicH.\arabic{ALG@line}}
\makeatother

\algnewcommand{\LeftComment}[1]{\State \(\triangleright\) #1}

\algdef{SnE}{Async}{EndAsync}{\textbf{do asynchronously}}
\algdef{SnE}{Try}{EndTry}{\textbf{try}}
\algdef{CnE}{Try}{Finally}{EndTry}{\textbf{finally}}

\algnewcommand{\Ifx}[1]{\State\algorithmicif\ #1 \algorithmicthen}
\newcommand{\Break}{\textbf{break}}
\newcommand{\Exception}{\textbf{exception}}

\def\algOwners{
  \begin{algorithm}[t]
    \caption{Promise Ownership Management}
    \label{alg:owners}
    %
    \begin{algorithmic}[1]
      \Procedure{New}{\null}
      \State $t \gets \mathit{currentTask}()$
      \State $p \gets \{ \fldowner:t \}$
      \Comment{C: atomic, Java: volatile}
      \label{ln:owners:new:A}
      \State append $p$ to $t.\fldowned$
      \label{ln:owners:new:B}
      \State \Return{$p$}
      \EndProcedure
      \Statex

      \Procedure{Async}{$P$, $f$}
      \State $t \gets \mathit{currentTask}()$
      \State \textbf{assert} $p.\fldowner = t$
      \label{ln:owners:async:X}
      \textbf{forall} $p \in P$
      \State $t' \gets \{ \fldowned:P,\,$
      \label{ln:owners:async:A}
      \State \quad$\fldwaitingOn:\Null \}$
      \Comment{C: atomic, Java: volatile}
      \label{ln:owners:async:B}
      \State remove all of $P$ from $t.\fldowned$
      \label{ln:owners:async:Y}
      \State $p.\fldowner \gets t'$ \textbf{forall} $p \in P$
      \label{ln:owners:async:C}
      \Async
      \State $\mathit{setCurrentTask}(t')$
      \label{ln:owners:async:D}
      \State $f()$
      \State \textbf{assert} $t'.\fldowned$ is empty
      \label{ln:owners:async:E}
      \EndAsync
      \State \Return{$t'$}
      \EndProcedure
      \Statex

      \Procedure{Init}{$\mathit{main}$}
      \State $\mathit{setCurrentTask}(\Null)$
      \State \Call{Async}{[], $\mathit{main}$}
      \EndProcedure
      \Statex

      \Procedure{Set}{$p$, $v$}
      \State $t \gets \mathit{currentTask}()$
      \State \textbf{assert} $p.\fldowner = t$
      \label{ln:owners:set:A}
      \State $p.\fldowner \gets \Null$
      \label{ln:owners:set:B}
      \State remove $p$ from  $t.\fldowned$
      \label{ln:owners:set:C}
      \State $\mathit{set\_impl}(p, v)$
      \label{ln:owners:set:D}
      \EndProcedure
    \end{algorithmic}
  \end{algorithm}
}

\def\algDetector{
  \begin{algorithm}[t]
    \caption{Deadlock Cycle Detection}
    \label{alg:detector}
    %
    \begin{algorithmic}[1]
      \Procedure{Get}{$p_0$}
      \State $t_0 \gets \mathit{currentTask}()$
      \State $t_0.\fldwaitingOn \gets p_0$
      \Comment{C: seq\_cst}
      \label{ln:detector:enter}
      \LeftComment{TSO: memory fence}
      \label{ln:detector:fence}
      \State $i \gets 0$
      \State $t_{i+1} \gets p_i.\fldowner$
      \label{ln:detector:owner1}
      \While{$t_{i+1} \ne t_0$}
      \label{ln:detector:loop}
      \Ifx{$t_{i+1} = \Null$} \Break
      \label{ln:detector:breakt}
      \State $p_{i+1} \gets t_{i+1}.\fldwaitingOn$
      \Comment{C: acquire}
      \label{ln:detector:waitingOn}
      \Ifx{$p_{i+1} = \Null$} \Break
      \label{ln:detector:breakp}
      \Ifx{$t_{i+1} \ne p_i.\fldowner$} \Break
      \label{ln:detector:changed}
      \State $i \gets i+1$
      \label{ln:detector:inc}
      \State $t_{i+1} \gets p_i.\fldowner$
      \label{ln:detector:owner2}
      \label{ln:detector:loopend}
      \EndWhile
      \Try
      \State \textbf{assert} $t_{i+1} \ne t_0$
      \label{ln:detector:fail}
      \State \Return{$\mathit{get\_impl}(p_0)$}
      \label{ln:detector:return}
      \Finally
      \State $t_0.\fldwaitingOn \gets \Null$
      \Comment{C: release}
      \label{ln:detector:final}
      \EndTry
      \EndProcedure
    \end{algorithmic}
  \end{algorithm}
}

%\input{content/proof_of_algorithms}
%\input{content/solver_analysis}
%\input{content/powermethod}
%\input{content/fastprojection}
%\input{content/graphs}

\section*{Acknowledgements}
The author would like to thank Prasad Raghavendra, Ankur Moitra, and Costis Daskalakis for helpful discussions.
\newpage
\bibliographystyle{alpha}
\bibliography{bibliography}

\appendix

%% The Appendices part is started with the command \appendix;
%% appendix sections are then done as normal sections
\appendix
% \onecolumn
\subsection{Prompt Template}
\label{app1}

Figs. \ref{appendix:prompt1}-\ref{appendix:prompt3} (due to the page length, we break it into three parts) show the prompt design for the information extraction in the context of construction project scheduling. 

\begin{figure}[t]
    \vspace{-4.5cm}
    \centering
    \begin{tcolorbox}[colback=gray!10!white, colframe=gray!50!gray, halign=left, boxrule=0.5pt, left=1mm, right=1mm, top=1mm, bottom=1mm]
    \fontsize{8pt}{8pt}\selectfont
    SYSTEM PROMPT: You are a project management assistant specializing in construction scheduling analysis. Your task is to analyze text descriptions of project changes and extract structured information about task relation changes in a construction project.
    \vspace{8pt}
    
    CONTEXT\\
    The project involves the following tasks and their relationships: \\
    \vspace{3pt}
    Task ID \textbar{} Predecessor \textbar{} Duration \textbar{} Description \textbar{} Robot Type
    \begin{itemize}
    \item T1 \textbar{} - \textbar{} 0.25 \textbar{} Move Electrical Conduit \textbar{} R1
    \item T2 \textbar{} - \textbar{} 0.25 \textbar{} Move Window Frame \textbar{} R1
    \item T3 \textbar{} - \textbar{} 0.25 \textbar{} Move Window \textbar{} R1
    \item T4 \textbar{} - \textbar{} 0.25 \textbar{} Move Duct Structural Materials \textbar{} R1
    \item T5 \textbar{} - \textbar{} 0.25 \textbar{} Move Duct \textbar{} R1
    \item T6 \textbar{} - \textbar{} 0.5 \textbar{} Drill Wall \textbar{} R4 or R2
    \item T7 \textbar{} T1, T6 \textbar{} 1 \textbar{} Install Electrical Conduit \textbar{} R5 or R2
    \item T8 \textbar{} T2 \textbar{} 1 \textbar{} Install Window Frame \textbar{} R4 or R2
    \item T9 \textbar{} T3, T8 \textbar{} 0.5 \textbar{} Install Window \textbar{} R3
    \item T10 \textbar{} T4 \textbar{} 2 \textbar{} Duct Structural Framing \textbar{} R4 or R2
    \item T11 \textbar{} T5, T10 \textbar{} 2 \textbar{} Install HVAC Duct \textbar{} R4 or R2
    \item T12 \textbar{} T7 \textbar{} 2 \textbar{} Install Wiring \textbar{} R5 or R2
    \item T13 \textbar{} T12 \textbar{} 1 \textbar{} Wall Painting \textbar{} R6
    \item T14 \textbar{} - \textbar{} 0.5 \textbar{} Construction Site Inspection \textbar{} R7
    \end{itemize}
    \vspace{8pt}
    
    The robot capabilities are listed below: \\
    \vspace{3pt}
    Robot ID \textbar{} Capabilities
    \begin{itemize}
    \item R1: Cargo container
    \item R2: High-payload, Precise parallel gripper, Normal parallel gripper
    \item R3: High-payload, Suction-based gripper
    \item R4: High-payload, Normal parallel gripper
    \item R5: Precise parallel gripper
    \item R6: Sprayer
    \item R7: Camera, IAQ sensors
    \end{itemize}
    \vspace{8pt}
    
    CONSTRAINT TYPES:
    \begin{enumerate}
    \item Task Dependency Adjustments
      \begin{itemize}
        \item Format: [task\_id, successor, +/-]
        \item task\_id: the target task
        \item successor: the successors of the target task
        \item +/-: ``+'' indicates a newly added successor, ``-'' means the dependency has been removed
      \end{itemize}
    \item Task Duration Variations
      \begin{itemize}
        \item Format: [task\_id, new\_duration]
        \item task\_id: the target task
        \item new\_duration: the new duration of the target task in hours
      \end{itemize}
    \item Task Starting Time Changes
      \begin{itemize}
        \item Format: [task\_id, start\_time\_change]
        \item task\_id: the target task
        \item start\_time\_change: the changes in start time of the target task (e.g., +2 means delayed by 2 hours; -2 means ahead by 2 hours)
      \end{itemize}
    \item Number of Robot Variations
      \begin{itemize}
        \item Format: [robot\_type\_id, robot\_number\_change]
        \item robot\_type\_id: the type of robot (e.g., R1, R2, R3, etc.)
        \item robot\_number\_change: the number changes of the robot (e.g., +1 means one more robot; -1 means one less robot)
      \end{itemize}
    \item Task Conflict Constraints
      \begin{itemize}
          \item Format: [task\_id1, task\_id2]
          \item task\_id1: the first task in the conflict
          \item task\_id2: the second task in the conflict
      \end{itemize}
    \end{enumerate}
    \end{tcolorbox}
    \caption{Prompt design for construction project scheduling - Part 1.}
    \label{appendix:prompt1}
\end{figure}


\begin{figure}[t]
    \centering
    \begin{tcolorbox}[colback=gray!10!white, colframe=gray!50!gray, halign=left, boxrule=0.5pt, left=1mm, right=1mm, top=1mm, bottom=1mm]
    \fontsize{8pt}{8pt}\selectfont
    STEP-BY-STEP INSTRUCTIONS:
    \begin{enumerate}
    \item Read through the entire description to understand the context.
    \item For each change mentioned in the description:
       \begin{itemize}
       \item[a.] Identify which task (T1-T14) or robot type (R1-R7) is being affected based on CONTEXT. 
        \begin{itemize}
          \item Be careful to distinguish between similar tasks, for example:
          \begin{itemize}
          \item T2 (Move Window Frame) vs. T3 (Move Window) vs. T8 (Install Window Frame) vs. T9 (Install Window) - These are different tasks.
          \item If text mentions ``window installation'', specifically, it refers to T9 (Install Window), not T3 or T8
          \item If text mentions "window frame installation," it refers to T8 (Install Window Frame), not T2
          \end{itemize}
        \item Be careful to distinguish between similar robots, for example:
          \begin{itemize}
          \item R2, R3, R4, and R5 are different robots. 
          \item Only R2 combines both high-payload and precise parallel gripper capabilities.
          \item If text only mentions ``high-payload and normal parallel gripper'', it refers to R4 not R2.
          \end{itemize}
       \end{itemize}
       \item[b.] Determine which constraint type (1-5) applies to the change based on CONSTRAINT TYPES.
       \item[c.] Extract the specific parameters needed for that constraint type.
       \item[d.] Format the parameters according to the required format for that constraint type.
       \end{itemize}
    \item Compile all identified changes into the JSON output format:
       \begin{itemize}
       \item[a.] Create a JSON object with a ``changes'' array.
       \item[b.] For each change, add an object with ``constraint\_type'' and ``parameters'' fields.
       \item[c.] Ensure numerical values (like durations and time changes) are formatted as numbers, not strings.
       \item[d.] Ensure task IDs, successors, and robot types are formatted as strings.
       \item[e.] For time-related values:
          \begin{itemize}
          \item Simplify all numerical values to their simplest form (e.g., 1.5 not 1.50, 2 not 2.0)
          \item Convert minutes to hours (e.g., 30 minutes = 0.5 hours, 45 minutes = 0.75 hours)
          \item Please be aware that if you identify the constraint as 3, the time change should be associated with ``+'' or ``-''. 
          \end{itemize}
       \item[g.] Please be aware that if you identify the constraint as 4, the robot change should be associated with ``+'' or ``-''.
       \end{itemize}
    \item Double-check your result to ensure all changes mentioned in the description have been captured.
       \begin{itemize}
       \item[a.] Please ensure that your output follows the required format; e.g., for constraint 1, the output should be [task\_id, successor, +/-] (do NOT nest successors in additional brackets) and the task\_id should be the predecessor of the successor. 
       \item[b.] Please ensure that if you identify the constraint as 1, you correctly identify the target task and the successor of the target task and put them in the right order [task\_id, successor, +/-].
       \item[c.] Please ensure that if you identify the constraint as 3, the time change should be associated with ``+'' or ``-''. 
       \item[d.] Please ensure that if you identify the constraint as 4, the robot change should be associated with ``+'' or ``-''. 
       \item[e.] Please ensure that the task description corresponds to the task\_id in the CONTEXT.
       \end{itemize}
    \end{enumerate}
    \end{tcolorbox}
    \caption{Prompt design for construction project scheduling - Part 2.}
    \label{appendix:prompt2}
\end{figure}


\begin{figure}[t]
    \vspace{-2cm}
    \centering
    \begin{tcolorbox}[colback=gray!10!white, colframe=gray!50!gray, halign=left, boxrule=0.5pt, left=1mm, right=1mm, top=1mm, bottom=1mm]
    \fontsize{8pt}{8pt}\selectfont
EXAMPLES:\\
    Example 1:\\
    Input: ``Due to how things are unfolding on-site, it's understood that the drilling machine is not functioning, so the wall will be drilled manually. The task is expected to take two hours, and in light of recent discussions, after coordinating with field staff, it seems that the original worker assigned to install the HVAC duct is no longer available; however, we have secured another worker who can arrive in 150 minutes.''\\
    Output:
    \begin{verbatim}
{"changes": [
{"constraint_type": 2, "parameters": [T6, 2]},
{"constraint_type": 3, "parameters": [T11, +2.5]}
]}
    \end{verbatim}

    Example 2:\\
    Input: ``Recent developments suggest that wall painting takes 1.5 hours instead of 1 hour due to the need for multiple coats, and in light of recent adjustments, a revised understanding across teams indicates that a specialist required for electrical conduit installation calls in sick, preventing work from starting for 2 hours., followed by further refinements as recent developments suggest that wall painting takes 1.5 hours instead of 1 hour due to the need for multiple coats.''\\
    Output:
    \begin{verbatim}
{"changes": [
{"constraint_type": 2, "parameters": [T13, 1.5]},
{"constraint_type": 3, "parameters": [T7, +2]},
{"constraint_type": 2, "parameters": [T13, 1.5]}
]}
    \end{verbatim}

    Example 3:\\
    Input: ``Task dependencies have shifted, and one of the robots capable of handling heavy loads and performing fine, precise tasks is currently out of service due to a mechanical failure. Additionally, in light of recent discussions and the evolving situation on-site, it appears that two robots with high-capacity arms and fine-movement grippers were not charged, and have now run out of power.''\\
    Output:
    \begin{verbatim}
{"changes": [
{"constraint_type": 4, "parameters": [R2, -1]},
{"constraint_type": 4, "parameters": [R2, -2]}
]}
    \end{verbatim}
    
    Now, analyze the following description and extract all task relation changes in the specified JSON format: \{description\}
    
    Please output your response in JSON format and do not output other things. 
    \begin{verbatim}
{"changes": [
{"constraint_type": <number>, 
 "parameters": [<value1>, <value2>, ...]},...
]}
    \end{verbatim}
    \end{tcolorbox}
    \caption{Prompt design for construction project scheduling - Part 3.}
    \label{appendix:prompt3}
\end{figure}

\end{document}