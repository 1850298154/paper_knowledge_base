\section{Conclusions}
\label{sec:conclusions}

This paper presented a provenance-based approach for improving the explainability of plans.
Our approach (1) extends the \shop HTN planner to generate dependency information, (2) transforms the dependency information into an established PROV-O representation, and (3) uses graph propagation and TMS-inspired algorithms support dynamic and counter-factual assessment of information flow, confidence, and support.

We qualified our approach's explanatory scope with respect to explanation targets from the automated planning literature \cite{MariaFoxExplainablePlanning2017} and the information analysis literature \cite{icd_203,icd_206,zelik2010measuring}.
We demonstrated that our approach answers questions of pertinence, sensitivity, risk, assumption support, diversity of evidence, and relative confidence.
Our approach is limited to explaining elements of the plan itself: it does \emph{not} explain why a given action was not planned or whether an action is plannable via the planner's internal model.

Our provenance approach might be able to help explain ``the road not taken'' if the planner represents decision points and constraints in the dependency-based plan.
This would not enable complete \emph{``what-if''} hypothetical explanations, but it would explain local rationale for planning decisions within context.

This paper used simple example plans.
Our underlying graph propagation and TMS algorithms easily scale to larger datasets, and they can execute incrementally when graphs (\ie{}, automated plans) are revised online \cite{forbus1993building}.
However, we face a non-computational scalability problem of user experience: the UI to display the associated provenance cannot intuitively display full plans with hundreds of nodes without graph summarization and graph filtering, which is one avenue of future work.

\subsection{Future Work}
This work demonstrates the explanatory value of provenance for analysis \emph{after} the planning process.
We see value in integrating these provenance-based analyses \emph{online} into a continuous and dynamic planning environment, interleaving  provenance analysis and planning.
%
Some possibilities for using provenance while planning include: heuristic guidance (\eg, preferring choices based on higher-confidence information); %, all else being equal);
guiding contingency planning (\eg{}, prepare for more likely nondeterministic outcomes based on reliability of sensors, information sources, \etc);
or to %replanning (TODO).
plan repair (\eg{} triggering the planner to make revisions when provenance changes for the worse).

Furthermore, as a source of explainability to other agents, there is a potential for novel uses in multi-agent planning scenarios such as decentralized planning (\eg{}, evaluating other agents' performance to assess reliability of their action outcomes and relayed information)
and mixed-initiative planning (\eg{}, using the interface to detail the current plan's provenance and receive iterative changes from the user to parameters such as DIVE \textbf{Appraisals}).

% I cut this because the provenance analysis DOES have access to the planner's world model -- we track back to action models, method preconditions, and axioms.
% Our present provenance analysis has no access to the planner's internal world model, but using the planner's model might improve the provenance analyses, \eg{}, to inform the cataloging of plan elements in \figref{ss-index} or provide a source of DIVE \textbf{Appraisals} (\eg{}, confidence and assumptions) for plan methods and operators.

%%% Local Variables:
%%% mode: latex
%%% TeX-master: "main"
%%% End:
