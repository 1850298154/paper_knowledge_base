\section{Background}
\label{sec:background}

\begin{figure}
  \begin{center}
  \includegraphics[width=\linewidth]{figures/prov-ontology.pdf}
  \caption{PROV ontology subset used in our approach.}
  \label{fig:prov}
  \end{center}
\end{figure}

\subsection{Provenance-Tracking}
\label{sec:prov}

We utilize the PROV-O ontology \cite{lebo2013prov}, which expresses PROV Data Model's entities and relationships using the OWL2 Web Ontology Language.
The PROV Data Model includes the following three primary classes of elements to express provenance:
\begin{enumerate}
  \item \textbf{Entities} are real or hypothetical things with some fixed aspects in physical or conceptual space.  These may be beliefs, documents, databases, inferences, \etc
  \item \textbf{Activities} occur over a period of time, processing and/or generating entities.  These may be inference actions, judgment actions, planned (not yet performed) actions, \etc
  \item \textbf{Agents} are responsible for performing activities or generating entities.  These may be humans, machines, rovers, web services, \etc
\end{enumerate}

\noindent
The primary relationships over these three classes in PROV are shown in \figref{prov}, as detailed in the W3C PROV-O recommendation.\footnote{https://www.w3.org/TR/2013/REC-prov-o-20130430/}

Systems that utilize PROV-O, as specified in \figref{prov}, can represent long inferential chains, formally linking conclusions (\eg, a downstream belief) through generative activities (\eg, inference operations) and antecedents, to source entities and assumptions.
This comprises a directed network of provenance that we can traverse in either direction to answer questions of foundations, derivations, and impact.

\begin{figure}[t]
\centering
\includegraphics[width=\columnwidth]{newhtn} \vspace{-0.2in}
\caption{Delivery planning
example.} \label{fig:travel}%\vspace{-0.2in}
\end{figure}

\subsection{The DIVE Ontology}
\label{sec:dive}

The DIVE ontology \cite{friedman_tapp_2020} extends the PROV ontology with additional classes and relationships to appraise information and validate information workflows.
For this work, we use DIVE's \textbf{Appraisal} class, which is an \textbf{Agent}'s judgment about an activity, entity, or other agent.

For example, we express a DIVE \textbf{Appraisal} about a GPS sensor---from which we derive beliefs about the world before planning and during plan execution---with moderate baseline confidence.
This baseline confidence in our GPS sensor may affect our confidence of the information it emits, all else being equal, which may ultimately impact our judgment of the success likelihood of our planned actions.

We also use DIVE to express \textit{collection disciplines} such as GEOINT (geospatial), IMINT (image), and other types of information for all relevant information sources, beliefs, and sensors involved in a plan.
DIVE is expressed at the meta-level of PROV.
DIVE expressions flow through the network to facilitate downstream quality judgments and interpretation, as we demonstrate in this work.

\hide{
\subsection{Truth-Maintenance Systems}
\label{sec:tms}

Truth-Maintenance Systems (TMSs) \cite{forbus1993building,friedman2018csj} explicitly store entities alongside \emph{justifications} that link antecedent entities (analogous to PROV entities) with consequent entities.
This explicitly encodes the rationale for each entity, so --- similar to the PROV ontology --- we can use a TMS to explore foundations, derivations, and impact.

TMSs track \emph{environments} as sets of elements that sufficiently justify an entity in its upstream lineage.
If the lineage changes (e.g., due to a new derivation of an entity), the TMS recomputes the affected environments.
Environments allow TMSs to efficiently recognize contradictions, retrieve logical rationale, and identify upstream assumptions \cite{de1986assumption}.
TMSs operate alongside inference engines to record the lineage and logical conditions for believing various assertions; they do not themselves generate inferences or derive entities.
Our approach utilizes TMS-like environments to efficiently refute information, propagate confidence, and visualize impact.
}

\begin{figure*}
  \begin{center}
  \includegraphics[width=\textwidth]{figures/prov-to-plan.pdf}
  \caption{How we represent (with PROV) and display (with D3.js) the provenance of \shop plans.}
  \label{fig:prov-to-plan}
  \end{center}
\end{figure*}

\subsection{The SHOP3 HTN Planner}

\label{sec:shop3}

\shop~\cite{GoldmanKuter:SHOP3ELS} is the successor to the \pred HTN
planner~\cite{nau2003shop2} developed at the University of Maryland.
Unlike a first principles planner, an HTN planner produces a sequence of actions that perform some activity or
\emph{task}, instead of finding a path to a goal state.
An HTN planning domain includes a set of planning {\em operators}
(actions) and
\emph{methods}, each of which is a prescription for how to decompose a task into its
\emph{subtasks} (smaller tasks). The description of a planning problem contains an initial state as
in classical planning. Instead of a goal formula, however, there is a partially-ordered set of
tasks to accomplish.
Planning proceeds by decomposing tasks recursively into subtasks, until
\emph{primitive tasks}, which can be performed directly using the planning operators, are reached.
For each task, the planner chooses an applicable method, instantiates it to decompose the task into
subtasks, and then chooses and instantiates other methods to decompose the subtasks even further.
If the constraints on the subtasks or the interactions among them prevent the plan from being
feasible, the planner will backtrack and try other methods.
\figref{travel} illustrates how \shop HTN domains are described and used for planning in a Delivery planning example.

\shop is an HTN planner that generates actions in the
order they will be executed in the world (hence ``hierarchical
\emph{ordered} planner'' in the name).
\pred is relatively efficient and well-tested, and
it performed well in the 2002 IPC--the last IPC in which HTN planners competed~\cite{long03:_inter_plann_compet}.
\pred also has been used in a number of planning applications, including recently at SIFT for
Air Operations and UAV planning \cite{mueller17human},
cyber security, cyber-physical systems,
planning for synthetic biology experiments \cite{kuter18xplan}, and more.
For an earlier survey of SHOP2 applications,
see Nau, \etal~\shortcite{nau05applications}.
\shop retains the essential
features of \pred, but has a modernized codebase, is easier to extend
(\eg{} with plan repair capabilities, new input languages, \etc), and
an alternative, more efficient search engine.


%%% Local Variables:
%%% mode: latex
%%% TeX-master: "main"
%%% End:
