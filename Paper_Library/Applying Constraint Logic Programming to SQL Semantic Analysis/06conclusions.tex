\section{Conclusions and Future Work}
\label{sect:conclusions}

We have presented a system using constraint logic programming for the semantic analysis of SQL statements (both DML and DDL).
With the aim of detecting possible misuses of syntactically correct SQL statements at compile-time, this system focuses on both metadata and statements, instead of data from tables.
There have been other approaches to SQL analysis (targeted at comparing results for concrete database instances, and based on CBM techniques), but ours mainly follows the same path as \mytt{sqllint}.
However, instead of using consistency techniques as in that work, we use CLP constraints and solver cooperation to develop a precise analysis, which can deal with non-linear conditions and queries as complex as needed.
Reasoning at the logic level eases the development of this approach, instead of using the more cumbersome SQL formulations for consistency checking.
Performance data show that the approach is practical, and well able to cope with queries that other systems cannot afford.

Despite we have successfully evaluated the tool in classroom and students have appreciated the semantic feedback, a more thorough evaluation must be done.
As part of a teaching innovation project, we are currently analysing the tool with both on-line questionnaires provided to students, and logging user sessions.
While questionnaires include selectable answers in a Likert scale (and also open answers to express additional specific comments), logs can be inspected to observe the reaction of students to the semantic warnings.
In addition, there is ample room for future work as, for example, the development (most likely, with CHR) of specific solvers for types such as strings and dates (in addition to the already used, but simpler, domain $\mathscr{H}$).
Taking into account bindings and domain pruning in negated CLP goals resulting from the translation of constructs such as \mytt{NOT IN} and \mytt{NOT EXISTS} would also increase the precision of the analysis.
Also, the tool might propose simplified versions of conditions by decompiling the CLP constraint store into SQL.
Other potential additions include: a Boolean type and its handling with a CLP($\mathscr{B}$) solver; taking advantage of subtypes (exact numeric types with limited range, and string subtypes with bounded size); and implementing projections \cite{DBLP:journals/corr/abs-0904-2136}.
Finally, the proposal in this work can be applied to the semantic analysis of Datalog queries and programs, and to other scenarios such as the verification of automatically-generated SQL queries.
%Finally, we plan to evaluate the tool in classroom by automatic analysis of student system session logs.

