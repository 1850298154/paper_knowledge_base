\section{Related Work}
\label{sect:related-work}

There are many tools targeted at learning SQL focusing on the answers of queries with respect to database instances, i.e., comparing the output of queries with the output of reference queries provided by experts 
(e.g.,
%ACME \cite{soler2006web}, 
SQLator \cite{Sadiq:2004:SOS:1026487.1008055}, 
WebSQL \cite{Allen2000WebSQLAI}, 
AsseSQL \cite{Prior:2014:AOB:2591708.2602682}, 
%SQLZoo\footnote{\url{https://sqlzoo.net}},
%HackerRank\footnote{\url{https://www.hackerrank.com/}},
%LearnSQL % No está disponible en abierto, no se encuentran referencias.
%SQLJudge \cite{SQLJudge}, 
and
QueryViz \cite{DBLP:journals/pvldb/Gatterbauer11}).
Other set of tools are focused on the semantic aspects of queries as SQL Tutor \cite{Mitrovic2012} which uses Constraint-Based Modelling (CBM) to form models of its students, enabling the automatic selection of problems based on these models.
Another one is SQL-LTM \cite{Dollinger20103i}, a tutoring module relying on reference queries to which student queries are compared.

The semantic-based system \mytt{sqllint} \cite{Brass:2006:SES:1183058.1183064} is the closest approach to ours. 
They introduce the concept of soft keys (attributes used in practice for identifying tuples, but that can have duplicates; e.g., the name of a person) which would be useful for some of their checks.
A description of the way to identify such errors is given in \cite{1579136}, relying on \textit{ad-hoc} consistency checks somewhat based on classical techniques \cite{Guo1996SolvingSA}.
With respect to subqueries, it only supports \mytt{EXISTS}  (no \mytt{IN}, \mytt{>= ALL}, \ldots)
Their approach neither supports aggregates, nor \mytt{UNION}, nor \mytt{LIKE}, and nor \mytt{IS [NOT] NULL}.
As types, it includes only strings and integers, % constants, and no other numeric constants.
and expressions are not allowed.
Finally, it does not support \mytt{CHECK} constraints in table definitions.
%Both \mytt{sqllint} and DES systems deal with error numbers 1, 2, 3, 4, 5, 8 and 27; in addition to this, \mytt{sqllint} deals with errors 34 and 39, while DES deals with errors 6, 7, 9, 11, 12, 13, 16, 17, 32, and 33.
Note also that, in contrast to DES, \mytt{sqllint} is only an analyzer, not a complete SQL system with a solving engine which could be used for teaching. 

