%\documentclass[draftclsnofoot,onecolumn,12pt,romanappendics]{IEEEtran}
%\documentclass[draftcls,onecolumn,12pt,romanappendics]{IEEEtran}
\documentclass[journal]{IEEEtran}
%\documentclass[conference]{IEEEtran}
\usepackage{amsfonts}
\usepackage{slashbox}
\usepackage{amsmath}
\usepackage{multirow}
\usepackage{graphicx}
\usepackage{amsfonts}
\usepackage{amsmath,epsfig}
\usepackage{mathbbold}
\usepackage{stfloats}
%\usepackage{stmaryrd}
\usepackage{amssymb}
%\usepackage{hyperref}
\usepackage{url}
%\usepackage{hyperref}
\usepackage{bm}
\usepackage{cite}
\usepackage{amsmath}
\usepackage{amssymb}
\usepackage{latexsym}
\usepackage{multirow}
\usepackage{epsfig}
\usepackage{graphics}
\usepackage{mathrsfs}
\usepackage{algorithmic}
\usepackage{algorithm}
\usepackage{subfigure}
\usepackage{slashbox}
\usepackage[all]{xy}


\usepackage{pifont}
\usepackage{bbding}
\usepackage{stmaryrd}
\usepackage{amssymb}
\usepackage{amsfonts}
\usepackage{epic}
%\usepackage{algpseudocode}
\usepackage{stfloats}
\usepackage{latexsym}
\usepackage{epstopdf}
\usepackage{epic}
\usepackage{bm}
\usepackage{xcolor}
%\documentclass[journal]{IEEEtran}
\usepackage{mathrsfs}
\usepackage{pifont}
\usepackage{bbding}
\usepackage{amsmath,epsfig}
\usepackage{mathbbold}
\usepackage{stmaryrd}
\usepackage{amssymb}
\usepackage{amsfonts}
\usepackage{epic}
\usepackage{graphicx}
\usepackage{subfigure}
\usepackage{enumerate}
\usepackage{stfloats}
\usepackage{latexsym}
\usepackage{epstopdf}
\usepackage{epic}
\usepackage{multirow}
\usepackage{stfloats}
\usepackage{bm}

\usepackage{booktabs}
\usepackage{color}


%\usepackage{threeparttable}
%\usepackage{booktabs}
%\usepackage{algorithmicx}
%\usepackage[numbers,sort&compress]{natbib}

%\setlength{\unitlength}{1mm} \setlength{\parindent}{3.5mm}
%
\newcommand{\reffig}[1]{Fig. \ref{#1}}
\newcommand{\figref}[1]{\figurename~\ref{#1}}
\newtheorem{definition}{\underline{Definition}}%[section]
\newtheorem{theorem}{\underline{Theorem}}%[section]
\newtheorem{corollary}{\underline{Corollary}}%[section]
\newtheorem{proposition}{\underline{Proposition}}%[section]
\newtheorem{lemma}{\underline{Lemma}}%[section]
\newtheorem{property}{Property}%[section]
\newtheorem{remark}{\underline{Remark}}%[section]
\renewcommand{\algorithmicrequire}{\textbf{Input:}}  %Use Input in the format of Algorithm
\renewcommand{\algorithmicensure}{\textbf{Output:}}  %UseOutput in the format of Algorithm
%%%%%%%%%%%%%%%%%%%%%%%

%%%% Kwan's operators :)
\DeclareMathOperator{\Tr}{\mathrm{Tr}}
\DeclareMathOperator{\zero}{\mathbf{0}}
\DeclareMathOperator{\Rank}{\mathrm{Rank}}
\DeclareMathOperator{\nullspace}{\mathrm{Null}}
\DeclareMathOperator{\diag}{\mathrm{diag}}
\DeclareMathOperator{\vect}{\mathrm{vec}}
\DeclareMathOperator{\maxo}{maximize}
\DeclareMathOperator{\mino}{minimize}
\DeclareMathOperator{\bigo}{\cal O}
 \newcommand{\qed}{\hfill \ensuremath{\blacksquare}}
\newcommand{\q}{\mathbf q}
\newcommand{\Q}{\mathbf Q}
\newcommand{\A}{\mathbf A}
\newcommand{\w}{\mathbf w}
\newcommand{\pow}{\mathbf P}
\newcommand{\myfont}{\fontsize{8.5pt}{\baselineskip}\selectfont}

%\newcommand{\}{\mathbf P}
%%%%%%%%%%%%%%%%%%%%%%%%%%%%%%%
\begin{document}

\title{Joint Trajectory and Communication Design for Multi-UAV Enabled Wireless Networks }
 %Overlaying cooperative cellular networks:  Energy-Efficient D2D Communications with spectrum-power trading
\author{\IEEEauthorblockN{Qingqing Wu,  \emph{Member, IEEE}, Yong Zeng,  \emph{Member, IEEE}, and Rui Zhang, \emph{Fellow, IEEE}
\thanks{ The authors are with the Department of Electrical and Computer Engineering, National University of Singapore, email:\{elewuqq, elezeng, elezhang\}@nus.edu.sg. This work was presented in part at IEEE GLOBECOM 2017 \cite{wu2017joint}. This work was supported by the National University of Singapore under Research Grant R-263-000-B62-112. }}  }



\maketitle

\begin{abstract}
Unmanned aerial vehicles (UAVs) have attracted significant interest recently in assisting  wireless communication due to their high maneuverability, flexible deployment, and low cost.  This paper considers a multi-UAV enabled wireless communication system, where multiple UAV-mounted  aerial base stations (BSs) are employed to serve a group of users on the ground.  To achieve fair performance among users, we maximize the minimum throughput over all ground users  in the downlink communication by optimizing the multiuser communication scheduling and association jointly with the  UAVs' trajectory and power control. The formulated problem is  a mixed integer non-convex optimization problem that is challenging  to solve.  As such, we propose an efficient iterative algorithm for solving it by applying the block coordinate descent and successive convex optimization techniques. Specifically, the user scheduling and association, UAV trajectory, and transmit power are alternately optimized in each iteration.  In particular, for the non-convex  UAV trajectory and transmit power optimization problems, two approximate convex optimization problems are solved, respectively. We further show that  the proposed algorithm  is guaranteed to converge. To speed up the algorithm convergence and achieve good throughput, a low-complexity and systematic initialization scheme is also proposed for the UAV trajectory design based on the simple circular trajectory and the circle packing scheme. Extensive simulation results are provided to demonstrate the significant throughput gains of the proposed design as compared to other benchmark schemes.
%achieved by exploiting the UAV mobility and the trajectory design for wireless networks.
%The proposed trajectory design also outperforms the circular trajectory significantly.

%We propose an iterative optimization framework which is guaranteed to converge to a local optimal solution. Specifically,  the original optimization problem is divided into into three sub-problems and in each iteration, we alternatively optimize one blocks of variables while keeping the other two fixed. By exploiting the relaxation and the successive convex optimization technique, all the three sub-problems are transformed into convex optimization problem.
\end{abstract}

\begin{IEEEkeywords}
UAV communications, throughput maximization, optimization, trajectory design, mobility control.
\end{IEEEkeywords}
%\newpage
\section{Introduction}
Unmanned aerial vehicles (UAVs), also commonly known as drones, have attracted significant attention in the past decade for various applications, such as surveillance and monitoring, aerial imaging, cargo delivery, etc \cite{valavanis2014handbook}.
%This is attributed to the significant advancements in the UAV technology, such as increased payload capacity, longer average flight duration.
As reported in \cite{globalUAVmarket}, the global market for commercial UAV applications, estimated at about 2 billion dollars in 2016, will skyrocket to as much as 127 billion dollars by 2020.
Equipped with advanced transceivers and batteries, UAVs are gaining increasing popularity in information technology (IT) applications  due to their high
maneuverability and flexibility for on-demand deployment. In particular, UAVs typically have high possibilities of line-of-sight (LoS) air-to-ground communication links, which are appealing to the wireless service providers \cite{lin2017sky}. To capitalize on this growing opportunity,  several leading IT companies have launched pilot projects,  such as Project Aquila by Facebook \cite{fb_UAV} and Project Loon by Google \cite{gl_UAV},   for providing ubiquitous internet access worldwide by leveraging the UAV/drone technology. The 3rd Generation Partnership Project (3GPP) is also looking up into the sky and studying aerial vehicles supported by Long Term Evolution (LTE) where the initial focus is on UAV \cite{3gpp}. In fact, with the approval of Federal Aviation Administration (FAA), Qualcomm and AT\&T have optimized  LTE networks for UAV communications  \cite{qualcom_UAV}, which aims to pave the way to a wide-scale deployment of UAVs in the upcoming fifth generation (5G) wireless networks, especially for mission-critical use cases.
 %in a mixed and real-time testing environment which includes  commercial zones and populated residential and uninhabited areas.
% It has been shown that LTE can provide coverage to UAVs at altitudes up to 400ft and the tested UAVs are capable of demonstrating seamless handovers between different base stations (BSs) during flights with zero link failures. %  It is also anticipated that  UAVs will be widely deployed in the upcoming fifth generation (5G) networks, specifically for mission-critical use cases.
 %Qualcomm and AT\&T also announced to test drones on cellular network to which would accelerate wide-scale deployment.
 Meanwhile, extensive research efforts from the academia have also been devoted to employing UAVs as different types of
   wireless communication platforms  \cite{zeng2016wireless}, such as aerial mobile base stations (BSs) \cite{mozaffari2016unmanned,mozaffari2016efficient,al2014optimal,lyu2016placement,bor2016efficient}, mobile relays \cite{zhan2011wireless,zeng2016throughput}, and flying computing cloudlets \cite{loke2015internet,jeong2016mobile}. In particular,  employing UAVs as aerial BSs is envisioned as a promising solution to enhance the performance of the existing cellular systems. Depending on whether the UAV's high mobility is exploited or not, two different lines of research can be identified in the literature, namely static-UAV or mobile-UAV enabled wireless networks.
 % to integrating UAVs into a wide range of applications in wireless networks \cite{zeng2016wireless}.
%and are also envisioned as a promising technology for wireless communications due to their capabilities for coverage extension and capacity enhancement, which is particularly important for
% As most aerial work requires low altitude and low speed operation, rotary  wing UAVs are more popular in the field of civilian UAVs.
%Small-scale fixed-wing UAVs as research platforms are generally less popular than the rotorcraft counterparts.
% have found a wide range of applications for wireless networks
%Recently, UAVs has been investigated as different kinds of wireless communication platforms, such as aerial base stations (BSs)
%unmanned aerial vehicles (UAVs) can serve a multitude of purposes such as surveillance,
%localization and communication, making them a flexible solution to augment and enhance the
%capabilities of the current cellular systems
%Employing UAV as a flying base station has many advantages. It is more likely to have a higher possibility of enjoying line-of-sight channels.
%to facilitate temporary hot spots and compensate network outages in case of public events and emergencies
%Use of single or multiple UAVs as communication relays,  aerial base stations,  for network provisioning in emergency situations and for public
%safety communications has been of particular interest due to their fast deployment and large coverage capabilities.
%In general, there are two main types of UAVs, rotary-wing and fixed-wing UAVs. rotary-wing UAV has a better maneuverability.
%UAVs have been exploited to achieve different functionalities for wireless communication systems, such as an aerial BS, a mobile relay, and a flying computing center.
%Depending on whether the UAV mobility is exploited or not, two different lines of research can be identified along this direction, i.e., static-UAV based and mobile-UAV based wireless networks.
%mozaffari2016efficient,mozaffari2016optimal,
%  ,chetlur2017downlink
%mozaffari2015drone, ,sharma2016uavs
%,kandeepan2014aerial

 The research on the static-UAV enabled wireless networks mainly focuses on the UAV deployment/placement optimization \cite{mozaffari2016unmanned,al2014optimal,mozaffari2016efficient,lyu2016placement,bor2016efficient}, with the UAVs serving as aerial quasi-static BSs to support ground users in a given area  from a certain altitude. %, which can be considered as an aerial static BS that transmits or receives signals
%However, the channel characteristic of the air-to-ground link is quite different from that of the conventional ground-to-ground link. In fact, a higher altitude, %although leading to a larger pathloss, also provides a higher probability of enjoying line-of-sight (LoS) channels.
As such, the altitude and the horizontal location of the UAV can be either separately or jointly optimized for different quality-of-sevice (QoS) requirements. In particular, the authors in \cite{al2014optimal} provide an analytical approach to optimize the altitude of a UAV for providing maximum coverage for ground users.
In contrast, by fixing the altitude, the horizontal positions of UAVs are optimized  in \cite{lyu2016placement} to minimize the number of required UAV BSs  to cover a given set of ground users.  In three-dimensional (3D) space, a drone-enabled small cell placement optimization problem is investigated in \cite{bor2016efficient} to maximize the number of users that can be covered.% The resulting problem is then solved via a numerical approach.
%these two works are further extended to \cite{bor2016efficient} where a drone small cell placement optimization problem in 3-D space is considered by
%A similar problem is also studied in \cite{bor2016efficient} for a drone-enabled small cell placement optimization in three-dimensional (3D)  space.

%By leveraging stochastic geometry, the authors in \cite{chetlur2017downlink} develop a general framework for the downlink coverage analysis of a network with multiple static UAVs, which also serves as a foundation for the investigation of co-existence of UAV networks and terrestrial cellular networks.

%the donwlink outage probabilities for  modeling the network with multiple UAVs  as a uniform binomial point process,


Besides the UAV placement optimization,  exploiting the UAV's high mobility in the mobile-UAV enabled wireless networks is anticipated to unlock the full potential of UAV-ground communications. With the fully controllable UAV mobility,  the communication distance between the UAV and ground users can be significantly shortened by proper UAV trajectory design and user scheduling. This is analogous and yet in sharp contrast to the existing small-cell technology  \cite{wu2016energy,wu2016overview,zhang2016fundamental,wang2017joint}, where the cell radius is  reduced by increasing the number of small-cell BSs deployed, but at the cost of increased infrastructure expenditure. Motivated by this, the UAV trajectory design is rigorously studied in \cite{zeng2016throughput} and \cite{zeng2016energy} for a mobile relaying system and point-to-point energy-efficient system, respectively, where sequential convex optimization techniques are applied to solve the non-convex trajectory optimization problems therein. Though providing a general framework for trajectory optimization in two-dimensional (2D) space, the studies in \cite{zeng2016throughput} and \cite{zeng2016energy} only focus on the setup with single UAV and single ground user. For UAV-enabled multi-user system,  a novel cyclical multiple access scheme is proposed in \cite{lyu2016cyclical}, where the UAV communicates with ground users when it flies sufficiently close to each of them in a periodic (cyclical) time-division manner.
An interesting throughput-access delay tradeoff is revealed and it has been shown that significant throughput gains can be achieved over the case of a static UAV for delay-tolerant applications. However,  only one single UAV with the constant flying speed is considered in \cite{lyu2016cyclical}, and the ground users are assumed to be uniformly located in a one-dimensional (1D) line, which simplifies the analysis but limits the applicability in practice.

% The research interest of the other line aims at exploiting the mobility of UAVs which is anticipated to further unlock the full potential of UAV-enabled communications in 3D free space. With the fully controllable UAV mobility,  the transmitter-receiver distance can be significantly shortened by proper UAV trajectory optimization. However, in the fifth generation (5G) wireless networks, such performance is in general realized by keeping decreasing the cell radius while at the cost of increasing the number of conventional BSs, which may largely increase the infrastructure expenditure \cite{wu2016overview}.
% %In fact, 80\% the BS are quite loaded for 80\% of the time. Employing UAV BS provide a flexible and efficient solution for handling dynamic traffics.
%%Therefore, the trajectory design is of paramount importance for fully exploring the advantages of the mobility.
%%However, the dual benefits brought by the mobility in 3D free space do not come for free and is in fact at the sacrifice of user access delay.
%%In \cite{lyu2016cyclical}, a fundamental \emph{throughput-delay tradeoff} has been characterized for a single UAV-enabled wireless network.
% Motivated by this, a novel cyclical multiple access scheme is proposed in  \cite{lyu2016cyclical} where a UAV schedules a user that it flies close to in a time division manner. It has been shown that significant throughput gains are achieved over the static UAV for delay-tolerant applications.
%%In practice, the user access delay arising from cyclical multiple access can be reduced by various approaches such as increasing the number and/or the maximum speed of UAVs.
%The trajectory design is also studied in \cite{mozaffari2016unmanned} where a single UAV is despatched to serve a set of users. However, the UAV is only assumed to transmit signals at some pre-defined locations and the trajectory is thereby heavily restricted while regardless of the transmission, leading to an underestimated performance gain of trajectory design.
%each user is only allowed to communicate with the UAV when the UAV gets close to it.

%Many ways can be employed to reduce the user access delay, such as . It is more suitable to periodic and  delay tolerant services.
%xxx
% \cite{bor2016efficient,gupta2016survey,zhan2011wireless}
% \cite{jeong2016mobile,kandeepan2014aerial,loke2015internet,chen2016caching}
%users are scheduled to communicate with the UAV in a cyclical time-division manner
%However, it has to be note that the performance does not come for free, which is in fact at the increase of user access delay, since each user has to wait for UAV to get close to it.


%% cost effective and time efficient
%in practice due to their relatively low costs.
 %However, the joint communication and trajectory coordination among multiple UAVs poses new challenges.
%Therefore, because
%of their versatility, flexibility, ease of deployment and,  the user scheduling, UAV trajectory, and power control have to be jointly considered.
%Compared to the single UAV network, a multi-UAV network has the advantage of improving network performance and reducing the user access delay.
%,li2016energy

In this paper, we study a general multi-UAV enabled wireless communication system, where multiple UAVs are employed to serve a group of users on the ground in a given 2D area.
Although a single UAV has demonstrated its advantages in performance enhancement for wireless networks  \cite{zeng2016throughput,zeng2016energy,wu2017joint, guangchi2016_UAV,JR:guangchi2016_UAV,wu2017ofdma,JR:wu2017_ofdm,yang2017energy},  it has limited capability in general and may not guarantee availability during the entire mission due to its practical size, weight and power (SWAP) constraints \cite{zeng2016wireless}.  This thus motivates the deployment of multiple or a swarm of UAVs which cooperatively serve the ground users to achieve more efficient communications.  For example, a group of UAVs may be deployed to keep track of the participants in a large-area event and to form a multi-hop communication network connecting to the ground audience.  More importantly, in a multi-UAV enabled network, users could be served in parallel with higher throughput and lower access delay, which could effectively alleviate the fundamental throughput-access delay tradeoff in single-UAV communications \cite{lyu2016cyclical}.% Therefore, to reap the benefits of UAV-enabled wireless networks, the use of a network employing multiple UAVs for information transmission has been foreseen as the main trend of UAV-enabled communications in future.

Without loss of generality, we consider that all UAVs share the same frequency band  for their communications with the ground users.
 By focusing on the downlink transmission from the UAVs to ground users, our goal is to maximize the minimum average rate among all users by jointly optimizing the user communication scheduling and association, and the UAV trajectory and transmit power control in a given finite period.
%Different from \cite{lyu2016cyclical}, users are freely located on the ground and  the trajectories of multiple UAVs can be optimized in 2D along with the multiuser communication scheduling and transmit power control.
%In general, civilian UAVs fall into two main categories: fixed-wing types or rotary-wing types.  Rotorcraft UAVs as research platforms are more favourable than its fixed-wing counterparts due to its better flight maneuverability such as hovering very close to the target. Thus, we assume rotary-wing UAVs for information transmission in this paper such that the hovering functionality can be exploited for performance improvement.
Such a joint optimization problem is practically appealing,  but has not been investigated in the literature to the authors'  best knowledge.
%Based on this, we formulate the max-min average data rate optimization problem by jointly considering the user scheduling, UAV trajectory, and power control.
 %Such a joint design is particularly attractive and expected to enhance the system performance substantially.
On one hand, by properly  designing the trajectories of different UAVs, not only short-distance LoS links can be proactively and dynamically established for those desired UAV-user pairs, but also the interfering channel distances between the undesired UAV-user pairs can be enlarged to
alleviate the co-channel interference. On the other hand, in the occasional scenarios when the UAVs have to get close with each other for serving nearby users,  their transmission power can be adjusted to reduce interference. While maximum transmission power is used for maximizing spectrum efficiency when the UAVs are far apart to serve users that are well separated. Therefore, the system performance can benefit from different design dimensions of the proposed joint optimization.
 However, such a joint  trajectory and adaptive communication design problem is non-trivial to solve. %On one hand, with any given user scheduling and association, it is intuitive that the UAV should visit users according to the order that users are scheduled for communication to achieve short-distance links. On the other hand, for any fixed UAV trajectory, the UAV should accordingly schedule the users for communication based on their distances to it.
 %In addition, the coordination among multiple UAVs for trajectory design and power control significantly complicate the problem. Finally, unlike the terrestrial wireless networks, a user with a fixed ground position may be scheduled by different UAVs in different time instants due to the UAV mobility.
 %As a consequence,
  This is because the user scheduling and association, UAV trajectory optimization, and transmit power control are closely coupled with each other in our considered problem, which makes it challenging to solve in general.

 To tackle the above challenges, we first relax the binary variables for user scheduling and association into continuous variables and solve the resulting problem with an efficient iterative algorithm  by leveraging the block coordinate descent method \cite{hong2016unified}.
 Specifically, the entire optimization variables are  partitioned into three blocks for the user scheduling and association, UAV trajectory, and transmit power control, respectively. Then, these three blocks of variables are alternately optimized in each iteration, i.e., one block is optimized at each time while keeping the other two blocks fixed.
  %Specifically, one of the three blocks of variables for the user scheduling, UAV trajectory, and transmit power control is optimized alternately in each iteration, while keeping the other two blocks fixed.
  However, even with fixed user scheduling and association, the UAV trajectory optimization problem with fixed power control and the UAV power control problem with fixed trajectory are still difficult to solve due to their non-convexity. We thus apply the successive convex optimization technique to solve them approximately.
  %To the best of our knowledge, this is the first work that considers joint trajectory and communication design for multi-UAV networks.
 %In this paper, we propose an efficient iterative optimization algorithm by leveraging the block coordinate descent method and the successive convex optimization technique.
 %It has been shown that the proposed algorithm is guaranteed to converge to at least a local optimal solution.
We also show that our proposed algorithm is guaranteed to converge. To speed up the algorithm convergence and achieve a superior performance, we propose an efficient and systematic trajectory initialization scheme based on the simple circular trajectory and the circle packing scheme. Numerical results show that significant throughput gains are achieved by our proposed joint design, as compared to conventional static UAV or other benchmark schemes with heuristic UAV trajectories. It is also shown that the throughput of the proposed mobile UAV system increases with the UAV trajectory design period, revealing the general throughput-access delay tradeoff \cite{wu2017joint,lyu2016cyclical} in multi-UAV enabled communications. In addition, compared to the single-UAV case, this  tradeoff is shown to be significantly improved by the use of multiple UAVs.


The rest of this paper is organized as follows. Section II introduces the system model and the problem formulation for a multi-UAV enabled wireless network.
In Section III, we propose an efficient iterative algorithm by applying the block coordinate descent and the successive convex optimization techniques.
Section VI presents the numerical results to demonstrate the performance of the proposed design. Finally, we conclude the paper  in Section VI.

\emph{Notations:} In this paper, scalars are denoted by italic letters, vectors and matrices are respectively denoted by bold-face lower-case and upper-case letters. $\mathbb{R}^{M\times 1}$ denotes the space of $M$-dimensional real-valued vector. For a vector $\mathbf{a}$, $\|\mathbf{a}\|$ represents its Euclidean norm and $\mathbf{a}^T$ denotes its transpose. For a time-dependent function $\mathbf{x}(t)$,  $\dot{\mathbf{x}}(t)$ denotes the derivative with respect to time $t$. For a set $\mathcal{K}$, $|\mathcal{K}|$ denotes its cardinality.
\section{System Model and Problem Formulation}

 \begin{figure}[!t]
\centering
\includegraphics[width=0.4\textwidth]{UAV_IFC7.eps}
\caption{A multi-UAV enabled wireless network. } \label{UAV}
\end{figure}
\subsection{System Model}
As shown in Fig. 1, we consider a wireless communication system where $M\geq 1 $ UAVs are employed to serve a group of $K>1$ ground users.  The user and UAV sets are denoted as $\mathcal{K}$ and $\mathcal{M}$, respectively, where $|\mathcal{K}|=K$ and $|\mathcal{M}|=M$. This practically corresponds to an information broadcast system enabled by UAVs.
 %Furthermore,  the solution and the information waiting to be broadcast are  embedded into the control and communication components of UAVs in advance. Thus, inter-UAV communications are not needed after they are dispatched.
  Assume that all the UAVs share the same frequency band for communication over consecutive periods each of duration  $T >0$ in second (s).  During any period,  each of the UAVs serves its associated ground users via a periodic/cyclical time-division multiple access (TDMA).
 Note that the choice of $T$ has a significant impact on the system performance. On one hand, thanks to the UAV mobility, a larger period  $T$ provides more time for each UAV to move closer to its served users  to achieve better communication channels, as well as to fly sufficiently away from the users served by other UAVs  for more effective interference mitigation,  thus achieving  a higher throughput.
% Intuitively, as $T$ gets sufficiently large so that the UAV flying time could be practically ignored, UAVs can stay stationary at the optimized positions above each user to maintain best channels and maximize the throughput.
{On the other hand, a larger $T$ in general implies a larger access delay for users since each user may need to wait for a longer time to be scheduled to communicate with a UAV between two periods.}
{Therefore, the period $T$ needs to be properly chosen in practice to strike a balance between the user throughput and access delay, i.e., there exists a fundamental throughput-access delay tradeoff \cite{lyu2016cyclical} in UAV-enabled communications.}
%We focus on a finite optimization period with duration $T$. Here, the choice of $T$ as well as its impact on the system is discussed as follows.
%Due to the UAV mobility, a larger optimization period  $T$ in general provides UAVs more time to achieve satisfactory direct signal links and avoid severe interference, which results in a higher performance. However, a larger $T$ correspondingly implies a larger access delay for users due to the fundamental throughput-delay tradeoff in cyclical multiple access \cite{zeng2016wireless}. Therefore, the optimization period $T$ can be properly chosen in practice to strike a balance between the system performance and the maximum user access delay.


Without loss of generality, we consider a 3D Cartesian coordinate system where  the horizontal coordinate of each ground user $k$ is fixed at ${\mathbf{w}}_{k}=[x_k,y_k]^T \in \mathbb{R}^{2\times 1}$, $k\in \mathcal{K}$. All UAVs are assumed to fly at a fixed altitude $H$ above ground and the time-varying horizontal coordinate of UAV $m\in \mathcal{M}$ at time instant $t$ is denoted by $\mathbf{q}_m(t)=[x_m(t), y_m(t)]^T\in \mathbb{R}^{2\times 1}$, with $0\leq t\leq T$. The UAV trajectories need to satisfy the following constraint
\begin{align}
\q_m(0) &= \q_m(T), \forall\, m, \label{eq01}
\end{align}
which implies that each UAV needs to return to its initial location by the end of each period $T$ such that users can be served periodically in the next period.  { In practice, the trajectories of UAVs are also subject to the maximum speed constraints\footnote{Here, we do not consider the minimum speed constraints, which is practically valid for the rotary-wing UAVs with the capability of keeping stationary at fixed positions, i.e., a minimum zero-speed is feasible. However, for the fixed-wing UAVs that must move forward to remain aloft, additional minimum speed constraints, i.e., $||\dot{\q}_m(t)||\geq V_{\min}>0$, $0\leq t \leq T, \forall\, m$, need to be imposed  \cite{zeng2016energy}, which can be handled by the proposed algorithm with only a minor modification.} and collision avoidance constraints, i.e,
\begin{align}
||\dot{\q}_m(t)|| &\leq V_{\max},  0\leq t \leq T, \forall\, m, \label{eq02}\\
|| \q_m(t)-  \q_{j}(t)|| &\geq d_{\min},  0\leq t \leq T, \forall\, j\neq m, \label{eq0220}
\end{align}
where $V_{\max}$ in \eqref{eq02} denotes the maximum UAV speed in meter/second (m/s) and $d_{\min}$ denotes the minimum inter-UAV distance in m to ensure collision avoidance.}
% which are assumed identical for all UAVs without loss of generality.
%Note that  the desired trajectories of UAVs $\q_m(t)$ are in general continuous functions of system parameters, which imposes much difficulty for the practical trajectory design.  To make it more tractable, linear discrete state-space approximation can be applied which has been widely adopted in simplifying a complicated continuous process such as energy harvesting. Specifically,  we assume that
For ease of exposition, the  period $T$ is  discretized into $N$ equal-time slots, indexed by $n=1,...,N$.  The elemental slot length $\delta_t = \frac{T}{N}$ is chosen to be sufficiently small such that a UAV's location is considered as approximately unchanged within each time slot even at the maximum speed $V_{\max}$. As a result, the trajectory of UAV $m$  can be approximated by the $N$ two-dimensional sequences $\mathbf{q}_m[n]=[x_m[n], y_m[n]]^T$, $n=1,\cdots, N$.  Furthermore, the trajectory constraints (\ref{eq01})--(\ref{eq0220}) can be equivalently written as
{
\begin{align}
\q_m[1] &= \q_m[N],\\
||\mathbf{q}_m[n+1]-\mathbf{q}_m[n]||^2 &\leq S_{\max}^2,  n=1,...,N-1, \\
|| \q_m[n]-  \q_{j}[n]||^2 &\geq d^2_{\min}, \forall\, n, m, j\neq m,  \label{eq022}
\end{align}}
\kern -1.6mm where $S_{\max} \triangleq V_{\max}\delta_t$ is the maximum horizontal distance that the UAV can travel in each time slot.
%we assume that the optimization period $T$ is further discretized into $N$ equally spaced time slots, indexed by $n=1,...,N$.  Then, the elemental slot length denoted as $\delta_t = \frac{T}{N}$ can be chosen sufficiently small such that a UAV's location is considered as unchanged within each time slot  \cite{zeng2016throughput}.  In this regard, the trajectory of UAV $m$, $\mathbf{q}_m(t)=[x_m(t), y_m(t)]^T$ over $T$ can be approximated by $\mathbf{q}_m[n]=[x_m[n], y_m[n]]^T$ over $N$ time slots where $(x_m[n],y_m[n])$ is the horizontal coordinate of the UAV $m$ in time slot $n$. As a result, the UAV trajectory constraints (\ref{eq01}) and (\ref{eq02}) reduce to a countable number of constraints as follows
%\begin{align}
%\q[1] &= \q[N],\\
%||\mathbf{q}[n+1]-\mathbf{q}[n]||^2 &\leq (V_{\max}\delta_t)^2,  n=1,...,N-1,
%\end{align}
%where $V_{\max}\delta_t$ is the largest horizontal distance that a UAV can travel in one time slot.
 In fact, any required accuracy of the adopted discrete-time approximation can be always satisfied by choosing a minimum $N$, as follows. To guarantee a certain accuracy, the ratio of $S_{\max}$ and $H$ can be restricted below a threshold, i.e., $\frac{S_{\max}}{H}\leq \varepsilon_{\max}$, where $\varepsilon_{\max}$ is the given threshold.  Then, the minimum number of time slots  required for achieving the accuracy with a given  $\varepsilon_{\max}$ can be obtained as
 \begin{align}
 N\geq \frac{V_{\max}T}{H \varepsilon_{\max}}.
 \end{align}
 However, further increasing  $N$ also increases our design complexity.  Therefore, the number of time slots $N$ can be properly chosen in practice to  balance between the accuracy and complexity.

%Assuming that all users' locations are known,
The distance from UAV $m$  to user $k$ in time slot $n$  can be expressed as
 \begin{align}
 d_{k,m}[n] =  \sqrt{H^2 +||\q_m[n]-\w_k||^2}.
 \end{align}
 For simplicity, we assume that the communication links from the UAV to the ground users are dominated by the LoS links where the channel quality depends only on the UAV-user distance. Furthermore, the Doppler effect caused by the UAV mobility  is assumed to be well compensated at the  receivers.  Thus, the channel power gain from UAV $m$ to user $k$ during slot $n$ follows the free-space path loss model, which can be  expressed as
   \begin{align}
h_{k,m}[n]& = \rho_0d^{-2}_{k,m}[n] =\frac{\rho_0}{ H^2 +||\q_m[n]-\w_k||^2},
 \end{align}
  where $\rho_0$ denotes the channel power at the reference distance $d_0=1$ m.
 Define a binary  variable $\alpha_{k,m}[n]$, which indicates that user $k$ is served by UAV $m$ in time slot $n$ if $\alpha_{k,m}[n]=1$; otherwise, $\alpha_{k,m}[n]=0$.
 As such, $\alpha_{k,m}[n]$ specifies not only the user communication scheduling across the different time slots, but also the UAV-user association for each time slot.
  We assume that in each time slot, each UAV only serves at most one user and each user is only served by at most one UAV, which yields the following constraints
\begin{align}
&\sum_{k=1}^{K}\alpha_{k,m}[n]\leq 1, \forall\,m, n,        \label{eq77}   \\
& \sum_{m=1}^{M}\alpha_{k,m}[n]\leq 1, \forall\,k,  n,    \label{eq88} \\
 &\alpha_{k,m}[n]\in\{0, 1\}, \forall\, k, m, n.
\end{align}
The downlink transmit power of UAV $m$, $m\in \mathcal{M}$ in time slot $n$ is denoted by $p_m[n]$, which is subject to the constraint $0 \leq p_m[n]\leq P_{\max}$, with $P_{\max}$ denoting the peak UAV transmission power.  Thus, if user $k$ is served by UAV $m$ in time slot $n$, i.e., $\alpha_{k,m}[n]=1$,  the corresponding received signal-to-interference-plus-noise ratio (SINR) at user $k$ can be expressed as
  \begin{align}\label{eq2}
  \gamma_{k,m}[n] = \frac{p_m[n]h_{k,m}[n]}{\sum_{j=1, j\neq m}^{M}p_j[n]h_{k,j}[n]+\sigma^2},
  \end{align}
%    \begin{align}\label{eq2}
%  \gamma_{i,m}[n] = \frac{P_m[n]h_{k,m}[n]}{\sum_{j=1, j\neq m}^{M}P_j[n]h_{k,j}[n]+\sigma^2},
%  \end{align}
 % where $h_{k,m}[n] = \frac{ \rho_0}{H^2+||\mathbf{q}_m[n]-\mathbf{w}_i||^2}$.
where $\sigma^2$ is the power of the additive white Gaussian noise (AWGN) at the receiver.
 The term $\sum_{j=1, j\neq m}^{M}p_j[n]h_{k,j}[n]$ in the denominator of (\ref{eq2}) represents the co-channel interference caused by the transmissions of all other UAVs in time slot $n$.
%Thus, if user $k$ is scheduled in time slot $n$, then the corresponding achievable throughput, $ R_i[n]$, can be expressed as
Thus, the achievable rate of user $k$ in time slot $n$, denoted by $R_{k}[n]$ in bits/second/Hertz (bps/Hz), can be expressed as
 \begin{align}
R_{k}[n] = \sum_{m=1}^{M}\alpha_{k,m}[n]  \log_2\left( 1 +  \gamma_{k,m}[n] \right).
 \end{align}
Thus, the achievable average rate of user $k$ over $N$ time slots is given by
  %\begin{align}\label{eq_133}
$R_k= \frac{1}{N}\sum_{n=1}^{N}R_{i}[n].$
% \nonumber \\
%&=\frac{1}{N}\sum_{n=1}^{N}\sum_{m=1}^{M}\alpha_{k,m}[n]  \log_2\left( 1 +\frac{\frac{p_m[n]\rho_0}{H^2+||\mathbf{q}_m[n]-\mathbf{w}_k||^2}}{\sum_{j=1, j\neq m}^{M}\frac{p_j[n]\rho_0}{H^2+||\mathbf{q}_j[n]-\mathbf{w}_k||^2}+\sigma^2} \right).
 %\end{align}
%  \gamma_{i,m}[n]
% \begin{figure}[!htbp]
%\centering
%\subfigure[Trajectory versus total available time $N$.]{\includegraphics[width=3.5in, height=2.8in]{UAV_IFC_single.eps}}
%\subfigure[Max-min throughput  versus total available time $N$.]{\includegraphics[width=3.5in,height=2.8in]{UAV_IFC3.eps}}
%\caption{Joint Trajectory and Communication Design for UAV Enabled Wireless Network.} \label{large_scale_paper}
%\end{figure}

%    \begin{figure}[!t]
%\centering
%\includegraphics[width=0.45\textwidth]{}
%\caption{UAV Communications for Broadcast Channel.} \label{UAV}
%\end{figure}


\subsection{Problem Formulation}
Let $\mathbf{A}=\{\alpha_{k,m}[n], \forall\, k,m,n\}$, $\mathbf{Q}=\{\mathbf{q}_{m}[n], \forall\,m,n\}$, and $\mathbf{P}=\{{p}_{m}[n], \forall\,m,n\}$.
By assuming that the locations of the ground users are known, our goal is to maximize the minimum average rate among all users by jointly optimizing the user scheduling and association (i.e., $\A$), UAV trajectory (i.e., $\Q$), and transmit power  (i.e., $\pow$) over all time slots. Define $\eta(\A,\Q, \pow)= \min \limits_{k\in \mathcal{K}}~ R_k$ as a function of $\A$, $\Q$, and $\pow$. The optimization problem is formulated as
\begin{subequations}  \label{probm6}
 \begin{align}%\label{probm11}
 &\max  \limits_{\eta,\mathbf{A},\mathbf{Q}, \mathbf{P}} ~ \eta  \label{probm06}  \\
&\text{s.t.} ~\frac{1}{N} \sum_{n=1}^{N}\sum_{m=1}^{M} \alpha_{k,m}[n]  \log_2\left( 1 +  \gamma_{k,m}[n] \right) \geq \eta, \forall\, k, \label{eq012} \\
 &~~~~ \sum_{k=1}^{K}\alpha_{k,m}[n]\leq 1, \forall\,m, n,        \label{eq7}   \\
&~~~~  \sum_{m=1}^{M}\alpha_{k,m}[n]\leq 1, \forall\,k,  n,    \label{eq8}  \\
&~ ~~~  \alpha_{k,m}[n]\in\{0, 1\}, \forall\, k,m,  n,     \label{eq9}  \\
&~ ~~~   ||\mathbf{q}_m[n+1]-\mathbf{q}_m[n]||^2\leq S_{\max}^2,  n=1,...,N-1,    \label{eq10} \\
&~~~~   \mathbf{q}_m[1]=\mathbf{q}_m[N],   \forall\, m,      \label{eq11} \\
&~ ~~~  {|| \q_m[n]-  \q_{j}[n]||^2 \geq d^2_{\min},   \forall\, n, m, j\neq m, }\label{eq11216}\\
&~ ~~~ 0\leq p_m[n]\leq P_{\max}, \forall\,m, n.   \label{eq12} %&~~~~~~~~ ~~~C6: \sum_{n=1}^{N}\sum_{k=1}^{K}\alpha_{k,m}[n]= \frac{N}{2}, \forall\,m.
 \end{align}
 \end{subequations}
%It is worth noting that when $V_{\max}=0$, problem  (\ref{probm6}) is simplified to a placement optimization problem for multi-static UAVs. Thus, this case serves as a benchmark for demonstrating the effectiveness of exploiting UAV mobility for performance improvement.
  %In this paper, the initial locations of all UAVs are also optimized for achieving a higher performance. While in some applications where the initial/final locations of UAVs are pre-specified due to certain physical restrictions, addition linear constraints can be integrated into problem (\ref{probm11}), which do not affect the applicability of the proposed optimization framework.

 Problem  (\ref{probm6}) is challenging to solve due to the following two main reasons. First, the optimization variables $\mathbf{A}$ for user scheduling and association are binary and thus (\ref{eq7})-(\ref{eq9}) involve integer constraints. Second, even with fixed user scheduling and association,  (\ref{eq012}) and \eqref{eq11216} are still non-convex constraints with respect to UAV trajectory variables $\Q$ and/or transmit power variables $\pow$.
 %optimization variables $\{\mathbf{Q}, \mathbf{P}\}$ are closely coupled in the objective function where no convexity structure can be exploited even with respect to each of them.
 Therefore, problem  (\ref{probm6}) is a mixed-integer non-convex problem, which is difficult to be optimally solved in general.
%Therefore, problem  (\ref{probm6}) is a combinatorial optimization problem that is neither concave nor quasi-concave. In general, there is no standard method for solving such highly non-convex optimization problems efficiently. Nevertheless, in the following, we propose an efficient iterative optimization algorithm, which is guaranteed to converge to a local optimal solution of problem  (\ref{probm6}). The proposed algorithm is mainly based on exploiting the block coordinate descent method and the successive convex optimization technique.
 %Before proceeding to solve problem  (\ref{probm6}), we first discuss two special cases for for interference-free scenarios.
% \subsection{Special Case I: A Single UAV}
%  First, when only one UAV is deployed to serve all the ground users, i.e., $M=1$, there is no co-channel interference in the system \cite{wu2017joint}.   For ease of exposition, we drop the UAV index and let $\bm{\alpha}=\{\alpha_i[n], \forall\, n\}$. % and  $\mathbf{Q}=\{\q[n], \forall\, n\}$.
%  Thus, the average achievable rate of user $k$ in (\ref{eq_133}) reduces to
% \begin{align}
% R^{I}_k=\frac{1}{N}\sum_{n=1}^{N}\alpha_{k}[n]  \log_2\left( 1 + {\frac{p[n]\rho_0}{(H^2+||\mathbf{q}[n]-\mathbf{w}_k||^2)\sigma^2}}\right).
% \end{align}
%It is not difficult to see that, the UAV should always transmit with its maximum power, i.e., $p[n]=P_{\max}, \forall\,n$. In this case, problem (\ref{probm6}) is simplified to a joint user scheduling and UAV trajectory optimization problem, i.e.,
% \begin{subequations}\label{probm_single}
% \begin{align}
%  &\max  \limits_{\bm{\alpha},\mathbf{Q}} ~~ \min \limits_{i\in \mathcal{K}}~~ R^{I}_i  \\
%  %\sum_{n=1}^{N}\alpha_{i}[n]  \log_2\left( 1 +\frac{\gamma_0}{H^2+||\mathbf{q}[n]-\mathbf{w}_i||^2} \right)     \nonumber  \\
%&~~~\text{s.t.} ~~~  \sum_{k=1}^{K}\alpha_{k}[n]\leq 1, \forall\, n,   \label{eq15}\\
%&~~~~~~~~ ~  \alpha_{i}[n]\in\{0, 1\}, \forall\, k, n,  \label{eq16} \\
%&~~~~~~~~ ~  ||\mathbf{q}[n+1]-\mathbf{q}[n]||^2\leq S^2_{\max},  n=1,...,N-1, \\
%&~~~~~~~~ ~    \mathbf{q}[1]=\mathbf{q}[N].
% \end{align}
% \end{subequations}
%Since problem (\ref{probm_single}) is a special case of problem (\ref{probm6}), the optimization framework proposed in Section III  is also applicable to problem (\ref{probm_single}).
%
% \subsection{Special Case II: Orthogonal  UAV Transmission}
% For the second special case, the multiple UAVs take turns to transmit information to their served ground users over orthogonal time slots, thus the system is also interference-free. This is achieved by imposing the following constraints\footnote{For convenience, we select the value of $N$ such that $\frac{N}{M}$ is an integer for a given $M$.},
% \begin{align}
%&\sum_{k=1}^{K}\alpha_{i,m}[M(\ell-1)+m]\leq 1, \forall\,m, \ell=1,2,\cdot\cdot\cdot,\frac{N}{M}, \label{eq220}\\
%&\sum_{k=1}^{K}\alpha_{i,j}[M(\ell-1)+m]= 0, \forall\,j\neq m, \ell=1,2,\cdot\cdot\cdot,\frac{N}{M}, \label{eq221}
% \end{align}
%which  guarantee that in each time slot, only one UAV is allowed to transmit.
% Accordingly, the achievable rate of user $k$ can be  expressed as
% \begin{align}
% R^{II}_i=\frac{1}{N}\sum_{n=1}^{N}\sum_{m=1}^{M}\alpha_{k,m}[n]  \log_2\left( 1 + {\frac{p_m[n]\rho_0}{(H^2+||\mathbf{q}_m[n]-\mathbf{w}_i||^2)\sigma^2}}\right).
% \end{align}
%Therefore,  problem (\ref{probm6}) can be reformulated as
% \begin{subequations}\label{probm622}
% \begin{align}
%  &\max  \limits_{\A,\mathbf{Q}, \mathbf{P}} ~~ \min \limits_{i\in \mathcal{K}}~~~~~R^{II}_i    \\
%&~~~\text{s.t.} ~~~~~  \text{(\ref{eq220})}, \text{(\ref{eq221}}),  \label{eq22} \\
%% \sum_{m=1}^{M}\sum_{k=1}^{K}\alpha_{k,m}[n]\leq 1, \forall\,n, \label{eq22} \\
%&~~~~~~~~ ~~~ \text{(\ref{eq9})}, \text{(\ref{eq10})}, \text{(\ref{eq11})}, \text{(\ref{eq12})}.% \nonumber
% \end{align}
%  \end{subequations}
% Note that problem (\ref{probm622}) is obtained from problem (\ref{probm6}) by replacing the constraints (\ref{eq7}) and (\ref{eq8}) with (\ref{eq22}). It can be shown that  (\ref{eq22}) and (\ref{eq9}) implies (\ref{eq7}) and (\ref{eq8}), but not vice versa. Thus, problem (\ref{probm622}) is regarded as a special case of problem (\ref{probm6}), and thus can be solved similarly as shown in the next section.
 % and can be thereby solved by the proposed optimization framework.
% \vspace{-0.5cm}
\section{Proposed Algorithm}
%Specifically, the optimization variables in original problem (\ref{probm6}) are partitioned into three blocks, i.e., $\{ \mathbf{A}, \Q, \mathbf{P}\}$.
%  We first investigate the three sub-problems of optimizing the user scheduling, UAV trajectory, and power control, respectively, for any given other two. Then, we integrate them into a complete iterative algorithm for original problem (\ref{probm6}) and discuss the convergence.

To make problem (\ref{probm6}) more tractable, we first relax the binary variables in (\ref{eq9}) into continuous variables, which yields the following problem
  \begin{subequations}\label{probm66}
 \begin{align}
&\max  \limits_{\eta,\mathbf{A},\mathbf{Q},\pow} ~~~ ~\eta  \\ %\label{probm6}
  %\sum_{n=1}^{N}\alpha_{i}[n]  \log_2\left( 1 +\frac{\gamma_0}{H^2+||\mathbf{q}[n]-\mathbf{w}_i||^2} \right)     \nonumber  \\
&~~\text{s.t.}  ~~ 0\leq\alpha_{k,m}[n]\leq 1, \forall\, k,m,n,   \label{eq13d} \\
&~~~~~~ ~ \text{(\ref{eq012}),  (\ref{eq7}),  (\ref{eq8}), (\ref{eq10}), (\ref{eq11}),  \eqref{eq11216}, (\ref{eq12})}.   \label{eq13d}
 \end{align}
 \end{subequations}
 Such a relaxation in general implies that the objective value of problem (\ref{probm66}) serves as an upper bound for that of problem  (\ref{probm6}).
 Although relaxed, problem (\ref{probm66}) is still a non-convex optimization problem due to the non-convex constraint (\ref{eq012}). In general, there is no standard method for solving such non-convex optimization problems efficiently. %In the following, we first propose an efficient iterative algorithm for problem (\ref{probm66}) which is guaranteed to converge to at least a locally optimal solution, and then show how to construct the solution of problem  (\ref{probm6}) based on that of problem (\ref{probm66}).
  In the following, we  propose an efficient iterative algorithm for the relaxed problem  (\ref{probm66}) by applying the block coordinate descent  \cite{hong2016unified} and successive convex optimization techniques. Specifically, for given UAV trajectory $\mathbf{Q}$ and transmit power $\pow$, we optimize the user scheduling and association $\A$ by solving a linear programming (LP). For any given user scheduling and association $\A$ and transmit power $\pow$ (UAV trajectory $\mathbf{Q}$), the UAV trajectory $\Q$ (transmit power $\pow$) is optimized based on the successive convex optimization technique \cite{zeng2016throughput,zeng2016energy}.
 Then, we present the overall algorithm and analytically show its convergence. Furthermore, we propose a low-complexity  initialization scheme for the UAV trajectory design. Finally, we show how to reconstruct a binary solution to the original problem (\ref{probm6}) based on the obtained solution to problem  (\ref{probm66}).
%\vspace{-0.3cm}
 \subsection{User Scheduling and Association Optimization}
 For any given UAV trajectory and transmit power  $\{\Q, \mathbf{P}\}$, the user scheduling and association of problem (\ref{probm66}) can be optimized by solving the following problem
 \begin{subequations}\label{probm25}
  \begin{align}
&\mathop {\text{max} }\limits_{\eta,\A}~~ \eta   \label{probm250}\\
&~\text{s.t.}~   \frac{1}{N} \sum_{n=1}^{N} \sum_{m=1}^{M} \alpha_{k,m}[n]  \log_2\left( 1 + \gamma_{k,m}[n]  \right)  \geq \eta, \forall\, k, \label{eq26}\\
&~~~~~\sum_{k=1}^{K}\alpha_{k,m}[n]\leq 1, \forall\,m, n,        \label{eq70}   \\
&~~~~~  \sum_{m=1}^{M}\alpha_{k,m}[n]\leq 1, \forall\,k,  n,    \label{eq80}  \\
&~~ ~~~   0\leq\alpha_{k,m}[n]\leq 1, \forall\, k,m,  n.     \label{eq90}
 \end{align}
  \end{subequations}
Since problem (\ref{probm25}) is a standard LP, it can be solved efficiently by existing  optimization tools such as CVX \cite{cvx}. Furthermore,  it is easy to see that the constraints (\ref{eq70}) and (\ref{eq80}) are met with equalities when the optimal solution $\A$ is attained for given $\{\Q, \mathbf{P}\}$.
% Note that problem (\ref{probm25}) is a mixed integer linear programming involving both integer and continuous optimization variables, which is challenging to solve in general. However, by relaxing binary variables to continuous ones, problem (\ref{probm25}) can be transformed into the following
%   \begin{align}%\label{probm33}
% &~\mathop {\text{max} }\limits_{t_{\rm asc},\A}~~~~~~~~ t_{\rm asc} \label{probm27}   \\
%&~~~\text{s.t.} ~~~~~ 0\leq\alpha_{k,m}[n]\leq 1, \forall\, k,m,  n, \\
%&~~~~~~ ~~~~~\text{(\ref{eq7}), (\ref{eq8}), (\ref{eq26})}. \nonumber
% \end{align}
%After the relaxation, problem (\ref{probm27}) is a linear programming (LP) problem, which can be solved by standard linear optimization tools such as the simplex method.
%%\begin{proposition}
%
%It is worth pointing out that with sufficiently large $N$:
%1) the optimal solution to problem (\ref{probm27}) are all binary;
%2) all users achieve equal data rates for problem (\ref{probm6}).
%%\end{proposition}
%The intuition is that for any non-integer solutions of problem  (\ref{probm27}), the time slot can be further divided into a smaller granularity such that there exists an integer solution with global optimality. The physical interpretation is to let the UAVs just hover at the their current positions while performing time sharing or frequency sharing among scheduled users.
%\vspace{-0.2cm}
  \subsection{UAV Trajectory Optimization}
%  In this subsection, we investigate the sub-problem of optimizing UAVs' trajectories $\{\mathbf{Q}\}$ for any given user scheduling and UAV transmit power $\{\A, \pow\}$.
%%The trajectories of UAVs are particularly important for achieving better direct link channels as well as reducing the system interference.
% Correspondingly, problem (\ref{probm6}) is reduced to
%
% \begin{align}\label{probm29}
%  &\max  \limits_{\Q} ~~ \min \limits_{i\in \mathcal{K}}~~\frac{1}{N}\sum_{n=1}^{N}\sum_{m=1}^{M}\alpha_{k,m}[n]  \log_2\left( 1 +\frac{\frac{p_m[n]\rho_0}{H^2+||\mathbf{q}_m[n]-\mathbf{w}_i||^2}}{\sum_{j=1, j\neq m}^{M}\frac{p_j[n]\rho_0}{H^2+||\mathbf{q}_j[n]-\mathbf{w}_i||^2}+\sigma^2} \right)     \\
%&~~~\text{s.t.} ~~~~~\text{(\ref{eq10}), (\ref{eq11})}. \nonumber
% \end{align}
% Similarly,  by introducing the slack variable $t_{\rm trj}\triangleq  \min \limits_{i \in \mathcal{K}}\{R_i\}$, problem (\ref{probm29}) reduces to
 For any given user scheduling and association as well as UAV transmit power \{$\A$, $\pow$\},  the UAV trajectory of problem (\ref{probm66}) can be optimized by solving the following problem
  \begin{subequations} \label{probm55}
    \begin{align}
 &\mathop {\text{max} }\limits_{\eta,\Q}~ \eta \label{probm550}  \\ %
&~\text{s.t.}  \frac{1}{N}\sum_{n=1}^{N}\sum_{m=1}^{M}\alpha_{k,m}[n] \log_2\left( 1 +  \gamma_{k,m}[n] \right)  \geq \eta, \forall\, k,\label{eq152} \\
&~~~~   ||\mathbf{q}_m[n+1]-\mathbf{q}_m[n]||^2\leq S_{\max}^2,  n=1,...,N-1,    \label{eq100} \\
&~~~~  \mathbf{q}_m[1]=\mathbf{q}_m[N],   \forall\, m,      \label{eq110} \\
&~~~~ || \q_m[n]-  \q_j[n]||^2 \geq d^2_{\min},   \forall\, n, m, j\neq m. \label{eq112}
 \end{align}
 \end{subequations}
Note that problem (\ref{probm55}) is neither a concave or quasi-concave maximization problem due to the non-convex constraints in (\ref{eq152}) and \eqref{eq112}. In general, there is no efficient method to obtain the optimal solution. %An interesting question is: can we reach a stationary point of the problem?
In the following, we adopt the successive convex optimization technique for the trajectory optimization. To this end, ${R}_{k,m}[n]$, in constraints (\ref{eq152}) can be written as
\begin{align}\label{eq291}
{R}_{k,m}[n]&=   \log_2\left( 1 +\frac{\frac{p_m[n]\rho_0}{H^2+||\mathbf{q}_m[n]-\mathbf{w}_k||^2}}{\sum_{j=1, j\neq m}^{M}\frac{p_j[n]\rho_0}{H^2+||\mathbf{q}_j[n]-\mathbf{w}_k||^2}+\sigma^2} \right)   \nonumber\\
&\kern -12mm =  \hat{R}_{k,m}[n]- \log_2\left(\sum_{j=1, j\neq m}^{M}\frac{p_j[n]\rho_0}{H^2+||\mathbf{q}_j[n]-\mathbf{w}_k||^2}+\sigma^2\right),
\end{align}
where
\begin{align}
  \hat{R}_{k,m}[n] &=   \log_2\left(\sum_{j=1}^{M} \frac{p_j[n]\rho_0}{H^2+||\mathbf{q}_j[n]-\mathbf{w}_k||^2}+ \sigma^2 \right). \label{eq29}
  \end{align}
  With (\ref{eq291}) and (\ref{eq29}), constraints (\ref{eq152}) are transformed into
  \begin{align}
 &\frac{1}{N}\sum_{n=1}^{N}\sum_{m=1}^{M}\alpha_{k,m}[n]\Bigg(  \hat{R}_{k,m}[n]-  \nonumber \\
& \log_2\bigg(\sum_{j=1, j\neq m}^{M}\frac{p_j[n]\rho_0}{H^2+||\mathbf{q}_j[n]-\mathbf{w}_k||^2}+\sigma^2\bigg)\Bigg)  \geq \eta, \forall\, k. \label{eq22b}
  \end{align}
 Note that constraints in (\ref{eq22b}) are still non-convex. By introducing slack variables $\mathbf{S}=\{S_{k,j}[n]=||\mathbf{q}_j[n]-\mathbf{w}_k||^2,\forall\, j\neq m, j\in\mathcal{M}, k, n\} $, problem (\ref{probm55}) can be reformulated as
 \begin{subequations}\label{probm34}
   \begin{align} %\label{probm67}
&\kern -3mm \mathop {\text{max} }\limits_{\eta,\mathbf{Q}, \mathbf{S}}~ \eta  \\
&\kern -3mm \text{s.t.} ~~ \frac{1}{N} \sum_{n=1}^{N}\sum_{m=1}^{M}\alpha_{k,m}[n] \Bigg(  \hat{R}_{k,m}[n] \nonumber\\
&- \log_2\bigg(\sum_{j=1, j\neq m}^{M} \frac{p_j[n]\rho_0}{H^2+S_{k,j}[n]}+\sigma^2\bigg)\Bigg) \geq \eta, \forall\, k, \label{eq35} \\
&S_{k,j}[n] \leq ||\mathbf{q}_j[n]-\mathbf{w}_k||^2,  \forall\, k, j \neq m, n, \label{eq36}\\
&||\mathbf{q}_m[n+1]-\mathbf{q}_m[n]||^2\leq S_{\max}^2,  n=1,...,N-1,    \label{eq100} \\
&\mathbf{q}_m[1]=\mathbf{q}_m[N],   \forall\, m,      \label{eq110} \\
&|| \q_m[n]-  \q_j[n]||^2 \geq d^2_{\min},   \forall\, n, m, j\neq m. \label{eq11220}
 \end{align}
  \end{subequations}
It can be verified that without loss of optimality to problem (\ref{probm34}), all constraints in (\ref{eq36}) can be met with equality, since otherwise we can always increase $S_{k,j}[n]$ without decreasing the objective value. Note that in (\ref{eq35}), $\hat{R}_{k,m}[n]$ is neither convex nor concave with respect to $\mathbf{q}_j[n]$. While in (\ref{eq36}), even though $||\mathbf{q}_j[n]-\mathbf{w}_k||^2$ is convex with respect to $\mathbf{q}_j[n]$, the resulting set is not  a convex set since the superlevel set of a convex quadratic function is not convex in general. Thus,  problem (\ref{probm34}) is still a non-convex optimization problem due to the non-convex feasible set.

% To tackle the non-convexity of  (\ref{eq152}), the successive convex optimization technique can be applied where in each iteration,  the left-hand-side (LHS) of (\ref{eq152}) is replaced by its concave lower bound at a given local point.
%Although simplified,  due to the non-convex feasible set created by constraints (\ref{eq35}) and (\ref{eq36}). This is because the superlevel set of a convex function may not be a convex set in general even if $\hat{R}_{i,m}[n]$ and $||\mathbf{q}_j[n]-\mathbf{w}_i||^2$ are convex functions of $||\mathbf{q}_j[n]-\mathbf{w}_i||^2$ and $\mathbf{q}_j[n]$, respectively.

To tackle the non-convexity of  (\ref{eq35}), (\ref{eq36}), and \eqref{eq11220}, the successive convex optimization technique can be applied where in each iteration, the original function is approximated by a more tractable function at a given local point. Specifically, define $\Q^r =\{\mathbf{q}_m^r[n], \forall\,m, n\}$ as the given trajectory of UAVs in the $r$-th iteration\footnote{{In Section III-D, we show that $\Q^r$ is in fact the solution obtained from the $(r-1)$th iteration.  } }.  The key observation is that in (\ref{eq29}),  although  $\hat{R}_{k,m}[n]$  is not concave with respect to $\mathbf{q}_j[n]$, it  is convex with respect to $||\mathbf{q}_j[n]-\mathbf{w}_k||^2$.  Recall that any convex function is globally lower-bounded by its first-order Taylor expansion at any point  \cite{Boyd}.  Therefore,  with given local point $\Q^r$ in the $r$-th iteration, we obtain the following  lower bound for  $\hat{R}_{k,m}[n]$ as in  \cite{zeng2016throughput,zeng2016energy}, i.e.,
% as motivated by the derivation in \cite{zeng2016energy}
%a local convex approximation can be applied where the original function is replaced by a convex function at a local point. Recall that any convex function is lower bounded by its first-order Taylor expansion at any point \cite{Boyd}.
% The key observation is that in constraints  (\ref{eq35}),  $\hat{R}_{i,m}[n]$ is convex with respect to $||\mathbf{q}_j[n]-\mathbf{w}_i||^2$, although not convex respect to $\mathbf{q}_j[n]$.  Therefore, we obtain the following global lower bound,

%\begin{align}
% \hat{R}_{i,m}[n] &=   \log_2\left(\sum_{j=1}^{M} \frac{p_j[n]\rho_0}{H^2+||\mathbf{q}_j[n]-\mathbf{w}_i||^2}+ \sigma^2 \right)
% \nonumber\\
%& \geq \sum_{j=1}^{M}C^r_{i,j}[n]\left(||\mathbf{q}_j[n]-\mathbf{w}_i||^2 -||\mathbf{q}^r_j[n]-\mathbf{w}_i||^2 \right)  \nonumber \\
%&+ \log_2\left( \sum_{j=1}^{M}A^r_{i,j}[n]+\sigma^2 \right)  \nonumber \\
%&\triangleq \hat{R}^{\rm lb}_{i,m}[n],  \label{eq37}
%\end{align}
%where $A^r_{i,j}[n]$ and $C^r_{i,j}[n]$ are constants that are given by
%\begin{align}
%A^r_{i,j}[n]&=  \frac{p_j[n]\rho_0}{H^2+||\mathbf{q}_j^r[n]-\mathbf{w}_i||^2}, \forall\, k,j, n, \\
%B^r_{i,j}[n]&=  \frac{A^r_{i,j}[n]}{H^2+||\mathbf{q}_j^r[n]-\mathbf{w}_i||^2}, \forall\, k,j, n,\\
%C^r_{i,j}[n]&= \frac{-B^r_{i,j}[n]\log_2(e)}{\sum_{l=1}^{M} A^r_{i,l}[n]+\sigma^2 }, \forall\, k,j, n. \label{eq17}
%\end{align}


{ \begin{align}
 \hat{R}_{k,m}[n] &=   \log_2\left(\sum_{j=1}^{M} \frac{p_j[n]\rho_0}{H^2+||\mathbf{q}_j[n]-\mathbf{w}_k||^2}+ \sigma^2 \right)
 \nonumber\\
&\kern -1cm \geq \sum_{j=1}^{M}-A^r_{k,j}[n]\left(||\mathbf{q}_j[n]-\mathbf{w}_k||^2 -||\mathbf{q}^r_j[n]-\mathbf{w}_k||^2 \right) \nonumber \\
& \kern -0.5cm+  B^r_{k,j}[n]  \triangleq \hat{R}^{\rm lb}_{k,m}[n],  \label{eq37}
\end{align} }
where $A^r_{k,j}[n]$ and $B^r_{k,j}[n]$ are constants that are given by
\begin{align}
A^r_{k,j}[n]&= \frac{ \frac{p_j[n]\rho_0}{(H^2+||\mathbf{q}_j^r[n]-\mathbf{w}_k||^2)^2}\log_2(e)}{\sum_{l=1}^{M} \frac{p_l[n]\rho_0}{H^2+||\mathbf{q}_l^r[n]-\mathbf{w}_k||^2} + \sigma^2  }, \forall\, k,j, n, \\
B^r_{k,j}[n]&= \log_2\left( \sum_{l=1}^{M}\frac{p_l[n]\rho_0}{H^2+||\mathbf{q}_l^r[n]-\mathbf{w}_k||^2}+\sigma^2 \right)  , \forall\, k,j, n.
\end{align}
In constraints (\ref{eq36}), since $||\mathbf{q}_j[n]-\mathbf{w}_k||^2$ is a convex function with respect to $\mathbf{q}_j[n]$,
we have the following inequality by applying the first-order Taylor expansion at the given point $\mathbf{q}_j^r[n]$,
\begin{align}
 & ||\mathbf{q}_j[n]-\mathbf{w}_k||^2\geq    |\mathbf{q}_j^r[n]-\mathbf{w}_k||^2  \nonumber\\
 &~~~~~+  2(\mathbf{q}_j^r[n] - \mathbf{w}_k)^T( \mathbf{q}_j[n] - \mathbf{q}_j^r[n] ), \forall\, k,j\neq m,n. \label{eq40}
\end{align}
{Similarly,  by applying the first-order Taylor expansion at the given point $\mathbf{q}_m^r[n]$ and $\mathbf{q}_j^r[n]$ to $|| \q_m[n]-  \q_j[n]||^2$, we obtain
\begin{align}
& ||\mathbf{q}_m[n]-\mathbf{q}_j[n]||^2 \geq    -||\mathbf{q}_m^r[n]-\mathbf{q}_j^r[n]||^2 \nonumber\\
 &~~~+  2(\mathbf{q}_m^r[n] - \mathbf{q}_j^r[n])^T( \mathbf{q}_m[n]-\mathbf{q}_j[n] ), \forall\, j\neq m,n. \label{eq400}
\end{align}}
%Then, C6 is written as
% \begin{align}
%C6: S_{i,j}[n] &\leq   ||\mathbf{q}_j^r[n]-\mathbf{w}_i||^2 +  2(\mathbf{q}_j^r[n] - \mathbf{w}_k)^T( \mathbf{q}_j[n] - \mathbf{q}_j^r[n] ) , \forall\, k, j\neq m,n.
%\end{align}



%For any given local point $\Q^r$, define the function $\eta^{{\rm lb}, r}(\A,\Q, \pow)= \min \limits_{i\in \mathcal{K}}~ \sum_{n=1}^{N}\alpha_{k,m}[n] {R}^{{\rm lb},r}_{i,m}[n]$.
 With any given local point $\Q^r$ as well as  the lower bounds  in  (\ref{eq37}) and (\ref{eq40}),  problem (\ref{probm34}) is approximated as the following problem %i%n the $r$-th iteration
%With the lower bounds  $\hat{R}^{\rm lb}_{i,m}[n]$ as well as the given local points $\mathbf{q}_j^r[n]$ in (\ref{eq37}) and (\ref{eq40}), problem (\ref{probm34}) is transformed into the following problem
 \begin{subequations} \label{probm41}
   \begin{align}
  &\kern -2mm \mathop {\text{max} }\limits_{\eta^{r}_{\rm trj},\mathbf{Q},\mathbf{S}}~~\eta^{r}_{\rm trj} \\
&\kern -2mm\text{s.t.} ~  \frac{1}{N}\sum_{n=1}^{N}\sum_{m=1}^{M}\alpha_{k,m}[n] \Bigg(  \hat{R}^{\rm lb}_{k,m}[n] \nonumber\\
&~~~- \log_2\bigg(\sum_{j=1, j\neq m}^{M} \frac{p_j[n]\rho_0}{H^2+S_{k,j}[n]}+\sigma^2\bigg)\Bigg) \geq \eta^{ r}_{\rm trj}, \forall\, k,   \label{eq41}\\
&\kern -2mm  S_{k,j}[n] \leq   ||\mathbf{q}_j^r[n]-\mathbf{w}_k||^2 \nonumber\\
&~~~+  2(\mathbf{q}_j^r[n] - \mathbf{w}_k)^T( \mathbf{q}_j[n] - \mathbf{q}_j^r[n] ),  \forall\, k, j \neq m, n,  \label{eq42} \\
&\kern -2mm  ||\mathbf{q}_m[n+1]-\mathbf{q}_m[n]||^2\leq S_{\max}^2,  n=1,...,N-1,    \label{eq100} \\
&\kern -2mm \mathbf{q}_m[1]=\mathbf{q}_m[N],   \forall\, m,     \label{eq110} \\
&\kern -2mm  d^2_{\min} \leq    -||\mathbf{q}_m^r[n]-\mathbf{q}_j^r[n]||^2 \nonumber\\
&\kern -2mm +  2(\mathbf{q}_m^r[n] - \mathbf{q}_j^r[n])^T( \mathbf{q}_m[n]-\mathbf{q}_j[n] ), \forall\, n, m, j\neq m. \label{eq110new}
 \end{align}
  \end{subequations}
  %where $\eta^{{\rm lb}, r}(\A,\Q, \pow)= \min \limits_{i\in \mathcal{K}}~ \sum_{n=1}^{N}\alpha_{k,m}[n] {R}^{{\rm lb},r}_{i,m}[n]$.
  Since the left-hand-side (LHS) of the constraint (\ref{eq41}) is jointly concave with respect to $\mathbf{q}_j^r[n]$ and $S_{k,j}[n]$,  it is convex now.  {Furthermore, (\ref{eq100}) is a convex quadratic constraint and (\ref{eq42}),  (\ref{eq110}), and \eqref{eq110new} are all linear constraints.} Therefore, problem (\ref{probm41}) is  a convex optimization problem that can be efficiently solved by standard convex optimization solvers such as  CVX  \cite{Boyd}. It is worth noting that the lower bounds adopted in (\ref{eq41}) and (\ref{eq42}) suggest that any feasible solution of problem (\ref{probm41}) is also feasible for problem (\ref{probm34}), but the reverse does not hold in general. As a result, the optimal objective value obtained from the approximate problem (\ref{probm41}) in general serves as a lower bound of that of problem (\ref{probm34}).
\subsection{UAV Transmit Power Control}
 For any given user scheduling and association as well as UAV trajectory \{$\A$, $\Q$\},  the UAV transmit power of problem (\ref{probm66}) can be optimized by solving the following problem
 \begin{subequations} \label{probm44}
  \begin{align}
 &~\mathop {\text{max} }\limits_{\eta,\mathbf{P}}~~ \eta  \\
&~\text{s.t.} ~  \frac{1}{N}\sum_{n=1}^{N}\sum_{m=1}^{M}\alpha_{k,m}[n]  \log_2\left( 1 +  \gamma_{k,m}[n] \right)   \geq \eta, \forall\, k, \label{eq45}\\
&~~~~~0\leq p_m[n]\leq P_{\max}, \forall\,m, n. \label{eq455}
 \end{align}
 \end{subequations}
 Problem (\ref{probm44}) is a non-convex optimization problem due to the non-convex constraint (\ref{eq45}) and in fact NP-hard for general $N$.
 Note that the LHS of (\ref{eq45}), i.e.,  ${R}_{k,m}[n]$, can be written as a difference of two concave functions with respect to the power control variables, i.e.,
\begin{align}
{R}_{k,m}[n]&=    \log_2\left( 1 +\frac{p_m[n]h_{k,m}[n]}{\sum_{j=1, j\neq m}^{M}p_j[n]h_{k,j}[n]+\sigma^2} \right)   \nonumber\\
&=    \log_2\left(\sum_{j=1}^{M} p_j[n]h_{k,j}[n]+ \sigma^2 \right) -   \check{R}_{k,m}[n], \label{eq370}
\end{align}
where
\begin{align}
  \check{R}_{k,m}[n] &= \log_2\left(\sum_{j=1, j\neq m}^{M}p_j[n]h_{k,j}[n]+\sigma^2\right).
  \end{align}
%By substituting (\ref{eq370}) into (\ref{eq45}), we can see that $\check{R}_{i,m}[n]$ which is concave with respect to $p_j[n]$, needs to be transformed into a convex function such that the feasible set created by (\ref{eq45}) is convex. To this end,
To handle the non-convex contraint of  (\ref{eq45}),  we apply the successive convex optimization technique to approximate $\check{R}_{k,m}[n]$ with a convex function in each iteration. Specifically, define $\pow^r =\{\mathbf{p}_m^r[n], \forall\,m, n\}$ as the given transmit power of UAV $m$ in the $r$-th iteration.   Recall that any concave function is globally upper-bounded by its first-order Taylor expansion at any point \cite{Boyd}.  Thus, we have the following convex upper bound at the given local point $p^r_j[n]$
\begin{align}
 \check{R}_{k,m}[n] &= \log_2\left(\sum_{j=1, j\neq m}^{M}p_j[n]h_{k,j}[n]+\sigma^2\right)
 \nonumber\\
& \leq \sum_{j=1, j\neq m}^{M} D_{k,j}[n]\left(p_j[n]-p_j^r[n] \right) \nonumber\\
&~~~~~~ + \log_2\left( \sum_{j=1, j\neq m}^{M}p_j^r[n]h_{k,j}[n]+\sigma^2 \right)  \nonumber \\
&\triangleq \check{R}^{\rm ub}_{k,m}[n],  \label{eq48}
\end{align}
where %$p^r_j[n]$ is transmit power of UAV $j$ in time slot $n$ at the $r$th iteration, and
\begin{align}
D_{k,j}[n]=  \frac{h_{k,j}[n]\log_2(e)}{ \sum_{l=1, l\neq m}^{M}p_j^r[n]h_{k,l}[n]+\sigma^2},  \forall\, k,j, n. %\label{eq17}
\end{align}

With any given local point  $\pow^r$ and  the upper bound $\check{R}^{\rm ub}_{k,m}[n]$  in (\ref{eq48}), problem (\ref{probm44})  is approximated as the following problem
\begin{subequations}\label{probm50}
  \begin{align}%\label{probm611}
 &~\mathop {\text{max} }\limits_{\eta^r_{\rm pow},\mathbf{P}}~ \eta^r_{\rm pow}   \\
&\text{s.t.} ~   \frac{1}{N}\sum_{n=1}^{N}\sum_{m=1}^{M}\alpha_{k,m}[n]\Bigg(   \log_2\bigg(\sum_{j=1}^{M} p_j[n]h_{k,j}[n]+ \sigma^2 \bigg) \nonumber\\
& ~~~~~~~~~~ -\check{R}^{\rm ub}_{k,m}[n] \Bigg) \geq \eta^r_{\rm pow}, \forall\, k, \label{eq50}\\
&~~0\leq p_m[n]\leq P_{\max}, \forall\,m, n. \label{eq455}
 \end{align}
 \end{subequations}
Problem (\ref{probm50}) is a convex optimization problem, which  can be efficiently solved  by standard convex optimization solvers such as CVX \cite{Boyd}. It is also worth noting that the upper bound adopted in (\ref{eq50}) suggests that the feasible set of problem (\ref{probm50}) is always a subset of that of problem (\ref{probm44}).  Therefore, the optimal objective value obtained from problem (\ref{probm50}) in general serves as a lower bound of that of problem (\ref{probm44}).

%In Subsection A-C, we have obtained the user association, UAV trajectory, and power control, individually, by solving either  the relaxed problem or the approximated problem.
\subsection{Overall Algorithm and Convergence}
%Based on Subsection A-C, we propose an efficient iterative optimization framework by exploiting the block coordinate descent method, also known as alternating optimization method. Specifically, the optimization variables in original problem (\ref{probm6}) are partitioned into three blocks, i.e., $\{ \mathbf{A}, \Q, \mathbf{P}\}$. In each iteration, we obtain only one block of variables by correspondingly solving either the relaxed problem or the approximated problem with optimality, i.e.,   (\ref{probm27}), (\ref{probm41}), or (\ref{probm50}),  while keeping the other two blocks of variables fixed. Furthermore, the obtained optimal solution of each iteration is used as the input of the next iteration.
%The details of the algorithm are summarized in Algorithm 1. %It is northing noting that step 4 and step 5 can be exchanged while step 3
%
Based on the results presented in the previous three subsections, we propose an overall  iterative algorithm for problem (\ref{probm66}) by applying the block coordinate descent method  \cite{bertsekas1999nonlinear}, also known as the alternating optimization method. Specifically, the entire optimization variables in original problem (\ref{probm66}) are partitioned into three blocks, i.e., $\{ \mathbf{A}, \Q, \mathbf{P}\}$. Then, the user scheduling and association $\A$, UAV trajectory $\Q$, and transmit power $\pow$ are alternately optimized, by  solving  problem (\ref{probm25}), (\ref{probm41}), and (\ref{probm50})  correspondingly, while keeping the other two blocks of variables fixed. {Furthermore, the obtained solution in each iteration is used as the input of the next iteration.}
The details of this algorithm are summarized in Algorithm 1.
It is worth pointing out that in the classical block coordinate descent method, the sub-problem for updating each block of variables is required to be solved exactly with optimality  in each iteration  in order to guarantee the convergence \cite{bertsekas1999nonlinear}. However, in our case, for the trajectory optimization problem (\ref{probm55}) and transmit power optimization problem  (\ref{probm44}), we only solve their approximate problems (\ref{probm41}) and (\ref{probm50}) optimally.  Thus, the convergence analysis for the classical  coordinate descent method cannot be directly applied and the convergence of Algorithm 1 needs to be proved, as shown next.


%Since the one-block subproblem can be easily solved, the BCD algorithms often perform very efficiently in practice. Furthermore, since only one block is updated at each iteration, the per-iteration storage and computational burden of the BCD algorithm could be very low, which is desirable for solving large-scale problems.


 \begin{algorithm}[t]
\caption{ Block coordinate descent algorithm for problem (\ref{probm66}).}\label{Algo:succ}
\begin{algorithmic}[1]
\STATE Initialize $\Q^0$ and $\mathbf{P}^0$. Let $r=0$.
\REPEAT
\STATE Solve problem (\ref{probm25}) for  given $\{\Q^r, \mathbf{P}^r\}$, and denote the optimal solution as $\{\mathbf{A}^{r+1}\}$.
\STATE {Solve problem (\ref{probm41}) for given $\{ \mathbf{A}^{r+1},\Q^r, \mathbf{P}^r\}$, and denote the optimal solution as $\{\Q^{r+1}\}$.}
\STATE Solve problem (\ref{probm50}) for given $\{\mathbf{A}^{r+1}, \Q^{r+1}, \mathbf{P}^r\}$, and denote the optimal solution as $\{\mathbf{P}^{r+1}\}$.
\STATE Update $r=r+1$.
\UNTIL{  The fractional increase of the objective value  is below a threshold $\epsilon>0$.}
\end{algorithmic}
\end{algorithm}

%It is worth pointing that in the classical block coordinate descent method, each sub-problem is required to be solved exactly with optimality in order to guarantee the convergence \cite{bertsekas1999nonlinear}. However, in our work, for the trajectory optimization problem (\ref{probm29}) and power control problem (\ref{probm44}), we only solve their corresponding approximated problems. Thus, the convergence shall depend the approximations adopted in Subsection III-B and -C.  Now, we analyze the convergence of the proposed algorithm as in the following proposition.
%Next, we discuss the convergence of Algorithm 1 as follows.
Define $\eta^{{\rm lb},r}_{\rm trj}(\A, \Q, \pow)=\eta^r_{\rm trj} $ and  $\eta^{{\rm lb},r}_{\rm pow}(\A, \Q, \pow)=\eta^r_{\rm pow} $ where $\eta^r_{\rm trj} $ and $\eta^r_{\rm pow}$ are respectively the objective values of problem (\ref{probm41}) and  (\ref{probm50}) based on $\A$, $\Q$, and $\pow$. First, in step 3 of Algorithm 1,  since the optimal solution of  (\ref{probm25}) is obtained for given $\Q^r$ and $\pow^r$, we have
\begin{align}\label{increase1}
\eta(\A^r, \Q^r, \pow^r) \leq \eta(\A^{r+1}, \Q^r, \pow^r),
\end{align}
where $\eta(\A, \Q, \pow)$ is defined prior to problem (\ref{probm6}).  Second, for given $\A^{r+1}$, $\Q^r$, and  $\pow^r$ in step 4 of Algorithm 1, it follows that
%This suggests that the objective function value of problem (\ref{probm66}) increases in this iteration.
{\begin{align}\label{increase2}
\eta(\A^{r+1}, \Q^{r}, \pow^{r}) &\overset{(a)} = \eta^{{\rm lb}, r}_{\rm trj}(\A^{r+1}, \Q^{r}, \pow^{r}) \nonumber\\
&\overset{(b)} \leq \eta^{{\rm lb}, r}_{\rm trj}(\A^{r+1}, \Q^{r+1}, \pow^{r})\nonumber \\
&\overset{(c)} \leq \eta(\A^{r+1}, \Q^{r+1}, \pow^{r}),
\end{align}}
where $(a)$ holds since the first-order Taylor expansions in (\ref{eq37}) and (\ref{eq40}) are tight at the given local points, respectively,  which means that problem (\ref{probm41})  at  $\Q^r$ has the same objective  value as that of problem  (\ref{probm55});   $(b)$ holds since in step 4 of Algorithm 1 with the given $\A^{r+1}$ and $\pow^{r}$, problem (\ref{probm41}) is solved optimally with solution $\Q^{r+1}$;  $(c)$ holds since the objective  value of problem (\ref{probm41}) is the lower bound of that of its original problem (\ref{probm55}) at $\Q^{r+1}$.
%due to inequality (15) where for any iteration $r$, $\eta^{{\rm lb},r}(\A,\Q)$ is always a lower bound of $\eta(\A,\Q)$ for any $\A$ and $\Q$.
%
The inequality in (\ref{increase2}) suggests that although only an approximate optimization problem (\ref{probm41}) is solved for obtaining the UAV trajectory, the objective  value of problem (\ref{probm55}) is still non-decreasing after each iteration.
{Third, for given $\A^{r+1}$, $\Q^{r+1}$, and $\pow^r$ in step 5 of Algorithm 1, it follows that
%This suggests that the objective function value of problem (\ref{probm66}) increases in this iteration.
\begin{align}\label{increase3}
\eta(\A^{r+1}, \Q^{r+1}, \pow^{r}) &= \eta^{{\rm lb}, r}_{\rm pow}(\A^{r+1}, \Q^{r+1}, \pow^{r}) \nonumber\\
&\leq \eta^{{\rm lb}, r}_{\rm pow}(\A^{r+1},\Q^{r+1}, \pow^{r+1})\nonumber \\
& \leq \eta(\A^{r+1}, \Q^{r+1}, \pow^{r+1}),
\end{align}
which can be similarly shown as in \eqref{increase2}.
%where $(d)$ holds since the first-order Taylor expansion in (\ref{eq48}) is tight at the given local point which means that problem (\ref{probm50})  at  $\pow^r$ has the same objective  value as that of problem  (\ref{probm44});   $(b)$ holds since in step 5 of Algorithm 1 with the given $\A^{r+1}$ and $\Q^{r+1}$, problem (\ref{probm41}) is solved optimally with solution $\pow^{r+1}$;  $(c)$ holds since the objective  value of problem (\ref{probm41}) is the lower bound of that of its original problem (\ref{probm55}) at $\pow^{r+1}$.
Based on (\ref{increase1})--(\ref{increase3}), we obtain
\begin{align}
\eta(\A^{r}, \Q^{r}, \pow^{r})\leq \eta(\A^{r+1}, \Q^{r+1}, \pow^{r+1}),
\end{align}
which indicates that the objective value of problem (\ref{probm66}) is non-decreasing after each iteration of Algorithm 1. } Since the objective value of problem  (\ref{probm66}) is upper bounded by a finite value, the proposed Algorithm 1 is guaranteed to converge. {Simulation results in Section IV show that the proposed block coordinate descent method converges quickly for our considered setup. Furthermore, since only convex optimization problems need to be solved  in each iteration of Algorithm 1, which are of polynomial complexity, Algorithm 1 can be practically implemented with fast convergence for wireless networks of a moderate number of users. }
%Furthermore, since the lower bound adopted in (\ref{eq155}), i.e.,  ${R}^{{\rm lb}, r}_i[n]$, has the same gradient as its original function ${R}_i[n]$ at the given local point  $\Q^r$. Thus, the convergence to a locally optimal solution  is guaranteed for Algorithm 1 based on the recent results in \cite{hong2016unified}.
%Furthermore, it can be verified that all the lower bound and upper bound functions adopted in (\ref{eq37}), (\ref{eq40}), and  (\ref{eq48})  have the same gradients as their original functions at the given local points, respectively. Thus, Algorithm 1 is guaranteed to converge to at least a locally optimal solution  based on the recent results in \cite{hong2016unified}.


%\begin{proposition}
%The max-min throughput obtained in each iteration of Algorithm 1 is monotonically non-decreasing and serves as a lower bound of problem (\ref{probm6}). Furthermore, the sequence $\{\mathbf{A}^{r}, \Q^{r},  \mathbf{P}^r\}$ converges to a point that fufills the KKT optimality conditions of the original non-convex problem (\ref{probm6}).
%\end{proposition}
%\begin{proof}
%Denote by $t$ the objective function of  problem (\ref{probm6}). Then, it follows that
%\begin{align}\label{eq51}
%t(\A, \Q^r, \pow^r)&\overset{(a)} \leq t(\A^{r+1}, \Q^r, \pow^r) \overset{(b)}= t_{\rm trj}(\A^{r+1}, \Q^r, \pow^r)\overset{(c)}\leq  t_{\rm trj}(\A^{r+1}, \Q^{r+1}, \pow^r) \nonumber\\
%&\overset{(d)}\leq t(\A^{r+1}, \Q^{r+1}, \pow^r)\overset{(e)} =  t_{\rm pow}(\A^{r+1}, \Q^{r+1}, \pow^r)\overset{(f)} \leq  t_{\rm pow}(\A^{r+1}, \Q^{r+1}, \pow^{r+1})  \nonumber\\
%&\overset{(g)}\leq  t(\A^{r+1}, \Q^{r+1}, \pow^{r+1}),
%\end{align}
%where  inequalities $(a)$,  $(c)$,  and $(f)$ hold due to the fact that the relaxed problem and the approximated problems, i.e., (\ref{probm27}), (\ref{probm41}), and (\ref{probm50}), are all solved with optimality; equalities $(b)$ and $(e)$ hold due to the fact that the approximated problems (\ref{probm41}) and (\ref{probm50}) resulting from Taylor expansion are both tight at the given local points, i.e.,  $\Q^{r}$ and $\pow^{r}$, respectively; inequalities $(d)$ and $(g)$ hold due to the fact that the objective function values of (\ref{probm41}) and (\ref{probm50}) are respectively the lower bounds of that of their original problems. Therefore, (\ref{eq51}) guarantees that Algorithm 1 monotonically improves the objective function value of problem (\ref{probm6}) in each iteration.
%%Denote $\mathcal{F}$ as the objective function of problem (\ref{probm6}).
%Furthermore, it can be verified that the lower bound and upper bound functions adopted in (\ref{eq37}), (\ref{eq40}), and  (\ref{eq48})  have the same gradients as their approximating functions at the given local points. Thus, the convergence to a local optimal point is guaranteed for Algorithm 1 based on the results developed in \cite{hong2016unified}. %The details of the proof are omitted here for brevity.
%\end{proof}

%In essence, the tightness of the approximation at any local points and the optimal solutions of the approximating problems suffice to guarantee that the monotonically non-decreasing property of the obtained objective value. Together with the fact that each approximating problem also has the identical gradient as its approximated one, the convergence for a local optimal point is eventually guaranteed.



%Since only one block of variables is updated in each iteration by solving a convex optimization problem, the per-iteration storage and computational complexity is relatively low, which is desirable in practice even when $M$, $K$, and $N$ are very large. In fact, the iterations required for convergence can be further decreased  by developing different coordinate selection rules, which however is out the scope of this paper  \cite{hong2016unified}.

Note that in Algorithm 1, the UAV trajectory has to be initialized. It is known that for such iterative algorithms, the converged solution and the ultimate system performance in general depend on the initialization schemes.  Thus, we further propose an efficient trajectory initialization scheme, which is elaborated in the next subsection.  % \cite{kha2012fast}


    \begin{figure}[!t]
\centering
\includegraphics[width=0.4\textwidth]{circle_packing3.eps}
\caption{An example of  UAV trajectories initialization based on circle packing  for $M=2$ (left) and $M=3$ (right). The black dots and the dashed blue circles are the results obtained from the circle packing scheme. The solid red circles are the initial circular trajectories of UAVs.}
\end{figure}
 \subsection{Trajectory Initialization Scheme}
In this subsection, we propose a low-complexity and systematic  initialization scheme for the trajectory design in Algorithm 1 based on the simple circular trajectory and the circle packing scheme. % Specifically, it is assumed that each UAV flies along a circle with the same radius $r$ at a constant speed $V\leq V_{\max}$.
%As a result,  two parameters, i.e., the circle radius $r$ and center $C_m$, $ m\in \mathcal{M}$, need to be specified for the initial circular trajectory.
Specifically, the initial  trajectory of each UAV is set to be a circular trajectory with the UAV speed taking a constant value $V$, with $0< V\leq V_{\max}$. Furthermore, the radius of the initial trajectory circles are assumed to be the same for all UAVs. The center and radius of the circular trajectories are denoted by $\mathbf{c}^m_{\rm trj}=[x^m_{\rm trj}, y^m_{\rm trj}]^T$ and $r_{\rm trj}$, respectively. Thus, for any given period $T$, we have $2\pi r_{\rm trj}=VT$.
Intuitively, circles that correspond to the initial trajectories of different UAVs should be sufficiently separated to minimize the co-channel interference, and at the same time, all circles together should cover the entire area as much as possible so as to better balance the users' rates.
Therefore, the initial circular trajectories are obtained based on  circle packing. To this end, we first determine the geometric center of users as $\mathbf{c}_{\rm g}=\frac{\sum_{k=1}^{K}\mathbf{w}_k}{K}$. The radius of the minimum circle with $\mathbf{c}_{\rm g}$ as the circle center which can cover all users is denoted by $r_{\rm u}$, which is equal to the maximum distance between  $\mathbf{c}_{\rm g}$ and all the users, i.e.,
%\begin{align}
$r_{\rm u} = \max\limits_{k\in \mathcal K } ||\mathbf{w}_k-\mathbf{c}_{\rm g}||.$
Given the number of UAVs $M$ and   $r_{\rm u}$, we exploit the circle packing (CP) scheme \cite{circle_packing}, also known as point packing, to obtain the center of each of the $M$ circles $\mathbf{c}^m_{\rm trj}$ as well as the corresponding radius $r^{\rm cp}$. To balance the number of users inside and outside the circular trajectory, $\frac{r^{\rm cp}}{2}$ is a reasonable choice for the trajectory circle radius. However, due to the maximum UAV speed constraint, the resulting radius $\frac{r^{\rm cp}}{2}$ may not be always achievable given the finite time $T$ if $\pi r^{\rm cp}>V_{\max}T$. In this case, the maximum allowed radius is computed as
\begin{align}
r_{\max} = \frac{V_{\max}T}{2\pi}.
\end{align}
%Therefore, the centers and the radius of the initial trajectory circles are obtained as $C_m= C_{m}^{\rm cp}$ and $r=\min (r_{\max},\frac{r^{\rm cp} }{2})$.
As such, the radius of the initial circular trajectory is set as $r_{\rm trj}=\min (r_{\max},\frac{r_{\rm cp} }{2} )$.  Let $\theta_n \triangleq  2\pi\frac{(n-1)}{N-1}$, $\forall\, n$, and  $\mathbf{Q}^0=\{\mathbf{q}^0_{m}[n], \forall\,m,n\}$. Based on $\mathbf{c}^m_{\rm trj}$ and $r_{\rm trj}$, the initial trajectory of UAV $m$ in time slot $n$ is then obtained as
\begin{align}\label{inital_trj}
\q^0_m[n] = \left[x^m_{\rm trj} + r_{\rm trj}\cos\theta_n,  y^m_{\rm trj} + r_{\rm trj}\sin\theta_n\right]^{T}, n=1,...,N.
\end{align}
Note that for $M\geq 2$, if the inter-UAV distance is larger than or equal to $d_{\min}$, then the trajectory obtained in \eqref{inital_trj} is feasible for original problem \eqref{probm6}. Otherwise, a feasible initial trajectory can be always obtained by scaling $r_{\rm u}$ such that $r_{\rm cp}$ is larger than or equal to $d_{\min}$.

% In this subsection,  we propose a low-complexity  trajectory initialization scheme for Algorithm 1 based on the simple circular trajectory. Specifically, the initial UAV trajectory is set to be a circular trajectory with the UAV speed taking a constant value $V$, with $0< V\leq V_{\max}$. The trajectory circle center and radius are denoted as $\mathbf{c}_{\rm trj}=[x_{\rm trj}, y_{\rm trj}]^T$ and $r_{\rm trj}$, respectively. Then, for any given period $T$, we have $2\pi r_{\rm trj}=VT$. %Thus,  two parameters, i.e., center $C=(x_{\rm trj}, y_{\rm trj})$ and circle radius $r_{\rm trj}$, need to be specified.\left[\frac{\sum_{k=1}^{K}x_i}{K}, \frac{\sum_{k=1}^{K}y_i}{K}\right]^T
%  To balance user rates, the geometric center  is a reasonable choice for the circle center of the initial UAV trajectory, i.e.,  $\mathbf{c}_{\rm trj}=\frac{\sum_{k=1}^{K}\mathbf{w}_i }{K}$. The minimum radius of a circle with $\mathbf{c}_{\rm trj}$ as the circle center which  can cover all users is denoted by $r_{\rm u}$, which is the maximum distance between  $\mathbf{c}_{\rm trj}$ and all the users, i.e.,
%%\begin{align}
%$r_{\rm u} = \max\limits_{i\in \mathcal K } ||\mathbf{w}_i-\mathbf{c}_{\rm trj}||.$
%%\end{align}
%To balance the number of users inside and outside the UAV trajectory circle, $\frac{r_{\rm u} }{2}$ is a reasonable candidate for the circle radius.
%However, due to the maximum UAV speed constraint, the resulting radius $\frac{r_{\rm u} }{2}$ may not be always achievable given a finite period $T$ if $\pi r_{\rm u} >V_{\max}T$. In this case, the maximum allowed radius is computed as
%%\begin{align}
%$r_{\max} = \frac{V_{\max}T}{2\pi}.$
%%\end{align}
%As such, the radius of the initial circular trajectory is obtained as $r_{\rm trj}=\min (r_{\max},\frac{r_{\rm u} }{2} )$. Let $\theta_n \triangleq  2\pi\frac{(n-1)}{N-1}$, $\forall\, n$, and $\Q^0=\{\q^0[n], \forall\,n\}$. Based on $\mathbf{c}_{\rm trj}$ and $r_{\rm trj}$, the initial UAV trajectory in time slot $n$  is obtained as $\q^0[n] = \left[x_{\rm trj} + r_{\rm trj}\cos\theta_n,  y_{\rm trj} + r_{\rm trj}\sin\theta_n\right]^{T}$,  $n=1,...,N$.


  \subsection{Reconstruct the Binary User Scheduling and Association Solution}
  Note that Algorithm 1 is to solve the relaxed problem (\ref{probm66}) where the binary user scheduling and association variables in the original problem  (\ref{probm6}) are relaxed to continuous variables between 0 and 1.
Thus, in the solution obtained by Algorithm 1, if the user scheduling and association variables $\alpha_{k,m}[n]$ are all binary, then the relaxation is tight and the obtained solution is also a feasible solution of problem (\ref{probm6}). Otherwise, the binary user scheduling and association solution needs to be reconstructed based on the solution obtained for (\ref{probm66}).  To this end, we further divide each time slot into $\tau$ sub-slots so that the new total number of sub-slots is $N' = \tau N$, $\tau\geq 1$. Then, the number of sub-slots assigned to user $k$ by UAV $m$ in time slot $n$ is  $N_{k,m}[n]=\lfloor\tau \alpha_{k,m}[n]\rceil$, where $\lfloor x\rceil$ denotes the nearest integer of $x$.
It is not difficult to see that as $\tau$ increases, $N_{k,m}[n]$ approaches an integer which allows a binary solution.
%we round the non-binary variables to zeros or ones to construct a binary solution for problem (\ref{probm6}). %It can be shown that the objective value gap due to such a rounding approaches to zero as long as $N$ is sufficiently large.
%%between the objective value achieved by the constructed binary solution and that of the original non-binary one approaches to zero as long as $N$ is sufficiently large.
%In fact, $N$ is a controllable parameter to model the UAV trajectory and it can be always further divided into a smaller granularity such that the gap arising from the rounding decreases. %which shows the asymptotical optimality of the proposed Algorithm 1.
%For the case when a user is scheduled for communication by two UAVs,
For example, consider a single-UAV enabled two-user system with $\alpha_1[\ell]=0.69$ and $\alpha_2[\ell]=0.31$ in time slot $\ell$, where the UAV index is dropped for convenience.
 If $\tau=1$, we have $N_1[\ell]=\lfloor0.69]=1$ and $N_2[\ell]=\lfloor0.31\rceil=0$, respectively. If each time slot is further divided into $10$ sub-slots, i.e., $\tau=10$, then $N_1[\ell]=\lfloor6.9\rceil=7$ and $N_2[\ell]=\lfloor3.1\rceil=3$, respectively. Although such a rounding  still causes a performance gap, the gap decreases as the duration of the sub-slot decreases. Alternatively, if each time slot is divided into $100$ sub-slots, i.e., $\tau=100$,  user 1 and user 2 will be assigned 69 and 31 sub-slots, respectively, i.e., $N_{1}[\ell]=\lfloor69\rceil=69$ and $N_{2}[\ell]=\lfloor31\rceil=31$, which permits a binary solution with zero relaxation gap.  Furthermore,  since constraints (\ref{eq70}) and (\ref{eq80}) are met with equalities in the optimal solution to problem (\ref{probm25}), a binary solution for the case of multiple UAVs can be easily reconstructed by applying the above procedure.

{It is worth pointing out that such a reconstructed binary solution is always feasible for problem \eqref{probm6} with the same larger $N'$ slots, while we do not need to resolve problem \eqref{probm6} with $N'>N$ directly to avoid high computational complexity.  Thus, the complexity of our proposed approach is lower compared to that of directly solving problem  \eqref{probm6}  with $N'$ slots.} {On the other hand, the case of $\tau=1$ which directly rounds off the continuous variables to binary ones,  is a special case of the proposed scheme but at the expense of certain performance loss in general. Therefore, our proposed scheme not only ensures to obtain a feasible solution to problem  \eqref{probm6}  with any given $N$ slots, but also can achieve higher accuracy and better performance by using $N'>N$ slots yet without increasing the complexity.} {In other words, if the number of time slots $N'$ is set very large initially, then directly solving problem  \eqref{probm6}   with $N'$ will incur very high complexity. In this case, we can first formulate and solve the problem with a smaller $N=N'/\tau$ by choosing a suitable $\tau>1$ (note that $\tau$ cannot be set too large as this may render the discrete-time approximation of the UAV trajectory inaccurate),  and then use our results to construct a feasible solution to problem \eqref{probm6}  with the larger number of times slots $N'$, to achieve lower complexity.}
% For problem (\ref{probm_single}) that considers a network with a single UAV,
%% , $p[n]= P_{\max}$ always holds and power control is thus not needed.
% %The constraints (\ref{eq7})-(\ref{eq9}) in user scheduling problem (\ref{probm25}) are replaced by (\ref{eq15}) and (\ref{eq16}).
%an iterative optimization between user scheduling and association and UAV trajectory as in Algorithm 1 is guaranteed to obtain a local optimal solution of (\ref{probm_single}). Problem (\ref{probm622}) can also be solved similarly by a slight modification of Algorithm 1 .
 \section{Numerical Results}
In this section, we provide numerical examples to demonstrate the effectiveness of the proposed algorithm. We consider a system with $K=6$ ground users that are randomly and uniformly distributed within a 2D area of $2\times 2$ km$^2$. The following results are obtained based on one random realization of the user locations as shown in Fig. \ref{single_trajectory}.  All the UAVs are assumed to fly at a fixed altitude $H=100$ m.
%The communication bandwidth is $B = 1$ MHz and the noise power spectrum density at the ground users is assumed to be identical and set as $N_0 = −170$ dBm/Hz.
 The receiver noise power is assumed to be $\sigma^2= -110$ dBm. The channel power gain at the reference distance $d_0= 1$ m is set as $\rho_0=-60$ dB. The maximum transmit power and the maximum speed of UAVs are assumed as $P_{\max}=0.1$ W and $V_{\max}=50$ m/s, respectively. The threshold $\epsilon$ in Algorithm 1 is set as $10^{-4}$. The transmit power of the UAVs is initialized by the maximum transmit power, i.e., $p_m[n] = P_{\max}, \forall\, m$. Other parameters are set as $d_{\min} =100$ m and $\tau=100$.


\subsection{Singe UAV Case}




\begin{figure}[!t]
\centering
\includegraphics[width=0.4\textwidth]{paper_single_trj9.eps}
\caption{Optimized UAV trajectories for different periods  $T$ for a single-UAV system. Each trajectory is sampled every 5 s and the sampled points are marked with `$\triangle$' by using the same colors as their corresponding trajectories. The user locations are marked by Blue circles `$\odot$'.} \label{single_trajectory}%\vspace{-0.5cm}
\end{figure}
   \begin{figure}[!t]
\centering
\includegraphics[width=0.4\textwidth]{s6.eps}
\caption{The UAV speed versus time for $T=210$ s.} \label{speed_paper}%\vspace{-0.6cm}
\end{figure}





%\subsection{UAV Trajectory versus Cyclical Multiple Access Period  $T$}
{We first consider the special case with one single UAV, i.e., $M=1$, where there is no co-channel interference in the system. It is not difficult to see that in this case, the UAV should always transmit with its maximum power, i.e., $p[n]=P_{\max}, \forall\,n$. Then, problem (\ref{probm6}) is simplified to a joint user scheduling and UAV trajectory optimization problem that can be solved by a slight modification of Algorithm \ref{Algo:succ}.} In Fig. \ref{single_trajectory}, we illustrate the optimized trajectories obtained by the proposed Algorithm 1 under different periods $T$. It is observed that as $T$ increases, the UAV
exploits its mobility to adaptively enlarge and adjust  its trajectory to move closer to the ground users. When $T$ is sufficiently large, e.g., $T=210$ s,  the UAV is able to sequentially visit all the users and stay stationary above each of them for a certain amount of time (i.e., with a zero speed), while  the UAV trajectory becomes a closed loop with segments connecting all the points right on top of the user locations. Except the time spent on traveling between the user locations, the UAV sequentially hovers above the users so as to enjoy the best communication channels. For example, for the case of $T=210$ s, it can be observed that the sampled points on the trajectory around each user have higher densities than those far way from the users. This means that when the UAV flies close to each user, it will reduce the speed accordingly such that more information can be transmitted over a better air-to-ground channel. This phenomenon can be more directly observed from Fig. \ref{speed_paper} for the case of $T=210$ s, where the UAV speed reduces  to zero when it flies right above  each user, such as $t=35$ s. While for $T=30$ and $60$ s, the UAV always flies at the maximum speed $V_{\max}$ in order to get as close to each user as possible for shorter LoS links within each limited period $T$.



%\vspace{-0.2cm}
%\subsection{Max-min Rate versus Cyclical Multiple Access Period $T$}
In Fig. \ref{single_throughput}, we compare the average max-min rate achieved by the following schemes: 1) Proposed trajectory, which is obtained by Algorithm 1; 2) Circular trajectory, which is obtained by the proposed  initialization scheme with $M=1$; and 3) Static UAV, where the UAV is placed at the geometric center of the user positions and remains static during the whole period $T$. For all the three schemes, the user scheduling is optimized by Algorithm 1 with given trajectory. It is observed from Fig. \ref{single_throughput} that the max-min rate of the static UAV is independent of $T$ since without mobility, the channel links between the UAV and users are time-invariant. In contrast, for the proposed trajectory and the circular trajectory schemes, the max-min rate increases with $T$ and eventually becomes saturated when $T$ is sufficiently large. This is expected since with the UAV mobility,  a larger $T$ provides the UAV more time to fly closer to the users to be served, which thus improves the max-min rate. {In addition, when $T$ and/or $V_{\max}$ is sufficiently large such that the UAV's travelling time between users is negligible, each ground user is sequentially served with equal time duration when the UAV is directly on top of it. In this case,  the max-min rate for each user can be obtained as }
 %small portion of time is spent for the UAV's flying and most of time is spent for hovering right on top of the users. Thus, the channel link quality, although better, again becomes time-invariant and thus leads a saturated max-min data rate.
% In light of this, the saturated max-min data rate achieved by the proposed scheme is upper bounded by the one when a UAV is hovering right on top of a user. In fact, by ignoring the flying time,  the upper bound of the max-min average data rate for each user can be obtained as
\begin{align}\label{upperbound}
R^{\rm ub} = \frac{1}{K}  \log_2\left( 1 + {\frac{P\rho_0}{H^2\sigma^2}}\right)= 1.6612\  \text{bps/Hz}.
\end{align}
{It is worth pointing out that since the travelling time in practice is always not negligible  for any finite UAV speed,  the maximum objective value of problem \eqref{probm6} is strictly upper-bounded by the rate  in \eqref{upperbound}. As the obtained trajectory by our proposed algorithm is able to move the UAV to be above of each user,  the asymptotic optimality of the proposed algorithm can be demonstrated with increasing $T$, which can be seen in Fig. \ref{single_throughput}. In Fig. \ref{user_delay}, we plot the access delay for two of the users versus the period $T$ based on the optimized user scheduling variables.
%Specifically, the access delay of a user is calculated by minus the corresponding user scheduling time from the period $T$.\\
One can observe that as $T$ increases, the user access delay also increases, which implies that each user needs to wait for a longer time to be scheduled for communication with the UAV. Based on Figs. \ref{single_throughput} and \ref{user_delay}, the fundamental delay-throughput tradeoff is demonstrated.  }
%the proposed algorithm achieves the performance upper bound of the
%The asymptotic optimality of the proposed algorithm is shown as $T$ increases



\begin{figure}[!t]
\centering
\includegraphics[width=0.4\textwidth]{paper_single_N10.eps}
\caption{Max-min rate  versus period $T$ for a single-UAV system with different trajectory designs.} \label{single_throughput}%\vspace{-0.6cm}
\end{figure}

\begin{figure}[!t]
\centering
\includegraphics[width=0.4\textwidth]{real_delay1.eps}
\caption{User access delay versus period $T$ for a single-UAV system. The locations of users 1 and 2 are $[-419, 400]^T$ and $[600, 1130]^T$ in m, respectively, which are shown in Fig. \ref{single_trajectory}. } \label{user_delay}%\vspace{-0.6cm}
\end{figure}

%%   \begin{figure}[!t]
%\centering
%\subfigure[$ \theta_2= 0.4$]{\includegraphics[width=4.5in, height=3.5in]{paper_single_N10.eps}}
%\subfigure[$\theta_2=0.8$]{\includegraphics[width=1.7in, height=1.1in]{hete083.eps}}
%\caption{Max-min rate and user access delay versus period $T$ for a single-UAV system with different trajectory designs.} \label{single_throughput}%\vspace{-0.6cm}
%\end{figure}

By comparing the performance of the proposed trajectory with that of the circular trajectory in Fig. \ref{single_throughput},  the advantage of fully exploiting the trajectory design  is also demonstrated. Since the circular trajectory restricts the UAV to fly along a circle, the users that are not around the circle suffer from worse channels.  As a result,  more time needs to be assigned to those users, which poses the bottleneck for the achievable max-min throughput.
While for the proposed trajectory with a sufficiently large period $T$, the UAV is able to fly closer to or even stays stationary above all users to serve them with better channels. Therefore, the max-min throughput is improved, but at the cost of longer access delay on average for the users.






%\newpage
%In Fig. \ref{single_trajectory}, we illustrate the trajectories obtained by the proposed algorithm for a single UAV network under different optimization horizon $T$. It is observed that as $T$ increases, the UAV
%exploits its mobility to adaptively enlarge and shape its trajectory based on the given user locations such that as many users as possible can be covered. When $T$ is sufficiently large, e.g., $T=120$ s,  such that the UAV can fly over each user, the UAV's trajectory becomes a closed loop with segments connecting all the points right on top of users together. Except the time spent on the way to each user, the UAV in turn hovers on top of users for the rest of available time. For example, when $T=120$ s, it can be observed that the densities of the sampling points around each user are larger than that of the sampling points far way from users. This means that when the UAV flies close to user, its speed will be accordingly lowered such that more data can be transmitted over a better air-ground channel. This phenomenon can be more directly observed from Fig. \ref{speed_paper} where the UAV's speed approaches to zero when it flies around a user for the case of $T=120$ s. While for the cases of $T=30$ and $60$ s, the UAV always flies at the maximum speed $V_{\max}$ in order to get as close to each user as possible for shorter LoS links.
%
%
%In Fig. \ref{single_throughput}, we compare the average max-min data rate achieved by the following trajectory designs: 1) Proposed trajectory: the trajectory obtained by Algorithm 1 for problem (\ref{probm_single}); 2) Circular trajectory: exhaustive search the optimal radius with the geometric center of users' locations as the circle radius; 3) Static UAV: the UAV is placed at the geometric center of locations of users. For all the three schemes, the user scheduling is optimized by Algorithm 1. It is observed from Fig. \ref{single_throughput} that the max-min rate of the static UAV scheme is independent of the optimization horizon since without mobility, the channel links between the UAV and users are time-invariant. In contrast, for the proposed trajectory and the circular trajectory schemes, the max-min rate increases with $T$ and eventually become saturated when $T$ is sufficiently large. This can be explained as follows. Due to the mobility of the UAV,  a larger $T$ provides the UAV more time to fly closer to users, which thereby improves the max-min data rate. However, when $T$ is sufficiently large, only a small portion of time is spent for the UAV's flying and most of time is spent for hovering right on top of the users. Thus, the channel link quality, although better, again becomes time-invariant and thus leads a saturated max-min data rate. In light of this, the saturated max-min data rate achieved by the proposed scheme is upper bounded by the one when a UAV is hovering right on top of a user. In fact, by ignoring the flying time,  the upper bound\footnote{Note that the upper bound is not the optimal solution of problem (\ref{probm_single}), but the solution without considering the flying time or the case when $V_{\max}$ approaches infinity.} of the max-min average data rate for each user can be obtained as
%\begin{align}
%R^{\rm ub} = \frac{1}{K}  \log_2\left( 1 + {\frac{P_{\max}\rho_0}{H^2\sigma^2}}\right)= 2.2146\  \text{bps/Hz}.
%\end{align}
%Therefore, the max-min rate achieved by the proposed algorithm approaches this value as $T$ increases which also demonstrates its asymptotic optimality.
%
%Furthermore, the performance gain of the proposed trajectory over the circular trajectory comes from the full exploration of the 2D free space. Since the circular trajectory restricts the UAV to fly along a circle, those users that are not around the circle suffer from degraded channels.  As a result, these users need to be compensated by assigning them more time slots, which constitute the bottleneck for achieving a higher max-min data rate.
%While for the proposed trajectory with sufficiently large optimization time $T$, the UAV is able to fly towards all users to serve them directly. Therefore, each user has the chance of enjoying short-distance LoS links, which improves the max-min data rate significantly. %The performance gain becomes more evident when $T>100$ s when the time is sufficient for the UAV in the proposed scheme to hover on top of the users.
%\newpage
   \begin{figure}[!t]
\centering
\includegraphics[width=0.4\textwidth]{convergence_joint3.eps}
\caption{Convergence behaviour of the proposed Algorithm 1.} \label{convergence}
\end{figure}


\subsection{Multi-UAVs Case}

Next, we study the max-min throughput of the  multi-UAV network.  Before the performance comparison, we show the convergence behaviour of the proposed Algorithm 1  in Fig. \ref{convergence} for the case of two UAVs under $T=90$ s.  It can be observed from the figure that the max-min rate achieved by the proposed algorithm increases quickly with the number of iterations and the algorithm converges in about 40 iterations.

   \begin{figure}[!t]
\centering
\includegraphics[width=0.4\textwidth]{joint_compare4.eps}
\caption{Max-min rate versus period $T$ for a two-UAV system with different optimization schemes.} \label{twoUAV_T}
\end{figure}



In order to show the performance gain brought by the optimization of the different design variables  in Algorithm 1,
 in Fig. \ref{twoUAV_T}, we compare the following three schemes for a two-UAV network, namely,  1) Scheme I: All variables are jointly optimized as in Algorithm 1; 2) Scheme II: Jointly optimized  user scheduling and association as well as UAV trajectory but with full transmit power (i.e., no transmit power control); and 3) Scheme III: Optimized user scheduling and association but with simple  circular trajectory and full transmit power of UAVs. Several important observations can be made from Fig. \ref{twoUAV_T}. First, as expected,  the max-min rates  of all the three schemes increase as the period $T$ becomes large. %, which demonstrates that the UAV mobility indeed benefits the system performance.
% Furthermore, by comparing Schemes II and III or Schemes I and II,  the effectiveness of trajectory optimization or  the power control is clearly demonstrated.
Second,  the performance gap between Scheme II and Scheme III demonstrates the throughput gain brought by the proposed trajectory design even without transmit power control applied, and the performance gap between the two schemes increases with increasing $T$.  This is because with larger $T$, the optimization of  UAVs' trajectories becomes more crucial for both achieving better direct links and avoiding severe co-channel interference links, especially when there is no transmit power control applied, whereas restricting the UAVs flying along circles limits the potential of UAV mobility.
  Second, by comparing Scheme I and Scheme II, the additional gain of power control is also demonstrated. When the power control can be optimized, it also provides more flexibility for designing UAVs' trajectories, which helps achieve better user rates.
%  This is reasonable since when more time are available for UAVs to capture and sustain strong LoS links, the effect of power control on mitigating co-channel interference becomes less significant.
Last but not the least, by comparing Scheme I and its counterpart for the case of a single UAV in Fig. \ref{single_throughput}, it is observed that the user access delay is significantly reduced by employing two UAVs to serve users jointly. For example, to achieve the same average max-min rate about $1.60$ bps/Hz, a single-UAV system requires more than $T=800$ s as shown in Fig. \ref{single_throughput}, whereas this value dramatically reduces to about $T=70$ s for a two-UAV system, both applying the proposed Algorithm 1.
 %the max-min rate achieved when $T=800$ s in a single-UAV network,  only requires  $T=100$ s in a two-UAV network.
 Such a performance gain is mainly attributed to two facts. On one hand,  the spectrum efficiency is improved by allowing concurrent transmissions of the  two UAVs with the same power budget. In fact, this can be directly observed by comparing  the upper bound of the max-min rate for a single-UAV system which  is $1.6612$ bps/Hz given in  (\ref{upperbound}) with the achievable max-min rate of the two-UAV system which is more than  $2.00$ bps/Hz as shown in Fig. \ref{twoUAV_T}.
  On the other hand, the traveling time of each UAV over its served  ground users is reduced and the average air-to-ground channels are also improved when the number of UAVs increases,  which saves more time for them to stay above  each user to maintain the best LoS channels. In summary, the above observations demonstrate the effectiveness of employing multiple UAVs  for improving the user throughput and/or reducing the access delay, which thus improves the fundamental throughput-access delay tradeoff.

%%strong, good, superior, short


\begin{figure}[!t]
\centering
\subfigure[Optimized UAV trajectories without power control.]{\includegraphics[width=0.4\textwidth]{trajectory_wopower4.eps}} %width=3.5in, height=2.7in
\subfigure[Optimized  UAV trajectories with power control.]{\includegraphics[width=0.4\textwidth]{trajectory_joint4.eps}}
%\subfigure[Optimized  UAV trajectories with orthogonal transmission.]{\includegraphics[width=3.2in, height=2.5in]{trajectory_joint2.eps}}
%\subfigure[Optimized  UAV trajectories with power control.]{\includegraphics[width=0.45\textwidth]{trajectory_joint2.eps}}
\caption{ Trajectory comparison for a two-UAV system when $T=90$ s. The initial locations of trajectories are marked with blue square `$\Box$'. Black arrows represent the directions of the trajectories. Each trajectory is sampled every 5 s and the sampling points are marked with `$\triangle$'s by using the same colors as their corresponding trajectories. } \label{twouav_trj}\vspace{-0.5cm}
\end{figure}


   \begin{figure}[!t]
\centering
\includegraphics[width=0.4\textwidth]{transmit_power1.eps}
\caption{UAV transmit power versus time for a two-UAV system.} \label{transmit_power}
\end{figure}



In Fig. \ref{twouav_trj}, we compare the optimized UAV trajectories obtained by Schemes I and II with the period $T=90$ s. {It can be observed from Fig. \ref{twouav_trj} (a)  that for Scheme II without power control, i.e., when the maximum transmit power is used by both UAVs, the trajectories of the two UAVs tend to keep away from each other as far as possible to avoid co-channel interference. However, at some pair of UAV locations, this is realized at the cost of sacrificing favourable direct  communication links, especially when they have to serve two users that are close to each other. As a result, the advantage of trajectory design is compromised so as to trade off between the direct channel and the co-channel interference. In contrast, in Fig \ref{twouav_trj} (b) for Scheme I when the transmit power is also optimized,  the two  UAVs can reduce the interference by properly adjusting the transmit power when they get close to each other to serve nearby users. As such,  strong direct links and weak co-channel interference can be achieved at the same time, which helps unlock the potential benefit brought by the trajectory design and thereby achieves a larger max-min rate ($R_k=1.8434$ bps/Hz, $\forall\, k$, with Scheme I versus $R_k=1.5947$ bps/Hz, $\forall\, k$, with Scheme II).}
{The corresponding UAV transmit power versus time is plotted in Fig. \ref{transmit_power}. First, it can be observed that  at any time instant, there is always one UAV that transmits with the maximum power.
 Second, when two UAVs are far away from each other, both of them tend to transmit with the maximum power so as to improve the spectrum efficiency, e.g., from $t=10$ s to $t=20$ s where two UAVs flight towards the opposite directions. In contrast, when the two UAVs are getting very close to each other, one UAV will reduce the transmit power to zero to avoid severe interference, e.g., from $t=40$ s to $t=45$ s where the two UAVs are serving the two nearby users in the center. Therefore, without power control, the communication interference can only be mitigated by adjusting the UAV trajectory, while a joint power control and trajectory  design provides more  flexibility to mitigate the co-channel interference  and thus achieves a higher max-min rate. }



   \begin{figure}[!t]
\centering
\includegraphics[width=0.4\textwidth]{twoUAV_N3.eps}
\caption{Max-min rate  versus period $T$ for a two-UAV system with different trajectory designs and the orthogonal transmission.} \label{twoUAV_throughput}%\vspace{-0.6cm}
\end{figure}




{In Fig. \ref{twoUAV_throughput}, we compare the average max-min rate achieved by the three trajectory designs in a two-UAV system similar to those  in Fig. \ref{single_throughput} for the single-UAV case, i.e., 1) Proposed trajectory; 2) Circular trajectory, which is obtained by the proposed  initialization scheme with $M=2$; and 3) Static UAV, where each  UAV  $m$ is placed at $\mathbf{c}^m_{\rm trj}$ as in the initialization scheme and remains static for the entire $T$.
%Furthermore, the special case II, i.e., dynamic TDMA scheme is also adopted as a benchmark In Fig. \ref{twoUAV_throughput}.
 For all the three schemes, both the user scheduling and association as well as power control are optimized by Algorithm 1 with given corresponding trajectory.} {In addition, an orthogonal UAV transmission scheme is also adopted for comparison. Specifically, the multiple UAVs take turns to transmit information to their served ground users over orthogonal time slots, thus the system is interference-free. This is achieved by imposing the following constraints\footnote{ {For convenience, we select the value of $N$ such that $\frac{N}{M}$ is an integer for a given $M$.}},
\begin{align}
\kern -2mm &\sum_{k=1}^{K}\alpha_{k,m}[M(\ell-1)+m]\leq 1, \forall\,m, \ell=1,\cdot\cdot\cdot,\frac{N}{M}, \label{eq220}\\
\kern -2mm &\sum_{k=1}^{K}\alpha_{k,j}[M(\ell-1)+m]= 0, \forall\,j\neq m, \ell=1,\cdot\cdot\cdot,\frac{N}{M}, \label{eq221}
 \end{align}}
{\kern -1.6mm which  guarantee that in each time slot, only one UAV is allowed to transmit.  Accordingly, the achievable rate of user $k$ can be  expressed as
{\myfont \begin{align}
 R^{II}_k=\frac{1}{N}\sum_{n=1}^{N}\sum_{m=1}^{M}\alpha_{k,m}[n]  \log_2\left( 1 + {\frac{p_m[n]\rho_0}{(H^2+||\mathbf{q}_m[n]-\mathbf{w}_k||^2)\sigma^2}}\right).
 \end{align}}
Since the above case  is a special case of problem (\ref{probm6}), the corresponding problem  can be solved similarly by Algorithm 1.}
  As can be seen, the max-min rate of the static-UAV case is still regardless of the period $T$ due to the time-invariant air-to-ground channels. In contrast, by exploiting the UAV mobility,  the max-min rates achieved by the other two trajectory designs  are non-decreasing with $T$, which further demonstrates the fundamental throughput-access delay tradeoff. Compared to Fig. \ref{single_throughput} with a single UAV, it can also be observed that such a tradeoff has been significantly improved (i.e., higher max-min rate is achieved with the same given $T$)  by employing more than one  UAVs. In addition, compared to the orthogonal transmission scheme, the spectrum sharing gain by the two UAVs  is also demonstrated.

\section{Conclusions}
In this paper, we have investigated a new type of multi-UAV enabled wireless networks. Specifically,  the user scheduling and association, UAV trajectories, and transmit power are jointly optimized with the objective of maximizing the minimum average rate among all users.  By means of the block coordinate descent and  the successive convex optimization techniques, an efficient iterative algorithm has been proposed, which is guaranteed to converge.
%Numerical results have demonstrated that the mobility of UAVs in 2D free space provides dual benefits of achieving satisfactory direct links and avoid severe interference. The proposed solution significantly outperforms static UAVs and mobile UAVs with circular trajectories.
 Numerical results demonstrate that the UAV mobility provides the benefit of achieving better  air-to-ground channels as well as additional flexibility for interference mitigation, and thereby improves the system throughput, compared to the conventional case with static BSs. Furthermore, the proposed trajectory design significantly outperforms the simple circular trajectory. The interesting throughput-access delay tradeoff is also shown for multi-UAV enabled communications.

 {Although we focus on the downlink communication scenario from the UAVs to ground users, the problem for the uplink communication scenario from ground users to the UAVs can be pursued by following a similar approach via optimizing the UAV trajectory alternately with the joint optimization of user scheduling and power control. However, how to integrate the solution of the joint optimization of user scheduling and power control into the framework of the block coordinate descent method to guarantee the convergence is challenging and needs further investigation.}
In addition, there are still many other interesting research directions that could be pursued in future work by extending the results of this paper, including e.g.  1) Co-existence design of a network with both aerial and ground BSs; 2) 3-D UAV trajectory design with both altitude and horizontal position optimization; and 3) Energy-efficient UAV trajectory design for the general multi-UAV and/or  multi-user scenario by taking into account the UAV movement energy consumption \cite{zeng2016energy}.  %%  guangchi de paper, my delay constrained paper.

\bibliographystyle{IEEEtran}
\bibliography{IEEEabrv,mybib}


%\begin{IEEEbiography}[{\includegraphics[width=1.8in,height=1.25in,clip,keepaspectratio]{qingqing.jpg}}]{Qingqing Wu}
%(S'13-M'16) received B.Eng. and the Ph.D. degrees in Electronic Engineering from South China University of Technology and Shanghai Jiao Tong University (SJTU), China, in 2012 and 2016 (in advance), respectively. From 2015 to 2016, He worked as a visiting research scholar at the School of Electrical and Computer Engineering, Georgia Institute of Technology, Atlanta, GA, USA.  He received the IEEE WCSP Best Paper Award in 2015 and Exemplary Reviewer of IEEE Communications Letters in 2016. He was also the recipient of outstanding Ph.D. thesis funding in SJTU, 2016. He served as a TPC member of IEEE VTC 2017, Globecom 2016,  etc. His research interests include convex and nonconvex optimization, energy-efficient wireless communications, wireless power transfer, and unmanned aerial vehicle (UAV) communications.
%\end{IEEEbiography}
%
%\begin{IEEEbiography}[{\includegraphics[width=1.8in,height=1.25in,clip,keepaspectratio]{YongZeng}}]{Yong Zeng}
%(S'12-M'14) received the Bachelor of Engineering (First-Class Honours) and Ph.D. degrees both from the Nanyang Technological University, Singapore, in 2009 and 2014, respectively. Since September 2013, he has been with the National University of Singapore, firstly as a Research Fellow and then as a Senior Research Fellow. His research interests include UAV communications, wireless power transfer, massive MIMO and millimeter wave communications for 5G, multi-user MIMO communications. He has published 34 IEEE journal papers (16 as first author) and 22 IEEE conference papers, including one invited paper on IEEE Transactions on Communications, 4 ESI highly cited papers and 2 ESI hot papers. He received the 2017 IEEE Communications Society Heinrich Hertz Award, 2015 IEEE Wireless Communications Letters Exemplary Reviewer Award, and the Best Paper Award for the 10th IEEE International Conference on Information, Communications and Signal Processing. He is currently serving as an Associate Editor of IEEE Access, Leading Guest Editor of IEEE Wireless Communications on ``Integrating UAVs into 5G and Beyond'' and China Communications on ``Network-Connected UAV Communications''. He is the workshop co-chair for two workshops in ICC 2018 and the 23rd Asia-Pacific Conference on Communications (APCC).
%\end{IEEEbiography}
%
%\begin{IEEEbiography}[{\includegraphics[width=1.8in,height=1.25in,clip,keepaspectratio] {RuiZhang}}]{Rui Zhang}
% (S'00-M'07-SM'15-F'17) received the B.Eng. (first-class Hons.) and M.Eng. degrees from the National University of Singapore, Singapore, and the Ph.D. degree from the Stanford University, Stanford, CA, USA, all in electrical engineering.
%
%From 2007 to 2010, he worked as a Research Scientist with the Institute for Infocomm Research, ASTAR, Singapore. Since 2010, he has joined the Department of Electrical and Computer Engineering, National University of Singapore, where he is now a Dean's Chair Associate Professor in the Faculty of Engineering. He has authored over 300 papers. He has been listed as a Highly Cited Researcher (also known as the World's Most Influential Scientific Minds), by Thomson Reuters since 2015. His research interests include wireless information and power transfer, drone communications, wireless information surveillance, energy-efficient and energy-harvesting-enabled wireless communications, multiuser MIMO, cognitive radio, and optimization methods.
%
%He was the recipient of the 6th IEEE Communications Society Asia-Pacific Region Best Young Researcher Award in 2011, and the Young Researcher Award of National University of Singapore in 2015. He was the co-recipient of the IEEE Marconi Prize Paper Award in Wireless Communications in 2015, the IEEE Communications Society Asia-Pacific Region Best Paper Award in 2016, the IEEE Signal Processing Society Best Paper Award in 2016, the IEEE Communications Society Heinrich Hertz Prize Paper Award in 2017, the IEEE Signal Processing Society Donald G. Fink Overview Paper Award in 2017, and the IEEE Technical Committee on Green Communications \& Computing (TCGCC) Best Journal Paper Award in 2017. He was a coauthor of the paper that received the IEEE Signal Processing Society Young Author Best Paper Award in 2017. He served for over 30 international conferences as TPC Co-Chair or Organizing Committee Member, and as the guest editor for 10 special issues in IEEE and other internationally refereed journals. He was an elected member of the IEEE Signal Processing Society SPCOM (2012-2017) and SAM (2013-2015) Technical Committees, and served as the Vice Chair of the IEEE Communications Society Asia-Pacific Board Technical Affairs Committee (2014-2015). He served as an Editor for the IEEE TRANSACTIONS ON WIRELESS COMMUNICATIONS (2012-2016) and the IEEE JOURNAL ON SELECTED AREAS IN COMMUNICATIONS: Green Communications and Networking Series (2015-2016). He is now an Editor for the IEEE TRANSACTIONS ON COMMUNICATIONS, the IEEE TRANSACTIONS ON SIGNAL PROCESSING, and the IEEE TRANSACTIONS ON GREEN COMMUNICATIONS AND NETWORKING.
%\end{IEEEbiography}







\end{document}


