The TAI formalization above gives rise to multiple hierarchies of
tentacular agents.  We discuss some of the these below.



\begin{small}
\begin{enumerate}
\item[\textbf{Syntactic Goal Complexity}] The goal $g$ can range in
  complexity from simple propositional statements,
  e.g. $clean{Kitchen}$, to first-order statements.  e.g.
  $\forall r:\mathsf{Room}: clean(r)$, and to intensional statements
  representing cognitive states of other agents
  $$\believes(a, now, \believes(b, now,\forall r: clean(r)))$$

  %%TODO: Make more precise

\item[\textbf{Goal Variation}] According to the definition
  above, an agent $a$ qualifies as being tentacular if it plans for
  just one goal $g$ in tentacular fashion as laid out in the
  conditions above. We could have agents that plan for a number of
  varied and different goals in tentacular fashion. 

  %%TODO: Make more precise

\item[\textbf{Plan Complexity}] For many goals, there will usually be
  multiple plans involving different actions (with different costs and
  resources used) and executed by different agents. 

  %%TODO: Make more precise



\end{enumerate}
\end{small}


%%++++++++++++++++++++++++++++++++++++++++++++++++++++++++++++++++++++
\begin{figure}[h!]
 \centering
 {
  \includegraphics[width=\linewidth]{./TAIagents.pdf}}
 \caption{Pictorial Overview.  \textit{A bit of explanation: That some
     agents are within agents indicates that the outer agent knows
     and/or believes everything relevant about the inner agent; hence
     as agents are increasingly cognitively powerful, the depth of
     their epistemic attitudes grows (reflected in formulae with
     iterated belief/knowledge operators).  Agents grow in
     size/intelligence in lockstep with the logical calculi upon which
     they are based increasing in expressivity and reasoning power;
     $\mathcal{L}_0$ is zero-order logic, $\mathcal{L}_1$ is
     e.g.\ first-order logic, and the particular cognitive calculus
     $\mathcal{DCEC}$ is shown.  Rotation indicates simply that,
     through time, agents perceive and act.}}
 \label{fig:pictorial_overview}
\end{figure}
%%++++++++++++++++++++++++++++++++++++++++++++++++++++++++++++++++++++

%%% Local Variables:
%%% mode: latex
%%% TeX-master: "main"
%%% End:
