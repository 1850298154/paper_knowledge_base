%%++++++++++++++++++++++++++++++++++++++++++++++++++++++++++++++++++++
\begin{figure}[h!]
 \centering
 {
  \includegraphics[width=\linewidth]{./TAIFlowchart.pdf}}
\caption{TAI Informal Overview: \textit{We have an architecture for
    how a TAI agent $\tau$ might operate. $\tau$ continuously comes up
    with goals based on its contract. If a goal is not achievable
    using $\tau$'s own resources, $\tau$ has to employ other agents in
    achieving this goal. To successfully do so $\tau$ would need to
    have one or more of $\mathbf{D_1} - \mathbf{D_6}$ attributes.  }}
 \label{fig: TAI_Flowchart}
\end{figure}
%%++++++++++++++++++++++++++++++++++++++++++++++++++++++++++++++++++++


We give a quick and informal overview of TAI. We have a set of
agents $a_1, \ldots, a_n$. Each agent has an associated (implicit or
explicit) contract that it should adhere to. Consider one particular
agent $\tau$. During the course of this agent's lifetime, the agent
comes up with goals to achieve so that its contract is not
violated. Some of these goals might require an agent to exercise some
or all of the six attributes $\mathbf{D_1} - \mathbf{D_6}$. We
formalize this using planning as shown in Figure ~\ref{fig:
  TAI_Flowchart}. As shown in the figure, if some goal is not
achievable on its own, $\tau$ can seek to recruit other agents by
leveraging their resources, beliefs, obligations etc.





%%% Local Variables:
%%% mode: latex
%%% TeX-master: "main"
%%% End:
