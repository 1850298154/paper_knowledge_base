%%%% ijcai18.tex

\typeout{IJCAI-18 Instructions for Authors}

% These are the instructions for authors for IJCAI-18.
% They are the same as the ones for IJCAI-11 with superficial wording
%   changes only.

\documentclass{article}
\pdfpagewidth=8.5in
\pdfpageheight=11in
% The file ijcai18.sty is the style file for IJCAI-18 (same as ijcai08.sty).
\usepackage{ijcai18}
\usepackage{enumerate} 
% Use the postscript times font!
\usepackage{times}
\usepackage[usenames,dvipsnames,svgnames,table]{xcolor}
\usepackage{soul}
\usepackage[utf8]{inputenc}
\usepackage[small]{caption}
   \usepackage{mdframed}
  \usepackage{graphicx}
  \usepackage{verbatim}
  \usepackage{fancybox}
  \usepackage{amssymb}
  \usepackage{url}
  \usepackage{amsmath}
  \usepackage{proof}  
  \usepackage{bussproofs}
  \usepackage{verbatim}
  \usepackage{algorithm2e}
  \usepackage{hyperref}
    \hypersetup{
      colorlinks,
      citecolor=blue,
      filecolor=blue,
      linkcolor=blue,
      urlcolor=blue
                 }
  \usepackage{paralist}
  \usepackage[]{algorithm2e}
  \usepackage{graphicx}
  \usepackage{subcaption}

\usepackage{booktabs}
%\usepackage{verbatim}
\usepackage{color}
\usepackage{amsfonts}
\usepackage{amsmath}
\usepackage{proof}
\usepackage[amsmath,thmmarks]{ntheorem}
\usepackage{relsize}
 
\urlstyle{same}
\usepackage{mathtools} 

\usepackage{colortbl}

 \newcommand{\citeaffixed}[1]{\cite{#1}}

 \newcommand{\todoinline}[1]{\todo[inline]{#1}}
%\newcommand{\an}[1]{\todo[inline,color=yellow]{AN: #1}}
\newcommand{\rarrow}{\ensuremath{\rightarrow}\xspace}
\newcommand{\blind}[1]{\texttt{\lipsum[#1]}}


\newcommand{\Classifier}[1]{\boldsymbol{$M_#1$}\xspace}

\newcommand{\Table}{Table\xspace}
\newcommand{\Figure}{Figure\xspace}

\newcommand{\refsec}[1]{Section~\ref{#1}\xspace}
\newcommand{\reffig}[1]{\Figure~\ref{#1}\xspace}
\newcommand{\reftab}[1]{\Table~\ref{#1}\xspace}

\newcommand{\RQ}[1]{RQ\ensuremath{_{#1}}}
\newcommand{\dbo}[1]{\href{http://dbpedia.org/ontology/#1}{\texttt{dbo:#1}}}
\newcommand{\dbr}[1]{\href{http://dbpedia.org/resource/#1}{\texttt{dbr:#1}}}
\newcommand{\owl}[1]{\href{https://www.w3.org/2001/sw/wiki/#1}{\texttt{owl:#1}}}
\newcommand{\ie}{i.e.,~}
\newcommand{\incl}{incl.~}
\newcommand{\wrt}{w.r.t.~}
\newcommand{\st}{s.t.,~}
\newcommand{\eg}{e.g.,~}
\newcommand{\egPlain}{e.g.,}
\newcommand{\Eg}{For example,~}
\newcommand{\cf}{cf.,~}
\newcommand{\etc}{etc.\xspace}
\newcommand{\qq}[1]{``#1''}

\newcommand{\WebQuestions}{WebQuestions\xspace}
\newcommand{\SimpleQuestions}{SimpleQuestions\xspace}
\newcommand{\Babelfy}{Babelfy\xspace}
\newcommand{\DBpediaSpotlight}{DBpedia Spotlight\xspace}
\newcommand{\Dandelion}{Dandelion\xspace}
\newcommand{\MeaningCloud}{Meaning Cloud\xspace}
\newcommand{\DBpedia}{DBpedia\xspace}
\newcommand{\Fscore}{F1 score\xspace}
\newcommand{\Precision}{Precision\xspace}
\newcommand{\Recall}{Recall\xspace}
\newcommand{\Babelscape}{WikiNEuRal\xspace}
\newcommand{\Flair}{Flair\xspace}
\newcommand{\Davlan}{Multilingual BERT-base NER\xspace}
\newcommand{\Qanary}{Qanary\xspace}
\newcommand{\mGenre}{mGENRE\xspace}
\newcommand{\LibreMT}{LibreTranslate\xspace}
\newcommand{\MAG}{MAG\xspace}
\newcommand{\nllbMT}{NLLB\xspace}
\newcommand{\mBART}{mBART-50\xspace}
\newcommand{\spaCy}{spaCy\xspace}
\newcommand{\OpusMT}{Opus MT\xspace}
\newcommand{\OpenNMT}{OpenNMT\xspace}
\newcommand{\LCQuAD}{LC-QuAD\xspace}
\newcommand{\TagMe}{TAGME\xspace}
\newcommand{\QALDplus}{QALD-9-plus\xspace}
\newcommand{\BLEU}{BLEU\xspace}
\newcommand{\chrFPP}{chrF++\xspace}
\newcommand{\Wikidata}{Wikidata\xspace}
\newcommand{\HelsinkiNLP}{Helsinki NLP\xspace}
\newcommand{\BERTScore}{BERTScore\xspace}


\newcommand{\mystar}{\ensuremath{\star}}

\newcommand{\dm}[1]{\textbf{\textcolor{green}{DM:#1}}} % Diego
\newcommand{\an}[1]{\textbf{\textcolor{cyan}{AN:#1}}} % Axel
\newcommand{\tm}[1]{\textbf{\textcolor{blue}{TM: #1}}} % Tatiana
\newcommand{\ns}[1]{\textbf{\textcolor{teal}{NS: #1}}} % Nikit
\newcommand{\hz}[1]{\textbf{\textcolor{cyan}{HZ: #1}}} % Hamada
\newcommand{\mr}[1]{\textbf{\textcolor{violet}{MR: #1}}} % Micha
\newcommand{\hd}[1]{\textbf{\textcolor{red}{Heidi: #1}}} % Heidi
\newcommand{\kd}[1]{\textbf{\textcolor{blue}{KD: #1}}} % Denis


\newcommand{\pendingcite}[1]{\textbf{\textcolor{orange}{cite #1}}} % For pending citations

\newcommand{\llms}{{\glspl{LLM}}\xspace} % For plural llms
\newcommand{\llm}{{\acrshort{LLM}}\xspace} % For llm
\newcommand{\mllms}{{\glspl{MLLM}}\xspace} % For plural mllms
\newcommand{\mllm}{{\acrshort{MLLM}}\xspace} % For mllm
\newcommand{\moe}{{\acrshort{MoE}}\xspace} % For MoE
\newcommand{\moefull}{{\acrfull{MoE}}\xspace} % For MoE full
\newcommand{\com}{\textit{the curse of multilinguality}\xspace} % For MoE
\newcommand{\lm}{{\acrshort{LM}}\xspace} % For Language Model
\newcommand{\nlp}{{\acr{NLP}}\xspace} % For NLP
\newcommand{\ffn}{{\acrshort{FFN}}\xspace} % For FFN
\newcommand{\ffns}{{\glspl{FFN}}\xspace} % For plural FFNs

\newcommand{\link}{\small{\faExternalLink{}}}


\newcommand{\approach}{LOLA\xspace}
\newcommand{\REPLACE}[1]{\textcolor{orange}{#1}\xspace}

\newcommand{\multiLineTabular}[1]{\begin{tabular}[c]{@{}c@{}}#1\end{tabular}}

% the following package is optional:
%\usepackage{latexsym} 

% Following comment is from ijcai97-submit.tex:
% The preparation of these files was supported by Schlumberger Palo Alto
% Research, AT\&T Bell Laboratories, and Morgan Kaufmann Publishers.
% Shirley Jowell, of Morgan Kaufmann Publishers, and Peter F.
% Patel-Schneider, of AT\&T Bell Laboratories collaborated on their
% preparation.

% These instructions can be modified and used in other conferences as long
% as credit to the authors and supporting agencies is retained, this notice
% is not changed, and further modification or reuse is not restricted.
% Neither Shirley Jowell nor Peter F. Patel-Schneider can be listed as
% contacts for providing assistance without their prior permission.

% To use for other conferences, change references to files and the
% conference appropriate and use other authors, contacts, publishers, and
% organizations.
% Also change the deadline and address for returning papers and the length and
% page charge instructions.
% Put where the files are available in the appropriate places.

%% \title{On Tentacular Artificial Intelligence}
\title{Tentacular Artificial Intelligence, and the Architecture Thereof, Introduced}

% Single author syntax
\author{}

% Multiple author syntax (remove the single-author syntax above and the \iffalse ... \fi here)

\author{
\small{Selmer Bringsjord}$^1$, 
Naveen Sundar G$^1$, 
Atriya Sen$^1$, 
Matthew Peveler$^1$, 
Biplav Srivastava$^2$
Kartik Talamadupula$^2$
\\ 
$^1$ \small{Rensselaer Polytechnic Institute (RPI); RAIR Lab}\\
$^2$ \small{IBM Research}\\
%
\small{\texttt{Selmer.Bringsjord@gmail.com}},
\small{\texttt{naveensundarg@gmail.com}},
\small{\texttt{atriya@atriyasen.com}},\\
\small{\texttt{matt.peveler@gmail.com}},
\small{\texttt{biplavs@us.ibm.com}},
\small{\texttt{krtalamad@us.ibm.com}}
}
% If your authors do not fit in the default space, you can increase it 
% by uncommenting the following (adjust the "2.5in" size to make it fit
% properly)
% \setlength\titlebox{2.5in}


\begin{document}

\maketitle

\begin{abstract}
\noindent
%
  We briefly introduce herein a new form of distributed, multi-agent
  artificial intelligence, which we refer to as ``tentacular.''
  Tentacular AI is distinguished by six attributes, which among other
  things entail a capacity for reasoning and planning based in highly
  expressive calculi (logics), and which enlists subsidiary agents
  across distances circumscribed only by the reach of one or more
  given networks.
\end{abstract}

\section{Introduction} 
\section{Introduction}

One of the most fundamental problems in combinatorial optimization is the traveling salesperson problem (TSP), formalized as early as 1832 (c.f. \cite[Ch 1]{ABCC07}).
In an instance of  TSP we are given a set of $n$ cities $V$ along with their pairwise symmetric distances, $c:V\times V \to\R_{\geq 0}$. The goal is to find a Hamiltonian cycle of minimum cost. In the metric TSP problem, which we study here, the distances satisfy the triangle inequality. Therefore, the problem is equivalent to finding a closed Eulerian connected walk of minimum cost.%\footnote{Given such an Eulerian cycle, we can use the triangle inequality to shortcut vertices visited more than once to get a Hamiltonian cycle.}

It is NP-hard to approximate TSP within a factor of $\frac{123}{122}$ \cite{KLS15}.  An algorithm of Christofides-Serdyukov~\cite{Chr76,Ser78} from four decades ago gives a $\frac32$-approximation for TSP.
Over the years there have been numerous attempts to improve the Christofides-Serdyukov algorithm and exciting progress has been made for various special cases of metric TSP, e.g., \cite{OSS11,MS11,Muc12,SV12,HNR21, KKO20, HN19, GLLM21}.
 Recently, ~\cite{KKO21} gave the first improvement for the general case by demonstrating that the so-called ``max entropy" algorithm of \cite{OSS11} gives a randomized $\frac{3}{2}-\epsilon$ approximation for some $\epsilon > 10^{-36}$.% (see \cite{VS20} for a historical note about TSP)

%After a long line of work %~\cite{Wol80,SW90,BP91,Goe95,CV00,GLS05,BM10,BC11,SWV12, HNR17,HN19, KKO20a} 
	%the best known approximation algorithm for the general case of the problem is $\frac{3}{2}-\epsilon$ for some $\epsilon > 10^{-36}$ due to ~\cite{KKO21}, a result that built upon the work of the third author, Saberi, and Singh ~\cite{OSS11}. 
	The method introduced in \cite{KKO21} exploits the optimum solution to the following linear programming relaxation of metric TSP studied by \cite{DFJ59,HK70,BG93}, also known as the subtour elimination LP:
\begin{equation}\label{eq:tsplp}
\begin{aligned}
	\min \quad& \sum_{u,v} x_{\{u,v\}} c(u,v)& \\
	\text{s.t.,} \quad &  \sum_{u} x_{\{u,v\}} = 2&\forall v\in V,\\
	& \sum_{u\in S, v\notin S} x_{\{u,v\}}\geq 2,&\forall S \subsetneq V, S\not= \emptyset\\
	& x_{\{u,v\}}\geq 0 &\forall u,v\in V.
\end{aligned}	
\end{equation} 
	
	 However, ~\cite{KKO21} did not show that the integrality gap of the subtour elimination polytope is bounded below $\frac{3}{2}$, and therefore did not make progress towards the ``4/3 conjecture" which posits that the integrality gap of LP \eqref{eq:tsplp} is $\frac{4}{3}$. In this work we remedy this discrepancy by proving the following theorem, improving upon the bound of $\frac{3}{2}$ from Wolsey~\cite{Wol80} in 1980:

\begin{theorem}\label{thm:main}
	Let $x$ be a solution to LP \eqref{eq:tsplp} for a TSP instance. For some absolute constant $\epsilon > 10^{-36}$, the \hyperlink{tar:alg}{max entropy algorithm} outputs a TSP tour with expected cost at most $\frac{3}{2}-\epsilon$ times the cost of $x$. Therefore the integrality gap of the subtour elimination LP is at most $\frac{3}{2} - \epsilon$. 
\end{theorem} 

To prove \cref{thm:main}, we amend Section 4 of \cite{KKO21} but keep the remainder of the analysis essentially the same. Unlike \cite{KKO21}, this argument now preserves the integrality gap by avoiding the use of the optimum solution in bounding the cost of the matching. See \cref{sec:overview} for a discussion of our new approach.
%We note that the analysis in this paper is not specialized to the max entropy algorithm (although we rely on many results from \cite{KKO21} to obtain \cref{thm:main} itself); instead, it is valid for any algorithm which samples a spanning tree from the support of a solution to LP \eqref{eq:tsplp} and then adds the minimum cost matching on the odd degree vertices of the tree.  
%Instead, we use the polygon representation of near minimum cuts \cite{Ben95,BG08} to bound  the cost of the matching (see the following section for an overview of our new findings). %An added benefit of avoiding the use of OPT in the analysis is  %We remark this makes the analysis constructive 
%We remark that this allows future analyses to explicitly compute and possibly utilize the relevant laminar family of near minimum cuts (whereas previously one needed to know OPT to find the laminar family used in the analysis in \cite{KKO21}).
%In particular, we show that to get a bound better than $\frac{3}{2}$ for this class of algorithm it is (essentially) sufficient to handle the case in which the near minimum cuts of $x$ are a laminar family.

\subsection{Other Consequences}
\paragraph{Path TSP} In recent exciting work, Traub, Vygen, Zenklusen \cite{TVZ20} showed that an $\alpha$-approximation algorithm for metric TSP can be used as a black box to get a $\alpha(1+\eps)$ approximation algorithm for Path TSP. This together with \cite{KKO21} implies that there is a $3/2-\eps$ approximation algorithm for Path TSP (for $\eps>10^{-36}$). On the other hand, it is a folklore result that the integrality gap of the natural LP relaxation of Path TSP is at least $3/2$.  Therefore, a consequence of the above theorem is that although the best possible approximation factors of the two problem are the same (up to polynomial reductions), the natural LP relaxation of metric TSP has a strictly smaller integrality gap.


\paragraph{2-ECSM} In the 2-edge-connected multi-subgraph problem, or 2-ECSM for short, we are given a weighted graph $G$ and we want to find a minimum cost 2-edge-connected spanning subgraph, where an edge can be chosen multiple times.
The classical Christofides-Serdyukov algorithm gives a 3/2-approximation for 2-ECSM and despite significant attempts \cite{CR98,BFS16,SV14,BCCGISW20} improved algorithms were designed only for special cases of the problem.
Since in \cite{BG93} it is shown that LP \eqref{eq:tsplp} is a valid relaxation for 2-ECSM, we obtain:

\begin{corollary}	
For some absolute constant $\epsilon > 10^{-36}$ the \hyperlink{tar:alg}{max entropy algorithm} is a randomized $\frac{3}{2}-\epsilon$ approximation for the 2-edge-connected multi-subgraph problem.
\end{corollary}
%Beyond these theorems, we believe the analysis in this paper will open new avenues to improve the arguments in ~\cite{KKO21}. The analysis in that work is by nature non-constructive because it uses information about the optimal solution. Here we remove this weakness and could in principle construct the proposed fractional matching in polynomial time. Although of course this has no practical benefit since the algorithm always finds the minimum cost matching, this may allow future works to manipulate the algorithm to better serve the analysis.

%We analyze the max-entropy rounding algorithm introduced in \cite{OSS11} and slightly modified in \cite{KKO20, KKO21}. 

%In other words, we design a feasible vector for the $O$-join polytope to ``satisfy'' all near min cuts ``crossed on both  sides'' 


%Whereas Section 4 of ~\cite{KKO21} only deals with the near minimum cuts of $x$ (where $x$ is a solution to LP \eqref{eq:tsplp}) which lie along the optimal Hamiltonian cycle, we deal with all near minimum cuts of $x$ using the so-called polygon representation of near minimum cuts ~\cite{Ben97,BG08}. %The results give new intuition for the structure of cuts that are within $\frac{6}{5}$ or less of the edge connectivity of the graph.

 %: we show that we can incur a cost of $O(\eta^2) \cdot c(x)$ to ensure that the set of cuts with $x(\delta(S)) \le 2+\eta$ is a laminar family.


\subsection{New techniques and contributions}\label{sub:newtechniques}

This paper can be seen as a case study on how to reason about and deal with {\em near} minimum cuts. One can deduce from the classical cactus representation of a graph $G$ \cite{DKL76} (i) the structure of {\em all} min cuts of $G$ and (ii) the structure of the edges of $G$ in the sense that every edge $\{u,v\}$ maps to a unique {\em path} in the cactus between the images of $u$ and $v$. Furthermore, such a path intersects every cycle of the cactus on at most one cactus edge. The theory has found many application from designing fast algorithms
\cite{Kar00,KP09} to the analysis of approximation algorithms for TSP \cite{KKO20} and connectivity augmentation \cite{BGJ20,CTZ21}.

Two decades later, the theory of min cuts was extended to near min cuts in works of Bencz\'ur and Goemans \cite{Ben95, BG08} where they introduced the polygon representation which represents all cuts of a graph with at most $\frac{6}{5}k$ edges, where $k$ is its edge connectivity. Although these works completely classify the structure of all near min cuts of a given graph $G$, they do not characterize the structure of the \textit{edges} of $G$ with respect to these cuts, which can be important in applications (for example, in many of the recent applications of min cuts,
 one also needs to exploit the structure of the edges in relation to the cactus).
The structure on the edges turns out to be highly relevant in this work as well, and as a byproduct of our analysis we make progress towards classifying the way in which the edges of $G$ relate to the structure of the polygon representation.
 
 % and (to some extent) a classification of the set of edges of $G$ with respect to the polygon representation of Bencz\'ur and Goemans.
 
  %i
 %s to give a better understanding of the structure of edges of $G$ with respect to its near min cuts.

  %One can partition the edges of $G$ into sets $F_1\dots,F_m$ such that the set of edges in every min cut $(S,\overline{S})$ of $G$ is the union of edges in a pair $F_i,F_j$ for $i\ neq j$.
%\Nathan{Shayan can add something} For example...

For motivation, consider a generic family of network design problems in which we want to construct a network such that every pair $u,v$ of vertices has connectivity at least $c_{u,v}$. A natural approach is to write an LP relaxation to find a (minimum cost) vector $x: E \to \R_{\ge 0}$ such that for every cut $S$ separating $u$ and $v$, $x(\delta(S))\geq c_{u,v}$. We can round this LP using independent rounding or a dependent rounding scheme such as sampling from max entropy distributions. Using classical concentration bounds one can show that if $x(\delta(S))\gg c_{u,v}$ then with high probability the rounded solution has at least $c_{u,v}$ edges across this cut. So the main challenge is to ``fix'' near tight cuts, i.e., cuts where $x(\delta(S))\approx c_{u,v}$.  For an explicit instantiation of this scheme see \cite{KKOZ22}. A better understanding of the global structure of the family of near tight cuts has the potential to significantly simplify or even improve the approximation factor of such rounding algorithms. A classical technique to design algorithms for such network design problems is to apply uncrossing to extreme point solutions of the LP. One can view our contribution as an approximate uncrossing technique that deals with all near tight cuts (instead of just tight cuts) as we explain next.
%Next, we explain how our main theorem can be used to give global structure for near tight cuts in the case that $c_{u,v}=2$ for all $u,v$ and we contrast it with the classical uncrossing technique which only deals with tight/min cuts. 


\paragraph{An Approximate Uncrossing Technique.} A fundamental technique in the field of approximation algorithms is the uncrossing technique\footnote{See e.g. \cite{LRS11} for a number of applications of this technique.} of Jain \cite{Jai01}. Given a graph $G=(V,E)$,  a weight vector $x:E\to\R_{\geq 0}$, and a  function $f:V\to\R$, suppose that $x(\delta(S))\geq f(S)$ for all $S\subseteq V$. Let $\cN$ be the family of sets $S$ such that $x(\delta(S)) = f(S)$, i.e., the family of {\em tight} sets with respect to $f$. The uncrossing technique says that if $f$ is (weakly) supermodular then we can refine $\cN$ to a laminar family of sets, $\cH$, such that if all sets of $\cH$ are tight, then all sets of $\cN$ are tight as well. For a concrete example, suppose $f$ is a constant function, say $f(S)=2$ for all $\emptyset\subsetneq S\subsetneq V$. Then, sets of $\cH$ can be constructed using the cactus representation \cite{DKL76} of cuts in $\cN$. The significance of this method is that if $x$ is a basic feasible solution to a LP with constraints $x(\delta(S))\geq f(S)$ for all $S$, one can use this machinery to argue that the support of $x$ has size $O(|V|)$.

Informally, we prove the following, which 
can be seen as  an {\em approximate uncrossing technique}: 
\begin{theorem}[Informal]\label{thm:uncrossing}Suppose we have a vector $x:E\to\R_{\geq 0}$ such that $x(\delta(S))\geq f(S)$ for all $S$; define $\cN$ to be sets $S$ where $x(\delta(S))\leq f(S)(1+\eps)$ for some fixed $\eps>0$. If $f(.)$ is constant, say $f(S)=2$ for all $S$, then there is a set $\cN^*\subseteq \cN$ and a collection of edge sets $F_1,\dots,F_m\subseteq E$ such that the following hold:
\begin{itemize}
	\item $|\cN^*|= O(|V|)$, $m= O(|V|)$.
	\item $x(F_i)\geq 1-\eps/2$ for all $1\leq i\leq m$.
	\item Every edge $e$ is in at most $O(1)$ of the $F_i$'s.
	\item For every set $S\in \cN\smallsetminus \cN^*$ there exists $1\leq i<j\leq m$ such that $F_i\cap F_j=\emptyset$ and $F_i\cup F_j\subseteq \delta(S)$ and for every $S\in \cN^*$, there exists $1\leq i\leq m$ such that $F_i\subseteq \delta(S)$. 
\end{itemize}
\end{theorem}
In words, although we cannot simply refine $\cN$ to a linear number of sets, we can refine the edges in cuts of $\cN$ to a linear number of sets $F_1,\dots, F_m$ such  that we can essentially capture the edges of $\delta(S)$ for any $S\in \cN\smallsetminus \cN^*$ by a pair of disjoint $F_i$'s. We give a slightly weaker condition for cuts in $\cN^*$; namely we only capture half of their edges by $F_i$'s.

\begin{example}For a simple example of the above theorem, suppose $\eps=0$, i.e. $\cN$ is the set of min cuts of a graph $G$. Furthermore, suppose that every proper  cut in $\cN$ is \hyperlink{tar:crossing}{crossed} (recall that $S$ is proper if $1<|S|<|V|-1$) and that $\cN$ has at least one proper cut. 
Then, one can use an uncrossing technique, namely that if $A,B\in \cN$ then $A\cap B\in \cN$, to prove that $G$ must be cycle, namely we can order vertices of $G$, $v_0,\dots,v_{n-1}$ such that $x_{\{v_i,v_{i+1\text{ mod n}}\}}=1$.
In such a case we let $\cN^*=\emptyset$ and $F_i=E(v_i,v_{i+1\text{ mod }n})$.
%partition $V$ into sets $a_0,\dots,a_{m-1}$ such that 
%Let $\C$ be a connected component of crossing cuts of $\cN$, namely, for any pair of sets $A,B\in \C$ there is a path of crossing cuts all from $\C$ that goes from $A$ to $B$.
% and further suppose that $\cN$ can be represented by a cycle $C$ in the sense every min cut of $\cN$ corresponds to a min cut of $C$ and vice versa. Here we assume $a_0,\dots,a_{m-1}$ are the nodes of $C$ where each $a_i$ is identified with a disjoint set of vertices where $V=\uplus_{i=1}^m a_i$. In such a case, we can simply let $\cN^*=\emptyset$ and $F_i=E(a_i,a_{i+1\text{ mod }m})$. 
\label{eg:cycle}\end{example}

\begin{example}\label{eg:laminar}
For a second example, suppose again $\eps=0$ and $\cN$ is the set of mincuts of a graph $G$ where $\cN$ forms a laminar family (no two cuts cross). It turns out that we cannot decompose edges of cuts of $\cN$ into a linear sized collection of sets where every edge appears only a constant number of times. The main reason is that some edges may appear in an unbounded number of cuts. In this case we let $\cN^*=\cN$ and for every $A\in \cN$ (with immediate parent $B\in \cN$ in the laminar family) we add a set $F_A=\delta(A)\smallsetminus \delta(B)$  to our collection.  It is straightforward to show, using the structure of min cuts, that $x(F_A)\geq 1$; furthermore, since the size of a laminar family is linear in $V$, this gives a valid decomposition in the sense of above theorem.
\end{example}
Lastly, if $\eps=0$ and $\cN$ is the set of min cuts of an arbitrary graph, one can represent all min cuts of $\cN$ by a cactus \cite{DKL76} which can be seen as a tree of cycles. In such a case, one can use a construction similar to \cref{eg:cycle} for each cycle where instead of a vertex $v_i$ we have a set $a_i \subseteq V$ and one similar to \cref{eg:laminar} for the tree part of the cactus. For a concrete application of such a decomposition of min cuts see \cite{KKO20}.
%More generally, if $\cN$ corresponds to the set of min cuts of an arbitrary graph, the cuts of $\cN$ can be represented by a {\em cactus graph}. In such a case we add one $F_i$ for every edge of a cycle of the cactus. 


%and further for simplicity assume that there is a single connected component of crossing cuts in $\cN$, namely we can go from any $A$ to $B$ for $A,B\in\cN$ simply following crossing cuts of $\cN$. Then, one can represent cuts in $\cN$ by the set of min cuts of a cycle, namely we can contract vertices of $G$ 

%For a concrete application , suppose we need at least two edges in every set in $\cN^*$, say in a network optimization problem. Then, if we make sure that we have at least one edge in each $F_i$, all typical cuts constraints, $\cN\smallsetminus \cN^*$,  are satisfied, so we  reduce the problem to cuts in $\cN^*$. 


One of the main challenges in dealing with near min cuts relative to min cuts is that if $x(\delta(A)),x(\delta(B))\leq 2+\eps$ then $x(\delta(A\cap B))\leq 2+2\eps$. Therefore, if $\eps=0$, then min cuts are closed under intersection, set difference and union, but this is no longer true when $\eps>0$. So, to employ the classical uncrossing machinery one should be very careful to "uncross" only a constant number of times (independent of $\eps$) to make sure that every cut remains within $2+O(\eps)$. This is the main reason that the polygon representation of near min cuts (see below) is more sophisticated, e.g., we can no longer argue $x(E(a_i, a_{i+1}))\approx 1$, see \cref{fig:nearmincutbadexample}.

Although we don't study it here, we believe it may be worthwhile to find generalizations of \cref{thm:uncrossing} which hold for any (weakly) supermodular function.% That could be helpful in many questions based on the network optimization framework of Jain \cite{Jai01}.

\begin{remark} 
 We do not explicitly prove \cref{thm:uncrossing} in this extended abstract, as it is not used to prove \cref{thm:main}. However it can be deduced from arguments in \cref{sec:twoside} and \cref{app:oneside}. 
%In \cref{sec:overview} we discuss the main ideas of the proof of \cref{thm:uncrossing}. Here, let us explain the main challenge: In principal one might try to simply extend the above decomposition for the case $\eps=0$. The main challenge is that if $x(\delta(A)),x(\delta(B))\leq 2+\eps$ then $x(\delta(A\cap B))\leq 2+2\eps$. Therefore, if $\eps=0$, then min cuts are closed under intersection, set difference and union, but this is no longer true when $\eps>0$. So, to employ the classical uncrossing machinery one should be very careful to "uncross" only a constant number of times (independent of $\eps$) to make sure that every cut remains within $2+O(\eps)$. This is the main reason that the polygon representation of near min cuts (see below) is more sophisticated, e.g., we can no longer argue $x(E(a_i, a_{i+1}))\approx 1$, see \cref{fig:nearmincutbadexample}.
\end{remark}





\paragraph{Extensions to the Polygon Representation} To obtain our uncrossing framework we prove new properties of the polygon representation.
Given a graph $G=(V,E)$, let $k$ be the edge-connectivity of $G$, i.e. the number of edges in a minimum cut of $G$. For $\eps>0$, consider the set of $(1+\eps)$-near minimum cuts of $G$: cuts $(S,\overline{S})$ where $|E(S,\overline{S})| < (1+\eps)k$.
Bencz\'ur \cite{Ben95} and Bencz\'ur, Goemans \cite{BG08} proved that if $\eps \le 1/5$ then the near minimum cuts of $G$ admit a {\em polygon representation}. Namely, every connected component $\cC$ of \hyperlink{tar:crossing}{crossing} $(1+\eps)$ near min cuts can be represented by the diagonals of a convex polygon. In this polygon, the vertices of $G$ are partitioned into sets called \textit{atoms}, and every atom is mapped to a cell of this polygon defined by the diagonals and the boundary of the polygon itself (see \cref{sec:polyrep} for more details). 

The polygon representation can be seen as a generalization of the well-known cactus representation \cite{DKL76} of minimum cuts where a cycle of the cactus is replaced by a convex polygon. Unlike a cycle, some vertices/atoms map to the interior of the polygon, which are called ``inside'' atoms. The inside atoms at first look like a mystery and one can ask many questions about them such as how many can exist and what structures they can exhibit.



 Here, we explain two lemmas we proved which might find further applications beyond TSP in the future. 
%Our results give new intuition and understanding about the structure of polygon representations. These guide our analysis of the integrality gap of the subtour LP.
 %For example, one of our new observations is a 
 First, we give a necessary condition for a cell of a polygon to contain an inside atom:
\begin{lemma}[Informal, see \cref{thm:halfplanes}]
	Consider a polygon $P$ for a connected component $\C$ of a family of $1+\eps$ near min cuts for $\eps \le 1/5$ (where representing diagonals correspond to cuts in $\C$). Any cell of $P$ that has an inside atom must have at least $\Omega(1/\eps)$ many sides. 
\end{lemma}
This can be seen as a generalization of \cite[Lem 22]{BG08} to the case in which the cell is allowed to be adjacent to vertices of the polygon $P$.

Now, we explain our second extension: it follows from the cactus representation of minimum cuts that for a graph $G$ and a min cut $S$ one can partition the set of all min cuts that cross $S$ into two groups ${\cal A}=\{A_1,\dots,A_k\}$ and ${\cal B}=\{B_1,\dots,B_l\}$ for some $k,l\geq 0$ such that $S\cap A_1\subseteq S\cap A_2 \subseteq \dots S\cap A_k$ and, similarly, $S\cap B_1\subseteq \dots\subseteq S\cap B_l$. We prove a generalization of this fact for near min cuts:
\begin{lemma}[Informal, see \cref{lem:crosschain}]
Consider the set of $1+\eps$ near min cuts of a graph $G$ for $\eps\leq 1/10$; for any such near min cut $S$, one can partition the $1+\eps$ near min cuts crossing $S$ into two groups ${\cal A}=\{A_1,\dots,A_k\}$ and ${\cal B}=\{B_1,\dots,B_l\}$ such that $S\cap A_1 \subseteq S\cap A_2\subseteq \dots \subseteq S\cap A_k$ and similarly for cuts in ${\cal B}$.
\end{lemma}

\subsection{Outline of rest of paper} After reviewing preliminaries in \cref{sec:prelims}, we give a high-level overview of our proof technique in \cref{sec:overview}. The main new technical contributions of this paper are in \cref{sec:polyrep} and  \cref{sec:twoside}. The remaining content of the paper essentially follows from ~\cite{KKO21}. %Therefore, the reader may want to skip \cref{sec:proof-of-main}. 






\section{Related Work}
\section{Related Work}
\label{relatedwork}
In this section, we discuss the most relevant work that apply formal methods in the context of Hadoop framework.
In~\cite{Performance2015} and~\cite{Petrinets2015}, the authors analyzed the behavior of MapReduce using Stochastic Petri Nets and Timed Colored Petri Nets, respectively. They modeled the mean delay in each time transition of the scheduled tasks as formulas, and the Hadoop jobs were simulated based on the used Petri Nets. The proposed approaches could evaluate the correctness of the system and analyze the performance of MapReduce. But, they lack several constraints about the scheduling of the jobs and cannot cover larger Hadoop clusters.
Su \textit{et al.}~\cite{Su2009} used the CSP language to formalize four key components in MapReduce. Specifically, they formally modeled the master, mapper, reducer and file system while considering the basic operations in Hadoop (\eg{} task state storing, error handling, progress controlling). However, none of the properties of the Hadoop framework is verified using the formalized components.
Xie \textit{et al.}~\cite{Xie2016} address the formal verification of the HDFS reading and writing operations using CSP and the PAT model checker.
For instance, they formally modeled the reading and writing operations for the HDFS based on the formalized components proposed in~\cite{Su2009}.
Moreover, they verified some of the HDFS properties including the deadlock-freeness, minimal distance scheme, mutual exclusion, write-once scheme and robustness. While this approach allows to detect unexpected traces generating errors and verify data consistency in the HDFS, a limitation of this work is that it only models the reading and writing operations for just one file system and requires to investigate the validity of these operations for distributed files as in HDFS.
In \cite{Towards-Reddy2013}, Reddy \emph{et al.} propose an approach to model Hadoop's system using the PAT model checker. They used CSP to model Hadoop software components including the ``NameNode", ``DataNode", task scheduler and cluster setup. They identified the benefits of some properties like data locality, deadlock-freeness and non-termination among others and proved the correctness of these properties. However, their proposed model is evaluated on a small workload, and none of the properties is verified to check the performance of the Hadoop scheduler.
%Although theorem provers are widely used to check the correctness and reliability of several distributed systems,
%To the best of our knowledge, we only find one theorem prover-based study that verifies the actual running code of MapReduce applications.
Kosuke \emph{et al.}~\cite{OnoCoq2011} used the proof assistant Coq to write an abstract computation model of MapReduce. They modeled the mapper and reducer as Coq functions and proved that the implementation of the mapper and reducer satisfies the specification of some applications such as WordCount~\cite{Wordcount2015}.
The authors present an abstracted model of MapReduce, where many details are not included
%hidden\Foutse{are you saying that they didn't provided all details?} 
such as task assignment or resources allocation. These issues can affect the performance of applications running on Hadoop.
\vspace{-10pt} 


\section{Quick Overview}
%%++++++++++++++++++++++++++++++++++++++++++++++++++++++++++++++++++++
\begin{figure}[h!]
 \centering
 {
  \includegraphics[width=\linewidth]{./TAIFlowchart.pdf}}
\caption{TAI Informal Overview: \textit{We have an architecture for
    how a TAI agent $\tau$ might operate. $\tau$ continuously comes up
    with goals based on its contract. If a goal is not achievable
    using $\tau$'s own resources, $\tau$ has to employ other agents in
    achieving this goal. To successfully do so $\tau$ would need to
    have one or more of $\mathbf{D_1} - \mathbf{D_6}$ attributes.  }}
 \label{fig: TAI_Flowchart}
\end{figure}
%%++++++++++++++++++++++++++++++++++++++++++++++++++++++++++++++++++++


We give a quick and informal overview of TAI. We have a set of
agents $a_1, \ldots, a_n$. Each agent has an associated (implicit or
explicit) contract that it should adhere to. Consider one particular
agent $\tau$. During the course of this agent's lifetime, the agent
comes up with goals to achieve so that its contract is not
violated. Some of these goals might require an agent to exercise some
or all of the six attributes $\mathbf{D_1} - \mathbf{D_6}$. We
formalize this using planning as shown in Figure ~\ref{fig:
  TAI_Flowchart}. As shown in the figure, if some goal is not
achievable on its own, $\tau$ can seek to recruit other agents by
leveraging their resources, beliefs, obligations etc.





%%% Local Variables:
%%% mode: latex
%%% TeX-master: "main"
%%% End:


\section{The Formal System}
% To make the above notions more concrete, we use a two-level
% formalism. The above scenarios have different AI agents interacting
% with each other in a concurrent manner in the real world. While Most
% AI formalizations deal with information processing, there are
% formalisms for modeling concurrent computation.  We need one formalism
% that interleaves both these different levels of abstraction. For
% modeling concurrent computation that goes across different agents, we
% use the actor calculus. The actor calculus has been used successfully
% to model and build concurrent computing systems in the real world. For
% the second level of information processing, we use a computational
% logic. 


%%++++++++++++++++++++++++++++++++++++++++++++++++++++++++++++++++++++
\begin{figure}[h!]
 \centering
 {
  \includegraphics[width=\linewidth]{./star_fig.pdf}}
 \caption{Space of Logical Calculi.  \textit{There are five dimensions
     that cover the entire, vast space of logical calculi.  The due
     West dimension holds those calculi powering the Semantic Web
     (which are generally short of first-order logic =
     $\mathcal{L}_1$), and include so-called \textbf{description
       logics}.  Both NW and NE include logical systems with wffs that
     are allowed to be infinitely long, and are needless to say hard
     to compute with and over.  SE is higher-order logic, which has
     a robust automated theorem-proving community gathered around it.
     It's the SW direction that holds the cognitive calculi described
     in the present paper, and associated with TAI; and the star refers
     to those specific cognitive calculi called out in these pages by
     us.}}
 \label{fig:star_fig}
\end{figure}
%%++++++++++++++++++++++++++++++++++++++++++++++++++++++++++++++++++++


To make the above notions more concrete, we use a version of a
computational logic.  The logic we use is \textbf{deontic cognitive
  event calculus} (\DCEC).  This calculus is a first-order modal
logic. Figure ~\ref{fig:star_fig} shows the region where \DCEC\ is
located in the overall space of logical calculi. \DCEC\ belongs to the
\textbf{cognitive calculi} family of logical calculi (denoted by a
star in Figure~\ref{fig:star_fig} and expanded in
Figure~\ref{fig:cc_family}). \DCEC\ has a well-defined syntax and
inference system; see Appendix A of \cite{nsg_sb_dde_2017} for a full
description. The inference system is based on natural deduction
\cite{gentzen_investigations_into_logical_deduction}, and includes all
the introduction and elimination rules for first-order logic, as well
as inference schemata for the modal operators and related structures

This system has been used previously in
\cite{nsg_sb_dde_2017,dde_self_sacrifice_2017} to automate versions of
the doctrine of double effect \DDE, an ethical principle with
deontological and consequentialist components.  While describing the
calculus is beyond the scope of this paper, we give a quick overview
of the system below.  Dialects of \DCEC\ have also been used to
formalize and automate highly intensional (i.e. cognitive) reasoning
processes, such as the false-belief task
\cite{ArkoudasAndBringsjord2008Pricai} and \textit{akrasia}
(succumbing to temptation to violate moral principles)
\cite{akratic_robots_ieee_n}. {Arkoudas and Bringsjord
  \shortcite{ArkoudasAndBringsjord2008Pricai} introduced the general
  family of \textbf{cognitive event calculi} to which \DCEC\ belongs,
  by way of their formalization of the false-belief task.} More
precisely, \DCEC\ is a sorted (i.e.\ typed) quantified modal logic
(also known as sorted first-order modal logic) that includes the event
calculus, a first-order calculus used for commonsense reasoning.

 
%%++++++++++++++++++++++++++++++++++++++++++++++++++++++++++++++++++++
\begin{figure}[h!]
 \centering
 {
  \includegraphics[width=0.75\linewidth]{./CCFamily.pdf}}
\caption{Cognitive Calculi. \textit{The \textbf{cognitive calculi}
    family is composed of a number of related
    calculi. \protect\citeauthor{ArkoudasAndBringsjord2008Pricai} introduced
    the first member in this family, $\mathcal{CEC}$, to model the
    false-belief task. The smallest member in this family,
    $\mu\mathcal{C}$, has been used to model uncertainty in quantified
    beliefs \protect\cite{govindarajulu2017strength}. \DCEC\ and variants have
    been used in the modelling of ethical principles and theories and
    their implementations.}}
 \label{fig:cc_family}
\end{figure}
%%++++++++++++++++++++++++++++++++++++++++++++++++++++++++++++++++++++



\subsection{Syntax}
\label{subsect:syntax}

As mentioned above, \DCEC\ is a sorted calculus.  A sorted system can
be regarded as analogous to a typed single-inheritance programming
language.  We show below some of the important sorts used in \DCEC.\\
 
\begin{footnotesize}
\rowcolors{2}{gray!10}{white}
\def\arraystretch{1.25}

\begin{tabular}{lp{5.8cm}}  
\toprule
\textbf{Sort}    & \textbf{Description} \\
\midrule
\type{Agent} & Human and non-human actors.  \\

\type{Time} &  The \type{Time} type stands for
time in the domain.  E.g.\ simple, such as $t_i$, or complex, such as
$birthday(son(jack))$. \\

 \type{Event} & Used for events in the domain. \\
 \type{ActionType} & Action types are abstract actions.  They are
  instantiated at particular times by actors.  Example: eating.\\
 \type{Action} & A subtype of \type{Event} for events that occur
  as actions by agents. \\
 \type{Fluent} & Used for representing states of the world in the
  event calculus. \\
\bottomrule
\end{tabular}
\end{footnotesize} \\

The syntax has two components: a first-order
core and a modal system that builds upon this first-order core.  The
figures below show the syntax and inference schemata of \DCEC.    The first-order core of \DCEC\ is
the \emph{event calculus} \cite{mueller_commonsense_reasoning}.
Commonly used function and relation symbols of the event calculus are
included.  Fluents, event and times are the three major sorts of the event
calculus. Fluents represent states of the world as first-order
terms. Events are things that happen in the world at specific instants
of time. Actions are events that are carried out by an agent. For any
action type $\alpha$ and agent $a$, the event corresponding to $a$
carrying out $\alpha$ is given by $action(a, \alpha)$. For instance
if $\alpha$ is \textit{``running''} and $a$ is \textit{``Jack'' },
$action(a, \alpha)$ denotes \textit{``Jack is running''}.
Other calculi (e.g.\ the \emph{situation calculus}) for
modeling commonsense and physical reasoning can be easily switched out
in-place of the event calculus.

 \begin{scriptsize}
\begin{mdframed}[linecolor=white, frametitle=Syntax,
  frametitlebackgroundcolor=gray!10, backgroundcolor=gray!05,
  roundcorner=8pt]
 \begin{equation*}
 \begin{aligned} 
    \mathit{S} &::= 
    \begin{aligned}
      & \Agent \sep \ActionType \sep \Action \sqsubseteq
      \Event \sep \Moment  \sep \Fluent \\
    \end{aligned} 
    \\ 
    \mathit{f} &::= \left\{
    \begin{aligned}
      & action: \Agent \times \ActionType \rightarrow \Action \\
      &  \initially: \Fluent \rightarrow \Boolean\\
      &  \holds: \Fluent \times \Moment \rightarrow \Boolean \\
      & \happens: \Event \times \Moment \rightarrow \Boolean \\
      & \clipped: \Moment \times \Fluent \times \Moment \rightarrow \Boolean \\
      & \initiates: \Event \times \Fluent \times \Moment \rightarrow \Boolean\\
      & \terminates: \Event \times \Fluent \times \Moment \rightarrow \Boolean \\
      & \prior: \Moment \times \Moment \rightarrow \Boolean\\
    \end{aligned}\right.\\
        \mathit{t} &::=
    \begin{aligned}
      \mathit{x : S} \sep \mathit{c : S} \sep f(t_1,\ldots,t_n)
    \end{aligned}
    \\ 
    \mathit{\phi}&::= \left\{ 
    \begin{aligned}
     & q:\Boolean \sep  \neg \phi \sep \phi \land \psi \sep \phi \lor
     \psi \sep \forall x: \phi(x) \sep \\\
 &\perceives (a,t,\phi)  \sep \knows(a,t,\phi) \sep     \\ 
& \common(t,\phi) \sep
 \says(a,b,t,\phi) 
     \sep \says(a,t,\phi) \sep  \believes(a,t,\phi) \\
& \desires(a,t,\phi)  \sep \intends(a,t,\phi) \\ & \ought(a,t,\phi,(\lnot)\happens(action(a^\ast,\alpha),t'))
      \end{aligned}\right.
  \end{aligned}
\end{equation*}
\end{mdframed}
\end{scriptsize}


The modal operators present in the calculus include the standard
operators for knowledge $\knows$, belief $\believes$, desire
$\desires$, intention $\intends$, etc.  The general format of an
intensional operator is $\knows\left(a, t, \phi\right)$, which says
that agent $a$ knows at time $t$ the proposition $\phi$.  Here $\phi$
can in turn be any arbitrary formula. Also,
note the following modal operators: $\mathbf{P}$ for perceiving a
state, 
$\mathbf{C}$ for common knowledge, $\mathbf{S}$ for agent-to-agent
communication and public announcements, $\mathbf{B}$ for belief,
$\mathbf{D}$ for desire, $\mathbf{I}$ for intention, and finally and
crucially, a dyadic deontic operator $\mathbf{O}$ that states when an
action is obligatory or forbidden for agents. It should be noted that
\DCEC\ is one specimen in a \emph{family} of extensible
cognitive calculi.
 
The calculus also includes a dyadic (arity = 2) deontic operator
$\ought$. It is well known that the unary ought in standard deontic
logic leads to contradictions.  Our dyadic version of the operator
blocks the standard list of such contradictions, and
beyond.\footnote{A overview of this list is given lucidly in
  \cite{sep_deontic_logic}.}

Declarative communication of $\phi$ between $a$ and $b$ at time $t$ is
represented using the $\says(a,b,t, \phi)$.


\subsection{Inference Schemata}


The figure below shows a fragment of the inference schemata for \DCEC.
First-order natural deduction introduction and elimination rules are
not shown. Inference schemata $I_\mathbf{K}$ and $I_\mathbf{B}$ let us
model idealized systems that have their knowledge and beliefs closed
under the \DCEC\ proof theory.  While humans are not deductively
closed, these two rules lets us model more closely how more deliberate
agents such as organizations, nations and more strategic actors
reason. (Some dialects of cognitive calculi restrict the number of
iterations on intensional
operators.) % $I_1$ and $I_2$ state respectively that it is common
% knowledge that perception leads to knowledge, and that it is common
% knowledge that knowledge leads to belief.  $I_3$ lets us expand out
% common knowledge as unbounded iterated knowledge.  $I_4$ states that
% knowledge of a proposition implies that the proposition holds.  $I_5$
% to $I_{10}$ provide for a more restricted form of reasoning for
% propositions that are common knowledge, unlike propositions that are
% known or believed.
$I_{13}$ ties intentions directly to perceptions
(This model does not take into account agents that could fail to carry
out their intentions).  $I_{14}$ dictates how obligations get
translated into known intentions.


\begin{scriptsize}


\begin{mdframed}[linecolor=white, frametitle=Inference Schemata
  (Fragment), nobreak=true, frametitlebackgroundcolor=gray!10, backgroundcolor=gray!05, roundcorner=8pt]
\begin{equation*}
\begin{aligned}
% &\mbox{Sample rules below. For more rules, see
%   \cite{akratic_robots_ieee_n}.}\\
  &\hspace{40pt} \infer[{[I_{\knows}]}]{\knows(a,t_2,\phi)}{\knows(a,t_1,\Gamma), \ 
    \ \Gamma\vdash\phi, \ \ t_1 \leq t_2}  \\ 
& \hspace{40pt} \infer[{[I_{\believes}]}]{\believes(a,t_2,\phi)}{\believes(a,t_1,\Gamma), \ 
    \ \Gamma\vdash\phi, \ \ t_1 \leq t_2} \\
%  &\infer[{[I_1]}]{\common(t,\perceives(a,t,\phi)
%    \lif\knows(a,t,\phi))}{}\\ 
% &\infer[{[I_2]}]{\common(t,\knows(a,t,\phi)
%     \lif\believes(a,t,\phi))}{}\\
 %   &\infer[{[I_3]}]{\knows(a_1, t_1, \ldots
 %     \knows(a_n,t_n,\phi)\ldots)}{\common(t,\phi) \ t\leq t_1 \ldots t\leq
 %     t_n}\hspace{5pt}
 % \infer[{[I_4]}]{\phi}{\knows(a,t,\phi)}\\
%   & \infer[{[I_5]}]{\common(t,\knows(a,t_1,\phi_1\lif\phi_2))
%     \lif \knows(a,t_2,\phi_1) \lif \knows(a,t_3,\phi_2)}{}\\
% & \infer[{[I_6]}]{\common(t,\believes(a,t_1,\phi_1\lif\phi_2))
%     \lif \believes(a,t_2,\phi_1) \lif \believes(a,t_3,\phi_2)}{}\\
% & \infer[{[I_7]}]{\common(t,\common(t_1,\phi_1\lif\phi_2))
%     \lif \common(t_2,\phi_1) \lif \common(t_3,\phi_2)}{} \\
% & \infer[{[I_8]}]{\common(t, \forall x. \  \phi \lif \phi[x\mapsto
%   t])}{} \hspace{6pt}
%   \infer[{[I_9]}]{\common(t,\phi_1 \liff \phi_2 \lif \neg
%     \phi_2 \lif \neg \phi_1)}{}\\
% & \infer[{[I_{10}]}] {\common(t,[\phi_1\land\ldots\land\phi_n\lif\phi]
%   \lif [\phi_1\lif\ldots\lif\phi_n\lif\psi])}{}\\
% & \infer[{[I_{11a}]}]{\believes(a,t,\psi)}{\believes(a,t,\phi)\ \ \phi
%   \lif \psi}\
% \hspace{6pt} \infer[{[I_{11b}]}]{\believes(a,t,\psi \land \phi)}{\believes(a,t,\phi)\ \ \believes(a,t,\psi)}\\ 
& \hspace{20pt} \infer[{[I_4]}]{\phi}{\knows(a,t,\phi)}
\hspace{18pt}\infer[{[I_{13}]}]{\perceives(a,t', \psi)}{t<t', \ \ \intends(a,t,\psi)}\\
&\infer[{[I_{14}]}]{\knows(a,t,\intends(a,t,\chi))}{\begin{aligned}\ \ \ \ \believes(a,t,\phi)
 & \ \ \
 \believes(a,t,\ought(a,t,\phi, \chi)) \ \ \ \ought(a,t,\phi,
 \chi)\end{aligned}}
\end{aligned}
\end{equation*}
\end{mdframed}
\end{scriptsize}



 \subsection{Semantics}

 The semantics for the first-order fragment is the standard
 first-order semantics. The truth-functional connectives
 $\land, \lor, \rightarrow, \lnot$ and quantifiers $\forall, \exists$
 for pure first-order formulae all have the standard first-order
 semantics. The semantics of the modal operators differs from what is
 available in the so-called Belief-Desire-Intention (BDI) logics
 {\cite{bdi_krr_1999}} in many important ways.  For example, \DCEC\
 explicitly rejects possible-worlds semantics and model-based
 reasoning, instead opting for a \textit{proof-theoretic} semantics
 and the associated type of reasoning commonly referred to as
 \textit{natural deduction}
 \cite{gentzen_investigations_into_logical_deduction,proof-theoretic_semantics_for_nat_lang}.
 Briefly, in this approach, meanings of modal operators are defined
 via arbitrary computations over proofs. 



%%% Local Variables:
%%% mode: latex
%%% TeX-master: "main"
%%% End:



\section{Defining TAI}
%%++++++++++++++++++++++++++++++++++++++++++++++++++++++++++++++++++++
\begin{figure}[h!]
 \centering
 {
  \includegraphics[width=0.75\linewidth]{./TAITime.pdf}}
\caption{TAI Working Through Time. \textit{A TAI agent initially
    considers a goal and then has to produce a proof for the
    non-existence of a non-tentacular plan that uses only this
    agent. Then $\tau$ recruits a set of other relevant agents to help
    with its goal.}}
 \label{fig: TAI_Architecture}
\end{figure}
%%++++++++++++++++++++++++++++++++++++++++++++++++++++++++++++++++++++

We denote the state-of-affairs at any time $t$ by a set of formulae
$\Gamma(t)$. This set of formulae will also contain any obligations
and prohibitions on different agents. For each agent $a_i$ at time
$t$, there is a contract $\mathbf{c}(a_i, t)\subseteq \Gamma(t)$ that
describes $a_i$'s obligations, prohibitions etc.  $a$ at any time $t$
then comes up with a goal $g$ so that its
contract is satisfied.\footnote{See \cite{nsg_sb_dde_2017} for an
  example of how obligations and prohibitions can be used in \DCEC.}
The agent believes that if  $g$ does not hold then its contract at some future
$t+\delta$  will be violated:

 $$\mathbf{B}\left(a, t, \lnot
 g \rightarrow  \lnot \bigwedge \mathbf{c}(a, t+\delta)\right)$$ Then the agent tries
 to come up with a plan involving a sequence of actions to satisfy the
 goal.


We make these notions more precise. An agent $a$ has a set of
actions that it \emph{can} perform at different time points. For instance, a
vacuuming agent can have movement along a plane as its possible
actions while an agent on a phone can have displaying a notification
as an action. We denote this by $can(a,\alpha, t)$ with the following
additional axiom:


 

$$ \colorbox{gray!10}{Axiom}  \lnot can(a,\alpha,
t) \rightarrow \lnot \happens(action(a, \alpha), t) $$

 
\noindent We now define a \emph{consistent plan} below:

 \begin{footnotesize}
\begin{mdframed}[linecolor=white, frametitle=Consistent Plan,
  frametitlebackgroundcolor=gray!10, backgroundcolor=gray!05,
  roundcorner=8pt]

  A \emph{consistent plan} $\rho_{\langle a_1, \ldots, a_n\rangle} $ at
  time $t$ is a sequence of agents $a_1,\ldots, a_n$ with
  corresponding actions $\alpha_1, \ldots, \alpha_n$ and times
  $t_1, \ldots, t_n$ such that
  $\Gamma \vdash (t<t_i < t_j) \mbox{ for } i<j$ and for all agents $a_i$
  we have:
\begin{enumerate} 
\item $can(a_i, \alpha_i, t_i)$
\item $happens(action(a_i,\alpha_i))$ is consistent with $\Gamma(t)$. 
\end{enumerate}
 \end{mdframed}
 \end{footnotesize}
 Note that a consistent plan $\rho_{\langle \ldots \rangle}$ can be
 represented by a term in our language. We introduce a new sort
 $\mathsf{Plan}$ and a variable-arity predicate symbol
 $\plan(\rho, a_1, \ldots, a_n)$ which says that $\rho$ is a plan
 involving $a_1\ldots, a_n$.


 A goal is also any formula $g$.  A consistent plan satisfies a goal
 $g$ if:

\begin{equation*}
 \left(\begin{aligned}
\Gamma(t) \cup \left\{ \begin{aligned}
   & happens(action(a_1,\alpha_1), t_1), \ldots,\\
 &happens(action(a_n,\alpha_n), t_n) \end{aligned}\right\} 
 \end{aligned}\right) \vdash g
 \end{equation*}
 We use $\Gamma\vdash (\rho \rightarrow g)$ as a shorthand for the
 above. The above definitions of plans and goals give us the following
 important constraint needed for defining TAI. This differentiates our
 planning formalism from other planning systems and makes it more
 appropriate for an architecture for a general-purpose tentacular AI
 system.


 \begin{footnotesize}
\begin{mdframed}[linecolor=white, frametitle=Uniform Planning Constraint,
  frametitlebackgroundcolor=yellow!25, backgroundcolor=yellow!10,
  roundcorner=8pt]
  Plans and goals should be represented and reasoned over in the
  language of the planning system.
\end{mdframed}
 \end{footnotesize}


 Leveraging the above requirement, we can define two levels of TAI
 agents. A $\level{1}{*}$ TAI system corresponding to an agent $\tau$
 is a system that comes up with goal $g$ at time $t'$ to satisfy its
 contract, produces a proof that there is no consistent plan that
 involves only the agent $\tau$. Then $\tau$ comes with a plan that
 involves one or more other agents.  A $\level{1}{*}$ TAI agent starts
 with knowledge about what actions are possible for other agents.



 \begin{footnotesize}
   \begin{mdframed}[linecolor=white, frametitle=$\level{1}{*}$ TAI Agents ,
     frametitlebackgroundcolor=gray!10, backgroundcolor=gray!05,
     roundcorner=8pt]
     \begin{enumerate}
       \item[\textbf{Prerequisite}] For any $a$, $\alpha$, $t$, we have:
         \begin{equation*}
           \begin{aligned}
             \Gamma \vdash
             & can(a, \alpha, t) \rightarrow \Knows\big(\tau, t',
             can(a, \alpha, t) \big) \end{aligned}
         \end{equation*}
\item[\textbf{Then}]
       \item $\tau$ produces a proof that no plan exists for $g$
         involving just itself and $\tau$ declares that there is no
         such plan.
         \begin{equation*}
           \begin{aligned}
             \Gamma \vdash
             & \says\big(\tau, t', \lnot \exists  \rho:\left( \plan(\rho,
             \tau) \land \rho \rightarrow g\right)\big) \end{aligned}
         \end{equation*}
       \item $\tau$ produces a plan for $g$ involving just itself and
         one or more agents and declares that plan.
         \begin{equation*}
           \begin{aligned}
             \Gamma \vdash
             & \says\Bigg(\tau, t', \Big( \plan(\rho
            , a_1, \ldots, \tau \ldots a_n) \land \rho \rightarrow g\Big)\Bigg) 
         \end{aligned}
         \end{equation*}
       \end{enumerate}
     \end{mdframed}
   \end{footnotesize}

   The agent may not always have knowledge about what other agents can
   do. The TAI agent may have imperfect knowledge about other
   agents. The agent can gain information about other agents' actions,
   their obligations, prohibitions, etc. by observing them or by reading
   specifications governing these agents. In this case, we get a
   $\level{2}{*}$ TAI agent. We need to modify only the prerequisite
   condition above.


 \begin{footnotesize}
   \begin{mdframed}[linecolor=white, frametitle=$\level{2}{*}$ TAI Agents ,
     frametitlebackgroundcolor=gray!10, backgroundcolor=gray!05,
     roundcorner=8pt]
     \begin{enumerate}
     \item[\textbf{Prerequisite}]  For any $a$, $\alpha$, $t$, we have:
         \begin{equation*}
           \begin{aligned}
             \Gamma \vdash
             & can(a, \alpha, t) \rightarrow \Believes\big(\tau, t',
             can(a, \alpha, t) \big) \end{aligned}
         \end{equation*}
      
       \end{enumerate}
     \end{mdframed}
   \end{footnotesize}
 
   The TAI agents above can be considered \textbf{first-order}
   tentacular agents. We can also have a \textbf{higher-order} TAI
   agent that intentionally engages in actions that trigger one or
   more other agents to act in tentacular fashion as described above.
   The need for having the uniform planning constraint is more clear
   when we consider higher-order agents.







%%% Local Variables:
%%% mode: latex
%%% TeX-master: "main.tex"
%%% End:


\section{A Hierarchy of TAI Agents}
The TAI formalization above gives rise to multiple hierarchies of
tentacular agents.  We discuss some of the these below.



\begin{small}
\begin{enumerate}
\item[\textbf{Syntactic Goal Complexity}] The goal $g$ can range in
  complexity from simple propositional statements,
  e.g. $clean{Kitchen}$, to first-order statements.  e.g.
  $\forall r:\mathsf{Room}: clean(r)$, and to intensional statements
  representing cognitive states of other agents
  $$\believes(a, now, \believes(b, now,\forall r: clean(r)))$$

  %%TODO: Make more precise

\item[\textbf{Goal Variation}] According to the definition
  above, an agent $a$ qualifies as being tentacular if it plans for
  just one goal $g$ in tentacular fashion as laid out in the
  conditions above. We could have agents that plan for a number of
  varied and different goals in tentacular fashion. 

  %%TODO: Make more precise

\item[\textbf{Plan Complexity}] For many goals, there will usually be
  multiple plans involving different actions (with different costs and
  resources used) and executed by different agents. 

  %%TODO: Make more precise



\end{enumerate}
\end{small}


%%++++++++++++++++++++++++++++++++++++++++++++++++++++++++++++++++++++
\begin{figure}[h!]
 \centering
 {
  \includegraphics[width=\linewidth]{./TAIagents.pdf}}
 \caption{Pictorial Overview.  \textit{A bit of explanation: That some
     agents are within agents indicates that the outer agent knows
     and/or believes everything relevant about the inner agent; hence
     as agents are increasingly cognitively powerful, the depth of
     their epistemic attitudes grows (reflected in formulae with
     iterated belief/knowledge operators).  Agents grow in
     size/intelligence in lockstep with the logical calculi upon which
     they are based increasing in expressivity and reasoning power;
     $\mathcal{L}_0$ is zero-order logic, $\mathcal{L}_1$ is
     e.g.\ first-order logic, and the particular cognitive calculus
     $\mathcal{DCEC}$ is shown.  Rotation indicates simply that,
     through time, agents perceive and act.}}
 \label{fig:pictorial_overview}
\end{figure}
%%++++++++++++++++++++++++++++++++++++++++++++++++++++++++++++++++++++

%%% Local Variables:
%%% mode: latex
%%% TeX-master: "main"
%%% End:



\section{Examples and Embryonic Implementation}
In this section, we present a formal sketch of a TAI agent and then
describe using another example ongoing work in implementing a TAI
system.

\subsection{Example } Consider the example given in the beginning.  We
have a human $j$ and three artificial agents: $a_{c}$ in the car,
$a_h$ in the home and $a_p$ an agent managing scheduling and calendar
information.  We present some of the formulae in $\Gamma$.


\begin{footnotesize}
\begin{equation*}
\begin{aligned}
& \mathbf{B}(a_c, t_0, crowded(store) \rightarrow unusal), \mbox{\colorbox{gray!20}{$\mathsf{f_1}$}}\\
&\mathbf{P}(a_c, t_1, crowded(store)),\mbox{\colorbox{gray!20}{$\mathsf{f_2}$}}\\
&\forall t: \mathbf{O}\left(\begin{aligned}a_c, t, & unusal,\\ &
    happens\big(action(a_c, check(weather)),
    t+1\big) \end{aligned}\right)\mbox{\colorbox{gray!20}{$\mathsf{f_3}$}}\\
&  \forall t: \mathbf{B}\left(a_c, t, \mbox{\colorbox{gray!20}{$\mathsf{f_3}$}}\right)\\
& \forall a: \left(\begin{aligned} happens\Big(& action(a, check(weather)), t_3\Big) 
\\ & \rightarrow\mathbf{K}(a, t_4, storm), \end{aligned}\right)\mbox{\colorbox{gray!20}{$\mathsf{f_4}$}}\\
&\forall t: \mathbf{O}\big(a_c, t, storm, \mathsf{S}(a_c, a_h, storm,
t+1))\big),\mbox{\colorbox{gray!20}{$\mathsf{f_5}$}}\\
&  \forall t: \mathbf{B}\left(a_c, t, \mbox{\colorbox{gray!20}{$\mathsf{f_5}$}}\right)\\
 \end{aligned} 
\end{equation*}
\end{footnotesize}
The above formulae first state the fact that $a_c$ observes the store
being crowded. $a_c$'s contract states that the agent should check a
weather service if it finds something unusual.  The formulae also
states that if an agent checks the weather at $t_3$, the agent will
get a prediction about an incoming storm.  $a_c$'s contract places an
obligation on it to inform $a_h$ if it believes that a storm is
incoming.
 
\begin{footnotesize}
\begin{equation*}
\begin{aligned}
&\forall t: \mathbf{O}\big(a_h, t, storm, \forall s: quantity(s) > 0\big),\mbox{\colorbox{gray!20}{$\mathsf{f_6}$}}\\
& \mathbf{K}\left(\begin{aligned} a_h, t_5, & shops(j, today) \lor shops(j,
tomorrow) \\ &  \rightarrow \forall s: quantity(s) > 0\end{aligned} \right), \mbox{\colorbox{gray!20}{$\mathsf{f_7}$}}\\
& \forall t: \mathbf{B}\left(\begin{aligned}a_h, t, & happens\big(action(a_c, recc(shops(j))), t\big)\\ & \rightarrow shops(j)\end{aligned}\right) \mbox{\colorbox{gray!20}{$\mathsf{f_8}$}}\\
& \forall t: \mathbf{B}\left(\begin{aligned} & a_h, t, happens\Big(action\big(a_h, req(a_c, shops(j))\big), t\Big)\\ & \rightarrow happens\big(action(a_c, recc(shops(j))), t\big)\end{aligned}\right) \mbox{\colorbox{gray!20}{$\mathsf{f_9}$}}\\
 \end{aligned} 
\end{equation*}
\end{footnotesize}

The first formula above states that $a_h$ ought to see to it that
supplies are stocked in the event of a storm. Then we have that 
$a_h$ knows that the human $j$ shopping today or tomorrow can result
in the supplies being stocked. $a_h$ gets information from $a_p$ that
shopping tomorrow is not possible (this formula is not shown). Then we
have formulae stating the effects of $a_c$ recommending the shopping
action to $j$.  The goal for $a_h$ is $\forall s: quantity(s)>0$ and a
plan for it is built up using $a_h$, $a_c$ and $j$.



%%% Local Variables:
%%% mode: latex
%%% TeX-master: "main"
%%% End:


\subsection{Experimental setup}

All the simulations were performed on a computer with an AMD Ryzen 7 2700 8-core processor and 16GB RAM, running Ubuntu 18.04. 
%\footnote{Videos for the simulations and demonstrations can be viewed at \url{https://www.youtube.com/channel/UCs-5zrF1oNrgCfntydgA39g/playlists}}. 
The MILP formulation was implemented in MATLAB using Yalmip \cite{lofberg2004yalmip} with MOSEK v8 as the solver. The learning-based approach was implemented in Python 3 with Tensorflow 1.14 and Keras API and Casadi with QPOASES as the solver. 
We implemented the CA-MPC using CVXGEN for a measurement of 
computation times and real-time implementation for experiments of actual hardware.

%\begin{figure}[tb]
%	\begin{center}
%		\includegraphics[width=0.42\textwidth,trim={0 0.2cm 0 3.5cm},clip]{figures/scenario1_w_ca_traj_plot.png}
%	\end{center}
%	\vspace{-10pt}
%	{\footnotesize
%		\caption{\small Trajectories for 2 UAS. The dotted (planned) trajectories have a collision at the halfway point. The solid ones, generated through L2F method, avoid the collision while remaining within the robustness tube of the original trajectories. Playback of scenario at: \url{https://tinyurl.com/sk5kyvl}}
%		\label{fig:scen1_w_ca}}
%\end{figure} 

%\begin{figure}[tb]
%	\begin{center}
%		\includegraphics[width=0.42\textwidth]{figures/non_parabola2.png}
%	\end{center}
%	\vspace{-10pt}
%	{\footnotesize
%		\caption{\small Trajectories for 2 UAS from different angles. The dashed (planned) trajectories have a collision at the halfway point. The solid ones, generated through L2F method, avoid the collision while remaining within the robustness tube of the original trajectories. Initial UAS positions marked as stars. Playback of scenario at: \url{https://tinyurl.com/y8cm65ya}}
%		\label{fig:scen1_w_ca}\vspace{-10pt}}
%\end{figure} 

For the experiments, we set minimum separation to $\delta = 0.1$m. 
The learning-based CR scheme was trained for $\rho = 0.055$ which is close to the lower bound in assumption \ref{assumption1}.

We have generated the data set of 14K training and 10K test conflicting trajectories using the minimum-jerk trajectory generation algorithm from~\cite{pant2018fly}.
The time horizon was set to $T=4$s and $dt=0.1$s. 
%Thus, each trajectory consists of $N+1=41$ time-steps. 
The initial and final waypoints were sampled uniformly at random from two 3D cubes close to the fixed collision point, initial velocities were set to zero.
%An example trajectories are presented in Fig.~\ref{fig:scen1_w_ca}.

%We have generated the data set of $27K$ conflicting trajectories $\mathbf{x}_1$, $\mathbf{x}_2$ with initial and final waypoints uniformly sampled from two 3D cubes close to the fixed collision point $(1,1,1)$. Duration was
%By solving the centralized MILP problem~\ref{eq:CentralMILP} for such conflicting trajectories, the sequence of binary decision variables $\mathbf{b}\in\{0,1\}^{6\times 41}$, and therefore, the deconfliction sequence $\mathbf{d}\in\{1,\ldots,6\}^{41}$ was obtained.
%For each pair of such trajectories the sequence $\mathbf{z}$ was defined using~\eqref{eq:noconf}.

We have trained and ran experiments for various network 
configurations. 
For each model, the number of training epochs was set to 2K with a batch size of 2K. Each network was trained to minimize categorical cross-entropy loss using Adam optimizer with training rate of $0.001$.
%and moment exponential decay rates of
%$\beta_1= 0.9$ and $\beta_2=0.999$.
The model with 3 LSTM layers with 128 neurons each has 
been chosen as the default learning-based CR model.


\section{Conclusion \& Future Work}
\label{sect:conclusion}
\section{Conclusions}
In this paper, we set out to address the problem of multi-tasking robots in multi-robot tasks. 
%A fundamental limitation of existing multi-robot systems was addressed by the removal of a restrictive assumption that was often made--robots are single-tasking.
%Our method allowed coalitions to overlap thus enabling multi-tasking robots. 
We observed that the key underlying challenge was to reason about the physical constraints that could be synergistically satisfied.
%which directly affected the feasibility of multi-tasking.
To address the challenge, we developed our method based on the information invariant theory and modeled constraints as information instances. 
%This allowed us to reason about the relationships between constraints by reasoning about those between information requirements. 
Thereby, a formal and general framework to achieve multi-tasking robots was developed. 
We showed that our algorithm was sound and complete under our problem settings. 
%Our method was integrated with a simple greedy heuristic for task allocation.
Simulation  results  were  provided  to  show  the  effectiveness  of  our approach under resource-constrained situations and in handling challenging situations. % in a multi-UAV simulator. 

% The idea of multi-tasking is attractive in many ways. 
% Humans are living in multi-tasking environments--at any point of time, 
% we are optimizing for more than one task. 
% Multi-task often leads to more efficient task performance since it allows us to exploit task synergies. 
% The work presented in this paper takes us one step forward in realizing multi-tasking robots. 
% In particular, we started looking at the feasibility of multi-tasking. 
% There are many potential directions to pursue along this direction. First, several limitations are present in the current approach. 
% For example, although our method guarantees that there exists a physical configuration that satisfies all the constraints, it does not explicitly take the environmental influence into account. For example, a narrow corridor may prevent a robot formation from passing through, even though all the constraints for the formation do not introduce any conflicts. In this sense, our work should better be characterized as establishing a necessary condition for multi-tasking. Also, our method is mainly focused on the ``{\it planning}'' phase and hence does not address how the robots reach the desired configuration and maintain the constraints. These issues are assumed to be handled by the execution layer.

% More generally, the question of how to execute the tasks with overlapping coalitions is not addressed in this work. 
% As we already discussed, executing individual tasks with non-overlapping coalitions is straightforward but task synergies impose additional requirements on the task execution: how should the robots that are assigned multiple tasks execute them? Should they consider them in a prioritized strategy~\cite{van2005prioritized}? Or should they combine the different tasks in a way that is similar to motor schemas~\cite{arkin2}. 
% Communication requirements for maintaining the constraints must also be taken into account. How should the robots optimize their communication to improve the task performance? 

% The stringency of the physical constraints is another interesting question. It may be desirable to relax the constraints in certain situations (e.g., due to environmental influences). In such cases, it may be important to consider the problem where the constraints are least violated~\cite{kim2012revision}, or specify task constraints in different ways to increase the diversity of the configurations~\cite{srivastava2007domain} so as to make it robust to different environments. 

\section{Acknowledgments}
\label{sect:ack}

The TAI project is made possible by joint support from RPI and IBM
under the AIRC Program; we are grateful for this support.  Some of the
research reported on herein has been enabled by support from ONR and
AFOSR, and for this too we are grateful.


 \bibliographystyle{named}
\bibliography{main72,naveen,kartik,atriya}

\end{document}
