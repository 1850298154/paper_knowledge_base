%%++++++++++++++++++++++++++++++++++++++++++++++++++++++++++++++++++++
\begin{figure}[h!]
 \centering
 {
  \includegraphics[width=0.75\linewidth]{./TAITime.pdf}}
\caption{TAI Working Through Time. \textit{A TAI agent initially
    considers a goal and then has to produce a proof for the
    non-existence of a non-tentacular plan that uses only this
    agent. Then $\tau$ recruits a set of other relevant agents to help
    with its goal.}}
 \label{fig: TAI_Architecture}
\end{figure}
%%++++++++++++++++++++++++++++++++++++++++++++++++++++++++++++++++++++

We denote the state-of-affairs at any time $t$ by a set of formulae
$\Gamma(t)$. This set of formulae will also contain any obligations
and prohibitions on different agents. For each agent $a_i$ at time
$t$, there is a contract $\mathbf{c}(a_i, t)\subseteq \Gamma(t)$ that
describes $a_i$'s obligations, prohibitions etc.  $a$ at any time $t$
then comes up with a goal $g$ so that its
contract is satisfied.\footnote{See \cite{nsg_sb_dde_2017} for an
  example of how obligations and prohibitions can be used in \DCEC.}
The agent believes that if  $g$ does not hold then its contract at some future
$t+\delta$  will be violated:

 $$\mathbf{B}\left(a, t, \lnot
 g \rightarrow  \lnot \bigwedge \mathbf{c}(a, t+\delta)\right)$$ Then the agent tries
 to come up with a plan involving a sequence of actions to satisfy the
 goal.


We make these notions more precise. An agent $a$ has a set of
actions that it \emph{can} perform at different time points. For instance, a
vacuuming agent can have movement along a plane as its possible
actions while an agent on a phone can have displaying a notification
as an action. We denote this by $can(a,\alpha, t)$ with the following
additional axiom:


 

$$ \colorbox{gray!10}{Axiom}  \lnot can(a,\alpha,
t) \rightarrow \lnot \happens(action(a, \alpha), t) $$

 
\noindent We now define a \emph{consistent plan} below:

 \begin{footnotesize}
\begin{mdframed}[linecolor=white, frametitle=Consistent Plan,
  frametitlebackgroundcolor=gray!10, backgroundcolor=gray!05,
  roundcorner=8pt]

  A \emph{consistent plan} $\rho_{\langle a_1, \ldots, a_n\rangle} $ at
  time $t$ is a sequence of agents $a_1,\ldots, a_n$ with
  corresponding actions $\alpha_1, \ldots, \alpha_n$ and times
  $t_1, \ldots, t_n$ such that
  $\Gamma \vdash (t<t_i < t_j) \mbox{ for } i<j$ and for all agents $a_i$
  we have:
\begin{enumerate} 
\item $can(a_i, \alpha_i, t_i)$
\item $happens(action(a_i,\alpha_i))$ is consistent with $\Gamma(t)$. 
\end{enumerate}
 \end{mdframed}
 \end{footnotesize}
 Note that a consistent plan $\rho_{\langle \ldots \rangle}$ can be
 represented by a term in our language. We introduce a new sort
 $\mathsf{Plan}$ and a variable-arity predicate symbol
 $\plan(\rho, a_1, \ldots, a_n)$ which says that $\rho$ is a plan
 involving $a_1\ldots, a_n$.


 A goal is also any formula $g$.  A consistent plan satisfies a goal
 $g$ if:

\begin{equation*}
 \left(\begin{aligned}
\Gamma(t) \cup \left\{ \begin{aligned}
   & happens(action(a_1,\alpha_1), t_1), \ldots,\\
 &happens(action(a_n,\alpha_n), t_n) \end{aligned}\right\} 
 \end{aligned}\right) \vdash g
 \end{equation*}
 We use $\Gamma\vdash (\rho \rightarrow g)$ as a shorthand for the
 above. The above definitions of plans and goals give us the following
 important constraint needed for defining TAI. This differentiates our
 planning formalism from other planning systems and makes it more
 appropriate for an architecture for a general-purpose tentacular AI
 system.


 \begin{footnotesize}
\begin{mdframed}[linecolor=white, frametitle=Uniform Planning Constraint,
  frametitlebackgroundcolor=yellow!25, backgroundcolor=yellow!10,
  roundcorner=8pt]
  Plans and goals should be represented and reasoned over in the
  language of the planning system.
\end{mdframed}
 \end{footnotesize}


 Leveraging the above requirement, we can define two levels of TAI
 agents. A $\level{1}{*}$ TAI system corresponding to an agent $\tau$
 is a system that comes up with goal $g$ at time $t'$ to satisfy its
 contract, produces a proof that there is no consistent plan that
 involves only the agent $\tau$. Then $\tau$ comes with a plan that
 involves one or more other agents.  A $\level{1}{*}$ TAI agent starts
 with knowledge about what actions are possible for other agents.



 \begin{footnotesize}
   \begin{mdframed}[linecolor=white, frametitle=$\level{1}{*}$ TAI Agents ,
     frametitlebackgroundcolor=gray!10, backgroundcolor=gray!05,
     roundcorner=8pt]
     \begin{enumerate}
       \item[\textbf{Prerequisite}] For any $a$, $\alpha$, $t$, we have:
         \begin{equation*}
           \begin{aligned}
             \Gamma \vdash
             & can(a, \alpha, t) \rightarrow \Knows\big(\tau, t',
             can(a, \alpha, t) \big) \end{aligned}
         \end{equation*}
\item[\textbf{Then}]
       \item $\tau$ produces a proof that no plan exists for $g$
         involving just itself and $\tau$ declares that there is no
         such plan.
         \begin{equation*}
           \begin{aligned}
             \Gamma \vdash
             & \says\big(\tau, t', \lnot \exists  \rho:\left( \plan(\rho,
             \tau) \land \rho \rightarrow g\right)\big) \end{aligned}
         \end{equation*}
       \item $\tau$ produces a plan for $g$ involving just itself and
         one or more agents and declares that plan.
         \begin{equation*}
           \begin{aligned}
             \Gamma \vdash
             & \says\Bigg(\tau, t', \Big( \plan(\rho
            , a_1, \ldots, \tau \ldots a_n) \land \rho \rightarrow g\Big)\Bigg) 
         \end{aligned}
         \end{equation*}
       \end{enumerate}
     \end{mdframed}
   \end{footnotesize}

   The agent may not always have knowledge about what other agents can
   do. The TAI agent may have imperfect knowledge about other
   agents. The agent can gain information about other agents' actions,
   their obligations, prohibitions, etc. by observing them or by reading
   specifications governing these agents. In this case, we get a
   $\level{2}{*}$ TAI agent. We need to modify only the prerequisite
   condition above.


 \begin{footnotesize}
   \begin{mdframed}[linecolor=white, frametitle=$\level{2}{*}$ TAI Agents ,
     frametitlebackgroundcolor=gray!10, backgroundcolor=gray!05,
     roundcorner=8pt]
     \begin{enumerate}
     \item[\textbf{Prerequisite}]  For any $a$, $\alpha$, $t$, we have:
         \begin{equation*}
           \begin{aligned}
             \Gamma \vdash
             & can(a, \alpha, t) \rightarrow \Believes\big(\tau, t',
             can(a, \alpha, t) \big) \end{aligned}
         \end{equation*}
      
       \end{enumerate}
     \end{mdframed}
   \end{footnotesize}
 
   The TAI agents above can be considered \textbf{first-order}
   tentacular agents. We can also have a \textbf{higher-order} TAI
   agent that intentionally engages in actions that trigger one or
   more other agents to act in tentacular fashion as described above.
   The need for having the uniform planning constraint is more clear
   when we consider higher-order agents.







%%% Local Variables:
%%% mode: latex
%%% TeX-master: "main.tex"
%%% End:
