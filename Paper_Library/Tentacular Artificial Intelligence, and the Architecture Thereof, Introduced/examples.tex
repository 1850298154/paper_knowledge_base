In this section, we present a formal sketch of a TAI agent and then
describe using another example ongoing work in implementing a TAI
system.

\subsection{Example } Consider the example given in the beginning.  We
have a human $j$ and three artificial agents: $a_{c}$ in the car,
$a_h$ in the home and $a_p$ an agent managing scheduling and calendar
information.  We present some of the formulae in $\Gamma$.


\begin{footnotesize}
\begin{equation*}
\begin{aligned}
& \mathbf{B}(a_c, t_0, crowded(store) \rightarrow unusal), \mbox{\colorbox{gray!20}{$\mathsf{f_1}$}}\\
&\mathbf{P}(a_c, t_1, crowded(store)),\mbox{\colorbox{gray!20}{$\mathsf{f_2}$}}\\
&\forall t: \mathbf{O}\left(\begin{aligned}a_c, t, & unusal,\\ &
    happens\big(action(a_c, check(weather)),
    t+1\big) \end{aligned}\right)\mbox{\colorbox{gray!20}{$\mathsf{f_3}$}}\\
&  \forall t: \mathbf{B}\left(a_c, t, \mbox{\colorbox{gray!20}{$\mathsf{f_3}$}}\right)\\
& \forall a: \left(\begin{aligned} happens\Big(& action(a, check(weather)), t_3\Big) 
\\ & \rightarrow\mathbf{K}(a, t_4, storm), \end{aligned}\right)\mbox{\colorbox{gray!20}{$\mathsf{f_4}$}}\\
&\forall t: \mathbf{O}\big(a_c, t, storm, \mathsf{S}(a_c, a_h, storm,
t+1))\big),\mbox{\colorbox{gray!20}{$\mathsf{f_5}$}}\\
&  \forall t: \mathbf{B}\left(a_c, t, \mbox{\colorbox{gray!20}{$\mathsf{f_5}$}}\right)\\
 \end{aligned} 
\end{equation*}
\end{footnotesize}
The above formulae first state the fact that $a_c$ observes the store
being crowded. $a_c$'s contract states that the agent should check a
weather service if it finds something unusual.  The formulae also
states that if an agent checks the weather at $t_3$, the agent will
get a prediction about an incoming storm.  $a_c$'s contract places an
obligation on it to inform $a_h$ if it believes that a storm is
incoming.
 
\begin{footnotesize}
\begin{equation*}
\begin{aligned}
&\forall t: \mathbf{O}\big(a_h, t, storm, \forall s: quantity(s) > 0\big),\mbox{\colorbox{gray!20}{$\mathsf{f_6}$}}\\
& \mathbf{K}\left(\begin{aligned} a_h, t_5, & shops(j, today) \lor shops(j,
tomorrow) \\ &  \rightarrow \forall s: quantity(s) > 0\end{aligned} \right), \mbox{\colorbox{gray!20}{$\mathsf{f_7}$}}\\
& \forall t: \mathbf{B}\left(\begin{aligned}a_h, t, & happens\big(action(a_c, recc(shops(j))), t\big)\\ & \rightarrow shops(j)\end{aligned}\right) \mbox{\colorbox{gray!20}{$\mathsf{f_8}$}}\\
& \forall t: \mathbf{B}\left(\begin{aligned} & a_h, t, happens\Big(action\big(a_h, req(a_c, shops(j))\big), t\Big)\\ & \rightarrow happens\big(action(a_c, recc(shops(j))), t\big)\end{aligned}\right) \mbox{\colorbox{gray!20}{$\mathsf{f_9}$}}\\
 \end{aligned} 
\end{equation*}
\end{footnotesize}

The first formula above states that $a_h$ ought to see to it that
supplies are stocked in the event of a storm. Then we have that 
$a_h$ knows that the human $j$ shopping today or tomorrow can result
in the supplies being stocked. $a_h$ gets information from $a_p$ that
shopping tomorrow is not possible (this formula is not shown). Then we
have formulae stating the effects of $a_c$ recommending the shopping
action to $j$.  The goal for $a_h$ is $\forall s: quantity(s)>0$ and a
plan for it is built up using $a_h$, $a_c$ and $j$.



%%% Local Variables:
%%% mode: latex
%%% TeX-master: "main"
%%% End:
