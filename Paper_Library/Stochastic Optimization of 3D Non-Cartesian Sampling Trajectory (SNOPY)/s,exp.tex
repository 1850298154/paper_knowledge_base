\subsection{Datasets}

We used two publicly available datasets;
both of them contain 3D multi-coil raw \kspace data. 
SKM-TEA \cite{desai2022skm}
is a 3D quantitative double-echo steady-state (qDESS\cite{welsch:2009:qdess}) knee dataset.
It was acquired by 3T GE MR750 scanners
and 15/16-channel receiver coils.
SKM-TEA includes 155 subjects.
We used 132 for training, 10 for validation,
and 13 for the test.
Calgary brain dataset \cite{souza2018open}
is a 3D brain T1w MP-RAGE \cite{brant-zawadzki:1992:MPRAGE} \kspace dataset.
It includes 67 available subjects,
acquired by an MR750 scanner
and 12-channel head coils.
We used 50 for training, 6 for validation, and 7 for testing.
All receiver coil sensitivity maps were calculated by
ESPIRiT \cite{espirit}.

\begin{table*}[bt]%
\caption{The quantitative reconstruction quality (PSNR) of the test set.
\label{tab:recon}}
\begin{tabular*}{\textwidth}{@{\extracolsep\fill}lccc@{\extracolsep\fill}}
\toprule
 & \textbf{CG-SENSE}  & \textbf{PLS}  & {\textbf{MoDL}} \\
\midrule
3D kooshball & 28.15 dB  & 28.16 dB  & 30.07 dB    \\
SNOPY & 32.47 dB  &  32.53 dB  & 33.68 dB    \\
\bottomrule
\end{tabular*}

\end{table*}

\subsection{Simulation experiments}
 
We experimented with multiple scenarios to show the broad
applicability of the proposed method. All the experiments used a server node equipped with an Nvidia Tesla A40 GPU for training.

\begin{figure*}
\centerline{\includegraphics[width=0.9\textwidth]{figure/fig,sos.png}}
\caption{~Prospective results of \ref{exp:sos}, optimizing the rotation angles of the stack-of-stars (6$\times$ acceleration).
`Best empirical' uses the design from previous study \cite{zhou:2017:GoldenratioRotatedStackofstars}.
The upper subfigure shows two slices from prospective in-vivo experiments.
The reconstruction algorithm was PLS.
\textbf{Avg. PSNR} is the average PSNR of the 4 subjects
compared to the fully sampled reference.
The lower subfigure shows the log-scaled PSF (single-coil) of two trajectories. 
\label{fig:sos}}
\end{figure*}

\begin{figure*}
\centerline{\includegraphics[width=\textwidth]{figure/fig,ossi.png}}
\caption{~Prospective results of \ref{exp:pns}. We showed three different trajectories: the unoptimized REPI, SNOPY-optimized with the PNS threshold of 80\%, and SNOPY-optimized with the PNS threshold of 70\%. The left subfigure shows one slice of reconstructed images. The reconstruction used PLS and 120 shots (volume TR = 2s). The right subfigure shows the subjective score of the PNS effect.\label{fig:ossi}}
\end{figure*}






\subsubsection{Optimizing 3D gradient waveform}
\label{exp:freeform}
We optimized the sampling trajectory with a 3D radial
(``kooshball'') initialization
\cite{barger:2002:TimeresolvedContrastenhancedImaging,herrmann:2016:TimeEfficient3D}.
As described in \ref{subsec:param},
we directly optimized the readout waveform of each shot.
The trajectory was parameterized by B-spline kernels
to reduce the number of degrees of freedom
and enable multi-scale optimization.
The initial 3D radial trajectory 
had a 5.1 2ms readout (raster time = 4 $\mu$s)
and 1024 spokes/shots (8$\times$ acceleration),
using the rotation angle described in \citestd{chaithya:2022:OptimizingFull3D}.
The training used the SKM-TEA dataset.
The FOV was 15.8$\times$15.8$\times$5.1cm
with 1mm$^3$ resolution.
The receiver bandwidth was $\pm$125kHz.
The training loss was
$$
\Loss = \Lrecon + 0.1 \Lg + 0.1 \Ls +  \Lpns.
$$
The gradient strength (\gmax),
slew rate (\smax),
and PNS threshold (\pmax)
were 50 mT/m, 150 T/m/s, 80\%,
respectively.
The learning rate
$\etao$ % macro...
decayed from 1e-4 to 0 linearly.
For the multi-level optimization,
we used 3 levels (with B-spline kernel widths = 32, 16, and 8 time samples),
and each level used 200 epochs.
The total training time was $\sim$180 hrs.
We also optimized the trajectory
for several image reconstruction algorithms.
We used a regularizer weight of 1e-3
and 30 CG iterations for CG-SENSE and PLS. 
For learning-based reconstruction,
we used the MoDL \cite{modl} approach
that alternates between a neural network-based denoiser
and data consistency updates.
We used a 3D version of the denoising network \cite{DIDN},
20 CG iterations for the data consistency update,
and 6 outer iterations.
Similar to previous investigations
\cite{aggarwal:2020:JointOptimizationSampling, wang:22:bjork-tmi},
SNOPY jointly optimized
the neural network's parameters
and the sampling trajectories
using \eqref{e:sgld}.


\subsubsection{Optimizing rotation angles of stack-of-stars trajectory}
\label{exp:sos}

This experiment optimized the rotation angles
of stack-of-stars,
which is a widely used volumetric imaging sequence.
The training used Calgary brain dataset.
We used PLS as the reconstruction method for simplicity,
with $\lambda=10^{-3}$ and 30 iterations.
We used 200 epochs and a learning rate
linearly decaying from 1e-4 to 0.
The FOV is 25.6$\times$21.8$\times$3.2 cm
and the resolution is 1mm$^3$.
We used 40 spokes per $kz$ location (6$\times$ acceleration),
and 1280 spokes in total.
The readout length is 3.5 ms.
The receiver bandwidth is $\pm$125kHz.
The trajectory was a stack of
32 stars,
so we optimized
1280
rotation angles \cc.
 
Since optimizing rotation angles
does not
impact the gradient strength, slew rate, PNS, and image contrast, 
we used only
the reconstruction loss
$\Loss = \Lrecon.$
We regard the method (RSOS-GR) proposed in previous work
\cite{zhou:2017:GoldenratioRotatedStackofstars} as the
best currently available scheme.
We applied 200 epochs with a linearly decaying learning rate 
from 1e-3 to 0.
The training cost  $\sim$20 hrs.

\subsubsection{PNS suppression of 
3D rotational EPI trajectory
for functional imaging}
\label{exp:pns}

The third application
optimizes the rotation EPI
(REPI) trajectory \cite{rettenmeier:2022:REPI},
which provides an efficient sampling strategy for fMRI.
For higher resolution (i.e., $\leq$1mm),
we found that subjects may experience
strong PNS effects introduced by REPI.
This experiment aimed to reduce the PNS effect of REPI
while preserving the original image contrast.
We optimized one shot/readout waveform of REPI
with a B-spline kernel with a width of 16 to parameterize the trajectory,
and rotated the optimized readout shot
using the angle scheme
similar to \cite{rettenmeier:2022:REPI}.

We designed the REPI readout
for an oscillating stead steady imaging (OSSI) sequence,
a novel fMRI signal model that can
improve the SNR \cite{guo:2020:OSSI,guo:2020:HOSSI}.
The FOV is 20$\times$20$\times$1.2 cm,
with 1 mm$^3$ isotropic resolution,
TR = 16 ms, and TE = 7.4 ms.
The readout length is 10.6 ms.
The receiver bandwidth is $\pm$250kHz.

To accelerate training,
the loss term here
excluded
the reconstruction loss $\Lrecon$:
$$
\Loss = 0.01 \Loss_{g} + 0.01 \Loss_{s} + \Loss_{pns} + \Loss_{c}.
$$
The training used 40,000 steps,
with a learning rate decaying linearly from 1e-4 to 0.
The training cost $\sim$1 hrs.

\subsection{In-vivo experiments}

We implemented the optimized trajectory prospectively
on a GE UHP 3.0T scanner
equipped with a Nova Medical 32-channel head coil.
Participants gave informed consent under local IRB approval. 
Because the cache in the MR system cannot load  
hundreds of distinct gradient waveforms,
the experiment \ref{exp:freeform}
was not implemented prospectively.
Readers may refer to the corresponding 2D prospective studies
\cite{wang:22:bjork-tmi}
for image quality improvement
and correction of eddy current effects.
For experiment \ref{exp:sos},
we programmed the sampling trajectory with
a 3D T1w fat-saturated GRE sequence \cite{nielsen:2018:TOPPEFrameworkRapid},
with TR/TE = 14/3.2ms and FA = 20\textdegree.
The experiment included 4 healthy subjects.
For experiment \ref{exp:pns},
to rate the PNS effect,
we asked 3 participants to score the nerve stimulation 
with a 5-point Likert scale
from `mild tingling' to `strong muscular twitch.'


\subsection{Reproducible research}

The code for 2D trajectory optimization is publicly available%
\footnote{\url{https://github.com/guanhuaw/Bjork}}.
As an accompanying project,
MIRTorch\footnote{\url{https://github.com/guanhuaw/MIRTorch}}
facilitates the differentiable programming
for MRI sampling and reconstruction.
When this paper is accepted, we will also provide the 3D version as a toolbox
on open-source platforms.
For the prospective in-vivo experiments,
we will provide open-source and vendor-agnostic 
sequences based on TOPPE \cite{nielsen:2018:TOPPEFrameworkRapid}.

