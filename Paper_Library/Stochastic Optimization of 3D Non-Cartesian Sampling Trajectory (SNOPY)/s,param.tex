As is shown in \citestd{wang:22:bjork-tmi},
directly optimizing
every \kspace sampling point
(or equivalently every gradient waveform time point)
may lead to sub-optimal results.
Additionally, in many applications one wants to
optimize certain
properties of existing sampling patterns,
such as the rotation angles of a multi-shot spiral trajectory,
so that the optimized trajectory can 
be easily integrated into existing workflows.
For these cases,
we propose two parameterization strategies.

The first approach,
spline-based freeform optimization, 
is to represent the sampling pattern
using a linear basis,
i.e.,
$\om = \B \cc$,
where \B is a matrix of samples
of a basis
such as quadratic B-spline kernels
and \cc denotes
the coefficients to be optimized
\cite{wang:22:bjork-tmi,pilot}.
This approach fully exploits the generality
of a gradient system.
Using a linear parameterization like B-splines
reduces the degrees of freedom
and facilitates applying hardware constraints \cite{wang:22:bjork-tmi,hao:2016:JointDesignExcitationa}.
Additionally, it enables multi-scale optimization 
for avoiding sub-optimal local minima
and further improving optimization results \cite{sparklingmrm,wang:22:bjork-tmi,pilot}.
However, the freeformly optimized trajectory could have implementation challenges. For example, some MRI systems can not restore hundreds of different gradient waveforms. 

The second approach is to optimize attributes (\cc) of existing trajectories
such as rotation angles,
where $\om(\cc)$ is a nonlinear function of the parameters.
The trajectory parameterization
should be differentiable in \cc
to enable differentiable programming.
This approach is easier to implement on scanners,
and can work with existing workflows, such as reconstruction and eddy-current correction,
with minimal modification.