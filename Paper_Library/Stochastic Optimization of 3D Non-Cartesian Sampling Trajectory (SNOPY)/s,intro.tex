Most magnetic resonance imaging systems
sample data in the frequency domain (\kspace)
following prescribed sampling trajectories.
Efficient sampling strategies can
accelerate acquisition and improve image quality.
Many well-designed sampling strategies and their variants,
such as spiral, radial, CAIPIRINHA, and PROPELLER
\cite{ahn1986spiral, lauterbur1973image, breuer:2006:caipi, pipe:1999:PROPELLER},
have enabled MRI's application to many areas \cite{glover:2001:SpiralinOutBOLD,johansson:2018:RigidbodyMotionCorrection, yu:2013:ClinicalApplicationControlled, forbes:2001:PROPELLERMRIClinical}.
Sampling patterns in \kspace
either are located on the Cartesian raster
or arbitrary locations (non-Cartesian sampling).
This paper focuses on optimizing 3D 
non-Cartesian trajectories
and introduces a generalized 
gradient-based optimization method
for automatic trajectory design/tailoring.

The design of sampling patterns usually considers
certain properties of \kspace signals.
For instance, the variable density (VD) spiral trajectory \cite{lee:2003:Fast3DImaging}
samples more densely in the central \kspace
where more energy is located.
For higher spatial frequency regions,
the VD spiral trajectory uses larger gradient strengths
and slew rates
to cover as much of \kspace as quickly as possible.
Compared to 2D sampling,
designing 3D sampling by hand
is more challenging for several reasons.
The number of parameters increases in 3D,
and thus the parameter selection is more difficult 
due to the larger search space.
Additionally, analytical designs usually are
based on the Shannon-Nyquist relationship \cite{gurney:2006:DesignAnalysisPractical,johnson:2017:HybridRadialconesTrajectory,zhou:2017:GoldenratioRotatedStackofstars}
that might not fully consider
properties of sensitivity maps
and advanced reconstruction methods.
For 3D sampling pattern with a high undersampling (acceleration) ratio,
there are limited analytical tools
for designing sampling patterns
having anisotropic FOV and resolution.
The peripheral nerve stimulation (PNS) effect
is also more severe in the 3D imaging \cite{ham:1997:PeripheralNerveStimulation},
further complicating analytical trajectory designs.
For these reasons, it is important to automate the design of 3D sampling trajectories.

Many 3D sampling approaches exist.
`stack-of-2D' is a commonly used 3D sampling strategy,
by stacking 2D sampling patterns
in the slice direction \cite{johansson:2018:RigidbodyMotionCorrection,ham:1997:PeripheralNerveStimulation}.
This approach is easier to implement 
and enables slice-by-slice 2D reconstruction.
Another design applies Cartesian sampling in the frequency-encoding direction
and non-Cartesian sampling in the phase-encoding direction \cite{aggarwal:2020:JointOptimizationSampling,bilgic:2015:WaveCAIPIHighlyAccelerated}.
However, these approaches do not
fully exploit the potential of modern gradient systems
and may not perform as well as true 3D sampling trajectories
\cite{chaithya:2022:OptimizingFull3D}. 

Recently, 3D SPARKLING \cite{chaithya:2022:OptimizingFull3D}
proposes to optimize 3D sampling trajectories
based on 
the goal of
conforming to a given density
while distributing samples as uniformly as possible \cite{sparklingmrm}.
That method demonstrated improved image quality
compared to the `stack-of-2D SPARKLING' approach.
In both 2D and 3D,
the SPARKLING approach
uses a pre-specified sampling density in \kspace
that is typically 
an isotropic radial function.
This density function
cannot readily capture distinct energy distributions
of different imaging protocols.
Additionally, the method does not control PNS effects explicitly.
SPARKLING optimizes the location of every sampling point,
or the gradient waveform
(freeform optimization),
and is not applicable to the optimization of
existing parameters of
existing sampling patterns,
which limits its practicability beyond
T2$^*$-weighted imaging.

In addition to analytical methods,
learning-based methods are also investigated
in MRI sampling trajectory design.
Since different anatomies have distinct energy
distributions in the frequency domain,
an algorithm may learn to optimize sampling trajectories 
from training datasets.
Several studies show that different anatomies
and different reconstruction algorithms
produce very different optimized sampling patterns,
and such optimized sampling trajectories can improve image quality \cite{huijben:2020:LearningSamplingModelBased, wang:22:bjork-tmi,bahadir:2020:DeepLearningBasedOptimizationUnderSampling, pilot, sanchez:2020:ScalableLearningBasedSampling,jin:2019:SelfSupervisedDeepActive,
rl:david,sherry:20:lts,gozcu2018learning}.
Several recent studies also applied learning-based approaches
to 3D non-Cartesian trajectory design.
J-MoDL \cite{aggarwal:2020:JointOptimizationSampling}
proposes to learn sampling patterns and model-based deep learning reconstruction algorithms jointly.
J-MoDL optimizes the sampling locations in the phase-encoding direction,
to avoid the computation cost 
of non-uniform Fourier transform.
PILOT/3D-FLAT \cite{alush-aben:2020:3DFLATFeasible}
jointly optimizes freeform 3D non-Cartesian trajectories
and a reconstruction neural network
with gradient-based methods. 
These studies use the standard %brute-force
auto-differentiation approach to calculate the gradient,
which can be inaccurate and 
lead to sub-optimal optimization results \cite{wang:22:bjork-tmi,wang:21:eao}.

This work extends our previous methods
\cite{wang:22:bjork-tmi, wang:21:eao}
and introduces a generalized
\textbf{S}tochastic optimization framework
for 3D \textbf{NO}n-Cartesian sam\textbf{P}ling trajector\textbf{Y} (\textbf{SNOPY}).
The proposed method can automatically tailor given trajectories
and learn \kspace features from training datasets.
We formulated several optimization objectives,
including image quality, hardware constraints,
PNS effect suppression, and image contrast.
Users can simultaneously optimize one or multiple characteristics
of a given sampling trajectory.
Similar to previous learning-based methods \cite{aggarwal:2020:JointOptimizationSampling,
wang:22:bjork-tmi,bahadir:2020:DeepLearningBasedOptimizationUnderSampling,pilot},
the sampling trajectory can be jointly optimized
with trainable reconstruction algorithms
to improve image quality.
The joint optimization approach
can exploit the synergy between
acquisition and reconstruction,
and learn optimized trajectories 
for different anatomies.
The algorithm can optimize various properties of a sampling trajectory,
such as the readout waveform, or the rotation angles of readout shots,
making it more practical and applicable.
We also introduced several techniques to
improve efficiency,
enabling large-scale 3D trajectory optimization.
We tested the proposed methods
with multiple imaging applications,
including structural and functional imaging.
These applications benefited from the
SNOPY-optimized sampling trajectories
in both simulation and prospective studies.
