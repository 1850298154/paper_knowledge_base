\begin{figure*}
\centerline{\includegraphics[width=0.8\textwidth]{figure/fig,workflow.png}}
\caption{~Diagram of SNOPY.
The sampling trajectory (and possibly reconstruction parameters)
are updated using gradient methods.
The training/optimization uses the differentiable programming approach
to get the gradient required in the update.
\label{fig:workflow}}
\end{figure*}

This section describes the proposed gradient-based methods for 
trajectory optimization.
We use the concept of differentiable programming
to compute the Jacobian/gradient
w.r.t. sampling trajectories required
in the gradient-based methods.
The sampling trajectory and reconstruction parameters
are differentiable parameters,
whose gradients can be calculated
by auto-differentiation or chain rule.
To learn/optimize these parameters,
one may apply (stochastic) gradient descent-like algorithms.
\fref{fig:workflow} illustrates the basic idea.
Here the sampling trajectory can also be jointly optimized with
the parameters of a learnable reconstruction algorithm,
so that the learned sampling trajectory and reconstruction method
are in synergy and can produce high-quality images.
The SNOPY algorithm combines
several optimization objectives,
to ensure that the optimized sampling trajectories 
have desired properties.
\ref{subsec:obj} delineates these objective functions.
\ref{subsec:param} shows that the proposed method
is applicable to multiple scenarios with
different parameterization strategies.
For non-Cartesian sampling,
the system model usually involves non-uniform fast Fourier transforms (NUFFT).
\ref{subsec:jacob} briefly describes an
efficient and accurate way to calculate the gradient involving NUFFTs.
\ref{subsec:efficient}
suggests several engineering approaches to 
make this large-scale optimization problem practical and efficient.



\subsection{Optimization objectives}
\label{subsec:obj}

This section describes the
optimization objectives. 
Since we propose to use a stochastic gradient descent-type
optimization algorithm,
the objective function, or loss function,
is by default defined on a mini-batch of data.
The final loss function can be a linear combination
of the following loss terms
to ensure the optimized trajectory
has multiple desired properties.


\subsubsection{Image quality}

For many MRI applications,
efficient acquisition and reconstruction aim to produce high-quality images.
Consequently, the learning objective should encourage
images reconstructed
from sampled \kspace signals to be close to the reference image.
We formulate this similarity objective
as the following image quality training loss:
\begin{equation}
    \label{eqn:image}
\Lrecon = 
\|f_{\thta, \om}(\A(\om)\x+\vveps) - \x\|.
\end{equation}
Here,
$\om(\cc) \in \reals^{\Nfe \times \Ns \times \Nd}$
denotes the trajectory to be optimized,
with \Ns shots, \Nfe sampling points in each shot, and \Nd image dimensions.
\cc denotes the parameterization coefficients introduced in \ref{subsec:param}.
For 3D MRI, $\Nd=3$.
\x is a mini-batch of data from the training set $\mathcal{X}$.
\vveps is simulated complex Gaussian noise.
$\A(\om)$ is the forward system matrix for sampling trajectory \om \cite{fessler:03:nff}.
\A can also incorporate multi-coil sensitivity information \cite{pruessmann:2001:AdvancesSensitivityEncoding}
and field inhomogeneity \cite{fessler:05:tbi}.
\thta denotes the reconstruction algorithm's parameters.
It can be kernel weights in a convolutional neural network (CNN),
or the regularizer coefficient in a model-based reconstruction method.
The term $\|\cdot\|$ can be $\ell_1$ norm,
$\ell_2$ norm,
or a combination of both.
There are also other ways to measure the distance between $\x$ and
$f_{\thta, \om}(\A(\om)\x+\vveps)$,
such as the structural similarity index (SSIM \cite{wang:2004:ssim}).
For simplicity,
this work used a linear combination of $\ell_1$ norm and square-of-$\ell_2$ norm.


\subsubsection{Hardware limits}

The gradient system of MR scanners
has physical constraints,
namely maximum gradient strength and slew rate.
Ideally, we would like to enforce
a set of constraints of the form
\[
\| \gi[j,:] \|_2 \leq g_{\mathrm{max}}
,\quad
\gi = \D_1 \om[:,i,:] / (\gamma \dt) \in \reals^{(\Nfe-1) \times \Nd}
,\]
for every shot $i = 1,\ldots,\Ns$
and time sample $j = 1,\ldots,\Nfe$,
where $\gi$ denotes the gradient strength
of the $i$ shot
and
\gmax
denotes the desired gradient upper bound.
We use a Euclidean norm along the spatial axis
so that any 3D rotation of the sampling trajectory 
still obeys the constraint.
A similar constraint is enforced
on the Euclidean norm
of the slew rate
\(
\si = \D_2 \om[:,i,:] / (\gamma \dt^2)
,\)
where
$\D_1$ and $\D_2$
denote first-order and second-order
finite difference operators applied along the readout dimension,
and
\dt is the raster time interval
and $\gamma$ is the gyromagnetic ratio.

To make the optimization more practical,
we follow previous studies
\cite{wang:22:bjork-tmi,pilot},
and formulate the hardware constraint as a soft penalty term:
\begin{equation}
    \label{eqn:grad}
    \Lg=
    \sum_{i=1}^{\Ns} \sum_{j=1}^{\Nfe-1} 
    \phi_{\gmax}(\| \gi[j,:] \|_2)
\end{equation}
\begin{equation}
    \label{eqn:slew}
    \Ls=
    \sum_{i=1}^{\Ns} \sum_{j=1}^{\Nfe-2} 
    \phi_{\smax}(\| \si[j,:] \|_2)
.\end{equation}
Here
$\phi$ is a penalty function,
and we use a soft-thresholding function
$\phi_\lambda(x)
= \max(|x|-\lambda, 0).$
Since $\phi$ here is a soft penalty
and the optimization results may exceed \smax and \gmax,
\smax and \gmax can be slightly lower than
the scanner's physical limits
to make the optimization results feasible on the scanner.

\subsubsection{Suppression of PNS effect}

3D imaging often
leads to stronger PNS
effects than 2D imaging
because of the additional gradient
axis.
To quantify possible PNS effects
of candidate gradient waveforms,
we used the convolution model in
\citestd{schulte:2015:PNS}:
\begin{equation}
    R_{id}(t)=\frac{1}{\smin} \int_0^t \frac{\sid(\theta) c}{(c+t-\theta)^2} d \theta,
\end{equation}
where $R_{id}$ denotes the PNS effect of the gradient waveform from the $i$th shot and the $d$th dimension.
$\sid$ is the slew rate of the $i$th shot in the $d$th dimension.
\xmath{c} (chronaxie) and \smin (minimum stimulation slew rate)
are scanner parameters.

Likewise, we discretize the convolution model
and formulate a soft penalty term 
as the following loss function:
\begin{equation} 
\pid[j] = \sum_{k=1}^{j}\frac{\sid[j] c  \dt}{\smin(c + j \dt - k \dt)^2},
\nonumber
\end{equation}
\begin{equation}
    \label{eqn:pns}
    \Lpns=
    \sum_{i=1}^{\Ns} \sum_{j=1}^{\Nfe} \phi_{\pmax}((\sum_{d=1}^{\Nd}\pid[j]^2)^{\frac{1}{2}}).
\end{equation}

Again, $\phi$ denotes the soft-thresholding function,
with PNS threshold \pmax (usually $\leq 80\%$\cite{schulte:2015:PNS}). 
This model combines the 3 spatial axes via the sum-of-squares manner, 
and does not consider the anisotropic response of PNS \cite{davids2019prediction}.
The implementation may use an FFT (with zero padding)
to implement this convolution efficiently.


\subsubsection{Image contrast}

In many applications,
the optimized sampling trajectory should 
maintain certain parameter-weighted contrasts.
For example,
ideally the (gradient) echo time (TE)
should be identical for each shot.
Again
we replace this hard constraint 
with an echo time penalty.
Other parameters, like repetition time (TR)
and inversion time (TI),
can be predetermined in the RF pulse design.
Specifically, the corresponding loss function
encourages the sampling trajectory to cross
the \kspace center at certain time points:
\begin{equation}
    \label{eqn:contrast}
    \Loss_c =
    \sum_{\{i,j,d\} \in C}
    \phi_{0}(|\om[i,j,d]|), 
\end{equation}

where $C$ is a collection of gradient 
time points that are constrained 
to cross k-space zero point.
$\phi$ is still a
soft-thresholding function,
with threshold 0.



\subsection{Reconstruction}
In \eqref{eqn:image},
the reconstruction algorithm $f_{\thta, \om}(\cdot)$
can be various algorithms.
Consider a typical cost function
for regularized MR image reconstruction
\begin{equation}
\label{recon}
\xh = \argmin_{\x} \|\A(\om)\x-\y\|_{2}^2 + \mathcal{R}(\x).
\end{equation}
$\mathcal{R}(\x)$ here can be a Tikhonov regularization $\lambda \|\x\|_2^2$
(CG-SENSE \cite{maier:2021:CGSENSERevisitedResults}),
a sparsity penalty $\lambda \|\T\x\|_1$ 
(compressed sensing \cite{lustig:2008:CompressedSensingMRI}, $\T$ is a finite-difference operator),
a roughness penalty $\lambda \|\T\x\|_2^2$ 
(penalized least squares, PLS),
or a neural network (model-based deep learning, MoDL \cite{modl}).
The Results section shows that different reconstruction algorithms
lead to distinct optimized sampling trajectories.

To get a reconstruction estimation $\xh$, 
one may use corresponding iterative reconstruction algorithms.
Specifically, the algorithm should be step-wise
differentiable (or sub-differentiable)
to enable differentiable programming.
The backpropagation
uses the chain rule to traverse 
every step of the iterative algorithm
to calculate the gradient w.r.t. variables
such as \om.



\subsection{Parameterization}
\label{subsec:param}
As is shown in \citestd{wang:22:bjork-tmi},
directly optimizing
every \kspace sampling point
(or equivalently every gradient waveform time point)
may lead to sub-optimal results.
Additionally, in many applications one wants to
optimize certain
properties of existing sampling patterns,
such as the rotation angles of a multi-shot spiral trajectory,
so that the optimized trajectory can 
be easily integrated into existing workflows.
For these cases,
we propose two parameterization strategies.

The first approach,
spline-based freeform optimization, 
is to represent the sampling pattern
using a linear basis,
i.e.,
$\om = \B \cc$,
where \B is a matrix of samples
of a basis
such as quadratic B-spline kernels
and \cc denotes
the coefficients to be optimized
\cite{wang:22:bjork-tmi,pilot}.
This approach fully exploits the generality
of a gradient system.
Using a linear parameterization like B-splines
reduces the degrees of freedom
and facilitates applying hardware constraints \cite{wang:22:bjork-tmi,hao:2016:JointDesignExcitationa}.
Additionally, it enables multi-scale optimization 
for avoiding sub-optimal local minima
and further improving optimization results \cite{sparklingmrm,wang:22:bjork-tmi,pilot}.
However, the freeformly optimized trajectory could have implementation challenges. For example, some MRI systems can not restore hundreds of different gradient waveforms. 

The second approach is to optimize attributes (\cc) of existing trajectories
such as rotation angles,
where $\om(\cc)$ is a nonlinear function of the parameters.
The trajectory parameterization
should be differentiable in \cc
to enable differentiable programming.
This approach is easier to implement on scanners,
and can work with existing workflows, such as reconstruction and eddy-current correction,
with minimal modification.



\subsection{Efficient and accurate Jacobian calculation}
\label{subsec:jacob}
In optimization,
the sampling trajectory is
embedded in the forward system matrix
within the similarity loss \eqref{eqn:image}.
The system matrix for non-Cartesian sampling
usually includes a NUFFT operation
\cite{fessler:03:nff}.
Updating the sampling trajectory
in each optimization step
requires the Jacobian, or the gradient w.r.t.
the sampling trajectory.
The NUFFT operator typically involves interpolation 
in the frequency domain,
which is non-differentiable in typical implementations
due to rounding operations.
Several previous works used auto-differentiation
(with sub-gradients)
to calculate an approximate
numerical gradient \cite{pilot,alush-aben:2020:3DFLATFeasible},
but that approach is inaccurate and slow \cite{wang:21:eao}.
We derived an efficient and accurate
Jacobian approximation method \cite{wang:21:eao}.
For example,
the efficient Jacobian of a forward system model \A is:
\begin{equation}
\frac{\partial \A\x}{\partial \omd} = -\imath \, \diag{\A (\x\odot\rd)}
\label{e,Ax}
,\end{equation}
where $d \in \{1,2,3\}$ denotes a spatial dimension,
and $\rd$ denotes the Euclidean spatial grid.
Calculating this Jacobian
simply uses another NUFFT, 
which is more efficient than the auto-differentiation approach.
See 
\citestd{wang:21:eao} for more cases,
such as $\frac{\partial \A'\A\x}{\partial \omd}$ and the detailed derivation.

\subsection{Efficient optimization}
\label{subsec:efficient}
\subsubsection{Optimizer}

\begin{table*}[t]%
\caption{~The memory/time use reduction brought by proposed techniques.
Here we used a 2D 400$\times$400 test case,
and CG-SENSE reconstruction (20 iterations). `+' means adding the technique to previous columns.
}
\label{tab:memory}
\begin{tabular*}{\textwidth}{@{\extracolsep\fill}lcccc@{\extracolsep\fill}}
\toprule
\textbf{Plain} & \textbf{+Efficient Jacobian}  & \textbf{+In-place ops}  & {\textbf{+Toeplitz embedding}}  & \textbf{+Low-res NUFFT}   \\
\midrule
5.7GB / 10.4s  &  272MB / 1.9s  & 253MB / 1.6s  & 268MB / 0.4s  & 136MB / 0.2s  \\
\bottomrule
\end{tabular*}
\begin{tablenotes}%%[341pt]
\end{tablenotes}
\end{table*}

Generally, to optimize the sampling trajectory $\om$
and other parameters (such as reconstruction parameters $\thta$) via
stochastic gradient descent-like methods,
each update needs to take a gradient step
(in the simplest form)
\begin{align*}
\thta^K &= \thta^{K-1} - \etat  \frac{\partial \Loss}{\partial\thta}(\om^{K-1}, \thta^{K-1})
\\
\om^K &= \om^{K-1} - \etao  \frac{\partial \Loss}{\partial\om}(\om^{K-1}, \thta^{K-1}),
\end{align*}
where $\Loss$ is the loss function described
in Section \ref{subsec:obj} 
and
%$\eta$ is the step size.
where \etat and \etao denote step-size parameters.


The optimization is highly non-convex
and may suffer from sub-optimal local minima.
We used stochastic gradient Langevin dynamics (SGLD)
\cite{welling:2011:sgld} as the optimizer to improve results and
accelerate training.
Each update of SGLD injects Gaussian noise into the gradient
to introduce randomness
\begin{align}
\thta^K &= \thta^{K-1} - \etat
\frac{\partial \Loss}{\partial\thta^{K-1}} + \sqrt{2 \eta_{\thta}} \mathcal{N}(0,\,1)
\nonumber \\
\om^K &= \om^{K-1} - \etao
\frac{\partial \Loss}{\partial\om^{K-1}} + \sqrt{2\etao} \mathcal{N}(0,\,1)
\label{e:sgld}
.\end{align}

Across most experiments, 
we observed that SGLD led to improved results 
and better convergence speeds
compared with SGD or Adam \cite{kingma:2017:AdamMethodStochastic}.
\fref{fig:loss} shows a loss curve of SGLD
and Adam of experiment \ref{exp:pns}.

\begin{figure}[htb]
\centerline{\includegraphics[width=\columnwidth]{figure/fig,loss.png}}
\caption{~The evaluation loss curve led by SGLD and Adam.}
\label{fig:loss}
\end{figure}

\begin{figure*}[htb]
\centerline{\includegraphics[width=0.9\textwidth]{figure/fig,bjork.png}}
\caption{~The optimized sampling trajectory of experiment \ref{exp:freeform}.
The training involves SKM-TEA dataset and MoDL \cite{modl} reconstruction.
The upper row shows a zoomed-in region from different viewing perspectives.
The lower row shows one shot from different perspectives.
\label{fig:bjork}}
\end{figure*}

\begin{figure}[htb]
\centerline{\includegraphics[width=0.9\columnwidth]{figure/fig,psf.png}}
\caption{~Visualization of the optimized trajectory in \ref{exp:freeform}.
The upper subfigure displays PSFs (log-scaled, single-coil)
of trajectories optimized with different reconstruction methods.
The lower subfigure shows the density of sampling trajectories,
by convolving the sampling points with a Gaussian kernel.
Three rows are central profiles from three perspectives.
\label{fig:psf}}
\end{figure}

\begin{figure*}[htb]
\centerline{\includegraphics[width=0.99\textwidth]{figure/fig,recon.png}}
\caption{~Examples of the reconstructed images for two knee slices in experiment \ref{exp:freeform}.
\label{fig:recon}}
\end{figure*}



\subsubsection{Memory saving techniques}

Due to the large dimension, 
the memory cost for naive 3D trajectory optimization
would be prohibitively intensive.
We developed several techniques to
reduce memory use and accelerate training.

As discussed above,
the efficient Jacobian approximation 
uses only 10\% of the memory used in 
the standard auto-differentiation approach \cite{wang:21:eao}.
%It is also possible to
We also used in-place operations
in certain reconstruction steps,
such as the conjugate gradient (CG) method,
because with careful design
it will still permit
auto-differentiation.
(See our open-source code\footnote{\url{https://github.com/guanhuaw/Bjork}} for details.)
The primary memory bottleneck is with the 3D NUFFT operators.
We pre-calculate the Toeplitz embedding kernel to 
save memory and accelerate computation \cite{fessler:05:tbi,muckley:20:tah}.
In the training phase,
we use a NUFFT with lower accuracy,
for instance, with a smaller oversampling ratio for gridding
\cite{wang:21:eao}.
\tref{tab:memory} shows the incrementally improved efficiency
achieved with these techniques.
Without the proposed techniques,
optimizing 3D trajectories
would require hundreds of gigabytes of memory,
which would be impractical.
SNOPY enables solving this otherwise prohibitively large problem
on a single graphic card (GPU).

