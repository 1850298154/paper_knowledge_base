For the spline-based freeform optimization experiment delineated in \ref{exp:freeform},
\fref{fig:bjork} shows an example of the optimized trajectory
with zoomed-in regions and plots of a single shot.
Similar to the 2D case \cite{wang:22:bjork-tmi}
and SPARKLING \cite{chaithya:2022:OptimizingFull3D, sparklingmrm},
the multi-level B-spline 
optimization
leads to a swirling trajectory
that can cover more \kspace in the fixed readout time,
to reduce large gaps between sampling locations
and thus help reduce aliasing artifacts.
\fref{fig:psf} displays point spread functions (PSF)
of trajectories optimized jointly with different
reconstruction algorithms.
To visualize the sampling density
in different regions of \kspace,
we convolved the trajectory with a Gaussian kernel,
and \fref{fig:psf}
shows the density 
of central profiles from different views.
Compared with 3D kooshball,
the SNOPY optimization led to fewer radial patterns in the PSF,
corresponding to fewer streak artifacts in \fref{fig:recon}.
Different reconstruction algorithms
generated distinct optimized PSFs
and densities, 
which agrees with previous studies \cite{wang:21:eao, zibetti2020fast, gozcu:2019:RSP}.
\tref{tab:recon} lists the quantitative reconstruction quality
of different trajectories.
The image quality metric is the average peak signal-to-noise ratio (PSNR) of the test set.
SNOPY led to
$\sim$4 dB higher PSNR than
the kooshball initialization.
\fref{fig:recon} includes examples of
reconstructed images.
Compared to kooshball, SNOPY's reconstructed images
have fewer artifacts and blurring.
Though MoDL reconstruction (and its variants) is one of the best reconstruction algorithms
based on the open fastMRI reconstruction challenge \cite{muckley2021results}, 
many important structures are misplaced with the kooshball reconstruction.
Using the SNOPY-optimized trajectory,
even a simple model-based reconstruction (CG-SENSE)
can reconstruct these structures.

For experiment \ref{exp:sos},
\fref{fig:sos} shows the PSF of the optimized angle 
and RSOS-GR angle scheme
\cite{zhou:2017:GoldenratioRotatedStackofstars}.
For the in-plane ($x$-$y$) PSF,
the SNOPY rotation shows noticeably reduced streak-like patterns.
In the $y$-$z$ direction, SNOPY optimization leads to 
a narrower central lobe
and suppressed aliasing.
The prospective in-vivo experiments also support this theoretical finding.
In \fref{fig:sos},
the example slices
(reconstructed by PLS) from prospective studies
show that SNOPY
reduces streaking artifacts
and blurring.
The average PSNR of SNOPY and RSOS-GR for the 4 participants 
were 39.23 dB and 37.84 dB, respectively.


\begin{figure*}
\centerline{\includegraphics[width=0.9\textwidth]{figure/fig,pns.png}}
\caption{~The first row plots the PNS effect
calculated by the convolution model (\ref{eqn:pns}) of the experiment \ref{exp:pns}.
The second row shows one readout trajectory
before/after the SNOPY optimization.\label{fig:pns}}
\end{figure*}


In experiment \ref{exp:pns},
we tested three settings:
unoptimized REPI,
optimized with PNS threshold (\pmax in \eqref{eqn:pns}) = 80\%,
and
optimized with \pmax= 70\%.
\fref{fig:pns} shows one shot before/after the optimization,
and a plot of simulated PNS effects.
For the subjective rating 
of PNS,
the first participant reported 
5,2,1;
the second participant reported
4,3,2;
the third participant reported 5, 4, 3.
The SNOPY optimization
effectively reduced the subjective PNS effect
of the given REPI readout
in both simulation and in-vivo experiments.
Intuitively, SNOPY smooths the trajectory
to avoid a constantly high slew rate,
preventing the high PNS effect.
\fref{fig:ossi} shows one slice of
reconstructed images by the CS-SENSE algorithm.
Though SNOPY suppressed the PNS effect,
the image contrast was well preserved
by the image contrast regularizer \eqref{eqn:contrast}.
