

\section{Conclusions}
We propose MALinZero, which leverages low-dimensional representational
structures to enable efficient MCTS in complex multi-agent planning. MALinZero can be viewed as projecting the joint-action returns into the low-dimensional space representable using a contextual linear bandit problem formulation, with a convex and $\mu$-smooth loss to place more importance on better actions. We employ an $(1-\tfrac1e)$-approximation algorithm for the joint action selection by maximizing a submodular objective. MALinZero demonstrates state-of-the-art performance on multi-agent benchmarks such as MatGame, SMAC, and SMACv2, outperforming MARL and MCTS baselines. 

{\bf Limitations:} MALinZero leverages a contextual linear bandit formulation in the low-dimensional space. The use of non-linear formulations that may also allow efficient MCTS could further improve the performance. Developing fully decomposable representations also remains an open problem.