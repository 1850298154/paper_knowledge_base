\documentclass{article}


% if you need to pass options to natbib, use, e.g.:
    \PassOptionsToPackage{numbers, compress}{natbib}
% before loading neurips_2024


% ready for submission
% \usepackage{neurips_2024}


% to compile a preprint version, e.g., for submission to arXiv, add add the
% [preprint] option:
    % \usepackage[preprint]{neurips_2024}


% to compile a camera-ready version, add the [final] option, e.g.:
    \usepackage[final]{neurips_2024}


% to avoid loading the natbib package, add option nonatbib:
   % \usepackage[nonatbib]{neurips_2024}

\usepackage[utf8]{inputenc} % allow utf-8 input
\usepackage[T1]{fontenc}    % use 8-bit T1 fonts
% \usepackage{hyperref}       % hyperlinks
\usepackage[pagebackref=true,breaklinks=true,letterpaper=true,colorlinks,bookmarks=false,citecolor=blue]{hyperref}
\usepackage{url}            % simple URL typesetting
\usepackage{booktabs}       % professional-quality tables
\usepackage{amsfonts}       % blackboard math symbols
\usepackage{nicefrac}       % compact symbols for 1/2, etc.
\usepackage{microtype}      % microtypography
\usepackage{xcolor}         % colors
\usepackage{bm}
% \usepackage[normalem]{ulem}
\usepackage{multirow}
\usepackage{graphicx}
\usepackage{amssymb}
\usepackage{amsmath}
\usepackage{bbding}
\usepackage{pifont}
\usepackage{utfsym}
\usepackage{wrapfig}
\def\UrlBreaks{\do\A\do\B\do\C\do\D\do\E\do\F\do\G\do\H\do\I\do\J
\do\K\do\L\do\M\do\N\do\O\do\P\do\Q\do\R\do\S\do\T\do\U\do\V
\do\W\do\X\do\Y\do\Z\do\[\do\\\do\]\do\^\do\_\do\`\do\a\do\b
\do\c\do\d\do\e\do\f\do\g\do\h\do\i\do\j\do\k\do\l\do\m\do\n
\do\o\do\p\do\q\do\r\do\s\do\t\do\u\do\v\do\w\do\x\do\y\do\z
\do\.\do\@\do\\\do\/\do\!\do\_\do\|\do\;\do\>\do\]\do\)\do\,
\do\?\do\'\do+\do\=\do\#}

\title{\raisebox{-0.2\height}{\includegraphics[width=0.06\textwidth,keepaspectratio]{log.png}} PediatricsGPT: Large Language Models as Chinese Medical Assistants for Pediatric Applications}



\author{%
  Dingkang Yang$^{1,3 \,\usym{1F396}}$\footnotemark[2]   $\quad$  Jinjie Wei$^{1,3}$\footnotemark[2]
  $\quad$  Dongling Xiao$^{2}$\footnotemark[2] $\quad$ Shunli Wang$^{1}$\footnotemark[4]
  $\quad$ Tong Wu$^{2}$\footnotemark[4] \\ 
  $\quad$ \textbf{Gang Li}$^{2}$\footnotemark[4]  $\quad$ \textbf{Mingcheng Li}$^{1}$\footnotemark[4]
  $\quad$ \textbf{Shuaibing Wang}$^{1}$\footnotemark[4] $\quad$ \textbf{Jiawei Chen}$^{1}$\footnotemark[4] $\quad$ \textbf{Yue Jiang}$^{1}$\footnotemark[4] \\
  $\quad$ \textbf{Qingyao Xu}$^{1}$\footnotemark[4] $\quad$ \textbf{Ke Li}$^{2}$\footnotemark[4] $\quad$ \textbf{Peng Zhai}$^{1,3}$\footnotemark[1]
  $\quad$ \textbf{Lihua Zhang}$^{1,3,4,5}$\footnotemark[1] \\
  \\
 \small $^{1}$Academy for Engineering and Technology, Fudan University, Shanghai, China  \\
 \small $^{2}$Tencent Youtu Lab, Shanghai, China  \\
  \small $^{3}$Cognition and Intelligent Technology Laboratory, Shanghai, China\\
  \small $^{4}$Engineering Research Center of AI and Robotics, Ministry of Education, Shanghai, China\\
  \small $^{5}$AI and Unmanned Systems Engineering Research Center of Jilin Province, Changchun, China\\
 \small \texttt{$\{$dkyang20, pzhai, lihuazhang$\}$@fudan.edu.cn}    \\
  \small \texttt{jjwei23@m.fudan.edu.cn, xdluestc@outlook.com}    \\
    \small \texttt{tristanli@tencent.com}    \\
}



\begin{document}
 

\maketitle

\renewcommand{\thefootnote}{\fnsymbol{footnote}} 
\footnotetext[2]{Equal first contributions. $^{\S}$Equal second contributions.  $^\ast$Corresponding authors. $^{\usym{1F396}}$Project lead.}  

\begin{abstract}
Developing intelligent pediatric consultation systems offers promising prospects for improving diagnostic efficiency, especially in China, where healthcare resources are scarce.
Despite recent advances in Large Language Models (LLMs) for Chinese medicine, their performance is sub-optimal in pediatric applications due to inadequate instruction data and vulnerable training procedures.
To address the above issues, this paper builds PedCorpus, a high-quality dataset of over 300,000 multi-task instructions from pediatric textbooks, guidelines, and knowledge graph resources to fulfil diverse diagnostic demands.
Upon well-designed PedCorpus, we propose PediatricsGPT, the first Chinese pediatric LLM assistant built on a systematic and robust training pipeline.
In the continuous pre-training phase, we introduce a hybrid instruction pre-training mechanism to mitigate the internal-injected knowledge inconsistency of LLMs for medical domain adaptation.
Immediately, the full-parameter Supervised Fine-Tuning (SFT) is utilized to incorporate the general medical knowledge schema into the models.
After that, we devise a direct following preference optimization to enhance the generation of pediatrician-like humanistic responses.
In the parameter-efficient secondary SFT phase,
a mixture of universal-specific experts strategy is presented to resolve the competency conflict between medical generalist and pediatric expertise mastery.
Extensive results based on the metrics, GPT-4, and doctor evaluations on distinct downstream tasks show that PediatricsGPT consistently outperforms previous Chinese medical LLMs.
The project and data will be released at~\url{https://github.com/ydk122024/PediatricsGPT}.

\end{abstract}

\section{Introduction}

As an essential component of medicine, pediatrics plays an indispensable role in ensuring children’s health growth~\cite{elzouki2011textbook,garrison1923history}.
The unbalanced distribution of healthcare resources~\cite{pu2021fairness} has resulted in a massive shortage of pediatricians, especially in populous countries led by China~\cite{qin2014economic,dong2021differences}.
With the rapid advances in LLMs exemplified by ChatGPT~\cite{ChatGPT}, developing intelligent pediatric consultation systems provides promise for enriching medical services.
Although Chinese LLMs~\cite{cui2023efficient,zhang2022fengshenbang,Baichuan,zeng2022glm,du2021glm} have exhibited progress in general language understanding, they are incompetent in the pediatric medical field due to the lack of domain-specific discipline and specialized expertise injection.

To fulfil the interactive demands of Chinese medicine, preliminary efforts~\cite{bao2023disc,wang2023huatuo,xiong2023doctorglm,chen2023bianque} have enhanced LLMs' healthcare mastery through Supervised Fine-Tuning (SFT) training and medically relevant corpus collection.
Despite improvements, challenges remain due to unavoidable dilemmas, including inadequate instruction data and vulnerable training procedures.
Specifically, (i) existing instruction data typically involve vanilla rephrasing of the general medical corpus~\cite{xiong2023doctorglm} or aggregation of doctor-like dialogues~\cite{yang2024zhongjing}, which loses the specialization and focus in pediatric applications.
More importantly, the current straightforward different round instruction construction paradigms~\cite{zhang2023huatuogpt,chen2023bianque} fail to accommodate multi-task healthcare services in real-world scenarios, limiting the model generalization and inducing response hallucination. 
(ii) Furthermore, prior methods mostly relied on SFT to compensate for medical instruction following capabilities, ignoring the discrepancies between inherent and externally absorbed knowledge within the models.
This single pattern causes secondary LLMs to lapse into excessive role-playing rather than understanding~\cite{ramamurthy2022reinforcement}.
Despite a few attempts in the pre-training and Reinforcement Learning from Human Feedback (RLHF) phases~\cite{bai2022training,ouyang2022training}, their performance is restricted by actor-critic instability~\cite{schulman2017proximal} and online sampling bias~\cite{zheng2023secrets}.



Motivated by these observations, we construct PedCorpus, a high-quality dataset with over 300,000 instructions across single-turn and multi-turn medical conversations.
Besides containing generalist healthcare data, PedCorpus incorporates multi-dimensional corpora from pediatric textbooks, guidelines, and knowledge graphs to ensure medical knowledge's accuracy.
Vanilla instructions can also be readily extended to seed instructions for generating specialized corpora to serve different training phases.
Furthermore, we integrate the well-presented GPT-4-distilled data with authentic doctor-patient dialogue data to standardize the fluency and faithfulness of instruction information.

Among our PedCorpus, we propose PediatricsGPT, the first Chinese pediatric LLM assistant with pediatric expertise and medical generalist.
PediatricsGPT is developed on a systematic training pipeline that includes Continuous Pre-Training (CPT), full-parameter SFT, human preference alignment, and parameter-efficient secondary SFT.
In this case, we introduce a hybrid instruction pre-training mechanism in CPT to bridge the capability weakening due to corpus format discrepancies between the internal and injected medical knowledge of foundation models, facilitating knowledge accumulation and extension.
Meanwhile, a Direct Following Preference Optimization (DFPO) in human preference alignment is devised to enhance response robustness and align human preferences.
Additionally, we present a mixture of universal-specific experts strategy to tackle the competency conflict between medical generalist and pediatric expertise in secondary SFT via Low-Rank Adaptation (LoRA)~\cite{hu2021lora}, which strengthens the model's adaptability to distinct downstream tasks.
We conduct three pragmatic pediatric tasks to evaluate the different capabilities of existing models.
Extensive experiments on pediatric and public benchmarks show that our PediatricsGPT outperforms open-source Chinese medical LLMs and baselines, yielding competitive performance compared to GPT-3.5-turbo.



\vspace{-5pt}
\section{Related Work}
\vspace{-5pt}
\textbf{Chinese Large Language Model Evolution.}
The emergence of Large Language Models (LLMs) dominated by ChatGPT~\cite{ChatGPT} and GPT-4~\cite{achiam2023gpt} has revolutionized the paradigm for novel human-machine interaction. Driven by learning-oriented technologies~\cite{chen2024can,chen2024miss,chen2024efficiency,yang2022disentangled,yang2022learning,yang2024robust,wei2024medaide}, pragmatic instruction~\cite{mishra2021natural,wei2021finetuned} and preference optimization~\cite{bai2022training,ouyang2022training} strategies enable LLMs to address complex generation tasks with aligned human intentions.
Despite improvements, large-scale resources for training general LLMs~\cite{jiang2023mistral,touvron2023llama,touvron2023llama-2} are anchored in the English corpora, limiting their abilities to respond reliably in extensive Chinese application scenarios.
Recently, researchers~\cite{cui2023efficient,zhang2022fengshenbang} have attempted to enhance the comprehension and execution of Chinese instructions in open-source LLMs by augmenting Chinese vocabulary and data (\textit{e.g.,} Chinese LLaMA and Alpaca~\cite{cui2023efficient}).
To facilitate Chinese-specific demands, several LLMs trained from scratch exhibit remarkable Chinese proficiency due to multilingual data resources, such as the Baichuan~\cite{Baichuan,yang2023baichuan}, General Language Model (GLM)~\cite{du2021glm,zeng2022glm}, and Qwen~\cite{bai2023qwen} families.
In this work, the Baichuan2-Base series is utilized as the foundation model for our PediatricsGPT, given its comprehensive potential among similar contenders.


\textbf{LLMs in Medical Applications.}
Current LLMs provide unprecedented opportunities to develop resource-efficient and diagnostic-comprehensive intelligent healthcare systems.
Despite universal models~\cite{achiam2023gpt,ChatGPT} equipped with certain internal knowledge regarding biomedicine, they are incompetent in real-world medical applications due to the absence of domain-specific disciplines.
% and specialized expertise injection.
In this context, several efforts~\cite{wang2023huatuo,xiong2023doctorglm,chen2023bianque,li2023chatdoctor} attempt to construct medically tailored LLMs from multiple perspectives.
For instance, ChatDoctor~\cite{li2023chatdoctor} uses patient-doctor conversation data based on LLaMA~\cite{touvron2023llama} to enhance the language model's accuracy in healthcare. DoctorGLM~\cite{xiong2023doctorglm} proves that a healthcare-purpose LLM can be implemented with affordable overhead by fine-tuning ChatGLM-6B~\cite{du2021glm}.
After that, more Chinese medical LLMs~\cite{MedicalGPT,bao2023disc,zhang2023huatuogpt,chen2023huatuogpt2,yang2024zhongjing} are progressively presented to generate doctor-like robust responses, such as HuatuoGPT~\cite{zhang2023huatuogpt}, DISC-MedLLM~\cite{bao2023disc}, and Zhongjing~\cite{yang2024zhongjing}.
Despite advances in general medical knowledge, current models are suboptimal for pressing pediatric applications. In comparison, our sophisticated training procedure and high-quality instruction datasets inject new insights and prospects for developing specialized LLMs with pediatric expertise.



\begin{table}[t]
% \setlength{\tabcolsep}{8pt}
\setlength{\tabcolsep}{3pt}
% \renewcommand{\arraystretch}{0.9}
\centering
\caption{Statistical information on the proposed dataset. PedCorpus is well extensible and adaptable by incorporating general domain data and as seed instructions to generate specialized corpora (\textit{i.e.,} PedCorpus-CPT and PedCorpus-DFPO). ``KG'' means the Knowledge Graphs.}
\vspace{-2pt}
\resizebox{\linewidth}{!}{%
\begin{tabular}{cccccccc}
\toprule
\multirow{2}{*}{Dataset}       & \multirow{2}{*}{Data Sources}     & \multirow{2}{*}{Department} & \multirow{2}{*}{Number/Size} & \multirow{2}{*}{\begin{tabular}[c]{@{}c@{}}Human \\ Preference\end{tabular}} & \multicolumn{3}{c}{Task Type} \\ \cline{6-8} 
                               &                                   &                             &                              &                                                                              &\rule{0pt}{10pt} MedKQ\&A & EviDiag & TreRecom \\ \midrule
\multirow{5}{*}{PedCorpus}     & Pediatric Textbooks               & Pediatrics                  & 37,284                       & \ding{52}                                                                          & \ding{52}      & --     & \ding{52}      \\
                               & Pediatric Guidelines              & Pediatrics                  & 63,129                       & \ding{52}                                                                          & \ding{52}       & --     & \ding{52}      \\
                               & Pediatric KG                      & Pediatrics                  & 46,320                       & \ding{52}                                                                          & \ding{52}      & --      & \ding{52}      \\
                               & Real Doctor-Patient Conversations & Multiple                    & 46,385                       & \ding{52}                                                                          & --      & \ding{52}     & \ding{52}      \\
                               & Distilled Medical Datasets        & Multiple                    & 107,177                      & --                                                                          & \ding{52}      & \ding{52}      & \ding{52}      \\ \midrule
\multirow{3}{*}{PedCorpus-CPT} & Plain Textbooks, Guidelines, KG      & Multiple                    & \multirow{3}{*}{975.8MB}     & --                                                                          & \ding{52}       & \ding{52}     & \ding{52}      \\
                               &  Filtered Chinese Wikipedia   & Multiple                    &                              & --                                                                          & --      & --      & --      \\
                               & Extended data from PedCorpus                 & Multiple                    &                              & --                                                                          & --      & --     & --      \\ \midrule
PedCorpus-DFPO                  & Pediatrics data from PedCorpus       & Pediatrics                  & 15,556                       & \ding{52}                                                                          & \ding{52}       & \ding{52}     & \ding{52}      \\ \toprule
\end{tabular}
}
\label{tab1}
\vspace{-12pt}
\end{table}



\vspace{-5pt}
\section{Methodology}
\vspace{-5pt}
This section describes the proposed PedCorpus dataset and the sequential pipeline for developing PediatricsGPT. Figure~\ref{arc} illustrates the comprehensive method workflow.

\vspace{-3pt}
\subsection{PedCorpus: Multi-task Medical Instruction Dataset}
\vspace{-3pt}
To endow the model with versatile diagnostic proficiency, PedCorpus is constructed through the multi-dimensional corpus across three application-oriented medical tasks, including Knowledge Question-Answer (MedKQ\&A), Evidence-based Diagnosis (EviDiag), and Treatment Recommendation (TreRecom). Table~\ref{tab1} shows the detailed statistical information from different data sources.
We explain the three patterns of PedCorpus construction below.

\textbf{Specialized Pediatric Data.}
Extracting pediatric data from textbooks, guidelines, and knowledge graphs ensures knowledge professionalism.
Specifically, we automatically extract standard medical definitions and descriptions from physical textbooks covering 131 disease types in 11 broad categories.
Over 500 corresponding disease guidelines are collected, including diagnostic protocols and treatment consensus. Additionally, extensive knowledge entities are sampled from ternary instances in the knowledge graphs.
Based on these resources, we introduce a role-playing-driven instruction building rule via GPT-4 API that produces well-organized instructions to enable \textbf{accurate} and \textbf{humanistic} model responses. 
The detailed building procedure is shown in Appendix~\ref{app_sec:1_1}.

\textbf{Real Doctor-patient Conversations.}
To avoid the model collapse dilemma~\cite{shumailov2023curse}, we incorporate authentic doctor-patient dialogues from online treatment platforms and voice transcriptions during medical consultations.
The single-/multi-turn instructions are jointly considered to equip the model with healthcare interrogation and contextual understanding.
Original responses from real doctors are usually terse and noisy, potentially worsening the generation quality~\cite{zhang2023huatuogpt}.
To this end, we craft 100 high-quality examples to guide the advanced language model by the in-context learning to regularize vanilla conversations in the self-instruct pattern~\cite{chen2023phoenix,wang2022self}. This approach ensures \textbf{doctor-like} and \textbf{patient-friendly} model responses.
More regularization details are shown in Appendix~\ref{app_sec:1_2}.


\textbf{Distilled Medical Datasets.}
Integrating general medical knowledge from existing datasets~\cite{li2023huatuo,he2020meddialog,zhang2018multi} is a common practice in previous efforts~\cite{chen2023bianque,xiong2023doctorglm,MedicalGPT,bao2023disc}.
However, we find numerous unclear and incomplete representations in the instruction instances from public benchmarks due to the absence of careful calibration, potentially triggering hallucinated outputs.
Consequently, we manually sample 107,177 knowledge-intensive instructions from three mainstream benchmarks (\textit{i.e.,} Huatuo-26M~\cite{li2023huatuo}, MedDialog~\cite{he2020meddialog}, and CMeKG~\cite{byambasuren2019preliminary}), adhering to the philosophy of quality over quantity~\cite{zhou2024lima}.
After that, a progressive instruction reconstruction rule is proposed to distill the sampled instructions to ensure \textbf{informative} and \textbf{logical} model responses.
The rule process can be found in Appendix~\ref{app_sec:1_3}.

\vspace{-3pt}
\subsection{Hybrid Instruction Pre-training in CPT}
\vspace{-3pt}
Continuous Pre-Training (CPT) is essential in developing domain-specific models~\cite{chen2023huatuogpt2,wen2023chathome,yang2024zhongjing} since it can break the scaling law~\cite{gunasekar2023textbooks} to a certain extent.
For this purpose, we introduce the PedCorpus-CPT dataset to ensure a high-quality pre-training corpus.
From Table~\ref{tab1}, PedCorpus-CPT consists of three-part data components.
(i) We integrate plain texts from vanilla pediatric textbooks, guidelines, and knowledge graphs. 
(ii) The filtered Chinese Wikipedia~\cite{Wikipedia} is also considered to achieve the model's trade-off for medical-general knowledge memory capacity.
(iii) In practice, we observe that CPT leads to catastrophic forgetting of the models at follow-up due to different data distribution and format discrepancies compared to the original pre-training and SFT.
Thus, we introduce a hybrid instruction pre-training mechanism to bridge these discrepancies. 
The core philosophy is to assemble instruction data from PedCorpus with {\tt Input-Output} forms into {\tt Completion} forms, which are then assimilated into plain texts to provide multi-task and complementary information.
This mechanism effectively mitigates inconsistencies between the internal-injected medical knowledge of the foundation model while reinforcing medical domain adaptation.
Moreover, we take PedCorpus as the seed instructions to improve multiple-department corpus density and breadth via knowledge-enhanced prompts.
The prompt template is shown in Appendix~\ref{app_sec:2}. 
 
We pre-train the foundation model to follow the causal language modelling paradigm. Given any input token sequence $\bm{t} = (t_0, t_1, t_2, ...) \in \mathcal{D}_{cpt}$ from the above multi-channel corpus $\mathcal{D}_{cpt}$, the next token $t_i$ is autoregressively predicted by minimizing the negative log-likelihood:
\begin{equation}
\mathcal{L}_{\mathrm{CPT}}(\theta,\mathcal{D}_{cpt}) = \mathbb{E}_{\bm{t}\sim \mathcal{D}_{cpt}}
\left [ -  {\textstyle \sum_{i}^{\left | \bm{t} \right |}}  \mathrm{log}\,p(t_{i} \mid t_0, t_1, ..., t_{i-1};\theta )  \right ],
\end{equation}
where $\theta$ is the model parameter and the input context consists of $t_0, t_1, ..., t_{i-1}$.

\vspace{-3pt}
\subsection{Full-parameter Supervised Fine-tuning}
\vspace{-3pt}


 \begin{figure}[t]
  \centering
  \includegraphics[width=\linewidth]{1.pdf}
  \caption{The sequential pipeline for developing PediatricsGPT. We begin by injecting intensive medical and world knowledge into the foundation model through the hybrid instruction mechanism in CPT phase.
  Then, full-parameter SFT is implemented to improve the model's instruction-following capabilities regarding medical generalists. After that, we introduce the direct following preference optimization to control the model behaviour to align with human preference.
  In the parameter-efficient SFT phase, the LoRA-based mixture of universal-specific experts is devised to mitigate conflicts across downstream tasks and competition between pediatric expertise and general mastery.}
  \label{arc}
  \vspace{-0.35cm}
\end{figure}

During this phase, we activate the model's ability to follow medical instructions by the Full-parameter Supervised Fine-tuning (FSFT).
The full-parameter pattern enables a fuller invocation of the intensive knowledge in CPT and promotes comprehension and logical reasoning about diverse structured instructions.
The training data at this phase is composed of the following three aspects.
(i) We utilize the multi-department medical data in the PedCorpus dataset to develop the medical generalist.
(ii)
Chinese instruction data (\textit{i.e.,} Alpaca dataset~\cite{peng2023instruction} and ShareGPT~\cite{ShareGPT}) from general domains are selectively integrated to avoid the potential overfitting risk.
(iii)
Providing safety measures is vital for LLM assistants yet overlooked by prior methods~\cite{zhou2024lima}.
In contrast, we write 200 training instructions with some degree of maliciousness, hallucinations, and counterfactuals. Correspondingly, the refusal responses with detailed explanations for disobedience are carefully crafted.
We also include 300 examples related to self-cognition content. These data significantly improve the robustness and security of the model against unfriendly commands.


Given any input instruction $\bm{x} = (x_0, x_1, x_2, ...) \in \mathcal{D}_{fsft}$ and corresponding target response $\bm{y} = (y_0, y_1, y_2, ...) \in \mathcal{D}_{fsft}$ from the above-integrated fine-tuning dataset $\mathcal{D}_{fsft}$, 
the optimization objective can be formulated as follows:
\begin{equation}
\mathcal{L}_{\mathrm{FSFT}}(\theta,\mathcal{D}_{fsft}) = \mathbb{E}_{(\bm{x},\bm{y})\sim \mathcal{D}_{fsft}}
\left [ -  {\textstyle \sum_{i=1}^{\left | \bm{y} \right | }}  \mathrm{log}\,p(y_{i} \mid \bm{x}, y_{<i}; \theta)  \right ].
\end{equation}

\vspace{-3pt}
\subsection{Direct Following Preference Optimization}
\vspace{-3pt}
Aligning human intention preferences facilitates the model to generate harmless responses.
To this end, we introduce PedCorpus-DFPO $\mathcal{D}_{dfpo}$, a preference dataset to guide the model in learning human preference behaviours.
PedCorpus-DFPO contains the input instruction set $\bm{x} = (x_0, x_1, x_2, ...) \in \mathcal{D}_{dfpo}$, which is selectively sampled from vanilla PedCorpus.
On the one hand, we perform a humanistic stylistic rephrasing of the outputs to generate preferred responses $\bm{y}^w = (y^w_0, y^w_1, y^w_2, ...) \in \mathcal{D}_{dfpo}$.
On the other hand, the corresponding low responses $\bm{y}^l = (y^l_0, y^l_1, y^l_2, ...) \in \mathcal{D}_{dfpo}$ are generated from the feedback of a low-capability medical assistant~\cite{wang2023huatuo} to maintain domain consistency.

Despite impressive improvements achieved by RLHF-based approaches~\cite{yang2024zhongjing,zhang2023huatuogpt}, challenges remain due to unstable reward modelling and significant computational costs~\cite{schulman2017proximal,zheng2023secrets}.
Inspired by single-stage preference learning~\cite{rafailov2024direct}, we propose a stable and lightweight method for domain-specific LLMs called Direct Following Preference Optimization (DFPO).
DFPO utilizes variable changes to formulate the preference loss as a policy function that efficiently optimizes the policy with a simple binary cross-entropy objective. 
Meanwhile, our method directly regularizes model behaviour boundaries in an instruction-following paradigm on medical demonstrations of preferred responses,  facilitating robustness and smoothing of the preference learning.

Theoretically, the observed probability of a particular preference pair usually follows the Bradley-Terry model~\cite{bradley1952rank}, and $\bm{y}^{w}$ is preferred over $\bm{y}^{l}$ (denoted $\bm{y}^w \succ \bm{y}^l$):
\begin{equation}
p(\bm{y}^w \succ \bm{y}^l) = \sigma(\gamma (\bm{x}, \bm{y}^w) - \gamma (\bm{x}, \bm{y}^l)),
\end{equation}
where $ \gamma (\bm{x},\bm{y}^{w/l})$ means the parameterized reward function and $\sigma(\cdot)$ is the sigmoid activation.
In this case, the overall optimization objective is expressed as:
\begin{equation}
\mathcal{L}_{\mathrm{DFPO}}(\theta,\mathcal{D}_{dfpo}) = 
-\mathbb{E}_{(\bm{x}, \bm{y}^{w}, \bm{y}^{l})\sim \mathcal{D}_{dfpo}}
\left [ 
\mathrm{log}\,\sigma (\beta\, \mathrm{log} \frac{\pi_{\theta}(\bm{y}^{w}\mid \bm{x}) }{\pi_{r}(\bm{y}^{w}\mid \bm{x})} - \beta\, \mathrm{log} \frac{\pi_{\theta}(\bm{y}^{l}\mid \bm{x}) }{\pi_{r}(\bm{y}^{l}\mid \bm{x})})
\right ] + \mu \Phi(\bm{x}, \bm{y}^w),
\end{equation}
where $\pi_{\theta}$ and $\pi_{r}$ are the desired optimal policy and the reference policy, respectively.
$\beta$ is the control parameter reflecting the deviation from the basic $\pi_{r}$.
For the fine-tuning regularization term $\Phi(\bm{x}, \bm{y}^w)$ with the scaling coefficient $\mu$, the implementation process is equivalent to maximizing the log probability $p(\bm{y}^w \mid \bm{x})$ regarding the preferred responses $\bm{y}^w$ given the input instructions $\bm{x}$:
\begin{equation}
\Phi(\bm{x}, \bm{y}^w) = \mathbb{E}_{(\bm{x},\bm{y}^w)\sim \mathcal{D}_{dfpo}}
\left [ -  {\textstyle \sum_{i=1}^{\left | \bm{y}^w \right | }}  \mathrm{log}\,p(y_{i}^w \mid \bm{x}, y_{<i}^w; \theta)  \right ].
\end{equation}



\subsection{Mixture of Universal-specific Experts in Parameter-efficient SFT}
This phase aims to reinforce the model performance for various pediatric applications through the LoRA-based Parameter-efficient SFT (PSFT). The used dataset $\mathcal{D}_{psft}$ is derived from the pediatric department in PedCorpus and partial general medical/world data.
In practice, we observe that competition across different pediatric tasks and the conflicts between medical generalization and specialized knowledge deteriorate instruction-following abilities.
Accordingly, we propose a mixture of universal-specific experts strategy to address these challenges.
Formally, LoRA adapters~\cite{hu2021lora} act as experts to replace the linear layers in the Feed-Forward Neural (FFN) networks of LLMs, providing trainable parameters.
Several specific experts $\left \{ E_j^{s} \right \}^{T}_{j=1}$ are assigned adaptive activations to master distinct pediatric expertise through soft routing. The routing gating is defined as follows:
\begin{equation}
G(\bm{x}) = \mathrm{Softmax} (\bm{x}\bm{W}_{g}+ \mathcal{S}(\varphi (\bm{x}\bm{W}_{n})).
\end{equation}
$\bm{W}_{g}$ and $\bm{W}_{n}$ are the learnable weights. $\mathcal{S}(\varphi (\bm{x}\bm{W}_{n})$ is the noise term for regularizing the expert utilization balance, where $\mathcal{S}(\cdot)$ and $\varphi(\cdot)$ represent the Standard Normal distribution sampling and Softplus function, respectively.
Moreover, we consistently activate a universal expert $E^{u}$ across all training data to prevent general knowledge forgetting and mitigate competency conflict.
The parameterized output $\bm{z}$ of all the experts in the forward process can be mathematized as follows:
\begin{equation}
\bm{z} = \frac{\alpha}{r}({\sum_{j=1}^{T}} G(\bm{x})_j E_j^{s}(\bm{x}) + E^{u}(\bm{x})),
\end{equation}
where $r$ is the rank value and $\alpha$ is a hyper-parameter for approximating the learning rate.

\vspace{-6pt}
\section{Experiments}

\begin{table}[t]
\setlength{\tabcolsep}{2pt}
\centering
\caption{Comparison results of different models on three pediatric medical benchmarks. In each benchmark, the best results are marked in \textbf{bold}, and the second-best results are marked \underline{underlined}.}
\vspace{-2pt}
\resizebox{\linewidth}{!}{%
\begin{tabular}{clccccccccccc}
\toprule
\multicolumn{2}{c}{Benchmark}                 & Model             & ROUGE-1        & ROUGE-2        & ROUGE-L        & BLEU-1         & BLEU-2         & BLEU-3         & BLEU-4         & GLEU           & Distinct-1     & Distinct-2     \\ \midrule
\multicolumn{2}{c}{\multirow{9}{*}{MedKQ\&A}}   & Baichuan2-7B      & 40.88          & 19.44          & 21.50           & 26.77          & 20.00             & 17.30           & 14.86          & 24.88          & 20.14          & 39.95          \\
\multicolumn{2}{c}{}                          & Baichuan2-13B     & 46.96          & 22.85          & 22.54          & 29.02          & 25.62          & 22.63          & 19.31          & 27.97          & 21.45          & 42.53          \\
\multicolumn{2}{c}{}                          & HuatuoGPT        & 48.52          & 23.44          & 25.13          & 43.00             & 41.25          & 36.31          & 29.82          & 34.60           & 20.42          & 41.27          \\
\multicolumn{2}{c}{}                          & DISC-MedLLM              & 53.83          & 25.98          & 27.71          & 47.91          & 44.57          & 37.65          & 30.07          & 37.11          & \underline{26.63} & \underline{51.98} \\
\multicolumn{2}{c}{}                          & Zhongjing         & 53.97          & 26.03          & 29.56          & 51.11          & 45.04          & 39.13          & 33.59          & 42.61          & \textbf{26.75} & \textbf{52.66} \\
\multicolumn{2}{c}{}                          & HuatuoGPT-II       & 55.27          & 26.59          & 27.95          & 59.07          & 51.49          & 45.38          & 38.70           & 39.18          & 20.97          & 41.34          \\
\multicolumn{2}{c}{}                          & Meditron-7B & 55.63 & 26.19 & 30.37 & 58.43 & 53.45 & 56.07 & 38.77 & 42.23 & 22.34 & 45.17 \\
\multicolumn{2}{c}{}                          & Llama3.1-8B &  53.18 &  24.74 &  28.26 &  45.07 &  42.45 &  36.57 &  29.73 &  35.63 &  22.74 &  46.52 \\
\multicolumn{2}{c}{}                          & ChatGPT               & 56.92          & 27.87          & 29.05          & 61.58 & 54.37          & 47.97          & 40.77          & 45.15          & 20.76          & 40.19          \\
\multicolumn{2}{c}{}                          & GPT-4 & \underline{58.79} & \underline{33.56} & \underline{32.15} & \textbf{62.53} & \underline{59.14} & 55.26 & 52.39 & 53.72 & 21.79 & 43.26 \\
\multicolumn{2}{c}{}                          &  PediatricsGPT-7B  & 58.08 & 31.78 & 31.11 & 59.41          & 56.88 & \underline{57.47} & \underline{55.34} & \underline{54.41} & 24.33          & 47.41          \\
\multicolumn{2}{c}{}                          & PediatricsGPT-13B & \textbf{60.85} & \textbf{36.56} & \textbf{35.64} & \underline{61.65} & \textbf{63.17} & \textbf{58.96} & \textbf{59.34} & \textbf{57.22} & 24.24          & 46.23          \\ \midrule
\multicolumn{2}{c}{\multirow{9}{*}{EviDiag}}  & Baichuan2-7B      & 26.81          & 7.75           & 11.22          & 15.18          & 11.51          & 9.19           & 6.72           & 13.44          & 23.65          & 46.93          \\
\multicolumn{2}{c}{}                          & Baichuan2-13B     & 39.14          & 12.06          & 12.44          & 47.65          & 36.02          & 28.82          & 21.19          & 28.28          & 25.45          & 50.43          \\
\multicolumn{2}{c}{}                          & HuatuoGPT        & 35.12          & 10.77          & 15.04          & 46.22          & 33.10           & 25.44          & 21.22          & 25.44          & 22.30           & 45.73          \\
\multicolumn{2}{c}{}                          & DISC-MedLLM              & 33.55          & 11.67          & 15.32          & 15.91          & 12.46          & 10.27          & 7.96           & 16.77          & \textbf{35.89} & \textbf{69.36} \\
\multicolumn{2}{c}{}                          & Zhongjing         & 40.92          & 14.26          & 17.41          & 48.64          & 37.52          & 30.17          & 22.44          & 27.03          & \underline{33.40}  & \underline{65.89} \\
\multicolumn{2}{c}{}                          & HuatuoGPT-II       & 39.52          & 12.14          & 16.38          & 49.58          & 37.62          & 30.66          & 23.34          & 28.98          & 21.97          & 43.62          \\
\multicolumn{2}{c}{}                          & Meditron-7B & 42.63 & 15.12 & 18.94 & 52.36 & 39.24 & 37.78 & 27.15 & 31.25 & 22.07 & 45.13 \\
\multicolumn{2}{c}{}                          & Llama3.1-8B &  37.25 &  13.07 &  16.23 &  44.54 &  32.29 &  23.72 &  20.12 &  23.57 &  24.43 &  45.67 \\
\multicolumn{2}{c}{}                          & ChatGPT               & 40.88          & 13.42          & 16.97          & 48.84          & 37.69          & 30.55          & 23.17          & 29.02          & 23.49          & 46.54          \\ 
\multicolumn{2}{c}{}                          & GPT-4 & \textbf{48.48} & \underline{16.74} & \underline{21.51} & \underline{57.59} & \underline{44.78} & \underline{37.94} & \underline{30.56} & \underline{36.79} & 25.69 & 50.13 \\
\multicolumn{2}{c}{}                          &  PediatricsGPT-7B  & 45.83 & 16.60  & 19.91 & 54.37 & 41.99 & 37.59 & 29.03 & 33.42 & 23.49          & 46.61          \\
\multicolumn{2}{c}{}                          & PediatricsGPT-13B & \underline{47.32} & \textbf{17.63} & \textbf{21.87} & \textbf{58.21} & \textbf{45.72} & \textbf{39.74} & \textbf{31.25} & \textbf{37.15} & 23.34          & 46.34          \\ \midrule
\multicolumn{2}{c}{\multirow{9}{*}{TreRecom}} & Baichuan2-7B      & 48.39          & 23.07          & 26.35          & 47.94          & 40.91          & 35.54          & 29.69          & 35.06          & 21.90           & 43.57          \\
\multicolumn{2}{c}{}                          & Baichuan2-13B     & 48.87          & 23.41          & 26.42          & 49.96          & 46.24          & 42.84          & 35.04          & 35.63          & 22.36          & 45.12          \\
\multicolumn{2}{c}{}                          & HuatuoGPT        & 53.48          & 25.41          & 27.08          & 58.14          & 49.64          & 42.93          & 35.16          & 41.63          & 23.26          & 46.21          \\
\multicolumn{2}{c}{}                          & DISC-MedLLM              & 52.77          & 24.26          & 28.89          & 58.73          & 50.05          & 42.96          & 35.59          & 42.44          & 24.30           & 51.95          \\
\multicolumn{2}{c}{}                          & Zhongjing         & 54.92          & 26.63          & 29.68          & 60.12          & 53.31          & 44.25          & 38.76          & 40.38          & 26.18          & 53.94          \\
\multicolumn{2}{c}{}                          & HuatuoGPT-II       & 58.44          & 30.47          & 32.02          & 59.91          & 54.26          & 45.73          & 38.92           & 42.28          & 28.88          & 57.15 \\
\multicolumn{2}{c}{}                          & Meditron-7B & 58.56 & 32.25 & 33.37 & 60.47 & 55.36 & 48.73 & 42.18 & 46.73 & 28.51 & \underline{57.45} \\
\multicolumn{2}{c}{}                          & Llama3.1-8B &  52.45 &  24.98 &  26.14 &  57.56 &  48.11 &  41.67 &  24.03 &  40.67 &  22.73 &  45.13 \\
\multicolumn{2}{c}{}                          & ChatGPT           & 59.59 & 33.34 & 35.79 & 62.81 & 55.79 & 49.85 & 43.29 & 47.59 & \underline{31.09} & 56.87          \\ 
\multicolumn{2}{c}{}                          & GPT-4 & \underline{61.94} & \underline{37.27} & \underline{36.73} & \underline{63.23} & \underline{56.24} & \underline{50.58} & \underline{44.07} & \textbf{55.26} & 30.27 & 56.48 \\
\multicolumn{2}{c}{}                          &  PediatricsGPT-7B  & 56.92          & 29.13          & 31.26          & 61.36          & 55.34          & 46.44          & 40.61          & 44.65          & 26.06          & 52.77          \\
\multicolumn{2}{c}{}                          & PediatricsGPT-13B & \textbf{62.83} & \textbf{39.32} & \textbf{40.82} & \textbf{63.56} & \textbf{56.68} & \textbf{50.80}  & \textbf{44.31} & \underline{54.65} & \textbf{31.94} & \textbf{57.56} \\ \bottomrule
\end{tabular}
}
\label{tab2}
\vspace{-6pt}
\end{table}




\vspace{-3pt}
\subsection{Datasets and Implementation Details}
\vspace{-3pt}
Extensive experiments are conducted on three application-oriented benchmarks to assess the model's pediatric medical abilities, including Knowledge Question-Answer (\textbf{MedKQ\&A}), Evidence-based Diagnosis (\textbf{EviDiag}), and Treatment Recommendation (\textbf{TreRecom}).
Each benchmark contains 300 held-out samples to reject data leakage during training.
In addition, we select two publicly available Chinese medical benchmarks to validate the model's generalizability in general healthcare.
Specifically, we sample 50 challenging instances of diagnostic queries from each department from the \textbf{webMedQA}~\cite{he2019applying} and \textbf{CMD}~\cite{CMD} benchmarks, respectively, leading to testing sets with 300 samples.

Our PediatricsGPT is developed upon the Baichuan2-Base~\cite{yang2023baichuan} models in two versions with 7 and 13 billion parameters.
The model training is accomplished through the PyTorch platform with Accelerate and DeepSpeed packages using eight Nvidia A800 GPUs.
The ZeRO strategy~\cite{rajbhandari2020zero} is employed to alleviate the memory overhead during full parameter training.
The AdamW optimizer~\cite{loshchilov2017decoupled} is adopted for network optimization, and the bf16 data accuracy is chosen. More detailed hyper-parameter configurations for different stages are shown in Appendix~\ref{app_sec:3}.


\vspace{-3pt}
\subsection{Model Zoo}
\vspace{-3pt}
We compare a series of LLMs for comprehensive evaluations. Concretely, \textbf{Baichuan2-7B/13B (Chat)} models~\cite{yang2023baichuan} are trained on 2.6 trillion tokens as the baselines, which have excellent abilities in different domains. 
\textbf{Meditron-7B}~\cite{chen2023meditron} is a 7 billion parameters model adapted to the medical domain from Llama2-7B through continued pre-training on a comprehensively curated medical corpus.
\textbf{Llama3.1-8B}~\cite{dubey2024llama} is a robust multilingual large language model through systematic training.
For reproducible Chinese medical works, \textbf{DISC-MedLLM (13B)}~\cite{bao2023disc} is fine-tuned through reconstructed medical dialogues and behavioural preference instructions.
\textbf{HuatuoGPT (13B)}~\cite{zhang2023huatuogpt} performs SFT based on mixed instruction data and introduce human feedback in RLHF.
\textbf{HuatuoGPT-II (13B)}~\cite{chen2023huatuogpt2} enhances the medical-specific domain adaptation of LLMs through one-stage unified training.
\textbf{Zhongjing (13B)}~\cite{yang2024zhongjing} implements a complete pipeline based on Ziya-LLaMA-13B to enhance the model's multi-turn medical conversation abilities.
\textbf{ChatGPT}~\cite{ChatGPT} and \textbf{GPT-4}~\cite{achiam2023gpt} have impressive performance in general medical fields as closed-source models developed by OpenAI.


\vspace{-3pt}
\subsection{Comparison with State-of-the-art Methods}
\vspace{-3pt}

 \begin{figure}[t]
  \centering
  \includegraphics[width=\linewidth]{gpt_eval_modify.pdf}
  \caption{Response comparisons of PediatricsGPT-13B with other baselines via GPT-4 evaluation.}
  \label{gpt_eval}
  % \vspace{-0.2cm}
\end{figure}

 \begin{figure}[t]
  \centering
  \includegraphics[width=\linewidth]{doctor_eval_modify.pdf}
  \caption{Response comparisons of PediatricsGPT-13B with other baselines via Doctor evaluation.}
  \label{doctor_eval}
  % \vspace{-0.2cm}
\end{figure}




\textbf{Metrics-based Evaluation.}
In Table~\ref{tab2}, we present the comparison results of different models on three pediatric benchmarks through multifaceted metrics, including ROUGE-1/2/L, BLEU-1/2/3/4, GLEU, and Distinct-1/2.
The key observations are listed below.
(i) PediatricsGPT-13B significantly outperforms the baselines and SOTA medical models on the vast majority of metrics across all benchmarks, demonstrating excellent pediatric expertise.
(ii) Our 7B version also achieves competitive results compared to the 13B models. For instance, PediatricsGPT-7B yields absolute improvements of 3.53\% and 4.44\% on metrics ROUGE-L and GLEU in the EviDiag task compared to HuatuoGPT-II, respectively, generating more accurate and informative content.
(iii) By contrast to Zhongjing and HuatuoGPT-II with massive training corpora, our method confirms that the training data quality outweighs quantity for performance gains.
(iv) The worst results at baselines emphasize that target-oriented fine-tuning is an effective strategy for improving domain-specific abilities.


\textbf{Automated GPT-4 Evaluation.}
Measuring model performance from multiple aspects is essential in the pediatric medical domain. 
To this end, we consider four dimensions to holistically assess response quality, including \textit{usefulness}, \textit{correctness}, \textit{consistency}, and \textit{smoothness}.
Advanced GPT-4~\cite{achiam2023gpt} is prompted to select the winning response between pairwise models based on these dimensions.
The dimension explanations and the prompt template for GPT-4 can be found in Appendix~\ref{app_sec:4}.
(i) As Figure~\ref{gpt_eval} shows, PediatricsGPT-13B wins all LLMs by large margins in the MedKQ\&A task, implying the necessity of implementing the knowledge-intensive CPT.
(ii) The favourable win rates on the TreRecom and EviDiag tasks compared to medical LLMs show the superiority of our model in both single-turn treatment recommendations and multi-turn medical diagnostics. 
For example, our model beats Zhongjing via the 59\% win rate on the EviDiag, which specializes in multi-round consultations.


\textbf{Manual Doctor Evaluation.}
Doctor approval of LLM assistants is a vital step toward realistic applications. 
We invite three doctors (each paid \$300) to determine the winner of pairwise models by the majority voting rule. The evaluation requires simultaneous consideration of the responses' \textit{professionalism}, \textit{factuality}, and \textit{safety}.
(i) Excluding ChatGPT, the dominance of our model in Figure~\ref{doctor_eval} shows the effectiveness of considering safety measure data while incorporating specialized pediatric knowledge.
(ii) The proposed direct following preference optimization makes PediatricsGPT-13B more favoured by human preferences compared to other behavioural alignment efforts~\cite{bao2023disc,yang2024zhongjing,zhang2023huatuogpt}.
(iii) The competitive performance of ChatGPT when human judgments indicate that the scaling law still holds, stemming from the high agreement between its behaviours and human intentions.

 \begin{figure}[t]
 \vspace{-5pt}
  \centering
  \includegraphics[width=\linewidth]{CMD_eval_modify.pdf}
  \caption{Comparison results of different models on the CMD benchmark.}
  \label{cmd_eval}
  % \vspace{-0.2cm}
\end{figure}

 \begin{figure}[t]
  \centering
  \includegraphics[width=\linewidth]{webmedqa_eval_modify.pdf}
  \vspace{-5pt}
  \caption{Comparison results of different models on the webMedQA benchmark.}
  \label{web_eval}
  \vspace{-0.2cm}
\end{figure}

\textbf{Generalization Ability Evaluation.}
We show the GLEU metric-based scores of different models on the Chinese medical benchmarks in Figure~\ref{cmd_eval} for CMD and Figure~\ref{web_eval} for webMedQA. (i) PediatricsGPT-13B achieves impressive results across diverse medical departments (including pediatrics), exhibiting medical generalist and pediatric competency mastery.
(ii) The 7B counterpart similarly outperforms most 13B Chinese medical LLMs and exceeds ChatGPT in some departments. For instance, PediatricsGPT-7B brings relative gains of 18.8\% and 7.1\% compared to ChatGPT in the Gynecology and Oncology tasks on the CMD benchmark.
These findings confirm the robust generalization of our model and its ability to capture the multifaceted medical dialogue distributions.

\begin{table}[t]
\setlength{\tabcolsep}{3pt}
\centering
\caption{Ablation study results on five medical benchmarks. ``w/'' and ``w/o'' are short for with and without, respectively. ``MUE'' means the Mixture of Universal-specific Experts strategy.}
\vspace{-2pt}
\resizebox{\linewidth}{!}{%
\begin{tabular}{ccccccccccc}
\toprule
\multirow{2}{*}{Components}                      & \multicolumn{2}{c}{MedKQ\&A} & \multicolumn{2}{c}{EviDiag} & \multicolumn{2}{c}{TreRecom} & \multicolumn{2}{c}{CMD}   & \multicolumn{2}{c}{webMedQA} \\ \cline{2-11} 
                                            & \rule{0pt}{10pt}GPT-4          & Doctor      & GPT-4          & Doctor       & GPT-4           & Doctor       & GPT-4         & Doctor      & GPT-4           & Doctor       \\ \midrule
\textbf{Full Model}                                  & \textbf{68}\%  & \textbf{56}\% & \textbf{54}\%  & \textbf{44}\%  & \textbf{58}\%   & \textbf{38}\%  & \textbf{46}\% & \textbf{40}\% & \textbf{53}\%   & \textbf{45}\%  \\ \midrule
\multicolumn{11}{c}{Importance of Continuous Pre-training}                                                                                                                                       \\ \midrule
w/o Continuous Pre-training                  & 61\%           & 50\%           & 46\%            & 39\%            & 51\%             & 31\%            & 39\%           & 33\%           & 47\%             & 38\%            \\
w/o Hybrid Instruction Pre-training          & 67\%            & 54\%           & 52\%            & 43\%            & 57\%             & 36\%            & 44\%           & 38\%          & 52\%             & 43\%            \\ \midrule
\multicolumn{11}{c}{Necessity of Supervised Fine-tuning}                                                                                                                                         \\ \midrule
w/o Full-parameter SFT                       & 65\%            & 53\%           & 50\%           & 42\%            & 55\%             & 36\%            & 42\%           & 36\%           & 49\%             & 41\%            \\
w/o Parameter-efficient SFT                  & 63\%            & 51\%           & 49\%            & 40\%            & 53\%             & 34\%            & 45\%           & 39\%           & 51\%             & 43\%            \\
w/o MUE Strategy                                      & 67\%   & \textbf{57}\% & 52\%            & 43\%            & 56\%             & 36\%            & 43\%           & 37\%           & 50\%             & 42\%            \\
w/o Universal Expert                         & 67\%            & \textbf{56}\% & 53\%            & \textbf{44}\%  & 57\%             & 37\%            & 44\%           & 38\%           & 51\%             & 42\%            \\ \midrule
\multicolumn{11}{c}{Effectiveness of Preference Alignment}                                                                                                                                       \\ \midrule
w/o DFPO & 67\%            & 53\%           & 52\%            & 41\%            & 57\%             & 36\%            & 45\%           & 38\%          & 51\%             & 41\%            \\
w/ Vanilla DPO            & 66\%            & 55\%           & 53\%            & 42\%            & 57\%             & 36\%            & 44\%           & 38\%           & 52\%             & 42\%            \\
w/  RLHF                                     & 67\%            & 55\%           & \textbf{54}\%  & 43\%            & 57\%            & 37\%            & 45\%           & 39\%           & 52\%             & 44\%            \\ \bottomrule
\end{tabular}
}
\label{tab3}
% \vspace{-5pt}
\end{table}

\subsection{Ablation Studies}
\vspace{-3pt}
We perform thorough ablation studies on five medical benchmarks to investigate the effects of different modelling components. Following~\cite{zhang2023huatuogpt}, we compare the responses from each of the proposed model variants with ChatGPT, and then calculate the win rate (\%) of our model in pairwise responses by GPT-4 and doctor evaluations. Table~\ref{tab3} shows the following observations.



\textbf{Importance of Continuous Pre-training.}
Firstly, we remove the complete continuous pre-training phase to observe performance variations. (i) The significantly deteriorated win rates reveal that injecting specialized knowledge into medical LLMs through rich corpora is indispensable. (ii) Meanwhile, our hybrid instruction pre-training mechanism provides valuable gains to the model.

\begin{wrapfigure}{r}{0.6\textwidth}
\vspace{-12pt}
  \centering
  \setlength{\abovecaptionskip}{0pt}
  \setlength{\belowcaptionskip}{-8pt}
  \includegraphics[width=0.6\textwidth]{abl3.pdf}
  \caption{(a) and (b) show the effect of specific expert numbers on model performance and specific expert utilization in different task data, respectively.}
  \label{vis}
\end{wrapfigure}


\textbf{Necessity of Supervised Fine-tuning.}
(i) We observe consistent performance gaps when removing the Full-parameter SFT (FSFT) and Parameter-efficient SFT (PSFT) phases, respectively. This makes sense since SFTs are necessary to activate the model's healthcare instruction-following capabilities.
(ii) Moreover, PSFT is more critical for three pediatric applications because it facilitates pediatric-related knowledge accumulation, while FSFT focuses on consolidating general medical semantic representations.
(iii) Then, we replace the proposed Mixture of Universal-specific Experts (MUE) version with the vanilla single LoRA.
The reduced performance on pediatric EviDiag and TreRecom benchmarks verifies that it is essential to introduce multiple LoRAs that act as specific experts on different tasks. 
A reasonable explanation is that the single-LoRA model suffers from the task competition between learning the knowledge question-answer and mastering the diagnostic recommendation abilities.
(iv) Furthermore, we find that the universal LoRA expert significantly improves the results on the general medical benchmarks (\textit{i.e.}, CMD and webMedQA), proving that it mitigates the competency conflict between general medical and pediatric knowledge.

\textbf{Effectiveness of Preference Alignment.}
(i) When the Direct Following Preference Optimization (DFPO) phase is removed, the model exhibits significant performance drops in doctor evaluations compared to the full version. This observation proves that DFPO effectively helps the model to align human preferences, reducing harmful content while generating doctor-like output.
(ii) As two candidates, the vanilla DPO and RLHF methods are inferior to the proposed DFPO, suggesting that our strategy can more safely control model behaviour, leading to more favoured humanistic responses.




\vspace{-5pt}
\subsection{Qualitative Analysis of LoRA Experts}
\vspace{-5pt}
\textbf{Effect of Specific Expert Numbers.}
As a complement to the ablation of LoRA experts, Figure~\ref{vis}(a) explores the gain effects of varying the number of specific experts while maintaining the universal expert. (i) Noticeably, our MCE strategy with three specific experts achieves a reasonable performance trade-off across the three tasks with only 0.95\% trainable parameters. (ii) Conversely, excessively introducing LoRA experts does not result in appreciable gains but increases the training overhead.

\textbf{Analysis of Expert Utilization.}
To confirm the duties of specific LoRA experts in the routing process, we visualize the normalized weights assigned by the routing gating when encountering data from different downstream tasks.
CMD and webMedQA data are merged to compose general healthcare data.
From Figure~\ref{vis}(b), (i) Experts 2 and 3 are emphatically activated on the TreRecom and MedKQ\&A tasks, respectively, implying their focal ability to tackle medical knowledge interpretations and treatment recommendations.
(ii) In contrast, Expert 1 is more proficient at learning multi-turn diagnosis semantics in the EviDiag task, which is different from the other tasks of instruction content.
(iii) Additionally, there is no clear difference in the specific expert utilization on general healthcare, implying that the general task is handled by the consistently universal expert.
The above observations demonstrate the effectiveness and necessity of the proposed MCE strategy.


% \vspace{-6pt}
\subsection{Visualization Analysis of Model Responses}
% \vspace{-6pt}
To intuitively compare the output quality of medical LLMs, we show the responses of different models for each of the three types of medical inquiries in 
Figures~\ref{tab_medqa}\&\ref{tab_diag}\&\ref{tab_recom} from Appendix~\ref{app_sec:5}.
From the results, Zhongjing offers insufficient information due to limited output content. Although HuatuoGPT-II gives well-organized responses, it lacks accuracy and informativeness. In comparison, our model can provide more specialized and detailed medical knowledge and diagnostic guidance in extended response contexts,  confirming its application potential in diverse healthcare services.

\vspace{-5pt}
\section{Conclusion and Discussion}
\vspace{-5pt}
This paper presents PediatricsGPT, a Chinese medical LLM assistant with medical generalist and pediatric expertise capabilities. Based on the well-designed PedCorpus dataset, PediatricsGPT undergoes a systematic and robust procedure ranging from continuous pre-training and supervised fine-tuning to human preference optimization, leading to competence in different pediatric and general healthcare service scenarios.
Extensive experimental results under multi-dimensional evaluation patterns demonstrate that our model outperforms currently available Chinese medical LLMs, providing a potential solution for promoting reliable and intelligent interactive diagnosis and treatment.

\textbf{Broader Impacts.}
(i) Our model has made meaningful contributions to pediatric medicine by integrating extensive medical data and emerging research. This integration facilitates more accurate and expedited diagnosis of complex pediatric conditions and aids in predicting treatment outcomes, enabling highly personalized and effective treatment strategies for young patients.
(ii) The proposed PediatricsGPT provides crucial decision support for medical professionals, giving evidence-based recommendations and specialized medical insights.  Additionally, it democratizes access to expert medical suggestions and accurate medical knowledge, empowering parents and caregivers with accurate health information, which is especially crucial in underserved areas.
(iii) The training pipeline of PediatricsGPT showcases exemplary generalizability, designed to be applicable across various medical and non-medical domains. This adaptability broadens the model’s applicability and pioneers the development of future AI solutions in healthcare and other fields.

\textbf{Limitations.}
(i) When deployed online, the proposed PediatricsGPT model, like other Large Language Models (LLMs), faces significant security risks, particularly from attacks aimed at manipulating its outputs. These attacks can be strategically designed to exploit the model's response mechanisms, allowing attackers to induce the model to generate unsafe, biased, or otherwise inappropriate content. 
(ii) Currently, our PediatricsGPT model does not support all languages. This linguistic barrier can prevent the model from reaching a global audience, particularly in diverse linguistic landscapes where localized medical information is crucial. 

\textbf{Ethical Issues.}
We fully recognize the critical importance of privacy and data protection. All data used has been meticulously de-identified, with all sensitive information removed, and this process has been verified by the partnering medical institutions.
For the public databases, we strictly follow specific license agreements for use and adaptation. For the constructed corpus, we underwent an internal ethical review by the ethical review board of the partnering medical institutions with license and approval. We will release relevant resources to the extent that they are controlled and permitted.

We provide more discussions of the future work in Appendix \ref{app_sec:8}.

\textbf{Acknowledgment.} This work is supported in part by the National Key R\&D Program of China under Grant 2021ZD0113502, in part by the Shanghai Municipal Science and Technology Major Project under Grant 2021SHZDZX0103.


\bibliographystyle{plain}
\bibliography{nips}


%%%%%%%%%%%%%%%%%%%%%%%%%%%%%%%%%%%%%%%%%%%%%%%%%%%%%%%%%%%%
\newpage
\appendix

\section{Implementation Details of PedCorpus Construction}
\subsection{Role-playing-driven Instruction Building Rule}
\label{app_sec:1_1}
After integrating pediatric textbooks, guidelines, and knowledge graphs into a consolidated textual database, the content is segmented according to individual diseases. Subsequently, two instances of the GPT-4 model are deployed, designated as the ``inquirer'' and the ``expert pediatrician'' respectively. Disease-specific segments are then fed into the ``inquirer'' GPT-4, tasked with formulating a series of relevant and scholarly pediatric inquiries. Following this, the original disease segments and the formulated inquiries are fed into the ``expert pediatrician'' GPT-4 to generate precise responses for each inquiry, leveraging the segmented text as the contextual reference.
We show the prompt templates for the ``inquirer'' and ``expert pediatrician'' in Figures~\ref{app_sec_1_1} and~\ref{app_sec_1_2}, respectively.



\begin{figure}[h]
  \centering
  \includegraphics[width=\linewidth]{data_prompt_1.pdf}
  \caption{Illustration of the GPT-4 prompt template for the ``inquirer'' role. The original Chinese content is translated into the corresponding English version to facilitate intuitive understanding.
  }
  \label{app_sec_1_1}
    % \vspace{-3pt}
\end{figure}

\begin{figure}[h]
  \centering
  \includegraphics[width=\linewidth]{data_prompt_2.pdf}
  \caption{Illustration of the GPT-4 prompt template for the ``expert pediatrician'' role. The original Chinese content is translated into the corresponding English version to facilitate intuitive understanding.
  }
  \label{app_sec_1_2}
\end{figure}

\subsection{Vanilla Doctor-patient Conversation Regularization}
\label{app_sec:1_2}


We guide GPT-4 to regularize concise and noisy doctor responses in authentic doctor-patient consultations by the context learning strategy.
Specifically, we manually craft 100 instruction examples with high-quality content to allow GPT-4 to learn doctor-like and patient-friendly behavioral styles.
In each round of regularization, we randomly sample 10 out of 100 examples to perform 10-shot context prompts. 
Immediately after that, vanilla dialogues are fed to GPT-4 as seed instructions to optimize instructions according to user requirements. 
Constrained by the space, we show the prompt case with one example in Figure~\ref{app_sec_1_3}.

\begin{figure}[t]
  \centering
  \includegraphics[width=\linewidth]{data_prompt_3.pdf}
  \caption{Illustration of the GPT-4 prompt template for vanilla doctor-patient conversation regularization. The original Chinese content is translated into the corresponding English version to facilitate intuitive understanding.
  }
  \label{app_sec_1_3}
\end{figure}

\begin{figure}[t]
  \centering
  \includegraphics[width=\linewidth]{data_prompt_4.pdf}
  \caption{Illustration of the GPT-4 prompt template for progressive instruction reconstruction rule. The original Chinese content is translated into the corresponding English version to facilitate intuitive understanding.
  }
  \label{app_sec_1_4}
\end{figure}


\subsection{Progressive Instruction Reconstruction Rule}
\label{app_sec:1_3}

Medical knowledge integration from existing datasets is common but frequently imprecise, resulting in unclear or incomplete instructions and potentially inaccurate outputs. Consequently, 107,177 instructions are selected from three significant benchmarks, prioritizing quality over quantity. In this case, we design a progressive instruction reconstruction rule to refine these instructions, ensuring informative and logical model responses.

As shown in Figure~\ref{app_sec_1_4}, we first prompt GPT-4 to take the perspective of the experienced doctor to complete Tasks 1 and 2 in the given instruction and answer scenarios. Task 1 focuses on bridging the gaps in the vanilla instructions and reinforcing the completeness, professionalism, and medical relevance. Based on the refined instructions, Task 2 requires the GPT-4 to make further targeted improvements to the answer parts. In practice, this progressive reconstruction rule can activate better instruction following capabilities in advanced language models.







\section{Knowledge-enhanced Prompt}
\label{app_sec:2} 
To enrich the density and breadth of the multiple-department CPT corpus, we transform the structured instruction data from the vanilla PedCorpus dataset into comprehensive medical knowledge texts using knowledge-enhanced prompts. The medical knowledge texts are integrated as complementary content to construct the PedCorpus-CPT dataset.
The prompt template is shown in Figure~\ref{app_sec_1_5}.


\begin{figure}[t]
  \centering
  \includegraphics[width=\linewidth]{data_prompt_5.pdf}
  \caption{Illustration of the GPT-4 prompt template for improving corpus density and breadth. The original Chinese content is translated into the corresponding English version to facilitate intuitive understanding.}
  \label{app_sec_1_5}
\end{figure}

\section{Training Details}
\label{app_sec:3} 
In this section, we list in detail the hyper-parameter configurations for the different training phases.

\textbf{Continuous Pre-training.} During this procedure, we train each model for just a single epoch, setting the learning rate at 1e-6 and the batch size at 128. We adopt a maximum cutoff length of 4096, enabling the model to process extensive text sequences in one batch. This significantly enhances the model's contextual understanding and coherence.

\textbf{Full-parameter Supervised Fine-tuning.} In this configuration, we train all models for three epochs with a learning rate adjusted to 5e-5 and a batch size of 64, capping the maximum sequence length at 2048. We introduce a warmup\_steps setting at 200 to gradually ramp up the learning rate from an initial lower value, aiding the optimizer in adapting to gradient changes. This approach boosts stability and performance and guides the model towards a better convergence path. Also, we specify eval\_steps at 100 and save the best-performing weights on the validation set to ensure optimal results.


\textbf{Human Preference Alignment.} In this setup, we train five epochs with the learning rate set to 1e-6 and the batch size maintained at 64. To enhance the robustness and smoothness of preference learning, we adjust the control parameter $\beta$ to 0.1 and the scaling coefficient $\mu$ to 1.0. We specify eval\_steps at 100, selecting the best-performing weights on the validation set.


\textbf{LoRA-based Parameter-efficient SFT.} Here, we train three epochs with a learning rate of 1e-6 and adjust the batch size to 32. We configure the LoRA parameters by setting the rank $r$ to 8, $\alpha$ to 16, and the Dropout rate to 0.05, targeting all modules.
The default number of LoRA adapters is set to 4, including one constant universal expert and three specific experts.
Ultimately, we select the adapters that perform best on the validation set.




\section{GPT-4 Evaluation Details}
\label{app_sec:4} 
We consider four complementary dimensions in the automated evaluation to guide GPT-4 in judging the quality of model responses from a comprehensive perspective. The full definitions of these dimensions are shown as follows.

\textbf{Usefulness}: measures the extent to which the model response has pediatric expertise and relevance to the instruction intention.

\textbf{Correctness}: measures the extent to which harmful, misleading, and inaccurate information is present in the model response.

\textbf{Consistency}: measures the degree to which the model response is logically self-contradictory and the information is coherent in context.

\textbf{Smoothness}: measures whether the response content is fluent, natural, and conforms to the language expression style of human habits.

In this case, we present GPT-4 with paired responses from different models, assessing various criteria such as pediatric expertise in the responses, presence of harmful, misleading, or inaccurate information, logical consistency, and the fluency and naturalness of the language, which should conform to human linguistic habits. GPT-4 assesses these responses on their merits and selects the superior one. To maintain fairness and mitigate potential position bias, the order of the responses is randomised. This methodology is supported by recent studies demonstrating GPT-4’s strong agreement with human judgment in evaluating responses.
Figure~\ref{tab_gpt_eval} demonstrates the prompt template used to evaluate the quality of paired model responses.


\begin{figure}[t]
  \centering
  \includegraphics[width=\linewidth]{GPT-4-Eval.pdf}
  % \vspace{-15pt}
  \caption{Illustration of the GPT-4 prompt template for evaluating paired model responses.
  }
  % \vspace{-6pt}
  \label{tab_gpt_eval}
\end{figure}







\vspace{-6pt}
\section{Comparison Results of Model Responses}
\label{app_sec:5} 
\vspace{-6pt}
In this section, we visualize the responses of the proposed PediatricsGPT-13B and two SOTA Chinese medical LLMs across three tasks from the same medical inquiries to provide intuitive qualitative comparisons.
Specifically, Figures~\ref{tab_medqa} and \ref{tab_recom} illustrate the medical knowledge question-answer and treatment recommendation tasks, respectively, which follow a single-turn dialogue pattern. The multi-turn conversation pattern is considered in the evidence-based diagnosis task from Figure~\ref{tab_diag}.


\section{Future Work}
\label{app_sec:8} 

We list future work below to provide potential optimization directions.

\textbf{Enhancing Security Against Model Manipulation.} To mitigate the security risks associated with online deployment, our future strategy involves implementing multi-layered security measures for the proposed PediatricsGPT model. This will include advanced input validation techniques to detect and neutralize potentially malicious inputs that could manipulate model outputs. Continuous updates and patches will also be prioritized to address emerging security threats and vulnerabilities.

\textbf{Expanding Language Support.} To overcome the challenge of incomplete language coverage, we are committed to broadening the linguistic capabilities of PediatricsGPT. This expansion will involve training the model on a more diverse dataset that includes a broader range of languages and dialects, particularly those prevalent in underserved regions. By doing so, we aim to make the model more accessible and useful to a global audience, ensuring that non-Chinese speakers also benefit from reliable and localized medical information.

\clearpage


\begin{figure}[t]
  \centering
  \includegraphics[width=\linewidth]{model_response1.pdf}
  \caption{Comparison results of model responses on the medical knowledge question-answer task.
  }
  \label{tab_medqa}
\end{figure}


\begin{figure}[t]
  \centering
  \includegraphics[width=\linewidth]{model_response2.pdf}
  \caption{Comparison results of model responses on the evidence-based diagnosis task.
  }
  \label{tab_diag}
\end{figure}
\clearpage

\clearpage
\begin{figure}[t]
  \centering
  \includegraphics[width=\linewidth]{model_response3.pdf}
  \caption{Comparison results of model responses on the treatment recommendation task. 
  }
  \label{tab_recom}
\end{figure}
\clearpage

\newpage
\section*{NeurIPS Paper Checklist}



\begin{enumerate}

\item {\bf Claims}
    \item[] Question: Do the main claims made in the abstract and introduction accurately reflect the paper's contributions and scope?
    \item[] Answer: \answerYes{} % Replace by \answerYes{}, \answerNo{}, or \answerNA{}.
    \item[] Justification: The Abstract and Section 1 show our paper's contributions and scopes.
    \item[] Guidelines:
    \begin{itemize}
        \item The answer NA means that the abstract and introduction do not include the claims made in the paper.
        \item The abstract and/or introduction should clearly state the claims made, including the contributions made in the paper and important assumptions and limitations. A No or NA answer to this question will not be perceived well by the reviewers. 
        \item The claims made should match theoretical and experimental results, and reflect how much the results can be expected to generalize to other settings. 
        \item It is fine to include aspirational goals as motivation as long as it is clear that these goals are not attained by the paper. 
    \end{itemize}

\item {\bf Limitations}
    \item[] Question: Does the paper discuss the limitations of the work performed by the authors?
    \item[] Answer: \answerYes{} % Replace by \answerYes{}, \answerNo{}, or \answerNA{}.
    \item[] Justification: We fully discuss the limitations in Section 5.
    \item[] Guidelines:
    \begin{itemize}
        \item The answer NA means that the paper has no limitation while the answer No means that the paper has limitations, but those are not discussed in the paper. 
        \item The authors are encouraged to create a separate "Limitations" section in their paper.
        \item The paper should point out any strong assumptions and how robust the results are to violations of these assumptions (e.g., independence assumptions, noiseless settings, model well-specification, asymptotic approximations only holding locally). The authors should reflect on how these assumptions might be violated in practice and what the implications would be.
        \item The authors should reflect on the scope of the claims made, e.g., if the approach was only tested on a few datasets or with a few runs. In general, empirical results often depend on implicit assumptions, which should be articulated.
        \item The authors should reflect on the factors that influence the performance of the approach. For example, a facial recognition algorithm may perform poorly when image resolution is low or images are taken in low lighting. Or a speech-to-text system might not be used reliably to provide closed captions for online lectures because it fails to handle technical jargon.
        \item The authors should discuss the computational efficiency of the proposed algorithms and how they scale with dataset size.
        \item If applicable, the authors should discuss possible limitations of their approach to address problems of privacy and fairness.
        \item While the authors might fear that complete honesty about limitations might be used by reviewers as grounds for rejection, a worse outcome might be that reviewers discover limitations that aren't acknowledged in the paper. The authors should use their best judgment and recognize that individual actions in favor of transparency play an important role in developing norms that preserve the integrity of the community. Reviewers will be specifically instructed to not penalize honesty concerning limitations.
    \end{itemize}

\item {\bf Theory Assumptions and Proofs}
    \item[] Question: For each theoretical result, does the paper provide the full set of assumptions and a complete (and correct) proof?
    \item[] Answer: \answerNA{} % Replace by \answerYes{}, \answerNo{}, or \answerNA{}.
    \item[] Justification: This paper focuses on providing results based on metrics, GPT-4 and doctor evaluations. The related results are presented in Sections 4.3 through 4.6.
    \item[] Guidelines:
    \begin{itemize}
        \item The answer NA means that the paper does not include theoretical results. 
        \item All the theorems, formulas, and proofs in the paper should be numbered and cross-referenced.
        \item All assumptions should be clearly stated or referenced in the statement of any theorems.
        \item The proofs can either appear in the main paper or the supplemental material, but if they appear in the supplemental material, the authors are encouraged to provide a short proof sketch to provide intuition. 
        \item Inversely, any informal proof provided in the core of the paper should be complemented by formal proofs provided in appendix or supplemental material.
        \item Theorems and Lemmas that the proof relies upon should be properly referenced. 
    \end{itemize}

    \item {\bf Experimental Result Reproducibility}
    \item[] Question: Does the paper fully disclose all the information needed to reproduce the main experimental results of the paper to the extent that it affects the main claims and/or conclusions of the paper (regardless of whether the code and data are provided or not)?
    \item[] Answer: \answerYes{} 
    \item[] Justification: We disclose the implementation details of reproducing the results of the paper in Section 4.1 and Appendix C. 
    \item[] Guidelines:
    \begin{itemize}
        \item The answer NA means that the paper does not include experiments.
        \item If the paper includes experiments, a No answer to this question will not be perceived well by the reviewers: Making the paper reproducible is important, regardless of whether the code and data are provided or not.
        \item If the contribution is a dataset and/or model, the authors should describe the steps taken to make their results reproducible or verifiable. 
        \item Depending on the contribution, reproducibility can be accomplished in various ways. For example, if the contribution is a novel architecture, describing the architecture fully might suffice, or if the contribution is a specific model and empirical evaluation, it may be necessary to either make it possible for others to replicate the model with the same dataset, or provide access to the model. In general. releasing code and data is often one good way to accomplish this, but reproducibility can also be provided via detailed instructions for how to replicate the results, access to a hosted model (e.g., in the case of a large language model), releasing of a model checkpoint, or other means that are appropriate to the research performed.
        \item While NeurIPS does not require releasing code, the conference does require all submissions to provide some reasonable avenue for reproducibility, which may depend on the nature of the contribution. For example
        \begin{enumerate}
            \item If the contribution is primarily a new algorithm, the paper should make it clear how to reproduce that algorithm.
            \item If the contribution is primarily a new model architecture, the paper should describe the architecture clearly and fully.
            \item If the contribution is a new model (e.g., a large language model), then there should either be a way to access this model for reproducing the results or a way to reproduce the model (e.g., with an open-source dataset or instructions for how to construct the dataset).
            \item We recognize that reproducibility may be tricky in some cases, in which case authors are welcome to describe the particular way they provide for reproducibility. In the case of closed-source models, it may be that access to the model is limited in some way (e.g., to registered users), but it should be possible for other researchers to have some path to reproducing or verifying the results.
        \end{enumerate}
    \end{itemize}


\item {\bf Open access to data and code}
    \item[] Question: Does the paper provide open access to the data and code, with sufficient instructions to faithfully reproduce the main experimental results, as described in supplemental material?
    \item[] Answer: \answerYes{}
    \item[] Justification: We provide the link to release the relevant available resources in the Abstract.
    \item[] Guidelines:
    \begin{itemize}
        \item The answer NA means that paper does not include experiments requiring code.
        \item Please see the NeurIPS code and data submission guidelines (\url{https://nips.cc/public/guides/CodeSubmissionPolicy}) for more details.
        \item While we encourage the release of code and data, we understand that this might not be possible, so “No” is an acceptable answer. Papers cannot be rejected simply for not including code, unless this is central to the contribution (e.g., for a new open-source benchmark).
        \item The instructions should contain the exact command and environment needed to run to reproduce the results. See the NeurIPS code and data submission guidelines (\url{https://nips.cc/public/guides/CodeSubmissionPolicy}) for more details.
        \item The authors should provide instructions on data access and preparation, including how to access the raw data, preprocessed data, intermediate data, and generated data, etc.
        \item The authors should provide scripts to reproduce all experimental results for the new proposed method and baselines. If only a subset of experiments are reproducible, they should state which ones are omitted from the script and why.
        \item At submission time, to preserve anonymity, the authors should release anonymized versions (if applicable).
        \item Providing as much information as possible in supplemental material (appended to the paper) is recommended, but including URLs to data and code is permitted.
    \end{itemize}


\item {\bf Experimental Setting/Details}
    \item[] Question: Does the paper specify all the training and test details (e.g., data splits, hyperparameters, how they were chosen, type of optimizer, etc.) necessary to understand the results?
    \item[] Answer: \answerYes{}
    \item[] Justification:  We provide detailed hyper-parameter configurations in Appendix C. Also, Section 4.1 provides dataset splits and other implementation details.
    \item[] Guidelines:
    \begin{itemize}
        \item The answer NA means that the paper does not include experiments.
        \item The experimental setting should be presented in the core of the paper to a level of detail that is necessary to appreciate the results and make sense of them.
        \item The full details can be provided either with the code, in appendix, or as supplemental material.
    \end{itemize}

\item {\bf Experiment Statistical Significance}
    \item[] Question: Does the paper report error bars suitably and correctly defined or other appropriate information about the statistical significance of the experiments?
    \item[] Answer: \answerNo{}
    \item[] Justification: Performing multiple repetitive experiments in order to compute error bars is labor-intensive and has significant overhead for the large language model development.
    \item[] Guidelines:
    \begin{itemize}
        \item The answer NA means that the paper does not include experiments.
        \item The authors should answer "Yes" if the results are accompanied by error bars, confidence intervals, or statistical significance tests, at least for the experiments that support the main claims of the paper.
        \item The factors of variability that the error bars are capturing should be clearly stated (for example, train/test split, initialization, random drawing of some parameter, or overall run with given experimental conditions).
        \item The method for calculating the error bars should be explained (closed form formula, call to a library function, bootstrap, etc.)
        \item The assumptions made should be given (e.g., Normally distributed errors).
        \item It should be clear whether the error bar is the standard deviation or the standard error of the mean.
        \item It is OK to report 1-sigma error bars, but one should state it. The authors should preferably report a 2-sigma error bar than state that they have a 96\% CI, if the hypothesis of Normality of errors is not verified.
        \item For asymmetric distributions, the authors should be careful not to show in tables or figures symmetric error bars that would yield results that are out of range (e.g. negative error rates).
        \item If error bars are reported in tables or plots, The authors should explain in the text how they were calculated and reference the corresponding figures or tables in the text.
    \end{itemize}

\item {\bf Experiments Compute Resources}
    \item[] Question: For each experiment, does the paper provide sufficient information on the computer resources (type of compute workers, memory, time of execution) needed to reproduce the experiments?
    \item[] Answer: \answerYes{} % Replace by \answerYes{}, \answerNo{}, or \answerNA{}. 我们在章节4.1中提供了为了复现实验所需要的计算资源。
    \item[] Justification: We provide the computational resources needed in order to reproduce the experiments in Section 4.1.
    \item[] Guidelines:
    \begin{itemize}
        \item The answer NA means that the paper does not include experiments.
        \item The paper should indicate the type of compute workers CPU or GPU, internal cluster, or cloud provider, including relevant memory and storage.
        \item The paper should provide the amount of compute required for each of the individual experimental runs as well as estimate the total compute. 
        \item The paper should disclose whether the full research project required more compute than the experiments reported in the paper (e.g., preliminary or failed experiments that didn't make it into the paper). 
    \end{itemize}
    
\item {\bf Code Of Ethics}
    \item[] Question: Does the research conducted in the paper conform, in every respect, with the NeurIPS Code of Ethics \url{https://neurips.cc/public/EthicsGuidelines}?
    \item[] Answer: \answerYes{} % Replace by \answerYes{}, \answerNo{}, or \answerNA{}.
    \item[] Justification: Our research follows the NeurIPS Code of Ethics.
    \item[] Guidelines:
    \begin{itemize}
        \item The answer NA means that the authors have not reviewed the NeurIPS Code of Ethics.
        \item If the authors answer No, they should explain the special circumstances that require a deviation from the Code of Ethics.
        \item The authors should make sure to preserve anonymity (e.g., if there is a special consideration due to laws or regulations in their jurisdiction).
    \end{itemize}


\item {\bf Broader Impacts}
    \item[] Question: Does the paper discuss both potential positive societal impacts and negative societal impacts of the work performed?
    \item[] Answer: \answerYes{} % Replace by \answerYes{}, \answerNo{}, or \answerNA{}. 我们在附录F中充分讨论了可能的社会影响。
    \item[] Justification: We fully discuss the potential social impacts in Section 5.
    \item[] Guidelines:
    \begin{itemize}
        \item The answer NA means that there is no societal impact of the work performed.
        \item If the authors answer NA or No, they should explain why their work has no societal impact or why the paper does not address societal impact.
        \item Examples of negative societal impacts include potential malicious or unintended uses (e.g., disinformation, generating fake profiles, surveillance), fairness considerations (e.g., deployment of technologies that could make decisions that unfairly impact specific groups), privacy considerations, and security considerations.
        \item The conference expects that many papers will be foundational research and not tied to particular applications, let alone deployments. However, if there is a direct path to any negative applications, the authors should point it out. For example, it is legitimate to point out that an improvement in the quality of generative models could be used to generate deepfakes for disinformation. On the other hand, it is not needed to point out that a generic algorithm for optimizing neural networks could enable people to train models that generate Deepfakes faster.
        \item The authors should consider possible harms that could arise when the technology is being used as intended and functioning correctly, harms that could arise when the technology is being used as intended but gives incorrect results, and harms following from (intentional or unintentional) misuse of the technology.
        \item If there are negative societal impacts, the authors could also discuss possible mitigation strategies (e.g., gated release of models, providing defenses in addition to attacks, mechanisms for monitoring misuse, mechanisms to monitor how a system learns from feedback over time, improving the efficiency and accessibility of ML).
    \end{itemize}
    
\item {\bf Safeguards}
    \item[] Question: Does the paper describe safeguards that have been put in place for responsible release of data or models that have a high risk for misuse (e.g., pretrained language models, image generators, or scraped datasets)?
    \item[] Answer: \answerYes{} % Replace by \answerYes{}, \answerNo{}, or \answerNA{}. 为确保安全的发布数据,我们在章节3.1中详细描述了数据集构建的措施。为确保提出模型的响应是无害和安全的,我们在章节3.3中提供了对抗性指令来控制模型的行为。同时,我们在章节3.4中Wie模型执行了人类偏好优化,进一步强化了模型的安全性和鲁棒性。
    \item[] Justification: To ensure the safe release of data, we describe measures for dataset construction in detail in Section 3.1. To ensure that the responses of the proposed model are harmless and safe, we provide adversarial instructions to control the model behaviour in Section 3.3. Meanwhile, we perform human preference optimization for the model in Section 3.4, which further strengthens the safety and robustness of the model.
    \item[] Guidelines:
    \begin{itemize}
        \item The answer NA means that the paper poses no such risks.
        \item Released models that have a high risk for misuse or dual-use should be released with necessary safeguards to allow for controlled use of the model, for example by requiring that users adhere to usage guidelines or restrictions to access the model or implementing safety filters. 
        \item Datasets that have been scraped from the Internet could pose safety risks. The authors should describe how they avoided releasing unsafe images.
        \item We recognize that providing effective safeguards is challenging, and many papers do not require this, but we encourage authors to take this into account and make a best faith effort.
    \end{itemize}

\item {\bf Licenses for existing assets}
    \item[] Question: Are the creators or original owners of assets (e.g., code, data, models), used in the paper, properly credited and are the license and terms of use explicitly mentioned and properly respected?
    \item[] Answer: \answerYes{} % Replace by \answerYes{}, \answerNo{}, or \answerNA{}.
    \item[] Justification: We provide reasonable references for the datasets and models used in Sections 4.1 and 4.2, respectively.
    \item[] Guidelines:
    \begin{itemize}
        \item The answer NA means that the paper does not use existing assets.
        \item The authors should cite the original paper that produced the code package or dataset.
        \item The authors should state which version of the asset is used and, if possible, include a URL.
        \item The name of the license (e.g., CC-BY 4.0) should be included for each asset.
        \item For scraped data from a particular source (e.g., website), the copyright and terms of service of that source should be provided.
        \item If assets are released, the license, copyright information, and terms of use in the package should be provided. For popular datasets, \url{paperswithcode.com/datasets} has curated licenses for some datasets. Their licensing guide can help determine the license of a dataset.
        \item For existing datasets that are re-packaged, both the original license and the license of the derived asset (if it has changed) should be provided.
        \item If this information is not available online, the authors are encouraged to reach out to the asset's creators.
    \end{itemize}

\item {\bf New Assets}
    \item[] Question: Are new assets introduced in the paper well documented and is the documentation provided alongside the assets?
    \item[] Answer: \answerYes{} % Replace by \answerYes{}, \answerNo{}, or \answerNA{}. 我们在附录中提供了相关的文档记录。
    \item[] Justification: We provide the relevant documentation in the Appendix.
    \item[] Guidelines:
    \begin{itemize}
        \item The answer NA means that the paper does not release new assets.
        \item Researchers should communicate the details of the dataset/code/model as part of their submissions via structured templates. This includes details about training, license, limitations, etc. 
        \item The paper should discuss whether and how consent was obtained from people whose asset is used.
        \item At submission time, remember to anonymize your assets (if applicable). You can either create an anonymized URL or include an anonymized zip file.
    \end{itemize}

\item {\bf Crowdsourcing and Research with Human Subjects}
    \item[] Question: For crowdsourcing experiments and research with human subjects, does the paper include the full text of instructions given to participants and screenshots, if applicable, as well as details about compensation (if any)? 
    \item[] Answer: \answerYes{}
    \item[] Justification: We offer \$300 each to participating experts. Expert doctors are asked to evaluate the quality of the models' responses. The relevant descriptions can be found in Section 4.3.
    \item[] Guidelines:
    \begin{itemize}
        \item The answer NA means that the paper does not involve crowdsourcing nor research with human subjects.
        \item Including this information in the supplemental material is fine, but if the main contribution of the paper involves human subjects, then as much detail as possible should be included in the main paper. 
        \item According to the NeurIPS Code of Ethics, workers involved in data collection, curation, or other labor should be paid at least the minimum wage in the country of the data collector. 
    \end{itemize}

\item {\bf Institutional Review Board (IRB) Approvals or Equivalent for Research with Human Subjects}
    \item[] Question: Does the paper describe potential risks incurred by study participants, whether such risks were disclosed to the subjects, and whether Institutional Review Board (IRB) approvals (or an equivalent approval/review based on the requirements of your country or institution) were obtained?
    \item[] Answer: \answerYes{}
    \item[] Justification: The research in the paper has no risks to be faced by participants who are only used as evaluators. Moreover, we underwent an internal ethical review by the ethical review board of the partnering medical institutions with license and approval. The relevant descriptions can be found in Section 5.
    \item[] Guidelines:
    \begin{itemize}
        \item The answer NA means that the paper does not involve crowdsourcing nor research with human subjects.
        \item Depending on the country in which research is conducted, IRB approval (or equivalent) may be required for any human subjects research. If you obtained IRB approval, you should clearly state this in the paper. 
        \item We recognize that the procedures for this may vary significantly between institutions and locations, and we expect authors to adhere to the NeurIPS Code of Ethics and the guidelines for their institution. 
        \item For initial submissions, do not include any information that would break anonymity (if applicable), such as the institution conducting the review.
    \end{itemize}

\end{enumerate}

\end{document}