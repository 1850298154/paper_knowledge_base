
\subsection{$k$-Unit Feasibility Constraints}
\label{sec:border_k}%

In this section, we consider environments where at most $k$ agents can
be simultaneously allocated to. First, we generalize Border's
characterization of interim feasibility to environments with $k$-unit
feasibility constraint. Our generalization implies that the space of
feasible normalized interim allocation rules is a polymatroid. Second,
we observe that optimization problems can be efficiently solved over
polymatroids; this allows us to optimize over feasible interim
allocation rules. Third, we show that the normalized interim
allocation rules corresponding to the vertices of this polymatroid are
implemented by simple deterministic ordered-subset-based allocation
mechanisms. Furthermore, for any point in this polymatroid, the
corresponding normalized interim allocation rule can be implemented
by, (i) expressing it as a convex combination of the vertices of the
polymatroid, (ii) sampling from this convex combination, and (iii)
using the ordered subset mechanism corresponding to the sampled
vertex. We present an efficient randomized rounding routine for
rounding a point in a polymatroid to a vertex which combines the steps
(i) and (ii). These approaches together yield efficient algorithms for
optimizing and implementing interim allocation rules.

\paragraph{Polymatroid Preliminaries.}
Consider an arbitrary set function $\SubFun : 2^{\Universe} \to \PosReals$ defined over an arbitrary finite set $\Universe$; let
$\PolyMat{\SubFun}$ denote the polytope associate with $\SubFun$ defined as
\begin{align*}
    \PolyMat{\SubFun} &= \left\{ \PolyVec \in \PosReals^\Universe \middle| \forall \USubset \subseteq \Universe : \PolyVec(\USubset) \le \SubFun(\USubset)
    \right\}
\end{align*}
where $\PolyVec(\USubset)$ denotes $\sum_{\Elem \in \USubset}
\PolyVec(\Elem)$. The convex polytope $\PolyMat{\SubFun}$ is called a
\emph{polymatroid} if $\SubFun$ is a submodular function. Even though
a polymatroid is defined by an exponential number of linear
inequalities, the separation problem for any given $\PolyVec \in
\PosReals^{\Universe}$ can be solved in polynomial time as follows:
find $\displaystyle \USubset^* = \argmin_{\USubset}
\SubFun(\USubset)-\PolyVec(\USubset)$; if $\PolyVec$ is infeasible,
the inequality $\PolyVec(\USubset^*) \le \SubFun(\USubset^*)$ must be
violated, and that yields a separating hyperplane for $\PolyVec$. Note
that $\SubFun(\USubset)-\PolyVec(\USubset)$ is itself submodular in
$\USubset$, so it can be minimized in strong polynomial
time. Consequently, optimization problems can be solved over
polymatroids in polynomial time.  Next, we describe a characterization
of the vertices of a polymatroid. This characterization plays an
important role in our proofs and also in our ex post implementation of
interim allocation rules.

\begin{definition}[ordered subset]
\label{def:ranking}%
For an arbitrary finite set $\Universe$, an {\em ordered subset}
$\Ranking \subset \Universe$ is given by an ordering on elements
$\Ranking = (\RElem_1,\ldots,\RElem_{|\Ranking|})$ where shorthand
notation $\RElem_r \in \Ranking$ denotes the $r$th element in
$\Ranking$.
\end{definition}

\begin{proposition}
\label{prop:vertex}%
Let $\SubFun : 2^\Universe \to \PosReals$ be an arbitrary
non-decreasing submodular function with $\SubFun(\emptyset) = 0$ and
let $\PolyMat{\SubFun}$ be the associated polymatroid with the set of
vertices $\VertexSet{\PolyMat{\SubFun}}$. Every ordered subset
$\Ranking$ of $\Universe$ (see \autoref{def:ranking}) corresponds to a
vertex of $\PolyMat{\SubFun}$, denoted by
$\Vertex{\Ranking}{\PolyMat{\SubFun}}$, which is computed as
follows.  
\begin{align*}
    \forall \Elem \in \Universe :& \qquad &
    \PolyVec(\Elem) &=
        \begin{cases}
        \SubFun(\{\RElem_1,\ldots,\RElem_r\})-\SubFun(\{\RElem_1,\ldots,\RElem_{r-1}\}) & \text{if $\Elem = \RElem_r \in \Ranking$} \\
        0       & \text{if $\Elem \not \in \Ranking$}
        \end{cases}
\end{align*}
Furthermore, for every $\PolyVec \in \VertexSet{\PolyMat{\SubFun}}$
there exist a corresponding $\Ranking$.
\end{proposition}
It is easy to see that for any $\PolyVec \in
\VertexSet{\PolyMat{\SubFun}}$, an associated $\Ranking$ can be found
efficiently by a greedy algorithm (see \citet{S03} for a comprehensive
treatment of polymatroids).

\paragraph{Ordered subset allocation mechanisms.}
The following class of allocation mechanisms are of particular importance both in our characterization of interim feasibility and
in ex post implementation of interim allocation rules.

\begin{definition}[ordered subset allocation mechanism]
\label{def:rank_alloc_k} Parameterized by a \emph{ordered subset} $\Ranking$ of $\UnionTypeSpace$ (see \autoref{def:ranking}),
the {\em ordered subset mechanism}, on profile of types $\Types \in
\TypeSpaces$, orders the agents based on their types according to
$\Ranking$, and allocates to the agents greedily (e.g., with $k$ units
available the $k$ first ordered agents received a unit).  If an agent
$i$ with type $\type_i \not\in \Ranking$ is never served.\footnote{The
  virtual valuation maximizing mechanisms from the classical
  literature on revenue maximizing auctions are ordered subset mechanisms,
  see, e.g., \citet{M81}, an observation made previousl by
  \citet{elk-07}. The difference between these ordered subset mechanisms and the
  classic virtual valuation maximization mechanisms is that our
  ordered subset will come from solving an optimization on the whole auction
  problem where as the Myerson's virtual values come directly from
  single-agent optimizations.}
\end{definition}

\paragraph{Characterization of interim feasibility.}
Border's characterization of interim feasibility for $k=1$ unit
auctions states that the probability of serving a type in a subspace
of type space is no more than the probability that a type in that
subspace shows up.  This upper bound is equivalent to the expected
minimum of one and the number of types from the subspace that show up;
furthermore, this equivalent phrasing of the upper bound extends to
characterize interim feasibility in $k$-unit auctions.

Consider expressing an ex post allocation for type profile $\types$ by
$\RvAlloc{\Types} \in \{0,1\}^\UnionTypeSpace$ as follows.  For all
$\type'_i \in \UnionTypeSpace$, $\RvAlloc{\Types}(\type'_i) = 1$ if
player $i$ is served and $\type_i = \type_i'$ and 0 otherwise.  This
definition of ex post allocations is convenient as the normalized
interim allocation rule is calculated by taking its expecation, i.e.,
$\PreAlloc(\type'_i) = \expect[\types]{\RvAlloc{\Types}(\type'_i)}$.
Ex post feasibility requires that,
\begin{align}
    \RvAlloc{\Types}(\TypeSubset) & \le \min(\Abs{\Types \cap \TypeSubset}, k), &
        &  \forall \Types \in \TypeSpaces, \forall \TypeSubset \subseteq \UnionTypeSpace
        \label{eq:bk1}
\end{align}
In other words: for any profile of types $\types$, the number of types
in $\TypeSubset$ that are served by $\RvAlloc{\types}$ must be at most
the number of types in $S$ that showed up in $\types$ and the upper
bound $k$.  Taking expectations of both sides of this equation with
respect to $\types$ motivates the following definition and theorem.


\begin{definition}
\label{def:incidence_k} The {\em expected rank function} for distribution $\denss$ and subspace $S \subset \UnionTypeSpace$ is
\begin{align}
    \DistOracle_k(\TypeSubset) &= \Ex[\Types \sim
      \DistProbs]{\min\left(\Abs{\Types \cap \TypeSubset}, k\right)}
    \tag{$\DistOracle_k$} \label{eq:dist_oracle}
\end{align}
where $\Types \cap \TypeSubset$ denotes $\{\Type_1,\ldots,\Type_n\}
\cap \TypeSubset$.
\end{definition}




\begin{theorem}
\label{thm:border_k}%
For supply constraint $k$ and distribution $\denss$, the space of all feasible normalized interim allocation rules, $\PreAllocSpace$, is the polymatroid associated with
$\DistOracle_k$, i.e., $\PreAllocSpace = \PolyMat{\DistOracle_k}$, i.e., for all $\PreAlloc \in \PreAllocSpace$, 
\begin{align}
    \PreAlloc(\TypeSubset) &\le \Ex[\Types]{\min(\Abs{\Types \cap \TypeSubset}, k)} = \DistOracle_k(\TypeSubset), & \forall \TypeSubset \subseteq \UnionTypeSpace
    \label{eq:bk2}
\end{align}
\end{theorem}

The proof of this theorem will be deferred to the next section where
we will derive a more general theorem.  A key step in the proof will
be relating the statement of the theorem to the polymatroid theory
described already.  To show that the constraint of the theorem is a
polymatroid, we observe that the expected rank function is submoduler.

\begin{lemma} 
\label{lem:g_k-submodular}
The expected rank function $\DistOracle_k$ is submodular.
\end{lemma}

\begin{proof}
Observe that for any fixed $\Types$, $\min(\Types \cap
\TypeSubset, k)$ is obviously a submodular function in $\TypeSubset$, and therefore  $\DistOracle_k$ is a convex
combination\footnote{Note that taking the expectation is the same as taking a convex combination.} of submodular functions, so
$\DistOracle_k$ is submodular.
\end{proof}



We now relate vertices of the polymatroid to ordered subset allocation
mechanisms.


\begin{theorem}
\label{thm:border_k-unique}
For supply constraint $k$, if $\PreAlloc \in \PreAllocSpace$ is the vertex
$\Vertex{\Ranking}{\PolyMat{\DistOracle_k}}$ of the polymatroid
$\PolyMat{\DistOracle_k}$ the unique ex post implementation is the
ordered subset mechanism induced by $\Ranking$
(\autoref{def:rank_alloc_k}).
\end{theorem}


\begin{proof}
Let $\PreAlloc=\Vertex{\Ranking}{\PolyMat{\DistOracle_k}}$ be an
arbitrary vertex of $\PolyMat{\DistOracle_k}$ with a corresponding
ordered subset $\Ranking$; by \autoref{prop:vertex}, such a $\Ranking$
exists for every vertex of a polymatroid.  For every integer $r \leq
|\Ranking|$, define $\TypeSubset^r =\{\RElem_1,\ldots,\RElem_r\}$ as
the $r$-element prefix of the ordering. By \autoref{prop:vertex},
inequality~\eqref{eq:bk2} must be tight for every $\TypeSubset^r$
which implies that inequality~\eqref{eq:bk1} must also be tight for
every $\TypeSubset^r$ and every $\Types \in \TypeSpaces$. Observe that
inequality~\eqref{eq:bk1} being tight for a subset $\TypeSubset$ of
types implies that any ex post allocation mechanism implementing
$\PreAlloc$ must allocate as much as possible to types in
$\TypeSubset$.  By definition, an ordered subset mechanism allocates
to as many types as possible (up to $k$) from each $\TypeSubset^r$;
this is the unique outcome given that inequality~\eqref{eq:bk1} is
tight for every $\TypeSubset^r$.
\end{proof}

\paragraph{Optimization over feasible interim allocation rules.}
The characterization of interim feasibility as a polymatroid
constraint immediately enables efficient solving of optimization
problems over the feasible interim allocation rules as long as we can
compute $\DistOracle_k$ efficiently (see \citet{S03} for optimization
over polymatroids). The following lemma states that $\DistOracle_k$
can be computed efficiently.

\begin{lemma}
\label{lem:g_k}%
For independent agent (i.e., if $\DistProbs$ is a product distribution), $\DistOracle_k(\TypeSubset)$ can be exactly computed in
time $O((n+\Abs{\TypeSubset})\cdot k)$ for any $\TypeSubset \in \UnionTypeSpace$ using dynamic programming.
\end{lemma}

%
%\begin{remark}
%If agents types are dependent, we can still solve optimization problems over $\PreAllocSpace$ in polynomial time, assuming that
%we have oracle access to $\DistOracle_k(\cdot)$, however with dependent agents an optimal multi-agent mechanism in general cannot
%be decomposed to single-agent mechanisms using interim allocation rules.
%\end{remark}


\paragraph{Ex post implementation of feasible interim allocation rules.}

We now address the task of finding an ex post implementation
corresponding to any $\PreAlloc \in \PreAllocSpace$. By
\autoref{thm:border_matroid}, if $\PreAlloc$ is a vertex of
$\PreAllocSpace$, it can be easily implemented by an ordered subset
allocation mechanism (\autoref{def:rank_alloc_k}). As any point in the
polymatroid (or any convex polytope) can be specified as a convex
combination of its vertices, to implement the corresponding interim
allocation rule it is enough to show that this convex combination can
be efficiently sampled from. An ex post implementation can then be
obtained by sampling a vertex and using the ordered subset mechanism
corresponding to that vertex. Instead of explicitly computing this
convex combination, we present a general randomized rounding routine
$\RandRound(\cdot)$ which takes a point in a polymatroid and returns a
vertex of the polymatroid such that the expected value of every
coordinate of the returned vertex is the same as the original
point. This approach is formally described next.

\begin{definition}[randomized ordered subset allocation mechanism]
\label{def:rra_k}%
Parameterized by a normalized interim allocation rule $\PreAlloc
\in \PreAllocSpace$, a \emph{randomized ordered subset allocation
  mechanism (RRA)} computes the allocation for a profile of types
$\Types \in \TypeSpaces$ as follows.
\begin{enumerate}
\item
Let $(\PreAlloc^*, \Ranking^*) \leftarrow \RandRound(\PreAlloc)$.
\item
Run the ordered subset mechanism (\autoref{def:rank_alloc_k}) with
ordered subset $\Ranking^*$.
\end{enumerate}
\end{definition}


\begin{theorem}
\label{thm:rra_k}%
Any normalized interim allocation rule $\PreAlloc \in \PreAllocSpace$
can be implemented by the randomized ordered subset allocation
mechanism (\autoref{def:rra_k}) as a distribution over deterministic
ordered subset allocation mechanisms.
\end{theorem}
\begin{proof}
The proof follows from linearity of expectation.
\end{proof}

\paragraph{Randomized rounding for polymatroids.}
We describe $\RandRound(\cdot)$ for general polymatroids. First, we
present a few definitions and give an overview of the rounding
operator. Consider an arbitrary finite set $\Universe$ and a
polymatroid $\PolyMat{\SubFun}$ associated with a non-decreasing
submodular function $\SubFun : 2^\Universe \to \PosReals$ with
$\SubFun(\emptyset) = 0$. A set $\USubset \subseteq \Universe$ is
called \emph{tight} with respect to a $\PolyVec \in
\PolyMat{\SubFun}$, if and only if
$\PolyVec(\USubset)=\SubFun(\USubset)$. A set
$\TightSets=\{\USubset^0,\ldots,\USubset^{m}\}$ of subsets of
$\Universe$ is called a \emph{nested family of tight sets} with
respect to $\PolyVec \in \PolyMat{\SubFun}$, if and only the elements
of $\TightSets$ can be or ordered and indexed such that $\emptyset =
\USubset^0 \subset \cdots \subset \USubset^m \subseteq \Universe$, and
such that $\USubset^r$ is tight with respect to $\PolyVec$ (for every
$r \in [m])$.


$\RandRound(\PolyVec)$ takes an arbitrary $\PolyVec \in
\PolyMat{\SubFun}$ and iteratively makes small changes to it until a
vertex is reached. At each iteration $\ell$, it computes
$\PolyVec^{\ell} \in \PolyMat{\SubFun}$, and a nested family of tight
sets $\TightSets^\ell$ (with respect to $\PolyVec^{\ell}$) such that
\begin{itemize}
\item
$\Ex{\PolyVec^\ell | \PolyVec^{\ell-1}}=\PolyVec^{\ell-1}$, and

\item
$\PolyVec^\ell$ is closer to a vertex (compared to $\PolyVec^{\ell-1}$) in the sense that either the number of non-zero
coordinates has decreased by one or the number of tight sets has increased by one.
\end{itemize}
Observe that the above process must stop after at most $2\Abs{\Universe}$ iterations\footnote{In fact we will show that it stops
after at most $\Abs{\Universe}$ iterations.}. At each iteration $\ell$ of the rounding process, a vector $\PolyVecDelta \in
\Reals^\Universe$ and $\delta, \delta' \in \PosReals$ are computed such that both $\PolyVec^{\ell-1}+\delta \cdot\PolyVecDelta$
and $\PolyVec^{\ell-1}-\delta'\cdot\PolyVecDelta$ are still in $\PolyMat{\SubFun}$, but closer to a vertex. The algorithm then
chooses a random $\delta'' \in \{\delta, -\delta'\}$ such that $\Ex{\delta''}=0$, and sets $\PolyVec^{\ell} \leftarrow
\PolyVec^{\ell-1}+\delta''\cdot\PolyVecDelta$.


\begin{definition}[$\RandRound(\PolyVec)$]
This operator takes as its input a point $\PolyVec \in \PolyMat{\SubFun}$ and returns as its output a pair $(\PolyVec^*,
\Ranking^*)$, where $\PolyVec^*$ is a random vertex of $\PolyMat{\SubFun}$ and $\Ranking^*$ is its associated ordered subset
(see \autoref{prop:vertex}), and such that $\Ex{\PolyVec^*}=\PolyVec$.

The algorithm modifies $\PolyVec$ iteratively until a vertex is reached. It also maintains a nested family of tight sets
$\TightSets$ with respect to $\PolyVec$. As we modify $\TightSets$, we always maintain an ordered labeling of its elements, i.e.,
if $\TightSets=\{\USubset^0,\ldots,\USubset^m\}$, we assume that $\emptyset=\USubset^0 \subset \cdots \subset \USubset^m
\subseteq \Universe$; in particular, the indices are updated whenever a new tight set is added. For each $\Elem \in \Universe$,
define $\IVec{\Elem} \in [0,1]^\Universe$ as a vector whose value is $1$ at coordinate $\Elem$ and $0$ everywhere else.

\begin{enumerate}
\item
Initialize $\TightSets \leftarrow \{\emptyset\}$.

\item
Repeat each of the following steps until no longer applicable:
\begin{itemize}
\item
If there exist distinct $\Elem, \Elem' \in \USubset^r \setminus \USubset^{r-1}$ for some $r \in [m]$:
\begin{enumerate}
\item
Set $\PolyVecDelta \leftarrow \IVec{\Elem}-\IVec{\Elem'}$, and compute $\delta,\delta' \in \PosReals$ such that
$\PolyVec+\delta\cdot \PolyVecDelta$ has a new tight set $\USubset$ and $\PolyVec-\delta'\cdot \PolyVecDelta$ has a new tight set
$\USubset'$, i.e,
\begin{itemize}
\item
set $\displaystyle \USubset \leftarrow \argmin_{\USubset^{r-1}+\Elem \subseteq \USubset \subseteq \USubset^r-\Elem'}
\SubFun(\USubset)-\PolyVec(\USubset)$, and $\delta \leftarrow \SubFun(\USubset)-\PolyVec(\USubset)$;

\item
set $\displaystyle  \USubset' \leftarrow \argmin_{\USubset^{r-1}+\Elem' \subseteq \USubset' \subseteq \USubset^r-\Elem}
\SubFun(\USubset')-\PolyVec(\USubset')$, and $\delta' \leftarrow \SubFun(\USubset')-\PolyVec(\USubset')$.
\end{itemize}

\item
$\begin{cases}
    \text{with prob. $\frac{\delta}{\delta+\delta'}$: \qquad
        set $\PolyVec \leftarrow \PolyVec+\delta\cdot\PolyVecDelta$, and add $\USubset$ to $\TightSets$.} & \\
    \text{with prob. $\frac{\delta'}{\delta+\delta'}$: \qquad
        set $\PolyVec \leftarrow \PolyVec-\delta'\cdot\PolyVecDelta$, and add $\USubset'$ to $\TightSets$.} &
\end{cases}$
\end{enumerate}

\item
If there exists $\Elem \in \Universe \setminus \USubset^m$ for which $\PolyVec(\Elem) > 0$:
\begin{enumerate}
\item
Set $\PolyVecDelta \leftarrow \IVec{\Elem}$, and compute $\delta,\delta' \in \PosReals$ such that $\PolyVec+\delta\cdot
\PolyVecDelta$ has a new tight set $\USubset$ and $\PolyVec-\delta'\cdot \PolyVecDelta$ has a zero at coordinate $\Elem$, i.e,
\begin{itemize}
\item
set $\displaystyle \USubset \leftarrow \argmin_{\USubset \supseteq \USubset^m+\Elem} \SubFun(\USubset)-\PolyVec(\USubset)$, and
$\delta \leftarrow \SubFun(\USubset)-\PolyVec(\USubset)$;

\item
set $\delta' \leftarrow \PolyVec(\Elem)$.
\end{itemize}

\item
$\begin{cases}
    \text{with prob. $\frac{\delta}{\delta+\delta'}$: \qquad
        set $\PolyVec \leftarrow \PolyVec+\delta\cdot\PolyVecDelta$, and add $\USubset$ to $\TightSets$.} & \\
    \text{with prob. $\frac{\delta'}{\delta+\delta'}$: \qquad
        set $\PolyVec \leftarrow \PolyVec-\delta'\cdot\PolyVecDelta$ } &
\end{cases}$

\end{enumerate}
%
%Find $\delta^+ \in [0,1]$ such that $\PreAlloc'=(\PreAlloc+\delta^+\cdot\IVec{a}) \in \PreAllocSpace$ has a new tight set
%$\TypeSubset'$ where $\TypeSubset^m \subset \TypeSubset' \subseteq  \PreAllocedTypes[\PreAlloc]$ (see \autoref{lem:round_out}).
%Let $\delta^- = \PreAlloc(c)$.
%
%\item
%Set $\PreAlloc^* \leftarrow
%    \begin{cases}
%    \PreAlloc+\delta^+ \cdot \IVec{c}       & \text{with prob.} \; \frac{\delta^+}{\delta^+ + \delta^-} \\
%    \PreAlloc-\delta^- \cdot \IVec{c}       & \text{with prob.} \; \frac{\delta^-}{\delta^+ +\delta^-}
%    \end{cases}$
%\item
%Either $\PreAlloc^*$ has a new tight set $\TypeSubset^*$, in this case add $\TypeSubset^*$ to $\TightSets$, otherwise it must be
%that $\PreAlloc^*(c)=0$. Set $\PreAlloc \leftarrow \PreAlloc^*$.
%\end{enumerate}
%
%
%Find $\delta_{ab} \in [0,1]$ such that for $\PreAlloc'=(\PreAlloc+\delta_{ab}\cdot(\IVec{a}-\IVec{b})) \in \PreAllocSpace$ either
%$\PreAlloc'(b)=0$, or $\PreAlloc'$ has a new tight set $\TypeSubset'$, such that $\TypeSubset^{r-1} \subset \TypeSubset' \subset
%\TypeSubset^r$ (see \autoref{lem:round_in}). Similarly, find $\delta_{ba} \in [0,1]$ such that
%$\PreAlloc''=(\PreAlloc+\delta_{ba}\cdot(\IVec{b}-\IVec{a})) \in \PreAllocSpace$ has either of those properties (switching roles
%of $a$ and $b$).
%
%\item
%Set $\PreAlloc^* \leftarrow
%    \begin{cases}
%    \PreAlloc+\delta_{ab} \cdot (\IVec{a}-\IVec{b})     & \text{with prob.} \; \frac{\delta_{ba}}{\delta_{ab}+\delta_{ba}} \\
%    \PreAlloc+\delta_{ba} \cdot (\IVec{b}-\IVec{a})     & \text{with prob.} \; \frac{\delta_{ab}}{\delta_{ab}+\delta_{ba}}
%    \end{cases}$
%
%\item
%Either $\PreAlloc^*$ has a new tight set $\TypeSubset^*$, in this case add $\TypeSubset^*$ to $\TightSets$, otherwise there must
%be $c \in \{a,b\}$ for which $\PreAlloc^*(c)=0$, remove $c$ from all tight sets in $\TightSets$. Set $\PreAlloc \leftarrow
%\PreAlloc^*$.
%\end{enumerate}
%
%\item
%If there exists $c \in \UnionTypeSpace \setminus \TypeSubset^m$ for which $\PreAlloc(c) > 0$:
%\begin{enumerate}
%\item
%Find $\delta^+ \in [0,1]$ such that $\PreAlloc'=(\PreAlloc+\delta^+\cdot\IVec{a}) \in \PreAllocSpace$ has a new tight set
%$\TypeSubset'$ where $\TypeSubset^m \subset \TypeSubset' \subseteq  \PreAllocedTypes[\PreAlloc]$ (see \autoref{lem:round_out}).
%Let $\delta^- = \PreAlloc(c)$.
%
%\item
%Set $\PreAlloc^* \leftarrow
%    \begin{cases}
%    \PreAlloc+\delta^+ \cdot \IVec{c}       & \text{with prob.} \; \frac{\delta^+}{\delta^+ + \delta^-} \\
%    \PreAlloc-\delta^- \cdot \IVec{c}       & \text{with prob.} \; \frac{\delta^-}{\delta^+ +\delta^-}
%    \end{cases}$
%\item
%Either $\PreAlloc^*$ has a new tight set $\TypeSubset^*$, in this case add $\TypeSubset^*$ to $\TightSets$, otherwise it must be
%that $\PreAlloc^*(c)=0$. Set $\PreAlloc \leftarrow \PreAlloc^*$.
%\end{enumerate}
\end{itemize}

\item Set $\PolyVec^* \leftarrow \PolyVec$ and define the ordered subset $\Ranking^* : \USubset^m \to [m]$ according to $\TightSets$, i.e.,
for each $r \in [m]$ and $\Elem \in \USubset^r \setminus \USubset^{r-1}$, define $\Ranking^*(\Elem) = r$.
\item Return $(\PolyVec^*,\Ranking^*)$.


%Order the elements of $\TypeSubset^m$ as
%$\TypeVar^1,\ldots, \TypeVar^m$ such that $\TypeSubset^r =
%\{\TypeVar^1,\ldots, \TypeVar^r\}$ for every $r \in
%\Range{0}{m}$. This is always possible assuming that neither of the
%above steps can be applied any more. Return $(\PreAlloc ,
%\{\TypeVar^1,\ldots, \TypeVar^m\})$.
\end{enumerate}
\end{definition}


\begin{theorem}
\label{thm:randround}%
For any non-decreasing submodular function $\SubFun : 2^\Universe \to \PosReals$ and any $\PolyVec \in \PolyMat{\SubFun}$, the
operator $\RandRound(\PolyVec)$ returns a random $(\PolyVec^*, \Ranking^*)$ such that $\PolyVec^* \in
\VertexSet{\PolyMat{\SubFun}}$, and $\Ranking^*$ is the ordered subset corresponding to $\PolyVec^*$ (see
\autoref{prop:vertex}), and such that $\Ex{\PolyVec^*}=\PolyVec$. Furthermore, the algorithm runs in strong polynomial time. In
particular, it runs for $O(\Abs{\Universe})$ iterations where each iteration involves solving two submodular minimizations.
\end{theorem}
