\section{Optimization and Implementation of Interim Allocation Rules}
\label{sec:inter}%

In this section we address the computational issues pertaining to (i) solving optimization problems over the space of feasible
interim allocation rules, and (ii) ex post implementation of such a feasible interim allocation rule.  We present computationally
tractable methods for both problems.

\paragraph{Normalized interim allocation rules.}
It will be useful to ``flatten'' the interim allocation rule $\InAllocs$ for which $\InAllocs_i (\type_i)$ denotes the
probability that $i$ with type $\Type_i$ is served (randomizing over the mechanism and the draws of other agent types); we do so
as follows. Without loss of generality, we assume that the type spaces of different agents are disjoint.\footnote{This can be
achieved by labeling all types of
  each agent with the name of that agent, i.e., for each $i\in [n]$ we
  can replace $\TypeSpace_i$ with $\TypeSpace'_i=\{(i,t) | t
  \in \TypeSpace_i\}$ so that $\TypeSpace'_1,\cdots, \TypeSpace'_n$
  are disjoint.}  Denoting the set of all types by $\UnionTypeSpace =
\bigcup_i \TypeSpace_i$, the interim allocation rule can be flattened as a vector in $[0,1]^{\UnionTypeSpace}$.

\begin{definition}
The {\em normalized interim allocation rule} $\PreAlloc \in
[0,1]^{\UnionTypeSpace}$ corresponding to interim allocation rule
$\intallocs$ under distribution $\DistProbs$ is defined as
\begin{align*}
    \PreAlloc(\Type_i) &= \intalloc_i(\Type_i) \DistProb_i(\Type_i)&
    &\forall \Type_i \in \UnionTypeSpace
\end{align*}
\end{definition}
For the rest of this section, we refer to interim allocation rules via $\PreAlloc$ instead of $\InAllocs$.  Note that there is a
one-to-one correspondence between $\PreAlloc$ and $\InAllocs$ as specified by the above linear equation; so any linear of convex
optimization problem involving $\InAllocs$ can be written in terms of $\PreAlloc$ without affecting its linearity or convexity.
As $\InAllocSpace$ denotes the space of feasible interim allocation rules $\InAllocs$, we will use $\PreAllocSpace$ to denote the
space of feasible normalized interim allocation rules.

%%
%% commenting out because this is only used in section border_1, where
%% it is redefined.

%\paragraph{Additional notation.}
%For any subset of agents $\AgentSubset \subseteq \Agents = \{1,\ldots,
%n\}$, define $\TypeSpace_\AgentSubset = \bigcup_{i \in
%  \AgentSubset} \TypeSpace_i$.  Define shorthand notation $\Type_i \in
%\TypeSubset$ where $\TypeSubset \subseteq \UnionTypeSpace$ to quantify
%over all types in $\TypeSubset$ and their corresponding agents (i.e.,
%$\forall \Type_i \in \TypeSubset$ is equivalent to $\forall i \in
%\Agents, \forall \Type_i \in \TypeSubset\cap \TypeSpace_i$).


In the remainder of this section we characterize interim feasibility
and show that normalized interim allocation rules can be optimized
over and implemented in polynomial time.
