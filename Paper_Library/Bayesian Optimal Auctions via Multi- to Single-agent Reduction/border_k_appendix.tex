\section{Proofs from Section~\ref{sec:border_k}}
\label{app:border_k}


\begin{figure}[h]
\centering
\includegraphics[width=0.7\textwidth]{flow_fig}
\caption{The bipartite graph used in the max-flow/min-cut argument of the proof of \autoref{thm:border_k}. The capacities are
indicated on the edges.\label{fig:kflow}}
\end{figure}

\begin{proof}[Rest of the proof of \autoref{thm:border_k}]
We give a proof of $\PolyMat{\DistOracle_k} \subseteq \PreAllocSpace $ based on the min-cut/max-flow theorem. We start by
constructing a directed bipartite graph as illustrated in \autoref{fig:kflow}. On one side we put a node $\SNode{\Types}$, for
each type profile $\Types \in \TypeSpaces$. On the other side we put a node $\SNode{\Type_i}$, for each type $\Type_i \in
\TypeSpace_{\Range{1}{n}}$. We also add a source node $\Src$ and a sink node $\Snk$. We add a directed edge from $\Src$ to the
node $\SNode{\Types}$, for each $\Types \in \TypeSpaces$ and set the capacity of this edge to $k\cdot\DistProbs(\Types)$. We also
add $n$ outgoing edges for every node $\SNode{\Types}$, each one going to one of the nodes
$\SNode{\Type_1},\ldots,\SNode{\Type_n}$ and with a capacity of $\DistProbs(\Types)$. Finally we add a directed edge from the
node $\SNode{\Type_i}$, for each $\Type_i \in \TypeSpace_{\Range{1}{n}}$, to $\Snk$ with capacity of $\PreAlloc(\Type_i)$.
Consider a maximum flow from $\Src$ to $\Snk$. It is easy to see that there exists a feasible ex post implementation for
$\PreAlloc$ if and only if all the edges to the sink node $\Snk$ are saturated. In particular, if $\rho(\Types,\Type_i)$ denotes
the amount of flow from $\SNode{\Types}$ to $\SNode{\Type_i}$, a feasible ex post implementation can be obtained by allocating to
each type $\Type_i$ with probability $\rho(\Types,\Type_i)/\DistProbs(\Types)$ when the type profile $\Types$ is reported by the
agents.

We show that if a feasible ex post implementation does not exist, then $\PreAlloc \not \in \PolyMat{\DistOracle_k}$. Observe that
if a feasible ex post implementation does not exist, then some of the incoming edges of $\Snk$ are not saturated by the max-flow.
Let $(A, B)$ be a minimum cut such that $\Src \in A$ and $\Snk \in B$. Let $B'=B \cap \UnionTypeSpace$. We show that the
polymatroid inequality
\begin{align}
    \PreAlloc(B') &\le \DistOracle_k(B') \label{eq:bk3}
\end{align}
must have been violated. It is easy to see that the size of the cut is given by the following equation.
\begin{align*}
    \Cut(A, B) &=
        \sum_{\Types \in \TypeSpaces \cap A}\#\left\{i \middle| \Type_i \in B\right\}\DistProbs(\Types)
            +\sum_{\Types \in \TypeSpaces\cap B}k\cdot\DistProbs(\Types)+\sum_{\TypeVar \in \UnionTypeSpace\cap A}\PreAlloc(\TypeVar)
\end{align*}
Observe that for each $\Types \in \TypeSpaces \cap A$, it must be that $\#\{i | \Type_i \in B\} \le k$, otherwise moving
$\SNode{\Types}$ to $B$ would decrease the size of the cut. So the size of the minimum cut can be in simply written as:
\begin{align*}
    \Cut(A, B) &=
        \sum_{\Types \in \TypeSpaces} \min\left(\#\left\{i \middle| \Type_i \in B\right\},k\right) \DistProbs(\Types)+\sum_{\TypeVar \in
        \UnionTypeSpace\cap A}\PreAlloc(\TypeVar)
\end{align*}
On the other hand, since some of the incoming edges of $\Snk$ are not saturated by the max-flow, it must be that
\begin{align*}
    \sum_{\TypeVar\in \UnionTypeSpace} \PreAlloc(\TypeVar) &= \Cut(A\cup B-\Snk, \Snk) > \Cut(A, B),
\intertext{so}
    \sum_{\TypeVar \in \UnionTypeSpace \cap B}\PreAlloc(\TypeVar) &>
        \sum_{\Types \in \TypeSpaces} \min\left(\#\left\{i \middle| \Type_i \in B\right\},k\right) \DistProbs(\Types).
\end{align*}
The right hand side of the above inequality is the same as $\Ex[\Types \sim \DistProbs]{\min(\#\{i | \Type_i \in B\}, k)}$ which
shows that polymatroid inequality~\eqref{eq:bk3} of $\PolyMat{\DistOracle_k}$ is violated so $\PreAlloc \not \in
\PolyMat{\DistOracle_k}$. That completes the proof.
\end{proof}
