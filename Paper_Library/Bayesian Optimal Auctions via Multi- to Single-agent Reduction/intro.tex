\section{Introduction}

Classical economics and game theory give fundamental characterizations
of the structure of competitive behavior.  For instance, Nash's
\citeyearpar{N51} theorem shows that mixed equilibrium gives a complete
description of strategic behavior, and the Arrow-Debreu \citeyearpar{AD54}
theorem shows the existence of market clearing prices in multi-party
exchanges.  In these environments computational complexity has offered
further perspective.  In particular, mixed equilibrium in general
games can be computationally hard to find \citep{CD06,DGP09}, whereas market
clearing prices are often easy to find \citep{DPSV08,J04}.  In this paper we
investigate an analogous condition for auction theory due to Kim
\citet{B91}, give a computationally constructive generalization that
further illuminates the structure of auctions, and thereby show that
the theory of optimal auctions is tractable.

Consider an abstract optimal auction problem.  A seller faces a set of
agents.  Each agent desires service and there may be multiple ways to
serve each agent (e.g., when renting a car, you can get a GPS or not,
you can get various insurance packages, and you will pay a total
price).  Each agent has preferences over the different possible ways
she can be served and we refer to this preference as her type.  The
seller is restricted by the {\em feasibility} constraint that at most
one agent can be served (e.g., only one car in the rental shop).  When
the agents' types are drawn independently from a known prior
distribution, the seller would like to design an auction to optimize
her objective, e.g., revenue, in expectation over this distribution,
subject to feasibility.  Importantly, in this abstract problem we have
not made any of the following standard assumptions on the agents'
preferences: quasi-linearity, risk-neutrality, or
single-dimensionality.


%
% Example
%


%
% IC
%
We assume that agents behave strategically and we will analyze an
auction's performance in {\em Bayes-Nash equilibrium}, i.e., where
each agent's strategy is a best response to the other agents'
strategies and the distribution over their preferences.  Without loss
of generality the revelation principle \citep{M81} allows for the
restriction of attention to {\em Bayesian incentive compatible} (BIC)
mechanisms, i.e., ones where the truthtelling strategy is a Bayes-Nash
equilibrium.


%
% decomposition to interim allocation rules.
%
Any auction the seller proposes can be decomposed across the agents as
follows.  From an agent's perspective, the other agents are random
draws from the known distribution.  Therefore, the composition of
these random draws (as inputs), the bid of the agent, and the
mechanism induce an {\em interim allocation rule} which specifies the
probability the agent is served as a function of her bid.  BIC implies
that the agent is at least as happy to bid her type
%% Tiger:  replaced "over" by "as"
%over
as
any other bid.

%
% feasibility
%
Applying the same argument to each agent induces a profile of interim
allocation rules.  These interim allocation rules are jointly feasible
in the sense that there exists an auction that, for the prior
distribution, induces them.  As an example, suppose an agent's type is
high or low with probability $1/2$ each.  Consider two interim
allocation rules: rule (a) serves the agent with probability one when
her type is high and with probability zero otherwise, and rule (b)
serves the agent with probability $1/2$ regardless of her type.  It is
feasible for both agents to have rule (b) or for one agent to have
rule (a) and the other to have rule (b); on the other hand, it is
infeasible for both agents to have rule (a).  This last combination is
infeasible because with probability one quarter both agents have high
types but we cannot simultaneously serve both of them.  An important
question in the general theory of auctions is to decide when a profile
of interim allocation rules is feasible, and furthermore, when it is
feasible, to find an auction that implements it.

%
% md-via-sd
%
A structural characterization of the necessary and sufficient conditions
for the aforementioned {\em interim feasibility} is important for the
construction of optimal auctions as it effectively allows the auction
problem to be decomposed across agents.  If we can optimally serve a
single agent for a given interim allocation rule and we can check
feasibility of a profile of interim allocation rules, then we can
optimize over auctions.
%% Tiger:  replaced the colon after "Effectively" by a comma
Effectively, we can reduce the multi-agent
auction problem to a collection of single-agent auction
problems.

%
% borders
%

We now informally describe Border's \citeyearpar{B91} characterization
of interim feasibility.  A profile of interim allocation rules is
implementable if for any subspace of the agent types the expected
number of items served to agents in this subspace is at most the
probability that there is an agent with type in this subspace.
Returning to our infeasible example above, the probability that there
is an agent with a high type is $3/4$ while the expected number of
items served to agents with high types is one; Border's condition is
violated.

%
% discussion
%
The straightforward formulation of interim feasibility via Border's
characterization has exponentially many constraints.  Nonetheless, it
can be simplified to a polynomial number of constraints in single-item
auctions with symmetric agents (where the agents' type space and
distribution are identical).  This simplification of the
characterization has lead to an analytically tractable theory of
auctions when agents have budgets \citep{LR96} or are risk averse
\citep{M84,MR84}.


%
% main theorem
%
\paragraph{Results}
Our main theorem is to show computationally tractable (i.e., in polynomial time in the total number of agents' types) methods for
each of the following problems. First, a given profile of interim allocation rules can be checked for interim feasibility.
Second, for any feasible profile of interim allocation rules, an auction (i.e., ex post allocation rule) that induces these
interim allocation rules can be constructed.  In particular, for problems where the seller can serve at most one agent, we show
that the exponentially-faceted polytope specified by the interim feasibility constraints is a projection of a
quadratically-faceted polytope in a higher dimension, and an ex post allocation rule implementing a feasible profile of interim
allocation rules is given immediately by the latter's preimage in the higher dimensional polytope. In particular, this implies
that optimal interim allocation rules can be computed by solving a quadratically sized linear/convex program. These results
combine to give a (computationally tractable) reduction from the multi-agent auction problem to a collection of single-agent
problems. Furthermore, our algorithmic procedure characterizes every single service auction as implementable by a simple token
passing game.

%
% additional results.
%
We also generalize the interim feasibility characterization and use it
to design optimal auctions for the cases where the seller faces a
$k$-unit feasibility constraint, i.e., at most $k$~agents can be
served, and more generally to matroid feasibility constraints.  Our
generalization of the feasibility characterization is based on a
simpler network-flow-based approach.  Although the number of
constraints in this characterization is exponential in the sizes of
type spaces, we observe that the constraints define a polymatroid,
whose vertices correspond to particularly simple auctions which can be
implemented by simple determinist allocation rules based on ranking.
Algorithms for submodular function minimization give rise to fast
separation oracles which, given a set of interim allocation rules,
detect a violated feasibility constraint whenever there is one;
expressing any point in the polymatroid as a convex combination of the
vertices allows us to implement any feasible interim allocation rule
as a distribution over the simple auctions.  These enable us again to
reduce the multi-agent problem to single-agent problems.


%From this generalization we are able to again reduce the multi-agent problem to a single-agent problem.
%but only in the case of symmetric agents (identical type space and distribution).

%
% conclusion
%
Auction theory is very poorly understood outside the standard
single-dimensional quasi-linear revenue maximization environment of
\citet{M81}.  The main consequence of this work is that even without
analytical understanding, optimal auctions can be computationally
solved for in environments that include non-quasi-linear utility
(e.g., budgets or risk aversion) and multi-dimensional preferences
(assuming that the corresponding single-agent problem can be solved).
Furthermore, unlike most work in auction theory with budgets or
risk-aversion, our framework permits the budgets or risk parameters to
be private to the agents.

%\paragraph{Our Contribution (Technical).}
%
%We consider optimal mechanism design problems in environments where agents types are distributed independently and the only
%coupling constraint among the agents is that at most $k$ agents can be served. In such environments, we show that the optimal
%multi-agent mechanism design problem can be reduced to single-agent problems. The decomposition is based on using \emph{interim
%allocation rules}. For each agent, an interim allocation rule specifies (for each type in the type space of the agent) an
%(upper-bound) interim probability of serving that agent. An optimal mechanism can be constructed by computing the optimal interim
%allocation rules and computing an optimal single-agent mechanism for each agent subject to her corresponding interim allocation
%rule. To obtain a computationally tractable construction, we need computationally efficient approaches for (a) solving the
%single-agent problems subject to any given interim allocation rules, (b) optimizing over the space of all jointly feasible
%interim allocation rules, and (c) computing an ex post implementation of the resulting interim allocation rules. Our main
%contribution is to provide computationally tractable methods for (b) and (c).
%
%For the case of $k=1$ (i.e., assuming that at most one agent can be served), it was known (\citet{B91}) that the space of
%feasible profiles of interim allocation rules is a $\NumTypes$-dimensional convex polytope represented by $2^\NumTypes$ linear
%inequalities where $\NumTypes$ is the total number of types of all agents. We show this polytope is in fact the projection of a
%higher dimensional polytope which can be represented by just $O(\NumTypes^2)$ linear constraints. Consequently, we can compute
%the optimal interim allocation rules by a linear/convex program over this polytope in polynomial time. Furthermore, we show that
%computing a pre-image of the optimal interim allocation rules in the higher dimensional polytope, immediately gives an ex post
%implementation.
%
%For the case of $k\ge 1$ (i.e., assuming that at most $k$ agent can be served), we present necessary and sufficient conditions
%for feasibility of profiles of interim allocation rules. Our characterization implies that the space of feasible profiles of
%interim allocation rules can still be represented by $2^\NumTypes$ linear inequalities. We show that this polytope can
%equivalently be represented by a polymatroid. Consequently, the optimal interim allocation rules can still be compute by a
%linear/convex program over this polytope in polynomial time. Furthermore, we show that any profile of interim allocation rules
%that correspond a vertex of this polytope can be implemented by a simple deterministic ex post allocation rule. Consequently, any
%feasible profile of interim allocation rule corresponding to a point inside the polymatroid can be implemented by a careful
%randomized rounding (using $O(\NumTypes)$ calls to the separation oracle) to obtain a vertex and by using using the deterministic
%allocation rule corresponding to that vertex. For the case of $k=1$, we also present a separation or that runs in time
%$O(\NumTypes \log \NumTypes)$, which implies that any feasible profile of interim allocation rules can also be implemented in
%time $O(\NumTypes^2 \log \NumTypes)$ using our randomized rounding implementation.

\Xcomment{
The contrast between the economic understanding of optimal mechanisms
in environments with multi- and single-dimensional preferences is
severe.

There are two main challenges imposed by traditional environments for
multi-dimensional mechanism design.  First, there is the inherent
multi-dimensionality of the preference space.  This
multi-dimensionality prevents simplification of the incentive
compatibility constraints to enable analytical tractability.  The
second, is the multi-dimensionality of the inter-agent externalities.
E.g., if there are two items for sale, an agent could impose
externality on the other by taking the first, the second, or both
items.

In contrast, environments with single-dimensional preferences are
analytically tractable because the incentive compatibility constraints
simplify and because the inter-agent externalities are
single-dimensional.  An agent is either served or not, and if an agent
is served then the same externality is imposed on the other agents.

We restrict attention to environments for which the inter-agent
externalities are single-dimensional but the agents may otherwise have
multi-dimensional preferences.  We refer to these environments as {\em
  service constrained environments}.  There are a number of relevant
examples of these environments.  First, single-item auctions wherein
agents have a private value and private budget.  Second, single-item
auctions wherein agents have a private value and private risk
preferences.  Third, single-item multi-unit auctions wherein agents
have preferences over unconstrained ``add-ons.''  Classic examples of
an add-on market is the one for automobiles where a dealer can provide
the automobile with combinations of various options such as audio
systems and colors, or a restaurant that has limited seating capacity
but an unlimited menu.

Our results are directly comparable in form to standard environments
with single-dimensional preferences.  First, for revenue maximization
with agents with private value for service and quasi-linear utilities,
\citet{M81} characterized the optimal mechanism as a {\em virtual
  surplus maximizer}.  I.e., the agents' values should first be
transformed by a weakly monotone virtual valuation function, and then
the agents with maximum virtual surplus should served.  For the
special case of single-item auctions, maximizing virtual surplus means
allocating the item to the agent with the highest positive virtual
value.  In fact, \citet{BR89} reinterpret these virtual values as
``marginal revenue.'' Second, for revenue maximization with
single-dimensional agents a common (publicly known) budget,
\citet{LR96} showed that the optimal mechanism is again based on
ordering the agents by some criteria (like Myerson's virtual values),
but this criteria depends on market conditions (unlike Myerson's
virtual values).

We show that multi-dimensional preferences can also be partitioned
into two classes (based only on each agent's distribution over
preferences and not on the designer's feasibility constraint).  We
refer to the two classes as {\em contextual} and {\em context free}.  In
the context-free case, the optimal mechanism is a virtual surplus
maximizer and furthermore, these virtual values have a ``marginal
revenue'' interpretation.  In the contextual case, the optimal
mechanism is based on a linear ordering of the agents, though this
ordering depends on market conditions, i.e., the context.

Our approach to these multi-dimensional mechanism design problems is
through reduction from multi- to single-agent problems.  This
reduction is closely related to the ``marginal revenue''
reinterpretation of optimal mechanisms by \citet{BR89}.  Given a
single agent with private type drawn from a distribution, we can
analyze the revenue obtainable from this agent as a function of the ex
ante probability of sale, i.e., the {\em revenue curve}.  This
analysis question is a single-agent problem.  Myerson's virtual values
are exactly the derivative of this revenue curve, i.e., the marginal
revenue.  We generalize the Myerson-Bulow-Roberts result by showing
for (multi-dimensional) context-free preferences, the multi-agent
optimal mechanism optimizes the marginal revenue of the agents served.
Therefore, if the single-agent problem of optimizing revenue for any
ex ante service probability can be solved then so can the multi-agent
problem.

To understand contextual environments, we must formalize the
constraint that the context imposes.  An {\em interim service rule} is
a monotone (non-increasing) function from type probabilities to
service probabilities.  More specifically, it gives an upper-bound on
the probability that a random type conditioned to be in a subspace of
type space can be served.  A natural single-agent problem is to
maximize expected revenue subject to an interim service rule.  To
reduce the multi-agent problem to this single-agent problem we need to
show how to combine interim service rules (from the different agents)
to produce an ex post service rule.  The inequality of \citet{B91}
characterizes when a set of interim service rules are combinable.  In
symmetric environments the Border's condition gives linear (in the
number of agents) number of constraints; while, in asymmetric
environments the number of constraints is generally exponential.  Our
structural results are enabled by an algorithmic characterization of
Border's inequality with polynomial complexity.  This algorithmic
characterization can (a) check for the feasibility of a collection of
interim service rules and (b) and find a corresponding ex post service
rule if the interim rules are feasible.

Notice that the single-agent problem discussed in the previous
paragraph is the special case where the interim service rule is a step
function.  The distinction between context-free and contextual
preferences, then, is the distinction between whether the expected
performance is linear in the interim service rule.  If it is linear,
then the optimal mechanism for any interim service rule is a convex
combination of the optimal mechanism for step functions and its
revenue is given by the (same) convex combination of the marginal
revenues.


}
