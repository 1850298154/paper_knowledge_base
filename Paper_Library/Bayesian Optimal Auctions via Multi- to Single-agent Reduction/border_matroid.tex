
\subsection{Matroid Feasibility Constraints}
\label{sec:border_matroid}%


In this section, we consider environments where the feasibility constraints are encoded by a matroid $\Mat=(\UnionTypeSpace,
\IndepSets)$. For every type profile $\Types \in \TypeSpaces$, a subset $\TypeSubset \subseteq \{\Type_1,\ldots, \Type_n\}$ can
be simultaneously allocated to if and only if $\TypeSubset \in \IndepSets$. We show that the results of \autoref{sec:border_k}
can be easily generalized to environments with matroid feasibility constraints.


\paragraph{Matroid Preliminaries.}
A matroid $\Mat=(\Universe, \IndepSets)$ consists of a ground set $\Universe$ and a family of independent sets $\IndepSets
\subseteq 2^\Universe$ with the following two properties.
\begin{itemize}
\item
For every $\IndepSet, \IndepSet'$, if $\IndepSet' \subset \IndepSet \in \IndepSets$, then $\IndepSet' \in \IndepSets$.

\item
For every $\IndepSet,\IndepSet' \in \IndepSets$, if $\Abs{\IndepSet'} < \Abs{\IndepSet}$, there exists $\Elem \in \IndepSet
\setminus \IndepSet'$ such that $\IndepSet'\cup \{\Elem\} \in \IndepSets$.
\end{itemize}

For every matroid $\Mat$, the rank function $\Rank{\Mat} : 2^\Universe \to \mathds{N}\cup\{0\}$ is defined as follows: for each
$\USubset \subseteq \Universe$, $\Rank{\Mat}(\USubset)$ is the size of the maximum independent subset of $\USubset$. A matroid
can be uniquely characterized by its rank function, i.e., a set $\IndepSet \subseteq \Universe$ is an independent set if and only
if $\Rank{\Mat}(\IndepSet) = \Abs{\IndepSet}$. A matroid rank function has the following two properties:
\begin{itemize}
\item $\Rank{\Mat}(\cdot)$ is a non-negative non-decreasing integral submodular function.
\item $\Rank{\Mat}(\USubset) \le \Abs{\USubset}$, for all $\USubset \subseteq \Universe$.
\end{itemize}
Furthermore, every function with the above properties defines a matroid.

Any set $\USubset \subseteq \Universe$ can be equivalently represented by its incidence vector $\chi^\USubset \in
\{0,1\}^\Universe$ which has a $1$ at every coordinate $\Elem \in \USubset$ and $0$ everywhere else.

\begin{proposition}
Consider an arbitrary finite matroid $\Mat=(\Universe, \IndepSets)$ with rank function $\Rank{\Mat}(\cdot)$. Let
$\PolyMat{\Rank{\Mat}}$ denote the polymatroid associated with $\Rank{\Mat}(\cdot)$ (see \autoref{sec:border_k}); the vertices of
$\PolyMat{\Rank{\Mat}}$ are exactly the incidence vectors of the independent sets of $\Mat$.
\end{proposition}

See~\citet{S03} for a comprehensive treatment of matroids.


%% \paragraph{Rank-based allocation mechanisms.}
%% As in \autoref{sec:border_k}, the class of rank-based allocation mechanisms are of particular importance. Assuming that the
%% feasibility constraints are given by a matroid $\Mat=(\UnionTypeSpace, \IndepSets)$ with rank function $\Rank{\Mat}$, the class
%% of deterministic rank-based allocation mechanisms of \autoref{def:incidence_k} can be generalized as follows.

%% \begin{definition}[deterministic rank-based allocation mechanism (matroid version)]
%% \label{def:rank_alloc_matroid} Parameterized by a \emph{partial ranking} $\Ranking$ over $\UnionTypeSpace$ (see
%% \autoref{def:ranking}), a {\em deterministic rank-based allocation mechanism}, on type profile $\Types \in \TypeSpaces$, ranks
%% the agents by their types according to $\Ranking$, and greedily allocates to them in the order of this ranking, i.e, the set of
%% allocated types is a maximal independent set for every prefix of this ranking. In particular, any type which is not in the domain
%% of $\Ranking$ is never allocated to. Formally: let $\RvAlloc{\Types} = \Vertex{\Ranking}{\PolyMat{\Rank[\Types]{\Mat}}}$, where
%% $\Rank[\Types]{\Mat}(\TypeSubset)=\Rank{\Mat}(\Types \cap \TypeSubset)$ (for every $\TypeSubset \subseteq \UnionTypeSpace$); the
%% ex post allocation for each agent $i$ is exactly $\RvAlloc{\Types}(\Type_i)$.
%% \end{definition}


\paragraph{Characterization of interim feasibility.}
We now generalize the characterization of interim feasibility as the
polymatroid given by the expected rank of the matroid.  From this
generalization the computational results of the preceeding section can
be extended from $k$-unit environments to matroids.

%Assuming that the feasibility constraints are given by a matroid
%$\Mat=(\UnionTypeSpace, \IndepSets)$ with rank function $\Rank{\Mat}$,
%the definition of expected rank, and the characterization of interim
%feasibility can be generalized as follows.

Let $\RandBits$ denote the random bits used by an ex post allocation
rule, and let $\RvAlloc{\Types,\RandBits} \in \{0,1\}^\UnionTypeSpace$
denote the ex post allocation rule (i.e., the incidence vector of the
subset of types that get allocated to) for type profile $\Types \in
\TypeSpaces$ and random bits $\RandBits$. It is easy to see that
$\RvAlloc{\Types,\RandBits}$ is a feasible ex post allocation if and
only if it satisfies the following class of inequalities.
\begin{align}
    \RvAlloc{\Types,\RandBits}(\TypeSubset) & \le \Rank{\Mat}(\Types \cap \TypeSubset), &
        &  \forall \Types \in \TypeSpaces, \forall \TypeSubset \subseteq \UnionTypeSpace
        \label{eq:bm1}
\end{align}
The above inequality states that the subset of types that get
allocated to must be an independent set of the restriction of matroid
$\Mat$ to $\{\Type_1,\ldots,\Type_n\}$.  The expectation of the
left-hand-side is exactly the normalized interim allocation rule,
i.e., for any $\type'_i \in \UnionTypeSpace$, $\PreAlloc(\type'_i) =
\expect[\Types,\RandBits]{\RvAlloc{\Types,\RandBits}(\type'_i)}$.
Taking expectations of both sides of \eqref{eq:bm1} then motivates the
following definition and theorem that characterize interim
feasibility.

\begin{definition}
\label{def:incidence_matroid} The {\em expected rank} for distribudion $\denss$, subspace $S \subset \UnionTypeSpace$, and matroid $\Mat$ with rank function $\Rank{\Mat}$ is
\begin{align}
    \DistOracle_\Mat(\TypeSubset) &= \Ex[\Types \sim
      \DistProbs]{\Rank{\Mat}(\Types \cap \TypeSubset)}
    \tag{$\DistOracle_\Mat$} \label{eq:dist_oracle_matroid}
\end{align}
where $\Types \cap \TypeSubset$ denotes $\{\Type_1,\ldots,\Type_n\} \cap \TypeSubset$.
\end{definition}

\begin{theorem}
\label{thm:border_matroid}%
For matroid $\Mat$ and distribution $\denss$,
the space of all feasible normalized interim allocation rules, $\PreAllocSpace$, is the polymatroid associated with
$\DistOracle_k$, i.e., $\PreAllocSpace = \PolyMat{\DistOracle_\Mat}$, i.e., for all $\PreAlloc \in \PreAllocSpace$, 
\begin{align}
    \PreAlloc(\TypeSubset) &\le \Ex[\Types]{\Rank{\Mat}(\Types \cap \TypeSubset)} = \DistOracle_\Mat(\TypeSubset), & \forall \TypeSubset \subseteq \UnionTypeSpace
    \label{eq:bm2}
\end{align}
\end{theorem}

\begin{theorem}
\label{thm:border_matroid-unique}
For matroid $\Mat$, if $\PreAlloc \in \PreAllocSpace$ is the vertex
$\Vertex{\Ranking}{\PolyMat{\DistOracle_\Mat}}$ of the polymatroid
$\PolyMat{\DistOracle_\Mat}$ the unique ex post implementation is the
ordered subset mechanism induced by $\Ranking$
(\autoref{def:rank_alloc_k}).
\end{theorem}

To prove the above theorems, we use the following decomposition lemma which applies to general polymatroids.

\begin{lemma}[Polymatroidal Decomposition]
\label{lem:decompose}%
Let $\Universe$ be an arbitrary finite set, $\SubFun^1,\ldots,\SubFun^m : 2^\Universe \to \PosReals$ be arbitrary non-decreasing
submodular functions, and $\SubFun^*=\sum_{j=1}^m \lambda^j \SubFun^j$ be an arbitrary convex combination of them. For every
$\PolyVec^*$ the following holds: $\PolyVec^* \in \PolyMat{\SubFun^*}$ if and only if  it can be decomposed as $\PolyVec^* =
\sum_{j=1}^m \lambda^j \PolyVec^j$ such that $\PolyVec^j \in \PolyMat{\SubFun^j}$ (for each $j \in [m]$). Furthermore, if
$\PolyVec^*$ is a vertex of $\PolyMat{\SubFun^*}$, this decomposition is unique. More precisely, if
$\PolyVec^*=\Vertex{\Ranking}{\PolyMat{\SubFun^*}}$ for some ordered subset $\Ranking$, then
$\PolyVec^j=\Vertex{\Ranking}{\PolyMat{\SubFun^j}}$ (for each $j \in [m])$.
\end{lemma}
\begin{proof}
First, observe that the only-if part is obviously true, i.e., if $\PolyVec^j \in \PolyMat{\SubFun^j}$ (for each $j \in [m]$), we
can write
\begin{align}
    \PolyVec^j(\USubset) & \le \SubFun^j(\USubset) & & \forall \USubset \subseteq \Universe, \label{eq:pmd1}
\intertext{multiplying both sides by $\lambda^j$ and summing over all $j\in [m]$ we obtain}
    \PolyVec^*(\USubset)=\sum_{i=1}^m \lambda^j \PolyVec^j(\USubset) & \le \sum_{j=1}^m \lambda^j \SubFun^j(\USubset)=\SubFun^*(\USubset)  & & \forall \USubset \subseteq
    \Universe,
    \label{eq:pmd2}
\end{align}
which implies that $\PolyVec^* \in \PolyMat{\SubFun^*}$.

Next, we prove that for every $\PolyVec^* \in \PolyMat{\SubFun^*}$
such a decomposition exists. Note that a polymatroid is a convex
polytope, so any $\PolyVec^* \in \PolyMat{\SubFun^*}$ can be written
as a convex combination of vertices as $\PolyVec^*=\sum_{\ell}
\alpha^\ell {\PolyVec^*}^\ell$, where each ${\PolyVec^*}^\ell$ is a
vertex of $\PolyMat{\SubFun^*}$; consequently, if we prove the claim
for the vertices of $\PolyMat{\SubFun^*}$, i.e., that
${\PolyVec^*}^\ell=\sum_{j=1}^m \lambda^j \PolyVec^{\ell,j}$ for some
$\PolyVec^{\ell,j} \in \PolyMat{\SubFun^j}$, then a decomposition of
$\PolyVec^* = \sum_{j=1}^m \lambda^j \PolyVec^j$ can be obtained by
setting $\PolyVec^j = \sum_\ell \alpha^\ell \PolyVec^{\ell,j}$.

Next, we prove the second part of the theorem which also implies that
a decomposition exists for every vertex of $\PolyMat{\SubFun^*}$. Let
$\PolyVec^*=\Vertex{\Ranking}{\PolyMat{\SubFun^*}}$ be an arbitrary
vertex of $\PolyMat{\SubFun^*}$ with a corresponding ordered subset
$\Ranking$; by \autoref{prop:vertex}, such a $\Ranking$ exists for
every vertex of a polymatroid. For every integer $r \leq |\Ranking|$,
define $\TypeSubset^r =\{\RElem_1,\ldots,\RElem_r\}$ as the
$r$-element prefix of the ordering.  By \autoref{prop:vertex},
inequality~\eqref{eq:pmd2} is tight for every $\USubset^r$, which
implies that inequality~\eqref{eq:pmd1} must also be tight for each
$\USubset^r$ and for every $j \in [m]$. Consequently, for each $r \in [|\Ranking|]$, and each $j \in [m]$, by taking the difference
of the inequality~\eqref{eq:pmd2} for $\USubset^{r}$ and
$\USubset^{r-1}$, given that they are tight, we obtain
\begin{align*}
    \PolyVec^j(\Elem) & =\SubFun^j(\USubset^{r})-\SubFun^j(\USubset^{r-1})
\end{align*}
Furthermore, for each $\Elem \not \in \Ranking$, $\PolyVec^*(\Elem)=0$ which implies that $\PolyVec^j(\Elem)=0$ for
every $j \in [m]$. Observe that we have obtained a unique $\PolyVec^j$ for each $j \in [m]$ which is exactly the vertex of
$\PolyMat{\SubFun^j}$ corresponding to $\Ranking$ as described in \autoref{prop:vertex}. It is easy to verify that indeed
$\PolyVec^* = \sum_{j=1}^m \lambda^j \PolyVec^j$.
\end{proof}

\begin{proof}[Proof of \autoref{thm:border_matroid}]
The inequality in equation \eqref{eq:bm1} states that the subset of types that get allocated to must be an independent set of the restriction of
matroid $\Mat$ to $\{\Type_1,\ldots,\Type_n\}$. Define $\Rank[\Types ]{\Mat}(\TypeSubset) = \Rank{\Mat}(\Types \cap \TypeSubset)$
(for all $(\TypeSubset \subseteq \UnionTypeSpace$). Notice that $\Rank[\Types ]{\Mat}$ is a submodular function. The above
inequality implies that $\RvAlloc{\Types,\RandBits} \in \PolyMat{\Rank[\Types ]{\Mat}}$. Define $\RvAlloc{\Types} =
\Ex[\RandBits]{\RvAlloc{\Types,\RandBits}}$. Observe that $\PreAlloc=\Ex[\Types]{\RvAlloc{\Types}}$, so $\PreAlloc$ is a
feasible normalized interim allocation rule if and only if it can be decomposed as $\PreAlloc = \sum_{\Types \in \TypeSpaces}
\DistProb(\Types) \RvAlloc{\Types}$ where $\RvAlloc{\Types} \in \PolyMat{\Rank[\Types]{\Mat}}$ for every $\Types \in
\TypeSpaces$; by \autoref{lem:decompose}, this is equivalent to $\PreAlloc \in \PolyMat{\DistOracle_{\Mat}}$ where
$\DistOracle_\Mat(\TypeSubset) = \sum_{\Types \in \TypeSpaces} \DistProb(\Types) \Rank[\Types]{
\Mat}(\TypeSubset)=\Ex[\Types]{\Rank[\Types]{\Mat}(\TypeSubset)}$ (for all $\TypeSubset \subseteq \UnionTypeSpace$), as defined
in \autoref{def:incidence_matroid}. That completes the proof of the first part of the theorem.
\end{proof}



\begin{proof}[Proof of \autoref{thm:border_matroid-unique}]
Suppose $\PreAlloc=\Vertex{\Ranking}{\PolyMat{\DistOracle_\Mat}}$ for
some ordered subset $\Ranking$. By \autoref{lem:decompose}, the
decomposition of $\PreAlloc$ is unique and is given by
$\RvAlloc{\Types} =
\Vertex{\Ranking}{\PolyMat{\Rank[\Types]{\Mat}}}$. Notice that this is
the same allocation obtained by the deterministic rank-based
allocation mechanism which ranks according to $\Ranking$ (see
\autoref{def:rank_alloc_k}).
\end{proof}


\paragraph{Optimization over feasible interim allocation rules.}

As in \autoref{sec:border_k}, the characterization of interim
feasibility as a polymatroid constraint immediately enables efficient
solving of optimization problems over the feasible normalized interim
allocation rules as long as we can compute $\DistOracle_\Mat$
efficiently (see \citet{S03} for optimization over
polymatroids). Depending on the specific matroid, it might be possible
to exactly compute $\DistOracle_\Mat$ in polynomial time (e.g., as in
\autoref{lem:g_k}); otherwise, it can be computed approximately within
a factor of $1-\epsilon$ and with probability $1-\delta$, by sampling,
in time polynomial in $\frac{1}{\epsilon}$ and $\frac{1}{\delta}$.


\paragraph{Ex post implementation of feasible interim allocation rules.}
An ex post implementation for any $\PreAlloc \in \PreAllocSpace$ can
be obtained exactly as in \autoref{sec:border_k}. 
%Note that by
%\autoref{thm:border_matroid}, if $\PreAlloc$ is a vertex of
%$\PreAllocSpace$, it can be easily implemented by a deterministic
%ranked-based allocation mechanism. Since $\PreAllocSpace$ is a convex
%polytope, any $\PreAlloc \in \PreAllocSpace$ can be specified as a
%convex combination of the vertices of $\PreAllocSpace$. Therefore, to
%implement $\PreAlloc$ it is enough to sample from this convex
%combination, and then use the deterministic ranked-based allocation
%mechanism corresponding to the sampled vertex.


%\begin{definition}[randomized rank-based allocation mechanism (matroid version)]
%\label{def:rra_matroid}%
%Parameterized by a normalized interim allocation rule $\PreAlloc \in \PreAllocSpace$, a \emph{randomized rank-based allocation
%mechanism (RRA)} computes the allocation for a type profile $\Types \in \TypeSpaces$ as follows.
%\begin{enumerate}
%\item
%Let $(\PreAlloc^*, \Ranking^*) \leftarrow \RandRound(\PreAlloc)$.
%\item
%Use the deterministic rank-based allocation mechanism(\autoref{def:rank_alloc_matroid}) with partial ranking $\Ranking^*$.
%\end{enumerate}
%\end{definition}


\begin{corollary}
\label{thm:rra_matroid}%
Any normalized interim allocation rule $\PreAlloc \in \PreAllocSpace$
can be implemented by the randomized rank-based allocation mechanism
(\autoref{def:rra_k}) as a distribution over deterministic
rank-based allocation mechanisms (\autoref{def:rank_alloc_k}).
\end{corollary}

