\begin{abstract}
We study an abstract optimal auction problem for a single good or
service.  This problem includes environments where agents have
budgets, risk preferences, or multi-dimensional preferences over
several possible configurations of the good (furthermore, it allows an
agent's budget and risk preference to be known only privately to the
agent). These are the main challenge areas for auction theory.  A
single-agent problem is to optimize a given objective subject to a
constraint on the maximum probability with which each type is
allocated, a.k.a., an allocation rule.  Our approach is a reduction
from multi-agent mechanism design problem to collection of
single-agent problems.  We focus on maximizing revenue, but our
results can be applied to other objectives (e.g., welfare).


An optimal multi-agent mechanism can be computed by a linear/convex
program on interim allocation rules by simultaneously optimizing several
single-agent mechanisms  subject to joint feasibility of the
allocation rules.  For single-unit auctions, Border \citeyearpar{B91}
showed that the space of all jointly feasible interim allocation rules
for $n$~agents is a $\NumTypes$-dimensional convex polytope which can
be specified by $2^\NumTypes$ linear constraints, where $\NumTypes$ is
the total number of all agents' types.  Consequently, efficiently
solving the mechanism design problem requires a separation oracle for
the feasibility conditions and also an algorithm for ex-post
implementation of the interim allocation rules.  We show that the
polytope of jointly feasible interim allocation rules is the
projection of a higher dimensional polytope which can be specified by
only $O(\NumTypes^2)$ linear constraints.  Furthermore, our proof
shows that finding a preimage of the interim allocation rules in the
higher dimensional polytope immediately gives an ex-post
implementation.


We generalize Border's result to the case of $k$-unit and matroid
auctions.  For these problems we give a separation-oracle based
algorithm for optimizing over feasible interim allocation rules and a
randomized rounding algorithm for ex post implementation.  These ex
post implementations have a simple form; they are randomizations over
simple greedy mechanisms.  Given a ordered subset of agent types, such
a greedy mechanisms serves types in the specified order.





%
%We study an abstract optimal auction problem for a single good or
%service.  This problem includes environments where agents have
%budgets, risk preferences, or multi-dimensional preferences over
%several possible configurations of the good (furthermore, it allows an
%agent's budget and risk preference to be only known privately to the
%agent).  These are the main challenge areas for auction theory.  Our
%approach is a reduction from multi-agent mechanism design problem to
%single-agent problems of revenue maximization given an interim
%allocation rule.  The reduction consists of an polynomial time algorithm
%that implements interim allocation rules whenever they are jointly
%feasible.  This joint feasibility, known as Border's \citeyearpar{B91}
%condition generally governs exponentially many constraints (and was
%previously known only for unit supply).  We give a computationally
%constructive proof of Border's condition using network flow and derive
%a separation oracle, which can determines joint feasibility and an
%allocation oracle which gives an ex post allocation and configuration.
%
%We further generalize Border's condition to the case of multi-unit
%supply.  From this generalization we give a similar computationally
%tractable reduction from the multi-agent auction problem to a
%single-agent problem when the agents are symmetric.



% We generalize the ``marginal revenue'' approach to optimal mechanism
% design to environments with multi-dimensional agent preferences.  This
% approach can be viewed as a reduction from multi- to single-agent
% mechanism design.  For a single agent, marginal revenue describes the
% change in performance as a function of the ex ante probability that
% the agent is served.  When the distribution of preferences is well
% behaved, the optimal mechanism for multiple agents is to serve the
% agents with the highest marginal revenue.  This is a generalization of
% Myerson's (1981) virtual-value-based construction.
%
% When the distribution of preferences is not well behaved, the marginal
% revenue approach fails, and so does any virtual-value-based
% construction.  In these environments we consider performance as a
% function of the interim allocation rule.  Given a construction of the
% performance-optimal mechanism for any interim allocation rule for a
% single agent, the optimal mechanism for multiple agents can be
% constructed.  The construction requires joint feasibility of the
% interim allocation rule.  This joint feasibility is known as Border's
% (1991) condition.  In general, Border's condition has exponentially
% many constraints; however, we give a polynomial time separation oracle
% from which the optimal mechanism can be tractably constructed.



%% We study optimal mechanism design for multi-parameter agents with
%% single-dimensional inter-agent externalities.  Such settings include
%% single item auctions where bidders have private budgets, and single
%% service auctions where bidders compete for one chance to be provided
%% one of multiple services.  Our approach is a reduction from
%% multi-agent mechanism design problem to single-agent problems of
%% revenue maximization given an interim allocation rule.  The reduction
%% consists of an efficient algorithm that implements interim allocation
%% rules whenever they are jointly feasible.  This joint feasibility,
%% known as Border's (1991) condition in single-dimensional settings,
%% includes exponentially many constraints in general.  We generalize the
%% condition to multi-dimensional cases and give an efficient separation
%% oracle.

%% We also give Border's condition for auctions selling $k$~chances of
%% service, and show an efficient algorithm that computes optimal
%% auctions in this case for agents whose types are drawn from
%% i.i.d.\@ distributions.
\end{abstract}
