\section{Conclusions and Extensions}

\label{sec:conc}

% linear utility in expected offer
% correlation.
% stochastic feasibility
% additive values
% symmetry

In this paper we have focused on binary allocation problems where an
agent is either served or not served.  For these binary allocation
problems distributions over allocations are given by a single number,
i.e., the probability that the agent is served.  Our results can be
extended to environments with multi-unit demand when the agents
utility is linear in the expected number of units the agent receives.

In Section~\ref{sec:inter} we described algorithms for
optimizing over feasible interim allocation rules and for (ex post)
implementation of the resulting rules.  Neither these algorithms nor
the generalization of Border's condition require the types of the
agents to be independently distributed.  However, our formulation of
incentive compatibility for interim allocation rules does require
independence.  For correlated distributions the interim allocation
rule is a function of the actual type of the agent (which conditions
the types of the other agents) and the reported type of the agent.
Therefore, this generalization of our theorem to correlated
environments has little relevance for mechanism design.

The algorithms in Section~\ref{sec:inter} do not require the
feasibility constraint to be known in advance.  A simple example where
this generalization is interesting is a multi-unit auction where the
supply $k$ is stochastically drawn from a known distribution.  Our
result shows that the optimal auction in such an environment can be
described by picking the random ordering on types and allocating
greedily by this ordering while supplies last.  We do not know of many
examples other than this where this generalization is interesting.

Our techniques can also be used in conjunction with the approach of
\citet{CDW11} for solving multi-item auction problems for agents with
additive values.

One important extension of our work is to scenarios where the type
space and distribution are only available via oracle access (and can
be very large or even infinite).  Given a polynomial time
approximation scheme for a variant of the single agent problem we can
construct a polynomial time approximation scheme for the multi-agent
problem.  Such a model is important, for instance, when the agents
type space is multi-dimensional but succinctly describable, e.g., for
unit demand agents with independently distributed values for various
items for sale.  Such a type space would be exponentially large in the
number of items but succinctly described in polynomial space in the
number of items.  While reduction can be applied to this scenario;
however, we do not know of any solution to the optimal single-agent
problem with which to instantiate the reduction.


%% Finally, in symmetric environments, i.e., when the agents' type
%% distribution and the designer's feasibility constraint are symmetric,
%% e.g., for i.i.d.~multi-unit auctions, the optimal interim allocation
%% rule is symmetric.  In terms of the normalized allocation rules of
%% Section~\ref{sec:ptas} the optimal feasible allocation rule is
%% invariant to the problem or objective.  It is induced by the
%% ``strongest (in terms of quantile) agent wins'' ex post allocation
%% rule.  Therefore, symmetric multi-agent problems reduce to solving the
%% single-agent problem for a very specific normalized interim allocation
%% rule.  In comparison to the computational task of optimizing over
%% feasible interim allocation rules and solving for ex post
%% implementations, e.g., from Section~\ref{sec:inter}, this multi-agent
%% reduction for symmetric environments is computationally trivial.
