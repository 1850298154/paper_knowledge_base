\paragraph{Related Work}

\citet{M81} 
characterized Bayesian optimal auctions in environments with
quasi-linear risk-neutral single-dimensional agent
preferences.  \citet{BR89} reinterpreted Myerson's approach as reducing
the multi-agent auction problem to a related single-agent problem.
Our work generalize this reduction-based approach to single-item
multi-unit auction problems with general preferences.

An important aspect of our approach is that it can be applied to
general multi-dimensional agent preferences.  Multi-dimensional
preferences can arise as distinct values different configurations of
the good or service being auctioned, in specifying a private budget
and a private value, or in specifying preferences over risk.  We
briefly review related work for agent preferences with multiple
values, budgets, or risk parameters.


Multi-dimensional valuations are well known to be difficult.  For
example, \citet{RC98}, showed that, because {\em
bunching}\footnote{Bunching refers to the situation in which a group
of distinct types are treated the same way in by the mechanism.} can
not be ruled out easily, the optimal auctions for multi-dimensional
valuations are dramatically different from those for single
dimensional valuations.  Because of this, most results are for cases with special structure \citep[e.g.,][]{A96, W93, MM88} and
often, by using such structures, reduce the problems to
single-dimensional ones \citep[e.g.,][]{S80, R79, MS80}.  Our
framework does not need any such structure.
%azarakhsh
%For example, in~\citet{A96}, the utility function is continuous,
%convex and homogeneous of degree one in the type of the
%agent (additive utility). Wilson~\citeyearpar{W93, W96, W97}
%used similar model as Armstrong.  He derived first order conditions for an incentive scheme
%to be optimal.

%I used the description of armstrong here for wilson work since it is a book
% in the \w96, he used single crossing property and then described the necessary conditions and basically
%tried to use the property that imposes a strict ordering over types
%\citet{MM88} proposed the ``generalized single crossing''
%property, under which the necessary condition for a differentiable
%incentive compatible scheme to be implementable is also sufficient;
%they also considered the case of single buyer and two items and 
%characterized the optimal solution under some hazard rate condition. \citet{S80} considered the general demand
%function with multiple products and multi-dimensional types.
%However, he only solves the problem for the subcase in which the
%allocation is fixed and the only problem left is how to charge consumers
%optimally. \citet{R79} considered an $n+1$ goods economy. However,
%the last good is used to produce the first $n$~items for a given public
%utility function of the seller. As a result, the problem can be cast
%as a single dimensional problem. \citet{MS80} again considered
%multiple products.  However the type of each agent is defined by a
%scalar (which translated into single dimensional type space).
%azarakhsh


A number of papers consider optimal auctions for agents with budgets
\citep[see, e.g.,][]{PV08, CG95, M00}.  These papers rely on
budgets being public or the agents being symmetric; our technique
allows for a non-identical prior distribution and private budgets.
Mechanism design with risk averse agents was studied by
\citet{MR84} and \citet{Matthews83}.  Both works assume i.i.d.\@ prior
distributions and have additional assumptions on risk attitudes; our
reduction does not require these assumptions.

Our work is also related to a line of work on approximating the
Bayesian optimal mechanism.  These works tend to look for simple
mechanisms that give constant (e.g., two) approximations to the optimal
mechanism.  \citet{CHK07}, \citet{BCKW-10}, and \citet{CD11} consider
item pricing and lottery pricing for a single agent; the first two
give constant approximations the last gives a
$(1+\epsilon)$-approximation for any $\epsilon$.  These problems are
related to the single-agent problems we consider.
\citet{CHMS10} and \citet{BGGM10} extend these approaches to
multi-agent auction problems.  The point of view of reduction from
multi- to single-agent presented in this paper bears close
relationship to recent work by \citet{A11} who gives a reduction from
multi- to single-agent mechanism design that loses at most a constant
factor of the objective.  Our reductions, employing entirely different
techniques, give rise to optimal mechanisms instead of approximations
thereof.  


Characterization of interim feasibility plays a vital role in this
work.  For single-item single-unit auctions, necessary and sufficient
conditions for interim feasibility were developed through a series of
works \citep{MR84, M84, B91, B07, M11}; this characterization has
proved useful for deriving properties of mechanisms, \citet{MV10}
being a recent example. \citet{B91} characterized symmetric interim
feasible auctions for single-item auctions with identically
distributed agent preferences.  His characterization is based on the
definition of ``hierarchical auctions.''  He observes that the space
of interim feasible mechanisms is given by a polytope, where vertices
of this polytope corresponding to hierarchical auctions, and interior
points corresponding to convex combinations of vertices.  \citet{M11}
generalize Border's approach and characterization to asymmetric
single-item auctions.  The characterization via hierarchical auctions
differs from our characterization via ordered subset auctions in that
hierarchical auctions allow for some types to be relatively unordered
with the semantics that these unordered types will be considered in a
random order; it is important to allow for this when solving for
symmetric auctions.  Of course convex combinations over hierarchical
auctions and ordered subset auctions provide the same generality.  Our
work generalizes the characterization from asymmetric single-unit
auctions to asymmetric multi-unit and matroid auctions.

Our main result provides computational foundations to the interim
feasibility characterizations discussed above.  We show that interim
feasibility can be checked, that interim feasible allocation rules can
be optimized over, and that corresponding ex post implementations can
be found.  Independently and contemporaneously \citet{CDW11} provided
similar computational foundations for the single-unit auction problem.
Their approach to the single-unit auction problem is most comparable
to our approach for the multi-unit and matroid auction problems where
the optimization problem is written as a convex program which can be
solved by the ellipsoid method; while these methods result in strongly
polynomial time algorithms they are not considered practical.  In
contrast, our single-unit approach, when the single-agent problems can
be solved by a linear program, gives a single linear program which can
be practically solved.


While our work gives computationally tractable interim feasibility
characterizations in ``service based'' environments like multi-unit
auctions and matroid auctions; \citet{CDW11} generalize the approach
to multi-item auctions with agents with additive preferences.  The
problem of designing an optimal auction for agents with
multi-dimensional additive preferences is considered one of the main
challenges for auction theory and their result, from a computational
perspective, solves this problem.



