\section{Limitations and Future Work} \label{sec:limitation}
\vspace{-0.2cm}
Despite good performance in perceptive locomotion and local navigation, our approach has several limitations. 
First, our high-level policy can suffer from mode collapse, habitually relying on a single low-level gait and using residuals only to adapt. 
Incorporating exploration strategies or diversity-driving rewards during training may alleviate this issue and encourage the policy to exploit the full range of low-level motion skills.

Second, our training scenarios remain somewhat constrained. 
Future work could be extending our low-level policy to a wider range of motor behaviors (e.g. jumping, crawling) and improving the adaptability for even more challenging terrains, such as gaps, stepping stones, and environments with overhanging obstacles. 

In addition, although this work primarily focuses on active learning to command low-level motions for high-level tasks and demonstrates adaptation with a limited set of motions, we believe the proposed training architecture also offers an efficient and scalable framework for enhancing skill adaptation.
This can be achieved by randomly sampling latent encodings of different motions during training while manually commanding fixed, specific motions for high-level task inference.