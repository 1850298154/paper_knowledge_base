\section{Appendix}

In the appendix we provide further qualitative analysis and implementation details, including the prompts used in our experiments.

\subsection{Qualitative Examples}

We discuss several other qualitative examples from the agent with search.


\begin{figure*}[h]
    \centering
    \includegraphics[width=\linewidth]{images/qualitative_wa_14.pdf}
    \vspace{-0.2in}
    \caption{WA task \#14 is an example where performing more exploration helps the model to identify a trajectory that is likely to be more successful than others.}
    \label{fig:qualitative_14}
\end{figure*}


\paragraph{Enabling exploration} A significant advantage of models with search is their ability to explore larger parts of the environment compared to models without search. Fig.~\ref{fig:qualitative_14} part of the search tree for WebArena task \#14 (in the CMS environment), where the model is able to take multiple plausible actions at the first step (actions 1, 2, 3, and 4 in the graph), and expand the search tree to find the best trajectory ($3 \rightarrow 5 \rightarrow 6 \rightarrow 10$, which achieves the highest value of 0.68). In this case, the model terminates after hitting the search budget $c$ (rather than finding a state with value of 1.0), committing to the best found trajectory thus far, which is successful. This also highlights that our value function does not need to be perfect for search to be helpful.


\begin{figure*}[t]
    \centering
    \includegraphics[width=\linewidth]{images/qualitative_96.pdf}
    \vspace{-0.2in}
    \caption{VWA shopping task \#96 is another example where search allows the model to be more robust to sampling bad actions. On this task, the baseline agent without search failed, but the agent with search is able to prune less promising trajectories (faded nodes in the figure) to identify the successful one.}
    \label{fig:qualitative_96}
\end{figure*}


\paragraph{Improving robustness} As discussed in Sec.~\ref{sec:qualitative}, the baseline agent can be prone to selecting bad samples from the language model due to randomness from nucleus sampling. Search allows the agent to explore each possibility and identify the best trajectories. VWA shopping task \#96 (shown in Fig.~\ref{fig:qualitative_96}) is another example. The baseline agent fails on this task, but the agent with search avoids the first two trajectories (ending at actions 3 and 4) due to low values assigned after exploring the subsequent states. It is able to prune these and identify a successful trajectory (highlighted in Fig.~\ref{fig:qualitative_96}).


\subsection{Additional Ablations}

\subsubsection{Value Function Ablations} \label{appendix:vf_ablations}

\begin{table*}[t]
\centering
\resizebox{1.0\linewidth}{!}{%
\begin{tabular}{lllcccccc}
  \toprule
   & \textbf{Agent Model} & \textbf{Value Function} & \textbf{Max Steps}  & \textbf{No Search} & \textbf{+ Search} & \textbf{$\Delta$} \\
  \midrule
  \multirow{7}{*}{VWA} & Llama-3-70B-Instruct~\citep{koh2024visualwebarena} & - & \multirow{2}{*}{30}  & 9.8\% & - & - \\
  % & GPT-4o-mini + SoM & - & & 7.3\% & - & - \\
  & GPT-4o + SoM~\citep{koh2024visualwebarena} & - & & 19.8\% & - & - \\
  \cmidrule(lr){2-7}
  & Llama-3-70B-Instruct + captions & LLaVA-1.6-34B &  \multirow{5}{*}{5} & 7.6\% & 13.5\% & +77.6\% \\
  & Llama-3-70B-Instruct + captions & GPT-4o & & 7.6\% & 16.7\% & +119.7\% \\
  & Llama-3.1-70B-Instruct + captions & GPT-4o & & 9.1\% & 16.2\% & +78.0\% \\
  & GPT-4o-mini + SoM & GPT-4o-mini & & 9.1\% & 14.4\% & +58.2\% \\
  & GPT-4o + SoM & GPT-4o &   & 18.9\% & \textbf{26.4\%} & +39.7\% \\
  \midrule
  \multirow{10}{*}{WA}  
  & GPT-4o~\citep{zhou2023webarena} & - & \multirow{6}{*}{30} & 13.1\% & - & - \\
  % & GPT-4~\citep{zhou2023webarena} & - &  & 14.9\% & - & - \\
  & GPT-4 + Reflexion~\citep{pan2024autonomous} & - &  & 15.6\% & - & - \\
  & AutoWebGLM~\citep{lai2024autowebglm} & - &  & 18.2\% & - & - \\
  & AutoEval~\citep{pan2024autonomous} & - &  & 20.2\% & - & - \\
  & BrowserGym~\citep{drouin2024workarena} & - &  & 23.5\% & - & - \\
  & SteP~\citep{sodhi2024step} & - &  & \textbf{35.8\%} & - & - \\
  \cmidrule(lr){2-7}
  & Llama-3-70B-Instruct & GPT-4o & \multirow{2}{*}{5} & 7.6\% & 10.1\% & +32.3\% \\
  % & GPT-4o-mini & GPT-4o-mini &  & \jykoh{X\%} & \jykoh{X\%} & \jykoh{+X\%} \\
  & GPT-4o & GPT-4o &  & 15.0\% & 19.2\% & +28.0\% \\
  \bottomrule
\end{tabular}
}
\caption{Success rates (SR) and relative change ($\Delta$) for baseline models and models that employ search on the VisualWebArena (VWA)~\citep{koh2024visualwebarena} and WebArena (WA)~\citep{zhou2023webarena} benchmarks. We also show other published approaches. Search substantially improves our baseline models, setting a new state-of-the-art on VWA.}
\label{tab:vf_results}
\end{table*}

In Sec.~\ref{sec:results} of the main paper, we experimented with using gpt-4o as our value function. In Tab.~\ref{tab:vf_results}, we present results using different language models as the agent models and the value functions. We observe that our tree search algorithm is effective across a range of different model sizes and capabilities. In particular, our approach applied to the Llama-3-70B-Instruct and LLaVA-1.6-34B value function yields a 77.6\% relative improvement over the baseline Llama-3-70B-Instruct agent on VWA (7.6\% to 13.5\%), and is a fully open sourced and reproducible baseline. For the GPT-4o-mini model (a relatively weaker model compared to GPT-4o) we also observed improvements when it is used as both the agent model and the value function, improving performance by 58.2\% over the no-search baseline on VWA (9.1\% to 14.4\%).



\subsection{Implementation Details}  \label{appendix:implementation}
\subsubsection{Language Model Agents}

\begin{figure*}[th]
\noindent\fbox{\parbox{\textwidth}{%
\small
You are an autonomous intelligent agent tasked with navigating a web browser. You will be given web-based tasks. These tasks will be accomplished through the use of specific actions you can issue.
\\~\\
Here's the information you'll have: \\
The user's objective: This is the task you're trying to complete.\\
The current web page screenshot: This is a screenshot of the webpage, with each interactable element assigned a unique numerical id. Each bounding box and its respective id shares the same color.\\
The observation, which lists the IDs of all interactable elements on the current web page with their text content if any, in the format [id] [tagType] [text content]. tagType is the type of the element, such as button, link, or textbox. text content is the text content of the element. For example, [1234] [button] ['Add to Cart'] means that there is a button with id 1234 and text content 'Add to Cart' on the current web page. [] [StaticText] [text] means that the element is of some text that is not interactable.\\
The current web page's URL: This is the page you're currently navigating.\\
The open tabs: These are the tabs you have open.\\
The previous action: This is the action you just performed. It may be helpful to track your progress.
\\~\\
The actions you can perform fall into several categories:
\\~\\
Page Operation Actions:\\
\texttt{\`}\texttt{\`}\texttt{\`}click [id]\texttt{\`}\texttt{\`}\texttt{\`}: This action clicks on an element with a specific id on the webpage.\\
\texttt{\`}\texttt{\`}\texttt{\`}type [id] [content]\texttt{\`}\texttt{\`}\texttt{\`}: Use this to type the content into the field with id. By default, the ``Enter'' key is pressed after typing unless press\_enter\_after is set to 0, i.e., \texttt{\`}\texttt{\`}\texttt{\`}type [id] [content] [0]\texttt{\`}\texttt{\`}\texttt{\`}.\\
\texttt{\`}\texttt{\`}\texttt{\`}hover [id]\texttt{\`}\texttt{\`}\texttt{\`}: Hover over an element with id.\\
\texttt{\`}\texttt{\`}\texttt{\`}press [key\_comb]\texttt{\`}\texttt{\`}\texttt{\`}:  Simulates the pressing of a key combination on the keyboard (e.g., Ctrl+v).\\
\texttt{\`}\texttt{\`}\texttt{\`}scroll [down]\texttt{\`}\texttt{\`}\texttt{\`} or \texttt{\`}\texttt{\`}\texttt{\`}scroll [up]\texttt{\`}\texttt{\`}\texttt{\`}: Scroll the page up or down.
\\~\\
Tab Management Actions:\\
\texttt{\`}\texttt{\`}\texttt{\`}new\_tab\texttt{\`}\texttt{\`}\texttt{\`}: Open a new, empty browser tab.\\
\texttt{\`}\texttt{\`}\texttt{\`}tab\_focus [tab\_index]\texttt{\`}\texttt{\`}\texttt{\`}: Switch the browser's focus to a specific tab using its index.\\
\texttt{\`}\texttt{\`}\texttt{\`}close\_tab\texttt{\`}\texttt{\`}\texttt{\`}: Close the currently active tab.
\\~\\
URL Navigation Actions:\\
\texttt{\`}\texttt{\`}\texttt{\`}goto [url]\texttt{\`}\texttt{\`}\texttt{\`}: Navigate to a specific URL.\\
\texttt{\`}\texttt{\`}\texttt{\`}go\_back\texttt{\`}\texttt{\`}\texttt{\`}: Navigate to the previously viewed page.\\
\texttt{\`}\texttt{\`}\texttt{\`}go\_forward\texttt{\`}\texttt{\`}\texttt{\`}: Navigate to the next page (if a previous 'go\_back' action was performed).
\\~\\
Completion Action:\\
\texttt{\`}\texttt{\`}\texttt{\`}stop [answer]\texttt{\`}\texttt{\`}\texttt{\`}: Issue this action when you believe the task is complete. If the objective is to find a text-based answer, provide the answer in the bracket.
\\~\\
Homepage:\\
If you want to visit other websites, check out the homepage at http://homepage.com. It has a list of websites you can visit.\\
http://homepage.com/password.html lists all the account name and password for the websites. You can use them to log in to the websites.
\\~\\
To be successful, it is very important to follow the following rules:\\
1. You should only issue an action that is valid given the current observation\\
2. You should only issue one action at a time.\\
3. You should follow the examples to reason step by step and then issue the next action.\\
4. Generate the action in the correct format. Start with a ``In summary, the next action I will perform is'' phrase, followed by action inside \texttt{\`}\texttt{\`}\texttt{\`}\texttt{\`}\texttt{\`}\texttt{\`}. For example, ``In summary, the next action I will perform is \texttt{\`}\texttt{\`}\texttt{\`}click [1234]\texttt{\`}\texttt{\`}\texttt{\`}''.\\
5. Issue stop action when you think you have achieved the objective. Don't generate anything after stop.
}
}
\caption{System message from \cite{koh2024visualwebarena} in our SoM agent.}  \label{fig:som_system_message}
\end{figure*}

\begin{figure*}[th]
\noindent\fbox{\parbox{\textwidth}{%
\small
{\color{red}{example\_user:}} \\
IMAGES: (1) current page screenshot\\
OBSERVATION:\\
\lbrack31\rbrack\ [IMG]\ \lbrack Image, description: hp fx-7010dn fax machine, url: B08GKZ3ZKD.0.jpg\rbrack\\
\lbrack32\rbrack\ [A]\ \lbrack HP CB782A\#ABA 640 Inkjet Fax Machine (Renewed)\rbrack\\
\lbrack\ \rbrack\ [StaticText]\ \lbrack\$279.49\rbrack\\
\lbrack33\rbrack\ [BUTTON]\ \lbrack Add to Cart\rbrack\\
\lbrack34\rbrack\ [A]\ \lbrack Add to Wish List\rbrack\\
\lbrack35\rbrack\ [A]\ \lbrack Add to Compare\rbrack\\
URL: http://onestopmarket.com/office-products/office-electronics.html\\
OBJECTIVE: What is the price of HP Inkjet Fax Machine?\\
PREVIOUS ACTION: None
\\
{\color{red}{example\_assistant:}} Let's think step-by-step. This page list the information of HP Inkjet Fax Machine, which is the product identified in the objective. Its price is \$279.49. I think I have achieved the objective. I will issue the stop action with the answer. In summary, the next action I will perform is \texttt{\'}\texttt{\'}\texttt{\'}stop [\$279.49]\texttt{\'}\texttt{\'}\texttt{\'}
\\
{\color{red}{example\_user:}} \\
IMAGES: (1) current page screenshot\\
OBSERVATION:\\
\lbrack\ \rbrack\ [StaticText]\ \lbrack/f/food\rbrack\\
\lbrack\ \rbrack\ [StaticText]\ \lbrack[homemade] Obligatory Halloween Pumpkin Loaf!\ \ \ Submitted by\ \ kneechalice\ \ t3\_yid9lu\ \ 1 year ago\rbrack\\
\lbrack9\rbrack\ [IMG]\ \lbrack\ \rbrack\\
\lbrack\ \rbrack\ [StaticText]\ \lbrack Submitted by\ \ kneechalice\ \ t3\_yid9lu\ \ 1 year ago\rbrack\\
\lbrack10\rbrack\ [A]\ \lbrack kneechalice\rbrack\\
\lbrack11\rbrack\ [A]\ \lbrack 45 comments\rbrack\\
\lbrack\ \rbrack\ [StaticText]\ \lbrack[I ate] Maple Pecan Croissant\ \ \ Submitted by\ \ AccordingtoJP\ \ t3\_y3hrpn\ \ 1 year ago\rbrack\\
\lbrack14\rbrack\ [IMG]\ \lbrack\ \rbrack\\
\lbrack\ \rbrack\ [StaticText]\ \lbrack Submitted by\ \ AccordingtoJP\ \ t3\_y3hrpn\ \ 1 year ago\rbrack\\
\lbrack15\rbrack\ [A]\ \lbrack AccordingtoJP\rbrack\\
\lbrack16\rbrack\ [A]\ \lbrack 204 comments\rbrack\\
URL: http://reddit.com\\
OBJECTIVE: Tell me what the top comment on the croissant post says.\\
PREVIOUS ACTION: None
\\
{\color{red}{example\_assistant:}} Let's think step-by-step. This page has a post titled '\lbrack I ate\rbrack \ Maple Pecan Croissant', which is the post mentioned in the objective. In order to find the top comment, I will navigate into the comments section of the post. In summary, the next action I will perform is \texttt{\'}\texttt{\'}\texttt{\'}click \lbrack11\rbrack\texttt{\'}\texttt{\'}\texttt{\'}
\\
{\color{red}{example\_user:}} \\
IMAGES: (1) current page screenshot\\
OBSERVATION:\\
\lbrack\ \rbrack\ [StaticText]\ \lbrack What are you looking for today?\rbrack\\
\lbrack5\rbrack\ [INPUT]\ \lbrack\ \rbrack\\
\lbrack6\rbrack\ [SELECT]\ \lbrack Select a category\rbrack\\
\lbrack7\rbrack\ [BUTTON]\ \lbrack Search\rbrack\\
\lbrack\ \rbrack\ [StaticText]\ \lbrack Latest Listings\rbrack\\
\lbrack\ \rbrack\ [StaticText]\ \lbrack Atlas Powered Audio System w/ Tripod\ \ 150.00 \$\ \ Music instruments\ \ Borough of Red Lion\ \ (Pennsylvania)\ \ 2023/11/16\rbrack\\
\lbrack8\rbrack\ [IMG]\ \lbrack Atlas Powered Audio System w/ Tripod\rbrack\\
\lbrack9\rbrack\ [A]\ \lbrack Atlas Powered Audio System w/ Tripod\rbrack\\
\lbrack\ \rbrack\ [StaticText]\ \lbrack 150.00 \$\rbrack\\
\lbrack\ \rbrack\ [StaticText]\ \lbrack Neptune Gaming Console\ \ 350.00 \$\ \ Video gaming\ \ Pennwyn\ \ (Pennsylvania)\ \ 2023/11/16\rbrack\\
URL: http://classifieds.com\\
OBJECTIVE: Help me find the cheapest dark colored guitar.\\
PREVIOUS ACTION: None\\
{\color{red}{example\_assistant:}} Let's think step-by-step. The objective is to find the cheapest dark colored guitar on the site. The site has a search box whose ID is \lbrack5\rbrack. I can search for guitars by entering ``guitar''. I can submit this by pressing the Enter afterwards. In summary, the next action I will perform is \texttt{\'}\texttt{\'}\texttt{\'}type \lbrack5\rbrack\ \lbrack guitar\rbrack\  \lbrack1\rbrack\texttt{\'}\texttt{\'}\texttt{\'}
}
}
\caption{In-context examples and prompt adopted from \cite{koh2024visualwebarena} for our SoM agents. The webpage screenshots with SoM are added to the end of each round of the {\color{red}{example\_user}} dialogue.}  \label{fig:som_prompt}
\end{figure*}
\begin{figure*}[th]
\noindent\fbox{\parbox{\textwidth}{%
\small
{\color{red}{example\_user:}} \\
OBSERVATION:\\
\lbrack1744\rbrack\ [link]\ \lbrack 'HP CB782A\#ABA 640 Inkjet Fax Machine (Renewed)'\rbrack\\
\lbrack1749\rbrack\ [StaticText]\ \lbrack \$279.49\rbrack\\
\lbrack1757\rbrack\ [button]\ \lbrack 'Add to Cart'\rbrack\\
\lbrack1760\rbrack\ [button]\ \lbrack 'Add to Wish List'\rbrack\\
\lbrack1761\rbrack\ [button]\ \lbrack 'Add to Compare'\rbrack\\
URL: http://onestopmarket.com/office-products/office-electronics.html \\
OBJECTIVE: What is the price of HP Inkjet Fax Machine?\\
PREVIOUS ACTION: None\\
{\color{red}{example\_assistant:}} Let's think step-by-step. This page list the information of HP Inkjet Fax Machine, which is the product identified in the objective. Its price is \$279.49. I think I have achieved the objective. I will issue the stop action with the answer. In summary, the next action I will perform is \texttt{\'}\texttt{\'}\texttt{\'}stop [\$279.49]\texttt{\'}\texttt{\'}\texttt{\'}
{\color{red}{example\_user:}} \\
IMAGES: (1) current page screenshot\\
OBSERVATION:\\
\lbrack204\rbrack\ [heading]\ \lbrack '/f/food'\rbrack\\
\lbrack593\rbrack\ [heading]\ \lbrack '[homemade] Obligatory Halloween Pumpkin Loaf!'\rbrack\\
\ \ \ \ \lbrack942\rbrack\ [link]\ \lbrack '[homemade] Obligatory Halloween Pumpkin Loaf!'\rbrack\\
\ \ \ \ \lbrack945\rbrack\ [StaticText]\ \lbrack 'Submitted by '\rbrack\\
\ \ \ \ \lbrack30\rbrack\ [link]\ \lbrack 'kneechalice' expanded: False\rbrack\\
\ \ \ \ \lbrack1484\rbrack\ [StaticText]\ \lbrack 't3\_yid9lu'\rbrack\\
\ \ \ \ \lbrack949\rbrack\ [time]\ \lbrack 'October 31, 2022 at 10:10:03 AM EDT'\rbrack\\
\ \ \ \ \lbrack1488\rbrack\ [StaticText]\ \lbrack '1 year ago'\rbrack\\
\ \ \ \ \lbrack1489\rbrack\ [link]\ \lbrack '45 comments'\rbrack\\
\lbrack605\rbrack\ [heading]\ \lbrack '[I ate] Maple Pecan Croissant'\rbrack\\
\ \ \ \ \lbrack963\rbrack\ [link]\ \lbrack '[I ate] Maple Pecan Croissant'\rbrack\\
\ \ \ \ \lbrack966\rbrack\ [StaticText]\ \lbrack 'Submitted by '\rbrack\\
\ \ \ \ \lbrack37\rbrack\ [link]\ \lbrack 'AccordingtoJP' expanded: False\rbrack\\
\ \ \ \ \lbrack1494\rbrack\ [StaticText]\ \lbrack 't3\_y3hrpn'\rbrack\\
\ \ \ \ \lbrack970\rbrack\ [time]\ \lbrack 'October 13, 2022 at 10:41:09 PM EDT'\rbrack\\
\ \ \ \ \lbrack1498\rbrack\ [StaticText]\ \lbrack '1 year ago'\rbrack\\
\ \ \ \ \lbrack1499\rbrack\ [link]\ \lbrack '204 comments'\rbrack\\
URL: \texttt{http://reddit.com}\\
OBJECTIVE: Tell me what the top comment on the croissant post says.\\
PREVIOUS ACTION: None\\
{\color{red}{example\_assistant:}} Let's think step-by-step. This page has a post titled '\lbrack I ate\rbrack \ Maple Pecan Croissant', which is the post mentioned in the objective. In order to find the top comment, I will navigate into the comments section of the post. In summary, the next action I will perform is \texttt{\'}\texttt{\'}\texttt{\'}click \lbrack1499\rbrack\texttt{\'}\texttt{\'}\texttt{\'}
{\color{red}{example\_user:}} \\
IMAGES: (1) current page screenshot\\
OBSERVATION:\\
\lbrack42\rbrack\ [link]\ \lbrack 'My account'\rbrack\\
\lbrack43\rbrack\ [link]\ \lbrack 'Logout'\rbrack\\
\lbrack44\rbrack\ [link]\ \lbrack 'Publish Ad'\rbrack\\
\lbrack25\rbrack\ [heading]\ \lbrack 'What are you looking for today?'\rbrack\\
\lbrack143\rbrack\ [StaticText]\ \lbrack 'Keyword'\rbrack\\
\lbrack81\rbrack\ [textbox]\ \lbrack 'e.g., a blue used car' required: False\rbrack\\
\lbrack146\rbrack\ [StaticText]\ \lbrack 'Category'\rbrack\\
\lbrack28\rbrack\ [heading]\ \lbrack 'Latest Listings'\rbrack\\
URL: \texttt{http://classifieds.com}\\
OBJECTIVE: Help me find the cheapest dark colored guitar.\\
PREVIOUS ACTION: None\\
{\color{red}{example\_assistant:}} Let's think step-by-step. The objective is to find the cheapest dark colored guitar on the site. The site has a search box whose ID is \lbrack5\rbrack. I can search for guitars by entering ``guitar''. I can submit this by pressing the Enter afterwards. In summary, the next action I will perform is \texttt{\'}\texttt{\'}\texttt{\'}type \lbrack81\rbrack\ \lbrack guitar\rbrack\  \lbrack1\rbrack\texttt{\'}\texttt{\'}\texttt{\'}
}
}
\caption{In-context examples and prompt used for the caption-augmented Llama-3 agent. The webpage accessibility tree is added to the end of each round of the {\color{red}{example\_user}} dialogue.}  \label{fig:llama_prompt}
\end{figure*}
\begin{figure*}[th]
\noindent\fbox{\parbox{\textwidth}{%
\small
{\color{red}{example\_user:}} \\
OBSERVATION:\\
\lbrack1744\rbrack\ [link]\ \lbrack 'HP CB782A\#ABA 640 Inkjet Fax Machine (Renewed)'\rbrack\\
\lbrack1749\rbrack\ [StaticText]\ \lbrack \$279.49\rbrack\\
\lbrack1757\rbrack\ [button]\ \lbrack 'Add to Cart'\rbrack\\
\lbrack1760\rbrack\ [button]\ \lbrack 'Add to Wish List'\rbrack\\
\lbrack1761\rbrack\ [button]\ \lbrack 'Add to Compare'\rbrack\\
URL: http://onestopmarket.com/office-products/office-electronics.html \\
OBJECTIVE: What is the price of HP Inkjet Fax Machine?\\
PREVIOUS ACTION: None\\
{\color{red}{example\_assistant:}} Let's think step-by-step. This page list the information of HP Inkjet Fax Machine, which is the product identified in the objective. Its price is \$279.49. I think I have achieved the objective. I will issue the stop action with the answer. In summary, the next action I will perform is \texttt{\'}\texttt{\'}\texttt{\'}stop [\$279.49]\texttt{\'}\texttt{\'}\texttt{\'}
{\color{red}{example\_user:}} \\
IMAGES: (1) current page screenshot\\
OBSERVATION:\\
\lbrack164\rbrack\ [textbox]\ \lbrack 'Search' focused: True required: False\rbrack\\
\lbrack171\rbrack\ [button]\ \lbrack 'Go'\rbrack\\
\lbrack174\rbrack\ [link]\ \lbrack 'Find directions between two points'\rbrack\\
\lbrack212\rbrack\ [heading]\ \lbrack 'Search Results'\rbrack\\
\lbrack216\rbrack\ [button]\ \lbrack 'Close'\rbrack\\
URL: \texttt{http://openstreetmap.org}\\
OBJECTIVE: Show me the restaurants near CMU\\
PREVIOUS ACTION: None\\
{\color{red}{example\_assistant:}} 
Let's think step-by-step. This page has a search box whose ID is \lbrack164\rbrack. According to the nominatim rule of openstreetmap, I can search for the restaurants near a location by ``restaurants near''. I can submit my typing by pressing the Enter afterwards. In summary, the next action I will perform is \texttt{\'}\texttt{\'}\texttt{\'} type \lbrack164\rbrack \lbrack restaurants near CMU\rbrack \lbrack1\rbrack \texttt{\'}\texttt{\'}\texttt{\'}
}
}
\caption{In-context examples and prompt used for the text-only GPT-4o agent on WebArena. The webpage accessibility tree is added to the end of each round of the {\color{red}{example\_user}} dialogue.}  \label{fig:wa_prompt}
\end{figure*}

For all experiments, we use a webpage viewport width of 1280, a viewport height of 2048, and truncate text observations to 3840 tokens. We sample from models using nucleus sampling with a temperature of 1.0 and a temperature of 1.0 and a top-p of 0.95. The system message used in all our experiments is provided in Fig.~\ref{fig:som_system_message}. This instructs the agent with the guidelines for the web navigation task, and list out all the possible actions that it can perform.

For the GPT-4o agent on VWA, we use the same prompt with SoM prompting from \cite{koh2024visualwebarena}, reproduced in Fig.~\ref{fig:som_prompt}. The model is provided with 3 in-context examples. A similar prompt (without the image screenshots) is used for the caption-augmented Llama-3-70B-Instruct agent which takes the caption-augmented accessibility tree as input (shown in Fig.~\ref{fig:llama_prompt}). 
On WA, the agents take the accessibility tree as input, and we use the same prompt from \cite{zhou2023webarena} that includes 2 in-context examples (reproduced in Fig.~\ref{fig:wa_prompt}).

\subsubsection{Value Function} \label{appendix:vf_prompt}

\begin{figure*}[th]
\noindent\fbox{\parbox{\textwidth}{%
\small

{\color{red}{system\_message:}}\\
You are an expert in evaluating the performance of a web navigation agent. The agent is designed to help a human user navigate a website to complete a task. Given the user's intent, the agent's action history, the final state of the webpage, and the agent's response to the user, your goal is to decide whether the agent's execution is successful or not. If the current state is a failure but it looks like the agent is on the right track towards success, you should also output as such.\\

There are three types of tasks:\\
1. Information seeking: The user wants to obtain certain information from the webpage, such as the information of a product, reviews, the text in a comment or post, the date of a submission, etc. This may be formulated in the intent as ``tell me'', ``what is'', or ``list out''. The agent's response must contain the information the user wants, or explicitly state that the information is not available. Otherwise, e.g. the agent encounters an exception and respond with the error content, the task is considered to be a failure. It is VERY IMPORTANT that the bot response is the stop action with the correct output. If the bot response is not stop (e.g., it is click, type, or goto), it is considered a failure for information seeking tasks.\\
2. Site navigation: The user wants to navigate to a specific page (which may also be specified in the intent as ``find'', ``show me'', ``navigate to''). Carefully examine the agent's action history and the final state of the webpage (shown in the LAST IMAGE) to determine whether the agent successfully completes the task. It is VERY IMPORTANT that the agent actually navigates to the specified page (reflected by the final state of the webpage, in the LAST IMAGE) and NOT just output the name of the item or post. Make sure that the final url is compatible with the task. For example, if you are tasked to navigate to a comment or an item, the final page and url should be that of the specific comment/item and not the overall post or search page. If asked to navigate to a page with a similar image, make sure that an image on the page is semantically SIMILAR to the intent image. If asked to look for a particular post or item, make sure that the image on the page is EXACTLY the intent image. For this type of task to be considered successful, the LAST IMAGE and current URL should reflect the correct content. No need to consider the agent's response.\\
3. Content modification: The user wants to modify the content of a webpage or configuration. Ensure that the agent actually commits to the modification. For example, if the agent writes a review or a comment but does not click post, the task is considered to be a failure. Carefully examine the agent's action history and the final state of the webpage to determine whether the agent successfully completes the task. No need to consider the agent's response.\\

*IMPORTANT*\\
Format your response into two lines as shown below:\\

Thoughts: \textless your thoughts and reasoning process\textgreater\\
Status: ``success'' or ``failure''\\
On the right track to success: ``yes'' or ``no''\\

{\color{red}{user:}}\\
{\color{blue}{\textless intent screenshots\textgreater}}\\
User Intent: {intent}\\
{\color{blue}{\textless obs\_screenshot\_1\textgreater}}~...~{\color{blue}{\textless obs\_screenshot\_d\textgreater}}\\
Action History: {last\_actions\_str}\\
Bot response to the user: {last\_response}\\
Current URL: {current\_url}\\
The images corresponding to the user intent are shown in the FIRST \{len(intent\_images)\} images (before the User Intent).\\
The last \{len(screenshots)\} snapshots of the agent's trajectory are shown in the LAST \{len(screenshots)\} images. The LAST IMAGE represents the current state of the webpage.\\
}
}
\caption{System message and prompt used for the value function. Blue text indicates items that will be replaced by image content during the call to the value function.}  \label{fig:vf_prompt}
\end{figure*}
As described in Sec.~\ref{sec:method_vf}, we implement the value function $f_v$ by prompting a multimodal language model with all current and previously seen observations $\{ o_1, \ldots, o_p\}$. We use a prompt similar to the one from \cite{pan2024autonomous}, but make several modifications:
\begin{itemize}
    \item Instead of just the current screenshot, we include the last-$d$ screenshots of the evaluated trajectory, to enable the value function to more accurately compute success or failure for tasks that involve multi-step reasoning (e.g., whether the final observation corresponds to the second item in the second row of the second last observation).
    \item We modify the instructions to include more detailed instructions about what constitutes a failure or a success crtieria. This is necessary as our search occurs over a denser graph (compared to generating and re-ranking trajectories), and requires a more accurate value function. We refer readers to \cite{chen2024tree} for more discussion.
    \item Rather than a binary output, we instruct the model to produce whether the given observations have succeeded at the task or failed. If it fails, we further prompt the model to output if it is possibly on the right track to success. This allows us to collect scores in $`\{0, 0.5, 1\}$, enabling more finegrained value outputs (in addition to the averaging of multiple reasoning paths described in Sec.~\ref{sec:implementation}).
\end{itemize} 

The full system message and prompt for the value function is provided in Tab.~\ref{fig:vf_prompt}. We also note that our value function is heavily visual, which may be one explanation for why our method is more effective on the multimodal VWA benchmark than on WA (Sec.~\ref{sec:experiments}). Including more finegrained textual information about the trajectory on top of the screenshots, such as the accessibility tree representations of each page, may further improve its performance (at greater compute and API cost).

\subsection{Search Algorithm} \label{appendix:search_algorithm}

\begin{algorithm*}[t]
\caption{Our proposed search algorithm at step $t$}  \label{alg:search}
\let\AND\relax
\begin{algorithmic}[1]
\REQUIRE depth $d$, branching factor $b$, search budget $c$, start state $s_t$
\STATE Initialize frontier $\mathcal{F} \leftarrow \{\}$ as a max priority queue
\STATE Initialize best state $\hat{s}_t \leftarrow s_t$
\STATE Initialize the best score $\hat{v}_t \leftarrow -\infty$
\STATE Initialize the search counter $s \leftarrow 0$
\WHILE{$s < c$}
    \STATE $s_p, v_{\text{prev}} \leftarrow \text{pop}(\mathcal{F})$
    \STATE Backtrack and execute new actions to get to state $s_p$
    \STATE Compute the score $v_p = f_v(I, \{ o_1, \ldots, o_p\})$ from current and previous observations
    \STATE $s \leftarrow s + 1$
    \IF{$v_p \geq \hat{v}_t$}
        \STATE $\hat{v}_t \leftarrow v_p$
        \STATE $\hat{s}_t \leftarrow s_p$
    \ENDIF
    \IF{$v_p \geq \theta$}
        \STATE \textbf{break} \COMMENT{Found a likely successful state}
    \ENDIF
    \IF{$s \geq c$}
        \STATE \textbf{break} \COMMENT{Search budget exceeded}
    \ENDIF
    \IF{$| s_0, ..., s_{p} | < d$}
    \STATE Sample $b$ candidates for the next action from the LM: $\{a_p^1, ..., a_p^b\} \sim f_{\theta}(o_p)$
    \FOR{$i \leftarrow 1 \text{ to } b$}
        \STATE Execute $a_p^i$ to get to state $s_p^i$
        \STATE Add new candidate state and the current value to the frontier: $\mathcal{F} \leftarrow \mathcal{F} \cup (s_p^i, v_p)$
    \ENDFOR
    \ENDIF
\ENDWHILE
\STATE Reset $\mathcal{F} \leftarrow \{\}$ and $s \leftarrow 0$
\STATE Go to the best state $\hat{s}_t$
\STATE Set $t \leftarrow t + \text{(\#actions to get from $s_t$ to $\hat{s}_t$)}$
\end{algorithmic}
\end{algorithm*}

Our search procedure described in Sec.~\ref{sec:method_search} is summarized in Algorithm.~\ref{alg:search}.

\subsubsection{Environment Reset} \label{appendix:reset}

In this section, we describe the implementation details of the backtracking used in our search procedure:

\begin{enumerate}
    \item We maintain a max priority queue that contains sequences of actions and their score $v$ (from the value function). Each element is a sequence of actions that the agent has to sequentially execute starting from the initial state (task dependent, but often the website homepage) to get to state $s$ that has the corresponding score $v$.
    \item After we execute a new action (L23 of Algorithm.~\ref{alg:search}), we append this action to the sequence of actions and add the new sequence to the priority queue with its corresponding score $v$.
    \item In order to reset the environment to get a clean slate for the next node to explore, we reset to the initial state again, and repeat the execution of the next sequence of actions starting from step 1.
\end{enumerate}

We implemented backtracking in this fashion, as we found that this was a substantially more complete way of resetting the state, as opposed to simply clicking the ``back'' button on the browser for example, as this does not persist certain web states such as the scroll offset, or retain text in text inputs. While our implementation does improve fidelity of backtracking and resets, it however does add significant overhead in terms of time (see Sec.~\ref{sec:limitations} for more discussion).

%The exact code implementation details can be found within the \texttt{removed\_for\_review} file of our publicly available code at \texttt{removed\_for\_review}.
