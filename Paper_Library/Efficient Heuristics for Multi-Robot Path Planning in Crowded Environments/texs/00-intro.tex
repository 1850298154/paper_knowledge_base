We study the \emph{labeled} Multi-Robot Motion Planning (\mpp) problem under a graph-theoretic setting, also known as Multi-Agent Path Finding (MAPF). 
%
The basic objective of \mpp is to find a set of collision-free paths to route multiple robots from a start configuration to a goal configuration.  
% 
In practice, solution optimality is also of key importance; yet optimally solving \mpp in terms of makespan and sum-of-cost is generally NP-hard~\cite{YuLav13AAAI,Sur10,Yu2015IntractabilityPlanar}. 
%
\mpp algorithms find many important large-scale applications, including, e.g., in warehouse automation for general order fulfillment \cite{wurman2008coordinating}, grocery order fulfillment \cite{mason2019developing}, and parcel sorting \cite{wan2018lifelong}.
% 
Other application scenarios include formation reconfiguration~\cite{PodSuk04}, agriculture~\cite{cheein2013agricultural}, object 
transportation~\cite{RusDonJen95}, swarm robotics \cite{preiss2017crazyswarm}.

%
Given the potential of employing its solutions in a wide range of impactful applications, even though \mpp had been studied since the 1980s in the robotics domain~\cite{KorMilSpi84,ErdLoz86,LavHut98b,GuoPar02}, it remains a highly active research topic. 
%
Many effective algorithms, for example~\cite{YuLav16TRO, boyarski2015icbs, cohen2016improved}, have been proposed recently that balance fairly well between computational efficiency and solution optimality.
% 
Existing \mpp algorithms have been tested on randomly generated instances and yield decent performance for instances with relatively limited robot-robot interactions, i.e., either the number of robots is limited, or the density of robots is relatively low.
%
However, they frequently fail in instances that are both large and dense.
% 
\begin{figure}[t]
\vspace{2mm}
    \centering
  \begin{overpic}               
        [width=1\linewidth]{./figures/dense_ins.pdf}
             \small
             \put(12.5, 36.5) {(a)}
             \put(47.5,36.5) {(b)}
             \put(82.5, 36.5) {(c)}
             \put(20.5, -3) {(d)}
             \put(75.5, -3) {(e)}
        \end{overpic}
\vspace{-3mm}
    \caption{(a)-(c) A challenging \emph{locally-dense} \mpp example on $20\times 20$ map with 49 robots. It requires rearranging the robots from the start configuration (a) to the goal configuration (c). By ``sparsifying'' the configuration using our methods, as shown in an intermediate step (b), the problem can be solved quickly with decent solution optimality. (d)-(e) A challenging \emph{globally-dense} \mpp example on a $24\times 18$ warehouse map with 203 robots. In both settings, each robot has a unique start and goal.}
    \label{fig:dense_example}
 %   \vspace{2mm}
\end{figure}% 
Recently, \mpp algorithms have been applied in high-density applications, such as autonomous vehicle parking systems \cite{Guo2023TowardEP,okoso2022high}, to increase space utilization efficiency.
% 
In such dense scenarios, robots' motions are strongly correlated and may block the paths of each other, which makes the problem extremely difficult for existing \mpp solvers.
% 

\textbf{Results and contributions.}
% 
This research proposes efficient heuristics and uses them to build complete solvers for tackling dense and difficult \mpp instances. 
% 
We address two classes of dense \mpp: \emph{globally} dense instances where the number of robots is large with high average robot density (more than $40\%$, see Fig.~\ref{fig:dense_example}(c)), and \emph{locally} dense instances where the robot distribution is unbalanced with high local robot density (i.e. $100\%$, see Fig.~\ref{fig:dense_example}(a)-(b)).
% 

We develop two hybrid \mpp algorithms to address the above-mentioned challenges.
% 
% In the first class, the total number of robots distributed on a map is large. 
% The resulting average robot density is very high but they can be uniformly randomly distributed .
% 
% 
% In the second class, the robot distribution is extremely unbalanced and the local robot density in some places can be very high .
% 
% We call these instances \emph{globally dense instances} and \emph{locally dense instances} respectively.
% 
% We propose two hybrid \mpp algorithms for these dense scenarios. 
% 
In the first algorithm, we introduce a (motion-primitive) database-based conflict resolution mechanism inspired by \cite{han2019ddm} to augment a conflict-based search \cite{sharon2015conflict}.
%
%
%\decbs greedily resolves all conflicts of an existing node and returns the solution.
% 
We also design a set of rules to maintain the solution quality as well as the completeness of the resulting algorithm. 
%
We call the algorithm \decbs, standing for \emph{database-accelerated enhanced conflict-based search}.  

While our first algorithm works for both globally dense and locally dense scenarios, our second algorithm is designed specifically for locally dense instances.
% 
Inspired by \cite{guo2022sub}, we first convert the challenging configuration to a \emph{sparsified} configuration, which is relatively easier to solve, using \emph{unlabeled} \mpp planning solutions. 
% 
To reduce the extra overhead of the conversion, we adopt a best-first heuristic for finding a proper sparsified configuration and a path refinement technique for better concatenating the intermediate paths.
% 
We call the second algorithm \secbs, standing for \emph{sparsified enhanced conflict-based search}.  

Experiments on diverse environment maps demonstrate the effectiveness of our proposed methods in solving instances with robot densities greater than $60\%$-$70\%$ with a high success rate and decent levels of solution quality. 
% 
\decbs and \secbs outperform previous \mpp algorithms in terms of combined speed and solution quality.
% 

\textbf{Organization.}
The rest of the paper is organized as follows. Sec.~\ref{sec:problem} covers the preliminaries, including the problem formulation and two suboptimal algorithms \ecbs and \ddm. 
In Sec.~\ref{sec:algo-trans}-Sec.~\ref{sec:algo-ddecbs}, we describe our heuristics and algorithms for solving dense \mpp. 
% 
We perform thorough evaluations and discussions of the proposed algorithms in Sec.~\ref{sec:evaluation} and conclude with Sec.~\ref{sec:conclusion}.




% 
