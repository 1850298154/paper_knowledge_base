In this paper, we present two novel heuristics-based algorithms for multi-robot path planning (\mpp) in dense and congested environments, with the goal to provide to quickly provide high-quality solutions for these problems. 
% 
The first method, \decbs, incorporates a database-driven conflict resolution mechanism to resolve node conflicts in dense setups. Optimality protection rules are also instilled to maintain reasonable solution quality. 
% 
Whereas \decbs addresses \emph{globally} dense scenarios, the second method, \secbs, tackles \emph{locally} dense settings by converting ultra-dense configurations into sparser ones through a greedy start-goal assignment and then solving an unlabeled \mpp. The sparsification step, while incurring some overhead, makes the overall problem significantly easier. 
% 
Through extensive experiments, we show that our proposed methods achieve excellent performance in balancing success rate, running time, and solution quality.
%

Currently, \decbs only uses a fairly basic solution database, which is limiting the speed and flexibility of \decbs.  
In future work, we plan to significantly expand the solution database while keeping it sufficiently small for fast look-ups. Portions of the database may also be augmented using machine learning. We expect this to provide a sizable performance boost for \decbs. 
 
% 
There are also many open questions that should be investigated further.
%
For example, as of now, the way we trigger the conflict resolution mechanism is somewhat rigid. Can we devise a better approach, e.g., using a data-driven method, to figure out the optimal time to trigger conflict resolution? 
% 
As another example, there is still a lack of understanding of the exact relationship between time complexity and robot density and distribution. Can we establish a deeper, or better yet, quantitative, relationship between the two? 


% In this work, we present two heuristics aiming at tackling \mpp in globally and locally dense scenarios. 
% %
% \decbs introduces the database conflict resolution mechanism to resolve all the conflicts of a high-level node when the \noc of that node is in the  stagnation state or is small enough.
% % 
% With the additional optimality protection mechanism, \decbs is able to be bounded-suboptimal and complete.
% %
% For locally dense scenarios, \secbs reduces the hard and congested configurations into sparsified  configurations  using unlabeled \mpp  which are easier to solve and only introduce a small overhead.
% % 
% The methods we proposed have a much higher success rate and lower running time than \ecbs, and better solution quality than rule-based \ddm.
% % 

% For future work, we intend to optimize database heuristics for better optimality.
% % 
% There are still some opening questions that remain to solve. 
% % 
% For example, is there a theory-guided method to decide the best time to trigger the database conflict resolution?
% % 
% Currently, it is empirically determined and may vary among different maps and densities.
% % 
% In addition, while dense and congested instances are more difficult to solve in an empirical sense, it remains unclear how the time complexity is exactly related to the robot density and robot distribution.
% 

%Last but not least, it's worth noting that our assignment and planning algorithms are not restricted to sorting warehouse but can also be applied in other types of grid-like settings, i,e, fulfillment warehouses. 

% For future work, we intend to further improve \csmp's scalability and flexibility, possibly leveraging the latest advances in  \mpp/MAPF research. 
% 
%Also, to further improve the efficiency, we would like to combine \csmp and other \mpp algorithms to get a dual-mode system. 
% 
%Specifically,  we may use optimal and non-scalable \mpp algorithms in low-density mode and use CSMP in high-density mode.
% We also plan to extend the automated garage design from 2D to 3D, supporting multiple levels of parking. Finally, we would like to build a small-scale test
% beds realizing the physical and algorithmic designs.

