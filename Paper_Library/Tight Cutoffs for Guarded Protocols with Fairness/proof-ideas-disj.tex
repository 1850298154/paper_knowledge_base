\section{Proof Techniques for Disjunctive Systems}
\label{sec:ideas-disj}
\ak{to simplify the notation, can we remove `$(x)$' in $\visInf{\mB}{x}$?} \sj{I think it is much clearer if we leave it - the notation is self-explanatory currently}

\subsection{\LTLmX\ Properties without Fairness: Existing Constructions}
%\paragraph*{\LTLmX\ Properties without Fairness}
\label{sec:ideas-disj-nofair}

We revisit the main technique of the original proof 
of Theorem~\ref{thm:disj-cutoff-pairs}~\cite{Emerson00}. 
It constructs an infinite run $y$ of $\cutoffsys$ 
with $y \models h(A,B^{(1)})$, 
based on an infinite run $x$ of $\largesys$ with $n>c$ and $x \models h(A,B^{(1)})$. 
The idea is to copy local runs $x(A)$ and $x(B_1)$ into $y$, 
and construct runs of other processes in a way 
that enables all transitions along $x(A)$ and $x(B_1)$. 
The latter is achieved with the flooding construction.

\myparagraph{Flooding Construction \cite{Emerson00}}
Given a run $x = (s_1,e_1,p_1), (s_2,e_2,p_2) \ldots$ of $\largesys$, let
$\visited_\mB(x)$ be the set of all local states visited by $B$-processes in $x$,
i.e., $\visited_\mB(x) = \{ q \in Q_B \| \exists m \exists i.\ s_m(B_i) = q \}$. 

For every $q \in \visited_\mB(x)$ there is a local run of \largesys, say $x(B_i)$,
that visits $q$ first, say at moment $m_q$. Then, saying that process 
$B_{i_q}$ of \cutoffsys \emph{floods $q$} means:
$$y(B_{i_q}) = x(B_i)\slice{1}{m_q}(q)^\omega.$$ 
In words: the run $y(B_{i_q})$ is the same as $x(B_i)$ until moment $\time_q$,
and after that the process never moves.

The construction achieves the following. 
If we copy local runs of $A$ and $B_1$ from $x$ to $y$, 
and in $y$ for every $q \in \visited_\mB(x)$ introduce one process that floods $q$, 
then: 
if in $x$ at some moment $\time$ there is a process in state $q'$, 
then in $y$ at moment $\time$ there will also be a process (different from 
$A$ and $B_1$) in state $q'$. Thus, every transition of $A$
 and $B_1$, which is enabled at moment $\time$ in $x$, will also be enabled in $y$. 

%\begin{proof}[Proof idea of the bounding lemma]
\myparagraph{Proof idea of the bounding lemma}
The lemma for disjunctive systems without fairness can be proved by
copying local runs $x(A)$ and $x(B_1)$, and flooding all states in
$\visited_\mB(x)$. To ensure that at least one process moves infinitely often
in $y$, we copy one additional (infinite) local run from $x$. Finally, it may
happen that the resulting collection of local runs violates the interleaving 
semantics requirement. To resolve this, we add stuttering steps into local 
runs whenever two or more processes move at the same time, and we 
remove global stuttering steps in $y$. Since the only difference between 
$x(A,B_1)$ and $y(A,B_1)$ are stuttering steps, $y$ and $x$ satisfy the same $
\LTLmX$-properties $h(A,B^{(1)})$. 
Since $\card{\visited_\mB(x)} \leq 
\card{B}$, we need at most $1+\card{B}+1$ copies of $B$ in \cutoffsys.
%\end{proof}



\subsection{\LTLmX\ Properties with Fairness: New Constructions}
%\paragraph*{\LTLmX\ Properties with Fairness}
\label{sec:ideas-disj-fair}

The flooding construction does not preserve fairness, 
 and also cannot be used to construct deadlocked runs since it does not 
preserve  disabledness of transitions of processes $A$ or $B_1$. 
For these cases, we provide new proof constructions.

Consider the proof task of the bounding lemma for disjunctive systems with 
fairness: given an unconditionally fair run $x$ of 
\largesys with 
$x \models h(A,B^{(1)})$, we want to construct an unconditionally fair run $y$ 
of \cutoffsys with $y \models h(A,B^{(1)})$. In contrast to unfair systems, we 
need to ensure that all processes move infinitely often in $y$. 
The insight is 
that after a finite time all processes will start looping 
around some set $\visited^\inf$ of states. We construct a run $y$ that
mimics this. To this end, we introduce two constructions. \emph{Flooding with
evacuation} is similar to flooding, but instead of keeping
processes in their flooding states forever it evacuates the processes into 
$\visited^\inf$. \emph{Fair extension} lets all processes move infinitely 
often without leaving $\visited^\inf$.

\myparagraph{Flooding with Evacuation}
Given a subset $\mF \subseteq \mB$
%\footnote{In this section we will use only $\mP_1 = \{B_1\}$, 
%          but for the case of deadlocks we will need a set.  
%          $\mP_1$ is a set of processes whose local runs we will copy 
%          from $x$ to $y$ later in the proof.} 
and an infinite run $x=(s_1,e_1,p_1)\ldots$ of \largesys, 
define
\begin{align}
& \visInf{\mF}{x} = \{ q \|\! \exists \text{ infinitely many } ~~~~ m\!:  
s_m(B_i) = q 
\text{ for some } B_i \in \mF \} \label{disj:def_vinf_wrt} \\
& \visFin{\mF}{x} = \{ q \|\! \exists \text{ only finitely many } m\!:  
s_m(B_i) = q
\text{ for some } B_i\in \mF \} \label{disj:def_vfin_wrt}
\end{align}
%
Let $q \in \visFin{\mF}{x}$.
In run $x$ there is a moment $f_q$ when $q$
is reached for the first time by some process from $\mF$, denoted $B_{\first_q}$. 
Also, in run $x$ there is a moment $l_q$ such that:
$s_{l_q}(B_{\last_q})=q$ for some process $B_{\last_q} \in \mF$, 
and $s_t(B_i)\neq q$ for all $B_i \in \mF$,
$t > l_q$ 
--- i.e., when some process from $\mF$ is in state $q$ for the last time in $x$. 
Then, saying that process $B_{i_q}$ of \cutoffsys 
\emph{floods $q \in \visFin{\mF}{x}$ and then evacuates into $\visInf{\mF}{x}$} 
means: 
$$
y(B_{i_q}) = x(B_{\first_q})\slice{1}{f_q} \ \cdot\ (q)^{(l_q - f_q + 1)} \cdot \ 
x(B_{\last_q})\slice{l_q}{m} \ \cdot \ (q')^\omega,
$$
where $q'$ is the state in $\visInf{\mF}{x}$ that $x(B_{\last_q})$ reaches first, 
at some moment $\time \geq l_q$.
In words, process $B_{i_q}$ mimics process $B_{\first_q}$ until it reaches $q$, 
then does nothing until process $B_{\last_q}$ starts leaving $q$, 
then it mimics $B_{\last_q}$ until it reaches $\visInf{\mF}{x}$.

The construction ensures: 
if we copy local runs of all processes not in $\mF$ from $x$ to $y$, 
then all transitions of $y$ are enabled. 
This is because: 
for any process $p$ of $\cutoffsys$ that takes a transition in $y$ at any moment, 
the set of states visible to process $p$ is a superset of the set of states 
visible to the original process in \largesys whose transitions process $p$ copies.


\myparagraph{Fair Extension} 
\ak{explain intuition about those three sets}
\ak{adapt to dead}
Here, we consider a path $x$ that is the postfix of an unconditionally fair run $x'$ of $\largesys$, 
starting from the moment where no local states from $\visFin{\mB}{x'}$ are visited anymore. 
We construct a corresponding unconditionally-fair path $y$ of $\cutoffsys$, 
where no local states from $\visFin{\mB}{x'}$ are visited.

Formally, let $n \geq 2|B|$, and $x$ an unconditionally-fair path of $\largesys$ such that
$\visFin{\mB}{x}=\emptyset$.
Let $c \geq 2|B|$, and $s_1'$ a state of \cutoffsys
with
\li
\- $s_1'(A_1)=s_1(A_1)$, $s_1'(B_1)=s_1(B_1)$

\- for every $q \in \visInf{B_2..B_n}{x} \smi \visInf{B_1}{x}$,
   there are two processes $B_{i_q}, B_{i_q'}$ of \cutoffsys
   that start in $q$, i.e., $s_1'(B_{i_q})=s_1'(B_{i_q'})=q$

\- for every $q \in \visInf{B_2..B_n}{x} \cap \visInf{B_1}{x}$,
   there is one process $B_{i_q}$ of \cutoffsys
   that starts in $q$

\- for some $\qstar \in \visInf{B_2..B_n}{x} \cap \visInf{B_1}{x}$,
   there is one additional process of \cutoffsys, 
   different from any in the above, 
   called $B_{i_\qstar'}$,
   that starts in $\qstar$.

\- any other process $B_i$ of \cutoffsys 
   starts in some state of $\visInf{B_2..B_n}{x}$.
\il
Note that if $\visInf{B_2..B_n}{x}\cap \visInf{B_1}{x} = \emptyset$, 
then the third and fourth pre-requisites are trivially satisfied.

The fair extension extends state $s_1'$ of \cutoffsys 
to an unconditionally-fair path $y=(s'_1,e'_1,p'_1)\ldots$ 
with $y(A_1,B_1) = x(A_1,B_1)$ as follows:
\li
\-[(a)] $y(A_1)=x(A_1)$, $y(B_1)=x(B_1)$

\-[(b)] for every $q \in \visInf{B_2..B_n}{x} \smi \visInf{B_1}{x}$: 
       in run $x$ there is $B_i \in \{B_2..B_n\}$ 
       that starts in $q$ and visits it infinitely often. 
       Let $B_{i_q}$ and $B_{i'_q}$ of \cutoffsys mimic $B_i$ in turns: 
       first $B_{i_q}$ mimics $B_i$ until it reaches $q$, 
       then $B_{i'_q}$ mimics $B_i$ until it reaches $q$, and so on.

\-[(c)] arrange states of $\visInf{B_2..B_n}{x}\cap \visInf{B_1}{x}$ 
       in some order $(\qstar, q_1, \ldots, q_l)$.  
       The processes $B_{i_\qstar'}, B_{i_\qstar}, B_{i_{q_1}}, \ldots, B_{i_{q_l}}$ 
       behave as follows.
       Start with $B_{i_\qstar'}$: 
       when $B_1$ enters $\qstar$ in $y$, it carries%
       \footnote{``Process $B_1$ starting at moment $m$ carries process $B_i$ 
                 from $q$ to $q'$'' means: process $B_i$ mimics 
                 the transitions of $B_1$ starting at moment $m$ at $q$ 
                 until $B_1$ first reaches $q'$.}
       $B_{i_\qstar'}$             from $\qstar$ to $q_1$, 
       then carries $B_{i_{q_1}}$ from $q_1$ to $q_2$, \ldots, 
       then carries $B_{i_{q_l}}$ from $q_l$ to $\qstar$, 
       then carries $B_{i_\qstar}$ from $\qstar$ to $q_1$, 
       then carries $B_{i_\qstar'}$ from $q_1$ to $q_2$, 
       then carries $B_{i_{q_1}}$ from $q_2$ to $q_3$,
       and so on.

%\-[c2.] otherwise, $\visited_{\inf\cap B_1}{x} = \{ q^\star \}$. Then $B_1$ only ever makes transitions $q^\star \to q^\star$, thus let process $B_{i_{q^\star}}$ mimic this.

\-[(d)] any other $B_i$ of \cutoffsys,
       starting in $q \in \visInf{B_2..B_n}{x}$,
       mimics $B_{i_q}$.
\il
Note that parts (b) and (c) of the constrution ensure that there is always at
       least one process in every state from $\visInf{B_2..B_n}{x}$. This
       ensures that the guards of all transitions of the construction are satisfied.
Excluding processes in (d), the fair extension uses up to $2|B|$ copies of $B$.%
\footnote{A careful reader may notice that if
          $|\visInf{B_1}{x}|=1$ and $|\visInf{B_2..B_n}{x}|=|B|$,
          then the construction uses $2|B|+1$ copies of $B$.
          But one can slightly modify the construction for this special case,
          and remove process $B_{i_\qstar'}$ from the pre-requisites.}
%\li
%\- if $\visInf{B_2..B_n}{x} \cap \visInf{B_1}{x} = \emptyset$, then 
%     $$\leq 1+2|\visInf{B_2..B_n}{x}| \leq 1+2(|B|-1) = 2|B|-1$$
%   (note that $\visInf{B_1}{x}$ contains at least one state)
%
%\- otherwise:\\
%   let $smi = \visInf{B_2..B_n}{x} \smi \visInf{B_1}{x}$, \\
%   let $inter = \visInf{B_2..B_n}{x} \cap \visInf{B_1}{x}$, \\
%   then
%   $$\leq 1+2|smi| + |inter| + 1 \leq 1+2(|B|-1) + 1 + 1 = 2|B|+1$$
%   $$\leq 1+2|smi| + |inter| + 1 \leq 1+2(|B|-2) + 2 + 1 = 2|B|+1$$
%\il
%
%
%%%%%%%%%%%%% OLD FAIR EXTENSION %%%%%%%%%%%%%%%%%%%%%%
%Let $x=(s_1,e_1,p_1)\ldots$ be an unconditionally-fair path $x$ of $\largesys$, let $\visited_\fin{x}$ and $\visited_\inf{x}$ be defined wrt. no processes, let $n\geq 2|\visited_\inf{x}|$, and let $x$ satisfy: 
%\li
%  \- every $B_i\neq B_1$ visits $s_1(B_i)$ infinitely often
%  \- $\visited_\fin{x} = \emptyset$,
%\il
%and let $s_1'$ be a state of \cutoffsys with $c \geq 2|\visited_\inf{x}|$ that satisfies: 
%\li
%  \- $s_1'(A)=s_1(A)$, $s_1'(B_1)=s_1(B_1)$
%  \- for every $q \in \visited_\inf{x}$ there are two processes of \cutoffsys called $B_{i_q}$, $B_{i'_q}$ with $s_1'(B_{i_q}) = s_1'(B_{i'_q}) = q$
%  \- for all other processes $B_i$ of \cutoffsys (if any): $s_1'(B_i) \in \visited_\inf{x}$.
%\il
%The fair extension extends state $s_1'$ of \cutoffsys to an unconditionally-fair path $y=(s'_1,e'_1,p'_1)\ldots$ with $x(A,B_1) = y(A,B_1)$ as follows. Let $\visited_{\inf\cap B_1}{x}$ be the set of states visited infinitely often by process $B_1$, and $q^\star=s_1(B_1)$:
%\li
%\-[a.] $y(A)=x(A)$, $y(B_1)=x(B_1)$
%\-[b.] for every $q \in \visited_\inf{x} \smi \visited_{\inf\cap B_1}{x}$: in run $x$ there is $B_i$ that starts in $q$ and visits it infinitely often. Let $B_{i_q}$ and $B_{i'_q}$ of \cutoffsys mimic $B_i$ in turns: first $B_{i_q}$ mimics $B_i$ until it reaches $q$, then $B_{i'_q}$ mimics $B_i$ until it reaches $q$,\dots 
%\-[c1.] if $\visited_{\inf\cap B_1}\smi \{ q^\star \} \neq \emptyset$:
%\li
%  \- order arbitrarily $\visited_{\inf\cap B_1}\smi \{ q^\star \} = (q_1, q_2, \ldots, q_k)$
%  \- the processes $\{ B_{i_{q^\star}}, B_{i_{q_1}}, B_{i'_{q_1}}, B_{i_{q_2}}, B_{i'_{q_2}}, \ldots, B_{i_{q_k}}, B_{i'_{q_k}} \}$ behave as follows:
%  \- start with $B_{i_{q_1}}$: when $B_1$ enters $q_1$, it carries $B_{i_{q_1}}$ from $q_1$ to $q_2$, then carries $B_{i_{q_2}}$ from $q_2$ to $q_3$, \ldots, then carries $B_{i_{q_k}}$ from $q_k$ to $q^\star$, then carries $B_{i_{q^\star}}$ from $q^\star$ to $q_1$, then carries $B_{i'_{q_1}}$ from $q_1$ to $q_2$, and so on.
%\il
%\-[c2.] otherwise, $\visited_{\inf\cap B_1} \!=\! \{ q^\star \}$: $B_1$ only transits $q^\star \!\to\! q^\star$; let $B_{i_{q^\star}}$ mimic $B_1$
%
%\-[d.] let other processes $B_i$ of \cutoffsys with $s_1'(B_i)=q$ (if any) mimic $B_{i_q}$.
%\il
%%%%%%%%%%%%% END OF OLD FAIR EXTENSION %%%%%%%%%%%%%%%%%%%%%%

%\begin{proof}[Proof idea of the bounding lemma]

\myparagraph{Proof idea of the bounding lemma}
\sj{for weak or strong fairness, the same construction can be used; evacuation is not necessary, but also doesn't increase the cutoff if we use it; difficulty: show that cutoff is still tight
}
Let $c=2\card{B}$. 
Given an unconditionally-fair run $x$ of $\largesys$ 
we construct an unconditionally-fair run $y$ of the cutoff system $\cutoffsys$ 
such that $y(A,B_1)$ is stuttering equivalent to $x(A,B_1)$.

Note that in $x$ there is a moment $m$ such that all local states that are visited after $m$ are in $\visInf{\mB}{x}$.

The construction has two phases. In the first phase, we apply flooding for states in $\visInf{\mB}{x}$, and flooding with evacuation for states in $\visFin{\mB}{x}$:
\li
\-[(a)] $y(A)=x(A)$, $y(B_1)=x(B_1)$

\-[(b)] for every $q \in \visInf{B_2..B_n}{x} \smi \visInf{B_1}{x}$, 
       devote two processes of $\cutoffsys$ that flood $q$

\-[(c)] for some $\qstar \in \visInf{B_2..B_n}{x} \cap \visInf{B_1}{x}$,
       devote one process of \cutoffsys that floods $\qstar$

\-[(d)] for every $q \in \visFin{B_2..B_n}{x}$, 
       devote one process of $\cutoffsys$ that 
       floods $q$ and evacuates into $\visInf{B_2..B_n}{x}$

\-[(e)] let other processes (if any) mimic process $B_1$
\il
The phase ensures that at moment $m$ in $y$, 
there are no processes in $\visFin{\mB}{x}$, 
and all the pre-requisites of the fair extension are satisfied.

The second phase applies the fair extension, 
and then establishes the interleaving semantics 
as in the bounding lemma in the non-fair case.
The overall construction uses up to $2|B|$ copies of $B$.
% Indeed: 
% fin&smi=0, fin&cap=0,
% smi = InfB2..Bn - cap
% InfB2..Bn <= B - fin
% Then:
% 1+2smi+1cap+1+fin <= 1+2(InfB2..Bn-cap)+cap+1+fin = 
%                      2+2InfB2..Bn-cap-fin <=
%                      2+2B-cap-fin
% now split case:
% everywhere fin>0 (otherwise 2|B| follows from the analysis of the fair extension)
% note: if cap=0, then we actually have (recall fair pre)
%       1+2smi+fin =< 1+2(B-InfB1-fin)+fin = 1+2B-2InfB1-2fin =< 2B-3
% thus cap>0,fin>0
% then 2+2B-cap-fin =< 2B

%\end{proof}


\subsection{Detection of Local and Global Deadlocks: New Constructions}
\label{sec:ideas-disj-deadlock}
%\paragraph*{Detection of Local and Global Deadlocks}

\myparagraphraw{Monotonicity Lemmas.}
The lemma for deadlock detection, for fair and unfair cases,
is proven for $n \geq |B|+1$.
In the case of local deadlocks, 
process $B_{n+1}$ mimics a process that moves infinitely often in $x$.
In the case of global deadlocks, 
by pigeon hole principle, 
in the global deadlock state there is a state $q$ with at least two processes in it---let process $B_{n+1}$ mimic a process that deadlocks in $q$.

\myparagraphraw{Bounding Lemmas.}
For the case of global deadlocks, fairness does not affect the proof of the bounding lemma. 
The insight is to divide deadlocked local states into two disjoint sets, 
$\dead_1$ and $\dead_2$, as follows.
Given a globally deadlocked run $x$ of \largesys, 
for every $q \in \dead_1$, 
there is a process of \largesys deadlocked in $q$ with input $i$,
that has an outgoing transition guarded ``$\exists q$''
-- hence, adding one more process into $q$ would unlock the process.
%\sj{do we always consider inputs correctly? what if $q \in \dead_1$ for some $e$, but $q \in \dead_2$ for $e'$?}\ak{thanks, modified, now it is impossible}
In contrast, $q \in \dead_2$ if any process deadlocked in $q$
stays deadlocked after adding more processes into $q$.
Let us denote the set of $B$-processes deadlocked in $\dead_1$ by $\mD_1$.
Finally, abuse the definition in Eq.~\ref{disj:def_vfin_wrt}
and denote by $\visFin{\mB\smi\mD_1}{x}$ the set of states
that are visited by $B$-processes not in $\mD_1$ before reaching a deadlocked state.

Given a globally deadlocked run $x$ of \largesys with $n\geq 2|B|-1$, 
we construct a globally deadlocked run $y$ of \cutoffsys with $c = 2|B|-1$ as follows:
\li
\- copy from $x$ into $y$ the local runs of processes in $\mD_1 \cup \{A\}$
\- flood every state of $\dead_2$
\- for every $q \in \visFin{\mB\smi\mD_1}{x}$, flood $q$ and evacuate into $\dead_2$.
\il
The construction ensures: 
(1) for any moment and any process in $y$,
    the set of local states that are visible to the process includes all the states that were visible 
    to the corresponding process in \largesys whose transitions we copy;
(2) in $y$, there is a moment when all processes deadlock in $\dead_1 \cup \dead_2$.

For the case of local deadlocks, 
the construction is similar but slightly more involved, 
and needs to distinguish between unfair and fair cases.
%The construction for local deadlocks is similar but slightly more involved, and needs to distinguish between unfair and fair cases.
In the unfair case, we also copy the behaviour of an infinitely moving process. 
In the strong-fair case,
we continue the runs of non-deadlocked processes with the fair extension. 
\iffinal \else See details in Appendix~\ref{sec:app-disj}.\fi

\ak{put here the tightness picture for deadlocks under fairness?}

%--- global deadlocks: fair and unfair:
%C: processes that dead1
%F: processes that dead2
%We copy local runs of dead1.
%We flood deadlocked states of dead2, and flood and evacuate non-deadlocked states of dead2.
%
%--- local deadlocks: unfair:
%I: processes that move infinitely often
%D: processes that dead
%copy local run of one process from I, 
%copy one local run of process from C,
%flood and evacuate finitely visited states by processes except copied
%
%--- local deadlocks: fair:
%I: processes that move infinitely often
%C: processes that dead1
%F: processes that dead2
%Copy local runs of C, 
%flood and evacuate finitely visited states of F\\C, 
%flood dead or infinitely often visited states of F\\C.
