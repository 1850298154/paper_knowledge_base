\section{Conclusion}
\label{sec:concl}

We have extended the cutoff results for guarded protocols of Emerson and 
Kahlon~\cite{Emerson00} to support local deadlock detection, fairness 
assumptions, and open systems. In particular, our results imply decidability of the parameterized model checking problem for this class of systems and specifications, which to the best of our knowledge was unknown before. 
%Our results allow us to model check
%guarded protocols that satisfy not 
%only safety, but also liveness conditions, for an arbitrary number of 
%components. 
Furthermore, the cutoff results can easily be integrated into 
the parameterized synthesis approach~\cite{Jacobs14,Khalimov13a}.
%~\cite{Jacobs14,Khalimov13,Khalimov13a}.

%An approach for using cutoff results in 
%synthesis has been introduced by 
%Jacobs and Bloem~\cite{Jacobs14}. It has been described in detail for the 
%case of 
%token-passing systems. Follow-up papers have shown how to make the approach 
%more efficient~\cite{KhalimovJB13b}, and how to use it for the synthesis of a 
%large 
%case study, the AMBA bus arbiter~\cite{BloemJK14}.

Since conjunctive guards can model atomic sections and read-write locks, 
and disjunctive guards can model pairwise rendezvous 
(for some classes of specifications, cp.~\cite{EmersonK03}), 
our results apply to a wide spectrum of systems models.
But the expressivity of the model %and flexibility of the results 
comes at a high cost: cutoffs are linear in the size of a process, and 
are shown to be tight (with respect to this parameter).
For conjunctive systems, our new results are restricted to systems with
1-conjunctive guards, effectively only allowing to model a single shared
resource. 
We conjecture that our proof methods can be extended to systems with
more general conjunctive guards, at the price of even bigger cutoffs.
We leave this extension and the question of finding cutoffs that are independent of the size of processes for future research.
%
%
%We are working on a prototype implementation (\url{https://bitbucket.org/parsy/guarded_synthesis/}), which however is currently limited to very small systems.\sj{we should re-formulate or remove the comment on implementation} 
%In future work, we will try to lift the restrictions of our results for conjunctive systems, and investigate cutoffs that are independent of the size of the components' state spaces.
%\ak{remove this promise?}
\ak{note that EK have better complexities for 'for all paths' properties. 
As a future work, one can look if our cutoffs can be improved.}
%This is due 
%to the growth of the cutoff (linearly) and the set of possible transition 
%guards (doubly exponential) in the size of process templates. 
%In the future, 
%we will look into cutoffs that are independent of the size of process 
%templates.

\begingroup
\footnotesize
\smallskip\noindent\textbf{Acknowledgment.}
We thank Roderick Bloem, Markus Rabe and Leander Tentrup for comments on drafts of this paper.
This work was supported by the Austrian Science Fund (FWF) through the 
RiSE project (S11406-N23, S11407-N23) and grant nr.~P23499-N23,
as well as by the German Research Foundation (DFG) through SFB/TR 14 AVACS and
project ASDPS (JA 2357/2-1).
%
%\endgroup
