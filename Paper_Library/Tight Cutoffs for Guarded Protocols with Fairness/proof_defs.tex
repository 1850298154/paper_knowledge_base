\noindent\hrulefill

For a given infinite path $x$ of $\largesys$ we distinguish between states of template $B$ that are visited finitely and infinitely often along $x$: 
\begin{itemize}
\item
$\visited = \{ q \in T_B \| \exists{j \in \bbN}\ \exists{p \in [1,n]}: s_j(B^p) = q \}$,
\sj{for deadlocks, these should only be the states visited by $p \in [2,n]$ --- is this OK in all cases?}

\item 
$\visited_\inf = \{ q\in T_B \| \forall{i} \in \bbN \ \exists{j>i}\ \exists{p \in [1,n]}: s_j(B^p)=q \}$,
 
\item 
$\visited_\fin = \visited \setminus \visited_\inf$.
\end{itemize}
%
Let $\last_\fin$ be the last point in run $x$ with a process in a state from $\visited_\fin$, i.e.: 
$$\last_\fin = max(\{ i \in \bbN \| \exists{p \in [1,n]}: s_i(B^p) \in \visited_\fin \}).$$
Let $\first_\inf$ be the first point in run $x$ where all states from $\visited_\inf$ have been visited at least once, i.e.:
$$\first_\inf = min(\{ i \in \bbN \| \forall{q\in \visited_\inf}\ \exists{j\le i}\ \exists{p \in [1,n]}: s_j(B^p)=q \}).$$
%
Since run $x$ is infinite, process $B^1$ visits some state $q \in \visited_\inf$ infinitely often. 
Denote this state by $\qstar$, and denote the moment when $B^1$ reached $\qstar$ for the first time after $max(\last_\fin+1,\first_\inf)$\footnote{Intuitively, $max(\last_\fin+1, \first_\inf)$ is the first moment in $x$ when $B$-processes exited $\visited_\fin$ and visited all states in $\visited_\inf$ at least once.} by $\pref$, i.e.:
\begin{align*}
& \qstar = any(\{ q \in T_B \| \forall{i \in \bbN} \ \exists{j \ge i}: s_j(B^1) = q\}), \\ 
% & \estar= any(\{ e \| \forall{i} \ \exists{j \ge i}: (s_j(B^1),e_j(B^1)=(\qstar,e) \}), \\ 
& \pref = min( \{ i \ge max(\last_\fin+1,\first_\inf) \| s_i(B^1)=\qstar \}).
\end{align*}\rb{isn't it $any(..) = any(visited_{inf})$ here?}

Intuitively, $\pref$ is the first moment of run $x$ when: all states in $\visited_\inf$ were visited, no process is in a state from $\visited_\fin$, and process $B^1$ is in the state $\qstar$ (which it visits infinitely often in $x$).
Note that all transitions in $x$ that happen after point $\pref$ must be enabled by some state in $\visited_\inf$, since only these states appear after $\pref$. Thus, if in the cutoff system there is at least one process (other than the one making the transition) in every state of $\visited_\inf$, then every transition that happened in $x$ after $\pref$ will be enabled.

For $q \in \visited$ let:
\begin{itemize}
\item $\first_q = min(\{ i \| \exists{p}: s_i(B^p) = q\})$ --- the first moment when state $q$ is reached, 

\item $\witfirst_q = any(\{ p \| s_{\f_q}(B^p)=q \})$ --- the index of a process that reached $q$ first,

\item 
$\last_q = max(\{ i \| \exists{p}: s_i(B^p) = q\})$ if $q\in \visited_\fin$ otherwise $\last_q = \pref$ --- the last point in $x[1:\pref]$ where state $q$ was seen,
 
\item $\witlast_q = any(\{ p \| s_{l_q}(B^p)=q \})$ --- the index of a process that was in $q$ last.
\end{itemize} 
Note that for any $q\in \visited_\fin$  $\first_q$ and $\last_q$ are in $[1,\last_\fin]$, and for any $q \in \visited_\inf$ $\first_q \in [1,\first_\inf]$, $\last_q \in [1,\pref]$.

% Finally, define $\ppi{q}$ and $\ei{q}$: for any state $q \in \visited_\inf$, there is a process of $\largesys$ and a local input $\ei{q}$ that appear infinitely often in $x=(s_1,e_1,sch_1)\ldots$\footnote{The existence of $\ei{q}$ follows from the fact that process inputs are finite while the run $x$ is infinite -- the construction can be adapted for inputs from infinite domains.}.
% Denote such process input by $\ei{q}$ and such process by $\ppi{q}$, i.e.:
% $$\ppi{q} = any(\{ B^p \| \forall{i}\ \exists{j\ge i}: s_j(B^p) = q \ \land \ e_j(B^p)=\ei{q}\}).$$


\noindent\hrulefill