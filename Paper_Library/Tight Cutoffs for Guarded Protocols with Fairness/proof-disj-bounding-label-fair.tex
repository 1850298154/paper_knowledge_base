\begin{proof}
The proof goes along the same lines as the proof of Lemma~\ref{disj:le:FairDisjunctiveBounding}, with modified flooding and fair extension constructions. 

{\bf Original construction.} Remember that for systems with fairness, the flooding construction separates states of the originial run into those that are visited finitely often ($\visited_{fin}$) and those that are visited infinitely often ($\visited_{fin}$). Then, we consider a moment in time $\pref$, which is the first moment such that no states from $\visited_{fin}$ will be visited anymore. The original construction then ensured that the transitions of all local runs are possible until moment $\pref$, and that at this moment there are at least two processes in any given state from $\visited_{inf}$.

Starting from $\pref$, the original proof continues with the \emph{fairing extension} construction that ensures that all local runs can move infinitely often. For the combination of flooding and fair extension, it is crucial that for any state $q$ in $\visited_{inf}$, we combine a prefix from a local run that reaches $q$ first with a postfix (from a possibly different local run) that reaches $q$ infinitely often.

{\bf Modified construction.}
Like in the proofs of Lemmas~\ref{disj:le:NonFairDisjunctiveBoundingLabels} and \ref{le:disj:deadlocks:label}, we define $\visited$ and flooding based on labelings. 

For labelings in $\visited_\fin$, we use the modified evacuation construction from Lemma~\ref{le:disj:deadlocks:label}. That is, we order labelings $\{L_1,\ldots,L_n\} = \visited_\fin$ according to the moment they are evacuated, and add $i$ copies of $B$ for each $L_i$, such that we have a copy that is in a state with labeling $L_i$ before and after all labels that are evacuated before have been evacuated. All of these process copies eventually move to a loop with only labelings from $\visited_\inf$, just like they did in $x$.

For labelings $L \in \visited_\inf$, we use a combination of the original fair extension construction with this modified evacuation construction. The reason we need evacuation is the same as in Lemma~\ref{le:disj:deadlocks:label}: the state with labeling $L$ that is reached first in $x$ may be different from the state $q'$ with labeling $L$ in which some process eventually loops. Therefore, we add one copy $B^{\witfirst_L}$ that takes the shortest path to $L$, and two copies of $B^{\witlast_L}$ that go to a state $q'$ in which there is a loop coming back to $q'$ (that only depends on labelings in $\visited_\inf$). Like in Lemma~\ref{le:disj:deadlocks:label}, we may have the case that a labeling from $\visited_\fin$ needed to leave $q$ will be evacuated before $q'$ is reached by $B^{\witlast_L}$, and therefore we in general need multiple copies of $B$ that reach states with $L$ at different times. Since $L \not\in \visited_\fin$, in general all labelings from $\visited_\fin$ could be evacuated before $q'$ is reached. Thus, for every labeling $L \in \visited_\inf$, we need up to $\card{\visited_\fin}+1$ copies of $B$ that reach $L$ and eventually leave it again, and two additional copies for the fair extension that ensures that $L$ will always be present after $\pref$.

Finally, we use the interleaving construction as before.

In summary, we need $\sum_{i=1}^{\card{\visited_\fin}} i = \frac{\card{\visited_\fin} \cdot (\card{\visited_\fin}+1)}{2}$ copies of $B$ for the labelings in $\visited_\fin$, and $(\card{\visited_\fin}+3)\cdot\card{\visited_\inf}$ copies of $B$ for the labelings in $\visited_\inf$. In the worst case, this means that the cutoff for process template $B$ is (bounded by) \ldots \remove{$3 \card{\labelings_B}$. By a closer analysis, we can see that for at least one labeling $l$ in $\visited_{inf}$, the local run that visits $l$ first must also visit it infinitely often, and we only need one additional copy of this run. Thus, we obtain a cutoff of $3 \card{\labelings_B} - 1$.}
\end{proof}