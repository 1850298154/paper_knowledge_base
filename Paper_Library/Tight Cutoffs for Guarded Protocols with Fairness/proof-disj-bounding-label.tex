\begin{proof}
The proof goes along the same lines as the proof of Lemma~\ref{disj:le:NonFairDisjunctiveBounding}. The main difference is that in the flooding construction, we only need to consider states with different labelings. That is, the flooding construction is modified, as to only include one local run for every state \emph{labeling} that is visited in the original run. Thus, the maximal number of local runs needed to ensure that the behavior of $A^1$ and $B^1$ is possible is $\card{\labelings_B}$.

The rest of the proof works just as before. In particular, we also need the local runs of $A^1$ and $B^1$, as well as possibly a copy $B^\infty$ of a local run that is infinite. Thus, the cutoff for process template $B$ is $\card{\labelings_B} + 2$.
\end{proof}