\section{Preliminaries}
\label{sec:prelim}\label{sec:definitions}

\sj{should we add a remark on which definitions are old, and which are new?}

\subsection{System Model}
\label{sec:model}

We consider systems $A {\parallel} B^n$, usually written $\largesys$, 
consisting of one copy of a process template $A$ and $n$ copies of a process template $B$, 
in an interleaving parallel composition.%
%% AK: moved to a separate note, to be able to explain why we _cannot_ generalize for 1-conj
%\footnote{As shown in \cite{Emerson00}, cutoffs for this case generalize to cutoffs 
%          for systems of the form $A^m {\parallel} B^n$, and further 
%          to systems with an arbitrary number of process templates 
%          $U_1^{n_1} {\parallel} \ldots {\parallel} U_m^{n_m}$.} 
We distinguish objects that belong to different templates by indexing them with
the template. E.g., for process template $U \in \{A,B\}$, $Q_U$ is the set of
states of $U$. For this section, fix two disjoint finite sets $Q_A$, $Q_B$ as
sets of states of process templates $A$ and $B$, and a positive integer $n$.

\smartpar{Processes.} A \emph{process template} 
 is a transition system
  $U=(\stateset, \init, \inputs, \trans)$ with 
	\begin{itemize}
	\item $\stateset$ is a finite set of states including the
  initial state $\init$,
	\item $\inputs$ is a finite input alphabet,
	\item $\trans: \stateset \times \inputs \times \mP(Q_A \cupdot Q_B) \times \stateset$ is a guarded transition relation.
	\end{itemize}
A process template is \emph{closed} if $\inputs = \emptyset$, and otherwise \emph{open}.

We define the size $\card{U}$ of a process template $U \in \{A,B\}$ as $\card{\stateset_U}$. A copy of template $U$ will be called a \emph{$U$-process}.
Different $B$-processes are distinguished by subscript, i.e., for $i \in [1..n]$, $B_i$ is the $i$th copy of $B$, and $\state_{B_i}$ is a state of $B_i$. A state of the $A$-process is denoted by $q_A$. 

For the rest of this subsection, fix templates $A$ and $B$. We assume that $\inputs_A \cap \inputs_B = \emptyset$. We will also write $p$ for a process in $\{ A, B_1, \ldots, B_n\}$, unless $p$ is specified explicitly.


\smartpar{Disjunctive and Conjunctive Systems.}
In a system $\largesys$, consider global state $s = (\state_A,\state_{B_1},\ldots,\state_{B_n})$ and global input $e=(\localin_A,\localin_{B_1},\ldots,\localin_{B_n})$.
We also write $s(p)$ for $q_p$, and $e(p)$ for $\sigma_p$.
A local transition $(\state_p,\localin_p,g,\state_p') \in \trans_U$ of $p$ is \emph{enabled for $s$ and $e$} if its \emph{guard} $g$ is satisfied for $p$ in $s$, written $(s,p) \models g$. 
Disjunctive and conjunctive systems are distinguished by the \emph{interpretation of guards}:
%
\begin{align*}
\text{In disjunctive systems: } & (s,p) \models g \text{~~~iff~~~} 
\exists p' \in \{A,B_1,\ldots,B_n\} \setminus \{p\}:\ \ \state_{p'} \in g. \\
\text{In conjunctive systems: } & (s,p) \models g \text{~~~iff~~~} 
\forall p' \in \{A,B_1,\ldots,B_n\} \setminus \{p\}:\ \ \state_{p'} \in g.
\end{align*}

Note that we check containment in the guard (disjunctively or conjunctively) 
only for local states of processes \emph{different from} $p$. A process is \emph{enabled} for $s$ and $e$ if at least one of its transitions is enabled for $s$ and $e$, otherwise it is \emph{disabled}.

Like Emerson and Kahlon~\cite{Emerson00}, 
we assume that in conjunctive systems $\init_A$ and $\init_B$ are contained in all guards,
i.e., they act as neutral states.
Furthermore, we call a conjunctive system \emph{$1$-conjunctive} if every guard is of the form $(Q_A \cupdot Q_B) \setminus \{q\}$ for some $q \in Q_A\cupdot Q_B$.

Then, \largesys is defined as the transition 
system $(S,\init_S,\globIn,\Trans)$ with 
\begin{itemize}
\item set of global states $S = (\stateset_A) \times (\stateset_B)^{n}$, 
\item global initial state $\init_S = (\initstate_A,\initstate_B,\ldots,\initstate_B)$, 
\item set of global inputs $\globIn = (\inputs_A) \times (\inputs_B)^{n}$,
\item and global transition relation $\Trans \subseteq S \times \globIn \times S$ with $(s,e,s') \in \Trans$ iff 
\begin{enumerate}[label=\roman*)] 
  \item $s=(\state_A,\state_{B_1},\ldots,\state_{B_n})$, 
  \item $e=(\localin_A, \localin_{B_1},\ldots,\localin_{B_n})$, and 
  \item $s'$ is obtained from $s$ by replacing one local state $\state_p$ with a new local state $\state_p'$, where $p$ is a $U$-process with local transition $(\state_{p},\localin_{p},g,\state_p') \in \trans_U$ and $(s,p) \models g$. 
\end{enumerate}
\end{itemize}
We say that a system $\largesys$ is \emph{of type} $(A,B)$. It is called a
\emph{conjunctive system} if guards are interpreted conjunctively, and a
\emph{disjunctive system} if guards are interpreted disjunctively. \sj{is
  previous sentence necessary? conj. and disj. systems are already defined before}
A system is \emph{closed} if all of its templates are closed.
We often denote the set $\{B_1,...,B_n\}$ as $\mB$.



\smartpar{Runs.} 
A \emph{configuration} of a system is a triple $(s,e,p)$, where $s \in S$, $e 
\in \globIn$, and $p$ is either a system process, or the special symbol $\bot$.
 A \emph{path} of a system is a configuration sequence 
$x = (s_1,e_1,p_1),(s_2,e_2,p_2),\ldots$ such that for all $\time < |x|$ there is a 
transition $(s_\time,e_\time,s_{\time+1}) \in \Trans$ based on a local 
transition of process $p_\time$. We say that process 
$p_\time$ \emph{moves} at \emph{moment} $\time$. 
Configuration $(s,e,\bot)$ appears
 iff all processes are disabled for $s$ and $e$.
Also, for every $p$ and $\time < |x|$: 
either $e_{\time+1}(p) = e_\time(p)$ or process $p$ moves at moment $\time$. 
That is, the environment keeps input to each process unchanged until 
the process can read it.\footnote{By only considering inputs that are actually processed, we 
approximate an 
action-based semantics. Paths that do not fulfill this requirement are not 
very interesting, since the environment can violate any interesting 
specification that involves input signals by manipulating them when the 
corresponding process is not allowed to move.} 

A system \emph{run} is a maximal path starting in the initial state. Runs are either infinite, or they end in a configuration $(s,e,\bot)$. We say that a run is \emph{initializing} if every 
%$B$-process 
process
that moves infinitely often also visits 
%$\initstate_B$ 
its $\initstate$ 
infinitely often.

Given a system path $x = (s_1,e_1,p_1),(s_2,e_2,p_2),\ldots$ and a process $p$, the \emph{local path} of $p$ in $x$ is the projection $x(p) = (s_1(p),e_1(p)),(s_2(p),e_2(p)),\ldots$ of $x$ onto local states and inputs of $p$.
Similarly define the projection on two processes $p_1,p_2$ denoted by $x(p_1,p_2)$.

%The \emph{destuttering} $\destutter(x)$\ak{make it work with inf runs} of a (local) path \sj{local path not defined} $x=x_0,x_1,\ldots$ is obtained by removing stuttering steps from the sequence, i.e., $\destutter(x)$ is the maximal subsequence $x'$ of $x$ such that for every $\time$ we have $x'_\time \neq x'_{\time+1}$. Two (local) paths $x$ and $y$ are \emph{stutter-equivalent}, written $x \simeq y$, if $\destutter(x)=\destutter(y)$. Define an extension of $\destutter$ to sets of paths in the obvious way. Then two systems $S_1, S_2$ are \emph{stutter-equivalent}, written $S_1 \simeq S_2$, if $\destutter(X_1) = \destutter(X_2)$, where $X_i$ is the set of all infinite runs of system $S_i$.

\smartpar{Deadlocks and Fairness.}
A run is \emph{globally deadlocked} if it is finite.
An infinite run is \emph{locally deadlocked} for process $p$ if there exists $\time$ such that $p$ is disabled for all $s_{\time'},e_{\time'}$ with $\time'\ge \time$. A run is \emph{deadlocked} if it is locally or globally deadlocked.
A system \emph{has a (local/global) deadlock} if it has a (locally/globally) deadlocked run. Note that absence of local deadlocks for all $p$ implies absence of global deadlocks, but not the other way around.

A run $(s_1,e_1,p_1), (s_2,e_2,p_2),...$ is \emph{unconditionally-fair} if every process moves infinitely often. 
A run is \emph{strong-fair} if it is infinite and for every process $p$, if $p$ is enabled infinitely often, then $p$ moves infinitely often.
%\sj{weak fairness needed?} Finally, $x$ is \emph{weak-fair} if it is infinite and for every process $p$, if there exists $t$ such that $p$ is enabled for every $s_{\time'}, e_{\time'}$ with $\time' \ge \time$, then $p$ moves infinitely often.
We will discuss the role of deadlocks and fairness in synthesis in Sect.~\ref{sec:paramsynt}.

\begin{remark}
Why do we consider systems $A {\parallel} B^n$?
Emerson and Kahlon~\cite{Emerson00} showed how to generalize cutoffs 
for such systems to systems of the form $A^m {\parallel} B^n$, 
and further to systems with an arbitrary number of process templates 
$U_1^{n_1} {\parallel} \ldots {\parallel} U_m^{n_m}$.
This generalization also works for our new results, except for the cutoffs for deadlock detection that are restricted to 1-conjunctive systems (see Section~\ref{sec:cutoffs}).
% AK: i had to change the def of initializing runs: 
% now A is also initializing -- although this is used only in the proof of Monotonicity(conj,dead,fair)
% SJ: if this unnecessarily reduces generality of our results for (conj,prop,fair), then we should not use the more restrictive version for both cases
\end{remark}

\subsection{Specifications}
\label{sec:semantics}
Fix templates $(A,B)$. We consider formulas in $\LTLmX$, i.e., $\LTL$ without the next-time operator $\nextt$.
Let $h(A,B_{i_1},\ldots,B_{i_k})$ be an $\LTLmX$ formula over atomic propositions from $Q_A \cup \Sigma_A$ and indexed propositions from $(Q_B \cup \Sigma_B) \times \{i_1,\ldots,i_k\}$. For a system $\largesys$ with $n \geq k$ and $i_j \in [1..n]$, satisfaction of $\pforall h(A,B_{i_1},\ldots,B_{i_k})$ and $\pexists h(A,B_{i_1},\ldots,B_{i_k})$ is defined in the usual way (see e.g. \cite{PrinciplesMC}).


\smartpar{Parameterized Specifications.} 	
\label{sec:parameterized}
A \emph{parameterized specification} is a temporal logic formula
with indexed atomic propositions and quantification over indices. 
We consider formulas of the forms
$\forall{i_1,\ldots,i_k.} \pforall h(A,B_{i_1},\ldots,B_{i_k})$ and\\ 
$\forall{i_1,\ldots,i_k.} \pexists h(A,B_{i_1},\ldots,B_{i_k})$. 
For given $n \geq k$, 
$$\largesys {\models} \forall{i_1,{\ldots},i_k.} \pforall h(A,B_{i_1},{\ldots},B_{i_k})$$
~iff~
$$\largesys {\models} \bigwedge_{j_1 \neq {\ldots} \neq j_k \in [1..n]} \pforall h(A,B_{j_1},{\ldots},B_{j_k}).$$ 
By symmetry of guarded protocols, this is equivalent 
(cp.\cite{Emerson00})
to $\largesys \models \pforall h(A,B_1,\ldots,B_k)$. 
The latter formula is denoted by $\pforall h(A,B^{(k)})$, 
and we often use it instead of the original $\forall{i_1,\ldots,i_k.} \pforall h(A,B_{i_1},...,B_{i_k})$. For formulas with path quantifier $\pexists$, satisfaction is defined analogously, and equivalent to satisfaction of $\pexists h(A,B^{(k)})$.



\smartpar{Specification of Fairness and Local Deadlocks.}
It is often convenient to express fairness assumptions and local deadlocks 
as parameterized specifications.
To this end,
define auxiliary atomic propositions $\sched_p$ and $\enabled_p$ for every process $p$ of system $(A,B)^{(1,n)}$. At moment $\time$ of a given run $(s_1,e_1,p_1),(s_2,e_2,p_2), \ldots$, let $\sched_p$ be true whenever $p_\time = p$, and let $\enabled_p$ be true if $p$ is enabled for $s_\time, e_\time$. Note that we only allow the use of these propositions to define fairness, but not in general specifications.
Then, an infinite run is 
\begin{itemize}
\item \emph{local-deadlock-free} if it satisfies $\forall{p}. \GF \enabled_p$, abbreviated as $\spec_{\neg dead}$,
\item \emph{strong-fair} if it satisfies $\forall{p}. \GF \enabled_p \impl \GF \sched_p$, abbreviated as $\spec_{strong}$, and 
\item \emph{unconditionally-fair} if it satisfies $\forall{p}. \GF \sched_p$, abbreviated as $\spec_{uncond}$.
%\item \sj{needed?:}\emph{weak-fair} if it satisfies $\forall{p}. \pforall \spec_{weak}$, where $\spec_{weak} = \FG \enabled_p \impl \GF \sched_p$.
\end{itemize}

If \emph{fair} is a fairness notion and 
$\pforall h(A,B^{(k)})$ 
a specification, then we write 
$\pforall_{fair} h(A,B^{(k)})$ for $\pforall (\spec_{fair} 
\rightarrow h(A,B^{(k)}))$. Similarly, we write $\pexists_{fair} h(A,B^{(k)})$ for $\pexists (\spec_{fair} \land h(A,B^{(k)}))$.



\subsection{Model Checking and Synthesis Problems}
\label{sec:nonparameterized_synthesis}
%
For a given system $\largesys$ and specification $h(A,B^{(k)})$ with $n \ge k$,
\begin{itemize}
\item the \emph{model checking problem} is to decide whether $\largesys \models \pforall h(A,B^{(k)})$,
\item the \emph{deadlock detection problem} is to decide whether $\largesys$
      does not have global nor local deadlocks,
%\item the \emph{deadlock detection problem} is to decide whether all runs of $\largesys$
%are infinite and $\largesys \models \pforall \spec_{\neg dead}$, 
%i.e., there are no local deadlocks,
\item the \emph{parameterized model checking problem} (PMCP) is to decide whether $\forall m \ge n:\ (A,B)^{(1,m)} \models \pforall h(A,B^{(k)})$, and 
%\item the \emph{parameterized deadlock detection problem} is to decide whether for all $m \ge n$, all runs of $(A,B)^{(1,m)}$ are infinite and $(A,B)^{(1,m)} \models \pforall \spec_{\neg dead}$.
\item the \emph{parameterized deadlock detection problem} is to decide whether 
      for all $m \ge n$, $(A,B)^{(1,m)}$ does not have global nor local deadlocks.
\end{itemize}
For a given number $n \in \bbN$ and specification $h(A,B^{(k)})$ with $n \ge k$,
\begin{itemize}
\item the \emph{template synthesis problem} is to find process templates $A,B$ such that
$\largesys \models \pforall h(A,B^{(k)})$ and $\largesys$ does not have global deadlocks. 
\item 
the \emph{bounded template synthesis problem} for a pair of bounds $(\bound_A,\bound_B) \in \bbN \times \bbN$ 
is to solve the template synthesis problem with 
$\card{A} \leq \bound_A$ and $\card{B} \leq \bound_B$.
\item the \emph{parameterized template synthesis problem} is to find process templates $A,B$ such that $\forall m \ge n:\ (A,B)^{(1,m)} \models \pforall h(A,B^{(k)})$ and $(A,B)^{(1,m)}$ does not have global deadlocks.
\end{itemize}
These definitions can be flavored with different notions of fairness 
(and similarly for the $\pexists$ path quantifier).
In the next section we clarify the problems studied.
