\section{Related Work}
\label{sec:related}
\iffinal
\else
\hrule
\li
  \- cite ``Parameterized Model-Checking of Timed Systems with Conjunctive Guards'' and another Sasha's paper that contain results for disj guards (including undec result for some CTL*-like logic).

  \- Delzanno in ``Towards Automatic Verification of Java Programs'' studied Monitor objects, and had some kind of boolean guards -- related and compare with our conjunctive guards.

  \- relate with Kahlon's papers on locks, bounded lock chains, etc.

  \- relate to ``Using branching time temporal logic to synthesize synchronization skeletons''

  \- The paper `Automatic Deductive Verification with invisible invariants' \sj{mentioned it below}\ak{yes, but my point was different -- they do have cutoffs. (minor: not for the conference paper)} \sj{seems like they have cutoffs, but only for checking inductive invariants, not finding them} contains cutoff $2b+2$ for 1-indexed properties and $2b+3$ for 2-indexed (pp.7,8) . Is there a connection with our model/cutoffs?

  \- Benedikt Bollig has a paper on ParamSynt: comm is some form of comm automata.
     Can we reduce disj guards to comm automata?

\il
\hrule
\

\fi

As mentioned, we extend the results of Emerson and Kahlon~\cite{Emerson00} who study PMC of guarded protocols, but do not support fairness assumptions, nor provide cutoffs for 
deadlock detection.
In \cite{EmersonK03} they extended their work to systems with limited forms of guards and broadcasts, and also proved undecidability 
of PMC of conjunctive guarded protocols wrt. $\LTL$ (including $\nextt$), 
and undecidability wrt. $\LTLmX$ for systems with both 
conjunctive and disjunctive guards.
%Their line of work has been extended to obtain decision procedures for 
%a combination of limited forms of guarded updates with broadcast communication,
%as well as undecidability results for conjunctive 
%systems with specifications from $\LTL$ (with the $\nextt$ operator), and for 
%systems with both conjunctive and disjunctive guards with specifications from 
%$\LTLmX$~\cite{EmersonK03}.
%, and use the resulting 
%procedures to verify safety of high-level cache coherence 
%protocols~\cite{EmersonK03,Emerson03}. Emerson and Kahlon~\cite{EmersonK03} 
%\remove{Esparza et al. {\protect\cite{EsparzaFM99}} consider the closely related class of 
%systems that communicate by pairwise rendezvous and broadcast in 
%cliques. They 
%show undecidability for liveness specifications, and give a decision 
%procedure for safety specifications that is not based on cutoffs.}

Bouajjani et al.~\cite{Bouajjani08} study parameterized model checking of
resource allocation systems (RASs). Such systems have a bounded number of resources, 
each owned by at most one process at any time. Processes are pushdown automata, 
and can request resources with high or normal priority.
RASs are similar to conjunctive guarded protocols in that certain
transitions are disabled unless a processes has a certain resource. 
RASs without priorities and with processes being finite state automata can 
be converted to conjunctive guarded protocols (at the price of blow up), 
but not vice versa. The authors study parameterized model checking wrt. 
$\LTLmX$ properties on arbitrary or on strong-fair runs, and (local or global) deadlock detection. The proof structure resembles that of \cite{Emerson00}, as does ours.
 
German and Sistla~\cite{German92} considered global deadlocks and strong fairness properties for systems with pairwise rendezvous communication in a clique.
Emerson and Kahlon~\protect\cite{EmersonK03} have shown that disjunctive guard systems can be reduced to such pairwise rendezvous systems.
However, German and Sistla \cite{German92} do not provide cutoffs, 
nor do they consider local deadlocks, 
and their specifications can talk about one process only.
%
Aminof et al.~\cite{AminofKRSV14} have 
recently extended these results to more general topologies, and have 
shown that for some decidable PMC problems there are 
no cutoffs, even in cliques. \ak{update the paragraphs, they also have some
  results on disj?}

\sj{we might remove this paragraph:}\ak{may be worth to compare with conj guards}
Emerson and Namjoshi provide cutoffs for systems that pass a valueless token 
in a ring~\cite{Emerso03}, 
which is essentially resource allocation of a 
single resource with a specific allocation scheme. Their results have been extended to more general topologies~\cite{Clarke04c,AminofJKR14}.
All of these results consider fairness of token passing in the sense that 
every process receives the token infinitely often.

Many of the decidability results above have recently been surveyed by Bloem
et al~\cite{BloemETAL15}. In addition, there are many methods based on
semi-algorithms. 

``Dynamic cutoff''
approaches~\cite{KaiserKW10,AbdullaHH13} support larger 
classes of systems, and try to find cutoffs for a 
concrete system and specification. These 
methods can find smaller cutoffs than those that are statically determined for 
a whole class of systems and specifications, but are currently limited to safety
properties.
The invisible invariants method~\cite{PnueliRZ01} tries to find
invariants in small systems, and applies a specialized cutoff result to prove
correctness of all instances, including an extension to liveness properties~\cite{FangPPZ06}.

Finally, there are methods that work completely without cutoffs, like regular model
checking~\cite{Bouajjani00}, network invariants~\cite{WolperL89,Kurshan95,KestenPSZ02},
and counter abstraction~\cite{PnueliXZ02}. They are in general
incomplete, but may provide decision procedures for certain classes of systems
and specifications, and support liveness to some extent.

%Regular model checking~\cite{Bouajjani00} represents the state space of an infinite 
%number of system instances as regular expressions over a finite alphabet, and 
%provides a decision procedure for certain classes of systems and specifications.
%The invisible 
%invariants method~\cite{PnueliRZ01,FangPPZ06} automatically detects candidate 
%invariants in small system instances, and then tries to verify them for the 
%parameterized case. 
%The approach of network invariants~\cite{WolperL89,Kurshan95,KestenPSZ02} 
%uses a combination of abstraction and parallel composition
%to find a finite system that exhibits all behaviours of the 
%parameterized system and satisfies the property to be 
%verified. The counter 
%abstraction approach~\cite{PnueliXZ02} tries to verify the 
%parameterized system by only keeping track of whether there are $0$, $1$, or 
%more processes in any given local state. 
