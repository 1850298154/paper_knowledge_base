\begin{proof}
The idea of the proof relies on the subtleties of the definition of a run: it is infinite (thus not globally deadlocked), and in each step of a run exactly one process moves. 

Consider the templates in the figure below and let $\pexists h(A,B_1) = \pexists (\eventually 3_{B_1} \land \eventually\always (2_{B_1} \land {end}_A))$. In words: there exists a run in a system where process $B_1$ visits $3_B$ and process $B_1$ with $A$ eventually always stay in $2_B$ and ${end}_A$.
\begin{figure}[h]
% \vspace{-20pt}
\centering
\subfloat[Template A]{
\centering
\makebox[0.4\textwidth][c]{
\scalebox{0.75}{\input{img/disj_tight_propAB_tmplA}}
\label{fig:disj:tight_propAB_tmplA}
}}
\subfloat[Template B]{
\centering
\makebox[0.6\textwidth][c]{
\scalebox{0.75}{\input{img/disj_tight_propAB_tmplB}}
\label{fig:disj:tight_propAB_tmplB}
}}
% \caption{Templates $(A,B)$ used to prove the tightness of the cutoffs for properties $\pexists h(A,B_1)$ (Observation~\ref{obs:disj:tight_prop})\ak{check me}.}
\label{fig:disj:tight_propAB_tmpl}
\end{figure}
% \vspace{-10pt}

We need one process in every state of $B$ to enable the transitions of $A$ to ${all}_A$. Only when $A$ in ${all}_A$, $B_1$ can move $3_B \to 1_B$, and then at some point to $2_B$. After $B_1$ moves $3_B \to 1_B$, $A$ moves ${all}_A \to {end}_A$ which requires process $B_{i \neq 1}$ in $3_B$. Finally, to make the run infinite there should be at least two processes in ${|B|}_B$.\sj{other cases are covered in general lemma below}
% 
\end{proof}
