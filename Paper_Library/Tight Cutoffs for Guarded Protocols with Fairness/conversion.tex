\subsection{Translation Between Open and Closed Systems}
\sj{short description of approach and drawbacks (translation blows up state-space (and thus, cutoffs); does not include any notion of fairness, i.e., only suitable for safety verification)}
In general, the cut-off results for guarded systems can only be used for synthesizing closed systems. However, we provide an algorithm that allows to convert any open system into an equivalent closed system and vice versa. In the corresponding closed system, the environment's behaviour is modelled by non-determinism regarding the transition relations. Applying the inverse conversion to a synthesized closed system yields an open system. To this end, we iteratively convert each open LTS $\mathcal{T}_i = (T_i, t_i, \delta_i, o_i)$ into an equivalent closed LTS in two steps. In the first step, we construct an input-preserving LTS $\mathcal{T}_i^{\text{ip}} = (T_i \times \mathcal{P}(I_i), t_i \times {\_}, \delta_i, o_i)$, such that for each transition $(t, i', t')$ of the given open LTS and each $(t,i) \in T_i \times \mathcal{P}(I_i)$ corresponding to an original state $t \in T_i$, there exists a corresponding transition $((t,i), i', (t',i))$ in the new system. Thus, the last read input values are stored in the current state. In the second step, we remove the input labels from all transitions defined by the transition relation $\delta$. The resulting system does not depend on environment inputs and thus is a closed system. However, each local state still contains information about input values. Whereas these input values are provided by the environment in the open system, the corresponding closed system decides about the input values by selecting a certain transition, and thus a particular set of input values provided by the successor state.

This conversion algorithm allows us to directly use the cut-off results for closed systems. However, a major drawback is the worst-case exponential state blow-up, which is caused by the conversion of the open system to the input-preserving LTS. As a result, there is also an exponential cut-off increase, because the cut-offs are proportional to the state size (i.e., bound). A further issue is the absence of fairness constraints, which allows to synthesize systems that only satisfy specifications that contain safety properties only.
