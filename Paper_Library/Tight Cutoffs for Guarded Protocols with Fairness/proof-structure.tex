%\myparagraph{Proof Structure}
\subsection*{Proof Structure}
To justify the entries in the table,
we first recapitulate the proof structure of the original 
Theorems~\ref{thm:disj-cutoff-pairs} and \ref{thm:conj-cutoff}.
The proofs are based on two lemmas, Monotonicity and Bounding.
We give some basic proof ideas of the lemmas from~\cite{Emerson00} and mention extensions to the cases with fairness and deadlock detection. For cases where this extension is not easy, we will introduce additional proof techniques and explain how to use them in Sections~\ref{sec:ideas-disj} and \ref{sec:ideas-conj}. 
Note that we only consider properties of the form $h(A,B^{(1)})$ --- the
proof ideas extend to general properties $h(A,B^{(k)})$ without
difficulty. Similarly, in most cases the proof ideas extend to open systems
without major difficulties --- mainly because when we construct a simulating
run, we have the freedom to choose the input that is needed. 
% AK: in the main text of the paper no difficulties are seen
Only for the case of deadlock detection we have to handle open systems explicitly.

\smallskip
\noindent
{\bf 1) \emph{Monotonicity} lemma:} if a behavior is possible in a system with $n \in \Nat$ copies of $B$, then it is also possible in a system with one additional process:
\[
\largesys \models \pexists h(A,B^{(1)}) 
~\implies~
(A,B)^{(1, n+1)} \models \pexists h(A,B^{(1)}), 
\]
and if a deadlock is possible in $(A,B)^{(1, n)}$, then it is possible in $(A,B)^{(1, n+1)}$.
%

\begin{proof}[Proof ideas] The lemma is easy to prove for properties 
$\pexists h(A,B^{(1)})$ in both disjunctive and conjunctive systems, by letting the 
additional process stay in its initial state $\init_B$ forever 
(cp.~\cite{Emerson00}). This cannot disable transitions with disjunctive guards, as 
these check for \emph{existence} of a local state in another process (and we 
do not remove any processes), and it cannot disable conjunctive guards since 
they contain $\init_B$ by assumption. 
However, this construction violates fairness, since the new process 
never moves. This can be resolved in the disjunctive case by letting the 
additional process mimic all transitions of an existing process. But in 
general this does not work in conjunctive systems (due to the non-reflexive 
interpretation of guards).
For this case and for deadlock detection, the proof is not 
trivial and may only work for $n \geq c$, for some lower bound $c \in \Nat$ 
(see Sect.~\ref{sec:ideas-disj}, \ref{sec:ideas-conj}).
\end{proof}


\smallskip
\noindent
{\bf 2) \emph{Bounding} lemma:} for a number $c \in \Nat$, a behavior is
possible in a system with $c$ copies of $B$ if it is possible in a system with
$n \geq c$ copies of process $B$:
\[
(A,B)^{(1, c)} \models \pexists h(A,B^{(1)})
~\impliedby~
(A,B)^{(1, n)} \models \pexists h(A,B^{(1)})
,
\]
and a deadlock is possible in \cutoffsys if it is possible in \largesys.

\begin{proof}[Proof ideas] 
  For disjunctive systems, the main difficulty is that removing processes might falsify guards of the local transitions of $A$ or $B_1$ in a given run (see Sect.~\ref{sec:ideas-disj}).
  For conjunctive systems, removing processes from a run is easy for the case of infinite runs, since a transition that was enabled before cannot become disabled. Here, the difficulty is in preserving deadlocks, because removing processes may enable processes that were deadlocked before (Sect.~\ref{sec:ideas-conj}).
\end{proof}

% AK: this remark was moved to the Section with the definition of cutoffs
%\begin{remark}
%Note that \cite{Emerson00} do not consider deadlocked runs in their Monotonicity lemmas, but only in proofs of Bounding lemmas. Therefore, they do not obtain cutoffs for deadlock detection that satisfy the property in Remark~\ref{re:EK_cutoffs}.
%\end{remark}
