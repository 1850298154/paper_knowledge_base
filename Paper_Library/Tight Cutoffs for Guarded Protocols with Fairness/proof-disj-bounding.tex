\begin{proof}
The following proof is similar to the proof for properties of \emph{closed} disjunctive systems by Emerson and Kahlon~\cite[Lemma 4.1.2]{Emerson00}. We separate the proof into two main constructions that will be re-used in later proofs.

Let $c = \card{B}+2$ and $n \geq c$. In the following, we will call $\largesys$ the \emph{large system}, and $\cutoffsys$ the \emph{cutoff system}.

\medskip
\noindent
Direction $\implies$:\\
Let $x=(s_1,e_1,p_1), (s_2,e_2,p_2) \ldots$ be a run \sj{should runs start with $0$ or $1$?} of $\largesys$ that satisfies $\pexists h(A^1,B^1)$.
In the following, we construct a run $y$ of the cutoff system $\cutoffsys$ such that $y(A^1, B^1)\simeq x(A^1, B^1)$.
Informally, the two steps are the following: 
\begin{enumerate}
  \item \emph{Flooding Construction}:\sj{TODO: separate flooding from other constructions}
  given a run $x$ of $\largesys$, we construct local runs $\yflood(A^1)$, $\yflood(B^1)$, $\yflood(B^2)$, \ldots, $\yflood(B^c)$ for the processes of the cutoff system. The construction ensures that
	\begin{enumerate}
	\item $\yflood(A^1)=x(A^1)$ and $\yflood(B^1)=x(B^1)$, and 
	\item when combined into a sequence of pseudo-configurations of $\cutoffsys$, the resulting sequence $\yflood$ is a pseudo-run that is not deadlocked, i.e., it evolves according to enabled transitions of the processes and at least one process moves infinitely often. 
	\end{enumerate}
	To ensure the latter, we devote to every state $q \in T_B$ that is reached in run $x$ one local run of template $B$ that moves into $q$ as fast as possible, and then stutters there forever.
  
  \item \emph{Interleaving Construction}:
The interleaving construction converts the pseudo-run $\yflood$ into the desired run $y$ by restoring an interleaving semantics. To this end, it may add local stuttering of one or several processes to ensure that at most one process moves at any time, and it may remove global stuttering to ensure that at least one process moves. The resulting run $y$ satisfies $y(A^1, B^1)\simeq x(A^1, B^1)$ and is not deadlocked.
\end{enumerate}

In the following, we formalize these two constructions. 

\subsubsection*{Flooding Construction.}
Set $\yflood(A^1)=x(A^1)$, $\yflood(B^1)=x(B^1)$.
Since $x$ satisfies the property, it must in particular be infinite, and hence there is a process of $\largesys$ that moves infinitely often in $x$.
If this process is a copy of $B$, say $B^\infty$, then set $\yflood(B^c) = x(B^\infty)$. Otherwise it is process $A^1$, and we set $\yflood(B^c)=\yflood(B^1)$.
To make sure that all local transitions in $\yflood(A^1)$, $\yflood(B^1)$ and $\yflood(B^c)$ are enabled, we construct additional local runs $\yflood(B^i)$ for $i\in [2,c-1]$ in the following way.

Let $\visited \subseteq T_B$ be the set of all local states of $B$ that are visited by at least one process in the run $x=(s_1,e_1,p_1) (s_2,e_2,p_2) \ldots$ of $\largesys$, i.e.:
$$\visited = \{ q \in T_B \| \exists{\time \in \bbN}\ \exists{i \in [1,n]}: s_\time(B^i) = q \}.$$

For $q \in \visited$ let:
\begin{itemize}
\item $\first_q = min(\{ \time \| \exists{i}: s_\time(B^i) = q\})$ be the first point in time where state $q$ is visited, and 

\item $\witfirst_q = any(\{ i \| s_{\first_q}(B^i)=q \})$ the index of a process that reached $q$ first (a ``witness'' process for $q$).

\end{itemize} 

Let $pr$ be an arbitrary bijection function from $\visited$ to $[2,|\visited|+1]$.
We use $pr$ to map a state $q \in \visited$ to a process $B^{pr(q)}$ of $\cutoffsys$.

For every $q \in \visited$, set
$$ \yflood(B^{pr(q)}) = \left(x(B^{\witfirst_q})\slice{1}{\first_q}\right)(q,e)^\omega,
$$
where $e$ is the local input of process $B^{pr(q)}$ at moment $\first_q$.
In words, to every state of $\visited$ we devote one process in $\cutoffsys$ that reaches it (on the shortest local path known from $x$) and then stays there.

For every $p \in [|\visited|+2,c-1]$ (this set is empty if $\visited=T_B$), set 
$$ \yflood(B^p) = \yflood(B^1),$$
i.e., all processes that do not `flood' states from $\visited$ repeat transitions of the process $B^1$.

\begin{figure}[h]
% \setlength{\abovecaptionskip}{-7pt plus 3pt minus 2pt} % Chosen fairly arbitrarily 
\centering
\scalebox{0.7}{
\input{img/disj_flooding_construction}
}
\caption{Illustration of the flooding construction from Lemma \ref{disj:le:NonFairDisjunctiveBounding}.\ak{Add flooding of the init state}}
\label{disj:fig:FloodingConstruction}
% \vspace{-5pt}
\end{figure}

Thus, we have defined all local runs $\yflood(A^1), \yflood(B^1), \yflood(B^2), \ldots$ in $\cutoffsys$. We obtain a sequence of pseudo-configurations $\yflood=(\sflood_1,\eflood_1,\Pflood_1)\ldots$ in the obvious way by defining the $\sflood_\time$ and $\eflood_\time$ as tuples of the corresponding local states, and letting the $\Pflood_\time$ contain all processes of $\cutoffsys$ that transit between state $\sflood_\time$ and $\sflood_{\time+1}$.\sj{$\leftarrow$ have this outside, in a definition?}
Any $\Pflood_\time$ may contain multiple processes that take a transition at the same time, e.g., if $\witfirst_q = i = \witfirst_{q'}$ for $q\neq q'$, then a prefix of $x(B^i)$ is contained in both $\yflood(B^{pr(q)})$ and in $\yflood(B^{pr(q')})$.
There might be also elements with $\Pflood_\time=\emptyset$. This happens if $p_\time \not\in \{ B^i \| \exists{q} \in \visited:\ i = \witfirst_q \} \cup \{B^1,B^\infty\}$, i.e., $p_i$ is a process that is not used to construct a local run of $\cutoffsys$. Figure~\ref{disj:fig:FloodingConstruction} illustrates the flooding construction.\ak{illustrate issues mentioned below?}

We show that the resulting sequence $\yflood = (\sflood_1,\eflood_1,\Pflood_1)\ldots$ is a pseudo-run of $\cutoffsys$. To this end, observe that the flooding construction ensures the following properties:
\begin{enumerate}
\item [P1.]
$\yflood(A^1,B^1) \simeq x(A^1,B^1).$

\item [P2.]
$\forall{q \in \visited}\ \forall{\time \ge \first_q}:\ \sflood_\time(B^{pr(q)}) = q.$

\item [P3.]
At any point of $\yflood$, if some process of $\cutoffsys$ transits $\transition{q}{q'}{e:g}$, then at the same point of the run $x$ of $\largesys$ there is a process that makes exactly the same transition $\transition{q}{q'}{e:g}$.
% $\forall{i} \ \forall{(p^*\!:\!\transition{q}{q'}{e:g})}:$
% $$(p^*\!:\!\transition{q}{q'}{e:g}) \in (s^*_i \rightarrow s^*_{i+1})
% ~~\impl~~
% \exists{p}\ (p\!:\!\transition{q}{q'}{e:g}) \in  (s_i \rightarrow s_{i+1}),$$
% where $p^*$ is a process of $\cutoffsys$ and $p$ is a process of $\largesys$.
\item [P4.]
At any point of $\yflood$, if more than one process transits, then they make exactly the same transition.
\end{enumerate}

We need to show that every element $(\sflood_{\time+1},\eflood_{\time+1},\Pflood_{\time+1})$ is derived from element $(\sflood_\time,\eflood_\time,\Pflood_\time)$ by local transitions of all processes in $\Pflood_\time$. 
Suppose that $p \in \Pflood_\time$, and the local transition from $q=\sflood_\time(p)$ to $q'=\sflood_{\time+1}(p)$ (with local input $\eflood_\time(p)$) is guarded by some guard $g \subseteq T_A \cup T_B$. Then by property P3 some process of $\largesys$ makes exactly the same transition, and hence in $s_\time$ of $x$ there is a process in some state $r \in g$. Consider two cases:
\begin{itemize}
  \item if $r \in T_A$, then by property P1, we have $\sflood_\time(A^1)=r$, and thus the transition is enabled.
  \item if $r \in T_B$, then $\time \ge \first_r$.
  Note that $p \neq B^{pr(r)}$ (since process $B^{pr(r)}$ does not move after reaching $r$ at time $\first_r$), and hence the transition is enabled.
\end{itemize}
 
\subsubsection*{Interleaving Construction.}
To obtain a run $y$ with $y(A^1, B^1)\simeq x(A^1, B^1)$ from $\yflood$, we first\rb{what is second?} sequentialize simultaneous transitions of processes.

Let $\interleave$ be the function that takes as input a pseudo-run $(\sflood_1,\eflood_1,\Pflood_1)$, $(\sflood_2,\eflood_2,\Pflood_2)$\ldots from the flooding construction and returns the pseudo-configuration sequence $\pi_1 \cdot \pi_2 \cdot \ldots$, where the $\pi_\time$ are non-empty sequences constructed from $(\sflood_\time, \eflood_\time, \Pflood_\time)$ in the following way:
\begin{itemize}
  \item if $|\Pflood_\time|\le 1$, then $\pi_\time = (\sflood_\time, \eflood_\time, \Pflood_\time)$,
 
  \item otherwise, $\Pflood_\time$ contains several processes that move simultaneously, say $p_1,\ldots,p_n$. By property P4, we know that in this case $p_\time \in \{ B^1,\ldots,B^c\}$ for all $\time$.
	%Then:
%  $\transition{s_\time(p_1)}{s_{\time+1}(p_1)}{e_\time(p_1)}$,
%  $\ldots$, 
%  $\transition{s_\time(p_n)}{s_{\time+1}(p_n)}{e_\time(p_n)}$.
  Then $\pi_\time = (s_{\time_1},e_{\time_1},P_{\time_1}) \ldots (s_{\time_n}, e_{\time_n}, P_{\time_n})$, where:
  \begin{itemize}
  \item 
  $(s_{\time_1}, e_{\time_1}, P_{\time_1}) = (\sflood_\time, \eflood_\time, \{ p_1 \})$, 
  \item 
  $s_{\time_2} = s_{\time_1}[s_{\time_1}(p_1) \gets \sflood_{\time+1}(p_1)]$, \\
  $e_{\time_2} = e_{\time_1}[e_{\time_1}(p_1) \gets \eflood_{\time+1}(p_1)]$, \\
  $P_{\time_2} = \{ {p_2} \}$;

  \item 
  \ldots
  \item 
  $s_{\time_n} = s_{\time_{n-1}}[s_{\time_{n-1}}(p_n) \gets \sflood_{\time+1}(p_n)]$, \\
  $e_{\time_n} = e_{\time_{n-1}}[e_{\time_{n-1}}(p_n) \gets \eflood_{\time+1}(p_n)]$, \\
  $P_{\time_n} = \{ {p_n} \}$.
  \end{itemize}
  Note that $\sflood_{\time+1} \equiv s_{\time_n}[s_{\time_n}(p_n) \gets \sflood_{\time+1}(p_n)]$.
\end{itemize}

It is straightforward to show that $\interleave(\yflood)$ is a pseudo-run of $\cutoffsys$ where at each moment either one or no process moves, and $y(A^1,B^1) \simeq x(A^1,B^1)$.
Thus, let $y = \destutter(\interleave(\yflood))$ to obtain the desired run\footnote{Note that $\destutter(x)$ is finite for a run $x$ that ends in an infinite repetition of the same configuration $(s,e,p)$. We abuse the notation and write $\destutter(x)$ for $\destutter(x)\ (s,e,p)^\omega$ in this case.}
of $\cutoffsys$.
This concludes the proof of $\implies$ direction.

\bigskip
\noindent Direction $\impliedby$\sj{if we keep Monotonicity Lemma, this is just repeated application}\ak{but then you need to prove Monotonicity Lemma?} (proof idea):\\
Construct the run $y$ of the large system $\largesys$ from a given run $x$ of the cutoff system $\cutoffsys$ as follows: $y(A^1)=x(A^1)$, $y(B^i)=x(B^i)$ for $i\in [1,c]$, and $y(B^{c+i}) = x(B^c)$ for $i \in [1,n-c]$. 
Use the interleaving construction to restore the interleaving semantics of the resulting pseudo-run to obtain $y$.
% 
\end{proof}