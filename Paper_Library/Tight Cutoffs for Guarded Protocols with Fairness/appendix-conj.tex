\section{Cutoffs for Conjunctive Systems}
\label{sec:app-conj}
%
%For convenience in the proofs, we {\bf inverse the meaning of conjunctive guard}, 
%i.e., given system state $s$, process $p$, and guard $guard \in Q_B \cupdot Q_A$:
%$$
%(s,p) \models guard 
%\ \iff\ 
%\forall p' \in \{A,B_1,\ldots,B_n\} \smi \{p\}:\ \state_{p'} \not\in guard.
%$$

\subsection{Conjunctive Systems without Fairness}

\begin{restatable}[Monotonicity: Conj, Props, Unfair]{lem}{ConjMonotonicityLemma}
\label{le:ConjMonotonicityLemma}
For conjunctive systems,
\begin{align*}
\forall n \geq 1:\ 
(A,B)^{(1,n)} \models \pexists h(A,B_1)
\ \ \Impl \ \ 
(A,B)^{(1,n+1)}\models \pexists h(A,B_1).
\end{align*}
\end{restatable}
\begin{proof}
Let the new process stutter in $\init$ state.
\end{proof}

\begin{restatable}[Bounding: Conj, Props, Unfair]{lem}{ConjBoundingLemma}
\label{le:ConjunctiveBoundingLemma}
For conjunctive systems,
\begin{align*}
\forall n \geq 2:\ 
(A,B)^{(1,2)} \models \pexists h(A,B_1)
\ \ \Implied \ \ 
\largesys \models \pexists h(A,B_1).
\end{align*}
\end{restatable}

\begin{proof}
The proof is inspired by the first part of the proof of \cite[Lemma 5.2]{Emerson00}.

Let $x=(s_1,e_1,p_1) (s_2,e_2,p_2) \ldots$ be a run of $\largesys$. 
Note that by the semantics of conjunctive guards, 
the transitions along any local run of $x$ will also be enabled 
in any system $\cutoffsys$ with $c \leq n$, 
where the processes exhibit a subset of the local runs of $x$. 
Thus, we obtain a run of $\cutoffsys$ by copying a subset of the local runs of $x$, 
and removing elements of the new global run where all processes stutter.
\sj{should we put this as a general lemma somewhere?}

Then, based on an infinite run $x$ of the original system, 
we construct an infinite run $y$ of the cutoff system. 
Let $y(A)=x(A)$ and $y(B_1)=x(B_1)$. 
The second copy of template $B$ in $(A,B)^{(1,2)}$ is needed to ensure that 
the run $y$ is infinite, i.e., at least one process moves infinitely often. 
If both $x(A)$ and $x(B_1)$ eventually deadlock, 
then there exists a process $B_i$ of $\largesys$ that makes infinitely many moves, 
and we set $y(B_2) = x(B_i)$. 
Otherwise, we set $y(B_2) = x(B_2)$.

\end{proof}

\begin{restatable}[Conj, Props, Unfair]{tightness}{TightConjPropRestricted}
\label{obs:conj:tight_prop}
The cutoff $c=2$ is tight for parameterized model checking of properties $\pexists h(A,B_1)$ in the 1-conjunctive systems, i.e., there is a system type $(A,B)$ and property $Eh(A,B_1)$ which is not satisfied by $(A,B)^{(1,1)}$ but is by $(A,B)^{(1,2)}$.
\end{restatable}
\begin{proof}
The figure below shows templates $(A,B)$, $\pexists h(A,B_1) = \pexists \eventually b$. An infinite run that satisfies the formula needs one copy of $B$ that stays in the initial state, and one that moves into $b$.
\begin{figure}[h]
\centering
\subfloat[Template A]{
\centering
%\makebox[0.3\textwidth][c]{
\scalebox{0.75}{\input{img/conj_tight_prop_tmplA}}
\label{fig:conj:tight_prop_fair_tmplA}
}%}
\hspace{1cm}
\subfloat[Template B]{
\centering
%\makebox[0.4\textwidth][c]{
\scalebox{0.75}{\input{img/conj_tight_prop_tmplB}}
\label{fig:conj:tight_prop_fair_tmplB}
}%}
\end{figure}
\end{proof}


\ifwithextensions %%%%%%%%%%%%%%%%%%%%%%%%%%%%%%%%%%%%%%%%%%%
\begin{restatable}[Generalized Bounding Lemma]{lem}{ConjBoundingLemmaGeneral}
\label{le:ConjunctiveBoundingLemmaGen}
For conjunctive systems,
\begin{align*}
&\forall n \geq k+1: \largesys \models \pexists h(A,B_{(k)}) 
\ \iff\ 
(A,B)^{(1,k+1)} \models \pexists h(A,B_{(k)})
\end{align*}
\end{restatable}

\begin{proof}
The proof is the same as for Lemma~\ref{le:ConjunctiveBoundingLemma}, except that we copy the $k$ local runs that exhibit the property, in addition to the local run that ensures that the resulting global run will be infinite.
\end{proof}

\begin{restatable}[Tightness]{obs}{TightConjPropRestrictedGeneral}
\label{obs:conj:tight_prop:general}
The cutoff $k+1$ for properties $\pexists h(A,B_{(k)})$ on unconstrained runs in 1-guard conjunctive systems, i.e., there is a system type $(A,B)$ and property $Eh(A,B_{(k)})$ which is not satisfied by $(A,B)^{(1,k)}$ but is satisfied by $(A,B)^{(1,k+1)}$.
\end{restatable}
\begin{proof}
Consider again Fig.~\ref{fig:conj:tight_prop_tmpl}, and the property $\pexists h(A,B_{(k)}) = \pexists \bigwedge_{i \in [1..k]} \eventually b_i$.
\end{proof}
\fi      %\ifwithextensions %%%%%%%%%%%%%%%%%%%%%%%%%%%%%%%%%%%%%%%%%%%

% \subsubsection{Deadlock Detection.}

% A conjunctive system $\largesys$ is called \emph{disjoint-guards system}\ak{bad name, negations of guards are disjoint, not guards} if for any transitions $t, t' \in \delta$: $\neg\guard(t)\cap \neg\guard(t') \neq \emptyset \implies \guard(t) = \guard(t')$.
% \ak{too restricted -- does not allow read-write locks. Try more general version: $\neg\guard(t) \cap \neg\guard(t') \neq \emptyset \implies \guard(t) \subseteq \guard(t') \vee \guard(t') \subseteq \guard(t)$.}
% 
% For a global state $s$ of a system $(A,B)^{(1,n)}$ and $U\in\{A,B\}$, let $Set(s,U)$ be the set of states of all processes of type $U$ in $s$.\ak{find home}
% 
% Let $(\stateset, \init, \inputs, \trans, \guard)$ be a process template. For transition $t \in \delta$, and $g\in\guard(t)$ let \emph{resource} be the the set $\neg g$.\ak{find home}\ak{better to define conj guards via negations of current guards?}

\begin{restatable}[Monotonicity: Conj, Deadlocks, Unfair]{lem}{ConjDeadlockMonotonicityLemmaRestricted}
\label{le:ConjunctiveMonotonicityLemmaDeadlocks}
For conjunctive systems:
$$
\forall n\geq 1: (A,B)^{(1,n)} \textit{ has a deadlock} 
\ \Impl\ 
(A,B)^{(1,n+1)} \textit{ has a deadlock}
$$
% \li
%   \- with $c=2|Q_B\smi \{ \init \}|$:\hl{XXX}
%   $$(A,B)^{(1,c)} \textit{ has a global deadlock} \ \Impl\ (A,B)^{(1,>c)} \textit{ has a global deadlock} $$
%   
%   \- with $c=1$:
%   $$(A,B)^{(1,c)} \textit{ has a local deadlock} \ \Impl\ (A,B)^{(1,>c)} \textit{ has a local deadlock}$$
%   
%   \- with $c=1$:
%   $$(A,B)^{(1,c)} \textit{ has a deadlock} \ \Impl\ (A,B)^{(1,>c)} \textit{ has a deadlock}$$

% \il
\end{restatable}
\begin{proof}
Given a deadlocked run $x$ of $(A,B)^{(1,n)}$, we construct a deadlocked run of $(A,B)^{(1,n+1)}$. Let $y$ copy run $x$, and keep the new process in $\init$. If $x$ is globally deadlocked and $d$ is the moment when the deadlock happens in $x$, then schedule the new process arbitrarily after moment $d$.
% 
% 
% \myparagraph{Local Deadlocks}
% Given a locally deadlocked run $x$ of \cutoffsys, we construct a locally deadlocked run of \largesys. Copy the run $x$, and keep excessive processes in $\init$.
% 
% \myparagraph{Deadlocks}
% Given a deadlocked run $x$ of \cutoffsys, we construct a deadlocked run of \largesys. Copy the run $x$, and keep excessive processes in $\init$. If $x$ is globally deadlocked and $d$ is the moment when the deadlock happens in $x$, then schedule excessive processes arbitrarily after moment $d$.
% 
% \myparagraph{Global Deadlocks}
% Given a globally deadlocked run $x$ of \cutoffsys, we construct a globally deadlocked run of \largesys:
% \hl{todo}
% 

\end{proof}


\begin{restatable}[Bounding: 1-Conj, Deadlocks, Unfair]{lem}{ConjDeadlockLemmaRestricted}
\label{le:ConjunctiveBoundingLemmaDeadlocks}
For 1-conjunctive systems:
\li
  \- with $c=2|Q_B\smi \{ \init \}|$ and any $n>c$ \footnote{This statement also applies to systems without restriction to $1$-conjunctive guards.}
  $$(A,B)^{(1,c)} \textit{ has a global deadlock} \ \Implied\ (A,B)^{(1,n)} \textit{ has a global deadlock} $$
  
  \- with $c=|Q_B\smi \{ \init \}|+2$ and any $n>c$:
  $$(A,B)^{(1,c)} \textit{ has a local deadlock} \ \Implied\ (A,B)^{(1,n)} \textit{ has a local deadlock}$$
  
  \- with $c=2|Q_B\smi \{ \init \}|$ and any $n>c$:
  $$(A,B)^{(1,c)} \textit{ has a deadlock} \ \Implied\ (A,B)^{(1,n)} \textit{ has a deadlock}$$
\il
\end{restatable}
\begin{proof}
The proof is inspired by the second part of the proof of \cite[Lemma 5.2]{Emerson00}, 
but in addition to global we consider local deadlocks. 

\myparagraph{Global Deadlocks} 
Let $c=2|Q_B\smi\{ \init \}|)$. 
Let run $x = (s_1,e_1,p_1)\ldots(s_d,e_d,\bot)$ of \largesys 
with $n>c$ be globally deadlocked. 
We construct a globally deadlocked run $y$ in $\cutoffsys$:
\li
  \-[a.] for every $q \in Set(s_d) \setminus \{\init\}$:
  \li
    \- if $s_d$ has two processes in state $q$, 
       then devote two processes of \cutoffsys that mimic the behaviour 
       of the two of \largesys correspondingly

    \- otherwise, $s_d$ has only one process in state $q$, 
       then devote one process of \cutoffsys that mimics the process of \largesys
  \il
  \-[b.] for any process of \cutoffsys not used in the construction (if any): 
         let it mimic an arbitrary $B$-process of \largesys 
         not used in the construction (including (b))
\il
The construction uses (if ignore (b)) $\leq 2|Q_B\smi \{ \init \}|$ processes $B$. 
Note that the proof does not assume that the system is 1-conjunctive.

\myparagraph{Local Deadlocks} 
Let $c = |Q_B\smi \{ \init \}|+2$. 
Let run $x = (s_1,e_1,p_1)\ldots$ of \largesys with $n>c$ be locally deadlocked. 
We will construct a run $y$ of \cutoffsys 
where at least one process deadlocks and exactly one process moves infinitely often.

Wlog. we distinguish three cases:
\li
\-[1.] $A$ moves infinitely often in $x$, and $B_1$ deadlocks
\-[2.] $A$ deadlocks, and $B_1$ moves infinitely often
\-[3.] $A$ neither deadlocks nor moves infinitely often, $B_1$ deadlocks, 
       $B_2$ moves infinitely often
\il

\myparagraph{1} ``$A$ moves infinitely often in $x$, and $B_1$ deadlocks''.

Let $q_\bot, e_\bot$ be the deadlocked state and input of $B_1$ in $x$, 
and let $d$ be the moment from which $B_1$ is deadlocked.

Let $DeadGuards=\{q_1,\ldots,q_k\}$ be the set of states
such that for every $q_i \in DeadGuards$ there is an outgoing transitions 
from $q_\bot$ with $e_\bot$ guarded ``$\forall \neg q_i$'',
and assume $DeadGuards \neq \emptyset$
(if it is empty, then we keep every process in $\init$ 
 until someone reaches $q_\bot$ and then schedule the rest arbitrarily). 
(Recall that $q_i \in Q_B \cupdot Q_A$).

The construction is:
\li
  \-[a.] $y(A)=x(A)$, $y(B_1)=x(B_1)$
  \-[b.] for each $q \in DeadGuards$, at moment $d$ in $x$
         there is a process $p_q$ in state $q$. 
         If $p_q \in \{B_1,...,B_n\}$, 
         then let one process of \cutoffsys mimic it till moment $d$, 
         and then stutter in $q$.
  \-[c.] let other processes of \cutoffsys (if any) stay in $\init$.
\il
The construction uses (if ignore (c)) $\leq |Q_B\smi \{ \init \}|+1$ processes $B$.

Note: 
the assumption of 1-conjunctive systems implies that,
in order to deadlock $B_1$,
we need a process in each state in $BlockGuards$.
This implies that having a process in each state of $BlockGuards$ does not disable 
any $A$'s transition after moment $d$.

\myparagraph{2} ``$A$ deadlocks, and $B_1$ moves infinitely often'': 
use the construction from (1).

\myparagraph{3} 
``$A$ neither deadlocks nor moves infinitely often, 
  $B_1$ deadlocks, $B_2$ moves infinitely often''. 
Use the construction from (1), and additionally: $y(B_2)=x(B_2)$. 
Thus, the construction uses (if ignore (c)) $\leq |Q_B \smi \{ \init \}|+2$ 
processes $B$.

\myparagraph{Deadlocks}
Take the higher value among the cases considered above $c=2|Q_B\smi \{ \init \}|$: 
if $x$ is locally deadlocked then the monotonicity lemma ensures 
that there is a deadlocked run in \cutoffsys.

\end{proof}

\ak{proof trial of the disjoint conj is commented out}
%\begin{lem}
%``Bounding: disjoint-conj, Deadlocks, Unfair''
%\end{lem}
%\input{other/disjoint-conj-dead-proof-trial}

\begin{restatable}[1-Conj, Deadlocks, Unfair]{tightness}{TightConjDeadlockRestricted}
\label{obs:conj:tight_deadlock}
The cutoff $c=2|B|-2$ is tight for parameterized deadlock detection in the 1-conjunctive systems, i.e., for any $k$ there is a system type $(A,B)$ with $|B|=k$ such that there is a deadlock in $(A,B)^{(1,2|B|-2)}$, but not in $(A,B)^{(1,2|B|-3)}$. 
\end{restatable}
% 
\begin{proof} 
The figure below provides templates $(A,B)$ that proves the observation. In the figure the edge with $\forall{\neg b_1},\ldots,\forall{\neg b_k}$ denotes edges with guards $\forall{\neg b_1},\ldots,\forall{\neg b_k}$. To get the global deadlock we need at least two processes in each $b_i \in \{b_1,\ldots,b_k\}$. Note that the system does not have local deadlocks.\ak{show that cutoffs for local deadlocks are also tight}
\begin{figure}[h]
\vspace{-10pt}
\centering
\subfloat[Template A]{
\centering
%\makebox[0.4\textwidth][c]{
\scalebox{0.75}{\input{img/conj_tight_deadlock_tmplA}}
\label{fig:conj:tight_deadlock_tmplA}
}%}
\hspace{1cm}
\subfloat[Template B]{
\centering
%\makebox[0.6\textwidth][c]{
\scalebox{0.75}{\input{img/conj_tight_deadlock_tmplB}}
\label{fig:conj:tight_deadlock_tmplB}
}%}
\label{fig:conj:tight_dead_tmpl}
\end{figure}
\end{proof}



\ifwithextensions   %%%%%%%%%%%%%%%%%%%%%%%%%%%%%%%%%%%%%%%%%%%%
Now consider the general case of guards of the original (unrestricted) form:
\begin{restatable}[Local Deadlock Detection, Unrestricted Guards]{lem}{ConjDeadlockUnrestricted}
\label{le:ConjunctiveBoundingLemmaDeadlocksUnrestricted}
Let $c=3^{\card{B}}$.
If there are no local deadlocks 
in the conjunctive system $\cutoffsys$, then for any $n \ge c$ the conjunctive system $\largesys$ has no local deadlocks.
\end{restatable}

\iffinal
\else
\begin{proof}
Suppose run $x= (s_0,e_0,p_0),(s_1,e_1,p_1),\ldots$ of $\largesys$ is locally deadlocked for at least one process, and at least one process keeps on moving forever. 
We construct a run $y= (s^*_0,e^*_0,p^*_0),(s^*_1,e^*_1,p^*_1),\ldots$ where at least one process deadlocks and at least one process moves on forever in the following way.

Let $y(A^1,B^1)=x(A^1,B^j)$, where $j$ is chosen such that either $A^1$ moves infinitely often and $B^j$ deadlocks, or the other way around. Let $U^1$ denote the deadlocked process in $\cutoffsys$.

Then, let $d$ be the point in time from which $A^1$ or $B^i$ are disabled 
forever in $x$. To ensure that all transitions of $U^1$ are disabled in $s_d$ 
of $\cutoffsys$, let $I = \{ i | s_d(B^i) = q \textit{ for some } q \in 
Set(s_d,B)\}$, let $pr: I \rightarrow [2:\card{Set(s_d,B)}]$, and define $y(B^{pr(i)
}) = x(B^i)$ for all $i \in I$. In the following, we want to ensure that: a) 
$U^1$ will always remain deadlocked, and b) all transitions on local run 
$y(\overline{U}^1)$ are enabled. To ensure a), it is sufficient to ensure that 
$Set(s^*_\time,B) \supseteq Set(s_\time,B)$, and to ensure b), it is sufficient to ensure $Set(s^*_\time,B) \subseteq Set(s_\time,B)$. 
Thus, the goal of our 
construction is to ensure $Set(s^*_\time,B) = Set(s_\time,B)$ for all $\time 
\ge d$.\ak{What is $Set$? Is a set of states or set of sets?}

For a transition of $\largesys$ where $A^1$ or $B^j$ move, we fire the same transition in $\cutoffsys$. Otherwise, consider the following cases for a transition of $\largesys$ from $s_\time$ to $s_{\time+1}$:\sj{formally, this is probably a (0,1,many)-counter abstraction}
\begin{enumerate}
\item $Set(s_{\time+1},B) = Set(s_\time,B)$: we drop this transition in $\cutoffsys$, unless it leads to the situation that only one process is in the state that was left in the original transition\sj{i.e., unless it changes the state in the abstraction}; in that case, let all but one processes move along that transition
\item $Set(s_{\time+1},B) = (Set(s_\time,B) \setminus \{ q \}) \cup \{q'\}$ for some $q \in Set(s_\time,B)$, $q' \not\in Set(s_\time,B)$: we simulate the transition from $q$ to $q'$ with the unique process in $\cutoffsys$ that is in state $q$
\item $Set(s_{\time+1},B) = Set(s_\time,B) \setminus \{ q \}$ for some $q \in Set(s_\time,B)$: we simulate the transition from $q$ to a state $q' \in Set(s_{\time+1},B)$ with the unique process in $\cutoffsys$ that is in state $q$
\item $Set(s_{\time+1},B) = (Set(s_\time,B) \cup \{q\}$ for some $q \not\in Set(s_\time,B)$: if possible, we simulate this by a number of transitions from a state $q'$ to $q$, where multiple processes are in $q'$; if this is not possible, we add another local run: $y(B^{c+1})=x(B^i)$ for some $i$ with $s_{\time+1}(B^i) = q$.
\end{enumerate}

Using this construction, we simulate $\largesys$ in $\cutoffsys$ until we have 
reached $\time$ such that there exists $\time'$ with $d < \time' < \time$ and 
$s_\time'=s_\time$. The maximal length of a subsequence of $x$ such that 
$s_\time \neq s_{\time'}$ for all $x_\time,x_\time'$ is bounded by $\card{B}^n$. 
However, we only simulate steps where $Set(s_\time,B)$ changes, or where all 
but one processes move out of a state. Furthermore, we can remove loops 
within the abstraction, i.e., if there are $\time,\time'$ such that 
$Set(s^*_\time)=Set(s^*_{\time'})$, then we cut $x^*[\time+1:\time']$ from the run. 
Thus, there are at most $3^{\card{B}}$ different abstract configurations that 
can be reached on such a path. The number of steps of case d) that add 
another run is thus bounded by $3^{\card{B}}$.\sj{somewhat better cutoff possible, but probably not worth the effort...}

Note that the abstract state at the end of this loop is the same as in the original configuration, and therefore arbitrarily many executions of the loop can be appended.
\end{proof}
\fi
\fi   %\ifwithextensions %%%%%%%%%%%%%%%%%%%%%%%%%%%%%%%%%%%%%%%%%%%%


\subsection{Conjunctive Systems with Fairness}
In this section, subscript $i$ in path quantifiers, $\pexists_i$ and $\pforall_i$, 
denotes the quantification over initializing runs.

\begin{restatable}[Monotonicity: Conj, Props, Fair]{lem}{ConjMonLemmaFair}
\label{le:ConjMonFair}
For unconditionally-fair initializing runs of conjunctive systems:\sj{generalization is obvious; $n \ge k+1$ in general case}
\begin{align*}
& \forall n \ge 2:\\
& (A,B)^{(1,n)} \models \pexists_{uncond,i} h(A,B_1)
\ \Impl \
(A,B)^{(1,n+1)} \models \pexists_{uncond,i} h(A,B_1).
\end{align*}
\end{restatable}
\begin{proof}
Given a unconditionally-fair initializing run $x$ of $\largesys$, we construct a unconditionally-fair initializing run $y$ in $(A,B)^{(1,n+1)}$, with one additional process $p$. 
First, copy all local runs of all processes of $(A,B)^{(1,n)}$ from the run $x$ into $y$.
Then, let process $p'$ stutter in $\init$ until some other process $p \neq B_1$ enters $\initstate$. 
Then, exchange the roles of processes $p'$ and $p$: let $p$ stutter in $\initstate$, while $p'$ takes the transitions of $p$ from the original run, until it enters $\initstate$. And so on.
In this way, we continue to interleave the run between $p'$ and $p$, and obtain a unconditionally-fair initializing run for all processes, with $y(A,B_1)=x(A,B_1)$. 
Thus, if $\largesys \models \pexists h(A,B_1)$, then $(A,B)^{(1,n+1)} \models \pexists h(A,B_1)$.
\end{proof}

\begin{restatable}[Bounding: Conj, Props, Fair]{lem}{ConjBoundingLemmaFair}
\label{le:FairConjunctiveBounding Lemma}
For unconditionally-fair initializing runs of conjunctive systems:
\begin{align*}
&\forall n \geq 1:\\
& (A,B)^{(1,1)} \models \pexists_{uncond} h(A,B_1)
\ \Implied \
(A,B)^{(1,n)} \models \pexists_{uncond} h(A,B_1)
\end{align*}
\end{restatable}
%

\begin{proof}
Given an unconditionally-fair [initializing] run $x$ of $\largesys$ with $n>c$ construct an unconditionally-fair [initializing] run $y$ in the cutoff system $(A,B)^{(1,1)}$: copy the local runs of processes $A$, $B_1$.
%  and copy the behaviour of a process of $\largesys$ that moves infinitely often in the run $x$. Since $y$ is the result of removal of a number of local runs from $x$, it is an unconditionally-fair initializing run of $\cutoffsys$.
\end{proof}

\begin{restatable}[1-Conj, Props, Fair]{tightness}{TightConjPropRestrictedFair}
\label{obs:conj:tight_prop_fair}
The cutoff $c=2$ is tight for parameterized model checking of $\pexists h(A,B_1)$ 
on unconditionally-fair initializing runs in 1-conjunctive systems, 
i.e., 
there is a system type $(A,B)$ and property $\pexists h(A,B_1)$ 
which is satisfied by $(A,B)^{(1,1)}$ but not by $(A,B)^{(1,2)}$.
\end{restatable}
\begin{proof}
The figure below shows $(A,B)$, $\pexists h(A,B_1) = \pexists \FG (b_{init} \impl a_1)$.
\begin{figure}[h]
\centering
\vspace{-20pt}
\subfloat[Template A]{
\centering
%\makebox[0.4\textwidth][c]{
\scalebox{0.75}{\input{img/conj_tight_prop_fair_tmplA}}
\label{fig:conj:tight_prop_fair_tmplA}
}%}
\hspace{1cm}
\subfloat[Template B]{
\centering
%\makebox[0.6\textwidth][c]{
\scalebox{0.75}{\input{img/conj_tight_prop_fair_tmplB}}
\label{fig:conj:tight_prop_fair_tmplB}
}%}
\label{fig:conj:tight_prop_fair_tmpl}
\end{figure}
\end{proof}

\begin{restatable}[Monotonicity: Conj, Deadlocks, Fair]{lem}{ConjDeadlockMonotonicityLemmaFair}
\label{le:FairConjunctiveMonotonicityLemmaDeadlocks}
For 1-conjunctive systems on strong fair initializing or finite runs:
$$
\forall n\geq 1: (A,B)^{(1,n)} \textit{ has a deadlock}
\ \Impl\ 
(A,B)^{(1,n+1)} \textit{ has a deadlock}
$$
\end{restatable}
\begin{proof}\ak{check the minimal value of $n$ (1 or 2?)}
Let $x$ be a globally deadlocked or locally deadlocked strong-fair initializing run of $(A,B)^{(1,n)}$.
We will build a globally deadlocked or locally deadlocked strong-fair initializing run 
of $(A,B)^{(1,n+1)}$.

If $x$ is finite, then $y$ is the copy of $x$, and the new process stays in $\init_B$
until every process become deadlocked, and then is scheduled arbitrarily.
Note that $y$ constructed this way may be locally deadlocked 
rather than globally deadlocked as $x$ is.

Now consider the case when $x$ is locally deadlocked strong-fair initializing.

Let $\mD$ be the set of deadlocked $B$-processes in $x$, and $d$ be the moment 
when the processes become deadlocked.

Consider the case $\visInf{\mB\smi\mD}{x} \neq \emptyset$:
copy $x$ into $y$, and let the new process $B_{n+1}$ wait in $\init_B$ 
and interleave the roles with a process $B$ that moves infinitely often in $x$, 
similarly to as described in the proof of Lemma~\ref{le:ConjMonFair}.

Consider the case $\visInf{\mB\smi\mD}{x} = \emptyset$:
every $B$ process of $(A,B)^{(1,n)}$ is deadlocked and thus $\mD = \mB$.
Define 
$$
DeadGuards = \{\ q \| \exists P \in \mD
                      \textit{ with a transition guarded ``\,}
                      {\forall \neg q} 
                      \textit{\!'' in } (s_d(P),e_d(P))\ \}.
$$
Note that $Q_A \cap DeadGuards = \emptyset$, because $A$ visits infinitely often $\init_A$
and we consider 1-conjunctive systems.
Hence, copy $x$ into $y$, and let the new process $B_{n+1}$ wait in $\init_B$ 
until every process $B_1,...,B_n$ become deadlocked, and then schedule $B_{n+1}$ arbitrarily.
%
% See the proof of Lemma~\ref{le:ConjunctiveMonotonicityLemmaDeadlocks}.
%%% AK: this won't work because the process B_{n+1} in init should move inf often or deadlock,
%%% the arbitrary scheduling can lead to unlocking of every one
\end{proof}

\begin{restatable}[Bounding: 1-Conj, Deadlocks, Fair]{lem}{ConjDeadlockLemmaFair}
\label{le:FairConjunctiveBoundingLemmaDeadlocks}
For 1-conjunctive systems on strong-fair initializing or finite runs:
\ak{no real need for initializing -- but easier to explain}
\li
  \- with $c=2|Q_B\smi \{ \init \}|$ and any $n>c$:
  $$
  \cutoffsys \textit{ has a global deadlock} 
  \ \Implied\ 
  \largesys \textit{ has a global deadlock}
  $$

  \- with $c=2|Q_B\smi \{ \init \}|+1$ and any $n>c$ (when $|Q_B|>2$):
  $$
  \cutoffsys \textit{ has a local deadlock} 
  \ \Implied\ 
  \largesys \textit{ has a local deadlock}
  $$

  \- with $c=2|Q_B\smi \{ \init \}|$ and any $n>c$:
  $$
  \cutoffsys \textit{ has a deadlock} 
  \ \Implied\ 
  \largesys \textit{ has a deadlock}
  $$
\il
\end{restatable}
\begin{proof}
\providecommand{\deadOne}{\dead_1}
\providecommand{\deadTwo}{\dead_2}

\myparagraph{Global Deadlocks}
$c=2|Q_B \smi \{\init_B\}|$, 
see Lemma~\ref{le:ConjunctiveBoundingLemmaDeadlocks}, 
the fairness does not matter on finite runs.

\myparagraph{Local Deadlocks}
Let $c=2|Q_B\smi \{ \init_B \}|$. 
Let $x= (s_1,e_1,p_1)\ldots$ be a locally deadlocked strong-fair intitializing run 
of $\largesys$ with $n>c$. 
We construct a locally deadlocked strong-fair initializing run $y$ of $\cutoffsys$.

Let $\mD$ be the set of deadlocked processes in $x$. 
Let $d$ be the moment in $x$ starting from which every process in $\mD$ is deadlocked.

Let $\dead(x)$ be the set of states in which processes $\mD$ of \largesys
are deadlocked.

Let $\deadTwo(x) \subseteq \dead(x)$ be the set of deadlocked states such that: 
for every $q \in \deadTwo(x)$, 
there is a process $P \in \mD$ with $s_d(P) = q$ 
and that for input $e_{\geq d}(P)$ has a transition guarded with ``$\forall \neg q$''.
Thus, a process in $q$ is deadlocked with $e_d(P)$
only if there is another process in $q$ in every moment $\geq d$.

Let $\deadOne(x) = \dead(x)\smi\deadTwo(x)$.
I.e., 
for any $q \in \deadOne(x)$, there is a process $P$ of \largesys 
which is deadlocked in $s_d(P) = q$ with input $e_d(P)$,
and no transitions from $q$ with input $e_d(P)$ are guarded with ``$\forall \neg q$''.
%%% AK:
%%% Note that this def of deadOne differs from the below one (that i used originally):
%%% for every $q \in \deadOne(x)$, 
%%% there is a process $P \in \mD$ of \largesys
%%% that for input $e_{\geq d}(P)$ does not have a transition guarded with ``$\forall \neg q$''.
%%% The definition currently used may define deadOne which has less states that the commented one.
%%% But in the commented version we used the trick "Wlog., assume deadOne \cap deadTwo = 0", 
%%% which seems to produce the same set of states!

%Similarly, let $\deadOne(x)$ be the set of deadlocked states such that: 
%for every $q \in \deadOne(x)$, 
%there is a process $P \in \mD$ of \largesys
%that for input $e_{\geq d}(P)$ does not have a transition guarded with ``$\forall \neg q$''.
%I.e., 
%for such process $P$ is deadlocked in $q \in \deadOne(x)$ with input $e_{\geq d}(P)$,
%even if there is no other process in $q$ at any moment after $d$.
%
%Wlog., assume that $\deadOne(x) \cap \deadTwo(x) = \emptyset$.
%\footnote{Note:
%it is possible that
%$\deadOne \cap \deadTwo \neq \emptyset$ 
%for some locally deadlocked strong-fair initializing run $x$.
%But from $x$ we can always produce a locally deadlocked strong-fair initializing
%run $x'$ with an empty intersection as follows.
%Let $q \in \deadOne \cap \deadTwo$.
%Then, there is a process, $p \in {B_1,...,B_n}$, 
%deadlocked in $q$ with input $e(p)$ such that 
%all outgoing transitions from $q$ with $e(p)$ are not guarded with 
%``$\forall \neg q$''.
%Then, to all processes deadlocked in $q$ we provide input $e(p)$.
%This removes state $q$ from $\deadTwo$.
%By modifying inputs to all processes deadlocked in a state in the intersection,
%we can produce $x'$ with the empty intersection.
%}

%Let $\visited^\inf(x)$ be the set of states that are entered and
%exited infinitely often in $x$
%(this definition is slightly different from that of the rest of the paper).

%\ak{redefine the previous $\visited^\inf$ to mean: ``entered and exited inf often''?}

%Note:
%it is possible that $\dead(x) \cap \visited^\inf(x) \neq \emptyset$:
%one process may be deadlocked in such state provided one input,
%while another process may enter and exit the state infinitely often
%provided a different input.
% we cannot deadlock those non-deadlocked processes that
% visit $dead$, 
% because they might be moving on the loop that deadlocks some states,
% and deadlocking this process would require to move others from their states
% and BOOM! everyone gets unlocked!

Define
$$
DeadGuards = \{\ q \| \exists P \in \mD
                      \textit{ with a transition guarded ``\,}
                      {\forall \neg q} 
                      \textit{\!'' in } (s_d(P),e_d(P))\ \}.
$$
We illustrate properties of sets 
$DeadGuards$, $\deadOne$, $\deadTwo$, $\visInf{\mB\smi\mD}{x}$ 
in Fig.~\ref{fig:conj-deadlocks-venn}.
\ak{check how $A$'s states affect all those sets, currently i assumed that they are all subsets of $Q_B$}

\begin{figure}[h]
\begin{mdframed}
\centering
\includegraphics[width=0.7\textwidth]{img/conj-deadlocks-venn.png}
\captionsetup{singlelinecheck=off}
\caption[fig:conj-deadlocks-venn]{%
Venn diagram for sets $DeadGuards$, $\deadOne$, $\deadTwo$, $\visInf{\mB\smi\mD}{x}$:
\begin{itemize}
\item[($q_1$)] $\deadOne \cap DeadGuards \cap \visInf{\mB\smi\mD}{x} \neq \emptyset$ is possible:
               in $x$, 
               there is a process deadlocked in state $q_1$,
               there is a non-deadlocked process that visits $q_1$ infinitely often,
               and there is a process deadlocked in a state $q \neq q_1$ 
               with a transition guarded ``$\forall \neg q_1$'' 

\item[($q_3$)] $\deadOne \cap DeadGuards \smi \visInf{\mB\smi\mD}{x} \neq \emptyset$ is possible:
               similarly to $q_1$, 
               except that no non-deadlocked processes visit $q_3$ infinitely often

\item[($q_2$)] $\deadOne \smi (\visInf{\mB\smi\mD}{x} \cup DeadGuards) \neq \emptyset$ is possible:
               in $x$, 
               there is a process deadlocked in state $q_2$,
               no other processes visit $q_2$ infinitely often,
               and no processes are deadlocked with a transition guarded ``$\forall \neg q_2$''

\item[($q_4$)] $DeadGuards \smi \dead \neq \emptyset$ is possible:
               there is a process deadlocked in a state $q \neq q_4$ 
               with a transition guarded  ``$\forall \neg q_4$''

\item[($q_5$)] $\deadTwo \cap \visInf{\mB\smi\mD}{x} \cap DeadGuards \neq \emptyset$ is possible:
               there is at least one process deadlocked in $q_5$ with a transition guarded ``$\forall \neg q_5$'',
               and some non-deadlocked process visits $q_5$ infinitely often
               (this process does not deadlock in $q_5$, 
                because in $q_5$ it receives an input different from that of the deadlocked processes)

\item[($q_6$)] $\deadTwo \cap DeadGuards \smi \visInf{\mB\smi\mD}{x} \neq \emptyset$ is possible:
               similarly to $q_5$, except no non-deadlocked processes visit $q_6$ infinitely often
\end{itemize}
}
\label{fig:conj-deadlocks-venn}
\end{mdframed}
\end{figure}

Let us assume $DeadGuards \neq \emptyset$ -- the other case is straightforward.\ak{check}

The construction has two phases, the setup and the looping.
The setup phase is:
\li
\-[a.] $y(A) = x(A)$

\-[b.] for every $q \in \deadOne$: 
   devote one process of \cutoffsys that copies 
   a process of \largesys deadlocked in $q$

\-[c.] for every $q \in \deadTwo \setminus \visInf{\mB\smi\mD}{x}$: 
   devote two processes of \cutoffsys that copy 
   the behaviour of two processes of \largesys that deadlock in $q$

\-[d.] for every $q \in \deadTwo \cap \visInf{\mB\smi\mD}{x}$:
   in $x$, 
   there is a process, $B_q^\inf \in \mB\smi\mD$, that visits $q$ infinitely often,
   and there is a process, $B_q^\bot \in \deadTwo$, deadlocked in $q$.
   Then:
\li
   \-[1.] devote one process of \cutoffsys that copies the behaviour of $B_q^\bot$
   \-[2.] devote one process of \cutoffsys that copies the behaviour of $B_q^\inf$ 
          until it reaches $q$ at a moment after $d$,
          and then provide the same input as $B_q^\bot$ receives at moment $d$.
          This will deadlock the process.
\il

\-[e.] for every $q \in DeadGuards \setminus \dead$:
       note that $q \in \visInf{\mB\smi\mD}{x}$ and, thus, there is a process, 
       $B_q^\inf \in \mB\smi\mD$, 
       that visits $q$ infinitely often.
       Devote one process of \cutoffsys that copies the behaviour of $B_q^\inf$ 
       until it reaches $q$ at a moment after $d$

\-[f.] if $DeadGuards \setminus \dead \neq \emptyset$ 
       or $A \in \mD$,
       then devote one process that stays in $\init_B$.
       The process will be used in the looping phase to ensure that the run $y$ is infinite,
       and that every process of \cutoffsys used in (e) 
       moves infinitely often (and thus $y$ is strong-fair).
%       Note that if $A$ moves infinitely often in $x$ and $DeadGuards \smi \dead = \emptyset$,
%       then there is no need for such additional infinitely moving process.

\-[g.] let any other process of \cutoffsys (if any) 
       copy behaviour of a process of \largesys 
       that was not used in the construction so far (including this step)
\il
\ak{go through every item, and prove it is necessary (by giving an example)}
The setup phase ensures: 
in every state $q \in \dead$,
there is at least one process deadlocked in $q$ at moment $d$ in $y$. 
Now we need to ensure that the non-deadlocked processes described 
in steps (e) and (f) move infinitely often.

The looping phase is applied to processes in (e) and (f) only\footnote%
{%
  If there are no such processes, then the setup phase produces the sought run $y$.
}.
Order arbitrarily 
$DeadGuards \smi \dead = (q_1,\ldots,q_k) \subseteq \visInf{\mB\smi\mD}{x}$.
Note that $\init_B \not\in (q_1,...,q_k)$.
Let $\mP$ be the set of processes of \cutoffsys used in steps (e) or (f).
Note that $|\mP| = |(q_1,...,q_k)| + 1$.

The looping phase is: set $i=1$, and repeat infinitely the following.
\li
  \- let $P_\init \in \mP$ be the process that is currently in $\init_B$, 
     and $P_{q_i} \in \mP$ -- in $q_i$
     
  \- let $B_{q_i} \in \visInf{\mB\smi\mD}{x}$ be a process of \largesys 
     that visits $q_i$ and $\init_B$ infinitely often.
     Let $P_\init$ of \cutoffsys copy transitions of $B_{q_i}$
     on some path $\init_B \to \ldots \to g_i$,
     then let $P_{g_i}$ copy transitions of $B_{q_i}$ on some path 
     $g_i \to \ldots \to \init_B$. 
     For copying we consider only the paths of $B_{q_i}$ that happen after moment $d$.

  \- $i=i \oplus 1$
\il

The number of copies of $B$ that the construction uses in the worst case is 
(if ignore (g), assume $Q_B>2$, $DeadGuards \smi \dead = \emptyset$, and $A \in \mD$):
$$
1_{(f)} + 2|\deadTwo|_{(c),(d)} + |\deadOne|_{(b)} 
 \leq 
2|Q_B \smi \{\init_B\}| + 1.
$$

\myparagraph{Deadlocks}
The largest value of $c$ among those for ``Local Deadlocks'' 
and for ``Global Deadlocks'' can be used as the sought value of $c$ 
for the case of general deadlocks.
But it will not be the smallest one.
In the proof of the case ``Local Deadlocks'', in the setup phase, 
item (e) can be modified for the case when $A \in \mD$:
since we do not need to ensure that $y$ is infinite, 
we avoid allocating a process in state $\init_B$.
For a given locally deadlocked strong-fair run, the setup phase may produce
the globally deadlocked run, but that is allright for the case of general deadlocks.
With this note, for the general case $c = 2|Q_B \smi \{\init_B\}|$.
%
%\li
%  \-[a.] $y(A^1) = x(A^1)$
%  \-[b.] for each $q \in \deadOne(x)$: devote one process of \cutoffsys that mimics a process of \largesys that deadlocks in $q$\ak{replace `mimic' -- copy?}
%
%  \-[c1.] for each $q \in \deadTwo(x) \smi BlockStates$: devote two processes of \cutoffsys that mimic two processes of \largesys that deadlock in $q$
%  \-[c2.] for each $q \in \deadTwo(x) \cap BlockStates$: devote one processes of \cutoffsys that mimic one processes of \largesys that deadlocks in $q$
%  \-[d.] for each $q \in BlockStates$ devote one process of \cutoffsys that mimics a process $p_q$ that is in state $q$ at moment $d$. Note that such process in \largesys exists. 
%  
%  \-[e.] if $BlockStates \neq \emptyset$, then devote one process of \cutoffsys that stays in $\init$
%
%%   \-[d2.] Note \largesys has a process $p_q'$ different from $p_q$ from (d1) that enters $q$ infinitely often. Process $p_q'$ also visits $\init$ infinitely often. Thus, devote one process of \cutoffsys that stays in $\init$. Let $m_{qq}$ be some moment after 
%%   \li
%%     \-[d1.] if there is a process in \largesys that loops $q\to q$ infinitely often, then devote one process of \cutoffsys that mimics this process
%
%%     \- every process of \largesys that enters $q$ later exits $q$. Hence there are two processes of \largesys that meet in $q$ at some moment $m_{qq}$: devote two processes $p_1,p_2$ of \cutoffsys that mimic the behaviour of such processes till they meet in $q$ at the moment $m_{qq}$.
%%     Now observe that there is an infinite number of looping paths from $q \to \ldots\to q$ in the run $x$ by processes that enter $q$ and exit $q$ infinitely often.
%%     Starting from moment $m_{qq}$ interleave loopings between processes $p_1,p_2$, namely, start with $p_1$: stutter them both until some process of \largesys does the looping $q\to \ldots \to q$, and let process $p_1$ repeat that looping while keeping process $p_2$ in $q$.
%%     Now change turns: stutter them both until the moment when some process of \largesys does the looping $q \to \ldots \to q$, and let $p_2$ repeat the looping while keeping $p_1$ in state $q$. And so on.\ak{needs formalization}
%%   \il 
%
%  \-[f.] let any other process of \cutoffsys (if any) mimic a process of \largesys that was not used in the construction so far (including step (e))
%\il
%The setup phase ensures that in every state in $\dead(x)$ there is at least one process deadlocked at moment $d$ in $y$. Now we need to ensure that non-deadlocked processes described in step (d) move infinitely often.
%
%The looping phase applies to processes in (d) and (e) only. Order arbitrarily $BlockStates=(g1,\ldots,g_k)$. Then, set $i=1$, and repeat infinitely:
%\li
%  \- let $B^\init$ be the process from step (d) or (e) that is currently in $\init$, and $B^{g_i}$ is the one from (d) or (e) that is in $g_i$
%  \- in $x$ state $g_i$ is visited infinitely often by a process of \largesys that starts in $\init$. Hence, let $B^\init$ mimic that process on its loop from $\init \to \ldots\to g_i$, then let $B^{g_i}$ mimic that process on $g_i \to \ldots \to \init$
%  \- $i=i \oplus 1$
%\il
%The construction uses (if ignore (f)) assuming $Q_B>2$ and in the worst case (when $BlockStates$ is empty) $|\deadOne(x)| + 2|\deadTwo(x)| \leq 2|Q_B\smi \{ \init \}|$ processes B.
%
% The upper bound on the number of processes used in the construction of $y$ is $2(|BlockSet|-1) \leq 2|B|-2$.
%
% \input{other/disjoint-conj-dead-fair-proof-trial}

\end{proof}

\begin{restatable}[1-Conj, Deadlocks, Fair]{tightness}{TightConjDeadlockRestrictedFair}
\label{obs:conj:tight_deadlock_fair}
The cutoff $c=2|B|-2$ is tight for deadlock detection on strong-fair initializing
or finite runs in the 1-conjunctive systems, 
i.e., 
for any $k>2$ there is a system type $(A,B)$ with $|B|=k$ such that 
there is a strong-fair initializing deadlocked run in $(A,B)^{(1,2|B|-2)}$, 
but not in $(A,B)^{(1,2|B|-3)}$.
\end{restatable}

% 
\begin{proof} 
Consider the same templates as in Observation~\ref{obs:conj:tight_deadlock}.
%
% AK: we can claim the below, but we need to note that 
% the monotonicity lemma for local deadlocks also holds (straightforward)
% let's comment this out for now.
%  Note that the cutoff $c=2|B|-1$ stated in the previous Lemma 
% for the case of local deadlocks is also tight.
% To prove this, take the templates from Observation~\ref{obs:conj:tight_deadlock}
% and modify slightly the template B:
% add the unguarded self-loop to $\init$.
%%%
% \begin{figure}[Htb]
% \centering
% \subfloat[Template A]{
% \centering
% \makebox[0.4\textwidth][c]{
% \scalebox{0.75}{\input{img/conj_tight_deadlock_tmplA}}
% \label{fig:conj:tight_deadlock_tmplA}
% }}
% \subfloat[Template B]{
% \centering
% \makebox[0.6\textwidth][c]{
% \scalebox{0.75}{\input{img/conj_tight_deadlock_tmplB}}
% \label{fig:conj:tight_deadlock_tmplB}
% }}
% \caption{Templates $(A,B)$ used to prove the tightness of the cutoff $c=2|B|-2$ for the deadlock detection in 1-guard conjunctive systems.
% In the figure the edge with $\forall{\neg b_1},\ldots,\forall{\neg b_k}$ denotes edges with guards $\forall{\neg b_1},\ldots,\forall{\neg b_k}$ (Observation~\ref{obs:conj:tight_deadlock}).\ak{check me}}
% \label{fig:conj:tight_dead_tmpl}
% \end{figure}
\end{proof}






\ifwithextensions      %%%%%%%%%%%%%%%%%%%%%%%%%%%%%%%%%%
\subsection{Label-dependent Cutoffs for Conjunctive Systems}

To obtain label-dependent cutoffs, we need to re-define the special role of the initial state for conjunctive guards: in this case, we assume that the initial state has a labeling $l_\initstate$, and this labeling must be part of any guard.

\begin{restatable}[Deadlock Detection, Label-based]{lem}{ConjDeadlockLemmaLabel}
\label{le:ConjunctiveBoundingLemmaDeadlocksLabel}
Let $c=2\card{\labelings_B}-2$.
If there are no local deadlocks 
in the conjunctive system $\cutoffsys$, then for any $n \ge c$ the conjunctive system $\largesys$ has no local deadlocks.
\end{restatable}

\iffinal
\else
\begin{proof}
The proof works in the same way as for Lemma~\ref{le:ConjunctiveBoundingLemmaDeadlocks}, except that we consider visited labelings instead of states.
\end{proof}
\fi
\fi     % \ifwithextensions   %%%%%%%%%%%%%%%%%%%%%%%%5

\iffinal
\else
\subsection{Cutoffs for 2-Conjunctive-Disjoint Guard Systems}
\ak{finished, but not polished}

\begin{restatable}[Deadlock Detection, 2-Disjoint Guards]{lem}{blabla}
For 2-conjunctive-disjoint systems on strong-fair initializing runs, 
with $c\approx 2|Q_B \smi \{\init\}|$\ak{calc} and any $n>c$:
  $$
  \cutoffsys \textit{ has a deadlock} 
  \ \Implied\ 
  \largesys \textit{ has a deadlock}
  $$
\end{restatable}
\begin{proof}
%
$BlockGuards$ are the guards that are not blocked by the deadlocked processes,
and, thus, must be blocked for the local deadlocks to occur.

Inf-moving processes visit such guards by looping 
$\init \rightsquigarrow \init$.
Thus, through every guard some process goes inf-often.
Link with a guard a process of the large system, called $B_m$, 
that visits it inf-often.

Roughly, the idea of the construction is the following.
We devote one process that starts in $\init$.
For every blocking guard, in the cutoff system, 
a process called $B_m'$ will emulate a loop 
$\init \rightsquigarrow \init$ of process $B_m$ of the large system.
In the loop of the process $B_m$, 
blocking guard states are visited in some order: 
some states are entered through and some are exited through.
First, we ensure that there is a process in every ``exit'' state of every 
blocking guard on the loop of $B_m$.
Second, let $B_m'$ mimick transitions of $B_m$ till it reaches the first
``enter'' state, 
then let the process, that is currently in the ``exit'' state, 
continue mimicking the transitions of $B_m$, and so on.
This way we reach the blocking guard under consideration and then reach $\init$.

The realitiy is more involved: 
the main complication is that the same blocking guard may be visited 
several times, thus, strictly speaking, ``enter'' and ``exit'' states
are not well-defined in the above.
Below we decribe the details.

\smallskip
\noindent
\emph{Notes.}
Since there is only a finite number of transitions and blocking guards 
but runs are infinite, after some moment:
\li
\-[1.] Any loop's transition fires inf-often.
\-[2.] Any loop's path between two blocking guards fire inf-often.
\il

Below we consider only such looping paths.

\smallskip
\noindent
\emph{Simplification.}
Given a blocking guard $g$ and a loop that visits $g$, 
divide the loop into three parts:
\li
\- $\init \rightsquigarrow g$ 
   (from $\init$ until the first visit of $g$),
\- $g \rightsquigarrow g$ 
   (from the first visit of $g$ until the last visit), and
\- $g \rightsquigarrow \init$ 
   (from the last visit until $\init$).
\il

Given a guard and a part, \emph{enter[exit]} state is 
the first[last] state of the guard visited on the path.
For the middle part, 
this equivalents to the first[last] state of the path.

Now the \emph{simplification}.
Given a part of the loop, 
remove all loops from the part that start and end in the same state.
Apply this to the first, middle, and last part.

% do we really need the simplification?
% > yes, due to several times firing of the internal transition of a guard
% > and we really need both: simplificaiton of inter- and intra-guard transitions,
% > because:
% 1. internal transition may be fired several times 
%    without visiting other blocking guards:
%    +-----+
%    |     |
%    [guard]
%
% 2. internal transition may be fired several times 
%    within the same guard:
%    +---------+
%    | .<--->. | (guard)
%    +---------+ 

Call the loop consisting of the simplified first, middle and last parts,
the \emph{simplified loop}.
Then, the simplification ensures:
\li
\- for every simplified part: 
   a blocking guard $g$ can be visited more than once
   only if $enter_g \neq exit_g$
   and the internal transitions of $g$ are not used.

\- in the simplified loop:
   if an internal blocking guard transition is fired more than once,
   then the firings are separated by a visit of another blocking guard.
\il

\smallskip
\noindent
\emph{Preparation.}
Fix a part of the simplified loop (first, middle, or last).
In this part, for each guard $g$, the transition through $g$ 
is of the form:
\li
\-[f1.] $enter \rightarrow exit$ ($g$ is visited once), or
\-[f2.] $enter \rightarrow \neg g \rightsquigarrow exit$ ($g$ is visited twice).
\il
%If the target guard $guard$ visit is of the first type, then,
%according to the notes, $enter \rightarrow exit$ is fired inf-often.
Given a blocking guard $g$ visited in the part,
the \emph{preparation of $g$ wrt. the part} is:
if transition through $g$ is of the second form -- do nothing,
if of the first form -- wait until the transition is fired
in the original run of the large system, then execute it.
The preparation ensures:
\li
\- if the process $B_g'$ is in state $enter$, 
   then in the part the transition through the guard is of the second form.
\il

\smallskip
\noindent
\emph{Main construction.}
For each blocking guard, fix a loop that visits it inf-often.
Fix a blocking guard $guard$, called the target guard, and, thus, fix the loop.
Derive the simplified loop.
Let the simplified loop be of the form:
$$ 
\underbrace{\init \ g_1' \ ... \ g_f'}_\text{the first part} \ 
\underbrace{g^\star \ g_1'' \ ... \ g_m'' \ g^\star}_\text{the middle part} \ 
\underbrace{g_1''' \ ... \ g_l''' \ \init}_\text{the last part}
$$
In the above formula, $f$, $m$, or $l$ can be $0$, but for simplicity
consider they are not.

%Prepare all processes, one by one, of the blocking guards visited in the first part 
%wrt. the first part,
%and the process of the target guard wrt. middle part.
%This ensures,
%for every $g$, 
%which is a blocking guard of the first part or the target guard,
%$B_{g}'$ is either in $exit$ or
%internal transitions of $g$ are not fired in the corresponding part.
%\footnotemark[1000].

Start with $\init$ and the process $B_\init'$.
Call the process of the cutoff system, we currently move, $B_m'$.
Initially $B_m'$ is $B_\init$.
Prepare the process in $g_1'$ wrt. the first part.
Make $B_m'$ mimick $B_m$ until it reaches $g_1'$:
\li
\- if transition through $g_1'$ in the first part is of form (f2),
   then let $B_m'$ continue mimicking $B_m$ until it reaches 
   the next blocking guard
\- if transition through $g_1'$ in the first part if of form (f1),
   then leave $B_m'$ in $enter_{g_1'}$, wait until $B_m$ reaches $exit_{g_1'}$ 
   (this ensures that we do not transit inside the guard 
   when there is another process, $B_{exit_{g_1'}}$, in it).
   Then, prepare the next blocking guard wrt. the corresponding part.
   Set $B_m' = B_{exit_{g_1'}}$.
   Then, let $B_m'$ mimick $B_m$ until it reaches the next blocking guard.
\il
Repeat this construction in a natural way until $B_m'$ reaches $\init$.

For a given target guard $g^\star$, the main construction ensures that the process 
$B_{g^\star}$ moves to the next blocking guard $g$ on the simplified loop.
But we need to ensure that $B_{q^\star}$ eventually reaches $\init$.
To this end, set $g^\star=g$ and repeat the construction.
And so on, until the desired process reaches $\init$.
Finally, do this for every blocking guard.

This concludes the description of the main construction.

Note: the setup phase of the main construction, 
      where we put one process into every blocking guard and 
      one or two processes into every deadlocked state,
      is straightforward.

Finally, the cutoff is defined by the maximal number of the blocking guards + 1,
and is of the scale $\approx 2|Q_B\smi\{\init\}|$.
\end{proof}

\fi

\iffinal
\else
\subsection{Cutoffs for Conjunctive-Embedding Guard Systems}
\ak{not finished -- run into the problem that I could not solve.
    Originally, I wanted to put every process into every guard, 
    into the least restrictive state, 
    and then do looping, but the problem happens if the transition inside the guard is self-guarded}

\begin{restatable}[Deadlock Detection, Embedding guards]{lem}{blabla}
For 2-conjunctive-embedding systems on strong-fair initializing runs, 
with $c=???$ and any $n>c$:
  $$
  \cutoffsys \textit{ has a deadlock} 
  \ \Implied\ 
  \largesys \textit{ has a deadlock}
  $$
\end{restatable}
\begin{proof}
Given a locally deadlocked strong-fair initializing run of the \largesys, 
we construct such run in \cutoffsys.

Examples of 2-conjunctive-embedding guards: 
$\{\{a,b\}, \{a\}\}$, but not $\{\{a,b\}, \{a\}, \{b\}\}$.
I.e., all guards that are embedded into some guard are comparable with $\subset$
\footnote{Why guards like $(a,b), (a), (b)$ are more difficult and $(a,b),(b)$ are easier?
          Because if $(a,b)$ needs to blocked (but not $(a)$ or $(b)$),
          then a system run may be such that forces processes to flit $a \leftrightarrow b$ 
          when some other process goes from $\init$ to $(a,b)$.
          And if the process template has only $a \rightarrow b$, 
          then the process that goes $\init \rightarrow (a,b)$ can throw the process in $(a,b)$ 
          out of the guard 
          (first, move it $a \rightarrow b$ with transition guarded $\forall \neg a$, 
           second, move it out of $b$ with $\forall \neg b$).
          In contrast, in embedded guard systems when the process moves $\init \rightarrow (a,b)$,
          it is safe to keep the blocking process in $a$ until some moment, 
          when we can move it into $b$ and then finally out of $(a,b)$.}
Also, for now, disallow more than one level of embedding: 
if $g_1 \subset g_2$, then no other $g_i \subset g_1$.

The construction is based on three notes:
\li
\-[1.] invariant state: every cycle starts and ends with the some special state of $B$ processe, 
       called invariant state
\-[2.] abstraction of transitions inside the guard:
       instead of literal repeating how processes move inside the guard,
       we repeat only ``important'' transitions
\-[3.] loops repeat: we might need two loops from the original run of the large system
       in order to complete one loop in the cutoff system
\-[4.] there is a special case of inside-the-guard transitions
\il

We copy local runs of deadlocked processes as usual.\ak{TODO}
For the deadlocked processes to be deadlocked, 
we need to block the guards $BlockGuards' = \{ g_1, ..., g_{k'} \}$.
Some of those guards are already blocked by the deadlocked processes, 
so let $BlockGuards=\{g_1, ..., g_k\} \subseteq BlockGuards'$ denote the guards that are blocked
by infinitely moving processes of the large system.
Thus, our task is to construct a run of \cutoffsys such that:
always there is a process in some state of every $g_1,...,g_k$,
\emph{and} such processes move infinitely often.

\myparagraph{1) Invariant state}

What are the loop transitions? 
Take an inf-moving process in the large system.
It visits \init inf-often, 
and it visits some of guards $\subseteq BlockGuards$ inf-often\footnotemark[10].
Thus, for every guard $g \in BlockGuards$ there is a process of the large system
that moves $\init \rightsquigarrow g \rightsquigarrow \init$ inf-often.

There may be loops of several types. 
Consider the loop type in the figure.
In this loop, 
a process enters guard $\{a,b\}$ via state $a$ and then singleton-guard $\{b\}$.
Call this process $B_m$ (``moving'' process), 
and let $B_m'$ be the process of the cutoff system that currently copies its transitions.
When $B_m$ (and $B_m'$) is outside of $\{a,b\}$ while moving along the loop towards the guard,
in the large system there is some process that blocks the guard.
In the cutoff system, suppose we have a process $B_{a,b}'$ in state $a$ 
at the moment when $B_m$ (and thus $B_m'$) is in $\init$\footnotemark[666].

Process $B_m$ moves along the loop into the guard, 
and some transitions may be guarded $\forall\neg b$\footnotemark[20]
(that is why we cannot place $B_{a,b}'$ into $b$).
What we would like is to move $B_m'$ into $\{a,b\}$ by copying transitions of $B_m$,
then move $B_{a,b}'$ out of $\{a,b\}$ using transitions of $B_m$,
while leaving $B_m'$ in ${a,b}$.

Note 1: if process $B_{a,b}'$ stays in $a$ until process $B_m'$ (and $B_m$) reaches $a$,
then the transitions of $B_m'$ on path $\init \rightsquigarrow a$ are enabled.
However, keeping $B_{a,b}$ in $a$ may result in disabling the subsequent transition
of $B_m$ from $a$ into $b$
(imagine $B_m$ does $\transition{a}{b}{\forall\neg \{a,b\}}$ ---
 $B_{a,b}'$ of the cutoff system cannot repeat this).

Since $B_m$ reaches $a$ (and has a transition guarded $\forall \neg b$), 
there is the last moment in the path $\init \rightsquigarrow a$ 
when $\forall \neg b$ needs to be enabled\footnotemark[200].
After that moment, we can move $B_{a,b}'$ from $a$ into $b$.
But this transition should be enabled.
Now two cases are possible:
\li
\- the transition $a \rightarrow b$ becomes enabled 
   while $B_m$ moves $\init \rightsquigarrow a$.
   Hence, we move process $B_{a,b}'$ when this happens for the last time.
   \footnotemark[333]

\- the transition $a \rightarrow b$ became enabled 
   before $B_m$ started $\init \rightsquigarrow a$.
\il

\hl{the following scenario is problemic: 
    imagine in the large system:
    $B_{a,b}$ stays in $a$, until $B_m$ reaches $a$, 
    then $B_{a,b}$ goes outside for a moment -- $B_m$ goes into $b$ -- $B_{a,b}$ returns into $a$, 
    and finally $B_m$ leaves.
    Later, $B_{a,b}$ will move to make it strong-fair. 
    Here is the picture ($a \rightarrow b$ is guarded $\forall \neg \{a,b\}$):}
    \includegraphics[width=2.5cm]{img/embedded-guards-problem.png}











\footnotetext[10]{If no such processes exists, copy one such process, and we are done.}
\footnotetext[20]{If no such transitions exists, we put $B_{a,b}$ into $b$, 
                  and the construction is similar to 1-conj guards looping, 
                  because the processes can pass the baton of copying $B_m$ when they meet in $b$.
                  Almost: what if there is $\transition{}{}{\forall\neg \{a,b\}}$?\ak{todo}}
\footnotetext[200]{\ak{account for the case when it was enabled before we started from \init}}
\footnotetext[333]{\ak{Here we use the assumption that process $B_{a,b}$ 
                       (of the large system) moved $a \rightarrow b$.}}
\footnotetext[666]{Careful here -- account for different schedulings and possibilities.}























\end{proof}

\fi

\iffinal
\else
\section{2-Conjunctive Proof Idea}

\paragraph{Given:} a (initializing) system with only 1-conjunctive guards, except a single 2-conjunctive guard.

\paragraph{Goal:} find cutoff for local deadlock detection.

\paragraph{Idea}:
Assume in a system \largesys there is a locally deadlocked run $x$. That is, at some time $d$ there is a set of processes with state/input combinations $\dead \subseteq Q \times \Sigma$ such that $\forall (q,e) \in \dead$ and $\forall \time \geq d$, no transition from $(q,e)$ is enabled in $x_\time$. Let $\visited_\bot$ be the set of state components of $\dead$, i.e., $\visited_\bot = \{ q \in Q \mid \exists e \in \Sigma: (q,e) \in \dead \}$. 

As in the 1-conjunctive case, this local deadlock may depend on
\begin{enumerate}[label=\alph*)]
\item at least two processes staying in each of the states in $\visited^2_\bot \subset \visited_\bot$, and
\item an additional process staying in each of the states of an additional set $BlockStates$.
\end{enumerate}

Since we want to simulate a strong-fair run $x$, we need to ensure that all processes that are not in $\dead$ at moment $d$ will move infinitely often in our simulation. In particular, we assume that the deadlock depends on the 2-conjunctive guard of the system, and none of the deadlocked processes is in either of the states $q_1$ or $q_2$ that are excluded by this guard (otherwise, the 2-conjunctive guard is not important for the local deadlock and we can fall back to the 1-conjunctive proof).

To allow infinite movement, note that for every $q \in Blockstates$, there must be a loop from $q$ to $\initstate$ and back to $q$ (written $\myloop{q}$) in the original run $x$. While moving along this loop is easy in the pure 1-conjunctive case if we remove all processes not in $\visited_\bot \cup BlockStates$, it is not so easy if we have a 2-conjunctive guard: transitions on the loop $\myloop{q}$ may depend on guards that exclude either $q_1$ or $q_2$, the two states of the 2-conjunctive guard.

First, consider the problem of a process $p$ that occupies (wlog) the state $q_1$ of the 2-conjunctive guard at moment $d$, and has to make a loop $\myloop{q_1}$. The transition out of the loop may depend on a guard that forbids $q_1$, so another process needs to move into $q_2$ before $p$ can move out of $q_1$. The next transition along the loop may depend on a guard that forbids $q_2$, so another process needs to move into $q_1$, and the first additional process then needs to move out of $q_2$, etc.

\paragraph{Question:} How many processes are necessary to simulate just the loop $\myloop{q_1}$?
\fi


