\section{Cutoffs for Disjunctive Systems}
\label{sec:app-disj}

\subsection{Disjunctive Systems without Fairness}

\begin{restatable}[Monotonicity: Disj, Properties, Unfair]{lem}{DisjMonoLemma}
\label{disj:le:NonFairDisjunctiveMono}
    For disjunctive systems:
    \begin{align*}
    &\forall n \geq 1:\\
    &(A,B)^{(1,n)} \models \pexists h(A,B_1)
    \ \Impl \
    (A,B)^{(1,n+1)} \models \pexists h(A,B_1).
    \end{align*}
\end{restatable}
\begin{proof}
Given run $x$ of $(A,B)^{(1,n)}$ we construct a run $y$ of $(A,B)^{(1,n+1)}$: 
copy $x$ into $y$ and keep the additional process in the initial state.
\end{proof}

\begin{restatable}[Bounding: Disj, Properties, Unfair]{lem}{DisjBoundingLemma}
\label{disj:le:NonFairDisjunctiveBounding}
    For disjunctive systems:
    \begin{align*}
    \forall n \geq |B|+2:\ 
    (A,B)^{(1,|B|+2)} \models \pexists h(A,B_1)
    \ \ \Implied \ \ 
    \largesys \models \pexists h(A,B_1).
    \end{align*}
\end{restatable}
\noindent 
The proof is from \cite[Lemma 4.1.2]{Emerson00}. 
We recapitulate it to introduce the notion of ``a process floods a state'', 
\destutter, \interleave, and ``process mimics another process'' 
which are used in our proofs later.
\ak{we could remove the proof and define these notions outside as i did for other constructions}

\begin{proof}
Let $c = |B|+2$ and $n \geq c$. Let $x=(s_1,e_1,p_1), (s_2,e_2,p_2) \ldots$ be a run of $\largesys$ that satisfies $\pexists h(A,B_1)$. We construct a run $y$ of the cutoff system $\cutoffsys$ with $y(A, B_1) \simeq x(A, B_1)$.

Let $\visited(x)$ be the set of all visited states by B-processes in run $x$: $\visited(x) = \{ q \| \exists m \exists i: s_m(B_i) = q \}$. 

Construct the run $y$ of \cutoffsys as follows:
\li
  \-[a.] copy runs of $A$ and $B_1$ from $x$ to $y$: $y(A)=x(A)$, $y(B_1)=x(B_1)$
  \-[b.] $x$ is infinite, hence it has at least one infinitely moving process, denoted $B_\infty$. Devote one unique process $B_\infty$ in \cutoffsys that copies the behaviour of $B_\infty$ of \largesys: $y(B_\infty)=x(B_\infty)$.
  \-[c.] for every $q \in \visited$ there is a process of \largesys, denoted $B_i$, that visits $q$ first, at moment denoted $m_q$. Then devote one unique process in \cutoffsys, denoted $B_{i_q}$, that \emph{floods $q$}: set $y(B_{i_q}) = x(B_i)\slice{1}{m_q}(q)^\omega$. In words: the run $y(B_{i_q})$ repeats exactly that of $x(B_i)$ till moment $m_q$, after which the process is never scheduled.
  \-[d.] let any other process $B_i$ of \cutoffsys not used in the previous steps (if any) \emph{mimic} the behavior of $B_1$ of \cutoffsys: $y(B_i) = y(B_1)$.
\il
The figure illustrates the construction.\ak{\init should be flooded}
\begin{figure}
\centering
\scalebox{0.7}{
\input{img/disj_flooding_construction}
}
\end{figure}
The correctness follows from the observation that any transition of any process at any moment $m$ of $y$ was done by some process in $x$ at moment $m$ and hence is enabled. Also note that if $\geq 2$ processes transit simultaneously in $y$, then the guards of their transitions will be enabled even if both of them are removed from the state space\ak{vague}. Note that it is possible that in $y$:
\li
  \- more than one process transits at the same moment. Then, \emph{\interleave} the transitions of such processes, namely arbitrarily sequentialize them. \ak{why are enabled}
  \- at some moment no processes move. Then remove elements of the run $y$ -- the resulting run is denoted $\destutter(y)$.
\il
This construction uses $|\visited| + 2 \leq |B|+2$ copies of B (ignoring case (d)).
\end{proof} 
%\gray{\ak{previous version is commented out}
% \begin{proof}
The following proof is similar to the proof for properties of \emph{closed} disjunctive systems by Emerson and Kahlon~\cite[Lemma 4.1.2]{Emerson00}. We separate the proof into two main constructions that will be re-used in later proofs.

Let $c = \card{B}+2$ and $n \geq c$. In the following, we will call $\largesys$ the \emph{large system}, and $\cutoffsys$ the \emph{cutoff system}.

\medskip
\noindent
Direction $\implies$:\\
Let $x=(s_1,e_1,p_1), (s_2,e_2,p_2) \ldots$ be a run \sj{should runs start with $0$ or $1$?} of $\largesys$ that satisfies $\pexists h(A^1,B^1)$.
In the following, we construct a run $y$ of the cutoff system $\cutoffsys$ such that $y(A^1, B^1)\simeq x(A^1, B^1)$.
Informally, the two steps are the following: 
\begin{enumerate}
  \item \emph{Flooding Construction}:\sj{TODO: separate flooding from other constructions}
  given a run $x$ of $\largesys$, we construct local runs $\yflood(A^1)$, $\yflood(B^1)$, $\yflood(B^2)$, \ldots, $\yflood(B^c)$ for the processes of the cutoff system. The construction ensures that
	\begin{enumerate}
	\item $\yflood(A^1)=x(A^1)$ and $\yflood(B^1)=x(B^1)$, and 
	\item when combined into a sequence of pseudo-configurations of $\cutoffsys$, the resulting sequence $\yflood$ is a pseudo-run that is not deadlocked, i.e., it evolves according to enabled transitions of the processes and at least one process moves infinitely often. 
	\end{enumerate}
	To ensure the latter, we devote to every state $q \in T_B$ that is reached in run $x$ one local run of template $B$ that moves into $q$ as fast as possible, and then stutters there forever.
  
  \item \emph{Interleaving Construction}:
The interleaving construction converts the pseudo-run $\yflood$ into the desired run $y$ by restoring an interleaving semantics. To this end, it may add local stuttering of one or several processes to ensure that at most one process moves at any time, and it may remove global stuttering to ensure that at least one process moves. The resulting run $y$ satisfies $y(A^1, B^1)\simeq x(A^1, B^1)$ and is not deadlocked.
\end{enumerate}

In the following, we formalize these two constructions. 

\subsubsection*{Flooding Construction.}
Set $\yflood(A^1)=x(A^1)$, $\yflood(B^1)=x(B^1)$.
Since $x$ satisfies the property, it must in particular be infinite, and hence there is a process of $\largesys$ that moves infinitely often in $x$.
If this process is a copy of $B$, say $B^\infty$, then set $\yflood(B^c) = x(B^\infty)$. Otherwise it is process $A^1$, and we set $\yflood(B^c)=\yflood(B^1)$.
To make sure that all local transitions in $\yflood(A^1)$, $\yflood(B^1)$ and $\yflood(B^c)$ are enabled, we construct additional local runs $\yflood(B^i)$ for $i\in [2,c-1]$ in the following way.

Let $\visited \subseteq T_B$ be the set of all local states of $B$ that are visited by at least one process in the run $x=(s_1,e_1,p_1) (s_2,e_2,p_2) \ldots$ of $\largesys$, i.e.:
$$\visited = \{ q \in T_B \| \exists{\time \in \bbN}\ \exists{i \in [1,n]}: s_\time(B^i) = q \}.$$

For $q \in \visited$ let:
\begin{itemize}
\item $\first_q = min(\{ \time \| \exists{i}: s_\time(B^i) = q\})$ be the first point in time where state $q$ is visited, and 

\item $\witfirst_q = any(\{ i \| s_{\first_q}(B^i)=q \})$ the index of a process that reached $q$ first (a ``witness'' process for $q$).

\end{itemize} 

Let $pr$ be an arbitrary bijection function from $\visited$ to $[2,|\visited|+1]$.
We use $pr$ to map a state $q \in \visited$ to a process $B^{pr(q)}$ of $\cutoffsys$.

For every $q \in \visited$, set
$$ \yflood(B^{pr(q)}) = \left(x(B^{\witfirst_q})\slice{1}{\first_q}\right)(q,e)^\omega,
$$
where $e$ is the local input of process $B^{pr(q)}$ at moment $\first_q$.
In words, to every state of $\visited$ we devote one process in $\cutoffsys$ that reaches it (on the shortest local path known from $x$) and then stays there.

For every $p \in [|\visited|+2,c-1]$ (this set is empty if $\visited=T_B$), set 
$$ \yflood(B^p) = \yflood(B^1),$$
i.e., all processes that do not `flood' states from $\visited$ repeat transitions of the process $B^1$.

\begin{figure}[h]
% \setlength{\abovecaptionskip}{-7pt plus 3pt minus 2pt} % Chosen fairly arbitrarily 
\centering
\scalebox{0.7}{
\input{img/disj_flooding_construction}
}
\caption{Illustration of the flooding construction from Lemma \ref{disj:le:NonFairDisjunctiveBounding}.\ak{Add flooding of the init state}}
\label{disj:fig:FloodingConstruction}
% \vspace{-5pt}
\end{figure}

Thus, we have defined all local runs $\yflood(A^1), \yflood(B^1), \yflood(B^2), \ldots$ in $\cutoffsys$. We obtain a sequence of pseudo-configurations $\yflood=(\sflood_1,\eflood_1,\Pflood_1)\ldots$ in the obvious way by defining the $\sflood_\time$ and $\eflood_\time$ as tuples of the corresponding local states, and letting the $\Pflood_\time$ contain all processes of $\cutoffsys$ that transit between state $\sflood_\time$ and $\sflood_{\time+1}$.\sj{$\leftarrow$ have this outside, in a definition?}
Any $\Pflood_\time$ may contain multiple processes that take a transition at the same time, e.g., if $\witfirst_q = i = \witfirst_{q'}$ for $q\neq q'$, then a prefix of $x(B^i)$ is contained in both $\yflood(B^{pr(q)})$ and in $\yflood(B^{pr(q')})$.
There might be also elements with $\Pflood_\time=\emptyset$. This happens if $p_\time \not\in \{ B^i \| \exists{q} \in \visited:\ i = \witfirst_q \} \cup \{B^1,B^\infty\}$, i.e., $p_i$ is a process that is not used to construct a local run of $\cutoffsys$. Figure~\ref{disj:fig:FloodingConstruction} illustrates the flooding construction.\ak{illustrate issues mentioned below?}

We show that the resulting sequence $\yflood = (\sflood_1,\eflood_1,\Pflood_1)\ldots$ is a pseudo-run of $\cutoffsys$. To this end, observe that the flooding construction ensures the following properties:
\begin{enumerate}
\item [P1.]
$\yflood(A^1,B^1) \simeq x(A^1,B^1).$

\item [P2.]
$\forall{q \in \visited}\ \forall{\time \ge \first_q}:\ \sflood_\time(B^{pr(q)}) = q.$

\item [P3.]
At any point of $\yflood$, if some process of $\cutoffsys$ transits $\transition{q}{q'}{e:g}$, then at the same point of the run $x$ of $\largesys$ there is a process that makes exactly the same transition $\transition{q}{q'}{e:g}$.
% $\forall{i} \ \forall{(p^*\!:\!\transition{q}{q'}{e:g})}:$
% $$(p^*\!:\!\transition{q}{q'}{e:g}) \in (s^*_i \rightarrow s^*_{i+1})
% ~~\impl~~
% \exists{p}\ (p\!:\!\transition{q}{q'}{e:g}) \in  (s_i \rightarrow s_{i+1}),$$
% where $p^*$ is a process of $\cutoffsys$ and $p$ is a process of $\largesys$.
\item [P4.]
At any point of $\yflood$, if more than one process transits, then they make exactly the same transition.
\end{enumerate}

We need to show that every element $(\sflood_{\time+1},\eflood_{\time+1},\Pflood_{\time+1})$ is derived from element $(\sflood_\time,\eflood_\time,\Pflood_\time)$ by local transitions of all processes in $\Pflood_\time$. 
Suppose that $p \in \Pflood_\time$, and the local transition from $q=\sflood_\time(p)$ to $q'=\sflood_{\time+1}(p)$ (with local input $\eflood_\time(p)$) is guarded by some guard $g \subseteq T_A \cup T_B$. Then by property P3 some process of $\largesys$ makes exactly the same transition, and hence in $s_\time$ of $x$ there is a process in some state $r \in g$. Consider two cases:
\begin{itemize}
  \item if $r \in T_A$, then by property P1, we have $\sflood_\time(A^1)=r$, and thus the transition is enabled.
  \item if $r \in T_B$, then $\time \ge \first_r$.
  Note that $p \neq B^{pr(r)}$ (since process $B^{pr(r)}$ does not move after reaching $r$ at time $\first_r$), and hence the transition is enabled.
\end{itemize}
 
\subsubsection*{Interleaving Construction.}
To obtain a run $y$ with $y(A^1, B^1)\simeq x(A^1, B^1)$ from $\yflood$, we first\rb{what is second?} sequentialize simultaneous transitions of processes.

Let $\interleave$ be the function that takes as input a pseudo-run $(\sflood_1,\eflood_1,\Pflood_1)$, $(\sflood_2,\eflood_2,\Pflood_2)$\ldots from the flooding construction and returns the pseudo-configuration sequence $\pi_1 \cdot \pi_2 \cdot \ldots$, where the $\pi_\time$ are non-empty sequences constructed from $(\sflood_\time, \eflood_\time, \Pflood_\time)$ in the following way:
\begin{itemize}
  \item if $|\Pflood_\time|\le 1$, then $\pi_\time = (\sflood_\time, \eflood_\time, \Pflood_\time)$,
 
  \item otherwise, $\Pflood_\time$ contains several processes that move simultaneously, say $p_1,\ldots,p_n$. By property P4, we know that in this case $p_\time \in \{ B^1,\ldots,B^c\}$ for all $\time$.
	%Then:
%  $\transition{s_\time(p_1)}{s_{\time+1}(p_1)}{e_\time(p_1)}$,
%  $\ldots$, 
%  $\transition{s_\time(p_n)}{s_{\time+1}(p_n)}{e_\time(p_n)}$.
  Then $\pi_\time = (s_{\time_1},e_{\time_1},P_{\time_1}) \ldots (s_{\time_n}, e_{\time_n}, P_{\time_n})$, where:
  \begin{itemize}
  \item 
  $(s_{\time_1}, e_{\time_1}, P_{\time_1}) = (\sflood_\time, \eflood_\time, \{ p_1 \})$, 
  \item 
  $s_{\time_2} = s_{\time_1}[s_{\time_1}(p_1) \gets \sflood_{\time+1}(p_1)]$, \\
  $e_{\time_2} = e_{\time_1}[e_{\time_1}(p_1) \gets \eflood_{\time+1}(p_1)]$, \\
  $P_{\time_2} = \{ {p_2} \}$;

  \item 
  \ldots
  \item 
  $s_{\time_n} = s_{\time_{n-1}}[s_{\time_{n-1}}(p_n) \gets \sflood_{\time+1}(p_n)]$, \\
  $e_{\time_n} = e_{\time_{n-1}}[e_{\time_{n-1}}(p_n) \gets \eflood_{\time+1}(p_n)]$, \\
  $P_{\time_n} = \{ {p_n} \}$.
  \end{itemize}
  Note that $\sflood_{\time+1} \equiv s_{\time_n}[s_{\time_n}(p_n) \gets \sflood_{\time+1}(p_n)]$.
\end{itemize}

It is straightforward to show that $\interleave(\yflood)$ is a pseudo-run of $\cutoffsys$ where at each moment either one or no process moves, and $y(A^1,B^1) \simeq x(A^1,B^1)$.
Thus, let $y = \destutter(\interleave(\yflood))$ to obtain the desired run\footnote{Note that $\destutter(x)$ is finite for a run $x$ that ends in an infinite repetition of the same configuration $(s,e,p)$. We abuse the notation and write $\destutter(x)$ for $\destutter(x)\ (s,e,p)^\omega$ in this case.}
of $\cutoffsys$.
This concludes the proof of $\implies$ direction.

\bigskip
\noindent Direction $\impliedby$\sj{if we keep Monotonicity Lemma, this is just repeated application}\ak{but then you need to prove Monotonicity Lemma?} (proof idea):\\
Construct the run $y$ of the large system $\largesys$ from a given run $x$ of the cutoff system $\cutoffsys$ as follows: $y(A^1)=x(A^1)$, $y(B^i)=x(B^i)$ for $i\in [1,c]$, and $y(B^{c+i}) = x(B^c)$ for $i \in [1,n-c]$. 
Use the interleaving construction to restore the interleaving semantics of the resulting pseudo-run to obtain $y$.
% 
\end{proof}
%}

\begin{restatable}[Disj, Props, Unfair]{tightness}{TightDisjBoundingLemma}
\label{obs:disj:tight_prop}
    The cutoff in Lemma~\ref{disj:le:NonFairDisjunctiveBounding} is tight, 
    i.e., for any $k$ there exist process templates $(A,B)$ with $|B| = k$ 
    and $\LTLmX$ formula $h(A,B_1)$ such that:
    $$
    (A,B)^{(1,|B|+2)} \models \pexists h(A,B_1) ~~and~~ 
    (A,B)^{(1,|B|+1)} \not\models \pexists h(A,B_1).
    $$
\end{restatable}
\begin{proof}
The idea of the proof relies on the subtleties of the definition of a run: it is infinite (thus not globally deadlocked), and in each step of a run exactly one process moves. 

Consider the templates in the figure below and let $\pexists h(A,B_1) = \pexists (\eventually 3_{B_1} \land \eventually\always (2_{B_1} \land {end}_A))$. In words: there exists a run in a system where process $B_1$ visits $3_B$ and process $B_1$ with $A$ eventually always stay in $2_B$ and ${end}_A$.
\begin{figure}[h]
% \vspace{-20pt}
\centering
\subfloat[Template A]{
\centering
\makebox[0.4\textwidth][c]{
\scalebox{0.75}{\input{img/disj_tight_propAB_tmplA}}
\label{fig:disj:tight_propAB_tmplA}
}}
\subfloat[Template B]{
\centering
\makebox[0.6\textwidth][c]{
\scalebox{0.75}{\input{img/disj_tight_propAB_tmplB}}
\label{fig:disj:tight_propAB_tmplB}
}}
% \caption{Templates $(A,B)$ used to prove the tightness of the cutoffs for properties $\pexists h(A,B_1)$ (Observation~\ref{obs:disj:tight_prop})\ak{check me}.}
\label{fig:disj:tight_propAB_tmpl}
\end{figure}
% \vspace{-10pt}

We need one process in every state of $B$ to enable the transitions of $A$ to ${all}_A$. Only when $A$ in ${all}_A$, $B_1$ can move $3_B \to 1_B$, and then at some point to $2_B$. After $B_1$ moves $3_B \to 1_B$, $A$ moves ${all}_A \to {end}_A$ which requires process $B_{i \neq 1}$ in $3_B$. Finally, to make the run infinite there should be at least two processes in ${|B|}_B$.\sj{other cases are covered in general lemma below}
% 
\end{proof}


\ifwithextensions
%\subsubsection{Generalized Bounding Lemma.}

Now, consider properties $h(A,B^{(n)})$ that may talk about ($A$ and) $n$ different copies of process template $B$. 

\begin{restatable}[Generalized Bounding Lemma]{lem}{DisjBoundingLemmaGeneral}
\label{disj:le:NonFairDisjunctiveBoundingGeneral}
For disjunctive systems without fairness:
\begin{align*}
&(A,B)^{(1,\geq \card{B} + k + 1)} \models \pexists h(A,B^{(k)}) &
&~\iff~& &
(A,B)^{(1,\card{B} + k + 1)} \models \pexists h(A,B^{(k)}).%,\\
\end{align*}
\end{restatable}
\begin{proof}
An inspection of the proof of Lemma~\ref{disj:le:NonFairDisjunctiveBounding} shows that it works almost without modification if we replicate the local run of $A^1$ along with $n$ local runs of copies of $B$, instead of only $A^1, B^1$. In particular, the rest of the flooding construction and the interleaving construction do not have to be changed.

Additionally, one can observe that the cutoff is independent of whether the property talks about $A^1$ or not.
\end{proof}

\begin{restatable}{obs}{TightDisjBoundingLemmaGeneral}
\label{obs:disj:tight_prop:general}
The cutoffs in Lemma~\ref{disj:le:NonFairDisjunctiveBoundingGeneral} are tight, i.e.,
for any $k$ there exist $(A,B)$ with $\card{B}=k$ and $\LTLmX$ formula $h(A,B^{(n)})$ such that: \sj{do we have example for generalized form?}\ak{agree, proof is needed}
%
$$
(A,B)^{(1,k+n+1))} \models h(A,B^{(n)}) ~~and~~ 
(A,B)^{(1,k+n)} \not\models h(A,B^{(n)}).
$$
\end{restatable}
\begin{proof}
We can again use the templates from the previous observation, with a formula $h(A^1,B^{(n)}) = \pexists \bigwedge_{i \in [1..n]} (\eventually b^i_3 \land \eventually\always (b^i_2 \land a^1_{end}))$.\footnote{The case of $n=0$ can be shown with a simpler templates: $A$ is the template with only one state $a_{end}$ without successors, and $B$ is the chain that ends in $b_k$ with the self guarded loop.}
\end{proof}
\fi  %\ifwithextensions

\begin{restatable}[Monotonicity: Disj, Deadlocks, Unfair]{lem}{mono_lem_disj_deadlocks_unfair}
\label{mono_lem_disj_deadlocks_unfair}
    For disjunctive systems:
    $$\forall n\geq |B|+1: (A,B)^{(1,n)} \textit{ has a deadlock} \ 
    \Impl\ 
    (A,B)^{(1,n+1)} \textit{ has a deadlock}$$
\end{restatable}
\begin{proof}
Given a deadlocked run $x$ of $(A,B)^{(1,n)}$ 
we build a deadlocked run of $(A,B)^{(1,n+1)}$. 
If the run $x$ is locally deadlocked,
then it has at least one infinitely moving process, 
thus let the additional process mimic that process. 
If the run $x$ is globally deadlocked run, 
then due to $n>|B|$ in some state there are at least two processes deadlocked. 
Thus, let the new process mimic a process deadlocked in that state -- 
the run constructed will also be globally deadlocked.
\end{proof}

\sj{should be tight: If we only have $\card{B}$, here is the counterexample: pipeline where transition $(q_i,q_{i+1})$ is guarded with $\exists q_i$, and $(q_{n-1},q_0)$ guarded with $\exists q_{n-1}$. This system has deadlock with up to $n$ processes, but not with $n+1$.}


\begin{restatable}[Bounding: Disj, Deadlocks, Unfair]{lem}{lem_disj_deadlocks_unfair}
\label{lem_disj_deadlocks_unfair}
For disjunctive systems:
\li
  \- with $c=|B|+2$ and any $n>c$:
  $$(A,B)^{(1,c)} \textit{ has a local deadlock} \ \Implied\ (A,B)^{(1,n)} \textit{ has a local deadlock}$$
  
  \- with $c=2|B| - 1$ and any $n>c$
  $$(A,B)^{(1,c)} \textit{ has a global deadlock} \ \Implied\ (A,B)^{(1,n)} \textit{ has a global deadlock} $$
  
  \- with $c=2|B|-1$ and any $n>c$:
  $$(A,B)^{(1,c)} \textit{ has a deadlock} \ \Implied\ (A,B)^{(1,n)} \textit{ has a deadlock}$$
\il
\ak{seems not tight}
\end{restatable}
%\noindent Let us define the notion of ``process floods state and evacuates'' used in the proof.
%
%Given $P_1 \subseteq \{B_1,\ldots,B_n\}$, an infinite run $x=(s_1,e_1,p_1)\ldots$ of \largesys, let \emph{$ \visited_{\fin-P_1}(x)$ wrt. $P_1$}, and \emph{$ \visited_{\inf-P_1}(x) $ wrt. $P_1$} be:
%\begin{align}
%& \visited_{\inf-{P_1}}(x) = \{ q \|\! \exists \text{ infinitely many } ~~~~ m\!:  
%s_m(B_p)\!=\!q 
%\text{ for some } B_p\!\not\in\! P_1 \} \label{disj:def_vinf_wrt} \\
%& \visited_{\fin-{P_1}}(x) = \{ q \|\! \exists \text{ only finitely many } m\!:  
%s_m(B_p)\!=\!q 
%\text{ for some } B_p\!\not\in\!P_1 \} \label{disj:def_vfin_wrt}
%\end{align}
%
%Let $q \in \visited_{\fin-P_1}(x) $. Note that in run $x$ there is a moment $f_q$ when $q$ is reached first by some B-process denoted $B_{\first_q}$. And in run $x$ there is a moment $l_q$ when $q$ is left last by some B-process denoted $B_{\last_q}$. Let $B_i$ be a process of another system \cutoffsys. Let $y$ be a run of \cutoffsys that we plan to construct from the run $x$. Then, \emph{process $B_i$ of \cutoffsys floods $q$ and then evacuates into $\visited_{\inf-B_1}(x)$} means: 
%$$y(B_i) = x(B_{\first_q})\slice{1}{f_q} \ \cdot\ (q)^{(l_q - f_q + 1)} \cdot \ x(B_{\last_q})\slice{l_q}{m} \ \cdot \ (q')^\omega,$$
%where $q'$ is a state in $\visited_{\inf-P_1}(x)$ that $B_{\last_q}$ reaches first at some moment $m \geq l_q$. In words, process mimics process $B_{\first_q}$ until it reaches $q$, then does nothing until process $B_{\last_q}$ starts leaving, then mimics $B_{\last_q}$ until it reaches $ \visited_{\inf-P_1}(x) $.
\begin{proof}
Given a (globally or locally) deadlocked run of $\largesys$ 
we construct (globally or locally) deadlocked run of $\cutoffsys$, 
where $c$ depends on the nature of the given run. 
We do this using the construction template. 

Let $\mB=\{B_1,...,B_n\}$.
The template depends on set $\mC \subseteq \{B_1,...,B_c\}$:
\li
  \-[a.] set $y(A)=x(A)$
  \-[b.] for every $B_i \in \mC$, set $y(B_i)=x(B_i)$
  \-[c.] for every $q \in \visInf{\mB\smi\mC}{x}$, 
         devote one process of \cutoffsys that floods $q$
  \-[d.] for every $q \in \visFin{\mB\smi\mC}{x}$, 
         devote one process of \cutoffsys that floods $q$ 
         and then evacuates into $\visInf{\mB\smi\mC}{x}$
  \-[e.] let other processes (if any) mimic some process from (c)
\il

\myparagraph{1) Local Deadlock}
We distinguish three cases: 
\li
  \-[1a)] $A$ deadlocks, $B_1$ moves infinitely often
  \-[1b)] $A$ moves infinitely often, $B_1$ deadlocks
  \-[1c)] $A$ neither deadlocks nor moves infinitely often, 
          $B_1$ deadlocks, $B_2$ moves infinitely often.
\il

\myparagraphraw{1a:} ``$A$ deadlocks, $B_1$ moves infinitely often''. 

Let $c=|B|+1$, and $\mC=\{B_1\}$.
Note that $\visInf{B_2..B_n}{x} \neq \emptyset$. 
The resulting construction uses 
$|\visFin{B_2..B_n}{x}| + |\visInf{B_2..B_n}{x}| + 1 
 \leq 
 |B| + 1$ 
copies of B.
\ak{seems tight}\ak{correctness}

\myparagraphraw{1b:} ``$A$ moves infinitely often, $B_1$ deadlocks''. 

Let $c=|B|+1$, and $\mC=\{B_1\}$.
Let $q_\bot$ be the state in which $B_1$ deadlocks.
Instantiate the construction template.

Process $B_1$ of \cutoffsys is deadlocked in $y$ starting from some moment $d$,
because any state it sees (in $\visInf{A,B_2..B_n}{x}$)
was also seen by $B_1$ in \largesys in $x$ at some moment $d' \geq d$
(note that $d'$ may be not the same moment as $d$).
%\footnote{Note about open systems: here we use the fact from the definitions 
%          that inputs to $B_1$ do not change.
%          This ensures that the set of states that $B_1$ should not see in order
%          to stay deadlocked does not change over time.}


\myparagraphraw{1c:} ``$A$ neither deadlocks nor moves infinitely often, 
                       $B_1$ deadlocks, $B_2$ moves infinitely often''. 

Instantiate the construction template with $c=|B|+2$ and $\mC = \{B_1,B_2\}$.
\ak{seems not tight}\ak{correctness}

\smallskip
Finally, $|B|+2$ is a (possibly not tight) cutoff for local deadlock detection problem.


\myparagraph{2) Global Deadlock}
Let $x=(s_1,e_1,p_1)...(s_d,e_d,\bot)$ be a globally deadlocked run of $\largesys$ 
with $n\geq c$.

Let us abuse the definition of $\visInf{\mF}{x}$ and $\visFin{\mF}{x}$,
in Eq.~\ref{disj:def_vinf_wrt} and \ref{disj:def_vfin_wrt} resp., 
and adapt it to the case of finite runs.
To this end, given a finite run $x=(s_1,e_1,p_1)...(s_d,e_d,\bot)$, 
extend it to the infinite sequence $(s_1,e_1,p_1)...(s_d,e_d,\bot)^\omega$, 
and apply the definition of $\visInf{\mF}{x}$ and $\visFin{\mF}{x}$ to the sequence.

Let $\mD_1$ be the set of processes deadlocked in unique states:
$\forall p\in \mD_1 \not\exists p' \neq p: s_d(p')=s_d(p)$.
Instantiate the construction template with $\mC = \mD_1$ and $c=2|B|-1$.
\footnote{$2|B|-1$ copies is enough, because: 
          $\visFin{\mB\smi\mC}{x} \cap \visInf{\mB\smi\mC}{x} = \emptyset$,
          $\visInf{\mB\smi\mC}{x} \cap \visInf{\mC}{x} = \emptyset$,
          and if $\visFin{\mB\smi\mC}{x} \neq \emptyset$, 
          then $\visInf{\mB\smi\mC}{x} \neq \emptyset$.}
\ak{seems not tight}

%The construction uses $|dead1| + |dead2| + |\visited_{\fin-P_\bot^1}(x)| \leq 2|B|-1$ copies of B.\ak{seems not tight}\ak{CHECK}\ak{correctness}

\myparagraph{3) Deadlocks}
As the cutoff for the deadlock detection problem we take the largest cutoff in (1)-(2), namely, $2|B|-1$, but it may be not tight -- finding the tight cutoffs for local deadlock and for deadlock detection problems is an open problem.

\ak{tried to refine but could not -- the trial is commented out}
% AK: try to utilize the existence of the order?
% Now let us refine the estimate:
% \li
% \- `unique finite state' is any state in $\visited_{\fin+P_1_\bot}$ that either is not in $ \visited_{\fin-P_1_\bot} $, or the state that appears not in $]f_q,l_q[_{-P_1_\bot}$
% 
% \- intuitively, state is unique finite if we cannot make it `visible' by using the flooding and evacuating construction wrt. $ \visited_{\fin-P_1_\bot} $, $\visited_{\inf-P_1_\bot}$. Otherwise, if state is not unique finite then  we can flood the state and then evacuate into $ \visited_{\inf-P_1_\bot} $ 
% 
% \- let $U^c \subseteq P_1_\bot$ be the set of processes that have unique finite states on its run to $q_\bot \in \visited_\bot^1$. Let $U^f = P_1_\bot\smi U^c$.
% 
% \- note that for any $p \in U^f$ any its finite state can be flooded and then evacuated into $ \visited_{\inf-P_1_\bot} $ using local runs of $\mathcal{B}\smi P_1_\bot$ processes. Intuitively, this means that $U^f$ processes' finite states are not needed.
% 
% \- let us divide $\visited_\bot^1$ into two disjiont sets: $\visited_\bot^c$ (and corr. processes are $U^c$) is the set of states whose deadlocked paths do have a unique finite state, and $\visited_\bot^f$ (and corr. processes are $U^f$) is the set of states whose deadlocked paths do not have a unique finite state.
% \il
% 
% Define $ \visited_{\fin-U^c} $ and $ \visited_{\inf-U^c} $ wrt. $U^c$.
% Then the construction is:
% \li
%   \-[a.] $y(A)=x(A)$
%   \-[b.] copy the runs of $U^c$ processes (that end in $\visited_\bot^c$)
%   \-[c.] flood states in $ \visited_{\inf-U^c} $
%   \-[d.] for any $ q \in \visited_{\fin-U^c} $ devote one process that floods it and then evacuates into $ \visited_{\inf-U^c} \smi \visited_\bot^f$. Note that we can evacuate a state from $ \visited_{\fin-U^c} $ into $ \visited_{\inf-U^c} \smi \visited_\bot^f $, because processes $U^f$ do not contribute to the evacuation by definition.\ak{vague}
%   \-[e.] let other processes (if any) mimic a process from (c)
% \il
% The construction uses $|\visited_\bot^c| + | \visited_{\inf-U^c} | + | \visited_{\fin-U^c} |$. Note that:
% \li
%   \- $\visited_\bot^c$, $ \visited_{\inf-U^c} $ are disjoint
%   \- $\visited_{\inf-U^c}$, $\visited_{\fin-U^c}$ are disjoint
%   \- possibly $\visited_\bot^c \cap \visited_{\fin-U^c} \neq \emptyset$
% \il 
% If some $p \in U^c$, then it has at least one unique finite state $q$: possibly $q \in  \visited_{\inf-U^c} $ or $q \in \visited_{\fin-U^c}$, but $q \not\in \visited_\bot^c$. \ak{stuck here -- it is possible that all $U^c$ processes share a single unique finite state}

\sj{my idea for smaller cutoff in comments}

%What about this:
%separate states into:
%
%\noindent
%$\visited^2_\bot$        (deadlock-2 states)\\
%$\visited^1_{unique}$        (deadlock-1 states that do not appear in paths to $\visited^2_\bot$)\\
%$\visited^1_{both}$        (deadlock-1 states that do appear in paths to $\visited^2_\bot$)\\
%$\visited_\fin$        (non-deadlock states visited in *any* local path)\\
%
%Now, $\visited^1_{unique}, \visited^1_{both}, \visited^2_\bot$ and $\visited_\fin$ are disjoint (and contain all states appearing in $x$).
%Then:
%\begin{itemize}
%\item copy paths from $\visited^1_{unique}$\\
%\item flood states from $\visited^1_{both}$ and $\visited^2_\bot$\\
%\item flood and evacuate states from $\visited_\fin$
%\end{itemize}
%
%Why is it OK to flood states from $\visited^1_{both}$ instead of copying them?
%Because we know that from these states, we can evacuate to some state in $\visited^2_\bot$, so it cannot happen that states from $\visited_\fin$ will be stuck in $\visited^1_{both}$ when evacuating.
%
%Only problem I see: evacuation may depend on state of $A$.








\end{proof}

\begin{restatable}[Disj, Deadlocks, Unfair]{tightness}{TightDisjDeadlockLemma}
\label{obs:disj:tight_deadlock}
The cutoff $c=2|B|-1$ for deadlock detection in disjunctive systems is \emph{asymptotically optimal but possibly not tight}, i.e.: for any $k$ there are templates $(A,B)$ with $|B|=k$ such that:
$$
(A,B)^{(1,|B|-1)} \textit{ does not have a deadlock, but } (A,B)^{(1,|B|)} \textit { does}.
$$
\end{restatable}

\begin{proof}
The figure below illustrates templates $(A,B)$ to prove the asymptotical optimality of cutoff $2|B|-1$ for deadlock detection problem. Template $A$ is any that never deadlocks. The system has a local deadlock only when there are at least $|B|$ copies of $B$, which is a constant factor of $2|B|-1$.
\begin{figure}[h]
\centering
\makebox[0.4\textwidth][c]{
\scalebox{0.75}{\input{img/disj_tight_deadlock_tmpl}}
}
\end{figure}
% 
% The same figure above can be used to prove the asymptotical optimalitity of the cutoff $|B|+2$ for local deadlock detection problem -- in this case the template $A$ is any that never deadlocks. As before, the system locally deadlocks only when there is at least $|B|$ processes $B$.
\end{proof}


% Figure~\ref{fig:disj:tight_deadlockA_tmpl} provides templates for the case (iii).\ak{hm, good to have a formal proof for the complicated template}
% % 
% % 
% % 
% % 
% % 
% \begin{figure}
% \centering
% \subfloat[Template A]{
% \centering
% \makebox[0.55\textwidth][c]{
% \scalebox{0.75}{\input{img/disj_tight_deadlockA_tmplA}}
% \label{fig:disj:tight_deadlockA_tmplA}
% }}
% \subfloat[Template B]{
% \centering
% \makebox[0.43\textwidth][c]{
% \scalebox{0.75}{\input{img/disj_tight_deadlockA_tmplB}}
% \label{fig:disj:tight_deadlockA_tmplB}
% }}
% \caption{Templates $(A,B)$ used to prove the tightness of the cutoff $c=|B|+1$ for strictly local deadlock by $A$ process detection (Observation~\ref{obs:disj:tight_deadlock})\ak{wrong}.}
% \label{fig:disj:tight_deadlockA_tmpl}
% \end{figure}
% % 




\subsection{Disjunctive Systems with Fairness}

\begin{restatable}[Monotonicity: Disj, Props, Fair]{lem}{DisjBoundingLemmaFair}
\label{disj:le:FairDisjunctiveBounding}
For disjunctive systems:
\begin{align*}
& \forall n \geq 1: \\
&(A,B)^{(1, n)} \models \pexists_{uncond} h(A,B_1) 
\implies
(A,B)^{(1,n+1)} \models \pexists_{uncond} h(A,B_1),\\
\end{align*}
% 
\end{restatable}
\begin{proof}
In run $x$ of $(A,B)^{(1,n)}$ with $n \geq 1$ all processes move infinitely often. 
Hence let the run $y$ of $(A,B)^{(1,n+1)}$ copy $x$, 
and let the new process mimic an infinitely moving B process of $(A,B)^{(1,n)}$.
\end{proof}

\begin{restatable}[Bounding: Disj, Props, Fair]{lem}{DisjBoundingLemmaFair}
\label{disj:le:FairDisjunctiveBounding}
For disjunctive systems:
\begin{align*}
&\forall n>2|B|: \\
&(A,B)^{(1,2|B|)} \models \pexists_{uncond} h(A,B_1) &
&\impliedby& &
(A,B)^{(1,n)} \models \pexists_{uncond} h(A,B_1),\\
\end{align*}
% 
\end{restatable}
The proof was given in the main text, in Section~\ref{sec:ideas-disj-fair}.

%\myparagraph{Fair Extension}
%\ak{adapt to dead}
Here, we consider a path $x$ that is the postfix of an unconditionally fair run $x'$ of $\largesys$, 
starting from the moment where no local states from $\visFin{\mB}{x'}$ are visited anymore. 
We construct a corresponding unconditionally-fair path $y$ of $\cutoffsys$, 
where no local states from $\visFin{\mB}{x'}$ are visited.

Formally, let $n \geq 2|B|$, and $x$ an unconditionally-fair path of $\largesys$ such that
$\visFin{\mB}{x}=\emptyset$.
Let $c \geq 2|B|$, and $s_1'$ a state of \cutoffsys
with
\li
\- $s_1'(A_1)=s_1(A_1)$, $s_1'(B_1)=s_1(B_1)$

\- for every $q \in \visInf{B_2..B_n}{x} \smi \visInf{B_1}{x}$,
   there are two processes $B_{i_q}, B_{i_q'}$ of \cutoffsys
   that start in $q$, i.e., $s_1'(B_{i_q})=s_1'(B_{i_q'})=q$

\- for every $q \in \visInf{B_2..B_n}{x} \cap \visInf{B_1}{x}$,
   there is one process $B_{i_q}$ of \cutoffsys
   that starts in $q$

\- for some $\qstar \in \visInf{B_2..B_n}{x} \cap \visInf{B_1}{x}$,
   there is one additional process of \cutoffsys, 
   different from any in the above, 
   called $B_{i_\qstar'}$,
   that starts in $\qstar$.

\- any other process $B_i$ of \cutoffsys 
   starts in some state of $\visInf{B_2..B_n}{x}$.
\il
Note that if $\visInf{B_2..B_n}{x}\cap \visInf{B_1}{x} = \emptyset$, 
then the third and fourth pre-requisites are trivially satisfied.

The fair extension extends state $s_1'$ of \cutoffsys 
to an unconditionally-fair path $y=(s'_1,e'_1,p'_1)\ldots$ 
with $y(A_1,B_1) = x(A_1,B_1)$ as follows:
\li
\-[(a)] $y(A_1)=x(A_1)$, $y(B_1)=x(B_1)$

\-[(b)] for every $q \in \visInf{B_2..B_n}{x} \smi \visInf{B_1}{x}$: 
       in run $x$ there is $B_i \in \{B_2..B_n\}$ 
       that starts in $q$ and visits it infinitely often. 
       Let $B_{i_q}$ and $B_{i'_q}$ of \cutoffsys mimic $B_i$ in turns: 
       first $B_{i_q}$ mimics $B_i$ until it reaches $q$, 
       then $B_{i'_q}$ mimics $B_i$ until it reaches $q$, and so on.

\-[(c)] arrange states of $\visInf{B_2..B_n}{x}\cap \visInf{B_1}{x}$ 
       in some order $(\qstar, q_1, \ldots, q_l)$.  
       The processes $B_{i_\qstar'}, B_{i_\qstar}, B_{i_{q_1}}, \ldots, B_{i_{q_l}}$ 
       behave as follows.
       Start with $B_{i_\qstar'}$: 
       when $B_1$ enters $\qstar$ in $y$, it carries%
       \footnote{``Process $B_1$ starting at moment $m$ carries process $B_i$ 
                 from $q$ to $q'$'' means: process $B_i$ mimics 
                 the transitions of $B_1$ starting at moment $m$ at $q$ 
                 until $B_1$ first reaches $q'$.}
       $B_{i_\qstar'}$             from $\qstar$ to $q_1$, 
       then carries $B_{i_{q_1}}$ from $q_1$ to $q_2$, \ldots, 
       then carries $B_{i_{q_l}}$ from $q_l$ to $\qstar$, 
       then carries $B_{i_\qstar}$ from $\qstar$ to $q_1$, 
       then carries $B_{i_\qstar'}$ from $q_1$ to $q_2$, 
       then carries $B_{i_{q_1}}$ from $q_2$ to $q_3$,
       and so on.

%\-[c2.] otherwise, $\visited_{\inf\cap B_1}{x} = \{ q^\star \}$. Then $B_1$ only ever makes transitions $q^\star \to q^\star$, thus let process $B_{i_{q^\star}}$ mimic this.

\-[(d)] any other $B_i$ of \cutoffsys,
       starting in $q \in \visInf{B_2..B_n}{x}$,
       mimics $B_{i_q}$.
\il
Note that parts (b) and (c) of the constrution ensure that there is always at
       least one process in every state from $\visInf{B_2..B_n}{x}$. This
       ensures that the guards of all transitions of the construction are satisfied.
Excluding processes in (d), the fair extension uses up to $2|B|$ copies of $B$.%
\footnote{A careful reader may notice that if
          $|\visInf{B_1}{x}|=1$ and $|\visInf{B_2..B_n}{x}|=|B|$,
          then the construction uses $2|B|+1$ copies of $B$.
          But one can slightly modify the construction for this special case,
          and remove process $B_{i_\qstar'}$ from the pre-requisites.}
%\li
%\- if $\visInf{B_2..B_n}{x} \cap \visInf{B_1}{x} = \emptyset$, then 
%     $$\leq 1+2|\visInf{B_2..B_n}{x}| \leq 1+2(|B|-1) = 2|B|-1$$
%   (note that $\visInf{B_1}{x}$ contains at least one state)
%
%\- otherwise:\\
%   let $smi = \visInf{B_2..B_n}{x} \smi \visInf{B_1}{x}$, \\
%   let $inter = \visInf{B_2..B_n}{x} \cap \visInf{B_1}{x}$, \\
%   then
%   $$\leq 1+2|smi| + |inter| + 1 \leq 1+2(|B|-1) + 1 + 1 = 2|B|+1$$
%   $$\leq 1+2|smi| + |inter| + 1 \leq 1+2(|B|-2) + 2 + 1 = 2|B|+1$$
%\il
%
%
%%%%%%%%%%%%% OLD FAIR EXTENSION %%%%%%%%%%%%%%%%%%%%%%
%Let $x=(s_1,e_1,p_1)\ldots$ be an unconditionally-fair path $x$ of $\largesys$, let $\visited_\fin{x}$ and $\visited_\inf{x}$ be defined wrt. no processes, let $n\geq 2|\visited_\inf{x}|$, and let $x$ satisfy: 
%\li
%  \- every $B_i\neq B_1$ visits $s_1(B_i)$ infinitely often
%  \- $\visited_\fin{x} = \emptyset$,
%\il
%and let $s_1'$ be a state of \cutoffsys with $c \geq 2|\visited_\inf{x}|$ that satisfies: 
%\li
%  \- $s_1'(A)=s_1(A)$, $s_1'(B_1)=s_1(B_1)$
%  \- for every $q \in \visited_\inf{x}$ there are two processes of \cutoffsys called $B_{i_q}$, $B_{i'_q}$ with $s_1'(B_{i_q}) = s_1'(B_{i'_q}) = q$
%  \- for all other processes $B_i$ of \cutoffsys (if any): $s_1'(B_i) \in \visited_\inf{x}$.
%\il
%The fair extension extends state $s_1'$ of \cutoffsys to an unconditionally-fair path $y=(s'_1,e'_1,p'_1)\ldots$ with $x(A,B_1) = y(A,B_1)$ as follows. Let $\visited_{\inf\cap B_1}{x}$ be the set of states visited infinitely often by process $B_1$, and $q^\star=s_1(B_1)$:
%\li
%\-[a.] $y(A)=x(A)$, $y(B_1)=x(B_1)$
%\-[b.] for every $q \in \visited_\inf{x} \smi \visited_{\inf\cap B_1}{x}$: in run $x$ there is $B_i$ that starts in $q$ and visits it infinitely often. Let $B_{i_q}$ and $B_{i'_q}$ of \cutoffsys mimic $B_i$ in turns: first $B_{i_q}$ mimics $B_i$ until it reaches $q$, then $B_{i'_q}$ mimics $B_i$ until it reaches $q$,\dots 
%\-[c1.] if $\visited_{\inf\cap B_1}\smi \{ q^\star \} \neq \emptyset$:
%\li
%  \- order arbitrarily $\visited_{\inf\cap B_1}\smi \{ q^\star \} = (q_1, q_2, \ldots, q_k)$
%  \- the processes $\{ B_{i_{q^\star}}, B_{i_{q_1}}, B_{i'_{q_1}}, B_{i_{q_2}}, B_{i'_{q_2}}, \ldots, B_{i_{q_k}}, B_{i'_{q_k}} \}$ behave as follows:
%  \- start with $B_{i_{q_1}}$: when $B_1$ enters $q_1$, it carries $B_{i_{q_1}}$ from $q_1$ to $q_2$, then carries $B_{i_{q_2}}$ from $q_2$ to $q_3$, \ldots, then carries $B_{i_{q_k}}$ from $q_k$ to $q^\star$, then carries $B_{i_{q^\star}}$ from $q^\star$ to $q_1$, then carries $B_{i'_{q_1}}$ from $q_1$ to $q_2$, and so on.
%\il
%\-[c2.] otherwise, $\visited_{\inf\cap B_1} \!=\! \{ q^\star \}$: $B_1$ only transits $q^\star \!\to\! q^\star$; let $B_{i_{q^\star}}$ mimic $B_1$
%
%\-[d.] let other processes $B_i$ of \cutoffsys with $s_1'(B_i)=q$ (if any) mimic $B_{i_q}$.
%\il
%%%%%%%%%%%%% END OF OLD FAIR EXTENSION %%%%%%%%%%%%%%%%%%%%%%

%\begin{proof}[Proof of Lemma \ref{disj:le:FairDisjunctiveBounding}]
%\sj{for weak or strong fairness, the same construction can be used; evacuation is not necessary, but also doesn't increase the cutoff if we use it; difficulty: show that cutoff is still tight
}
Let $c=2\card{B}$. 
Given an unconditionally-fair run $x$ of $\largesys$ 
we construct an unconditionally-fair run $y$ of the cutoff system $\cutoffsys$ 
such that $y(A,B_1)$ is stuttering equivalent to $x(A,B_1)$.

Note that in $x$ there is a moment $m$ such that all local states that are visited after $m$ are in $\visInf{\mB}{x}$.

The construction has two phases. In the first phase, we apply flooding for states in $\visInf{\mB}{x}$, and flooding with evacuation for states in $\visFin{\mB}{x}$:
\li
\-[(a)] $y(A)=x(A)$, $y(B_1)=x(B_1)$

\-[(b)] for every $q \in \visInf{B_2..B_n}{x} \smi \visInf{B_1}{x}$, 
       devote two processes of $\cutoffsys$ that flood $q$

\-[(c)] for some $\qstar \in \visInf{B_2..B_n}{x} \cap \visInf{B_1}{x}$,
       devote one process of \cutoffsys that floods $\qstar$

\-[(d)] for every $q \in \visFin{B_2..B_n}{x}$, 
       devote one process of $\cutoffsys$ that 
       floods $q$ and evacuates into $\visInf{B_2..B_n}{x}$

\-[(e)] let other processes (if any) mimic process $B_1$
\il
The phase ensures that at moment $m$ in $y$, 
there are no processes in $\visFin{\mB}{x}$, 
and all the pre-requisites of the fair extension are satisfied.

The second phase applies the fair extension, 
and then establishes the interleaving semantics 
as in the bounding lemma in the non-fair case.
The overall construction uses up to $2|B|$ copies of $B$.
% Indeed: 
% fin&smi=0, fin&cap=0,
% smi = InfB2..Bn - cap
% InfB2..Bn <= B - fin
% Then:
% 1+2smi+1cap+1+fin <= 1+2(InfB2..Bn-cap)+cap+1+fin = 
%                      2+2InfB2..Bn-cap-fin <=
%                      2+2B-cap-fin
% now split case:
% everywhere fin>0 (otherwise 2|B| follows from the analysis of the fair extension)
% note: if cap=0, then we actually have (recall fair pre)
%       1+2smi+fin =< 1+2(B-InfB1-fin)+fin = 1+2B-2InfB1-2fin =< 2B-3
% thus cap>0,fin>0
% then 2+2B-cap-fin =< 2B

%\end{proof}

\begin{restatable}[Disj, Props, Fair]{tightness}{TightDisjBoundingLemmaFair}
\label{obs:disj:fair_tight_prop}
The cutoff in Lemma~\ref{disj:le:FairDisjunctiveBounding} is tight, i.e., 
for any $k$ there exist process templates $(A,B)$ with $|B| = k$ 
and $\LTLmX$ formula $h(A,B_1)$ such that:
$$
(A,B)^{(1,2|B|)} \models \pexists h(A,B_1) ~~and~~ 
(A,B)^{(1,2|B|-1)} \not\models \pexists h(A,B_1).
$$
\ak{what happens if we bound $T_A$?}
\end{restatable}
The proof was described in the main text, in Section~\ref{sec:ideas-disj-fair}.
\begin{proof}
Consider process templates $A,B$ in the figure below 
and property $\pexists \true$.
% 
\begin{figure}[h]
\centering
\subfloat[Template A]{
\centering
\scalebox{0.75}{\input{img/disj_fair_tight_propAB_tmplA}}
\label{fig:disj:fair_tight_propAB_tmplA}
}
\hspace{1cm}
\subfloat[Template B]{
\centering
\scalebox{0.75}{\input{img/disj_fair_tight_propAB_tmplB}}
\label{fig:disj:fair_tight_propAB_tmplB}
}
\end{figure}

\end{proof}

\ifwithextensions
\begin{restatable}[Generalized Bounding Lemma, Unconditionally Fair]{lem}{DisjBoundingLemmaFairGeneral}
\label{disj:le:FairDisjunctiveBounding:general}
For disjunctive systems:
\begin{align*}
&(A,B)^{(1,> 2\card{B}+n-1)} \models \pexists_{uncond} h(A,B^{(n)}) &
&~\iff~& &
(A,B)^{(1,2\card{B}+n-1)} \models \pexists_{uncond} h(A,B^{(n)}),\\
\end{align*}
% 
\end{restatable}
\iffinal
\begin{proof}
Again, an inspection of the proof of Lemma~\ref{disj:le:FairDisjunctiveBounding} shows that it works almost without modification if we replicate the local run of $A^1$ along with $k$ local runs of copies of $B$, instead of only $A^1, B^1$. In particular, the rest of the flooding construction and the interleaving construction do not have to be changed.
\end{proof}
\else
\fi

\begin{restatable}{obs}{TightDisjBoundingLemmaFairGeneral}
\label{obs:disj:fair_tight_prop:general}
The cutoff in Lemma~\ref{disj:le:FairDisjunctiveBounding:general} is tight.
\end{restatable}

\begin{proof}
Consider again process templates $A,B$ in Figure~\ref{fig:disj:fair_tight_propAB_tmpl}, with property $\pexists \bigwedge_{i \in [1..n]} \GF b_1$, i.e., in addition to enabling an infinite run, we need to keep $n$ processes in $b_1$.
\end{proof}
\fi % \ifwithextensions


\begin{restatable}[Monotonicity: Disj, Deadlocks, Fair]{lem}{mono_lem_disj_deadlocks_fair}
\label{mono_lem_disj_deadlocks_fair}
For disjunctive systems, on strong-fair or finite runs:
$$
\forall n\geq |B|+1: (A,B)^{(1,n)} \textit{ has a deadlock} 
\ \Impl\ 
(A,B)^{(1,n+1)} \textit{ has a deadlock}
$$
\end{restatable}
\begin{proof}
See proof of Lemma~\ref{mono_lem_disj_deadlocks_unfair}.
\end{proof}

\begin{restatable}[Bounding: Disj, Deadlocks, Fair]{lem}{DisjDeadlockLemmaFair}
\label{le:disj:fair_tight_deadlock}
For disjunctive systems, on strong-fair or finite runs:
\li
  \- with $c=2|B|-1$ and any $n>c$:
  $$(A,B)^{(1,c)} \textit{ has a local deadlock} \ \Implied\ (A,B)^{(1,n)} \textit{ has a local deadlock}$$
  
  \- with $c=2|B| - 1$ and any $n>c$
  $$(A,B)^{(1,c)} \textit{ has a global deadlock} \ \Implied\ (A,B)^{(1,n)} \textit{ has a global deadlock} $$
  
  \- with $c=2|B|-1$ and any $n>c$:
  $$(A,B)^{(1,c)} \textit{ has a deadlock} \ \Implied\ (A,B)^{(1,n)} \textit{ has a deadlock}$$
\il
\end{restatable}
\begin{proof}
\providecommand{\deadOne}[1]{\dead_{<2}(#1)}
\providecommand{\deadTwo}[1]{\dead_2(#1)}

If $\largesys$ has a global deadlock, 
then the fairness does not influence the cutoff, 
and the proof from Lemma~\ref{lem_disj_deadlocks_unfair}, 
case ``Global Deadlocks'', applies giving the cutoff $2|B|-1$. 
Hence below consider only the case of local deadlocks. 

Given a strong-fair deadlocked run $x$ of $\largesys$, 
we first construct a strong-fair deadlocked run $y$ of $\cutoffsys$ 
with $c=2|B|$ and then argue that $c$ can be reduced to $2|B|-1$. 
The construction is similar to that in Lemma~\ref{lem_disj_deadlocks_unfair} 
-- the differences originate from the need to infinitely move 
non deadlocked processes.

Let $\deadOne{x}$ be the set of deadlocked states in the run $x$ 
that are only deadlocked if there is no other process in the same state, 
and let $\mD_1$ be the set of processes deadlocked 
in the run $x$ in $\deadOne{x}$. 
Let $\deadTwo{x}$ be the set of states that are deadlocked 
in the run $x$ even if there is another process in the same state. 

Notes: 
\li
\- $|\mD_1| = |\deadOne{x}| \leq |B|$

\- $\deadOne{x} \cap \deadTwo{x} = \emptyset$

\- $\visFin{\mB\smi \mD_1}{x} \cap \deadOne{x} \neq \emptyset$
   is possible, because a state from $\visFin{\mB\smi \mD_1}{x}$ 
   can first be visited by a process in $\mB \smi \mD_1$, 
   and later deadlocked by the process in $\mD_1$.

\- $\deadTwo{x} \subseteq \visInf{\mB\smi \mD^1}{x}$,
   and hence $\visFin{\mB \smi \mD_1}{x} \cap \deadTwo{x} = \emptyset$.
\il

The construction has two phases. 
The first phase:
\li
\-[a.] for every $p \in \{A\} \cup \mD_1$, set $y(p)=x(p)$ 

\-[b.] for every $q \in \deadTwo{x}$, 
       devote one process of $\cutoffsys$ that floods it

\-[c.] for every $q \in \visInf{\mB \smi \mD_1}{x} \smi \deadTwo{x}$,
       devote two processes of $\cutoffsys$ that flood it

\-[d.] for every $q \in \visFin{\mB \smi \mD_1}{x}$, 
       devote one process of $\cutoffsys$ that floods it
       and then evacuates into $\visInf{\mB \smi \mD_1}{x}$

\-[e.] let other processes (if any) mimic some process from (c)
\il
After this phase all $B$ processes will be in 
$\visInf{\mB \smi \mD_1}{x} \cup \deadOne{x}$. 

The second phase applies to processes in 
$\visInf{\mB \smi \mD_1}{x} \smi \deadTwo{x}$ the fair extension%
\footnote{The fair extension requires run $x$ to be unconditionally-fair, 
          but here we have a run in which all processes that are not deadlocked
          move infinitely often.
          To adapt the construction to this case:
          copy local runs of processes $\{A\} \cup \mD_1$,
          and do not extend local runs of processes that are in 
          state in $\dead_2$.}.

% The resulting configuration sequence is a run of $\cutoffsys$ by correctness of the flooding, evacuation, fair extension, interleaving and destuttering constructions\ak{prove}. Furthermore, for every $q \in \deadOne$ we have exactly one process in $\cutoffsys$ that eventually stays in $q$, and for every $q \in \deadTwo$ -- at least one such process; they are eventually deadlocked in $q$ because the states that appear infinitely often in the run $x$ of $\cutoffsys$ are the same as in the resulting run of $\largesys$.

How many processes does the construction use? 
Note that the sets 
$\deadOne{x} \cup \visFin{\mB \smi \mD_1}{x}$, 
$\deadTwo{x}$, 
$\visInf{\mB \smi \mD_1}{x} \smi \deadTwo{x}$ 
are disjoint, thus:
\begin{align}
& | \visFin{\mB \smi \mD_1}{x} | + |\deadOne{x}| + |\deadTwo{x}| + 2| \visInf{\mB \smi \mD_1}{x} \smi \deadTwo{x} | \leq \label{disj:eq:1} \\
% 
& 2|\visFin{\mB \smi \mD_1}{x} \cup \deadOne{x}| + |\deadTwo{x}| + 2| \visInf{\mB \smi \mD_1}{x} \smi \deadTwo{x} | \leq \label{disj:eq:2} \\
% 
& |B| + |\visFin{\mB \smi \mD_1}{x} \cup \deadOne{x}| + | \visInf{\mB \smi \mD_1}{x} \smi \deadTwo{x} | \leq 2|B|  \nonumber
\end{align}
Let us reduce the estimate to $\leq 2|B|-1$:
\li
  \- assume that $\deadTwo{x} = \emptyset$ 
     (otherwise, Eq.\ref{disj:eq:1} and the sets disjointness give $2|B|-1$)

  \- assume that $ \visFin{\mB \smi \mD_1}{x} \neq \emptyset$ 
     (the other case together with eq.\ref{disj:eq:2}, 
      the sets disjointness, and the first item gives $2|B|-1$)

  \- hence, the construction in step (d) evacuates the process in 
     $q \in \visFin{\mB \smi \mD_1}{x}$ 
     into 
     $ \visInf{\mB \smi \mD_1}{x} \smi \deadTwo{x}$. 
     Hence modify step (c) of the construction 
     and for $q$ devote a single process of $\cutoffsys$ that floods it. 
     This will give $\leq 2|B|-1$.
\il
This concludes the proof.

\end{proof}

\begin{restatable}[Disj, Deadlocks, Fair]{tightness}{TightDisjDeadlockLemmaFair}
\label{obs:disj:fair_tight_deadlock}
The cutoff $c=2|B|-1$ for deadlock detection in disjunctive systems on strong-fair or finite runs is tight, i.e.: for any $k$ there are templates $(A,B)$ with $|B|=k$ such that:
$$
(A,B)^{(1,2|B|-2)} \textit{ does not have a deadlock, but } (A,B)^{(1,2|B|-1)} \textit { does}.
$$
\end{restatable}

\begin{proof}
The figure below shows process templates $(A,B)$ such that any system $\largesys$ with $n\leq 2|B|-2$ does not deadlock on strong-fair runs, but larger systems do.
% 
\begin{figure}[h]
\centering
\vspace{-10pt}
\subfloat[Template A]{
\centering
\scalebox{0.75}{\input{img/disj_tight_fair_deadlock_tmplA}}
\label{fig:disj:tight_fair_deadlock_tmplA}
}
\hspace{1cm}
\subfloat[Template B]{
\centering
\scalebox{0.75}{\input{img/disj_tight_fair_deadlock_tmplB}}
\label{fig:disj:tight_fair_deadlock_tmplB}
}
\label{fig:disj:fair_tight_dead_tmpl}
\end{figure}
% 
\end{proof}


\ifwithextensions   %%%%%%%%%%%%%%%%%%%%%%%%%%%%%%%%%%%%%%%%%%%%%
\subsection{Label-dependent Cutoffs for Disjunctive Systems}
\sj{To Do: compare to synchronization skeletons}

The results presented in Lemmas~\ref{disj:le:NonFairDisjunctiveBounding} and \ref{disj:le:FairDisjunctiveBounding} give cutoffs that depend only on the size of templates, without considering the number of output variables that can be used to distinguish states in transition guards. In the following, we analyze the effect of a limited number of output variables on cutoffs. Let $L_B$ be the set of state labels of process template $B$, and $\labelings_B = \bbB^{L_B}$ the set of labelings of $B$.

\begin{restatable}[Bounding Lemma, Label-dependent, Unfair]{lem}{DisjBoundingLemmaLabel}
\label{disj:le:NonFairDisjunctiveBoundingLabels}
For disjunctive systems:
\begin{align*}
&(A,B)^{(1,> \card{\labelings_B} + 2)} \models \pexists h(A,B_1) &
&~\iff~& &
(A,B)^{(1,\card{\labelings_B} + 2)} \models \pexists h(A,B_1).
\end{align*}
\end{restatable}
\iffinal
\else
\begin{proof}
The proof goes along the same lines as the proof of Lemma~\ref{disj:le:NonFairDisjunctiveBounding}. The main difference is that in the flooding construction, we only need to consider states with different labelings. That is, the flooding construction is modified, as to only include one local run for every state \emph{labeling} that is visited in the original run. Thus, the maximal number of local runs needed to ensure that the behavior of $A^1$ and $B^1$ is possible is $\card{\labelings_B}$.

The rest of the proof works just as before. In particular, we also need the local runs of $A^1$ and $B^1$, as well as possibly a copy $B^\infty$ of a local run that is infinite. Thus, the cutoff for process template $B$ is $\card{\labelings_B} + 2$.
\end{proof}
\fi

\begin{restatable}[Deadlock Detection, Label-dependent, Unfair]{lem}{DisjDeadlockLemmaLabel}
\label{le:disj:deadlocks:label}
If the open disjunctive system $(A,B)^{(1,2\card{\labelings_B})}$ has no local deadlocks, then no disjunctive system $(A,B)^{(1,n)}$ with $n \geq 2\card{\labelings_B}$ has local deadlocks.
\end{restatable}
\iffinal
\else
\begin{proof}
The proof works in the same way as the proof of Theorem~\ref{le:disj:deadlocks}, except:
\begin{enumerate}
\item instead of visited states we need to consider visited state labelings
\item we keep $x(A^1,B^1)$, wlog assuming that one of them is locally deadlocked in $x$
\item we use flooding for all state labelings
\item The main technical difference is that for \emph{finitely} often visited state labelings we need a modified evacuation construction, with multiple copies per labeling: 
\begin{itemize}
\item for a process $B^{\witfirst_L}$ that visits label $L$ first, and process $B^{\witlast_L}$ that visits labeling $L$ last in $x$, we cannot be sure that we can append the path from $B^{\witlast_L}$ to the path of $B^{\witfirst_L}$, since they may be in different states $q,q'$ with labeling $L$. 
\item Furthermore, another process with labeling $L'$ may be required to leave state $q$, and $L'$ may itself be a finitely visited state labeling. If $L'$ is evacuated before $B^{\witlast_L}$ reaches state $q'$ in $x$, then $B^{\witfirst_L}$ also needs to leave $q$. To ensure that we will still have a process with labeling $L$ (if it exists in $x$), we need another copy $B^{\witsecond_L}$ that is in a state $q''$ labeled with $L$ after $L'$ is evacuated (and thus, its evacuation cannot depend on labeling $L'$). If such a process does not exist in $x$, then it will also not be needed in the cutoff system. Since the evacuation from $q''$ may depend on another labeling $L''$, we may have to repeat the procedure, adding more copies as witnesses for labeling $L$.
\item To compute how many copies of $B$ are needed in total, let $\{L_1,\ldots,L_n\} = \visited_{fin}$, where $L_i$ is evacuated before $L_j$ in $x$ if $i < j$. Then for $L_1$ we only need one copy of $B$, since evacuation of $L_1$ cannot depend on any labelings that are evacuated before. In general, for labeling $L_i$ we need $i$ copies of $B$\ak{why $i$ is enough? why cannot happen smth like this -- }, since there are $i-1$ labelings that are evacuated before $L_i$, and for each of these we may need an additional copy of $B$ as a witness for $L_i$.
\end{itemize}
\item Overall, we thus need (roughly) $\card{\visited_{inf}} + \sum_{i=1}^{\card{\visited_{fin}}} i$ copies of $B$. The worst case is $\visited_{fin} = \labelings_B \setminus \{L\}$, since at least one labeling $L$ needs to be present infinitely often. Thus, the cutoff is $1 + \frac{\card{\labelings_B}\cdot(\card{\labelings_B}+1)}{2}$.
\end{enumerate}
\end{proof}
\fi

\begin{restatable}[Bounding Lemma, Unconditional Fairness, Label-dependent]{lem}{DisjBoundingLemmaLabelFair}
\label{disj:le:FairDisjunctiveBoundingLabels}
For unconditionally-fair runs of disjunctive systems:
\begin{align*}
&(A,B)^{(1,> 3 \card{\labelings_B} - 1)} \models \pexists h(A,B_1) &
&~\iff~& &
(A,B)^{(1,3 \card{\labelings_B} - 1)} \models \pexists h(A,B_1).
\end{align*}
\end{restatable}
\iffinal
\else
\begin{proof}
The proof goes along the same lines as the proof of Lemma~\ref{disj:le:FairDisjunctiveBounding}, with modified flooding and fair extension constructions. 

{\bf Original construction.} Remember that for systems with fairness, the flooding construction separates states of the originial run into those that are visited finitely often ($\visited_{fin}$) and those that are visited infinitely often ($\visited_{fin}$). Then, we consider a moment in time $\pref$, which is the first moment such that no states from $\visited_{fin}$ will be visited anymore. The original construction then ensured that the transitions of all local runs are possible until moment $\pref$, and that at this moment there are at least two processes in any given state from $\visited_{inf}$.

Starting from $\pref$, the original proof continues with the \emph{fairing extension} construction that ensures that all local runs can move infinitely often. For the combination of flooding and fair extension, it is crucial that for any state $q$ in $\visited_{inf}$, we combine a prefix from a local run that reaches $q$ first with a postfix (from a possibly different local run) that reaches $q$ infinitely often.

{\bf Modified construction.}
Like in the proofs of Lemmas~\ref{disj:le:NonFairDisjunctiveBoundingLabels} and \ref{le:disj:deadlocks:label}, we define $\visited$ and flooding based on labelings. 

For labelings in $\visited_\fin$, we use the modified evacuation construction from Lemma~\ref{le:disj:deadlocks:label}. That is, we order labelings $\{L_1,\ldots,L_n\} = \visited_\fin$ according to the moment they are evacuated, and add $i$ copies of $B$ for each $L_i$, such that we have a copy that is in a state with labeling $L_i$ before and after all labels that are evacuated before have been evacuated. All of these process copies eventually move to a loop with only labelings from $\visited_\inf$, just like they did in $x$.

For labelings $L \in \visited_\inf$, we use a combination of the original fair extension construction with this modified evacuation construction. The reason we need evacuation is the same as in Lemma~\ref{le:disj:deadlocks:label}: the state with labeling $L$ that is reached first in $x$ may be different from the state $q'$ with labeling $L$ in which some process eventually loops. Therefore, we add one copy $B^{\witfirst_L}$ that takes the shortest path to $L$, and two copies of $B^{\witlast_L}$ that go to a state $q'$ in which there is a loop coming back to $q'$ (that only depends on labelings in $\visited_\inf$). Like in Lemma~\ref{le:disj:deadlocks:label}, we may have the case that a labeling from $\visited_\fin$ needed to leave $q$ will be evacuated before $q'$ is reached by $B^{\witlast_L}$, and therefore we in general need multiple copies of $B$ that reach states with $L$ at different times. Since $L \not\in \visited_\fin$, in general all labelings from $\visited_\fin$ could be evacuated before $q'$ is reached. Thus, for every labeling $L \in \visited_\inf$, we need up to $\card{\visited_\fin}+1$ copies of $B$ that reach $L$ and eventually leave it again, and two additional copies for the fair extension that ensures that $L$ will always be present after $\pref$.

Finally, we use the interleaving construction as before.

In summary, we need $\sum_{i=1}^{\card{\visited_\fin}} i = \frac{\card{\visited_\fin} \cdot (\card{\visited_\fin}+1)}{2}$ copies of $B$ for the labelings in $\visited_\fin$, and $(\card{\visited_\fin}+3)\cdot\card{\visited_\inf}$ copies of $B$ for the labelings in $\visited_\inf$. In the worst case, this means that the cutoff for process template $B$ is (bounded by) \ldots \remove{$3 \card{\labelings_B}$. By a closer analysis, we can see that for at least one labeling $l$ in $\visited_{inf}$, the local run that visits $l$ first must also visit it infinitely often, and we only need one additional copy of this run. Thus, we obtain a cutoff of $3 \card{\labelings_B} - 1$.}
\end{proof}
\fi

\begin{restatable}[Deadlock Detection, Strong Fairness, Label-dependent]{lem}{DisjDeadlockLemmaLabelFair}
\sj{wait for clarification in fair state-based case}
\end{restatable}
\fi %ifwithextensions   %%%%%%%%%%%%%%%%%%%%%%%%%%%%%%%%%%%%%%%%%%%%%%%%%
