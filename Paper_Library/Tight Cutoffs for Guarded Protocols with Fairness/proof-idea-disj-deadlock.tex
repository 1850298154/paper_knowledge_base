\begin{proof}[Proof idea]
\ak{status -- possibly unite/shorten parts of the proofs}
We prove that if there is a deadlocked run $x$ of the large system $\largesys$, then there also is a deadlocked run of the cutoff system $\cutoffsys$ ($c$ is defined later)\rb{c is $B+1$? otherwise introduce $c_1$ and $c_2$ -- cutoffs for two different cases}. 
To this end, distinguish two cases: either $x$ is strictly locally deadlocked (and thus infinite), or it is globally deadlocked (and thus finite).

\subsubsection*{Strictly Locally Deadlocked Run.}
Again, we can distinguish two cases: 
\begin{itemize}
\item [a)] process $A$ deadlocks, hence wlog. process $B_1$ moves infinitely often,
\item [b)] process $A$ moves infinitely often, hence wlog. process $B_1$ deadlocks.
\end{itemize}
Then we construct an infinite run $y$ of the cutoff system $\cutoffsys$ such that process $U \in \{A,B_1\}$ is deadlocked in $y$ iff it is deadlocked in $x$. In the following, let $U$ be the process that eventually deadlocks, and $V$ the process that moves infinitely often.

Let $c=|B|+1$.

Recall that process $U$ is locally deadlocked in $x=(s_1,e_1,p_1) \ldots$ iff there is a $\time$ such that all transitions of $U$ from $s_{\time'}(U)$ with $e_{\time'}(U)$ are disabled in all states $s_{\time'}$ with $\time' \geq \time$. Note that by definition, if $U$ does not move then $s_{\time'}(U) = s_\time(U)$ and $e_{\time'}(U)=e_\time(U)$ for $\time' \geq \time$.

Informally, the construction consists of the following three steps:
\begin{enumerate}
\item \emph{Flooding Construction.} To ensure that processes $U$ and $V$ 
can exhibit the same behavior as in $x$ until the moment where one of them 
deadlocks, we first use the flooding construction from the proof of 
Lemma~\ref{disj:le:NonFairDisjunctiveBounding}, except that we do not need an 
additional process that moves infinitely often.\sj{here it is important 
that flooding construction only considers processes different from $B^1$ --- 
but this should be OK in other cases also?}\ak{todo: flooding here is applied to the whole run, not to local runs as i imagined before}

\item \emph{Evacuation Construction.} The evacuation construction modifies some of the local runs that result from the flooding construction. Since the local deadlock of $U$ may 
depend on the fact that eventually some states $q \in \visited(x)$ will not be 
visited anymore, we distinguish between states that are visited finitely often 
($\visited_\fin(x)$) or infinitely often $(\visited_\inf(x))$ in $x$. For states 
$q \in \visited_\fin(x)$, we require that their witness processes leave $q$ 
eventually and move to some state from $\visited_\inf(x)$. This is always 
possible by mimicking the behavior of the last process that left $q$ in the 
original run $x$, retaining the order in which states are moved out of 
$\visited_\fin(x)$ in the original run $x$.

If all processes are eventually in states from $\visited_\inf$, then 
$U$ is also deadlocked in the constructed run: if it was not, then one of the 
states from $\visited_\inf$ would enable one of its transitions, and 
therefore $U$ would not be deadlocked in $x$. Thus, we obtain a pseudo-run 
in which $U$ deadlocks.

\item \emph{Interleaving Construction.} Like in the proof of Lemma~\ref{disj:le:NonFairDisjunctiveBounding}, we use the interleaving construction to convert the resulting pseudo-run into a run.
\end{enumerate}

\subsubsection*{Globally Deadlocked Run $x$.}
In a similar way, we can construct a globally deadlocked run $y$ of $\cutoffsys$ with $c = 2 \card{B} - 1$ from a globally deadlocked run of $\largesys$. The additional copies are needed since we have to copy all local runs that end in a state $q$ that is only deadlocked if there is not other process in $q$ (bounded by $\card{B}$), and in addition use the flooding and evacuation construction for all states that are visited in the other local runs of $x$ (bounded by $\card{B}$). A closer investigation of the possible cases gives a cutoff of XXX.
\end{proof} 