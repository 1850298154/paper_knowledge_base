% 
\begin{proof} 
Exactly the same example (Fig.~\ref{fig:conj:tight_dead_tmpl}) as in Observation~\ref{obs:conj:tight_deadlock} can be used to prove the observation.
Note that if change the template $A$ to any that never deadlocks, then we get the example to prove the tightness for the special case of strictly local deadlocks too.
% \begin{figure}[Htb]
% \centering
% \subfloat[Template A]{
% \centering
% \makebox[0.4\textwidth][c]{
% \scalebox{0.75}{\input{img/conj_tight_deadlock_tmplA}}
% \label{fig:conj:tight_deadlock_tmplA}
% }}
% \subfloat[Template B]{
% \centering
% \makebox[0.6\textwidth][c]{
% \scalebox{0.75}{\input{img/conj_tight_deadlock_tmplB}}
% \label{fig:conj:tight_deadlock_tmplB}
% }}
% \caption{Templates $(A,B)$ used to prove the tightness of the cutoff $c=2|B|-2$ for the deadlock detection in 1-guard conjunctive systems.
% In the figure the edge with $\forall{\neg b_1},\ldots,\forall{\neg b_k}$ denotes edges with guards $\forall{\neg b_1},\ldots,\forall{\neg b_k}$ (Observation~\ref{obs:conj:tight_deadlock}).\ak{check me}}
% \label{fig:conj:tight_dead_tmpl}
% \end{figure}
\end{proof}