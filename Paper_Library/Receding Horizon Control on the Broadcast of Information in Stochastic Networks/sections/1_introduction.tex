
%{\bf \large{Some references}}\\
%\begin{itemize}
%    \item\cite{Wang2015} Optimal control for epidemic routing of two files with different priorities in Delay Tolerant Networks
%  A Multi-Layer Swarm Control Model for Information Propagation and Multi-Tasking 2019 ACC Fuming Zhang
%\end{itemize}
%Outline - 
%The problem is defined as controlling the performance of a network composed of mobile robots by controlling the the individual performance (define performance) of each of the node.

%The multi-robot system has an architecture that can be described by a graph determined by their positions and interactions. The topology and the features (define features) of the interactions together delivers the performance (of information propagation, routing, etc.) of a network.

%Uncertainty brings in randomness. Exsting work only considers certain stochasticity - whether a link is there or not. There are other issues to be consider. e.g. - The distances between different pairs of robots only only defined the topology, which specifies who can talk with who, but also how good they are communicating. The transmission quality can be described by a transmission rate, which converged the completion of transmission on one link into a Poisson process (that the completion is seen as an event).

%Existing graph theory tools can handle either a static graph or a time-varying graph with known dynamics (or random graphs). When the uncertainty disrupts the dynamics, existing tools all fail. We need new tools analyzing graphs that :1) with stochasticity in diffrent ~ , 2) with structues / patterns that we can leverage.

%So that we can deliver control schemes that can guarantee/optimize the performance of the network in a decentralized fashion.


%\subsection{Existing results}
%Li found a way of determining the most important links considering the chronological order.

%\subsection{Challenge}
%(maybe future work - If we want to improve the transmission quality, we essentially want to move that pair of robots close to each other.

%However, moving robots will affect the relative position to other robots. 

%Improving transmission rate all around may cause conglomeration.

%We may want to find a way of moving robots to improve the topology / transmission but not to cause conglomeration.)

%\subsection{Approach}

%\paragraph{Stage I:}
%The topology of the communication network is fixed. Any robot improving its communication quality with one of its neighbours will suffer from a decreased communication quality with other neighbors. 

%Model - Fixed topology, robot $i$ has $d_1$ neighbors, the delivery of a piece of message along a communication link is modeled as a Poisson process, each communication channel is associated with a parameter $\lambda_{ij}$ determining the Poisson process. We have $\sum_{\forall j} \lambda_{ij} = 1$.

%Aim - control each node's distribution of $\lambda$ to optimize the performance of the system.

%To be added - the model of information resources?

%\paragraph{Stage II:}
%The robots are moving on a 1-D or 2-D space that moving closer towards one neighbor with risk distancing itself with other neighbors.

%The 'perimeter' nodes are fixed.

%Communication qualify is related to the distance between the pair of robots. 

We are interested in leveraging spatio-temporal sensing capabilities of robotic teams to explore or monitor large-scale complex environments
(\textit{e.g.,} forests, oceans, and underground caves and tunnels \cite{breitenmoser2010voronoi,cassandras2016smart,gusrialdi2008voronoi,wei2018,yu2019synchronous}).
%\xyucomment{I mean 'deliver the tasks'. I edit the sentence accordingly. Thanks!}
While individual robots can each monitor or explore limited regions, a team of them can efficiently cover larger areas.  Naturally, individual robots in swarms are expected
to exchange information and make certain decisions based on the information gathered by itself and received from other robots \cite{hsieh2008decentralized,moarref2020automated}.
It is, therefore, essential for any robot to be able to
transmit its gathered data to others in the team. 

As the demand of information transmission can emerge between any pair of nodes, we use a propagation model to study the efficiency of the connectivity of the whole team. 
In such models, robots are represented as nodes of a graph,
and interactions 
%(\textit{i.e.} communication)
between nodes are represented by edges.
%The nodes and the edges together form a network.
At the beginning of a propagation, the nodes 
%(\textit{i.e.} robots) 
are considered as `non-informed', and will be `informed’ by the information propagated through the graph. A faster convergence of all nodes into the status of `informed' indicates a more \textit{efficiently connected} network. 
Existing works abstract the propagation of information from any individual agent to the rest of the team as compartmental models \cite{huang2006information,khelil2002epidemic,zanette2002dynamics}, which are also broadly used in modeling the spread of an infectious diseases \cite{brauer2008compartmental}.
%In such models, robots are represented as nodes of a graph,
%and interactions 
%(\textit{i.e.} communication)
%between nodes are represented by edges.
%The nodes and the edges together form a network.
% The nodes 
% %(\textit{i.e.} robots) 
% are labeled according to their status of whether the robots are `informed’ by the information propagated through the swarm.
The flow of the nodes from one compartment to another
%(\textit{i.e.} uninformed robots get `informed’ with the information.)
occurs when active links exist between robots that are
`informed’ and `non-informed’ %the information 
(\textit{i.e.,} 
only `informed' robots %carrying the information 
can transmit the information to `non-informed' ones).%robots that are not yet informed). 


%\thales{However, epidemic models often operate under the assumption that the networks are static and deterministic, so that the flow between compartments can be analyzed with ordinary differential equations \cite{}. In reality, neither human contacts nor robots working in communication challenging environments can guarantee such consistent connectivity in large-scale networks. The classic compartmental-models-based control strategies for such networks also focus on the statistical behavior of the nodes, ignoring the topology of the networks. The analysis and solutions are, therefore, generally valid with giant node numbers reaching thermodynamic limits \cite{}. } 


However, compartmental models often focus on the statistical behavior of the nodes, overlooking the capabilities of the \textit{individual-level} decision making that may impact the network's performance.
%\xyucomment{I commented out this paragraph and edit it as a couple of bridge sentences. }
Recent works in robotics have noticed and embraced the potential of controlling individual robots so that the performance of the whole network (\textit{i.e.,} the transmission rates, if a propagation model is considered,) can be impacted or improved. %swarm can jointly achieve the desired status. 
The topology of networks was analyzed in \cite{Preciado2014} and control strategies were proposed to identify 
and isolate the nodes that have a higher impact on
the propagation process. The design of {\it time-varying} networks where robots leverage their mobility to move into each other's communication ranges to form and maintain temporal communication links has been addressed in  \cite{hollinger2012multirobot,khodayi2019distributed,yu2020synthesis}. Information can be propagated in such networks by relying on time-respect routing paths formed by a sequence chronologically ordered temporal links \cite{yu2020synthesis}. 
Nonetheless, such approaches still require thorough and perfect knowledge of the network's topology and the robots' ability to maintain active links and requires robots to execute their motion plans precisely both in time and space. As such, the resulting networks lack robustness to uncertainties in link formation timings that may result from errors in motion plan execution due to localization errors and/or actuation uncertainties.

%\thales{Stochastic models, such as networks based on percolation theory concepts \cite{}, have also been introduced in works in epidemiology in recent years to carry out a more realistic analysis. Percolation theory assumes a growing collection of edges between nodes. When the addition of edges fits a stochastic process, the growing graphs yielded are modeled as Erd\"os–R\'enyi networks, and have been well studied in the field of random graphs \cite{}. Noticed that such stochastic models focus on a random \textit{existence} of edges on a graph, echoing the assumption of Canadian Traveler Problems, in which the graphs are partially visible, and the edges are randomly blocked \cite{}. }

%Real-world communication networks synthesized in robot 
%swarms face uncertainties causing stochastic performance
%of the networks.
%that can not always be directly addressed by existing graph theory tools. 
Existing stochastic graph models focus on random changes in a network's topology centered around random creation and removal of edges within a graph. For example, the Canadian Traveller Problem assumes partially visible graphs where edges randomly appear and disappear \cite{bar1991canadian,nikolova2008route}.
For time-varying networks,
\cite{knizhnik2022flow,shen2022topology} assumed that any temporal link
may suffer from deviations from
its scheduled timing due to the uncertain arrival time of one robot into another robot’s communication range.
%\xyucomment{I commented out the original paragraph and edit the following graph as shown in blue. }
%Consider robot swarms moving in challenging environments
%with network topology changing over time.
%Any temporal link that occurs and disappears in such a network may suffer from deviations in its scheduled timing \cite{ACC2022} due to the uncertain arrival time of one robot into another robot’s communication range \cite{ICRA 2022}.
Information routing or propagation planned for such networks may experience severe delays if subsequent links along a routing path appear out of chronological order in the presence of link formation uncertainties resulting from uncertainties in robot motions \cite{yu2020synthesis}.  These challenges are addressed in \cite{shen2022topology} for networks with periodically time-varying interconnection topologies.  Control strategies to `fix’ nodes with 
higher impact on the whole network’s performance were also proposed in \cite{shen2022topology} similar in spirit to those presented in \cite{Preciado2014}.

In the Canadian Traveler Problems and \cite{shen2022topology}, messages are assumed to be transmitted instantaneously between nodes whenever a link appears. As such, the resulting strategies are solely based on the \textit{existence} of an available link between a certain pair of robots at a certain time point. When the time to transmit a message is non-trivial, the question goes beyond whether the information can be transmitted or not and must consider the quality of the transmission (\textit{e.g.,} safe, fast, confident). Consider a pair of robots flying through a Poisson forest maintaining a communication link that requires line-of-sight (LOS). The randomly distributed obstacles may intermittently disrupt the LOS between the robots causing randomized delay in the completion of the information transmission task. In these situations, it becomes difficult to determine whether a robot \textit{is} informed or not at a given time point which then impacts the robots ability to plan subsequent actions.

In this paper, we aim to develop receding horizon control schemes that allows individual robots to re-direct their transmission resources 
(\textit{e.g.,} time, power) to different neighboring robots on the communication network with stochastic
links to guarantee an exponentially fast
convergence status in the information propagation. We model the completion of the message transmission across a link as a Poisson Point Process. The density parameter of this process is determined by the transmission resources invested by the nodes.  Each node carries limited transmission resources that is distributed among all the links connecting to it.
The completion of the transition of a 
message from one node to another is then modeled as a Markov process. 
All robots would then be able to update the distribution of their own transmission resources according to the neighboring current 
states and follow the control 
directions. Therefore, the control
strategy takes into account the 
transmission resource capabilities of the
robots and acts on them to fulfill
performance requirements.
Such a set of resources changes 
according to the application, for
example, in the Poisson forest we
might require that the robots stay close
together at the cost of decreasing the
covered area; in a situation in which
the robots need to move back and forth
to carry information ({\it e.g.,} 
surveillance
around buildings or in tunnels), we
might require the robots to increase their
frequency of visits among each other.

The paper is organized as follows. Sec.~\ref{sec:graph_theory} and Sec.~\ref{sec:SI_model} provides theoretical backgrounds of the graph structure and the propagation model. The problem is formally stated at the end of Sec.~\ref{ProbForm}. Sec.~\ref{section3} introduces our approach of developing the receding horizon control scheme for the stochastic network. Sec.~\ref{simulation} validates our proposed control schemes with numerical examples. Sec.~\ref{conclusion} concludes the paper and proposes future directions.