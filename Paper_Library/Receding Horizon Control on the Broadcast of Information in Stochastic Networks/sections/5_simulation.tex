In this section,
we illustrate the approach proposed in Section
\ref{section3} with two simulation results.
The first simulation was carried out on a 15 robots static 
topology. The topology of the network was generated using Erd\"{o}s-R\'{e}nyi model
with connection probability of 0.3. 
We assumed that each robot has an uniform total transmission rate $\mu$ to distribute to its connected neighbor.
% In other words, once a robot become informed, 
% it can transmit the message to
% its neighbors with different
% transmission rates $\omega_{ij}(t)$ such that 
% the sum its transmission resources 
% is equal to an uniform fixed value $\mu$.
% In this simulation, we set the total
% transmission rate $\mu$ to be 2.
In other words, an informed robot 
can become connected to
its neighbors according to different
transmission rates $\omega_{ij}(t)$
such that 
the sum of its transmission rates 
between neighbors
is limited by fixed value $\mu$.
This parameter, along with the connection probability of Erd\"{o}s-R\'{e}nyi model, has a big impact on the information propagation speed of the whole network. In this simulation, we set the total
transmission rate $\mu$ to be 2 for better comparison results.
% and $\omega(t)$ is randomly sampled so that  $\omega(t) \in (0,\mu)$,  $\sum_{j}^ {j\in\VV , (j,i)\in\EE} \omega_{ij} = \mu$.
% \thales{and we  each edge's transmission rate following this assumption. }
% \cthales{Not sure what do you mean.}
% \begin{align}
% \label{eq:pred_ctrl}
% & \min_{{\omega_{ij}}\in U} C(X(t))
% \\
% &~s.t.~ 
% n-E\big[\Scal(\boldsymbol{X}(t+\Delta t))\big] - 
%      (n-\Scal(\boldsymbol{X}(t))e^{-r\Delta t} \leq 0,
%     \\
% \big(1-E\big[X_i(t+\Delta t):X(t)\big]\big)
% -\big(1-X_i(t)\big)e^{-r\Delta t}\leq 0,
% \\
% & \sum_{j}^ {j\in\VV , (j,i)\in\EE} \omega_{ij} = \Omega
% \\
% & \omega_{ij} \in (0, \Omega)
% \nonumber
% \end{align}

We applied the control strategy \eqref{eq:pred_ctrl} proposed in 
Section \ref{sec:control_strategy} with the
Con-tinuous-Time Markov Chain (CTMC) representation
and with the Robust Moment Closure (RMC), model as
described in Section \ref{sec:robust_moment_closure}.
We can choose from a myriad of cost functions that align with different design purposes.
Here we chose $C(X_i(t))=\sum_{j\in\NN_i}\omega_{ij}^2$. If we were not seeking an exponential propagation, this objective function, together with the constraints on the transmission
resource $\omega_{ij}(t) \in (0,\mu)$, and
$\sum_{j\in\NN_i}\omega_{ij} = \mu$ for
each $i\in \VV$, would push for an even distribution of every robot's resources between its neighbors. We chose $\Delta t=1$, 
and $r=0.22$ for both CTMC and RMC models, and simulated our proposed controller which guarantees an exponential propagation while 
minimizes the chosen cost.
%The control method was defined as 
%in \eqref{eq:pred_ctrl} with
%$C(X_i(t))=\sum_{j\in\NN_i}\omega_{ij}^2$, $\Delta t=1$, 
%and $r=0.22$ for both CTMC and RMC models.
%The set of admissible control actions $U$ 
%is defined by the constraints $\omega_{ij}(t) \in (0,\mu)$, and
%$\sum_{j\in\NN_i}\omega_{ij} = \mu$ for
%each $i\in \VV$.
The original exact model is also simulated
without any control method in the loop. 
We run all three models 10 times each, and we took their average as the performance results. The performance comparison result is shown in Fig. \ref{fig:15_simulation_result} and the numerical result at different stages of the propagation process is in Table.~\ref{tab:results}.

 From Fig.\ref{fig:15bar},
the CTMC model with control outperforms two other models,
the RMC model with control is slightly slower than
the CTMC, but still faster than the original model without control.
The CTMC model was in average  39.6$\%$ faster than the open-loop model to informed all robots, and RMC model was 27.3$\%$ faster than the open-loop model completion time.
% \cthales{Instead of the \% of the total time, can you compute how \% faster it was? It's just the complement and is simpler for the reader}
% The CTMC model took 71.62$\%$ of the time comparing with the open-loop model completion time to informed all robots, and RMC model finished using 78.52$\%$ of the open-loop model completion time.
% \thales{The CTMC model informed all robots in 4.5902 seconds, RMC model finished in 5.0324 seconds, and epidemic model in 6.4092 seconds.}\cthales{Put those as percentage in relation to wo/ controls.}
Notice that the two patterns in black
and in red lines shown in Fig. \ref{fig:15diff} are similar--the biggest difference number between CTMC and RMC is 3 robots which occurs around 2.5 seconds.
%\shen{here the two - is not shown on the paper}

% \begin{figure}[h]
% % Use the relevant command for your figure-insertion program
% % to insert the figure file.
% % For example, with the graphicx style use
% \centering
% \begin{subfigure}[t]{0.49\textwidth}
% \centering
% \includegraphics[width=1\textwidth]{figures/15nodes_final.pdf}
% \caption{}
% \label{fig:15nodes}
% \end{subfigure}
% \hfill
% \begin{subfigure}[t]{0.49\textwidth}
% \centering
% \includegraphics[width=1\textwidth]{figures/100node_final.pdf}
% \caption{}
% \label{fig:100nodes}
% \end{subfigure}

\begin{figure}[h]
\centering
\begin{subfigure}[t]{0.49\textwidth}
\centering
\includegraphics[width=1\textwidth]{figures/15bar.pdf}
\caption{}
\label{fig:15bar}
\end{subfigure}
\hfill
\begin{subfigure}[t]{0.49\textwidth}
\centering
\includegraphics[width=1\textwidth]{figures/15diff.pdf}
\caption{}
\label{fig:15diff}
\end{subfigure}

\caption{{
Numerical results comparison of 15 nodes network’s broadcasting speed. 
In (a), the bar chart showcases the total number of robots that received information after the end of each 1.0833 seconds. The error bars represent the maximum and minimum informed robots in our 10 different data-set. In (b), the Line chart demonstrates the evolution of the difference in informed robots with respect to time. The red plot shows the difference between CTMC and the exact model, and the black plot shows the difference between RMC and the exact model.
} 
}
\label{fig:15_simulation_result} 

%
     
\end{figure}

\begin{figure}[h]
\centering
\begin{subfigure}[t]{0.49\textwidth}
\centering
\includegraphics[width=1\textwidth]{figures/100bar.pdf}
\caption{}
\label{fig:100bar}
\end{subfigure}
\hfill
\begin{subfigure}[t]{0.49\textwidth}
\centering
\includegraphics[width=1\textwidth]{figures/100diff.pdf}
\caption{}
\label{fig:100diff}
\end{subfigure}

\caption{{
Performance comparison of 100 nodes network’s broadcasting speed. 
In (a), the bar chart depicts the total number of informed robots after every 0.2609 seconds for both RMC and exact models. The error bars indicate the variation in informed robots of 10 different simulations.
In (b), the line plot represents 
the evolution of the difference
in informed robots between RMC and the open-loop method.
} 
}
\label{fig:100_simulation_result} 
\end{figure}

%Table \ref{tab:results} shows the average accumulated times that all models take to inform $20\%, 40\%, 60\%, 80\%$, and $100\%$ of the robots during the propagation process of different topology networks.

\begin{table}[h]
    \centering
    \caption{Average completion time at each propagation stage}
    \begin{tabular}{c|c|c | c | c | c | c}
    \toprule
    Topology & Model & $20\%$ & $40\%$ & $60\%$ &$80\%$& $100\%$ \\ \midrule 
    \multirow{3}{*}{\parbox{2cm}{\centering 15 nodes}}&
    CTMC &$\ 2.0571\ $ & $\ 2.4084\ $  &$\ 2.7997\ $ & $\ 3.7673\ $ & $\ 4.5902\ $ \\
    & RMC &$\ 2.2174\ $ & $\ 2.7436\ $  &$\ 3.3202\ $ & $\ 4.1394\ $ & $\ 5.0324\ $ \\
    &\ Open-Loop\ &$\ 2.6085\ $ & $\ 3.1619\ $  &$\ 3.7573\ $ & $\ 5.1964\ $ & $\ 6.4092\ $ \\ \midrule
    \multirow{2}{*}{\parbox{2cm}{\centering 100 nodes}}&
    RMC &$\ 0.2325\ $ & $\ 0.2849\ $  &$\ 0.3131\ $ & $\ 0.3525\ $ & $\ 0.9364\ $ \\
    &\ Open-Loop\ &$\ 0.2515\ $ & $\ 0.3343\ $  &$\ 0.4603\ $ & $\ 0.6428\ $ & $\ 2.9845\ $ \\ \bottomrule
    \end{tabular}
    \label{tab:results}
\end{table}
%


% \begin{figure}[h]
% % Use the relevant command for your figure-insertion program
% % to insert the figure file.
% % For example, with the graphicx style use
% \centering
% \begin{subfigure}[t]{0.49\textwidth}
% \centering
% \includegraphics[width=1\textwidth]{figures/15nodes_final.pdf}
% \caption{}
% \label{fig:15nodes}
% \end{subfigure}
% \hfill
% \begin{subfigure}[t]{0.49\textwidth}
% \centering
% \includegraphics[width=1\textwidth]{figures/100node_final.pdf}
% \caption{}
% \label{fig:100nodes}
% \end{subfigure}


% \caption{{
% Numerical results comparison of network’s broadcasting speed. The upper line chart demonstrates the evolution of the difference in the number of informed robots between our approaches and the open-loop approach with respect to time. The lower bar chart showcases the total number of robots that received information after the end of each time interval. (a) The red plot in the upper graph represents the difference between CTMC and the exact model, and the black plot represents the difference between RMC and the exact model. (b) The upper graph shows the number of informed robots’ difference between RMC and the open-loop method over time. The red bar in the lower graph is the open-loop model without control, and the blue bar is RMC with control.
% } 
% }
% \label{fig:simulation_result} 

% %
     
% \end{figure}

We conducted the second simulation in a
larger static topology network with 100 nodes.
The CTMC model is not appropriate in this setting as the state-space evolves in
$\real^{2^{100}\times2^{100}}$. The topology of the network is again generated using Erd\"{o}s-R\'{e}nyi model
with connection probability of 0.05,
and total transmission rate $\mu=4$.
We compared the performance between the open-loop
model and RMC model in broadcasting speed.
The control strategy is implemented
considering $\Delta t =0.1$ and 
exponential convergence rate
$r=2.8$. We run 10
different trials for each model
using random seeds and use their mean values as the numerical results.

% \begin{figure}[h]
% \sidecaption[t]
% % Use the relevant command for your figure-insertion program
% % to insert the figure file.
% % For example, with the graphicx style use
% \includegraphics[scale=.6]{figures/100node_comparision.pdf}
% %
% \caption{{\small Numerical result of network's broadcasting speed with Robust Moment Closure with control and the original epidemic control. The lower bar chart showcases the total number of robots have received information after each time interval. The upper plot demonstrates the evolution in number of informed robots difference using Robust Moment Closure model with control compared to epidemic model with respecting to time. \normalsize}
% }
% \label{fig:2method}      
% \end{figure}

Fig. \ref{fig:100_simulation_result} shows the performance in
information broadcasting for those two models.
Notice that at the beginning in Fig. \ref{fig:100bar}, before 0.3 seconds,
there was not much difference between the two
methods.  
This is because the constraint of the optimization problem was satisfied with the initial trivial solution and 
there was not many actions in $U$ that could further
improve the information propagation during
that period.
This might be why there is a fluctuating pattern at the beginning in Fig. \ref{fig:100diff}.
After that, the control strategy
starts to show its advantage,
and the average time to broadcast the information to all robots is 
%0.9364 
 0.93 seconds, which is in average 218.7$\%$ faster than the open-loop network, which has
 an average of 2.98 seconds.
%  The exact model finishes in 2.98 seconds,
%  \thales{that is \{\} faster
%  than the open loop network. \{add \% of how fast this is\}}.
 %2.9845 seconds.

Both numerical examples show a significantly better performance of our proposed methods in the scenarios assuming a random node is propagating information to the rest of the team. Notice that such scenarios are not usually seen in real-world applications. A faster propagation indicates a more efficiently connected network, providing a tighter upper bound of the information transmission time between any pair of nodes in the network. Therefore the propagation models were chosen to demonstrate a more efficient transmission is expected for any shortest path connecting two nodes in the network. A limitation of this propagation model is that all shortest paths are considered equally important, while in real-world applications, it makes sense that certain paths are with higher priorities due to the actual variation of demands. Our method can be easily tailored to fit different objective functions. 

