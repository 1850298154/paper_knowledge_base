We have investigated a control structure for shaping the transmission rates on networked systems
to increase the connectivity in networks, the broadcast of information can be modeled as stochastic jump processes. As a first result, we have shown that by using an epidemiological model to represent the transmission of information among robots 
we can describe the probability of future transmissions based on the current joint state of the network and its topological structure. Then, based on this finding, we examined the applicability of a receding control strategy to account for possible drifts in performance on the current realization and, subsequently, actuate on transmission rates whenever necessary. This approach provides efficient adjustments to the network. Finally, numerical experiments were implemented, illustrating the effectiveness of the method. Possible future work includes the extension of the strategy for distributed optimization. It is also essential to connect the robot dynamics with the changing transition rates. One possible approach to bridging those two models is through maps defined {\it a priori} and then accounting for them on the admissible set of actions. Finally, our methodology only accounts for the expectation of the transmission of information. This parameter can be a weak decision parameter, an extension to consider the variance would be appealing.
