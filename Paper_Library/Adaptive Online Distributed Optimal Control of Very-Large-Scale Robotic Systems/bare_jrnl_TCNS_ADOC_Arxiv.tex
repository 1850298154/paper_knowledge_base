
%% bare_jrnl.tex
%% V1.4b
%% 2015/08/26
%% by Michael Shell
%% see http://www.michaelshell.org/
%% for current contact information.
%%
%% This is a skeleton file demonstrating the use of IEEEtran.cls
%% (requires IEEEtran.cls version 1.8b or later) with an IEEE
%% journal paper.
%%
%% Support sites:
%% http://www.michaelshell.org/tex/ieeetran/
%% http://www.ctan.org/pkg/ieeetran
%% and
%% http://www.ieee.org/

%%*************************************************************************
%% Legal Notice:
%% This code is offered as-is without any warranty either expressed or
%% implied; without even the implied warranty of MERCHANTABILITY or
%% FITNESS FOR A PARTICULAR PURPOSE! 
%% User assumes all risk.
%% In no event shall the IEEE or any contributor to this code be liable for
%% any damages or losses, including, but not limited to, incidental,
%% consequential, or any other damages, resulting from the use or misuse
%% of any information contained here.
%%
%% All comments are the opinions of their respective authors and are not
%% necessarily endorsed by the IEEE.
%%
%% This work is distributed under the LaTeX Project Public License (LPPL)
%% ( http://www.latex-project.org/ ) version 1.3, and may be freely used,
%% distributed and modified. A copy of the LPPL, version 1.3, is included
%% in the base LaTeX documentation of all distributions of LaTeX released
%% 2003/12/01 or later.
%% Retain all contribution notices and credits.
%% ** Modified files should be clearly indicated as such, including  **
%% ** renaming them and changing author support contact information. **
%%*************************************************************************


% *** Authors should verify (and, if needed, correct) their LaTeX system  ***
% *** with the testflow diagnostic prior to trusting their LaTeX platform ***
% *** with production work. The IEEE's font choices and paper sizes can   ***
% *** trigger bugs that do not appear when using other class files.       ***                          ***
% The testflow support page is at:
% http://www.michaelshell.org/tex/testflow/



%\documentclass[journal]{IEEEtran}
\documentclass[10pt,twocolumn,twoside]{IEEEtran}
%\documentclass[10pt,draftclsnofoot, onecolumn]{IEEEtran}
%
% If IEEEtran.cls has not been installed into the LaTeX system files,
% manually specify the path to it like:
% \documentclass[journal]{../sty/IEEEtran}





% Some very useful LaTeX packages include:
% (uncomment the ones you want to load)


% *** MISC UTILITY PACKAGES ***
%
%\usepackage{ifpdf}
% Heiko Oberdiek's ifpdf.sty is very useful if you need conditional
% compilation based on whether the output is pdf or dvi.
% usage:
% \ifpdf
%   % pdf code
% \else
%   % dvi code
% \fi
% The latest version of ifpdf.sty can be obtained from:
% http://www.ctan.org/pkg/ifpdf
% Also, note that IEEEtran.cls V1.7 and later provides a builtin
% \ifCLASSINFOpdf conditional that works the same way.
% When switching from latex to pdflatex and vice-versa, the compiler may
% have to be run twice to clear warning/error messages.






% *** CITATION PACKAGES ***
%
\usepackage{cite}
% cite.sty was written by Donald Arseneau
% V1.6 and later of IEEEtran pre-defines the format of the cite.sty package
% \cite{} output to follow that of the IEEE. Loading the cite package will
% result in citation numbers being automatically sorted and properly
% "compressed/ranged". e.g., [1], [9], [2], [7], [5], [6] without using
% cite.sty will become [1], [2], [5]--[7], [9] using cite.sty. cite.sty's
% \cite will automatically add leading space, if needed. Use cite.sty's
% noadjust option (cite.sty V3.8 and later) if you want to turn this off
% such as if a citation ever needs to be enclosed in parenthesis.
% cite.sty is already installed on most LaTeX systems. Be sure and use
% version 5.0 (2009-03-20) and later if using hyperref.sty.
% The latest version can be obtained at:
% http://www.ctan.org/pkg/cite
% The documentation is contained in the cite.sty file itself.






% *** GRAPHICS RELATED PACKAGES ***
%
\ifCLASSINFOpdf
   \usepackage[pdftex]{graphicx}
  % declare the path(s) where your graphic files are
  % \graphicspath{{../pdf/}{../jpeg/}}
  % and their extensions so you won't have to specify these with
  % every instance of \includegraphics
  % \DeclareGraphicsExtensions{.pdf,.jpeg,.png}
\else
  % or other class option (dvipsone, dvipdf, if not using dvips). graphicx
  % will default to the driver specified in the system graphics.cfg if no
  % driver is specified.
  % \usepackage[dvips]{graphicx}
  % declare the path(s) where your graphic files are
  % \graphicspath{{../eps/}}
  % and their extensions so you won't have to specify these with
  % every instance of \includegraphics
  % \DeclareGraphicsExtensions{.eps}
\fi
% graphicx was written by David Carlisle and Sebastian Rahtz. It is
% required if you want graphics, photos, etc. graphicx.sty is already
% installed on most LaTeX systems. The latest version and documentation
% can be obtained at: 
% http://www.ctan.org/pkg/graphicx
% Another good source of documentation is "Using Imported Graphics in
% LaTeX2e" by Keith Reckdahl which can be found at:
% http://www.ctan.org/pkg/epslatex
%
% latex, and pdflatex in dvi mode, support graphics in encapsulated
% postscript (.eps) format. pdflatex in pdf mode supports graphics
% in .pdf, .jpeg, .png and .mps (metapost) formats. Users should ensure
% that all non-photo figures use a vector format (.eps, .pdf, .mps) and
% not a bitmapped formats (.jpeg, .png). The IEEE frowns on bitmapped formats
% which can result in "jaggedy"/blurry rendering of lines and letters as
% well as large increases in file sizes.
%
% You can find documentation about the pdfTeX application at:
% http://www.tug.org/applications/pdftex





% *** MATH PACKAGES ***
%
\usepackage{amsmath}
\usepackage{amssymb}
\usepackage{mathrsfs} 
% A popular package from the American Mathematical Society that provides
% many useful and powerful commands for dealing with mathematics.
%
% Note that the amsmath package sets \interdisplaylinepenalty to 10000
% thus preventing page breaks from occurring within multiline equations. Use:
%\interdisplaylinepenalty=2500
% after loading amsmath to restore such page breaks as IEEEtran.cls normally
% does. amsmath.sty is already installed on most LaTeX systems. The latest
% version and documentation can be obtained at:
% http://www.ctan.org/pkg/amsmath





% *** SPECIALIZED LIST PACKAGES ***
%
\usepackage{algorithmic}
% algorithmic.sty was written by Peter Williams and Rogerio Brito.
% This package provides an algorithmic environment fo describing algorithms.
% You can use the algorithmic environment in-text or within a figure
% environment to provide for a floating algorithm. Do NOT use the algorithm
% floating environment provided by algorithm.sty (by the same authors) or
% algorithm2e.sty (by Christophe Fiorio) as the IEEE does not use dedicated
% algorithm float types and packages that provide these will not provide
% correct IEEE style captions. The latest version and documentation of
% algorithmic.sty can be obtained at:
% http://www.ctan.org/pkg/algorithms
% Also of interest may be the (relatively newer and more customizable)
% algorithmicx.sty package by Szasz Janos:
% http://www.ctan.org/pkg/algorithmicx

% *** THEOREM PACKAGES ***
\newtheorem{theorem}{Theorem}
\newtheorem{corollary}{Corollary}[theorem]
\newtheorem{assumption}{Assumption}

% *** ALIGNMENT PACKAGES ***
%
%\usepackage{array}
% Frank Mittelbach's and David Carlisle's array.sty patches and improves
% the standard LaTeX2e array and tabular environments to provide better
% appearance and additional user controls. As the default LaTeX2e table
% generation code is lacking to the point of almost being broken with
% respect to the quality of the end results, all users are strongly
% advised to use an enhanced (at the very least that provided by array.sty)
% set of table tools. array.sty is already installed on most systems. The
% latest version and documentation can be obtained at:
% http://www.ctan.org/pkg/array


% IEEEtran contains the IEEEeqnarray family of commands that can be used to
% generate multiline equations as well as matrices, tables, etc., of high
% quality.




% *** SUBFIGURE PACKAGES ***
\ifCLASSOPTIONcompsoc
  \usepackage[caption=false,font=normalsize,labelfont=sf,textfont=sf]{subfig}
\else
  \usepackage[caption=false,font=footnotesize]{subfig}
\fi
% subfig.sty, written by Steven Douglas Cochran, is the modern replacement
% for subfigure.sty, the latter of which is no longer maintained and is
% incompatible with some LaTeX packages including fixltx2e. However,
% subfig.sty requires and automatically loads Axel Sommerfeldt's caption.sty
% which will override IEEEtran.cls' handling of captions and this will result
% in non-IEEE style figure/table captions. To prevent this problem, be sure
% and invoke subfig.sty's "caption=false" package option (available since
% subfig.sty version 1.3, 2005/06/28) as this is will preserve IEEEtran.cls
% handling of captions.
% Note that the Computer Society format requires a larger sans serif font
% than the serif footnote size font used in traditional IEEE formatting
% and thus the need to invoke different subfig.sty package options depending
% on whether compsoc mode has been enabled.
%
% The latest version and documentation of subfig.sty can be obtained at:
% http://www.ctan.org/pkg/subfig




% *** FLOAT PACKAGES ***
%
%\usepackage{fixltx2e}
% fixltx2e, the successor to the earlier fix2col.sty, was written by
% Frank Mittelbach and David Carlisle. This package corrects a few problems
% in the LaTeX2e kernel, the most notable of which is that in current
% LaTeX2e releases, the ordering of single and double column floats is not
% guaranteed to be preserved. Thus, an unpatched LaTeX2e can allow a
% single column figure to be placed prior to an earlier double column
% figure.
% Be aware that LaTeX2e kernels dated 2015 and later have fixltx2e.sty's
% corrections already built into the system in which case a warning will
% be issued if an attempt is made to load fixltx2e.sty as it is no longer
% needed.
% The latest version and documentation can be found at:
% http://www.ctan.org/pkg/fixltx2e


%\usepackage{stfloats}
% stfloats.sty was written by Sigitas Tolusis. This package gives LaTeX2e
% the ability to do double column floats at the bottom of the page as well
% as the top. (e.g., "\begin{figure*}[!b]" is not normally possible in
% LaTeX2e). It also provides a command:
%\fnbelowfloat
% to enable the placement of footnotes below bottom floats (the standard
% LaTeX2e kernel puts them above bottom floats). This is an invasive package
% which rewrites many portions of the LaTeX2e float routines. It may not work
% with other packages that modify the LaTeX2e float routines. The latest
% version and documentation can be obtained at:
% http://www.ctan.org/pkg/stfloats
% Do not use the stfloats baselinefloat ability as the IEEE does not allow
% \baselineskip to stretch. Authors submitting work to the IEEE should note
% that the IEEE rarely uses double column equations and that authors should try
% to avoid such use. Do not be tempted to use the cuted.sty or midfloat.sty
% packages (also by Sigitas Tolusis) as the IEEE does not format its papers in
% such ways.
% Do not attempt to use stfloats with fixltx2e as they are incompatible.
% Instead, use Morten Hogholm'a dblfloatfix which combines the features
% of both fixltx2e and stfloats:
%
% \usepackage{dblfloatfix}
% The latest version can be found at:
% http://www.ctan.org/pkg/dblfloatfix




%\ifCLASSOPTIONcaptionsoff
%  \usepackage[nomarkers]{endfloat}
% \let\MYoriglatexcaption\caption
% \renewcommand{\caption}[2][\relax]{\MYoriglatexcaption[#2]{#2}}
%\fi
% endfloat.sty was written by James Darrell McCauley, Jeff Goldberg and 
% Axel Sommerfeldt. This package may be useful when used in conjunction with 
% IEEEtran.cls'  captionsoff option. Some IEEE journals/societies require that
% submissions have lists of figures/tables at the end of the paper and that
% figures/tables without any captions are placed on a page by themselves at
% the end of the document. If needed, the draftcls IEEEtran class option or
% \CLASSINPUTbaselinestretch interface can be used to increase the line
% spacing as well. Be sure and use the nomarkers option of endfloat to
% prevent endfloat from "marking" where the figures would have been placed
% in the text. The two hack lines of code above are a slight modification of
% that suggested by in the endfloat docs (section 8.4.1) to ensure that
% the full captions always appear in the list of figures/tables - even if
% the user used the short optional argument of \caption[]{}.
% IEEE papers do not typically make use of \caption[]'s optional argument,
% so this should not be an issue. A similar trick can be used to disable
% captions of packages such as subfig.sty that lack options to turn off
% the subcaptions:
% For subfig.sty:
% \let\MYorigsubfloat\subfloat
% \renewcommand{\subfloat}[2][\relax]{\MYorigsubfloat[]{#2}}
% However, the above trick will not work if both optional arguments of
% the \subfloat command are used. Furthermore, there needs to be a
% description of each subfigure *somewhere* and endfloat does not add
% subfigure captions to its list of figures. Thus, the best approach is to
% avoid the use of subfigure captions (many IEEE journals avoid them anyway)
% and instead reference/explain all the subfigures within the main caption.
% The latest version of endfloat.sty and its documentation can obtained at:
% http://www.ctan.org/pkg/endfloat
%
% The IEEEtran \ifCLASSOPTIONcaptionsoff conditional can also be used
% later in the document, say, to conditionally put the References on a 
% page by themselves.




% *** PDF, URL AND HYPERLINK PACKAGES ***
%
%\usepackage{url}
% url.sty was written by Donald Arseneau. It provides better support for
% handling and breaking URLs. url.sty is already installed on most LaTeX
% systems. The latest version and documentation can be obtained at:
% http://www.ctan.org/pkg/url
% Basically, \url{my_url_here}.




% *** Do not adjust lengths that control margins, column widths, etc. ***
% *** Do not use packages that alter fonts (such as pslatex).         ***
% There should be no need to do such things with IEEEtran.cls V1.6 and later.
% (Unless specifically asked to do so by the journal or conference you plan
% to submit to, of course. )


% correct bad hyphenation here
\hyphenation{op-tical net-works semi-conduc-tor}


\begin{document}
%
% paper title
% Titles are generally capitalized except for words such as a, an, and, as,
% at, but, by, for, in, nor, of, on, or, the, to and up, which are usually
% not capitalized unless they are the first or last word of the title.
% Linebreaks \\ can be used within to get better formatting as desired.
% Do not put math or special symbols in the title.
\title{Adaptive Online Distributed Optimal Control of Very-Large-Scale Robotic Systems}
%
%
% author names and IEEE memberships
% note positions of commas and nonbreaking spaces ( ~ ) LaTeX will not break
% a structure at a ~ so this keeps an author's name from being broken across
% two lines.
% use \thanks{} to gain access to the first footnote area
% a separate \thanks must be used for each paragraph as LaTeX2e's \thanks
% was not built to handle multiple paragraphs
%

\author{Pingping~Zhu,~\IEEEmembership{Member,~IEEE,}
        Chang~Liu,~\IEEEmembership{Member,~IEEE,}
        and~Silvia~Ferrari,~\IEEEmembership{Senior~Member,~IEEE}% <-this % stops a space
\thanks{P. Zhu, C. Liu, and S. Ferrari are with the Sibley School of Mechanical
	and Aerospace Engineering, Cornell University, Ithaca, NY 14853 USA (e-mail: pingping.zhu@cornell.edu; cl775@cornell.edu; sferrari@cornell.edu)}% <-this % stops a space
%\thanks{J. Doe and J. Doe are with Anonymous University.}% <-this % stops a space
%\thanks{Manuscript received April 19, 2005; revised August 26, 2015.}
}

% note the % following the last \IEEEmembership and also \thanks - 
% these prevent an unwanted space from occurring between the last author name
% and the end of the author line. i.e., if you had this:
% 
% \author{....lastname \thanks{...} \thanks{...} }
%                     ^------------^------------^----Do not want these spaces!
%
% a space would be appended to the last name and could cause every name on that
% line to be shifted left slightly. This is one of those "LaTeX things". For
% instance, "\textbf{A} \textbf{B}" will typeset as "A B" not "AB". To get
% "AB" then you have to do: "\textbf{A}\textbf{B}"
% \thanks is no different in this regard, so shield the last } of each \thanks
% that ends a line with a % and do not let a space in before the next \thanks.
% Spaces after \IEEEmembership other than the last one are OK (and needed) as
% you are supposed to have spaces between the names. For what it is worth,
% this is a minor point as most people would not even notice if the said evil
% space somehow managed to creep in.



% The paper headers
%\markboth{Journal of \LaTeX\ Class Files,~Vol.~14, No.~8, FEBRUARY~2020}%
%{Shell \MakeLowercase{\textit{et al.}}: Bare Demo of IEEEtran.cls for IEEE Journals}
% The only time the second header will appear is for the odd numbered pages
% after the title page when using the twoside option.
% 
% *** Note that you probably will NOT want to include the author's ***
% *** name in the headers of peer review papers.                   ***
% You can use \ifCLASSOPTIONpeerreview for conditional compilation here if
% you desire.




% If you want to put a publisher's ID mark on the page you can do it like
% this:
%\IEEEpubid{0000--0000/00\$00.00~\copyright~2015 IEEE}
% Remember, if you use this you must call \IEEEpubidadjcol in the second
% column for its text to clear the IEEEpubid mark.



% use for special paper notices
%\IEEEspecialpapernotice{(Invited Paper)}




% make the title area
\maketitle

% As a general rule, do not put math, special symbols or citations
% in the abstract or keywords.
\begin{abstract}
This paper presents an adaptive online distributed optimal control approach that is applicable to optimal planning for very-large-scale robotics systems in highly uncertain environments. This approach is developed based on the optimal mass transport theory. It is also viewed as an online reinforcement learning and approximate dynamic programming approach in the Wasserstein-GMM space, where a novel value functional is defined based on the probability density functions of robots and the time-varying obstacle map functions describing the changing environmental information. The proposed approach is demonstrated on the path planning problem of very-large-scale robotic systems where the approximated layout of obstacles in the workspace is incrementally updated by the observations of robots, and compared with some existing state-of-the-art approaches. The numerical simulation results show that the proposed approach outperforms these approaches in aspects of the average traveling distance and the energy cost.      


\end{abstract}

% Note that keywords are not normally used for peerreview papers.
\begin{IEEEkeywords}
Very-Large-Scale Robotic (VLSR), Adaptive Distributed Optimal Control (ADOC), Multi-Agent Reinforcement Learning (MARL), Path Planning.
\end{IEEEkeywords}






% For peer review papers, you can put extra information on the cover
% page as needed:
% \ifCLASSOPTIONpeerreview
% \begin{center} \bfseries EDICS Category: 3-BBND \end{center}
% \fi
%
% For peerreview papers, this IEEEtran command inserts a page break and
% creates the second title. It will be ignored for other modes.
\IEEEpeerreviewmaketitle



\section{Introduction}
% The very first letter is a 2 line initial drop letter followed
% by the rest of the first word in caps.
% 
% form to use if the first word consists of a single letter:
% \IEEEPARstart{A}{demo} file is ....
% 
% form to use if you need the single drop letter followed by
% normal text (unknown if ever used by the IEEE):
% \IEEEPARstart{A}{}demo file is ....
% 
% Some journals put the first two words in caps:
% \IEEEPARstart{T}{his demo} file is ....
% 
% Here we have the typical use of a "T" for an initial drop letter
% and "HIS" in caps to complete the first word.
\IEEEPARstart{W}{ith} the development of low-cost sensor, wireless communication, and embedded computational systems, very-large-scale robotic (VLSR) systems comprised of hundreds of autonomous
robots are becoming a promising research area, where a large number of mobile robots can cooperate to complete a given task more efficiently and effectively than a single or a few mobile robots. In the past two decades, significant progress of VLSR systems has been made in different research methodologies, including the optimal control (OC) \cite{StengelOptimalControl1994,ParkerPathPlanningMotionCoordination2009,RuddIndirectDOC2013,FoderaroDOC2014,FoderaroDOC2016,FerrariDOCtutorial2016,RuddIndirectDOC2017} and multi-agent reinforcement learning (MARL) \cite{YangMeanFieldMARL2018,KhanScalableCentralizedDMARL2018,HuttenrauchDeepRLforSR2019,HernandezSurveyMultiagentDRL2019,CarmonaLinearQuadraticMeanField2019,CarmonaModelFreeMeanFieldRL2019,ZhangReviewMARL2019,RenRepresentedVFforLSMARL2020}. Some similar or identical tasks and applications of VLSR systems are considered and solved from different points of view, including the multi-agent path planning \cite{FerrariMultirobotMotionCoordination1998,BennewitzDecoupledPathPlanning2002} and the control of swarm robotics (SR) \cite{YogeswaranSwarmRoboticsReview2010,RubensteinProgrammableSelfAssembly2014,BandyopadhyayProbabilisticSwarmGuidanceOT2014,KrishnanDistributedOTforSwarm2018,BayindirSwarmRoboticsReview2016,NedjahSwarmRoboticsReview2019}.

One significant challenge of VLSR systems is the scalability issue, which is also referred to as the combinatorial nature of MARL in \cite{HernandezSurveyMultiagentDRL2019}. Even in a given and known environment, the optimization of the plans of $N$ cooperative robots is PSPACE-hard \cite{ParkerPathPlanningMotionCoordination2009}, which is not acceptable for a VLSR system with a very large $N$. Thus, several approaches are proposed to represent the macroscopic state of the VLSR system instead of the microscopic state of every individual robot, including the Nash Certainty Equivalence (NCE) or mean field, \cite{HuangNCE2006,HuangNEC2010}, and the distributed optimal control (DOC) \cite{RuddIndirectDOC2013,FoderaroDOC2014,FoderaroDOC2016,FerrariDOCtutorial2016,RuddIndirectDOC2017} approaches.  In the NCE approach, the macroscopic state is the mass of all robots, and the key idea is to specify a certain consistency relationship between the individual robot kinodynamics and the mass effect, where the mean field coupling is produced by the averaging of the microscopic robot kinodynamics and costs. While, in the DOC approach, the macroscopic state is the time-varying probability density function (PDF) of robots and the cost does not depend on the microscopic state of every individual robot, and, thus, the coupling is not needed. Several variants of the DOC approach have been developed to solve the path planning problem of VLSR systems \cite{RuddIndirectDOC2013,FoderaroDOC2014,FoderaroDOC2016,FerrariDOCtutorial2016,RuddIndirectDOC2017}, where the optimal trajectory of the robot PDFs from a given initial distribution to a given target distribution is generated as a reference via nonlinear programming (NLP) to guide these robots to travel in the region of interest (ROI) with fixed and known  obstacles. 

Another significant challenge of the VLSR systems is the adaptability issue. In many emerging systems, the goal is to control multiple assets and resources in the presence of significant uncertainties that cannot be modeled a priori. The VLSR systems are required to respond to significant changes in the environment and target information that occur over long time scales and re-plan and adapt to local uncertainties while preventing network-level instabilities. Although the DOC approach has been shown to overcome the computational complexity associated with classical optimal control by representing the state of the VLSR systems by the robot PDFs, %and generate the optimal trajectory of robot PDFs for a given layout of obstacles and given initial and target robot distributions, 
its computational complexity is relatively high because of the NLP on the continuous spaces of robot microscopic state and control. 
It is not feasible for the DOC approaches to re-plan the optimal trajectory of robot PDFs in real time. Thus, existing DOC approaches do not satisfy the requirements in the applications involved in the  uncertain and changing environments. Recently, a model-free MARL approach was proposed based on mean field control (MFC) \cite{CarmonaModelFreeMeanFieldRL2019}, where the macroscopic state of the VLSR system is also the robot PDF and the rewards or Lagrangian terms depend on both microscopic and macroscopic states. This MARL approach can be recast as a Markov decision process (MDP) on the Wasserstein space of measures and implemented by learning a  deterministic control law offline. The control law is a functional with the arguments of the macroscopic and microscopic states. However, the value functional and the corresponding control law cannot be approximated efficiently, especially for the VLSR systems with continuous states and controls. Thus, this MARL approach cannot be applied in the uncertain and changing environments neither.       

In this paper, an adaptive DOC (ADOC) theory and approach for VLSR systems are proposed to carry out online cooperative sensing and navigation tasks in highly uncertain environments. Since the similarities between the optimal control and reinforcement learning, the proposed ADOC approach can be treated as a reinforcement learning-approximate dynamic programming (RL-ADP) approach \cite{LewisRLAPD2013} for which the environment has to be explored online.



Compared to existing works, this paper makes the following contributions: 1) The time-varying  environmental information is expressed by a map function and reflected in the Lagrangian term (or reward). 2) A new value functional is defined based on the time-varying environmental information. 3) The ADOC approach is formulated in a Wasserstein-GMM space based on the optimal mass transport (OMT) theorem where the robot PDFs are all assumed to be Gaussian mixture distributions. 4) The ADOC approach is implemented online via solving linear programming (LP) problems in a subspace of the Wasserstein-GMM space.  5) The effectiveness of the proposed ADOC approach is demonstrated on the problem of VLSR systems path planning with uncertain environmental information, and the results show that the ADOC approach outperforms the other three existing state-of-the-art approaches. %Due to the limited space, some simulation results and the comparison of computational complexity are omitted in the paper. The interested readers are referred to \cite{ZhuADOCArxiv2020} for more details.

  

%\subsection{Subsection Heading Here}
%Subsection text here.
%
%% needed in second column of first page if using \IEEEpubid
%%\IEEEpubidadjcol
%
%\subsubsection{Subsubsection Heading Here}
%Subsubsection text here.


% An example of a floating figure using the graphicx package.
% Note that \label must occur AFTER (or within) \caption.
% For figures, \caption should occur after the \includegraphics.
% Note that IEEEtran v1.7 and later has special internal code that
% is designed to preserve the operation of \label within \caption
% even when the captionsoff option is in effect. However, because
% of issues like this, it may be the safest practice to put all your
% \label just after \caption rather than within \caption{}.
%
% Reminder: the "draftcls" or "draftclsnofoot", not "draft", class
% option should be used if it is desired that the figures are to be
% displayed while in draft mode.
%
%\begin{figure}[!t]
%\centering
%\includegraphics[width=2.5in]{myfigure}
% where an .eps filename suffix will be assumed under latex, 
% and a .pdf suffix will be assumed for pdflatex; or what has been declared
% via \DeclareGraphicsExtensions.
%\caption{Simulation results for the network.}
%\label{fig_sim}
%\end{figure}

% Note that the IEEE typically puts floats only at the top, even when this
% results in a large percentage of a column being occupied by floats.


% An example of a double column floating figure using two subfigures.
% (The subfig.sty package must be loaded for this to work.)
% The subfigure \label commands are set within each subfloat command,
% and the \label for the overall figure must come after \caption.
% \hfil is used as a separator to get equal spacing.
% Watch out that the combined width of all the subfigures on a 
% line do not exceed the text width or a line break will occur.
%
%\begin{figure*}[!t]
%\centering
%\subfloat[Case I]{\includegraphics[width=2.5in]{box}%
%\label{fig_first_case}}
%\hfil
%\subfloat[Case II]{\includegraphics[width=2.5in]{box}%
%\label{fig_second_case}}
%\caption{Simulation results for the network.}
%\label{fig_sim}
%\end{figure*}
%
% Note that often IEEE papers with subfigures do not employ subfigure
% captions (using the optional argument to \subfloat[]), but instead will
% reference/describe all of them (a), (b), etc., within the main caption.
% Be aware that for subfig.sty to generate the (a), (b), etc., subfigure
% labels, the optional argument to \subfloat must be present. If a
% subcaption is not desired, just leave its contents blank,
% e.g., \subfloat[].


% An example of a floating table. Note that, for IEEE style tables, the
% \caption command should come BEFORE the table and, given that table
% captions serve much like titles, are usually capitalized except for words
% such as a, an, and, as, at, but, by, for, in, nor, of, on, or, the, to
% and up, which are usually not capitalized unless they are the first or
% last word of the caption. Table text will default to \footnotesize as
% the IEEE normally uses this smaller font for tables.
% The \label must come after \caption as always.
%
%\begin{table}[!t]
%% increase table row spacing, adjust to taste
%\renewcommand{\arraystretch}{1.3}
% if using array.sty, it might be a good idea to tweak the value of
% \extrarowheight as needed to properly center the text within the cells
%\caption{An Example of a Table}
%\label{table_example}
%\centering
%% Some packages, such as MDW tools, offer better commands for making tables
%% than the plain LaTeX2e tabular which is used here.
%\begin{tabular}{|c||c|}
%\hline
%One & Two\\
%\hline
%Three & Four\\
%\hline
%\end{tabular}
%\end{table}


% Note that the IEEE does not put floats in the very first column
% - or typically anywhere on the first page for that matter. Also,
% in-text middle ("here") positioning is typically not used, but it
% is allowed and encouraged for Computer Society conferences (but
% not Computer Society journals). Most IEEE journals/conferences use
% top floats exclusively. 
% Note that, LaTeX2e, unlike IEEE journals/conferences, places
% footnotes above bottom floats. This can be corrected via the
% \fnbelowfloat command of the stfloats package.



\section{Problem Formulation}
\label{sec:Problem_Formulation}

Consider the problem of adaptively planning the trajectories of a VLSR system comprised of $N$ cooperative robots deployed according to a known distribution and tasked with forming a given target distribution through a large and obstacle-deployed region of interest (ROI), denoted by $\mathcal{W}\subset \mathbb{R}^2$. In the ROI, there are $N_B$ obstacles, $\mathcal{B}_1,\ldots,\mathcal{B}_{N_B} \subset\mathcal{W}$. The location, the geometry, and the number of obstacles are unknown a priori.  The actual layout of obstacles is the union of all obstacles denoted by $\mathcal{B} = \cup_{n_b = 1}^{N_{B}}\mathcal{B}_{n_b}$, and is unknown a priori as well. However, an approximate map is available for the VLSR system to indicate the layout of obstacles in the ROI denoted by $\hat{\mathcal{B}}_0 = \hat{\mathcal{B}}(t_0)\subset \mathcal{W}$ at the initial time $t_0$.  The approximate layout of obstacles $\hat{\mathcal{B}}(t) \subset \mathcal{W}$ can be updated by observations from all robots. The approximate layout of obstacles is represented by a time-varying function, $m(\mathbf{x},t)$ defined on  $\mathcal{W} \times \mathbb{R}$, referred to as obstacle map function. Let  $\mathscr{M}$ be a collection of all functions $m(\cdot,t)$ representing all possible layouts of obstacles in the ROI, such that $m(\cdot,t)\in \mathscr{M}$.

% There are several options for defining this map function, including traditional occupancy, the Gaussian process occupancy map, and the Hilbert occupancy map \cite{RamosHilbertMap2016}. For simplicity, the map function $m(\mathbf{x},t)$ at a certain time $t$ is defined as a binary occupancy map defined on $\mathcal{W}$. Given a Hilbert occupancy map $h(\mathbf{x},t)$ defined on $\mathcal{W}$, which can be updated incrementally by observations from sensors \cite{ZhuGDM2019,MorelliMapping2019}, the corresponding 
%map function is obtained by
%\begin{equation}
%m(\mathbf{x},t) = 
%\begin{cases}
%1, \quad \text{if } h(\mathbf{x},t) > 0.5 \\
%0, \quad \text{otherwise}
%\end{cases}
%\label{eq:map_function}
%\end{equation} 




The dynamics of each robot are governed by a stochastic
differential equation (SDE),
\begin{align}
\label{eq:dynamics}
\dot{\mathbf{x}}_n(t) &= \mathbf{f}[\mathbf{x}_n(t), \mathbf{u}_n(t),t] + \mathbf{w}(t),\\
\mathbf{x}_n(t_0) &= \mathbf{x}_{n_0}, \hspace{10pt} n=1,...,N,
\end{align}
where $\mathbf{x}_n(t) \in \mathcal{W}$ denotes the $n$th robot's configuration at time $t$,
$\mathbf{u}_n(t) \in \mathcal{U}$ denotes the $n$th robot action or control, and
$\mathbf{x}_{n_0}$ denotes the initial configuration of the $n$th robot at the initial time
$t_0$. The robot dynamics in (\ref{eq:dynamics}) are characterized by an
additive Gaussian white noise, denoted by $\mathbf{w}(t) \in \mathbb{R}^2$. In this paper, for simplicity, assume that the position of the $n$th robot, $\mathbf{x}_n(t)$, can be measured accurately by an equipped GPS. %and the position measurement at time $t$ is denoted by $\hat{\mathbf{x}}_n(t)$. In this paper, the symbol $\hat{\cdot}$ is applied to denote a measurement variable. For simplicity in this paper, assume that the sensors can obtain accurate position measurements, such that $\hat{\mathbf{x}}_n(t) = \mathbf{x}_n(t)$.  

To update the layout of obstacles, all robots are equipped
with identical omnidirectional range sensors. The field of view (FOV) of the $n$th sensor denoted by $\mathcal{F}_n(t) \subset \mathcal{W}$ can be represented by a circle of radius $r$ around $\mathbf{x}_n(t)$. Then, the whole FOV of all robots at time $t$ is denoted by $\mathcal{F}(t)=\cup_{n=1}^N \mathcal{F}_n(t)$. Assume that all robots can share information at any time, the obstacle at the position $\mathbf{x}\in\mathcal{W}$ is observed and updated at time $t$ if and only if $\mathbf{x} \in \mathcal{F}(t)$.


%then the layout of obstacles $\mathcal{B}(t) \subset \mathcal{W}$ can be updated by observations from all robots. The obstacle at the position $\mathbf{x}\in\mathcal{W}$ is observed and updated at time $t$ if and only if $\mathbf{x} \in \mathcal{F}(t)$. The layout of obstacles is represented by a time-varying function $m(\mathbf{x},t) \in \mathscr{M}$ referred to as obstacle map function, where $\mathscr{M}$ is a collection of all functions $m(\mathbf{x},t)$ representing all possible layouts of obstacles in the ROI. The definition of the map function is described in detail in Section \ref{sec:Distributed_Mapping}.

%For simplicity, a time-varying indicator function $b(\mathbf{x},t) =\boldsymbol{1}_{\mathcal{B}(t)}(\mathbf{x},t) \in \mathscr{B}$, which is referred to as obstacle function, is applied to represent the layout of obstacles, such that 
%\begin{equation}
% b(\mathbf{x},t) = \boldsymbol{1}_{\mathcal{B}(t)}(\mathbf{x},t) = \begin{cases}
%1 & \text{ if } \mathbf{x} \in \mathcal{B}(t)\\
%0 & \text{ if } \mathbf{x} \notin \mathcal{B}(t)
%\end{cases}
%\label{eq:obstacles_func}
%\end{equation} 
%where $\mathscr{B}$ is a collection of all functions $b(\mathbf{x},t)$ representing all possible layouts of obstacles in the ROI. There are many other options to represent the layout of obstacles, e.g. the Hilbert map function [ref], which can also be applied straightforwardly in the proposed approach. 

Since the robot positions, $\mathbf{x}_n(t) \in \mathcal{W}$, $n=1,\ldots,N$, at time $t $, are time-varying continuous vectors, assume that there exists a time-varying continuous probability distribution associated with a probability density function (PDF), $\wp(\mathbf{x},t) \in \mathscr{P}(\mathcal{W})$, such that these robot positions at time $t$ can be treated as random samples generated according to $\wp(\mathbf{x},t)$, where $\mathscr{P}(\mathcal{W})$ is the space of PDFs defined on the ROI $\mathcal{W}$. Denote the initial and target PDFs of robot positions by $\wp_0$ and $\wp_{targ}$, respectively. Then, the macroscopic objective of the VLSR system in this paper is to generate and adaptively update a trajectory of PDFs of robot positions from $\wp_0$ to $\wp_{targ}$ while preserving the robots from collisions with each other and obstacles which are incrementally observed and updated. Let $t_f$ denote the final time of the task. This macroscopic objective can be achieved by minimizing the following cost function,
\begin{equation}
J=\phi[\wp(t_f),\wp_{targ}]+\int_{t_0}^{t_f}\mathscr{L}[\wp(\mathbf{x},t),m(\mathbf{x},t)]dt
\label{eq:costFunc_cont}
\end{equation}
where the functional terms, $\phi[\wp(t_f),\wp_{targ}]$ and  $\mathscr{L}[\wp(\mathbf{x},t),m(\mathbf{x},t)]$, indicate the final term and the Lagrangian term, respectively. Because the layout of obstacles is updated according to the observations of sensors incrementally,  it is not guaranteed that the task can be completed within a given time period $[t_0, t_f]$. In other words, the final time step, $t_f$, cannot be given in advance. The task stops if and only if the robots achieve the target distribution.




%%%%%%%%%%%%%%%%%%%%%%%%%%% Delete for short %%%%%%%%%%%%%%%%%%%%%%%%%%%%%%%%%%%%%%%
\section{Background on Occupancy Mapping}
\label{sec:Distributed_Mapping}

As mentioned in Section \ref{sec:Problem_Formulation}, the approximate layout of obstacles is represented by a time-varying obstacle map function, $m(\cdot,t) \in \mathscr{M}$. There are several options for defining this map function, including traditional occupancy, the Gaussian process occupancy map, and the Hilbert occupancy map. For simplicity, the map function $m(\mathbf{x},t)$ at a certain time $t$ is defined as a binary occupancy map defined on the ROI $\mathcal{W}$, which is generated from the Hilbert occupancy map $h(\mathbf{x},t) \in [0,\, 1]$ \cite{RamosHilbertMap2016}.

A Hilbert occupancy map is a continuous probability map developed by formulating the mapping problem as a binary classification task. Let $\mathbf{x} \in \mathcal{W}$ be any point in ROI and $Y = \{0,1 \}$  be defined as a categorical random variable reflecting if the position $\mathbf{x}$ is occupied by obstacles, such that
\begin{equation}
Y = \begin{cases}
1, \quad \text{if } \mathbf{x} \in \mathcal{B}\\
0, \quad \text{otherwise}
\end{cases}
\end{equation} 
The Hilbert map $h(\mathbf{x},t)$ describes the probabilities $P(Y = 1|\mathbf{x}\in\mathcal{W})$ at the position x at the certain time $t$, as follows,
\begin{equation}
h(\mathbf{x},t) = P(Y = 1|\mathbf{x}\in\mathcal{W}) = \frac{e^{f(\mathbf{x},t)}}{1 + e^{f(\mathbf{x},t)}}
\end{equation}
where $P(Y = 1|\mathbf{x}\in\mathcal{W})$ indicates the probability that the location $\mathbf{x}\in \mathcal{W}$ is occupied by obstacles, and  $f(\mathbf{x},t)$ is a function defined on $\mathcal{W}$ at the certain time $t$. The function  $f(\mathbf{x},t)$ is learned and updated from all obtained observations up to time $t$ by minimizing a loss function of negative log-likelihood (NLL). Since the Hilbert occupancy map learning is not the key contribution, its implementation details are omitted in this paper. The interested readers are referred to \cite{ZhuGDM2019,MorelliMapping2019} for more details. Based on the updated Hilbert occupancy map $h(\mathbf{x},t)$, the obstacle map function is defined as a binary function, such that
\begin{equation}
m(\mathbf{x},t) = 
\begin{cases}
1, \quad \text{if } h(\mathbf{x},t) > 0.5 \\
0, \quad \text{otherwise}
\end{cases}
\label{eq:map_function}
\end{equation}  
%%%%%%%%%%%%%%%%%%%%%%%%%%% Delete for short %%%%%%%%%%%%%%%%%%%%%%%%%%%%%%%%%%%%%%%

%%%%%%%%%%%%%%%%%%%%% Delete for short    %%%%%%%%%%%%%%%%%%%%%%%%%%%%%%%
The example of the Hilbert function occupancy map $h(\mathbf{x},t)$ and the corresponding obstacle map functions $m(\mathbf{x},t)$ at three different times are  presented in Fig. \ref{fig:HM}. 

\begin{figure*}[htp]
	\centering
	\subfloat[]{\includegraphics[width=2in]{fig_TrajHM_100}}
	\hfil
	\subfloat[]{\includegraphics[width=2in]{fig_TrajHM_400}}
	\hfil
	\subfloat[]{\includegraphics[width=2in]{fig_TrajHM_700}}
	\hfil
	\subfloat[]{\includegraphics[width=2in]{fig_TrajBW_100}}
	\hfil
	\subfloat[]{\includegraphics[width=2in]{fig_TrajBW_400}}
	\hfil
	\subfloat[]{\includegraphics[width=2in]{fig_TrajBW_700}}
	\caption{The examples of the Hilbert occupancy map $h(\mathbf{x},t)$ and the corresponding obstacle map functions $m(\mathbf{x},t)$ at three different times. The sub-figures on the first row (a)-(c) represent the Hilbert occupancy map, and the sub-figures on the second row (d)-(f) represent the binary obstacle map function. The white polygons on the first row indicate the initial knowledge of the obstacles. }
	\label{fig:HM}
\end{figure*}
%%%%%%%%%%%%%%%%%%%%% Delete for short    %%%%%%%%%%%%%%%%%%%%%%%%%%%%%%%




\section{Adaptive DOC Approach}
\label{sec:Adaptive_DOC_Approach}

In this section, the problem described in (\ref{eq:costFunc_cont}) is discretized in the temporal scale, and an adaptive DOC approach is proposed in the discrete time framework. Given a fixed time interval $0 < \triangle t \ll (t_f - t_0)$,  the task time interval $[t_0,t_f]$ can be discretized into $T_f = (t_f - t_0) / \triangle t$ equal time steps. Let $t_k = t_0 + k \triangle t$ denote the discrete-time index with $k=1,\ldots,T_f$. Then, like the DOC approach \cite{RuddIndirectDOC2013,FoderaroDOC2014,FoderaroDOC2016,FerrariDOCtutorial2016,RuddIndirectDOC2017}, the macroscopic cost function in (\ref{eq:costFunc_cont}) can be rewritten by
\begin{equation}
J=\phi(\wp_{T_f},\wp_{targ})+\sum_{k=0}^{T_f - 1} \mathscr{L}(\wp_k,m_k)
\label{eq:discrete_cost_func}
\end{equation}   
where $\wp_k = \wp(\mathbf{x},t_k)$ and $m_k = m(\mathbf{x},t_k)$ are both functions defined on $\mathcal{W}$, and $\wp_{0}$ indicates the initial robot PDF.

Because the obstacle map function is time-varying and cannot be predicted, the macroscopic objective described in (\ref{eq:discrete_cost_func}) cannot be solved by directly optimizing the trajectories of robot PDFs to minimize the objective function like the DOC approaches \cite{RuddIndirectDOC2013,FoderaroDOC2014,FoderaroDOC2016,FerrariDOCtutorial2016,RuddIndirectDOC2017}. In this paper, the optimization of the trajectory of robot PDFs is modeled as an RL-ADP problem \cite{LewisRLAPD2013}, and the objective function is reformulated by
\begin{equation}
J \triangleq \phi(\wp_{T_f},\wp_{targ} ) + \sum_{k=0}^{T_f -1} \mathscr{L} \left[\wp_k,m_k,\mathcal{C}(\wp_k,m_k)\right] 
\label{eq:RL_cost_func}
\end{equation}           
where $\mathcal{C} : \mathscr{P} \times \mathscr{M} \mapsto \mathscr{P}$ is the functional control law, such that
\begin{equation}
\wp_{k+1} \triangleq \mathcal{C}(\wp_k,m_k)
\label{eq:functional_control_law}
\end{equation}
%The functional Lagrangian terms $\mathscr{L}(\wp_k,m_k)$ and $\mathscr{L} (\wp_k,m_k,\mathcal{C})$ are exchangeable in this paper. 
In addition, the final time step $T_f$ is not specified in advance. The final time step $T_f$ tends to infinity if the task cannot be completed.  

It is worthy to note that because the output of the functional control law in (\ref{eq:functional_control_law}) is the robot PDF at the $(k+1)$th time step, $\wp_{k+1}$ is the functional control and is also the next functional macroscopic state in the VLSR system. In the rest of this paper, the term $\mathscr{L} \left[\wp_k,m_k,\mathcal{C}(\wp_k,m_k)\right]$ is abbreviated as $\mathscr{L} (\wp_k,m_k,\mathcal{C})$. 

The goal of the RL-ADP problem is to adaptively determine an optimal functional control law of robot PDFs to minimize the objective function defined in (\ref{eq:RL_cost_func}). Specifically, the functional control law, $\mathcal{C} (\cdot,m_k) : \mathscr{P} \mapsto \mathscr{P}$, is updated online according to the obstacle map function $m_k$ at the $k$th time step. Then, an optimal robot PDF $\wp_{k+1}$ is generated by the updated functional control law, according to (\ref{eq:functional_control_law}). Finally, the robots can navigate from the initial distribution to target distribution according to the corresponding robot PDF $\wp_{k+1}$ by utilizing the microscopic control, which will be described in Section \ref{subsec:Microscopic_Control_Law}.   
% Moreover,  an optimal trajectory of robot PDFs $\mathbf{P}_k = [\wp_1,\ldots,\wp_k]$, $1 \leq k \leq T_f$, is generated by the optimal functional control law according the incrementally obtained obstacles functions $\mathbf{B}_k = [b_1,\ldots,b_k]$. Therefore, the objective of the VLSR system  is, first, to online find a map from trajectory of observed obstacle functions $\mathbf{B}_k$ to the optimal trajectory of robot PDFs $\mathbf{P}_k$, which is referred to as the optimal policy $\boldsymbol{\Pi}_k$ and expressed by
%\begin{align}
%\boldsymbol{\Pi}_k : \mathscr{B}^k \mapsto \mathscr{P}^k
%\end{align}
%Once the functional control law is obtained adaptively at the $k$th time step, the robots can navigate from the initial distribution to target distribution according to the corresponding robot PDF $\wp_k$ by utilizing the microscopical control.    

\subsection{Value Functional}
\label{subsec:Value_Function}
Unlike the traditional RL-ADP problem \cite{LewisRLAPD2013}, the functional control law in (\ref{eq:RL_cost_func}) is also dependent on the obstacle map function $m_k$ at the $k$th time step, which is updated by the incremental observations from all sensors. Assume that the obstacle map functions are static if the update of obstacles is not available. Thus, let $\mathbf{M}_k = [m_0,\ldots,m_{k-1},m_k,\ldots,m_k]$ denote a $1 \times T_f$ vector of obstacle map functions obtained at the $k$th time step. Here, $\mathbf{M}_k(\tau) = m_{\tau}$, $0 \leq \tau \leq k$, represent the observed and updated obstacle map functions up to the $k$th time step, and $\mathbf{M}_k(\tau) \equiv m_k$, $k < \tau \leq T_f-1$, represent the predicted obstacle map functions after the $k$th time step. 
Let $\mathcal{C}_{\tau}$, $\tau \leq k$, indicate the updated control law at the $\tau$th time step.
Thus, at the $k$th time step, the value functional associated to the control law $\mathcal{C}_k$ can be defined by   
\begin{align}
& \mathcal{V}_k(\wp_l,\mathbf{M},\mathcal{C}_k,l \vert \mathcal{C}_0,\ldots,\mathcal{C}_{k-1})  \nonumber\\
\triangleq & \begin{cases}
\phi(\wp_{T_f},\wp_{targ}) + \sum_{\tau=l}^{k -1} \mathscr{L}(\wp_{\tau},\mathbf{M}(\tau),\mathcal{C}_{\tau})\\
+ \sum_{\tau=k}^{T_f -1} \mathscr{L}(\wp_{\tau},\mathbf{M}(\tau),\mathcal{C}_k),
& 0 \leq l < k \\
\\
\phi(\wp_{T_f},\wp_{targ}) + \sum_{\tau=l}^{T_f -1} \mathscr{L}(\wp_{\tau},\mathbf{M}(\tau),\mathcal{C}_k), &  
k \leq l < T_f \\
\\
\phi(\wp_{T_f},\wp_{targ}), & l=T_f
\end{cases} 
\label{eq:VF_m_k}    
\end{align}
where $\mathbf{M}$ indicate the vector of obstacle map functions at any time step. 
For simplicity, the value functional is abbreviated as $\mathcal{V}_k(\wp_l,\mathbf{M})$ or $\mathcal{V}_k(\wp_l,\mathbf{M}, \mathcal{C}_k)$. 
Besides, because $\mathbf{M}_k(\tau) \equiv m_k$ for $k \leq \tau$, the term $\mathcal{V}_k(\wp_l,\mathbf{M}_k)$ is replaced by $\mathcal{V}_k(\wp_l,m_k)$ or $\mathcal{V}_k(\wp_l,m_k,\mathcal{C}_k)$ for $k \leq l < T_f$. 
 
Furthermore, a state-action functional, referred to as the Q-functional, is defined for $k \leq l < T_f$, such that
\begin{align}
\mathcal{Q}_k(\wp_l,m_k,\wp_{l+1}) &\triangleq \mathscr{L}(\wp_l,m_k,\wp_{l+1}) + \mathcal{V}_{k}(\wp_{l+1},m_k), \nonumber\\ 
&  k \leq l < T_f 
\label{eq:Q_m_k}
\end{align}
The Q-functional is a prediction of the cost-to-go from the robot PDF $\wp_l$ to $\wp_{T_f}$ at the $k$th time step, where the obstacle map function is fixed as $m_k$ and  the functional control law $\mathcal{C}_k(\cdot,m_k)$ is applied.
%%%%%%%%%%%%%%%%%%%%%%%%%%%%%%%% Delete for short %%%%%%%%%%%%%%%%%%%%%%%%%%%%%%%%%%%
%For simplicity, first, consider the case where the layout is known precisely in advance and is represented by a constant function $m$, such that  $m_k \equiv m$ for $k = 0,\ldots,T_f$, which is the case of the DOC problems \cite{RuddIndirectDOC2013,FoderaroDOC2014,FoderaroDOC2016,FerrariDOCtutorial2016,RuddIndirectDOC2017}. Thus, given a functional control law $\mathcal{C}_m(\wp_k) = \mathcal{C}(\wp_k,m)$, the value functional can be expressed by
%\begin{align} 
%\mathcal{V}_{\mathcal{C}_m}(\wp_k) & \triangleq 
%\phi(\wp_{T_f},\wp_{targ}) + \sum_{\tau=k}^{T_f -1} \mathscr{L}(\wp_{\tau},m,\mathcal{C}_m), \nonumber\\
%& k = 0,\ldots,T_f
%\label{eq:VF_fixed_m}
%\end{align}
%which represents the cost-to-go from the robot PDF $\wp_k$ at the $k$th time step to the target distribution $\wp_{targ}$ given the fixed obstacle function $m$. Here, the subscript ``$\mathcal{C}_m$" in the term ``$\mathcal{V}_{\mathcal{C}_m}$" indicates that the functional control law $\mathcal{C}_m$ is applied to obtain the value functional.  
%Then, consider the case of the time-varying obstacle function $m_k$, which is updated by the incremental observations from all sensors.
% 
%Given a functional control law $\mathcal{C}(\wp_k,m_k)$, the functional value functional can be defined as
%by straightforwardly modifying (\ref{eq:VF_fixed_m}), such that
%\begin{align}
%& \mathcal{V}_k(\wp_l,\mathbf{M}_k,\mathcal{C}_k)  \nonumber\\
%\triangleq & \begin{cases}
%\phi(\wp_{T_f},\wp_{targ}) + \sum_{\tau=l}^{k -1} \mathscr{L}(\wp_{\tau},m_{\tau},\mathcal{C}_{\tau})\\
%+ \sum_{\tau=k}^{T_f -1} \mathscr{L}(\wp_{\tau},m_k,\mathcal{C}_k),
%& 0 \leq l < k \\
%%\\
%\phi(\wp_{T_f},\wp_{targ}) + \sum_{\tau=l}^{T_f -1} \mathscr{L}(\wp_{\tau},m_k,\mathcal{C}_k), &  
%k \leq l < T_f \\
%%\\
%\phi(\wp_{T_f},\wp_{targ}), & k=T_f
%\end{cases} 
%\label{eq:VF_m_k}    
%\end{align}
%where $\mathcal{C}_k(\wp_{\tau}) = \mathcal{C}(\wp_{\tau},m_k)$ and $\mathcal{C}_{\tau}(\wp_{\tau}) = \mathcal{C}(\wp_{\tau},m_{\tau})$ are the functional control laws based on the obstacle function $m_k$ and $m_l$, respectively, and $\mathbf{M}_k = [m_0,\ldots,m_k]$ is all of the available obstacle functions up to the $k$th time step. The value functional with three arguments defined in (\ref{eq:VF_m_k}) represents the cost-to-go from the robot PDF $\wp_l$ to the target distribution $\wp_{targ}$ given all available obstacle functions $\mathbf{M}_k$ at the $k$th time step. Since the approach is designed to implement online and the control law cannot be updated for the past robot PDFs $\{\wp_l\}_{l=0}^{k-1}$, the evaluation of the value functional $\mathcal{V}_k(\wp_l,\mathbf{M}_k,\mathcal{C}_k)$ is not only dependent on the control law $\mathcal{C}_k$, but also dependent on the control law $\mathcal{C}_{\tau}$ for $l \leq \tau < k$ when $l < k$. Thus, the subscript of the value functional is $k$ instead of $\mathcal{C}_k$. Moreover, the argument, control law $\mathcal{C}_k$, is also omitted when no confusion is possible, such that  $\mathcal{V}_k(\wp_l,\mathbf{M}_k,\mathcal{C}_k)$ is replaced by $\mathcal{V}_k(\wp_l,\mathbf{M}_k)$ for short. 

%Similarly to the value functional, a Q-functional is defined as a prediction of the cost-to-go at the $k$th time step if the obstacle function is fixed as $m_k$ and  the functional control law $\mathcal{C}_k$ is applied to all robot PDFs from $\wp_l$ to $\wp_{T_f}$. 
%%%%%%%%%%%%%%%%%%%%%%%%%%%%%%%% Delete for short %%%%%%%%%%%%%%%%%%%%%%%%%%%%%%%%%%%
%Because the control, $\wp_{l+1} = \mathcal{C}(\wp_l,m_l) = \mathcal{C}_l(\wp_l)$ for $0\leq l \leq k$, is not dependent on the functional control law $\mathcal{C}_k(\cdot) = \mathcal{C}(\cdot,m_k)$ at the $k$th time step.  
%Furthermore, a state-action function, referred to as the Q-functional, is defined for $k \leq l < T_f$, such that
%\begin{align}
%\mathcal{Q}_k(\wp_l,m_k,\wp_{l+1}) &\triangleq \mathscr{L}(\wp_l,m_k,\wp_{l+1}) + \mathcal{V}_{k}(\wp_{l+1},m_k), \nonumber\\ 
%&  k \leq l < T_f 
%\label{eq:Q_m_k}
%\end{align}
%The Q-functional is a prediction of the cost-to-go at the $k$th time step if the obstacle function is fixed as $m_k$ and  the functional control law $\mathcal{C}_k$ is applied to all robot PDFs from $\wp_l$ to $\wp_{T_f}$.
%Because the control, $\wp_{l+1} = \mathcal{C}(\wp_l,m_l) = \mathcal{C}_l(\wp_l)$ for $0\leq l \leq k$, is not dependent on the functional control law $\mathcal{C}_k(\cdot) = \mathcal{C}(\cdot,m_k)$ at the $k$th time step,  the state-action function, Q-function is only defined for $k \leq l < T_f$, such that
%\begin{align}
%\mathcal{Q}_k(\wp_l,m_k,\wp_{l+1}) &\triangleq \mathscr{L}(\wp_l,m_k,\wp_{l+1}) + \mathcal{V}_{k}(\wp_{l+1},m_k), \nonumber\\ 
%&  k \leq l < T_f 
%\label{eq:Q_m_k}
%\end{align}
%which is a prediction of the cost-to-go at the $k$th time step if the obstacle function is fixed as $m_k$ and  the functional control law $\mathcal{C}_k$ is applied to all robot PDFs from $\wp_l$ to $\wp_{T_f}$.


\subsection{Derivation of Optimal Functional Control Law}

Although the functional control law $\mathcal{C}_k$ is an operator as described before and the functional operator learning has been studied in \cite{ZhuVFA2016,ZhuMultiKernelProbDistRegression2015}, it is still challenging
to approximate this operator online and approximate $\wp_{k+1}$ according to (\ref{eq:functional_control_law}). Thus, in this paper, the critic-only Q-learning (CoQL) method is applied to obtain the optimal control where the approximation of control law $\mathcal{C}_k$ is not required \cite{LuoCoQL2016}.
Let the superscript ``*" indicate the optimal terms. According to the Bellman equation, the optimal Q-function can be expressed by
\begin{align}
\mathcal{Q}_k^* (\wp_l,m_k,\wp^*_{l+1}) &= \min_{\wp_{l+1}} \left[\mathscr{L}(\wp_l,m_k,\wp_{l+1}) + \mathcal{V}_k^* (\wp_{l+1},m_k) \right], \nonumber\\
& k\leq l < T_f
\end{align}
Thus, the output of the optimal functional control law, $\wp^{*}_{k+1}$, can be obtained by
\begin{align}
\wp^{*}_{k+1} &= \mathcal{C}_k^*(\wp_k,m_k) \nonumber\\&
 = \underset{\wp_{k+1}}{\arg\min} \left[ \mathcal{Q}_k^* (\wp_k,m_k,\wp_{k+1})\right]\nonumber\\
& =\underset{\wp_{k+1}}{\arg\min} \left[ \mathscr{L}(\wp_k,m_k,\wp_{k+1}) + \mathcal{V}_k^* (\wp_{k+1},m_k)\right]
\label{eq:optimization_optimal_control}
\end{align}
where $\mathcal{C}_k^*$ indicate the optimal control law obtained the $k$th time step. 
%Because (\ref{eq:optimization_optimal_control}) is applied to updated the control law at $k$th time step, the superscript $(k)$ is omitted when  
%%%%%%%%%%%%%%%%%%%%%%%%%%%%%%% Delete for short %%%%%%%%%%%%%%%%%%%%%%%%%%%%%%%%%%%%
%In the proposed approach, only the optimal value function is approximated, and the approximate is denoted by $\hat{\mathcal{V}}_k^*$. Then, the optimal control can be approximated by solving the following optimization problem,
%\begin{equation}
%\wp^{*}_{k+1}  \approx \underset{\wp_{k+1}}{\arg\min} \left[ \mathscr{L}(\wp_k,m_k,\wp_{k+1}) + \hat{\mathcal{V}}_k^* (\wp_{k+1},m_k)\right]
%\label{eq:optimization_approximate_optimal_control}
%\end{equation} 
%%%%%%%%%%%%%%%%%%%%%%%%%%%%%%% Delete for short %%%%%%%%%%%%%%%%%%%%%%%%%%%%%%%%%%%%
\subsection{Analysis of Convergence}

Given the vector of obstacle map functions at the $k$th time step $\mathbf{M}_k$, the lower bound of the optimal value functions $\mathcal{V}_{k^{\prime}}^*(\wp_l,\mathbf{M}_k)$, $0 \leq k^{\prime} \leq k$, is provided by the following theorem. 

\begin{theorem}[Lower bound of optimal control law]
	\label{theorem: optimal_control_law_lower_bound}
	At the $k$th time step, $0 < k < T_f$, given the vector of obstacle map functions $\mathbf{M}_k$, the optimal value functional $\mathcal{V}_k^*(\wp_l,\mathbf{M}_k)$ is the lower bound of all of the previous optimal value functionals $\mathcal{V}_{k^{\prime}}^*(\wp_l,\mathbf{M}_k)$ obtained at the $k^{\prime}$th time step, $0 \leq k^{\prime}\leq k$, for all robot PDFs $\wp_l$, such that
	\begin{equation}
	\mathcal{V}_k^*(\wp_l,\mathbf{M}_k) \leq \mathcal{V}_{k^{\prime}}^*(\wp_l,\mathbf{M}_k), \;0 \leq k^{\prime} \leq k \text{ and } 0 \leq l \leq T_f 
	\end{equation}
\end{theorem}
The proof of Theorem \ref{theorem: optimal_control_law_lower_bound} is provided in Appendix \ref{Appedix:lower_bound_of_optimal_control_law}. 



%According to the definition of the cost function in (\ref{eq:RL_cost_func}), it can be found that 
%\begin{equation}
%J = \mathcal{V}_{T_f-1} (\wp_0,\mathbf{M}_{T_f-1})
%\end{equation}

Let $\tilde{J}(k)$, $0 \leq k <T_f$, denote the cost function with respect to the functional control law $\mathcal{C}_k$, such that
\begin{equation}
\tilde{J}(k) \triangleq \mathcal{V}_{k} (\wp_0,\mathbf{M}_{T_f-1})
\label{eq:tilde_V}
\end{equation}
%
%Let $\tilde{J}(k)$, $0 \leq k <T_f$, denote the prediction of the cost function at the $k$th time step, such that
%\begin{equation}
%\tilde{J}(k) \triangleq \mathcal{V}_{k} (\wp_0,\mathbf{M}_{k})
%\label{eq:tilde_V}
%\end{equation}
According to the definition of the cost function in (\ref{eq:RL_cost_func}), it can be found that 
\begin{equation}
J = \tilde{J}(T_f-1) = \mathcal{V}_{T_f-1} (\wp_0,\mathbf{M}_{T_f-1})
\end{equation}
%Furthermore, the optimal prediction of the cost function can be defined by
%\begin{equation}
%\tilde{J}^*(k) \triangleq  \mathcal{V}_{k}^* (\wp_0,\mathbf{M}_{k})
%\label{eq:prediction_cost_func}
%\end{equation}

If the optimal value function $\mathcal{V}_k^*$ is available at the $k$th time step, the optimal control $\wp_{k+1}^*$ can be obtained using (\ref{eq:optimization_optimal_control}). Then, the optimal functional control law $\mathcal{C}_k^*$ is updated for $k=1,\ldots,(T_f-1)$. 
%A trajectory of robot PDFs, $\{\wp_{\tau}\}_{\tau = 0}^k$, can also be generated sequentially according to the updated value function $\mathcal{V}_k^*$ for $k=1,\ldots,(T_f-1)$.
%, which provides a policy $\boldsymbol{\Pi}_k$. 
%Meanwhile, the optimal prediction of the cost function can be updated at each time step according to (\ref{eq:prediction_cost_func}) until the time step $k = T_f-1$.
The following corollary shows that the optimal cost function $\tilde{J}^*(k)$ converges to the minimum cost function $J^*$ for $0\leq k <T_f$. 

\begin{corollary}[Convergence of cost function prediction]
	\label{corollary:Convergence_cost_function_prediction}
	The optimal cost function $\tilde{J}^*(k)$, $0 \leq k < T_f$ monotonically  converges to  $\tilde{J}^*(T_f - 1)$. For any two optimal control laws, $\mathcal{C}_k^*$ and $\mathcal{C}_{k^\prime}^*$, obtained at the $k$th and $k^{\prime}$th time steps, respectively, $0\leq k^{\prime} \leq k \leq (T_f - 1)$, such that
	\begin{equation}
	  J^* = \tilde{J}^*(T_f - 1) \leq \tilde{J}^*(k) \leq \tilde{J}^*(k^{\prime})
	\end{equation}
\end{corollary}
According to (\ref{eq:tilde_V}) and \textit{Theorem} \ref{theorem: optimal_control_law_lower_bound}, \textit{Corollary} \ref{corollary:Convergence_cost_function_prediction} can be proved straightforwardly. Therefore, by applying the optimal functional control law in (\ref{eq:optimization_optimal_control}), the cost function is minimized at each time step until the time step $(T_f - 1)$. %However, it is noteworthy to mention that 

%%%%%%%%%%%%%%%%%%%%% Delete for short    %%%%%%%%%%%%%%%%%%%%%%%%%%%%%%%
\section{Background on optimal Mass Transport}
\label{sec:OMT}
%Optimal mass transport (OMT) is an active and rapidly developing area
%of research that deals with the geometry of probability densities.
Because Optimal mass transport (OMT) can deal with problems of transporting masses from an initial
distribution to a terminal one in a mass preserving manner with minimum cost, it can be applied to our proposed problem to measure the energy cost of the VLSR system from the current distribution to the next distribution. In this section, some critical concepts and properties in OMT are introduced briefly. The interested readers are referred to \cite{VillaniTopicsInOptimalTransportation2003, ChenOptimalTransportGMM2018} for more
details.

\subsection{Wasserstein Metric}
%%%%%%%%%%%%%%%%%%%%%%%%%%%% Delete for short %%%%%%%%%%%%%%%%%%%%%%%%%%%%%%%%%%%%%%%
%Let $\mathcal{X}$ and $\mathcal{Y}$ be two measurable spaces and let
%$\mathscr{P}(\mathcal{X})$ and $\mathscr{P}(\mathcal{Y})$ denote
%the sets of all probability measures on $\mathcal{X}$ and $\mathcal{Y}$, respectively. Consider two probability measures, \textit{e.g.},  PDFs for continuous random variables, $p\in\mathscr{P}(\mathcal{X})$ and $q\in\mathscr{P}(\mathcal{Y})$,
%which are defined on $\mathcal{X}$ and $\mathcal{Y}$, respectively. A transport
%map
%\begin{align}
%\mathcal{T} \colon \mathcal{X} & \to \mathcal{Y} \nonumber\\
%\mathbf{x} & \mapsto \mathbf{y}
%\end{align}
%specifies where mass $p(\mathbf{x})d\mathbf{x}$ at $\mathbf{x}\in\mathcal{X}$
%is transported to $q(\mathbf{y})d\mathbf{y}$ at $\mathbf{y}\in\mathcal{Y}$ so as to match the final distribution in the sense
%that $q$ is the ``push-forward'' of $p$ under $\mathcal{T}$,
%meaning
%\begin{equation}
%\int_{\mathcal{M}}p(\mathbf{x})d\mathbf{x} = \int_{\mathcal{T}(\mathcal{M})} q(\mathbf{y}) d\mathbf{y} 
%\end{equation}
%for every Borel set $\mathcal{M}\subset\mathcal{X}$ \cite{ChenOptimalTransportGMM2018}. Let $c(\mathbf{x},\mathbf{y})$
%defined on $\mathcal{X}\times\mathcal{Y}$ denote the transportation
%cost per unit mass from point $\mathbf{x}\in\mathcal{X}$ to $\mathbf{y}\in\mathcal{Y}$,
%which is a non-negative function. The map $\mathcal{T}$should achieve
%a minimum cost of transportation,
%\begin{equation}
%J_{\mathcal{T}}(p,q)=\int_{\mathcal{X}}c(\mathbf{x},\mathcal{T}(\mathbf{x}))d\mathbf{x}
%\end{equation}
%Because the dependence of the transportation cost on $\mathcal{T}$
%is highly nonlinear and a minimum may not exist in general, Kantorovich
%proposed a relaxed formulation in 1942 to seek a transport map, which
%is a joint probability measure \cite{VillaniTopicsInOptimalTransportation2003,KantorovichOMT1942}. Let $\mathcal{P}(\mathcal{X}\times\mathcal{Y})$
%denote the sets of all joint probability measure (or joint PDF for continuous random variables) on $\mathcal{X}\times\mathcal{Y}$
%and 
%%%%%%%%%%%%%%%%%%%%%%%%%%%% Delete for short %%%%%%%%%%%%%%%%%%%%%%%%%%%%%%%%%%%%%%%
Let $\wp_p, \wp_q \in \mathcal{P}(\mathcal{W})$ denote two PDFs defined on $\mathcal{W}$, and let $\Pi(\wp_p,\wp_q)\subset\mathcal{P}(\mathcal{W}\times\mathcal{W})$
denote the set of all joint PDF $\pi\in\mathcal{P}(\mathcal{W}\times\mathcal{W})$
such that their marginals measures along the two coordinate directions
coincide with $\wp_p$ and $\wp_q$, respectively, such that
\begin{align}
&\Pi(\wp_p,\wp_q) \triangleq  \bigg\{ \pi\in\mathcal{P}(\mathcal{W}\times\mathcal{W}), \nonumber\\ 
&\int_{\mathbf{x}^{\prime}\in\mathcal{W}}\pi(\cdot,\mathbf{x}^{\prime})d\mathbf{x}^{\prime}  =\wp_p, \text{ and } \int_{\mathbf{x}\in\mathcal{X}}\pi(\mathbf{x},\cdot)d\mathbf{x}  = \wp_q\bigg\}
\end{align}
%\begin{align}
%&\Pi(p,q) \triangleq \pi\in\mathcal{P}(\mathcal{X}\times\mathcal{Y}) \nonumber\\
%\text{s.t. } &\int_{\mathbf{y}\in\mathcal{Y}}\pi(\cdot,\mathbf{y})d\mathbf{y}  =p,\nonumber\\
%&	\int_{\mathbf{x}\in\mathcal{X}}\pi(\mathbf{x},\cdot)d\mathbf{x}  =q
%\end{align}
%Kantorovich's formulation is to find optimal joint probability measure
%in $\Pi(p,q)$ minimizing the expected cost, such that
%\begin{equation}
%W_{\ell}(p,q)\triangleq\left\{\underset{\pi\in\Pi(p,q)}{\inf}\int_{\mathcal{X\times\mathcal{Y}}}\left[c(\mathbf{x},\mathbf{y})\right]^{\ell}d\pi(\mathbf{x},\mathbf{y})\right\}^{1/\ell}
%\end{equation}
%where $\ell$ is a positive integer, e.g. $\ell = 2$.
%
%If the cost function $c(\mathbf{x},\mathbf{y})$ is a metric on $\mathcal{X}\equiv\mathcal{Y}$,
%then $W_{\ell}(p,q)$ is a metric on the set $\mathcal{P}(\mathcal{X})$
%of all probability measures on $\mathcal{X}$, which is referred to as 
%Wasserstein metric.
%
%In this paper, consider a special metric, $c(\mathbf{x},\mathbf{y})=\Vert\mathbf{x}-\mathbf{y}\Vert$ and $\ell =2$, where $\Vert\cdot\Vert$ indicates the $\ell-2$ norm in $\mathcal{W}$.
%To ensure finite cost, it is standard to assume that $p$ and $q$
%belong to the space of probability measure with finite second moments,
%denoted by $\mathcal{P}_{2}(\mathcal{X})$. 
The Wasserstein
metric $W_{2}(\wp_p,\wp_q)$ is defined by \cite{VillaniTopicsInOptimalTransportation2003}
\begin{equation}
W_{2}(\wp_p,\wp_q)\triangleq\left[\underset{\pi\in\Pi(\wp_p,\wp_q)}{\inf}\int_{\mathcal{X}\times\mathcal{X}}\Vert\mathbf{x}-\mathbf{x}^{\prime}\Vert^{2}d\pi(\mathbf{x},\mathbf{x}^{\prime})\right]^{1/2}
\label{eq:W2_metric}
\end{equation}
where $\Vert \cdot \Vert$ indicate the Euclidean distance.  

It has been shown in \cite{ChenOptimalTransportGMM2018} that if both of the marginals $\wp_p \sim\mathcal{N}(\mu_{p},\Sigma_{p})$ and $\wp_q \sim\mathcal{N}(\mu_{q},\Sigma_{q})$ are
Gaussian distributions, the Wasserstein metric $W_{2}(\wp_p,\wp_q)$ can be expressed
in a closed form, such that
%\begin{equation}
%W_{2}(p,q)=\left\{ \Vert\mu_{p}-\mu_{q}\Vert^{2}+tr\left[\Sigma_{p}+\Sigma_{q}-2\left(\Sigma_{p}^{1/2}\Sigma_{q}\Sigma_{p}^{1/2}\right)^{1/2}\right]\right\} ^{1/2}
%\end{equation}
%\begin{align}
%W_{2}(\wp_p,\wp_q)=& \bigg\{ \Vert\mu_{p}-\mu_{q}\Vert^{2}  \nonumber\\
%& +tr\left[\Sigma_{p}+\Sigma_{q}-2\left(\Sigma_{p}^{1/2}\Sigma_{q}\Sigma_{p}^{1/2}\right)^{1/2}\right]\bigg\} ^{1/2}
%\end{align}
\begin{align}
&W_{2}(\wp_p,\wp_q)= \bigg\{ \Vert\mu_{p}-\mu_{q}\Vert^{2}  \nonumber\\
& +tr\left[\Sigma_{p}+\Sigma_{q}-2\left(\Sigma_{p}^{1/2}\Sigma_{q}\Sigma_{p}^{1/2}\right)^{1/2}\right]\bigg\} ^{1/2}
\end{align}
where $tr[\cdot]$ indicates the trace operator.  %$p\sim\mathcal{N}(\mu_{p},\Sigma_{p})$ and $q\sim\mathcal{N}(\mu_{q},\Sigma_{q})$,
%$\mu_{p}$ and $\mu_{q}$ are the means, and $\Sigma_{p}$ and $\Sigma_{q}$
%are the covariance matrices, respectively.

There exists a consequent displacement interpolation $\wp_s(\epsilon)\in\mathcal{P}(\mathcal{W})$,
$\epsilon\in[0,1]$, between $\wp_p$ and $\wp_q$, where $\wp_s(0)=\wp_p$ and $\wp_s(1)=\wp_q$,
and $W_{2}(\wp_s(\epsilon),\wp_p)=\epsilon\cdot W_{2}(\wp_p,\wp_q)$ and $W_{2}(\wp_s(\epsilon),\wp_q)=(1-\epsilon)\cdot W_{2}(\wp_p,\wp_q)$ \cite{ChenOptimalTransportGMM2018}.
If $\wp_p$ and $\wp_q$ are both Gaussian distributions, then $\wp_s(\epsilon) \sim \mathcal{N}(\mu_{s}(\epsilon),\Sigma_{s}(\epsilon))$ is
also Gaussian distributed, such that
%with the mean $\mu_{s}(\epsilon)$ and the covariance
%matrix $\Sigma_{s}(\epsilon)$ defined by
\begin{align}
\mu_{s}(\epsilon)  &= (1-\epsilon)\mu_{p}+\epsilon \mu_{q}\label{eq:mean_displacement}\\
\Sigma_{s}(\epsilon) & =  \Sigma_{p}^{-1/2}  \left[(1-\epsilon)\Sigma_{p}+\epsilon \left(\Sigma_{p}^{1/2}\Sigma_{q}\Sigma_{p}^{1/2}\right)^{1/2}\right]^{2}\Sigma_{p}^{-1/2}\label{eq:covariance_displacement}
\end{align}
%By using (\ref{eq:mean_displacement}) and (\ref{eq:covariance_displacement}), a consequent displacement interpolation between Gaussian distributions, $p$ and $q$, can be generated. Furthermore, this interpolation method can also be utilized to generate the displacement interpolation between two general distributions  represented by Gaussian mixture models (GMM) in the next subsection.   

\subsection{Metric on Space of Gaussian Mixture Model}

Although the Wasserstein metric $W_{2}$ can be calculated efficiently in closed
form when $\wp_p$ and $\wp_q$ are both Gaussian distributions, there is
no such closed form for general distributions, even if $\wp_p$ and $\wp_q$ are both GMMs \cite{ChenOptimalTransportGMM2018,AuricchioWassersteinMetricComputing2018}. 

Assume that $\wp_p$ and $\wp_q$ are both GMM, which can be expressed by
\begin{align}
\wp_p&=\sum_{i=1}^{N_p}\omega^{i}_{p}g_p^{i}, \quad i=1,\ldots,N_p \\
\wp_q&=\sum_{j=1}^{N_q}\omega^{j}_{q}g_q^{j}, \quad j=1,\ldots,N_q
\end{align}
where $g_p^i \sim\mathcal{N}(\mu_{p}^i,\Sigma_{p}^i)$ and $g_q^j \sim\mathcal{N}(\mu_{q}^j,\Sigma_{q}^j)$ denote Gaussian distributions, $N_p$ and $N_q$ are the numbers of Gaussian components, and $\boldsymbol{\omega}_p=[\omega^{1}_p,\ldots,\omega^{N_p}_p]$ and $\boldsymbol{\omega}_q=[\omega^{1}_q,\ldots,\omega^{N_q}_q]$ are corresponding positive probability vectors, such that $\sum_{i=1}^{N_p}\omega^{i}_p=1$ and $\sum_{j=1}^{N_q}\omega^{j}_q=1$ \cite{BishopPRML2006}. Let the space of all Gaussian mixture distributions defined on $\mathcal{W}$ be denoted by $\mathcal{G}(\mathcal{W})$.   

%A Gaussian mixture model is a probability density
%consisting of several Gaussian components and can be expressed by   \begin{equation}
%p=\sum_{i=1}^{N_p}\omega^{i}_{p}g_p^{i}, \quad i=1,\ldots,N_p
%\end{equation}
%where each $g_p^{i}$ is a Gaussian distribution with the mean $\mu^{i}_p$
%and the covariance matrix $\Sigma^{i}_p$, $N_p$
%is the number of Gaussian components, and $\boldsymbol{\omega}_p=[\omega^{1}_p,\ldots,\omega^{N_p}_p]$
%is a probability vector such that $\sum_{i=1}^{N_p}\omega^{i}_p=1$ \cite{BishopPRML2006}. Let the space of all Gaussian mixture distributions defined on $\mathcal{X}$ be denoted by $\mathcal{G}(\mathcal{X})$.

Recently, a new metric on $\mathcal{G}(\mathcal{W})$ was proposed in \cite{ChenOptimalTransportGMM2018} to approximate $W_{2}(\wp_p,\wp_q)$
for $\wp_p,\wp_q\in\mathcal{G}(\mathcal{W})$, such that 
%Because the new metric is defined based on the Wasserstein metrics between Gaussian components of the GMMs and the linear optimization, the metric between two arbitrary GMMs can be calculated in a closed form efficiently.  
%
%Consider the other GMM, $q\in\mathcal{G}(\mathcal{X})$, which is expressed by
%\begin{equation}
%q  =\sum_{j=1}^{N_{q}}\omega^{j}_{q}g^{j}_{q}
%\end{equation}
%where similarly $g^j_q$ is a Gaussian distribution with the mean $\mu^j_q$ and the covariance matrix $\Sigma^j_q$, $N_{q}$ is the number of Gaussian components, and $\boldsymbol{\omega}_q=[\omega^{1}_q,\ldots,\omega^{N_q}_q]$ is a probability vector such that $\sum_{i=1}^{N_q}\omega^{j}_q=1$. Then, given these Gaussian
%components, $\{g^{i}_{p}\}_{i=1}^{N_{p}}$ and $\{g^{j}_{q}\}_{j=1}^{N_{q}}$,
%the distribution $p$ and $q$ are equivalent to two discrete probability
%measures $\boldsymbol{\omega}_{p}$ and $\boldsymbol{\omega}_{q}$,
%respectively. Based on these two discrete probability measures, an
%approximation of the Wasserstein metric $W_{2}(p,q)$ can be obtained
%by
\begin{equation}
d(\wp_p,\wp_q) \triangleq \left\{\underset{\pi\in\Pi(\boldsymbol{\omega}_{p},\boldsymbol{\omega}_{q})}{\min}\sum_{i=1}^{N_{p}}\sum_{j=1}^{N_{q}}[W_{2}(g^{i}_{p},g^{j}_{q})]^2\pi(i,j)\right\}^{1/2}
\label{eq:Approimate_WG_Metric}
\end{equation}
where $\Pi(\boldsymbol{\omega}_{p},\boldsymbol{\omega}_{q})$ denotes
the space of joint probability distributions between $\boldsymbol{\omega}_{p}$
and $\boldsymbol{\omega}_{q}$. 
%and the cost function is defined by
%\begin{equation}
%\tilde{c}(i,j)=W_{2}(g^{i}_{p},g^{j}_{q})
%\end{equation}
%
%By standard linear programming theory, the discrete OMT problem in
%(\ref{eq:Approimate_WG_Metric}) always has at least one solution.
%Let $\pi^{*}$ be a minimizer in (\ref{eq:Approimate_WG_Metric}) and
%define
%\begin{equation}
%d(p,q)\triangleq\left\{\sum_{i=1}^{N_{p}}\sum_{j=1}^{N_{q}} \left[\tilde{c}(i,j)\right]^2\pi^{*}(i,j)\right\}^{1/2}
%\label{eq:WG_metric}
%\end{equation}
It has been proved that $d(\cdot,\cdot)$ defines a metric on $\mathcal{G}(\mathcal{W})$
in \cite{ChenOptimalTransportGMM2018}.
%which means that the following four properties are satisfied
%for any $p,q,s\in\mathcal{G}(\mathcal{X})$
%\begin{enumerate} 
%\item non-negativity or separation axiom: 
%\begin{equation}
%    d(p,q) \geq 0
%    \label{eq:WG_nonnegativity}
%\end{equation}
%\item identity of indiscernibles: 
%\begin{equation}
%   d(p,q)=0\Leftrightarrow p=q 
%   \label{eq:WG_identity_of_indiscernibles}
%\end{equation}
%\item symmetry: 
%\begin{equation}
%    d(p,q)=d(q,p)
%    \label{eq:WG_symmetry}
%\end{equation}
%\item triangle inequality: 
%\begin{equation}
%    d(p,q)\leq d(p,s)+d(s,q)
%    \label{eq:WG_triangle_inequality}
%\end{equation}
%\end{enumerate}
In this paper, this metric is referred to as Wasserstein-GMM (WG)
metric and the GMM space associated with this metric is referred to as Wasserstein-GMM space.

Furthermore, the geodesic connecting $\wp_p$ and $\wp_q$ is given by
\begin{equation}
\wp_s(\epsilon)=\sum_{i,j}\pi^{*}(i,j)g^{ij}_{pq}(\epsilon), \quad 0\leq \epsilon \leq 1
\label{eq:distribution_interpolation}
\end{equation}
where $\pi^*(i,j)$ is the optimal joint distribution defined in (\ref{eq:Approimate_WG_Metric}), and   $g^{ij}_{pq}(\epsilon) \sim\mathcal{N}\left(\mu^{j}_{q},\Sigma^{j}_{q}\right)$ is a consequent displacement interpolation between $g^{i}_{p}$
and $g^{j}_{q}$, 
which can be specified 
according to ($\text{\ref{eq:mean_displacement}}$) and ($\text{\ref{eq:covariance_displacement}}$).
%such that
%\begin{align}
%\mu^{ij}_{pq}(\epsilon) & =(1-\epsilon)\mu^{i}_{p}+\epsilon\mu^{j}_{q}\label{eq:mean_displacement_GMM}\\
%\Sigma^{ij}_{pq}(\epsilon) & =\left(\Sigma^{i}_{p}\right)^{-1/2} \bigg\{ (1-\epsilon)\left(\Sigma^{i}_{p}\right)  \nonumber\\
%& + \epsilon \left(\left(\Sigma^{i}_{p}\right)^{1/2}\Sigma^{j}_{q}\left(\Sigma^{i}_{p}\right)^{1/2}\right)^{1/2}\bigg\} ^{2}\left(\Sigma^{i}_{p}\right)^{-1/2}
%\label{eq:covariance_displacement_GMM}
%\end{align}
%%%%%%%%%%%%%%%%%%%%% Delete for short    %%%%%%%%%%%%%%%%%%%%%%%%%%%%%%%

\section{Adaptive Distributed Optimal Control Based in Wasserstein-GMM Space}
\label{sec:ADOC_WG}

%As described in Section \ref{sec:OMT}, the Wasserstein metric $W_2$ defined in (\ref{eq:W2_metric}) and its approximation, WG metric, defined in (\ref{eq:WG_metric}), are both defined based on the expectation of Euclidean distance between two distributions, because of $c(\mathbf{x},\mathbf{y}) = \Vert \mathbf{x} - \mathbf{y} \Vert$. 
Considering two sequential robot PDFs, $\wp_k$ and $\wp_{k+1}$, the Wasserstein metric $W_2(\wp_k,\wp_{k+1})$ indicates the distance between these two distributions. Since the time interval $\triangle t$ between two distribution is fixed, $W_2(\wp_k,\wp_{k+1})$ is proportional to the distribution velocity at time step $k$ defined by
\begin{equation}
    \nu^{\wp}_{k} = \frac{W_2(\wp_k,\wp_{k+1})}{\triangle t}
    \label{eq:distribution_velocity_W2}
\end{equation}
%where the subscript ``$\wp$" indicates the distribution velocity. 
Moreover, the squared of $W_2$ is proportional to the energy-cost from $\wp_k$ to $\wp_{k+1}$, which is denoted by $E_k$, such that
\begin{equation}
    E_k \propto (\nu^{\wp}_k)^2 \propto [W_2(\wp_k,\wp_{k+1})]^2
    \label{eq:Engery_cost}
\end{equation}

Furthermore, because the WG metric $d(\wp_k,\wp_{k+1})$ can be obtained via a linear optimization in  (\ref{eq:Approimate_WG_Metric}), it is reasonable to apply  $[d(\wp_k,\wp_{k+1})]^2$ as an energy-cost term in the Lagrangian term  (\ref{eq:RL_cost_func}). Compared with the divergences, such as Kullback–Leibler (KL) divergence and Cauchy-Schwarz (CS) divergence,  WG metric is more suitable for this problem because of its clear physical meaning of energy-cost and metric properties\cite{ChenOptimalTransportGMM2018}.
%%%%%%%%%%%%%%%%%%%%%%%% Delete for short %%%%%%%%%%%%%%%%%%%%%%%%%%%%%%%%%%%%%%
%
%Because the WG metrics $d(\wp_k,\wp_{k+1})$ is defined on  $\mathcal{G}(\mathcal{W})$, a value functional based on the  WG metrics is defined and approximated in $\mathcal{G}(\mathcal{W})$ in this section. In addition, the corresponding functional control law is presented and the optimal control law is also approximated to develop the ADOC algorithm in $\mathcal{G}(\mathcal{W})$.  
%%%%%%%%%%%%%%%%%%%%%%%% Delete for short %%%%%%%%%%%%%%%%%%%%%%%%%%%%%%%%%%%%%%

\subsection{Value functional in Wasserstein-GMM Space}

First, consider \textit{Assumption \ref{ass:wp_in_space_GMM}} presented below. 
\begin{assumption}
	\label{ass:wp_in_space_GMM}
	Assume that all of the robot PDFs belong to the GMM space, such that $\mathcal{G}(\mathcal{W})$, 
%	including the initial robot PDF $\wp_0$, the target robot PDF $\wp_{targ}$,  and the trajectory of robot PDFs, $\{\wp_k\}_{k=1}^{T_f}$, 
%	such that
	\begin{align}
	\wp_k &= \sum_{i=1}^{N_k} \omega^i_k g^i_k, \quad k = 0,\ldots,T_f \label{eq:wp_k_in_GMM}\\
	\wp_{targ} &= \sum_{j=1}^{N_{targ}} \omega^j_{targ} g^j_{targ} \label{eq:wp_targ_in_GMM}   
	\end{align}
	where $N_k$ and $N_{targ}$ are the numbers of Gaussian components, $\boldsymbol{\omega}_{k} = [\omega^{1}_{k},\ldots,\omega^{N_{k}}_{k}]$ and $\boldsymbol{\omega}_{targ} = [\omega^{1}_{targ},\ldots,\omega^{N_{targ}}_{targ}]$ are all probability vectors, and $g^i_k$ and $g^j_{targ}$ are all Gaussian components specified by the means, $\mu^i_k$ and $\mu^j_{targ}$, and the covariance matrices, $\Sigma^i_k$ and $\Sigma^j_{targ}$, respectively.   
	Thus, the PDFs, $\wp_k$ and $\wp_{targ}$, can be specified by the tuples of parameters, $\Theta_{\wp_k} = (N_k,  \boldsymbol{g}_{k},\boldsymbol{\omega}_{k})$ and $\Theta_{\wp_{targ}}= (N_{targ}, \boldsymbol{g}_{targ}, \boldsymbol{\omega}_{targ})$, respectively, where $\boldsymbol{g}_{k} = [g^1_{k},\ldots,g^{N_{k}}_{k}]$ and $\boldsymbol{g}_{targ} = [g^1_{targ},\ldots,g^{N_{targ}}_{targ}]$ are vectors of Gaussian components. 
\end{assumption}

Then, given the robot PDF,  $\wp_k = \sum_{i=1}^{N_k} \omega^i_k g^i_k$, the functional control law  can be expressed by
\begin{align}
\wp_{k+1} &= \mathcal{C}_k(\wp_{k},m_k) \nonumber\\
&= \sum_{\imath=1}^{N_{k+1}} \omega^{\imath}_{k+1} g^{\imath}_{k+1} \nonumber\\
&= \sum_{i=1}^{N_{k}}\sum_{\imath=1}^{N_{k+1}} \pi_k(i,\imath) g^{\imath}_{k+1}
\label{eq:expression_of_control_law}
\end{align}
where $\pi_k \in \Pi(\boldsymbol{\omega}_{k},\boldsymbol{\omega}_{k+1})$
is  the  joint  probability distribution, such that 
\begin{equation}
\omega^{\imath}_{k+1} = \sum_{i=1}^{N_k} \pi_k(i,\imath), \quad \imath = 1,\ldots,N_{k+1}
\end{equation} 
Thus, given $\wp_k$ and $m_k$, the functional control law $\mathcal{C}_k$ in (\ref{eq:expression_of_control_law}) is specified by the tuple of parameters, $\Theta_{\mathcal{C}_k}=(N_{k+1}, \boldsymbol{g}_{k+1}, \pi_k )$. 

The energy-cost associated with the obstacle map function, $m_k$, and the functional control law, $\mathcal{C}_k$, can be defined by
\begin{equation}
\tilde{e}(\wp_k,m_k,\mathcal{C}_k) \triangleq \sum_{i=1}^{N_k}\sum_{\imath=1}^{N_{k+1}} [W_2(g^i_k,g^{\imath}_{k+1})]^2 \pi_k(i,\imath)
 \label{eq:tilde_d_sq_def}
\end{equation}
Like the WG metric, given $\mathcal{C}_k$ and $m_k$, a distance can be defined by
\begin{equation}
\tilde{d}(\wp_k,m_k,\mathcal{C}_k) \triangleq [\tilde{e}(\wp_k,m_k,\mathcal{C}_k)]^{1/2}
\end{equation}
According to (\ref{eq:Approimate_WG_Metric}), then, given $\wp_{k+1}$, the WG metric is the minimum of $\tilde{d}(\wp,\mathcal{C}_k)$, such that
\begin{align}
d(\wp_k,\wp_{k+1}) &=   \underset{\pi_k}{\min} \bigg\{ \sum_{i=1}^{N_k}\sum_{\imath=1}^{N_{k+1}} [W_2(g^i_k,g^{\imath}_{k+1})]^2 \pi_k(i,\imath) \bigg\}^{1/2}\nonumber\\
& = \underset{\pi_k}{\min} [\tilde{d}(\wp_k,m_k,\mathcal{C}_k)]
\label{eq:def_WG_Metric}
\end{align}

Furthermore, considering the constraint of the obstacles to the robot PDF, the Lagrangian term is defined by
\begin{equation}
%    \mathscr{L}[\wp_k,m_k,\mathcal{C}] = \left[d (\wp_k,\wp_{k+1}) \right]^2 +  \langle   \wp_{k+1}, m_k \rangle_{\mathcal{W}} 
	\mathscr{L}(\wp_k,m_k,\mathcal{C}_k) = [\tilde{d}(\wp_k,m_k,\mathcal{C}_k)]^2 +  \langle   \mathcal{C}_k(\wp_{k},m_k), m_k \rangle_{\mathcal{W}}
	\label{eq:Lagrangian_GW}
\end{equation}
where $\langle \cdot,\cdot \rangle_{\mathcal{W}}$
indicates the inner product on the ROI $\mathcal{W}$. Here, the second term reflects the probability that the sensors at the $(k+1)$th time step are deployed inside the approximate obstacles $\hat{\mathcal{B}}(t_k)$.


Finally, the final term in (\ref{eq:RL_cost_func}) is defined by the WG metric, such that
\begin{equation}
\phi(\wp_{T_f},\wp_{targ} ) \triangleq [d(\wp_{T_f},\wp_{targ} )]^2
\end{equation}
Therefore, the cost function in (\ref{eq:RL_cost_func}) can be rewritten by
\begin{align}
\label{eq:RL_cost_func_WG} 
J & \triangleq %[d(\wp_{T_f},\wp_{targ} )]^2 + \sum_{k=0}^{T_f - 1} \mathscr{L}(\wp_k,m_k,\mathcal{C}_k) \nonumber\\
%& = 
[d(\wp_{T_f},\wp_{targ} )]^2 \nonumber\\
& + \sum_{k=0}^{T_f -1} \left\{[\tilde{d}(\wp_k,m_k,\mathcal{C}_k)]^2 +  \langle   \wp_{k+1}, m_k \rangle_{\mathcal{W}} \right\} 
\end{align} 
According to (\ref{eq:VF_m_k}), thus, the corresponding value functional $\mathcal{V}_k(\wp_{k},m_k,\mathcal{C}_k)$ can be expressed by
\begin{align}
    \mathcal{V}_k(\wp_{k},m_k,\mathcal{C}_k) &= [d(\wp_{T_f},\wp_{targ} )]^2 %\nonumber\\&
    + \sum_{\tau=k}^{T_f -1} \bigg\{[\tilde{d}(\wp_{\tau},m_k,\mathcal{C}_k)]^2 \nonumber\\
    &+  \langle   \mathcal{C}_k(\wp_{\tau},m_k), m_k \rangle_{\mathcal{W}} \bigg\} \nonumber\\
    &= \mathscr{L}(\wp_k,m_k,\mathcal{C}_k) + \mathcal{V}_k(\wp_{k+1},m_k,\mathcal{C}_k)
    \label{eq:VF_WG}
\end{align}


\subsection{Approximation of Optimal Value functional in Wasserstein-GMM Space}

According to (\ref{eq:VF_m_k}), the value functional $\mathcal{V}_k(\wp_{k+1},m_k,\mathcal{C}_k)$ depends on the trajectory of robot PDFs, $\{ \wp_{\tau}\}_{\tau = k+1}^{T_f}$, which are not available at the $k$th time step. Therefore, the optimal value functional  $\mathcal{V}_k^*(\wp_{k+1},m_k)$ in (\ref{eq:optimization_optimal_control}) is not available.  An approximation of the optimal value functional is required. Similar to many other RL-ADP approaches \cite{LewisRLAPD2013}, an upper bound of the optimal value functional is applied as the approximation. 

% Assume that all of the robot PDFs are belong to the space of GMM, $\mathscr{G}(\mathscr{W})$, including the initial robot PDF $\wp_0$, the target robot PDF $\wp_{targ}$,  and the trajectory of robot PDFs, $\{\wp_k\}_{k=1}^{T_f}$, such that
% \begin{align}
%     \wp_k &= \sum_{i=1}^{N_k} \omega_i^k g_i^k, \quad k = 0,\ldots,T_f \label{eq:wp_k_in_GMM}\\
%     \wp_{targ} &= \sum_{j=1}^{N_{targ}} \omega_j^{targ} g_j^{targ} \label{eq:wp_targ_in_GMM}   
% \end{align}
 
Consider a special control law, $\tilde{\mathcal{C}}_k(\cdot,m_k) : \mathscr{P} \mapsto \mathscr{P}$. For any time step $\tau$, $k+1 \leq \tau \leq T_f-1$,  the next step robot PDF is obtained by $\wp_{\tau+1} = \tilde{C}_k(\wp_{\tau},m_k)$, such that
% \begin{align}
%     N_{\tau} &= N_{\tau+1}, \quad k+1 \leq \tau \leq T_f-1 \label{eq:number_GC}\\
%     \omega_{\imath}^{\tau} &= \omega_{\imath}^{\tau+1}, \quad \imath = 1,\ldots,N_{\tau} \label{eq:weight_GC}
% \end{align}
\begin{align}
    N_{\tau+1} &= N_{\tau}, \quad k+1 \leq \tau \leq T_f-1 \label{eq:number_GC}\\
    \pi_{\tau}(i,\imath) &= \begin{cases}
    \omega^i_{\tau}, & \text{if } i = \imath \\
    0, & \text{otherwise}
    \end{cases}, \quad i,\imath = 1,\ldots, N_{\tau}
    \label{eq:weight_GC}
\end{align}
It means that from the ($k+1$)th time step to the final time step, the number and the corresponding weights of the Gaussian components of  $\wp_{\tau}$ are fixed. 

By recursively applying $\tilde{\mathcal{C}}_k(\cdot,m_k)$, a trajectory of robot PDFs can be generated from $\wp_{k+1}$, such that
\begin{equation}
    \wp_{\tau} = \sum_{\imath=1}^{N_{k+1}}\omega^{\imath}_{k+1} g^{\imath}_{\tau}, \quad \tau = k+1,\ldots,T_f
\end{equation}
Thus, there are $N_{k+1}$ trajectories of Gaussian components from $g^{\imath}_{k+1}$ to $g^{\imath}_{T_f}$, $\imath = 1,\ldots,N_{k+1}$. Next, consider the $N_{targ}$ Gaussian components of the target PDF, $\{g^j_{targ}\}_{j=1}^{N_{targ}}$. There are $N_{k+1} \times N_{targ}$ trajectories of Gaussian components denoted by the trajectory set $\tilde{\mathscr{Tr}}_k = \big\{(g^{\imath}_{k+1},\ldots,g^{\imath}_{T_f}, g^j_{targ})\big\}_{\imath,j}^{N_{k+1},N_{targ}}$, where each element of the set $\tilde{\mathscr{Tr}}_k$ represents a trajectory of Gaussian components denoted by $\tilde{\mathscr{Tr}}^{\imath,j}_k =  (g^{\imath}_{k+1},\ldots,g^{\imath}_{T_f}, g^j_{targ})$. Let $\tilde{\mathcal{L}}^{\imath,j}_k$ denote the minimum cost function of the trajectory of Gaussian components, $\tilde{\mathscr{Tr}}^{\imath,j}_k$, at the $k$th time step, defined by
% \begin{align}
%     &\mathcal{L}^k(\mathscr{Tr}_{\imath,j})  \nonumber\\
%     &= \begin{cases}
%      \underset{\mathscr{Tr}_{\imath,j}}{\min}\bbig\{[W_2(g_{\imath}^{T_f},g_{\imath}^{targ})]^2 \\
%     + \sum_{\tau = k+1}^{T_f - 1} [W_2(g_{\imath}^{\tau},g_{\imath}^{\tau+1})]^2 & \text{if } k+1 < T_f\\
%     + \sum_{\tau = k+1}^{T_f - 1}\langle g_{\imath}^{\tau+1},m_k \rangle \bbig\},   \\
%     \\
%     [W_2(g_{\imath}^{T_f},g_{\imath}^{targ})]^2 & \text{if } k+1 = T_f
%     \end{cases}
% \end{align}
\begin{align}
    \tilde{\mathcal{L}}^{\imath,j}_k 
    &= \begin{cases}
     \underset{\tilde{\mathscr{Tr}}^{\imath,j}_k}{\min}\big\{[W_2(g^{\imath}_{T_f},g^{j}_{targ})]^2 \\
     + \sum_{\tau = k+1}^{T_f - 1} [W_2(g^{\imath}_{\tau},g^{\imath}_{\tau+1})]^2 & \text{if } k+1 < T_f\\
     + \sum_{\tau = k+1}^{T_f - 1}\langle g^{\imath}_{\tau+1},m_k \rangle \big\},   \\
    \\
    [W_2(g^{\imath}_{T_f},g^{j}_{targ})]^2, & \text{if } k+1 = T_f
    \end{cases}
    \label{eq:L_k}
\end{align}
% For short, in the rest of the paper, $\mathcal{L}^k(\mathscr{Tr}_{\imath,j})$ is denoted by $\mathcal{L}^k_{{\imath},j}$ when no confusion is possible.  
% 
% Given the special control law, $\tilde{\mathcal{C}}$, and the generated trajectory of robot PDFs, $\wp_{k+1},\ldots,\wp_{T_f}$, an minimum of the value functional $\mathcal{V}_k(\wp_{k+1},m_k,\tilde{\mathcal{C}})$ is provided by the following theorem, which is obviously also the upper bound of the optimal value functional  $\mathcal{V}_k^*(\wp_{k+1},m_k)$

By applying  $\tilde{\mathcal{C}}_k(\cdot,m_k)$, an upper bound of the optimal value functional  $\mathcal{V}_k^*(\wp_{k+1},m_k)$ is provided in the following theorem. 
\begin{theorem}[Upper bound of optimal value functional]
    \label{theorem: Upper_bound_of_optimal_value_functional}
Given the robot PDF $\wp_{k+1}$ and the target robot PDF $\wp_{targ}$ defined in (\ref{eq:wp_k_in_GMM}) and (\ref{eq:wp_targ_in_GMM}), respectively, there exists an upper bound of the optimal value functional $\mathcal{V}_{k}^*(\wp_{k+1},m_k)$, which is denoted by $\tilde{\mathcal{V}}_{k}(\wp_{k+1},m_k)$, such that
\begin{equation}
\mathcal{V}_{k}^*(\wp_{k+1},m_k) \leq \tilde{\mathcal{V}}_{k}(\wp_{k+1},m_k) 
\label{eq:upper_bound_optimal_VF}
\end{equation}
where 
\begin{equation}
    \tilde{\mathcal{V}}_{k}(\wp_{k+1},m_k) \triangleq \sum_{\imath=1}^{N_{k+1}} \sum_{j=1}^{N_{targ}}\tilde{\mathcal{L}}^{\imath,j}_k\tilde{\pi}_{k}(\imath,j)
\label{eq:def_upper_bound_VF}
\end{equation}
and 
$\tilde{\pi}_{k}(\imath,j) \in \Pi(\boldsymbol{\omega}_{k+1},\boldsymbol{\omega}_{targ})$ is the joint probability distribution.
% and $\boldsymbol{\omega}_{k+1} = [\omega_{1}^{k+1},\ldots,\omega_{N_{k+1}}^{k+1}]^{T}$ and $\boldsymbol{\omega}_{targ} = [\omega_{1}^{targ},\ldots,\omega_{N_{targ}}^{targ}]^{T}$.
\end{theorem}
The proof of \textit{Theorem} \ref{theorem: Upper_bound_of_optimal_value_functional} is provided in Appendix \ref{Appedix: Upper_bound_of_optimal_value_functional}. 

In this paper, the upper bound of the optimal value functional in (\ref{eq:def_upper_bound_VF}) is applied to approximate the optimal value functional $\mathcal{V}_k^*(\wp_{k+1},m_k)$, such that
%\begin{align}
%\hat{\mathcal{V}}_k^*(\wp_{k+1},m_k) &\approx \tilde{\mathcal{V}}_{k}(\wp_{k+1},m_k) \nonumber\\
%&= \sum_{\imath=1}^{N_{k+1}} \sum_{j=1}^{N_{targ}} \tilde{\mathcal{L}}^{\imath,j}_k \tilde{\pi}_{k}(\imath,j)
%\label{eq:appr_optimal_VF}
%\end{align}
\begin{equation}
\hat{\mathcal{V}}_k^*(\wp_{k+1},m_k) 
\triangleq \sum_{\imath=1}^{N_{k+1}} \sum_{j=1}^{N_{targ}} \tilde{\mathcal{L}}^{\imath,j}_k \tilde{\pi}_{k}(\imath,j)
\label{eq:appr_optimal_VF}
\end{equation}

\subsection{Optimal Functional Control Law in Wasserstein-GMM Space}

According to (\ref{eq:VF_WG}) and (\ref{eq:appr_optimal_VF}), the optimal control law can be approximated by solving the following optimization problem,
\begin{align}
\mathcal{C}_k^* 
%&= \underset{\mathcal{C}_k}{\arg\min} \left[\mathscr{L}(\wp_k,m_k,\mathcal{C}_k) + \mathcal{V}_k^*(\wp_{k+1},m_k,\mathcal{C}_k) \right] \nonumber\\
&\approx \underset{\mathcal{C}_k}{\arg\min} \left[\mathscr{L}(\wp_k,m_k,\mathcal{C}_k) + \hat{\mathcal{V}}_k^*(\wp_{k+1},m_k) \right] \nonumber\\
&= \underset{\mathcal{C}_k}{\arg\min} \bigg\{ [\tilde{d}(\wp_k,m_k,\mathcal{C}_k)]^2 +  \langle   \mathcal{C}_k(\wp_{k},m_k), m_k \rangle_{\mathcal{W}} \nonumber\\
&+ \sum_{\imath=1}^{N_{k+1}} \sum_{j=1}^{N_{targ}} \tilde{\mathcal{L}}^{\imath,j}_k \tilde{\pi}_{k}(\imath,j) \bigg\}
\label{eq:optimization_C_k}
\end{align}
Because the control law is specified by the tuple of parameter, $\Theta_{\mathcal{C}_k}=(N_{k+1}, \boldsymbol{g}_{k+1}, \pi_k )$ in (\ref{eq:expression_of_control_law}), and the optimal value functional is approximated by $\{ \tilde{\mathcal{L}}^{\imath,j}_k\}_{\imath,j}^{N_{k+1},N_{targ}}$ and $\tilde{\pi}_k$ in (\ref{eq:appr_optimal_VF}), the optimization problem in (\ref{eq:optimization_C_k}) can be expressed as,
\begin{align}
\hat{\Theta}^*_{\mathcal{C}_k} &= \underset{\Theta_{\mathcal{C}_k}}{\arg\min} \bigg\{  \sum_{i=1}^{N_{k}}\sum_{\imath=1}^{N_{k+1}}[W_2(g^i_k,g^{\imath}_{k+1})]^2\pi_k(i,\imath) \nonumber\\
&+ \sum_{i=1}^{N_k}\sum_{\imath = 1}^{N_{k+1}} \langle g^{\imath}_{k+1},m_k \rangle_{\mathcal{W}} \pi_k(i,\imath) \nonumber\\
&+ \sum_{\imath=1}^{N_{k+1}} \sum_{j=1}^{N_{targ}} \tilde{\mathcal{L}}^{\imath,j}_k \tilde{\pi}_{k}(\imath,j) \bigg\}
\end{align}
where $\hat{\Theta}^*_{\mathcal{C}_k}$ is the approximation of $\Theta^*_{\mathcal{C}_k}$.

There are $N_k \times N_{k+1}$ trajectories of Gaussian components denoted by the trajectory set $\mathscr{T}_k = \{(g^i_k,g^{\imath}_{k+1})\}_{i,\imath}^{N_k,N_{k+1}}$. Similarly, let $\mathcal{L}^{i,\imath}_k$ denote the cost function of the 
trajectory of Gaussian components, 
$\mathscr{T}^{i,\imath}_k = (g^{i}_k,g^{\imath}_{k+1})$,
with respect to the map function $m_k$, which is defined by
\begin{equation}
\mathcal{L}^{i,\imath}_k = [W_2(g^i_k,g^{\imath}_{k+1})]^2 + \langle g^{\imath}_{k+1}, m_k \rangle_{\mathcal{W}}
\label{eq:tilde_L_k}
\end{equation}
 
Finally, considering the following constraints of the joint probabilities, 
\begin{align}
\omega^i_k &= \sum_{\imath=1}^{N_{k+1}} \pi_k(i,\imath) \label{eq:constraint_1}\\
\omega^j_{targ} &= \sum_{\imath=1}^{N_{k+1}} \tilde{\pi}_k(\imath,j) \label{eq:constraint_2}\\
\omega^{\imath}_{k+1} &= \sum_{i=1}^{N_k} \pi_k(i,\imath) = \sum_{j=1}^{N_{targ}} \tilde{\pi}_k(\imath,j) \label{eq:constraint_3}
\end{align} 
the optimal control law can be approximated by solving the following optimization problem,





\begin{align}
\hat{\Theta}^*_{\mathcal{C}_k} &= \underset{\Theta_{\mathcal{C}_k}}{\arg\min} \sum_{\imath=1}^{N_{k+1}} \bigg [ \sum_{i=1}^{N_{k}} \mathcal{L}^{i,\imath}_k
\pi_k(i,\imath) %\nonumber\\
+  \sum_{j=1}^{N_{targ}} \tilde{\mathcal{L}}^{\imath,j}_k \tilde{\pi}_k(\imath,j) \bigg ]\nonumber\\
\text{s.t. } & \quad (\ref{eq:constraint_1}) - (\ref{eq:constraint_3}) 
\label{eq:Theta_k_next_2}
\end{align}

After obtaining $\hat{\Theta}^*_{\mathcal{C}_k} = (\hat{N}^*_{k+1}, \hat{\boldsymbol{g}}^*_{k+1}, \hat{\pi}^*_k )$, the optimal robot PDF $\wp^*_{k+1}$ can be approximated by
%\begin{align}
%\hat{\wp}^*_{k+1} &= \sum_{i=1}^{N_k}\sum_{\imath = 1}^{\hat{N}^*_{k+1}}
%\hat{\pi}_k^*(i,\imath) (\hat{g}^{\imath}_{k+1})^* \nonumber\\
%&= \sum_{\imath = 1}^{\hat{N}^*_{k+1}} (\hat{\omega}^{\imath}_{k+1})^* (\hat{g}^{\imath}_{k+1})^*
%\end{align}  
\begin{equation}
\hat{\wp}^*_{k+1} = \sum_{i=1}^{N_k}\sum_{\imath = 1}^{\hat{N}^*_{k+1}}
\hat{\pi}_k^*(i,\imath) (\hat{g}^{\imath}_{k+1})^* = \sum_{\imath = 1}^{\hat{N}^*_{k+1}} (\hat{\omega}^{\imath}_{k+1})^* (\hat{g}^{\imath}_{k+1})^*
\end{equation}
where $(\hat{\omega}^{\imath}_{k+1})^* = \sum_{i=1}^{N_k} \hat{\pi}_k^*(i,\imath)$ and $\hat{\boldsymbol{g}}^*_{k+1} = [(\hat{g}^1_{k+1})^*, \ldots, (\hat{g}^{\hat{N}_{k+1}^*}_{k+1})^*]$.

Given $\hat{\wp}^*_{k+1}$ and the approximated optimal control law, $\hat{\mathcal{C}}^*_k$, which is specified by $\hat{\Theta}^*_{\mathcal{C}_k}$,  the corresponding Lagrangian term defined in (\ref{eq:Lagrangian_GW}) can be expressed by
\begin{align}
\mathscr{L}(\wp_k,m_k,\hat{\mathcal{C}}^*_k) &= [\tilde{d}(\wp_k,m_k,\hat{\mathcal{C}}^*_k)]^2 +  \langle   \hat{\mathcal{C}}^*_k(\wp_{k},m_k), m_k \rangle_{\mathcal{W}} \nonumber\\
&= \sum_{i=1}^{N_k}\sum_{\imath=1}^{\hat{N}^*_{k+1}} [W_2\big(g^i_k,(\hat{g}^{\imath}_{k+1})^*\big)]^2 \hat{\pi}^*_k(i,\imath) \nonumber\\
&+ \langle   \hat{\wp}^*_{k+1}, m_k \rangle_{\mathcal{W}} \nonumber\\
&= [d(\wp_k,\hat{\wp}^*_{k+1})]^2 + \langle   \hat{\wp}^*_{k+1}, m_k \rangle_{\mathcal{W}}
\end{align}
Thus, the Lagrangian term under $\hat{\mathcal{C}}^*_k$ is a function of the WG metric between $\wp_k$ and $\hat{\wp}^*_{k+1}$. Let $\wp_{k+1} = \hat{\wp}^*_{k+1}$ and $v^{\wp}_k$ denote the velocity of robot PDFs at the $k$th time step, defined by
\begin{equation}
v^{\wp}_k \triangleq \frac{d(\wp_k,\wp_{k+1})}{\Delta t}
\label{eq:distribution_velocity_WG}
\end{equation}  
which is an approximation of distribution velocity defined in (\ref{eq:distribution_velocity_W2}). Then, the Lagrangian term, $\mathscr{L}(\wp_k,m_k,\hat{\mathcal{C}}^*_k)$,  reflects the energy-cost $E_k$ in (\ref{eq:Engery_cost}). 

\section{Implementation of Adaptive Distributed Optimal Control}
\label{sec:Implementation}
%Although an approximated optimal control law is proposed in (\ref{eq:Theta_k_next_2}), it is not feasible to solve the complicated nonlinear optimization online because of the very high computation complexity. 
To obtain the approximated optimal functional control law online,  an numerical approximation approach is presented in this section.

\subsection{Approximation of Optimal Control Law in Sub-space of GMM}
\label{subsec:Approximation_in_SubSpace}
%%%%%%%%%%%%%%%%%%%%%%%%% Delete for short %%%%%%%%%%%%%%%%%%%%%%%%%%%%%%%%%%%%%%%%%%
According to  (\ref{eq:Theta_k_next_2}), there are three problems result in the very high computational complexity, including 
\begin{enumerate}
    \item unknown number of Gaussian components $N_{k+1}$
    \item nonlinear calculations of $W_2(\cdot,\cdot)$ in (\ref{eq:L_k}) and (\ref{eq:tilde_L_k}) 
    \item nonlinear programming (NLP) in (\ref{eq:L_k})
\end{enumerate}
To simplify these problems, the following assumption is made.
%%%%%%%%%%%%%%%%%%%%%%%%% Delete for short %%%%%%%%%%%%%%%%%%%%%%%%%%%%%%%%%%%%%%%%%%
\begin{assumption}
\label{ass:collocation_GC} 
Assume that at the $k$th time step, $0\leq k < T_f$, the robot PDFs, $\wp_{\tau} \in \mathcal{G}(\mathcal{W})$, $\tau = k+1,\ldots,T_f$, can  be expressed by
\begin{equation}
    \wp_{\tau} = \sum_{\imath=1}^{N_c} \omega^{\imath}_{\tau} g^{\imath}_c
    \label{eq:wp_in_subspace}
\end{equation}
where these parameters $N_c$ and $g^{\imath}_c$, $\imath=1,\ldots,N_c$, are all fixed and known a priori, which specify a set of collocation Gaussian components denoted by $\boldsymbol{G} = \{g^1_{c},
\ldots,g^{N_{c}}_{c}\}$. Here, the subscript ``c" indicates the collocation Gaussain components. 
\end{assumption}

Specifically, these collocation Gaussian components, $g^{\imath}_c$ $\imath = 1,\ldots,N_c$, are all specified by the corresponding means $\mu^{\imath}_c$ and covariance matrices $\Sigma^{\imath}_c$. There are several methods to set these parameters, including random sampling and uniform deploying. In this paper, a common covariance matrix is set for all collocation Gaussian components, such that $\Sigma^{\imath}_c = \Sigma_c$, and the means of the collocation Gaussian components are uniformly deployed on the ROI with the same spatial interval.   
 
\textit{Assumption} \ref{ass:collocation_GC} is an extension of \textit{Assumption} \ref{ass:wp_in_space_GMM} by assuming that the Gaussian components,  
$g^{\imath}_{\tau}$, $k+1 \leq \tau \leq T_f$,
in  (\ref{eq:wp_k_in_GMM})  all belong to the set of collocation Gaussian components, such that  $g^\imath_{\tau} \in \boldsymbol{G}$. According to (\ref{eq:wp_in_subspace}), thus,  $\wp_{\tau}$ belongs to a sub-space of the GMM, $\tilde{\mathcal{G}}(\mathcal{W},\boldsymbol{G}) \subset \mathcal{G}(\mathcal{W})$, which is defined by
\begin{align}
    \tilde{\mathcal{G}}(\mathcal{W},\boldsymbol{G}) & \triangleq \bigg\{\wp \bigg\vert \wp = \sum_{\imath=1}^{N_c} \omega_{\imath} g^{\imath}_c, \sum_{\imath}^{N_c} \omega_{\imath} = 1, \nonumber\\
    & 0 \leq \omega_{\imath} \leq 1, \imath = 1,\ldots,N_c \bigg\}
\end{align}

With \textit{Assumption} \ref{ass:collocation_GC}, first, the number of Gaussian component of $\wp_{k+1}$ is known, such that $N_{k+1} = N_c$. Second, the metric $W_2(\cdot,\cdot)$ in (\ref{eq:L_k}) and (\ref{eq:tilde_L_k}) can be calculated in advance. 
%, including $W_2(g^{\imath}_c,g^{\imath^{\prime}}_c)$ and $W_2(g^i_c,g^j_{targ})$, $\imath,\imath^{\prime} = 1,\ldots,N_c$ and $j = 1,\ldots,N_{targ}$.
Finally, due to $g^{\imath}_{\tau} \in \boldsymbol{G}$, $\tau = k+1,\ldots,T_f$, the NLP in (\ref{eq:L_k}) can be solved by using the shortest-path-planing algorithms \cite{ThorupShortestPathPlanning2004}. 

Let $\boldsymbol{V} = \boldsymbol{G} \cup \{ g^j_{targ}\}_{j=1}^{N_{targ}}$ denote a set of Gaussian components, which are treated as nodes in a graph. Let $\mathcal{E}_k$ denote a set of edges between nodes with respect to the obstacle map function $m_k$, such that 
\begin{equation}
    \mathcal{E}_k = \big \{ e_{i,j} \big \vert e_{i,j} = [W_2(g_i,g_j)]^2 + \langle g_j, m_k \rangle_{\mathcal{W}}, \text{ and } g_i,g_j \in \boldsymbol{V} \big\}
\end{equation}
Then, a directed graph $\mathcal{DG}_k = (\boldsymbol{V}, \mathcal{E}_k)$ is defined. The cost of Gaussian component trajectory, $\tilde{\mathcal{L}}^{\imath,j}_k$ in (\ref{eq:L_k}) can be explained as the shortest path from the node $g^{\imath}_{k+1} \in \boldsymbol{V} $ to the node $g_j^{targ}\in \boldsymbol{V}$ in the directed graph $\mathcal{DG}_k$.

Therefore, given the set  of  collocation Gaussian components $\boldsymbol{G}$, under \textit{Assumption} \ref{ass:collocation_GC} the  functional control law is only specified by $\pi_k$. The optimal functional control law, thus, can be approximated by solving the following optimization problem,
\begin{align}
\hat{\pi}_k^* &= \underset{\pi_k}{\arg\min} \sum_{\imath=1}^{N_{c}} \bigg[ \sum_{i=1}^{N_k} \mathcal{L}^{i,\imath}_k
\pi_k(i,\imath) %\nonumber\\
+  \sum_{j=1}^{N_{targ}} \tilde{\mathcal{L}}^{\imath,j}_k \tilde{\pi}_k(\imath,j) \bigg]\nonumber\\
\text{s.t. } & \quad (\ref{eq:constraint_1}) - (\ref{eq:constraint_3}) 
\label{eq:Theta_k_next_3}
\end{align}
and the optimal robot PDF $\wp^*_{k+1}$ can be approximated by
\begin{equation}
\hat{\wp}^*_{k+1} = \sum_{i=1}^{N_k}\sum_{\imath =1}^{N_c} \hat{\pi}^*_k(i,\imath) g^{\imath}_c = \sum_{\imath=1}^{N_c}(\hat{\omega}_{\imath}^{k+1})^* g^{\imath}_c
\label{eq:hat_pi_k_next}
\end{equation}

In addition, given the costs of trajectories of Gaussian components, $\{\mathcal{L}^{i,\imath}_k \}_{i=1,\imath=1}^{N_k,N_c}$ and $\{\tilde{\mathcal{L}}^{\imath,j}_k\}_{\imath=1,j=1}^{N_c,N_{targ}}$, the approximated optimal value functional only depends on $\pi_k$ and $\tilde{\pi}_k$ with the constraints in (\ref{eq:constraint_1}), (\ref{eq:constraint_2}) and (\ref{eq:constraint_3}). Therefore, the optimal control law $\mathcal{C}_k^*$ can be approximated by using linear programming (LP) algorithms.

According to (\ref{eq:hat_pi_k_next}), the approximated optimal robot PDF, $\hat{\wp}_{k+1} \in \tilde{\mathcal{G}}(\mathcal{W},\boldsymbol{G})$, is specified by $N_c$ weight coefficients, $(\hat{\omega}^{\imath}_{k+1})^*$, $\imath = 1,\ldots,N_c$, where many weight coefficients are zeros or very small numbers. To reduce the computational complexity,  these Gaussian components with the weight coefficients that are smaller than a given threshold, $\omega_{th}$, are removed, and the remaining weight coefficients are normalized to generate the robot PDF at the $(k+1)$th time step, $\wp_{k+1} = \sum_{i=1}^{N_{k+1}} \omega^i_{k+1} g^{i}_{k+1} $, where $N_{k+1} \ll N_c$. 





\subsection{Simplified Approximation of Optimal Control Law}
 Although the optimal $\wp_{k+1}$ can be approximated in (\ref{eq:hat_pi_k_next}) by using LP algorithms, this optimization problem cannot be implemented online with acceptable performance. Because the computational complexities of $\{\mathcal{L}^{i,\imath}_k \}_{i=1,\imath=1}^{N_k,N_c}$ and $\{\tilde{\mathcal{L}}^{\imath,j}_k\}_{\imath=1,j=1}^{N_c,N_{targ}}$ increase dramatically with the increase of the number of collocation Gaussian components.
The following assumption is considered to reduce the computational complexity of the approximation of the optimal control law. 
%\begin{assumption}
%\label{ass:number_next_GC}
%Consider the control law, $\mathcal{C}_k$, 
%%under \textit{Assumption} \ref{ass:collocation_GC}, 
%such that $\wp_{l+1} = \mathcal{C}_k(\wp_l)$, $l = k, \ldots, T_f-1$.  Given the robot PDF, $\wp_l = \sum_{i=1}^{N_l} \omega^i_l g^i_l$, and $\wp_{l+1}=\sum_{\imath=1}^{N_c} \omega^{\imath}_{l+1} g^{\imath}_{c} \in \tilde{\mathcal{G}}(\mathcal{W},\boldsymbol{G})$, the control law can be specified by the joint probability $\pi_l$, an $N_l \times N_c$ matrix. Assume that the transportation distance of Gaussian compoenents under the control law at any time step is less than a given distance threshold, $d_{th}$, such that  
%\begin{align}
%\pi_l(i,\imath) = 0  \text{ if } W_2(g^i_l, g^{\imath}_{l+1}) > d_{th}, \nonumber\\
%i = 1,\ldots,N_l, \text{ and } \imath = 1,\ldots,N_c
%\end{align} 
%\end{assumption}
\begin{assumption}
	\label{ass:number_next_GC}
	  Given the robot PDF, $\wp_l = \sum_{i=1}^{N_l} \omega^i_l g^i_l$, and $\wp_{l+1}=\sum_{\imath=1}^{N_c} \omega^{\imath}_{l+1} g^{\imath}_{c} \in \tilde{\mathcal{G}}(\mathcal{W},\boldsymbol{G})$, the control law, $\wp_{l+1} = \mathcal{C}_k(\wp_l)$, $l = k, \ldots, T_f-1$, can be specified by the joint probability $\pi_l$, which is an $N_l \times N_c$ matrix. Assume that the transportation distance of Gaussian compoenents under the control law at any time step is less than a given distance threshold, $d_{th}$, such that  
	\begin{align}
	\pi_l(i,\imath) = 0  \text{ if } W_2(g^i_l, g^{\imath}_{l+1}) > d_{th}, \nonumber\\
	i = 1,\ldots,N_l, \text{ and } \imath = 1,\ldots,N_c
	\end{align} 
\end{assumption}

Let $\mathcal{I} = \{1,\ldots,N_c\}$ denote the index set of the collocation Gaussian components. For each Gaussian component of $\wp_l = \sum_{i=1}^{N_l} \omega^i_l g^i_l$, $l = k,\ldots,T_f-1$, these exists a subset of the index set $\mathcal{I}$, denoted by $\mathcal{I}^i_l \subset \mathcal{I}$, such that
\begin{equation}
\mathcal{I}^i_l = \{ \imath \vert \imath \in \mathcal{I}, \, g^{\imath}_c \in \boldsymbol{G}, \, \text{and } W_2(g^i_l, g^{\imath}_c) \leq d_{th}\}
\end{equation}  
The union of these subsets are denoted by $\mathcal{I}_l = \cup_{i = 1}^{N_l} \mathcal{I}^i_l$.
Also,  there exists a subset $\mathcal{I}^C_l \in \mathcal{I}$, such that
\begin{equation}
\mathcal{I}^C_l = \{\imath \vert \imath \in \mathcal{I} \text{ and } \imath \notin \mathcal{I}_l\}
\end{equation}
which is the complement of $\mathcal{I}_l$. 

%The cardinality of the subset $\boldsymbol{G}_i$ is denoted by $\vert \boldsymbol{G}_i \vert$. Therefore, under \textit{Assumption} \ref{ass:number_next_GC}, the robot PDF, $\wp_{l+1} \in \boldsymbol{G}_i$, is expressed by
%\begin{equation}
%\wp_{l+1} = \sum_{\imath = 1}^{\vert \boldsymbol{G}_i \vert} \tilde{\omega}_{\imath}^{l+1} g_{\imath}^{l+1}
%\end{equation}
%where the weight coefficients, $\tilde{\omega}_{\imath}^{l+1}$, are applied to distinguish the weigh coefficients, $\omega_{\imath}^{l+1}$, and the Gaussian components $g_{\imath}^{l+1} \in \boldsymbol{G}_i$. 
Then, with  \textit{Assumption} \ref{ass:number_next_GC} the approximation of the optimal control law in (\ref{eq:Theta_k_next_3}) can be rewitten as
\begin{align}
\hat{\pi}_k^* &= \underset{\pi_k}{\arg\min} \sum_{i=1}^{N_k} \sum_{j=1}^{N_{targ}} \bigg\{\sum_{\imath \in \mathcal{I}^i_k} \bigg[  \mathcal{L}^{i,\imath}_k
\pi_k(i,\imath) %\nonumber\\
+   \tilde{\mathcal{L}}^{\imath,j}_k \tilde{\pi}_k(\imath,j) \bigg]\bigg\}\nonumber\\
\text{s.t. } & \quad (\ref{eq:constraint_1}) - (\ref{eq:constraint_3}), \nonumber\\
& \quad \pi_k(i,\imath) = 0,  \; i = 1,\ldots,N_k \text{ and } \imath \notin \mathcal{I}^i_k, \nonumber\\
& \quad \tilde{\pi}_k(\imath,j) = 0, \, j = 1,\ldots,N_{targ} \text{ and } \imath \in \mathcal{I}_k^C
\label{eq:hat_pi_k_d_th}
\end{align} 
To solve the LP problem in (\ref{eq:hat_pi_k_d_th}), only $\vert \mathcal{I}_k \vert \times N_{targ}$ shortest-paths in the directed graph $\mathcal{DG}_k$ are required to calculate $\tilde{\mathcal{L}}_k^{\imath,j}$ at each time step, where ``$\vert \cdot \vert$" indicates the cardinality of a set. 


\subsection{Interpolation of Robot PDFs in Sub-space of GMM} 
By using (\ref{eq:hat_pi_k_d_th}), the optimal control law and the corresponding control can be approximated based on the collocation Gaussian components. However, the distribution velocity defined in (\ref{eq:distribution_velocity_WG}) is affected by the setting of the collocation Gaussian components.
%, specifically, which primarily dependents on $W_2(g^{\imath}_c,g^{\imath^{\prime}}_c)$. 
To remove the affect from the collocation Gaussian components, the distribution distance between two sequentially obtained robot PDFs is divided evenly by a user-defined distribution interval, $\bar{d}$, and more robot PDFs are interpolated between them.    

Let the approximate optimal control, $\wp_k^{goal} = \hat{\mathcal{C}}_k^*(\wp_k,m_k)$, denote the goal PDF at the $k$th time step, instead of the PDF at $(k+1)$th time step. The distribution distance, $d(\wp_k,\wp_k^{goal})$, is divided into $T_k = \lceil d(\wp_k,\wp_k^{goal})/\bar{d} \rceil$ segments, where ``$\lceil \cdot \rceil$" indicates the ceiling operator. Then, $(T_k-1)$ robot PDFs are interpolated between $\wp_k$ and $\wp_k^{goal}$ according to (\ref{eq:distribution_interpolation}), and the PDF $\wp_k^{goal}$ is treated as $\wp_{(k+T_k)}$. Therefore, the distribution velocity can be expressed by
\begin{equation}
v_{\tau}^{\wp} = \frac{d(\wp_k,\wp_k^{goal})}{T_k \cdot \Delta t} \approx \frac{\bar{d}}{\Delta t}\triangleq\bar{v}^{\wp}, \quad k\leq \tau \leq k+T_k
\label{eq:appro_distribution_velocity}
\end{equation} 
It means that the robot PDFs travel at a relatively smooth velocity. If $\bar{d} \ll d(\wp_k,\wp_k^{goal})$, the distribution velocity can be treated as a constant. 

\subsection{Optimal Microscopic Control Law}
\label{subsec:Microscopic_Control_Law}
Given the matrix of the robot microscopic states $\mathbf{X}_k = [\mathbf{x}_1(t_k),\ldots,\mathbf{x}_N(t_k)]$ at the $k$th time step and the approximated optimal control $\hat{\wp}^*_{k+1}$, the optimal microscopic control law can be determined by the artificial potential field method. The attractive potential is 
\begin{equation}
U^{attr}_k = %\underset{\mathbf{U}_k}{\arg\min} 
\int_{\mathcal{W}} \big[\hat{\wp}^*_{k+1} - \gamma \tilde{\wp}_{k+1}(\mathbf{X}_k, \mathbf{U}_k))\big]^2(\mathbf{x}) d\mathbf{x}
\label{eq:distribution_potential}
\end{equation} 
where $\mathbf{U}_k = [\mathbf{u}_1(t_k),\ldots,\mathbf{u}_N(t_k)]$ is the matrix of the microscopic controls at the $k$th time step, and $\tilde{\wp}_{k+1}(\mathbf{X}_k, \mathbf{U}_k)$ is the estimated robot PDF at the $(k+1)$th time step given the microscopic states $\mathbf{X}_k$ and the microscopic controls $\mathbf{U}_k$. In this paper, the kernel density estimation (KDE) method is applied to estimate the PDF from microscopic robot states. Here, $0 < \gamma \leq 1$ is a scalar parameter to control the scattering strength of the robots. 
A larger $\gamma$ results in a more scattered distribution of robots. Meanwhile, for every individual robot, the repulsive forces from  the observed obstacles and the other robots are also considered. Let $\rho \left(\mathbf{x}_n(t_k),\mathbf{u}_n(t_k)\right)$ denote the shortest distance between the robot $\mathbf{x}_n(t_{k+1})$ and the obstacles or the other robots, given the control input $\mathbf{u}_n(t_k)$. The repulsive potential for the $n$th robot is defined by
\begin{equation}
U^{rep}_{k,n} (\rho) = \begin{cases}
\frac{1}{2} \left( \frac{1}{\rho} - \frac{1}{\rho_{rep}}\right)^2, & \text{ if } \rho \leq \rho_{rep} \\
0, & \text{otherwise} 
\end{cases}
\label{eq:repulsitive_potential}
\end{equation}
where $\rho_{rep}$ is the repulsive distance threshold to create a repulsion effect on the robot. The microscopic control inputs can be determined according to the sum of the attractive and repulsive gradients. More details can be found in \cite{ZhuGDM2019}. 

\section{Simulations and Results}
\label{sec:Simulations_Results}
In this section, a synthetic simulation is presented to evaluate the proposed ADOC approach, and the performance of the ADOC approach is compared with three state-of-the-art approaches in this section.

\subsection{Simulation Objectives and Settings}
The effectiveness of the ADOC approach presented in the previous sections is demonstrated on a network of $N=500$ mobile robots with single-integrator dynamics
\begin{equation}
\dot{\mathbf{x}}_n(t) = \mathbf{u}_n(t), \mathbf{x}_n(t_0) = \mathbf{x}_{n_0}, \quad n = 1,\ldots,N
\end{equation}
where $\mathbf{x}_n = [x_n,\,y_n]^T$ is the robot state, $x_n$ and $y_n$ are the robot $xy$-coordinates in the inertial frame, $\mathbf{u}_n = [u_n^x, u_n^y]^T$ is the microscopic control input, and $u_n^x$ and $u_n^y$ indicate the linear velocities in the $x$- and $y$-direction, respectively. Here, the noisy $\mathbf{w}(t)$ in (\ref{eq:dynamics}) is ignored for simplicity. 

As mentioned in Section \ref{sec:Problem_Formulation}, the objectives of the VLSR system is to travel from a given initial distribution $\wp_0 = \sum_{i=1}^{N_0}\omega_0^i g^i_0$ to a target distribution $\wp_{targ} = \sum_{j=1}^{N_{targ}} \omega^j_{targ} g^j_{targ}$ while avoiding collisions with the obstacles deployed in an ROI, $\mathcal{W} = [0,L_x] \times [0,L_y]$, where $N_0 = 4$, $N_{targ} = 3$, $L_x = 20$ km and $L_y = 16$ km. The initial and target robot distributions are presented in Fig. \ref{fig:Initial_Target_PDFs}. At the initial time, the a priori layout of obstacles is inaccurate (Fig. \ref{fig:prioiri_layout}). The actual layout of obstacles (Fig. \ref{fig:actual_layout}) is not available and needs to be observed and updated by the identical omnidirectional range sensors equipped on the mobile robots at every time step. In this simulation, the radius of the sensor FOV is $r = 1$ km. 
\begin{figure}[htp]
	\centering
	\subfloat[]{\includegraphics[width=2.5in]{fig_InitialPDF}}
	\hfil
	\subfloat[]{\includegraphics[width=2.5in]{fig_TargetPDF}}
	\caption{Goal of VLSR system is to travel from an initial robot distribution shown in (a) to a goal robot distribution shown in (b).}
	\label{fig:Initial_Target_PDFs}
\end{figure}
%%%%%%%%%%%%%%%%%%%%%%%%%%%%%% Delete for short %%%%%%%%%%%%%%%%%%%%%%%%%%%%%%%%%%%%%
\begin{figure}[htp]
	\centering
	\subfloat[]{\includegraphics[width=2.5in]{fig_PrioriLayout}\label{fig:prioiri_layout}}
	\hfil
	\subfloat[]{\includegraphics[width=2.5in]{fig_TrueLayout}\label{fig:actual_layout}}
	\caption{Layout of obstacles. (a) A-priori layout of obstacles (b) Actual in situ obstacles }
	\label{fig:Priori_True_layout}
\end{figure}
%%%%%%%%%%%%%%%%%%%%%%%%%%%%%% Delete for short %%%%%%%%%%%%%%%%%%%%%%%%%%%%%%%%%%%%%

\subsection{Simulation of ADOC Approach}
%%%%%%%%%%%%%%%%% Delete for short %%%%%%%%%%%%%%%%%%%%%%%%%%%%%%%%%%%%%%%%%%%%%%%%
For the ADOC approach, the observed and updated layout of obstacles at the $k$th time step is represented by the map function defined in (\ref{eq:map_function}), %Fig. \ref{fig:HM}, 
which is evaluated at the collocation points. These collocation points are deployed on the ROI $\mathcal{W}$ with the even spatial intervals, $\Delta x = \Delta y = 0.1$ km. Thus, there are $160 \times 200 = 32000$ collocation points in total. 
%%%%%%%%%%%%%%%%% Delete for short %%%%%%%%%%%%%%%%%%%%%%%%%%%%%%%%%%%%%%%%%%%%%%%%

As described in \textit{Assumption} \ref{ass:collocation_GC}, the collocation Gaussian components, $\boldsymbol{G} = \{g^1_{c},
\ldots,g^{N_{c}}_{c}\}$, are set in the ROI $\mathcal{W}$, where $\mu^{\imath}_c = [\imath_x - 0.5, \, \imath_y - 0.5]$ km, $\imath_x = 1,\ldots,L_x$ and $\imath_y = 1,\ldots, L_y$, and $\Sigma_c = \begin{bmatrix}
0.5 & 0 \\ 0 & 0.5
\end{bmatrix}$ km\textsuperscript{2}. Thus, there are $N_c = 16 \times 20 = 320$ collocation Gaussian components deployed in the ROI and $N_c + N_{targ} = 320 + 3 = 323$ nodes in the directed graph, $\mathcal{DG}_k = (\boldsymbol{V}, \mathcal{E}_k)$, described in Section \ref{subsec:Approximation_in_SubSpace}. The distance threshold for the directed graph defined in \textit{Assumption} \ref{ass:number_next_GC} is set to $d_{th} = 4$ km. 

The fixed time interval is set to $\Delta t = 0.01$ hr. The user-defined distribution interval in (\ref{eq:appro_distribution_velocity}) is set to $\bar{d} = 0.05$ km to obtain a relatively constant distribution velocity, such that $\bar{v}^{\wp} = \frac{\bar{d}}{\Delta t} = \frac{0.05}{0.01} = 5$ km/hr. 

For the microscopic control, the scalar parameter in (\ref{eq:distribution_potential}) is set to $\gamma = 0.85$. Furthermore, considering that the uncertainty of the observed obstacles, different repulsive distance thresholds are set to create the repulsion effects from the obstacles and the other robots, respectively, including the repulsive distance threshold for obstacles, $\rho_{rep}^{obs} = 0.3$ km, and the repulsive distance threshold for the other robots, $\rho_{rep}^{rob} = 0.1$ km.          
%%%%%%%%%%%%%%%%%%%%%%%%%%%%% Delete for short %%%%%%%%%%%%%%%%%%%%%%%%%%%%%%
The proposed ADOC approach is applied to generate the approximated optimal macroscopic control law and the corresponding microscopic control inputs, which control the VLSR system to travel online in the ROI. The generated trajectory of robot PDFs are presented in Fig. \ref{fig:PDF_Trajectory_ADOC}, where the gray areas indicate the obstacles updated at each time step. From Fig. \ref{fig:PDF_ADOC_300}, it can be observed that at around $3$th hr, the new obstacle is observed and an original open path is blocked and the VLSR system can find new path adaptively to the target distribution. 
%%%%%%%%%%%%%%%%%%%%%%%%%%%%% Delete for short %%%%%%%%%%%%%%%%%%%%%%%%%%%%%%
%%%%%%%%%%%%%%%%%%%%%%%%%%%%% Delete for short %%%%%%%%%%%%%%%%%%%%%%%%%%%%%%
Positions and FOVs of robots at $t_1 = 2.5$ hr and $t_2 = 2.8$ hr are presented in Fig. \ref{fig:Trajectory_Take_Turning}, where the FOVs of three robots are plotted in three different colors. 
The microscopic state and control histories of these three robots are plotted in Fig. \ref{fig:Microscoic_state_control_history}. 
%%%%%%%%%%%%%%%%%%%%%%%%%%%%% Delete for short %%%%%%%%%%%%%%%%%%%%%%%%%%%%%%
%%%%%%%%%%%%%%%%%%%%%%%%%%%%% Delete for short %%%%%%%%%%%%%%%%%%%%%%%%%%%%%%
These results show that the VLSR system takes $T_f = 701$ steps and $T_f \Delta t = 701 \times 0.01 = 7.01$ hrs to successfully travel from the initial distribution to the target distribution under the control of the proposed ADOC approach.    
\begin{figure}[htp]
	\centering
	\subfloat[]{\includegraphics[width=2.5in]{fig_PDF_ADOC_100}}
	\hfil
	\subfloat[]{\includegraphics[width=2.5in]{fig_PDF_ADOC_300} \label{fig:PDF_ADOC_300}}
	\hfil
	\subfloat[]{\includegraphics[width=2.5in]{fig_PDF_ADOC_500}}
	\hfil
	\subfloat[]{\includegraphics[width=2.5in]{fig_PDF_ADOC_700}}
	\caption{Evolution of sensor PDFs generated by the ADOC approach at four moments, where the gray areas indicate the obstacles updated at each time step.}
	\label{fig:PDF_Trajectory_ADOC}
\end{figure}
%%%%%%%%%%%%%%%%%%%%%%%%%%%%% Delete for short %%%%%%%%%%%%%%%%%%%%%%%%%%%%%%

%%%%%%%%%%%%%%%%%%%%%%%%%%%%% Delete for short %%%%%%%%%%%%%%%%%%%%%%%%%%%%%%
\begin{figure*}[htp]
	\centering
	\subfloat[]{\includegraphics[width=2in]{fig_FOV_all_250}}
	\hfil
	\subfloat[]{\includegraphics[width=2in]{fig_FOV_250}}
	\hfil
	\subfloat[]{\includegraphics[width=2in]{fig_ZoomIn_250}}
	\hfil
	\subfloat[]{\includegraphics[width=2in]{fig_FOV_all_280}}
	\hfil
	\subfloat[]{\includegraphics[width=2in]{fig_FOV_280}}
	\hfil
	\subfloat[]{\includegraphics[width=2in]{fig_ZoomIn_280}}
	\caption{ Positions and FOVs of robots at two different moments are presented in two rows, including $t_1=2.5$ hrs and $t_2=2.8$ hrs. The figures in the first column, (a) and (d), present the robot positions in the whole ROI. The figures in the second column, (b) and (e),are generated by zooming in the red bounding boxes in (a) and (d), respectively, where the FOVs of three robots are plotted in three different colors. The cyan areas indicate the obstacles which are observed during the past $0.1$ hrs. The figures in the last column, (c) and (f), are generated by zooming in the red bounding boxes in (b) and (e), respectively, where the robots and obstacles are plotted from a 3D viewpoint.}
	\label{fig:Trajectory_Take_Turning}
\end{figure*}
%%%%%%%%%%%%%%%%%%%%%%%%%%%%% Delete for short %%%%%%%%%%%%%%%%%%%%%%%%%%%%%%

%%%%%%%%%%%%%%%%%%%%%%%%%%%%% Delete for short %%%%%%%%%%%%%%%%%%%%%%%%%%%%%%
\begin{figure}[htp]
	\centering
	\includegraphics[width=3.2in]{fig_Position_Velocity}
	\caption{Microscopic state and control histories for three mobile robots chosen from the VLSR system.}
	\label{fig:Microscoic_state_control_history}
\end{figure}
%%%%%%%%%%%%%%%%%%%%%%%%%%%%% Delete for short %%%%%%%%%%%%%%%%%%%%%%%%%%%%%%

\subsection{Comparison of Performance}
To the best of our knowledge, there is no existing hoc ad approach for the problem described in Section \ref{sec:Problem_Formulation}, because of the number of robots is too large. Thus, three different benchmark approaches are developed based on the existing state-of-the-art techniques and compared with the proposed ADOC approach, including  probability-density-function-based artificial potential field (PDF-APF) approaches,
 Sampling-based artificial potential field (SAPF), and the sampling-based path-planning (SPP). 

\begin{itemize} 
	\item PDF-APF: Because the target PDF $\wp_{targ}$ is given, the attractive potential function in (\ref{eq:distribution_potential}) can be modified by
	\begin{equation}
	U^{attr}_k = %\underset{\mathbf{U}_k}{\arg\min} 
	\int_{\mathcal{W}} \big[\wp_{targ} - \gamma \tilde{\wp}_{k+1}(\mathbf{X}_k, \mathbf{U}_k))\big]^2(\mathbf{x}) d\mathbf{x}
	\label{eq:distribution_potential_target_PDF}
	\end{equation}
	Meanwhile, consider the repulsive potential function with respect to the updated obstacles and robots in (\ref{eq:repulsitive_potential}). Then, the microscopic control inputs can be obtained according to the sum of the gradients of potential functions similar to Section \ref{subsec:Microscopic_Control_Law}.
	
	\item SAPF: Since the robot target PDF and the number of robots are both known, the set of the target robot positions $\mathcal{X}^{targ} = \{\mathbf{x}_n^{targ}\}_{n=1}^N$ can be obtained by sampling according to $\wp_{targ}$. Then, these target positions can be treated as attractive points generating the individual potentials to every robot, such that
	\begin{equation}
		U_{n,n^{\prime},k}^{attr} = -\Vert \mathbf{x}_{n}(t_k) - \mathbf{x}_{n^{\prime}}^{targ} \Vert^{-2}, \; n, n^{\prime} = 1,\ldots,N
	\end{equation} 
	which is the attractive potential from the $n^{\prime}$th attractive point to the $n$th robot. The attractive point exist if and only if the attractive position has not been occupied by any robot yet. Similarly, the repulsive potentials in (\ref{eq:repulsitive_potential}) are also considered to calculate the microscopic control inputs. 
    \item SPP: Like the SAPF approach, the target robot positions are obtained by sampling  according to $\wp_{targ}$. Then, the initial robot positions, $\mathcal{X}_0 = \{\mathbf{x}_n(t_0)\}_{n=1}^N$ and $\mathcal{X}^{targ}$ are paired according to their relative distances. Finally, the shortest-path-planing algorithm is applied with the time-varying map function $m_k$ to obtain the $N$ paths from the initial robot positions to the corresponding target positions while avoiding the collisions by using the repulsive potentials in (\ref{eq:repulsitive_potential}). 
\end{itemize}

Since there is no macroscopic state $\wp_k$ and the corresponding velocity $v^{\wp}_k$ for these compared approaches,  the microscopic velocities for robots are all set to $v^{rob} = 5$ km/hr. For the SPP approach, the shortest-path-planning algorithms are implemented based on a graph where the set of nodes is defined by $\mathbf{V}_{SPP} = \{\mu_c^{\imath}\}_{\imath = 1}^{N_c} \cup \mathcal{X}^{targ}$. For fairness reasons, the same parameters, \textit{e.g.}, $\rho_{rep}^{obs}$, $\rho_{rep}^{rob}$, and $d_{th}$, are applied in these approaches. Furthermore, although the time during $t_f$ is not required by these approaches, a maximum of the time steps is set to $T_f^{\max} = 2000$, which means that the algorithms will stop at the $T_f^{\max} = 2000$th time step even if the task is not completed.  
%%%%%%%%%%%%%%%%%%%%%%%%%%%%% Delete for short %%%%%%%%%%%%%%%%%%%%%%%%%%%%%%
%%%%%%%%%%%%%%%%%%%%% Comparison of Computational Complexities %%%%%%%%%%%%%%%%%%%%%%

Among these approaches, ADOC and SPP approaches involve the path-planning algorithms, while all of the approaches use artificial potentiates to obtain the microscopic controls, including the attractive and repulsive potentials. Given the number nodes $N_{node}$ and number of edges $N_{edge}$ in a directed graph, the computational complexity of planning a shortest path from a one single source to a single target is $O\big(N_{edge} + N_{node} \log\log(N_{node})\big)$ according to \cite{ThorupShortestPathPlanning2004}. Moreover, the calculations of artificial potentials between attractive or repulsive points and every robots are all considered. Let $N_k^{obs}$ denote the number of collocation points occupied by obstacles up to the $k$th time step. Thus, the computational complexities at the $k$th time step of these four approaches are tabulated in TABLE \ref{tab:computational_complexity}. It is noteworthy that for the microscopic control, since $N_k \ll N$ and $N_{targ} \ll N$, the computational complexities for all approaches are the same. Furthermore, for the case where $N_k \cdot \vert \mathcal{I}_k \vert \cdot N_{targ} < N$, the proposed ADOC approach outperforms the SPP approach in the aspect of computational complexity. 
\begin{table*}
	\caption{Comparison of Computational Complexity}
	\label{tab:computational_complexity}
	\centering
	\begin{tabular}{c|c|c}
		\hline
		Approach & Path Planning & Microscopic Control  \\
		\hline
		ADOC     & 	$O\bigg( \big( \vert \mathcal{E}_k \vert  + \vert \boldsymbol{V} \vert  \log \log \vert \boldsymbol{V} \vert \big) N_k \vert \mathcal{I}_k\vert N_{targ} \bigg)$			 &			$O(NN_k + N N_k^{obs} + N^2)$			\\
		\hline	 
		PDF-APF &		N/A		 &	$O(NN_{targ} + N N_k^{obs} + N^2)$					\\
		\hline
		SAPF	&       N/A      &  $O(N N_k^{obs} + N^2)$      \\
		\hline
		SPP	    &    $O\bigg( \big( \vert \mathcal{E}_k^{SPP} \vert  + \vert \mathbf{V}^{SPP} \vert  \log \log \vert \mathbf{V}^{SPP} \vert \big) N \bigg)$ & $O(N N_k^{obs} + N^2)$\\
		\hline
	\end{tabular}
\end{table*}   
%%%%%%%%%%%%%%%%%%%%% Comparison of Computational Complexities %%%%%%%%%%%%%%%%%%%%%%
%%%%%%%%%%%%%%%%%%%%%%%%%%%%% Delete for short %%%%%%%%%%%%%%%%%%%%%%%%%%%%%%



The trajectories of robots are plotted in Fig. \ref{fig:Robot_Trajectories}, which are generated by the different  approaches, including ADOC, PDF-APF, SAPF, and SPP. It can be observed that the ADOC and the SPP approaches complete the task successfully, while the PDF-APF and the SAPF approaches fail within $T_f^{\max} = 2000$ time steps. More numerical performances of these approaches are tabulated in TABLE \ref{tab:numerical_performances}, including the total time steps, $T_f$, the running time, the average distance-to-go $\bar{D}_{rob}(k)$, and the average energy-cost per kg, $\bar{E}_{rob}(k)$, which are defined by
\begin{equation}
\bar{D}_{rob}(k) = \frac{1}{N} \sum_{n=1}^N \sum_{\tau = k}^{T_f-1} \Vert   \mathbf{x}_n\big((\tau+1)\Delta t\big) - \mathbf{x}_n(\tau \Delta t) \Vert 
\end{equation}

\begin{equation}
\bar{E}_{rob}(k) = \frac{\eta}{2N} \sum_{n=1}^N \sum_{\tau = 1}^{k} \left[\frac{ \Vert \mathbf{x}_n(\tau \Delta t) - \mathbf{x}_n\big((\tau-1)\Delta t\big) \Vert }{\Delta t} \right]^2
\end{equation}
where $\eta = (1000/3600)^2 $ is applied for unit conversion. Here, these approaches run on the same PC with 18 cores. 
%%%%%%%%%%%%%%%%%%%%%%%%%%%%% Delete for short %%%%%%%%%%%%%%%%%%%%%%%%%%%%%%
The curves of the distance-to-go and the energy-cost-per-kg are  plotted in Fig. \ref{fig:Robot_performances}. 
%%%%%%%%%%%%%%%%%%%%%%%%%%%%% Delete for short %%%%%%%%%%%%%%%%%%%%%%%%%%%%%%
These results show that the proposed ADOC approach complete the task more efficiently and effectively than the other approaches. 

\begin{table}
	\caption{Comparison of Numerical Performances}
	\label{tab:numerical_performances}
	\centering
	\begin{tabular}{c|c|c|c|c}
		\hline
		         &       &                       &                 & \\
		Approach & $T_f$ & Running Time & $\bar{D}_{rob}(0) $ & $\bar{E}_{rob}(T_f)$  \\
		         &       & (min)               &  (km)    &  (J/kg)      \\
		\hline
		ADOC     & 	 \textbf{701} &	\textbf{37.0544} &  \textbf{28.5623}   & \textbf{545.1525}		\\
		\hline	 
		PDF-APF & 	2000 & 60.2447 &  99.95    &	1928.0478 	\\
		\hline
		SAPF	& 	2000 &   19.6193 &  100.8671   &  3747.0543   \\
		\hline
		SPP	    & 	1414 &  141.2573 &  41.9044   &  808.1635   \\
		\hline
	\end{tabular}
\end{table}   


\begin{figure}[htp]
	\centering
	\subfloat[]{\includegraphics[width=2.5in]{fig_sensorTrajector_finalPosition_ADOC}}
	\hfil
	\subfloat[]{\includegraphics[width=2.5in]{fig_sensorTrajector_finalPosition_PDF_APF}} 
	\hfil
	\subfloat[]{\includegraphics[width=2.5in]{fig_sensorTrajector_finalPosition_SAPF}}
	\hfil
	\subfloat[]{\includegraphics[width=2.5in]{fig_sensorTrajector_finalPosition_SPP}}
	\caption{Trajectories of robots generated by the different four approaches, including (a) ADOC, (b) PDF-APF, (c) SAPF, and (d) SPP, where the initial and final positions of robots are indicated by circles and diamonds, respectively, and the gray areas indicate the true obstacles in the workspace.}
	\label{fig:Robot_Trajectories}
\end{figure}


%%%%%%%%%%%%%%%%%%%%%%%%%%%%% Delete for short %%%%%%%%%%%%%%%%%%%%%%%%%%%%%% 
\begin{figure}[htp]
	\centering
	\subfloat[]{\includegraphics[width=2.5in]{fig_average_distance}}
	\hfil
	\subfloat[]{\includegraphics[width=2.5in]{fig_average_engery_cost}} 
	\caption{Performances of the VLSR systems generated by four approaches. (a) the average distance-to-go and (b) the engery-cost-per-kg.}
	\label{fig:Robot_performances}
\end{figure}
%%%%%%%%%%%%%%%%%%%%%%%%%%%%% Delete for short %%%%%%%%%%%%%%%%%%%%%%%%%%%%%%


%\subsection{Comparison of Computational Complexities}
%\label{subsec:computational_complexity}

 

%\section{Discussion}




\section{Conclusion and Future Research}
In this paper, an ADOC approach is proposed to carry out online cooperative sensing and navigation tasks for the VLSR systems in highly uncertain environments, which is an extension of the DOC approach, where the VLSR system is described by the robot PDFs referred to as the macroscopic state. Because This approach is formulated as an RL-ADP problem in the Wasserstein-GMM space based on the optimal mass transport (OMT) theorem, it can also be considered as an online MARL approach. To the best of our knowledge, this is the first implementation of online MARL for continuous states and controls, which provide a novel research direction of online MARL approaches. 

The proposed ADOC is a centralized approach where the optimal control law and the corresponding microscopic controls are all generated based on observations from all robots. Considering that the GMM is a linear combination of Gaussian components, it is feasible to divide the generation process of optimal control law into several groups according to the Gaussian components. Then, a decentralized version of the ADOC approach can be proposed in future research and can be applied to more general applications, \textit{e.g.}, the problems in swarm robotics.  





% if have a single appendix:
%\appendix[Proof of the Zonklar Equations]
% or
%\appendix  % for no appendix heading
% do not use \section anymore after \appendix, only \section*
% is possibly needed

% use appendices with more than one appendix
% then use \section to start each appendix
% you must declare a \section before using any
% \subsection or using \label (\appendices by itself
% starts a section numbered zero.)
%

%%%%%%%%%%%%%%%%%%%%%%%%%%%%%%%%% Delete for short %%%%%%%%%%%%%%%%%%%%%%%%%%%%%%%%%%
\appendices
\section{Lower bound of optimal control law}
\label{Appedix:lower_bound_of_optimal_control_law}
\begin{IEEEproof}
First, consider the case of $k^{\prime} = k-1$, $0 < k <T_f$ and $k \leq l < T_f$. From any robot PDF $\wp_l$,  
%and the optimal functional control law $\mathcal{C}_{k^{\prime}}^*(\cdot) = \mathcal{C}^*(\cdot,m_{k^{\prime}})$ defined in (\ref{eq:optimization_optimal_control}), 
by using the optimal functional control law obtained at the $k^{\prime}$th time step,  $\mathcal{C}_{k^{\prime}}^*$, recursively, a trajectory of robot PDFs $\{\wp_{\tau}\}_{\tau = l}^{T_f}$ are generated associated with the obstacle map function $m_k$.%, where $\wp_{\tau+1} = \mathcal{C}_{k^\prime}(\wp_{\tau},m_k)$. 

In addition, consider the optimal value functional $\mathcal{V}_k^*(\wp_l,m_k)$, $k \leq l < T_f$. According to (\ref{eq:VF_m_k}) and the Bellman equation,  $\mathcal{V}_k^*(\wp_l,m_k)$ can be expressed by
\begin{align}
    \mathcal{V}_k^*(\wp_l,m_k) &= \min_{\wp_{l+1}} \left[ \mathscr{L}(\wp_l,m_k,\wp_{l+1})+ \mathcal{V}_k^*(\wp_{l+1},m_k) \right] \nonumber\\
    &= \mathscr{L}(\wp_l,m_k,\wp_{l+1}^*) + \mathcal{V}_k^*(\wp_{l+1}^*,m_k) \nonumber\\
    &\leq \mathscr{L}(\wp_l,m_k,\wp_{l+1}) + \mathcal{V}_k^*(\wp_{l+1},m_k)
    \label{eq:Bellman_equation_mk}
\end{align}
where $\wp_{l+1} = \mathcal{C}^*_{k^\prime}(\wp_l,m_k)$ and $\wp^{*}_{l+1} = \mathcal{C}^*_k(\wp_l,m_k)$. %according to (\ref{eq:optimization_optimal_control}). 

By recursively utilizing  (\ref{eq:Bellman_equation_mk}), the following inequality is obtained
\begin{align}
    \mathcal{V}_k^*(\wp_l,m_k) &\leq \mathscr{L}(\wp_l,m_k,\wp_{l+1}) + \mathcal{V}_k^*(\wp_{l+1},m_k) \nonumber\\
    &\leq \mathscr{L}(\wp_l,m_k,\wp_{l+1})  \nonumber\\
    &+ \mathscr{L}(\wp_{l+1},m_k,\wp_{l+2}) +  \mathcal{V}_k^*(\wp_{l+2},m_k) \nonumber\\
    &\ldots \nonumber\\
    &\leq \sum_{\tau = l}^{T_f - 1} \mathscr{L}(\wp_{\tau},m_k,\wp_{\tau+1}) + \mathcal{V}_k^*(\wp_{T_f},m_k)\nonumber\\
%    &= \sum_{\tau = l}^{T_f - 1} \mathscr{L}(\wp_{\tau},m_k,\wp_{\tau+1}) + [d(\wp_{T_f},\wp_{targ})]^2 \nonumber\\
    &= \sum_{\tau = l}^{T_f - 1} \mathscr{L}(\wp_{\tau},m_k,\wp_{\tau+1}) + \mathcal{V}_{k^{\prime}}^*(\wp_{T_f},m_k)\nonumber\\
    &= \mathcal{V}_{k^{\prime}}^*(\wp_l,m_k), \; k \leq l < T_f
    \label{eq:lower_bound_1}
\end{align}
where $\mathcal{V}_k^*(\wp_{T_f},m_k) = \mathcal{V}_{k^{\prime}}^*(\wp_{T_f},m_k) = [d(\wp_{T_f},\wp_{targ})]^2$ according to (\ref{eq:VF_m_k}). Because this holds for $l = T_f$ as well,  according to (\ref{eq:VF_m_k}), (\ref{eq:lower_bound_1}) can be rewritten by
\begin{equation}
	\mathcal{V}_k^*(\wp_l,m_k) \leq \mathcal{V}_{k^{\prime}}(\wp_l,m_k), \; k \leq l \leq T_f
	\label{eq:lower_bound_2}
\end{equation}

Next, consider the case of $k^{\prime} = k - 1$ and $0 \leq l \leq k^{\prime}$. Assume that there exists a sequence of optimal functional control laws $\mathcal{C}_{\tau}^*$, $\tau = l,\ldots,k^{\prime}$. By using these optimal functional control laws sequentially, a trajectory of robot PDFs $\{\wp_\tau\}_{\tau = l}^{k^{\prime}}$ is generated associated with $\{m_{\tau}\}_{\tau=l}^{k^\prime}$. %$\mathbf{M}_k$. 
Moreover, similarly, by using the optimal functional control law $\mathcal{C}_{k^\prime}^*$ recursively, a trajectory of robot PDFs $\{\wp_{\tau}\}_{\tau = k^{\prime}}^{T_f}$ is generated from  $\wp_{k^{\prime}}$ to $\wp_{T_f}$ associate with $m_k$. Thus, a trajectory of robot PDFs, $\{\wp_{\tau}\}_{\tau = l}^{T_f}$ is generated from $\wp_l$ to $\wp_{T_f}$. 

Again, according to (\ref{eq:VF_m_k}) and the Bellman equation,  $\mathcal{V}_k^*(\wp_l,\mathbf{M}_k)$ can be expressed by  
\begin{align}
\mathcal{V}_k^*(\wp_l,\mathbf{M}_k) &=  \sum_{\tau=l}^{k^{\prime}} \mathscr{L}(\wp_{\tau},m_{\tau},\mathcal{C}_{\tau}^*) + \mathcal{V}_k^*(\wp_k,m_k) \nonumber\\
&\leq   \sum_{\tau=l}^{k^{\prime}} \mathscr{L}(\wp_{\tau},m_{\tau},\mathcal{C}_{\tau}^*) + \mathcal{V}_{k^{\prime}}^*(\wp_k,m_k) \nonumber\\
&= \mathcal{V}_{k^{\prime}}^*(\wp_l,\mathbf{M}_k), \; 0 \leq l \leq k^{\prime}
\label{eq:lower_bound_3}
\end{align}
where the inequality is obtained by applying (\ref{eq:lower_bound_2}).

Merging (\ref{eq:lower_bound_2}) and (\ref{eq:lower_bound_3}), it is shown that for $k^{\prime} = k-1$, 
\begin{equation}
\mathcal{V}_k^*(\wp_l,\mathbf{M}_k) \leq \mathcal{V}_{k^\prime}^*(\wp_l,\mathbf{M}_k), \;  0 \leq l \leq T_f
\label{eq:lower_bound_4}
\end{equation} 

Finally, by recursively applying (\ref{eq:lower_bound_4}), the theorem is proved, such that
\begin{equation}
\mathcal{V}_k^*(\wp_l,\mathbf{M}_k) \leq \mathcal{V}_{k^\prime}^*(\wp_l,\mathbf{M}_k), \;0 \leq k^{\prime} \leq k \text{ and } 0 \leq l \leq T_f
\end{equation}
\end{IEEEproof}


% you can choose not to have a title for an appendix
% if you want by leaving the argument blank
\section{Upper bound of optimal value functional}
\label{Appedix: Upper_bound_of_optimal_value_functional}
\begin{IEEEproof}
	First, consider the case of $k+1 < T_f$. According to (\ref{eq:VF_WG}), the value functional $\mathcal{V}_k(\wp_{k+1},m_k,\tilde{\mathcal{C}}_k)$ associated to %the special control law 
	$\tilde{\mathcal{C}}_k$, which is described in (\ref{eq:number_GC}) and (\ref{eq:weight_GC}), can be expressed by 
% 	\begin{align}
% 	\mathcal{V}_k(\wp_{k+1},m_k,\tilde{\mathcal{C}}) &= \phi(\wp_{T_f},\wp_{targ}) + \sum_{\tau = k+1}^{T_f-1}\mathscr{L}(\wp_{\tau},m_k,\tilde{\mathcal{C}}) \nonumber\\
% 	&= \left[ d(\wp_{T_f},\wp_{targ}) \right]^2 + \sum_{\tau =k+1}^{T_f -1} \left[\tilde{d} (\wp_{\tau},\mathcal{C}_k) \right]^2  \nonumber\\ 
% 	& + \sum_{\tau = k+1}^{T_f -1}   \langle   \wp_{\tau+1}, m_k \rangle_{\mathcal{W}} 	  
% 	\label{eq:VF_tilde_C}
% 	\end{align} 
\begin{align}
	\mathcal{V}_k(\wp_{k+1},m_k,\tilde{\mathcal{C}}_k) &= \phi(\wp_{T_f},\wp_{targ}) + \sum_{\tau = k+1}^{T_f-1}\mathscr{L}(\wp_{\tau},m_k,\tilde{\mathcal{C}}) \nonumber\\
	&= \left[ d(\wp_{T_f},\wp_{targ}) \right]^2 \nonumber\\ 
	&+ \sum_{\tau =k+1}^{T_f -1} \left[\tilde{d} (\wp_{\tau},m_k,\tilde{\mathcal{C}}_k) \right]^2  \nonumber\\ &
	 + \sum_{\tau = k+1}^{T_f -1}   \langle   \wp_{\tau+1}, m_k \rangle_{\mathcal{W}} 	  
	\label{eq:VF_tilde_C}
	\end{align} 
According to (\ref{eq:def_WG_Metric}), the following inequality is obtained,
\begin{equation}
\left[ d(\wp_{T_f},\wp_{targ}) \right]^2 \leq 
\sum_{\imath=1}^{N_{k+1}}\sum_{j=1}^{N_{targ}} \left[ W_2(g^{\imath}_{T_f},g^j_{targ}) \right]^2 \tilde{\pi}_k(\imath,j)
\label{eq:d_sq_ineq}
\end{equation} 
By recursively applying (\ref{eq:number_GC}) and (\ref{eq:weight_GC}), the term, $\left[\tilde{d} (\wp_{\tau},m_k,\tilde{\mathcal{C}}_k) \right]^2$, $k+1 \leq \tau \leq T_f - 1$,  can be expressed as
\begin{align}
\left[\tilde{d} (\wp_{\tau},m_k,\tilde{\mathcal{C}}_k) \right]^2 &= \sum_{\imath=1}^{N_{k+1}}\sum_{\imath^{\prime}=1}^{N_{k+1}} [W_2(g^{\imath}_{\tau},g^{\imath^{\prime}}_{\tau+1})]^2 \pi_k(\imath,\imath^{\prime}) \nonumber\\
&= \sum_{\imath=1}^{N_{k+1}} \left[ W_2(g^{\imath}_{\tau},g^{\imath}_{\tau+1}) \right]^2 \omega_{\imath}^{k+1} \nonumber\\
&= \sum_{\imath=1}^{N_{k+1}} \sum_{j=1}^{N_{targ}} \left[ W_2(g^{\imath}_{\tau},g^{\imath}_{\tau+1}) \right]^2 \tilde{\pi}_k(\imath,j)
\label{eq:tilde_d_sq}
\end{align}

Substituting (\ref{eq:d_sq_ineq}) and (\ref{eq:tilde_d_sq}) into (\ref{eq:VF_tilde_C}), one can have
\begin{align}
\mathcal{V}_k(\wp_{k+1},m_k,\tilde{\mathcal{C}}_k) &\leq \sum_{\imath=1}^{N_{k+1}}\sum_{j=1}^{N_{targ}} \bigg\{ \left[ W_2(g^{\imath}_{T_f},g^j_{targ}) \right]^2 \nonumber\\
&+ \sum_{\tau =k+1}^{T_f -1} \left[ W_2(g^{\imath}_{\tau},g^{\imath}_{\tau+1}) \right]^2  \nonumber\\&
+ \sum_{\tau=k+1}^{T_f -1}  \langle   g^{\imath}_{\tau}, m_k \rangle_{\mathcal{W}} \bigg \} \tilde{\pi}_k(\imath,j)
\label{eq:upper_bound_1}
\end{align} 
Because (\ref{eq:upper_bound_1}) holds for any trajectories of Gaussian components from $g^{\imath}_{k+1}$ to $g^{j}_{targ}$, $\imath = 1,\ldots,N_{k+1}$ and $j = 1,\ldots,N_{targ}$, 
the following inequality can be obtained,
\begin{align}
\mathcal{V}_k(\wp_{k+1},m_k,\tilde{\mathcal{C}}_k) &\leq \sum_{\imath=1}^{N_{k+1}}\sum_{j=1}^{N_{targ}} \underset{\mathscr{Tr}^{\imath,j}_k}{\min}\bigg\{ \left[ W_2(g^{\imath}_{T_f},g^j_{targ}) \right]^2 \nonumber\\
&+ \sum_{\tau =k+1}^{T_f -1} \left[ W_2(g^{\imath}_{\tau},g^{\imath}_{\tau+1}) \right]^2  \nonumber\\&
+ \sum_{\tau=k+1}^{T_f -1}  \langle   g^{\imath}_{\tau}, m_k \rangle_{\mathcal{W}} \bigg \} \tilde{\pi}_k(\imath,j)\nonumber\\
&= \sum_{\imath=1}^{N_{k+1}} \sum_{j=1}^{N_{targ}} \tilde{\mathcal{L}}^{\imath,j}_k \tilde{\pi}_k(\imath,j) 
\label{eq:upper_bound_2}
\end{align} 

Next, consider the case of $k+1 = T_f$. According to (\ref{eq:d_sq_ineq}), the upper bound of $\mathcal{V}_k(\wp_{T_f},m_k,\tilde{\mathcal{C}}_k)$ is expressed by, 
\begin{align}
    \mathcal{V}_k(\wp_{T_f},m_k,\tilde{\mathcal{C}}_k) 
%    &= \left[ d(\wp_{T_f},\wp_{targ}) \right]^2 \nonumber\\
    &\leq  \sum_{\imath=1}^{N_{k+1}}\sum_{j=1}^{N_{targ}} \left[ W_2(g^{\imath}_{T_f},g^j_{targ}) \right]^2 \tilde{\pi}_k(\imath,j) \nonumber\\
    &= \sum_{\imath=1}^{N_{k+1}}\sum_{j=1}^{N_{targ}} \tilde{\mathcal{L}}^{\imath,j}_k \tilde{\pi}_k(\imath,j)
\end{align}

Finally, considering the definition of the optimal value functional, the theorem is proved, such that
%%%%%%%%%%%%%%%%%%%%%%% Delete for short %%%%%%%%%%%%%%%%%%%%%%%%%%%%%%%%%%%%%%%%%%%%
\begin{align}
\mathcal{V}_k^*(\wp_{k+1},m_k) & \leq \mathcal{V}_k(\wp_{k+1},m_k,\tilde{\mathcal{C}}_k) \nonumber\\
& \leq  \sum_{\imath=1}^{N_{k+1}} \sum_{j=1}^{N_{targ}} \tilde{\mathcal{L}}^{\imath,j}_k  \tilde{\pi}_k(\imath,j) \nonumber\\
& = \tilde{\mathcal{V}}_k(\wp_{k+1},m_k)
\end{align}
%%%%%%%%%%%%%%%%%%%%%%% Delete for short %%%%%%%%%%%%%%%%%%%%%%%%%%%%%%%%%%%%%%%%%%%%
\end{IEEEproof}
%%%%%%%%%%%%%%%%%%%%%%%%%%%%%%%%% Delete for short %%%%%%%%%%%%%%%%%%%%%%%%%%%%%%%%%%

%%%%%%%%%%%%%%%%%%%%%%%%%%%%%%%%% Delete for short %%%%%%%%%%%%%%%%%%%%%%%%%%%%%%%%%%
% use section* for acknowledgment
\section*{Acknowledgment}
This research was partially funded by National Science Foundation grant ECCS-1556900 and the Office 550 of Naval Research, Code 321.
%%%%%%%%%%%%%%%%%%%%%%%%%%%%%%%%% Delete for short %%%%%%%%%%%%%%%%%%%%%%%%%%%%%%%%%%


% Can use something like this to put references on a page
% by themselves when using endfloat and the captionsoff option.
\ifCLASSOPTIONcaptionsoff
  \newpage
\fi



% trigger a \newpage just before the given reference
% number - used to balance the columns on the last page
% adjust value as needed - may need to be readjusted if
% the document is modified later
%\IEEEtriggeratref{8}
% The "triggered" command can be changed if desired:
%\IEEEtriggercmd{\enlargethispage{-5in}}

% references section

% can use a bibliography generated by BibTeX as a .bbl file
% BibTeX documentation can be easily obtained at:
% http://mirror.ctan.org/biblio/bibtex/contrib/doc/
% The IEEEtran BibTeX style support page is at:
% http://www.michaelshell.org/tex/ieeetran/bibtex/
\bibliographystyle{IEEEtran}
% argument is your BibTeX string definitions and bibliography database(s)
\bibliography{Ping_ADP_DOC_refs}
%
% <OR> manually copy in the resultant .bbl file
% set second argument of \begin to the number of references
% (used to reserve space for the reference number labels box)
%\begin{thebibliography}{1}
%
%\bibitem{IEEEhowto:kopka}
%H.~Kopka and P.~W. Daly, \emph{A Guide to \LaTeX}, 3rd~ed.\hskip 1em plus
%  0.5em minus 0.4em\relax Harlow, England: Addison-Wesley, 1999.
%
%\end{thebibliography}

%\bibliographystyle{IEEEtran}
%\bibliography{Ping_ADP_DOC_refs} 

% biography section
% 
% If you have an EPS/PDF photo (graphicx package needed) extra braces are
% needed around the contents of the optional argument to biography to prevent
% the LaTeX parser from getting confused when it sees the complicated
% \includegraphics command within an optional argument. (You could create
% your own custom macro containing the \includegraphics command to make things
% simpler here.)
%\begin{IEEEbiography}[{\includegraphics[width=1in,height=1.25in,clip,keepaspectratio]{mshell}}]{Michael Shell}
% or if you just want to reserve a space for a photo:
%\begin{IEEEbiography}{Pingping Zhu}



%\begin{IEEEbiography}[{\includegraphics[width=1in,height=1.25in,clip,keepaspectratio]{Bios/PingpingZhu}}]
%{Pingping Zhu}
%(S’10-M’13) is currently a Research Associate with the Department of Mechanical and Aerospace Engineering, Cornell University, Ithaca, NY, USA. 
%Prior to that, he was a Postdoctoral Associate with the Department of Mechanical Engineering and Material Science, Duke University. 
%His research interests include approximate dynamic programming, reinforcement learning, signal processing, information theoretical learning, machine learning, artificial intelligence, and neural networks. 
%He received the B.S. degree in electronics and information engineering and M.S. degree from the Institute for Pattern Recognition and Artificial Intelligence, Huazhong University of Science and Technology, Wuhan, China, in 2006 and 2008, respectively, and the M.S. and Ph.D. degrees in electrical and computer engineering from the University of Florida, Gainesville, FL, USA, in 2010 and 2013, respectively.
%\end{IEEEbiography}
%
%\begin{IEEEbiography}[{\includegraphics[width=1in,height=1.25in,clip,keepaspectratio]{Bios/ChangLiu}}]
%{Chang Liu}
%(S’15-M’17) is a Postdoctoral Associate in the Sibley School of Mechanical and Aerospace Engineering at Cornell University.
%%, where he works on the decentralized perception and planning of multi-agent systems. 
%His research interests include planning and decision-making of robots, multi-agent systems, state estimation and prediction, computer vision, and human-robot collaboration.
%He received the B.S. degrees in Electrical Engineering and in Applied Mathematics (double major) in 2011 from Peking University, China. He received the M.S. degrees in Mechanical Engineering and in Computer Science in 2014 and 2016 from the University of California, Berkeley. He received his Ph.D. in Mechanical Engineering from the University of California, Berkeley in 2017. 
%\end{IEEEbiography}
%
%\begin{IEEEbiography}[{\includegraphics[width=1in,height=1.25in,clip,keepaspectratio]{Bios/SilviaFerrari}}]
%{Silvia Ferrari}
%(S’01-M’02-SM’08) is John Brancaccio Professor of Mechanical and Aerospace Engineering at Cornell University.  
%Prior to that, she was Professor of Engineering and Computer Science at Duke University and Founder and Director of the NSF Integrative Graduate Education and Research Traineeship (IGERT) and Fellowship program on Wireless Intelligent Sensor Networks (WISeNet).  
%Currently, she is the Director of the Laboratory for Intelligent Systems and Controls (LISC) at Cornell University, and her principal research interests include robust adaptive control of aircraft, learning and approximate dynamic programming, and optimal control of mobile sensor networks. She received the B.S. degree from Embry–Riddle Aeronautical University and the M.A. and Ph.D. degrees from Princeton University. 
%She is a senior member of the IEEE, and a member of ASME, SPIE, and AIAA. She is the recipient of the ONR young investigator award (2004), the NSF CAREER award (2005), and the Presidential Early Career Award for Scientists and Engineers (PECASE) award (2006).
%\end{IEEEbiography}


%
%% insert where needed to balance the two columns on the last page with
%% biographies
%%\newpage
%


% You can push biographies down or up by placing
% a \vfill before or after them. The appropriate
% use of \vfill depends on what kind of text is
% on the last page and whether or not the columns
% are being equalized.

%\vfill

% Can be used to pull up biographies so that the bottom of the last one
% is flush with the other column.
%\enlargethispage{-5in}



% that's all folks
\end{document}


