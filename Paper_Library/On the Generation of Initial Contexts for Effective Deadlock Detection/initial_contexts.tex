% ;; -*- coding: iso-latin-1; TeX-PDF-mode: t; TeX-master: "main" -*-%

In our asynchronous programs, the most \emph{general} initial contexts
consist of sets of locations with \emph{free} variables in their fields, and
initial tasks in each location queue with \emph{free} variables as
parameters, i.e., neither the fields nor the parameters have concrete
values.  A first approach to systematically generate initial contexts
could consist in generating, on backtracking, all possible multisets
of initial tasks (method names), and for each one, generate all
aliasing combinations with the locations of the tasks belonging to the
same type of location. They are multisets because there can be
multiple occurrences of the same task.  To guarantee termination of
this process we need to impose some limit in the generation of the
multisets. For this, we could simply set a limit on the multiset
global size. However it would be more reasonable and useful to set a
limit on the maximum cardinality of each element in the multiset. To
allow further flexibility, let us also set a limit on the minimum
cardinality of each element. For instance, if we have a program with just one
location type $A$ with just one method $m$, and we set $1$ and $2$ as
the minimum and maximum cardinalities respectively, then there are two
possible multisets, namely, $\{m\}$ and $\{m,m\}$. The first one leads
to one initial context with one location of type $A$ with an instance
of task $m$ in its queue. The second one leads to two contexts, one
with one location of type $A$ with two instances of task $m$ in its
queue, and the other one with two different locations, each with an
instance of task $m$ in its queue.

On the other hand, it makes sense to allow specifying which tasks
should be considered as initial tasks and which should not. A typical
scenario is that the user knows which are the main tasks of the
application and does not want to consider auxiliary or internal tasks
as initial tasks. Another scenario is in the context of integration
testing, where the tester might want to try out together different
groups of tasks to observe how they interfere with each other. Also,
the use of  static analysis can help determine a subset of
tasks of interest to detect some specific property. This is the
case of our deadlock-guided approach of
Section~\ref{sec:initialtasks}.
%
With all this, the input to our automatic generation of initial
contexts is: (1) a set of \emph{abstract tasks} $\inittasks$ such that
each task is abstracted by the method name that is executing, (2) the
minimum and maximum cardinalities. Thus, an initial context is a set of tuples
(\lst{C.M},$\mincard,\maxcard$), where \lst{C} and \lst{M} are the
class and method name resp., and $\mincard$ resp. $\maxcard$ is the
associated minimum resp. maximum cardinality.
%
Note that this does not limit the approach in any way since one could
just include in $\inittasks$ all methods in the program and set
$\mincard = 0$ and a sufficiently large $\maxcard$. 

%The goal of
%starting from this input is hence usefulness and flexibility.

\begin{example}\label{ex:cmincmax}
  % Let us consider the set $\inittasks = \{$(\lst{DB.register},0,2)$\}$
  % from which we derive the multisets $\sf\{\{\emptyset\},\{register\},
  % \{register,register\}\}$. The empty multiset is useless since it
  % does not lead to any initial context of interest. From multiset $\sf
  % \{register\}$ we just have to consider an initial context containing
  % an instance of class \lst{DB} and a task \lst{register} in its
  % queue. The multiset $\sf \{register,register\}$ requires an initial
  % context containing exactly two instances of task \lst{register},
  % which leads to two different contexts: (1) a context composed by two
  % instances of class \lst{DB}, each of them with a task \lst{register}
  % in its queue, and, (2) a context composed by an instance of class
  % \lst{DB} with two tasks \lst{register} in its queue.
  Let us consider the set $\inittasks = \sf
  \{(DB.register,1,1),(DB.getData,0,1)\}$. The corresponding
  multisets are $\sf \{register\}$ and $\sf
  \{register,getData\}$. All contexts must contain exactly one
  instance of task \textsf{register} and at most one instance of task
  \textsf{getData}. This leads to three possible contexts: (1) a
  \textsf{DB} location instance with a task \textsf{register} in its queue,
  (2) a \textsf{DB} location instance with tasks \textsf{register} and
  \textsf{getData} in its queue, and, (3) two different \textsf{DB}
  location instances, one of them with an instance of task
  \textsf{register} and the other one with an instance of task
  \textsf{getData}. For instance, the state corresponding to the latter
  context would be: 

\vspace{-.25cm}
\[ \small\begin{array}{llll}
\tt
 S & =\! &\tt [loc(DB1,bot, &\tt [data \mapsto D1, clients
             \mapsto Cl1, checkOn \mapsto B1], \\
   &   &   &\tt [tsk(1,register,[this \mapsto r(DB1), m \mapsto
             W1],body(register))]) \\
   &   &\tt ~loc(DB2,bot, &\tt [data \mapsto D2, clients \mapsto
    Cl2, checkOn \mapsto B2], \\
   &   &   &\tt  [tsk(2,getData,[this
    \mapsto r(DB2), m \mapsto W2],body(getData))])],
\end{array} \]
%
\noindent where $\sf D1, Cl1,$ and $\sf B1$ (resp. $\sf D2, Cl2,$ and
$\sf B2$) are the fields \textsf{data, clients,} and \textsf{checkOn}
of location $\sf DB1$ (resp. $\sf DB2$), and \textsf{W1}
resp. \textsf{W2} the parameter of the task \textsf{register} resp.
\textsf{getData}, and \textsf{body(m)} is the sequence of instructions
in method \textsf{m}. Note that both fields and task parameters are
fresh variables so that the context is the most general possible.
Note that the first parameter of a task is always the location
\textsf{this} and it is therefore fixed. \hfill $\Box$
\end{example}

% \begin{example}\label{ex:system}
%   \fxnote{byMZ: Le anadiria al ejemplo al principio algunas otras
%     configuraciones de $\maxcard$ y $\mincard$ mostrando a que
%     multiconjuntos dar�an lugar para luego entrar en detalles con el
%     caso $\maxcard=1$ y $\mincard=1$ }
%Let us consider the set
%   $\{$\textsf{DB.register},\textsf{Worker.work}$\}$ and $\maxcard=1$ for the
%   system testing of our motivating example. From this input, we must
%   consider all possible contexts to be performed. This simple example
%   only needs to consider one initial context containing an instance of
%   class \textsf{Database} with a pending task \textsf{register} and an
%   instance of class \textsf{Worker} with a pending task \textsf{work} and
%   their respective future variables. That is:

% \texttt{\footnotesize \hspace{-0.35cm}Locs \!=\! [loc(1,\textit{bot},[field(data,D),field(clients,C),field(connected,B)], \\
% {\color{white}{aaaaaaaaaaaaaaaaaaaaaaaaa}}   [tsk(1,register,register([r(1),r(Worker)],Ret1),\ref{pp:reginit},call)])\!,\\
% {\color{white}{000000}}  loc(2,\textit{bot},[field(data,N)], \\
% {\color{white}{aaaaaaaaaaaaaaaaaaaaaaaaa}}
% [tsk(2,work,work([r(2),r(DB)],Ret2),\ref{pp:workinit},call)])\!,\\
% {\color{white}{0000000}})]}, and,\newline 
% \texttt{\footnotesize Futs
%   =
%   [fut(0,1,1,\ref{pp:reginit},\textit{bot}),fut(1,2,2,\ref{pp:workinit},\textit{bot})]},
% \fxnote{check futures}
% where \texttt{D, C, B} are the database's unknown fields, \texttt{N} is the
% worker's unknown field, \texttt{Worker}
% and \texttt{DB} are unknown references to any worker and database,
% respectively.
% We can find several situations depending on the value of \texttt{B} and the
% aliasing of the \texttt{Worker} and \texttt{DB} locations.
% \begin{itemize}
% \item Once task \textsf{register} is chosen to be executed, \texttt{B} is a variable, so
%   perfoming symbolic execution, it results in two different executions
%   with constraints: \texttt{\{B = true\}} and
%   \texttt{\{B = false\}}, respectively. In case of execution with path condition \texttt{\{B = false\}}, \textsf{register} finishes
%   without making any asynchronous call and, thus, after executing tasks
%   \textsf{work} and \textsf{getData}, we obtain a correct test case with no
%   bugs.
% %
% \item In the case of path condition \texttt{\{B  = true\}}, an asynchronous call is made
%   on the variable \textsf{Worker}. Symbolic execution needs to try all
%   possible aliasings. \textsf{Worker} is firstly instantiated to the initial
%   worker (\textsf{Worker = 2}) and then, database $1$ gets blocked.
%   Now, the initial worker can execute \textsf{work} or \textsf{ping}. 
%   The most interesting case is when \textsf{work} is executed (the other
%   one does not lead to deadlock). Here,
%   again an asynchronous call is made on the variable \texttt{DB}. If \texttt{DB} is
%   instantiated to the initial db (\texttt{DB = 1} is added to the
%   current condition path) then it also gets blocked and, thus, we will obtain several test
%   cases of deadlock. On the other hand, if \texttt{DB} is
%   bound to a new data base, then we will not find any deadlock
%   situation.
% %
% \item In a similar way if \texttt{Worker} is instantiated to a new worker
%   instead of the worker $2$, we cannot find any deadlock either and
%   the execution finishes correctly.
% %
% \item If \textsf{work} is executed before \textsf{register}, we obtain
%   analagous situations to the ones explained aforementioned.
% \end{itemize}
% \end{example}


In the following, we formally define the contexts that must be
produced from a set of abstract tasks $\inittasks$ with associated
cardinalities, and a procedure (as a Prolog rule) that generates these
contexts as partially instantiated states. 
%
We use the notation $\{[m_1,...,m_n]_{o_i}\}$ for an initial context
where there exists a location
$\buffer(o_i,\bot,\heap,\{\tkid(tk_1,m_1,\mapping_1,body(m_1))\}\cup...\cup
\{\tkid(tk_n,m_n,\mapping_n,body(m_n))\})$. Note that  we can have $m_i =
m_j$ with $i \neq j$. For instance, the three contexts in
Example~\ref{ex:cmincmax} are written as $\sf
\{[register]_{db_1}\},\{[register,getData]_{db_1}\}$ and $\sf \{[register]_{db_1},[getData]_{db_2}\}$,
respectively.
%
% Therefore, an initial state is a multiset of terms $o.m$, that is, a
% set of locations containing tasks inside their queues.  Example
% \ref{ex:system} is simple as $\mincard = \maxcard=1$ and tasks
% belong to different classes, so we only need to consider the initial
% context $\sf \{db.register,w.work\}$.  However, in those cases of
% $\maxcard > 1 $ or there are several abstract tasks belongs to the
% same class (Example \ref{ex:cmincmax}), we must consider all
% possible distributions among tasks and instances of the class
% containing such tasks, some of them can be repeated and, thus, this
% is the reason why initial contexts are multisets.
Let us first define the set of initial contexts from a given
$\inittasks$ when all tasks belong to the same class.
% Then, we define how to generate the set when abstract tasks belong
% to different classes.
\begin{definition}[Superset of initial contexts (same class $C_i$)]
  Let
  $\inittasks=\{(C_i.m_{1},\mincard_1,\maxcard_1),\ldots,(C_i.m_{n},\mincard_n,\maxcard_n)\}$
  be the set of abstract tasks with associated
  cardinalities.  
  % We will obtain all possible configurations related to class $C_i$.
% 
  Let us have $\sum\limits_{i=1}^n\maxcard_i$ different identifiers:
  $o_{1,1},\ldots,o_{1,\maxcard_1}, \\
  \ldots,o_{n,1},\ldots,o_{n,\maxcard_n}$.  We can find at most
  $\sum\limits_{i=1}^n\maxcard_i$ instances of class $C_i$, that is,
  each abstract task $m_{i}$ ($i \in [1,n]$) has at most $\maxcard_i$
  instances and each of them can be inside a different instance of
  class $C_i$.  Let $u^{m_k}_{i,j}$ be an integer variable that
  denotes the number of instances of task $m_k$ inside the location
  $o_{i,j}$ and let us consider the following integer system:
\begin{equation*}
\begin{cases}
\mincard_1 \leq
u^{m_1}_{1,1}+\ldots+u^{m_1}_{1,\maxcard_1}+\ldots+u^{m_1}_{n,1}+\ldots+u^{m_1}_{n,\maxcard_n}
\leq \maxcard_1 \\ 
 \ldots \\
\mincard_n \leq
u^{m_n}_{1,1}+\ldots+u^{m_n}_{1,\maxcard_1}+\ldots+u^{m_n}_{n,1}+\ldots+u^{m_n}_{n,\maxcard_n}
\leq \maxcard_n 
\end{cases} 
\end{equation*}
Each formula requires at least $\mincard_k$ and at most $\maxcard_k$
instances of task $m_k$. Each solution to this system corresponds to
an initial context.
%
\newline Let
$(d^{m_1}_{1,1},\ldots,d^{m_1}_{n,\maxcard_n},\ldots,d^{m_n}_{1,1},\ldots,d^{m_n}_{n,\maxcard_n})$
be a solution, then the corresponding initial context contains:
\begin{itemize}
\item $\buffer(o_{i,j},\bot,\heap,\queue)$, that is, a location
  $o_{i,j}$ whose lock is free, the fields in $\heap$ are mapped to
  fresh variables, and the queue $\queue$ contains:
  $d^{m_1}_{i,j}$ instances of abstract task $m_1$,\ldots, and
  $d^{m_n}_{i,j}$ instances of $m_n$, if $i \in [1,n]$, $j \in
  [1,\maxcard_i]$ and $\exists d^{m_k}_{i,j} > 0, k \in [1,n]$, where
  each instance of $m_i$ is $tsk(\tkid,m_i,\mapping,body(m_i))$ and
  every argument in $\mapping$ is mapped to a fresh variable.
\end{itemize} 
\end{definition}

\begin{example}\label{ex:superset}
 \noindent Let us consider the example $\inittasks {=} \sf
  \{(DB.register,0,1),(DB.getData,1,1)\}$. The identifiers are
  $o_{1,1}$ and $o_{2,1}$, and the variables of the system are $u^{reg}_{1,1}$, $u^{reg}_{2,1}$,
  $u^{get}_{1,1}$ and $u^{get}_{2,1}$. Finally, we obtain the next system:
\begin{equation*}
\begin{cases}
0 \leq u^{reg}_{1,1} + u^{reg}_{2,1} \leq 1 \\
1 \leq u^{get}_{1,1} + u^{get}_{2,1} \leq 1
\end{cases}
\end{equation*}

\noindent We obtain $6$ solutions: $(0,0,1,0), (0,0,0,1), (1,0,1,0),
(1,0,0,1), (0,1,1,0)$ and $(0,1,0,1)$. Then, the superset of initial
contexts is $$\sf \{\{[getData]_{o_{1,1}}\},\{[getData]_{o_{2,1}}\},
\{[register,getData]_{o_{1,1}}\},\{[register,getData]_{o_{2,1}}\},$$ $$\sf
\{[register]_{o_{2,1}},[getData]_{o_{1,1}}\},\{[register]_{o_{1,1}},[getData]_{o_{2,1}}\}\}  $$ 
 \hfill $\Box$
\end{example}
%
Let us observe that the two last contexts are equivalent since they
are both composed of two instances of \lst{DB} with tasks
\lst{register} and \lst{getData} respectively. Therefore, we only need
to consider one of these two contexts
% , namely the most \emph{general} one,\fxnote{MZ: Pero son igual de
% generales...}
for symbolic execution. Considering both would lead to
\emph{redundancy}. The notion of minimal set of initial contexts below
eliminates redundant contexts, hence avoiding useless executions.

\begin{definition}[Equivalence relation $\sim$]
  Two contexts $C_1$ and $C_2$ are equivalent, written $C_1 \sim C_2$,
  if $C_1 = C_2 = \emptyset$ or $C_1 =
  \{\buffer(\objid_1,\bot,\heap_1,\queue_1)\} \cup C_1'$, and
  $\exists~ \objid_2 \in C_2$ such that:
\begin{enumerate} 
\item $C_2 = \{\buffer(\objid_2,\bot,\heap_2,\queue_2)\} \cup C_2'$, 
\item $\queue_1$ and $\queue_2$ contain the same number of instances of
  each task, and
\item $C_1' \sim C_2'$ .
\end{enumerate}
\end{definition}

\begin{example}
  The superset in Example \ref{ex:superset} contains 3 equivalence
  classes induced by the relation $\sim$: (1) the class $\sf
  \{\!\{[getData]_{o_{1,1}}\},\{[getData]_{o_{2,1}}\}\!\},$ where both
  contexts are composed of a location with a task \lst{getData},
  (2) the class $\sf \{\!\{[register,getData]_{o_{1,1}}\}, \\
  \{[register,getData]_{o_{2,1}}\}\!\},$ whose locations have two
  tasks \lst{register} and \lst{getData}.  and, finally, (3) the class
  $\sf
  \{\!\{[register]_{o_{2,1}},[getData]_{o_{1,1}}\},\{[register]_{o_{1,1}},[getData]_{o_{2,1}}\}\!\}
  $, where both contexts have two locations with a task \lst{register}
  and a task \lst{getData}, respectively.  \hfill $\Box$
\end{example}

\begin{definition}[Minimal set of initial contexts
  $\initcontexts{C_i}$ (same class $Cl_i$)]
  Let $\inittasks$ be the set of abstract tasks, then the
  \emph{minimal set of initial contexts} $\initcontexts{Cl_i}$ is
  composed of a representative of each equivalence class induced by
  the relation $\sim$ over the superset of initial contexts for the
  input $\inittasks$.
\end{definition}

\begin{example} 
  As we have seen in the previous example, there are three different
  equivalence classes. So, the minimal set of initial contexts is
  composed of a representative of each class (we have renamed the
  identifiers for the sake of clarity): $$\initcontexts{DB} = \sf
  \{\{[getData]_{{db_1}}\}, \{[register,getData]_{{db_1}}\},
  \{[register]_{{db_1}},[getData]_{{db_2}}\}\} $$ \hfill $\Box$
\end{example}

Let us now define the set of initial contexts $\initcontexts{}$ when
the input set $\inittasks$ contains tasks of different types of
locations.

\begin{definition}[Minimal set of initial contexts  $\initcontexts{}$  (Different classes)]\label{def:different}
  \noindent \\ Let $\inittasks = 
  \{(C_1.m_1,\mincard_1,\maxcard_1),\ldots, (C_n.m_n,\mincard_n,\maxcard_n)\}$
  be the set of abstract tasks with associated cardinalities, and let
  us consider a partition of this set where every equivalence class is
  composed of abstract tasks of the same class. Hence, we have: 
  $\inittasks^{C_1}=\{C_1.m'_1,{\ldots},C_1.m'_{j_1}\},\ldots,\inittasks^{C_n}=\{C_n.m{''}_1,\ldots,C_n.m{''}_{j_n}\}$
  where $C_i \neq C_j, \forall i,j \in [1,n], i\neq j$. \\
Then, 
  let $\initcontexts{C_i}$ be the minimal set of initial contexts for the input
  $\inittasks^{C_i},~ i \in [1,n]$ and $U : \initcontexts{C_1} \times \ldots \times
  \initcontexts{C_n} \rightarrow \initcontexts{}$, defined by
  $U(s_1,\ldots,s_n) = s_1 \cup \ldots \cup s_n$. The set $\initcontexts{}$ is defined by the
  image set of application $U$.
\end{definition} 

\begin{example}\label{ex:initialcontexts}
  Let us consider the set 
  $\inittasks = \sf
  \{(DB.register,1,1),(DB.getData,1,1),\\(Worker.work,1,1)\}$
  from which we get the initial contexts $\initcontexts{Worker}= \sf
  \{\{[work]_{{w_1}}\}\}$ and $\initcontexts{DB} = \sf
\{\{[register,getData]_{{db,1}}\},
\{[register]_{{db_1}},[getData]_{{db_2}}\}\}$. Then, by Def.~\ref{def:different},  
  $$\initcontexts{} \!= \!\sf
  \{ \{[register,getData]_{{db_1}},[work]_{{w_1}}\},\{[register]_{{db_1}},
   [getData]_{{db_2}},[work]_{{w_1}}\}\!\}$$ \hfill $\Box$
% \begin{example}
%   Let us consider now the following simple example to illustrate this syntax and how
%   to generate the appropiated context.

% \begin{minipage}{1\textwidth}
% \begin{center}
% \begin{tabular}{c}
% \begin{lstlisting}[name=smallexamples]
% class O{
%   ${\text{\bf Int}}$ M(){
%     return 0;
%   }
% }
% \end{lstlisting}
% \end{tabular}
% \end{center}
% \end{minipage}
%   The previous code defines a class \lst{O} with a method \lst{M} which returns the
%   value $0$. If the system testing receives as
%   input the set $\{$\lst{O.M}$\}$ and $C^{max}=1$ we only need to consider the
%   context $\{o.m\}$, which means: the initial context requires an
%   initial object $o$ which is an instance of the class \lst{O} whose
%   pending tasks queue contains an instance $m$ of the abstract task
%   \lst{M}. The fields of the object $o$ and the paramaters that
%   task $m$ requires (in case it does) are unknown variables as usual in the
%   context of symbolic execution.
\end{example}
% However, If the maximum cardinality is increased, the number of initial
% contexts also does.

% \begin{example}
%   Let us consider now the input $\mathfrak{S} = \sf \{$Worker.ping$\}$ and $C^{max}=3$, then
%   $\mathfrak{I^S} =$ $\{o_1.work\}$ ,$\{o_1.work.,o_1.work\}$
%   ,$\{o_1.work,o_2.work\}$, 
% $\{o_1.work,o_1.work,o_1.work\},$ 

% $\{o_1.work,o_1.work,o_2.work\},$ $ \{o_1.work,o_2.work,o_3.work\}$, being
%   $o_1,o_2,o_3$ instances of class \textsf{Worker} and m a instance
%   of abstract task \textsf{Worker}.
% \end{example}
% %

% For testing, it is not needed though to take into account all these initial
% contexts but only the ones where the maximum cardinality is reached,
% as test cases of the discarded contexts will be obtained during the
% execution of the bigger ones. We discard those initial contexts that
% are prefixes of others. E.g. $\{o_1.work\}$
% and $ \{o_1.work,o_1.work\} $ are discarded by $ \{o_1.work,o_1.work,o_1.work\}$ and
% $\{o_1.work,o_2.work\} $ by $\{o_1.work,o_2.work,o_3.work\}$.
% Then, during the execution of system testing of the bigger context,
% we will obtain the desired test cases for the discarded contexts 
% where all the tasks present in the
% discards are executed but none of the tasks which are not.
% %
% However, the bigger the input set is, the harder the combinatorial explosion
% becomes, as we can see in the following example.
% %


\begin{figure}[t]
\begin{center}
\begin{tabular}{ll}
\begin{lstlisting}[style=PrologStyle]
generate_contexts([(M,MinC,MaxC)|Methods],SOut) :-
   add_calls([],[(M,0,MinC,MaxC)|Methods],SOut),
   normal_form(SOut,N),
   (prev_generated(N) -> fail ; assertz(prev_generated(N))).

add_calls(SIn,[(M,Instances,MinC,MaxC)|Ms],SOut) :-
   Instances < MaxC, 
   add_task(SIn,M,SAux),$\label{pp:add-calls12}$
   I2 is Instances + 1, 
   add_calls(SAux,[(M,I2,MinC,MaxC)|Methods],SOut). $\label{pp:add-calls15}$
add_calls(SIn,[(_,I,Min,_),(M,MinC,MaxC)|Methods],SOut) :-
   Min <= I,
   add_calls(SIn,[(M,0,MinC,MaxC)|Methods],SOut). $\label{pp:add-calls22}$
add_calls(SIn,[(_,I,Min,_)],SIn) :-
   Min <= I. $\label{pp:add-calls31}$

add_task([],M,SIn) :-
   fresh_location(LocId),fresh_task(TkId),
   initialize_fields(M,Fields),initialize_mapping(M,L),
   SIn = [loc(LocId,$\bot$,Fields,[tsk(TkId,M,L,body(M))])]. $\label{pp:add-task14}$
add_task([Loc|SIn],M,[Loc2|SIn]) :-
   Loc = loc(Id,Lock,Fields,Q),
   class(Id,Class), class(M,Class), $\label{pp:add-task22}$
   fresh_task(TkId), initialize_mapping(M,L),$\label{pp:add-task23}$
   Loc2 = loc(Id,Lock,Fields,[tsk(TkId,M,L,body(M))|Q]). $\label{pp:add-task24}$
add_task([Loc|SIn],M,[Loc|SOut]) :-
   add_task(SIn,M,SOut). $\label{pp:add-task31}$
\end{lstlisting}
\end{tabular}
\end{center}
\caption{Prolog predicate to generate minimal set of initial contexts}
\label{fig:clp-builtins}
\end{figure}

% \begin{figure}[t]
% \begin{center}
% \begin{tabular}{ll}
% \begin{lstlisting}[style=PrologStyle]
% add_task(SIn,M,SIn):-
%   var(SIn),!,
%   fresh_location(LocId),fresh_task(TkId),
%   initialize_fields(M,Fields),initialize_mapping(M,L),
%   SIn = [loc(LocId,$\bot$,Fields,[tsk(TkId,M,L,body(M))])|SVar]. $\label{pp:add-task14}$

% add_task([Loc|SIn],M,[Loc2|SIn]):-
%   Loc = loc(Id,Lock,Fields,Q),
%   class(Id,Class),class(M,Class), $\label{pp:add-task22}$
%   fresh_task(TkId),initialize_mapping(M,L),$\label{pp:add-task23}$
%   Loc2 = loc(Id,Lock,Fields,[tsk(TkId,M,L,body(M))|Q]). $\label{pp:add-task24}$

% add_task([Loc|SIn],M,[Loc|SOut]):-
%   add_task(SIn,M,SOut). $\label{pp:add-task31}$
% \end{lstlisting}
% \end{tabular}
% \end{center}
% \caption{CLP operations for handling reference aliasing.}
% \label{fig:clp-builtins4}
% \end{figure}

We now define a Prolog predicate that generates the minimal set of
initial contexts as partially instantiated states. Predicate
\texttt{generate\_contexts/2} in Figure~\ref{fig:clp-builtins}
receives a set of abstract tasks with their associated maximum and
minimum cardinalities, and generates on backtracking all generated
initial contexts by means of \texttt{add\_calls/3}. Predicate
\texttt{normal\_form/2} produces a normal form for the new context
which is the same for all initial contexts in the same equivalence
class. The new context is therefore only generated if it has not been
previously generated (i.e., if the call \texttt{prev\_generated/1}
fails).  The first rule of \texttt{add\_calls/3} checks if the number
of instances \texttt{Instances} of task \texttt{M} is smaller than the
maximum cardinality \texttt{MaxC}, in which case we add a new instance
of \texttt{M}, \texttt{Instances} is incremented, and,
\texttt{add\_calls/3} is recursively invoked. The second rule checks
if the number of instances is greater than or equal to \texttt{Min},
it initializes the number of instances for the next method
(\texttt{M}) and makes the recursive call to
\texttt{add\_calls/3}. Finally, the third rule corresponds to the base
case when we are processing the last method of the list and the number
of instances if greater than or equal to \texttt{Min}.
%
Predicate \texttt{add\_task/3} adds a new instance of method
\texttt{M} to the current location.  Note here that it can add the new
task to one of the existing locations in \texttt{Locs} or create a new
one to add it.  The first rule checks if \texttt{SIn} is a variable
(the end of the locations list) and then, it creates a new location to
add the task \texttt{M}. To do so, we initialize the location fields,
the method arguments and its tasks queue with a new task with fresh
identifier \texttt{TkId}, and the instructions of method \texttt{M}
(\texttt{body(M)}). The second rule checks if method \texttt{M} can be
added to the first location by checking if the class of location
\texttt{Id} matches with the class of \texttt{M}. If it does, then we
add a new task to its tasks queue \texttt{Q}. The third rule ignores
the first location and tries to add \texttt{M} to \texttt{SIn}.

 
\begin{example}\label{ex:gen-contexts}
  Let us show predicate \texttt{generate\_contexts/2} in action for
  the set $\inittasks = \sf \{(DB.reg,1,1),(DB.make,1,1), 
  Worker.work,1,1)\}$.  The first rule of \\\texttt{add\_calls/3} is
  applied, as \texttt{0 = Instances < MaxC = 1}. Then,
  \texttt{add\_task/3} is called with variable \texttt{Locs} and
  \texttt{M = DB.register}, at line~\ref{pp:add-calls12}. As
  \texttt{Locs} is a variable, only the first rule of
  \texttt{add\_task/3} can be applied and then, a new location is
  created (line~\ref{pp:add-task14}). Once this predicate has
  finished, \texttt{Instances} is incremented and \texttt{add\_calls}
  is recursively called   (line~\ref{pp:add-calls15}). Now, the second
  rule is applied, as \texttt{0 = Min < Instances = 1},
  and \texttt{add\_calls} is called with \texttt{M = DB.makesConnection}
  whose number of instances is initialized to 0 (line
  \ref{pp:add-calls22}).
%
  Again, at line\ref{pp:add-calls12}, \texttt{add\_call/3} is
  called with \texttt{M = DB.makesConnection} and \texttt{Locs}
  containing an instance of \lst{DB}. Here we get to a branching point
  which gives rise to the two different initial contexts in
  Example~\ref{ex:initialcontexts}. In the first branch, \texttt{SIn}
  contains a location whose class is equal to that of the method
  \texttt{makesConnection}, so \texttt{LocVar} is the existing
  location and a new instance is added to its queue (lines
  \ref{pp:add-task23} and \ref{pp:add-task24}).
%
  Finally, \texttt{add\_calls/3} is called with \texttt{M =
    Worker.work} (line \ref{pp:add-calls22}), it creates a new
  instance of class \texttt{Worker} with a task \texttt{work} (line
  \ref{pp:add-task14}), it finishes correctly at line
  \ref{pp:add-calls31}, and returns an initial context containing an
  instance of \lst{DB} with tasks \lst{register} and
  \lst{makesConnection}, and an instance of \lst{Worker} with task
  \lst{work}.
%
  Now, it fails and the backtracking goes back to the branching
  point. Here, the third rule is applied and then, the first location
  is ignored and task \texttt{makesConnection} is added to a new
  location at line~\ref{pp:add-task14}. It finishes in a similar
  way. In this case, the initial context returned contains two
  instances of \lst{DB} containing a task \lst{register} and
  \lst{makesConnection}, respectively, and an instance of \lst{Worker}
  with task \lst{work}. \hfill $\Box$
\end{example}
