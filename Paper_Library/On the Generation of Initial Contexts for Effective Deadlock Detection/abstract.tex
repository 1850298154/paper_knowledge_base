\begin{abstract}
  It has been recently proposed that testing based on symbolic
  execution can be used in conjunction with static deadlock analysis
  to define a deadlock detection framework that: (i) can show deadlock
  presence, in that case a concrete test-case and trace are obtained,
  and (ii) can also prove deadlock freedom.
%
 Such symbolic execution starts from an 
  \emph{initial distributed context}, i.e., a set of locations and their
  initial tasks.
% , which was provided manually by the user.
%, hence limiting the approach potential.
%
  Considering all possibilities results in a combinatorial explosion
  on the different distributed contexts that must be considered. This
  paper proposes a technique to effectively generate initial contexts
  that can lead to deadlock, using the possible conflicting task
  interactions identified by static analysis, discarding other
  distributed contexts that cannot lead to deadlock.
%
  The proposed technique has been integrated in the above-mentioned
  deadlock detection framework hence enabling it  to analyze systems
  without the need of any user supplied initial context.
%
  % In the context of unit level testing, the method under test is
  % generally analyzed from a completely unknown context, without
  % assuming any knowledge on the input. However, when lifting up to the
  % integration or the system level, an initial (possibly partial)
  % context should be ideally provided. This is specially relevant in
  % the case of distributed programs, where the initial context includes
  % a set of locations, their connections and initial tasks.
  % Considering all possibilities results in a combinatorial explosion
  % on the different distributed contexts that must be considered.
  % Therefore, it is crucial to provide support to eliminate those
  % contexts that cannot expose errors.  This paper proposes the
  % combined use of static analysis and symbolic execution for the
  % generation of the initial contexts which might lead to deadlock
  % errors in asynchronous programs. When the static analysis detects a
  % potential deadlock, we generate initial contexts based on the
  % possible conflicting task interactions detected by the deadlock
  % analyzer, discarding other distributed contexts that cannot lead to
  % deadlock.
\end{abstract}%
