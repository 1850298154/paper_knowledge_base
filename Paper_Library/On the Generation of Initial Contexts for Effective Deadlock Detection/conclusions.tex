% ;; -*- coding: iso-latin-1; TeX-PDF-mode: t; TeX-master: "main" -*-%

We have proposed a framework for the automatic generation of initial
contexts for deadlock-guided symbolic execution. Such initial contexts
are composed of the interfering tasks which, according to a static
deadlock analyzer, might lead to deadlock. Given the initial contexts,
we can drive symbolic execution towards paths that are more likely to
manifest a deadlock, discarding safe contexts.
%
There is a large body of work on deadlock detection including both
dynamic and static approaches.  Much of the existing work, both for
asynchronous programs \cite{FloresAG13-short,GGLLW13} and
thread-based programs
\cite{DBLP:conf/pdd/MasticolaR91,DBLP:journals/tocs/SavageBNSA97}, is
based on static analysis techniques. Although we have used the static
analysis of \cite{FloresAG13-short}, the information provided by other
deadlock analyzers could be used in an analogous way.
Deadlock detection has been also studied in the context of dynamic
testing and model checking
\cite{DBLP:conf/icst/ChristakisGS13,DBLP:conf/pldi/JoshiPSN09,lockout},
where sometimes has been combined with static information
\cite{DBLP:conf/hvc/AgarwalWS05,DBLP:conf/sigsoft/JoshiNSG10}. 
The initial contexts generated by our framework are of interest also
in these approaches. Deadlock detection is even more challenging in
the context of thread-based concurrency model. As future work, we plan to
investigate how our framework could be adapted to this model.

% Static analysis can ensure the
% absence of errors, however it works on approximations (especially for
% pointer aliasing) which might lead to a ``don't
% know'' answer. Our work complements static analysis techniques and can
% be used to look for deadlock paths when static analysis is not able to
% prove deadlock freedom. Using our method, we try to find a deadlock by
% generating initial contexts which include the conflicting tasks given
% by our deadlock detection algorithm that relies on the static
% information.




 %  The
%  approach in \cite{lockout} consists in generating binaries that are
%  more likely to manifest a deadlock such that by relying on standard
%  testing it is easier to capture deadlock derivations. 
% As regards combined approaches, the approach in
% \cite{DBLP:conf/sigsoft/JoshiNSG10} first performs a transformation of
% the program into a trace program that only keeps the instructions that
% are relevant for deadlock and then dynamic testing is performed on
% such program.


% The approach is fundamentally different from ours: in
% their case, since model checking is performed on the trace program
% (that over-approximates the deadlock behaviour), the method can detect
% deadlocks that do not exist in the program, while in our case this is
% not possible since the testing is performed on the original program
% and the analysis information is only used to drive the execution. 


