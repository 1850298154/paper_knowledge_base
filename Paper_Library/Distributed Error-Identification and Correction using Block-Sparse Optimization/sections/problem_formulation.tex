\subsection{Sensing and Communication Topology}
\label{subsec:graphs}
An (undirected) \textit{hypergraph} refers to a pair $(\mathcal V, \mathcal E)$ constituting a set of vertices $\mathcal V$ and a set of hyperedges $\mathcal E$, where each element of $\mathcal E$ (called a hyperedge) is a subset of $\mathcal V$ \cite[Sec. 1.10]{diestel2017}. Consider a multi-agent system represented as a hypergraph, $\mathcal G=(\mathcal V, \mathcal E)$, such that the vertices correspond to the agents (of which there are $|\mathcal V|$ in total), and each hyperedge represents the availability of a measurement that depends on the states of the corresponding agents. As hypergraphs generalize graphs by allowing more than two (or even zero or one) vertices to be connected, they enable us to represent measurements that are functions of three or more agents' states, such as subtended angle and time-difference-of-arrival (TDoA) measurements \cite{weak_rigidity,tdoa_2017}. Let $\mathcal V$ and $\mathcal E$ be endowed with arbitrary orderings, so that we may write $\mathcal V = \lbrace 1, 2, \dots, |\mathcal V|\rbrace$ and $\mathcal E= \lbrace \mathcal E^{(1)}, \mathcal E^{(2)}, \dots, \mathcal E^{(|\mathcal E|} \rbrace$.
Agents $i$ and $j$ are said to be \textit{neighbors} of each other if, for some $l\in\lbrace 1, \dots, |\mathcal E|\rbrace$, both $i$ and $j$ are in $\mathcal E^{(l)}$. The set of neighbors of agent $i$ is denoted by $\mathcal N_i \subseteq \mathcal V$.
% ; we also say that agent $i$ is able to \textit{sense} agent $j$, and vice versa. 
Each agent in $\mathcal G$ is assumed to be a Cyber-Physical System (CPS) that has a physical state, an embedded computer, and the capabilities to communicate and (potentially) obtain measurements through a variety of sensors. We make the following assumption about the communication capabilities of the agents.

\vspace{2pt}
\begin{assumption}[Communication Topology]
Each agent is able to communicate with its neighbors, and the agents are able to synchronize\footnote{The meaning of `synchronize' as it is used in Assumption \ref{ass:1} can be understood unambiguously by studying the algorithm which we develop in Section \ref{sec:distributed}.} their communications.
\label{ass:1}
\end{assumption}
\vspace{2pt}

% \begin{remark} Furthermore, the inter-agent communications can be synchronized using the distributed algorithm presented in \cite{sync2010}.\end{remark}\vspace{2pt}

Thus, we have implicitly made the assumption that if agent $i$ is able to sense agent $j$, then agents $i$ and $j$ are able to establish a bidirectional communication channel between them as well. 
% We can represent the resulting communication topology using the \textit{adjacency graph} of $\mathcal G$, which is defined as the graph $\mathcal G_{Ad}=(\mathcal V, \mathcal E_{Ad})$ where $\mathcal E_{Ad}\subseteq \mathcal V \times \mathcal V$, such that $(i, j)$ and $(j,i)$ are in $\mathcal E_{Ad}$ if and only if agents $i$ and $j$ are neighbors in $\mathcal G$.
% While the hypergraph $\mathcal G$ simplifies the forthcoming notation, the graph $\mathcal G_{Ad}$ is easier to visualize.

\subsection{Agent States and Estimates}
The collective state of the multi-agent system is represented by a block vector $\mathbf p$, which has the form 
\begin{equation}
\mathbf p = \begin{bmatrix} \mathbf p[1]^\top & \mathbf p [2]^\top & \dots & \mathbf p [|\mathcal V|]^\top \end{bmatrix}^\top \in \mathbb R^n
\end{equation}
where $\mathbf p[i] \in \mathbb R^{n_i}$ is the $i^{th}$ agent's state, $n_1, n_2, \dots, n_{|\mathcal V|}$ are positive integers representing the dimensions of each of the agents' states, and $n \coloneqq \sum_{i\in \mathcal V} n_i$. We call the pair $(\mathcal G, \mathbf p)$ (or equivalently,  the vector $\mathbf p$) a \textit{configuration}, and $\mathbb R^n$ is called the configuration space.
% Similar to \cite{whiteley1985generating} and \cite{anderson2010formal}, we say that a configuration is \textit{generic} if its state vector $\mathbf p$ does not lie in a measure-zero subset of $\mathbb R^n$ where a certain property fails to hold. The corresponding property is said to be a generic property of the configuration space. An example of a nongeneric configuration is one where the states of three of the agents are precisely collinear; such cases are excluded when we study generic configurations and their properties.


Each agent uses a suite of onboard sensors to estimate its own state and self-reports the estimated state to its neighbors. The reported state of agent $i$ is denoted by $\hat {\mathbf p}[i]$, which may or may not coincide with $\mathbf p[i]$. The discrepancy between $\mathbf p$ and $\hat {\mathbf p}$ is encapsulated in the error vector, defined as $\mathbf x \coloneqq \mathbf p - \hat {\mathbf p}$.  We say that there is a \textit{fault} or an \textit{error} at agent $i$ to mean that $\mathbf p[i] \neq \hat{\mathbf p}[i]$, or equivalently, $\mathbf x[i] \neq \mathbf 0$. This can occur if agent $i$ has wrongly estimated its state due to sensor bias, miscalibration, modeling errors, or adversarial sensor spoofing attacks.
% \begin{remark}
% Our paper does not consider the problem of categorizing the source of the error into either of the preceding cases; rather, we develop an algorithm for reconstructing $\mathbf x$ in a distributed manner, so that the error at agent $i$, $\mathbf x[i]$, can be determined irrespective of its source.
% \end{remark}
Let $\mathcal D$ be the set of agents that have errors. We have,
\begin{equation}
|\mathcal D| = \sum_{i\in \mathcal V} \mathbb I\left(\mathbf p[i] \neq \hat{\mathbf p}[i]\right) = \|\mathbf x\|_{2,0},
\end{equation}
which is in turn
equal to the number of non-zero blocks of $\mathbf x$, referred to as its \textit{block-sparsity}.
We say that the errors are sparse, and that $\mathbf x$ is block-sparse, if 
$\|\mathbf x\|_{2,0} \ll|\mathcal V|$.

\vspace{2pt}
\begin{remark}
In the literature on rigidity theory, it is assumed that the identities of the agents in $\mathcal D^\complement$, which are called the \textit{anchors}, are known to all the agents in the network \cite{zhao2019bearing}.
However, we have dropped the assumption that any of the elements of $\mathcal D$ or $\mathcal D^\complement$ are known \textit{a priori}, allowing us to diagnose the entire network for errors, rather than ruling out a subset of the agents (i.e., the anchors) from having errors. 
\label{rem:anchors}
\end{remark}
\vspace{2pt}

In the single-agent fault detection, identification, and reconstruction (FDIR) problem, the objective is to identify the subset of sensors that are faulty, as well as recover the nominal state estimation performance. We can extend this idea to the multi-agent setting by treating each agent of the network as a \textit{sensor} that is observing the collective state vector, $\mathbf p$. With this interpretation, the onboard state estimates of the agents can also be thought of as measurements of $\mathbf p$:
\begin{align}
    \hat {\mathbf p}[i] = \bigl[\ 
        \mathbf 0 \ \ \mathbf 0 \ \ &\dots \ \ 
        \underset{
            \substack{\vspace{1pt}\\
                        \textstyle \uparrow \vspace{1pt}\\
                        \scalebox{0.8}{$i^{th}\ \textrm{block}$}
                    }
                }{\mathbf I} 
        \ \ \dots \ \ \mathbf 0\ \bigr]\ \mathbf p + \mathbf x[i], 
    \label{eq:onboard}
\end{align}
where $i\in \mathcal V$. Similarly, $\mathbf x[i]$ represents an unknown additive signal whose presence or absence must be determined at each sensor, as part of the multi-agent FDIR problem. However, the measurements in (\ref{eq:onboard}) are decoupled, as only agent $i$ is able to measure $\mathbf p[i]$ and $\mathbf x[i]$. This makes it impossible for the multi-agent network to collaboratively identify faults unless an additional set of measurements is available.


\subsection{Inter-Agent Measurements}

A distinctive feature of many real-world multi-agent systems is the availability of relative measurements between them, which are often nonlinear functions of the agents' states. We allow each hyperedge in $\mathcal E$ to correspond to a different type of nonlinear measurement.
Let the inter-agent measurement model corresponding to $\mathcal E^{(l)}$, which is the $l^{th}$ edge in $\mathcal E$, be denoted by $
\mathbf \Phi^{(l)}: U^{(l)} \rightarrow \mathbb R^{m_l}$, where $m_l$ is the dimension of the measurement and $U^{(l)}$ is an open subset of $\mathbb R^n$. Some examples of inter-agent measurement models that arise in practical applications are given in Table \ref{tab:iamms}, in each of which the state vector $\mathbf p[i]$ represents the position of agent $i$ in $2$ or $3$-dimensional space.

\begin{table}[h!]
\centering
\caption{Examples of inter-agent measurement models, where $\lbrace\mathbf p[i]\rbrace_{i\in \mathcal V}$ represent positions of the agents in $\mathbb R^2$ or $\mathbb R^3$}
\bgroup
\def\arraystretch{2.0}
\setlength{\tabcolsep}{1.0em}
% \vspace{10pt}
{\rowcolors{2}
{gray!3}{gray!12}
% {green!80!yellow!50}{green!70!yellow!40}
\begin{tabular}{|p{2.5cm}|p{5.0cm}|}
\hline
Measurement Type & Expression for $\mathbf \Phi^{(l)}$ \\
\hline
Displacement \cite{oh2015survey} & $\ \mathbf p[i] - \mathbf p[j]$ \vspace{4pt}  \\
Distance \cite{oh2015survey,topology_const2015observability} & $\  \big\|\mathbf p[i] - \mathbf p[j]\big\|$ \vspace{4pt}  \\
Bearing \cite{zhao2019bearing} & $(\mathbf p[i] - \mathbf p[j])\Big/\big\|\mathbf p[i] - \mathbf p[j]\big\|$ \vspace{4pt}\\
Time-Difference-of-Arrival (TDoA) \cite{tdoa_2017} & $\big\|\mathbf p[i] - \mathbf p[j]\big\| 
- \big\|\mathbf p[i] - \mathbf p[k]\big\|$ \\
Subtended Angle \cite{weak_rigidity} &  $\bigstrut[t]
\arccos
\biggl(
\cfrac{\mathbf p[i] - \mathbf p[j]}{\|\mathbf p[i] - \mathbf p[j]\|}\overset{^\top}{^{\ }}
\cfrac{\mathbf p[i] - \mathbf p[k]}{\|\mathbf p[i] - \mathbf p[k]\|}
\biggr)$ \vspace{3pt} \\
\hline
\end{tabular}}
\egroup
\label{tab:iamms}
\end{table}

\begin{remark}
Other inter-agent measurement models, such as Frequency-Difference-of-Arrival (FDoA) \cite{tdoa_2017}, may involve the relative velocities between agents, in which case $\mathbf p[i]$ may be a block vector constituting the $i^{th}$ agent's position and velocity vectors. Similarly, the agents' orientations can be appended to their state vectors in order to model relative pose measurements (which can be obtained using cameras) \cite{aragues2011relative_pose}.
\end{remark}
\vspace{2pt}

For the measurement models given in Table \ref{tab:iamms}, the domain $U^{(l)}$ of the $l^{th}$ measurement model $\mathbf \Phi^{(l)}$ can be chosen to exclude the set of points in $\mathbb R^n$ where two agents' positions are coincident, as $\mathbf \Phi ^{(l)}$ and/or its derivatives may not be well-defined at these points. With the appropriate choices of the domains, each of the measurement models discussed thus far satisfy the following assumption.

\vspace{2pt}
\begin{assumption}[Smoothness]
For all $l=1, 2, \dots, |\mathcal E|$, $\mathbf \Phi^{(l)}$ is a smooth (i.e., infinitely differentiable) map.
\end{assumption}
\vspace{2pt}

The inter-agent measurements are collectively represented by the block vector-valued function,
\begin{align}
    \mathbf \Phi:\ &U \rightarrow \ \mathbb R^m\\
    & \mathbf p \mapsto \begin{bmatrix}
        \mathbf \Phi^{(1)}(\mathbf p)\\
        \mathbf \Phi^{(2)}(\mathbf p)\\
        \vdots\\
        \mathbf \Phi^{(|\mathcal E|)}(\mathbf p)
    \end{bmatrix}
\end{align}
where $U=U^{(1)}\times \dots \times U^{(|\mathcal E|)} \subseteq \mathbb R^n$ is the domain of $\mathbf \Phi$. Let $\mathbf y = \mathbf \Phi(\mathbf p)$ be the block vector of inter-agent measurements, which is partitioned such that $\mathbf y[l]= \mathbf \Phi^{(l)}(\mathbf p)$  $\forall l=1, 2, \dots, |\mathcal E|$. We assume that agent $i$ has access to the set of inter-agent measurements obtained in its neighborhood in $\mathcal G$, which is the set $\lbrace \mathbf y[l] \hsp \vert \hsp i \in \mathcal E^{(l)}\rbrace$. Letting, $\mathbf J_{\mathbf \Phi}(\mathbf p)$ denote the Jacobian of $\mathbf \Phi$ evaluated at the point $\mathbf p$, which is partitioned such that $\mathbf J_{\mathbf \Phi}(\mathbf p)[l,i]\in \mathbb R^{m_l \times n_i}$, the following assumption formalizes the functional dependence between the measurement models and the agents.

\vspace{2pt}
\begin{assumption}[Inter-Agent Sensing Topology]
We have, $\mathbf J_{\mathbf \Phi}(\mathbf p)[l, i] \neq \mathbf 0\ \Leftrightarrow\ i \in \mathcal E^{(l)}$. This means that the $l^{th}$ block in $\mathbf \Phi(\mathbf p)$ only depends on the states of the agents in $\mathcal E^{(l)}$.
\label{ass:jacobian}
\end{assumption}
\vspace{2pt}

With the above definitions in place, the objective of multi-agent FDIR can be stated as follows: the agents must collectively reconstruct an error vector $\hat{\mathbf x}\in \mathbb R^n$ that solves the equation $\mathbf y = \mathbf \Phi(\hat{\mathbf p} + \hat {\mathbf x})$, i.e., it explains the observed inter-agent measurements. Moreover, we require the error reconstruction to be done in a distributed manner, while ensuring that the computational and communication costs of the algorithm do not scale with $|\mathcal V|$. Suppose the error vector is reconstructed correctly, such that $\hat{\mathbf x} = \mathbf x$, then the indices of the non-zero blocks of $\hat{\mathbf x}$ can be identified with the faulty agents (i.e., the agents in $\mathcal D$). Finally, the true configuration of the multi-agent system can be determined as $(\mathcal G, \hat{\mathbf p} + \hat {\mathbf x})$, thereby solving the FDIR problem. However, as we show in the next section, the set of error vectors that explain the observed measurements, $\lbrace\hat{\mathbf x} \in \mathbb R^n \hsp \vert \hsp\mathbf y = \mathbf \Phi(\hat{\mathbf p} + \hat {\mathbf x})\rbrace$, may have infinitely many elements.
Therefore, additional assumptions and/or regularization techniques are needed to uniquely reconstruct ${\mathbf x}$.

% \vspace{2pt}
% \begin{remark}[Localization using Anchors]
% In order to cast our problem formulation into the special case where the identities of the anchors in $\mathcal D^\complement$ is known \textit{a priori}, we can include the anchors' measurements in $\mathbf \Phi$. For instance, letting the $l^{th}$ hyperedge in $\mathcal E$ correspond to an anchor's measurement of its own state, we have $\mathbf \Phi^{(l)}=\mathbf p[i]$, where $\mathcal E^{(l)} = \lbrace i\rbrace$, and $i \in \mathcal D^\complement$.
% \end{remark}
% \vspace{2pt}