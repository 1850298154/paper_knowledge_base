\subsection{Notational Preliminaries}
\label{subsec:notation}

Let $n_1, n_2, \dots, n_k$ be a sequence of $k$ positive integers whose sum is denoted by $n \coloneqq \sum_{i=1}^k n_i$.
Given a block vector $\mathbf v\in\mathbb R^{n}$ that is partitioned into $k$ blocks of lengths $n_1, n_2, \dots, n_k$, $\mathbf v[i]\in \mathbb R^{n_i}$ refers to the $i^{th}$ block of $\mathbf v$. Similarly, given a block matrix $\mathbf A$, its $(i,j)^{th}$ block (whose dimensions are understood from context) is denoted as $\mathbf A[i,j]$. The $l_2$ norm (or the Euclidean norm) of $\mathbf v$ is denoted by $\|\mathbf v\|$.
Similar to \cite{efficient_block_sparse_2010, robust_NSP_2017}, we define the following notation:
\[
\|\mathbf v\|_{2,q}=
\begin{array}{lc}
     \Big( \sum_{i=1}^{k} \|\mathbf v[\small i]\|^q\Big)^{1/q} 
\end{array}
\]
where $0<q<\infty$. It follows from the definition that $\|\mathbf v\|_{2,2} = \|\mathbf v\|$.
The case of $q=0$ can be evaluated using limits, giving us the definition,
\[
\|\mathbf v\|_{2,0} = \sum_{i=1}^{k} \mathbb I\Big(\|\mathbf v[\small i]\|>0\Big)
\]
where $\mathbb I(\hspace{1pt}\cdot\hspace{1pt})$ is the indicator function which is equal to $1$ when the inequality holds,
and $0$ otherwise. 
% The quantity $\|\bold v\|_{2,0}$ equals the number of blocks in $\bold v$ that are non-zero, and is referred to as its \textit{block sparsity}.
% $\|\cdot\|_{2,q}$ is a norm for $q\geq 1$, with $\|\cdot\|_{2,2}=\|\cdot\|_2$. 
Given a set $\mathcal S$, $|\mathcal S|$ denotes its cardinality.
Given an index set $\mathcal S\subset \{1, 2, \dots, k\}$, $\mathcal S^\complement$ denotes its complement, given by $\{1, 2, \dots, k\}\backslash \mathcal S$. 
% $\bold v_{\mathcal S}$ is the vector whose support is restricted to the blocks corresponding to $\mathcal S$, i.e., 
% \[\bold v_{\mathcal S}[i] = \begin{cases}
% \begin{array}{ll}
% \bold v[i] \quad & i\in \mathcal S\\
% \bold 0 & i\in \mathcal S^\complement,
% \end{array}
% \end{cases}
% \]
% where $\bold 0$ denotes a vector of zeros. 
% $\bold 1_d$ denotes a vector of ones of length $d$ and $\bold I_d$ is the $d\times d$ identity matrix. For a matrix $\bold A$, we let 
% % $\vrange(\bold A)$ refer to its column space (i.e., the linear span of its columns), whereas 
% $\ker(\bold A)$ denote its kernel or null space. 
% Lastly, `$\otimes$' is the Kronecker product operation for vectors and matrices.