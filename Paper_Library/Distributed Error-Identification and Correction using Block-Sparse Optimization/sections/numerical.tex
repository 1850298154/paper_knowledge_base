In this section, we consider the example of a multi-agent system consisting of unmanned aerial vehicles (UAVs) that are able to measure their distances from each other (which is the same scenario that was considered in the example in Section \ref{subsec:example}). In practice, the inter-agent distances can be measured using the time-of-arrival (ToA) of inter-agent communications \cite{wen2020multi}.
Each agent is also equipped with a GNSS receiver which enables it to estimate its own position.
However, some of the agents (selected at random) have estimation errors due to multipath effects and/or loss of GNSS tracking loops, which are two types of GNSS faults that are known to occur in practice \cite{siebert2022multipath}. Using numerical simulations, it is demonstrated that the proposed distributed multi-agent FDIR algorithm (Algs. \ref{alg:admm} and \ref{alg:inner}) can process the inter-agent distances to correctly identify the faulty agents and reconstruct their error vectors. To keep within the scope of this paper, it is assumed that there is no noise in the inter-agent distance measurements\footnote{The effect of measurement noise was investigated, both analytically and numerically, in \cite{khan2023recovery}.}.

\subsubsection*{Simulation Scenario} The parameters of the simulation are as follows. The multi-UAV configuration $(\mathcal G, \mathbf x)$ consists of $20$ UAVs, where $\lbrace\mathbf p[i] \in \mathbb R^3\rbrace_{i\in\mathcal V}$ represent their position vectors, depicted by the black dots in Fig. \ref{fig:single-run}. In the figure, the black lines are used to connect the pairs of agents that are able to measure their distances from each other. 
For the configuration in Fig. \ref{fig:single-run}, the rank of $\mathbf J_{\mathbf \Phi}(\mathbf p)$ was determined to be $54$; its first six eigenvalues are close to $0$ (smaller than $10^{-13}$ in magnitude) and its seventh-smallest eigenvalue is non-zero (on the order of $10^{-4}$). Moreover, we have $n=3|\mathcal V|=60$ as the dimension of the configuration space.
Hence (using Theorem \ref{thm:generalized}), the search-space of the error reconstruction problem is a $6$-dimensional embedded submanifold in $\mathbb R^{60}$.

In each simulation, the set of faulty agents, $\mathcal D \subseteq \mathcal V$, is constructed by selecting $6$ agents at random. Given a faulty agent $i\in\mathcal D$, we let its position estimate be $\hat{\mathbf p}[i] = \mathbf p[i] + \mathbf x[i]$, where $\mathbf x[i]\in\mathbb R^3$ is sampled from a random vector having a uniform distribution over the domain $[-1,1]^3$ (i.e., each element of $\mathbf x[i]$ lies between $[-1,1]$). Given a fault-free agent $i\in\mathcal D^\complement$, $\mathbf x[i] $ is set to $\mathbf 0$.
The position estimates of the agents are depicted by the yellow discs in Fig. \ref{fig:single-run}.

\subsubsection*{Multi-Agent FDIR Performance} The proposed distributed multi-agent FDIR algorithm is implemented using the following parameters. We set $\rho=1$ and $N_{\textrm{ADMM}}=10$, i.e., the nonlinear constraint is re-linearized after every $10$ ADMM iterations. Figure \ref{fig:single-run} shows the reconstructed error vector $\mathbf x^*$ after $6$ SCP iterations (which corresponds to a total of $60$ ADMM iterations). Only the blocks of $\mathbf x^*$ whose Euclidean norm is larger than $10^{-5}$ are visualized in this figure; these `non-zero' blocks are seen to correspond precisely to the faulty agents. 

Figure \ref{fig:block_errors} visualizes the accuracy of the reconstructed error vector at each agent, which is represented by the quantity $\|\mathbf x[i] - \mathbf x^*[i]\|$; the red lines correspond to the agents in $\mathcal D$, i.e., the faulty agents. It can be observed that the error at each agent asymptotically tends to $0$. Furthermore, a defining feature of the proposed multi-agent FDIR algorithm is that it is interpretable; specifically, we showed in Theorem \ref{thm:threshold} that 
each agent does something akin to \textit{residual testing} in order to determine whether it has a fault. Figure \ref{fig:thresholds} depicts the residuals of each agent, as well as the fault-detection threshold (blue dashed line), whose expressions can be found in Theorem \ref{thm:threshold}. As expected, the faulty agents are precisely the ones whose residuals have an energy greater than the detection threshold. Figure \ref{fig:thresholds_2} depicts the same plot for $\rho = 0.1$, confirming our observation in Theorem \ref{thm:threshold} that the penalty parameter is inversely proportional to the fault-detection threshold. Furthermore, observe that the faults are correctly identified within $40$ and $20$ ADMM iterations in Figures \ref{fig:thresholds} and \ref{fig:thresholds_2}, respectively, even though the error reconstruction mechanism takes longer (about $80$ ADMM iterations) to converge (cf. Fig. \ref{fig:block_errors}). This is consistent with the existing literature on ADMM, which notes that the dual variables of ADMM converge rapidly to near-optimal values, and asymptotically to their optimal values.
On the other hand, convergence of the primal variables of ADMM is not guaranteed in general (see Sec. 3.2.1 of \cite{boyd2011distributed}). 

\subsubsection*{Monte Carlo Simulations} To further investigate the accuracy of the error reconstruction mechanism, the numerical simulation was repeated for $100$ Monte Carlo (MC) simulations. We select $6$ faulty agents and execute $10$ SCP iterations in each MC simulation, such that the identities of the faulty agents and their error vectors are randomly sampled at the beginning of each MC simulation. Figure \ref{fig:all_errors} visualizes the quantity $\|\mathbf x - \mathbf x^*\|$, which represents the accuracy of the reconstructed multi-agent error vector; the black line represents its mean and the shaded region corresponds to $\pm 1$ standard deviation. It can be seen that the reconstructed error vector $\mathbf x^*$ approximates the true error vector $\mathbf x$ quite well and that the standard deviation of the quantity $\|\mathbf x - \mathbf x^*\|$ tends to $0$. Thus, the proposed multi-agent FDIR algorithm is able to accurately reconstruct the multi-agent error vector with high probability, irrespective of the identities of the faulty agents.

\begin{figure}
         \centering
         \includegraphics[trim={0.7cm, 1.5cm, 0.8cm, 2.0cm},clip, width=0.9\columnwidth]{images/single_run.png}
        \caption{The black dots correspond to the true states of the multi-agent system, whereas the yellow discs represent the agents' estimated states.
        The red arrows depict the non-zero blocks of the reconstructed error vector after $60$ iterations (in total) of the ADMM loop.}
        \label{fig:single-run}
\end{figure}

\begin{figure}
         \centering
         \includegraphics[width=0.49\textwidth]{images/single_run_block_errors.png}
        \caption{The accuracy of the reconstructed error vector at each agent. The red lines correspond to the agents in $\mathcal D$ (i.e., the faulty agents), and the vertical grey lines indicate the iterations where the constraint was re-linearized.}
        \label{fig:block_errors}
\end{figure}

\begin{figure}[h]
         \centering
         \begin{subfigure}[t]{\linewidth}
         \centering
         \includegraphics[width=\linewidth]{images/single_run_thresholds.png}
         \caption{$\rho = 1.0$}
         \label{fig:thresholds}
         \end{subfigure}
         \begin{subfigure}[t]{\linewidth}
         \centering
         \includegraphics[width=\linewidth]{images/single_run_thresholds_2.png}
         \caption{$\rho = 0.1$}
         \label{fig:thresholds_2}
         \end{subfigure}
        \caption{The fault-detection threshold $1/\rho$ (blue dashed line) and the residual energy $\|\mathbf A_i^\top \mathbf b_i\|$ are visualized for each agent; the definitions of $\mathbf A_i$ and $\mathbf b_i$ are given in (\ref{eq:A_i-b_i}). The red lines correspond to the agents in $\mathcal D$ (i.e., the faulty agents), and the vertical grey lines indicate the iterations where the constraint was re-linearized.}
\end{figure}

\begin{figure}
         \centering
         \includegraphics[width=0.49\textwidth]{images/single_run_all_errors.png}
        \caption{The accuracy of the error reconstruction mechanism is visualized in terms of the mean and $\pm 1$ standard deviation of $\|{\mathbf x} - \mathbf x^*\|$ (depicted by the black line and the shaded region, respectively) across $100$ Monte Carlo simulations. The vertical grey lines indicate the iterations where the constraint was re-linearized.}
        \label{fig:all_errors}
\end{figure}