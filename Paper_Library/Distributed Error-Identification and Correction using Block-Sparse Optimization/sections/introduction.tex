A Cyber-Physical System (CPS) is an autonomous agent that has a physical state, as well as onboard computers and sensors that enable it to observe and navigate its environment safely.
Two or more CPS can be collectively deployed as a multi-agent system to achieve a series of objectives, such that the capabilities of the multi-agent system far exceed those of the individual CPS. 
% As these objectives are typically collaborative in their nature, the agents communicate with each other in order to allocate tasks, maintain prescribed safety margins, and monitor and improve the overall performance of the network. Examples of engineering applications where multi-agent systems have been successfully deployed include autonomous search-and-rescue missions for disaster response \cite{mas_review_S&R2021}. 
However, the increased capabilities of multi-agent systems go hand in hand with their increased complexity, making the design of efficient estimation and decision-making algorithms for multi-agent systems a challenging task. This is especially true in the case of algorithms that have a combinatorial aspect to them, such as the conventional algorithms for fault-detection, identification, and reconstruction (FDIR)\footnote{In the literature, the acronym FDIR is sometimes expanded as fault diagnosis, isolation, and recovery \cite{naderi2017datadriven_FDIR, guo2012distributed}.}. Fault detection refers to the mechanism by which an agent's sensor data is monitored for the presence of systematic anomalies
that manifest as a discrepancy between the estimated state and the true state of the agent. We refer to the presence of such a discrepancy at a given agent as an \textit{error}. In practice, errors can arise due to various types of sensor or actuator faults, including sensor bias, miscalibration, cyberattacks, modeling errors, and programming bugs \cite{gao2022survey}. 
The objective of multi-agent FDIR is then to identify the agents which have faults and correct their error vectors, respectively. In particular, the multi-agent FDIR problem does not discriminate between the types of faults at the agent level, but rather focuses on fault-identification at the network level, which involves identifying all the agents in the network that have faults (see, for example, \cite{kazumune2020distributed}).
% In a multi-agent system, faults that are not detectable using the intrinsic sensors of the individual agents (which could include IMUs, magnetometers, and GNSS receivers), may instead be detected using the relative measurements that are available between the agents; for instance, interconnected CPS that are able to measure their distances from each other can use this information to correct their localization errors \cite{wen2020multi}.

Historically, FDIR algorithms were first developed for the single-agent, multi-sensor scenario. As single-agent systems are lower in dimension and have fewer bottlenecks than multi-agent systems, the FDIR problem can be solved using data-driven methods \cite{naderi2017datadriven_FDIR}, banks of observers \cite{hwang2009survey},  or
combinatorial algorithms like sensor subset search \cite{mishra2016secure}. In general, the FDIR mechanisms designed for single-agent systems are not feasible for use in multi-agent systems due to the collective state vector of a multi-agent system being relatively high-dimensional.
Moreover, the agents are typically limited in their communication and computational capabilities, which rules out the possibility of processing the measurement information in a centralized manner. 
Therefore, several authors have proposed alternative FDIR strategies that are better suited for multi-agent applications, including those based on $H_\infty$ performance indices \cite{chadli2017distributed_hinf_iet, gallehdari2017h}, residual testing \cite{guo2012distributed}, $l_1$ norm minimization \cite{kazumune2020distributed}, and interval observers \cite{zhang2018distributed} among others \cite{zhang2021physical_survey}. 
A limitation of the preceding works is that they do not accommodate the nonlinear measurement models that typically arise in real-world CPS applications, such as inter-agent distance and bearing measurements,
which present unique challenges to the multi-agent FDIR problem. 
For instance, it is known that the inter-agent distances are not sufficient to uniquely determine the identities and states of the faulty agents, which means that additional information channels, assumptions, or regularization techniques are needed in order to solve the FDIR problem \cite{khan2023recovery}. 
%

The study of the observability and controllability of multi-agent systems which are able to obtain nonlinear inter-agent measurements, including distance \cite{oh2015survey,topology_const2015observability}, bearing \cite{zhao2019bearing}, and subtended angle \cite{weak_rigidity} measurements, is collectively referred to as \textit{rigidity theory}. In the literature on rigidity theory, it is assumed that some of the agents, called the anchors, are able to observe their true states, and that the identities of the anchors are known to all the agents \textit{a priori} \cite{zhao2019bearing}. Under this assumption, the agents that are classified as the anchors are exempted from having faults, which ensures that the faults in the remaining agents can be corrected by processing the inter-agent measurements.
% For example, in multi-agent localization using inter-agent distances, the presence of $2$ or $3$ anchors, combined with a condition on the network topology (namely, its rigidity) is used to ensure that each of the agents can be uniquely localized. 
% Note that multi-agent localization is dually related to the multi-agent formation control problem, in which some of the agents are designated as being the leaders \cite{zhao2019bearing}.
However, under the anchor-based approach to multi-agent FDIR, the presence of undetected faults at the anchors can cause the FDIR algorithm to misidentify the faulty agents, since the anchors were assumed to be fault-free. This fact is especially problematic when the faults are introduced in an adversarial manner, e.g., by a cyberattacker who perturbs some of the anchors' state estimates using sensor spoofing attacks, thereby compromising the FDIR capabilities of the entire network.

To address these limitations of existing multi-agent FDIR mechanisms, we propose a novel framework for multi-agent FDIR that does not require any of the agents to be classified as the anchors, and can therefore diagnose the entire multi-agent network for faults. In the proposed framework, the multi-agent FDIR problem is reformulated as one of reconstructing a block-sparse error vector. Thereafter, the nonlinearity of the inter-agent measurements is accommodated using sequential convex programming (SCP) \cite{scp_zillober2004}, and a distributed implementation of the error-reconstruction algorithm is developed using the alternating direction method of multipliers (ADMM) optimization procedure \cite{boyd2011distributed}.
Once the block-sparse multi-agent error vector is reconstructed,
the presence of non-zero blocks in the reconstructed error vector indicates the presence of faults, the positions of the non-zero blocks identify the faulty agents, and the non-zero blocks are precisely the error vectors of the faulty agents; thus, the objectives of multi-agent FDIR are accomplished in an efficient, distributed manner. A secondary goal of this paper is to explain why the proposed approach works, and how it relates to the existing literature on rigidity theory. 

In summary, the contributions of this work are as follows:
\begin{enumerate}
    \item The multi-agent FDIR problem is reformulated as an optimization problem that searches for a block-sparse vector subject to nonlinear equality constraints.
    \item The search-space (i.e., the feasible set) of the preceding problem is characterized using tools from differential geometry. It is shown that our characterization generalizes (and moreover, reinterprets) some of the existing results in rigidity theory.
    \item A combination of the SCP and ADMM optimization techniques is used to develop a distributed algorithm for reconstructing the error vector, which exploits the unique structure of the multi-agent FDIR problem.
    \item The theory of subgradient methods is used to show that the proposed algorithm exhibits a \textit{thresholding property}, by which each agent compares the norm of a residual vector against a pre-specified threshold in order to detect and identify the faults.
    \item Finally, a numerical example of multi-agent FDIR using inter-agent distance measurements is used to demonstrate the effectiveness of the proposed approach.
\end{enumerate}

The remainder of this paper is organized as follows. Section \ref{sec:prob_form} presents some preliminary definitions and assumptions, and formulates the multi-agent FDIR problem. In Section \ref{sec:reconstruction}, we characterize the search-space of the multi-agent FDIR problem, explain its connections to rigidity theory, and motivate the assumption of block-sparsity of the error vector. In Section \ref{sec:distributed}, the distributed multi-agent FDIR algorithm is developed and analyzed. Section \ref{sec:numerical} presents numerical simulations that validate our theoretical results and demonstrate the effectiveness of the proposed algorithm. Finally, the conclusions and future research directions are discussed in Section \ref{sec:conclusions}.
% \iffalse
% , we 
% , one desires a multi-agent FDIR mechanism that does not rely on the anchors' state estimates
% the problem of identifying and reconstructing the faults in a multi-agent system in the absence of anchors (i.e., agents whose position estimates can be trusted), is an important problem.
% To address the limitations, we develop a novel multi-agent FDIR mechanism which can identify the faulty agents and reconstruct their error vectors in a distributed manner. 
% In particular, our analysis accommodates the nonlinearity of the inter-agent measurement models, and does not
% assume that any of the agents can be assumed to be free of faults.
% . To develop the proposed algorithm, we formulate the multi-agent FDIR problem as a compressive sensing problem, in which the collective error vector of the multi-agent system is treated as a block-sparse vector that needs to be reconstructed using the inter-agent measurements. 
% Using tools from differential geometry, it is shown that the assumption of block-sparsity of the error vector
% We then apply this framework to the multi-agent FDIR problem
% In \cite{kazumune2020distributed}, the authors were able to uniquely and efficiently reconstruct the faults by introducing the assumption that the faults are sparse, which is inspired by the literature on compressive sensing. This is similar to the approach used in our paper, with some crucial differences: (i) we consider the nonlinearity of the inter-agent measurements
% In particular, Theorem \ref{thm:generalized} generalizes the notion of global rigidity of configurations. Similar results about local and global rigidity were obtained in \cite{Asimow1978} and \cite{krick2009stabilisation}, respectively, for the special case of distance-based localization. The most widely studied property is that of infinitesimal rigidity, which does not require the theory of smooth manifolds; rather, it can be studied using linear algebra by linearizing the measurement model.
% \fi
% \red{Pending...}