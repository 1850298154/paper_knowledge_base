
When in the function $f(\bold v)$ in Theorem \ref{thm:proximal} we set $\bold A=\bold I$,
we get the proximal operator for $\|{}\cdot{}\|_2$, defined by:

\begin{align}
\textit{prox}_{\|{}\cdot{}\|_2}(\bold b) &= \argmin_{\bold v} \left(\frac{1}{2}\|\bold v - \bold b\|_2^2 + \|\bold v\|_2\right)\\&= \begin{cases}
\begin{array}{cl}
    \left(1- \frac{1}{\|\bold b\|_2}\right)\bold b &   \|\bold b\|_2 > 1\\
     \bold 0 &  \|\bold b\|_2 \leq 1
\end{array}
\end{cases}
\end{align}

which has an easier derivation by way of the Moreau decomposition. Our derivation additionally handles the cases of $\bold A$ being rank deficient, and is consistent with the proximal operator of $\|{}\cdot{}\|_2$ when $\bold A = \bold I$. More generally, we have the following corollaries, which are proved via substitution.

\begin{corollary}
If $\bold b$ is an eigenvector of $\bold A\bold A^T$ with eigenvalue $\lambda$, then the minimizer $\bold v^*$ of $f(\bold v)$ in Theorem \ref{thm:proximal} is given by
\begin{equation}
    \bold v^* = 
\begin{cases}
 \begin{array}{cl}
    \frac{1}{\lambda}\left(1- \frac{1}{\|\bold A^T\bold b\|_2}\right)\bold A^T\bold b &   \|\bold A^T \bold b\|_2 > 1\\
     \bold 0 &  \|\bold A^T\bold b\|_2 \leq 1
 \end{array}
\end{cases}
\label{eq:corollary_condition}
\end{equation}
\end{corollary}

% \begin{corollary}
% The minimizer $\bold v^*$ is given by
% \begin{equation}
%     \bold v^* = 
% \begin{cases}
%  \begin{array}{cl}
%     \left(\bold A^T \bold A + \sfrac{1}{k} \bold I\right)^{-1} \bold A^T \bold b &   \|\bold A^T \bold b\|_2 > 1\\
%      \bold 0 &  \|\bold A^T\bold b\|_2 \leq 1
%  \end{array}
% \end{cases}
% \label{eq:corollary2_condition}
% \end{equation}
% where $k>0$ is the solution to
% \begin{equation}
%     \|\left(k\bold A^T \bold A + \bold I\right)^{-1} \bold A^T \bold b\|_2^2 = 1,
% \end{equation}
% \end{corollary}

% Thus, we have a nice characterization of the solution for special cases, such as when $\bold A$ is orthogonal (in which (\ref{eq:corollary_condition}) holds independent of $\bold b$). In general, we only need to find $k$, which requires solving a polynomial with one variable. A closed-form expression for $k$ may be possible.

% Note that the convergence rate of proximal gradient descent is $O(1/k)$, accelerated proximal gradient descent is $O(1/k^2)$ \cite{parikh2014proximal}, which are both faster and more robust than subgradient descent.