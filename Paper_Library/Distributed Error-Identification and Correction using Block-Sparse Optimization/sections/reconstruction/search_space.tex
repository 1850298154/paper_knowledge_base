Following the definitions in \cite[p. 182]{lee2012}, a vector $\tilde {\mathbf p} \in \mathbb R^n$ is said to be a \textit{regular point} of $\mathbf \Phi$ if $\mathbf J_{\mathbf \Phi}(\tilde {\mathbf p})$ has full row rank, i.e., is a surjective map. A \textit{regular value} of $\mathbf \Phi$ is a vector $\tilde {\mathbf y}\in \mathbb R^m$ in the codomain, such that every element of $\mathbf \Phi^{-1}[\lbrace \tilde{\mathbf y} \rbrace]$ is a regular point. The relevance of regular values is due to the following lemma.

% \vspace{2pt}
% \begin{lemma}[Constant-Rank Level Set Theorem \cite{lee2012}]
% Suppose $\mathbf J_{\mathbf \Phi}(\mathbf p)$ has a constant rank (for all $\mathbf p\in U$) equal to $k$, $\mathbf \Phi^{-1}[\lbrace \mathbf z \rbrace]$ is an $(n-k)$-dimensional embedded submanifold of $\mathbb R^n$.
% \end{lemma}

\vspace{2pt}
\begin{lemma}[Regular Level Set Theorem \protect{\cite[Corr. 8.10]{lee2012}}]
If $\tilde {\mathbf y}$ is a regular value of $\mathbf \Phi$, then $\mathbf \Phi^{-1}[\lbrace \tilde{\mathbf y} \rbrace]$ is an $(n-m)$-dimensional embedded submanifold of $\mathbb R^n$.
\label{lem:regular}
\end{lemma}
\vspace{2pt}

Lemma \ref{lem:regular} can be used to show that, while the measurement vector $\mathbf y$ does not uniquely determine the configuration, it effectively restricts the set of possible configurations (which is given by the preimage, $\mathbf \Phi^{-1}[\lbrace {\mathbf y}\rbrace]$), to an embedded submanifold in $\mathbb R^n$, i.e., a smooth `surface'.
This means that we can reconstruct the true configuration $(\mathcal G, \mathbf p)$ more efficiently by restricting our search-space to this lower-dimensional submanifold, by starting at a point on this submanifold and moving in tangential directions to search for the true configuration.

\vspace{2pt}
\begin{theorem}[Characterization of the Search-Space]
Let $k$ denote the rank of $\mathbf J_{\mathbf \Phi}(\mathbf p)$. There exists a point $\tilde{\mathbf p}$ arbitrarily close to $\mathbf p$,  such that the points in $\mathbf \Phi^{-1}[\lbrace   {\mathbf \Phi(\tilde{\mathbf p})}\rbrace]$ lie on an $(n-k)$-dimensional embedded submanifold of $\mathbb R^n$.
\label{thm:generalized}
\end{theorem}
\begin{proof}
As $\mathbf J_{\mathbf \Phi}(\mathbf p)$ is a rank $k$ matrix, we can construct a subset $\mathcal I \subseteq \lbrace 1, 2, \dots, m\rbrace$ consisting of $k$ elements, such that the rows of $\mathbf J_{\mathbf \Phi}({\mathbf p})$ which are indexed by $\mathcal I$ are linearly independent. Thereafter, define the vector-valued map $\mathbf \Phi_{\mathcal I}:\mathbb R^n \rightarrow \mathbb R^k$ whose components are precisely the components of $\mathbf \Phi(\cdot)$ which are indexed by $\mathcal I$. Let $\mathbf y_{\mathcal I}\in \mathbb R^k$ denote the vector constructed by choosing the components of $\mathbf y$ which are indexed by $\mathcal I$.

We observe that ${\mathbf p}$ is a regular point of $\mathbf \Phi_{\mathcal I}$. However, the application of Lemma \ref{lem:regular} requires ${\mathbf y}_{\mathcal I}$ to be a regular value, which may not be the case.
To fix this issue, we invoke \textit{Sard's theorem} \cite[Thm. 10.7]{lee2012}, which states that the set of elements of $\mathbb R^k$ that are not regular values of $\mathbf \Phi_{\mathcal I}  $ constitute a measure-zero subset of $\mathbb R^k$. Thus, any (arbitrarily small) open neighborhood of ${\mathbf y}_{\mathcal I}$ in $\mathbb R^k$ contains a regular value in it; in fact, almost all vectors in such a neighborhood are regular values. As $\mathbf \Phi_{\mathcal I}$ is surjective in the neighborhood of $\mathbf p$ \cite[Prop. 2]{Asimow1978}, there exists a point $\tilde {\mathbf p}$ arbitrarily close to $\mathbf p$ that maps to a regular value, i.e., $\mathbf \Phi_{\mathcal I}(\tilde{\mathbf p})$ is a regular value of $\mathbf \Phi_{\mathcal I}$. Finally, observe that
\begin{equation}
\mathbf \Phi^{-1} \bigl[\lbrace  \mathbf \Phi(\tilde{\mathbf p}) \rbrace\bigr] \subseteq \mathbf \Phi_{\mathcal I}^{-1} \bigl[\lbrace  \mathbf \Phi_{\mathcal I}(\tilde{\mathbf p})\rbrace\bigr].
\label{eq:subsets}
\end{equation}
Applying Lemma \ref{lem:regular} to the right-hand side of (\ref{eq:subsets})
completes the proof.
\end{proof}
\vspace{2pt}


% \vspace{2pt}
% \begin{theorem}[Characterization of the Search-Space]
% Let $k$ be equal to the maximal rank attained by $\mathbf J_{\mathbf \Phi}(\cdot)$ in its domain. There exists an arbitrarily small perturbation $\delta {\mathbf y}\in\mathbb R^m$ of the measurement vector $\mathbf y$, such that the points in $\mathbf \Phi^{-1}[\lbrace  {\mathbf y + \delta \mathbf y}\rbrace]$ lie on an $(n-k)$-dimensional embedded submanifold of $\mathbb R^n$.
% \label{thm:generalized}
% \end{theorem}
% \begin{proof}
% By the definition of $k$, there exists $\tilde{\mathbf p}\in\mathbb R^n$ such that $\mathbf J_{\mathbf \Phi}(\tilde{\mathbf p})$ has rank $k$. This means that we can construct a subset $\mathcal I \subseteq \lbrace 1, 2, \dots, m\rbrace$ consisting of $k$ elements, such that the rows of $\mathbf J_{\mathbf \Phi}(\tilde{\mathbf p})$ which are indexed by $\mathcal I$ are linearly independent. Thereafter, define the vector-valued map $\tilde {\mathbf \Phi}:\mathbb R^n \rightarrow \mathbb R^k$ whose components are precisely the components of $\mathbf \Phi(\cdot)$ which are indexed by $\mathcal I$. The rest of the proof relies on the observation that $\mathbf J_{\tilde {\mathbf \Phi}}(\tilde {\mathbf p})$ has full row rank, i.e., $\tilde{\mathbf p}$ is a regular point of $\tilde {\mathbf \Phi}$. 

% Let the vector $\delta \mathbf y \in \mathbb R^m$ be chosen such that the components of $\mathbf y + \delta \mathbf y$ which are indexed by $\mathcal I$ agree with the corresponding components of $\tilde{\mathbf \Phi} (\tilde {\mathbf p})$. Then, we have

% \begin{equation}
% \mathbf \Phi^{-1} \bigl[\lbrace  \mathbf y + \delta \mathbf y \rbrace\bigr] \subseteq \tilde{\mathbf \Phi}^{-1} \bigl[\lbrace \tilde{\mathbf \Phi} (\tilde {\mathbf p})\rbrace\bigr].
% \label{eq:subsets}
% \end{equation}
% On the left-hand side of (\ref{eq:subsets}), it remains to be shown that $\delta \mathbf y$ can be chosen to be arbitrarily small. On the right-hand side of (\ref{eq:subsets}), we need to show that $\tilde{\mathbf \Phi} (\tilde {\mathbf p})$ is a regular value of $\tilde{\mathbf \Phi}$, enabling the use of Lemma \ref{lem:regular}. Together, the two steps would imply the main claim of the theorem, which is that the left-hand side of $(\ref{eq:subsets})$ is contained in an $(n-k)$-dimensional embedded submanifold of $\mathbb R^n$.
% To show both of these facts, we invoke \textit{Sard's theorem} \cite[Thm. 10.7]{lee2012}; it states that the set of elements of $\mathbb R^k$ which are not regular values of $\tilde {\mathbf \Phi}  $ constitute a measure-zero subset of $\mathbb R^k$. Consequently, any given (arbitrarily small) open neighborhood of $\tilde {\mathbf \Phi} (\mathbf p)$ in $\mathbb R^k$ must contain regular values; if it did not, then Sard's theorem would fail to hold, as 
% one would have found a subset of non-zero measure that contains no regular values. Choosing $\tilde {\mathbf p}$ to lie in the preimage (under $\tilde{\mathbf \Phi}$) of one such regular value completes the proof.
% \end{proof}
% \vspace{2pt}

\vspace{2pt}
\begin{remark}[Regular Values are Almost-Everywhere]
When $\mathbf y$ has some amount of randomness in each of its components (e.g., due to measurement noise), Sard's theorem establishes that the vector $\mathbf y_{\mathcal I}$ is a regular value of $\mathbf \Phi_{\mathcal I}$ with probability one.
Hence, the fact that $\tilde{\mathbf p}$ is not identically equal to $\mathbf p$ in the claim of Theorem \ref{thm:generalized} can be viewed as a technicality. Similar technicalities arise in the study of multi-agent localization, which motivates the study of \textit{generic} configurations, as opposed to the special, measure-zero subset of configurations for which certain properties may fail to hold \cite{whiteley1985generating, anderson2010formal, hendrickson1992}. 
\label{rem:technicality}
\end{remark}