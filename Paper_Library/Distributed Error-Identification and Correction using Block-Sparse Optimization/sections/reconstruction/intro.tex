Given a subset $Y\subseteq \mathbb R^m$ of the codomain of $\mathbf \Phi$, we let
 $\mathbf \Phi^{-1}\bigl[Y\bigr]$ denote its preimage under $\mathbf \Phi$, defined as follows:
 \begin{equation}
     \mathbf \Phi^{-1}\bigl[Y\bigr] = \lbrace \mathbf v \in \mathbb R^n \hsp\vert\hsp \mathbf \Phi (\mathbf v) \in Y \rbrace
 \end{equation}
%
Thus, $\mathbf \Phi^{-1}[\lbrace \mathbf y\rbrace]\subseteq \mathbb R^n$ is the set of all state vectors that may have generated the measurement $\mathbf y$, which we refer to as the \textit{search-space} of the error reconstruction problem. In particular, we have that $\mathbf p \in \mathbf \Phi^{-1}[\lbrace \mathbf y\rbrace]$.

For each of the inter-agent measurement models listed in Table \ref{tab:iamms}, the function $\mathbf \Phi$ is neither injective (i.e., one-one) nor surjective (i.e., onto), making it non-invertible. To see that it may fail to be injective, observe that rigid translations and rotations of Euclidean space preserve the inter-agent distances, while scaling (expansion and shrinking) of Euclidean space preserves inter-agent bearings, which means that two different configurations may generate the same measurement vector \cite{zhao2019bearing}.
To see why $\mathbf \Phi$ may fail to be surjective, consider $3$ agents organized in a triangular formation; their inter-agent distances (or displacements) must then satisfy the triangle inequality, and therefore $\mathbf \Phi$ does not attain the points in its codomain where the triangle inequality fails. Therefore, we have that $\mathbf \Phi^{-1}[\lbrace \mathbf y\rbrace] \neq \lbrace \mathbf p \rbrace$ in general. Nevertheless, it is still possible to uniquely reconstruct $\mathbf p$ using $\mathbf y$ if additional assumptions are introduced. 
% In the existing literature on the localization of multi-agent systems, the additional assumption which enables the unique reconstruction of the error vector $\mathbf x$ is that a subset $\mathcal A \subseteq \mathcal V$ of the agents, called as the set of anchors, are known to not have any faults. 

In the remainder of this section, we use tools from differential geometry to show that $\mathbf x$ may be uniquely reconstructed if the errors are assumed to be sparse, i.e., $|\mathcal D|\ll|\mathcal V|$. We also draw parallels between the main result of this section (Theorem \ref{thm:generalized}) and the existing literature on rigidity theory, showing that our analysis generalizes the concept of rigidity in distance and bearing-based multi-agent localization (which are both special cases of our problem formulation).