A special case of the error reconstruction problem is that of localizing a multi-agent system using inter-agent measurements, which has been previously studied using \textit{rigidity theory} \cite{zhao2019bearing}. In this subsection, we consider the specific example of multi-agent localization using inter-agent distances, using it to illustrate the relevance of Theorem \ref{thm:generalized}, as well as motivate our forthcoming analysis. 

Consider the example where $\mathbf p[i]\in \mathbb R^3$ is the $i^{th}$ agent's position vector and each component of $\mathbf \Phi$ is of the form
\begin{equation}
    \mathbf \Phi^{(l)}(\mathbf p)=\|\mathbf p[i] - \mathbf p[j]\|
\end{equation}
where $\mathcal E^{(l)}=\lbrace i, j\rbrace$. This gives us the dimensions $n=3|\mathcal V|$ and $m=|\mathcal E|$ of the domain and codomain of $\mathbf \Phi$, respectively. In this case, the matrix $\mathbf J_{\mathbf \Phi}(\mathbf p)\in \mathbb R^{|\mathcal E|\times 3|\mathcal V|}$ is called the \textit{rigidity matrix}, and its maximal rank is well-known to be equal to $(3|\mathcal V|-6)$ \cite{eren2004rigidity, zelazo2015decent, hendrickson1992}. A multi-agent configuration is said to be \textit{infinitesimally rigid} if the rank of its rigidity matrix is maximal. Similar to our observation in Remark \ref{rem:technicality}, infinitesimal rigidity fails to hold for a measure-zero subset of the configuration space \cite[Thm. 2.1]{hendrickson1992}. 

Suppose an infinitesimally rigid configuration generates the inter-agent distance vector, $\mathbf y \in \mathbb R^{|\mathcal E|}$. Using Theorem \ref{thm:generalized} with $n=3|\mathcal V|$ and $k=(3|\mathcal V|-6)$, we see that the set of all configurations that generate the measurement $\mathbf y$ constitute a $6$-dimensional embedded submanifold of $\mathbb R^3$. Denote this submanifold as $\mathcal M_{\mathbf y}$. 
As rigid translations and rotations (i.e., the isometries) of the multi-agent configuration preserve the inter-agent distances, it follows that each connected component of $\mathcal M_{\mathbf y}$ corresponds to the $SE(3)$ Lie group\footnote{To state this more precisely, each of the connected components of $\mathcal M_{\mathbf y}$ is the orbit of some configuration $(\mathcal G,\tilde{\mathbf p})$ under $SE(3)$ equipped with an appropriate Lie group action on $\mathbb R^n$. A similar observation is made in \cite[Lemma 3]{krick2009stabilisation} for the two-dimensional case.} (which are precisely the isometries of the three-dimensional Euclidean space, excluding reflections). Indeed, $SE(3)$ is 6-dimensional as a manifold \cite[Sec. 1.2.5]{hall2013lie}. 

Note that the manifold $\mathcal M_{\mathbf y}$ may have two or more disconnected components. The disconnected components of $\mathcal M_{\mathbf y}$ correspond to what are called \textit{flip ambiguities} in the literature on rigidity theory, which is the phenomenon that is visualized in Fig. \ref{fig:example}. The two configurations shown in Fig. \ref{fig:example} both lie on $\mathcal M_{\mathbf y}$ as they produce the same vector of inter-agent distances, $\mathbf y$. However, there is no continuous path on $\mathcal M_{\mathbf y}$ that connects the two configurations in $\mathbb R^{3|\mathcal V|}$, for the same reason that no continuous motion of the vertices of one configuration can make it into the other while preserving the inter-agent distance measurements.

\begin{figure}
     \centering
     \begin{minipage}[b]{0.22\textwidth}
         \centering
         \includegraphics[trim={0.95cm 1.0cm 0.95cm 1.0cm},clip, width=\textwidth]{images/flip_ambiguity_1.png}
         % \caption{$y=x$}
     \end{minipage}
     % \hfill
     \ 
     \begin{minipage}[b]{0.22\textwidth}
         \centering
         \includegraphics[trim={0.95cm 1.0cm 0.95cm 1.0cm},clip, width=\textwidth]{images/flip_ambiguity_2.png}
         % \caption{$y=3\sin x$}
     \end{minipage}
        \caption{Two configurations of a rigid graph that produce the same measurement vector $\mathbf y$, but are not related to each other via rigid translations or rotations, are said to be \textit{flip ambiguous}.}
        \label{fig:example}
\end{figure}

As we mentioned in Remark \ref{rem:anchors}, the existing literature assumes that the anchors are known to be correctly localized. Equivalently, they assume that the components of $\mathbf x$ which correspond to the anchors are equal to $\mathbf 0$. Observe that the foregoing assumption defines a subspace of the configuration space. By intersecting this subspace with $\mathcal M_{\mathbf y}$, the true configuration may be recovered (up to flip ambiguity), as the intersection of two manifolds is (typically) a lower-dimensional manifold \cite[p. 203]{lee2012}. In particular, a zero-dimensional manifold is a set of discrete points that correspond to configurations that are infinitesimally rigid.

In the following subsection, we show that the assumption that the anchors' identities are known (i.e., that the locations of the non-zero blocks of $\mathbf x$ are known) can be relaxed in favor of a milder assumption: the errors are sparse (i.e., some of the blocks of $\mathbf x$ are equal to $\mathbf 0$). The latter assumption is less restrictive in the context of the FDIR problem, as none of the agents' states need to be trusted, but rather, the entire multi-agent network may be diagnosed for faults.
