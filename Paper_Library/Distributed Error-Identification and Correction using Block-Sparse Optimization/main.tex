\documentclass[journal,twoside,web]{ieeecolor}
\usepackage{generic,cite}
\usepackage{amsmath,amssymb,amsfonts, mathtools}
\usepackage{algorithm, algpseudocode}
\usepackage[pdftex]{graphicx}
\usepackage{textcomp, subcaption}

\usepackage{soul}
\newcommand{\hlt}[1]{\hl{#1}}
\newcommand{\red}[1]{\textcolor{red}{#1}}

% \usepackage{amsthm}
% \newtheoremstyle{thmspaced}
%   {\topsep} % Space above
%   {\topsep} % Space below
%   {} % Body font
%   {} % Indent amount
%   {\bfseries} % Theorem head font
%   {.} % Punctuation after theorem head
%   {.5em} % Space after theorem head
%   {} % Theorem head spec (can be left empty, meaning `normal')

% \theoremstyle{thmspaced}
\newtheorem{theorem}{Theorem}
\newtheorem{lemma}{Lemma}
\newtheorem{proposition}{Proposition}
\newtheorem{corollary}{Corollary}[theorem]

\newtheorem{assumption}{Assumption}
\newtheorem{definition}{Definition}
\newtheorem{remark}{Remark}

\usepackage[table]{xcolor}
\setlength{\arrayrulewidth}{0.5mm}
\setlength{\tabcolsep}{18pt}
% \renewcommand{\arraystretch}{2.5}
\newcommand{\hsp}{\hspace{1pt}}
\DeclareMathOperator*{\argmin}{arg\,min}

\usepackage{bigstrut, xfrac, subfiles}
\hyphenation{non-linear}

% \usepackage{tikz}
% \usetikzlibrary{matrix,decorations.pathreplacing, calc, positioning,fit, decorations.pathreplacing,calligraphy}
\def\BibTeX{{\rm B\kern-.05em{\sc i\kern-.025em b}\kern-.08em
T\kern-.1667em\lower.7ex\hbox{E}\kern-.125emX}}
\markboth{\journalname, VOL. XX, NO. XX, XXXX 2017}
{Khan \MakeLowercase{\textit{et al.}}: Distributed Error-Identification and Correction using Block-Sparse Optimization}

\begin{document}
\title{
Distributed Error-Identification and Correction using Block-Sparse Optimization}
\author{
Shiraz Khan and Inseok Hwang, \IEEEmembership{Member, IEEE}
\thanks{\textcolor{blue}{© 2023 IEEE. Personal use of this material is permitted. Permission from IEEE must be obtained for all other uses, in any current or future media, including reprinting/republishing this material for advertising or promotional purposes, creating new collective works, for resale or redistribution to servers or lists, or reuse of any copyrighted component of this work in other works.}}
\thanks{The authors are with the School of Aeronautics and Astronautics, Purdue University,
West Lafayette, IN 47906. Email: {\tt\small shiraz, ihwang@purdue.edu}}
\thanks{This research is funded by the Secure Systems Research Center (SSRC) at the Technology Innovation Institute (TII), UAE. The authors are grateful to Dr. Shreekant (Ticky) Thakkar and his team members at the SSRC for their valuable comments and support.}
}
%
\maketitle
%
\begin{abstract}
% An outstanding challenge that arises in the operation of multi-agent systems is that of fault-detection, identification, and reconstruction (FDIR), where the objective is to use efficient sensor fusion and communication protocols to identify the agents that have erroneous state estimates and uniquely reconstruct their true states.
% Faults can arise due to sensor bias, modeling errors, or the presence of malicious agents in the environment that may compromise the onboard estimators of one or more CPS through cyberattacks.
% While there are existing algorithms for distributed multi-agent localization using inter-agent measurements, each of them requires the identities of one or more anchors (i.e., agents that do not have errors) to be known; in contrast, the proposed algorithm relaxes the foregoing requirement, allowing it to identify and recover from a larger class of faults. 
The conventional solutions for fault-detection, identification, and reconstruction (FDIR) require centralized decision-making mechanisms which
are typically combinatorial in their nature, necessitating the design of an efficient distributed FDIR mechanism that is suitable for multi-agent applications.
To this end, we develop a general framework for efficiently reconstructing a sparse vector being observed over a sensor network via nonlinear measurements. The proposed framework is used to design a distributed multi-agent FDIR algorithm based on a combination of the sequential convex programming (SCP) and the alternating direction method of multipliers (ADMM) optimization approaches. The proposed distributed FDIR algorithm can process a variety of inter-agent measurements (including distances, bearings, relative velocities, and subtended angles between agents) to identify the faulty agents and recover their true states.
The effectiveness of the proposed distributed multi-agent FDIR approach is demonstrated by considering a numerical example in which the inter-agent distances are used to identify the faulty agents in a multi-agent configuration, as well as reconstruct their error vectors.
\end{abstract}
%
\begin{IEEEkeywords}
fault-identification and reconstruction, networked systems, rigidity theory, compressive sensing 
\end{IEEEkeywords}
%
% -------------------------------------
%
\section{Introduction}
\label{sec:introduction}
\subfile{sections/introduction}
%
% -------------------------------------
%
\section{Problem Formulation}
\label{sec:prob_form}
\subfile{sections/notation.tex}
\subfile{sections/problem_formulation}
%
% -------------------------------------
%
\section{Error Reconstruction using Sparsity}
\label{sec:reconstruction}
\subfile{sections/reconstruction/intro}
%
\subsection{Characterization of the Search-Space}
\label{subsec:search-space}
\subfile{sections/reconstruction/search_space}
%
\subsection{Connections to Rigidity Theory}
\label{subsec:example}
\subfile{sections/reconstruction/example}
%
\subsection{The Role of Sparsity}
\label{subsec:sparsity}
\subfile{sections/reconstruction/sparsity}
%
% -------------------------------------
%
\section{Distributed FDIR using ADMM}
\label{sec:distributed}
\subfile{sections/distributed/intro}
%
\subsection{Reduction to Convex Sub-Problems}
\label{subsec:scp}
\subfile{sections/distributed/scp}
%
\subsection{Derivation of the Distributed Algorithm}
\label{subsec:admm}
\subfile{sections/distributed/admm}
%
\subsection{Further Analysis of Algorithm \ref{alg:admm}}
\label{subsec:interpretation}
\subfile{sections/distributed/interpretation}
%
% -------------------------------------
%
\section{Numerical Simulations}
\label{sec:numerical}
\subfile{sections/numerical}
%
% -------------------------------------

\section{Conclusions}
\label{sec:conclusions}
In this work, we studied the problem of fault detection, identification and reconstruction (FDIR) of multi-agent systems, which presents distinctive challenges due to the combinatorial complexity of the conventional FDIR algorithms.
The proposed approach for multi-agent FDIR borrows ideas from compressive sensing to reformulate the FDIR problem as one of reconstructing a block-sparse error vector. The search-space of the error-reconstruction problem was characterized and its connections to the exiting literature on rigidity theory were explored, which explains why the proposed approach is able to reconstruct the error vector without requiring the identities of the anchors (i.e., the agents that do not have faults) to be known \textit{a priori}. Thereafter, we developed a distributed algorithm that solves the error-reconstruction problem at each agent. Numerical simulations were used to validate the effectiveness of the proposed distributed multi-agent FDIR algorithm.

One possibility for future work is to consider each agent to evolve according to a dynamical model, similar to \cite{kazumune2020distributed}. Another future research direction is to cast the multi-agent FDIR problem in the Bayesian setting.
%
\appendix[Proof of Lemma \ref{lem:proximal}]
\subfile{sections/lemma}

\bibliographystyle{ieeetr}
\bibliography{IEEEabrv,refs}
% \begin{IEEEbiography} {Author} insert biographical sketch here.
% \end{IEEEbiography}
% \begin{IEEEbiography}[{\includegraphics[width=1in,height=1.25in,clip,keepaspectratio]{images/Shiraz.png}}]
\begin{IEEEbiographynophoto}
{Shiraz Khan} received his MS degree in Aeronautics and Astronautics from Purdue University in 2020, and his Bachelor's degree in Aerospace Engineering from IIT Madras in 2018. He is currently pursuing a Ph.D. degree at Purdue University, as part of which he investigates the problems of robustness, resilience, and scalability of state estimation. The potential applications of his Ph.D. research include decentralized and cyberattack-resilient state estimation in single and multi-agent scenarios.
One of his current research interests is in exploring and leveraging the underlying structure (such as sparsity, geometry, and/or symmetry) of various control theory and signal processing applications.
\end{IEEEbiographynophoto}
% \end{IEEEbiography}

% \begin{IEEEbiography}[{\includegraphics[width=1in,height=1.25in,clip,keepaspectratio]{images/Hwang.pdf}}]
\begin{IEEEbiographynophoto}
{Inseok Hwang} received the Ph.D. degree in Aeronautics and Astronautics from Stanford University, and is currently a professor in the School of Aeronautics and Astronautics and a university faculty scholar at Purdue University. His research interests lie in high assurance autonomy (which guarantees safety, security and performance) for Cyber-Physical Systems (CPS) based on the hybrid systems approach and its applications to safety critical systems such as aircraft/spacecraft/unmanned aerial systems (UAS), air traffic management, and multi-agent systems, which are complex systems with interacting cyber (or logical or human) elements and physical components. For his research, he leads the Flight Dynamics and Control/Hybrid Systems Laboratory at Purdue University. He received various awards, including the National Science Foundation (NSF) CAREER award in 2008, recognition as one of the nation’s brightest young engineers by the National Academy of Engineering (NAE) in 2008, the AIAA Special Service Citation in 2010, the University Faculty Scholar by Purdue university in recognition of outstanding scholarship in 2017, Outstanding Graduate Faculty Mentor award 2018, and the C.T. Sun Award in recognition of excellence in research in 2019. He is an Associate Fellow of AIAA and a member of IEEE Control Systems Society and Aerospace and Electronic Systems Society. He is currently an associate editor of the IEEE Transactions on Aerospace and Electronic Systems.
\end{IEEEbiographynophoto}
% \end{IEEEbiography}

\end{document}
