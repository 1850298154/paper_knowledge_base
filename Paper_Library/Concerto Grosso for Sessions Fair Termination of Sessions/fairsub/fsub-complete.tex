\beginbass
%
\Cref{thm:fsub_sound} alone suffices to justify the adoption of $\subt$ as
fair subtyping relation, but we are interested in understanding to which extent
$\subt$ covers $\ssubt$. In this respect, it is quite easy to see that there
exist session types that are related by $\ssubt$ but not by $\subt$. For
example, consider $S = \role\Out\Tag[a].S$ and $T = \role\In\Tag[b].T$ and
observe that these two session types describe completely different protocols
(the output of infinitely many $\Tag[a]$'s in the case of $S$ and the input of
infinitely many $\Tag[b]$'s in the case of $T$). In particular, we have $S
\not\subt T$ and $T \not\subt S$ but also $S \ssubt T$ and $T \ssubt S$. That
is, $S$ and $T$ are \emph{unrelated} according to $\subt$ but they are
\emph{equivalent} according to $\ssubt$. This equivalence is justified by the
fact that there exists no coherent session map in which $S$ and $T$ could play
any role, because none of them can ever terminate.

This discussion hints at the possibility that, if we restrict the attention to
those session types that \emph{can} terminate, which are the interesting ones as
far as this work is concerned, then we can establish a tighter correspondence
between $\subt$ and $\ssubt$. We call such session types \emph{bounded}, because
they describe protocols for which termination is always within reach.

\begin{definition}[Bounded session type]
	\label{def:bounded_type}
	We say that a session type is \emph{bounded} if all of its subtrees contain a
	$\End$ leaf.
\end{definition}

Note that a \emph{finite} session type is always bounded but not every bounded
session type is finite. If we consider the reduction system in which states are
session types and we have $S \red T$ if $T$ is an immediate subtree of $S$, then
$S$ is bounded if and only if $S$ is fairly terminating.
%
Now, for the family of bounded session types we can prove a \emph{relative
completeness} result for $\subt$ with respect to $\ssubt$.

\begin{theorem}[Relative completeness]
	\label{thm:fsub_complete}
 	$S$ bounded, $S \ssubt T$ imply $S \subt T$.
\end{theorem}

The proof of \Cref{thm:fsub_complete} is done by contradiction. We show
that, for any bounded $S$, if $S\subt T$ does not hold then we can build a
session map $M$ called \emph{discriminator} such that $M\parop \Map\rolep S$ is
coherent and $M\parop \Map\rolep T$ is not, which contraddicts the hypothesis $S
\ssubt T$.
%
The boundedness of $S$ is necessary to make sure that it is always possible to
find a session map $N$ such that $N\parop \Map\rolep S$ is coherent.

For this proof we need some auxiliary notions and notation.
%
First of all, we consider \emph{unfair subtyping} $\usubt$ from 
\Cref{fig:usub} and we fix the invariance of the type of the
channel being sent by substituting rules \refrule{us-channel-in}
and \refrule{us-channel-out} with
\[
	\inferrule[us-channel]{
      S \usubt T
    }{
      \role\Pol U.S \usubt \role\Pol U.T
    }
\]
It is straightforward to see that ${\subt} \subseteq {\usubt}$.
Then, we introduce some convenient notation for building session maps. To do
this, we assume the existence of an arbitrary total order $<$ on the set of
roles.
%
Now, given a finite set of roles $\set{\role_1,\dots,\role_n}$ where $\role_1 <
\cdots < \role_n$, we write $\set{\role_1,\dots,\role_n}\Out\Tag.S$ for the
session type $\role_1\Out\Tag\cdots\role_n\Out\Tag.S$.
%
Given a finite family $\set{M_i}_{i\in I}$ of session maps all having the same
domain $\set\roleq \subseteq D \subseteq \RoleSet\setminus\set\rolep$, we write
$\Map\roleq{\Tags{\rolep\Pol\Tag_i.M_i}}$ for the session map $M$ having domain
$D$ and such that
\[
  M(\roler) \eqdef
  \begin{cases}
    \Tags\rolep\Pol\Tag_i.  D\setminus\set{\roleq}\Out\Tag_i.
    M_i(\roleq) & \text{if $\roler = \roleq$}
  \\
  \Tags\roleq\In\Tag_i.M_i(\roler) & \text{if
    $\roler \neq \roleq$}
  \end{cases}
\]
for every $\roler \in D$.
%
As suggested by the notation, this session map realizes a conversation in which
$\roleq$ first interacts with $\rolep$ by exchanging a tag $\Tag_i$ and then it
informs all the other participants about the tag that has been exchanged. This
session map has the property
\[
  M \xlred{\roleq:\rolep\Pol\Tag_k}\wred M_i
\]
for every $k\in I$.

Similarly, given a session map $N$ with domain $D \supseteq\set\roleq$, we write
$\Map{\roleq}{\rolep\Pol{U}.N}$ for the session map $M$ with domain $D$ such
that
\[
  M(\roler) \eqdef
  \begin{cases}
    \rolep\Pol{U}.N(\roleq) & \text{if $\roler = \roleq$}
    \\
    N(\roler) & \text{if $\roler \in D\setminus\set{\rolep,\roleq}$}
  \end{cases}
\]
for every $\roler \in D$. Note that $M$ has the property
\[
  M \xlred{\roleq:\rolep\Pol{U}} N
\]

The first key step is showing that $S \ssubt T$ implies $S \usubt T$ when $S$ is
a bounded session type. That is, unfair subtyping is a \emph{necessary
condition} for fair subtyping to hold.

\begin{lemma}
  \label{lem:usub_complete}
  If\/ $S$ is bounded and $S \ssubt T$ then $S \usubt T$.
\end{lemma}
\begin{proof}
  Using the coinduction principle (\Cref{prop:coindp}) 
  it suffices to show that each judgment in the set
  \[
    \srel \eqdef \set{ S \usubt T \mid \text{$S$ is bounded and $S \ssubt T$} }
  \]
  is derivable by the rules in \Cref{fig:usub} from premises that
  satisfy the same property. Let $S \usubt T \in \srel$. Then $S$ is
  bounded and $S \ssubt T$.
  %
  We reason by cases on the shape of $S$.

  \proofcase{Case $S = \End$}
  %
  Consider $M \eqdef \Map{\roleq}{\End[\co\Pol]}$ and observe that
  $M \parop \Map{\rolep} S$ is coherent. Then $M \parop \Map{\rolep} T$ is
  coherent as well, which implies $T = \End$.
  %
  We conclude by observing that $\End \usubt \End$ is derivable with
  \refrule{us-end}.

  \proofcase{Case $S = \Tags\roleq\Out\Tag_i.S_i$}
  %
  Let $\set{M_i}_{i\in I}$ be a family of session maps such that
  $M_i \parop \Map{\rolep} S_i$ is coherent for every $i\in I$. Such
  family is guaranteed to exist from the hypothesis that $S$ is
  bounded.
  %
  Without loss of generality we may assume that the $M_i$ all have
  the same domain $D \supseteq \set\roleq$.
  %
  Let $M \eqdef \Map{\roleq}{\Tags\rolep\In\Tag_i.M_i}$ and observe that
  $M \parop \Map{\rolep}{S}$ is coherent by definition of $M$.  Then
  $M \parop \Map{\rolep}{T}$ is coherent as well.
  %
  We deduce that $T = \Tags[i\in J]\roleq\Out\Tag_i.T_i$ and
  $J \subseteq I$ and also that $M_i \parop \Map{\rolep}{T_i}$ is coherent
  for every $i\in J$. Hence $S_i \ssubt T_i$ for every $i\in J$,
  namely $S_i \usubt T_i \in \srel$ for every $i\in J$ by definition
  of $\srel$.
  %
  We conclude by observing that $S \usubt T$ is derivable by
  \refrule{us-tag-out}.

  \proofcase{Case $S = \Tags\roleq\In\Tag_i.S_i$}
  %
  Let $\set{M_i}_{i\in I}$ be a family of session maps such that
  $M_i \parop \Map{\rolep}{S_i}$ is coherent for every $i\in I$. Such
  family is guaranteed to exist from the hypothesis that $S$ is
  bounded. Without loss of generality we may assume that the $M_i$
  all have the same domain $D \supseteq \set\roleq$.
  %
  Let $M \eqdef \Map{\roleq}{\Tags\rolep\Out\Tag_i.M_i}$ and observe that
  $M \parop \Map{\rolep}{S}$ is coherent by definition of $M$.
  %
  We deduce that $T = \Tags[i\in J]\roleq\In\Tag_i.T_i$ and
  $I \subseteq J$ and also that $M_i \parop \Map{\rolep}{T_i}$ is coherent
  for every $i\in I$. Hence $S_i \ssubt T_i$ for every $i\in I$,
  namely $S_i \usubt T_i \in \srel$ for every $i\in I$ by definition
  of $\srel$.
  %
  We conclude by observing that $S \usubt T$ is derivable by
  \refrule{us-tag-in}.

  \proofcase{Case $S = \roleq\Pol{U}.S'$}
  %
  Let $N$ be a session map such that $N \parop \Map{\rolep}{S'}$ is
  coherent. Such $N$ is guaranteed to exist from the hypothesis that
  $S$ is bounded.
  %
  Let $M \eqdef \Map{\roleq}{\rolep\co\Pol{U}.N}$ and observe that
  $M \parop \Map{\rolep}{S}$ is coherent by definition of $M$.
  %
  We deduce that $T = \roleq\Pol{U}.T'$ and also that
  $N \parop \Map{\rolep}{T'}$ is coherent. Hence $S' \ssubt T'$, namely
  $S' \usubt T' \in \srel$ by definition of $\srel$.
  %
  We conclude by observing that $S \usubt T$ is derivable by
  \refrule{us-channel}.
\end{proof}

\newcommand{\dualof}[3][D]{\mathsf{dual}_{#1}(\Map{#2}#3)}

Next we show that every bounded session type may be part of a coherent session
map. This result is somewhat related to the notion of \emph{duality} in binary
session type theories \citep{Honda93,HondaVasconcelosKubo98}, showing that every
behavior can be completed by a matching -- dual -- one.

\begin{definition}[Duality]
  Let $\targets{\cdot}$ be the function that yields the set of roles occurring
  in a session type, let $S$ be a bounded session type and $D$ be a non-empty
  set of roles that includes $\targets{S}$ but not $\rolep$.
  %
  Let $\dualof\rolep{S}$ be the session map corecursively defined by
  the following equations:
  \[
    \begin{array}{r@{~}c@{~}ll}
      \dualof\rolep{\End[\In]} & = & \set{\Map\roleq \End[\Out]}_{\roleq\in D}
      \\
      \dualof\rolep{\End[\Out]} & = & \Map{\min D} \End[\In] \parop \set{\Map\roleq \End[\Out]}_{\roleq\in D \setminus \set{\min D}}
      \\
      \dualof\rolep{\Tags\roleq\Pol\Tag_i.S_i} & = &
      \Map\roleq \Tags\rolep\co\Pol\Tag_i.\dualof\rolep{S_i}
      \\
      \dualof\rolep{\roleq\Pol{U}.S} & = &
      \Map\roleq \rolep\co\Pol{U}.\dualof\rolep{S}
    \end{array}
  \]
\end{definition}

\begin{lemma}[Duality]
	\label{lem:duality}
	$\dualof\rolep{S} \parop \Map\rolep S$ is coherent.
\end{lemma}
\begin{proof}
  Follows from the definition of $\dualof\rolep{S}$.
\end{proof}

Now we provide an algorithmic way of computing the ``difference'' between two
session types related by unfair subtyping.

\begin{definition}[Subtyping weight]
	\label{def:fsub_wg}
  Under the hypothesis $S\usubt T$, let $\rk(S,T)\in\N\union\set\infty$ 
  be the least solution of the system of equations below:
\[
  \begin{array}{@{}r@{~}c@{~}ll@{}}
    \rk(\End, \End) & = & 0
    \\
    \rk(\Tags\role\In\Tag_i.S_i, \Tags[i\in J]\role\In\Tag_i.T_i)
    & = & \max_{i\in I} \rk(S_i, T_i)
    & I \subseteq J
    \\
    \rk(\Tags\role\Out\Tag_i.S_i, \Tags[i\in J]\role\Out\Tag_i.T_i)
    & = & 1 + \min_{i\in J} \rk(S_i, T_i)
    & J \subsetneq I
    \\
    \rk(\Tags\role\Out\Tag_i.S_i, \Tags\role\Out\Tag_i.T_i)
    & = & \min\set{ \\
    	& & ~~ 1 + \min_{i\in I} \rk(S_i,T_i), \\
    	& & ~~ \max_{i\in I} \rk(S_i,T_i)}
    \\
    \rk(\role\Pol{U}.S', \role\Pol{U}.T') & = & \rk(S', T')
  \end{array}
\]
\end{definition}

To see that $\rk(S,T)$ is well defined, observe that the system of equations
defining $\rk(S,T)$ under the hypothesis $S \usubt T$ contains finitely many
equations, say $n$, by regularity of $S$ and $T$. The system is representable as
a monotone endofunction $F$ on the complete lattice $(\Nat\union\set\infty)^n$.
Thus, $F$ has a least solution of which $\rk(S,T)$ is a component.
We call two session types $S$ and $T$ divergent if they are related by unfair
subtyping and have infinite rank.

\begin{definition}[Divergence]
	\label{def:diverge}
	\brk
 	 We write $S \diverge T$ if $S \usubt T$ and $\rk(S, T) = \infty$.
\end{definition}

\begin{lemma}
	\label{lem:diverge}
  	If $S \diverge T$ then the derivation of $S \usubt T$ contains at least one application of \refrule{u-tag-out} with $J \subsetneq I$ and one of the following holds:
  	\begin{enumerate}
  	\item $S = \Tags \role\In\Tag_i.S_i$ and
    	$T = \Tags[i \in J] \role\In\Tag_i.T_i$ with $I \subseteq J$ and
    	$S_k \diverge T_k$ for some $k\in I$, or
  	\item $S = \Tags \role\Out\Tag_i.S_i$ and
    	$T = \Tags[i\in J] \role\Out\Tag_j.T_j$ with $J \subseteq I$ and
    	$S_i \diverge T_i$ for every $i\in J$, or
  	\item $S = \role\Pol{U}.S'$ and $T = \role\Pol{U}.T'$ and
    	$S' \diverge T'$.
  	\end{enumerate}
\end{lemma}
\begin{proof}
  If the derivation of $S \usubt T$ contained no application of
  \refrule{u-tag-out} with $J \subsetneq I$ we would have
  $\rk(S, T) = 0$. Now we reason by cases on the last rule used to
  derive $S \usubt T$.

  \proofrule{us-end}
  %
  Then $S = T = \End$.  This case is impossible because
  $\rk(\End, \End) = 0$ by definition.

  \proofrule{us-tag-in}
  %
  Then $S = \Tags\role\In\Tag_i.S_i$ and
  $T = \Tags[i\in J] \role\In\Tag_i.T_i$ with $I \subseteq J$ and
  $S_i \usubt T_i$ for every $i\in I$ and
  $\infty = \rk(S, T) = \max_{i\in I} \rk(S_i, T_i)$.
  %
  That is, $\rk(S_k,T_k) = \infty$ for some $k \in I$, hence we
  conclude $S_k \diverge T_k$.

  \proofrule{us-tag-out}
  %
  Then $S = \Tags\role\Out\Tag_i.S_i$ and
  $T = \Tags[i\in J]\role\Out\Tag_i.T_i$ with $J \subseteq I$ and
  $S_i \usubt T_i$ for every $i\in J$.
  %
  We distinguish two sub-cases.
  %
  If $J \subsetneq I$ then
  $\infty = \rk(S, T) = 1 + \min_{i\in J} \rk(S_i, T_i)$, that is
  $\rk(S_i, T_i) = \infty$ for every $i\in J$.
  %
  If $J = I$ then
  $\infty = \rk(S, T) = \min\set{1 + \min_{i\in I} \rk(S_i, T_i),
    \max_{i\in I} \rk(S_i, T_i)} \leq 1 + \min_{i\in I} \rk(S_i,
  T_i)$ and we have $\rk(S_i, T_i) = \infty$ for every $i\in I$.
  %
  Therefore, in both cases, we conclude $S_i \diverge T_i$ for every
  $i \in J$.

  \proofrule{us-channel}
  %
  Then $S = \role\Pol{U}.S'$ and $T = \role\Pol{U}.T'$ and
  $S' \usubt T'$ and $\infty = \rk(S, T) = \rk(S', T')$ hence we
  conclude $S' \diverge T'$.
\end{proof}

\newcommand{\discriminator}[3]{\mathsf{disc}(#1,#2,#3)}

Finally, the key aspect of the proof of \Cref{lem:divergence} is 
how we build the session map $M$ such that $M \parop \Map{\rolep}{S}$ 
is coherent while $M \parop \Map{\rolep}{T}$ is not.

\begin{definition}[Discriminator]
	\label{def:discriminator}
	Let $\discriminator\rolep{S}{T}$ be the session map
  	corecursively defined by the following equations:
  	%
  	\[
    	\begin{array}{r@{~}c@{~}l}
      	\discriminator\rolep{
        	\Tags\roleq\In\Tag_i.S_i
      	}{
        	\Tags[i\in J]\roleq\In\Tag_i.T_i
      	}
      	& = &
      	\Map\roleq \Tags[i\in I, S_i \diverge T_i]\rolep\Out\Tag_i.\discriminator\rolep{S_i}{T_i}
      	\\ & & \hfill \text{if $I \subseteq J$}
      	\\
      	\discriminator\rolep{
        	\Tags\roleq\Out\Tag_i.S_i
      	}{
        	\Tags[i\in J]\roleq\Out\Tag_i.T_i
      	}
      	& = &
      	\Map\roleq \Tags[i\in J]\roleq\In\Tag_i.\discriminator\rolep{S_i}{T_i} 
        \\
        & + & \Tags[i\in I\setminus J] \roleq\In\Tag_i.\dualof\rolep{S_i}
      	\\ & & \hfill \text{if $J \subseteq I$}
      	\\
      	\discriminator\rolep{
        	\roleq\Pol U.S 
      	}{
        	\roleq\Pol U.T
      	}
      	& = &
      	\Map\roleq \rolep\co\Pol U.\discriminator\rolep{S}{T}
    	\end{array}
  	\]
\end{definition}

\begin{lemma}
	\label{lem:divergence}
  If $S$ is bounded and $S \diverge T$ then $S \not\ssubt T$.
\end{lemma}
\begin{proof}
  %
  From the hypothesis that
  $S \diverge T$ and \Cref{lem:diverge} we deduce that the
  derivation of $S \usubt T$ contains at least one application of
  \refrule{us-tag-out} with $J \subsetneq I$.
  %
  Consider $\discriminator\rolep{S}{T}$ from \Cref{def:discriminator}.
  Note that $\discriminator\rolep{S}{T}$ always sends a subset of
  the labels accepted by $S$, it is willing to receive any label
  sent by $S$, and it can always terminate successfully when
  interacting with $S$. Note also that it terminates successfully
  only after receiving a label from $S$ that $T$ cannot sent.
  %
  Therefore, we have
  $\discriminator\rolep{S}{T} \parop \Map\rolep{S}$ coherent and
  $\discriminator\rolep{S}{T} \parop \Map\rolep{T}$ incoherent,
  which proves $S \not\ssubt T$.
\end{proof}

\begin{proof}[Proof of \Cref{thm:fsub_complete}]
  \newcommand{\foosubt}{\subt[@]}
  %
  Let $S \foosubt T$ if $S \usubt T$ and $\rk(U, T) < \infty$ for
  every judgment $U \usubt V$ in the derivation of $S \usubt T$.
  %
  From \Cref{lem:usub_complete,lem:divergence} we have that
  $S \ssubt T$ implies $S \foosubt T$. Indeed, if there is a
  judgment $U \usubt V$ in the derivation of $S \usubt T$ such that
  $\rk(U,V) = \infty$, then it is possible to build a session $M$
  such that $M \parop \Map\role{S}$ is coherent and
  $M \parop \Map\role{T}$ is not by induction on the minimum depth
  of the judgment $U \usubt V$ in the derivation using the
  hypothesis that $S$ is bounded and \Cref{lem:divergence}.

  Now, using the principle of coinduction, it suffices to show that
  each judgment in the set
  \[
    \srel \eqdef \set{ S\subt[\rk(S,T)] T \mid S \foosubt T}
  \]
  is derivable using one of the rules in \Cref{fig:fsub_multi} whose
  premises all belong to $\srel$.
\end{proof}
