\beginalto
%
Fair subtyping is a \emph{liveness}-preserving variant
of the subtyping proposed by \cite{GayHole05}.
In this section we present such relation as a purely
coinductive predicate. However, in \Cref{ssec:fsub_gis}
we will show an equivalent and more involved formulation
obtained by equipping the rules in \Cref{fig:usub} with
a \emph{corule} (see \Cref{sec:gis}). While the generalized 
inference system is intriguing since it points out the safety 
and liveness aspects of the predicate, the formulation we give in
this section significantly simplifies the correctness proofs and
it will be easily integrated in the type systems in \Cref{pt:type_systems}.
Since binary sessions are the simplest multiparty ones, we refer to the binary
case when it allows to simplify the presentation of some features of the property
(\eg examples, description of the rules) while we present all the results
in the more general multiparty one.

\begin{figure}[t]
	\framebox[\textwidth]{
  \begin{mathpar}
    \inferrule[fsb-end]{\mathstrut}{
      \End \subt[n] \End
    } \defrule[fsb-end]{}
    \and
    \inferrule[fsb-tag-in]{
      \forall i\in I: S_i \subt[n_i] T_i
      \\
      \forall i\in I: n_i \leq n
    }{
      \textstyle
      \Tags\In\Tag_i.S_i \subt[n] \Tags[i\in I \union J]\In\Tag_i.T_i
    } \defrule[fsb-tag-in]{}
    \\
    \inferrule[fsb-channel]{
      S \subt[n] T
    }{
      \Pol U.S \subt[n] \Pol U.T
    } \defrule[fsb-channel]{}
    \and
    \inferrule[fsb-tag-out-1]{
      \forall i\in I: S_i \subt[n_i] T_i
      \\
      \forall i\in I: n_i \leq n
    }{
      \textstyle
      \Tags[i\in I]\Out\Tag_i.S_i \subt[n] \Tags\Out\Tag_i.T_i
    } \defrule[fsb-tag-out-1]{}
    \and
    \inferrule[fsb-tag-out-2]{
      \forall i\in I: S_i \subt[n_i] T_i
      \\
      \exists i\in I: n_i < n
    }{
      \textstyle
      \Tags[i \in I \union J]\Out\Tag_i.S_i \subt[n] \Tags\Out\Tag_i.T_i
    } \defrule[fsb-tag-out-2]{}
  \end{mathpar}
  }
  \caption{Inference system for fair subtyping - Binary}
  \label{fig:fsub_bin}
\end{figure}

\begin{figure}[t]
	\framebox[\textwidth]{
  \begin{mathpar}
    \inferrule[fsm-end]{\mathstrut}{
      \End \subt[n] \End
    } \defrule[fsm-end]{}
    \and
    \inferrule[fsm-tag-in]{
      \forall i\in I: S_i \subt[n_i] T_i
      \\
      \forall i\in I: n_i \leq n
    }{
      \textstyle
      \Tags\rolep\In\Tag_i.S_i \subt[n] \Tags[i\in I \union J]\rolep\In\Tag_i.T_i
    } \defrule[fsm-tag-in]{}
    \and
    \inferrule[fsm-channel]{
      S \subt[n] T
    }{
      \role\Pol U.S \subt[n] \role\Pol U.T
    } \defrule[fsm-channel]{}
    \and
    \inferrule[fsm-tag-out-1]{
      \forall i\in I: S_i \subt[n_i] T_i
      \\
      \forall i\in I: n_i \leq n
    }{
      \textstyle
      \Tags[i\in I]\role\Out\Tag_i.S_i \subt[n] \Tags\role\Out\Tag_i.T_i
    } \defrule[fsm-tag-out-1]{}
    \and
    \inferrule[fsm-tag-out-2]{
      \forall i\in I: S_i \subt[n_i] T_i
      \\
      \exists i\in I: n_i < n
    }{
      \textstyle
      \Tags[i \in I \union J]\role\Out\Tag_i.S_i \subt[n] \Tags\role\Out\Tag_i.T_i
    } \defrule[fsm-tag-out-2]{}
  \end{mathpar}
  }
  \caption{Inference system for fair subtyping - Multiparty}
  \label{fig:fsub_multi}
\end{figure}


Fair subtyping for binary session types is defined as the
relation $\subt[n]$ coinductively defined by the inference system
in \Cref{fig:fsub_bin}, where $n$ ranges over natural numbers. The characterization
of fair subtyping that we consider is the relation ${\subt} \eqdef
\bigcup_{n\in\Nat} {\subt[n]}$.
Fair subtyping for multiparty session types is analogously defined
(see \Cref{fig:fsub_multi}).
%
The rules for deriving $S \subt[n] T$ are quite similar to those of the standard
subtyping relation for session types \citep{GayHole05}: \refrule{fsb-end} states
reflexivity of subtyping on terminated session types; \refrule{fsb-channel}
relates higher-order session types with the same polarity and payload type;
\refrule{fsb-tag-in} is the usual covariant rule for the input of tags (the set
of tags in the larger session type includes those in the smaller one);
\refrule{fsb-tag-out-2} is the usual contravariant rule for the output of tags
(the set of tags in the smaller session type includes those in the larger one).
Overall, these rules entail a ``simulation'' between the behaviors described by
$\S_b$ and $\S_b'$ whereby all inputs offered by $\S_b$ are also offered by $\S_b'$ and all
outputs performed by $\S_b'$ are also performed by $\S_b$.
%
The main differences between $\subt$ and the subtyping relation of \cite{GayHole05}
are the presence of an invariant rule for outputs
\refrule{fsb-tag-out-1} and the natural number $n$ annotating each subtyping
judgment $S \subt[n] T$. Intuitively, this number estimates how much $\S_b$ and $\S_b'$
differ in terms of performed outputs. In all rules but \refrule{fsb-tag-out-2},
the annotation in the conclusion of the rule is just an upper bound of the
annotations found in the premises. In \refrule{fsb-tag-out-2}, where the sets of
output tags in related session types may differ, the annotation $n$ is required
to be a \emph{strict} upper bound for at least one of the premises. That is,
there must be at least one premise in which the annotation strictly decreases,
while no restriction is imposed on the others. Intuitively, this ensures the
existence of a tag shared by the two related session types whose corresponding
continuations are slightly less different. So, the annotation $n$ provides an
upper bound to the number of applications of \refrule{fsb-tag-out-2} along any
path (\ie any sequence of actions) shared by $\S_b$ and $\S_b'$ that leads to
termination.  In the particular case when $n=0$, the rule \refrule{fsb-tag-out-2}
cannot be applied, so that $\S_b'$ may perform all the outputs also performed by
$\S_b$.

\begin{remark}[Invariant delegation]
	\label{rmk:fsub_invariant}
	As it can be noted from \refrule{fsb-channel} and \refrule{fsm-channel},
	we require that the type of the channel being exchanged in the supertype
	matches that in the subtype. We made such a decision
	since we found out that allowing co/contravariance of input/output of channels
  may break the liveness of the session. 
	We will give more details in \Cref{ssec:invariant_ch}.
	%
	\eor
\end{remark}

Fair subtyping allows us to reject the subtyping instance in \Cref{ex:unfair_sub}
that breaks the liveness of the involved session.

\begin{example}
	\label{ex:bsc_fair_sub}
  Consider the session types 
  \[
  \begin{array}{rclrcl}
  	S_b & = & \Out\tadd.S_b + \Out\tpay.\End[\Out]
  	\qquad &
  	\S_b' & = & \Out\tadd.\Out\tadd.\S_b' + \Out\tpay.\End[\Out]
  \end{array}
  \]
  from \Cref{ex:bin_bsc,ex:original_sub} 
  describing the \actor{buyer} purchasing an arbitrary number of items
  and the behavior of the \actor{buyer} always purchasing an even
  number of items, respectively. Consider also
  \[
  	\S_b^\infty = \Out\tadd.\S_b^\infty
  \]
  from \Cref{ex:unfair_sub}, 
  which describes the behavior of a \actor{buyer} attempting to $\tadd$ an 
  infinite number of items without ever $\tpay$ing the \actor{seller}.
  %
  We have $\S_b \subt \S_b'$ and $\S_b \not\subt \S_b^\infty$. Indeed, we can derive
  \[
    \begin{prooftree}
      \[
        \[
          \mathstrut\smash\vdots
          \justifies
          \S_b \subt[1] \S_b' 
        \]
        \justifies
        \S_b \subt[2] \Out\tadd.\S_b'
        \using\refrule{fsb-tag-out-2}
      \]
      \[
        \justifies
        \End[\Out] \subt[0] \End[\Out]
        \using\refrule{fsb-end}
      \]
      \justifies
      \S_b \subt[1] \S_b'
      \using\refrule{fsb-tag-out-2}
    \end{prooftree}
  \]
  %
  but there is no derivation for $\S_b \subt[n] \S_b^\infty$ no matter how large $n$ is
  chosen.
  %
  Note that there are infinitely many sequences of actions of $\S_b$ that cannot be
  performed by both $\S_b'$ and $\S_b^\infty$. In particular, $\S_b'$ cannot perform any sequence
  of actions consisting of an odd number of $\Tag[add]$ outputs followed by a
  $\Tag[pay]$ output, whereas $\S_b^\infty$ cannot perform any sequence of $\Tag[add]$
  outputs followed by a $\Tag[pay]$ output. Nonetheless, there is a path shared
  by $\S_b$ and $\S_b'$ that leads into a region of $\S_b$ and $\S_b'$ in which no more
  differences are detectable. The annotations in the derivation tree measures
  the distance of each judgment from such region. In the case of $\S_b$ and $\S_b^\infty$,
  there is no shared path that leads to a region where no differences are
  detectable.
  %
  \eoe
\end{example}

\begin{example}
  \label{ex:slot_fair_sub}
  Consider the session types 
	\[
	\begin{array}{rcl}
		S & = & \In\tplay.(\Out\twin.S + \Out\tlose.S) + \In\tquit.\End[\Out]
		\\
		T & = & \In\tplay.\Out\tlose.T + \In\tquit.\End[\Out]
	\end{array}
	\]  
  describing
  the behavior of two slot machines, an unbiased one in which the player may win
  at every play and a biased one in which the player never wins. If we try to
  build a derivation for $S \subt[n] T$ we obtain
  \[
    \begin{prooftree}
      \[
        \[
          \mathstrut\smash\vdots
          \justifies
          S \subt[n-1] T
        \]
        \justifies
        \Out\twin.S + \Out\tlose.S \subt[n] \Out\tlose.T
        \using\refrule{f-tag-out-1}
      \]
      \[
        \justifies
        \End[\Out] \subt[n] \End[\Out]
        \using\refrule{f-end}
      \]
      \justifies
      S \subt[n] T
      \using\refrule{f-tag-in}
    \end{prooftree}
  \]
  %
  which would contain an infinite branch with strictly decreasing annotations.
  Therefore, we have $S \not\subt T$.
  %
  In this case there exists a shared path leading into a region of $S$ and $T$
  in which no more differences are detectable between the two protocols, but
  this path starts from an input. The fact that $S$ is \emph{not} a fair subtype
  of $T$ has a semantic justification. Think of a player that deliberately
  insists on playing until it wins. This is possible when the player interacts
  with the unbiased slot machine $S$ but not with the biased one $T$.
  %
  \eoe
\end{example}

Now we can provide a semantic characterization of fair subtyping for binary and multiparty
session types as the relation preserving \emph{compliance} (\Cref{def:compatibility}) and
\emph{coherence} (\Cref{def:coherence}) of the involved session, respectively.
Since the notion of \emph{coherence} boils down to \emph{compatibility} when there 
are exactly two participants, we only state the property in the multiparty scenario. 

\begin{definition}[Semantic fair subtyping - Multiparty]
  \label{def:ssubt}
  We say that $\S$ is a \emph{fair subtype} of\/ $\T$, notation
  $\S \ssubt \T$, if\/ $M \parop \Map\role\S$ \emph{coherent} implies
  $M \parop \Map\role\T$ \emph{coherent} for every $M$ and $\role$.
\end{definition}

We conclude this section by stating and proving a fundamental property of
fair subtyping.

\begin{theorem}
	\label{thm:fsub_preorder}
  	$\subt$ is a preorder.
\end{theorem}

While reflexivity of $\subt$ is trivial to prove (since \refrule{fsm-tag-out-2} is
never necessary, it suffices to only consider judgments with a $0$ annotation),
transitivity is surprisingly complex. The challenging part of proving that from
$S \subt[m] U$ and $U \subt[n] T$ we can derive $S \subt[k] T$ is to come up
with a feasible annotation $k$. As it turns out, such $k$ depends not only on
$m$ and $n$, but also on annotations found in different regions of the
derivation trees that prove $S \subt[m] U$ and $U \subt[n] T$. In particular,
the ``difference'' of $S$ and $T$ is not simply the ``maximum difference'' or
``the sum of the differences'' of $S$ and $U$ and of $U$ and $T$.
%
More in detail, we first show that we can always find a derivation of $S\subt[m]
U$ where the rank annotations of all judgements occurring in it are below some
$h \geq m$; then, the judgement $S\subt[k] T$ is provable for $k = m + (1+h)n$. 

\begin{lemma}
  \label{lem:bounded_derivation}
  Let $S \subt[n]^m T$ if and only if $S \subt[n] T$ is the
  conclusion of a derivation in which every rank annotation is at
  most $m$. Then $S \subt[n] T$ if and only $S \subt[n]^m T$ for
  some $m$.
\end{lemma}
\begin{proof}
  \newcommand{\msubt}[1]{\subt[#1{\leq}]}
  %
  The ``if'' part is obvious. Concerning the ``only if'' part, it
  suffices to show that each judgment in the set
  \[
    \srel \eqdef \set{U \subt[m] V \mid U \subt[m] V \wedge \nexists
      n < m: U \subt[n] V}
  \]
  is derivable from premises that are also in $\srel$. This is
  enough to prove $S \subt[n]^m T$ from $S \subt[n] T$, because in
  $\srel$ there is at most one judgment $U \subt[n] V$ for each pair
  of session types $U$ and $V$ and, by regularity of $U$ and $V$,
  the derivation of $U \subt[n] V$ obtained using judgments in
  $\srel$ contains finitely many annotations, which must have a
  maximum.

  Suppose $U \subt[m] V \in \srel$. Then $U \subt[m] V$ is
  derivable. We reason by cases on the last rule applied to derive
  this judgment.

  \proofrule{fsm-end}
  %
  Then $U = V = \End$ and there is nothing left to prove since
  \refrule{fsm-end} has no premises.

  \proofrule{fsm-tag-in}
  %
  Then $U = \Tags\role\In\Tag_i.U_i$ and
  $V = \Tags[i\in J]\role\In\Tag_i.V_i$ and $I \subseteq J$ and
  $U_i \subt[n_i] V_i$ and $n_i \leq m$ for every $i\in I$.
  %
  By definition of $\srel$ we have that, for every $i\in I$, there
  exists $m_i \leq n_i$ such that $U \subt[m_i] V_i \in \srel$.
  %
  Then $U \subt[m] V$ is derivable by \refrule{fsm-tag-in} using
  premises in $\srel$.

  \proofrule{fsm-tag-out-1}
  %
  Analogous to the previous case.

  \proofrule{fsm-tag-out-2}
  %
  Then $U = \Tags\role\Out\Tag_i.U_i$ and
  $V = \Tags[i\in J]\role\Out\Tag_i.V_i$ and $J \subseteq I$ and
  $U_i \subt[n_i] V_i$ for every $i\in J$ and $n_k < m$ for some
  $k\in I$.
  %
  By definition of $\srel$ we have that, for every $i\in J$, there
  exists $m_i \leq n_i$ such that $U \subt[m_i] V_i \in \srel$.
  %
  In particular, $m_k \leq n_k < m$.
  %
  Then $U \subt[m] V$ is derivable by \refrule{fsm-tag-out-2} using
  premises in $\srel$.
\end{proof}

\begin{proof}[Proof of \Cref{thm:fsub_preorder}]
The proof that $\subt$ is reflexive is trivial, since
  $S \subt[n] S$ is derivable for every $n$. Concerning
  transitivity, by \cref{lem:bounded_derivation} it suffices to show
  that each judgment in the set
  \[
    \srel \eqdef \set{
      S \subt[n_1 + (1 + m)n_2] T \mid S \subt[n_1]^m U \wedge U \subt[n_2] T
    }
  \]
  is derivable using the rules in \Cref{fig:fsub_multi} from premises that
  are also in $\srel$.
  %
  Suppose $S \subt[n] T \in \srel$. Then there exist $U$, $n_1$, $m$
  and $n_2$ such that $S \subt[n_1]^m U$ and $U \subt[n_2] T$ and
  $n = n_1 + (1 + m)n_2$.
  %
  We reason by cases on the last rules applied to derive
  $S \subt[n_1] U$ and $U \subt[n_2] T$.

  \proofrule{fsm-end}
  %
  Then $S = U = T = \End$ hence $S \subt[n] T$ is derivable by \refrule{fsm-end}.

  \proofrule{fsm-tag-in}
  %
  Then $S = \Tags\role\In\Tag_i.S_i$ and $U = \Tags[i\in J]\role\In\Tag_i.U_i$
  and $T = \Tags[i\in K]\role\In\Tag_i.T_i$ and $I \subseteq J \subseteq K$ and
  $S_i \subt[n_{1i}]^m U_i$ and $n_{1i} \leq n_1$ for every $i\in I$ and $U_i
  \subt[n_{2i}] T_i$ and $n_{2i} \leq n_2$ for every $i\in J$.
  %
  By definition of $\srel$ we have that $S_i \subt[n_{1i} + (1 + m)n_{2i}] T_i
  \in \srel$ for every $i\in I$.
  %
  Observe that $n_{1i} + (1 + m)n_{2i} \leq n_1 + (1 + m)n_2 = n$ for every
  $i\in I$ hence $S \subt[n] T$ is derivable by \refrule{fsm-tag-in}.

  \proofrule{fsm-tag-out-1}
  %
  Then $S = \Tags\role\Out\Tag_i.S_i$ and
  $U = \Tags\role\Out\Tag_i.U_i$ and $T = \Tags\role\Out\Tag_i.T_i$
  and $S_i \subt[n_{1i}]^m U_i$ and $n_{1i} \leq n_1$ and
  $U_i \subt[n_{2i}] T_i$ and $n_{2i} \leq n_2$ for every $i\in I$.
  %
  By definition of $\srel$ we have that
  $S_i \subt[n_{1i} + (1 + m)n_{2i}] T_i \in \srel$ for every
  $i\in I$.
  %
  Observe that $n_{1i} + (1 + m)n_{2i} \leq n_1 + (1 + m)n_2 = n$
  for every $i\in I$ hence $S \subt[n] T$ is derivable by
  \refrule{fsm-tag-out-1}.

  \proofcase{Case \refrule{fsm-tag-out-1} and \refrule{fsm-tag-out-2}}
  %
  Then $S = \Tags\role\Out\Tag_i.S_i$ and
  $U = \Tags\role\Out\Tag_i.U_i$ and
  $T = \Tags[i\in J]\role\Out\Tag_i.T_i$ and $J \subseteq I$ and
  $S_i \subt[n_{1i}]^m U_i$ and $n_{1i} \leq n_1$ for every $i\in I$
  and $U_i \subt[n_{2i}] T_i$ for every $i\in J$ and $n_{2k} < n_2$
  for some $k\in J$.
  %
  By definition of $\srel$ we have that
  $S_i \subt[n_{1i} + (1 + m)n_{2i}] T_i \in \srel$ for every
  $i\in J$.
  %
  Observe that $n_{1k} + (1 + m)n_{2k} < n_1 + (1 + m)n_2 = n$ hence
  $S \subt[n] T$ is derivable by \refrule{f-tag-out-2}.

  \proofcase{Case \refrule{fsm-tag-out-2} and \refrule{fsm-tag-out-1}}
  %
  Then $S = \Tags\role\Out\Tag_i.S_i$ and
  $U = \Tags[i\in J]\role\Out\Tag_i.U_i$ and
  $T = \Tags[i\in J]\role\Out\Tag_i.T_i$ and $J \subseteq I$ and
  $S_i \subt[n_{1i}]^m U_i$ for every $i\in J$ and $n_{1k} < n_1$
  for some $k\in J$ and $U_i \subt[n_{2i}] T_i$ for every $i\in J$
  and $n_{2i} \leq n_2$ for every $i\in J$.
  %
  By definition of $\srel$ we have that
  $S_i \subt[n_{1i} + (1 + m)n_{2i}] T_i \in \srel$ for every
  $i\in J$.
  %
  Observe that $n_{1k} + (1 + m)n_{2k} < n_1 + (1 + m)n_2 = n$ hence
  $S \subt[n] T$ is derivable by \refrule{fsm-tag-out-2}.

  \proofrule{fsm-tag-out-2}
  %
  Then $S = \Tags\role\Out\Tag_i.S_i$,
  $U = \Tags[i\in J]\role\Out\Tag_i.U_i$ and
  $T = \Tags[i\in K]\role\Out\Tag_i.T_i$ and
  $K \subseteq J \subseteq I$ and $S_i \subt[n_{1i}]^m U_i$ for
  every $i\in J$ and $n_{1j} < n_1$ for some $j\in J$ and
  $U_i \subt[n_{2i}] T_i$ for every $i\in K$ and $n_{2k} < n_2$ for
  some $k\in K$.
  %
  By definition of $\srel$ we have that
  $S_i \subt[n_{1i} + (1 + m)n_{2i}] T_i \in \srel$ for every
  $i\in K$.
  %
  Observe that
  \[
    \begin{array}{rcll}
      n_{1k} + (1 + m)n_{2k} & \leq & m + (1 + m)n_{2k} & \text{since $n_{1k} \leq m$}
      \\
      & < & 1 + m + (1 + m)n_{2k}
      \\
      & = & (1 + m)(1 + n_{2k})
      \\
      & \leq & (1 + m)n_2 & \text{since $n_{2k} < n_2$}
      \\
      & < & n_1 + (1 + m)n_2 & \text{since $n_{1j} < n_1$}
    \end{array}
  \]
  hence $S \subt[n] T$ is derivable by \refrule{fsm-tag-out-2}.
\end{proof}