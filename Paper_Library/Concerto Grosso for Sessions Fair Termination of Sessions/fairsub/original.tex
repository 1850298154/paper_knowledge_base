\beginalto
%
\begin{figure}[t]
  \framebox[\textwidth]{
    \begin{mathpar}
        \inferrule[us-end]{\mathstrut}{
          \End \usubt \End
        } \defrule[us-end]{}
        \and
        \inferrule[us-channel-in]{
          \S \usubt \T \\ \U \usubt \V
        }{
          \In\U.\S \usubt \In\V.\T
        } \defrule[us-channel-in]{}
        \and
        \inferrule[us-channel-out]{
          \S \usubt \T \\ \V \usubt \U
        }{
          \Out\U.\S \usubt \Out\V.\T
        } \defrule[us-channel-out]{}
        \and
        \inferrule[us-tag-in]{
           \forall i\in I: \S_i \usubt \T_i
        }{
        	\textstyle
          \Tags\In\Tag_i.S_i \usubt 
          \Tags[i \in I \union J]\In\Tag_i.T_i
        } \defrule[us-tag-in]{}
        \and
        \inferrule[us-tag-out]{
           \forall i\in I: \S_i \usubt \T_i
        }{
        	\textstyle
          \Tags[i \in I \union J]\Out\Tag_i.S_i \usubt 
          \Tags[i \in I]\Out\Tag_i.T_i
        } \defrule[us-tag-out]{}
    \end{mathpar}
  }
	\caption{Rules for subtyping \citep{GayHole05}}
	\label{fig:usub}
\end{figure}
%
The original subtyping relation for session types has 
been introduced by \cite{GayHole05} and it is obtained
by coinductively interpreting the rules in \Cref{fig:usub}.
Notably, we only show the relation for binary session types 
since in the multiparty case it is defined in the same way and it
can be obtained by simply adapting the syntax.
%
The inference system in \Cref{fig:usub} derives judgments of
the form $\S \usubt \T$ meaning that $\S$ is a \emph{subtype}
of $\T$. Let us analyze the rules in details.

Rule \refrule{us-end} is used to relate terminated sessions.
%
\refrule{us-channel-in} and \refrule{us-channel-out} relate 
input and output of channels, respectively. Note that they
show a different behavior. While the former is \emph{covariant}
with respect to both the channel being received and the continuation,
the second one is \emph{contravariant} in the exchanged channel.
%
Rules \refrule{us-tag-in} and \refrule{us-tag-out} relate input
and output of message tags, respectively. Similarly to the 
exchange of a channel, the former is covariant while the second
contravariant in the set of messages being exchanged.

The rules are consistent with the informal example that we gave
at the beginning of \Cref{ch:fs} by referring to \Cref{ex:bsc}.
Indeed, the \actor{seller} in could handle a $\tsearch$ message in
addition to $\tadd$ and $\tpay$ while the \actor{buyer} 
could always purchase an even/odd number of items.

\begin{example}
	\label{ex:original_sub}
	Consider the session types $\S_b$ and $\S_s$ from 
	\Cref{ex:bin_bsc} describing the communication protocol between
	the \actor{buyer} and the \actor{seller}.
	We can derive
	\[
	\begin{array}{lcl}
		\S_b = \Out\tadd.\S_b \choice \Out\tpay.\End[\Out]
				 & \usubt & 
				 \S_b' = \Out\tadd\Out\tadd.\S'_b \choice \Out\tpay.\End[\Out]
		\\
		\S_s = \In\tadd.\S_s \choice \In\tpay.\End[\In]
				 & \usubt & 
				 \S_s' = \In\tadd.\S_s' \choice \In\tpay.\End[\In]
				 						\choice \In\tsearch.\S_s'
	\end{array}
	\]
	We can focus on the infinite derivation trees.
	\[
	\begin{prooftree}
		\[
			\[
				\vdots
				\justifies
				\S_b \usubt \S_b'
			\]
			\justifies
			\Out\tadd.\S_b \choice \Out\tpay.\End[\Out] \usubt
			\Out\tadd.\S_b'
			\using\refrule{us-tag-out}
		\]
		\[
			\justifies
			\End[\Out] \usubt \End[\Out]
			\using\refrule{us-end}
		\]
		\justifies
		\Out\tadd.\S_b \choice \Out\tpay.\End[\Out] \usubt
		\Out\tadd\Out\tadd.\S'_b \choice \Out\tpay.\End[\Out]
		\using\refrule{us-tag-out}
	\end{prooftree}
	\]\[
	\begin{prooftree}
		\[
			\vdots
			\justifies
			\S_s \usubt \S_s'
		\]
		\[
			\justifies
			\End[\In] \usubt \End[\In]
			\using\refrule{us-end}
		\]
		\justifies
		\In\tadd.\S_s \choice \In\tpay.\End[\In] \usubt
		\In\tadd.\S_s' \choice \In\tpay.\End[\In]
				 						\choice \In\tsearch.\S_s'
		\using\refrule{us-tag-in}
	\end{prooftree}
	\]
	
	As previously noted, the same judgments can be derived
	for the multiparty session types in \Cref{ex:bsc_ty_multi}.
	%
	\eoe
\end{example}

The subtyping relation we presented induces a substitution
principle.

\begin{proposition}[\emph{Safe} substitution principle]
	\label{prop:safe_sub}
	If $\S \usubt \T$, then a process that uses an endpoint according
	to $\S$ can be \emph{safely} substituted with a process that
	uses the endpoint according to $\T$.
\end{proposition}

Note that we highlight the word \emph{safe} as we want to
point out that the subtyping relation under analysis is
studied to preserve the safety of the communication.
We will give more details in \Cref{ssec:unfair_sub}.

\begin{remark}
	The reader might be confused by the formulation of \Cref{prop:safe_sub}
	as it seems to treat the substitution in the wrong direction 
	(left-to-right). However, it is important to think about the
	subject of the principle, \ie processes in \Cref{prop:safe_sub}. 
	Indeed, the same principle
	can be formulated in a right-to-left fashion as in \cite{LiskovWing94} 
	by taking into account the endpoints and not the processes. 
	The two formulations turn out to be equivalent (see \cite{Gay16} for 
	more details).
	%
	\eor
\end{remark}