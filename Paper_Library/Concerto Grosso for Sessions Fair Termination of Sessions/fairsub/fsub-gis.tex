\beginbass
%
The purely \emph{coinductive} characterization of fair subtyping 
presented in \Cref{fig:fsub_bin,fig:fsub_multi} has been first used in \cite{CicconeDP22}.
Previous works 
\citep{Padovani13,Padovani16,CicconePadovani21,CicconePadovani22}
relied on a different characterization based on a generalized
inference system (see \Cref{sec:gis}) that consists of the same rules
of the original subtyping relation (\Cref{fig:usub}) equipped with
a corule. The generalized inference system for binary session types
is presented in \Cref{fig:fsub_gis}. Note the invariance in the type
of the channel being exchanged according to \Cref{rmk:fsub_invariant}.
In addition to $\subt$, we write $\isubt$ for the
relation defined by the \emph{inductive} interpretation of the same inference
system and $\csubt$ for the relation defined by the \emph{coinductive}
interpretation of the inference system in \cref{fig:fsub_gis} 
excluding the corule \refrule{fs-converge}. We introduce the
notions of \emph{paths} and \emph{residual} of a session type.

\begin{figure}[t]
  \framebox[\textwidth]{
    \begin{mathpar}
        \inferrule{\mathstrut}{
          \End \usubt \End
        }
        \and
        \inferrule{
          \S \usubt \T
        }{
          \Out\U.\S \usubt \Out\U.\T
        }
        \\
    		\inferrule{
           \forall i\in I: \S_i \usubt \T_i
        }{
        	\textstyle
          \Tags\In\Tag_i.S_i \usubt 
          \Tags[i \in I \union J]\In\Tag_i.T_i
        }
        \and
        \inferrule{
           \forall i\in I: \S_i \usubt \T_i
        }{
        	\textstyle
          \Tags[i \in I \union J]\Out\Tag_i.S_i \usubt 
          \Tags[i \in I]\Out\Tag_i.T_i
        }
        \and
        \infercorule[fs-converge]{
          \forall\actionsA\in\paths\S\setminus\paths\T:
          \exists\actionsB \prefix \actionsA, \Tag:
          S(\actionsB\Out\Tag) \subt T(\actionsB\Out\Tag)
        }{
          S \subt T
        } \defrule[fs-converge]{}
    \end{mathpar}
  }
  \caption{
    Generalized inference system for fair subtyping}
  \label{fig:fsub_gis}
\end{figure}

\begin{definition}[Paths of a session type]
	\label{def:path}
  We say that $\actions$ is a \emph{path} of $S$ if $S \wlred\actions$. We write
  $\paths\S$ for the (prefix-closed) set of paths of $S$, that is $\paths\S
  \eqdef \set{\actions \mid S \wlred\actions }$.
\end{definition}

\begin{definition}[Residual of a session type]
  \label{def:residual}
  The \emph{residual} of a session type $S$ with respect to a path $\actions \in
  \paths\S$, denoted by $S(\actions)$, is the unique session type $T$ such that
  $S \wlred\actions T$.
\end{definition}

The subtle difference between fair and unfair subtyping is due to the corule
\refrule{fs-converge}. Since this corule is somewhat obscure, 
we explain it gradually starting with the following observations:
\begin{enumerate}
\item Recall from \Cref{sec:gis} that $S \subt T$ implies $S \csubt T$ and $S
  \isubt T$. Hence, $\subt$ is a refinement of $\csubt$ such that, for each pair
  of related session types $S$ and $T$, there exists a \emph{finite-depth}
  derivation tree for the judgment $S \subt T$ using the rules and possibly the
  corule \refrule{fs-converge}.
\item When $S \csubt T$ holds, it is not possible to establish a general
  correlation between $\paths\S$ and $\paths\T$. Indeed, \refrule{us-tag-in}
  entails that some paths of $T$ may not be present in $S$ and
  \refrule{us-tag-out} entails that some paths of $S$ may not be present in $T$.
\item The judgment $S \subt T$ is trivially derivable using \refrule{fs-converge}
  if the path inclusion relation $\paths\S \subseteq \paths\T$ holds. Since
  \refrule{us-tag-out} is the only rule that allows $T$ to have fewer paths
  than $S$, we deduce that \refrule{fs-converge} limits (but does not always
  forbid) applications of \refrule{us-tag-out}.
\item In general \refrule{fs-converge} requires that, whenever a path $\actions$
  of $S$ is no longer present in $T$, it must be possible to find a prefix
  $\actionsB$ of $\actions$ and an output $\Out\Tag$ shared by both $S$ and $T$
  such that $S(\actionsB\Out\Tag)$ and $T(\actionsB\Out\Tag)$ are one step closer to
  the region of $S$ and $T$ where path inclusion holds.
\end{enumerate}

The reason why path inclusion plays such an important role in the definition of
fair subtyping is that a process using a channel of type $T$ keeps using it
according to $T$ even if it is replaced by another channel of type $S \subt T$,
without even realizing that the replacement has taken place. After all, this is
what the ``safe substitution principle'' (\Cref{prop:safe_sub}) is based on. 
As a consequence, none of
the paths in $S$ that have disappeared in $T$ will be offered to the process at
the other end of the session. If there are ``too few'' paths in $T$ compared
to $S$, then the replacement might compromise the termination of the process at
the other end of the session, should it crucially rely on those paths to
terminate.
When $S \subt T$ (and therefore $S \isubt T$) holds, the corule
\refrule{fs-converge} makes sure that the process using the channel of type $S$ 
believing
that it has type $T$ is always at \emph{finite distance} from the region where
path inclusion between (some subtrees of) $S$ and (the corresponding subtrees
of) $T$ holds. Moreover, this region is always reachable by means of
\emph{output actions} (those $\Out\l$ mentioned in \refrule{fs-converge}) which
are performed actively by the process using the channel. In other words, the process
using such channel is always able, in a finite amount of time and relying on choices and
actions it can perform autonomously, to steer the interaction towards a region
of the protocol where path inclusion holds, hence where a common path to session
termination is guaranteed to exist.

To conclude, the generalized inference system in \Cref{fig:fsub_gis}
has the advantage of highlighting the \emph{liveness}-preserving 
feature of fair subtyping by using \refrule{fs-converge}. However,
the purely coinductive characterization that we gave in \Cref{sec:fair_sub}
allowed us to to provide a direct proof of \Cref{thm:fsub_preorder}.
Indeed, in previous works
\citep{Padovani13,Padovani16,CicconePadovani21,CicconePadovani22},
transitivity has been established indirectly by relating the 
generalized inference system of
fair subtyping with its semantic definition (\Cref{def:ssubt}).

In \Cref{pt:agda} we will characterize properties of session types
mixing safety and liveness aspects. Starting from the definitions
of the safety (coinductive) parts, we will show how to use corules
to obtain the desired predicates. Hence, for what concerns
fair subtyping, we will refer to \Cref{fig:fsub_gis} to provide
a sound and complete Agda mechanization of such predicate (see \Cref{sec:agda_fs}).