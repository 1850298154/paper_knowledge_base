\beginbass
%
Now we prove the \emph{soundness} of $\subt$ with respect
to $\ssubt$ (\Cref{def:ssubt}). That is, we prove that $\subt$ 
is \emph{coherence}-preserving just like $\ssubt$ is.
The proof of this result relies on a key property of $\subt$ not enjoyed by the
usual subtyping relation on session types \citep{GayHole05}: when $S\subt T$ and
$M \parop \Map\rolep S$ is coherent, the session map $M \parop \Map\rolep T$ can
successfully terminate (\Cref{lem:fsub_term}).
The rank annotation on subtyping judgements is used to
set up an appropriate inductive argument for proving this property.

\begin{theorem}[Soundness]
	\label{thm:fsub_sound}
 	If $S \subt T$ then $S \ssubt T$.
\end{theorem}

We start with an auxiliary result formalizing the simulation entailed by the
relation $S \subt T$.

\begin{lemma}
	\label{lem:subt_sim}
  	If\/ $S \subt T$ and $M \parop \Map\rolep{S}$ is coherent and $M \parop
  	\Map\rolep{T} \wred N \parop \Map\rolep{T'}$, then $M \parop \Map\rolep{S}
  	\wred N \parop \Map\rolep{S'}$ for some $S' \subt T'$.
\end{lemma}
\begin{proof}
  We prove the result for a single reduction
  $M \parop \Map\rolep{T} \lred\tau N \parop \Map\rolep{T'}$. The
  general statement then follows by a straightforward induction on
  the length of the reduction
  $M \parop \Map\rolep{T} \wred N \parop \Map\rolep{T'}$ using the
  fact that coherence is preserved by reductions.

  \proofcase{Case $M \lred\tau N$}
  %
  Then $T' = T$ and we conclude by taking $S' \eqdef S$.

  \proofcase{Case $\Map\rolep{T} \lred\tau \Map\rolep{T''}$}
  %
  Then
  $\Map\rolep{T} = \Map\rolep{\Tags\roleq\Out\Tag_i.T_i} \lred\tau
  \Map\rolep{\roleq\Out{\Tag_k}.T_k} = \Map\rolep{T'}$ for
  some $k\in I$.
  %
  From the hypothesis $S \subt T$ we deduce
  $S = \Tags[i\in J]\roleq\Out\Tag_i.S_i$ where $I \subseteq J$ and
  $S_i \subt T_i$ for every $i\in I$.
  %
  Now we have
  $M \parop \Map\rolep{S} \lred\tau M \parop
  \Map\rolep{\roleq\Out\Tag_k.S_k}$ and also
  $\roleq\Out\Tag_k.S_k \subt T'$.
  %
  We conclude by taking $S' \eqdef \roleq\Out\Tag_k.S_k$.

  \proofcase{Case $M \xlred{\Map\roleq{\rolep\Out\Tag}} N$ and
    $\Map\rolep{T} \xlred{\Map\rolep{\roleq\In\Tag}}
    \Map\rolep{T'}$}
  %
  Then $T = \Tags\roleq\In\Tag_i.T_i$ and $\Tag = \Tag_k$ and
  $T' = T_k$ for some $k\in I$.
  %
  From the hypothesis $S \subt T$ we deduce
  $S = \Tags[i\in J]\roleq\In\Tag_i.S_i$ and $J \subseteq I$ and
  $S_i \subt T_i$ for every $i\in J$.
  %
  From the hypothesis $M \parop \Map\rolep{S}$ coherent we deduce
  $k\in J$ or else the participant $\rolep$ would not be able to
  receive the $\Tag$ tag.
  %
  We conclude by taking $S' \eqdef S_k$.

  \proofcase{Case $M \xlred{\Map\roleq{\rolep\In\Tag}} N$ and
    $\Map\rolep{T} \xlred{\Map\rolep{\roleq\Out\Tag}}
    \Map\rolep{T'}$}
  %
  Then $T = \Tags\roleq\Out\Tag_i.T_i$ and $\Tag = \Tag_k$ and
  $T' = T_k$ for some $k\in I$.
  %
  From the hypothesis $S \subt T$ we deduce
  $S = \Tags[i\in J]\roleq\Out\Tag_i.S_i$ and $I \subseteq J$ and
  $S_i \subt T_i$ for every $i\in I$.
  %
  We conclude by taking $S' \eqdef S_k$.

  \proofcase{Case $M \xlred{\Map\roleq\rolep\Out U} N$ and
    $\Map\rolep{T} \xlred{\Map\rolep\roleq\In U} \Map\rolep{T'}$}
  %
  Then $T = \roleq\In U.T'$.
  %
  From the hypothesis $S \subt T$ we deduce $S = \roleq\In U.S'$
  and $S' \subt T'$.
  %
  We conclude by observing that
  $M \parop \Map\rolep{S} \lred\tau N \parop \Map\rolep{S'}$.

  \proofcase{Case $M \xlred{\Map\roleq\rolep\In U} N$ and
    $\Map\rolep{T} \xlred{\Map\rolep\roleq\Out U} \Map\rolep{T'}$}
  %
  Then $T = \roleq\Out U.T'$.
  %
  From the hypothesis $S \subt T$ we deduce $S = \roleq\Out U.S'$
  and $S' \subt T'$.
  %
  We conclude by observing that
  $M \parop \Map\rolep{S} \lred\tau N \parop \Map\rolep{S'}$.
\end{proof}

Next we show that $S \subt T$ preserves the termination of any session map that
completes $S$ into a coherent one.

\begin{lemma}
	\label{lem:fsub_term}
  	If\/ $S \subt[n] T$ and $M \parop \Map\rolep{S}$ is coherent, then $M \parop
  	\Map\rolep{T} \wlred{\In\terminated}$.
\end{lemma}
\begin{proof}
  By induction on the lexicographically ordered tuple
  $(n, |\actions|)$ where $\actions$ is any string of actions such
  that $M \xwlred{\co\actions\co\Pol\terminated}$ and
  $\Map\role{S} \xwlred{\actions\Pol\terminated}$.  We know that at
  least one such $\actions$ least does exist from the hypothesis
  $M \parop \Map\rolep{S}$ is coherent.
  %
  We now reason by cases on the shape of $\actions$.

  \proofcase{Case $\actions = \varepsilon$}
  %
  Then $S = \End$.
  %
  From the hypothesis $S \subt[n] T$ and \refrule{fsm-end} we deduce
  $T = \End$ and we conclude
  $M \parop \Map\rolep{T} \wlred{\In\terminated}$.

  \proofcase{Case $\actions = \Map\rolep\roleq\In\Tag\actionsB$}
  %
  Then $S = \Tags\roleq\In\Tag_i.S_i$ and $\Tag = \Tag_k$ for some
  $k\in I$.
  %
  From the hypothesis $S \subt[n] T$ and \refrule{fsm-tag-in} we
  deduce $T = \Tags[i\in J]\roleq\In\Tag_i.T_i$ and $I \subseteq J$
  and $S_i \subt[n_i] T_i$ and $n_i \leq n$ for every $i\in I$.
  %
  We conclude using the induction hypothesis.

  \proofcase{Case $\actions = \Map\rolep\roleq\Out\Tag\actionsB$}
  %
  Then $S = \Tags\roleq\Out\Tag_i.S_i$ and $\Tag = \Tag_k$ for some
  $k\in I$. We distinguish two sub-cases, according to the last rule
  used in the derivation of $S \subt[n] T$.
  %
  If the last rule was \refrule{fsm-tag-out-1}, then
  $T = \Tags[i\in I]\roleq\Out\Tag_i.T_i$ and $S_i \subt[n_i] T_i$
  and $n_i \leq n$ for every $i\in I$.
  %
  In particular, $S_k \subt[n_k] T_k$ and $n_k \leq n$ and we
  conclude using the induction hypothesis.
  %
  If the last rule was \refrule{fsm-tag-out-2}, then
  $T = \Tags[i\in J]\roleq\Out\Tag_i.T_i$ with $J \subseteq I$ and
  $S_i \subt[n_i] T_i$ for every $i\in J$ and $n_j < n$ for some
  $j\in J$.
  %
  In particular, we have $S_j \subt[n_j] T_j$ and we conclude using
  the induction hypothesis.
\end{proof}

\begin{proof}[Proof of \Cref{thm:fsub_sound}]
  Consider a run
  $M \parop \Map\rolep{T} \wred N \parop \Map\rolep{T'}$.
  %
  From \Cref{lem:subt_sim} we deduce that there exists $S' \subt T'$
  such that $M \parop \Map\rolep{S} \wred N \parop \Map\rolep{S'}$.
  %
  From the hypothesis that $M \parop \Map\rolep{S}$ is coherent we
  deduce $N \parop \Map\rolep{S'}$ is also coherent.
  %
  From \Cref{lem:fsub_term} we conclude
  $N \parop \Map\rolep{T'} \wlred{\In\terminated}$.
\end{proof}