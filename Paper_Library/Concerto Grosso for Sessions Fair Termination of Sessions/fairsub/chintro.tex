\begintreble
%
The mere \emph{assumption} of fairness (see \Cref{def:fair_run}) does not
turn an ordinary session type system into one that ensures fair session
termination because the correspondence imposed by the type system between the
structure of processes and that of the protocols they implement is generally
(often necessarily) a loose one.
%
Indeed, processes may be ``more accommodating'' than the protocols they
implement by handling more messages than those mentioned in the protocols. For
example, the \actor{seller} in \Cref{ex:bsc} could handle a $\tsearch$ message in
addition to $\tadd$ and $\tpay$, even if the session type associated with $x$
does not mention $\tsearch$.
%
At the same time, processes may also be ``less demanding'' than the protocols they 
implement by sending fewer messages than those allowed by the protocols. For example, 
the \actor{buyer} in \Cref{ex:bsc} could always purchase an even/odd number of items, 
or at least $n$ items, or no more than $n$ items, even if the session type 
associated with the channel allows sending an arbitrary number of $\tadd$ messages.
%
These mismatches between processes and protocols are usually reconciled by a
\emph{subtyping relation} for session types \citep{GayHole05,BernardiHennessy16}. 
The problem is that this subtyping relation is \emph{too coarse} because 
it has been conceived to preserve the
\emph{safety} properties of sessions but not termination, which is a
\emph{liveness} property: if session types are not sufficiently precise
descriptions of the actual behavior of processes, a session that appears to be
fairly terminating at the level of types may not terminate at all at the level
of processes.
%
To solve this problem we adopt \emph{fair subtyping}
\citep{Padovani13,Padovani16,BravettiLangeZavattaro21}, a
\emph{liveness-preserving} refinement of the subtyping relation defined by
\cite{GayHole05}.

The chapter is organized as follows. First, in \Cref{sec:original_sub} we present the original
subtyping relation for session types \citep{GayHole05}.
We conclude such section showing why such relation is \emph{unfair} by applying it
to a variant of \Cref{ex:bsc} (see \Cref{ssec:unfair_sub}).
Then, in \Cref{sec:fair_sub} we introduce \emph{fair subtyping}
\citep{Padovani13,Padovani16,BravettiLangeZavattaro21} .
We show an inference system for characterizing it and we prove its correctness
with respect to a semantic definition. At last, we
show a characterization of fair subtyping by using a generalized inference
system (see \Cref{ssec:fsub_gis}). We will refer to this definition
in \Cref{sec:agda_fs}.