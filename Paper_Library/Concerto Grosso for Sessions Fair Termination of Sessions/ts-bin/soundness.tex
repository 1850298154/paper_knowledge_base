\beginalto
%
In this section we detail the proof of \Cref{thm:ts_bin_sound}.
The most intriguing aspect in such proof is that a
closed, well-typed process admits a reduction sequence to $\Done$.
%
The proof technique is related to the \emph{method of helpful directions} \citep{Francez86}:
we define a well-founded \emph{measure} for (well-typed) processes and we prove
that this measure decreases strictly as the result of ``helpful'' reductions.
In our case, the measure of a (well-typed) process $P$ is a lexicographically
ordered pair $(m, n)$ of natural numbers such that $m$ is an upper bound to the
number of sessions that $P$ may need to create and of weights of casts that $P$ may need to
perform \emph{in the future} in order to terminate, whereas $n$ is the
cumulative efforts to terminate the sessions that $P$ has created \emph{in the past} and that
are not terminated yet. 
We account for this effort by measuring the shortest reduction that terminates a
compatible session (\Cref{def:compatibility}).
Notably, a session terminates by \refrule{rb-close}; a cast is
performed when it is absorbed by the corresponding restriction, namely by
\refrule{sb-cast-new}.

We first show two standard results, that is, typing is preserved by structural
precongruence and reduction.
Then, we formally introduce the measure and we characterize some \emph{normal forms} that we
need in order to achieve the proof that such measure reduces by following the right
reductions.

%%%%%%%%%%%%%%%%%%%%%%%%%
%%% SUBJECT REDUCTION %%%
%%%%%%%%%%%%%%%%%%%%%%%%%

\subsection{Subject Reduction}
\beginbass
%
We show the proof of a standard subject reduction theorem (see \Cref{lem:subj_red_multi}). For this sake,
we need an additional result, that we dub subject congruence,
stating that well-typedness is preserved by the structural precongruence
relation for processes in \Cref{fig:pcong_multi} (\Cref{lem:subj_cong_multi}).
Notably, \Cref{lem:subj_cong_multi} tells that the rank does not incrase.
Such result is not provable in \Cref{lem:subj_red_multi} as the non deterministic choice
can reduce to a branch that increases the rank.

\begin{lemma}
\label{lem:substitution_multi}
	If $\wtp[n]{\Ctx, x : S}{P}$ and $\Ctx, u : S$ is defined, then $\wtp[n]{\Ctx, u : S}{P \subst{u}{x}}$.
	A typing context is \emph{defined} if the endpoints occurring in it all have different session names.
\end{lemma}
\begin{proof}
By bounded coinduction (see \Cref{prop:bcp}).
\end{proof}

\begin{lemma}[Subject Congruence]
\label{lem:subj_cong_multi}
	If\/ $\wtp[n] \Ctx {P}$ and $P \pcong Q$, then $\wtp[m] \Ctx {Q}$ for some $m \leq n$.
\end{lemma}
\begin{proof}
By induction on the derivation of $P \pcong Q$ and by cases on the last rule applied.

\proofrule{sm-par-comm} 
%
Then $P = \pres{s}{\procs{P} \ppar P' \ppar Q' \ppar \procs{Q}} \pcong \pres{s}{\procs{P} \ppar Q' \ppar P' \ppar \procs{Q}} = Q$.
%
From rule \refrule{tm-par} we deduce that there exist $\Ctx_i, \role_i, S_i, n_i$ for $i = 1,\dots,h$ such that
\begin{itemize}
\item $\Ctx = \Ctx_1,\dots,\Ctx_h$
\item $n = 1 + \sum_{i=1}^h n_i$
\item $\prod_{i=1}^h \Map{\role_i}{S_i} \ft$
\item $\wtp[n_i]{\Ctx_i, \ep{s}{\role_i} : S_i}{P_i}$ for $i = 1,\dots,k$
\item $\wtp[n_{k+1}]{\Ctx_{k+1}, \ep{s}{\role_{k+1}} : S_{k+1}}{P'}$
\item $\wtp[n_{k+2}]{\Ctx_{k+2}, \ep{s}{\role_{k+2}} : S_{k+2}}{Q'}$
\item $\wtp[n_i]{\Ctx_i, \ep{s}{\role_i} : S_i}{Q_i}$ for $i = k+3,\dots,h$
\end{itemize}

We conclude $\wtp[m]\Ctx{Q}$ with one application of \refrule{tm-par} by taking $m \eqdef n$.

\proofrule{sm-par-assoc}
%
Then $P = \pres{s}{\procs{P} \ppar \pres{t}{R \ppar \procs{Q}}} \pcong \pres{t}{\pres{s}{\procs{P} \ppar R} \ppar \procs{Q}} = Q$ and $s \in \fn{R}$.
%
From rule \refrule{tm-par} we deduce that there exist $\Ctx_i, \role_i, S_i, n_i$ for $i = 1,\dots,h$ such that
\begin{itemize}
\item $\Ctx = \Ctx_1,\dots,\Ctx_h$
\item $n = 1 + \sum_{i=1}^h n_i$
\item $\prod_{i=1}^h \Map{\role_i}{S_i} \ft$
\item $\wtp[n_i]{\Ctx_i, \ep{s}{\role_i} : S_i}{P_i}$ for $i = 1,\dots,h - 1$
\item $\wtp[n_h]{\Ctx_h, \ep{s}{\role_h} : S_h}{\pres{t}{R \ppar \procs{Q}}}$
\end{itemize}
From rule \refrule{tm-par} and the hypothesis that $s \in \fn{R}$ we deduce that there exist $\CtxD_i, \roleq_i, T_i, m_i$ for $i = 1,\dots,k$ such that
\begin{itemize}
\item $\Ctx_h = \CtxD_1,\dots,\CtxD_k$
\item $n_h = 1 + \sum_1^k m_i$
\item $\prod_1^k \Map{\roleq_i}{T_i} \ft$
\item $\wtp[m_1]{\CtxD_1, \ep{s}{\role_h} : S_h, \ep{t}{\roleq_1} : T_1}{R}$
\item $\wtp[m_{i+1}]{\CtxD_{i+1}, \ep{t}{\roleq_{i+1}} : T_{i+1}}{Q_i}$ for $i = 1,\dots,k-1$
\end{itemize}
Using \refrule{tm-par} we deduce 
$\wtp[1 + \sum_{i=1}^{h-1}{n_i} + m_1]{\Ctx_1,\dots,\Ctx_{h-1},\CtxD_1, \ep{t}{\roleq_1} : T_1}{\pres{s}{\procs{P} \ppar R}}$.
We conclude $\wtp[m]{\Ctx}{\pres{t}{\pres{s}{\procs{P} \ppar R} \ppar \procs{Q}}}$ with another application of \refrule{tm-par} by taking $m \eqdef n$.

\proofrule{sm-cast-comm} 
%
Then $P = \pcast{u}{\pcast{v}{R}} \pcong \pcast{v}{\pcast{u}{R}} = Q$. We can assume $u \ne v$ or else $P = Q$.
%
From rule \refrule{tm-cast} we deduce that there exist $\Ctx_1, S, T, n_1, n_u$ such that
\begin{itemize}
\item $\Ctx = \Ctx_1, u : S$
\item $S \subt[n_u] T$
\item $n = n_u + n_1$
\item $\wtp[n_1]{\Ctx_1, u : T}{\pcast{v}{R}}$
\end{itemize} 
From rule \refrule{tm-cast} we deduce that there exist $\Ctx_2, S', T', n_2, n_v$ such that
\begin{itemize}
\item $\Ctx_1 = \Ctx_2, v : S'$
\item $S' \subt[n_v] T'$
\item $n_1 = n_v + n_2$
\item $\wtp[n_2]{\Ctx_2, u : T, v : T'}{R}$
\end{itemize}
We derive $\wtp[n_u + n_2]{\Ctx_2, u : S, v : T'}{\pcast{u}{R}}$ with one application of \refrule{tm-cast} 
and we conclude with another application of \refrule{tm-cast} by taking $m \eqdef n$.

\proofrule{sm-cast-new}
%
Then $P = \pres{s}{\pcast{\ep{s}{\role}}{R} \ppar \procs{P}} \pcong \pres{s}{R \ppar \procs{P}} = Q$.
%
From rule \refrule{tm-par} we deduce that there exist $\CtxD, n'$ and $\Ctx_i, \roleq_i, n_i$ for $i = 1,\dots,h$ such that
\begin{itemize}
\item $\Ctx = \CtxD, \Ctx_1, \dots, \Ctx_h$ for $i = 1,\dots,h$
\item $n = 1 + n' + \sum_{i=1}^h n_i$
\item $\Map{\role}{S} \ppar \prod_{i = 1}^h \Map{\roleq_i}{S_i} \ft$
\item $\wtp[n']{\CtxD, \ep{s}{\role} : S}{\pcast{\ep{s}{\role}}{R}}$
\item $\wtp[n_i]{\Ctx_i, \ep{s}{\roleq_i} : S_i}{P_i}$ for $i = 1,\dots,h$
\end{itemize} 
From rule \refrule{tm-cast} we deduce that there exist $T, m', m_s$ such that
\begin{itemize}
\item $S \subt[m_s] T$
\item $n' = m_s + m'$
\item $\wtp[m']{\CtxD, \ep{s}{\role} : T}{R}$
\end{itemize}
From $\Map{\role}{S} \ppar \prod_{i = 1}^h \Map{\roleq_i}{S_i} \ft$, $S \subt[m_s] T$ and \cref{def:ssubt} we deduce 
$\Map{\role}{T} \ppar \prod_{i = 1}^h \Map{\roleq_i}{S_i} \ft$.
We conclude with an application of \refrule{tm-par} by taking $m \eqdef 1 + m' + \sum_{i=1}^h n_i \le n$.

\proofrule{sm-cast-swap}
%
Then $P = \pres{s}{\pcast{\ep{t}{\role}}{R} \ppar \procs{P}} \pcong \pcast{\ep{t}{\role}}{\pres{s}{R \ppar \procs{P}}} = Q$ and $t \ne s$.
%
From rule \refrule{tm-par} we deduce that there exist $\Ctx_i, \roleq_i, n_i$ for $i = 1,\dots,h$ such that
\begin{itemize}
\item $\Ctx = \Ctx_1, \dots, \Ctx_h$ for $i = 1,\dots,h$
\item $n = 1 + \sum_{i=1}^h n_i$
\item $\prod_{i = 1}^h \Map{\roleq_i}{S_i} \ft$
\item $\wtp[n_1]{\Ctx_1, \ep{s}{\roleq_1} : S_1}{\pcast{\ep{t}{\role}}{R}}$
\item $\wtp[n_i]{\Ctx_i, \ep{s}{\roleq_i} : S_i}{P_i}$ for $i = 2,\dots,h$
\end{itemize} 
From rule \refrule{tm-cast} we deduce that there exist $\CtxD, T, n', m_t$ such that
\begin{itemize}
\item $\Ctx_1 = \CtxD, \ep{t}{\role} : S$
\item $S \subt[m_t] T$
\item $n_1 = m_t + n'$
\item $\wtp[n']{\CtxD, \ep{t}{\role} : T, \ep{s}{\roleq_1} : S_1}{R}$
\end{itemize}
We derive $\wtp[1 + n' + \sum_{i=2}^h n_i]{\CtxD, \ep{t}{\role} : T,\Ctx_2,\dots,\Ctx_h}{\pres{s}{R \ppar \procs{P}}}$ 
with an application of \refrule{tm-par}. We conclude with an application of \refrule{tm-cast} by taking $m \eqdef n$.

\proofrule{sm-call} 
%
Then $P = \pinvk{A}{\seqof u} \pcong R\subst{\seqof u}{\seqof x} = Q$ and
$\Definition{A}{\seqof x}{R}$.
%
From \refrule{tm-call} we conclude that there exist $\seqof S$ and $m$ such that
$\tass{A}{\seqof{S}}{m}$ and $\Ctx = \seqof{u : S}$ and $\wtp[m]{\seqof{u :
S}}{Q}$ and $m \leq n$.
\end{proof}

\begin{lemma}[Subject Reduction]
	\label{lem:subj_red_multi}
	If\/ $\wtp[n] \Ctx {P}$ and $P \red Q$, then $\wtp[m] \Ctx {Q}$ for some $m$.
\end{lemma}
\begin{proof}
By induction on the derivation of $P \red Q$ and by cases on the last rule applied.

\proofrule{rm-choice}
%
Then $P = P_1 \pchoice P_2 \red P_k = Q$ and $k \in \set{1,2}$.
%
From \refrule{tm-choice} we deduce that $\wtp[m]{\Ctx}{Q}$ for some $m$.

\proofrule{rm-signal}
%
Then $ P = \pres{s}{\pwait{\ep{s}{\role}}{Q} \parop \pclose{\ep{s}{\roleq_1}} \parop \cdots \parop \pclose{\ep{s}{\roleq_h}}} \red Q$.
%
From \refrule{tm-par}, \refrule{tm-wait} and \refrule{tm-close} we deduce that there exist $m$ and $n_i$ for $i=1,\dots,h$ such that
\begin{itemize}
\item $n = 1 + m + \sum_{i=1}^h n_i$
\item $\wtp[m]{\Ctx, \ep{s}{\role} : \End[\In]}{\pwait{\ep{s}{\role}}{Q}}$
\item $\wtp[m]{\Ctx}{Q}$
\item $\wtp[n_i]{\ep{s}{\roleq_i} : \End[\Out]}{\pclose{\ep{s}{\roleq_i}}}$ for
$i=1,\dots,h$
\end{itemize}
There is nothing left to prove.

\proofrule{rm-channel}
%
Then $P = \pres{s}{\poch{\ep{s}{\rolep}}{\roleq}{v}{P'} \parop
\pich{\ep{s}{\roleq}}{\rolep}{x}{Q'} \parop \procs{R}} \red \pres{s}{P' \parop
Q'\subst{v}{x} \parop \procs{R}} = Q$.
%
From \refrule{tm-par} we deduce that there exist $\Ctx_i, S_i, \role_i, n_i$ for $i=1,\dots,h$ such that
\begin{itemize}
\item $\Ctx = \Ctx_1,\dots,\Ctx_h$
\item $n = 1 + \sum_{i=1}^h n_i$
\item $\prod_{i=1}^h \Map{\role_i}{S_i} \ft$
\item $\role = \role_1$ and $\roleq = \role_2$
\item $\wtp[n_1]{\Ctx_1, \ep{s}{\role} : S_1}{\poch{\ep{s}{\rolep}}{\roleq}{v}{P'}}$
\item $\wtp[n_2]{\Ctx_2, \ep{s}{\roleq} : S_2}{\pich{\ep{s}{\roleq}}{\rolep}{x}{Q'}}$
\item $\wtp[n_i]{\Ctx_i, \ep{s}{\role_i} : S_i}{R_i}$ for $i = 3,\dots,h$
\end{itemize}
From \refrule{tm-channel-out} and \refrule{tm-channel-in} we deduce that there exist $S_v, T_1, T_2, \CtxD_1$ such that
\begin{itemize}
\item $S_1 = \roleq\Out{S_v}.T_1$
\item $\Ctx_1 = \CtxD_1, v : S_v$
\item $\wtp[n_1]{\CtxD_1, \ep{s}{\rolep} : T_1}{P'}$
\item $S_2 = \rolep\In{S_v}.T_2$
\item $\wtp[n_2]{\Ctx_2, \ep{s}{\roleq} : T_2, x : S_v}{Q'}$
\end{itemize} 
Using \cref{lem:substitution_multi} we deduce $\wtp[n_2]{\Ctx_2, \ep{s}{\roleq} : T_2,
v : S_v}{Q'\subst{v}{x}}$. Using \cref{def:coherence} we deduce $\Map{\rolep}{T_1}
\parop \Map{\roleq}{T_2} \parop \prod_{i=3}^h \Map{\role_i}{S_i} \ft$. We conclude with
one application of \refrule{tm-par} taking $m \eqdef n$.

\proofrule{rm-pick}
%
Then $P = \pres{s}{\pobranch[i\in I]{\ep{s}{\role}}{\roleq}{\Tag_i}{\PP_i} \parop \procs{Q}} \red 
\pres{s}{\pobranch{\ep{s}{\role}}{\roleq}{\Tag_k}{\PP_k}\parop \procs{Q}} = Q$ and $k \in I$. %$|I| > 1$
%
From \refrule{tm-par} we deduce that there exist $\Ctx_i, \role_i, n_i, S_i$ for $i=1,\dots,h$ such that
\begin{itemize}
\item $\Ctx = \Ctx_1, \dots, \Ctx_h$
\item $n = 1 + \sum_{i=1}^h n_i$
\item $\prod_{i=1}^h \Map{\role_i}{S_i} \ft$
\item $\role = \role_1$ and $\roleq = \role_i$ for some $i \in \set{2,\dots,h}$
\item $\wtp[n_1]{\Ctx_1, \ep{s}{\role} : S_1}{\pobranch[i\in I]{\ep{s}{\role}}{\roleq}{\Tag_i}{\PP_i}}$
\item $\wtp[n_i]{\Ctx_i, \ep{s}{\role_i} : S_i}{Q_i}$ for $i=2,\dots,h$
\end{itemize}
From \refrule{tm-tag} we deduce that there exist $T_i$ for all $i \in I$ such that
\begin{itemize}
\item $S_1 = \Tags\roleq\Out \Tag_i.T_i$
\item $\wtp[n_1]{\Ctx_1, \ep{s}{\role} : T_i}{P_i}~{}^{(i\in I)}$
\end{itemize}
From the hypothesis that $k \in I$ we deduce that $\wtp[n_1]{\Ctx_1, \ep{s}{\role} : T_k}{P_k}$ and from \refrule{tm-tag} we deduce 
$\wtp[n_1]{\Ctx_1, \ep{s}{\role} : \roleq\Out\Tag_k.T_k}{\pobranch{\ep{s}{\role}}{\roleq}{\Tag_k}{\PP_k}}$. 
From \cref{def:coherence} we deduce that $\roleq\Out\Tag_k.T_k \parop \prod_{i=2}^h \Map{\role_i}{S_i} \ft$. 
We conclude with an application of \refrule{tm-par} taking $m \eqdef n$.

\proofrule{rm-tag}
%
Then $P = \pres{s}{\pobranch{\ep{s}{\role}}{\roleq}{\Tag_k}{\PP'} \parop \pibranch[i\in I]{\ep{s}{\roleq}}{\role}{\Tag_i}{Q_i} \parop \procs{R}} 
        \red 
        \pres{s}{P' \parop Q_k \parop \procs{R}} = Q$ and $k \in I$.
%
From \refrule{tm-par} we deduce that there exist $\Ctx_i, S_i, \role_i, n_i$ for $i=1,\dots,h$ such that
\begin{itemize}
\item $\Ctx = \Ctx_1,\dots,\Ctx_h$
\item $n = 1 + \sum_{i=1}^h n_i$
\item $\prod_{i=1}^h \Map{\role_i}{S_i} \ft$
\item $\role = \role_1$ and $\roleq = \role_2$
\item $\wtp[n_1]{\Ctx_1, \ep{s}{\role} : S_1}{\pobranch{\ep{s}{\role}}{\roleq}{\Tag_k}{\PP'}}$
\item $\wtp[n_2]{\Ctx_2, \ep{s}{\roleq} : S_2}{\pibranch[i\in I]{\ep{s}{\roleq}}{\role}{\Tag_i}{Q_i}}$
\item $\wtp[n_i]{\Ctx_i, \ep{s}{\role_i} : S_i}{R_i}$ for $i = 3,\dots,h$
\end{itemize}
From \refrule{tm-tag} we deduce that there exist $S'_1$ and $T_i$ for every $i \in I$ such that
\begin{itemize}
\item $S_1 = \roleq\Out \Tag_k.S'_1$
\item $\wtp[n_1]{\Ctx_1, \ep{s}{\role} : S'_1}{P'}$
\item $S_2 = \Tags\rolep\In \Tag_i.T_i$
\item $\wtp[n_2]{\Ctx_2, \ep{s}{\roleq} : T_i}{Q_i}~{}^{(i\in I)}$
\end{itemize}
From \cref{def:coherence} we deduce that $\Map{\role}{S'_1} \parop \Map{\roleq}{T_k} \parop \prod_{i=2}^h \Map{\role_i}{S_i} \ft$. 
We conclude with an application of \refrule{tm-par} by taking $m \eqdef n$.

\proofrule{rm-par}
%
Then $P = \pres{s}{P' \parop \procs{R}} \red \pres{s}{Q' \parop \procs{R}} = Q$ and $P' \red Q'$.
%
From \refrule{tm-par} we deduce that there exist $\Ctx_i, \role_i, S_i, n_i$ for $i=1,\dots,h$ such that
\begin{itemize}
\item $\Ctx = \Ctx_1,\dots,\Ctx_h$
\item $n = 1 + \sum_{i=1}^h n_i$
\item $\prod_{i=1}^h \Map{\role_i}{S_i} \ft$
\item $\wtp[n_1]{\Ctx_1, \ep{s}{\role_1} : S_1}{P'}$
\item $\wtp[n_i]{\Ctx_i, \ep{s}{\role_i} : S_i}{R_i}$ for $i = 2,\dots, h$
\end{itemize}
Using the induction hypothesis on $\wtp[n_1]{\Ctx_1, \ep{s}{\role_1} : S_1}{P'}$ and $P' \red Q'$
we deduce $\wtp[n'_1]{\Ctx_1, \ep{s}{\role_1} : S_1}{Q'}$ for some $n'_1$. We conclude with an application 
of \refrule{tm-par} taking $m \eqdef 1 + n'_1 + \sum_{i=2}^h n_i$.

\proofrule{rm-cast}
%
Then $P = \pcast{u}{P'} \red \pcast{u}{Q' } = Q$ and $P' \red Q'$.
%
From \refrule{tm-cast} we deduce that there exist $S, T, \Ctx', n', m_u$ such that
\begin{itemize}
\item $\Ctx = \Ctx', u : S$
\item $S \subt[m_u] T$
\item $n = m_u + n'$
\item $\wtp[n']{\Ctx', u : T}{P'}$
\end{itemize}
Using the induction hypothesis on $\wtp[n']{\Ctx', u : T}{P'}$ and $P' \red Q'$ we deduce $\wtp[m']{\Ctx', u : T}{Q'}$ for some $m'$.
We conclude with an application of \refrule{tm-cast} taking $m \eqdef m_u + m'$.

\proofrule{rm-struct}
%
Then $P \pcong P' \red Q' \pcong Q$.
%
From \Cref{lem:subj_cong_multi} we deduce that $\wtp[n']{\Ctx}{P'}$ for some $n' \le n$.
Using the induction hypothesis on $\wtp[n']{\Ctx}{P'}$ and $P' \red Q'$ we deduce $\wtp[m']{\Ctx}{Q'}$ for some $m'$. 
We conclude using \cref{lem:subj_cong_multi} once more.
\end{proof}

%%%%%%%%%%%%%%%
%%% MEASURE %%%
%%%%%%%%%%%%%%%

\subsection{Measure}
\beginbass
%
We introduce two fundamental notions for the soundness proof of the type system.
First, we introduce the \emph{rank} of a session map $M$ as the minimum length to reach
successful termination of session $M$. 
%
Then, we introduce the \emph{measure} of a process which takes into account the rank in the typing judgment
as well as the ranks of the session that have been already opened.
We embed such measure in the typing derivations by using a refined set of rules.
%
At last, we compare the typing judgments labeled with the usual rank with those including the measure (see \Cref{lem:measure_rank_multi})
and we prove that structural precongruence of processes does not increase the measure (see \Cref{lem:measure_pcong_multi}).

\begin{figure}[t]
\framebox[\textwidth]{
\begin{mathpar}
    \inferrule[mtm-thread]{
        \mathstrut
    }{
        \wtpn{(n, 0)}\Ctx{P}
    }
    \wtp[n]\Ctx{P}
    \defrule[mtm-thread]{}
    \and
    \inferrule[mtm-cast]{
        \wtpn\Measure{\Ctx, u : T}{P}
    }{
        \wtpn{\Measure + (n,0)}{\Ctx, u : S}{\pcast{u} P}
    }
    ~
    S \subt[n] T
    \defrule[mtm-cast]{}
    \and
    \inferrule[mtm-par]{
        \wtpn{\Measure_i}{\Ctx_i, \ep{s}{\role_i} : S_i}{P_i}~{}^{(i=1,\dots,h)}
    }{
        \wtpn{\sum_{i=1}^h \Measure_i + (0, \rank{\set{\Map{\role_i}{S_i}}_{i=1,\dots,h}})}{
            \Ctx_1,\dots,\Ctx_h
        }{
            \pres{s}{P_1 \parop \dots \parop P_h}
        }
    }
    ~ \coherent{\set{\Map{\role_i}{S_i}}_{i=1..h}}
    \defrule[mtm-par]{}
\end{mathpar}
}
\caption{Typing rules with measure}
\label{fig:measure_multi}
\end{figure}

\begin{definition}[rank]
  \label{def:rank_multi}
  The \emph{rank} of a session map $M = \prod_{i=1}^h \Map{\role_i}{S_i}$, written $\rank{M}$, is the element of $\Nat
  \union \set\infty$ defined as
  \begin{center}
  	\begin{math}
  		\rank{M} \eqdef \min | M \wlred{\In\terminated}|
  	\end{math}
  \end{center}
  where $|M \wlred{\action} N|$ denotes the length of the sequence $\tau,\dots,\tau,\action$ and we postulate that $\min\emptyset = \infty$.
\end{definition}

\begin{definition}[Measure]
\label{def:measure_multi}
The measure of a process is a lexicographically ordered pair of natural numbers
$(m , n)$ where:
\begin{itemize}
\item $m$ is an upper bound to the number of sessions that the process may open
and of weights of casts that the process may perform \emph{in the future} before
it terminates;
\item $n$ is the overall effort for terminating the sessions that have been
already opened \emph{in the past}, \ie the sum of their rank (\Cref{def:rank_multi}).
\end{itemize}
\end{definition}

In \Cref{fig:measure_multi} we introduce a refined set of typing rules for processes that allow us to
associate them with their measure, not just with their rank.
The idea behind these rules (similarly to \Cref{fig:measure_bin}) is that they distinguish between \emph{past} and
\emph{future} of a process by looking at its structure. Indeed, unguarded
sessions have been created, casts have not been performed yet and sessions that
occur guarded have not been created yet.
% 
\refrule{mtm-thread} adopts the rank of the process inside the usual typing
judgment (\Cref{fig:ts_multi}) as first component of the measure. This rule has lower
priority with respect to the other rules so that it is applied to processes that
are not casts or restrictions.
%
In \refrule{mtm-cast} the first component of the measure is increased by the
weight of the cast.
%
\refrule{mtm-par} increases the second component of the measure by the rank of
the involved session.

\begin{lemma}
    \label{lem:measure_rank_multi}
    The following properties hold:
    \begin{enumerate}
        \item $\wtp[n]\Ctx{P}$ implies $\wtpn\Measure\Ctx{P}$ for some
        $\Measure \leq (n, 0)$;
        \item $\wtpn\Measure\Ctx{P}$ implies $\wtp[n]\Ctx{P}$ for some $n$ such that $\Measure \leq (n, 0)$.
    \end{enumerate}
\end{lemma}
\begin{proof}
    We prove item 1 by induction on the structure of $P$. 
    The proof of item 2 is by a straightforward induction over $\wtpn\Measure\Ctx{P}$.
    
\proofcase{Case $P = \pres{s}{\procs{P}}$}
From \refrule{tm-par} we deduce that there exist $\Ctx_i, \role_i, S_i, n_i$ for $i = 1,\dots,h$ such that
\begin{itemize}
\item $\Ctx = \Ctx_1,\dots,\Ctx_h$
\item $n = 1 + \sum_{i=1}^h n_i$
\item $\prod_{i=1}^h \Map{\role_i}{S_i} \ft$
\item $\wtp[n_i]{\Ctx_i, \ep{s}{\role_i} : S_i}{P_i}~{}^{(i=1,\dots,h)}$
\end{itemize} 
Using the induction hypothesis on $\wtp[n_i]{\Ctx_i, \ep{s}{\role_i} : S_i}{P_i}~{}^{(i=1,\dots,h)}$ we deduce 
that there exist $\Measure_i$ for $i=1,\dots,h$ such that 
\begin{itemize}
\item $\wtpn{\Measure_i}{\Ctx_i, \ep{s}{\role_i} : S_i}{P_i}~{}^{(i=1,\dots,h)}$
\item $\Measure_i \le (n_i,0)$ for $i=1,\dots,h$
\end{itemize}
We conclude with one application of \refrule{mtm-par} by taking 
$\Measure \eqdef \sum_{i=1}^h \Measure_i + (0, \rank{\prod_{i=1}^h \Map{\role_i}{S_i}})$ and observing that 
$\Measure < (n_1,0) + (n_2,0) + \dots + (n_h,0) + (1,0) = (n,0)$.

\proofcase{Case $P = \pcast{u}{Q}$}
From \refrule{tm-cast} we deduce that there exist $\CtxD, S, T, m$ and $m_u$ such that
\begin{itemize}
\item $\Ctx = \CtxD, u : S$
\item $S \subt[m_u] T$
\item $n = m_u + m$
\item $\wtp[m]{\CtxD, u : T}{Q}$
\end{itemize}
Using the induction hypothesis on $\wtp[m]{\CtxD, u : T}{Q}$ we deduce $\wtpn{\MeasureN}{\CtxD, u : T}{Q}$ for some $\MeasureN \le (m,0)$.
We conclude with an application of \refrule{mtm-cast} by taking $\Measure \eqdef \MeasureN + (m_u,0)$ 
and observing that $\Measure \le (m,0) + (m_u,0) = (n,0)$.

\proofcase{In all the other cases} We conclude with an application of \refrule{mtm-thread} by taking $\Measure \eqdef (n,0)$.
\end{proof}

\begin{lemma}
	\label{lem:measure_pcong_multi}
	If $\wtpn\MeasureM\Ctx P$ and $P \pcong Q$, then there exists $\MeasureN \le \MeasureM$ such that $\wtpn\MeasureN\Ctx Q$.
\end{lemma}
\begin{proof}
By induction on the derivation of $P \pcong Q$ and by cases on the last rule applied. We only consider the base cases.

\proofrule{sm-par-comm} 
%
Then $P = \pres{s}{\procs{P} \ppar P' \ppar Q' \ppar \procs{Q}} \pcong \pres{s}{\procs{P} \ppar Q' \ppar P' \ppar \procs{Q}} = Q$.
%
From rule \refrule{mtm-par} we deduce that there exist $\Ctx_i, \role_i, S_i, \Measure_i$ for $i = 1,\dots,h$ such that
\begin{itemize}
\item $\Ctx = \Ctx_1,\dots,\Ctx_h$
\item $\Measure = \sum_{i=1}^h \Measure_i + (0 , \rank{\prod_{i=1}^h \Map{\role_i}{S_i}})$
\item $\prod_{i=1}^h \Map{\role_i}{S_i} \ft$
\item $\wtpn{\Measure_i}{\Ctx_i, \ep{s}{\role_i} : S_i}{P_i}$ for $i = 1,\dots,k$
\item $\wtpn{\Measure_{k+1}}{\Ctx_{k+1}, \ep{s}{\role_{k+1}} : S_{k+1}}{P'}$
\item $\wtpn{\Measure_{k+2}}{\Ctx_{k+2}, \ep{s}{\role_{k+2}} : S_{k+2}}{Q'}$
\item $\wtpn{\Measure_i}{\Ctx_i, \ep{s}{\role_i} : S_i}{Q_i}$ for $i = k+3,\dots,h$
\end{itemize}
We conclude $\wtpn{\MeasureN}\Ctx{Q}$ with one application of \refrule{mtm-par} by taking $\MeasureN \eqdef \Measure$.

\proofrule{sm-par-assoc}
%
Then $P = \pres{s}{\procs{P} \ppar \pres{t}{R \ppar \procs{Q}}} \pcong \pres{t}{\pres{s}{\procs{P} \ppar R} \ppar \procs{Q}} = Q$ and $s \in \fn{R}$.
%
From rule \refrule{mtm-par} we deduce that there exist $\Ctx_i, \role_i, S_i, \Measure_i$ for $i = 1,\dots,h$ such that
\begin{itemize}
\item $\Ctx = \Ctx_1,\dots,\Ctx_h$
\item $\Measure = \sum_{i=1}^h \Measure_i + (0 , \rank{\prod_{i=1}^h \Map{\role_i}{S_i}})$
\item $\prod_{i=1}^h \Map{\role_i}{S_i} \ft$
\item $\wtpn{\Measure_i}{\Ctx_i, \ep{s}{\role_i} : S_i}{P_i}$ for $i = 1,\dots,h - 1$
\item $\wtpn{\Measure_h}{\Ctx_h, \ep{s}{\role_h} : S_h}{\pres{t}{R \ppar \procs{Q}}}$
\end{itemize}
From rule \refrule{mtm-par} and the hypothesis that $s \in \fn{R}$ we deduce that there exist 
$\CtxD_i, \roleq_i, T_i, \MeasureN_i$ for $i = 1,\dots,k$ such that
\begin{itemize}
\item $\Ctx_h = \CtxD_1,\dots,\CtxD_k$
\item $\Measure_h = \sum_1^k \MeasureN_i + (0 , \rank{\prod_{i=1}^k \Map{\roleq_i}{T_i}})$
\item $\prod_{i=1}^k \Map{\roleq_i}{T_i} \ft$
\item $\wtpn{\MeasureN_1}{\CtxD_1, \ep{s}{\role_h} : S_h, \ep{t}{\roleq_1} : T_1}{R}$
\item $\wtpn{\MeasureN_{i+1}}{\CtxD_{i+1}, \ep{t}{\roleq_{i+1}} : T_{i+1}}{Q_i}$ for $i = 1,\dots,k-1$
\end{itemize}
Using \refrule{tm-par} we deduce
\begin{itemize}
\item $\wtpn{\sum_{i=1}^{h-1}{\Measure_i} 
	+ \MeasureN_1 
	+ \rank{\prod_{i=1}^h \Map{\role_i}{S_i}}}{\Ctx_1,\dots,\Ctx_{h-1},\CtxD_1, \ep{t}{\roleq_1} : T_1}{\pres{s}{\procs{P} \ppar R}}$
\end{itemize}   
We conclude $\wtpn{\MeasureN}{\Ctx}{\pres{t}{\pres{s}{\procs{P} \ppar R} \ppar \procs{Q}}}$ with another 
application of \refrule{mtm-par} by taking $\MeasureN \eqdef \Measure$.

\proofrule{sm-cast-comm} 
%
Then $P = \pcast{u}{\pcast{v}{R}} \pcong \pcast{v}{\pcast{u}{R}} = Q$. We can assume $u \ne v$ or else $P = Q$.
%
From rule \refrule{mtm-cast} we deduce that there exist $\Ctx_1, S, T, \Measure_1, m_u$ such that
\begin{itemize}
\item $\Ctx = \Ctx_1, u : S$
\item $S \subt[m_u] T$
\item $\Measure = \Measure_1 + (m_u , 0)$
\item $\wtpn{\Measure_1}{\Ctx_1, u : T}{\pcast{v}{R}}$
\end{itemize} 
From rule \refrule{tm-cast} we deduce that there exist $\Ctx_2, S', T', \Measure_2, m_v$ such that
\begin{itemize}
\item $\Ctx_1 = \Ctx_2, v : S'$
\item $S' \subt[m_v] T'$
\item $\Measure_1 = \Measure_2 + (m_v , 0)$
\item $\wtpn{\Measure_2}{\Ctx_2, u : T, v : T'}{R}$
\end{itemize}
We derive $\wtpn{\Measure_2 + (m_u,0)}{\Ctx_2, u : S, v : T'}{\pcast{u}{R}}$ with one application of \refrule{mtm-cast} 
and we conclude with another application of \refrule{mtm-cast} by taking $\MeasureN \eqdef \Measure$.

\proofrule{sm-cast-new}
%
Then $P = \pres{s}{\pcast{\ep{s}{\role}}{R} \ppar \procs{P}} \pcong \pres{s}{R \ppar \procs{P}} = Q$.
%
From rule \refrule{mtm-par} we deduce that there exist $\CtxD, \Measure', S$ and $\Ctx_i, \roleq_i, S_i, \Measure_i$ for $i = 1,\dots,h$ such that
\begin{itemize}
\item $\Ctx = \CtxD, \Ctx_1, \dots, \Ctx_h$
\item $\Measure = \Measure' + \sum_{i=1}^h \Measure_i + (0 , \rank{\Map{\role}{S} \ppar \prod_{i = 1}^h \Map{\roleq_i}{S_i}})$
\item $\Map{\role}{S} \parop \prod_{i = 1}^h \Map{\roleq_i}{S_i} \ft$
\item $\wtpn{\Measure'}{\CtxD, \ep{s}{\role} : S}{\pcast{\ep{s}{\role}}{R}}$
\item $\wtpn{\Measure_i}{\Ctx_i, \ep{s}{\roleq_i} : S_i}{P_i}$ for $i = 1,\dots,h$
\end{itemize} 
From rule \refrule{mtm-cast} we deduce that there exist $T, \MeasureN', m_s$ such that
\begin{itemize}
\item $S \subt[m_s] T$
\item $\MeasureN' = \Measure' + (m_s , 0)$
\item $\wtpn{\MeasureN'}{\CtxD, \ep{s}{\role} : T}{R}$
\end{itemize}
From $\Map{\role}{S} \ppar \prod_{i = 1}^h \Map{\roleq_i}{S_i} \ft$, $S \subt[m_s] T$ and \cref{def:ssubt} 
we deduce $\Map{\role}{T} \ppar \prod_{i = 1}^h \Map{\roleq_i}{S_i} \ft$. 
We conclude with an application of \refrule{mtm-par} by taking 
$\MeasureN = \MeasureN' + \sum_{i=1}^h \Measure_i + (0 , \rank{\Map{\role}{S} \ppar \prod_{i = 1}^h \Map{\roleq_i}{S_i}}) \le n$.

\proofrule{sm-cast-swap}
%
Then $P = \pres{s}{\pcast{\ep{t}{\role}}{R} \ppar \procs{P}} \pcong \pcast{\ep{t}{\role}}{\pres{s}{R \ppar \procs{P}}} = Q$ and $t \ne s$.
%
From rule \refrule{mtm-par} we deduce that there exist $\Ctx_i, \roleq_i, \Measure_i$ for $i = 1,\dots,h$ such that
\begin{itemize}
\item $\Ctx = \Ctx_1, \dots, \Ctx_h$
\item $\Measure = \sum_{i=1}^h \Measure_i + (0 , \rank{\prod_{i = 1}^h \Map{\roleq_i}{S_i}})$
\item $\prod_{i = 1}^h \Map{\roleq_i}{S_i} \ft$
\item $\wtpn{\Measure_1}{\Ctx_1, \ep{s}{\roleq_1} : S_1}{\pcast{\ep{t}{\role}}{R}}$
\item $\wtpn{\Measure_i}{\Ctx_i, \ep{s}{\roleq_i} : S_i}{P_i}$ for $i = 2,\dots,h$
\end{itemize} 
From rule \refrule{mtm-cast} we deduce that there exist $\CtxD, T, \Measure', m_t$ such that
\begin{itemize}
\item $\Ctx_1 = \CtxD, \ep{t}{\role} : S$
\item $S \subt[m_t] T$
\item $\Measure_1 = \Measure' + (m_t , 0)$
\item $\wtpn{\Measure'}{\CtxD, \ep{t}{\role} : T, \ep{s}{\roleq_1} : S_1}{R}$
\end{itemize}
We derive 
$\wtpn{\Measure' 
	+ \sum_{i=2}^h \Measure_i 
	+ (0 , \rank{\prod_{i = 1}^h \Map{\roleq_i}{S_i}})}{\CtxD, \ep{t}{\role} : T,\Ctx_2,\dots,\Ctx_h}{\pres{s}{R \ppar \procs{P}}}$ 
with an application of \refrule{mtm-par}. We conclude with an application of \refrule{mtm-cast} by taking $m \eqdef n$.

\proofrule{sm-call} 
%
Then $P = \pinvk{A}{\seqof u} \pcong R\subst{\seqof x}{\seqof u} = Q$ and $\Definition{A}{\seqof x}{R}$.
%
From \refrule{mtm-thread} we deduce that $\wtpn{n}{\Ctx}{\pinvk{A}{\seqof u}}$ for some $n$ such that $\Measure = (n , 0)$. 
Using \Cref{lem:subj_cong_multi} we deduce $\wtp[m]{\Ctx}{Q}$ for some $m \le n$.
Using \Cref{lem:measure_rank_multi} we deduce that $\wtpn{\MeasureN}{\Ctx}{Q}$ for some $\MeasureN \le (m , 0)$. 
We conclude observing that $\MeasureN \le (m , 0) \le (n , 0) = \Measure$.
\end{proof}

%%%%%%%%%%%%%%%%%%%%
%%% NORMAL FORMS %%%
%%%%%%%%%%%%%%%%%%%%

\subsection{Normal Forms}
\label{ssec:nf_bin}
\beginbass
%
In this section we introduce some normal forms that are instrumental to the
soundness proof of the type system. In particular, they allow us to prove
a \emph{quasi} deadlock freedom result which states that a well types process
can be rearranged in such a way that one of the reduction rule is ready to be 
applied. Deadlock freedom is achieved by considering the process in such
shape and the fact that the session under analysis is \emph{compatible}.

To this aim, it is useful to also introduce
\emph{process contexts} as a convenient way of referring to sub-processes. A
process context $\PCtxC$ is essentially a process with a \emph{hole} denoted by
$\Hole$:
\[
	\textbf{Process context}
	\quad
	\PCtxC, \PCtxD ~~::=~~ \Hole \mid \NewPar\x\PCtxC{P} \mid \NewPar\x{P}\PCtxC \mid \Cast\x\PCtxC
\]

As usual, we write $\PCtxC[P]$ for the process obtained by replacing the hole in $\PCtxC$ with $P$. 
Note that this operation may capture channel names that occur free in $P$ and that are bound by $\PCtxC$.

%%%%%%%%%%%%%%%%%%%
%%% DEFINITIONS %%%
%%%%%%%%%%%%%%%%%%%

\begin{definition}[Choice Normal Form]
	\label{def:cnf_bin}
	We say that $P_1 \pchoice P_2$ is an \emph{unguarded choice} of $P$ if there exists 
	$\PCtxC$ such that $P \pcong \PCtxC[P_1 \pchoice P_2]$. 
	We say that $P$ is in \emph{choice normal form} if it has no unguarded choices.
\end{definition}

We introduce a normal form that makes it easier to locate the components 
of a process that may interact with each other. 
Intuitively, a process is in \emph{thread normal form} if it consists of 
an initial prefix of casts followed by a parallel composition of threads, 
where a thread is either $\Done$ or a process waiting to perform an input/output 
action on some channel $x$. In this latter case, we say that the thread is an $x$-thread. 
Note that a process invocation $\Call A {\seqof\x}$ is \emph{not} a thread. Formally:

\begin{definition}[Thread Normal Form]
	\label{def:tnf_bin}
	A process is in \emph{thread normal form} if it is generated by the grammar below:
	\[
		\begin{array}{@{}rcl@{}}
			\Pnf, \Qnf & ::= & \Cast\x\Pnf \mid \Ppar
			\\
			\Ppar, \Qpar & ::= & \NewPar\x\Ppar\Qpar \mid \Pth
			\\
			\Pth & ::= &
			\Done \mid
			\Close\x \mid \Wait\x{P} \mid 
			\x\Pol\set{\l_i : P_i}_{i \in I} \mid
			\POutput\x\y.P \mid \PInput\x{(\y)}.P
		\end{array}
	\] 
\end{definition}

At last, a process in proximity normal form is such that there exist at least two 
$x$-threads that are next to each other. Since each thread is waiting to perform 
an operation on the same session $x$, the two thread may potentially 
reduce if the operations are complementary ones.

\begin{definition}[Proximity Normal Form]
	\label{def:pnf}
	We say that $\Pnf$ is in \emph{proximity normal form} if $\Pnf =
	\PCtxC[\NewPar\x\Pth\Qth]$ for some $\PCtxC$, $x$, $\Pth$ and $\Qth$ where
	$\Pth$ and $\Qth$ are $x$-threads.
\end{definition}

%%%%%%%%%%%%%%
%%% PROOFS %%%
%%%%%%%%%%%%%%

We can reduce any well-typed process into a process that is in choice normal form. 
The fact that the original process is well typed guarantees that this reduction 
eventually terminates when all the unguarded choices have been resolved.

\begin{lemma}
	\label{lem:cnf2_bin}
	If $\wtp[n]\Ctx{P}$ and $\wtpi\Ctx{P}$, then there exists $Q$ in choice
	normal form such that $P \wred Q$ and $\wtp[m]\Ctx{Q}$ for some $m \le n$.
\end{lemma}
\begin{proof}
By induction on $\wtpi\Ctx{P}$ and by cases on the last rule applied.

\proofcase{Case $P$ is already in choice normal form} 
We conclude taking $Q \eqdef P$ and $m \eqdef n$.

\proofrule{tb-call}
Then $P = \pinvk{A}{\seqof{u}}$ and $\Definition{A}{\seqof{x}}{R}$.
We deduce $\Ctx = \seqof{u : S}$, $\tass{A}{\seqof{S}}{n'}$ and $\wtpi\Ctx{R\subst{\seqof u}{\seqof x}}$. Moreover, it must be the case that
$\wtp[n']\Ctx{R\subst{\seqof u}{\seqof x}}$ and $n' \leq n$ since \refrule{tb-call} is used in the coinductive judgment as well.
Using the induction hypothesis we deduce that there exist $Q$ in choice normal form and $m \le n'$ such that $R\subst{\seqof u}{\seqof x} \wred Q$ 
and $\wtp[m]\Ctx{Q}$.
We conclude by observing that $P \wred Q$ using \refrule{rb-struct} and that $m \leq n' \leq n$.

\proofrule{cob-choice}
Then $P = P_1 \pchoice P_2$.
We deduce $\wtpi\Ctx{P_k}$ with $k \in \set{1,2}$.
Moreover, it must be the case that $\wtp[n]\Ctx{P_k}$ since \refrule{tb-choice} is used in the coinductive judgment.
Using the induction hypothesis we deduce that there exist $Q$ in choice normal form and $m \leq n$ such that $P_k \wred Q$ and $\wtp[m]\Ctx{Q}$.
We conclude by observing that $P \red P_k$ by \refrule{rb-choice}.

\proofrule{tb-choice}
Analogous to the previous case but we consider the premise in which the rank is the same of the conclusion to keep sure that it does not increase.

\proofrule{tb-par}
Then $P = \pres{x}{P_1 \parop P_2}$. 
We deduce that there exist $\Ctx_1, \Ctx_2, S_1, S_2$ such that
\begin{itemize}
\item $\Ctx = \Ctx_1, \Ctx_2$
\item $\wtpi{\Ctx_1, x : S_1}{P_1}$
\item $\wtpi{\Ctx_2, x : S_2}{P_2}$
\item $S_1 \compatible S_2$
\end{itemize}
Furthermore, it must be the case that there exist $n_1$ and $n_2$ such that
$\wtp[n_i]{\Ctx_i, x : S_i}{P_i}$ for $i=1,2$ 
and $n = 1 + n_1 + n_2$ since \refrule{tb-par} is used in the coinductive judgment as well.
Using the induction hypothesis we deduce that there exist $Q_i$ in choice normal form and $m_i \leq n_i$ such that 
$P_i \wred Q_i$ and $\wtp[m_i]{\Ctx_i, x : S_i}{Q_i}$ for $i=1,2$.
We conclude by taking $m \eqdef 1 + m_1 + m_2$ and $Q \eqdef \pres{s}{Q_1 \parop Q_2}$ 
with one application of \refrule{tb-par}, observing that 
$m = 1 + + m_1 + m_2 \leq 1 + n_1 + n_2 = n$ and that $P \wred Q$ by \refrule{rb-par}.

\proofrule{tb-cast}
Then $P = \pcast{x} P'$.
Analogous to the previous case, just simpler.
\end{proof}

\begin{lemma}
	\label{lem:cnf1_bin}
	If $\wtp[n]\Ctx{P}$, then there exists $Q$ in choice normal form such that $P \wred Q$ and $\wtp[m]\Ctx{Q}$ for some $m \le n$.
\end{lemma}
\begin{proof}
	\brk
	Consequence of \Cref{lem:cnf2_bin} noting that $\wtp[n]\Ctx{P}$ implies $\wtpi\Ctx{P}$.
\end{proof}

\begin{lemma}
	\label{lem:cnf_exists_bin}
	If\/ $\wtpn\MeasureM\Ctx{P}$, then there exist $Q$ in choice normal form and 
	$\MeasureN \leq \MeasureM$ such that $P \wred Q$ and $\wtpn\MeasureN\Ctx{Q}$.
\end{lemma}
\begin{proof}
By induction on $\wtpn\Measure\Ctx{P}$ and by cases on the last rule applied.

\proofrule{mtb-thread}
Then $P$ is a thread. We deduce that
\begin{itemize}
\item $\Measure = (n , 0)$ for some $n$
\item $\wtp[n]\Ctx{P}$
\end{itemize}
From \Cref{lem:cnf1_bin} we deduce that there exist $Q$ and $m \le n$ such that $P \wred Q$ and $\wtp[m]\Ctx{Q}$. 
From \Cref{lem:measure_rank_bin} we deduce $\wtpn\MeasureN\Ctx{Q}$ for some $\MeasureN \le (m , 0)$. 
We conclude observing that $\MeasureN \le (m , 0) \le (n , 0) = \Measure$.

\proofrule{mtb-cast}
Then $P = \pcast{x}{P'}$. We deduce that
\begin{itemize}
\item $\Ctx = \CtxD, x : S$
\item $S \subt[n] T$
\item $\Measure = \Measure' + (n,0)$
\item $\wtpn{\Measure'}{\Ctx', x : T}{P'}$
\end{itemize}
Using the induction hypothesis we deduce that there exist $Q'$ and $\MeasureN' \le \Measure'$ such that $P' \wred Q'$ 
and $\wtpn{\MeasureN'}{\Ctx', x : T}{Q'}$. We conclude with an application of \refrule{mtb-cast} taking 
$Q \eqdef \pcast{x}{Q'}$, $\MeasureN \eqdef \MeasureN' + (n,0)$ and observing that $P \wred Q$ using \refrule{rb-cast}. 

\proofrule{mtb-par}
Then $P = \pres{s}{P_1 \parop P_2}$. 
We deduce that there exist $\Ctx_1, \Ctx_2, S_1, S_2, \Measure_1, \Measure_2$ such that
\begin{itemize}
\item $\Ctx = \Ctx_1, \Ctx_2$
\item $\wtpn{\Measure_1}{\Ctx_1, x : S_1}{P_1}$
\item $\wtpn{\Measure_2}{\Ctx_2, x : S_2}{P_2}$
\item $\Measure = \Measure_1 + \Measure_2 + (0, \rankb{S_1}{S_2})$
\item $S_1 \compatible S_2$
\end{itemize}
Using the induction hypothesis we deduce that there exist 
$Q_i$ in choice normal form and $\MeasureN_i \leq \Measure_i$ such that $P_i \wred Q_i$ and 
$\wtpn{\MeasureN_i}{\Ctx_i, x : S_i}{Q_i}$ for $i=1,2$.
We conclude by taking $\MeasureN \eqdef \MeasureN_i + \MeasureN_2 + (0, \rankb{S_1}{S_2})$ and 
$Q \eqdef \pres{s}{Q_1 \parop Q_2}$ with one application of \refrule{mtb-par}, 
observing that 
\[
\MeasureN = \MeasureN_1 + \MeasureN_2 + (0, \rankb{S_1}{S_2}) \leq 
						\Measure_1 + \Measure_2 + (0, \rankb{S_1}{S_2}) = \Measure
\]
and that $P \wred Q$ by \refrule{rb-par}.
\end{proof}

It is easy to rewrite any \emph{well-typed} process that is in choice normal 
form into thread normal form using structural pre-congruence. 
The hypothesis that the process is well typed, at least according to the 
inductive interpretation of the typing rules with the corule \refrule{cob-tag},
is necessary to guarantee that a process invocation may eventually be expanded 
to a term other than another process invocation. 
For example, the process $A$ defined by $\Definition{A}{}{A}$ has 
no thread normal form and is ill typed. By combining this result with 
\Cref{lem:subj_cong_bin} we can deduce that the obtained thread normal form is also well typed.

\begin{lemma}
\label{lem:tnf_exists_bin}
	If\/ $\wtpi\Ctx P$ and $P$ is in choice normal form, then there exists
	$\Pnf$ such that $P \pcong \Pnf$. 
\end{lemma}
\begin{proof}
	By induction on $\wtpi\Ctx P$ and by cases on the last rule applied.

	\proofcase{Cases \refrule{tb-choice} and \refrule{co-choice}} 
	%
	These cases are impossible from the hypothesis that $P$ is in choice normal form.

	\proofcase{Cases \refrule{tb-done}, \refrule{tb-wait}, \refrule{tb-close}, 
	\refrule{tb-channel-in}, \refrule{tb-channel-out}, \refrule{tb-label}, \refrule{co-label}}
	%
	Then $P$ is a thread and is already in thread normal form and we conclude by
	reflexivity of $\pcong$.

	\proofrule{tb-call}
	%
	Then there exist $A$, $Q$, $\seqof\x$ and $\seqof\S$ such that
	\begin{itemize}
	\item $P = \Call{A}{\seqof\x}$
	\item $\Definition{A}{\seqof\x}Q$
	\item $\Ctx = \seqof{x : S}$
	\item $\wtpi{\seqof{x : S}}{Q}$
	\end{itemize}

	Using the induction hypothesis on $\wtpi{\seqof{x : S}}{Q}$ 
	we deduce that there exists $\Pnf$ such that $Q \pcong \Pnf$.
	%
	We conclude $P \pcong \Pnf$ using \refrule{sb-call} and the transitivity of $\pcong$.

	\proofrule{tb-par}
	%
	Then there exist $x$, $P_1$, $P_2$, $\Ctx_1$, $\Ctx_2$, $S_1$ and $S_2$ such that
	\begin{itemize}
	\item $P = \NewPar\x{P_1}{P_2}$
	\item $\Ctx = \Ctx_1, \Ctx_2$
	\item $\wtpi{\Ctx_i, x : S_i}{P_i}$ for $i=1,2$
	\end{itemize}

	Using the induction hypothesis on $\wtpi{\Ctx_i, x : S_i}{P_i}$ we deduce 
	that there exists $\Pnf_i$ such that $P_i \pcong \Pnf_i$ for $i=1,2$.
	%
	By definition of thread normal form, it must be the case that 
	$\Pnf_i = \Cast{\seqof{x_i}} \Ppar_i$ for some $\seqof{x_i}$ and $\Ppar_i$.
	Let $\seqof{y_i}$ be the same sequence as $\seqof{x_i}$ except that occurrences of $x$ have been removed.
	%
	We conclude by taking 
	$\Pnf \eqdef \Cast{\seqof{y_1}\seqof{y_2}}\NewPar\x{\Ppar_1}{\Ppar_2}$ and observing that
	\[
		\begin{array}{rcll}
			P & = & \NewPar\x{P_1}{P_2} & \text{by definition of $P$}
			\\
			& \pcong & \NewPar\x{\Pnf_1}{\Pnf_2} & \text{using the induction hypothesis}
			\\
			& = & \NewPar\x{\Cast{\seqof{x_1}}\Ppar_1}{\Cast{\seqof{x_2}}\Ppar_2}
			& \text{by definition of thread normal form}
			\\
			& \pcong & \Cast{\seqof{y_1}\seqof{y_2}}\NewPar\x{\Ppar_1}{\Ppar_2}
			& \text{by \refrule{sb-cast-new},} \\
			& & & \text{\refrule{sb-cast-swap} and \refrule{sb-par-comm}}
			\\
			& = & \Pnf & \text{by definition of $\Pnf$}
		\end{array}
	\]

	\proofrule{tb-cast}
	%
	Then there exist $x$, $Q$, $\Ctx'$, $S$ and $T$ such that
	\begin{itemize}
	\item $P = \Cast\x Q$
	\item $\Ctx = \Ctx', x : S$
	\item $\wtpi{\Ctx', x : T}{Q}$
	\item $S \subt T$
	\end{itemize}

	Using the induction hypothesis on $\wtpi{\Ctx', x : T}{Q}$ we 
	deduce that there exists $\Qnf$ such that $Q \pcong \Qnf$.
	%
	We conclude by taking $\Pnf \eqdef \Cast\x\Qnf$ using the fact that $\pcong$ is a pre-congruence.
\end{proof}

In order to show that every well-typed, closed process in thread normal form can 
also be rewritten in proximity normal form we prove \Cref{lem:proximity_bin}, 
which pushes a restriction $(x)$ next to a process in which $x$ occurs free, 
which might as well be an $x$-thread.

\begin{lemma}[Proximity]
  \label{lem:proximity_bin}
  If $x\in\fn{P} \setminus \bn\PCtxC$, then $\NewPar\x{\PCtxC[P]}{Q}
  \pcong \PCtxD[\NewPar\x{P}{Q}]$ for some $\PCtxD$.
\end{lemma}
\begin{proof}
  By induction on the structure of $\PCtxC$ and by cases on its shape.

  \proofcase{Case $\PCtxC = \Hole$}
  %
  We conclude by taking $\PCtxD \eqdef \Hole$ using the reflexivity
  of $\pcong$.

  \proofcase{Case $\PCtxC = \NewPar\y{\PCtxC'}{R}$}
  %
  From the hypothesis $x \in \fn{P} \setminus \bn\PCtxC$ 
  we deduce $x \ne y$ and $x \in \fn{P} \setminus \bn{\PCtxC'}$.
  %
  Using the induction hypothesis we deduce that there exists $\PCtxD'$ 
  such that $\NewPar\x{\PCtxC'[P]}{Q} \pcong \PCtxD'[\NewPar\x{P}{Q}]$.
  %
  Take $\PCtxD \eqdef \NewPar\y{\PCtxD'}{R}$. We conclude
  \[
    \begin{array}{rcll}
      \NewPar\x{\PCtxC[P]}{Q}
      & = & \NewPar\x{\NewPar\y{\PCtxC'[P]}{R}}{Q}
      & \text{by definition of $\PCtxC$}
      \\
      & \pcong & \NewPar\x{Q}{\NewPar\y{\PCtxC'[P]}{R}}
      & \text{by \refrule{sb-par-comm}}
      \\
      & \pcong & \NewPar\y{\NewPar\x{Q}{\PCtxC'[P]}}{R}
      & \text{by \refrule{sb-par-assoc}} \\
      & & & \text{and $x \in \fn{\PCtxC'[P]}$}
      \\
      & \pcong & \NewPar\y{\NewPar\x{\PCtxC'[P]}{Q}}{R}
      & \text{by \refrule{sb-par-comm}}
      \\
      & \pcong & \NewPar\y{\PCtxD'[\NewPar\x{P}{Q}]}{R}
      & \text{by induction hypothesis}
      \\
      & = & \PCtxD[\NewPar\x{P}{Q}]
      & \text{by definition of $\PCtxD$}
    \end{array}
  \]
  where, in using \refrule{sb-par-assoc}, we note that
  $x \in \fn{\PCtxC'[P]}$ since $x \in \fn{P} \setminus \bn\PCtxC$.

  \proofcase{Case $\PCtxC = \NewPar\y{R}{\PCtxC'}$}
  %
  Symmetric of the previous case.

  \proofcase{Case $\PCtxC = \Cast\y\PCtxC'$ and $x \ne y$}
  %
  Using the induction hypothesis we deduce that there exists $\PCtxD'$ such that
  $\NewPar\x{\PCtxC'[P]}{Q} \pcong \PCtxD'[\NewPar\x{P}{Q}]$.
  %
  Take $\PCtxD \eqdef \Cast\y\PCtxD'$. We conclude
  \[
    \begin{array}{rcll}
      \NewPar\x{\PCtxC[P]}{Q}
      & = & \NewPar\x{\Cast\y\PCtxC'[P]}{Q}
      & \text{by definition of $\PCtxC$}
      \\
      & \pcong & \Cast\y\NewPar\x{\PCtxC'[P]}{Q}
      & \text{by \refrule{sb-cast-swap} and $x \ne y$}
      \\
      & \pcong & \Cast\y\PCtxD'[\NewPar\x{P}{Q}]
      & \text{using the induction hypothesis}
      \\
      & = & \PCtxD[\NewPar\x{P}{Q}]
      & \text{by definition of $\PCtxD$}
    \end{array}
  \]

  \proofcase{Case $\PCtxC = \Cast\x\PCtxC'$}
  %
  Using the induction hypothesis we deduce that there exists $\PCtxD$ such that
  $\NewPar\x{\PCtxC'[P]}{Q} \pcong \PCtxD[\NewPar\x{P}{Q}]$.
  %
  We conclude
  \[
    \begin{array}[b]{rcll}
      \NewPar\x{\PCtxC[P]}{Q}
      & = & \NewPar\x{\Cast\x\PCtxC'[P]}{Q}
      & \text{by definition of $\PCtxC$}
      \\
      & \pcong & \NewPar\x{\PCtxC'[P]}{Q}
      & \text{by \refrule{sb-cast-new}}
      \\
      & \pcong & \PCtxD[\NewPar\x{P}{Q}]
      & \text{using the induction hypothesis}
    \end{array}
    \qedhere
  \]
\end{proof}

We can now prove the fact that every well-typed, closed process in thread normal form can 
be rewritten using structural pre-congruence either to $\Done$ or to a process in proximity normal form. 
Note that this property the one that, combined with the \emph{compatibility} of the involved session,
guarantees deadlock freedom.

\begin{lemma}[Quasi Deadlock Freedom]
	\label{lem:dl_freedom_bin}
	If\/ $\wtpn\Measure\EmptyCtx\Pnf$, then $\Pnf = \Done$ or $\Pnf
	\pcong \Qnf$ for some $\Qnf$ in proximity normal form.
\end{lemma}
\begin{proof}
	A simple induction on the derivation of $\wtpn\MeasureM\EmptyCtx\Pnf$
	allows us to deduce that $\Pnf$ consists of $k$ sessions and $k+1$ threads.
  	%
	If $k = 0$, then we conclude $\Pnf = \Done$.
  	%
  	If $k > 0$, then each of the $k+1$ threads is an $x_i$-thread for some
	$x_i$. Since there are $k+1$ threads but only $k$ distinct sessions, it must
	be the case that $x_i = x_j$ for some $1 \leq i < j \leq k+1$. In other
	words, there exist $\PCtxC$, $\PCtxC_1$, $\PCtxC_2$, $\Pth_1$ and $\Pth_2$ such
	that $\Pth_1$ and $\Pth_2$ are $x$-threads and $\Pnf =
	\PCtxC[\NewPar\x{\PCtxC_1[\Pth_1]}{\PCtxC_2[\Pth_2]}]$.
	%
	We conclude
	\[
		\begin{array}{rcll}
			\Pnf & = & \PCtxC[\NewPar\x{\PCtxC_1[\Pth_1]}{\PCtxC_2[\Pth_2]}]
			& \text{by definition of $\Pnf$}
			\\
			& \pcong & \PCtxC[\PCtxD_1[\NewPar\x{\Pth_1}{\PCtxC_2[\Pth_2]}]]
			& \text{for some $\PCtxD_1$ by \Cref{lem:proximity_bin}}
			\\
			& \pcong & \PCtxC[\PCtxD_1[\NewPar\x{\PCtxC_2[\Pth_2]}{\Pth_1}]]
			& \text{by \refrule{s-par-comm}}
			\\
			& \pcong & \PCtxC[\PCtxD_1[\PCtxD_2[\NewPar\x{\Pth_2}{\Pth_1}]]]
			& \text{for some $\PCtxD_2$ by \Cref{lem:proximity_bin}}
			\\
			& \eqdef & \Qnf
		\end{array}
	\]
	where $x = x_i = x_j$. The fact that $\Qnf$ is in thread normal form follows
	 from the observation that $\Pnf$ does not have unguarded casts (it is a closed process in thread normal form) 
	 so the pre-congruence rules applied here and in \Cref{lem:proximity_bin} do not move casts around. 
	 We conclude that $\Qnf$ is in proximity normal form by its shape.
\end{proof}


%%%%%%%%%%%%%%%%%
%%% SOUNDNESS %%%
%%%%%%%%%%%%%%%%%

\subsection{Soundness}
\beginbass
%
The next key result we prove is inspired by the \emph{helpful direction}. Given a well typed process in
proximity normal form (see \Cref{def:pnf_multi}), we prove that there exists a reduct that is well typed
with a \emph{strictly smaller} measure (see \Cref{lem:pnf_helpful_direction_multi}). 
Notably, this result implies deadlock freedom.
%
Then, it is easy to prove that a well typed process is either $\pdone$ or it can reach $\pdone$ in finitely
many steps by induction over the measure.

\begin{lemma}
	\label{lem:pnf_helpful_direction_multi}
	If $\wtpn\Measure\Ctx\Pnf$ where $\Pnf$ is in proximity normal form, then there exist $Q$ and $\MeasureN < \Measure$ 
	such that $\Pnf \wred^+ Q$ and $\wtpn\MeasureN\Ctx{Q}$.
\end{lemma}
\begin{proof}
From the hypothesis that $\Pnf$ is in proximity normal form we know that 
$\Pnf = \PCtxC[\pres{s}{\Pth_1 \parop \cdots \parop \Pth_h}]$ for some $\PCtxC$, $s$ and $\Pth_1, \dots, \Pth_h$ $s$-threads.
We reason by induction on $\PCtxC$ and by cases on its shape.
	
\proofcase{Case $\PCtxC = \Hole$}
%
From \refrule{mtm-thread} and \refrule{mtm-par} we deduce that there exist $\Ctx_i, S_i, \role_i, n_i$ for $i=1,\dots,h$ such that
\begin{itemize}
\item $\Ctx = \Ctx_1,\dots,\Ctx_h$
\item $\prod_{i=1}^h \Map{\role_i}{S_i} \ft$
\item $\Measure = (\sum_{i=1}^h n_i , \rank{\prod_{i=1}^h \Map{\role_i}{S_i}})$
\item $\wtp[n_i]{\Ctx_i, \ep{s}{\role_i} : S_i}{\Pth_i}$ for $i=1,\dots,h$
\end{itemize}
From the hypothesis that $\prod_{i=1}^h \Map{\role_i}{S_i} \ft$ we deduce $\prod_{i=1}^h \Map{\role_i}{S_i} \wlred{\In\terminated}$. 
We now reason on the rank of the session and on the shape of $S_i$. 
For the sake of simplicity, we implicitly apply \refrule{s-par-comm} at process level.\\\\
%
If $\rank{\prod_{i=1}^h \Map{\role_i}{S_i}} = 1$, then $\prod_{i=1}^h \Map{\role_i}{S_i} \lred{\In\terminated}$ using \refrule{lm-terminate}.
%
\begin{itemize}
\item \proofcase{Case $S_1 = \End[\In]$ and $S_j = \End[\Out]$ for $j=2,\dots,h$}
%
Then 
	\begin{itemize}
	\item $\Ctx_j = \EmptyCtx$ and $\Pth_j = \pclose{\ep{s}{\role_j}}$ for $j=2,\dots,h$
	\item $\Pth_1 = \pwait{\ep{s}{\role_1}}{Q}$
	\item $\wtp[n_1]{\Ctx_1}{Q}$
	\end{itemize}
From \Cref{lem:measure_rank_multi} we deduce that $\wtpn{\MeasureN}{\Ctx_1}{Q}$ for some $\MeasureN \le (n_1 , 0)$. 
We conclude observing that $\Pnf \red Q$ by \refrule{rm-signal} and that 
$\MeasureN \le (n_1 , 0) < (\sum_{i=1}^h n_i , \rank{\prod_{i=1}^h \Map{\role_i}{S_i}}) = \Measure$.
\end{itemize}
%
If $\rank{\prod_{i=1}^h \Map{\role_i}{S_i}} > 1$, then $\prod_{i=1}^h \Map{\role_i}{S_i} \lred{\tau} \dots \lred{\In\terminated}$ 
first using \refrule{lm-tau} and \refrule{lm-sync}. 
Observe that \refrule{lm-pick} is never used since we are considering the minimum reduction sequence; 
a synchronization through \refrule{lm-pick} and \refrule{lm-sync} would lead to a longer reduction. T
hen $S_1 \xlred{\Map{\role_1}{\role_2\Out\Tag_k}}$ and $S_2 \xlred{\Map{\role_2}{\role_1\In\Tag_k}}$ for some 
$\Tag_k$ or $S_1 \xlred{\Map{\role_1}{\role_2}\Out S}$ and $S_2 \xlred{\Map{\role_1}{\role_2}\In S}$.
%
\begin{itemize}
\item \proofcase{Case $S_1 = \Tags\role_2\Out \Tag_i.S'_i$ and $S_2 = \JTags\role_1\In \Tag_j.T_j$ with $k \in I$}
%
From the hypothesis that $\prod_{i=1}^h \Map{\role_i}{S_i} \ft$ we deduce $I \subseteq J$. 
From \Cref{def:coherence} we deduce 
$\prod_{i = 3}^h \Map{\role_i}{S_i} \parop \Map{\role_1}{S'_{k}} \parop \Map{\role_2}{T_{k}} \ft$ 
and from \refrule{tm-tag} we deduce that 
	\begin{itemize}
	\item $\Pth_1 = \pbranch[i \in I]{\ep{s}{\role_1}}{\role_2}{\Out}{\Tag_i}{P'_i}$
	\item $\Pth_2 = \pbranch[j \in J]{\ep{s}{\role_2}}{\role_1}{\In}{\Tag_j}{Q_j}$
	\item $\wtp[n_1]{\Ctx_1, \ep{s}{\role_1} : S'_i}{P'_i}$ for all $i \in I$
	\item $\wtp[n_2]{\Ctx_2, \ep{s}{\role_2} : T_j}{Q_j}$ for all $j \in J$
	\end{itemize}
Let $Q \eqdef \pres{s}{P'_k \parop Q_k \parop P_3 \parop \dots \parop P_h}$ and observe that $\Pnf \wred^+ Q$ by \refrule{rm-pick} and \refrule{rm-tag}. 
From \Cref{lem:measure_rank_multi} we deduce that there exist $\Measure_1 \le (n_1 , 0), \Measure_2 \le (n_2 , 0)$ such that
	\begin{itemize}
	\item $\wtpn{\Measure_1}{\Ctx_1, \ep{s}{\role_1} : S'_k}{P'_k}$
	\item $\wtpn{\Measure_2}{\Ctx_2, \ep{s}{\role_2} : T_k}{Q_k}$
	\end{itemize}
Let 
$\MeasureN \eqdef \Measure_1 
	+ \Measure_2 
	+ (\sum_{i=3}^h n_i, \rank{\Map{\role_1}{S'_k} \parop \Map{\role_2}{T_k} \parop \prod_{i=3}^h \Map{\role_i}{S_i}})$. 
We conclude with one application of \refrule{mtm-par} observing that
\[
\begin{array}{rcll}
	\MeasureN & = & \Measure_1 + \Measure_2 + \\
	& & (\sum_{i=3}^h n_i, \rank{\Map{\role_1}{S'_k} \parop \Map{\role_2}{T_k} \parop \prod_{i=3}^h \Map{\role_i}{S_i}})
	& \text{by def. of $\MeasureN$}
	\\
	& \le & (\sum_{i=1}^h n_i, \rank{\Map{\role_1}{S'_k} \parop \Map{\role_2}{T_k} \parop \prod_{i=3}^h \Map{\role_i}{S_i}})
	& \text{by \Cref{lem:measure_rank_multi}}
	\\
	& < & (\sum_{i=1}^h n_i, \rank{\prod_{i=1}^h \Map{\role_i}{S_i}})
	& \text{before $\red$}
	\\
	& = & \Measure
\end{array}
\]
\end{itemize}
%
\begin{itemize}
\item \proofcase{Case $S_1 = \role_2\Out{S}.T_1$ and $S_2 = \role_1\In{S}.T_2$}
\end{itemize}
%
From the hypothesis that $\prod_{i=1}^h \Map{\role_i}{S_i} \ft$ and \Cref{def:coherence} we deduce 
$\Map{\role_1}{T_1} \parop \Map{\role_2}{T_2} \parop \prod_{i=3}^h \Map{\role_i}{S_i} \ft$ and from 
\refrule{tm-channel-out} and \refrule{tm-channel-in} we deduce that
	\begin{itemize}
	\item $\Pth_1 = \poch{\ep{s}{\role_1}}{\role_2}{u}{P'_1}$
	\item $\Pth_2 = \pich{\ep{s}{\role_2}}{\role_1}{x}{P'_2}$
	\item $\wtp[n_1]{\Ctx_1, \ep{s}{\role_1} : T_1}{P'_1}$
	\item $\wtp[n_2]{\Ctx_2, \ep{s}{\role_2} : T_2, x : S}{P'_2}$
	\end{itemize}
Let $Q \eqdef \pres{s}{P'_1 \parop P'_2\subst{u}{x} \parop \Pth_3 \parop \cdots \parop \Pth_h}$ and observe that $\Pnf \red Q$ by \refrule{rm-channel}. 
Using \Cref{lem:substitution_multi} we deduce $\wtp[n_2]{\Ctx_2, \ep{s}{\role_2} : T_2, u : S}{P'_2\subst{u}{x}}$ and 
from \Cref{lem:measure_rank_multi} we deduce that there exist $\Measure_1 \le (n_1 , 0), \Measure_2 \le (n_2 , 0)$ such that
	\begin{itemize}
	\item $\wtpn{\Measure_1}{\Ctx_1, \ep{s}{\role_1} : T_1}{P'_1}$
	\item $\wtpn{\Measure_2}{\Ctx_2, \ep{s}{\role_2} : T_2, u : S}{P'_2\subst{u}{x}}$
	\end{itemize}
Let $\MeasureN \eqdef \Measure_1 
	+ \Measure_2 
	+ (\sum_{i=3}^h n_i, \rank{\Map{\role_1}{T_1} \parop \Map{\role_2}{T_2} \parop \prod_{i=3}^h \Map{\role_i}{S_i}})$. 
We conclude with one application of \refrule{mtm-par} observing that
\[
\begin{array}{rcll}
	\MeasureN & = & \Measure_1 + \Measure_2 + \\
	& & (\sum_{i=3}^h n_i, \rank{\Map{\role_1}{T_1} \parop \Map{\role_2}{T_2} \parop \prod_{i=3}^h \Map{\role_i}{S_i}})
	& \text{by definition of $\MeasureN$}
	\\
	& \le & (\sum_{i=1}^h n_i, \rank{\Map{\role_1}{T_1} \parop \Map{\role_2}{T_2} \parop \prod_{i=3}^h \Map{\role_i}{S_i}})
	& \text{by \Cref{lem:measure_rank_multi}}
	\\
	& < & (\sum_{i=1}^h n_i, \rank{\prod_{i=1}^h \Map{\role_i}{S_i}}) 
	& \text{before reductions}
	\\
	& = & \Measure
\end{array}
\]

\proofcase{Case $\PCtxC = \pres{t}{\procs{\Ppar} \parop \PCtxD \parop \procs{\Qpar}}$}
%
Let $\Rnf \eqdef \PCtxD[\pres{s}{\Pth_1 \parop \dots \parop \Pth_h}]$ and observe that $\Rnf$ is in proximity normal form. 
From \refrule{mtm-par} we deduce that there exist $\Ctx_i, S_i, \Measure_i, \role_i$ for $i=1,\dots,h$ and $k \le h$ such that
\begin{itemize}
\item $\Ctx = \Ctx_1,\dots,\Ctx_h$
\item $\wtpn{\Measure_1}{\Ctx_1, \ep{t}{\role_1} : S_1}{\Rnf}$
\item $\wtpn{\Measure_i}{\Ctx_i, \ep{t}{\role_i} : S_i}{\Ppar_i}$ for $i=1,\dots,k$
\item $\wtpn{\Measure_i}{\Ctx_i, \ep{t}{\role_i} : S_i}{\Qpar_i}$ for $i=k+1,\dots,h$
\item $\Measure = \sum_{i=1}^h \Measure_i + (0 , \rank{\prod_{i=1}^h \Map{\role_i}{S_i}})$
\end{itemize}
Using the induction hypothesis on $\wtpn{\Measure_1}{\Ctx_1, \ep{t}{\role_1} : S_1}{\Rnf}$ we deduce that there 
exists $Q'$ and $\MeasureN' < \Measure_1$ such that
\begin{itemize}
\item $\Rnf \wred^+ Q'$
\item $\wtpn{\MeasureN'}{\Ctx_1, \ep{t}{\role_1} : S_1}{Q'}$
\end{itemize}
We conclude taking $Q \eqdef \pres{t}{Q' \parop \procs\Ppar \parop \procs\Qpar}$ and
\[
	\MeasureN \eqdef \MeasureN' + \sum_{i=2}^h \Measure_i + (0 , \rank{\prod_{i=1}^h \Map{\role_i}{S_i}})
\] 
and observing that $\MeasureN < \Measure$ and $\Pnf \wred^+ Q$ by \refrule{rm-par}.

\proofcase{Case $\PCtxC = \pcast{\ep{t}{\roleq}}{\PCtxD}$}
%
Observe that $t \ne s$. Let $\Rnf \eqdef \PCtxD[\pres{s}{\Pth_1 \parop \cdots \parop \Pth_h}]$ 
and note that $\Rnf$ is in proximity normal form. 
From \refrule{mtm-cast} we deduce that there exists $\CtxD, \Measure', S, T, m_t$ such that
\begin{itemize}
\item $\Ctx = \CtxD, \ep{t}{\roleq} : S$
\item $S \subt[m_t] T$
\item $\Measure = \Measure' + (m_t,0)$
\item $\wtpn{\Measure'}{\CtxD, \ep{t}{\roleq} : T}{\Rnf}$
\end{itemize}
Using the induction hypothesis on $\wtpn{\Measure'}{\CtxD, \ep{t}{\roleq} : T}{\Rnf}$ we deduce that 
there exist $Q'$ and $\MeasureN' < \Measure'$ such that $\Rnf \wred^+ Q'$ and $\wtpn{\MeasureN'}{\CtxD, \ep{t}{\roleq} : T}{Q'}$. 
We conclude taking $Q \eqdef \pcast{\ep{t}{\roleq}}{Q'}$ and $\MeasureN \eqdef \MeasureN' + (m_t , 0)$ 
and observing that $\MeasureN < \Measure$ and $\Pnf \wred^+ Q$ by \refrule{rm-cast}.
\end{proof}

\begin{lemma}
	\label{lem:helpful_direction_multi}
	If $\wtpn{\Measure}{\EmptyCtx}{P}$, then either $P \pcong \pdone$ or $P \wred^+ Q$ 
	and $\wtpn{\MeasureN}{\EmptyCtx}{Q}$ for some $Q$ and $\MeasureN < \Measure$.
\end{lemma}
\begin{proof}
Using \Cref{lem:cnf_exists_multi} we deduce that there exist $P'$ in choice normal form such that
$P \wred P'$ and $\wtpn{\MeasureM'}\EmptyCtx{P'}$ and $\MeasureM' \leq \MeasureM$.
By \Cref{lem:measure_rank_multi} we deduce $\wtp\EmptyCtx{P'}$. 
Using \Cref{lem:tnf_exists_multi} we deduce that there exist $\Pnf$ such that $P' \pcong \Pnf$.

If $\Pnf = \pdone$ there is nothing left to prove.

If $\Pnf \neq \pdone$, by \Cref{lem:dl_freedom_multi} we deduce $\Pnf \pcong \Qnf$ for some $\Qnf$ in proximity normal form.
From \Cref{lem:measure_pcong_multi} we deduce $\wtpn{\MeasureM''}\EmptyCtx\Qnf$ for some $\MeasureM'' \leq \MeasureM'$.
Using \Cref{lem:pnf_helpful_direction_multi} we conclude that $\Qnf \wred^+ Q$ and $\wtpn\MeasureN\EmptyCtx Q$ 
for some $Q$ and $\MeasureN < \MeasureM'' \leq \MeasureM' \leq \MeasureM$.
\end{proof}

\begin{lemma}
	\label{lem:weak_termination_multi}
	If $\wtp[n]\EmptyCtx{P}$, then either $P \pcong \pdone$ or $P \wred^+ \pdone$.
\end{lemma}
\begin{proof}
From \Cref{lem:measure_rank_multi} we deduce that there exists $\MeasureM \le (n,0)$ such that $\wtpn\MeasureM\EmptyCtx P$.
We proceed doing an induction on the lexicographically ordered pair $\MeasureM$.
%
From \Cref{lem:helpful_direction_multi} we deduce either $P \pcong \pdone$ or $P \wred^+ Q$ 
and $\wtpn\MeasureN\EmptyCtx{Q}$ for some $\MeasureN < \MeasureM$.
%
In the first case there is nothing left to prove.
%
In the second case we use the induction hypothesis to deduce that either $Q \pcong \pdone$ or $Q \wred^+ \pdone$.
%
We conclude using either \refrule{rm-struct} or the transitivity of $\wred^+$, respectively.
\end{proof}

\begin{proof}[Proof of \Cref{thm:ts_multi_sound}]
	\brk
	Immediate consequence of \Cref{lem:subj_red_multi,lem:weak_termination_multi}.
\end{proof}