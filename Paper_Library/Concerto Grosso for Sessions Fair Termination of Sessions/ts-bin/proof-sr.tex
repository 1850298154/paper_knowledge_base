\beginbass
%
We first prove that typing is preserved by structural pre-congruence.
This lemma is required to prove subject reduction when dealing with
the reduction under structural pre-congruence (\refrule{rb-struct}).

\begin{lemma}
  \label{lem:subj_cong_bin}
  If\/ $\wtp[n]\Ctx{P}$ and $P \pcong Q$, then $\wtp[m]\Ctx{Q}$ for some $m \leq n$.
\end{lemma}
\begin{proof}
  The proof is by induction on the derivation of $P \pcong Q$ and by cases on
  the last rule applied. We only discuss a few representative cases, the
  remaining ones are analogous.

  \proofrule{sb-par-comm}
  %
  Then $P = \NewPar\x{P_1}{P_2} \pcong \NewPar\x{P_2}{P_1} = Q$.
  %
  From \refrule{tb-par} we deduce that there exist $\Ctx_1$,
  $\Ctx_2$, $x$, $S_1$, $S_2$, $n_1$ and $n_2$ such that:
  \begin{itemize}
  \item $\Ctx = \Ctx_1, \Ctx_2$
  \item $\wtp[n_i]{\Ctx_i, x : S_i}{P_i}$ for $i=1,2$
  \item $S_1 \compatible S_2$
  \item $n = 1 + n_1 + n_2$  
  \end{itemize}

  We conclude $\wtp\Ctx{Q}$ with one application of \refrule{tb-par} by taking $m \eqdef n$.

  \proofrule{sb-par-assoc}
  %
  Then $P = \NewPar\x{P_1}{\NewPar\y{P_2}{P_3}} \pcong
  \NewPar\y{\NewPar\x{P_1}{P_2}}{P_3} = Q$ and $x \in \fn{P_2}$.
  %
  From \refrule{tb-par} we deduce that there exist $\Ctx_1$, $\Ctx_{23}$,
  $T_1$, $S_1$, $n_1$ and $n_{23}$ such that:
  \begin{itemize}
  \item $\Ctx = \Ctx_1, \Ctx_{23}$
  \item $\wtp[n_1]{\Ctx_1, x : T_1}{P_1}$
  \item $\wtp[n_{23}]{\Ctx_{23}, x : S_1}{\NewPar\y{P_2}{P_3}}$
  \item $T_1 \compatible S_1$
  \item $n = 1 + n_1 + n_{23}$
  \end{itemize}

  From \refrule{tb-par} we deduce that there exist $\Ctx_2$, $\Ctx_3$,
  $T_2$, $S_2$, $n_2$ and $n_3$ such that:
  \begin{itemize}
  \item $\Ctx_{23} = \Ctx_2, \Ctx_3$
  \item $\wtp[n_2]{\Ctx_2, x : S_1, y : T_2}{P_2}$
  \item $\wtp[n_3]{\Ctx_3, y : S_2}{P_3}$
  \item $T_2 \compatible S_2$
  \item $n_{23} = 1 + n_2 + n_3$
  \end{itemize}

  Using \refrule{tb-par} we derive $\wtp[1 + n_1 + n_2]{\Ctx_1, \Ctx_2, y
  : T_2}{\NewPar\x{P_1}{P_2}}$.
  %
  We conclude $\wtp[m]\Ctx{\NewPar\y{\NewPar\x{P_1}{P_2}}{P_3}}$ with another application of \refrule{tb-par} by taking $m \eqdef n$.

  \proofrule{sb-cast-new}
  %
  Then $P = \NewPar\x{\Cast\x P_1}{P_2} \pcong \NewPar\x{P_1}{P_2}$.
  %
  From \refrule{tb-par} we deduce that there exist $\Ctx_1$, $\Ctx_2$, $S_1$, $T$, $n_1$ and $n_2$ such that:
  \begin{itemize}
  \item $\Ctx = \Ctx_1, \Ctx_2$
  \item $\wtp[n_1]{\Ctx_1, x : S_1}{\Cast\x P_1}$
  \item $\wtp[n_2]{\Ctx_2, x : T}{P_2}$
  \item $S_1 \compatible T$
  \item $n = 1 + n_1 + n_2$
  \end{itemize}

  From \refrule{tb-cast} we deduce that there exist $S_2$, $n_3$, $m_x$ such that
  \begin{itemize}
  \item $\wtp[n_3]{\Ctx_1, x : S_2}{P_1}$
  \item $S_1 \subt[m_x] S_2$
  \item $n_1 = m_x + n_3$
  \end{itemize}

  From $S_1 \compatible T$ and $S_1 \subt S_2$ and \Cref{thm:fsub_sound} we
  deduce $S_2 \compatible T$. We conclude with one application of
  \refrule{tb-par} by taking $m \eqdef 1 + n_3 + n_2$ and observing that
  \[
  	m \eqdef 1 + n_3 + n_2 \le 1 + m_x + n_3 + n_2 = 1 + n_1 + n_2 = n
  \]

  \proofrule{sb-cast-swap}
  %
  Then $P = \NewPar\x{\Cast\y P_1}{P_2} \pcong \Cast\y\NewPar\x{P_1}{P_2}$ and $\x \ne \y$.
  %
  From \refrule{tb-par} we deduce that there exist $\Ctx_1$, $S_1$, $S_2$,
  $T_1$, $n_1$ and $n_2$ such that:
  \begin{itemize}
  \item $\Ctx = \Ctx_1, \Ctx_2, y : T_1$
  \item $\wtp[n_1]{\Ctx_1, x : S_1, y : T_1}{\Cast\y P_1}$
  \item $\wtp[n_2]{\Ctx_2, x : S_2}{P_2}$
  \item $S_1 \compatible S_2$
  \item $n = 1 + n_1 + n_2$
  \end{itemize}

  From \refrule{tb-cast} we deduce that there exist $T_2$, $n_3$ and $m_y$ such that
  \begin{itemize}
  \item $T_1 \subt[m_y] T_2$
  \item $\wtp[n_3]{\Ctx_1, x : S_1, y : T_2}{P_1}$
  \item $n_1 = m_y + n_3$
  \end{itemize}

  We derive $\wtp[1 + n_3 + n_2]{\Ctx_1, \Ctx_2, y :
  T_2}{\NewPar\x{P_1}{P_2}}$ with one application of \refrule{tb-par} and we
  conclude with one application of \refrule{tb-cast} by taking $m \eqdef n$.

  \proofrule{sb-cast-comm}
  %
  Then $P = \Cast\x\Cast\y P' \pcong \Cast\y\Cast\x P' = Q$. We can assume $x \ne y$ or else $P = Q$ and there is nothing to prove.
  %
  From \refrule{tb-cast} we deduce that there exist $\Ctx_1$, $S_1$, $S_2$, $n_1$ and $m_x$ such that
  \begin{itemize}
  \item $\Ctx = \Ctx_1, x : S_1$
  \item $\wtp[n_1]{\Ctx_1, x : S_2}{\Cast\y P'}$
  \item $S_1 \subt[m_x] S_2$
  \item $n = m_x + n_1$
  \end{itemize}
  
  From \refrule{tb-cast} and the hypothesis $x \neq y$ we deduce that there exist $\Ctx_2$, $T_1$, $T_2$, $n_2$ and $m_y$ such that
  \begin{itemize}
  \item $\Ctx_1 = \Ctx_2, y : T_1$
  \item $\wtp[n_2]{\Ctx_2, x : S_2, y : T_2}{P'}$
  \item $T_1 \subt[m_y] T_2$
  \item $n_1 = m_y + n_2$
  \end{itemize}

  We derive $\wtp[n_1]{\Ctx_2, x : S_1, y : T_2}{\Cast\x P'}$ with one application of 
  \refrule{tb-cast} and we conclude with another application of \refrule{tb-cast} by taking $m \eqdef n$.

  \proofrule{sb-call}
  %
  Then $P = \Call{A}{\seqof\x} \pcong Q$ and $\Definition{A}{\seqof\x}Q$.
  %
  From \refrule{tb-call} we conclude that there exist $\seqof\S$ and $m$ such
  that $\tass{A}{\seqof{S}}{m}$ and $\Ctx = \seqof{x : S}$ and
  $\wtp[m]{\seqof{x : S}}{Q}$ and $m \leq n$.
\end{proof}

Then we have subject reduction, stating that typing is preserved also by reductions. 
Note that in this case we are not able to establish a general relation between the rank 
of the reducible process and that of the reduct. In particular, the rank may increase.

\begin{lemma}[subject reduction]
  \label{lem:subj_red_bin}
  If\/ $\wtp[n]\Ctx{P}$ and $P \red Q$, then $\wtp[m]\Ctx{Q}$ for some $m$.
\end{lemma}
\begin{proof}
  By induction on the derivation of $P \red Q$ and by cases on the last rule
  applied.

  \proofrule{rb-choice}
  %
  Then $P = P_1 \pchoice P_2 \red P_k = Q$ where $k\in\set{1,2}$.
  %
  From \refrule{tb-choice} we deduce that $\wtp[m]\Ctx{Q}$ for some $m$, which is all we need to conclude.

  \proofrule{rb-pick}
  %
  Then $P = \NewPar\x{\PSend\x{\l_i : P_i}_{i\in I}}{R} \red
  \NewPar\x{\POutput\x\l_k.P_k}{R} = Q$ where $k\in I$ and $|I| > 1$.
  %
  From \refrule{tb-par} we deduce that there exist $\Ctx_1$, $\Ctx_2$, $S$, $T$, $n_1$ and $n_2$ such that
  \begin{itemize}
  \item $\Ctx = \Ctx_1, \Ctx_2$
  \item $\wtp[n_1]{\Ctx_1, x : S}{\PSend\x{\l_i : P_i}_{i\in I}}$
  \item $\wtp[n_2]{\Ctx_2, x : T}{R}$
  \item $S \compatible T$
  \item $n = 1 + n_1 + n_2$
  \end{itemize}

  From \refrule{tb-tag} we deduce that there exists a family $S_{i\in I}$ such
  that:
  \begin{itemize}
  \item $S = \Choice{\l_i : S_i}_{i\in I}$
  \item $\wtp[n_1]{\Ctx_1, x : S_i}{P_i}$ for every $i\in I$
  \end{itemize}

  From $S \compatible T$ and $S \red \Out\l_k.S_k$ and
  \Cref{def:compatibility} we deduce $\Out\l_k.S_k \compatible T$.
  %
  We conclude with one application of \refrule{tb-tag} and one application of \refrule{tb-par} by taking $m \eqdef n$.
 	
  \proofrule{rb-signal}
  %
  Then $P = \NewPar\x{\Close\x}{\Wait\x{Q}} \red Q$.
  %
  From \refrule{tb-par}, \refrule{tb-close} and \refrule{tb-wait} we deduce that there exist $n'$ and $m$ such that:
  \begin{itemize}
  \item $\wtp[n']{x : \End[\Out]}{\Close\x}$
  \item $\wtp[m]{\Ctx, x : \End[\In]}{\Wait\x{Q}}$
  \item $\wtp[m]\Ctx{Q}$
  \item $n = 1 + n' + m$
  \end{itemize}

  There is nothing left to prove.

  \proofrule{rb-label}
  %
  Then $P = \NewPar\x{\POutput\x\l_k.R}{\PRecv\x{\l_i : Q_i}_{i \in I}} \red
  \NewPar\x{R}{Q_k} = Q$ with $k \in I$.
  %
  From \refrule{tb-par} we deduce that there exist $\Ctx_1$, $\Ctx_2$,
  $S$, $T$, $n_1$ and $n_2$ such that:
  \begin{itemize}
  \item $\Ctx = \Ctx_1, \Ctx_2$
  \item $\wtp[n_1]{\Ctx_1, x : S}{\POutput\x\l_k.R}$
  \item $\wtp[n_2]{\Ctx_2, x : T}{\PRecv\x{\l_i : Q_i}_{i \in I}}$
  \item $S \compatible T$
  \item $n = 1 + n_1 + n_2$
  \end{itemize}

  From \refrule{tb-tag} we deduce that there exists $S_1$ such that
  $S = \Out\l_k. S_1$ and $\wtp[n_1]{\Ctx_1, x : S_1}{R}$.
  %
  From \refrule{tb-tag} we deduce that there exists a family $T_{i\in I}$ such
  that:
  \begin{itemize}
  \item $T = \Branch{\l_i : T_i}_{i \in I}$
  \item $\wtp[n_2]{\Ctx_2, x : T_i}{Q_i}$ for every $i\in I$
  \end{itemize}

  From $S \compatible T$ we deduce $S_1 \compatible T_k$.
  %
  We conclude with one application of \refrule{tb-par} by taking $m \eqdef n$.

  \proofrule{rb-channel}
  %
  Then $P = \NewPar\x{\POutput\x\y.P'}{\PInput\x{(\y)}.Q'} \red
  \NewPar\x{P'}{Q'} = Q$.
  %
  From \refrule{tb-par} we deduce that there exist $\Ctx_1$, $\Ctx_2$, $S$, $T$, $n_1$ and $n_2$ such that
  \begin{itemize}
  \item $\Ctx = \Ctx_1, \Ctx_2$
  \item $\wtp[n_1]{\Ctx_1, x : S}{\POutput\x\y.P'}$
  \item $\wtp[n_2]{\Ctx_2, x : T}{\PInput\x{(\y)}.Q'}$
  \item $S \compatible T$
  \item $n = 1 + n_1 + n_2$
  \end{itemize}

  From \refrule{tb-channel-out} we deduce that there exist $\Ctx_1'$, $S_1$ and $S_2$ such that
  \begin{itemize}
  \item $\Ctx_1 = \Ctx_1', y : S_1$
  \item $S = \Out\S_1.S_2$
  \item $\wtp[n_1]{\Ctx_1', x : S_2}{P'}$
  \end{itemize}

  From \refrule{tb-channel-in} we deduce that there exist $T_1$ and $T_2$ such that
  \begin{itemize}
  \item $T = \In\T_1.T_2$
  \item $\wtp[n_2]{\Ctx_2, x : T_2, y : T_1}{Q'}$
  \end{itemize}

  From $S \compatible T$ we deduce $S_1 = T_1$ and $S_2 \compatible T_2$.
  %
  We conclude with one application of \refrule{tb-par} by taking $m \eqdef n$.

  \proofrule{rb-cast}
  %
  Then $P = \Cast\x P' \red \Cast\x Q' = Q$ and $P' \red Q'$.
  %
  From \refrule{tb-cast} we deduce that there exist $\Ctx'$, $S$, $T$, $n'$ 
  and $m_x$ such that
  \begin{itemize}
  \item $\Ctx = \Ctx', x : S$
  \item $\wtp[n']{\Ctx', x : T}{P'}$
  \item $S \subt[m_x] T$
  \item $n = m_x + n'$
  \end{itemize}

  Using the induction hypothesis on $\wtp[n']{\Ctx', x : T}{P'}$ and $P'
  \red Q'$ we derive $\wtp[m']{\Ctx', x : T}{Q'}$ for some $m'$. We conclude
  with one application of \refrule{tb-cast} by taking $m \eqdef m_x + m'$.
  
  \proofrule{rb-par}
  %
  Then $P = \NewPar\x{P'}{R} \red \NewPar\x{Q'}{R} = Q$ and $P' \red Q'$.
  %
  From \refrule{tb-par} we deduce that there exist $\Ctx_1$, $\Ctx_2$,
  $S$, $T$, $n_1$ and $n_2$ such that:
  \begin{itemize}
  \item $\Ctx = \Ctx_1, \Ctx_2$
  \item $\wtp[n_1]{\Ctx_1, x : S}{P'}$
  \item $\wtp[n_2]{\Ctx_2, x : T}{R}$
  \item $S \compatible T$
  \item $n = 1 + n_1 + n_2$
  \end{itemize}

  Using the induction hypothesis we deduce that $\wtp[m_1]{\Ctx_1, x :
  S}{Q'}$ for some $m_1$. We conclude with one application of \refrule{tb-par} by
  taking $m \eqdef m_1 + n_2$.

  \proofrule{rb-struct}
  %
  Then $P \pcong P' \red Q' \pcong Q$.
  %
  From \Cref{lem:subj_cong_bin} we deduce $\wtp[n']\Ctx{P'}$ for some $n' \leq n$.
  %
  Using the induction hypothesis we deduce $\wtp[m']\Ctx{Q'}$ for some $m'$.
  %
  We conclude using \Cref{lem:subj_cong_bin} once more.
\end{proof}