\beginalto
%
In this section we present the type system for binary sessions that we introduced in \Cref{sec:ts_bin_proc}.
%
Before looking at the typing rules we motivate, through a series of examples, the key properties enforced by the type system that, 
taken together, guarantee fair termination. There are two families of problems that can compromise fair termination.
%
First of all, the process (or part thereof) may be unable to reduce further but is not $\Done$. 
In our model, this can happen for many reasons, for example: a process attempts at sending a 
tag on a session that the receiver is not willing to accept; a process attempts at sending a 
termination signal when the receiver expects a channel; the processes at the two ends of the same 
session are both waiting for a message from that session. These are all examples of \emph{safety violations}, 
which are prevented by any ordinary session type system. 
%
In \Cref{ssec:boundedness} we focus instead on \emph{liveness violations}. Roughly
speaking, liveness is violated when a process (or part thereof) engages an
infinite computation that cannot possibly terminate.

%%%%%%%%%%%%%%%%%%%
%%% BOUNDEDNESS %%%
%%%%%%%%%%%%%%%%%%%

\subsection{Boundedness Properties}
\label{ssec:boundedness}
\beginbass
%
In \Cref{ch:fs}
we have introduced a fair subtyping relation that is liveness preserving but,
as we will see in a moment, the adoption of fair subtyping alone is not enough
to rule out all potential liveness violations. The type system must also enforce
three properties that we call \emph{action boundedness}, \emph{session
boundedness} and \emph{cast boundedness} guaranteeing that the overall effort
required to terminate the process is finite. In the rest of the section we
describe informally these properties and we show that violating even just one of
them may compromise fair process termination.

%%%%%%%%%%%%%%
%%% ACTION %%%
%%%%%%%%%%%%%%

\begin{definition}[Action Boundedness]
	We say that a process is \emph{action bounded} if there is a finite upper bound
	to the number of actions it has to perform in order to terminate. An
	action-unbounded process cannot terminate.
\end{definition}

\begin{example}
	\label{ex:action_boundedness}
	Compare the following processes
	\[
  \begin{array}{ll}
    \Definition{A}{}{A \pchoice \Done}
  	\qquad & \qquad
  	\Definition{B}{}{B \pchoice B}
  \end{array}
  \]
  and observe that $A$ may always reduce to $\Done$, whereas $B$ can only reduce
	forever into itself. So $A$ is action bounded whereas $B$ is not. 
	\eoe
\end{example}

We consider a
parallel composition action bounded if so are \emph{both} processes composed in
parallel.
%
Action boundedness is a necessary condition for (fair) process termination,
hence the type system must guarantee that well-typed processes are action
bounded. As we will see in \Cref{ssec:ts_bin_rules}, this can be easily achieved by means
of \emph{typing corules} (see \Cref{sec:gis}). 
Besides, action boundedness carries along two welcome side effects.
%
The first one is that degenerate process definitions such as
$\Definition{A}{}{A}$ are not action bounded and therefore are flagged as ill
typed by the type system. This guarantees that finitely many unfoldings of
recursive process invocations always suffice to expose some observable process
behavior.
%
The second is that action boundedness allows us to detect recursive processes
that claim to use a channel in a certain way when in fact they never do so. 
\begin{example}
	Consider the following processes
	\[
	\begin{array}{ll}
		\Definition{A}{x,y}{
    	\POutput\x\la.\Call{A}{x,y} \pchoice \POutput\x\lb.\Close\x
  	}
  	\qquad & \qquad
  	\Definition{B}{x,y}{\POutput\x\la.\Call{B}{x,y}}
	\end{array}
	\]
	where $\Call{A}{x,y}$ is action bounded and $\Call{B}{x,y}$ is not. An ordinary
	session type system with coinductively interpreted typing rules would accept
	$\Call{B}{x,y}$ regardless of $y$'s type on the grounds that $y$ occurs once in
	the body of $B$, hence it is ``used'' linearly. This is unfortunate, since $y$
	is not used in any meaningful way other than being passed as an argument of $B$.
	In $A$, the same linearity check promptly detects that $y$ is not used along the
	path to $\Close\x$ that proves the boundedness of $\Call{A}{x,y}$.
	\eoe
\end{example}

%%%%%%%%%%%%%%%
%%% SESSION %%%
%%%%%%%%%%%%%%%

\begin{definition}[Session Boundedness]
	We say that a process is \emph{session bounded} if there is a finite upper bound
	to the number of sessions it has to create in order to terminate.
\end{definition}

 It is easy
to construct non-terminating processes by chaining together an infinite number
of finite (or fairly terminating) sessions.

\begin{example}
  \label{ex:session_bounded}
  Compare the following processes
  \[
  \begin{array}{rl}
  	\Definition{A}{}{& \NewPar\x{\Close\x}{\Wait\x{\Call{A}{}}} \pchoice \Done}
  	\\
  	\Definition{B_1}{}{& \NewPar\x{\Close\x}{\Wait\x{\Call{B_1}{}}}}
  \end{array}
  \]
	where $A$ always has a possibility to terminate without creating new sessions
	(it is session bounded) while $B_1$ does not (it is session unbounded). It could
	be argued that $B_1$ is already ruled out because it is not action bounded.
	Indeed, while the left-hand side of the
	parallel composition in $B_1$ is finite, the right hand side is not (recall that
	we require \emph{both} sides of a parallel composition to admit a finite path to
	either $\Done$ or $\Close\x$). 
	\eoe
\end{example}

\begin{example}
	Below is a slightly more complex variation of
	$B_1$ that is action bounded and session unbounded. The trick is to have a
	finite branch on one side of the parallel composition matched by an infinite one
	on the other side:
	\[
	  \Definition{B_2}{}{
    \NewPar\x{
      \PSend\x{
        \la : \Close\x,
        \lb : \Wait\x{\Call{B_2}{}}
      }
    }{
      \PRecv\x{
        \la : \Wait\x{\Call{B_2}{}},
        \lb : \Close\x
      }
    }
  }
  \]
  \eoe
\end{example}

\Cref{ex:session_bounded} shows that a session bounded process like $A$ may
still create an unbounded number of sessions. Below is another example of
session bounded process that creates unboundedly many \emph{nested} sessions,
such that the first session being created is also the last one being completed.
\begin{example}
  \label{ex:infinite_sessions}
  \[
  \begin{array}{ll}
    \NewPar\x{\Call{C}{x}}{\Wait\x\Done}
  	&
  	\Definition{C}{x}{
    	\NewPar\y{\Call{C}{y}}{\Wait\y{\Close\x}} \pchoice \Close\x
  	}
  \end{array}
  \]
	While both $A$ and $C$ \emph{may} create an arbitrary number of sessions, they
	do not \emph{have to} do so in order to terminate. This is what sets them apart
	from $B_1$ and $B_2$.
	\eoe
\end{example}

To ensure session boundedness, we check that no new sessions are found in loops that 
occur along inevitable paths leading to the termination of a process. For example, 
the creation of a new session $x$ is inevitable in both $B_1$ and $B_2$ but it is not in $A$ and $C$.

%%%%%%%%%%%%
%%% CAST %%%
%%%%%%%%%%%%

\begin{definition}[Cast Boundedness]
	We say that a process is \emph{cast bounded} if there is a finite upper bound to
	the number of casts it has to perform in order to terminate.
\end{definition}

Performing a cast
means applying \refrule{sm-cast-new}, which corresponds to a usage of fair
subtyping. The reason why cast boundedness is fundamental is that the
liveness-preserving property of fair subtyping holds as long as fair
subtyping is used finitely many times. Conversely, infinitely many usages of
fair subtyping may have the overall effect of a single usage of unfair subtyping
(see \Cref{ssec:unfair_sub}). By ``infinitely many usages'' we mean usages of
fair subtyping that occur within a loop in a recursive process. 

\begin{remark}[0-weight casts]
	Notably, cast boundedness refers to those situations in which a cast with a strictly
	positive weight is applied inside a loop. It might be the case that a 0-weighted cast
	is performed (\eg no output contravariance). 
	In this case we allow such application of fair subtyping as the 
	liveness of the session is preserved. This is a substantial difference with respect
	to \cite{CicconePadovani22}.
	%
	\eor
\end{remark}

\begin{example}
  \label{ex:infinite_fair}
  Consider the following variants of \actor{buyer} and \actor{seller} from \Cref{ex:bsc} 
  where we can assume that the \actor{seller} closes the session as soon as he receives 
  a tag $\lpay$.
  \[
  \begin{array}{ll}
    \NewPar\x{\Call \Buyer x}{\Call \Seller x}
  	&
  	\begin{array}{r@{~}l}
    	\Definition{\Buyer}{x}{& \Cast\x \POutput\x\ladd.\Call \Buyer x} \\
    	\Definition{\Seller}{x}{&
      	\PRecv\x{
        	\ladd : \Call \Seller x,
        	\lpay : \dots
      	}
    	}
  	\end{array}
  \end{array}
  \]
  and the session type $S_b = \Out\ladd.S_b \choice \Out\lpay.\End[\Out]$.
	%
	It can be argued that the channel $x$ is used according to $S_b$ in $\Buyer(x)$ and
	according to $\co{\S_b}$ in $\Seller(x)$. Indeed, the structure of $\Seller(x)$ matches
	perfectly that of $\co{\S_b}$ and note that $\POutput\x\ladd.\Buyer(x)$ uses $x$ according to
	$\Out\ladd.S_b$, which is a fair supertype of $S_b$ accounted for by the cast
	$\Cast\x$ in $\Buyer$. With this cast it is as if $\Buyer(x)$ promises to make a choice
	between sending $\ladd$ and sending $\lpay$ at each iteration, but
	systematically favors $\ladd$ over $\lpay$. The overall effect of these
	unfulfilled promises is that the actual behavior of $\Buyer(x)$ over $x$ is better
	described by the session type $S_b^\infty = \Out\ladd.S_b^\infty$, which is
	\emph{not} a fair supertype of $S_b$ as we have seen in
	\Cref{ex:unfair_sub}.
	\eoe
\end{example}

Although $\Buyer(x)$ could be rejected on the grounds that it is not action bounded,
it is possible to find an action-bounded (but slightly more involved) variation
of \Cref{ex:infinite_fair} in which the same phenomenon
occurs.

\begin{example}
  \label{ex:infinite_fair_bounded}
  \[
  \begin{array}{r@{~}l}
    \Definition{A}{x}{&
      \Cast\x
      \POutput\x\lmore.
      \PRecv\x{
        \lmore : \Call A x,
        \lstop : \Wait\x\Done
      }
    }
    \\
    \Definition{B}{x}{&
      \PRecv\x{
        \lmore : \Cast\x \POutput\x\lmore. \Call B x,
        \lstop : \Wait\x\Done
      }
    }
  \end{array}
  \]
  both $A(x)$ and $B(x)$ have a chance to continue or to terminate the session by 
  sending either $\lmore$ or $\lstop$, except that they systematically favor $\lmore$ over $\lstop$.
	%
	Now, if we consider the session types
	\[
	\begin{array}{lcl}
		S & = & \Out\lmore.(\In\lmore.S \branch \In\lstop.\End[\In]) \choice \Out\lstop.\End[\Out]
		\\
		S_A & = & \Out\lmore.(\In\lmore.S \branch \In\lstop.\End[\In])
		\\
		S_B & = & \In\lmore.\Out\lmore.\co\S \branch \In\lstop.\End[\In]
	\end{array}
	\]
	it can be argued that $A(x)$
	uses $x$ according to $S_A$, which is a fair supertype of $S$, and that $B(x)$ uses $x$
	according to $S_B$, which
	is a fair supertype of $\co\S$. The casts
	account for the differences between $S$ and $S_A$ in $A(x)$ and between $\co\S$
	and $S_B$ in $B(x)$, but they occur within loops along paths that lead to
	process termination, hence $A$ and $B$ are not cast bounded.
	\eoe
\end{example}

It is worth discussing one last attempt to work around the problem, by moving
the casts outward from within $A(x)$ and $B(x)$.

\begin{example}
  \label{ex:finite_unfair}
  \[
  	\begin{array}{@{}r@{~}l@{}}
  		& \NewPar\x{\Cast\x \Call A x}{\Cast\x \Call B x}
  		\\
    	\Definition{A}{x}{&
      	\POutput\x\lmore.
      	\PRecv\x{
        	\lmore : \Call A x,
        	\lstop : \Wait\x\Done
      	}
    	}
    	\\
    	\Definition{B}{x}{&
      	\PRecv\x{
        	\lmore : \POutput\x\lmore. \Call B x,
        	\lstop : \Wait\x\Done
      	}
    	}
  	\end{array}
  \]
  Now $A(x)$ uses $x$  according to $T_A = \Out\lmore.(\In\lmore.T_A
	\branch \In\lstop.\End[\In])$ and $B(x)$ uses $x$ according to $T_B =
	\In\lmore.\Out\lmore.T_B \branch \In\lstop.\End[\In]$, but while $S \usubt T_A$
	and $\co\S \usubt T_B$ both hold neither $S \subt T_A$ nor $\co\S \subt T_B$
	does.
	\eoe
\end{example}

In summary, the non-terminating process in \Cref{ex:finite_unfair} is
action bounded, session bounded and cast bounded, but it is typeable only using
\emph{unfair} subtyping.
% 
As we will see in \Cref{ssec:ts_bin_rules}, we enforce cast boundedness using the same 
technique already introduced for session boundedness. 
That is, we check that casts do not occur along ``inevitable'' paths leading to recursive process invocations.


%%%%%%%%%%%%%
%%% RULES %%%
%%%%%%%%%%%%%

\subsection{Typing Rules}
\label{ssec:ts_bin_rules}
\beginbass
%
The typing rules resemble those of a traditional session type system but differ
in a few key aspects. First of all, they establish a tighter-than-usual
correspondence between types and processes so that any discrepancy between
actual and expected types is accounted for by explicit casts. This way, we make
sure that actions leading to the termination of a session \emph{at the type
level} are matched by corresponding actions \emph{at the process level}, a key
property used in the soundness proof of the type system.
In addition, the typing rules enforce the boundedness properties informally
described in the previous section.
%
Action boundedness is enforced by specifying the typing rules as a generalized
inference system and using two corules to make sure that every well-typed
process is at finite distance from $\Done$ or a $\Close\x$.
%
Concerning session and cast boundedness, we annotate typing judgments with a
\emph{rank}, that is an upper bound to the \emph{weights} of casts that must be
performed and of sessions that must be created in order to terminate the process
in the judgment.

\begin{figure}[t]
  \framebox[\textwidth]{
      \begin{mathpar}
        \displaystyle
          \inferrule[tb-done]{\mathstrut}{
            \wtp[n]\EmptyCtx\Done
          } \defrule[tb-done]{}
          \and
          \inferrule[tb-wait]
          {
            \wtp[n]\Ctx{P}
          }{
            \wtp[n]{\Ctx, x :  \End[\In]}{\Wait\x{P}}
          } \defrule[tb-wait]{}
          \and
          \inferrule[tb-close]{\mathstrut}
          {
            \wtp[n]{x : \End[\Out]}{\Close\x}
          } \defrule[tb-close]{}
          \and
          \inferrule[tb-channel-in]{
            \wtp[n]{\Ctx, x : S, y : T}{P}
          }{
            \wtp[n]{\Ctx, x :  \In\T.S}{\PInput\x{(y)}.P}
          } \defrule[tb-channel-in]{}
          \and
          \inferrule[tb-channel-out]{
            \wtp[n]{\Ctx, x : S}{P}
          }{
            \wtp[n]{\Ctx, x : \Out\T.S, y : T}{\POutput\x\y.P}
          } \defrule[tb-channel-out]{}
          \and
          \inferrule[tb-tag]
          {
          	\forall i\in I:
            \wtp[n]{\Ctx, x : S_i}{P_i}
          }{
            \textstyle
            \wtp[n]{
              \Ctx, x : \Pol\set{\l_i : S_i}_{i \in I}
            }{
              x\Pol\set{\l_i : P_i}_{i \in I}
            }
          } \defrule[tb-tag]{}
          \and
          \inferrule[tb-choice]{
            \wtp[n_1]\Ctx{P}
            \\
            \wtp[n_2]\Ctx{Q}
          }{
            \wtp[n_k]\Ctx{P \pchoice_k Q}
          }
          ~
          k \in \set{1,2}
          \defrule[tb-choice]{}
          \and
          \inferrule[tb-cast]{
            \wtp[n]{\Ctx, x : T}{P}
          }{
            \wtp[n+m]{\Ctx, x : S}{\Cast\x P}
          }
          ~ S \subt[m] T \defrule[tb-cast]{}
          \and
          \inferrule[tb-par]{
            \wtp[m]{\Ctx, x : S}{P}
            \\
            \wtp[n]{\CtxD, x : T}{Q}
          }{
            \wtp[1+m+n]{
              \Ctx, \CtxD
            }{
              \NewPar\x{P}{Q}
            }
          }
          ~
          S \compatible T \defrule[tb-par]{}
          \and
          \inferrule[tb-call]{
            \wtp[n]{\seqof{x:S}}{P}
          }{
            \wtp[m+n]{\seqof{x:S}}{\Call{A}{\seqof\x}}
          }
          ~
          \tass{A}{\seqof{S}}{n},
          \Definition{A}{\seqof\x}{P} \defrule[tb-call]{}
          \and
          \infercorule[cob-tag]{
            \wtp[n]{\Ctx, x : S_k}{P_k}
          }{
            \wtp[n]{
              \Ctx, x : \Pol\set{\l_i : S_i}_{i \in I}
            }{
              x\Pol\set{\l_i : P_i}_{i \in I}
            }
          }
          ~
          k \in  I \defrule[cob-tag]{}
          \and
          \infercorule[cob-choice]{
            \wtp[n]\Ctx{P_k}
          }{
            \wtp[n]\Ctx{P_1 \pchoice_k P_2}
          } \defrule[cob-choice]{}
      \end{mathpar}
    }
    \caption{Typing rules}
    \label{fig:ts_bin}
\end{figure}

The typing rules are defined by the generalized inference system in
\Cref{fig:ts_bin} and derive judgements of the form $\wtp[n]\Ctx{P}$, meaning
that $P$ is well typed in the \emph{typing context} $\Ctx$ and has rank $n$.
A typing context is a finite map from channels to session types written $x_1 :
S_1, \dots, x_n : S_n$ or $\seqof{x : S}$. We use $\Ctx$ and $\CtxD$ to
range over typing contexts, we write $\EmptyCtx$ for the empty context and
$\Ctx,\CtxD$ for the union of $\Ctx$ and $\CtxD$ when they
have disjoint domains.
%
We type check a program $\set{\Definition{A_i}{\seqof{x_i}}{P_i}}_{i\in I}$
under a global set of type assignments $\set{\tass{A_i}{\seqof{S_i}}{n_i}}_{i\in
I}$ associating each process name $A_i$ with a tuple of session types
$\seqof{S_i}$ and a rank $n_i$. The program is well typed if
$\wtp[n_i]{\seqof{x_i : S_i}}{P_i}$ for every $i\in I$, establishing that the
tuple $\seqof{S_i}$ corresponds to the way the channels $\seqof{x_i}$ are used
by $P_i$ and that $n_i$ is a feasible rank annotation for $P_i$. Hereafter, we
omit the rank from judgments when it is not important. 

Let us look at the typing (co)rules in detail.
%
\refrule{tb-done} is the usual axiom requiring that the terminated process leaves no unused channels behind. 
Since $\Done$ performs no casts and creates no sessions, it can have any rank.
%
Rules \refrule{tb-wait} and \refrule{tb-close} concern the exchange of session
termination signals. There is nothing remarkable here except noting once again
that the rank of $\Close\x$ can be arbitrary.
%
Rules \refrule{tb-channel-in} and \refrule{tb-channel-out} are similar, but they
concern the exchange of channels. Note that, in \refrule{tb-channel-out}, the
type $T$ of the message $y$ is required to match \emph{exactly} that in the type
of the channel $x$ used for the communication, whereas \citep{GayHole05} allow
the type of $y$ to be a subtype of $T$. This is one instance of the ``tight
correspondence'' that we mentioned earlier (see \Cref{ex:invariant_ch}).
%
The rule \refrule{tb-label} deals with the input/output of labels. As usual, any
channel other than the one affected by the communication must be used in exactly
the same way in every branch. However, the rule is stricter than that of
\citet{GayHole05} because it requires an exact correspondence between the labels
that can be exchanged on $x$ by the process and those in the type of $x$. The
fact that a conclusion and premises are all annotated with the same rank $n$
means that $n$ is an upper bound for the rank of all branches of a label
input/output.
%
The corule \refrule{cob-label} does not impose additional constraints compared to
\refrule{tb-label} and has \emph{exactly one premise}, corresponding to one
branch of the process in the conclusion. The effect of \refrule{cob-label}, when
interpreted inductively together with the other rules, is to ensure the
existence of a finite typing derivation whose leaves are applications of
\refrule{tb-done} or \refrule{tb-close}, hence action boundedness.

Rule \refrule{tb-choice} is a standard typing rule for non-deterministic choices,
requiring that both branches are well typed in exactly the same typing context.
Notice that the rank of a choice $P_1 \pchoice_k P_2$ is determined by the branch
indexed by the $k$ annotation, which is elected as the branch that leads to
termination. Like \refrule{cob-label}, the associated corule
\refrule{cob-choice} ensures that the same branch gets closer to $\Done$ or a
$\Close\x$ to enforce action boundedness. Without this corule, it would not be
possible to find a \emph{finite-depth} derivation tree for an action-bounded
process such as $A$ in \Cref{ex:action_boundedness}. Coherently with
\refrule{tb-choice}, the same branch that leads to termination is also the one
that determines the rank of the choice as a whole.

Rule \refrule{tb-cast} is Liskov's substitution principle formulated as an
inference rule. It states that a channel $x$ of type $S$ can be safely used
where a channel of type $T$ is expected, provided that $S \subt T$. The most
important detail to notice here is that the rank of a cast is the \emph{weight}
of the subtyping judgment plus that of
the process in which the cast has effect. This way we account for this cast in
the rank of the process so as to guarantee cast boundedness.
%
Rule \refrule{tb-par} concerns parallel composition and session creation. The
rule is shaped after the cut rule of linear logic also adopted in other session
type systems based on linear logic
\citep{CairesPfenningToninho16,Wadler14,LindleyMorris16}. In particular, the
parallel processes $P$ and $Q$ share no channel other than the session $x$ that
connects them, so as to prevent mutual dependencies between sessions and
guarantee deadlock freedom. The side condition $S \compatible T$ requires that
the way in which $P$ and $Q$ use channel $x$ is such that the session $x$ can
fairly terminate (see \Cref{def:compatibility}). We \emph{do not} require that $S$
and $T$ are dual to each other because reductions (see \refrule{rb-pick}) and
structural pre-congruence (see \refrule{sb-cast-new}) do not necessarily preserve
session type duality. Also, duality does not always imply compatibility.
%
The rank of a parallel composition is one plus that of the composed processes.
By accounting for each occurrence of parallel compositions in the rank, we
guarantee that well-typed processes are session bounded.

Finally, rule \refrule{tb-call} states that a process invocation
$\Call{A}{\seqof\x}$ is well typed provided that the types associated with
$\seqof\x$ match those of the global assignment $\tass{A}{\seqof{S}}{n}$. Note
that \refrule{tb-call} is \emph{not} an axiom: its premise (re)checks that the
body $P$ in the definition of $A$ is coherent with the global type assignment
$\tass{A}{\seqof{S}}{n}$. With this formulation of \refrule{tb-call}, the only
axioms are \refrule{tb-done} and \refrule{tb-close} so that the inductive
interpretation of the typing (co)rules ensures action boundedness. Note also
that the rank of the conclusion may be greater than the rank $n$ associated with
$A$. This overapproximation grants more flexibility when typing different
branches in \refrule{tb-label}.

\begin{remark}[On structural pre-congruence...continuation]
	Now we have all the ingredients to understand why the choice of a pre-congruence
	over a congruence relation is just a design one (see \Cref{rmk:pcong}).
	Indeed, the such choice was compulsory in \cite{CicconePadovani22}.
	As mentioned before, in such work we relied on the characterization of fair subtyping based on
	a generalized inference system (see \Cref{ssec:fsub_gis}) and the subsumption rule
	\refrule{tb-cast} always increased the \emph{rank} by one. This way, \refrule{sb-cast-new} interpreted
	in a congruence way would increase the rank of the process due to the introduced cast.
	Using the actual notions, a reflexive application of fair subtyping has weight zero.
	%
	\eor
\end{remark}

Well-typed processes enjoy the expected properties, including typing
preservation under structural pre-congruence and reduction. Most importantly,
they fairly terminate:

\begin{theorem}{Soundness}
  \label{thm:ts_bin_sound}
  If $\wtp[n]\EmptyCtx P$ and $P \wred Q$, then $Q \wred\pcong \pdone$.
\end{theorem}

The proof of \Cref{thm:ts_bin_sound} follows \Cref{thm:fair_termination}. 
Moreover the proof that all the reducts of a process are \emph{weakly terminating}
(see \Cref{lem:weak_termination_bin})
is loosely based on the method of helpful directions
\citep{Francez86}, namely on the property that a (well-typed) process \emph{may}
reduce in such a way that its measure strictly decreases
(see \Cref{lem:helpful_direction_bin}). Recall that this
property is not true for every reduction.

There are several valuable implications of \Cref{thm:ts_multi_sound} on a well-typed,
closed process $P$:
\begin{description}
  \item[Deadlock freedom.] If $Q$ cannot reduce any further, then it must be
  $\pdone$ (structurally precongruent to), namely there are no residual
  input or output actions.
  \item[Fair termination.] Under the fairness assumption,
  \Cref{thm:fair_termination} assures that $P$ eventually reduces to $\pdone$.
  This also implies that every session created by $P$ eventually terminates.
  \item[Junk freedom.] Each message produced as $P$ executes is eventually
  consumed. Indeed, if $Q$ contains a pending message, the fact that $Q$ may
  reduce to $\pdone$ means that some process is able to consume the message and
  will eventually do so under the fairness assumption.
  \item[Progress.] If $Q$ contains a sub-process with pending input/output
  actions, the fact that $Q$ may reduce to $\pdone$ means that these actions are
  eventually performed.
\end{description}

\begin{remark}
  \label{rem:internal_choice_rank}
  The rank of a non-deterministic choice $P \pchoice Q$ can usually be chosen to
  be the minimum among those of the branches $P$ and $Q$, so that the type
  system can handle processes like those in \cref{ex:infinite_sessions}, which
  \emph{may} create new sessions or perform casts but they need not do so in
  order to terminate.
  %
  On the contrary, the rank of a label output $\PSend\x{\l_i:P_i}_{i\in I}$ has
  to be an upper bound of that of all branches $P_i$.
  %
  The motivation for such different ways of determining the rank of these
  process forms, despite both represent an \emph{internal choice}, lies in the
  proof of \Cref{lem:helpful_direction_bin}.
  %
  In $P \pchoice Q$, both branches are typed in \emph{exactly the same} typing
  context, meaning that the choice of one branch or the other has no substantial
  impact on the shortest paths that terminate the sessions used by $P$ and $Q$.
  Thus, the ``helpful'' reduction can be solely driven by the rank of the chosen
  branch.
  %
  In a label output $\PSend\x{\l_i:P_i}_{i\in I}$ it could happen that all
  branches with minimum rank increase the length of the shortest path that leads
  to the termination of $x$. In this case, the choice of the ``helpful''
  reduction must prioritize the termination of $x$, but then the rank of the
  whole process has to be an upper bound of that of the branches to be sure that
  the measure of the reduct decreases.
  %
  \eor
\end{remark}

%%%%%%%%%%%%%%%%
%%% EXAMPLES %%%
%%%%%%%%%%%%%%%%

\subsection{Examples}
\beginbass
%
We dedicate the rest of \Cref{sec:ts_multi_ts} to the analysis of some examples
that integrate all the features of the presented type system. We start from some
basic examples and then we move to more involved ones. First, in \Cref{ex:bsc_ts_multi} we take into
account our slightly different variant of the running example (\Cref{ex:bsc_multi}). 
For what concerns the problematic processes in \Cref{ssec:boundedness}, they are still
valid in the multiparty context (see \Cref{rm:boundedness_multi}). 
We use the rest of the examples to deal with the processes introduced in \Cref{ssec:proc_ex_multi}.

%%%%%%%%%%%%%%%%%%%%%%%%%%%%%%%%%%%%%%%
%%% Action/Session/Cast-Boundedness %%%
%%%%%%%%%%%%%%%%%%%%%%%%%%%%%%%%%%%%%%%

\begin{remark}[Boundedness]
\label{rm:boundedness_multi}
	All the problematic processes that we presented in \Cref{ssec:boundedness} are still valid
	in the multiparty scenario and can be dealt with using the techniques that we mentioned for
	the binary case. In particular
	\begin{description}
	\item[Action-boundedness.] Type system with corules.
	\end{description}
	\[
    A \peq A
    \qquad
    B \peq {B \pchoice B}
    \qquad
    C \peq {C \pchoice \pdone}
  \]
	\begin{description}
	\item[Session-boundedness.] Rule \refrule{tm-par} increases the rank by one.
	\end{description}
	\[
    A \peq
      \pres{s}{
        \act{\ep{s}{\rolep}}{\roleq}\oact\set{
          \Tag[a].\pclose{\ep{s}{\rolep}},
          \Tag[b].\pwait{\ep{s}{\rolep}}{A}
        }
        \parop
        \act{\ep{s}{\roleq}}{\rolep}\iact\set{
          \Tag[a].\pwait{\ep{s}{\roleq}}{A},
          \Tag[b].\pclose{\ep{s}{\roleq}}
        }
      }
  \]
	\begin{description}
	\item[Cast-boundedness.] Rule \refrule{tm-cast} increases the rank by the weight of the subtyping
		being applied.
	\end{description}
	\[
  	B(x) \peq \pcast{x}\act{x}\seller\oact\tadd.\pinvk{B}{x}
  \]
	%
	\eor
\end{remark}

%%%%%%%%%%%%%%%%%%%%%%%
%%% Running Example %%%
%%%%%%%%%%%%%%%%%%%%%%%

\begin{example}
  \label{ex:bsc_ts_multi} 
  Let us show some typing derivations for fragments of \Cref{ex:bsc_multi}. 
  Let $\S_b$, $S_s$ and $S_c$ be the types from \Cref{ex:bsc_ty_multi}.
  We collapse roles to their initials.
  Let $\S'_b = \rseller\Out\tadd\rseller\Out\tadd.\S'_b + \rseller\Out\tpay.\End[\Out]$.
  Concerning $\Buyer$, we obtain the infinite derivation
  \[
    \begin{prooftree}
      \[
        \[
          \mathstrut\smash\vdots
          \justifies
          \wtp[0]{
            x : \S'_b
          }{
            \pinvk\Buyer{x}
          }
          \using\refrule{tm-call}
        \]
        \justifies
        \wtp[0]{
          x : \rseller\Out\tadd.\S'_b
        }{
          \act{x}\rseller\oact\tadd.\pinvk\Buyer{x}
        }
        \using\refrule{tm-tag}
      \]
      \[
        \justifies
        \wtp[0]{
          x : \End[\Out]
        }{
          \pclose{x}
        }
        \using\refrule{tm-close}
      \]
      \justifies
      \wtp[0]{
        x : \S'_b
      }{
        \act{x}\rseller\oact\set{
          \tadd.\act{x}\rseller\oact\tadd.\pinvk\Buyer{x},
          \tpay.\pclose{x}
        }
      }
      \using\refrule{tm-tag}
    \end{prooftree}
  \]
  %
  and, for each judgment in it, it is easy to find a finite derivation possibly
  using \refrule{com-tag}. Concerning $\Main$ we obtain
  \[
    \begin{prooftree}
      \[
        \mathstrut\smash\vdots
        \justifies
        \wtp[0]{
          \ep{s}\rbuyer : \S'_b
        }{
          \pinvk\Buyer{\ep{s}\rbuyer}
        }
        \using\refrule{tm-call}
      \]
      \[
        \smash\vdots
        \justifies
        \wtp[0]{
          \ep{s}\rseller : S_s
        }{
          \pinvk\Seller{\ep{s}\rseller}
        }
      \]
      \vdots
      \justifies
      \wtp[1]{
        \EmptyCtx
      }{
        \pres\sn{
          \pinvk\Buyer{\ep{s}\rbuyer} \ppar \pinvk\Seller{\ep{s}\rseller} \ppar \pinvk\Carrier{\ep{s}\rcarrier}
        }
      }
      \using\refrule{tm-par}
    \end{prooftree}
  \]
  %
  where the application of \refrule{tm-par} is justified by the fact that
  $\Map\rbuyer{\S'_b} \parop \Map\rseller{S_s} \parop \Map\rcarrier{S_c}$ is coherent.
  We recall that $\S_b \subt[1] \S'_b$ (\Cref{ex:bsc_fair_sub}).
  %
  No participant creates new sessions or performs casts, so they all have zero
  rank. The rank of $\Main$ is 1 since it creates the session $s$.
  %
  \eoe
\end{example}

%%%%%%%%%%%%%%%%%%%%%%%%%%%%%%
%%% 2Buyers-Seller-Shipper %%%
%%%%%%%%%%%%%%%%%%%%%%%%%%%%%%

\begin{example}
\label{ex:2bsc-ts}
In this example we show that the process $\Buyer_1$ playing the role $\rbuyer_1$
in the inner session of \Cref{ex:2bsc_multi} is well typed. For clarity, we recall its
definition here: 
\[
Buyer_1(x,y) \peq \act{y}{\rbuyer_2}\oact\{
	\begin{lines}
	  \tsplit.\act{y}{\rbuyer_2}\iact\{
		\begin{lines}
			\tyes.\pcast{x}
				\act{x}\rseller\oact\tok.
				\act{x}\rcarrier\iact\tbox.
				\pwait{x}
				\pwait{y}
				\pdone,
				\\
			\tno.\pinvk{Buyer_1}{x,y} \},
		\end{lines}
	  \\
	  \tgiveup.
		\pwait{y}
		\pcast{x}
		\act{x}\rseller\oact\tcancel.
		\pwait{x}
		\pdone \}
	\end{lines}
\]

We wish to build a typing derivation showing that $Buyer_1$ has rank $1$ and
uses $x$ and $y$ respectively according to $S$ and $T$, where $S =
\rseller\Out\tok.\rcarrier\In\tbox.\End[\In] + \rseller\Out\tcancel.\End[\In]$
and $T = \rbuyer_2\Out\tsplit.(\rbuyer_2\In\tyes.\End[\In] + \rbuyer_2\In\tno.T)
+ \rbuyer_2\Out\tgiveup.\End[\In]$.
%
As it has been noted previously, what makes this process interesting is that it
uses the endpoint $x$ differently depending on the messages it exchanges with
$\rbuyer_2$ on $y$. Since rule \refrule{tm-tag} requires any endpoint other
than the one on which messages are exchanged to have the same type, the only way
$\Buyer_2$ can be declared well typed is by means of the casts that occur in its
body.
%
For the branch in which $\Buyer_1$ proposes to $\tsplit$ the payment we obtain
the following derivation tree (we show only the $\tyes$ branch, the $\tno$ one is trivial):
\[
	\begin{prooftree}
		\[
			\[
				\[
					\[
						\[
							\[
								\justifies
								\wtp[0]\EmptyCtx\pdone
								\using\refrule{tm-done}
							\]
							\justifies
							\wtp[0]{
								y : \End[\In]
							}{
								\pwait{y}\pdone
							}
							\using\refrule{tm-wait}
						\]
						\justifies
						\wtp[0]{
							x : \End[\In],
							y : \End[\In]
						}{
							\pwait[\dots]{x}
						}
						\using\refrule{tm-wait}
					\]
					\justifies
					\wtp[0]{
						x : \rcarrier\In\tbox.\End[\In],
						y : \End[\In]
					}{
						\act{x}\rcarrier\iact\tbox\dots
					}
					\using\refrule{tm-tag}
				\]
				\justifies
				\wtp[0]{
					x : \rseller\Out\tok.\rcarrier\In\tbox.\End[\In],
					y : \End[\In]
				}{
					\act{x}\rseller\oact\tok\dots
				}
				\using\refrule{tm-tag}
			\]
			\justifies
			\wtp[1]{
				x : S,
				y : \End[\In]
			}{
				\pcast{x}\dots
			}
			\using\refrule{tm-cast}
		\]
		\vdots
		\justifies
		\wtp[1]{
			x : S,
			y : \rbuyer_2\In\tyes.\End[\In] + \rbuyer_2\In\tno.T
		}{
			\act{y}{\rbuyer_2}\iact\set{\tyes\dots, \tno\dots}
		}
		\using\refrule{tm-tag}
	\end{prooftree}
\]

Note how the application of \refrule{tm-cast} is key to change the type of $x$ in
the branch where the proposed split is accepted by $\rbuyer_2$. In that branch,
$x$ is deterministically used to send an $\tok$ message and we leverage on the
fair subtyping relation $S \subt[1] \rseller\Out\tok.\rcarrier\In\tbox.\End[\In]$.
%

For the branch in which $\Buyer_1$ sends $\tgiveup$ we obtain the following
derivation tree:
\[
	\begin{prooftree}
		\[
			\[
				\[
					\[
						\justifies
						\wtp[0]\EmptyCtx{
							\pdone
						}
						\using\refrule{tm-done}
					\]
					\justifies
					\wtp[0]{
						x : \End[\In]
					}{
						\pwait{x}\pdone
					}
					\using\refrule{tm-wait}
				\]
				\justifies
				\wtp[0]{
					x : \rseller\Out\tcancel.\End[\In]
				}{
					\act{x}\rseller\oact\tcancel.
					\pwait{x}
					\pdone
				}
			\]
			\justifies
			\wtp[1]{
				x : S
			}{
				\pcast{x}
				\act{x}\rseller\oact\tcancel.
				\pwait{x}
				\pdone
			}
			\using\refrule{tm-cast}
		\]
		\justifies
		\wtp[1]{
			x : S,
			y : \End[\In]
		}{
			\pwait{y}
			\pcast{x}
			\act{x}\rseller\oact\tcancel.
			\pwait{x}
			\pdone
		}
		\using\refrule{tm-wait}
	\end{prooftree}
\]

Once again the cast is necessary to change the type of $x$, but this time
leveraging on the fair subtyping relation $S \subt[1]
\rseller\Out\tcancel.\End[\In]$.
%
These two derivations can then be combined to complete the proof that the body
of $\Buyer_1$ is well typed:
\[
	\begin{prooftree}
		\qquad
		\mathstrut\smash\vdots
		\qquad
		\qquad
		\qquad
		\smash\vdots
		\qquad
		\justifies
		\wtp[1]{
			x : S,
			y : T
		}{
			\act{y}{\rbuyer_2}\oact\set{\tsplit\dots, \tgiveup\dots}
		}
		\using\refrule{tm-tag}
	\end{prooftree}
\]

Clearly, it is also necessary to find finite derivation trees for all of the
judgments shown above. This can be easily achieved using the corule
\refrule{com-tag}.
%
\eoe
\end{example}

\begin{example}
	\label{ex:non-det}
	Casts can be useful to reconcile the types of a channel that is used
	differently in different branches of a non-deterministic choice. For
	example, below is an alternative modeling of $\Buyer$ from \Cref{ex:bsc_multi}
	where we abbreviate $\role[\seller]$ to $\rseller$ for convenience:
	\[
		\Definition{B}{x}{
			\pcast{x}
			\act{x}\rseller\oact\tadd.
			\act{x}\rseller\oact\tadd.\pinvk{B}{x}
			\pchoice
			\pcast{x}
			\act{x}\rseller\oact\tpay.
			\pclose{x}
		}
	\]

	Note that $x$ is used for sending two $\tadd$ messages in the left branch of
	the non-deterministic choice and for sending a single $\tpay$ message in the
	right branch. Given the session type $S = \rseller\Out\tadd.S +
	\rseller\Out\tpay.\End[\Out]$ and using the fair subtyping relations $S
	\subt[2] \rseller\Out\tadd.\rseller\Out\tadd.S$ and $S \subt[1]
	\rseller\Out\tpay.\End[\Out]$ we can obtain the following typing derivation
	for the body of $B$ (we show only the left branch as the right one contains a
	straightforward application of $S \subt[1] \rseller\Out\tpay.\End[\Out]$):
	\[
		\begin{prooftree}
			\[
				\[
					\[
						\[
							\mathstrut\smash\vdots
							\justifies
							\wtp[1]{
								x : S
							}{
								\pinvk{B}{x}
							}
							\using\refrule{tm-call}
						\]
						\justifies
						\wtp[1]{
							x : \rseller\Out\tadd.S
						}{
							\act{x}\rseller\oact\tadd.\pinvk{B}{x}
						}
						\using\refrule{tm-tag}
					\]
					\justifies
					\wtp[1]{
						x : \rseller\Out\tadd.\rseller\Out\tadd.S
					}{
						\act{x}\rseller\oact\tadd.
						\act{x}\rseller\oact\tadd.\pinvk{B}{x}
					}
					\using\refrule{tm-tag}
				\]
				\justifies
				\wtp[3]{
					x : S
				}{
					\pcast{x}
					\act{x}\rseller\oact\tadd.
					\act{x}\rseller\oact\tadd.\pinvk{B}{x}	
				}
				\using\refrule{tm-cast}
			\]
			\vdots
			\justifies
			\wtp[1]{
				x : S
			}{
				\pcast{x}
				\act{x}\rseller\oact\tadd.
				\act{x}\rseller\oact\tadd.\pinvk{B}{x}
				\pchoice
				\pcast{x}
				\act{x}\rseller\oact\tpay.
				\pclose{x}	
			}
			\using\refrule{tm-choice}
		\end{prooftree}
	\]
	%
	In general, the transformation $\pbranch[i=1..n]{u}{\role}\oact{\Tag_i}{P_i}
	\leadsto \pcast{u}\act{u}{\role}\oact{\Tag_1}.P_1 \pchoice \cdots \pchoice
	\pcast{u}\act{u}\role\oact{\Tag_n}.P_n$ does not always preserve typing, so
	it is not always possible to encode the output of tags using casts and
	non-deterministic choices. As an example, the definition
	%
	\[
		\Definition\SlotMachine{x}{
			\act{x}\rplayer\iact\set{
				\tplay.\act{x}\rplayer\oact\set{
					\twin.\pinvk\SlotMachine{x},
					\tlose.\pinvk\SlotMachine{x}
				},
				\tquit.\pclose{x}
			}
		}
	\]
	implements the unbiased slot machine of \Cref{ex:slot_fair_sub} that waits
	for a message indicating whether a $\rplayer$ wants to $\tplay$ another game or
	to $\tquit$ (we assign role $\rplayer$ to the player). 
	In the former case, the slot machine notifies $\rplayer$ of the
	outcome (either $\twin$ or $\tlose$).
	%
	It is easy to see that $\SlotMachine$ is well typed under the global type
	assignment $\tass\SlotMachine{T}{0}$ where $T =
	\rplayer\In\tplay.(\rplayer\Out\twin.T + \rplayer\Out\tlose.T) +
	\rplayer\In\tquit.\End[\Out]$. In particular, $\SlotMachine$ has rank $0$
	since it performs no casts and it creates no sessions. If we encode the tag
	output in $\SlotMachine$ using casts and non-deterministic choices we end up
	with the following process definition, which is ill typed because it cannot
	be given a finite rank:
	%
	\[
		\Definition{\SlotMachine}{x}{
			\act{x}\rplayer\iact\set{
				\tplay.(
					\pcast{x}
					\act{x}\rplayer\oact\twin.
					\pinvk{\SlotMachine}{x}
					\pchoice
					\pcast{x}
					\act{x}\rplayer\oact\tlose.
					\pinvk{\SlotMachine}{x}
				),
				\tquit.\pclose{x}
			}
		}
	\]

	The difference between this version of $\SlotMachine$ and the above
	definition of $B$ is that $\SlotMachine$ always recurs after a cast, so it
	is not obvious that finitely many casts suffice in order for $\SlotMachine$
	to terminate. 
	%
	\eoe
\end{example}

%%%%%%%%%%%%%%%%%%%%%%%%%%%%%%%%%%%%
%%% Buyer-Seller-Shipper Refined %%%
%%%%%%%%%%%%%%%%%%%%%%%%%%%%%%%%%%%%

\begin{example} 
	\Cref{ex:2bsc-ts}
	shows that casts are essential in the type derivation.
	However, the process would be well typed if we considered a subtyping relation that does not preserve coherence
	\citep{GayHole05} for the involved types are finite.
	Now we refine the buyer from \Cref{ex:bsc_multi} in order to consider more involved sessions. Again, we
	collapse role names to their initials.
	\begin{align*}
		B(x) & \peq
			{\pcast{x}\pinvk{B_1}{x} \pchoice \pcast{x}\pinvk{B_2}{x}} 
		\\
		B_1(x) & \peq
			{\act{x}{\rseller}\oact\Tag[add].\act{x}{\rseller}\oact\set{
  			\Tag[add].\pinvk{B_1}{x},\, 
  			\Tag[pay].\pwait{x}{\pdone} 
			}} 
	  \\  
		B_2(x) & \peq
			{\act{x}{\rseller}\oact\set{
				\Tag[add].\act{x}{\rseller}\oact\Tag[add]. \pinvk{B_2}{x},\, 
  				\Tag[pay].\pwait{x}{\pdone} 
			}}
	\end{align*}
	$B_2$ corresponds to the buyer in \Cref{ex:bsc_multi} while $B_1$ is the acquirer that adds an odd number of items to the cart. 
	$B$ non deterministically chooses to behave according to $B_1$ or $B_2$.
	%
	Let $S_{b_1}$ and $S_{b_2}$ be the session types such that 
	$x : S_{b_1}$ in $B_1$ and $x : S_{b_2}$ in $B_2$ respectively:
	\[
	\begin{array}{ll}
		S_{b_1} = \rseller\Out\Tag[add].(\rseller\Out\Tag[add].S_{b_1} + \rseller\Out\Tag[pay].\End[\Out])
		&
		S_{b_2} = \rseller\Out\Tag[add].\rseller\Out\Tag[add].S_{b_2} + \rseller\Out\Tag[pay].\End[\Out]
	\end{array}
	\]
	In \cref{ex:bsc_fair_sub} we showed that $S \subt[1] S_{b_2}$ where $S =\rseller\Out\tadd.S + \rseller\Out\tpay.\End[\Out]$
	models the acquirer that adds arbitrarily many items to the cart. Analogously, we can prove that $S \subt[2] S_{b_1}$.
	Hence we derive
	\[
	\begin{prooftree}
		\[
			\[
				\vdots
				\justifies
				\wtp[0]{x : S_{b_1}}{\pinvk{B_1}{x}}
				\using\refrule{tm-call}
			\]
			\justifies
			\wtp[2]{x : S}{\pcast{x}\pinvk{B_1}{x}}
			\using\refrule{tm-cast}
		\]
		\[
			\[
				\vdots
				\justifies
				\wtp[0]{x : S_{b_2}}{\pinvk{B_2}{x}}
				\using\refrule{tm-call}
			\]
			\justifies
			\wtp[1]{x : S}{\pcast{x}\pinvk{B_2}{x}}
			\using\refrule{tm-cast}
		\]
		\justifies
		\wtp[1]
			{x : S}
			{\pcast{x}\pinvk{B_1}{x} \pchoice \pcast{x}\pinvk{B_2}{x}}
		\using\refrule{tm-choice}
	\end{prooftree}
	\]
	Again, the casts are crucial to obtain the type derivation of process $B$ because rule 
	\refrule{tm-choice} requires that $B_1$ and $B_2$ are typed in the same context.
	Note that $B_1$ and $B_2$ are typed with rank 0 since no sessions are created and no casts are performed by the processes.
\end{example}

%%%%%%%%%%%%%%%%%%%%%%%%%%%%%%%%%
%%% Unbounded No. Of Sessions %%%
%%%%%%%%%%%%%%%%%%%%%%%%%%%%%%%%%

\begin{example}
	\label{ex:pms_ts}
	Here we provide evidence that the process definitions in \Cref{ex:pms_multi} are
	well typed, even if they model processes that can open arbitrarily many
	sessions. In that example, the most interesting process definition is that
	of the worker $\Sort$, which is recursive and may create a new session. In
	contrast, $\Merge$ is finite and $\Main$ only refers to $\Sort$. We claim
	that these process definitions are well typed under the global type
	assignments
	\[
		\tass\Main{}{1}
		\qquad
		\tass\Sort{U}{0}
		\qquad
		\tass\Merge{T, V}{0}
	\]
	where 
	\[
	\begin{array}{lll}
		T = \rmaster\Out\tres.\End[\Out]
		&
		U = \rmaster\In\treq.T
		&
		V = \rworker_1\Out\treq.\rworker_2\Out\treq.\rworker_1\In\tres.\rworker_2\In\tres.\End[\In]
	\end{array}
	\]
	For the branch of $\Sort$ that creates a new session we obtain the
	derivation tree
	\[
		\begin{prooftree}
			\[
				\mathstrut\smash\vdots
				\justifies
				\wtp[0]{
					x : T,
					\ep{t}\rmaster : V
				}{
					\pinvk\Merge{x,\ep{t}\rmaster}
				}
				\using\refrule{tm-call}
			\]
			\[
				\smash\vdots
				\justifies
				\wtp[0]{
					\ep{t}{\rworker_i} : U
				}{
					\pinvk\Sort{\ep{t}{\rworker_i}}
				}
			\]
			\justifies
			\wtp[1]{
				x : T
			}{
				\pres{t}{
					\pinvk\Merge{x,\ep{t}\rmaster} \parop
					\pinvk\Sort{\ep{t}{\rworker_1}} \parop
					\pinvk\Sort{\ep{t}{\rworker_2}}
				}
			}
			\using\refrule{tm-par}
		\end{prooftree}
	\]
	%
	where $i=1,2$. The rank $1$ derives from the fact that the created session involves
	three zero-ranked participants.
	%
	For the body of $\Sort$ we obtain the following derivation tree:
	\[
		\begin{prooftree}
			\[
				\[
					\smash\vdots
					\justifies
					\wtp[1]{
						x : T
					}{
						\pres{t}{
							\pinvk\Merge{x,\ep{t}\rmaster} \parop
							\cdots
						}
					}
					\using\refrule{tm-par}
				\]
				\[
					\[
						\justifies
						\wtp[0]{
							x : \End[\Out]
						}{
							\pclose{x}
						}
						\using\refrule{tm-close}
					\]
					\justifies
					\wtp[0]{
						x : T
					}{
						\act{x}\rmaster\oact\tres\dots
					}
					\using\refrule{tm-tag}
				\]
				\justifies
				\wtp[0]{
					x : T
				}{
					\pres{t}{
						\pinvk\Merge{x,\ep{t}\rmaster} \parop
						\cdots
					}
					\pchoice
					\act{x}\rmaster\oact\tres\dots
				}
				\using\refrule{tm-choice}
			\]
			\justifies
			\wtp[0]{
				x : U
			}{
				\act{x}\rmaster\iact\treq.(
					\pres{t}{
						\pinvk\Merge{x,\ep{t}\rmaster} \parop
						\cdots
					}
					\pchoice
					\act{x}\rmaster\oact\tres\dots
				)
			}
			\using\refrule{tm-tag}
		\end{prooftree}
	\]
	
	In the application of the rule \refrule{tm-choice}, the rank of the whole
	choice coincides with that of the branch in which no new sessions are
	created. This way we account for the fact that, even though $\Sort$
	\emph{may} create a new session, it does not \emph{have to} do so in order
	to terminate.
	%
	\eoe
\end{example}

%%%%%%%%%%%%%%%%%%%%
%%% HIGHER ORDER %%%
%%%%%%%%%%%%%%%%%%%%

\subsection{On Higher-Order Session Types}
\label{ssec:invariant_ch}
\beginbass
%
We have defined fair subtyping in such a way that higher-order session types are \emph{invariant} 
with respect to the type of the channel being exchanged (see \refrule{fs-channel}). 
This is a limitation compared to traditional presentations of unfair subtyping 
\citep{GayHole05,CastagnaDezaniGiachinoPadovani09,BernardiHennessy16}, 
where the covariant/contravariant rules shown below are adopted (\refrule{us-channel-in}, \refrule{us-channel-out}):
\[
        \inferrule{
          \S \usubt \T \\ \U \usubt \V
        }{
          \In\U.\S \usubt \In\V.\T
        }
        \qquad\qquad
        \inferrule{
          \S \usubt \T \\ \V \usubt \U
        }{
          \Out\U.\S \usubt \Out\V.\T
        }
\]

The problem of these rules is that a single application of fair subtyping allowing for 
co-/contra-variance of higher-order session types may have the same overall effect of 
infinitely many applications of fair subtyping on first-order session types and, 
as we have seen in \Cref{ssec:boundedness}, unbounded applications of fair subtyping may compromise fair termination.
%
Below is an example that illustrates the problem. The example is not large \emph{per se}, 
but it is a bit contrived because it has to involve two sessions 
(or else there would be no need for higher-order session types), 
it must be bounded (or else it could be ruled out by the 
action/session/cast boundedness requirements) and non-terminating.

\begin{example}
    \label{ex:invariant_ch}
    \[
    	\begin{array}{@{}r@{~}l@{}}
    	& \NewPar\y{\NewPar\x{\Call A {x,y}}{\Call B x}}{\Call B y}
    	\\
    	\Definition{A}{x,y}{& \POutput\x\lmore.\POutput\x\y.\Call B x}
    	\\
    	\Definition{B}{x}{&
        \PRecv\x{
        	\lmore : \PInput\x{(y)}.\Call A {y,x},
        	\lstop : \Wait\x\Done
        }
    	}	
    	\end{array}
    \]
    The process models a \emph{master} $\Call A {x,y}$ connected with a
	\emph{primary slave} $\Call B x$ and a \emph{secondary slave} $\Call B y$
	through the sessions $x$ and $y$. The interaction among the three processes
	proceeds in rounds. At each round, the master may decide whether to continue or
	stop the interaction by sending either $\lmore$ or $\lstop$ on the session $x$
	to the primary slave. If the master decides to continue the interaction (which
	it does deterministically), it also delegates $y$ to the primary slave so that,
	at the next round, the roles of the three processes rotate: the master is
	downgraded to secondary slave, the primary slave is promoted to master, and the
	secondary slave becomes the primary one.
	%
	\eoe
\end{example}

Below is a graphical representation of
the network topology modeled by the process in \Cref{ex:invariant_ch} and of its evolution:
\[
    \tikzset{baseline=1.25em}
    \tikz[thick]{
    \node (M) at ( 0   ,1) {$\Call A {x,y}$};
    \node (P) at (-0.75,0) {$\Call B x$};
    \node (S) at ( 0.75,0) {$\Call B y$};
    \draw (P) -- (M) -- (S);
    }
    \wred
    \tikz[thick]{
    \node (S) at ( 0   ,1) {$\Call B x$};
    \node (M) at (-0.75,0) {$\Call A {y,x}$};
    \node (P) at ( 0.75,0) {$\Call B y$};
    \draw (P) -- (M) -- (S);
    }
    \wred
    \tikz[thick]{
    \node (P) at ( 0   ,1) {$\Call B x$};
    \node (S) at (-0.75,0) {$\Call B y$};
    \node (M) at ( 0.75,0) {$\Call A {x,y}$};
    \draw (P) -- (M) -- (S);
    }
    \wred
    \tikz[thick]{
    \node (M) at ( 0   ,1) {$\Call A {y,x}$};
    \node (P) at (-0.75,0) {$\Call B y$};
    \node (S) at ( 0.75,0) {$\Call B x$};
    \draw (P) -- (M) -- (S);
    }
    %\wred\cdots
\] 

It is clear that the process in \Cref{ex:invariant_ch} does not terminate since 
there is no $\Close\x$ to match the $\Wait\x\Done$.
%
It is also relatively easy to infer the types of $x$ and $y$ from the structure of 
$A(x,y)$ and $B(x)$. In particular, if we call $S_A$ and $T_A$ the types of $x$ and 
$y$ in $A(x,y)$ and $S_B$ the type of $x$ in $B(x)$ we see that these types must satisfy the equations
\[
\begin{array}{rcl}
	 	S_A & = & \Out\lmore.\Out\T_A.S_B
    \\
    S_B & = & \In\lmore.\In\S_A.T_A \branch \In\lstop.\End[\In]
    \\
    T_A & = & \Out\lmore.\Out\T_A.S_B \choice \Out\lstop.\End[\Out]
\end{array}
\]

Note that $T_A \subt[1] S_A$ holds because $T_A$ and $S_A$ differ only for the topmost output. 
The validity of this relation is unquestionable as it relies on the definition of fair subtyping 
that we have given in \Cref{sec:fair_sub}, which is invariant with respect to higher-order session types.
%
\begin{remark}
	In the following we assume to have two rules for fair subtyping allowing for 
	co-/contra-variance of channel input and output, respectively.
	\[
        \inferrule[ch-in]{
          \S \subt[m] \T \\ \U \subt[n] \V
        }{
          \In\U.\S \subt[k] \In\V.\T
        }\defrule[ch-in]{}
        \qquad\qquad
        \inferrule[ch-out]{
          \S \subt[m] \T \\ \V \subt[n] \U
        }{
          \Out\U.\S \subt[k] \Out\V.\T
        }\defrule[ch-out]{}
	\]
	where $k \in F(m,n)$ concerning the input and $k \in G(m,n)$ concerning the output.
	Note that $F,G : \Nat \times \Nat \rightarrow \mathcal{P_*}(\Nat)$.
	However, \refrule{tb-channel-in} and \refrule{tb-channel-out} are still invariant.
	%
	\eor
\end{remark}
%
If fair subtyping allowed for covariance of higher-order inputs (see \refrule{us-channel-in}, \refrule{us-channel-out}),
then $\co{T_A} \subt S_B$, $\co{S_B} \subt T_A$ (along with $\co{S_B} \subt S_A$ by transitivity of $\subt$)  
would also hold and we would be able to establish that the process 
in \Cref{ex:invariant_ch} is well typed, provided that casts are placed appropriately. 

\begin{example}
	We show the derivation for the judgment $\co{T_A} \subt[k] S_B$ for some $k$.
	\[
	\begin{prooftree}
		\[
			\[
				\vdots
				\justifies
				T_A \subt[1] S_A
			\]
			\[
				\vdots
				\justifies
				\co{T_A} \subt[k] S_B
			\]
			\justifies
			\Out{S_A}.\co{T_A} \subt[m] \Out{T_A}.S_b
			\using\refrule{ch-out}
		\]
		\[
			\vdots
			\justifies
			\Out\lstop.\End[\Out] \subt[0] \Out\lstop.\End[\Out] 
		\]
		\justifies
		\co{S_B} \subt[1] T_A
		\using\refrule{fsb-tag-out-2}
	\end{prooftree}
	\]
	\[
	\begin{prooftree}
		\[
			\[
				\vdots
				\justifies
				T_A \subt[1] S_A
			\]
			\[
				\vdots
				\justifies
				\co{S_B} \subt[1] T_A
				%\using\refrule{fsb-tag-out-2}
			\]
			\justifies
			\In{T_A}.\co{S_B} \subt[k] \In{S_A}.T_A
			\using\refrule{ch-in}
		\]
		\[
			\vdots
			\justifies
			\In\lstop.\End[\In] \subt[0] \In\lstop.\End[\In]
		\]
		\justifies
		\co{T_A} \subt[k] S_B
		\using\refrule{fsb-tag-in}
	\end{prooftree}
	\]
	Note that $m \in G(1,k)$ and $k \in F(1,1)$. Hence, a solution is guaranteed to exist.
	For this reason, in the following we omit the weight of the subtyping judgments involving
	\refrule{ch-in} and \refrule{ch-out}. It suffices to know that such judgments are derivable.
	%
	\eoe
\end{example}

\begin{example}
    \label{ex:annotated_delegation}
    We show a version of the process in \Cref{ex:invariant_ch} in which 
    we have annotated restrictions with the types $\Sys\S\T$ 
		of the two endpoints and casts with the target type of the channel affected by subtyping. 
    \[
    \NewPar{y : \hl{\Sys{T_A}{\co{T_A}}}}{
        \NewPar{x : \hl{\Sys{T_A}{\co{T_A}}}}{
            \Cast{x : \hl{S_A}}
            \Call A {x,y}
        }{
            \Cast{x : \hl{S_B}}
            \Call B x
        }
    }{
        \Cast{y : \hl{S_B}}
        \Call B y
    }
    \]
    We provide another formulation of such process which is well-typed as well.
    \[
    \NewPar{y : \hl{\Sys{\co{S_B}}{S_B}}}{
        \NewPar{x : \hl{\Sys{\co{S_B}}{S_B}}}{
            \Cast{x : \hl{S_A}}
            \Cast{y : \hl{T_A}}
            \Call A {x,y}
        }{
            \Call B x
        }
    }{
        \Call B y
    }
		\]
    %
    \eoe
\end{example}

\begin{example}
	We show that the process in \Cref{ex:annotated_delegation} is well typed.
	Below is the partial proof tree showing that $\Call{A}{x,y}$ is well typed. 
	Each judgment is implicitly annotated with the rank $0$:
	\[
    \begin{prooftree}
    \[
        \[
            \[
                \smash\vdots\mathstrut
                \justifies
                \wtp{x : S_B}{\Call{B}{x}}
                \using\refrule{tb-call}
            \]
            \justifies
            \wtp{x : \Out\T_A.S_B, y : T_A}{
                \POutput\x\y.\Call{B}{x}
            }
            \using\refrule{tb-channel-out}
        \]
        \justifies
        \wtp{x : S_A, y : T_A}{
            \POutput\x\lmore.\POutput\x\y.\Call{B}{x}
        }
        \using\refrule{tb-tag}
    \]
    \justifies
    \wtp{x : S_A, y : T_A}{\Call{A}{x,y}}
    \using\refrule{tb-call}
    \end{prooftree}
	\]
	Below is the partial proof tree showing that $\Call{B}{x}$ is well typed. Each judgment is implicitly annotated with the rank $0$:
	\[
    \begin{prooftree}
    \[
        \[
            \[
                \smash\vdots\mathstrut
                \justifies
                \wtp{x : T_A, y : S_A}{
                \Call{A}{y,x}
                }
                \using\refrule{tb-call}
            \]
            \justifies
            \wtp{x : \In\S_A.T_A}{
                \PInput\x{(y)}.\Call{A}{y,x}
            }
            \using\refrule{tb-channel-in}
        \]
        \[
            \[
                \justifies
                \wtp\emptyset\Done
                %\using\refrule{tb-done}
            \]
            \justifies
            \wtp{x : \End[\In]}{
                \Wait\x\Done
            }
            %\using\refrule{tb-wait}
        \]
        \justifies
        \wtp{x : S_B}{
        \PRecv\x{
            \lmore : \PInput\x{(y)}.\Call{A}{y,x},
            \lstop : \Wait\x\Done
        }
        }
        %\using\refrule{tb-tag}
    \]
    \justifies
    \wtp{x : S_B}{
        \Call{B}{x}
    }
    %\using\refrule{tb-call}
    \end{prooftree}
	\]
	Finally, here is the partial proof tree showing that the process shown in \Cref{ex:annotated_delegation} is well typed
	(again, we omit rank information since \refrule{ch-in} and \refrule{ch-out} are used):
	\[
    \begin{prooftree}
    \[
        \[
            \[
                \smash\vdots\mathstrut
                \justifies
                \wtp{x : T_A, y : S_A}{\Call{A}{x,y}}
                %\using\refrule{tb-call}
            \]
            \justifies
            \wtp{
                x : S_A,
                y : S_A
            }{
                \Cast\x \Call{A}{x,y}
            }
            %\using\refrule{tb-cast}
        \]
        \[
            \[
                \smash\vdots\mathstrut
                \justifies
                \wtp{x : T_B}{\Call{B}{x}}
                %\using\refrule{tb-call}
            \]
            \justifies
            \wtp{
                x : \co{S_A}
            }{
                \Cast\x \Call{B}{x}
            }
            %\using\refrule{tb-cast}
        \]
        \justifies
        \wtp{y : S_A}{
            \NewPar\x{
                \Cast\x \Call{A}{x,y}
            }{
                \Cast\y \Call{B}{x}
            }
        }
       % \using\refrule{tb-par}
    \]
    \[
        \[
            \smash\vdots\mathstrut
            \justifies
            \wtp{
                y : T_B
            }{
                \Call{B}{y}
            }
            %\using\refrule{tb-call}
        \]
        \justifies
        \wtp{
            y : \co{S_A}
        }{
            \Cast\y \Call{B}{y}
        }
        %\using\refrule{tb-cast}
    \]
    \justifies
    \wtp\emptyset{
        \NewPar\y{
            \NewPar\x{
                \Cast\x \Call{A}{x,y}
            }{
                \Cast\x \Call{B}{x}
            }
        }{
            \Cast\y \Call{B}{y}
        }
    }
    %\using\refrule{tb-par}
    \end{prooftree}
	\]
	%
	\eoe
\end{example}

By restricting fair subtyping of higher-order session types to invariant inputs and outputs, 
the only chance we have to build a typing derivation for the process in \Cref{ex:invariant_ch} 
is by casting $y$ each time it is delegated, either before it is sent or after it is received.

\begin{example}
	Consider the following variant of \Cref{ex:invariant_ch}:
	\[
    \begin{array}{r@{~}l}
    \Definition{A}{x : \hl{U_A},y : \hl{V_A}}{&
        \POutput\x\lmore.
        \Cast{y : \hl{U_A}}
        \POutput\x\y.
        \Call B x
    } \\
    \Definition{B}{x : \hl{U_B}}{&
        \PRecv\x{
            \lmore : \PInput\x{(y : \hl{U_A})}.\Call A {y,x},
            \lstop : \Wait\x\Done
        }
    }
    \end{array}
	\]
	where
	\[
	\begin{array}{rcl}
		U_A & = & \Out\lmore.\Out\U_A.U_B \\
		U_B & = & \In\lmore.\In\U_A.V_A \branch \In\lstop.\End[\In] \\
		V_A & = & \Out\lmore.\Out\U_A.U_B \choice \Out\lstop.\End[\Out]
	\end{array}
	\]
	Note that $\Cast{y : U_A}$ is a ``first-order'' cast, in the sense that the relation $V_A \subt[1] U_A$ 
	holds for fair subtyping as defined in \Cref{sec:fair_sub} without using \refrule{us-channel-in} or \refrule{us-channel-out}, 
	but the cast is now placed in a region within the definition of $A$ that prevents finding a finite rank for $A$.
	%
	\eoe
\end{example}
