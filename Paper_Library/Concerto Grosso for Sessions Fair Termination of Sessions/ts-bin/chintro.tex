\begintreble
%
In this chapter we present the first type system for enforcing fair
termination of \emph{binary sessions}. Although in \Cref{ch:ft_multi} we will
present the same approach applied to multiparty ones, hence to a more general context,
the binary case succeeds in highlighting the main requirements, properties and
challenges of such a type system.
%
For this reason, in this chapter we focus on paradigmatic examples in order
to guide the reader across the main developed features. 
%
In \Cref{ssec:proc_ex_multi,ssec:ts_multi_ex} we will show some more involved scenarios.
%
It is worth noting that the proof technique that we adopt to prove the soundness
of the type system is shared by both the binary and the multiparty case. However,
the two proofs differ in some technicalities (\eg deadlock freedom).
%
The type system that we present in this chapter is a refined version of the one
used by \cite{CicconePadovani22}. Indeed, \cite{CicconePadovani22} rely on
the characterization of fair subtyping presented in \Cref{ssec:fsub_gis}.
For the sake of uniformity with respect to \cite{CicconeDP22} on which
\Cref{ch:ft_multi} will be based on,
we refer to the purely coinductive characterization presented in \Cref{sec:fair_sub}.

The chapter is organized as follows.
In \Cref{sec:ts_bin_proc} we present the syntax and the semantics of the
session based calculus on which we apply static analysis.
\Cref{sec:ts_bin_ts} shows the type system for such calculus and introduce
some additional properties that are required to enforce fair termination.
Finally, in \Cref{sec:ts_bin_corr} we detail the soundness proof the type system. 

Concerning a comparison of the type system we show in this chapter with respect
to existing works, we delay the discussion to \Cref{ch:ft_multi}.