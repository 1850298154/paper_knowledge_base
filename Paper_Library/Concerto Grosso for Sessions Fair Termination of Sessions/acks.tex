%%%%%%%%%%%%%%%%%%%%%
%%% Institutional %%%
%%%%%%%%%%%%%%%%%%%%%

\section*{\textit{Institutional}}

\paragraph{To University of Torino and the entire PhD board}
2020 and 2021 have been hard years for everyone. However, UniTo successfully managed to provide
courses and side activities in the best way possible. At the beginning of 2020 TYPES conference should have been
held in Torino but COVID was spreading in Italy. The conference moved to a virtual space. 
No one knew it was the first of a long sequence of remote events. 
PhD related events have been obviously affected as well. 
Nonetheless, courses have been well organized fast in a remote fashion.

\paragraph{To my supervisor Luca Padovani}
I skip the COVID part as it would be obvious and I move to mere professional considerations.
As soon as my PhD started Luca and I discussed about the ways in which we could have taken advantage of
the work I did during the master thesis. This way we immediately found the right intersection between by knowledge
and my PhD project and we started working together. I never worked \emph{for} but \emph{with} him. 
Such an approach led to fast and satisfactory research outcomes. Moreover Luca always shared his connections with other 
research groups, \eg at University of Glasgow.
I have been his first PhD student so I hope that these lines will give him a feedback about his supervision.

\paragraph{To University of Genova and Elena Zucca}
My master thesis supervisor Elena Zucca agreed with Luca for giving me an office at DIBRIS, UniGe. 
This way we could start a fruitful research collaboration. 
During last year we managed to publish together a paper at ECOOP that earned the \emph{distinguished paper} award.
Hence, I'm grateful to both the University of Genova and Elena for allowing me to use such office.

\paragraph{To EuroProofNet, University of Glasgow and Ornela Dardha}
After two years of remote human interactions, in-person events started again. During the pandemic I have been in touch with 
Ornela Dardha whose research interests were very close to mine. Unfortunately I could not have met her research group.
In the meantime EuroProofNet, a european project aimed at connecting researchers on formal proofs, was born.  
Such project gave me the funds for making two visits to Ornela Dardha at University of Glasgow. 
I have been welcomed in the best way. 

\paragraph{To the reviewers}
I'm grateful to Eugenio Moggi and Ornela Dardha for having spent a lot of time 
reading this thesis. They provided useful comments and important references.
I hope this work will be useful for their research groups.

%%%%%%%%%%%%
%%% Pt 2 %%%
%%%%%%%%%%%%

\section*{\textit{...a little flattery}}

\paragraph{To my super-visor Luca Padovani}
Coming to personal considerations, I am grateful to Luca for many reasons.
In general, as an empathetic person I found his approach of sharing not only his ideas
but also his feelings very inspiring. I learned how to face a variety of situations 
thanks to him, from the anxiety in meeting deadlines to the tricks in providing 
successful presentations. These skills will be helpful for my career.
Last but not least, I'd like to thank Luca for his support during the pandemic when I had 
serious COVID problems in my family; we were writing the ICALP paper and the result has been 
successful.

\paragraph{To my unofficial supervisor Elena Zucca}
I'd like to express my gratitude to Elena for her precious suggestions during the PhD and before
starting it. She put me in touch with Luca Padovani at the end of my master thesis.
Although she was not officially reported as my co-supervisor, she demonstrated a high 
involvement in working together as well as with my colleague and friend Francesco Dagnino (UniGe).
As for Luca, I'd like to thank her for her support when I had COVID problems in my family. 
When I was working on the ICALP paper with Luca, in the meantime I was writing the ITP one
with Elena and Francesco. Such paper has been accepted and together with the PPDP one it 
introduced me to the formal proofs community.

\paragraph{To Ornela Dardha}
I had the opportunity of meeting Ornela only last year. She welcomed me in Glasgow 
as a friend more than as a professor. 
I immediately discovered that it was not just 
a special treatment for visitors as she has a very special way of dealing with her research group.
\begin{wrapfigure}{r}{0.45\textwidth}
\centering
    \includegraphics[scale=0.23]{other/ornela_cut.jpg}
    \par\vspace{2pt}
    \textit{SPLS, 26th October 2022.}
\end{wrapfigure}
Although I'm usually not so keen on those situations in which 
\textit{colleagues} and \textit{friends} automatically overlap, I'm sure that Ornela's approach 
is fantastic both for her students and visitors. The reason behind my opinion is that
Ornela is a very empathetic person and she cares about the mood of others a lot.
For this reason I decided to visit her two more times the same year and I really hope to meet her again
even if I will not be involved in the research world anymore.\\
\textit{That's okay for today. Now you should go to the pub! (cit.)}

%%%%%%%%%%%%%%%%
%%% Personal %%%
%%%%%%%%%%%%%%%%

\section*{\textit{Personal}}

\paragraph{To my family}
This is a special paragraph. I'm usually in trouble when I have to write something
to express my gratitude to my family as putting on paper deep feelings is a very hard task 
(and I also have to deal with my eyes filled with tears). What lies behind the usual sentence
\textit{for their support and blah blah}? I would like to completely change the point of view
and try something new. Obviously, the focus is on my parents Renato Ciccone and Marirosa Gaggero.
Let me start with a consideration: I trust stereotypes and in Genova we are famous as people that 
do not want to spend money. My studies were not free of course, so I'm very grateful to my parents 
for the sacrifices they made for paying a lot for so many years. 
Although I managed to win scolarships different times,
I always used such money for whatever but paying university subscriptions; I was younger and naive.
Furthermore, when I decided to start a PhD, I always asked myself whether such a title would have
been useful for my career outside the academy. 
I honestly shared my doubts with my parents. In the end I found 
a satisfactory number of job opportunities positively influenced by these three years of research.
At last, the years of PhD, and in particular the last one, made me grow up a lot as a person.
Hence, I hope to have made my parents proud of the results.

\paragraph{To my grandmother Anna (Ciccone) Peruzzi}
Things become harder. She is a dresskmaker for children, she loves classical music, 
she attends theatres and music associations...more in general, she loves beauty in its essence.
\textit{What is a person without any cultural interest?}, she always asked.
From her I learned the things to appreciate and study in my free time and that I deepened every day.
I'm grateful to her for her ambitions towards me that I hope to have satisfied.

\paragraph{To my ggf Ruggero Peruzzi}
\begin{wrapfigure}{r}{0.35\textwidth}
\centering
    \includegraphics[scale=0.3, angle=180]{other/quadro.jpg}
    \par\vspace{2pt}
    \textit{Ruggero Peruzzi after Edmé Bouchardon.}
\end{wrapfigure}
He was born on 19 Novemeber, 1891. 
He was an antiquarian in Genova. 
My grandmother learned from him the love of beauty. 
During a journey he met the painter C. Tafuri and he paid for him an office in
Palazzo della Borsa in Genova. 
During WW2 he has been captured by fascists as they wanted the names of 
some partisans. He never told such information and fascists started torturing him. 
He had only one opportunity to say goodbye to her daughter.
He has been killed by fascists and his body has been found on 18 February 1945
in front of Brignole train station in Genova where you can find his tombstone.

\vfill

\paragraph{Concerto Grosso?}
This last acknowledgment goes to A. Schnittke, A. Schönberg, A. Berg, J. S. Bach, B. Galuppi, A. Pärt
as well as to many other great composers. In particular, Schnittke's music has been the soundtrack of
the last three years. His \emph{Concerti Grossi} obviously recall baroque music but at the same time they
are progressively cracked and filled with XXth century influences.




