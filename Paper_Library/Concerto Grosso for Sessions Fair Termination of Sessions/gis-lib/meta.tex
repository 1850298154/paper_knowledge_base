\beginalto
%
In this section we describe the main definitions of the library. 
Concerning the meta-theory, the reader can refer to \Cref{sec:gis}.
Notably, we first present an approach mimicking meta-rules. Then we introduce
the notions of \emph{interpretations} and we prove the \emph{principles}.
At last, we formalize the examples we presented in \Cref{sec:gis}.

%%%%%%%%%%%%%%%
%%% META-IS %%%
%%%%%%%%%%%%%%%

\subsection{(Meta-)Rules/Inference System}
\beginbass
%
As anticipated, the aim of the Agda library is to allow a user to write meta-rules as ``on paper''. 
To illustrate this format, let us consider, e.g., the $\allpos$ example from \Cref{sec:gis}:
\[
\inferrule{\allpos{\xs}}{\allpos {\x{:}\xs}}
	~ \x > 0
\]
In a meta-rule, we have \emph{meta-variables}, which range over certain sets, in a way possibly restricted by a \emph{side condition}. 
We call \emph{context} the set of the instantiations of meta-variables which satisfy the side-condition, 
hence produce a rule of the inference system.  
In the example, there are two meta-variables, $\x$ and $\xs$, which range over $\N$ and $\FIList{\N}$, 
respectively, with the restriction that $\x$ should be positive. 
Hence the context is $\{\ple{x,l}\in\N\times\FIList{\N} \mid x > 0 \}$, 
see \Cref{ssec:lib_examples} for the Agda version of this meta-rule.
%
\begin{figure}[t]
\begin{lstlisting}[frame=single]  
record MetaRule {$\ell$c $\ell$p : Level} (U : Set $\ell$u) : Set _ where 
  field 
    Ctx : Set $\ell$c
    Pos : Set $\ell$p 
    prems : Ctx $\rightarrow$ Pos $\rightarrow$ U
    conclu : Ctx $\rightarrow$ U 

  RF[_] : $\forall${$\ell$} $\rightarrow$ (U $\rightarrow$ Set $\ell$) $\rightarrow$ (U $\rightarrow$ Set _)
  RF[_] P u = 
  	$\Sigma$[ c $\in$ Ctx ] (u $\equiv$ conclu c $\times$ ($\forall$ p $\rightarrow$ P (prems c p)))

  RClosed : $\forall${$\ell$} $\rightarrow$ (U $\rightarrow$ Set $\ell$) $\rightarrow$ Set _
  RClosed P = $\forall$ c $\rightarrow$ ($\forall$ p $\rightarrow$ P (prems c p)) $\rightarrow$ P (conclu c)
  
record IS {$\ell$c $\ell$p $\ell$n : Level} (U : Set $\ell$u) : Set _ where
  field
    Names : Set $\ell$n            
    rules : Names $\rightarrow$ MetaRule {$\ell$c} {$\ell$p} U 

  ISF[_] : $\forall${$\ell$} $\rightarrow$ (U $\rightarrow$ Set $\ell$) $\rightarrow$ (U $\rightarrow$ Set _)
  ISF[_] P u = $\Sigma$[ rn $\in$ Names ] RF[ rules rn ] P u

  ISClosed : $\forall${$\ell$} $\rightarrow$ (U $\rightarrow$ Set $\ell$) $\rightarrow$ Set _
  ISClosed P = $\forall$ rn $\rightarrow$ RClosed (rules rn) P  
\end{lstlisting}
\caption{MetaRule datatype}
\label{fig:metarule-dt}
\end{figure} 
%
Correspondingly, the Agda declaration in \Cref{fig:metarule-dt} defines a meta-rule as a record, parametric on the universe \lstinline{U}. 
The first two components are the context and a set of positions for premises. 
For each element of the context (instantiation of meta-variables satisfying the side condition), 
the last two components produce the premises, one for each position, and the conclusion of the 
rule obtained by this instantiation.

Recall that in Agda the declaration \lstinline{U : Set} introduces the type (set) 
\lstinline{U}, and \lstinline{P : U $\rightarrow$ Set} the dependent type (predicate on \lstinline{U}) \lstinline{P}.  
For each element \lstinline{u} of \lstinline{U}, \lstinline{P u} is the type of the proofs that 
\lstinline{u} satifies \lstinline{P}, hence \lstinline{P u} inhabited means that \lstinline{u} satisfies \lstinline{P}. 
To avoid paradoxes, not every Agda type is in \lstinline{Set}; there is an infinite sequence \lstinline{Set 0},
\lstinline{Set 1}, \ldots, \lstinline{Set $\ell$}, \ldots\ such that \lstinline{Set $\ell$ : Set (suc $\ell$)}, 
where $\ell$ is a \emph{level}, and \lstinline{Set} is an abbreviation for \lstinline{Set 0}.  
The programmer can write a wildcard for a level which can be inferred; to make the Agda code 
reported in the paper lighter, we sometimes use a wildcard even for a level which is explicit in the real code. 

\begin{remark}
	In the Agda code in this section, predicates  \lstinline{P : U $\rightarrow$ Set} encode subsets 
	of the universe, so we speak of subsets and membership, 
	rather than of predicates and satisfaction, to closely follow  the previous formulation. 
	%
	\eor
\end{remark}

The function  \lstinline{RF[_]} encodes the inference operator associated with the meta-rule. 
Given a subset \lstinline{P} of the universe, \lstinline{u} belongs to the resulting subset if we can 
find an instantiation \lstinline{c} of meta-variables satisfying the side condition, producing \lstinline{u} 
as conclusion, and, for each position, a premise in \lstinline{P}. 
Note the use of existential quantification \lstinline{$\Sigma$[ x $\in$ A ] B} where \lstinline{B} depends on \lstinline{x}.  
%
The predicate \lstinline{RClosed} encodes the property of being closed with respect to the meta-rule.
A subset \lstinline{P} of the universe is closed if, for each instantiation \lstinline{c} of the 
meta-variables satisfying the side-condition,  if all the premises are in \lstinline{P} then the 
conclusion is in \lstinline{P} as well. 
Note the use of universal quantification 
\lstinline{$\forall$  (x : A) $\rightarrow$ B}, where \lstinline{B} depends on \lstinline{x}.
%
Finally, an inference system is defined in \Cref{fig:metarule-dt} as a record, parametric on the universe \lstinline{U}, 
consisting of a set of meta-rule names and a family of meta-rules. 
The function \lstinline{ISF[_]} and the predicate \lstinline{ISClosed} 
are defined composing those given for a single meta-rule. 
%
\begin{figure}[t]
\begin{lstlisting}[frame=single]
record FinMetaRule {$\ell$c n} (U : Set $\ell$u) : Set _ where
  field
    Ctx : Set $\ell$c
    comp : Ctx $\rightarrow$ Vec U n $\times$ U

  from : MetaRule {$\ell$c} {zero} U
  from .MetaRule.Ctx = Ctx
  from .MetaRule.Pos = Fin n
  from .MetaRule.prems c i = get (proj$_1$ (comp c)) i
  from .MetaRule.conclu c = proj$_2$ (comp c)
\end{lstlisting}
\caption{Finitary meta-rule}
\label{fig:finmetarule-dt}
\end{figure}

Since in practical cases metarules are very often \emph{finitary}, that is,
premises are a finite set, the library also offers an interface to write a (
finitary) meta-rule (see \Cref{fig:finmetarule-dt}),
by providing, besides the context, %the number of premises \lstinline{n}, and 
two components which are the \emph{vector} of premises, with fixed length \lstinline{n}, and the conclusion. 
The injection \lstinline{from} transforms this more concrete format in the generic one for meta-rules, by     
specifying that the set of positions is \lstinline{Fin n} (the set of indexes from $0$ to $n-1$). 



%%%%%%%%%%%%%%%%%%%%%%%%%%%%%%%%%%%%
%%% INTERPRETATIONS / PRINCIPLES %%%
%%%%%%%%%%%%%%%%%%%%%%%%%%%%%%%%%%%%

\subsection{Interpretations and Principles}
\beginbass
%
Recall that the inductive interpretation $\Inductive{\mis}$ of an inference system $\mis$ is the set of 
elements of the universe which have a finite proof tree, and finite proof trees are, in turn, 
inductively defined, that is, by a least fixed point operator.  
In Agda, inductive structures are encoded as \emph{datatypes} (see \Cref{fig:coind-dt}), which specify their constructors. 
%
\begin{figure}[t]
\begin{lstlisting}[frame=single]
data Ind$\llbracket$_$\rrbracket$ {$\ell$c $\ell$p $\ell$n : Level} 
(is : IS {$\ell$c} {$\ell$p} {$\ell$n} U) : U $\rightarrow$  Set _ where
  fold : $\forall$ {u} $\rightarrow$ ISF[ is ] Ind$\llbracket$ is $\rrbracket$ u $\rightarrow$ Ind$\llbracket$ is $\rrbracket$ u
 
record CoInd$\llbracket$_$\rrbracket$ {$\ell$c $\ell$p $\ell$n : Level} 
 (is : IS {$\ell$c} {$\ell$p} {$\ell$n} U) (u : U) : Set _ where
  coinductive
  constructor cofold_
  field
    unfold : ISF[ is ] CoInd$\llbracket$ is $\rrbracket$ u

data SCoInd$\llbracket$_$\rrbracket$ {$\ell$c $\ell$p $\ell$n : Level} 
(is : IS {$\ell$c} {$\ell$p} {$\ell$n} U) : U $\rightarrow$ Size $\rightarrow$ Set _ where
  sfold : $\forall$ {u i} $\rightarrow$ ISF[ is ] ($\lambda$ u $\rightarrow$ Thunk (SCoInd$\llbracket$ is $\rrbracket$ u) i) u 
      $\rightarrow$ SCoInd$\llbracket$ is $\rrbracket$ u i
\end{lstlisting}
\caption{(Co)inductive interpretations - datatype}
\label{fig:coind-dt}
\end{figure}
%
For each \lstinline{u}, \lstinline{Ind$\llbracket$ is $\rrbracket$ u} is the type of the proofs that \lstinline{u}
satisfies \lstinline{Ind$\llbracket$ is $\rrbracket$},  which are essentially the finite proof trees
\footnote{With some more structure, since the Agda proofs keep trace of the applied meta-rules.} for \lstinline{u}. 
Indeed, the \lstinline{fold} constructor, given a proof that \lstinline{u} can be derived  by applying a 
rule from premises belonging to \lstinline{Ind$\llbracket$ is $\rrbracket$},   %that is, having finite proof trees,
which essentially  consists of a rule with conclusion \lstinline{u} and finite proof trees for its premises, 
builds a finite proof tree for \lstinline{u}. 

The coinductive interpretation $\CoInductive{\mis}$ (see \Cref{fig:coind-dt}), instead, is the set of elements of the universe 
which have a possibly infinite proof tree, and possibly infinite proof trees are, in turn, 
coinductively defined, that is, by a greatest fixed point operator. 
For each \lstinline{u}, \lstinline{CoInd$\llbracket$ is $\rrbracket$ u} is the type of the 
proofs that \lstinline{u} satisfies \lstinline{CoInd$\llbracket$ is $\rrbracket$},  
which are essentially the possibly infinite proof trees for \lstinline{u}, 
and analogously for \lstinline{SCoInd$\llbracket$ is $\rrbracket$}. 

\begin{remark}
	\label{rm:agda-coind}
	In Agda, coinductive structures can be encoded in two different ways: 
	either as \emph{coinductive records} \citep{AbelPTS13}, 
	or as  datatypes  by using the mechanism of \emph{thunks} (suspended computations) 
	together with \emph{sized types} \citep{Abel12,AbelP16,AbelVW17}  to ensure termination. 
\end{remark}  

To allow compatibility with existing code implemented in either way, 
both versions in \Cref{rm:agda-coind} are supported by the library.
%
In the first version, a possibly infinite proof tree for \lstinline{u} is a record with 
only one field \lstinline{unfold} containing an element of \lstinline{ISF[ is ] CoInd$\llbracket$ is $\rrbracket$ u}, 
that is, a proof that \lstinline{u} can be derived by applying a rule from premises 
belonging to \lstinline{CoInd$\llbracket$ is $\rrbracket$}, %that is, with possibly infinite proof trees.  
which  essentially consists of a rule with conclusion \lstinline{u} and possibly infinite proof trees for its premises.  
%
In the second version, a possibly infinite proof tree is obtained by a \lstinline{data} constructor, 
analogously to a finite one in the inductive interpretation; 
however, since proof trees are encoded as thunks, hence evaluated lazily,
this encoding represents infinite trees as well. 
In other words, coinduction is ``hidden'' in the library type \lstinline{Thunk}, 
which is a coinductive record with only one field \lstinline{force}, 
intuitively representing the suspended computation.

The interpretation of a generalized inference system (see \Cref{fig:fcoind-dt}) can then be encoded following exactly 
the definition in \Cref{sec:gis}: it is the coinductive interpretation of \lstinline{I}, restricted to
rules whose conclusion is in the inductive interpretation of the (standard) inference 
system consisting of both rules \lstinline{I} and corules \lstinline{C}.   
%
\begin{figure}[t]
\begin{lstlisting}[frame=single]
_$\sqcap$_ : $\forall$ {$\ell$c $\ell$p $\ell$n $\ell$}{U : Set $\ell$u} $\rightarrow$ IS {$\ell$c} {$\ell$p} {$\ell$n} U 
  $\rightarrow$ (U $\rightarrow$ Set $\ell$) $\rightarrow$ IS {$\ell$c $\sqcup$ $\ell$} {_} {_} U
(is $\sqcap$ P) .Names = is .Names
(is $\sqcap$ P) .rules rn = addSideCond (is .rules rn) P

_$\cup$_ : $\forall${$\ell$c $\ell$p $\ell$n $\ell$n'}{U : Set $\ell$} $\rightarrow$ IS {$\ell$c} {$\ell$p} {$\ell$n} U 
  $\rightarrow$ IS {_} {_} {$\ell$n'} U $\rightarrow$ IS {_} {_} {$\ell$n $\sqcup$ $\ell$n'} U
(is1 $\cup$ is2) .Names = (is1 .Names) $\uplus$ (is2 .Names)
(is1 $\cup$ is2) .rules = [ is1 .rules , is2 .rules ]

FCoInd$\llbracket$_,_$\rrbracket$ : $\forall${$\ell$c $\ell$p $\ell$n $\ell$n'} $\rightarrow$ (I : IS {$\ell$c} {$\ell$p} {$\ell$n} U) 
  $\rightarrow$ (C : IS {$\ell$c} {$\ell$p} {$\ell$n'} U) $\rightarrow$ U $\rightarrow$ Set _
FCoInd$\llbracket$ I , C $\rrbracket$ = CoInd$\llbracket$ I $\sqcap$ Ind$\llbracket$ I $\cup$ C $\rrbracket$ $\rrbracket$

SFCoInd$\llbracket$_,_$\rrbracket$ : $\forall${$\ell$c $\ell$p $\ell$n $\ell$n'} $\rightarrow$ (I : IS {$\ell$c} {$\ell$p} {$\ell$n} U) 
  $\rightarrow$ (C : IS {$\ell$c} {$\ell$p} {$\ell$n'} U) $\rightarrow$ U $\rightarrow$ Size $\rightarrow$ Set _
SFCoInd$\llbracket$ I , C $\rrbracket$ = SCoInd$\llbracket$ I $\sqcap$ Ind$\llbracket$ I $\cup$ C $\rrbracket$ $\rrbracket$
\end{lstlisting}
\caption{Interpretation generated by corules - datatype}
\label{fig:fcoind-dt}
\end{figure}
%    
The definition is provided in two flavours where the coinductive interpretation 
is encoded by coinductive records and thunks, respectively, and uses two operators on inference systems, 
restriction $\sqcap$ and union $\cup$. We report the codes in \Cref{fig:fcoind-dt}. 
The former adds  to each rule the side condition that the conclusion should satisfy \lstinline{P}, 
as specified by the function \lstinline{addSideCond} (here omitted). On the other hand, $\cup$ joins two inference
systems.

The library also provides the proofs of relevant properties, e.g., that closed sets coincide with pre-fixed points, 
and consistent sets coincide with post-fixed points. Moreover, it is shown that the two versions of encoding of the
coinductive interpretation (by coinductive records and thunks) are equivalent.    
Finally, the library provides  the induction, coinduction, and bounded coinduction principles (see \Cref{prop:indp,prop:coindp,prop:bcp}). 
We only report the statements in \Cref{fig:principles} and we briefly recall their meaning.
%
\begin{figure}[t]
\begin{lstlisting}[frame=single]
ind[_] : $\forall${$\ell$c $\ell$p $\ell$n $\ell$} 
    $\rightarrow$ (is : IS {$\ell$c} {$\ell$p} {$\ell$n} U)		-- IS
    $\rightarrow$ (S : U $\rightarrow$ Set $\ell$)			-- specification
    $\rightarrow$ ISClosed is S			-- S is closed
    $\rightarrow$ Ind$\llbracket$ is $\rrbracket$ $\subseteq$ S

coind[_] : $\forall${$\ell$c $\ell$p $\ell$n $\ell$}
    $\rightarrow$ (is : IS {$\ell$c} {$\ell$p} {$\ell$n} U) 
    $\rightarrow$ (S : U $\rightarrow$ Set $\ell$)
    $\rightarrow$ (S $\subseteq$ ISF[ is ] S)		-- S is consistent
    $\rightarrow$ S $\subseteq$ CoInd$\llbracket$ is $\rrbracket$
 
bounded-coind[_,_] : $\forall${$\ell$c $\ell$p $\ell$n $\ell$n' $\ell$} 
    $\rightarrow$ (I : IS {$\ell$c} {$\ell$p} {$\ell$n} U)
    $\rightarrow$ (C : IS {$\ell$c} {$\ell$p} {$\ell$n'} U)
    $\rightarrow$ (S : U $\rightarrow$ Set $\ell$)                   
    $\rightarrow$ S $\subseteq$ Ind$\llbracket$ I $\cup$ C $\rrbracket$		-- S is bounded w.r.t. I $\cup$ C
    $\rightarrow$ S $\subseteq$ ISF[ I ] S		-- S is consistent w.r.t. I
    $\rightarrow$ S $\subseteq$ FCoInd$\llbracket$ I , C $\rrbracket$
\end{lstlisting}
\caption{Proof principles}
\label{fig:principles}
\end{figure}

\begin{itemize}
\item If \lstinline{S} is closed, then each element of the inductively defined 
set \lstinline{Ind$\llbracket$ is $\rrbracket$} satisfies  \lstinline{S}.
\item If \lstinline{S} is consistent, then each element satisfying \lstinline{S} is in the
coinductively defined set \lstinline{CoInd$\llbracket$ is $\rrbracket$}. 
\item If \lstinline{S} is bounded, and  consistent with respect to \lstinline{I},
then each element which satisfies \lstinline{S} belongs to the set 
\lstinline{FCoInd$\llbracket$ I , C $\rrbracket$} defined by flexible coinduction. 
\end{itemize}
 
Another useful theorem is that 
$\FlexCo{\mis}{\mcois}\subseteq\Ind{\mis\cup\mcois}$ (see \Cref{fig:fcoind-to-ind}). 

\begin{figure}[t]
\begin{lstlisting}[frame=single]
fcoind-to-ind : $\forall${$\ell$c $\ell$p $\ell$n $\ell$n'}
    {is : IS {$\ell$c} {$\ell$p} {$\ell$n} U}{cois : IS {$\ell$c} {$\ell$p} {$\ell$n'} U} 
    $\rightarrow$ FCoInd$\llbracket$ is , cois $\rrbracket$ $\subseteq$ Ind$\llbracket$ is $\cup$ cois $\rrbracket$
\end{lstlisting}
	\caption{Extract inductive proof}
	\label{fig:fcoind-to-ind}
\end{figure}

%%%%%%%%%%%%%%%%%%%%%%
%%% BASIC EXAMPLES %%%
%%%%%%%%%%%%%%%%%%%%%%

\subsection{Basic Examples}
\label{ssec:lib_examples}
\beginbass
%
We continue this section by showing how to use the library to define specific inference systems and prove their properties. 
In particular, we consider the basic examples in \Cref{sec:gis} that allow us to cover
all the cases that we investigated before.

\begin{example}
	Consider the predicate $\member$. We first recall its inference system and we give names to the (meta-)rules.
		\begin{mathpar}
		\inferrule[mem-h]{\mathstrut}{\member(\x , \cons\x\xs)}
		\and
		\inferrule[mem-t]{\member(\x ,\xs)}{\member(\x , \cons\y\xs)}
	\end{mathpar} 
	The universe consists of pairs of elements and possibly infinite lists, 
	implemented by the Agda library \lstinline{Colist} which uses thunks.
	
\begin{lstlisting}
U = A $\times$ Colist A $\infty$
data memberRN : Set where mem$\textrm{-}$h mem$\textrm{-}$t : memberRN

mem$\textrm{-}$h$\textrm{-}$r : FinMetaRule U
mem$\textrm{-}$h$\textrm{-}$r .Ctx = A $\times$ Thunk (Colist A) $\infty$
mem$\textrm{-}$h$\textrm{-}$r .comp (x , xs) =
 	[] ,
 	----------------
 	(x , x :: xs) 

mem$\textrm{-}$t$\textrm{-}$r : FinMetaRule U
mem$\textrm{-}$t$\textrm{-}$r .Ctx = A $\times$ A $\times$ Thunk (Colist A) $\infty$
mem$\textrm{-}$t$\textrm{-}$r .comp (x , y , xs) =
 	((x , xs .force) :: []) ,
 	----------------
 	(x , y :: xs) 

memberIS : IS U
memberIS .Names = memberRN
memberIS .rules mem$\textrm{-}$h = from mem$\textrm{-}$h$\textrm{-}$r
memberIS .rules mem$\textrm{-}$t = from mem$\textrm{-}$t$\textrm{-}$r
\end{lstlisting}

	Here \lstinline{memberRN} are the rule names, and each rule name has an associated element of 
	\lstinline{FinMetaRule U}, which exactly encodes the meta-rule in the inference system at the beginning. 
	Note, in \lstinline{mem$\textrm{-}$t$\textrm{-}$r}, the use of the \lstinline{force} field 
	of \lstinline{Thunk} to actually obtain the tail colist. 
	%
	This inference system is expected to define exactly the pairs \lstinline{(x , xs)} such that 
	\lstinline{x} belongs to \lstinline{xs}, that is, those satisfying the following specification
	%
\begin{lstlisting}	
memSpec : U $\rightarrow$ Set
memSpec (x , xs) = $\Sigma$[ i $\in$ $\N$ ] (Colist.lookup i xs = just x)
\end{lstlisting} 
	%
	where \lstinline{lookup : $\N$ $\rightarrow$ Colist A $\infty$ $\rightarrow$ Maybe A} 
	is the (standard) library function that returns the \lstinline{i}-th element of \lstinline{xs}, if any. 
	%
	As said in \Cref{sec:gis}, to obtain the desired meaning this inference system has to be 
	interpreted inductively, and soundness can be proved by the induction principle, 
	that is, by providing a proof that the specification is closed with respect to the two meta-rules, as shown below. 

\begin{lstlisting}
_member_ : A $\rightarrow$ Colist A $\infty$ $\rightarrow$ Set
x member xs = Ind$\llbracket$ memberIS $\rrbracket$ (x , xs)

memSpecClosed : ISClosed memberIS memSpec
memSpecClosed mem$\textrm{-}$h _ _ = zero , refl
memSpecClosed mem$\textrm{-}$t _ pr =
 	let (i , proof) = pr Fin.zero in (suc i) , proof

memberSound : $\forall$ {x xs} $\rightarrow$ x member xs $\rightarrow$ memSpec (x , xs)
memberSound = ind[memberIS] memSpec memSpecClosed
\end{lstlisting}

	For completeness there is no canonical technique; in this example, it can be proved by induction 
	on the position (the index \lstinline{i} in the specification). 
	For the complete proof see \cite{Ciccone20}.
	%
	\eoe
\end{example}

%%%%%%%%%%%%%%
%%% ALLPOS %%%
%%%%%%%%%%%%%%

\begin{example}
	Consider the predicate $\allpos$ from \Cref{sec:gis}. We first recall its inference system and we give names to the (meta-)rules.
	\begin{mathpar}
	\inferrule[allP-$\Lambda$]{\mathstrut}{\allpos\nil}
	\and
	\inferrule[allP-t]{\allpos\xs}{\allpos \cons\x\xs} ~ \x > 0
	\end{mathpar}
	Then the universe consists of possibly infinite lists. 

\begin{lstlisting}
U : Set
U = Colist $\N$ $\infty$
data allPosRN : Set where allP$\textrm{-}\Lambda$ allP$\textrm{-}$t : allPosRN

allP$\textrm{-}\Lambda\textrm{-}$r : FinMetaRule U
allP$\textrm{-}\Lambda\textrm{-}$r .Ctx = $\top$
allP$\textrm{-}\Lambda\textrm{-}$r .comp c =
  [] ,
  -----------------
  [] 

allP$\textrm{-}$t$\textrm{-}$r : FinMetaRule U
allP$\textrm{-}$t$\textrm{-}$r .Ctx = $\Sigma$[ (x , _) $\in$ $\N$ $\times$ Thunk (Colist $\N$) $\infty$ ] x > 0
allP$\textrm{-}$t$\textrm{-}$r .comp ((x , xs) , _) =
  ((xs .force) :: []) ,
  -----------------
  (x :: xs)

allPosIS : IS U
allPosIS .Names = allPosRN
allPosIS .rules allP$\textrm{-}\Lambda$ = from allP$\textrm{-}\Lambda\textrm{-}$r
allPosIS .rules allP$\textrm{-}$t = from allP$\textrm{-}$t$\textrm{-}$r
\end{lstlisting}

	This inference system is expected to define exactly the lists such that all elements are positive, 
	that is, those satisfying the following specification 
	(where for simplicity, we use the predicate $\in$, omitted, directly defined inductively). 
	Notably, we could use $\member$ instead of $\in$.

\begin{lstlisting}
allPosSpec : U $\rightarrow$ Set
allPosSpec xs = $\forall$ {x} $\rightarrow$ x $\in$ xs $\rightarrow$ x > 0
\end{lstlisting}

	As said in \Cref{sec:gis}, to obtain the desired meaning this inference system has 
	to be interpreted coinductively, and completeness can be proved by the coinduction principle, 
	that is, by providing a proof that the specification is consistent with respect to the inference system, as shown below. 

\begin{lstlisting}
allPos : U $\rightarrow$ Set
allPos = CoInd$\llbracket$ allPosIS $\rrbracket$

allPosSpecCons : $\forall$ {xs} 
	$\rightarrow$ allPosSpec xs $\rightarrow$ ISF[ allPosIS ] allPosSpec xs
allPosSpecCons {[]} _ = allP$\textrm{-}\Lambda$ , (tt , (refl , tt , $\lambda$ ()))
allPosSpecCons {(x :: xs)} Sxs = 
  allP$\textrm{-}$t , 
  ((x , xs) , (refl , 
  		(Sxs here , 
  		$\lambda$ {Fin.zero $\rightarrow$ $\lambda$ mem $\rightarrow$ Sxs (there mem)})))

allPosComplete : allPosSpec $\subseteq$ allPos
allPosComplete = coind[ allPosIS ] allPosSpec allPosSpecCons
\end{lstlisting}

	For what concerns the soundness, there is no canonical technique; in this example, 
	when the colist is empty the proof that the specification holds is trivial.
	If the colist is not empty, then the proof proceeds by induction on 
	the position of the element to be proved to be positive.
	For the complete proof see \cite{Ciccone20}.
 	%
 	\eoe
\end{example}

\begin{example}
	Consider the predicate $\maxelem$ from \Cref{sec:gis}. We recall once again its meta-(co)rules.
	\begin{mathpar}
		\inferrule[max-h]{\mathstrut}{\maxelem(\x , \cons\x\nil)}
		\and
		\inferrule[max-t]{\maxelem(\x , \xs)}{\maxelem(max(\x , \y) , \cons\y\xs)}
		\and
		\infercorule[co-max-h]{\mathstrut}{\maxelem(\x , \cons\x\xs)}
	\end{mathpar}
	Then the universe consists of pairs of natural numbers and possibly infinite lists.

\begin{lstlisting}
U : Set
U = $\N$ $\times$ Colist $\N$ $\infty$
data maxElemRN : Set where max$\textrm{-}$h max$\textrm{-}$t : maxElemRN
data maxElemCoRN : Set where co$\textrm{-}$max$\textrm{-}$h : maxElemCoRN

max$\textrm{-}$h$\textrm{-}$r : FinMetaRule U
max$\textrm{-}$h$\textrm{-}$r .Ctx = 
  $\Sigma$[ (_ , xs) $\in$ $\N$ $\times$ Thunk (Colist $\N$) $\infty$ ] xs .force $\equiv$ []
max$\textrm{-}$h$\textrm{-}$r .comp ((x , xs) , _) =
  [] ,
  --------------
  x , x :: xs

max$\textrm{-}$t$\textrm{-}$r : FinMetaRule U
max$\textrm{-}$t$\textrm{-}$r .Ctx = 
  $\Sigma$[ (x , y , z , _) $\in$ 
  $\N$ $\times$ $\N$ $\times$ $\N$ $\times$ Thunk (Colist $\N$) $\infty$ ] z $\equiv$ max x y 
max$\textrm{-}$t$\textrm{-}$r .comp ((x , y , z , xs) , _) =
  (x , xs .force) :: [] ,
  --------------
  z , y :: xs

co$\textrm{-}$max$\textrm{-}$h$\textrm{-}$r : FinMetaRule U
co$\textrm{-}$max$\textrm{-}$h$\textrm{-}$r .Ctx = $\N$ $\times$ Thunk (Colist $\N$) $\infty$
co$\textrm{-}$max$\textrm{-}$h$\textrm{-}$r .comp (x , xs) =
  [] ,
  --------------
  (x , x :: xs) 

maxElemIS : IS U
maxElemIS .Names = maxElemRN
maxElemIS .rules max$\textrm{-}$h = from max$\textrm{-}$h$\textrm{-}$r
maxElemIS .rules max$\textrm{-}$t = from max$\textrm{-}$t$\textrm{-}$r

maxElemCoIS : IS U
maxElemCoIS .Names = maxElemCoRN
maxElemCoIS .rules co$\textrm{-}$max$\textrm{-}$h = from co$\textrm{-}$max$\textrm{-}$h$\textrm{-}$r
\end{lstlisting}

	Note that in this example we have defined two inference systems, the rules and the corules. 
	This generalized inference system is expected to define exactly the pairs 
	\lstinline{(x , xs)} such that \lstinline{x} is the maximal element of \lstinline{xs}, 
	that is, those satisfying the following specification, where to be the maximal element 
	\lstinline{x} should belong to \lstinline{xs}, and be greater or equal than any \lstinline{n} in \lstinline{xs}.

\begin{lstlisting}
maxSpec inSpec geqSpec : U $\rightarrow$ Set
inSpec (x , xs) = x $\in$ xs
geqSpec (x , xs) = $\forall${n} $\rightarrow$ n $\in$ xs $\rightarrow$ x $\equiv$ max x n
maxSpec u = inSpec u $\times$ geqSpec u
\end{lstlisting}

	As said in \Cref{sec:gis}, the desired meaning is provided by the interpretation of the generalized inference system.
	
\begin{lstlisting}
_maxElem_ : $\N$ $\rightarrow$ Colist $\N$ $\infty$ $\rightarrow$ Set
x maxElem xs = FCoInd$\llbracket$ maxElemIS , maxElemCoIS $\rrbracket$ (x , xs)
\end{lstlisting}

	and the completeness can be proved by the bounded coinduction principle. 

\begin{lstlisting}
maxElemComplete : $\forall${x xs} $\rightarrow$ maxSpec (x , xs) $\rightarrow$ x maxElem xs
maxElemComplete =
  bounded$\textrm{-}$coind[ maxElemIS , maxElemCoIS ] maxSpec 
    ($\lambda${(x , xs) $\rightarrow$ maxSpecBounded x xs}) 
    $\lambda${(x , xs) $\rightarrow$ maxSpecCons x xs}
\end{lstlisting}

	Notably, we have to prove that the specification is:
	\begin{itemize}
	\item \emph{bounded}, that is, contained in \lstinline{_maxElem$_i$_ }, the inductive interpretation of the 
	standard inference system consisting of both rules and corules, as shown below:
	
\begin{lstlisting}
_maxElem$_i$_ : $\N$ $\rightarrow$ Colist $\N$ $\infty$ $\rightarrow$ Set
x maxElem$_i$ xs = Ind$\llbracket$ maxElemIS $\cup$ maxElemCoIS $\rrbracket$ (x , xs)

maxSpecBounded : $\forall${x xs} $\rightarrow$ inSpec (x , xs) 
  $\rightarrow$ geqSpec (x , xs) $\rightarrow$ x maxElem$_i$ xs
\end{lstlisting}

	\item \emph{consistent} with respect to the inference system consisting of only rules, as shown below:
	
\begin{lstlisting}
maxSpecCons : $\forall${x xs} $\rightarrow$ inSpec (x , xs) $\rightarrow$ 
  geqSpec (x , xs) $\rightarrow$ ISF[ maxElemIS ] maxSpec (x , xs)
\end{lstlisting}

	\end{itemize}
	These proofs are omitted for the sake of brevity. See \cite{Ciccone20} for more details.
	%
	Concerning the soundness there is no canonical technique. 
	The proof can be split for the two components of the specification. 
	It is worth noting that, for the soundness with respect to \lstinline{inSpec}, 
	we  first use \lstinline{fcoind$\textrm{-}$to$\textrm{-}$ind} (see \Cref{fig:fcoind-to-ind}),
	and then define \lstinline{maxElemSound$\textrm{-}$in$\textrm{-}$ind}, omitted, 
	by induction on the inference system consisting of rules and corules.  
	The use of \lstinline{fcoind$\textrm{-}$to$\textrm{-}$ind} in the proof corresponds to the fact 
	that without corules unsound judgments could be derived. 
	
\begin{lstlisting}
maxElemSound$\textrm{-}$in : $\forall$ {x xs} $\rightarrow$ x maxElem xs $\rightarrow$ inSpec (x , xs)
maxElem$\textrm{-}$sound$\textrm{-}$in max = maxElemSound$\textrm{-}$in$\textrm{-}$ind (fcoind$\textrm{-}$to$\textrm{-}$ind max)
\end{lstlisting}

	Soundness with respect to \lstinline{geqSpec} is proved by induction on the position, 
	that is, the proof of membership, of the element that must be proved to be less or equal.  
	In this case, soundness would hold even in the purely coinductive case. 
	%
	\eoe
\end{example}

\begin{remark}[Code Duplication]
	\label{rm:agda_dup}
	Of course, as Agda supports both inductive and coinductive dependent types, 
	one could directly write Agda code for inductive, coinductive and 
	even flexible coinductive definitions of concrete examples. 
	We have explored this possibility in \cite{Ciccone20}. 
	However, in this way, the definition is hard-wired with its semantics, and, for flexible coinduction, 
	one has to manually construct the interpretation by combining in the correct way an inductive and a 
	coinductive type and to prove the bounded coinduction principle for each example. 
	For instance, the definition of \lstinline{maxElem} will look as follows: 
	
\begin{lstlisting}
data _maxElem_ : $\N$ $\to$ CoList $\N$ $\infty$ $\to$ Size $\to$ Set where 
  max$\textrm{-}$h : $\forall$ {x xs i} $\to$force xs $\equiv$ [] $\to$ x maxElem (x :: xs) i 
  max$\textrm{-}$t : $\forall$ {x y xs i} $\to$ Thunk (x maxElem (force xs)) i 
                       $\to$ z $\equiv$ max x y 
                       $\to$ z maxElem$_i$ (y :: xs) 
                       $\to$ z maxElem (y :: xs) i  

data _maxElem$_i$_ : $\N$ $\to$ CoList $\N$ $\infty$ $\to$ Set where 
  imax$\textrm{-}$h : $\forall$ {x xs} $\to$force xs $\equiv$ [] $\to$ x maxElem$_i$ (x :: xs) 
  imax$\textrm{-}$t : $\forall$ {x y xs} $\to$ x maxElem$_i$ (force xs)) $\to$ z $\equiv$ max x y 
                      $\to$ z maxElem$_i$ (y :: xs)  
  co$\textrm{-}$max$\textrm{-}$h : $\forall$ {x xs} $\to$ x maxElem$_i$ (x :: xs) 
\end{lstlisting} 

	Clearly, this approach causes duplication of rules and code, as rules of the coinductive type have to be duplicated in the inductive one, 
	making things rather complex. 
	Our library instead hides all these details, exposing interfaces for interpretations and proof principles, 
	so that the user only has to write code describing rules. 
	%
	\eor
\end{remark}
