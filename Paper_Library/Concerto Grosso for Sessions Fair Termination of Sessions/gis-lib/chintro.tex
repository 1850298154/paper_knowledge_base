\begintreble
%
In this chapter we present a library for supporting generalized inference systems (see \Cref{sec:gis}) in Agda \citep{Agda}.
Summarizing from \Cref{sec:gis},
an \emph{inference system} \citep{Aczel77,LeroyG09,Sangiorgi11}, that is, a set of (meta-)rules stating that 
a consequence can be derived from a set of premises, is a simple, general and widely-used way to express 
and reason about a recursive definition. In most cases such recursive definition is seen as inductive, 
that is,  the denoted set consists of the elements with a finite derivation. 
This  enables \emph{inductive reasoning}, that is, to prove that the elements an inductively 
defined set satisfy a property, it is enough to show that, for each (meta-)rule, the property 
holds for the consequence assuming that it holds for the premises.
In other cases, the recursive definition is seen as coinductive, 
that is, the denoted set  consists of the elements with a possibly infinite derivation. 
This enables \emph{coinductive reasoning}, that is, to prove that all the elements satisfying a 
property belong to the coinductively defined set, it is enough to show that, when the property 
holds for an element, it can be derived from premises for which the property holds as well. 
Recently, a generalization of inference systems has been proposed \citep{AnconaDagninoZucca17,Dagnino19,Dagnino21} 
which handles cases where neither the inductive, nor the purely coinductive intepretation provides the desired meaning. 
This approach is called \emph{flexible coinduction}, and, correspondingly, coinductive reasoning is 
generalized as well by a principle which is called \emph{bounded coinduction}.

The Agda proof assistant \citep{Agda} offers language constructs to inductively/coinductively define predicates, 
and correspondingly  built-in proof principles. 
However, in this way the recursive definition is monolithic, and hard-wired with its chosen interpretation.
%
\begin{remark}
	In \cite{Ciccone20} we deeply investigated the built-in features of Agda by
	inspecting all the different ways a (co)inductive predicate can be defined.
	We found out that, while pure (co)inductive definitions are easily supported, a predicate mixing 
	both approaches led to complex codes with many duplicated notions due to the lack 
	of modularity (see \Cref{rm:agda_dup}).
\end{remark} 
%
Our aim, instead, is to provide an Agda library allowing  the user to express a recursive definition 
as an \emph{instance of a parametric type} of inference systems.  
In this way, the user is not committed from the beginning to a given interpretation but, rather, 
gets for free a bunch of properties which have been proved once and for all, including the inductive and 
coinductive intepretation and the corresponding proof principles. 
Moreover, it is possible to define composition operators on inference systems, 
for instance union and restriction.
Finally, flexible coinduction is modularly obtained as well, 
by composing in a certain way the interpretations of two inference systems. 

\emph{Indexed containers} \citep{AltenkirchGHMM15} provide a way to specify possibly recursive definitions 
of predicates independently from their interpretation and are supported in the Agda standard library. 
An Agda implementation of inference systems can be provided by seeing them as indexed containers. 
However, this approach requires to structure definitions in an unusual way. 
Indeed, inference systems are usually presented through a (finite) set of \emph{meta-rules}, 
denoting all the rules which can be obtained by instantiating meta-variables with values satisfying the side condition. 
Hence, we provide a different implementation following this schema, to allow users to write their own inference 
system in an Agda format which closely resembles  that ``on paper''. 
We then prove that the two implementations are equivalent, showing that every  indexed container can be 
encoded in terms of meta-rules and viceversa.

\begin{remark}[Agda Version \& Reference]
	\label{rm:agda_v_link}
	It is important to point out that the material presented in \Cref{ch:gis_lib,ch:agda_prop}
	holds as long as Agda is consistent.
	During the development of the notions we are going to present, Agda received many updates.
	As an example, an inconsistency in mixing \emph{sized types} and \emph{coinductive records}
	has been found.
	%
	The library is available on GitHub (see \cite{GisLib}) and has been tested
	with the Agda 2.6.2.2. In case of future Agda updates, we will try to keep the
	material updated.  
	%
	\eor
\end{remark}

The chapter is structured as follows.
In \Cref{sec:agda_gis_meta} we show the mechanization of the meta theory of (generalized) inference systems (see \Cref{sec:gis}).
In particular, we present the main datatypes to encode inference systems in Agda, how to obtain their interpretations and,
most important, the proof principles. We provide the formalization of the examples that we introduce in \Cref{sec:gis} since
they give an overview of all the features of the library.
Then, in \Cref{sec:agda_gis_lambda} we show a more involved example, that is, a lambda calculus with divergence and
we prove soundness and completeness of its big-step semantics.
At last, in \Cref{sec:agda_gis_container} we investigate the relation between inference systems and indexed containers
\citep{AltenkirchGHMM15} that are supported by the standard library. 