\beginbass
%
Recall that the inductive interpretation $\Inductive{\mis}$ of an inference system $\mis$ is the set of 
elements of the universe which have a finite proof tree, and finite proof trees are, in turn, 
inductively defined, that is, by a least fixed point operator.  
In Agda, inductive structures are encoded as \emph{datatypes} (see \Cref{fig:coind-dt}), which specify their constructors. 
%
\begin{figure}[t]
\begin{lstlisting}[frame=single]
data Ind$\llbracket$_$\rrbracket$ {$\ell$c $\ell$p $\ell$n : Level} 
(is : IS {$\ell$c} {$\ell$p} {$\ell$n} U) : U $\rightarrow$  Set _ where
  fold : $\forall$ {u} $\rightarrow$ ISF[ is ] Ind$\llbracket$ is $\rrbracket$ u $\rightarrow$ Ind$\llbracket$ is $\rrbracket$ u
 
record CoInd$\llbracket$_$\rrbracket$ {$\ell$c $\ell$p $\ell$n : Level} 
 (is : IS {$\ell$c} {$\ell$p} {$\ell$n} U) (u : U) : Set _ where
  coinductive
  constructor cofold_
  field
    unfold : ISF[ is ] CoInd$\llbracket$ is $\rrbracket$ u

data SCoInd$\llbracket$_$\rrbracket$ {$\ell$c $\ell$p $\ell$n : Level} 
(is : IS {$\ell$c} {$\ell$p} {$\ell$n} U) : U $\rightarrow$ Size $\rightarrow$ Set _ where
  sfold : $\forall$ {u i} $\rightarrow$ ISF[ is ] ($\lambda$ u $\rightarrow$ Thunk (SCoInd$\llbracket$ is $\rrbracket$ u) i) u 
      $\rightarrow$ SCoInd$\llbracket$ is $\rrbracket$ u i
\end{lstlisting}
\caption{(Co)inductive interpretations - datatype}
\label{fig:coind-dt}
\end{figure}
%
For each \lstinline{u}, \lstinline{Ind$\llbracket$ is $\rrbracket$ u} is the type of the proofs that \lstinline{u}
satisfies \lstinline{Ind$\llbracket$ is $\rrbracket$},  which are essentially the finite proof trees
\footnote{With some more structure, since the Agda proofs keep trace of the applied meta-rules.} for \lstinline{u}. 
Indeed, the \lstinline{fold} constructor, given a proof that \lstinline{u} can be derived  by applying a 
rule from premises belonging to \lstinline{Ind$\llbracket$ is $\rrbracket$},   %that is, having finite proof trees,
which essentially  consists of a rule with conclusion \lstinline{u} and finite proof trees for its premises, 
builds a finite proof tree for \lstinline{u}. 

The coinductive interpretation $\CoInductive{\mis}$ (see \Cref{fig:coind-dt}), instead, is the set of elements of the universe 
which have a possibly infinite proof tree, and possibly infinite proof trees are, in turn, 
coinductively defined, that is, by a greatest fixed point operator. 
For each \lstinline{u}, \lstinline{CoInd$\llbracket$ is $\rrbracket$ u} is the type of the 
proofs that \lstinline{u} satisfies \lstinline{CoInd$\llbracket$ is $\rrbracket$},  
which are essentially the possibly infinite proof trees for \lstinline{u}, 
and analogously for \lstinline{SCoInd$\llbracket$ is $\rrbracket$}. 

\begin{remark}
	\label{rm:agda-coind}
	In Agda, coinductive structures can be encoded in two different ways: 
	either as \emph{coinductive records} \citep{AbelPTS13}, 
	or as  datatypes  by using the mechanism of \emph{thunks} (suspended computations) 
	together with \emph{sized types} \citep{Abel12,AbelP16,AbelVW17}  to ensure termination. 
\end{remark}  

To allow compatibility with existing code implemented in either way, 
both versions in \Cref{rm:agda-coind} are supported by the library.
%
In the first version, a possibly infinite proof tree for \lstinline{u} is a record with 
only one field \lstinline{unfold} containing an element of \lstinline{ISF[ is ] CoInd$\llbracket$ is $\rrbracket$ u}, 
that is, a proof that \lstinline{u} can be derived by applying a rule from premises 
belonging to \lstinline{CoInd$\llbracket$ is $\rrbracket$}, %that is, with possibly infinite proof trees.  
which  essentially consists of a rule with conclusion \lstinline{u} and possibly infinite proof trees for its premises.  
%
In the second version, a possibly infinite proof tree is obtained by a \lstinline{data} constructor, 
analogously to a finite one in the inductive interpretation; 
however, since proof trees are encoded as thunks, hence evaluated lazily,
this encoding represents infinite trees as well. 
In other words, coinduction is ``hidden'' in the library type \lstinline{Thunk}, 
which is a coinductive record with only one field \lstinline{force}, 
intuitively representing the suspended computation.

The interpretation of a generalized inference system (see \Cref{fig:fcoind-dt}) can then be encoded following exactly 
the definition in \Cref{sec:gis}: it is the coinductive interpretation of \lstinline{I}, restricted to
rules whose conclusion is in the inductive interpretation of the (standard) inference 
system consisting of both rules \lstinline{I} and corules \lstinline{C}.   
%
\begin{figure}[t]
\begin{lstlisting}[frame=single]
_$\sqcap$_ : $\forall$ {$\ell$c $\ell$p $\ell$n $\ell$}{U : Set $\ell$u} $\rightarrow$ IS {$\ell$c} {$\ell$p} {$\ell$n} U 
  $\rightarrow$ (U $\rightarrow$ Set $\ell$) $\rightarrow$ IS {$\ell$c $\sqcup$ $\ell$} {_} {_} U
(is $\sqcap$ P) .Names = is .Names
(is $\sqcap$ P) .rules rn = addSideCond (is .rules rn) P

_$\cup$_ : $\forall${$\ell$c $\ell$p $\ell$n $\ell$n'}{U : Set $\ell$} $\rightarrow$ IS {$\ell$c} {$\ell$p} {$\ell$n} U 
  $\rightarrow$ IS {_} {_} {$\ell$n'} U $\rightarrow$ IS {_} {_} {$\ell$n $\sqcup$ $\ell$n'} U
(is1 $\cup$ is2) .Names = (is1 .Names) $\uplus$ (is2 .Names)
(is1 $\cup$ is2) .rules = [ is1 .rules , is2 .rules ]

FCoInd$\llbracket$_,_$\rrbracket$ : $\forall${$\ell$c $\ell$p $\ell$n $\ell$n'} $\rightarrow$ (I : IS {$\ell$c} {$\ell$p} {$\ell$n} U) 
  $\rightarrow$ (C : IS {$\ell$c} {$\ell$p} {$\ell$n'} U) $\rightarrow$ U $\rightarrow$ Set _
FCoInd$\llbracket$ I , C $\rrbracket$ = CoInd$\llbracket$ I $\sqcap$ Ind$\llbracket$ I $\cup$ C $\rrbracket$ $\rrbracket$

SFCoInd$\llbracket$_,_$\rrbracket$ : $\forall${$\ell$c $\ell$p $\ell$n $\ell$n'} $\rightarrow$ (I : IS {$\ell$c} {$\ell$p} {$\ell$n} U) 
  $\rightarrow$ (C : IS {$\ell$c} {$\ell$p} {$\ell$n'} U) $\rightarrow$ U $\rightarrow$ Size $\rightarrow$ Set _
SFCoInd$\llbracket$ I , C $\rrbracket$ = SCoInd$\llbracket$ I $\sqcap$ Ind$\llbracket$ I $\cup$ C $\rrbracket$ $\rrbracket$
\end{lstlisting}
\caption{Interpretation generated by corules - datatype}
\label{fig:fcoind-dt}
\end{figure}
%    
The definition is provided in two flavours where the coinductive interpretation 
is encoded by coinductive records and thunks, respectively, and uses two operators on inference systems, 
restriction $\sqcap$ and union $\cup$. We report the codes in \Cref{fig:fcoind-dt}. 
The former adds  to each rule the side condition that the conclusion should satisfy \lstinline{P}, 
as specified by the function \lstinline{addSideCond} (here omitted). On the other hand, $\cup$ joins two inference
systems.

The library also provides the proofs of relevant properties, e.g., that closed sets coincide with pre-fixed points, 
and consistent sets coincide with post-fixed points. Moreover, it is shown that the two versions of encoding of the
coinductive interpretation (by coinductive records and thunks) are equivalent.    
Finally, the library provides  the induction, coinduction, and bounded coinduction principles (see \Cref{prop:indp,prop:coindp,prop:bcp}). 
We only report the statements in \Cref{fig:principles} and we briefly recall their meaning.
%
\begin{figure}[t]
\begin{lstlisting}[frame=single]
ind[_] : $\forall${$\ell$c $\ell$p $\ell$n $\ell$} 
    $\rightarrow$ (is : IS {$\ell$c} {$\ell$p} {$\ell$n} U)		-- IS
    $\rightarrow$ (S : U $\rightarrow$ Set $\ell$)			-- specification
    $\rightarrow$ ISClosed is S			-- S is closed
    $\rightarrow$ Ind$\llbracket$ is $\rrbracket$ $\subseteq$ S

coind[_] : $\forall${$\ell$c $\ell$p $\ell$n $\ell$}
    $\rightarrow$ (is : IS {$\ell$c} {$\ell$p} {$\ell$n} U) 
    $\rightarrow$ (S : U $\rightarrow$ Set $\ell$)
    $\rightarrow$ (S $\subseteq$ ISF[ is ] S)		-- S is consistent
    $\rightarrow$ S $\subseteq$ CoInd$\llbracket$ is $\rrbracket$
 
bounded-coind[_,_] : $\forall${$\ell$c $\ell$p $\ell$n $\ell$n' $\ell$} 
    $\rightarrow$ (I : IS {$\ell$c} {$\ell$p} {$\ell$n} U)
    $\rightarrow$ (C : IS {$\ell$c} {$\ell$p} {$\ell$n'} U)
    $\rightarrow$ (S : U $\rightarrow$ Set $\ell$)                   
    $\rightarrow$ S $\subseteq$ Ind$\llbracket$ I $\cup$ C $\rrbracket$		-- S is bounded w.r.t. I $\cup$ C
    $\rightarrow$ S $\subseteq$ ISF[ I ] S		-- S is consistent w.r.t. I
    $\rightarrow$ S $\subseteq$ FCoInd$\llbracket$ I , C $\rrbracket$
\end{lstlisting}
\caption{Proof principles}
\label{fig:principles}
\end{figure}

\begin{itemize}
\item If \lstinline{S} is closed, then each element of the inductively defined 
set \lstinline{Ind$\llbracket$ is $\rrbracket$} satisfies  \lstinline{S}.
\item If \lstinline{S} is consistent, then each element satisfying \lstinline{S} is in the
coinductively defined set \lstinline{CoInd$\llbracket$ is $\rrbracket$}. 
\item If \lstinline{S} is bounded, and  consistent with respect to \lstinline{I},
then each element which satisfies \lstinline{S} belongs to the set 
\lstinline{FCoInd$\llbracket$ I , C $\rrbracket$} defined by flexible coinduction. 
\end{itemize}
 
Another useful theorem is that 
$\FlexCo{\mis}{\mcois}\subseteq\Ind{\mis\cup\mcois}$ (see \Cref{fig:fcoind-to-ind}). 

\begin{figure}[t]
\begin{lstlisting}[frame=single]
fcoind-to-ind : $\forall${$\ell$c $\ell$p $\ell$n $\ell$n'}
    {is : IS {$\ell$c} {$\ell$p} {$\ell$n} U}{cois : IS {$\ell$c} {$\ell$p} {$\ell$n'} U} 
    $\rightarrow$ FCoInd$\llbracket$ is , cois $\rrbracket$ $\subseteq$ Ind$\llbracket$ is $\cup$ cois $\rrbracket$
\end{lstlisting}
	\caption{Extract inductive proof}
	\label{fig:fcoind-to-ind}
\end{figure}