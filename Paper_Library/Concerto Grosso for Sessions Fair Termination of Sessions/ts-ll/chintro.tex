\begintreble
%
%The \emph{linear $\pi$-calculus} \citep{KobayashiPierceTurner99} is a typed
%refinement of Milner's $\pi$-calculus in which linear channels can be used only
%once, for a one-shot communication. As it turns out, the linear $\pi$-calculus
%is the fundamental model underlying a broad family of communicating processes.
%In particular, all \emph{binary
%sessions} \citep{Honda93,HondaVasconcelosKubo98,HuttelEtAl16} and some
%\emph{multiparty sessions} \citep{HondaYoshidaCarbone16} can be encoded into the
%linear
%$\pi$-calculus \citep{Kobayashi02b,CairesPerez16,DardhaGiachinoSangiorgi17,CicconePadovani20}.
%The key insight for encoding sessions into the linear
%$\pi$-calculus is to encode linearized channels in a continuation-passing style:
%when some payload is transmitted over a linear channel, it can be paired with
%another linear channel (the continuation) on which the subsequent interaction
%takes place.
%
In this chapter we propose a type system for \piLIN, a linear $\pi$-calculus with
(co)recursive types, such that well-typed processes are \emph{fairly
terminating}.
Our type system is a conservative extension of
\muMALL \citep{BaeldeDoumaneSaurin16,Doumane17,BaeldeEtAl22}, the infinitary
proof system for the multiplicative additive fragment of linear logic with least
and greatest fixed points. In fact, the modifications we make to \muMALL are
remarkably small: we add one (standard) rule to deal with
\emph{non-deterministic choices}, those performed autonomously by a process, and
we relax the validity condition on \muMALL proofs so that it only considers the
``fair behaviors'' of the program it represents.
%
The fact that there is such a close correspondence between the typing rules of
\piLIN and the inference rules of \muMALL is not entirely surprising. After all,
there have been plenty of works investigating the relationship between
$\pi$-calculus terms and linear logic proofs, from those of \cite{Abramsky94}, 
\cite{BellinScott94} to those on
the interpretation of linear logic formulas as session
types \citep{DeYoungCairesPfenningToninho12,Wadler14,CairesPfenningToninho16,LindleyMorris16,RochaCaires21,QianKavvosBirkedal21}.

Nonetheless, we think that the connection between \piLIN and \muMALL stands out
for two reasons.
%
First, \piLIN is conceptually simpler and more general than the session-based
calculi that can be encoded in it. In particular, all the session calculi based
on linear logic rely on an asymmetric interpretation of the multiplicative
connectives $\tfork$ and $\tjoin$ so that $\FormulaF \tfork \FormulaG$
(respectively, $\FormulaF \tjoin \FormulaG$) is the type of a session endpoint
used for sending (respectively, receiving) a message of type $\FormulaF$ and
then used according to $\FormulaG$. In our setting, the connectives $\tfork$ and
$\tjoin$ retain their symmetry since we interpret $\FormulaF \tfork \FormulaG$
and $\FormulaF \tjoin \FormulaG$ formulas as the output/input of pairs, in the
same spirit of the original encoding of linear logic proofs proposed by 
\cite{BellinScott94}. This interpretation gives \piLIN the ability of
modeling \emph{bifurcating protocols} of which binary sessions are just a
special case.
%
The second reason why \piLIN and \muMALL get along has to do with the cut
elimination result for \muMALL. In finitary proof systems for linear logic, cut
elimination may proceed by removing \emph{topmost cuts}. In \muMALL there is no
such notion as a topmost cut since \muMALL proofs may be infinite. As a
consequence, the cut elimination result for \muMALL is proved by eliminating
\emph{bottom-most cuts} \citep{BaeldeEtAl22}. This strategy fits perfectly with
the reduction semantics of {\piLIN} -- and that of any other conventional
process calculus, for that matter -- whereby reduction rules act only on the
exposed (\ie unguarded) part of processes but not behind prefixes. As a result,
the reduction semantics of \piLIN is completely ordinary, unlike other
logically-inspired process calculi that incorporate commuting conversions
\citep{Wadler14,LindleyMorris16}, perform reductions behind
prefixes \citep{QianKavvosBirkedal21} or swap prefixes \citep{BellinScott94}.

In \Cref{ch:ft_bin,ch:ft_multi} we have proposed a type system
ensuring the fair termination of binary/multiparty sessions. 
In the present chapter we achieve the
same objective using a more basic process calculus and exploiting its strong
logical connection with \muMALL. In fact, the soundness proof of our type system
piggybacks on the cut elimination property of \muMALL.
Other session typed calculi based on linear logic with fixed points have been
studied by \cite{LindleyMorris16} and \cite{DerakhshanPfenning19,Derakhshan21}.
%
The type systems described in these works respectively guarantee termination and
strong progress, whereas our type system guarantees fair termination which is
somewhat in between these properties. Overall, our type system seems to hit a
sweet spot: on the one hand, it is deeply rooted in linear logic and yet it can
deal with common communication patterns (like the buyer/seller interaction
described above) that admit potentially infinite executions and therefore are
out of scope of other logic-inspired type systems; on the other hand, it
guarantees lock freedom \citep{Kobayashi02,Padovani14}, strong
progress \citep[Theorem 12.3]{DerakhshanPfenning19} and also termination, under a
suitable fairness assumption.

The chapter is organized as follows.
In \Cref{sec:ts_ll_types} we show the definition of the types bu relying on
the formulas of linear logic. Notably, these contain least/greatest fixed points.
In \Cref{sec:ts_ll_proc} we present syntax and semantics of \piLIN.
\Cref{sec:ts_ll_ts} introduces the type system for \piLIN as well as the 
validity conditions.
As usual, we dedicate \Cref{sec:ts_ll_corr} to detail the soundness proof 
of the type system and \Cref{sec:ts_ll_related} to discuss about related works.
Finally, we are now able to make a comparison between the present type system and
those presented in \Cref{ch:ft_bin,ch:ft_multi}.
We make such comparison by examples in \Cref{sec:comparison}.
