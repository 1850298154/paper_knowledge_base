\beginbass
%
Here we informally explain the main results that are needed in order to prove 
\Cref{thm:ts_ll_sound}. Roughly, the most important lemmas tell that
typing is preserved by reduction (\emph{subject reduction}) and that 
a well typed process can always reach termination (\emph{weak termination}).

\begin{theorem}[Subject Reduction]
	\label{thm:subj_red_ll}
  If $\piwtp\Ctx{P}$ and $P \red Q$ then $\piwtp\Ctx{Q}$.
\end{theorem}

All reductions in \Cref{fig:red_ll} except those for non-deterministic
choices correspond to cut-elimination steps in a quasi typing derivation. As
an illustration, below is a fragment of derivation tree for two processes
exchanging a pair of $y$ and $z$ on channel $x$.
\[
    \begin{prooftree}
        \[
            \[
                \mathstrut\smash\vdots
                \justifies
                \qtp{\CtxC, y : S}{P}
            \]
            \[
                \mathstrut\smash\vdots
                \justifies
                \qtp{\CtxD, z : T}{Q}
            \]
            \justifies
            \qtp{
                \CtxC, \CtxD, x : S \tfork T
            }{
                \Fork[z]\x\y{P}{Q}
            }
            \using\refrule[fork]\ForkRule
        \]
        \[
            \[
                \mathstrut\smash\vdots
                \justifies
                \qtp{\CtxC', y : \dual{S}, z : \dual{T}}{R}
            \]
            \justifies
            \qtp{\CtxC', x : \dual{S} \tjoin \dual{T}}{\Join[z]\x\y.R}
            \using\refrule[join]\JoinRule
        \]
        \justifies
        \qtp{
            \CtxC, \CtxD, \CtxC'
        }{
            \Cut\x{
                \Fork[z]\x\y{P}{Q}
            }{
                \Join[z]\x\y.R
            }
        }
        \using\refrule\CutRule
    \end{prooftree}
\]

As the process reduces, the quasi typing derivation is rearranged so that
the cut on $x$ is replaced by two cuts on $y$ and $z$. The resulting quasi
typing derivation is shown below.
\[
    \begin{prooftree}
        \[
            \mathstrut\smash\vdots
            \justifies
            \qtp{\CtxC, y : S}{P}
        \]
        \[
            \[
                \mathstrut\smash\vdots
                \justifies
                \qtp{\CtxD, z : T}{Q}
            \]
            \[
                \mathstrut\smash\vdots
                \justifies
                \qtp{\CtxC', y : \dual{S}, z : \dual{T}}{R}
            \]
            \justifies
            \qtp{
                \CtxD, \CtxC', y : \dual{S}
            }{
                \Cut\z{Q}{R}
            }
            \using\refrule\CutRule
        \]
        \justifies
        \qtp{
            \CtxC, \CtxD, \CtxC'
        }{
            \Cut\y{P}{\Cut\z{Q}{R}}
        }
        \using\refrule\CutRule            
    \end{prooftree}
\]

It is also interesting to observe that, when $P \red Q$, the reduct $Q$ is well
typed in the same context as $P$ but its rank may be different. In particular,
the rank of $Q$ can be \emph{greater} than the rank of $P$. Recalling that the
rank of a process estimates the number of choices that the process must perform
to terminate, the fact that the rank of $Q$ increases means that $Q$ \emph{moves
away} from termination instead of getting closer to it (we will see an instance
where this phenomenon occurs in \Cref{ex:parallel_programming}). What really
matters is that a well-typed process is weakly terminating. This is the second
key property ensured by our type system.

\begin{lemma}[Weak Termination]
	\label{lem:weak_termination_ll}
  If $\piwtp{x : \One}{P}$ then $P \wred \PiClose\x$.
\end{lemma}

The proof of \Cref{lem:weak_termination_ll} is a refinement of the cut elimination
property of \muMALL. Essentially, the only new case we have to handle is when a
choice $\PiChoice{P_1}{P_2}$ ``emerges'' towards the bottom of the typing
derivation, meaning that it is no longer guarded by any action. In this case, we
reduce the choice to the $P_i$ with smaller rank, which is guaranteed to lay on
a fair branch of the derivation.
%
An auxiliary result used in the proof of \Cref{lem:weak_termination_ll} 
is that our type system is a conservative extension of \muMALL.

\begin{lemma}[Conservativity]
	\label{lem:conservativity}
  If $\piwtp{x_1:S_1,\dots,x_n:S_n}{P}$ is derivable then 
  $\vdash S_1,\dots,S_n$ is derivable in \muMALL.
\end{lemma}

Then, the proof of \Cref{thm:ts_ll_sound} is a
simple consequence of \Cref{thm:subj_red_ll} and \Cref{lem:weak_termination_ll}.
%
The combination of \Cref{thm:fair_termination,thm:ts_ll_sound} also guarantees the
termination of every fair run of the process.

\begin{corollary}[Fair Termination]
	\label{cor:fair_termination_ll}
   If $\piwtp{x : \One}{P}$ then $P$ is fairly terminating.
\end{corollary}

Observe that zero-ranked process do not contain any non-deterministic choice. In
that case, every infinite branch in their typing derivation is fair and our
validity condition coincides with that of \muMALL. As a consequence, we obtain
the following strengthening of \Cref{cor:fair_termination_ll}:

\begin{proposition}
    If $\piwtp{x : \One}{P}$ and $\rankof{P} = 0$ then $P$ is terminating.
\end{proposition}

For regular processes (those consisting of finitely many distinct sub-trees, up
to renaming of bound names) it is possible to adapt the algorithm that
decides the validity of a \muMALL proof so that it decides the validity of a
\piLIN typing derivation.