\beginalto
%
On account of the known encodings of sessions into the linear
$\pi$-calculus \citep{Kobayashi02b,DardhaGiachinoSangiorgi17,ScalasDardhaHuYoshida17},
\piLIN belongs to the family of process calculi providing logical foundations to
sessions and session types. Some representatives of this family are
\piDILL \citep{CairesPfenningToninho16} and its variant equipped with a circular
proof theory \citep{DerakhshanPfenning19,Derakhshan21}, \CP \citep{Wadler14} and
\muCP \citep{LindleyMorris16}, among others.
%
There are two main aspects distinguishing \piLIN from these calculi. The first
one is that these calculi take sessions as a native feature. This fact can be
appreciated both at the level of processes, where session endpoints are
\emph{linearized} resources that can be used \emph{multiple times} albeit in a
sequential way, and also at the level of types, where the interpretation of the
$\FormulaF \tfork \FormulaG$ and $\FormulaF \tjoin \FormulaG$ formulas is skewed
so as to distinguish the type $\FormulaF$ of the message being sent/received on
a channel from the type $\FormulaG$ of the channel after the exchange has taken
place.
%
In contrast, \piLIN adopts a more fundamental communication model based on
\emph{linear} channels, and is thus closer to the spirit of the encoding of
linear logic proofs into the $\pi$-calculus proposed by \cite{BellinScott94} 
while retaining the same expressiveness of the
aforementioned calculi. To some extent, \piLIN is also more general than those
calculi, since a formula $\FormulaF \tfork \FormulaG$ may be interpreted as a
protocol that bifurcates into two independent sub-protocols $\FormulaF$ and
$\FormulaG$ (we have seen an instance of this pattern in
\Cref{ex:parallel_programming}). So, \piLIN is natively equipped with the
capability of modeling some multiparty interactions, in addition to all of the
binary ones.
%
A session-based communication model identical to \piLIN, but whose type system
is based on intuitionistic rather than classical linear logic, has been
presented by \cite{DeYoungCairesPfenningToninho12}. In that work,
the authors advocate the use of explicit continuations with the purpose of
modeling an asynchronous communication semantics and they prove the equivalence
between such model and a buffered session semantics. However, they do not draw a
connection between the proposed calculus and the linear
$\pi$-calculus \citep{KobayashiPierceTurner99} through the encoding of binary
sessions \citep{Kobayashi02b,DardhaGiachinoSangiorgi17} and, in the type system,
the multiplicative connectives are still interpreted asymmetrically.
%
The second aspect that distinguishes \piLIN from the other calculi is its type
system, which is still deeply rooted into linear logic and yet it ensures fair
termination instead of
progress \citep{CairesPfenningToninho16,Wadler14,QianKavvosBirkedal21},
termination \citep{LindleyMorris16} or strong
progress \citep{DerakhshanPfenning19,Derakhshan21}. Fair termination entails
progress, strong progress and lock freedom \citep{Kobayashi02,Padovani14}, but at
the same time it does not always rule out processes admitting infinite
executions. Simply, infinite executions are deemed unrealistic because they are
unfair.

Another difference between \piLIN and other calculi based on linear logic is
that its operational semantics is completely ordinary, in the sense that it does
not include commuting conversions, reductions under prefixes, or the swapping of
communication prefixes. The cut elimination result of \muMALL, on which the
proof of \Cref{thm:ts_ll_sound} is based, works by reducing cuts from the bottom
of the derivation instead of from the
top \citep{BaeldeDoumaneSaurin16,Doumane17,BaeldeEtAl22}. As a consequence, it is
not necessary to reduce cuts guarded by prefixes or to push cuts deep into the
derivation tree to enable key reductions in \piLIN processes.

The extension of calculi based on linear logic with non-deterministic features
has recently received quite a lot of attention. \cite{RochaCaires21} have proposed 
a session calculus with shared cells
and non-deterministic choices that can model mutable state. Their typing rule
for non-deterministic choices is the same as our own, but in their calculus
choices do not reduce. Rather, they keep track of every possible evolution of a
process to be able to prove a confluence result.
%
\cite{QianKavvosBirkedal21} introduce \emph{coexponentials}, a new
pair of modalities that enable the modeling of concurrent clients that compete
in order to interact with a shared server that processes requests sequentially.
In this setting, non-determinism arises from the unpredictable order in which
different clients are served. Interestingly, the coexponentials are derived by
resorting to their semantics in terms of least and greatest fixed points. For
this reason, the cut elimination result of \muMALL might be useful to attack the
termination proof in their setting.