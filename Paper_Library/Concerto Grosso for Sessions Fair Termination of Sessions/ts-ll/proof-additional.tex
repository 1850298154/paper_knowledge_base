\beginbass
%
Most of the soundness proof of the type system relies on the cut elimination
result of \muMALL. Here we gather some auxiliary definitions and properties of
well-typed processes.
%
First of all, we define a function that makes a ``fair choice'' among two
processes $P$ and $Q$, by selecting the one with smaller rank. If $P$ and $Q$
happen to have the same rank, we choose $P$ by convention.

\begin{definition}
    \label{def:fc}
    The \emph{fair choice} among $P$ and $Q$, written $\fc{P}{Q}$, is defined by
    \[
        \fc{P}{Q} \eqdef
        \begin{cases}
            P & \text{if $\rankof{P} \leq \rankof{Q}$} \\
            Q & \text{otherwise}
        \end{cases}
    \]
\end{definition}

From its definition we have $\rankof{\fc{P}{Q}} = \min\set{ \rankof{P},
\rankof{Q} }$.
%
Next, we show that in a well-typed process there cannot be an infinite sequence
of choices if we follow the fair ones. To this aim, we define a total function
on processes that computes the length of the longest chain of subsequent fair
choices.

\begin{definition}
    \label{def:depth}
    Let $\depth\cdot$ be the function from processes to $\RankSet$ such that
    \[
        \depth{P} =
        \begin{cases}
            1 + \depth{\fc{P_1}{P_2}} & \text{if $P = \PiChoice{P_1}{P_2}$} \\
            0 & \text{otherwise}
        \end{cases}
    \]
\end{definition}

Note that $\depth{\fc{P}{Q}} < 1 + \depth{\fc{P}{Q}} = \depth{\PiChoice{P}{Q}}$.
We prove that, for well-typed processes, $\depth\cdot$ always yields a natural
number. That is, in a well-typed process there is no infinite chain of fair
choices.

\begin{lemma}
    \label{lem:depth}
    If $\piwtp\Ctx{P}$ then $\depth{P} \in \Nat$.
\end{lemma}
\begin{proof}
    Suppose that $\depth{P} = \infty$. Then the derivation for $\qtp\Ctx{P}$
    has an infinite branch $\gamma = (\qtp\Ctx{P_i})_{i\in\omega}$ solely
    consisting of choices such that $\rankof{P_{i+1}} < \rankof{P_i}$ for every
    $i\in\omega$. Therefore, $\rankof{P_i} = \infty$ for every $i\in\omega$ and
    $\gamma$ is fair.
    %
    From the hypothesis that $P$ is well typed we deduce that $\gamma$ is also
    valid (see \Cref{def:valid_branch}), namely it has a non-stationary
    $\tnu$-thread. This contradicts the fact that the contexts in $\gamma$ are
    all equal to $\Ctx$.
    %
    We conclude that $\depth{P} \in \Nat$.
\end{proof}

Now we define a function $\Resolve\cdot$ on well-typed processes to statically
and fairly resolve all the choices of a process.

\begin{definition}
    \label{def:resolve}
    Let $\Resolve\cdot$ be the function on well-typed processes such that
    $\Resolve{\PiChoice{P}{Q}} = \Resolve{\fc{P}{Q}}$ and extended
    homomorphically to all the other process forms.
\end{definition}

The fact that $\Resolve{P}$ is uniquely defined when $P$ is well typed is a
consequence of \Cref{lem:depth}. Indeed, any branch of $P$ that contains
infinitely many choices also contains infinitely many forms other than choices.
%
Note that the range of $\Resolve\cdot$ only contains zero-ranked processes.

The next two results prove that $\Resolve{P}$ is well typed if so is $P$.

\begin{lemma}
    \label{lem:resolved_quasi_typed}
    If $\piwtp\Ctx{P}$ then $\qtp\Ctx{\Resolve{P}}$.
\end{lemma}
\begin{proof}
    \newcommand\rrel{\mathcal{R}}
    %
    We apply the coinduction principle to show that every judgment in the set
    \[
        \rrel \eqdef \set{\qtp\Ctx{\Resolve{P}} \mid
        \piwtp\Ctx{P}}
    \]
    is the conclusion of a rule in \Cref{fig:ts_ll} whose premises are
    also in $\rrel$.
    %
    Let $\qtp\Ctx{Q} \in \rrel$. Then $Q = \Resolve{P}$ for some $P$ such
    that $\piwtp\Ctx{P}$.
    %
    From \Cref{lem:depth} we deduce that $\depth{P} \in \Nat$.
    %
    We reason by induction on $\depth{P}$ and by cases on the shape of $P$ to
    show that $\qtp\Ctx{Q}$ is the conclusion of a rule in
    \Cref{fig:ts_ll} whose premises are also in $\rrel$. We only discuss
    a few cases, the others being similar or simpler.
    
      \proofcase{Case $P = \Cut\x{P_1}{P_2}$}
        %
        Then $Q = \Resolve{P} = \Cut\x{\Resolve{P_1}}{\Resolve{P_2}}$.
        %    
        From \refrule{\CutRule} we deduce that there exists $\Ctx_1$,
        $\Ctx_2$, $S_1$ and $S_2$ such that $\piwtp{\Ctx_i, x : S_i}{P_i}$
        for $i=1,2$ and $\Ctx = \Ctx_1, \Ctx_2$ and $S_1 =
        \dual{S_2}$.
        %
        Then $\qtp{\Ctx_i, x : S_i}{\Resolve{P_i}} \in \rrel$ by definition
        of $\rrel$ and we conclude observing that $\qtp\Ctx{Q} \in \rrel$ is
        the conclusion of \refrule{\CutRule}.
    
      \proofcase{Case $P = \Choice{P_1}{P_2}$}
        %
        Then $Q = \Resolve{P} = \Resolve{\fc{P_1}{P_2}}$. From
        \refrule{\PiChoiceRule} we deduce $\piwtp\Ctx{\fc{P_1}{P_2}}$. Since
        $\depth{\fc{P_1}{P_2}} < \depth{P} \in \Nat$, we conclude using the
        induction hypothesis.
\end{proof}

\begin{lemma}
    \label{lem:resolved_well_typed}
    If $\piwtp\Ctx{P}$ then $\piwtp\Ctx{\Resolve{P}}$.
\end{lemma}
\begin{proof}
    Using \Cref{lem:resolved_quasi_typed} we deduce $\qtp\Ctx{\Resolve{P}}$.
    %
    Now, consider an infinite branch $\gamma$ in the derivation for
    $\qtp\Ctx{\Resolve{P}}$ and observe that $\gamma$ is necessarily fair,
    since it does not traverse any choice.
    %
    The branch $\gamma$ corresponds to another infinite branch $\gamma'$ in the
    derivation for $\qtp\Ctx{P}$ which always makes fair choices whenever it
    traverses a process of the form $\PiChoice{P_1}{P_2}$. The only differences
    between $\gamma$ and $\gamma'$ are the judgments for the choices in $P$ that
    have been resolved. Nonetheless, all of the contexts that occur in $\gamma'$
    also occur in $\gamma$ because \refrule{\PiChoiceRule} does not affect typing
    contexts.
    %
    If $\gamma'$ traverses infinitely many choices, then $\gamma'$ traverses
    infinitely many processes with strictly decreasing ranks, which must all be
    $\infty$. Therefore, $\gamma'$ is fair.
    %
    Since $P$ is well typed we know that $\gamma'$ is valid. Then, the
    $\tnu$-thread that witnesses the validity of $\gamma'$ corresponds to a
    $\tnu$-thread that witnesses the validity of $\gamma$.
    %
    We conclude that $\Resolve{P}$ is well typed.
\end{proof}

\begin{proof}[Proof of \Cref{lem:conservativity}]
    By \Cref{lem:resolved_well_typed} we deduce
    $\piwtp{x_1:S_1,\dots,x_n:S_n}{\Resolve{P}}$. Since $\rankof{\Resolve{P}} =
    0$, there are no applications of \refrule{\PiChoiceRule} in the derivation for
    $\qtp{x_1:S_1,\dots,x_n:S_n}{\Resolve{P}}$. That is, this derivation
    corresponds to a \muMALL derivation and every infinite branch in it is fair.
    Since $\Resolve{P}$ is well typed, we deduce that every infinite branch in
    this derivation is valid. That is, $\vdash S_1, \dots, S_n$ is derivable in
    \muMALL.
\end{proof}

The key property of zero-ranked processes that are well typed in a context $x :
\One$ is that they are weakly terminating and they eventually reduce to
$\PiClose\x$.

\begin{lemma}
    \label{lem:zero_rank_weak_termination}
    If $\piwtp{x:\One}{P}$ and $\rankof{P} = 0$ then $P \wred \Close\x$.
\end{lemma}
\begin{proof}
    Without loss of generality we may assume that $P$ does not contain links.
    Indeed, the axiom \refrule{\LinkRule} is admissible in
    \muMALL \citep[Proposition 10]{BaeldeDoumaneSaurin16}, so we may assume that
    links have been expanded into equivalent (choice-free) processes. We just
    note that, if the statement holds for link-free processes, then it holds
    also in the case $P$ does contain links, the only difference being that the
    sequence of reductions that are needed to fully eliminate all the cuts
    resulting from an expanded link are replaced by a single occurrence of
    \refrule{rp-link}.

    From the hypothesis that $P$ has rank 0 we deduce that $P$ does not contain
    non-determimnistic choices and the derivation of $\qtp{x:\One}{P}$ does not
    contain applications of the \refrule{\PiChoiceRule} rule. So, every infinite
    branch of this derivation is fair. From the hypothesis that every infinite
    fair branch of this derivation is valid we deduce that every infinite branch
    of this derivation is valid. Then this derivation corresponds to a \muMALL
    proof of $\vdash \One$.
    %
    Observe that every principal reduction rule of \muMALL \cite[Figure
    3.2]{Doumane17} corresponds to a reduction rule of \piLIN
    (see \Cref{fig:red_ll}).
    %
    Also, \Cref{lem:proximity_ll} guarantees that, whenever there are two processes
    in which the same channel name $x$ occurs free we are always able to rewrite
    the process using structural precongruence so that the two sub-processes are
    the children of a cut on $x$.
    %
    Finally, the cut elimination result for \muMALL is proved by reducing
    bottom-most cuts, meaning that principal reductions (also called
    \emph{internal reductions} \citep{Doumane17}) are only applied at the bottom
    of a \muMALL proof.
    %
    Therefore, each step in which a principal reduction is applied in the cut
    elimination result of \muMALL can be mimicked by a reduction of \piLIN.
    %
    From the cut elimination result for \muMALL \cite[Proposition
    3.5]{Doumane17} we deduce that there exists no (fair) infinite sequence of
    principal reductions. That is, there exists a stuck $Q$ such that $P \wred
    Q$. From \Cref{lem:quasi_subj_red_ll} we deduce $\qtp{x:\One}{Q}$.
    Since the only cut-free \muMALL proof of $\vdash \One$ consists of a single
    application of \refrule[close]{\PiCloseRule}, and since this proof corresponds
    to the proof that $Q$ is quasi typed, we conclude that $Q = \Close\x$.
\end{proof}
