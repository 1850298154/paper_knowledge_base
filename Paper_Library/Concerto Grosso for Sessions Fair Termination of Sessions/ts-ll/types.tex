\beginalto
%
The types of \piLIN are built using the multiplicative additive fragment of
linear logic enriched with least and greatest fixed points. In this section we
specify the syntax of types along with all the auxiliary notions that are needed
to present the type system and prove its soundness.

\begin{definition}[Pre-formula]
	The syntax of pre-formulas relies on an infinite set of \emph{propositional
	variables} ranged over by $X$ and $Y$ and is defined by the grammar below:
\[
\begin{array}{rcl}
    \FormulaF, \FormulaG &
    ::= &
    \Zero \mid
    \Top \mid
    \One \mid
    \Bot \mid
    \FormulaF \plinchoice \FormulaG \mid
    \FormulaF \plinbranch \FormulaG \mid
    \FormulaF \tfork \FormulaG \mid
    \FormulaF \tjoin \FormulaG \mid
    \tmu\X.\Formula \mid
    \tnu\X.\Formula \mid
    X
\end{array}
\]
\end{definition}

As usual, $\tmu$ and $\tnu$ are the binders of propositional variables and the
notions of free and bound variables are defined accordingly. We assume that the
body of fixed points extends as much as possible to the right of a pre-formula,
so $\tmu\X.X \plinchoice \One$ means $\tmu\X.(X \plinchoice \One)$ and not $(\tmu\X.X)
\plinchoice \One$. We write $\subst\Formula\X$ for the capture-avoiding substitution
of all free occurrences of $X$ with $\Formula$. 

\begin{definition}[Dual of a formula]
	We write $\dual\Formula$ for the
	\emph{dual} of $\Formula$, which is the involution defined by the equations
\[
    \begin{array}{*{6}{@{}r@{~}c@{~}l@{\quad}}@{}}
        \dual\Zero & = & \Top & \qquad
        \dual{(\FormulaF \plinchoice \FormulaG)} & = & \dual\FormulaF \plinbranch \dual\FormulaG & \qquad
        \dual{(\tmu\X.\Formula)} & = & \tnu\X.\dual\Formula &
        \\
        \dual\One & = & \Bot &
        \dual{(\FormulaF \tfork \FormulaG)} & = & \dual\FormulaF \tjoin \dual\FormulaG &
        \dual\X & = & X
    \end{array}
\]
\end{definition}

\begin{definition}[Formula]
	A \emph{formula} is a closed pre-formula.
\end{definition}

In the context of \piLIN, formulas
describe how linear channels are used. Positive formulas (those built with the
constants $\Zero$ and $\One$, the connectives $\plinchoice$ and $\tfork$ and the
least fixed point) indicate output operations whereas negative formulas (the
remaining forms) indicate input operations. The formulas $\FormulaF \plinchoice
\FormulaG$ and $\FormulaF \plinbranch \FormulaG$ describe a linear channel used for
sending/receiving a tagged channel of type $\FormulaF$ or $\FormulaG$. The tag
(either $\LeftTag$ or $\RightTag$) distinguishes between the two possibilities.
The formulas $\FormulaF \tfork \FormulaG$ and $\FormulaF \tjoin \FormulaG$
describe a linear channel used for sending/receiving a pair of channels of type
$\FormulaF$ and $\FormulaG$; $\tmu\X.\Formula$ and $\tnu\X.\Formula$ describe a
linear channel used for sending/receiving a channel of type
$\Formula\subst{\tmu\X.\Formula}\X$ or $\Formula\subst{\tnu\X.\Formula}\X$
respectively. The constants $\One$ and $\Bot$ describe a linear channel used for
sending/receiving the unit. Finally, the constants $\Zero$ and $\Top$
respectively describe channels on which nothing can be sent and from which
nothing can be received.

\begin{example}
    \label{ex:bsc_ll_formulas}
    Looking at the structure of \actor{buyer} and \actor{seller} in
    \Cref{ex:bsc}, we can make an educated guess on the type of the
    channel they use. Indeed, we see that it is used according to
    $\FormulaF \eqdef \tmu\X.X \plinchoice \One$ by \actor{buyer} and according to
    $\FormulaG \eqdef \tnu\X.X \plinbranch \Bot$ in \actor{seller}. 
    Note that $\FormulaF = \dual\FormulaG$, 
    suggesting that \actor{seller} and \actor{seller} may interact correctly when connected.
    %
    \eoe
\end{example}

\begin{definition}[Subformula ordering]
	We write $\subf$ for the \emph{subformula ordering}, that is the least partial
	order such that $\FormulaF \subf \FormulaG$ if $\FormulaF$ is a subformula of
	$\FormulaG$.
\end{definition}

For example, consider $\FormulaF \eqdef \tmu\X.\tnu\Y.X \plinchoice Y$
and $\FormulaG \eqdef \tnu\Y.\FormulaF \plinchoice Y$. Then we have $\FormulaF \subf
\FormulaG$ and $\FormulaG \not\subf \FormulaF$. When $\Formulas$ is a set of
formulas, we write $\minf\Formulas$ for its $\subf$-minimum formula if it
is defined.
%
Occasionally we let $\star$ stand for an arbitrary binary connective $\plinchoice$,
$\tfork$, $\plinbranch$, or $\tjoin$ and $\sigma$ stand for an arbitrary fixed point
operator $\tmu$ or $\tnu$.

When two \piLIN processes interact on some channel $x$, they may exchange other
channels on which their interaction continues. We can think of these subsequent
interactions stemming from a shared channel $x$ as being part of the same
conversation (the literature on \emph{sessions} \citep{Honda93,HuttelEtAl16}
builds on this idea \citep{Kobayashi02b,DardhaGiachinoSangiorgi17}). The
soundness proof of the type system is heavily based on the proof of the cut
elimination property of \muMALL, which relies on the ability to uniquely
identify the types of the channels that belong to the same conversation and to
trace conversations within typing derivations. Following the literature on
\muMALL \citep{BaeldeDoumaneSaurin16,Doumane17,BaeldeEtAl22}, we annotate
formulas with addresses.
%
We assume an infinite set $\AddressSet$ of \emph{atomic addresses},
$\dual\AddressSet$ being the set of their duals such that $\AddressSet \cap
\dual\AddressSet = \emptyset$ and $\dual{\dual\AddressSet{}} = \AddressSet$. We
use $a$ and $b$ to range over elements of $\AddressSet \cup \dual\AddressSet$.

\begin{definition}[Address]
	An \emph{address} is a string $aw$ where $w \in \set{i,l,r}^*$. The dual of an
	address is defined as $\dual{(aw)} = \dual{a}w$.
\end{definition}

We use $\addressA$ and $\addressB$ to range over addresses, we write $\prefix$
for the prefix relation on addresses and we say that $\addressA$ and $\addressB$
are \emph{disjoint} if $\addressA \not\prefix \addressB$ and $\addressB
\not\prefix \addressA$.

\begin{definition}[Type]
	A \emph{type} is a formula $\Formula$ paired with an address $\address$ written
	$\Formula_\address$.
\end{definition}

We use $S$ and $T$ to range over types and we extend to
types several operations defined on formulas: we use logical connectives to
compose types so that $\FormulaF_{\address l} \mathbin\star \FormulaG_{\address
r} \eqdef (\FormulaF \mathbin\star \FormulaG)_\address$ and
$\sigma\X.\Formula_{\address i} \eqdef (\sigma\X.\Formula)_\address$; the dual
of a type is obtained by dualizing both its formula and its address, that is
$\dual{(\Formula_\address)} \eqdef \dual\Formula_{\dual\address}$; type
substitution preserves the address in the type within which the substitution
occurs, but forgets the address of the type being substituted, that is
$\FormulaF_\addressA\subst{\FormulaG_\addressB}\X \eqdef
\FormulaF\subst\FormulaG\X_\addressA$.

We often omit the address of constants (which represent terminated
conversations) and we write $\strip{S}$ for the formula obtained by forgetting
the address of $S$. Finally, we write $\tred$ for the least reflexive relation
on types such that $S_1 \star S_2 \tred S_i$ and $\sigma\X.S \tred
S\subst{\sigma\X.S}\X$.

\begin{example}
    \label{ex:bsc_ll_types}
    Consider once again the formula $\FormulaF \eqdef \tmu\X.X \plinchoice \One$ that
    describes the behavior of \actor{buyer} (\Cref{ex:bsc_ll_formulas}) and let
    $a$ be an arbitrary atomic address. We have
    \[
        \Formula_a
        \tred (\Formula \plinchoice \One)_{ai}
        \tred \Formula_{ail}
        \tred (\Formula \plinchoice \One)_{aili}
        \tred \One_{ailir}
    \]
    where the fact that the types in this sequence all share a common non-empty
    prefix `$a$' indicates that they belong to the same conversation.
    %
    Note how the symbols $i$, $l$ and $r$ composing an address indicate the step
    taken in the syntax tree of types for making a move in this sequence: $i$
    means ``inside'', when a fixed point operator is unfolded, whereas $l$ and
    $r$ mean ``left'' and ``right'', when the corresponding branch of a
    connective is selected.
    %
    \eoe
\end{example}