\beginbass
%
We detail the proof of the \emph{subject reduction} theorem (see \Cref{thm:subj_red_ll}).
Notably, the presence of \refrule{rp-struct} in the reduction rules
implies a \emph{subject congruence} lemma, stating that typing is preserved by structural
pre-congruence. The same happened in \Cref{sec:ts_bin_corr,sec:ts_multi_corr}.

\begin{lemma}[Substitution]
\label{lem:substitution}
	If $\qtp{\Ctx, x : S}{P}$ and $y \not\in \dom\Ctx$, then
	$\qtp{\Ctx, y : S}{P\subst\y\x}$.
\end{lemma}
\begin{proof}
	A simple application of the coinduction principle.
\end{proof}

%%%%%%%%%%%%%%%%%%%%%%%%%
%%% Subject Conguence %%%
%%%%%%%%%%%%%%%%%%%%%%%%%

\begin{lemma}[Quasi Subject Congruence]
\label{lem:quasi_subj_cong_ll}
	If\/ $\qtp\Ctx{P}$ and $P \pcong Q$, then $\qtp\Ctx{Q}$.
\end{lemma}
\begin{proof}
	By induction on the derivation of $P \pcong Q$ and by cases on the last rule
	applied. We only discuss the base cases.

	\proofrule{sp-link}
		%
		Then $P = \Link\x\y \pcong \Link\y\x = Q$.
		%
		From \refrule{\LinkRule} we deduce that there exist $\Formula$,
		$\addressA$ and $\addressB$ such that $\Ctx = x :
		\Formula_\addressA, y : \dual{\Formula_\addressB}$. We conclude with an
		application of \refrule{\LinkRule}.

	\proofrule{sp-comm}
		%
		Then $P = \Cut\x{P_1}{P_2} \pcong \Cut\x{P_2}{P_1} = Q$. From
		\refrule{\CutRule} we deduce that there exist $\Ctx_1$,
		$\Ctx_2$, and $S$ such that
		\begin{itemize}
		\item $\Ctx = \Ctx_1, \Ctx_2$
		\item $\qtp{\Ctx_1, x : S}{P_1}$
		\item $\qtp{\Ctx_2, x : \dual{S}}{P_2}$
		\end{itemize}
		We conclude with an application of \refrule{\CutRule}.

	\proofrule{sp-assoc}
		%
		Then $P = \Cut\x{P_1}{\Cut\y{P_2}{P_3}} \pcong
		\Cut\y{\Cut\x{P_1}{P_2}}{P_3} = Q$ and $\x \in \fn{P_2}$.
		%
		From \refrule{\CutRule} we deduce that there exist $\Ctx_1$,
		$\Ctx_2$, $\Ctx_3$, $S$, $T$ such that
		\begin{itemize}
		\item $\Ctx = \Ctx_1, \Ctx_2, \Ctx_3$
		\item $\qtp{\Ctx_1, x : S}{P_1}$
		\item $\qtp{\Ctx_2, x : \dual{S}, y : T}{P_2}$
		\item $\qtp{\Ctx_3, y : \dual{T}}{P_3}$
		\end{itemize}
		We conclude with two applications of \refrule{\CutRule}.
\end{proof}

%%%%%%%%%%%%%%%%%%%%%%%%%
%%% Subject Reduction %%%
%%%%%%%%%%%%%%%%%%%%%%%%%

\begin{lemma}[Quasi Subject Reduction]
	\label{lem:quasi_subj_red_ll}
	If $P \red Q$ then $\qtp\Ctx{P}$ implies $\qtp\Ctx{Q}$.
\end{lemma}
\begin{proof}
	By induction on $P \red Q$ and by cases on the last rule applied.

	\proofrule{rp-link}
		%
		Then $P = \Cut\x{\Link\x\y}{R} \red R\subst\y\x = Q$.
		%
		From \refrule{\CutRule} and \refrule{\LinkRule} we deduce that there
		exist $\Formula_\addressA$, $\Formula_\addressB$ and $\CtxD$ such
		that $\Ctx = y : \dual\Formula_{\dual\addressB}, \CtxD$ and
		$\qtp{\CtxD, x : \dual\Formula_{\dual\addressB}}{R}$
		%
		We conclude by applying \cref{lem:substitution}.

	\proofrule{rp-unit}
		%
		Then $P = \Cut\x{\PiClose\x}{\PiWait{x}.Q} \red Q$.
		%
		From \refrule\CutRule, \refrule[close]{\PiCloseRule} and
		\refrule[wait]{\PiWaitRule} we conclude $\qtp\Ctx{Q}$.

	\proofrule{rp-pair}
		%
		Then $P = \Cut\x{\Fork[z]\x\y{P_1}{P_2}}{\Join[z]\x\y.P_3} \red
		\Cut\y{P_1}{\Cut\z{P_2}{P_3}} = Q$.
		%
		From \refrule{\CutRule}, \refrule[fork]{\ForkRule} and
		\refrule[join]{\JoinRule} we deduce that there exist $\Ctx_1$,
		$\Ctx_2$, $\Ctx_3$, $S$ and $T$ such that
		\begin{itemize}
		\item $\Ctx = \Ctx_1,\Ctx_2,\Ctx_3$
		\item $\qtp{\Ctx_1, y : S}{P_1}$
		\item $\qtp{\Ctx_2, z : T}{P_2}$
		\item $\qtp{\Ctx_3, y : \dual{S}, z : \dual{T}}{P_3}$
		\end{itemize}
		We conclude with two applications of \refrule{\CutRule}.
		
	\proofrule{rp-sum}
		%
		Then $P = \Cut\x{\Select[y]{\InTag_i}\x.R}{\Case[y]\x{P_1}{P_2}} \red
		\Cut\y{R}{P_i} = Q$ for some $i\in\set{1,2}$.
		%
		From \refrule{\CutRule}, \refrule[select]{\SelectRule} and
		\refrule[case]{\CaseRule} we deduce that there exist $\Ctx_1$,
		$\Ctx_2$, $S_1$ and $S_2$ such that
		\begin{itemize}
		\item $\Ctx = \Ctx_1, \CtxD$
		\item $\qtp{\Ctx_1, y : S_i}{R}$
		\item $\qtp{\Ctx_2, y : \dual{S_1}}{P_1}$
		\item $\qtp{\Ctx_2, y : \dual{S_2}}{P_2}$
		\end{itemize}
		We conclude with an application of \refrule{\CutRule}.

	\proofrule{rp-rec}
		%
		Then $P = \Cut\x{\Rec[y]\x.P_1}{\Corec[y]\x.P_2} \red \Cut\y{P_1}{P_2} = Q$.
		%
		From \refrule{\CutRule}, \refrule[rec]{\RecRule} and
		\refrule[corec]{\CorecRule} we deduce that there exist $\Ctx_1$,
		$\Ctx_2$ and $S$ such that
		\begin{itemize}
		\item $\Ctx = \Ctx_1, \Ctx_2$
		\item $\qtp{\Ctx_1, y : S\subst{\tmu\X.S}\X}{P_1}$
		\item $\qtp{\Ctx_2, y : \dual{S}\subst{\tnu\X.\dual{S}}{\X}}{P_2}$
		\end{itemize}
		We conclude with an application of \refrule{\CutRule}.

	\proofrule{rp-choice}
		%
		Then $P = \PiChoice{P_1}{P_2} \red P_i = Q$ for some $i \in \set{1,2}$.
		%
		From \refrule{\PiChoiceRule} we conclude that $\qtp\Ctx{P_i}$.

		\item[Case \refrule{rp-cut}.]
		%
		Then $P = \Cut\x{P'}{R} \red \Cut\x{Q'}{R} = Q$ and $P' \red Q'$.
		%
		From \refrule{\CutRule} we deduce that there exist $\Ctx_1$,
		$\Ctx_2$ and $S$ such that
		\begin{itemize}
		\item $\Ctx = \Ctx_1, \Ctx_2$
		\item $\qtp{\Ctx_1, x : S}{P'}$
		\item $\qtp{\Ctx_2, x : \dual{S}}{R}$
		\end{itemize}
		Using the induction hypothesis on $P' \red Q'$ we deduce
		$\qtp{\Ctx_1, x : S}{Q'}$.
		%
		We conclude with an application of \refrule{\CutRule}.

	\proofrule{rp-struct}
		%
		Then $P \pcong P'$, $P' \red Q'$ and $Q' \pcong Q$ for some $P'$, $Q'$.
		%
		From \Cref{lem:quasi_subj_cong_ll} we deduce $\qtp\Ctx{P'}$.
		Using the induction hypothesis on $P' \red Q'$ we deduce
		$\qtp\Ctx{Q'}$. We conclude by applying \Cref{lem:quasi_subj_cong_ll}
		once more.
\end{proof}

\begin{proof}[Proof of \Cref{thm:subj_red_ll}]
	From the hypothesis $\piwtp\Ctx{P}$ we deduce $\qtp\Ctx{P}$. Using
	\Cref{lem:quasi_subj_red_ll} we deduce $\qtp\Ctx{Q}$.
	%
	Now, let $\gamma$ be an infinite fair branch in the typing derivation for
	$\qtp\Ctx{Q}$. By inspecting the proof of
	\Cref{lem:quasi_subj_red_ll} we observe that $\gamma$ can be
	decomposed as $\gamma = \gamma_1\gamma_2$ where $\gamma_2$ is a branch in
	the typing derivation for $\qtp\Ctx{Q}$. From the fact that $\gamma$ is
	fair we deduce that so is $\gamma_2$.
	%
	From the hypothesis $\piwtp\Ctx{P}$ we deduce that $\gamma_2$ is valid,
	namely there is a $\tnu$-thread $t$ in it. Then $t$ is a $\tnu$-thread also of
	$\gamma$, that is $\gamma$ is also valid.
	%
	We conclude $\piwtp\Ctx{Q}$.
\end{proof}