\beginalto
%
In this section we present the typing rules for \piLIN.
%
As usual we introduce typing contexts to track the type of the names occurring
free in a process. A \emph{typing context} is a finite map from names to types
written $x_1 : S_1, \dots, x_n : S_n$. We use $\CtxC$ and $\CtxD$ to
range over contexts, we write $\dom\Ctx$ for the domain of $\Ctx$ and
$\Ctx,\CtxD$ for the union of $\Ctx$ and $\CtxD$ when
$\dom\Ctx \cap \dom\CtxD = \emptyset$.
%
Typing judgments have the form $\qtp\Ctx{P}$ and we say that $P$ is \emph{quasi
typed} in $\Ctx$.
For the time being we say ``quasi typed'' and not ``well typed'' because some
infinite derivations using the rules in \Cref{ssec:ts_ll} are invalid.
Well-typed processes are quasi-typed processes whose typing derivation satisfies
some additional validity conditions that we detail in \Cref{ssec:ts_ll_wtp}.

%%%%%%%%%%%%%
%%% RULES %%%
%%%%%%%%%%%%%

\subsection{Typing Rules}
\label{ssec:ts_ll}
\beginbass
%
The typing rules resemble those of a traditional session type system but differ
in a few key aspects. First of all, they establish a tighter-than-usual
correspondence between types and processes so that any discrepancy between
actual and expected types is accounted for by explicit casts. This way, we make
sure that actions leading to the termination of a session \emph{at the type
level} are matched by corresponding actions \emph{at the process level}, a key
property used in the soundness proof of the type system.
In addition, the typing rules enforce the boundedness properties informally
described in the previous section.
%
Action boundedness is enforced by specifying the typing rules as a generalized
inference system and using two corules to make sure that every well-typed
process is at finite distance from $\Done$ or a $\Close\x$.
%
Concerning session and cast boundedness, we annotate typing judgments with a
\emph{rank}, that is an upper bound to the \emph{weights} of casts that must be
performed and of sessions that must be created in order to terminate the process
in the judgment.

\begin{figure}[t]
  \framebox[\textwidth]{
      \begin{mathpar}
        \displaystyle
          \inferrule[tb-done]{\mathstrut}{
            \wtp[n]\EmptyCtx\Done
          } \defrule[tb-done]{}
          \and
          \inferrule[tb-wait]
          {
            \wtp[n]\Ctx{P}
          }{
            \wtp[n]{\Ctx, x :  \End[\In]}{\Wait\x{P}}
          } \defrule[tb-wait]{}
          \and
          \inferrule[tb-close]{\mathstrut}
          {
            \wtp[n]{x : \End[\Out]}{\Close\x}
          } \defrule[tb-close]{}
          \and
          \inferrule[tb-channel-in]{
            \wtp[n]{\Ctx, x : S, y : T}{P}
          }{
            \wtp[n]{\Ctx, x :  \In\T.S}{\PInput\x{(y)}.P}
          } \defrule[tb-channel-in]{}
          \and
          \inferrule[tb-channel-out]{
            \wtp[n]{\Ctx, x : S}{P}
          }{
            \wtp[n]{\Ctx, x : \Out\T.S, y : T}{\POutput\x\y.P}
          } \defrule[tb-channel-out]{}
          \and
          \inferrule[tb-tag]
          {
          	\forall i\in I:
            \wtp[n]{\Ctx, x : S_i}{P_i}
          }{
            \textstyle
            \wtp[n]{
              \Ctx, x : \Pol\set{\l_i : S_i}_{i \in I}
            }{
              x\Pol\set{\l_i : P_i}_{i \in I}
            }
          } \defrule[tb-tag]{}
          \and
          \inferrule[tb-choice]{
            \wtp[n_1]\Ctx{P}
            \\
            \wtp[n_2]\Ctx{Q}
          }{
            \wtp[n_k]\Ctx{P \pchoice_k Q}
          }
          ~
          k \in \set{1,2}
          \defrule[tb-choice]{}
          \and
          \inferrule[tb-cast]{
            \wtp[n]{\Ctx, x : T}{P}
          }{
            \wtp[n+m]{\Ctx, x : S}{\Cast\x P}
          }
          ~ S \subt[m] T \defrule[tb-cast]{}
          \and
          \inferrule[tb-par]{
            \wtp[m]{\Ctx, x : S}{P}
            \\
            \wtp[n]{\CtxD, x : T}{Q}
          }{
            \wtp[1+m+n]{
              \Ctx, \CtxD
            }{
              \NewPar\x{P}{Q}
            }
          }
          ~
          S \compatible T \defrule[tb-par]{}
          \and
          \inferrule[tb-call]{
            \wtp[n]{\seqof{x:S}}{P}
          }{
            \wtp[m+n]{\seqof{x:S}}{\Call{A}{\seqof\x}}
          }
          ~
          \tass{A}{\seqof{S}}{n},
          \Definition{A}{\seqof\x}{P} \defrule[tb-call]{}
          \and
          \infercorule[cob-tag]{
            \wtp[n]{\Ctx, x : S_k}{P_k}
          }{
            \wtp[n]{
              \Ctx, x : \Pol\set{\l_i : S_i}_{i \in I}
            }{
              x\Pol\set{\l_i : P_i}_{i \in I}
            }
          }
          ~
          k \in  I \defrule[cob-tag]{}
          \and
          \infercorule[cob-choice]{
            \wtp[n]\Ctx{P_k}
          }{
            \wtp[n]\Ctx{P_1 \pchoice_k P_2}
          } \defrule[cob-choice]{}
      \end{mathpar}
    }
    \caption{Typing rules}
    \label{fig:ts_bin}
\end{figure}

The typing rules are defined by the generalized inference system in
\Cref{fig:ts_bin} and derive judgements of the form $\wtp[n]\Ctx{P}$, meaning
that $P$ is well typed in the \emph{typing context} $\Ctx$ and has rank $n$.
A typing context is a finite map from channels to session types written $x_1 :
S_1, \dots, x_n : S_n$ or $\seqof{x : S}$. We use $\Ctx$ and $\CtxD$ to
range over typing contexts, we write $\EmptyCtx$ for the empty context and
$\Ctx,\CtxD$ for the union of $\Ctx$ and $\CtxD$ when they
have disjoint domains.
%
We type check a program $\set{\Definition{A_i}{\seqof{x_i}}{P_i}}_{i\in I}$
under a global set of type assignments $\set{\tass{A_i}{\seqof{S_i}}{n_i}}_{i\in
I}$ associating each process name $A_i$ with a tuple of session types
$\seqof{S_i}$ and a rank $n_i$. The program is well typed if
$\wtp[n_i]{\seqof{x_i : S_i}}{P_i}$ for every $i\in I$, establishing that the
tuple $\seqof{S_i}$ corresponds to the way the channels $\seqof{x_i}$ are used
by $P_i$ and that $n_i$ is a feasible rank annotation for $P_i$. Hereafter, we
omit the rank from judgments when it is not important. 

Let us look at the typing (co)rules in detail.
%
\refrule{tb-done} is the usual axiom requiring that the terminated process leaves no unused channels behind. 
Since $\Done$ performs no casts and creates no sessions, it can have any rank.
%
Rules \refrule{tb-wait} and \refrule{tb-close} concern the exchange of session
termination signals. There is nothing remarkable here except noting once again
that the rank of $\Close\x$ can be arbitrary.
%
Rules \refrule{tb-channel-in} and \refrule{tb-channel-out} are similar, but they
concern the exchange of channels. Note that, in \refrule{tb-channel-out}, the
type $T$ of the message $y$ is required to match \emph{exactly} that in the type
of the channel $x$ used for the communication, whereas \citep{GayHole05} allow
the type of $y$ to be a subtype of $T$. This is one instance of the ``tight
correspondence'' that we mentioned earlier (see \Cref{ex:invariant_ch}).
%
The rule \refrule{tb-label} deals with the input/output of labels. As usual, any
channel other than the one affected by the communication must be used in exactly
the same way in every branch. However, the rule is stricter than that of
\citet{GayHole05} because it requires an exact correspondence between the labels
that can be exchanged on $x$ by the process and those in the type of $x$. The
fact that a conclusion and premises are all annotated with the same rank $n$
means that $n$ is an upper bound for the rank of all branches of a label
input/output.
%
The corule \refrule{cob-label} does not impose additional constraints compared to
\refrule{tb-label} and has \emph{exactly one premise}, corresponding to one
branch of the process in the conclusion. The effect of \refrule{cob-label}, when
interpreted inductively together with the other rules, is to ensure the
existence of a finite typing derivation whose leaves are applications of
\refrule{tb-done} or \refrule{tb-close}, hence action boundedness.

Rule \refrule{tb-choice} is a standard typing rule for non-deterministic choices,
requiring that both branches are well typed in exactly the same typing context.
Notice that the rank of a choice $P_1 \pchoice_k P_2$ is determined by the branch
indexed by the $k$ annotation, which is elected as the branch that leads to
termination. Like \refrule{cob-label}, the associated corule
\refrule{cob-choice} ensures that the same branch gets closer to $\Done$ or a
$\Close\x$ to enforce action boundedness. Without this corule, it would not be
possible to find a \emph{finite-depth} derivation tree for an action-bounded
process such as $A$ in \Cref{ex:action_boundedness}. Coherently with
\refrule{tb-choice}, the same branch that leads to termination is also the one
that determines the rank of the choice as a whole.

Rule \refrule{tb-cast} is Liskov's substitution principle formulated as an
inference rule. It states that a channel $x$ of type $S$ can be safely used
where a channel of type $T$ is expected, provided that $S \subt T$. The most
important detail to notice here is that the rank of a cast is the \emph{weight}
of the subtyping judgment plus that of
the process in which the cast has effect. This way we account for this cast in
the rank of the process so as to guarantee cast boundedness.
%
Rule \refrule{tb-par} concerns parallel composition and session creation. The
rule is shaped after the cut rule of linear logic also adopted in other session
type systems based on linear logic
\citep{CairesPfenningToninho16,Wadler14,LindleyMorris16}. In particular, the
parallel processes $P$ and $Q$ share no channel other than the session $x$ that
connects them, so as to prevent mutual dependencies between sessions and
guarantee deadlock freedom. The side condition $S \compatible T$ requires that
the way in which $P$ and $Q$ use channel $x$ is such that the session $x$ can
fairly terminate (see \Cref{def:compatibility}). We \emph{do not} require that $S$
and $T$ are dual to each other because reductions (see \refrule{rb-pick}) and
structural pre-congruence (see \refrule{sb-cast-new}) do not necessarily preserve
session type duality. Also, duality does not always imply compatibility.
%
The rank of a parallel composition is one plus that of the composed processes.
By accounting for each occurrence of parallel compositions in the rank, we
guarantee that well-typed processes are session bounded.

Finally, rule \refrule{tb-call} states that a process invocation
$\Call{A}{\seqof\x}$ is well typed provided that the types associated with
$\seqof\x$ match those of the global assignment $\tass{A}{\seqof{S}}{n}$. Note
that \refrule{tb-call} is \emph{not} an axiom: its premise (re)checks that the
body $P$ in the definition of $A$ is coherent with the global type assignment
$\tass{A}{\seqof{S}}{n}$. With this formulation of \refrule{tb-call}, the only
axioms are \refrule{tb-done} and \refrule{tb-close} so that the inductive
interpretation of the typing (co)rules ensures action boundedness. Note also
that the rank of the conclusion may be greater than the rank $n$ associated with
$A$. This overapproximation grants more flexibility when typing different
branches in \refrule{tb-label}.

\begin{remark}[On structural pre-congruence...continuation]
	Now we have all the ingredients to understand why the choice of a pre-congruence
	over a congruence relation is just a design one (see \Cref{rmk:pcong}).
	Indeed, the such choice was compulsory in \cite{CicconePadovani22}.
	As mentioned before, in such work we relied on the characterization of fair subtyping based on
	a generalized inference system (see \Cref{ssec:fsub_gis}) and the subsumption rule
	\refrule{tb-cast} always increased the \emph{rank} by one. This way, \refrule{sb-cast-new} interpreted
	in a congruence way would increase the rank of the process due to the introduced cast.
	Using the actual notions, a reflexive application of fair subtyping has weight zero.
	%
	\eor
\end{remark}

Well-typed processes enjoy the expected properties, including typing
preservation under structural pre-congruence and reduction. Most importantly,
they fairly terminate:

\begin{theorem}{Soundness}
  \label{thm:ts_bin_sound}
  If $\wtp[n]\EmptyCtx P$ and $P \wred Q$, then $Q \wred\pcong \pdone$.
\end{theorem}

The proof of \Cref{thm:ts_bin_sound} follows \Cref{thm:fair_termination}. 
Moreover the proof that all the reducts of a process are \emph{weakly terminating}
(see \Cref{lem:weak_termination_bin})
is loosely based on the method of helpful directions
\citep{Francez86}, namely on the property that a (well-typed) process \emph{may}
reduce in such a way that its measure strictly decreases
(see \Cref{lem:helpful_direction_bin}). Recall that this
property is not true for every reduction.

There are several valuable implications of \Cref{thm:ts_multi_sound} on a well-typed,
closed process $P$:
\begin{description}
  \item[Deadlock freedom.] If $Q$ cannot reduce any further, then it must be
  $\pdone$ (structurally precongruent to), namely there are no residual
  input or output actions.
  \item[Fair termination.] Under the fairness assumption,
  \Cref{thm:fair_termination} assures that $P$ eventually reduces to $\pdone$.
  This also implies that every session created by $P$ eventually terminates.
  \item[Junk freedom.] Each message produced as $P$ executes is eventually
  consumed. Indeed, if $Q$ contains a pending message, the fact that $Q$ may
  reduce to $\pdone$ means that some process is able to consume the message and
  will eventually do so under the fairness assumption.
  \item[Progress.] If $Q$ contains a sub-process with pending input/output
  actions, the fact that $Q$ may reduce to $\pdone$ means that these actions are
  eventually performed.
\end{description}

\begin{remark}
  \label{rem:internal_choice_rank}
  The rank of a non-deterministic choice $P \pchoice Q$ can usually be chosen to
  be the minimum among those of the branches $P$ and $Q$, so that the type
  system can handle processes like those in \cref{ex:infinite_sessions}, which
  \emph{may} create new sessions or perform casts but they need not do so in
  order to terminate.
  %
  On the contrary, the rank of a label output $\PSend\x{\l_i:P_i}_{i\in I}$ has
  to be an upper bound of that of all branches $P_i$.
  %
  The motivation for such different ways of determining the rank of these
  process forms, despite both represent an \emph{internal choice}, lies in the
  proof of \Cref{lem:helpful_direction_bin}.
  %
  In $P \pchoice Q$, both branches are typed in \emph{exactly the same} typing
  context, meaning that the choice of one branch or the other has no substantial
  impact on the shortest paths that terminate the sessions used by $P$ and $Q$.
  Thus, the ``helpful'' reduction can be solely driven by the rank of the chosen
  branch.
  %
  In a label output $\PSend\x{\l_i:P_i}_{i\in I}$ it could happen that all
  branches with minimum rank increase the length of the shortest path that leads
  to the termination of $x$. In this case, the choice of the ``helpful''
  reduction must prioritize the termination of $x$, but then the rank of the
  whole process has to be an upper bound of that of the branches to be sure that
  the measure of the reduct decreases.
  %
  \eor
\end{remark}

%%%%%%%%%%%%%%%%%%%%%%%%%%%
%%% QUASI TO WELL TYPED %%%
%%%%%%%%%%%%%%%%%%%%%%%%%%%

\subsection{From Quasi Typed to Well Typed Processes}
\label{ssec:ts_ll_wtp}
\beginbass
%
As we have anticipated, there exist infinite typing derivations that are unsound
from a logical standpoint, because they allow us to prove $\Zero$ or the empty
sequent. Hence, the typing rules presented in \Cref{fig:ts_ll} must be combined
with additional \emph{validity conditions}.

\begin{example}
	\label{ex:omega}
	Consider the non-terminating process $\Omega(x) = \PiChoice{\Call\Omega\x}{\Call\Omega\x}$.
	We obtain the following infinite derivation showing that $\Call\Omega\x$ is quasi typed.
	\[
    \begin{prooftree}
        \[
            \mathstrut\smash\vdots
            \justifies
            \qtp{x : \Zero}{\Call\Omega\x}
        \]
        \qquad
        \[
            \mathstrut\smash\vdots
            \justifies
            \qtp{x : \Zero}{\Call\Omega\x}
        \]
        \justifies
        \qtp{x : \Zero}{\Call\Omega\x}
        \using\refrule\PiChoiceRule
    \end{prooftree}
    \]
%
\end{example}

As illustrated by the next example,
there exist non-terminating processes that are quasi typed also in logically
sound contexts.

\begin{example}[Compulsive Buyer]
    \label{ex:compulsive_buyer_ll}
    Consider the following variant of the $\Buyer$ process
    \[
        \Buyer(x) = \Rec\x.\Select\AddTag\x.\Call\Buyer{x}
    \] 
    that models a ``compulsive buyer'', namely a buyer that adds infinitely many
    items to the shopping cart but never pays. Using $\FormulaF \eqdef \tmu\X.X
    \plinchoice \One$ and an arbitrary atomic address $a$ we can build the following
    infinite derivation
    \[
        \begin{prooftree}
            \[
                \[
                    \mathstrut\smash\vdots
                    \justifies
                    \qtp{
                        x : \Formula_{ail}
                    }{
                        \Call\Buyer{x}
                    }
                \]
                \justifies
                \qtp{
                    x : (\Formula \plinchoice \One)_{ai}
                }{
                    \Select\AddTag\x.\Call\Buyer{x}
                }
                \using\refrule[select]{\SelectRule}
            \]
            \justifies
            \qtp{
                x : \Formula_a
            }{
                \Call\Buyer{x}
            }
            \using\refrule[rec]{\RecRule}
        \end{prooftree}
    \]
    %
    showing that this process is quasi typed. By combining this derivation with
    the one for $\Seller$ in \Cref{ex:bsc_ll_ts} we obtain a
    derivation establishing that $\Cut\x{\Call\Buyer\x}{\Call\Seller{x,y}}$ is
    quasi typed in the context $y : \One$, although this composition cannot
    terminate.
    %
    \eoe
\end{example}

To rule out unsound derivations like those in
\Cref{ex:omega,ex:compulsive_buyer_ll} it is necessary to impose a validity
condition on derivations \citep{BaeldeDoumaneSaurin16,Doumane17}. Roughly
speaking, \muMALL's validity condition requires every infinite branch of a
derivation to be supported by the continuous unfolding of a greatest fixed
point. In order to formalize this condition, we start by defining
\emph{threads}, which are sequences of types describing sequential interactions
at the type level.

\begin{definition}[Thread]
    \label{def:thread}
    A \emph{thread} of $S$ is a sequence of types $(S_i)_{i\in o}$ for some
    $o\in\omega + 1$ such that $S_0 = S$ and $S_i \tred S_{i+1}$ whenever
    $i+1\in o$.
\end{definition}

Hereafter we use $t$ to range over threads. 

\begin{example}
	\label{ex:bsc_ll_thread}
	Consider
	$\Formula \eqdef \tmu\X.X \plinchoice \One$ from \Cref{ex:bsc_ll_formulas} we have that
	$t \eqdef (\Formula_a,(\Formula\plinchoice\One)_{ai},\Formula_{ail},\dots)$ is an
	infinite thread of $\Formula_a$.
\end{example}

A thread is \emph{stationary} if it has an
infinite suffix of equal types. The thread $t$ from \Cref{ex:bsc_ll_thread} is not stationary.
Among all threads, we are interested in finding those in which a $\tnu$-formula
is unfolded infinitely often. These threads, called $\tnu$-threads, are precisely
defined thus:

\begin{definition}[$\tnu$-thread]
    \label{def:nu-thread}
    Let $t = (S_i)_{i\in\omega}$ be an infinite thread, let $\strip{t}$ be the
    corresponding sequence $(\strip{S_i})_{i\in\omega}$ of formulas and let
    $\InfOften{t}$ be the set of elements of $\strip{t}$ that occur infinitely
    often in $\strip{t}$. We say that $t$ is a \emph{$\tnu$-thread} if
    $\minf\InfOften{t}$ is defined and is a $\tnu$-formula.
\end{definition}

\begin{example}
	Consider the infinite thread $t$ from \Cref{ex:bsc_ll_thread}.
	We have $\InfOften{t} = \set{\Formula, \Formula \choice \One}$ 
	and $\minf\InfOften{t} = \Formula$, so $t$
	is \emph{not} a $\tnu$-thread because $\Formula$ is not a $\tnu$-formula.
\end{example}

\begin{example}
	Consider the following formulas 
	\[
	\begin{array}{rclrcl}
		\FormulaF & \eqdef & \tnu\X.\tmu\Y.X \choice Y 
		& \qquad
		\FormulaG & \eqdef & \tmu\Y.\FormulaF \choice Y
	\end{array}
	\]	
	Observe that $\FormulaG$ is the ``unfolding'' of $\FormulaF$. 
	Now 
	\[
	t_1 \eqdef (\FormulaF_a, \FormulaG_{ai},
	(\FormulaF \choice \FormulaG)_{aii}, \FormulaF_{aiil}, \dots)
	\] is a thread of
	$\FormulaF_a$ such that $\InfOften{t_1} = \set{\FormulaF, \FormulaG, \FormulaF
	\choice \FormulaG}$ and we have $\minf\InfOften{t_1} = \FormulaF$ because
	$\FormulaF \subf \FormulaG$, so $t_1$ is a $\tnu$-thread.
	%
	If, on the other hand, we consider the thread 
	\[
	t_2 \eqdef (\FormulaF_a,
	\FormulaG_{ai}, (\FormulaF \choice \FormulaG)_{aii}, \FormulaG_{aiir},
	(\FormulaF \choice \FormulaG)_{aiiri}, \dots)
	\]
	such that $\InfOften{t_2} =
	\set{\FormulaG, \FormulaF \choice \FormulaG}$ we have $\minf\InfOften{t_2} =
	\FormulaG$ because $\FormulaG \subf \FormulaF \choice \FormulaG$, so $t_2$ is
	not a $\tnu$-thread.
\end{example}

Intuitively, the $\subf$-minimum formula among those that occur infinitely often
in a thread is the outermost fixed point operator that is being unfolded
infinitely often. It is possible to show that this minimum formula is always
well defined \citep{Doumane17}. If such minimum formula is a greatest fixed point
operator, then the thread is a $\tnu$-thread.

Now we proceed by identifying threads along branches of typing derivations. To
this aim, we provide a precise definition of \emph{branch}.

\begin{definition}[Branch]
    \label{def:branch}
    A \emph{branch} of a typing derivation is a sequence
    $(\qtp{\Ctx_i}{P_i})_{i\in o}$ of judgments for some $o\in\omega+1$ such
    that $\qtp{\Ctx_0}{P_0}$ occurs somewhere in the derivation and
    $\qtp{\Ctx_{i+1}}{P_{i+1}}$ is a premise of the rule application that
    derives $\qtp{\Ctx_i}{P_i}$ whenever $i+1\in o$.
\end{definition}

An infinite branch is valid if supported by a $\tnu$-thread that originates
somewhere therein.

\begin{definition}[Valid Branch]
    \label{def:valid_branch}
    Let $\gamma = (\qtp{\Ctx_i}{P_i})_{i\in\omega}$ be an infinite branch in
    a derivation. We say that $\gamma$ is \emph{valid} if there exists
    $j\in\omega$ such that $(S_k)_{k\geq j}$ is a non-stationary $\tnu$-thread
    and $S_k$ is in the range of $\Ctx_k$ for every $k \geq j$.
\end{definition}

\begin{example}
	The infinite branch in the typing derivation for $\Seller$ of
	\Cref{ex:bsc_ll_formulas} is valid since it is supported by the $\tnu$-thread
	$(\FormulaG_{\dual{a}}, (\FormulaG \plinbranch \Bot)_{\dual{a}i},
	\FormulaG_{\dual{a}il},\dots)$ where $\FormulaG \eqdef \tnu\X.X \plinbranch \Bot$
	happens to be the $\subf$-minimum formula that is unfolded infinitely often.
\end{example}

\begin{example}
	The infinite branch in the typing derivation for $\Buyer$ of
	\Cref{ex:compulsive_buyer_ll} is invalid, because the only infinite thread
	in it is $(\FormulaF_a, (\FormulaF \plinchoice \One)_{ai}, \FormulaF_{ail}, \dots)$
	which is not a $\tnu$-thread.
\end{example}

A \muMALL derivation is valid if so is every infinite branch in
it \citep{BaeldeDoumaneSaurin16,Doumane17}. For the purpose of ensuring fair
termination, this condition is too strong because some infinite branches in a
typing derivation may correspond to unfair executions that, by definition, we
neglect insofar its termination is concerned. For example, the infinite branch
in the derivation for $\Buyer$ of \Cref{ex:bsc_ll_formulas} corresponds to
an unfair run in which the buyer insists on adding items to the shopping cart,
despite it periodically has a chance of paying the seller and terminate the
interaction. That typing derivation for $\Buyer$ would be considered an invalid
proof in \muMALL because the infinite branch is not supported by a $\tnu$-thread
(in fact, there is a $\tmu$-formula that is unfolded infinitely many times along
that branch, as in \Cref{ex:compulsive_buyer_ll}).

It is generally difficult to understand if a branch corresponds to a fair or
unfair run because the branch describes the evolution of an incomplete process
whose behavior is affected by the interactions it has with processes found in
other branches of the derivation.
%
However, we can detect (some) unfair branches by looking at the
non-deterministic choices they traverse, since choices are made autonomously by
processes. To this aim, we introduce the notion of \emph{rank} to estimate the
least number of choices a process can possibly make during its lifetime.

\begin{definition}[Rank]
    \label{def:rank}
    Let $\rankR$ and $\rankS$ range over the elements of $\RankSet \eqdef
    \Nat\cup\set\infty$ equipped with the expected total order $\leq$ and
    operation $+$ such that $\rankR + \infty = \infty + \rankR = \infty$.
    %
    The \emph{rank} of a process $P$, written $\rankof{P}$, is the least element
    of $\RankSet$ such that
    \[
    	\begin{array}{lr}
        \begin{array}{@{}r@{~}c@{~}l@{}}
            \rankof{\Link\x\y} & = & 0 \\
            \rankof{\Fail\x} & = & 0 \\
            \rankof{\PiClose\x} & = & 0 \\
            \rankof{\PiWait\x.P} & = & \rankof{P} \\
            \rankof{\Join[z]\x\y.P} & = & \rankof{P} \\
            \rankof{\Select[y]{\InTag_i}\x.P} & = & \rankof{P}
        \end{array}
				& \qquad
        \begin{array}{@{}r@{~}c@{~}l@{}}
        		\rankof{\Rec[y]\x.P} & = & \rankof{P} \\
            \rankof{\Corec[y]\x.P} & = & \rankof{P} \\
            \rankof{\Case[y]\x{P}{Q}} & = & \max\set{\rankof{P},\rankof{Q}} \\
            \rankof{\PiChoice{P}{Q}} & = & 1 + \min\set{\rankof{P},\rankof{Q}} \\
            \rankof{\Cut\x{P}{Q}} & = & \rankof{P} + \rankof{Q} \\
            \rankof{\Fork[z]\x\y{P}{Q}} & = & \rankof{P} + \rankof{Q}
        \end{array}
    	\end{array}
    \]
\end{definition}

Roughly, the rank of terminated processes is $0$, that of processes with a
single continuation $P$ coincides with the rank of $P$, and that of processes
spawning two continuations $P$ and $Q$ is the sum of the ranks of $P$ and $Q$.
Then, the rank of a sum input with continuations $P$ and $Q$ is conservatively
estimated as the maximum of the ranks of $P$ and $Q$, since we do not know which
one will be taken, whereas the rank of a choice with continuations $P$ and $Q$
is 1 plus the minimum of the ranks of $P$ and $Q$.

\begin{example}
	Consider $\Buyer$ and $\Seller$ from \Cref{ex:bsc_ll_proc} and $\Omega$ from \Cref{ex:omega}.
 	Then have $\rankof{\Call\Buyer\x} = 1$, $\rankof{\Call\Seller{x,y}} = 0$
 	and $\rankof{\Call\Omega\x} = \infty$.
\end{example}

Note that $\rankof{P}$ only depends on the structure of $P$ but not on the
actual names occurring in $P$. As a consequence, when $P$ is defined by means of a
\emph{finite} system of equations, the value of $\rankof{P}$ too can be
determined by a \emph{finite} system of equations.

\begin{example}
	\label{ex:eq_system}
	Consider the definition of $\Buyer$ found in
	\Cref{ex:bsc_ll_proc}. In order to compute $\rankof{\Call\Buyer\x}$ we consider
	the system of equations
	{
    \renewcommand{\x}{\bullet}
    \renewcommand{\y}{\bullet}
    \renewcommand{\z}{\bullet}
    %
    \[
    \begin{array}{rcl}
    	\rankof{\Call\Buyer\x} & = & \rankof{
            \PiChoice{
                \Select[\z]\AddTag\y.\Call\Buyer\z
            }{
                \Select[\z]\PayTag\y.\PiClose\z
            }
        }
      \\
      \rankof{
            \PiChoice{
                \Select[\z]\AddTag\y.\Call\Buyer\z
            }{
                \Select[\z]\PayTag\y.\PiClose\z
            }
        } & = & 1 + \min\set{ 
            \\ & & \qquad\quad \rankof{\Select[\z]\AddTag\y.\Call\Buyer\z},
            \\ & & \qquad\quad \rankof{\Select[\z]\PayTag\y.\PiClose\z}
        }
        \\
        \rankof{\Select[\z]\AddTag\y.\Call\Buyer\z} & = & \rankof{\Call\Buyer\z}
        \\
        \rankof{\Select[\z]\PayTag\y.\PiClose\z} & = & \rankof{\PiClose\z}
        \\
        \rankof{\PiClose\z} & = & 0
    \end{array}
    \]
	}
	%
	where we have used a placeholder $\bullet$ in place of every channel name
	occurring in these terms.
	%
	\eoe
\end{example}

Every such system of equations can be thought of as a function $\ffun :
\parens\RankSet^n \to \parens\RankSet^n$ on the complete lattice
$\parens\RankSet^n$ ordered by the pointwise extension of $\leq$ in $\RankSet$.
%
Note that $\ffun$ is monotone, because all the operators occurring in the
definition of rank (see \Cref{def:rank}) are monotone. So, $\ffun$ has a least fixed
point by the Knaster-Tarski theorem and the rank of $P$ is the component of this
fixed point that corresponds to $\rankof{P}$.

\begin{example}
	For system of equations in \Cref{ex:eq_system} we have $n = 5$ and we have
	\[
    \ffun(x_1,x_2,x_3,x_4,x_5) = (x_2,1+\min\set{x_3,x_4},x_1,x_5,0)
	\]
	whose least solution is $(1,1,1,0,0)$. Now $\rankof{\Call\Buyer\x}$ corresponds
	to the first component of this solution, that is $1$.
	%
	\eoe
\end{example}

\begin{definition}
    \label{def:fair_branch}
    A branch is \emph{fair} if it traverses finitely many, finitely-ranked
    choices.
\end{definition}

A finitely-ranked choice is at finite distance from a region of the process in
which there are no more choices. An \emph{unfair} branch gets close to such
region infinitely often, but systematically avoids entering it.
%
Note that every finite branch is also fair, but there are fair branches that are
infinite. 

\begin{example}
	All the infinite branches inside the derivation of
	\Cref{ex:omega} and the only infinite branch in the derivation for
	$\Call\Seller{x,y}$ of \Cref{ex:bsc_ll_ts} are fair since they do not
	traverse any finitely-ranked choice. On the contrary, the only infinite branch
	in the derivation for $\Call\Buyer\x$ of the \Cref{ex:bsc_ll_ts} is
	unfair since it traverses infinitely many finitely-ranked choices. All fair
	branches in the same derivation for $\Buyer$ are finite.
\end{example}

At last we can define our notion of well-typed process.

\begin{definition}[Well-Typed Process]
    \label{def:wtp}
    We say that $P$ is \emph{well typed} in $\Ctx$, written
    $\piwtp\Ctx{P}$, if the judgment $\qtp\Ctx{P}$ is derivable and each
    fair, infinite branch in its derivation is valid.
\end{definition}

\begin{example}
	 $\Omega$ is ill typed since the fair, infinite branches in
	\Cref{ex:omega} are all invalid.
\end{example}

\begin{theorem}[Soundness]
	\label{thm:ts_ll_sound}
    If $\piwtp{x : \One}{P}$ and $P \wred Q$ then $Q \wred \PiClose\x$.
\end{theorem}

As for the session-based calculi (\Cref{ch:ft_bin,ch:ft_multi}) 
\Cref{thm:ts_ll_sound} entails all the good properties we expect from well-typed
processes: \emph{failure freedom} (no unguarded sub-process $\Fail\y$ ever
appears), \emph{deadlock freedom} (if the process stops it is terminated),
\emph{lock freedom} \citep{Kobayashi02,Padovani14} (every pending action can be
completed in finite time) and \emph{junk freedom} (every channel can be
depleted).

%%%%%%%%%%%%%%%%
%%% EXAMPLES %%%
%%%%%%%%%%%%%%%%

\subsection{Examples}
\label{ssec:ts_ll_ex}
\beginbass
%
We dedicate the rest of \Cref{sec:ts_multi_ts} to the analysis of some examples
that integrate all the features of the presented type system. We start from some
basic examples and then we move to more involved ones. First, in \Cref{ex:bsc_ts_multi} we take into
account our slightly different variant of the running example (\Cref{ex:bsc_multi}). 
For what concerns the problematic processes in \Cref{ssec:boundedness}, they are still
valid in the multiparty context (see \Cref{rm:boundedness_multi}). 
We use the rest of the examples to deal with the processes introduced in \Cref{ssec:proc_ex_multi}.

%%%%%%%%%%%%%%%%%%%%%%%%%%%%%%%%%%%%%%%
%%% Action/Session/Cast-Boundedness %%%
%%%%%%%%%%%%%%%%%%%%%%%%%%%%%%%%%%%%%%%

\begin{remark}[Boundedness]
\label{rm:boundedness_multi}
	All the problematic processes that we presented in \Cref{ssec:boundedness} are still valid
	in the multiparty scenario and can be dealt with using the techniques that we mentioned for
	the binary case. In particular
	\begin{description}
	\item[Action-boundedness.] Type system with corules.
	\end{description}
	\[
    A \peq A
    \qquad
    B \peq {B \pchoice B}
    \qquad
    C \peq {C \pchoice \pdone}
  \]
	\begin{description}
	\item[Session-boundedness.] Rule \refrule{tm-par} increases the rank by one.
	\end{description}
	\[
    A \peq
      \pres{s}{
        \act{\ep{s}{\rolep}}{\roleq}\oact\set{
          \Tag[a].\pclose{\ep{s}{\rolep}},
          \Tag[b].\pwait{\ep{s}{\rolep}}{A}
        }
        \parop
        \act{\ep{s}{\roleq}}{\rolep}\iact\set{
          \Tag[a].\pwait{\ep{s}{\roleq}}{A},
          \Tag[b].\pclose{\ep{s}{\roleq}}
        }
      }
  \]
	\begin{description}
	\item[Cast-boundedness.] Rule \refrule{tm-cast} increases the rank by the weight of the subtyping
		being applied.
	\end{description}
	\[
  	B(x) \peq \pcast{x}\act{x}\seller\oact\tadd.\pinvk{B}{x}
  \]
	%
	\eor
\end{remark}

%%%%%%%%%%%%%%%%%%%%%%%
%%% Running Example %%%
%%%%%%%%%%%%%%%%%%%%%%%

\begin{example}
  \label{ex:bsc_ts_multi} 
  Let us show some typing derivations for fragments of \Cref{ex:bsc_multi}. 
  Let $\S_b$, $S_s$ and $S_c$ be the types from \Cref{ex:bsc_ty_multi}.
  We collapse roles to their initials.
  Let $\S'_b = \rseller\Out\tadd\rseller\Out\tadd.\S'_b + \rseller\Out\tpay.\End[\Out]$.
  Concerning $\Buyer$, we obtain the infinite derivation
  \[
    \begin{prooftree}
      \[
        \[
          \mathstrut\smash\vdots
          \justifies
          \wtp[0]{
            x : \S'_b
          }{
            \pinvk\Buyer{x}
          }
          \using\refrule{tm-call}
        \]
        \justifies
        \wtp[0]{
          x : \rseller\Out\tadd.\S'_b
        }{
          \act{x}\rseller\oact\tadd.\pinvk\Buyer{x}
        }
        \using\refrule{tm-tag}
      \]
      \[
        \justifies
        \wtp[0]{
          x : \End[\Out]
        }{
          \pclose{x}
        }
        \using\refrule{tm-close}
      \]
      \justifies
      \wtp[0]{
        x : \S'_b
      }{
        \act{x}\rseller\oact\set{
          \tadd.\act{x}\rseller\oact\tadd.\pinvk\Buyer{x},
          \tpay.\pclose{x}
        }
      }
      \using\refrule{tm-tag}
    \end{prooftree}
  \]
  %
  and, for each judgment in it, it is easy to find a finite derivation possibly
  using \refrule{com-tag}. Concerning $\Main$ we obtain
  \[
    \begin{prooftree}
      \[
        \mathstrut\smash\vdots
        \justifies
        \wtp[0]{
          \ep{s}\rbuyer : \S'_b
        }{
          \pinvk\Buyer{\ep{s}\rbuyer}
        }
        \using\refrule{tm-call}
      \]
      \[
        \smash\vdots
        \justifies
        \wtp[0]{
          \ep{s}\rseller : S_s
        }{
          \pinvk\Seller{\ep{s}\rseller}
        }
      \]
      \vdots
      \justifies
      \wtp[1]{
        \EmptyCtx
      }{
        \pres\sn{
          \pinvk\Buyer{\ep{s}\rbuyer} \ppar \pinvk\Seller{\ep{s}\rseller} \ppar \pinvk\Carrier{\ep{s}\rcarrier}
        }
      }
      \using\refrule{tm-par}
    \end{prooftree}
  \]
  %
  where the application of \refrule{tm-par} is justified by the fact that
  $\Map\rbuyer{\S'_b} \parop \Map\rseller{S_s} \parop \Map\rcarrier{S_c}$ is coherent.
  We recall that $\S_b \subt[1] \S'_b$ (\Cref{ex:bsc_fair_sub}).
  %
  No participant creates new sessions or performs casts, so they all have zero
  rank. The rank of $\Main$ is 1 since it creates the session $s$.
  %
  \eoe
\end{example}

%%%%%%%%%%%%%%%%%%%%%%%%%%%%%%
%%% 2Buyers-Seller-Shipper %%%
%%%%%%%%%%%%%%%%%%%%%%%%%%%%%%

\begin{example}
\label{ex:2bsc-ts}
In this example we show that the process $\Buyer_1$ playing the role $\rbuyer_1$
in the inner session of \Cref{ex:2bsc_multi} is well typed. For clarity, we recall its
definition here: 
\[
Buyer_1(x,y) \peq \act{y}{\rbuyer_2}\oact\{
	\begin{lines}
	  \tsplit.\act{y}{\rbuyer_2}\iact\{
		\begin{lines}
			\tyes.\pcast{x}
				\act{x}\rseller\oact\tok.
				\act{x}\rcarrier\iact\tbox.
				\pwait{x}
				\pwait{y}
				\pdone,
				\\
			\tno.\pinvk{Buyer_1}{x,y} \},
		\end{lines}
	  \\
	  \tgiveup.
		\pwait{y}
		\pcast{x}
		\act{x}\rseller\oact\tcancel.
		\pwait{x}
		\pdone \}
	\end{lines}
\]

We wish to build a typing derivation showing that $Buyer_1$ has rank $1$ and
uses $x$ and $y$ respectively according to $S$ and $T$, where $S =
\rseller\Out\tok.\rcarrier\In\tbox.\End[\In] + \rseller\Out\tcancel.\End[\In]$
and $T = \rbuyer_2\Out\tsplit.(\rbuyer_2\In\tyes.\End[\In] + \rbuyer_2\In\tno.T)
+ \rbuyer_2\Out\tgiveup.\End[\In]$.
%
As it has been noted previously, what makes this process interesting is that it
uses the endpoint $x$ differently depending on the messages it exchanges with
$\rbuyer_2$ on $y$. Since rule \refrule{tm-tag} requires any endpoint other
than the one on which messages are exchanged to have the same type, the only way
$\Buyer_2$ can be declared well typed is by means of the casts that occur in its
body.
%
For the branch in which $\Buyer_1$ proposes to $\tsplit$ the payment we obtain
the following derivation tree (we show only the $\tyes$ branch, the $\tno$ one is trivial):
\[
	\begin{prooftree}
		\[
			\[
				\[
					\[
						\[
							\[
								\justifies
								\wtp[0]\EmptyCtx\pdone
								\using\refrule{tm-done}
							\]
							\justifies
							\wtp[0]{
								y : \End[\In]
							}{
								\pwait{y}\pdone
							}
							\using\refrule{tm-wait}
						\]
						\justifies
						\wtp[0]{
							x : \End[\In],
							y : \End[\In]
						}{
							\pwait[\dots]{x}
						}
						\using\refrule{tm-wait}
					\]
					\justifies
					\wtp[0]{
						x : \rcarrier\In\tbox.\End[\In],
						y : \End[\In]
					}{
						\act{x}\rcarrier\iact\tbox\dots
					}
					\using\refrule{tm-tag}
				\]
				\justifies
				\wtp[0]{
					x : \rseller\Out\tok.\rcarrier\In\tbox.\End[\In],
					y : \End[\In]
				}{
					\act{x}\rseller\oact\tok\dots
				}
				\using\refrule{tm-tag}
			\]
			\justifies
			\wtp[1]{
				x : S,
				y : \End[\In]
			}{
				\pcast{x}\dots
			}
			\using\refrule{tm-cast}
		\]
		\vdots
		\justifies
		\wtp[1]{
			x : S,
			y : \rbuyer_2\In\tyes.\End[\In] + \rbuyer_2\In\tno.T
		}{
			\act{y}{\rbuyer_2}\iact\set{\tyes\dots, \tno\dots}
		}
		\using\refrule{tm-tag}
	\end{prooftree}
\]

Note how the application of \refrule{tm-cast} is key to change the type of $x$ in
the branch where the proposed split is accepted by $\rbuyer_2$. In that branch,
$x$ is deterministically used to send an $\tok$ message and we leverage on the
fair subtyping relation $S \subt[1] \rseller\Out\tok.\rcarrier\In\tbox.\End[\In]$.
%

For the branch in which $\Buyer_1$ sends $\tgiveup$ we obtain the following
derivation tree:
\[
	\begin{prooftree}
		\[
			\[
				\[
					\[
						\justifies
						\wtp[0]\EmptyCtx{
							\pdone
						}
						\using\refrule{tm-done}
					\]
					\justifies
					\wtp[0]{
						x : \End[\In]
					}{
						\pwait{x}\pdone
					}
					\using\refrule{tm-wait}
				\]
				\justifies
				\wtp[0]{
					x : \rseller\Out\tcancel.\End[\In]
				}{
					\act{x}\rseller\oact\tcancel.
					\pwait{x}
					\pdone
				}
			\]
			\justifies
			\wtp[1]{
				x : S
			}{
				\pcast{x}
				\act{x}\rseller\oact\tcancel.
				\pwait{x}
				\pdone
			}
			\using\refrule{tm-cast}
		\]
		\justifies
		\wtp[1]{
			x : S,
			y : \End[\In]
		}{
			\pwait{y}
			\pcast{x}
			\act{x}\rseller\oact\tcancel.
			\pwait{x}
			\pdone
		}
		\using\refrule{tm-wait}
	\end{prooftree}
\]

Once again the cast is necessary to change the type of $x$, but this time
leveraging on the fair subtyping relation $S \subt[1]
\rseller\Out\tcancel.\End[\In]$.
%
These two derivations can then be combined to complete the proof that the body
of $\Buyer_1$ is well typed:
\[
	\begin{prooftree}
		\qquad
		\mathstrut\smash\vdots
		\qquad
		\qquad
		\qquad
		\smash\vdots
		\qquad
		\justifies
		\wtp[1]{
			x : S,
			y : T
		}{
			\act{y}{\rbuyer_2}\oact\set{\tsplit\dots, \tgiveup\dots}
		}
		\using\refrule{tm-tag}
	\end{prooftree}
\]

Clearly, it is also necessary to find finite derivation trees for all of the
judgments shown above. This can be easily achieved using the corule
\refrule{com-tag}.
%
\eoe
\end{example}

\begin{example}
	\label{ex:non-det}
	Casts can be useful to reconcile the types of a channel that is used
	differently in different branches of a non-deterministic choice. For
	example, below is an alternative modeling of $\Buyer$ from \Cref{ex:bsc_multi}
	where we abbreviate $\role[\seller]$ to $\rseller$ for convenience:
	\[
		\Definition{B}{x}{
			\pcast{x}
			\act{x}\rseller\oact\tadd.
			\act{x}\rseller\oact\tadd.\pinvk{B}{x}
			\pchoice
			\pcast{x}
			\act{x}\rseller\oact\tpay.
			\pclose{x}
		}
	\]

	Note that $x$ is used for sending two $\tadd$ messages in the left branch of
	the non-deterministic choice and for sending a single $\tpay$ message in the
	right branch. Given the session type $S = \rseller\Out\tadd.S +
	\rseller\Out\tpay.\End[\Out]$ and using the fair subtyping relations $S
	\subt[2] \rseller\Out\tadd.\rseller\Out\tadd.S$ and $S \subt[1]
	\rseller\Out\tpay.\End[\Out]$ we can obtain the following typing derivation
	for the body of $B$ (we show only the left branch as the right one contains a
	straightforward application of $S \subt[1] \rseller\Out\tpay.\End[\Out]$):
	\[
		\begin{prooftree}
			\[
				\[
					\[
						\[
							\mathstrut\smash\vdots
							\justifies
							\wtp[1]{
								x : S
							}{
								\pinvk{B}{x}
							}
							\using\refrule{tm-call}
						\]
						\justifies
						\wtp[1]{
							x : \rseller\Out\tadd.S
						}{
							\act{x}\rseller\oact\tadd.\pinvk{B}{x}
						}
						\using\refrule{tm-tag}
					\]
					\justifies
					\wtp[1]{
						x : \rseller\Out\tadd.\rseller\Out\tadd.S
					}{
						\act{x}\rseller\oact\tadd.
						\act{x}\rseller\oact\tadd.\pinvk{B}{x}
					}
					\using\refrule{tm-tag}
				\]
				\justifies
				\wtp[3]{
					x : S
				}{
					\pcast{x}
					\act{x}\rseller\oact\tadd.
					\act{x}\rseller\oact\tadd.\pinvk{B}{x}	
				}
				\using\refrule{tm-cast}
			\]
			\vdots
			\justifies
			\wtp[1]{
				x : S
			}{
				\pcast{x}
				\act{x}\rseller\oact\tadd.
				\act{x}\rseller\oact\tadd.\pinvk{B}{x}
				\pchoice
				\pcast{x}
				\act{x}\rseller\oact\tpay.
				\pclose{x}	
			}
			\using\refrule{tm-choice}
		\end{prooftree}
	\]
	%
	In general, the transformation $\pbranch[i=1..n]{u}{\role}\oact{\Tag_i}{P_i}
	\leadsto \pcast{u}\act{u}{\role}\oact{\Tag_1}.P_1 \pchoice \cdots \pchoice
	\pcast{u}\act{u}\role\oact{\Tag_n}.P_n$ does not always preserve typing, so
	it is not always possible to encode the output of tags using casts and
	non-deterministic choices. As an example, the definition
	%
	\[
		\Definition\SlotMachine{x}{
			\act{x}\rplayer\iact\set{
				\tplay.\act{x}\rplayer\oact\set{
					\twin.\pinvk\SlotMachine{x},
					\tlose.\pinvk\SlotMachine{x}
				},
				\tquit.\pclose{x}
			}
		}
	\]
	implements the unbiased slot machine of \Cref{ex:slot_fair_sub} that waits
	for a message indicating whether a $\rplayer$ wants to $\tplay$ another game or
	to $\tquit$ (we assign role $\rplayer$ to the player). 
	In the former case, the slot machine notifies $\rplayer$ of the
	outcome (either $\twin$ or $\tlose$).
	%
	It is easy to see that $\SlotMachine$ is well typed under the global type
	assignment $\tass\SlotMachine{T}{0}$ where $T =
	\rplayer\In\tplay.(\rplayer\Out\twin.T + \rplayer\Out\tlose.T) +
	\rplayer\In\tquit.\End[\Out]$. In particular, $\SlotMachine$ has rank $0$
	since it performs no casts and it creates no sessions. If we encode the tag
	output in $\SlotMachine$ using casts and non-deterministic choices we end up
	with the following process definition, which is ill typed because it cannot
	be given a finite rank:
	%
	\[
		\Definition{\SlotMachine}{x}{
			\act{x}\rplayer\iact\set{
				\tplay.(
					\pcast{x}
					\act{x}\rplayer\oact\twin.
					\pinvk{\SlotMachine}{x}
					\pchoice
					\pcast{x}
					\act{x}\rplayer\oact\tlose.
					\pinvk{\SlotMachine}{x}
				),
				\tquit.\pclose{x}
			}
		}
	\]

	The difference between this version of $\SlotMachine$ and the above
	definition of $B$ is that $\SlotMachine$ always recurs after a cast, so it
	is not obvious that finitely many casts suffice in order for $\SlotMachine$
	to terminate. 
	%
	\eoe
\end{example}

%%%%%%%%%%%%%%%%%%%%%%%%%%%%%%%%%%%%
%%% Buyer-Seller-Shipper Refined %%%
%%%%%%%%%%%%%%%%%%%%%%%%%%%%%%%%%%%%

\begin{example} 
	\Cref{ex:2bsc-ts}
	shows that casts are essential in the type derivation.
	However, the process would be well typed if we considered a subtyping relation that does not preserve coherence
	\citep{GayHole05} for the involved types are finite.
	Now we refine the buyer from \Cref{ex:bsc_multi} in order to consider more involved sessions. Again, we
	collapse role names to their initials.
	\begin{align*}
		B(x) & \peq
			{\pcast{x}\pinvk{B_1}{x} \pchoice \pcast{x}\pinvk{B_2}{x}} 
		\\
		B_1(x) & \peq
			{\act{x}{\rseller}\oact\Tag[add].\act{x}{\rseller}\oact\set{
  			\Tag[add].\pinvk{B_1}{x},\, 
  			\Tag[pay].\pwait{x}{\pdone} 
			}} 
	  \\  
		B_2(x) & \peq
			{\act{x}{\rseller}\oact\set{
				\Tag[add].\act{x}{\rseller}\oact\Tag[add]. \pinvk{B_2}{x},\, 
  				\Tag[pay].\pwait{x}{\pdone} 
			}}
	\end{align*}
	$B_2$ corresponds to the buyer in \Cref{ex:bsc_multi} while $B_1$ is the acquirer that adds an odd number of items to the cart. 
	$B$ non deterministically chooses to behave according to $B_1$ or $B_2$.
	%
	Let $S_{b_1}$ and $S_{b_2}$ be the session types such that 
	$x : S_{b_1}$ in $B_1$ and $x : S_{b_2}$ in $B_2$ respectively:
	\[
	\begin{array}{ll}
		S_{b_1} = \rseller\Out\Tag[add].(\rseller\Out\Tag[add].S_{b_1} + \rseller\Out\Tag[pay].\End[\Out])
		&
		S_{b_2} = \rseller\Out\Tag[add].\rseller\Out\Tag[add].S_{b_2} + \rseller\Out\Tag[pay].\End[\Out]
	\end{array}
	\]
	In \cref{ex:bsc_fair_sub} we showed that $S \subt[1] S_{b_2}$ where $S =\rseller\Out\tadd.S + \rseller\Out\tpay.\End[\Out]$
	models the acquirer that adds arbitrarily many items to the cart. Analogously, we can prove that $S \subt[2] S_{b_1}$.
	Hence we derive
	\[
	\begin{prooftree}
		\[
			\[
				\vdots
				\justifies
				\wtp[0]{x : S_{b_1}}{\pinvk{B_1}{x}}
				\using\refrule{tm-call}
			\]
			\justifies
			\wtp[2]{x : S}{\pcast{x}\pinvk{B_1}{x}}
			\using\refrule{tm-cast}
		\]
		\[
			\[
				\vdots
				\justifies
				\wtp[0]{x : S_{b_2}}{\pinvk{B_2}{x}}
				\using\refrule{tm-call}
			\]
			\justifies
			\wtp[1]{x : S}{\pcast{x}\pinvk{B_2}{x}}
			\using\refrule{tm-cast}
		\]
		\justifies
		\wtp[1]
			{x : S}
			{\pcast{x}\pinvk{B_1}{x} \pchoice \pcast{x}\pinvk{B_2}{x}}
		\using\refrule{tm-choice}
	\end{prooftree}
	\]
	Again, the casts are crucial to obtain the type derivation of process $B$ because rule 
	\refrule{tm-choice} requires that $B_1$ and $B_2$ are typed in the same context.
	Note that $B_1$ and $B_2$ are typed with rank 0 since no sessions are created and no casts are performed by the processes.
\end{example}

%%%%%%%%%%%%%%%%%%%%%%%%%%%%%%%%%
%%% Unbounded No. Of Sessions %%%
%%%%%%%%%%%%%%%%%%%%%%%%%%%%%%%%%

\begin{example}
	\label{ex:pms_ts}
	Here we provide evidence that the process definitions in \Cref{ex:pms_multi} are
	well typed, even if they model processes that can open arbitrarily many
	sessions. In that example, the most interesting process definition is that
	of the worker $\Sort$, which is recursive and may create a new session. In
	contrast, $\Merge$ is finite and $\Main$ only refers to $\Sort$. We claim
	that these process definitions are well typed under the global type
	assignments
	\[
		\tass\Main{}{1}
		\qquad
		\tass\Sort{U}{0}
		\qquad
		\tass\Merge{T, V}{0}
	\]
	where 
	\[
	\begin{array}{lll}
		T = \rmaster\Out\tres.\End[\Out]
		&
		U = \rmaster\In\treq.T
		&
		V = \rworker_1\Out\treq.\rworker_2\Out\treq.\rworker_1\In\tres.\rworker_2\In\tres.\End[\In]
	\end{array}
	\]
	For the branch of $\Sort$ that creates a new session we obtain the
	derivation tree
	\[
		\begin{prooftree}
			\[
				\mathstrut\smash\vdots
				\justifies
				\wtp[0]{
					x : T,
					\ep{t}\rmaster : V
				}{
					\pinvk\Merge{x,\ep{t}\rmaster}
				}
				\using\refrule{tm-call}
			\]
			\[
				\smash\vdots
				\justifies
				\wtp[0]{
					\ep{t}{\rworker_i} : U
				}{
					\pinvk\Sort{\ep{t}{\rworker_i}}
				}
			\]
			\justifies
			\wtp[1]{
				x : T
			}{
				\pres{t}{
					\pinvk\Merge{x,\ep{t}\rmaster} \parop
					\pinvk\Sort{\ep{t}{\rworker_1}} \parop
					\pinvk\Sort{\ep{t}{\rworker_2}}
				}
			}
			\using\refrule{tm-par}
		\end{prooftree}
	\]
	%
	where $i=1,2$. The rank $1$ derives from the fact that the created session involves
	three zero-ranked participants.
	%
	For the body of $\Sort$ we obtain the following derivation tree:
	\[
		\begin{prooftree}
			\[
				\[
					\smash\vdots
					\justifies
					\wtp[1]{
						x : T
					}{
						\pres{t}{
							\pinvk\Merge{x,\ep{t}\rmaster} \parop
							\cdots
						}
					}
					\using\refrule{tm-par}
				\]
				\[
					\[
						\justifies
						\wtp[0]{
							x : \End[\Out]
						}{
							\pclose{x}
						}
						\using\refrule{tm-close}
					\]
					\justifies
					\wtp[0]{
						x : T
					}{
						\act{x}\rmaster\oact\tres\dots
					}
					\using\refrule{tm-tag}
				\]
				\justifies
				\wtp[0]{
					x : T
				}{
					\pres{t}{
						\pinvk\Merge{x,\ep{t}\rmaster} \parop
						\cdots
					}
					\pchoice
					\act{x}\rmaster\oact\tres\dots
				}
				\using\refrule{tm-choice}
			\]
			\justifies
			\wtp[0]{
				x : U
			}{
				\act{x}\rmaster\iact\treq.(
					\pres{t}{
						\pinvk\Merge{x,\ep{t}\rmaster} \parop
						\cdots
					}
					\pchoice
					\act{x}\rmaster\oact\tres\dots
				)
			}
			\using\refrule{tm-tag}
		\end{prooftree}
	\]
	
	In the application of the rule \refrule{tm-choice}, the rank of the whole
	choice coincides with that of the branch in which no new sessions are
	created. This way we account for the fact that, even though $\Sort$
	\emph{may} create a new session, it does not \emph{have to} do so in order
	to terminate.
	%
	\eoe
\end{example}