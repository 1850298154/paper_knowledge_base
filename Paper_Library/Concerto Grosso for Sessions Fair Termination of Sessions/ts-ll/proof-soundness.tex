\beginbass
%
The last auxiliary result for proving \Cref{lem:weak_termination_ll} is to show
that $P$ weakly simulates $\Resolve{P}$, namely that every reduction of a
process $\Resolve{P}$ can be mimicked by one or more reductions of $P$.

\begin{lemma}
    \label{lem:resolved_red}
    If $\piwtp\Ctx{P}$ and $\Resolve{P} \red Q$ then $P \wred R$ for some $R$
    such that $Q = \Resolve{R}$.
\end{lemma}
\begin{proof}
    From \Cref{lem:depth} we deduce $\depth{P} \in \Nat$. We proceed by double
    induction on $\depth{P}$ and on the derivation of $\Resolve{P} \red Q$ and
    by cases on the last rule applied in the derivation of $\qtp\Ctx{P}$.
    Most cases are straightforward since the structure of $\Resolve{P}$ and that
    of $P$ only differ for non-deterministic choices, so we only discuss the
    case \refrule{\PiChoiceRule} in which $P = \Choice{P_1}{P_2}$. Then
    $\qtp\Ctx{\fc{P_1}{P_2}}$ and $\Resolve{P} = \Resolve{\fc{P_1}{P_2}}
    \red Q$. Recall that $\depth{\fc{P_1}{P_2}} < \depth{P} \in \Nat$. Using the
    induction hypothesis we deduce $\fc{P_1}{P_2} \wred R$ for some $R$ such
    that $Q = \Resolve{R}$. We conclude by observing that $P \red \fc{P_1}{P_2}
    \wred R$.
\end{proof}

\Cref{lem:resolved_red} can be easily generalized to arbitrary sequences of
reductions.

\begin{lemma}
    \label{lem:resolved_reds}
    If $\piwtp\Ctx{P}$ and $\Resolve{P} \wred Q$ then $P \wred R$ for some $R$
    such that $Q = \Resolve{R}$.
\end{lemma}
\begin{proof}
    A simple induction on the number of reductions in $\Resolve{P} \wred Q$
    using \Cref{lem:resolved_red}.
\end{proof}

\begin{proof}[Proof of \Cref{lem:weak_termination_ll}]
    Using \Cref{lem:resolved_well_typed} we deduce $\piwtp{x:\One}{\Resolve{P}}$.
    Using \Cref{lem:zero_rank_weak_termination} we deduce $\Resolve{P} \wred
    \Close\x$. Using \Cref{lem:resolved_reds} and
    \Cref{lem:quasi_subj_red_ll} we conclude $P \wred \Close\x$.
\end{proof}

Finally, we can prove \Cref{thm:ts_ll_sound,cor:fair_termination_ll}.

\begin{proof}[Proof of \Cref{thm:ts_ll_sound}]
    By induction on the number of reductions in $P \wred Q$, from the hypothesis
    $\piwtp{x : \One}{P}$ and \Cref{thm:subj_red_ll} we deduce $\piwtp{x :
    \One}{Q}$. We conclude using \Cref{lem:weak_termination_ll}.
\end{proof}

\begin{proof}[Proof of \Cref{cor:fair_termination_ll}]
		\brk
    Immediate consequence of \Cref{thm:ts_ll_sound,thm:fair_termination}.
\end{proof}
