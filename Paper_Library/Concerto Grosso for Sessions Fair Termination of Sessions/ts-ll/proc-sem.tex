\beginbass
%
\begin{figure}[t]
		\framebox[\textwidth]{
    \begin{math}
        \displaystyle
        \begin{array}{@{}lr@{~}c@{~}ll@{}}
            \defrule{sp-link} &
            \Link\x\y & \pcong & \Link\y\x
            \\
            \defrule{sp-comm} &
            \Cut\x{P}{Q} & \pcong & \Cut\x{Q}{P}
            \\
            \defrule{sp-assoc} &
            \Cut\x{P}{\Cut\y{Q}{R}} & \pcong & \Cut\y{\Cut\x{P}{Q}}{R} &
            \text{if $x\in\fn{Q}$,}
            \\
            & & & & \text{$y\not\in\fn{P}$, $x\not\in\fn{R}$}
        \end{array}
    \end{math}
    }
    \caption{Structural pre-congruence of \piLIN}
    \label{fig:pcong_ll}
\end{figure}
%
\begin{figure}[t]
    \framebox[\textwidth]{
    \begin{mathpar}
        \displaystyle
        \inferrule[rp-link]{\mathstrut}{\Cut\x{\Link\x\y}{P} \red P\subst\y\x}
        \defrule[rp-link]{}
        \and
        \inferrule[rp-unit]{\mathstrut}{\Cut\x{\PiClose\x}{\PiWait\x.P} \red P}
        \defrule[rp-unit]{}
        \and
        \inferrule[rp-pair]
        	{\mathstrut}
        	{\Cut\x{\Fork[y]\x\z{P_1}{P_2}}{\Join[y]\x\z.Q} \red \Cut\z{P_1}{\Cut\y{P_2}{Q}}}
        	\defrule[rp-pair]{}
        \and
        \inferrule[rp-sum]
        	{\mathstrut}
        	{\Cut\x{\Select[y]{\InTag_i}\x.P}{\Case[y]\x{P_1}{P_2}} \red \Cut\y{P}{P_i}}
          ~ i\in\set{1,2}
          \defrule[rp-sum]{}
        \and
        \inferrule[rp-rec]
        	{\mathstrut}
        	{\Cut\x{\Rec[y]\x.P}{\Corec[y]\x.Q} \red \Cut\y{P}{Q}}
        	\defrule[rp-rec]{}
        \and
        \inferrule[rp-choice]
        	{\mathstrut}
        	{\PiChoice{P_1}{P_2} \red P_i}
        	~ i\in\set{1,2}
        	\defrule[rp-choice]{}
				\and
				\inferrule[rp-cut]
					{P \red Q}
					{\Cut\x{P}{R} \red \Cut\x{Q}{R}}
					\defrule[rp-cut]{}
        \and
        \inferrule[rp-struct]
        	{P \pcong P' \\ P' \red Q' \\ Q' \pcong Q}
        	{P \red Q}
        	\defrule[rp-struct]{}
    \end{mathpar}
    }
    \caption{Reduction of \piLIN}
    \label{fig:red_ll}
\end{figure}
%
The operational semantics of the calculus is given in terms of the
\emph{structural precongruence} relation $\pcong$ and the \emph{reduction
relation} $\red$ defined in \Cref{fig:pcong_ll,fig:red_ll}.
%
As usual, structural precongruence relates processes that are syntactically
different but semantically equivalent. In particular, \refrule{sp-link} states
that linking $x$ with $y$ is the same as linking $y$ with $x$, whereas
\refrule{sp-comm} and \refrule{sp-assoc} state the expected  commutativity and
associativity laws for parallel composition. Concerning the latter, the side
condition $x\in\fn{Q}$ makes sure that $Q$ (the process brought closer to $P$
when the relation is read from left to right) is indeed connected with $P$ by
means of the channel $x$.
%
Note that \refrule{sp-assoc} only states the right-to-left associativity of
parallel composition and that the left-to-right associativity law
$\Cut\x{\Cut\y{P}{Q}}{R} \pcong \Cut\y{P}{\Cut\x{Q}{R}}$ is derivable when
$x\in\fn{Q}$.
%
The reduction relation is mostly unremarkable. Links are reduced with
\refrule{rp-link} by effectively merging the linked channels. All the reductions
that involve the interaction between processes except \refrule{rp-unit} create
new continuations channels that connect the reducts. The rule \refrule{rp-choice}
models the non-deterministic choice between two behaviors. Finally,
\refrule{rp-cut} and \refrule{rp-struct} close reductions by cuts and structural
precongruence.
%
In the following we write $\wred$ for the reflexive, transitive closure of
$\red$ and we say that $P$ is \emph{stuck} if there is no $Q$ such that $P \red
Q$.