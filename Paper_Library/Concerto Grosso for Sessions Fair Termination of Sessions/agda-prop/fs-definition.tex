\beginbass
%
As we noted at the beginning, the aim of this chapter is to show how to obtain
a refined, liveness enforcing, property by adding corules to the inference systems
that alone characterize well known safety properties.
Hence, for what concerns subtyping, we refer to the \GIS that we presented in \Cref{ssec:fsub_gis}.
%
However, we have to take into account that the compliance relation the we illustrated in
\Cref{sec:agda_fc} is \emph{asymmetric}. Thus, we first recall the semantic
definitions of (un)fair subtyping and then we show the asymmetric variant of
the \GIS in \Cref{fig:fsub_gis}. 

\begin{definition}[Subtyping]
  \label{def:sub}
  We say that $S$ is a \emph{subtype} of $T$ if $R$ compliant with
  $S$ implies $R$ compliant with $T$ for every $R$.
\end{definition}

\begin{definition}[Fair Subtyping]
  \label{def:fsub}
  We say that $S$ is a \emph{fair subtype} of $T$ if $R$ fairly compliant with
  $S$ implies $R$ fairly compliant with $T$ for every $R$.
\end{definition}


\begin{figure}[t]
  \framebox[\textwidth]{
    \begin{mathpar}
      \inferrule[s-nil]{\mathstrut}{
        \TNil\subt\T
      } \defrule[s-nil]{}
      \and
      \inferrule[s-end]{\mathstrut}{
        {\End[\Pol]}\subt\T
      }
      ~T \neq \TNil \defrule[s-end]{}
      \\
      \inferrule[s-inp]{
        \forall x \in X : {S_x}\subt{T_x}
      }{
        {\In\set{x:S_x}_{x\in X}}\subt{\In\set{x:T_x}_{x\in X \cup Y}}
      } \defrule[s-inp]{}
      \and
      \inferrule[s-out]{
        \forall x \in X : {S_x}\subt{T_x}
      }{
        {\Out\set{x:S_x}_{x\in X \cup Y}}\subt{\Out\set{x:T_x}_{x\in X}}
      } \defrule[s-out]{}
      \\
      \infercorule{
        \forall\actionsA\in\traces\S\setminus\traces\T:
        \exists\actionsB \leq \actionsA, x \in \Message:
        {S(\actionsB\Out\x)}\subt{T(\actionsB\Out\x)}
      }{
        \S\subt\T
      }
    \end{mathpar}
  }
  \caption{Generalized inference system $\gis{\is[F]}{\cois[F]}$ for fair subtyping}
  \label{fig:subt}
\end{figure}

The \GIS in \Cref{fig:subt} corresponds to that in \Cref{fig:fsub_gis} where the 
two axioms \refrule{s-nil} and \refrule{s-end} model the asymmetry.
The corule is defined exactly as \refrule{fs-converge} in \Cref{fig:fsub_gis}.
The explanation of the \emph{convergence} is given in \Cref{ssec:fsub_gis}.

\begin{example}
  \label{ex:subt}
  Consider once again the session types $T_i$ and $S_i$ from
  \Cref{ex:agda_st_ex}.
  %
  The (infinite) derivation
  \[
    \begin{prooftree}
      \[
        \[
          \vdots
          \justifies
          {T_1}\subt{S_1}
        \]
        \justifies
        {\Out\Nat.T_1}\subt{\Out\NatPlus.S_1}
        \using\refrule{s-out}
      \]
      \quad
      \[
        \justifies
        \End[\In]\subt\End[\In]
        \using\refrule{s-end}
      \]
      \justifies
      {T_1}\subt{S_1}
      \using\refrule{s-out}
    \end{prooftree}
  \]
  proves that ${T_1}\subt{S_1} \in \CoInductive{\is[F]}$ and the (infinite) derivation
  \[
    \begin{prooftree}
      \[
        \[
          \vdots
          \justifies
          {T_2}\subt{S_2}
        \]
        \justifies
        {\Out\Nat.T_2}\subt{\Out\NatPlus.S_2}
        \using\refrule{s-out}
      \]
      \quad
      \[
        \justifies
        \End[\In]\subt\End[\In]
        \using\refrule{s-end}
      \]
      \justifies
      {T_2}\subt{S_2}
      \using\refrule{s-inp}
    \end{prooftree}
  \]
  proves that ${T_2}\subt{S_2} \in \CoInductive{\is[F]}$.
  %
  In order to derive ${\T_i}\subt{\S_i}$ in the \GIS $\gis{\is[F]}{\cois[F]}$ we
  must find a well-founded proof tree in $\is[F] \cup \cois[F]$
  and the only hope to do so is by means of \refrule{fs-converge}, since $T_i$
  and $S_i$ share traces of arbitrary length.
  %
  Observe that every trace $\actionsA$ of $T_1$ that is not a trace
  of $S_1$ has the form $(\Out\vtrue\Out{p_k})^k\Out\vtrue\Out0\dots$
  where $p_k \in \NatPlus$. Thus, it suffices to take
  $\actionsB = \es$ and $x = 0$, noted that
  $T_1(\Out0) = S_1(\Out0) = \End$, to derive
  \[
    \begin{prooftree}
      \[
        \justifies
        \End[\In]\subt\End[\In]
        \using\refrule{fs-converge}
      \]
      \justifies
      {T_1}\subt{S_1}
      \using\refrule{fs-converge}
    \end{prooftree}
  \]
  %
  On the other hand, traces $\actionsA \in \traces{T_2} \setminus
  \traces{S_2} = (\In\vtrue\Out{p_k})^k\In\vtrue\Out0\dots$ where
  $p_k \in \NatPlus$.  All the prefixes of such traces that are followed by an
  output and are shared by both $T_2$ and $S_2$ have the form
  $(\In\vtrue\Out{p_k})^k\In\vtrue$ where $p_k \in \NatPlus$, and
  $T_2(\actionsB\Out{p}) = T_2$ and $S_2(\actionsB\Out{p}) = S_2$ for all such
  prefixes and $p \in \NatPlus$. It follows that we are unable to derive
  ${T_2}\subt{S_2}$ with a well-founded proof tree in $\is[F] \cup \cois[F]$.
  This is consistent with the fact that, in \Cref{ex:fcomp}, we have
  found a client $R_2$ that is fairly compliant with $T_2$ but not with $S_2$.
  Intuitively, $R_2$ insists on poking the server waiting to receive $0$. This
  may happen with $T_2$, but not with $S_2$.
  %
  In the case of $T_1$ and $S_1$ no such client can exist, since the server may
  decide to interrupt the interaction at any time by sending a $\vfalse$ message
  to the client.
  %
  \eoe
\end{example}

\begin{remark}
  Part of the reason why rule \refrule{fs-converge} is so contrived and hard to
  understand is that the property it enforces is fundamentally \emph{non-local}
  and therefore difficult to express in terms of immediate subtrees of a session
  type as we saw for the purely coinductive formulation of fair subtyping (see \Cref{sec:fair_sub}).
  %  
  To better illustrate the point, consider the following alternative set of
  corules meant to replace \refrule{fs-converge}:
  
  \begin{mathpar}
    \infercorule[co-inc]{
      \mathstrut
    }{
      \S\subt\T
    } \defrule[co-inc]{}
    ~ \traces\S \subseteq \traces\T
    \and
    \infercorule[co-inp]{
      \forall x \in \Message {S_x}\subt{T_x}
    }{
      {\In\set{x:S_x}_{x\in\Message}}\subt{\In\set{x:T_x}_{x\in\Message}}
    } \defrule[co-inp]{}
    \and
    \infercorule[co-out]{
      {S}\subt{T}
    }{
      {\Out x.S + S'}\subt{\Out x.T + T'}
    } \defrule[co-out]{}
  \end{mathpar}

  It is easy to see that these rules provide a sound approximation of
  \refrule{s-converge}, but they are not complete. Indeed, consider the session
  types $S = \In\vtrue.S \branch \In\vfalse.(\Out\vtrue.\End[\In] +
  \Out\vfalse.\End[\In])$ and $T = \In\vtrue.T \branch \In\vfalse.\Out\vtrue.\End[\In]$.
  %
  We have $\S\subt\T$ and yet $\S\isubt\T$ cannot be proved with the above corules: 
  it is not possible to prove $\S\isubt\T$ using \refrule{co-inc} because $\traces\S \not\subseteq \traces\T$. 
  If, on the other hand, we insist on visiting both branches of the topmost input as required by \refrule{co-inp}, 
  we end up requiring a proof of $\S\isubt\T$ in order to derive $\S\isubt\T$.
  %
  \eor
\end{remark}

\begin{theorem}
  \label{thm:sub}
  For every $S,T \in \SessionType$ the following properties hold:
  \begin{enumerate}
  \item $S$ is a subtype of $T$ if and only if
    $\S\subt\T \in \CoInductive{\is[F]}$;
  \item $S$ is a fair subtype of $T$ if and only if
    $\S\subt\T \in \FlexCo{\is[F]}{\cois[F]}$.
  \end{enumerate}
\end{theorem}
\begin{proof}[Proof sketch]
  As usual we focus on item~(2), which is the most interesting property.
	We have already presented the correctness proofs for the purely coinductive
	formulation of fair subtyping in \Cref{ssec:fsub_sound,ssec:fsub_complete}. 
	So we just sketch the proofs for the \GIS one.
  %
  For the ``if'' part, we consider an arbitrary $R$ that fairly complies with
  $S$ and show that it fairly complies with $T$ as well. More specifically, we
  consider a reduction $\session{R}{T} \wred \session{R'}{T'}$ and show that it
  can be extended so as to achieve client satisfaction.
  %
  The first step is to ``unzip'' this reduction into $R \wlred{\co\actions} R'$
  and $T \wlred\actions T'$ for some string $\actions$ of actions. Then, we show
  by induction on $\actions$ that there exists $S'$ such that ${S'}\subt{T'} \in
  \FlexCo{\is[F]}{\cois[F]}$ and $S \wlred\actions S'$, using the hypothesis
  $\S\subt\T \in \CoInductive{\is[F]}$ and the hypothesis that $R$ complies
  with $S$. This means that $R$ and $S$ may synchronize just like $R$ and $T$,
  obtaining a reduction $\session{R}{S} \wred \session{R'}{S'}$. At this point
  the existence of the reduction $\session{R'}{T'} \wred \session\Win{T''}$ is
  proved using the arguments in the discussion of rule \refrule{fs-converge}
  given earlier.

  For the ``only if'' part we use once again the bounded coinduction principle.
  In particular, we use the hypothesis that $S$ is a fair subtype of $T$ to show
  that $S$ and $T$ must have one of the forms in the conclusions of the rules in
  \Cref{fig:subt}. This proof is done by cases on the shape of $S$,
  constructing a canonical client of $S$ that must succeed with $T$ as well.
  %
  Then, the coinduction principle allows us to conclude that $\S\subt\T \in
  \CoInductive{\is[F]}$.
  %
  The fact that $\S\subt\T \in \Inductive{\is[F] \cup \cois[F]}$ also holds is
  by far the most intricate step of the proof. First of all, we establish that
  $\Inductive{\is[F] \cup \cois[F]} = \Inductive{\cois[F]}$. That is, we
  establish that rule \refrule{fs-converge} subsumes all the rules in $\is[F]$
  when they are inductively interpreted. Then, we provide a characterization of
  the \emph{negation} of \refrule{fs-converge}, which we call divergence. At this
  point we proceed by contradiction: under the hypothesis that $\S\subt\T \in
  \CoInductive{\is[F]}$ and that $S$ and $T$ ``diverge'', we are able to
  corecursively define a \emph{discriminating} (see \Cref{def:discriminator}) client $R$ that fairly complies
  with $S$ but not with $T$. This contradicts the hypothesis that $S$ is a fair
  subtype of $T$ and proves, albeit in a non-constructive way, that $\S\subt\T
  \in \Inductive{\cois[F]}$ as requested.
\end{proof}

\begin{remark}
  \label{rem:duality}
  Most session type theories adopt a symmetric form of session type
  compatibility whereby client and server are required to terminate
  the interaction at the same time.
  %
  It is easy to define a notion of \emph{symmetric compliance} 
  by turning $T' \ne \TNil$ into $T' = \End[\In]$ in \Cref{def:comp}. 
  The subtyping relation induced
  by symmetric compliance has essentially the same characterization of
  \Cref{def:sub}, except that the axiom \refrule{s-end} is replaced by
  the more familiar \refrule{fs-end} \citep{GayHole05}.
  %
  On the other hand, the analogous change in \Cref{def:fcomp} has much
  deeper consequences: the requirement that client and server must end the
  interaction at the same time induces a large family of session types that are
  syntactically very different, but semantically equivalent. For example, the
  session types $S$ and $T$ such that $S = \In\Nat.S$ and $T = \Out\Bool.T$,
  which describe completely unrelated protocols, would be equivalent for the
  simple reason that no client successfully interacts with them (they are not
  fairly terminating, since they do not contain any occurrence of $\End[]$).
  %
  \eor
\end{remark}