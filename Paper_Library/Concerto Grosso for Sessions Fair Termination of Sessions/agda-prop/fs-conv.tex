\beginbass
%
We dedicate this last part of the section answering a couple of questions
about the corule \refrule{fs-converge} and its mechanization.
%
First, the reader might be confused about whether \refrule{fs-converge} is
actually a (meta)rule for it contains an existential quantifier.
Hence, \refrule{fs-converge} is not a proper metarule written in that form.
Indeed, in the premise both a universal and an existential quantifier are mixed.
The issue can be solved by turning such premise in \emph{Skolem normal form}, that is,
without existential quantifiers.
%
The drawback of this approach is that quantifiers are replaced by functions making
the defined predicate hard to understand.
Although for the sake of clarity we could present \refrule{fs-converge}
in a \emph{wrong format}, in the Agda mechanization we should have followed the
Skolemization approach because the
\lstinline{MetaRule} datatype requires a precise formulation of all the components. 
For the sake of simplicity, we decided to opt for the more convenient approach that
we presented in \Cref{ssec:agda_fs_formalization} and that we comment here.

The second question that we try to answer is whether the approach that we adopted in 
\Cref{ssec:agda_fs_formalization} for defining \refrule{fs-converge}, that is by using
a coaxiom with a side condition, is equivalent to defining the actual corule.
Notably, the side condition \lstinline{S $\downarrow$ T} is obtained by
considering \refrule{fs-converge} as a stand-alone predicate. 
Let us show such rule forgetting about subtyping.
We use the notation with message tags that we adopted in \Cref{sec:st}.

\[
  \inferrule{
    \forall\actionsA\in\traces\S\setminus\traces\T:
    \exists\actionsB\prefix\actionsA, \l:
    S(\actionsB\Out\l) \converge T(\actionsB\Out\l)
  }{
    S \converge T
  }
\]

The Agda predicate \lstinline{_$\downarrow$_} is defined analogously.
Now we can prove that the predicate obtained by inductively interpreting such rule
is equivalent to the inductive interpretation of the whole inference system in \Cref{fig:subt}.

\begin{lemma}
	 $S \isubt T$ if and only if $S \converge T$.
\end{lemma}
\begin{proof}
  The ``if'' part is trivial since the sole rule defining
  $\converge$ is the same as \refrule{fs-converge}.  We prove the
  ``only if'' part by induction on the derivation of $S \isubt T$
  and by cases on the last rule applied.
	
	\proofcase{Case \refrule{s-nil}}
	%
	Then $S = \TNil$ and $T \ne \TNil$. We conclude $S \converge T$
  observing that $\emptyset \setminus \traces\T = \emptyset$.
  
  \proofcase{Case \refrule{s-end}}
  %
  Then $S = \End[\Pol]$ for some $\Pol$ and $T \ne \TNil$. 
  Notably, $\traces{\End[\Pol]} = \emptyset$ by \Cref{def:traces}.
  Hence, we conclude $S \converge T$
  observing that $\emptyset \setminus \traces\T = \emptyset$.
 
  \proofcase{Case \refrule{s-inp}}
  %
  Then $S = \Branch{\l_i : S_i}_{i\in I}$ and
  $T = \Branch{\l_j : T_j}_{j\in J}$ and $I \subseteq J$ and
  $S_i \isubt T_i$ for every $i \in I$.
  %
  Using the induction hypothesis we deduce that $S_i \converge T_i$
  for every $i\in I$.
  %
  We conclude $S \converge T$ observing that
  $\actionsA \in \traces\S \setminus \traces\T$ implies
  $\actionsA = \In\l_k\actionsB$ for some $k\in I$ and
  $\actionsB \in \traces{S_k} \setminus \traces{T_k}$.

  \proofcase{Case \refrule{s-out}}
  %
  Then $S = \Choice{\l_i : S_i}_{i\in I}$ and
  $T = \Choice{\l_j : T_j}_{j\in J}$ and $J \subseteq I$ and
  $S_j \isubt T_j$ for every $j\in J$.
  %
  Using the induction hypothesis we deduce that $S_j \converge T_j$
  for every $j\in J$.
  %
  In order to conclude $S \converge T$ we have to show that, for
  every $\actionsA \in \traces\S \setminus \traces\T$, we are able to
  find $\actionsB \prefix \actionsA$ and $\l \in \Message$ such that
  $S(\actionsB\Out\l) \converge T(\actionsB\Out\l)$.
  %
  We distinguish two sub-cases:
  \begin{itemize}
  \item Sub-case $\actions = \Out\l_j\actions'$ where
    $\actions' \in \traces{S_j} \setminus \traces{T_j}$ for some
    $j\in J$.
    %
    From $S_j \converge T_j$ we deduce that there exist
    $\actionsB' \prefix \actionsA'$ and $\l \in \Message$ such that
    $S_j(\actionsB'\Out\l) \converge T_j(\actionsB'\Out\l)$.
    %
    We conclude by taking $\actionsB \eqdef \Out\l_j\actionsB'$
    observing that $S(\actionsB\Out\l) = S_j(\actionsB'\Out\l)$ and
    $T(\actionsB\Out\l) = T_j(\actionsB'\Out\l)$.
  \item Sub-case $\actions = \Out\l_i\actions'$ where
    $\actions' \in \traces{S_i}$ for some $i\in I \setminus J$.
    %
    We conclude by taking $\actionsB \eqdef \es$ and $\l \eqdef \l_j$
    for some $j\in J$ and observing that $S(\actionsB\Out\l) = S_j$
    and $T(\actionsB\Out\l) = T_j$.
  \end{itemize}

  \proofcase{Case \refrule{fs-converge}}
  %
  Then, for every $\actionsA \in \traces\S \setminus \traces\T$, there
  exist $\actionsB \prefix \actions$ and $\l\in\Message$ such that
  $S(\actionsB\Out\l) \isubt T(\actionsB\Out\l)$.
  %
  We conclude immediately using the induction hypothesis to deduce
  that for each such $\actionsB$ and $\l$ we have
  $S(\actionsB\Out\l) \converge T(\actionsB\Out\l)$.
\end{proof}

In the Agda mechanization it happens that we take the inductive interpretation of the \GIS with
the corule replace by the coaxion.
Hence, in order to prove the equivalence between the two approaches, we need to prove
that the inductive interpretation of the \GIS in Agda is equivalent to the $\converge$ predicate.

\begin{lstlisting}
$\subt$F$_i$$\rightarrow\downarrow$ : $\forall${S T} $\rightarrow$ S $\subt$F$_i$ T $\rightarrow$ S $\downarrow$ T
\end{lstlisting}

We omit the detailed proof (it can be found in \cite{FairSubtypingAgda}). 
The other direction is trivial since we can directly apply the coaxiom providing the convergence proof.
