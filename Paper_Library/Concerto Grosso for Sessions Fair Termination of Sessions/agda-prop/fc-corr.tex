\beginalto
%
In this section we detail the mechanized correctness proof of \emph{fair compliance}
(see point 2. of \Cref{thm:compliance}).
As usual, correctness is expressed in terms of \emph{soundness} and \emph{completeness}
of \lstinline{_$\compliance{}{}$_} with respect to some specification.
Such specification is defined in \Cref{def:fcomp}. 
Before looking at such proofs, we formalize \Cref{def:fcomp}.
To formalize fair compliance, we define a \lstinline{Success} predicate that 
characterizes those configurations $\session{R}{S}$ in which the client has succeeded 
($R = \Win$) and the server has not failed ($S \neq \TNil$).

\begin{lstlisting}
data Success : Session $\rightarrow$ Set where
  success : $\forall${R S} $\rightarrow$ Win R $\rightarrow$ Defined S $\rightarrow$ Success (R , S)
\end{lstlisting}

We can weaken \lstinline{Success} to \lstinline{MaySucceed}, 
to characterize those configurations \emph{that can be extended} so as to become successful ones.
For this purpose we make use of the \lstinline{Satisfiable} predicate and of the 
intersection \lstinline{$\cap$} of two sets from Agda's standard library.

\begin{lstlisting}
MaySucceed : Session $\rightarrow$ Set
MaySucceed Se = 
	Relation.Unary.Satisfiable (Reductions Se $\cap$ Success)
\end{lstlisting}

In words, \lstinline{Se} may succeed if there exists \lstinline{Se'}
such that \lstinline{Se $\wred$ Se'} and \lstinline{Se'} is a successful configuration. 
We can now formulate fair compliance as the property of those sessions that may succeed 
no matter how they reduce (see \Cref{def:fcomp}). This is the specification against 
which we prove soundness and completeness of the \GIS for fair compliance (\Cref{fig:compliance}).

\begin{lstlisting}
FCompS : Session $\rightarrow$ Set
FCompS Se = $\forall${Se'} $\rightarrow$ Reductions Se Se' $\rightarrow$ MaySucceed Se'
\end{lstlisting}

\begin{remark}
	As we did in \Cref{sec:agda_ft}, assume that we want to instaciate \lstinline{Specification}
	from \Cref{sec:agda_fairt} by using \lstinline{Session} (see \Cref{ssec:agda_st_formalization_types})
	as set of states and \lstinline{Reduction} (see \Cref{ssec:agda_st_formalization_lts}) as transition system.
	Such instance of \lstinline{Specification} would not be equivalent to \lstinline{FCompS} as fari compliance
	corresponds to a successful form of fair termination.
	In order to make them equivalent we would need to parametrize \lstinline{WeaklyTerminating} on a predicate
	that the reducts have to satisfy. In this case we would say that a state is weakly terminating
	if it reduces to another one that is in normal form and that satisfies the input predicate.
	%
	\eor
\end{remark}

%%%%%%%%%%%%%%%%%
%%% SOUNDNESS %%%
%%%%%%%%%%%%%%%%%

\subsection{Soundness}
\beginbass
%
\GISs provide no canonical way for proving the soundness of the generalized 
interpretation of an inference system, so we have to handcraft the proof. 
We start by proving that the inductive interpretation of the inference system 
with the corules implies the existence of a reduction leading to a successful configuration.

\begin{lstlisting}
FCompI$\rightarrow$MS : $\forall${Se} $\rightarrow$ FCompI Se $\rightarrow$ MaySucceed Se
FCompI$\rightarrow$MS (fold (inj$_1$ success , (_ , succ) , refl , _)) = 
  _ , $\epsilon$ , succ
FCompI$\rightarrow$MS (fold (inj$_1$ out-inp , (_ , _ , fx) , refl , pr)) =
  let _ , reds , succ = FCompI$\rightarrow$MS (pr (_ , fx)) in
  _ , sync (out fx) inp $\triangleleft$ reds , succ
FCompI$\rightarrow$MS (fold (inj$_1$ inp-out , (_ , _ , gx) , refl , pr)) =
  let _ , reds , succ = FCompI$\rightarrow$MS (pr (_ , gx)) in
  _ , sync inp (out gx) $\triangleleft$ reds , succ
FCompI$\rightarrow$MS (fold (inj$_2$ out-inp , (_ , _ , fx) , refl , pr)) =
  let _ , reds , succ = FCompI$\rightarrow$MS (pr Data.Fin.zero) in
  _ , sync (out fx) inp $\triangleleft$ reds , succ
FCompI$\rightarrow$MS (fold (inj$_2$ inp-out , (_ , _ , gx) , refl , pr)) =
  let _ , reds , succ = FCompI$\rightarrow$MS (pr Data.Fin.zero) in
  _ , sync inp (out gx) $\triangleleft$ reds , succ 
\end{lstlisting}

where \lstinline{$\epsilon$} and \lstinline{red $\triangleleft$ reds} 
are the constructors of \lstinline{Star}. While the former represents 
the base case (when there are no reductions), the second
represents a chain of reductions starting with the single reduction 
\lstinline{red} followed by the reductions \lstinline{reds}.
There are two things worth noting here. First, in the union of the two inference 
systems each rule is identified by a name of the form \lstinline{inj$_1$ n} 
or \lstinline{inj$_2$ n} where \lstinline{n} is either the name of a rule 
or of a corule, respectively.
%
Also, we use the function \lstinline{pr} to access the premises of a 
(co)rule by their position. For the plain rules \lstinline{out-inp} 
and \lstinline{inp-out} the position is the witness \lstinline{fx}
or \lstinline{gx} that the value exchanged in the synchronization belongs to the 
domain of the continuation function of the sender. For the corules 
\lstinline{out-inp} and \lstinline{inp-out}, 
we use the position \lstinline{Data.Fin.Zero} 
to access the first and only premise in their list of premises.

The next auxiliary result establishes a ``subject reduction'' property for 
fair compliance: if $\compliance{R}{S}$ and $\session{R}{S} \wred \session{R'}{S'}$, 
then $\compliance{R'}{S'}$. Note that this property is trivial to prove when we 
consider the specification of fair compliance (see \Cref{def:fcomp}), 
but here we are referring to the predicate defined by the \GIS. 
The proof consists of a simple induction on the reduction $\session{R}{S} \wred \session{R'}{S'}$.

\begin{lstlisting}
sr : $\forall${Se Se'} $\rightarrow$ FCompG Se $\rightarrow$ Reductions Se Se' $\rightarrow$ FCompG Se'
sr fc $\epsilon$ = fc
sr fc (_ $\triangleleft$ _) with fc .CoInd$\llbracket$_$\rrbracket$.unfold
sr _ (sync (out fx) inp $\triangleleft$ _) | 
  success , ((_ , success (out e) _) , _) , refl , _ = 
  $\bot$-elim (e _ fx)
sr _ (sync inp (out gx) $\triangleleft$ reds) | inp-out , _ , refl , pr = 
  sr (pr (_ , gx)) reds
sr _ (sync (out fx) inp $\triangleleft$ reds) | out-inp , _ , refl , pr = 
  sr (pr (_ , fx)) reds
\end{lstlisting}

Note that we have an absurd case, in which a successful configuration apparently reduces,
which we rule out using false elimination (\lstinline{$\bot$-elim}).
%
The soundness proof is a simple combination of the above auxiliary results. We
use the library function \lstinline{fcoind-to-ind} (see \Cref{fig:fcoind-to-ind})
to extract an inductive derivation of \lstinline{FCompI Se} from a
derivation of \lstinline{FCompG Se} in the \GIS for fair
compliance.

\begin{lstlisting}
sound : $\forall${Se} $\rightarrow$ FCompG Se $\rightarrow$ FCompS Se
sound fc reds = FCompI$\rightarrow$MS (fcoind-to-ind (sr fc reds))
\end{lstlisting}

%%%%%%%%%%%%%%%%%%%%
%%% COMPLETENESS %%%
%%%%%%%%%%%%%%%%%%%%

\subsection{Completeness}
\beginbass
%
For the completeness result we appeal to the bounded coinduction principle of 
\GISs (\Cref{prop:bcp}, \Cref{fig:principles}), which requires us to prove boundedness and 
consistency of \lstinline{FCompS}. Concerning boundedness, 
we start by computing a proof of \lstinline{FCompI Se}
for every session \lstinline{Se} that may reduce a successful 
configuration, by induction on the reduction.

\begin{lstlisting}
MS$\rightarrow$FCompI : $\forall${Se} $\rightarrow$ MaySucceed Se $\rightarrow$ FCompI Se
MS$\rightarrow$FCompI (_ , reds , succ) = aux reds succ
  where
    aux : $\forall${Se Se'} $\rightarrow$ Reductions Se Se' $\rightarrow$ 
      	Success Se' $\rightarrow$ FCompI Se
    aux $\epsilon$ succ = apply-ind (inj$_1$ success) (_ , succ) $\lambda$ ()
    aux (sync (out fx) inp $\triangleleft$ red) succ =
      apply-ind (inj$_2$ out-inp) 
        (_ , _ , fx) $\lambda${Data.Fin.zero $\rightarrow$ aux red succ}
    aux (sync inp (out gx) $\triangleleft$ red) succ =
      apply-ind (inj$_2$ inp-out) 
        (_ , _ , gx) $\lambda${Data.Fin.zero $\rightarrow$ aux red succ}
\end{lstlisting}

where \lstinline{apply-ind} is the function in the \GIS library that applies 
a rule for an inductively defined predicate.
%
Then, boundedness follows by observing that $R$ fairly compliant with $S$ 
implies the existence of a successful configuration reachable from $\session{R}{S}$.

\begin{lstlisting}
bounded : $\forall${Se} $\rightarrow$ FCompS Se $\rightarrow$ FCompI Se
bounded fc = MS$\rightarrow$FCompI (fc $\epsilon$)
\end{lstlisting}

Showing that \lstinline{FCompS} is consistent means showing that every 
configuration \lstinline{Se} that satisfies \lstinline{FCompS} is 
found in the conclusion of a rule in the inference system \lstinline{FCompIS}
whose premises are all configurations that in turn satisfy \lstinline{FCompS}. 
This follows by a straightforward case analysis on the \emph{first} reduction 
of \lstinline{Se} that leads to a successful configuration.

\begin{lstlisting}
consistent : $\forall${Se} $\rightarrow$ FCompS Se $\rightarrow$ ISF[ FCompIS ] FCompS Se
consistent fc with fc $\epsilon$
... | _ , $\epsilon$ , succ = success , (_ , succ) , refl , $\lambda$ ()
... | _ , sync (out fx) inp $\triangleleft$ _ , _ =
out-inp , (_ , _ , fx) , refl , 
  $\lambda$ (_ , gx) reds $\rightarrow$ fc (sync (out gx) inp $\triangleleft$ reds)
... | _ , sync inp (out gx) $\triangleleft$ _ , _ =
inp-out , (_ , _ , gx) , refl , 
  $\lambda$ (_ , fx) reds $\rightarrow$ fc (sync inp (out fx) $\triangleleft$ reds)
\end{lstlisting}

We obtain the completeness proof using the bounded coinduction principle, 
\ie the library function
\lstinline{bounded-coind} (see \Cref{fig:principles})
which applies the principle to the boundedness and
consistency proofs.

\begin{lstlisting}
complete : $\forall${Se} $\rightarrow$ FCompS Se $\rightarrow$ FCompG Se
complete = 
  bounded-coind[ FCompIS , FCompCOIS ] FCompS bounded consistent
\end{lstlisting}

