\begintreble
%
In this chapter we take into account three interesting properties of \emph{binary} session types. 
All the properties under analysis mix safety and liveness. Hence, we use \GISs (\Cref{sec:gis}) as a reference framework
for their definitions and proofs.
Furthermore, for each property we first investigate its safety counterpart and we show how to obtain the desired
one by using corules.
Notably, all the results have been formalized in Agda \citep{CicconePadovani21,FairSubtypingAgda,CicconeP22@lmcs}.
We chose to restrict to the binary case to simplify the mechanizations as much as possible.

The properties that we study are the following:
\begin{itemize}
	\item \emph{Fair Termination} of a session type. A fairly terminating session can always eventually
	reach termination
	\item \emph{Fair Compliance} of two session types. Compliant sessions can always eventually interact to reach
	termination (asymmetric variant of \Cref{def:compatibility})
	\item \emph{Fair Subtyping} between session types. This property is a compliance-preserving subtyping relation
	(see \Cref{def:ssubt})
\end{itemize}

We begin the chapter by mechanizing in \Cref{sec:agda_fairt} the notions presented in \Cref{sec:ft_formally}
and that will be used later.
%
Then, in \Cref{sec:agda_st} we introduce an alternative definition of (binary) session
types and its formalization.
%
Then the chapter is split according to the property under analysis 
(\Cref{sec:agda_ft,sec:agda_fc,sec:agda_fs}).
Furthermore, for each property we show its Agda definition and we explain how
the proof principles (see \Cref{sec:gis}) can be used to prove the correctness. 
%
At last, in \Cref{sec:agda_fc_corr} we
detail the soundness and completeness proof of \emph{fair compliance}.
The formalization of all the results is available on GitHub \citep{FairSubtypingAgda}.

\begin{remark}
	\label{rm:agda_st}
	In this chapter we refer to \emph{binary session types} but we rely on a syntax 
	different from the more conventional one presented
	in \Cref{sec:st}. This choice is mainly motivated by the fact we aimed at
	simplifying the mechanization as much as possible. In \Cref{sec:agda_st} we describe all
	the advantages of using such representation.
\end{remark}