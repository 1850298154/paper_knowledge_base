\beginbass
%
To formalize fair termination we need the notion of \emph{trace}, which is a
finite sequence of actions performed on a session channel while preserving
usability of the channel.

\begin{definition}[(Maximal) traces]
  \label{def:traces}
  The \emph{traces} of a session $S$ are defined as $\traces\S \eqdef \set{ \actions \mid
  \exists\T : S \wlred\actions T \ne \TNil }$.  We say that $\actions \in
  \traces\S$ is \emph{maximal} if $\actions\actionsB \in \traces\S$ implies
  $\actionsB = \es$.
\end{definition}

\begin{remark}
	\Cref{def:traces} differs from the $\paths$ of a session type (\Cref{def:path})
	since we require that the involved session type does not reduce to $\TNil$.
	This difference is motivated by the different definition of session types.
	%
	\eor
\end{remark}

\begin{example}
	We have $\traces\TNil = \emptyset$ and $\traces\Win = \traces{\End[\In]} =
	\set\es$. Note that $\Win$ and $\End[\In]$ have the same traces but different
	transitions (hence different behaviors).
	%
	\eoe
\end{example}

A \emph{maximal trace} is a trace that cannot be extended any further. 

\begin{definition}[Fair Termination]
  \label{def:wt}
  We say that $S$ is \emph{fairly terminating} if, for every
  $\actionsA\in\traces\S$, there exists $\actionsB$ such that
  $\actionsA\actionsB \in \traces\S$ and $\actionsA\actionsB$ is maximal.
\end{definition}

\begin{example}
 	$\es$ is a maximal trace of both $\Win$ and $\End[\In]$ but not of
 	$\Out\Bool.\End[\In]$ whereas $\Out\vtrue$ and $\Out\vfalse$ are maximal traces of
 	$\Out\Bool.\End[\In]$.
	%
	\eoe
\end{example}

\begin{example}
  \label{ex:termination}
  All of the session types presented in \Cref{ex:agda_st_ex} are fairly
  terminating. The session type $R = \Out\Bool.R$, which describes a channel
  used for sending an infinite stream of boolean values, is not fairly
  terminating because no trace of $R$ can be extended to a maximal one. Note
  that also $R' \eqdef \Out\vtrue.R + \Out\vfalse.\Win$ is not fairly
  terminating, even though there is a path leading to $\Win$, because fair
  termination must be \emph{preserved} along all possible transitions of the
  session type, whereas $R' \xlred{\Out\vtrue} R$ and $R$ is not fairly
  terminating.
  %
  Finally, $\TNil$ is trivially fairly terminating because it has no trace.
  %
  \eoe
\end{example}

\begin{figure}[t]
	\framebox[\textwidth]{
 		\begin{mathpar}
      \inferrule[t-nil]{\mathstrut}{
        \terminates\TNil
      } \defrule[t-nil]{}
      \and
      \inferrule[t-all]{
        \forall x \in \Message : \terminates{T_x}
      }{
        \terminates{\Pol\set{x:T_x}_{x\in\Message}}
      } \defrule[t-all]{}
      \and
      \infercorule[t-any]{
        \terminates{S}
      }{
        \Pol x.S + T
      }
      ~
      S \ne \TNil \defrule[t-any]{}
    \end{mathpar}
  }
  \caption{Generalized inference system $\gis{\is[T]}{\cois[T]}$ for fair termination}
  \label{fig:ft}
\end{figure}

To find an inference system for fair termination observe that the set
$\FairlyTerminating$ of fairly terminating session types is the largest one that
satifies the following two properties:
\begin{enumerate}
  \item it must be possible to reach either $\Win$ or $\End[\In]$ from every $S \in
\FairlyTerminating \setminus \set\TNil$;
  \item the set $\FairlyTerminating$ must be closed by transitions, namely if $S
\in \FairlyTerminating$ and $S \lred\action T$ then $T \in \FairlyTerminating$.
\end{enumerate}
%
Neither of these two properties, taken in isolation, suffices to define
$\FairlyTerminating$: the session type $R'$ in \Cref{ex:termination}
enjoys property (1) but is not fairly terminating; the set $\SessionType$ is
obviously the largest one with property (2), but not every session type in it is
fairly terminating.
%
This suggests the definition of $\FairlyTerminating$ as the largest subset of
$\SessionType$ satisfying (2) and whose elements are \emph{bounded} by property
(1), which is precisely what corules allow us to specify.

\Cref{fig:ft} shows a \GIS $\gis{\is[T]}{\cois[T]}$ for fair termination with
the usual notation for single-lined rules and doubly-lined corules.
%
The axiom \refrule{t-nil} indicates that $\TNil$ is fairly terminating in a
trivial way (it has no trace), while \refrule{t-all} indicates that fair
termination is closed by all transitions. Note that these two rules, interpreted
coinductively, are satisfied by all session types, hence $\set{ \S \mid
\terminates\S \in \CoInductive{\is[S]} } = \SessionType$.

\begin{theorem}
  \label{thm:termination}
  $T$ is fairly terminating if and only if $\terminates\S \in
  \FlexCo{\is[S]}{\cois[S]}$.
\end{theorem}
\begin{proof}[Proof sketch]
  For the ``if'' part, suppose $\terminates\S \in \FlexCo{\is[S]}{\cois[S]}$ and
  consider a trace $\actions\in\traces\S$. That is, $S \wlred\actions T$ for
  some $T \ne \TNil$. Using \refrule{t-all} we deduce $\terminates\T \in
  \FlexCo{\is[S]}{\cois[S]}$ by means of a simple induction on $\actions$. Now
  $\terminates\T \in \FlexCo{\is[S]}{\cois[S]}$ implies $\terminates\T \in
  \Inductive{\is[S] \cup \cois[S]}$. Another
  induction on the (well-founded) derivation of this judgment, along with the
  witness message $x$ of \refrule{t-any}, allows us to find $\actionsB$ such
  that $\actions\actionsB$ is a maximal trace of $S$.

  For the ``only if'' part, we apply the bounded coinduction principle (see
  \Cref{prop:bcp}). Since we have already argued that the coinductive
  interpretation of the GIS in \Cref{fig:ft} includes all session types, it
  suffices to show that $S$ fairly terminating implies $\terminates\S \in
  \Inductive{\is[S] \cup \cois[S]}$.
  %
  From the assumption that $S$ is fairly terminating we deduce that there exists
  a maximal trace $\actions\in\traces{S}$. An induction on $\actions$ allows us
  to derive $\terminates{S}$ using repeated applications of \refrule{t-any}, one
  for each action in $\actions$, topped by a single application of
  \refrule{t-nil}.
\end{proof}