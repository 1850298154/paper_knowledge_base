\beginbass
%
\GISs provide no canonical way for proving the soundness of the generalized 
interpretation of an inference system, so we have to handcraft the proof. 
We start by proving that the inductive interpretation of the inference system 
with the corules implies the existence of a reduction leading to a successful configuration.

\begin{lstlisting}
FCompI$\rightarrow$MS : $\forall${Se} $\rightarrow$ FCompI Se $\rightarrow$ MaySucceed Se
FCompI$\rightarrow$MS (fold (inj$_1$ success , (_ , succ) , refl , _)) = 
  _ , $\epsilon$ , succ
FCompI$\rightarrow$MS (fold (inj$_1$ out-inp , (_ , _ , fx) , refl , pr)) =
  let _ , reds , succ = FCompI$\rightarrow$MS (pr (_ , fx)) in
  _ , sync (out fx) inp $\triangleleft$ reds , succ
FCompI$\rightarrow$MS (fold (inj$_1$ inp-out , (_ , _ , gx) , refl , pr)) =
  let _ , reds , succ = FCompI$\rightarrow$MS (pr (_ , gx)) in
  _ , sync inp (out gx) $\triangleleft$ reds , succ
FCompI$\rightarrow$MS (fold (inj$_2$ out-inp , (_ , _ , fx) , refl , pr)) =
  let _ , reds , succ = FCompI$\rightarrow$MS (pr Data.Fin.zero) in
  _ , sync (out fx) inp $\triangleleft$ reds , succ
FCompI$\rightarrow$MS (fold (inj$_2$ inp-out , (_ , _ , gx) , refl , pr)) =
  let _ , reds , succ = FCompI$\rightarrow$MS (pr Data.Fin.zero) in
  _ , sync inp (out gx) $\triangleleft$ reds , succ 
\end{lstlisting}

where \lstinline{$\epsilon$} and \lstinline{red $\triangleleft$ reds} 
are the constructors of \lstinline{Star}. While the former represents 
the base case (when there are no reductions), the second
represents a chain of reductions starting with the single reduction 
\lstinline{red} followed by the reductions \lstinline{reds}.
There are two things worth noting here. First, in the union of the two inference 
systems each rule is identified by a name of the form \lstinline{inj$_1$ n} 
or \lstinline{inj$_2$ n} where \lstinline{n} is either the name of a rule 
or of a corule, respectively.
%
Also, we use the function \lstinline{pr} to access the premises of a 
(co)rule by their position. For the plain rules \lstinline{out-inp} 
and \lstinline{inp-out} the position is the witness \lstinline{fx}
or \lstinline{gx} that the value exchanged in the synchronization belongs to the 
domain of the continuation function of the sender. For the corules 
\lstinline{out-inp} and \lstinline{inp-out}, 
we use the position \lstinline{Data.Fin.Zero} 
to access the first and only premise in their list of premises.

The next auxiliary result establishes a ``subject reduction'' property for 
fair compliance: if $\compliance{R}{S}$ and $\session{R}{S} \wred \session{R'}{S'}$, 
then $\compliance{R'}{S'}$. Note that this property is trivial to prove when we 
consider the specification of fair compliance (see \Cref{def:fcomp}), 
but here we are referring to the predicate defined by the \GIS. 
The proof consists of a simple induction on the reduction $\session{R}{S} \wred \session{R'}{S'}$.

\begin{lstlisting}
sr : $\forall${Se Se'} $\rightarrow$ FCompG Se $\rightarrow$ Reductions Se Se' $\rightarrow$ FCompG Se'
sr fc $\epsilon$ = fc
sr fc (_ $\triangleleft$ _) with fc .CoInd$\llbracket$_$\rrbracket$.unfold
sr _ (sync (out fx) inp $\triangleleft$ _) | 
  success , ((_ , success (out e) _) , _) , refl , _ = 
  $\bot$-elim (e _ fx)
sr _ (sync inp (out gx) $\triangleleft$ reds) | inp-out , _ , refl , pr = 
  sr (pr (_ , gx)) reds
sr _ (sync (out fx) inp $\triangleleft$ reds) | out-inp , _ , refl , pr = 
  sr (pr (_ , fx)) reds
\end{lstlisting}

Note that we have an absurd case, in which a successful configuration apparently reduces,
which we rule out using false elimination (\lstinline{$\bot$-elim}).
%
The soundness proof is a simple combination of the above auxiliary results. We
use the library function \lstinline{fcoind-to-ind} (see \Cref{fig:fcoind-to-ind})
to extract an inductive derivation of \lstinline{FCompI Se} from a
derivation of \lstinline{FCompG Se} in the \GIS for fair
compliance.

\begin{lstlisting}
sound : $\forall${Se} $\rightarrow$ FCompG Se $\rightarrow$ FCompS Se
sound fc reds = FCompI$\rightarrow$MS (fcoind-to-ind (sr fc reds))
\end{lstlisting}