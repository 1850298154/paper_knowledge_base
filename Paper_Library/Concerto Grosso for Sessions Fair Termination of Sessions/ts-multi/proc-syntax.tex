\beginbass
%
\begin{figure}[t]
	\framebox[\textwidth]{
  \begin{math}
    \begin{array}[t]{@{}rcll@{}}
      P, Q, R & ::= & & \textbf{Process} \\
      &   & \pdone & \text{termination} \\
      & | & \pwait\chvar{P} & \text{signal in} \\
      & | & \pich\chvar\role{x}{P} & \text{channel in} \\
      & | & \pbranch[i\in I]\chvar\role\Pol{\Tag_i}{P_i} & \text{tag in/out} \\
      & | & \pres\sn{P_1\ppar\cdots\ppar P_n} & \text{session} \\
    \end{array}
    ~
    \begin{array}[t]{@{}rcll@{}}
      \\
      & | & \pinvk\pdn{\seqof\chvar} & \text{invocation} \\
      & | & \pclose\chvar & \text{signal out} \\
      & | & \poch\chvar\role\achvar{P} & \text{channel out} \\
      & | & P \pchoice Q &\text{choice} \\
      & | & \pcast\chvar P & \text{cast} \\
    \end{array}
  \end{math}
  }
  \caption{Syntax of processes}
  \label{fig:proc_syntax_multi}
\end{figure}
%
A \emph{program} is a finite set of \emph{definitions} of the form
$\pdef\pdn{\seqof\var}{P}$, at most one for each process name, where $P$ is a
term generated by the syntax shown in \Cref{fig:proc_syntax_multi}.
%
The term $\pdone$ denotes the terminated process that performs no action.
%
The term $\pinvk\pdn{\seqof\chvar}$ denotes the invocation of the process with
name $\pdn$ passing the channels $\seqof\chvar$ as arguments. When
$\seqof\chvar$ is empty we just write $A$ instead of $\pinvk{A}{}$.
%
The term $\pclose\chvar$ denotes the process that sends a termination signal on
the channel $\chvar$, whereas $\pwait\chvar P$ denotes the process that waits
for a termination signal from channel $\chvar$ and then continues as $P$.
%
The term $\poch\chvar\role\achvar{P}$ denotes the process that sends the channel
$\achvar$ on the channel $\chvar$ to the role $\rolep$ and then continues as
$P$. Dually, $\pich\chvar\role\var{P}$ denotes the process that receives a
channel from the role $\rolep$ on the channel $\chvar$ and then continues as $P$
where $\var$ is replaced with the received channel.
%
The term $\pbranch[i\in I]\chvar\role\Pol{\Tag_i}{P_i}$ denotes a process that
exchanges one of the tags $\Tag_i$ on the channel $\chvar$ with the role $\role$
and then continues as $P_i$. Whether the tag is sent or received depends on the
polarity $\Pol$ and, as it will be clear from the operational semantics, the
polarity $\Pol$ also determines whether the process behaves as an internal
choice (when $\Pol$ is $\oact$) or an external choice (when $\Pol$ is $\iact$).
In the first case the process chooses \emph{actively} the tag being sent,
whereas in the second case the process reacts \emph{passively} to the tag being
received.
%
We assume that $I$ is finite and non-empty and also that the tags $\Tag_i$ are
pairwise distinct. For brevity, we write $\pbranch\chvar\role\Pol{\Tag_k}{P_k}$
instead of $\pbranch[i\in I]\chvar\role\Pol{\Tag_i}{P_i}$ when $I$ is the
singleton set $\set{k}$.
%
The term $P \pchoice Q$ denotes a process that non-deterministically behaves
either as $P$ or as $Q$.

A term $\pres{s}{P_1\ppar\cdots\ppar P_n}$ with $n\geq 1$ denotes the parallel
composition of $n$ processes, each of them being a participant of the session
$s$. Each process is associated with a distinct a role $\role_i$ and
communicates in $s$ through the endpoint $\ep{s}{\role_i}$. Combining session
creation and parallel composition in a single form is common in session type
systems based on linear
logic \citep{CairesPfenningToninho16,Wadler14,LindleyMorris16} and helps
guaranteeing deadlock freedom. 
%
Finally, a \emph{cast} $\pcast\chvar P$ denotes a process that behaves exactly
as $P$. This form is only relevant for the type system and
denotes the fact that the type of $\chvar$ is subject to an application of
subtyping.

The free and bound names of a process are defined as usual, the latter ones
being easily recognizable as they occur within round parenteses. We write
$\fn{P}$ for the set of free names of $P$ and we identify processes modulo
renaming of bound names. Note that $\fn{P}$ may contain variables and session
names, but not endpoints.
%
Occasionally we write $\pdef{A}{\seqof{x}}{P}$ as a predicate or side condition,
meaning that $P$ is the process associated with the process name $A$. For each
of such definitions we assume that $\fn{P} \subseteq \set{\seqof{x}}$.