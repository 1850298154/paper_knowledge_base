\beginalto
%
In this section we define the calculus for multiparty sessions on which we apply
our static analysis technique. The calculus is an extension of the one presented
by \cite{CicconePadovani22} to multiparty sessions in the
style of \cite{ScalasYoshida19} and that has been presented in \cite{CicconeDP22}.
%
We recall some basic notions.
We use an infinite set of \emph{variables} ranged over by $x$, $y$, $z$, an
infinite set of \emph{session names} ranged over by $s$ and $t$, a set of
\emph{roles} ranged over by $\rolep$, $\roleq$, $\roler$, a set of \emph{message
tags} ranged over by $\Tag$, and a set of \emph{process names} ranged over by
$A$, $B$, $C$. As in the binary case, the different terminology for labels 
is needed to avoid confusion with the labels in \Cref{sec:st}.
%
We use roles to distinguish the participants of a session. In particular, an
\emph{endpoint} $\ep\sn\role$ consists of a session name $\sn$ and a role
$\role$ and is used by the participant with role $\role$ to interact with the
other participants of the session $s$.
%
We use $u$ and $v$ to range over \emph{channels}, which are either variables or
session endpoints.
%
We write $\seqof x$ and $\seqof u$ to denote possibly empty sequences of
variables and channels, extending this notation to other entities.
%
We use $\Pol$ to range over the elements of the set $\set{\iact,\oact}$ of
\emph{polarities}, distinguishing input actions ($\iact$) from output actions
($\oact$).

%%%%%%%%%%%%%%
%%% SYNTAX %%%
%%%%%%%%%%%%%%

\subsection{Syntax of Processes}
\beginbass
%
\begin{figure}[t]
	\framebox[\textwidth]{
    \begin{math}
        \displaystyle
        \begin{array}[t]{@{}r@{~}c@{~}ll@{}}
            P, Q
            & ::= & \Link\x\y & \text{link}
            \\
            & | & \Fail\x & \text{empty in}
            \\
            & | & \PiWait\x.P & \text{unit in}
            \\
            & | & \Join[y]\x\z.P & \text{pair in}
            \\
            & | & \Case[y]\x{P}{Q} & \text{sum in}
            \\
            & | & \Corec[y]\x.P & \text{corec}
        \end{array}
        ~
        \begin{array}[t]{@{}r@{~}c@{~}lll@{}}
            & | & \Cut\x{P}{Q} & \text{comp}
            \\
            & | & \PiChoice{P}{Q} & \text{choice}
            \\
            & | & \PiClose\x & \text{unit out}
            \\
            & | & \Fork[y]\x\z{P}{Q} & \text{pair out}
            \\
            & | & \Select[y]{\InTag_i}\x.P & \text{sum out} & i\in\set{1,2}
            \\
            & | & \Rec[y]\x.P & \text{rec}
        \end{array}
    \end{math}
  	}
    \caption{Syntax of \piLIN}
    \label{fig:proc_syntax_ll}
\end{figure}
%
We assume an infinite set of \emph{channels} ranged over by $x$, $y$ and
$z$. \piLIN processes are coinductively generated by the productions of the
grammar shown in \Cref{fig:proc_syntax_ll}. %and their informal meaning is given below.
A \emph{link} $\Link\x\y$ acts as a \emph{linear
forwarder} \cite{GardnerLaneveWischik07} that forwards a single message either
from $x$ to $y$ or from $y$ to $x$. The uncertainty in the direction of the
message is resolved once the term is typed and the polarity of the types of $x$
and $y$ is fixed (as we will show in \Cref{sec:ts_ll_ts}).
%
The term $\Fail\x$ represents a process that receives an empty message from $x$
and then fails. This form is only useful in the metatheory: the type system
guarantees that well-typed processes never fail, since it is not possible to
send empty messages.
%
The term $\PiClose\x$ models a process that sends the unit on $x$, effectively
indicating that the interaction is terminated, whereas $\PiWait\x.P$ models a
process that receives the unit from $x$ and then continues as $P$.
%
The term $\Fork[z]\x\y{P}{Q}$ models a process that creates two new channels $y$
and $z$, sends them in a pair on channel $x$ and then forks into two parallel
processes $P$ and $Q$. Dually, $\Join[z]\x\y.P$ models a process that receives a
pair containing two channels $y$ and $z$ from channel $x$ and then continues as
$P$.
%
The term $\Select[y]{\InTag_i}\x.P$ models a process that creates a new channel
$y$ and sends $\InTag_i\parens\y$ (that is, the $i$-th injection of $y$ in a
disjoint sum) on~$x$. Dually, $\Case[y]\x{P_1}{P_2}$ receives a disjoint sum
from channel $x$ and continues as either $P_1$ or $P_2$ depending on the tag
$\InTag_i$ it has been built with. For clarity, in some examples we will use
more descriptive labels such as $\AddTag$ and $\PayTag$ instead of $\LeftTag$
and $\RightTag$.
%
The terms $\Rec[y]\x.P$ and $\Corec[y]\x.P$ model processes that respectively
send and receive a new channel $y$ and then continue as $P$. They do not
contribute operationally to the interaction being modeled, but they indicate the
points in a program where (co)recursive types are unfolded.
%
A term $\Cut\x{P}{Q}$ denotes the parallel composition of two processes $P$ and
$Q$ that interact through the fresh channel $x$.
%
Finally, the term $\PiChoice{P}{Q}$ models a non-deterministic choice between two
behaviors $P$ and $Q$.

\piLIN binders are easily recognizable because they enclose channel names in
round parentheses. Note that all outputs are in fact \emph{bound outputs}. The
output of free channels can be modeled by combining bound outputs with
links \citep{LindleyMorris16}. For example, the output $\FreeFork\x\y\z$ of a
pair of free channels $y$ and $z$ can be modeled as the term
$\Fork[z']\x{y'}{\Link\y{y'}}{\Link\z{z'}}$.
%
We identify processes modulo renaming of bound names, we write $\fn{P}$ for the
set of channel names occurring free in $P$ and we write $\subst\y\x$ for the
capture-avoiding substitution of $y$ for the free occurrences of $x$.
%
We impose a well-formedness condition on processes so that, in every sub-term of
the form $\Fork[z]\x\y{P}{Q}$, we have $y \not\in\fn{Q}$ and $z\not\in\fn{P}$.

We omit any concrete syntax for representing infinite processes. Instead, we
work directly with infinite trees obtained by corecursively unfolding
contractive equations of the form $A(x_1,\dots,x_n) = P$. For each such
equation, we assume that $\fn{P} \subseteq \set{x_1,\dots,x_n}$ and we write
$\Call{A}{y_1,\dots,y_n}$ for its unfolding $P\subst{y_i}{x_i}_{1\leq i\leq n}$.

\begin{notation}
    \label{not:sessions}
    To reduce clutter due to the systematic use of bound outputs, by convention
    we omit the continuation called $y$ in \Cref{fig:proc_syntax_ll} when its name is
    chosen to coincide with that of the channel $x$ on which $y$ is
    sent/received.
    %
    For example, with this notation we have $\Join\x\z.P = \Join[x]\x\z.P$ and
    $\Select{\InTag_i}\x.P = \Select[x]{\InTag_i}\x.P$ and $\Case\x{P}{Q} =
    \Case[x]\x{P}{Q}$.
    %
    \eoe
\end{notation}

A welcome side effect of adopting \Cref{not:sessions} is that it gives the
illusion of working with a session calculus in which the same channel $x$ may be
used repeatedly for multiple input/output operations, while in fact $x$ is a
linear channel used for exchanging a single message along with a fresh
continuation that turns out to have the same name. If one takes this notation as
native syntax for a session calculus, its linear $\pi$-calculus
encoding \citep{DardhaGiachinoSangiorgi17} turns out to be precisely the \piLIN
term it denotes.
%
Besides, the idea of rebinding the same name over and over is widespread in
session-based functional languages \citep{GayVasconcelos10,Padovani17} as it
provides a simple way of ``updating the type'' of a session endpoint after each
use.

\begin{example}
    \label{ex:bsc_ll_proc}
    Below we model the interaction informally described in
    \Cref{ex:bsc} between \actor{buyer} and \actor{seller} using  the syntactic sugar
    defined in \Cref{not:sessions}:
    \[
    \begin{array}{@{}r@{~}l@{}}
    				& \Cut\x{\Call\Buyer\x}{\Call\Seller{x,y}} \\
            \Buyer(x) & =
            \Rec\x.\parens{
                \PiChoice{
                    \Select\AddTag\x.\Call\Buyer\x
                }{
                    \Select\PayTag\x.\PiClose\x
                }
            }
            \\
            \Seller(x,y) & =
            \Corec\x.
            \Case\x{
                \Call\Seller{x,y}
            }{
                \PiWait\x.\PiClose\y
            }
        \end{array}
    \]

    At each round of the interaction, the buyer decides whether to $\AddTag$ an
    item to the shopping cart and repeat the same behavior (left branch of the
    choice) or to $\PayTag$ the seller and terminate (right branch of the
    choice). The seller reacts dually and signals its termination by sending a
    unit on the channel $y$. As we will see in \Cref{sec:ts_ll_ts}, $\Rec\x$
    and $\Corec\x$ identify the points within processes where (co)recursive
    types are unfolded.

    If we were to define $\Buyer$ using distinct bound names we would write an
    equation like
    \[
        \Buyer(x) =
        \Rec[y]\x.
        \parens{
            \PiChoice{
                \Select[z]\AddTag\y.\Call\Buyer\z
            }{
                \Select[z]\PayTag\y.\PiClose\z
            }
        }
    \]
    and similarly for $\Seller$. 
    %
    \eoe
\end{example}

%%%%%%%%%%%%%%%%%
%%% SEMANTICS %%%
%%%%%%%%%%%%%%%%%

\subsection{Operational Semantics}
\beginbass
%
\begin{figure}[t]
		\framebox[\textwidth]{
    \begin{math}
        \displaystyle
        \begin{array}{@{}lr@{~}c@{~}ll@{}}
            \defrule{sp-link} &
            \Link\x\y & \pcong & \Link\y\x
            \\
            \defrule{sp-comm} &
            \Cut\x{P}{Q} & \pcong & \Cut\x{Q}{P}
            \\
            \defrule{sp-assoc} &
            \Cut\x{P}{\Cut\y{Q}{R}} & \pcong & \Cut\y{\Cut\x{P}{Q}}{R} &
            \text{if $x\in\fn{Q}$,}
            \\
            & & & & \text{$y\not\in\fn{P}$, $x\not\in\fn{R}$}
        \end{array}
    \end{math}
    }
    \caption{Structural pre-congruence of \piLIN}
    \label{fig:pcong_ll}
\end{figure}
%
\begin{figure}[t]
    \framebox[\textwidth]{
    \begin{mathpar}
        \displaystyle
        \inferrule[rp-link]{\mathstrut}{\Cut\x{\Link\x\y}{P} \red P\subst\y\x}
        \defrule[rp-link]{}
        \and
        \inferrule[rp-unit]{\mathstrut}{\Cut\x{\PiClose\x}{\PiWait\x.P} \red P}
        \defrule[rp-unit]{}
        \and
        \inferrule[rp-pair]
        	{\mathstrut}
        	{\Cut\x{\Fork[y]\x\z{P_1}{P_2}}{\Join[y]\x\z.Q} \red \Cut\z{P_1}{\Cut\y{P_2}{Q}}}
        	\defrule[rp-pair]{}
        \and
        \inferrule[rp-sum]
        	{\mathstrut}
        	{\Cut\x{\Select[y]{\InTag_i}\x.P}{\Case[y]\x{P_1}{P_2}} \red \Cut\y{P}{P_i}}
          ~ i\in\set{1,2}
          \defrule[rp-sum]{}
        \and
        \inferrule[rp-rec]
        	{\mathstrut}
        	{\Cut\x{\Rec[y]\x.P}{\Corec[y]\x.Q} \red \Cut\y{P}{Q}}
        	\defrule[rp-rec]{}
        \and
        \inferrule[rp-choice]
        	{\mathstrut}
        	{\PiChoice{P_1}{P_2} \red P_i}
        	~ i\in\set{1,2}
        	\defrule[rp-choice]{}
				\and
				\inferrule[rp-cut]
					{P \red Q}
					{\Cut\x{P}{R} \red \Cut\x{Q}{R}}
					\defrule[rp-cut]{}
        \and
        \inferrule[rp-struct]
        	{P \pcong P' \\ P' \red Q' \\ Q' \pcong Q}
        	{P \red Q}
        	\defrule[rp-struct]{}
    \end{mathpar}
    }
    \caption{Reduction of \piLIN}
    \label{fig:red_ll}
\end{figure}
%
The operational semantics of the calculus is given in terms of the
\emph{structural precongruence} relation $\pcong$ and the \emph{reduction
relation} $\red$ defined in \Cref{fig:pcong_ll,fig:red_ll}.
%
As usual, structural precongruence relates processes that are syntactically
different but semantically equivalent. In particular, \refrule{sp-link} states
that linking $x$ with $y$ is the same as linking $y$ with $x$, whereas
\refrule{sp-comm} and \refrule{sp-assoc} state the expected  commutativity and
associativity laws for parallel composition. Concerning the latter, the side
condition $x\in\fn{Q}$ makes sure that $Q$ (the process brought closer to $P$
when the relation is read from left to right) is indeed connected with $P$ by
means of the channel $x$.
%
Note that \refrule{sp-assoc} only states the right-to-left associativity of
parallel composition and that the left-to-right associativity law
$\Cut\x{\Cut\y{P}{Q}}{R} \pcong \Cut\y{P}{\Cut\x{Q}{R}}$ is derivable when
$x\in\fn{Q}$.
%
The reduction relation is mostly unremarkable. Links are reduced with
\refrule{rp-link} by effectively merging the linked channels. All the reductions
that involve the interaction between processes except \refrule{rp-unit} create
new continuations channels that connect the reducts. The rule \refrule{rp-choice}
models the non-deterministic choice between two behaviors. Finally,
\refrule{rp-cut} and \refrule{rp-struct} close reductions by cuts and structural
precongruence.
%
In the following we write $\wred$ for the reflexive, transitive closure of
$\red$ and we say that $P$ is \emph{stuck} if there is no $Q$ such that $P \red
Q$.

%%%%%%%%%%%%%%%%
%%% EXAMPLES %%%
%%%%%%%%%%%%%%%%

\subsection{Examples}
\label{ssec:proc_ex_multi}
\beginbass
%
In the rest of this section we illustrate the main features of the calculus with
some examples. For none of them the existing multiparty session type systems are
able to guarantee progress.
%
First of all, we formally define in our calculus the processes corresponding to 
(a slightly different) \actor{buyer}, \actor{seller} and \actor{carrier} from \Cref{ex:bsc} that was only informally 
presented. 

\begin{example}[Buyer - Seller - Carrier]
  \label{ex:bsc_multi}
  Consider the following definitions:
  \begin{align*}
    \Main & \peq \pres\sn{ \pinvk\Buyer{\ep\sn\buyer} \ppar \pinvk\Seller{\ep\sn\seller} \ppar \pinvk\Carrier{\ep\sn\carrier}} \\ 
    \Buyer(x) & \peq \act{x}\seller\oact\set{
    			  \tadd.\act{x}\seller\oact\tadd.\pinvk\Buyer{x},
                  \tpay.\pclose{x}
                } \\ 
    \Seller(x) & \peq \act{x}\buyer\iact\set{
                  \tadd.\pinvk\Seller{x},
                  \tpay.\act{x}\carrier\oact\tship.\pclose{x} 
                } \\ 
    \Carrier(x) & \peq \act{x}\seller\iact\tship.\pwait{x}\pdone
  \end{align*}
  Note that the buyer either sends $\tpay$ or it sends two $\tadd$ messages in a
  row before repeating this behavior. That is, this particular buyer always adds
  an even number of items to the shopping cart.
  %
  Nonetheless, the buyer periodically has a chance to send a $\tpay$ message and
  terminate. Therefore, the execution of the program in which the buyer only
  sends $\tadd$ is unfair according to \Cref{def:fair_run} hence this program is
  fairly terminating. 
  %
  \eoe
\end{example}

\begin{example}[Purchase with negotiation]
  \label{ex:2bsc_multi}
  Consider a variation of \Cref{ex:bsc} in which the buyer, before making the
  payment, negotiates with a secondary buyer for an arbitrarily long time. The
  interaction happens in two nested sessions, an outer one involving the primary
  buyer, the seller and the carrier, and an inner one involving only the two
  buyers. We model the interaction as the program below, in which we collapse
  role names to their initials.
    \begin{align*}
    \Main & \peq \pres\sn{ \pinvk\Buyer{\ep\sn\rbuyer} \ppar \pinvk\Seller{\ep\sn\rseller} \ppar \pinvk\Carrier{\ep\sn\rcarrier} }
    \\ 
    \Buyer(x) & \peq \act{x}\rseller\oact\tquery.
                \act{x}\rseller\iact\tprice.
                \pres{t}{ \pinvk{\Buyer_1}{x,\ep\asn{\rbuyer_1}} \ppar \pinvk{\Buyer_2}{\ep\asn{\rbuyer_2}}} 
    \\ 
    \Seller(x) & \peq \act{x}\rbuyer\iact\tquery.
                \act{x}\rbuyer\oact\tprice.
                \act{x}\rbuyer\iact\{
                	\begin{lines}
                		\tpay.\act{x}\rcarrier\oact\tship.\pclose{x}, \\
                		\tcancel.\act{x}\rcarrier\oact\tcancel.\pclose{x} \}
                	\end{lines}
    \\     
    \Carrier(x) & \peq \act{x}\rseller\iact\set{
                  \tship.\act{x}\rbuyer\oact\tbox.\pclose{x},
                  \tcancel.\pclose{x} 
                }
    \\
    \Buyer_1(x,y) & \peq \act{y}{\rbuyer_2}\oact\{
                      \begin{lines}
                        \tsplit.\act{y}{\rbuyer_2}\iact\{
                          \begin{lines}
                            \tyes.\pcast{x}
                                  \act{x}\rseller\oact\tok.
                                  \act{x}\rcarrier\iact\tbox.
                                  \pwait{x}
                                  \pwait{y}
                                  \pdone,
                                  \\
                            \tno.\pinvk{\Buyer_1}{x,y} \},
                          \end{lines}
                        \\
                        \tgiveup.
                          \pwait{y}
                          \pcast{x}
                          \act{x}\rseller\oact\tcancel.
                          \pwait{x}
                          \pdone \}
                      \end{lines}
    \\
    \Buyer_2(y) & \peq \act{y}{\rbuyer_1}\iact\set{
                    \tsplit.\act{y}{\rbuyer_1}\oact\set{
                      \tyes.\pclose{y},
                      \tno.\pinvk{\Buyer_2}{y} 
                    },
                    \tgiveup.\pclose{y} 
                  }
  \end{align*} 

  The buyer queries the seller which replies with a price. At this point,
  $\Buyer$ creates a new session $t$ and forks as a primary buyer $\Buyer_1$ and
  a secondary buyer $\Buyer_2$. The interaction between the two sub-buyers goes
  on until either $\Buyer_1$ gives up or $\Buyer_2$ accepts its share of the
  price. In the former case, the primary buyer waits for the internal session to
  terminate and $\tcancel$s the order with the seller which, in turn, aborts the
  transaction with the carrier. In the latter case, the buyer confirms the order
  to the seller, which then instructs the carrier to $\tship$ a $\tbox$ to the
  buyer.
  
  Note that the outermost session $s$, taken in isolation, terminates in a
  bounded number of interactions, but its progress cannot be established without
  assuming that the innermost session $t$ terminates. In particular, if the two
  buyers keep negotiating forever, the seller and the carrier starve.
  %
  However, the innermost session can terminate if $\Buyer_1$ sends $\tgiveup$ to
  $\Buyer_2$ or if $\Buyer_2$ sends $\tyes$ to $\Buyer_1$. Thus, the run in
  which the two buyers negotiate forever is unfair, the session $t$ fairly
  terminates and the session $s$ terminates as well.
  
  On the technical side, note that the definition of $\Buyer_1$ contains two
  casts on the variable $x$. As we will see in \cref{ex:2bsc-ts}, these casts
  are necessary for the typeability of $\Buyer_1$ to account for the fact that
  $x$ is used \emph{differently} in two distinct branches of the process.
  %
  \eoe
\end{example}

\begin{example}[Parallel merge sort]
  \label{ex:pms_multi}
  To illustrate an example of program that creates an unbounded number of
  sessions we model a parallel version of the merge sort algorithm.
  \begin{align*}
    \Main & \peq \pres{s}{
                    \act{\ep{s}\rmaster}\rworker\oact\treq.
                    \act{\ep{s}\rmaster}\rworker\iact\tres.
                    \pwait{s}
                    \pdone \parop
                    \pinvk\Sort{\ep{s}\rworker}
                  } \\ 
    \Sort(x) & \peq \act{x}\rmaster\iact\treq.
    \\
    & (
                  \pres{t}{
                    \pinvk\Merge{x,\ep{t}\rmaster} \parop
                    \pinvk\Sort{\ep{t}{\rworker_1}} \parop
                    \pinvk\Sort{\ep{t}{\rworker_2}}
                  }
                  \pchoice
                  \act{x}\rmaster\oact\tres.\pclose{x}
                )
    \\ 
    \Merge(x,y) & \peq \act{y}{\rworker_1}\oact\treq.
                  \act{y}{\rworker_2}\oact\treq.
                  \act{y}{\rworker_1}\iact\tres.
                  \act{y}{\rworker_2}\iact\tres.
                  \pwait{y}
                  \act{x}\rmaster\oact\tres.
                  \pclose{x}
  \end{align*}

  The program starts as a single session $s$ in which a master $\rmaster$ sends
  the initial collection of data to the worker $\rworker$ as a $\treq$ message
  and waits for the $\tres$ult. The worker is modeled as a process $\Sort$ that
  decides whether to sort the data by itself (right branch of the choice in
  $\Sort$), in which case it sends the $\tres$ult directly to the master, or to
  partition the collection (left branch of the choice in $\Sort$). In the latter
  case, it creates a new session $t$ in which it sends $\treq$uests to two
  sub-workers $\rworker_1$ and $\rworker_2$, it gathers the partial $\tres$ults
  from them and gets back to the master with the complete $\tres$ult.

  Since a worker may always choose to start two sub-workers in a new session,
  the number of sessions that may be created by this program is unbounded. At
  the same time, each worker may also choose to complete its task without
  creating new sessions. So, while in principle there exists a run of this
  program that keeps creating new sessions forever, this run is unfair according
  to \Cref{def:fair_run}.
  %
  \eoe
\end{example}