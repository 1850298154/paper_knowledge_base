\beginbass
%
We introduce two fundamental notions for the soundness proof of the type system.
First, we introduce the \emph{rank} of a session map $M$ as the minimum length to reach
successful termination of session $M$. 
%
Then, we introduce the \emph{measure} of a process which takes into account the rank in the typing judgment
as well as the ranks of the session that have been already opened.
We embed such measure in the typing derivations by using a refined set of rules.
%
At last, we compare the typing judgments labeled with the usual rank with those including the measure (see \Cref{lem:measure_rank_multi})
and we prove that structural precongruence of processes does not increase the measure (see \Cref{lem:measure_pcong_multi}).

\begin{figure}[t]
\framebox[\textwidth]{
\begin{mathpar}
    \inferrule[mtm-thread]{
        \mathstrut
    }{
        \wtpn{(n, 0)}\Ctx{P}
    }
    \wtp[n]\Ctx{P}
    \defrule[mtm-thread]{}
    \and
    \inferrule[mtm-cast]{
        \wtpn\Measure{\Ctx, u : T}{P}
    }{
        \wtpn{\Measure + (n,0)}{\Ctx, u : S}{\pcast{u} P}
    }
    ~
    S \subt[n] T
    \defrule[mtm-cast]{}
    \and
    \inferrule[mtm-par]{
        \wtpn{\Measure_i}{\Ctx_i, \ep{s}{\role_i} : S_i}{P_i}~{}^{(i=1,\dots,h)}
    }{
        \wtpn{\sum_{i=1}^h \Measure_i + (0, \rank{\set{\Map{\role_i}{S_i}}_{i=1,\dots,h}})}{
            \Ctx_1,\dots,\Ctx_h
        }{
            \pres{s}{P_1 \parop \dots \parop P_h}
        }
    }
    ~ \coherent{\set{\Map{\role_i}{S_i}}_{i=1..h}}
    \defrule[mtm-par]{}
\end{mathpar}
}
\caption{Typing rules with measure}
\label{fig:measure_multi}
\end{figure}

\begin{definition}[rank]
  \label{def:rank_multi}
  The \emph{rank} of a session map $M = \prod_{i=1}^h \Map{\role_i}{S_i}$, written $\rank{M}$, is the element of $\Nat
  \union \set\infty$ defined as
  \begin{center}
  	\begin{math}
  		\rank{M} \eqdef \min | M \wlred{\In\terminated}|
  	\end{math}
  \end{center}
  where $|M \wlred{\action} N|$ denotes the length of the sequence $\tau,\dots,\tau,\action$ and we postulate that $\min\emptyset = \infty$.
\end{definition}

\begin{definition}[Measure]
\label{def:measure_multi}
The measure of a process is a lexicographically ordered pair of natural numbers
$(m , n)$ where:
\begin{itemize}
\item $m$ is an upper bound to the number of sessions that the process may open
and of weights of casts that the process may perform \emph{in the future} before
it terminates;
\item $n$ is the overall effort for terminating the sessions that have been
already opened \emph{in the past}, \ie the sum of their rank (\Cref{def:rank_multi}).
\end{itemize}
\end{definition}

In \Cref{fig:measure_multi} we introduce a refined set of typing rules for processes that allow us to
associate them with their measure, not just with their rank.
The idea behind these rules (similarly to \Cref{fig:measure_bin}) is that they distinguish between \emph{past} and
\emph{future} of a process by looking at its structure. Indeed, unguarded
sessions have been created, casts have not been performed yet and sessions that
occur guarded have not been created yet.
% 
\refrule{mtm-thread} adopts the rank of the process inside the usual typing
judgment (\Cref{fig:ts_multi}) as first component of the measure. This rule has lower
priority with respect to the other rules so that it is applied to processes that
are not casts or restrictions.
%
In \refrule{mtm-cast} the first component of the measure is increased by the
weight of the cast.
%
\refrule{mtm-par} increases the second component of the measure by the rank of
the involved session.

\begin{lemma}
    \label{lem:measure_rank_multi}
    The following properties hold:
    \begin{enumerate}
        \item $\wtp[n]\Ctx{P}$ implies $\wtpn\Measure\Ctx{P}$ for some
        $\Measure \leq (n, 0)$;
        \item $\wtpn\Measure\Ctx{P}$ implies $\wtp[n]\Ctx{P}$ for some $n$ such that $\Measure \leq (n, 0)$.
    \end{enumerate}
\end{lemma}
\begin{proof}
    We prove item 1 by induction on the structure of $P$. 
    The proof of item 2 is by a straightforward induction over $\wtpn\Measure\Ctx{P}$.
    
\proofcase{Case $P = \pres{s}{\procs{P}}$}
From \refrule{tm-par} we deduce that there exist $\Ctx_i, \role_i, S_i, n_i$ for $i = 1,\dots,h$ such that
\begin{itemize}
\item $\Ctx = \Ctx_1,\dots,\Ctx_h$
\item $n = 1 + \sum_{i=1}^h n_i$
\item $\prod_{i=1}^h \Map{\role_i}{S_i} \ft$
\item $\wtp[n_i]{\Ctx_i, \ep{s}{\role_i} : S_i}{P_i}~{}^{(i=1,\dots,h)}$
\end{itemize} 
Using the induction hypothesis on $\wtp[n_i]{\Ctx_i, \ep{s}{\role_i} : S_i}{P_i}~{}^{(i=1,\dots,h)}$ we deduce 
that there exist $\Measure_i$ for $i=1,\dots,h$ such that 
\begin{itemize}
\item $\wtpn{\Measure_i}{\Ctx_i, \ep{s}{\role_i} : S_i}{P_i}~{}^{(i=1,\dots,h)}$
\item $\Measure_i \le (n_i,0)$ for $i=1,\dots,h$
\end{itemize}
We conclude with one application of \refrule{mtm-par} by taking 
$\Measure \eqdef \sum_{i=1}^h \Measure_i + (0, \rank{\prod_{i=1}^h \Map{\role_i}{S_i}})$ and observing that 
$\Measure < (n_1,0) + (n_2,0) + \dots + (n_h,0) + (1,0) = (n,0)$.

\proofcase{Case $P = \pcast{u}{Q}$}
From \refrule{tm-cast} we deduce that there exist $\CtxD, S, T, m$ and $m_u$ such that
\begin{itemize}
\item $\Ctx = \CtxD, u : S$
\item $S \subt[m_u] T$
\item $n = m_u + m$
\item $\wtp[m]{\CtxD, u : T}{Q}$
\end{itemize}
Using the induction hypothesis on $\wtp[m]{\CtxD, u : T}{Q}$ we deduce $\wtpn{\MeasureN}{\CtxD, u : T}{Q}$ for some $\MeasureN \le (m,0)$.
We conclude with an application of \refrule{mtm-cast} by taking $\Measure \eqdef \MeasureN + (m_u,0)$ 
and observing that $\Measure \le (m,0) + (m_u,0) = (n,0)$.

\proofcase{In all the other cases} We conclude with an application of \refrule{mtm-thread} by taking $\Measure \eqdef (n,0)$.
\end{proof}

\begin{lemma}
	\label{lem:measure_pcong_multi}
	If $\wtpn\MeasureM\Ctx P$ and $P \pcong Q$, then there exists $\MeasureN \le \MeasureM$ such that $\wtpn\MeasureN\Ctx Q$.
\end{lemma}
\begin{proof}
By induction on the derivation of $P \pcong Q$ and by cases on the last rule applied. We only consider the base cases.

\proofrule{sm-par-comm} 
%
Then $P = \pres{s}{\procs{P} \ppar P' \ppar Q' \ppar \procs{Q}} \pcong \pres{s}{\procs{P} \ppar Q' \ppar P' \ppar \procs{Q}} = Q$.
%
From rule \refrule{mtm-par} we deduce that there exist $\Ctx_i, \role_i, S_i, \Measure_i$ for $i = 1,\dots,h$ such that
\begin{itemize}
\item $\Ctx = \Ctx_1,\dots,\Ctx_h$
\item $\Measure = \sum_{i=1}^h \Measure_i + (0 , \rank{\prod_{i=1}^h \Map{\role_i}{S_i}})$
\item $\prod_{i=1}^h \Map{\role_i}{S_i} \ft$
\item $\wtpn{\Measure_i}{\Ctx_i, \ep{s}{\role_i} : S_i}{P_i}$ for $i = 1,\dots,k$
\item $\wtpn{\Measure_{k+1}}{\Ctx_{k+1}, \ep{s}{\role_{k+1}} : S_{k+1}}{P'}$
\item $\wtpn{\Measure_{k+2}}{\Ctx_{k+2}, \ep{s}{\role_{k+2}} : S_{k+2}}{Q'}$
\item $\wtpn{\Measure_i}{\Ctx_i, \ep{s}{\role_i} : S_i}{Q_i}$ for $i = k+3,\dots,h$
\end{itemize}
We conclude $\wtpn{\MeasureN}\Ctx{Q}$ with one application of \refrule{mtm-par} by taking $\MeasureN \eqdef \Measure$.

\proofrule{sm-par-assoc}
%
Then $P = \pres{s}{\procs{P} \ppar \pres{t}{R \ppar \procs{Q}}} \pcong \pres{t}{\pres{s}{\procs{P} \ppar R} \ppar \procs{Q}} = Q$ and $s \in \fn{R}$.
%
From rule \refrule{mtm-par} we deduce that there exist $\Ctx_i, \role_i, S_i, \Measure_i$ for $i = 1,\dots,h$ such that
\begin{itemize}
\item $\Ctx = \Ctx_1,\dots,\Ctx_h$
\item $\Measure = \sum_{i=1}^h \Measure_i + (0 , \rank{\prod_{i=1}^h \Map{\role_i}{S_i}})$
\item $\prod_{i=1}^h \Map{\role_i}{S_i} \ft$
\item $\wtpn{\Measure_i}{\Ctx_i, \ep{s}{\role_i} : S_i}{P_i}$ for $i = 1,\dots,h - 1$
\item $\wtpn{\Measure_h}{\Ctx_h, \ep{s}{\role_h} : S_h}{\pres{t}{R \ppar \procs{Q}}}$
\end{itemize}
From rule \refrule{mtm-par} and the hypothesis that $s \in \fn{R}$ we deduce that there exist 
$\CtxD_i, \roleq_i, T_i, \MeasureN_i$ for $i = 1,\dots,k$ such that
\begin{itemize}
\item $\Ctx_h = \CtxD_1,\dots,\CtxD_k$
\item $\Measure_h = \sum_1^k \MeasureN_i + (0 , \rank{\prod_{i=1}^k \Map{\roleq_i}{T_i}})$
\item $\prod_{i=1}^k \Map{\roleq_i}{T_i} \ft$
\item $\wtpn{\MeasureN_1}{\CtxD_1, \ep{s}{\role_h} : S_h, \ep{t}{\roleq_1} : T_1}{R}$
\item $\wtpn{\MeasureN_{i+1}}{\CtxD_{i+1}, \ep{t}{\roleq_{i+1}} : T_{i+1}}{Q_i}$ for $i = 1,\dots,k-1$
\end{itemize}
Using \refrule{tm-par} we deduce
\begin{itemize}
\item $\wtpn{\sum_{i=1}^{h-1}{\Measure_i} 
	+ \MeasureN_1 
	+ \rank{\prod_{i=1}^h \Map{\role_i}{S_i}}}{\Ctx_1,\dots,\Ctx_{h-1},\CtxD_1, \ep{t}{\roleq_1} : T_1}{\pres{s}{\procs{P} \ppar R}}$
\end{itemize}   
We conclude $\wtpn{\MeasureN}{\Ctx}{\pres{t}{\pres{s}{\procs{P} \ppar R} \ppar \procs{Q}}}$ with another 
application of \refrule{mtm-par} by taking $\MeasureN \eqdef \Measure$.

\proofrule{sm-cast-comm} 
%
Then $P = \pcast{u}{\pcast{v}{R}} \pcong \pcast{v}{\pcast{u}{R}} = Q$. We can assume $u \ne v$ or else $P = Q$.
%
From rule \refrule{mtm-cast} we deduce that there exist $\Ctx_1, S, T, \Measure_1, m_u$ such that
\begin{itemize}
\item $\Ctx = \Ctx_1, u : S$
\item $S \subt[m_u] T$
\item $\Measure = \Measure_1 + (m_u , 0)$
\item $\wtpn{\Measure_1}{\Ctx_1, u : T}{\pcast{v}{R}}$
\end{itemize} 
From rule \refrule{tm-cast} we deduce that there exist $\Ctx_2, S', T', \Measure_2, m_v$ such that
\begin{itemize}
\item $\Ctx_1 = \Ctx_2, v : S'$
\item $S' \subt[m_v] T'$
\item $\Measure_1 = \Measure_2 + (m_v , 0)$
\item $\wtpn{\Measure_2}{\Ctx_2, u : T, v : T'}{R}$
\end{itemize}
We derive $\wtpn{\Measure_2 + (m_u,0)}{\Ctx_2, u : S, v : T'}{\pcast{u}{R}}$ with one application of \refrule{mtm-cast} 
and we conclude with another application of \refrule{mtm-cast} by taking $\MeasureN \eqdef \Measure$.

\proofrule{sm-cast-new}
%
Then $P = \pres{s}{\pcast{\ep{s}{\role}}{R} \ppar \procs{P}} \pcong \pres{s}{R \ppar \procs{P}} = Q$.
%
From rule \refrule{mtm-par} we deduce that there exist $\CtxD, \Measure', S$ and $\Ctx_i, \roleq_i, S_i, \Measure_i$ for $i = 1,\dots,h$ such that
\begin{itemize}
\item $\Ctx = \CtxD, \Ctx_1, \dots, \Ctx_h$
\item $\Measure = \Measure' + \sum_{i=1}^h \Measure_i + (0 , \rank{\Map{\role}{S} \ppar \prod_{i = 1}^h \Map{\roleq_i}{S_i}})$
\item $\Map{\role}{S} \parop \prod_{i = 1}^h \Map{\roleq_i}{S_i} \ft$
\item $\wtpn{\Measure'}{\CtxD, \ep{s}{\role} : S}{\pcast{\ep{s}{\role}}{R}}$
\item $\wtpn{\Measure_i}{\Ctx_i, \ep{s}{\roleq_i} : S_i}{P_i}$ for $i = 1,\dots,h$
\end{itemize} 
From rule \refrule{mtm-cast} we deduce that there exist $T, \MeasureN', m_s$ such that
\begin{itemize}
\item $S \subt[m_s] T$
\item $\MeasureN' = \Measure' + (m_s , 0)$
\item $\wtpn{\MeasureN'}{\CtxD, \ep{s}{\role} : T}{R}$
\end{itemize}
From $\Map{\role}{S} \ppar \prod_{i = 1}^h \Map{\roleq_i}{S_i} \ft$, $S \subt[m_s] T$ and \cref{def:ssubt} 
we deduce $\Map{\role}{T} \ppar \prod_{i = 1}^h \Map{\roleq_i}{S_i} \ft$. 
We conclude with an application of \refrule{mtm-par} by taking 
$\MeasureN = \MeasureN' + \sum_{i=1}^h \Measure_i + (0 , \rank{\Map{\role}{S} \ppar \prod_{i = 1}^h \Map{\roleq_i}{S_i}}) \le n$.

\proofrule{sm-cast-swap}
%
Then $P = \pres{s}{\pcast{\ep{t}{\role}}{R} \ppar \procs{P}} \pcong \pcast{\ep{t}{\role}}{\pres{s}{R \ppar \procs{P}}} = Q$ and $t \ne s$.
%
From rule \refrule{mtm-par} we deduce that there exist $\Ctx_i, \roleq_i, \Measure_i$ for $i = 1,\dots,h$ such that
\begin{itemize}
\item $\Ctx = \Ctx_1, \dots, \Ctx_h$
\item $\Measure = \sum_{i=1}^h \Measure_i + (0 , \rank{\prod_{i = 1}^h \Map{\roleq_i}{S_i}})$
\item $\prod_{i = 1}^h \Map{\roleq_i}{S_i} \ft$
\item $\wtpn{\Measure_1}{\Ctx_1, \ep{s}{\roleq_1} : S_1}{\pcast{\ep{t}{\role}}{R}}$
\item $\wtpn{\Measure_i}{\Ctx_i, \ep{s}{\roleq_i} : S_i}{P_i}$ for $i = 2,\dots,h$
\end{itemize} 
From rule \refrule{mtm-cast} we deduce that there exist $\CtxD, T, \Measure', m_t$ such that
\begin{itemize}
\item $\Ctx_1 = \CtxD, \ep{t}{\role} : S$
\item $S \subt[m_t] T$
\item $\Measure_1 = \Measure' + (m_t , 0)$
\item $\wtpn{\Measure'}{\CtxD, \ep{t}{\role} : T, \ep{s}{\roleq_1} : S_1}{R}$
\end{itemize}
We derive 
$\wtpn{\Measure' 
	+ \sum_{i=2}^h \Measure_i 
	+ (0 , \rank{\prod_{i = 1}^h \Map{\roleq_i}{S_i}})}{\CtxD, \ep{t}{\role} : T,\Ctx_2,\dots,\Ctx_h}{\pres{s}{R \ppar \procs{P}}}$ 
with an application of \refrule{mtm-par}. We conclude with an application of \refrule{mtm-cast} by taking $m \eqdef n$.

\proofrule{sm-call} 
%
Then $P = \pinvk{A}{\seqof u} \pcong R\subst{\seqof x}{\seqof u} = Q$ and $\Definition{A}{\seqof x}{R}$.
%
From \refrule{mtm-thread} we deduce that $\wtpn{n}{\Ctx}{\pinvk{A}{\seqof u}}$ for some $n$ such that $\Measure = (n , 0)$. 
Using \Cref{lem:subj_cong_multi} we deduce $\wtp[m]{\Ctx}{Q}$ for some $m \le n$.
Using \Cref{lem:measure_rank_multi} we deduce that $\wtpn{\MeasureN}{\Ctx}{Q}$ for some $\MeasureN \le (m , 0)$. 
We conclude observing that $\MeasureN \le (m , 0) \le (n , 0) = \Measure$.
\end{proof}