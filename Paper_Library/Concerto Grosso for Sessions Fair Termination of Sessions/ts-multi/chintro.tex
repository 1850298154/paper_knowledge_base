\begintreble
%
In this chapter we generalize the type system in \Cref{ch:ft_bin} to multiparty session types.
In particular, we provide a type system for enforcing (successful) fair termination
of multiparty sessions. In this case we refer to the notion of \emph{coherence} (\Cref{def:coherence}).
Again, we embed \emph{fair subtyping} as the coherence-preserving relation.
Notably, the \emph{boundedness} properties mentioned in \Cref{ssec:boundedness} are still required.
For the sake of simplicity we do not analyze them step by step as the motivating examples can be
straightforwardly adapted. Instead, we directly show how to enforce them.

\begin{remark}
	The need of the type system that we present in this chapter comes from the fact the there 
	exist multiparty sessions that cannot be decomposed in binary ones.
	Consider the following local types
	\[
	\begin{array}{lcl}
		\S_{\rolep} & : & \roleq\Out\set{\Tag[a].\roler\Out\Tag[c].\End[\Out],\, \Tag[b].\End[\Out]}
		\\
		\S_{\roleq} & : & \rolep\In\set{\Tag[a].\roler\Out\Tag[ok].\End[\Out],\, \Tag[b].\roler\Out\Tag[no].\End[\Out]}
		\\
		\S_{\roler} & : & \roleq\In\set{\Tag[ok].\rolep\In\Tag[c].\End[\In],\, \Tag[no].\End[\In]}
	\end{array}
	\]
	This example is paradigmatic because of the dependencies in the communications.
	Indeed, $\roler$ receives $\Tag[c]$ from $\role$ only if $\roleq$ receives $\Tag[a]$ from $\role$.
	Furthermore, the class of multiparty sessions that can be decomposed into linear
	ones consists of those sessions whose local types can be \emph{projected} to all the participants
	appearing in them.
	(\ie the \emph{projections} exist).
	This is not true for the example above because the projections of $\S_{\role}$ on $\roler$ and
	of $\S_{\roler}$ on $\role$ are not defined as is such types the communication with $\roler$ and $\role$
	does not take place in all the branches. 
	%
	\eor
\end{remark}

The chapter is organized as follows.
In \Cref{sec:ts_multi_proc} we present the syntax and the semantics of the calculus based on multiparty sessions. 
\Cref{sec:ts_multi_ts} shows the typing rules for such calculus. Differently from \Cref{sec:ts_bin_ts}, we focus
on some involved examples of processes instead of analyzing additional required properties (see \Cref{ssec:boundedness}).
Then, in \Cref{sec:ts_multi_corr} we detail the soundness proof of the type system.
At last, in \Cref{sec:ts_multi_related} we compare the present type system, and consequently
the one in \Cref{ch:ft_ll}, to existing works.
