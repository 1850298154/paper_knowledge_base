\beginbass
%
\begin{figure}[t]
	\framebox[\textwidth]{
  \begin{math}
    \begin{array}{@{}lr@{~}c@{~}ll@{}}
      \defrule{sm-par-comm} & \pres\sn{\procs{P} \ppar P \ppar Q \ppar \procs{Q}} 
      & \pcong & 
      \pres\sn{\procs{P} \ppar Q \ppar P \ppar \procs{Q}}
      \\
      \defrule{sm-par-assoc} & \pres\sn{\procs{P} \ppar \pres\asn{R \ppar \procs{Q}}} 
      & \pcong & 
      \pres\asn{\pres\sn{\procs{P} \ppar R} \ppar \procs{Q}}
      & \text{if $s \in \fn{R}$}
      \\
      & & & & \text{and $t \not\in \fn{P}$,}
      \\
      & & & & \text{$\forall Q. s \not\in \fn{Q}$}
      \\
      \defrule{sm-cast-comm} & \pcast{u}{\pcast{v}{P}} & \pcong & \pcast{v}{\pcast{u}{P}}
      \\
      \defrule{sm-cast-new} & \pres{s}{\pcast{\ep{s}\role} P \ppar \procs{Q}}
      & \pcong & 
      \pres{s}{P \ppar \procs{Q}}
      \\
      \defrule{sm-cast-swap} & \pres{s}{\pcast{\ep{t}\role}{P} \ppar \procs{Q}} 
      & \pcong & 
      \pcast{\ep{t}\role}{\pres{s}{P \ppar \procs{Q}}}
      & \text{if $s \ne t$} 
      \\
      \defrule{sm-call} & \pinvk{A}{\seqof{u}} & \pcong & P\subst{\seqof{u}}{\seqof{x}}
      & \text{if $\pdef{A}{\seqof{x}}{P}$}
    \end{array}
  \end{math}
  }
  \caption{Structural precongruence of processes}
  \label{fig:pcong_multi}
\end{figure}
%
\begin{figure}[t]
	\framebox[\textwidth]{
  \begin{mathpar}
    \inferrule[rm-choice]{ }{
      P_1 \pchoice P_2 \red P_k
    }
    ~ k\in\set{1,2} \defrule[rm-choice]{}
    \and
    \inferrule[rm-signal]{ }{
      \pres{s}{\pwait{\ep{s}\role}{P} \parop \pclose{\ep{s}{\roleq_1}} \parop \cdots \parop \pclose{\ep{s}{\roleq_n}}} 
      \red
      P
    } \defrule[rm-signal]{}
    \and
    \inferrule[rm-channel]{ }{
      \pres{s}{\poch{\ep{s}{\rolep}}{\roleq}{v}{P} \parop \pich{\ep{s}{\roleq}}{\rolep}{x}{Q} \parop \procs{R}}
      \red
      \pres{s}{P \parop Q\subst{v}{x}	\parop \procs{R}}
    } \defrule[rm-channel]{}
    \and
    \inferrule[rm-pick]{ }{
      \pres{s}{\pobranch[i\in I]{\ep{s}\role}\roleq{\Tag_i}{P_i} \parop \procs{Q}} 
      \red
      \pres{s}{\pobranch{\ep{s}\role}\roleq{\Tag_k}{P_k}\parop \procs{Q}}
    }
    ~ k\in I \defrule[rm-pick]{}
    \and
    \inferrule[rm-tag]{ }{
      \pres{s}{\pobranch{\ep{s}\role}{\roleq}{\Tag_k}{P} \parop \pibranch[i\in I]{\ep{s}\roleq}\rolep{\Tag_i}{Q_i} \parop \procs{R}} 
      \red
      \pres{s}{P \parop Q_k \parop \procs{R}}
    }
    ~ k\in I \defrule[rm-tag]{}
    \and
    \inferrule[rm-par]{
      P \red Q
    }{
      \pres{s}{P \parop \procs{R}} \red \pres{s}{Q \parop \procs{R}}
    } \defrule[rm-par]{}
    \and
    \inferrule[rm-cast]{
      P \red Q
    }{
      \pcast{u}{P} \red \pcast{u}{Q}
    } \defrule[rm-cast]{}
    \and
    \inferrule[rm-struct]{
      P \pcong P'
      \\
      P' \red Q'
      \\
      Q' \pcong Q
    }{
      P \red Q
    } \defrule[rm-struct]{}
  \end{mathpar}
  }
  \caption{Reduction of processes}
  \label{fig:red_multi}
\end{figure}
%
The operational semantics of processes is given by the structural precongruence
relation $\pcong$ defined in \Cref{fig:pcong_multi} and the reduction relation $\red$
defined in \Cref{fig:red_multi}. As usual, structural precongruence allows us to
rearrange the structure of processes without altering their meaning, whereas
reduction expresses an actual computation or interaction step.
%
The adoption of a structural \emph{pre}congruence (as opposed to a more common
congruence relation) is not strictly necessary, but it simplifies the technical
development by reducing the number of cases we have to consider in proofs
without affecting the properties of the calculus in any way (see \Cref{rmk:pcong}).

Rules \refrule{sm-par-comm} and \refrule{sm-par-assoc} state commutativity and
associativity of parallel composition of processes (we write $\seqof{P}$ to
denote possibly empty parallel compositions of processes). In
\refrule{sm-par-assoc}, the side condition $s \in \fn{R}$ makes sure that $R$ is
indeed a participant of the session $s$. 
We write $\forall Q. s \not\in \fn{Q}$
to state that $s$ is not free in each of the processes $\procs{Q}$.
Moreover, note that this rule only states
right-to-left associativity. Left-to-right associativity is derivable from this
rule and repeated uses of \refrule{sm-par-comm}.
%
Rule \refrule{sm-cast-comm} allows us to swap two consecutive casts.
%
Rule \refrule{sm-cast-new} removes an unguarded cast on an endpoint of the
restricted session (we refer to this operation as ``performing the cast'').
%
Rule \refrule{sm-cast-swap} swaps a cast and a restricted session as long as the
endpoint in the cast refers to a different session.
%
Finally, rule \refrule{sm-call} unfolds a process invocation to its definition.
Hereafter, we write $\subst{u}{x}$ for the capture-avoiding substitution of each
free occurrence of $x$ with $u$ and $\subst{\seqof{u}}{\seqof{x}}$ for its
natural extension to equal-length tuples of variables and names.
%
The rules \refrule{sm-cast-new}, \refrule{sm-cast-swap} and \refrule{sm-call} are
not invertible: by \refrule{sm-cast-new} casts can only be removed but never
added; by \refrule{sm-cast-swap} casts can only be moved closer to their
restriction, so that they can be eventually performed by \refrule{sm-cast-new};
by \refrule{sm-call} process invocations can only be unfolded.

The reduction relation is quite standard.
%
Rule \refrule{rm-choice} reduces $P_1\pchoice P_2$ to either $P_1$ or $P_2$, non
deterministically.
%
Rule \refrule{rm-signal} terminates a session in which all participants
($\roleq_1,\ldots,\roleq_n$) but one ($\rolep$) are sending a termination signal
and $\rolep$ is waiting for it; the resulting process is the continuation of the
participant $\rolep$.
%
Rule \refrule{rm-channel} models the exchange of a channel among two participants
of a session.
%
Rule \refrule{rm-pick} models an internal choice whereby a process picks one
particular tag $\Tag_k$ to send on a session.
%
Rule \refrule{rm-tag} synchronizes two participants $\rolep$ and $\roleq$ on
the tag chosen by $\rolep$.
%
Finally, rules \refrule{rm-par}, \refrule{rm-cast} and \refrule{rm-struct} close
reductions under parallel compositions and casts and by structural precongruence. 