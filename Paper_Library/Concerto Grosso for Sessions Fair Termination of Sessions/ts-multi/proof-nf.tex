\beginbass
%
As in \Cref{ssec:nf_bin} we need to prove that a well typed process is deadlock free. The key lemma of this is \Cref{lem:dl_freedom_multi}
that we dub \emph{quasi deadlock freedom}. Informally, we want to rearrange the process under analysis 
in order to put together the creation of some session $s$ with all the subprocesses that start with
some communication on $s$. This way we can try to apply a reduction rule (see \Cref{fig:red_multi}). 
This is what \Cref{lem:dl_freedom_multi} does.
However, a process organized as described is not guaranteed to make progress.
Indeed, for example all the subprocesses can be waiting for a message leading to a stuck process.
Hence, we called the lemma \emph{quasi} deadlock freedom. 
Deadlock freedom is achieved by taking into account the \emph{coherence} of the session as well.
%
We called the shape of the process mentioned before \emph{proximity normal form} (\Cref{def:pnf_multi}).
%
In order to obtain a process in such form from a well typed one we require several steps.
We introduce additional normal forms (\Cref{def:cnf_multi,def:tnf_multi}) to describe those processes in the middle of the procedure.
We introduce \emph{process contexts} to easily refer to unguarded sub-processes:
\[
	\textbf{Process context}
	\quad
	\PCtxC, \PCtxD ~~::=~~ \Hole \mid \pres{s}{\procs{P} \parop \PCtxC \parop \procs{Q}} \mid \pcast{u}\PCtxC
\]

%%%%%%%%%%%%%%%%%%%%%%%%%%%%%
%%%%     DEFINITIONS     %%%%
%%%%%%%%%%%%%%%%%%%%%%%%%%%%%

\begin{definition}[Choice Normal Form]
	\label{def:cnf_multi}
	We say that $P_1 \pchoice P_2$ is an \emph{unguarded choice} of $P$ if there
	exists $\PCtxC$ such that $P \pcong \PCtxC[P_1 \pchoice P_2]$. We say that
	$P$ is in \emph{choice normal form} if it has no unguarded choices.
\end{definition}


\begin{definition}[Thread Normal Form]
	\label{def:tnf_multi}
	A process is in \emph{thread normal form} if it is generated by the grammar below:
	\[
		\begin{array}{@{}rcl@{}}
			\Pnf, \Qnf & ::= & \pcast{u}\Pnf \mid \Ppar
			\\
			\Ppar, \Qpar & ::= & \pres{s}{\procs{\Ppar}} \mid \Pth
			\\
			\Pth & ::= &
				\pdone \mid
				\pclose{u} \mid \pwait{u}{P} \mid 
				\pbranch[i\in I]{\chvar}{\rolep}\Pol{\la_i}{\PP_i} \\
				 
			& &	\mid \poch{u}{\rolep}{v}{P} \mid
					\pich{u}{\rolep}{x}{P}
		\end{array}
	\] 
\end{definition}

Intuitively, a process is in \emph{thread normal form} if it consists of an initial prefix of casts followed 
by a parallel composition of threads, where a thread is either $\pdone$ or a process waiting to perform an input/output 
action on some channel $u = \ep{s}{\role}$ for some $\role$. In this latter case, we say that the thread is an $s$-thread.

\begin{definition}[Proximity Normal Form]
	\label{def:pnf_multi}
	We say that $\Pnf$ is in \emph{proximity normal form} if $\Pnf = \PCtxC[\pres{s}{\procs{\Pth}}]$ for some $\PCtxC$, $s$, $\procs{\Pth}$ where
	each $\Pth_i$ for $i=1,\dots,h$ is a $s$-thread.
\end{definition}

%%%%%%%%%%%%%%%%%%%%%%%%%
%%%%     PROOOFS     %%%%
%%%%%%%%%%%%%%%%%%%%%%%%%

\begin{lemma}
	\label{lem:cnf2_multi}
	If $\wtp[n]\Ctx{P}$ and $\wtpi\Ctx{P}$, then there exists $Q$ in choice
	normal form such that $P \wred Q$ and $\wtp[m]\Ctx{Q}$ for some $m \le n$.
\end{lemma}
\begin{proof}
By induction on $\wtpi\Ctx{P}$ and by cases on the last rule applied.

\proofcase{Case $P$ is already in choice normal form} 
We conclude taking $Q \eqdef P$ and $m \eqdef n$.

\proofrule{tm-call}
Then $P = \pinvk{A}{\seqof{u}}$ and $\Definition{A}{\seqof{x}}{R}$.
We deduce $\Ctx = \seqof{u : S}$, $\tass{A}{\seqof{S}}{n'}$ and $\wtpi\Ctx{R\subst{\seqof u}{\seqof x}}$. Moreover, it must be the case that
$\wtp[n']\Ctx{R\subst{\seqof u}{\seqof x}}$ and $n' \leq n$ since \refrule{tm-call} is used in the coinductive judgment as well.
Using the induction hypothesis we deduce that there exist $Q$ in choice normal form and $m \le n'$ such that $R\subst{\seqof u}{\seqof x} \wred Q$ 
and $\wtp[m]\Ctx{Q}$.
We conclude by observing that $P \wred Q$ using \refrule{rm-struct} and that $m \leq n' \leq n$.

\proofrule{com-choice}
Then $P = P_1 \pchoice P_2$.
We deduce $\wtpi\Ctx{P_k}$ with $k \in \set{1,2}$.
Moreover, it must be the case that $\wtp[n]\Ctx{P_k}$ since \refrule{tm-choice} is used in the coinductive judgment.
Using the induction hypothesis we deduce that there exist $Q$ in choice normal form and $m \leq n$ such that $P_k \wred Q$ and $\wtp[m]\Ctx{Q}$.
We conclude by observing that $P \red P_k$ by \refrule{rm-choice}.

\proofrule{tm-choice}
Analogous to the previous case but we consider the premise in which the rank is the same of the conclusion to keep sure that it does not increase.

\proofrule{tm-par}
Then $P = \pres{s}{P_1 \parop \dots \parop P_h}$. 
We deduce 
\begin{itemize}
\item $\Ctx = \Ctx_1, \dots, \Ctx_h$
\item $\wtpi{\Ctx_i, \ep{s}{\role_i} : S_i}{P_i}$ for $i=1,\dots,h$
\item $\prod_{i=1}^h \Map{\role_i}{S_i} \ft$
\end{itemize}
Furthermore, it must be the case that $\wtp[n_i]{\Ctx_i, \ep{s}{\role_i} : S_i}{P_i}$ for $i=1,\dots,h$ 
and $n = 1 + \sum_{i=1}^h n_i$ since \refrule{tm-par} is used in the coinductive judgment as well.
Using the induction hypothesis we deduce that there exist $Q_i$ in choice normal form and $m_i \leq n_i$ such that 
$P_i \wred Q_i$ and $\wtp[m_i]{\Ctx_i, \ep{s}{\role_i} : S_i}{Q_i}$ for $i=1,\dots,h$.
We conclude by taking $m \eqdef 1 + \sum_{i=1}^h m_i$ and $Q \eqdef \pres{s}{Q_1 \parop \cdots \parop Q_h}$ 
with one application of \refrule{tm-par}, observing that 
$m = 1 + \sum_{i=1}^h m_i \leq 1 + \sum_{i=1}^h n_i = n$ and that $P \wred Q$ by \refrule{rm-par}.

\proofrule{tm-cast}
Then $P = \pcast{u} P'$.
Analogous to the previous case, just simpler.
\end{proof}

\begin{lemma}
	\label{lem:cnf1_multi}
	If $\wtp[n]\Ctx{P}$, then there exists $Q$ in choice normal form such that $P \wred Q$ and $\wtp[m]\Ctx{Q}$ for some $m \le n$.
\end{lemma}
\begin{proof}
	\brk
	Consequence of \Cref{lem:cnf2_multi} noting that $\wtp[n]\Ctx{P}$ implies $\wtpi\Ctx{P}$.
\end{proof}

\begin{lemma}
	\label{lem:cnf_exists_multi}
	If $\wtpn\Measure\Ctx{P}$, then there exist $Q$ in choice normal form and $\MeasureN \leq \Measure$ such that $P \wred Q$ and $\wtpn\MeasureN\Ctx{Q}$.
\end{lemma}
\begin{proof}
By induction on $\wtpn\Measure\Ctx{P}$ and by cases on the last rule applied.

\proofrule{mtm-thread}
Then $P$ is a thread. We deduce that
\begin{itemize}
\item $\Measure = (n , 0)$ for some $n$
\item $\wtp[n]\Ctx{P}$
\end{itemize}
From \Cref{lem:cnf1_multi} we deduce that there exist $Q$ and $m \le n$ such that $P \wred Q$ and $\wtp[m]\Ctx{Q}$. 
From \Cref{lem:measure_rank_multi} we deduce $\wtpn\MeasureN\Ctx{Q}$ for some $\MeasureN \le (m , 0)$. 
We conclude observing that $\MeasureN \le (m , 0) \le (n , 0) = \Measure$.

\proofrule{mtm-cast}
Then $P = \pcast{u}{P'}$. We deduce that
\begin{itemize}
\item $\Ctx = \CtxD, u : S$
\item $S \subt[n] T$
\item $\Measure = \Measure' + (n,0)$
\item $\wtpn{\Measure'}{\Ctx', u : T}{P'}$
\end{itemize}
Using the induction hypothesis we deduce that there exist $Q'$ and $\MeasureN' \le \Measure'$ such that $P' \wred Q'$ 
and $\wtpn{\MeasureN'}{\Ctx', u : T}{Q'}$. We conclude with an application of \refrule{mtm-cast} taking 
$Q \eqdef \pcast{u}{Q'}$, $\MeasureN \eqdef \MeasureN' + (n,0)$ and observing that $P \wred Q$ using \refrule{rm-cast}. 

\proofrule{mtm-par}
Then $P = \pres{s}{P_1 \parop \dots \parop P_h}$. 
We deduce 
\begin{itemize}
\item $\Ctx = \Ctx_1, \dots, \Ctx_h$
\item $\Measure = \sum_{i=1}^h \Measure_i + (0, \rank{\prod_{i=1}^h \Map{\role_i}{S_i}})$
\item $\wtpn{\Measure_i}{\Ctx_i, \ep{s}{\role_i} : S_i}{P_i}$ for $i=1,\dots,h$
\item $\prod_{i=1}^h \Map{\role_i}{S_i} \ft$
\end{itemize}
Using the induction hypothesis we deduce that there exist $Q_i$ in choice normal form and $\MeasureN_i \leq \Measure_i$ such that $P_i \wred Q_i$ and 
$\wtpn{\MeasureN_i}{\Ctx_i, \ep{s}{\role_i} : S_i}{Q_i}$ for $i=1,\dots,h$.
We conclude by taking $\MeasureN \eqdef \sum_{i=1}^h \MeasureN_i + (0, \rank{\prod_{i=1}^h \Map{\role_i}{S_i}})$ and 
$Q \eqdef \pres{s}{Q_1 \parop \cdots \parop Q_h}$ with one application of \refrule{mtm-par}, 
observing that 
$\MeasureN = \sum_{i=1}^h \MeasureN_i + (0, \rank{\prod_{i=1}^h \Map{\role_i}{S_i}}) \leq 
						 \sum_{i=1}^h \Measure_i + (0, \rank{\prod_{i=1}^h \Map{\role_i}{S_i}}) = \Measure$ 
and that $P \wred Q$ by \refrule{rm-par}.
\end{proof}

\begin{lemma}
	\label{lem:tnf_exists_multi}
	If $\wtpi\Ctx P$ and $P$ is in choice normal form, then there exists $\Pnf$ such that $P \pcong \Pnf$. 
\end{lemma}
\begin{proof}
By induction on $\wtpi\Ctx P$ and by cases on the last rule applied.

\proofcase{Cases \refrule{tm-choice} and \refrule{com-choice}} These cases are impossible from the hypothesis that $P$ is in choice normal form.

\proofcase{Cases \refrule{tm-done}, \refrule{tm-wait} and\refrule{tm-close}}
Then $P$ is a thread and is already in thread normal form and we conclude by reflexivity of $\pcong$.

\proofcase{\refrule{tm-channel-in}, \refrule{tm-channel-out}, \refrule{tm-tag} and \refrule{com-tag}}
Then $P$ is a thread and is already in thread normal form and we conclude by reflexivity of $\pcong$.

\proofrule{tm-call}
Then there exist $A$, $Q$, $\seqof u$ and $\seqof S$ such that
\begin{itemize}
\item $P = \pinvk{A}{\seqof u}$
\item $\Definition{A}{\seqof x}Q$
\item $\Ctx = \seqof{u : S}$
\item $\wtpi{\seqof{u : S}}{Q\subst{\seqof u}{\seqof x}}$
\end{itemize}
Using the induction hypothesis on $\wtpi{\seqof{u : S}}{Q\subst{\seqof u}{\seqof x}}$ we deduce that there exists 
$\Pnf$ such that $Q\subst{\seqof u}{\seqof x} \pcong \Pnf$.
We conclude $P \pcong \Pnf$ using \refrule{sm-call} and the transitivity of $\pcong$.

\proofrule{tm-par}
Then there exist $s$ and $P_i, \Ctx_i, S_i, \role_i$ for $i=1,\dots,h$ such that
\begin{itemize}
\item $P = \pres{s}{P_1 \parop \cdots \parop P_h}$
\item $\Ctx = \Ctx_1, \dots,\Ctx_h$
\item $\wtpi{\Ctx_i, \ep{s}{\role_i} : S_i}{P_i}$ for $i=1,\dots,h$
\end{itemize}
Using the induction hypothesis on $\wtpi{\Ctx_i, \ep{s}{\role_i} : S_i}{P_i}$ we deduce that there exist 
$\Pnf_i$ such that $P_i \pcong \Pnf_i$ for $i=1,\dots,h$.
By definition of thread normal form, it must be the case that $\Pnf_i = \pcast{\seqof{u_i}} \Ppar_i$ 
for some $\seqof{u_i}$ and $\Ppar_i$. Let $\seqof{v_i}$ be the same sequence as $\seqof{u_i}$
 except that occurrences of $\ep{s}{\role_i}$ have been removed.
We conclude by taking $\Pnf \eqdef \pcast{\seqof{v_1}\dots\seqof{v_h}}\pres{s}{\Ppar_1 \parop \dots \parop \Ppar_h}$ 
and using the fact that $\pcong$ is a pre-congruence and observing that
	\[
		\begin{array}{rcll}
			P & = & \pres{s}{P_1 \parop \cdots \parop P_h} & \text{by definition of $P$}
			\\
			& \pcong & \pres{s}{\Pnf_1 \parop \cdots \parop \Pnf_h} & \text{using the induction hypothesis}
			\\
			& = & \pres{s}{\pcast{\seqof{u_1}}\Ppar_1 \parop \cdots \parop \pcast{\seqof{u_h}}\Ppar_h}
			& \text{by \Cref{def:tnf_multi}}
			\\
			& \pcong & \pcast{\seqof{v_1} \dots \seqof{v_h}}\pres{s}{\Ppar_1 \parop \cdots \parop \Ppar_h}
			& \text{by \refrule{sm-cast-new},} \\
			& & & \text{\refrule{sm-cast-swap}, \refrule{sm-par-comm}}
			\\
			& = & \Pnf & \text{by definition of $\Pnf$}
		\end{array}
	\]

\proofrule{tm-cast}
Then there exist $u$, $Q$, $\Ctx'$, $S$ and $T$ such that
\begin{itemize}
\item $P = \pcast{u} Q$
\item $\Ctx = \Ctx', u : S$
\item $\wtpi{\Ctx', u : T}{Q}$
\item $S \subt T$
\end{itemize}
Using the induction hypothesis on $\wtpi{\Ctx', u : T}{Q}$ we deduce that there exists $\Qnf$ such that $Q \pcong \Qnf$.
We conclude by taking $\Pnf \eqdef \pcast{u}\Qnf$ using the fact that $\pcong$ is a pre-congruence.
\end{proof}

\begin{lemma}[Proximity]
  \label{lem:proximity_multi}
  If $s\in\fn{P} \setminus \bn\PCtxC$, then $\pres{s}{\PCtxC[P] \parop \procs{Q}} \pcong \PCtxD[\pres{s}{P \parop \procs{Q}}]$ for some $\PCtxD$.
\end{lemma}
\begin{proof}
By induction on the structure of $\PCtxC$ and by cases on its shape.

\proofcase{Case $\PCtxC = \Hole$}
%
We conclude by taking $\PCtxD \eqdef \Hole$ using the reflexivity of $\pcong$. 

\proofcase{Case $\PCtxC = \pres{t}{\procs{P'} \parop \PCtxC' \parop \procs{Q'}}$}
%
From the hypothesis $s \in \fn{P} \setminus \bn\PCtxC$ we deduce $s \ne t$ and $s \in \fn{P} \setminus \bn{\PCtxC'}$.
Using the induction hypothesis and \refrule{sm-par-comm} we deduce that there exists $\PCtxD'$ such that 
$\pres{s}{\PCtxC'[P] \parop \procs{Q}} \pcong \PCtxD'[\pres{s}{P \parop \procs{Q}}]$.
Take $\PCtxD \eqdef \pres{t}{\PCtxD' \parop \procs{P'} \parop \procs{Q'}}$. We conclude
  \[
    \begin{array}{rcll}
      \pres{s}{\PCtxC[P] \parop \procs{Q}}
      & = & \pres{s}{\pres{t}{\procs{P'} \parop \PCtxC'[P] \parop \procs{Q'}} \parop \procs{Q}}
      & \text{by definition of $\PCtxC$}
      \\
      & \pcong & \pres{s}{\procs{Q} \parop \pres{t}{\PCtxC'[P] \parop \procs{P'} \parop \procs{Q'}}}
      & \text{by \refrule{sm-par-comm}}
      \\
      & \pcong & \pres{t}{\pres{s}{\procs{Q} \parop \PCtxC'[P]} \parop \procs{P'} \parop \procs{Q'}}
      & \text{by \refrule{sm-par-assoc}} \\
      & & & \text{and $s \in \fn{\PCtxC'[P]}$}
      \\
      & \pcong & \pres{t}{\pres{s}{\PCtxC'[P] \parop \procs{Q}} \parop \procs{P'} \parop \procs{Q'}}
      & \text{by \refrule{sm-par-comm}}
      \\
      & \pcong & \pres{t}{\PCtxD'[\pres{s}{P \parop \procs{Q}}] \parop \procs{P'} \parop \procs{Q'}}
      & \text{by induction hypothesis}
      \\
      & = & \PCtxD[\pres{s}{P \parop \procs{Q}}]
      & \text{by definition of $\PCtxD$}
    \end{array}
  \]
  
\proofcase{Case $\PCtxC = \pcast{\ep{t}{\role}}\PCtxC'$ and $s \ne t$}
%
Using the induction hypothesis we deduce that there exists $\PCtxD'$ such that 
$\pres{s}{\PCtxC'[P] \parop \procs{Q}} \pcong \PCtxD'[\pres{s}{P \parop \procs{Q}}]$.
Take $\PCtxD \eqdef \pcast{\ep{t}{\role}}\PCtxD'$. We conclude
  \[
    \begin{array}{rcll}
      \pres{s}{\PCtxC[P] \parop \procs{Q}}
      & = & \pres{s}{\pcast{\ep{t}{\role}}\PCtxC'[P] \parop \procs{Q}}
      & \text{by definition of $\PCtxC$}
      \\
      & \pcong & \pcast{\ep{t}{\role}}\pres{s}{\PCtxC'[P] \parop \procs{Q}}
      & \text{by \refrule{sm-cast-swap} and $t \ne s$}
      \\
      & \pcong & \pcast{\ep{t}{\role}}\PCtxD'[\pres{s}{P \parop \procs{Q}}]
      & \text{using the induction hypothesis}
      \\
      & = & \PCtxD[\pres{s}{P \parop \procs{Q}}]
      & \text{by definition of $\PCtxD$}
    \end{array}
  \]

\proofcase{Case $\PCtxC = \pcast{\ep{s}{\role}}\PCtxC'$}
%
Using the induction hypothesis we deduce that there exists $\PCtxD$ such that 
$\pres{s}{\PCtxC'[P] \parop \procs{Q}} \pcong \PCtxD[\pres{s}{P \parop \procs{Q}}]$.
We conclude
  \[
    \begin{array}[b]{rcll}
      \pres{s}{\PCtxC[P] \parop \procs{Q}}
      & = & \pres{s}{\pcast{\ep{s}{\role}}\PCtxC'[P] \parop \procs{Q}}
      & \text{by definition of $\PCtxC$}
      \\
      & \pcong & \pres{s}{\PCtxC'[P] \parop \procs{Q}}
      & \text{by \refrule{sm-cast-new}}
      \\
      & \pcong & \PCtxD[\pres{s}{P \parop \procs{Q}}]
      & \text{using the induction hypothesis}
    \end{array}
    \qedhere
  \]
\end{proof}


\begin{lemma}[Quasi - Deadlock Freedom]
	\label{lem:dl_freedom_multi}
	If $\wtpn\Measure\EmptyCtx\Pnf$, then $\Pnf = \pdone$ or $\Pnf \pcong \Qnf$ for some $\Qnf$ in proximity normal form.
\end{lemma}
\begin{proof}
By induction on the derivation of $\wtpn\MeasureM\EmptyCtx\Pnf$ we deduce that 
$\Pnf$ consists of $s_1,\dots,s_h$ sessions and $\sum_{i=1}^h k_i - h + 1$ threads where 
$k_i$ is the number of roles in $s_i$. The scenarios in which no communication is possible are those 
in which for each session $s_i$ there are less than $k_i$ $s_i$-threads. If we assume that 
for each $s_i$ there are $k_i - 1$ threads, then we obtain 
\[
\sum_{i=1}^h k_i - h + 1 - \sum_{i=1}^h(k_i - 1) = \sum_{i=1}^h k_i - h + 1 - \sum_{i=1}^hk_i + h = 1
\]
$s_i$-thread for some $s_i$; hence, there exist $k_i$ $s_i$-threads. 
In other words, there exist $\PCtxD,\PCtxC_1,\dots,\PCtxC_{k_i}$ and $\Pth_1,\dots,\Pth_{k_i}$ $s_i$-threads 
such that 
\[
	\Pnf = \PCtxD[\pres{s_i}{\PCtxC_1[\Pth_1] \parop \cdots \parop \PCtxC_{k_i}[\Pth_{k_i}]}]
\] 
We conclude
	\[
		\begin{array}{rcl}
			\Pnf & = & \PCtxD[\pres{s_i}{\PCtxC_1[\Pth_1] \parop \cdots \parop \PCtxC_{k_i}[\Pth_{k_i}]}] \\
			& & \hfill\text{by definition of $\Pnf$}
			\\
			& \pcong & \PCtxD[\PCtxD_1[\pres{s_i}{\Pth_1 \parop \PCtxC_2[\Pth_2] \parop \cdots \parop \PCtxC_{k_i}[\Pth_{k_i}]}]] \\
			& & \hfill\text{by \Cref{lem:proximity_multi}}
			\\
			& \pcong & \PCtxD[\PCtxD_1[\pres{s_i}{\PCtxC_2[\Pth_2] \parop \Pth_1 \parop \cdots \parop \PCtxC_{k_i}[\Pth_{k_i}]}]] \\
			& & \hfill\text{by \refrule{sm-par-comm}}
			\\
			& \dots \\
			& \pcong & \PCtxD[\PCtxD_1[\PCtxD_2[ \dots \PCtxD_{k_i}[\pres{s_i}{\Pth_{k_i} \parop \cdots \parop \Pth_2 \parop \Pth_1}] \dots ]]] \\
			& & \hfill\text{for some $\PCtxD_2,\dots,\PCtxD_{k_i}$ by \Cref{lem:proximity_multi}}
			\\
			& \eqdef & \Qnf
		\end{array}
	\]
	The fact that $\Qnf$ is in thread normal form follows from the observation
	that $\Pnf$ does not have unguarded casts (it is a closed process in thread
	normal form) so the pre-congruence rules applied here and in
	\Cref{lem:proximity_multi} do not move casts around. We conclude that $\Qnf$ is in
	proximity normal form by its shape.
\end{proof}

As mentioned at the beginning, 
\Cref{lem:dl_freedom_multi} is dubbed ``quasi-deadlock freedom'' because it does not
say that $\Qnf$ reduces. Indeed, a process in proximity normal form is only
\emph{ready to communicate} thanks to its shape (see reduction rules). We can
prove that a well typed process of such kind actually reduces by observing that
\refrule{tm-par} requires that the involved session is coherent. This result is
the key ingredient for proving \Cref{lem:pnf_helpful_direction_multi}.
