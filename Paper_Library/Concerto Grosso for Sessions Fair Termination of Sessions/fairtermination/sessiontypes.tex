\beginalto
%
In this section we instantiate \emph{fair termination} in session type based scenarios.
%
First, we consider binary sessions and then we generalize to the multiparty case.
We instantiate \Cref{ex:bsc} as well since it will be the running example
in \Cref{ch:ft_bin} and \Cref{ch:ft_multi}.
%
For what concerns the used syntaxes and labeled transition systems, we refer to \Cref{sec:st}.

\begin{definition}[Compatibility of a binary session]
	\label{def:compatibility}
	We say that $S$ and $T$ are \emph{compatible}, notation
  $S \compatible T$, if $\Sys\S\T \wred \Sys{S'}{T'}$ implies
  $\Sys{S'}{T'} \wred \Sys{\End[\co\Pol]}{\End[\Pol]}$ for some $\Pol$.
\end{definition}

\begin{definition}[Coherence of a session map]
	\label{def:coherence}
  We say that a session map $M$ is \emph{coherent}, notation $\coherent M$, if $M \wred N$
  implies $N \wlred{\In\terminated}$.
\end{definition}

The term ``coherence'' is borrowed from \cite{CarboneLMSW16,CarboneMontesiSchurmannYoshida17}, 
although the property is actually stronger than the one of 
\cite{CarboneLMSW16,CarboneMontesiSchurmannYoshida17}
as it entails fair termination of multiparty sessions 
through \cref{thm:fair_termination}.
%
In particular, if we consider the reduction system whose states are session maps
and whose reduction relation is $\lred\tau$, then $\coherent M$ implies $M$
fairly terminating. The same applies to a $\S \compatible \T$ binary session.

\begin{remark}[Successful fair termination]
	The properties stated in \Cref{def:compatibility,def:coherence}
	are stronger than fair termination. Indeed, a \emph{deadlocked} session
	is \emph{fairly terminating} as well as it cannot reduce. 
	\Cref{def:compatibility,def:coherence} are equivalent to a
	\emph{successful} form of fair termination since we ask that the
	involved sessions correctly terminate. 
\end{remark}

Now we can revise \Cref{ex:bsc} in both scenarios.

\begin{example}[Binary Buyer - Seller - Carrier]
	\label{ex:bin_bsc}
	Consider the types $\S_b,\S_s$ and $\T_s,\T_c$ that model 
	the two binary sessions connecting \actor{buyer} - \actor{seller}
	and \actor{seller} - \actor{carrier}, respectively, according to 
	\Cref{ex:bsc}.
	\[
	\begin{array}{lcllcl}
		\S_b & = & \Out\tadd.\S_b \choice \Out\tpay.\End[\Out] \qquad & \qquad \T_s & = & \Out\tship.\End[\Out] \\
		\S_s & = & \In\tadd.\S_s \choice \In\tpay.\End[\In] \qquad & \qquad \T_c & = & \In\tship.\End[\In]
	\end{array}
	\]
	We focus on the session $\Sys{\S_b}{\S_s}$. According to
	\Cref{def:compatibility}, $\S_b$ and $\S_s$ are \emph{compatible}
	because, no matter how many times the \actor{buyer} adds an item to the cart,
	he always has the possibility to $\tpay$ the amount. The infinite
	run in which the \actor{buyer} only adds items is \emph{unfair} according to
	\Cref{def:fair_run}.
	
	Note that $\S_b \compatible \S_s$ implies \emph{progress}
	of the session $\Sys{\T_s}{\T_c}$. Indeed, the communication between 
	the \actor{seller} and the \actor{carrier} takes place after the \actor{buyer}
	sends $\tpay$ which is guaranteed to happen by $\S_b \compatible \S_s$. 
	Furthermore, although this example can be easily adapted to the multiparty case, 
	it shows a very simple and realistic scenario in which more sessions are
	involved, no matter if they are binary or multiparty.
	%
	\eoe
\end{example}

\Cref{ex:bin_bsc} clearly holds in the multiparty context as well,
now we can model the same communication protocol using a single
multiparty session.

\begin{example}[Multiparty Buyer - Seller - Carrier]
\label{ex:bsc_ty_multi}
  Consider the session types
  \[
  \begin{array}{l@{~}c@{~}l}
    S_b & = & \seller\Out\tadd.S_b + \seller\Out\tpay.\End[\Out]\\
    S_s & = & \buyer\In\tadd.S_s + \buyer\In\tpay.\carrier\Out\tship.\End[\Out]\\
    S_c & = & \seller\In\tship.\End[\In]
  \end{array} 
  \]
  which describe the behavior of the three participants
  in \cref{ex:bsc}. The session map $\Map{\buyer}{S_b} \parop
  \Map{\seller}{S_s} \parop  \Map{\carrier}{S_c}$ is \emph{coherent}. 
  The explanation is just the same as the one given in \Cref{ex:bin_bsc}.
  %
  \eoe
\end{example}

Notably, in the multiparty context, fair termination implies \emph{progress}
of all non terminated participants (see $\carrier$ in \Cref{ex:bsc_ty_multi}).

