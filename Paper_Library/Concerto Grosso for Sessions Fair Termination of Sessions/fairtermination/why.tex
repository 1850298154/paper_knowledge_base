\beginbass
%
The reader might wonder why we focus on fair termination instead of considering
some fair version of progress. There are three reasons why we think that fair
termination is overall more appropriate than just progress.
%
First of all, ensuring that sessions (fairly) terminate is consistent with the
usual interpretation of the word ``session'' as an activity that lasts for a
\emph{finite amount of time}, even when the maximum duration of the activity is
not known \emph{a priori}.
%
Second, \emph{fair termination implies progress} when it is guaranteed along
with the usual safety properties of sessions. Indeed, if the session eventually
terminates, it must be the case that any non-terminated participant (think of
the carrier waiting for a \textit{"ship"} message) is guaranteed to eventually make
progress, even when such progress \emph{depends} on choices made by others
(like the buyer sending \textit{"pay"} to the seller).
%
Last but not least, \emph{fair session termination enables compositional
reasoning} in the presence of multiple sessions. This is not true for progress:
if an action on a session $s$ is blocked by actions on a different session $t$,
then knowing that the session $t$ enjoys progress does not necessarily guarantee
that the action on $s$ will eventually be performed (the interaction on $t$
might continue forever). On the contrary, knowing that $t$ fairly terminates
guarantees that the action on $s$ will eventually be scheduled and performed, so
that $s$ may in turn progress towards termination.