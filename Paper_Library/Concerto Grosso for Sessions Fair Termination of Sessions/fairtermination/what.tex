\beginbass
%
Fair termination is \emph{termination} under a \emph{fairness assumption}. We decided to
investigate such property since current type systems cannot deal with those scenarios 
in which the communication between two entities \emph{depends} on the communication between 
others.
To explain why,
we show a paradigmatic scenario that we will instantiate in different ways in the next chapters.

\begin{example}[Buyer - Seller - Carrier]
	\label{ex:bsc}
	\brk
	Consider the interaction between the following three entities:
	\begin{itemize}
	\item \actor{buyer}: he can tell a \actor{seller} either that he adds an item to the cart or that he pays the total amount.
	We assume that the messages being sent are $\tadd$ and $\tpay$, respectively
	\item \actor{seller}: if the \actor{buyer} decides to pay the amount (\ie $\tpay$ is received) he contacts the \actor{carrier}
	to ship the items. The message being sent is $\tship$
	\item \actor{carrier}: he sends the items as soon as he is contacted by the \actor{seller} (\ie $\tship$ is received)
	\end{itemize}
\end{example}

At the moment we do not care about the technicalities on how the three actors interact.
What makes this scenario somewhat difficult to reason
about is that \emph{the progress of the carrier is not unconditional but depends
on the choices performed by the buyer}: the carrier can make progress only if
the buyer eventually pays the seller.

The \emph{fairness assumption} that we used can be informally stated as
\begin{center}
	\emph{If termination is always possible, then it is inevitable}
\end{center}

Note that \Cref{ex:bsc} admits an infinite execution in which the \actor{buyer} only
adds items to the cart and never pays the amount. Such execution is \emph{unfair}
according to our assumption since the \actor{buyer} can always send $\tpay$ (and terminate)
but he always avoid to do so.