\beginalto
%
Since the notion of fair termination will apply to several different entities in this chapter
(session types, binary and multiparty sessions, processes) here we formally define it for a generic
reduction system. Later on we will show various instantiations of this
definition.

\begin{definition}[Reduction System]
	A \emph{reduction system} is a pair $(\States, {\red})$ where 
	\begin{itemize}
	\item $\States$ is a set of \emph{states}
	\item ${\red} \subseteq \States \times \States$ is a \emph{reduction relation}
	\end{itemize}
	We adopt the following notation:
	%
	we let $C$ and $D$ range over states;
	%
	we write $C \red$ if there exists $D \in \States$ such that $C \red D$; we write
	$C \nred$ if not $C \red$; we write $\wred$ for the reflexive, transitive
	closure of $\red$.
	%
	We say that $D$ is \emph{reachable} from $C$ if $C \wred D$.
\end{definition}

As an example, the reduction system $(\set{A,B}, \set{(A,A),(A,B)})$
models an entity that can be in two states, $A$ or $B$, and such that the entity
may perform a reduction to remain in state $A$ or a reduction to move from state
$A$ to state $B$. To formalize the evolution of an entity from a particular
state we define \emph{runs}.

\begin{definition}[runs and maximal runs]
  \label{def:run}
  A \emph{run} of $C$ is a (finite or infinite) sequence
  $C_0C_1\dots C_i\dots$ of states such that $C_0 = C$ and
  $C_i \red C_{i+1}$ for every valid $i$. A run is \emph{maximal} if
  either it is infinite or if its last state $C_n$ is such that
  $C_n \nred$.
\end{definition}

Hereafter we let $\run$ range over runs. Each run in the previously defined
reduction system is either of the form $A^n$ -- a finite sequence of $A$ -- or
of the form $A^nB$ -- a finite sequence of $A$ followed by one $B$ -- or
$A^\omega$ -- an infinite sequence of $A$. Among these, the runs of the form
$A^nB$ and $A^\omega$ are maximal, whereas no run of the form $A^n$ is maximal.

We now use runs to define different termination properties of states:
%
we say that $C$ is \emph{weakly terminating} if there exists a maximal run of
$C$ that is finite;
%
we say that $C$ is \emph{terminating} if every maximal run of $C$ is finite;
%
we say that $C$ is \emph{diverging} if every maximal run of $C$ is infinite.
%
\emph{Fair termination} \citep{Francez86} is a termination property that only
considers a subset of all (maximal) runs of a state, those that are considered
to be ``realistic'' or ``fair'' according to some fairness assumption.
%
The assumption that we make in this work, and that we stated in words in
\Cref{sec:ft_intro}, is formalized thus:

\begin{definition}[Fair run]
  \label{def:fair_run}
  A run is \emph{fair} if it contains finitely many weakly terminating states.
  Conversely, a run is \emph{unfair} if it contains infinitely many weakly
  terminating states.
\end{definition}

Continuing with the previous example, the runs of the form $A^n$ and $A^nB$ are
fair, whereas the run $A^\omega$ is unfair. In general, an unfair run is an
execution in which termination is always within reach, but is never reached.

A key requirement of any fairness assumption is that it must be possible to
extend every finite run to a maximal fair one. This property is called
\emph{feasibility} \citep{AptFrancezKatz87,GlabbeekHofner19} or \emph{machine
closure} \citep{Lamport00}.
%
It is easy to see that our fairness assumption is feasible:

\begin{lemma}
  \label{lem:feasibility}
  If $\run$ is a finite run, then there exists $\run'$ such that $\run\run'$ is
  a maximal fair run.
\end{lemma}
\begin{proof}
    Let $D$ be the last state of $\run$. We distinguish two possibilities: if $D$
    is weakly terminating, then there exists a finite maximal run $D\run'$ of $D$;
    if $D$ is diverging, then there exists an infinite run $D\run'$ of $D$ such
    that no state in $\run'$ is weakly terminating. In both cases we conclude by
    noting that $\run\run'$ is a maximal fair run. 
\end{proof}

Fair termination is finiteness of all maximal fair runs:

\begin{definition}[Fair Termination]
  \label{def:fair_termination}
  We say that $C$ is \emph{fairly terminating} if every maximal fair run of $C$
  is finite.
\end{definition}

In the reduction system given above, $A$ is fairly terminating. Indeed, all the
maximal runs of the form $A^nB$ are finite whereas $A^\omega$, which is the only
infinite run of $A$, is unfair.

For the particular fairness assumption that we make, it is possible to provide a
sound and complete characterization of fair termination that does not mention
fair runs. This characterization will be useful to relate fair termination with
the notion of correct session (\Cref{def:compatibility,def:coherence}) 
and the soundness property of the type systems we are going 
to introduce in \Cref{pt:type_systems}. 

\begin{theorem}
  \label{thm:fair_termination}
  Let $(\States, {\red})$ be a reduction system and $C\in\States$. Then $C$ is
  fairly terminating if and only if every state reachable from $C$ is weakly
  terminating.
\end{theorem}
\begin{proof}
	
  \proofcase{$\Rightarrow$} Let $D$ be a state reachable from $C$. That is, there exists a
  finite run $\run$ of $C$ ending with $D$. By \Cref{lem:feasibility} we deduce
  that this run can be extended to a maximal fair one $\run\run'$. From the
  hypothesis that $C$ is fairly terminating we deduce that $\run\run'$ is
  finite. Hence, $D$ is weakly terminating.

  \proofcase{$\Leftarrow$} Let $C_0C_1\dots$ be an infinite fair run of $C$.  Using the
  hypothesis we deduce that each $C_i$ is weakly terminating, which is absurd.
  Hence, either there are no maximal fair runs or 
  every maximal fair run of $C$ is finite, but the first case is not possible 
  by \Cref{lem:feasibility}, thus $C$ is fairly terminating. 
\end{proof}

\begin{remark}[Fair Reachability of Predicates]
  Most fairness assumptions have the form ``if \emph{something} is infinitely
  often possible then \emph{something} happens infinitely often'' and, in this
  respect, our formulation of fair run (\cref{def:fair_run}) looks slightly
  unconventional. However, it is not difficult to realize that
  \cref{def:fair_run} is an instance of the notion of fair reachability of
  predicates as defined by \cite[Definition 3]{QueilleSifakis83}. 
  According to Queille and Sifakis, a run $\run$ is fair
  with respect to some predicate $\StateSet \subseteq \States$ if, whenever in
  $\run$ there are infinitely many states from which a state in $\StateSet$ is
  reachable, then in $\run$ there are infinitely many occurrences of states in
  $\StateSet$. When we take $\StateSet$ to be $\nred$, that is the set of
  terminated states that do not reduce, pretending that irreducible states
  should occur infinitely often in the run is nonsensical. So, the fairness
  assumption boils down to assuming that such states should \emph{not} be
  reachable infinitely often, which is precisely the formulation of
  \cref{def:fair_run}.
  %
  \eor
\end{remark}