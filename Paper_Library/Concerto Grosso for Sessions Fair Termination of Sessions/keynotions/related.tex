\beginalto
%
Now we have all the ingredients for discussing about some well known properties
that can be found in the literature. 
Notably, such properties have been studied both on the pure $\pi$-calculus and
on session-based calculi. We split the discussion according the property and
the scenario under analysis.

\paragraph{Termination of Binary Sessions.}
%
Termination is a liveness property that can be guaranteed when finite session
types are considered \citep{PerezCairesPfenningToninho12}. As soon as infinite
session types are considered, many session type systems weaken the guaranteed
property to deadlock freedom.
\cite{LindleyMorris16} define a type system for a functional
language with session primitives and recursive session types that is strongly
normalizing.

\paragraph{Liveness Properties in the $\pi$-Calculus.}
%
\cite{Kobayashi02} defines a behavioral type system that guarantees
lock freedom in the $\pi$-calculus. Lock freedom is a liveness property akin to
progress for sessions, except that it applies to \emph{any} communication
channel (shared or private). 
As a paradigmatic example of lock we can consider a variant of \Cref{ex:bsc}
in which the \actor{buyer} never pays the products. Thus, the \actor{carrier}
is locked.
%
\cite{Padovani14} adapts and extends the type system of
\cite{Kobayashi02} to enforce lock freedom in the \emph{linear
$\pi$-calculus} \citep{KobayashiPierceTurner99}, into which binary sessions can
be encoded \citep{DardhaGiachinoSangiorgi17}.
All of these works annotate types with numbers representing finite upper bounds
to the number of interactions needed to unblock a particular input/output
action.

\paragraph{Deadlock Freedom.}
%
Deadlock freedom is a safety properties according to which a communication
cannot get stuck. A paradigmatic example of deadlocked process is (we adopt binary session
type syntax)
\[
\x\iact\Tag[true].\y\oact\Tag[false]\dots \parop \y\iact\Tag[false].\x\oact\Tag[true]\dots
\]
where we have a couple of processes running in parallel.
The former is waiting for $\Tag[true]$ on $\x$ which is sent by the second 
as soon as it receives $\Tag[false]$ by the former.

\cite{Kobayashi02,Padovani14} guarantee deadlock freedom using the same technique for
lock freedom.
There exist a number of session based type systems inspired by linear logic
\citep{Wadler14,CairesPfenningToninho16,LindleyMorris16,CarboneLMSW16,CarboneMontesiSchurmannYoshida17}
in which deadlock freedom is dealt with
by using a \emph{process composition} rule that resembles the \emph{cut} rule.

\paragraph{Liveness Properties of Multiparty Sessions.}
%
The enforcement of liveness properties has always been a key aspect of session
type systems, although previous works have almost exclusively focused on
progress rather than on (fair) termination.
%
\cite{ScalasYoshida19} define a general framework for
ensuring safety and liveness properties of multiparty sessions. In particular,
they define a hierarchy of three liveness predicates to characterize ``live''
sessions that enjoy progress. They also point out that the coarsest liveness
property in this hierarchy, which is the one more closely related to fair
termination, cannot be enforced by their type system.
%
The work of \cite{GlabbeekHofnerHorne21} presents a type
system for multiparty sessions that ensures progress and is not only sound but
also complete.
%
\cite{CarboneDM14} characterize progress in terms 
of the standard notion of lock-freedom by using catalysers.

