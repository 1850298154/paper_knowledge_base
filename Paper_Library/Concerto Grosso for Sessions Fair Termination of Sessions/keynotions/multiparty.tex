\beginbass
%
A \emph{multiparty} session type describes the communication protocol between
at least two participants. For this reason, the types associated to each
endpoint are exactly those presented in \Cref{ssec:bin} with the addition of
\emph{roles} ($\rolep$,$\roleq\dots)$. Indeed, a binary session can be considered
as the simplest multiparty one with only two participants; hence, the roles
can be omitted in the types.

\begin{definition}[Local Types]
	A \emph{local session type} is a regular tree \cite{Courcelle83} coinductively
	generated by the productions
	\[
  	S, T, U, V ::= \End \mid \Tags\rolep\Pol\Tag_i.S_i \mid \rolep\Pol{S}.T
	\]
\end{definition}

The session type $\End$ describes the behavior of a process that sends or receives
a termination signal.
%
The session type $\Tags\rolep\Pol\Tag_i.S_i$ describes the behavior of a process
that sends to or receives from the participant $\rolep$ one of the tags $\Tag_i$
and then behaves according to $S_i$. Note that the source or destination role
$\rolep$ and the polarity $\Pol$ are the same in every branch. We require that
$I$ is not empty and $i, j \in I$ with $i \ne j$ implies $\Tag_i \ne \Tag_j$.
Occasionally we write $\rolep\Pol\Tag_1.S_1 + \cdots + \rolep\Pol\Tag_n.S_n$
instead of $\Tags[i=1]^n \rolep\Pol\Tag_i.S_i$.
%
Finally, a session type $\rolep\Pol{S}.T$ describes the behavior of a process
that sends to or receives from the participant $\rolep$ an endpoint of type $S$
and then behaves according to $T$.
%
We often specify infinite session types as solutions of equations of the form $S
= \cdots$ where the metavariable $S$ may occur on the right hand side of $=$
guarded by at least one prefix. A regular tree satisfying such equation is
guaranteed to exist and to be unique~\cite{Courcelle83}.

In order to describe a whole multiparty session at the level of types we
introduce the notion of \emph{session map}.

\begin{definition}[session map]
  \label{def:session_map}
  A \emph{session map} is a finite, partial map from roles to session types
  written $\set{\Map{\role_i} S_i}_{i\in I}$.  We let $M$ and $N$ range over
  session maps, we write $\dom{M}$ for the domain of $M$, we write $M \parop N$
  for the union of $M$ and $N$ when $\dom{M} \cap \dom{N} = \emptyset$, and we
  abbreviate the singleton map $\set{\Map\rolep{S}}$ as $\Map\rolep{S}$.
\end{definition}

Again, we describe the evolution of a session at the level of types by means of a
\emph{labeled transition system} for session maps. Labels are generated by the
grammar below:
\[
  \textbf{Label}
  \qquad
  \ell ::= \tau \mid \action
  \qquad\qquad
  \textbf{Action}
  \qquad
  \actionA, \actionB ::= \Pol\terminated \mid \Map\rolep{\roleq\Pol\Tag} \mid \Map\rolep{\roleq\Pol S}
\]

The label $\tau$ represents either an internal action performed by a participant
independently of the others or a synchronization between two participants.
%
The labels of the form $\Pol\terminated$ describe the input/output of
termination signals, whereas the labels of the form $\Map\rolep\roleq\Pol\Tag$
and $\Map\rolep\roleq\Pol S$ represent the input/output of a tag $\Tag$ or of an
endpoint of type $S$. 

\begin{figure}[t]
\framebox[\textwidth]{
  \begin{mathpar}
    \inferrule[lm-end]{ }{
      \Map\role\End \xlred{\Pol\terminated} \Map\role\End
    } \defrule[lm-end]{}
    \and
    \inferrule[lm-channel]{ }{
      \Map\rolep\roleq\Pol U.S \xlred{\Map\rolep\roleq\Pol U} \Map\rolep{S}
    } \defrule[lm-channel]{}
    \and
    \inferrule[lm-pick]{ }{
      \textstyle
      \Map\rolep{\Tags\roleq\Out\Tag_i.S_i}
      \lred\tau
      \Map\rolep{\roleq\Out\Tag_k.S_k}
    }
    ~k\in I \defrule[lm-pick]{}
    \and
    \inferrule[lm-tag]{ }{
      \textstyle
      \Map\rolep{\Tags\roleq\Pol\Tag_i.S_i}
      \xlred{\Map\rolep{\roleq\Pol\Tag_k}}
      \Map\rolep{S_k}
    }
    ~ k \in I \defrule[lm-tag]{}
    \and
    \inferrule[lm-tau]{
      M \lred\tau M'
    }{
      M \parop N \lred\tau M' \parop N
    } \defrule[lm-tau]{}
    \and
    \inferrule[lm-terminate]{
      M \lred{\In\terminated} M'
      \\
      N \lred{\Out\terminated} N'
    }{
      M \parop N \lred{\In\terminated} M' \parop N'
    } \defrule[lm-terminate]{}
    \and
    \inferrule[lm-sync]{
      M \lred{\co\action} M'
      \\
      N \lred{\action} N'
    }{
      M \parop N \lred\tau M' \parop N'
    } \defrule[lm-sync]{}
  \end{mathpar}
  }
  \caption{Labeled Transition System for Session Maps}
  \label{fig:lts_multi} 
\end{figure}

The labeled transition system is defined by the rules in \Cref{fig:lts_multi}, most of
which are straightforward. Rule \refrule{lm-pick} models the fact that the
participant $\rolep$ may internally choose one particular tag $\Tag_k$ before
sending it to $\roleq$. The chosen tag is not negotiable with the receiver.
%
Rule \refrule{lm-terminate} models termination of a session. A session terminates
when there is exactly one participant waiting for the termination signal and all
the others are sending it. This property follows from a straightforward
induction on the derivation of $M \lred{\In\terminated} N$ using
\refrule{lm-terminate} and \refrule{lm-end}.

The existence of a single participant waiting for the termination signal ensures
that there is a uniquely determined continuation process after the session has
been closed.
%
Finally, rule \refrule{lm-sync} models the synchronization between two
participants performing complementary actions. The complement of an action
$\action$, denoted by $\co\action$, is the partial operation defined by the
equations
\[
  \co{\Map\rolep\roleq\Pol\Tag} \eqdef \Map\roleq\rolep{\co\Pol}\Tag
  \qquad
  \co{\Map\rolep\roleq\Pol S} \eqdef \Map\roleq\rolep{\co\Pol} S
\]
where $\co\Pol$ denotes the complement of the polarity $\Pol$. The complement of
actions of the form $\Pol\terminated$ is undefined, so rule \refrule{lm-sync}
cannot be applied to terminated sessions.
%
Hereafter we write $\wred$ for the reflexive, transitive closure of $\lred\tau$
and $\wlred\action$ for the composition ${\wred}{\lred\action}$.