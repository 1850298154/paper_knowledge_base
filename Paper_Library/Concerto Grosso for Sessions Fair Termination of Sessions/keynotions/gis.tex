\beginalto
%
A different way of specifying (and enforcing) properties is by means of
\emph{inference systems} \citep{Aczel77}. Inference systems admit two natural
interpretations, an inductive and a coinductive one respectively corresponding
to the least and the greatest fixed points of their associated inference
operator. Unlike the $\mu$-calculus, however, they lack the flexibility of
mixing their different interpretations since the inference rules are interpreted
either all inductively or all coinductively. For this reason, it is generally
difficult to specify properties that combine safety and liveness aspects by
means of a single inference system.
%
\emph{Generalized Inference Systems} (\GISs)
\citep{AnconaDagninoZucca17,Dagnino19} admit a wider range of interpretations,
including intermediate fixed points of the inference operator associated with
the inference system different from the least or the greatest one.  This is made
possible by the presence of \emph{corules}, whose purpose is to provide an
inductive definition of a space within which a coinductive definition is used.
%
Although \GISs do not achieve the same flexibility of the modal $\mu$-calculus
in combining different fixed points, they allow for the specification of
properties that can be expressed as the intersection of a least and a greatest
fixed point. This feature of \GISs resonates well with one of the fundamental
results in model checking stating that every property can be decomposed into a
conjunction of a safety property and a liveness
one \citep{AlpernSchneider85,AlpernSchneider87,BaierKatoen08}.

\subsection{Definitions and interpretations}  

\beginbass
%
We first summarize the main notions and definitions of inference systems.
Then we generalize them by introducing (meta)corules. Given an inference
system, one has to take an \emph{interpretation} that identifies
the set of defined judgments.
When dealing with a generalized inference system, such interpretation is obtained
by combining both the inductive and the coinductive ones and by 
taking into account corules in the proper way.

\begin{definition}[Inference Systems \& Rules]
	\label{def:infsys}
	An \emph{inference system} \citep{Aczel77} $\mis$ over a \emph{universe}
	$\universe$ of \emph{judgments} is a set of \textit{rules}, which are pairs
	$\RulePair\prem\judg$ where $\prem\subseteq\universe$ is the set of
	\emph{premises} of the rule and $\judg\in\universe$ is the \emph{conclusion} of
	the rule. A rule without premises is called \emph{axiom}. Rules are typically
	presented using the syntax
	\[
  	\Rule\prem\judg
	\]
	where the line separates the premises (above the line) from the
	conclusion (below the line).
\end{definition}

Since rules may be infinite, inference systems are usually described 
by using \emph{meta-rules} written in a meta-language. 
Informally, meta-rules group rules according to their shape.

\begin{definition}[Metarule]
	A \emph{meta-rule} $\mr$  is a tuple \ple{\MCtx,\mpr,\mj,\sd} where 
 	\begin{itemize}
 	\item $\MCtx$ is a set called \emph{context}
 	\item $\mpr = \ple{\mj_1,\ldots,\mj_n}$ is a finite sequence of functions 
 	of type $\MCtx\rightarrow\universe$ called \emph{(meta-)premises}
 	\item $\mj$ is a function of type $\MCtx\rightarrow\universe$ called \emph{(meta-)conclusion} 
 	\item $\sd$ is a subset of $\MCtx$  called \emph{side condition}
 	\end{itemize}
\end{definition}

Given $\mr=\ple{\MCtx,\ple{\mj_1,\ldots,\mj_n},\mj,\sd}$,
the inference system $\Ground{\mr}$ \emph{denoted by $\mr$} is defined by:\\
\[\Ground{\mr}=\{\RulePair{\{\mj_1(\mc),\ldots,\mj_n(\mc)\}}{\mj(\mc)}\mid c\in\sd\}\]

\begin{definition}[Meta System]
	An \emph{inference meta-system} $\mis$ is a set of meta-rules. 
	The inference system $\Ground{\mis}$ is the union of $\Ground{\mr}$ for all $\mr\in\mis$.  
\end{definition}

A \emph{predicate} on $\universe$ is any subset of $\universe$. An
\emph{interpretation} of an inference system $\mis$ identifies a
\emph{predicate} on $\universe$ whose elements are called \emph{derivable
judgments}.

To define the interpretation of an inference system $\mis$, consider
the \emph{inference operator} associated with $\mis$, which is the function
$\InfOp\mis : \wp(\universe) \to \wp(\universe)$ such that
\[
  \InfOp\mis(X)=\set{ \judg\in\universe \mid \exists\prem\subseteq X: \RulePair\prem\judg\in\mis }
\]
for every $X \subseteq \universe$.
%
Intuitively, $\InfOp\mis(X)$ is the set of judgments that can be
derived in one step from those in $X$ by applying a rule of $\mis$.
%
Note that $\InfOp\mis$ is a monotone endofunction on the complete
lattice $\wp(\universe)$, hence it has least and greatest fixed
points.
It can also be defined for a meta-rule
$\mr$ and an inference meta-system $\mis$ as
\[
\begin{array}{cc}
	\InfOp{\mr}(X) = \set{ \mj(\mc) \mid \mc\in\sd, \mj_i(\mc) \in X \text{ for all } i \in 1..n }
	&
	\InfOp{\mis}(X) = \bigcup_{\mr\in\is} \InfOp{\mr}(X)
\end{array}
\]

\begin{definition}[(Co)Inductive Interpretations]
  The \emph{inductive interpretation} $\Inductive\mis$ of an
  inference system $\mis$ is the least fixed point of $\InfOp\mis$
  and the \emph{coinductive interpretation} $\CoInductive\mis$ is
  the greatest one.
\end{definition}

From a proof theoretical point of view, $\Inductive\mis$ and $\CoInductive\mis$
are the sets of judgments derivable with well-founded and non-well-founded proof
trees, respectively.

Generalized Inference Systems enable the definition of (some)
predicates for which neither the inductive interpretation nor the
coinductive one give the expected meaning.

\begin{definition}[Generalized Inference System]
  \label{def:gis}
  A \emph{generalized inference system} is a pair $\Pair\mis\mcois$ where $\mis$
  and $\mcois$ are inference systems (over the same $\universe$) whose elements
  are called \emph{rules} and \emph{corules}, respectively.
  %
  The interpretation of a generalized inference system $\Pair\mis\mcois$,
  denoted by $\FlexCo\mis\mcois$, is the greatest post-fixed point of
  $\InfOp\mis$ that is included in $\Inductive{\mis\cup\mcois}$.
  
  Note that meta-corules are usually represented with a thicker line.
\end{definition}

From a proof theoretical point of view, a \GIS $\ple{\mis,\mcois}$ identifies
those judgments derivable with an arbitrary (not necessarily well-founded) proof
tree in $\mis$ and whose nodes (the judgments occurring in the proof tree) are
derivable with a well-founded proof tree in $\mis\cup\mcois$.
For more details we refer to \cite{Dagnino19}.

\subsection{Proving correctness - Proof principles}

\beginbass
%
The advantage of using inference systems is that they provide canonical
techniques for proving the correctness of a given definition.
%
Consider now a \emph{specification} $\Spec\subseteq\universe$, that is an
arbitrary subset of $\universe$. We can relate $\Spec$ to the interpretation of
a (generalized) inference system using one of the following proof principles.
%
The \emph{induction principle} \citep[Corollary 2.4.3]{Sangiorgi11} allows us to
prove the \emph{soundness} of an inductively defined predicate by showing that
$\Spec$ is \emph{closed} with respect to $\mis$. That is, whenever the premises
of a rule of $\mis$ are all in $\Spec$, then the conclusion of the rule is also
in $\Spec$.

\begin{proposition}[Induction]
	\label{prop:indp}
	$\InfOp\mis(\Spec) \subseteq \Spec$
	implies $\Inductive\mis \subseteq \Spec$.
\end{proposition}

The \emph{coinduction principle} \citep[Corollary 2.4.3]{Sangiorgi11} allows us
to prove the \emph{completeness} of a coinductively defined predicate by showing
that $\Spec$ is \emph{consistent} with respect to $\mis$.  That is, every
judgment of $\Spec$ is the conclusion of a rule whose premises are also in
$\Spec$.

\begin{proposition}[Coinduction]
	\label{prop:coindp}
	$\Spec \subseteq \InfOp\mis(\Spec)$ implies $\Spec \subseteq
  \CoInductive\mis$.
\end{proposition}

The \emph{bounded coinduction principle} \citep{AnconaDagninoZucca17} allows us
to prove the \emph{completeness} of a predicate defined by a generalized
inference system $\ple{\mis,\mcois}$. In this case, one needs to show not only
that $\Spec$ is consistent with respect to $\mis$, but also that $\Spec$ is
\emph{bounded} by the inductive interpretation of the inference system $\mis
\cup \mcois$. Formally:

\begin{proposition}[Bounded Coinduction]
	\label{prop:bcp}
  $\Spec \subseteq \Inductive{\mis\cup\mcois}$ and
  $\Spec \subseteq \InfOp\mis(\Spec)$ imply
  $\Spec\subseteq\FlexCo\mis\mcois$.
\end{proposition}

Proving the boundedness of $\Spec$ amounts to proving the completeness of
$\mis\cup\mcois$ (inductively interpreted) with respect to $\Spec$.

\subsection{Examples}

\beginbass
%
We now characterize the examples from \Cref{sec:sft_lvn}. For the sake of simplicity
we only present the definitions and we informally state where the proof principles
are used to prove the correctness (see \cite{Ciccone20} for the detailed proofs).
In \Cref{sec:agda_gis_meta} we will show Agda mechanizations of all the definitions.
In the following we write $\listin\xs{i}$ for the i-th element of list $\xs$.

\begin{example}
	Recall the inductive property from \Cref{ex:memberof}. Consider the inference system $\is$
	
	\begin{mathpar}
		\inferrule{\mathstrut}{\member(\x , \cons\x\xs)}
		\and
		\inferrule{\member(\x ,\xs)}{\member(\x , \cons\y\xs)}
	\end{mathpar}
	%
	where the axiom states that $\x$ is inside the list containing only $\x$ while the rule tells that,
	if $\x$ is inside $\xs$, then it is in the same list with an additional element $\y$.
	Such inference system $\is$ must be \emph{inductively interpreted}. Hence, we consider only those finite
	proof trees whose leaves are applications of the axiom.
	
	Let $\Spec = \set{(\x , \xs) \mid \exists i \in \Nat. \listin\xs{i} = \x}$. 
	We use the \emph{induction principle} to prove the \emph{soundness} of the definition
	(\ie $\Inductive\is \subseteq \Spec$).
	\eoe
\end{example}

\begin{example}
	Recall the coinductive property from \Cref{ex:allpos}. Consider the inference system $\is$
	
	\begin{mathpar}
	\inferrule{\mathstrut}{\allpos\nil}
	\and
	\inferrule{\allpos\xs}{\allpos \cons\x\xs} ~ \x > 0
	\end{mathpar}
	%
	where the axiom tells that the predicate trivially holds on the empty list and the rule
	states that, if a list $\xs$ is made of strictly positive numbers, then the predicate holds
	on such list with an additional element $\x$ provided that $\x > 0$.
	Such inference system $\is$ must be \emph{coinductively interpreted}. Hence, we consider either those finite
	proof trees whose leaves are the axiom or the infinite ones that are obtained by applying infinitely
	many times the rule. Note that is not compulsory to interpret the rule coinductively. On the other hand,
	if we interpret it inductively we obtain the expected predicate if and only if the lists under analysis are
	finite.
	
	Let $\Spec = \set{\xs \mid \forall i \in \Nat. \listin\xs{i} > 0}$. 
	We use the \emph{coinduction principle} to prove the \emph{completeness} of the definition
	(\ie $\Spec \subseteq \CoInductive\is$).
	%
	\eoe
\end{example}

\begin{example}
	Recall the property from \Cref{ex:maxelem}. We first consider the following inference system $\is$
	
	\begin{mathpar}
		\inferrule{\mathstrut}{\maxelem(\x , \cons\x\nil)}
		\and
		\inferrule{\maxelem(\x , \xs)}{\maxelem(max(\x , \y) , \cons\y\xs)}
	\end{mathpar}
	%
	where the axiom states that $\x$ is the maximum of the list that contains only $\x$, while the rule
	states that, if $\x$ is the maximum of $\xs$, then, when we add a new number $\y$ to $\xs$, the maximum
	becomes the greatest between $\x$ and $\y$.
	Such inference system cannot be inductively interpreted since we have to entirely inspect a possibly infinite
	list. On the other hand, the coinductive interpretation does not give us the expected meaning. Indeed, for example 
	we can derive $\maxelem(2 , \xs)$ where $\xs = \cons{1}\xs$ and $2$ actually does not belong to $\xs$.
	The infinite proof tree is obtained by applying the rule infinitely many times. 
	\[
	\begin{prooftree}
		\[
			\vdots
			\justifies
			\maxelem(2 , \xs)
		\]
		\justifies
		\maxelem(2 , \cons{1}{\xs})
	\end{prooftree}
	\]	
	Hence, $\is$ characterize only the \emph{safety} component of $\maxelem$ which states that the maximum must be
	greater than all the elements in the list.
	
	Now we consider the following $\cois$ consisting of a single \emph{coaxiom}
	\begin{mathpar}
		\infercorule{\mathstrut}{\maxelem(\x , \cons\x\xs)}
	\end{mathpar}  
	Such coaxiom forces the membership of the maximum into the list.
	The interpretation $\FlexCo\is\cois$ characterize the expected predicate. If we manage to find a finite proof tree
	in the inductive interpretation of the inference system with the additional coaxiom it means that 
	the maximum belongs to the list.
	
	Let $\Spec = \set{(\x , \xs) \mid \forall i \in \Nat. \x \geq \listin\xs{i}$ and $\member(\x , \xs)}$. 
	We use the \emph{bounded coinduction principle} to prove the \emph{completeness} of the definition
	(\ie $\Spec \subseteq \FlexCo\is\cois$).
	\eoe
\end{example}

As noted at the beginning of the section, all the proofs of the examples are detailed in \citep{Ciccone20}.