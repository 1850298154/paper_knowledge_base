\begintreble
%
This chapter is dedicated to the analysis of the main notions that are treated throughout the
thesis.
% 
First, we recall the usual distinction of properties of concurrent systems in 
\emph{safety} and \emph{liveness} ones \citep{OwickiLamport82}.
According to the class to which it belongs, a property must be defined using a different (dual) technique.
The same happens when we want to prove the correctness of such definitions.
%
Then, we introduce \emph{Generalized Inference Systems} \citep{AnconaDagninoZucca17}. 
Inference systems are a well established framework in the literature to define purely (co)inductive predicates 
as they provide canonical techniques for proving the correctness of a definition.
In this thesis we mainly focus on their generalization with \emph{corules} in order to treat those predicates
that require a mix of induction and coinduction.
%
After that we move to the main topics of the thesis.
We give an overview of \emph{session types} by describing what such types are and what they are used for.
In a few words, they are used to model the communication in message passing systems and to statically
enforce some desired properties. They have been originally introduced by \cite{Honda93} in a binary context
and later generalized by \cite{HondaYoshidaCarbone08} to \emph{multiparty} to deal with those scenarios
in which more entities participate to the communication.
We then review some interesting desirable properties that fall in the intersection between safety and liveness ones.
%
%Among such kind of properties we investigate \emph{fair termination}, \ie termination under a fairness
%assumption. We introduce the property in its general form and then we instantiate it in the context of session types.
%In general, we want to guarantee that the communication can always eventually reach (successful) termination.
%The assumption is used to consider \emph{unfair}, \ie unrealistic, those interactions that last forever without
%terminating. 

The chapter is organized as follows.
In \Cref{sec:sft_lvn} we review the usual classification of properties into safety and liveness ones and we
show that some fall in the intersection of the two classes.
In \Cref{sec:gis} we introduce generalized inference systems as a reference framework that can be used to deal with
such kind of properties.
At last, in \Cref{sec:st} we introduce session types in both their binary and multiparty variants.