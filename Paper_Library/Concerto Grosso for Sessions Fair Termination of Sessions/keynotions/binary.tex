\beginbass
%
We use \emph{polarities} $\Pol \in \set{\In,\Out}$ to
distinguish \emph{input actions} ($\In$) from \emph{output actions} ($\Out$) and
we write $\co\Pol$ for the \emph{opposite} or \emph{dual} polarity of $\Pol$ so that
$\co\In = \Out$ and $\co\Out = \In$.
We use $\Tag$ to denote an element of a given set of \emph{message tags} 
which may include values with a specific
interpretation such as booleans, natural numbers, and so forth.
The different terminology for labels 
is needed to avoid confusion with the labels used in the operational semantics.

\begin{definition}[Binary Session Types]
	Session types \citep{Honda93} are the possibly infinite, regular trees \citep{Courcelle83}
	coinductively generated by the grammar
	\[
    S, T, U, V ::= \End[\Pol] \mid \Tags\Pol\Tag_i.\S_i \mid \Pol \S.\T
	\]
\end{definition}

Session types of the form $\End[\Pol]$ describe channels used for exchanging a
session termination signal and on which no further communication takes place.
%
Session types of the form $\Pol\U.\S$ describe channels used for exchanging another
channel of type $U$ and then behaving according to $S$.
%
Finally, session types of the form $\Tags\Pol\Tag_i.S_i$ describe channels
used for exchanging a tag $\Tag_k$ and then behaving according to $S_k$. Session types of
the form $\Tags\In\Tag_i.S_i$ and $\Tags\Out\Tag_i.S_i$ are
sometimes referred to as \emph{external} and \emph{internal} choices
respectively, to emphasize that the label/tag being received or sent is always
chosen by the sender process. In a session type $\Tags\Pol\Tag_i.S_i$ we
assume that $I$ is not empty and that $i \ne j$ implies $\Tag_i \ne \Tag_j$ for every
$i,j\in I$. Note that $I$ is not necessarily finite, although regularity implies
that there must be finitely many \emph{distinct} $S_i$.

To improve readability we write $\Pol\Tag.\S$ when $I$ is the singleton $\set{\Tag}$
(when the choice is trivial) and we define the partial $\choice$ such
that

\[
  \begin{array}{lcl}
  \Tags\Pol\Tag_i.\S_i \branch \Tags[i \in J]\Pol\Tag_i.S_i & = &
  \Tags[i \in I \union J]\Pol\Tag_i.S_i
  \end{array}
\]

when $I, J \ne \emptyset$ and $I \cap J = \emptyset$.
%
Hereafter we specify possibly infinite session types by means of equations $S = \dots$ where
the right hand side of the equation may contain guarded occurrences of the metavariable $S$. 
Guardedness guarantees that a session type $S$ satisfying the equation exists and is unique \citep{Courcelle83}.

We equip session types with a \emph{labeled transition system} (LTS) that allows
us to describe, at the type level, the sequences of actions performed by a
process on a channel. We distinguish two kinds of transitions:
\emph{unobservable transitions} $S \lred\tau T$ are made autonomously by the process;
\emph{observable transitions} $S \lred{\smash\action} T$ are made by the process
in cooperation with the one it is interacting with through the channel. The
label $\action$ describes the kind of interaction and has either the form $\Pol U$
(indicating the exchange of a channel of type $U$) or the form $\Pol\Tag$ (indicating
the exchange of tag $\Tag$). The polarity $\Pol$ indicates whether the message is
received ($\In$) or sent ($\Out$).
If we use a term of the form $\Sys\S\T$ to describe the binary session as a whole, with
the two interacting processes behaving as $S$ and $T$, then we can formalize
their interaction (at the type level) using the LTS for session types and the
reduction rules in \Cref{fig:lts_bin}.

\begin{figure}[t]
\framebox[\textwidth]{
	\begin{mathpar}
		\inferrule[lb-channel]{\mathstrut}{\Pol\U.\S \lred{\mathstrut\Pol\U} \S} \defrule[lb-channel]{}
  	\\
  	\inferrule[lb-pick]{\mathstrut}
  		{\textstyle
  		\Tags\Out\Tag_i.\S_i \lred\tau \Out\Tag_k.\S_k}
  	~ k \in I \defrule[lb-pick]{}
  	\and
  	\inferrule[lb-tag]{\mathstrut}
  		{\textstyle
  		\Tags\Pol\Tag_i.\S_i \lred{\Pol\Tag_k} \S_k} 
  		~ k \in I \defrule[lb-tag]{}
  	\\
  	\inferrule[lb-tau-l]{
    	\S \lred\tau  \S'
  	}{
    	\Sys\S\T \lred\tau \Sys{\S'}\T
  	} \defrule[lb-tau-l]{}
  	\and
  	\inferrule[lb-tau-r]{
    	\T \lred\tau  \T'
  	}{
    	\Sys\S\T \lred\tau \Sys\S{\T'}
  	} \defrule[lb-tau-r]{}
  	\and
  	\inferrule[lb-sync]{
    	\S \lred{\co\action} \S'
    	\\
    	\T \lred\action  \T'
  	}{
    	\Sys\S\T \lred\tau \Sys{\S'}{\T'}
  	} \defrule[lb-sync]{}
	\end{mathpar}
}
\caption{Labeled Transition System for Binary Session Types}
\label{fig:lts_bin}
\end{figure}

We write $\co\action$ for the dual of $\action$, obtained by
changing the polarity of $\action$ with the opposite one.
%
A reduction occurs whenever one of the connected processes performs an
unobservable transition (\refrule{lb-tau-l} and \refrule{lb-tau-r}) 
or when the two processes exchange a message by
proposing complementary actions (\refrule{lb-sync}).
%
As usual, we let $\wred$ stand for the reflexive, transitive closure of $\lred\tau$ and 
we write $\Sys\S\T \nred$ if there are no $\S'$ and $\T'$ such that $\Sys\S\T \lred\tau \Sys{\S'}{\T'}$.
Note the different behaviors described by session types of the form
$\Tags\Pol\Tag_i.\S_i$ depending on the polarity $\Pol$.
%
According to \refrule{lb-tag}, a process using a channel of type $\Tags\In\Tag_i.\S_i$ performs an
observable transition for each of the tags $\Tag_i$ it is willing to receive.
%
On the contrary, a process using a channel of type $\Tags\Out\Tag_i.\S_i$
can first \emph{choose} a particular tag $\Tag = \Tag_k$ for some $k\in I$ by \refrule{lb-pick} (this choice
is internal to the process and is therefore unobservable) and then \emph{send}
the tag $\Tag$. As an example, the chain of transitions
\[
  \Out\Tag.\S \choice \T \lred\tau \Out\Tag.\S \lred{\Out\Tag} \S
\]
models a process that first chooses and then sends the tag $\Tag$. The choice of
the tag is irrevocable and not negotiable with the receiver process. Note
that, according to the definition of $\choice$, $\T$ must
be an internal choice of tags different from $\Tag$, hence $\Out\Tag.\S \choice \T$
is a non-trivial choice among two or more tags.