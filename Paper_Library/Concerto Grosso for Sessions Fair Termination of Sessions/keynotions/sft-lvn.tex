\beginalto
%
It is well known that properties can be
distinguished between \emph{safety} and \emph{liveness} one.
%
Informally, the two classes are identified by the mottos
``nothing bad ever happens'' and
``something good eventually happens'' \citep{OwickiLamport82}.
%
For example, in a network of communicating processes, the absence of
communication errors and of deadlocks are safety properties, whereas the fact
that a protocol or a process can always successfully terminate is a liveness
property.
%
Because of their different nature, characterizations and proofs of safety and
liveness properties rely on fundamentally different (dual) techniques: safety
properties are usually based on invariance (coinductive) arguments, whereas
liveness properties are usually based on well foundedness (inductive)
arguments \citep{AlpernSchneider85,AlpernSchneider87}.

The correspondence and duality between safety/coinduction and liveness/induction
is particularly apparent when properties are specified as formulas in the modal
\emph{$\mu$-calculus} \citep{Kozen83,Stirling01,BradfieldStirling07}, a modal
logic equipped with least and greatest fixed points: safety properties are
expressed in terms of greatest fixed points, so that the ``bad things'' are
ruled out along all (possibly infinite) program executions; liveness properties
are expressed in terms of least fixed points, so that the ``good things'' are
always within reach along all program executions. Since the $\mu$-calculus
allows least and greatest fixed points to be interleaved arbitrarily, it makes
it possible to express properties that combine safety and liveness aspects,
although the resulting formulas are sometimes difficult to understand.

In the next examples we informally state some (co)inductive properties that 
we will characterize in details in \Cref{sec:gis} and that we will characterize in
Agda in \Cref{sec:agda_gis_meta}.

\begin{example}
	\label{ex:memberof}
	\emph{``$\x$ belongs to the possibly infinite list $\listl$''}
	
	This is an example of a liveness property. Indeed, even if the list $\listl$ is infinite,
	if $\x$ belongs to $\listl$, then we need to inspect only a finite portion of
	the list. Hence, induction is required.
	\eoe
\end{example}

\begin{example}
	\label{ex:allpos}
	\emph{``All the numbers in the possibly infinite list $\listl$ are positive''}
	
	This is an example of a safety property. Indeed, in order to actually state that
	a list is made of only positive numbers, we have to inspect it entirely. It is
	clear that coinduction is required since the list can be infinite.
	\eoe
\end{example}

\begin{example}
	\label{ex:maxelem}
	\emph{`` $\x$ is the maximum element of the possibly infinite list $\listl$''}
	
	The predicate is composed by a safety and a liveness part.
	\begin{itemize}
	\item \emph{Safety}: $\x$ is greater that all the elements in $\listl$
	\item \emph{Liveness}: $\x$ belongs to $\listl$
	\end{itemize}
	Hence, this property requires a mix of induction and coinduction.
	\eoe
\end{example}
