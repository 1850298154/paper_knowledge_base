\section{Motivating Applications}\label{sec.Examples}

%We describe two representative examples of time-sensitive applications and how they could benefit from the networking middleware proposed in this work. 
%We describe how the proposed networking middleware could benefit two representative time-sensitive applications. %and how they could benefit from the networking middleware proposed in this work. 

We use two representative examples of time-sensitive applications to discuss the simplicity of integration and performance gains of the networking middleware.

%Two examples of this important class of applications are \emph{automated search and rescue} (SAR) missions and \emph{automated exploration of unknown environments} using collaborative teams of UAVs and ground robots. 

\textbf{Search and Rescue (SAR).} Consider the SAR mission described in \cite{SAR} in which a team of autonomous UAVs are supervised by a human rescue operator. The operator provides sparse high-level inputs to the UAVs via a Human-Robot Interaction (HRI) system. The HRI system displays real-time information about the mission and can be used to define regions of high/low interest, switch between search patterns (e.g., from expanding square search to parallel track search~\cite[Sec.~3]{SARManual}), modify UAV trajectories that were autonomously planned, and, if necessary, directly teleoperate the UAVs. The authors of \cite{SAR} propose an HRI system that allows the operator to interact with the UAVs in an efficient way. The HRI system is tested using two UAVs equipped with on-board cameras and connected to the operator via a WiFi base station.

%NATO, ATP-10 (C). Manual on Search and Rescue. Annex H of Chapter 6, 1988.
% Information freshness is clearly central to the automated SAR mission in \cite{SAR}, as outdated mission information at the HRI system may mislead the rescue operator, and outdated trajectory information at the UAVs may degrade group coordination, increasing the risk of UAV collisions. In addition to information freshness, operator-defined priorities are also important for the success of the SAR mission. For example, images from UAVs within high interest regions should have high priority and trajectory updates to teleoperated UAVs should also have high priority.
% %
% %In a SAR mission with a small-scale underloaded WiFi network (e.g., an HRI system with one or two UAVs), most data packets from all types of traffic should be delivered without delay, irrespective of their priority, thus, satis. 
% As we add more UAVs to the HRI system, the traffic load in the network increases, leading to a sharp degradation of WiFi's performance, which  can rapidly become insufficient to support the SAR mission. 

% Our proposed networking middleware can be easily integrated into the SAR mission in \cite{SAR} to improve its scalability by alleviating congestion in the WiFi network and by controlling the information flow. Specifically, the centralized resource allocation mechanism prevents packet collisions and can be programmed to adaptively prioritize transmissions, taking into account both information freshness and operator-defined priorities, in a similar way as the description in Sec.~\ref{sec.Scheduler}.


\textbf{Exploration of underground environments.} The DAR\-PA Subterranean Challenge is an international competition in which research teams develop multi-robot systems that can \emph{rapidly} search for, detect, and geolocalize artifacts (e.g., manikins, cellphones, and backpacks) placed in environments such as tunnels, mines, and caves. The CERBERUS team (who won the competition in 2021) deployed a system \cite{CERBERUS} with five robots: (i) two aerial robots that can explore spaces inaccessible to ground robots; (ii) two legged robots with long endurance and the ability to deploy WiFi breadcrumbs; and (iii) one wheeled roving robot with an on-board directional WiFi antenna and a $300$\thinspace{m}-long optical fiber reel connected to the WiFi base station located outside of the underground environment. Each robot is equipped with multiple cameras (regular, FLIR, and/or LIDAR) and powerful compute (CPUs, GPUs, and/or VPUs) that enable on-board multi-modal terrain perception, object detection, and path planning for exploration. The base station, the wheeled roving robot, and the WiFi breadcrumbs create a mesh WiFi network that allows robots to share maps and artifact detections with the human supervisor at the base station. The supervisor can teleoperate the roving robot (via the fiber connection) and send high-level controls to override the autonomously planned robots' behavior. Notice from the experimental evaluation in \cite[Sec.~10]{CERBERUS} that not all five robots were deployed in every mission run. The performance of the WiFi network was identified as a critical challenge. %in the experimental evaluation. %described in \cite[Sec.~10]{CERBERUS}.

Automated SAR \cite{SAR} and automated exploration \cite{CERBERUS} are clear examples of how information freshness can be important for time-sensitive applications. In both applications, outdated contextual information at the WiFi base station may mislead the human supervisor, decreasing the chances of finding survivors or artifacts, and outdated trajectory/control information at the robots may degrade group coordination, increasing the chances of collisions. In addition to information freshness, application-defined priorities also play an important role. For example, transmission of control updates should have high priority, and images from robots within high interest regions should have higher priority than images from low interest regions. 
%
In small-scale applications with an underloaded WiFi network (e.g., a SAR mission with one or two UAVs), most data packets from every source should be delivered promptly. However, as the application size scales, the sharp degradation of WiFi's performance drives the need for a networking solution that can alleviate congestion and prioritize the transmissions that are most valuable to the application, taking into account both information freshness and application-defined priorities.

The networking middleware proposed in this paper can be directly used to tailor WiFi to the needs of automated SAR \cite{SAR} and automated exploration \cite{CERBERUS}, without requiring hardware modifications. Specifically, the middleware for the leader compute node (described in Sec.~\ref{sec.Scheduler}) can be deployed at the operator's terminal, next to the WiFi base station, and the middleware for the followers (described in Sec.~\ref{sec.Queueing}) can be deployed at the autonomous robots. Similarly to the results obtained in Sec.~\ref{sec.Evaluation} for the mobility tracking application, we expect that our networking middleware can significantly improve throughput, information freshness, and scalability of the underlying WiFi networks in \cite{SAR} and \cite{CERBERUS}. %, similar to the results obtained in Sec.~\ref{sec.Evaluation} for the mobility tracking application.

%Our proposed networking middleware can be easily integrated into the automated SAR \cite{SAR} and the automated exploration \cite{CERBERUS} systems without hardware modifications. In particular, the programmable resource allocation mechanism described in Sec.~\ref{sec.Scheduler} can be deployed at the terminal used by the human operator (next to the WiFi base station) and the customizable information queues described in Sec.~\ref{sec.Queueing} can be deployed at the autonomous robots. Similarly to the results achieved for the mobility tracking application in Sec.~\ref{sec.Evaluation}, we believe that our networking middleware can improve information freshness and customize the underlying WiFi network to the needs of the specific time-sensitive application, thus, enabling the wireless network to support larger-scale time-sensitive systems.

%Add citations from Search and Rescue literature and from the DARPA SubT Challenge. Describe the automated search example in detail. Discuss the \emph{direct} relationship between automated search and object tracking. Discuss the importance of information freshness. Our system could have been directly used for this purpose. \color{blue} Insight from Luca: even if communication is not a major constraint, human validator becomes a constraint and requires scheduling. Not crucial, but might be worthwhile to add.\color{black}

%For example, in an autonomous search and rescue mission using a UAV swarm, the team of UAVs should collaboratively explore a geographical location in order to find people in distress as fast as possible. It is essential that the underlying communication network allows the team of UAVs to share information about the environment, e.g. images of the ground, and about themselves, e.g. position, attitude, and heading of the UAVs, in a timely manner. Notice that outdated information loses its value and can lead to system failures. Designing communication networks that can support time-sensitive applications that rely on collaborative multi-agent systems is clearly a challenging task with a high societal impact.
%The underlying communication network should... [Time is at the essence... \url{https://www.huffpost.com/entry/11-robotic-applications-for-search-and-rescue_b_5a173c9ae4b0bf1467a845c4}]. 

%\subsection{Smart Factories and Smart Cities}\label{sec.SmartApplications}
%\igor{Describe these two examples briefly. Show that our system could be used to improve the performance. Some adaptations may be necessary.} 
%\subsection{Computational Offloading}

%\textbf{Automated Fulfillment Warehouses.} Describe Amazon/Alibaba Warehouses. Add citations showing that they are using WiFi. Discuss how our system could be used to improve information freshness and allow for a better rumba/router ratio.

%\textbf{Smart-City Intersection.} Describe Smart-City intersection. Discuss how improved information freshness is needed. Add citations showing that US DoT and NY DoT are planing to use WiFi for V2X communication. Say that in a Smart-City intersection using WiFi-like technology (i.e. DSRC) could benefit from this type of system.