\section{Related Work}\label{sec.RelatedWork}

%\textbf{UAV Networks.} The literature on communication protocols for UAV networks is vast (see surveys in \cite{Drone_Surv_1,Drone_Surv_2,Drone_Surv_3,Drone_Surv_4}) and mostly theory-oriented (e.g., \cite{Drone_Theory_1,Drone_Theory_2,Drone_Theory_3,Drone_Theory_4}). Practical deployments of communication networks with multiple UAVs often use: (i) radios for unidirectional broadcast of control information to the UAVs \cite{Crazyswarm}; (ii) commercial off-the-shelf networking solutions such as WiFi and/or ZigBee \cite{CERBERUS,SAR,DistributedRobotFormation,CooperativeSLAM,ASTRO,DroneCinema,DOOR-SLAM,BeeCluster,Vijay_1,Vijay_2,hu2020hivemind}, both of which use CSMA and FIFO queues; or (iii) commercial LTE base-stations \cite{yuan2018ultra, moradi2018skycore, van2019androne, ahmad2021ares}. Communication is often mentioned as a bottleneck for scaling the multi-UAV system \cite{Drone_Surv_1,Drone_Surv_4,Vijay_2, chinchali2021network, honig2017flying}. To improve scalability, a common approach is to avoid/limit communication whenever possible \cite{Drone_Surv_4,Vijay_1,SLAM_review,Vijay_2} either by performing on-board computation; by compressing the data before transmitting \cite{mohanarajah2015cloud, choudhary2017distributed,chang2021kimera}, and/or by separating the UAV swarm into smaller communication groups \cite{hu2018centralize}. In this paper, we provide a novel solution to improve scalability that could be easily integrated to existing UAV networks.

\textbf{Age-of-Information.} 
Over the past decade there has been a rapidly growing body of theoretical works analyzing AoI in different contexts (see surveys in \cite{yates2021age,sun2019age_book,kosta2017age}). More recently, a few works \cite{AoI_measure_1,AoI_measure_3,kadota2021age,kadota2021wifresh,shreedhar2018acp,AoI_Wierman,AoI_SDR,ayan2021experimental} have considered system implementations. These system-oriented works can be separated into two categories: (i) measurement of AoI in real networks %using devices connected via Ethernet, WiFi, or LTE
\cite{AoI_measure_1,AoI_measure_3}; and (ii) systems that attempt to minimize AoI; by looking at congestion control \cite{shreedhar2018acp}, traffic engineering \cite{AoI_Wierman} and medium access via Software Defined Radios (SDRs) \cite{ayan2021experimental,kadota2021age,kadota2021wifresh,AoI_SDR}. However, there has been no prior work on minimizing AoI at the application layer or making readily deployable systems for real applications.
% removed AoI_measure_2, AoI_measure_4, AoI_vehicular
%and (iii) simulation of a time-sensitive application running over a real communication network that attempts to minimize AoI \cite{ayan2021experimental}. 
To the best of our knowledge, this is the first work to 
(i) build a system that can be easily deployed on current WiFi networks to minimize AoI and (ii) evaluate the performance of an AoI-based system using a time-sensitive application running over a real communication network. 

%We can discuss paper on AoI (theory and systems). Hardest part will be to compare this paper with the ICCCN paper. WiSwarm is not simply an extension of WiFresh. We should emphasize the many important differences. For example, the middleware is general and can be implemented for different applications. Its goal is to provide information freshness and application-defined dynamic prioritization in order to improve coordination. WiSwarm is the first system that optimizes freshness in a real application. Other system papers simply measure AoI or optimize AoI. They do not measure the impact of AoI-based design on the end application.

%A few theory-oriented papers \cite{Papers} have addressed the problem of finding the optimal rate in simple and idealized settings (e.g., M/M/1 queue \cite{kaul2012real} and ALOHA network \cite{ }) and a system-oriented paper \cite{shreedhar2018acp} developed a mechanism that iteratively adapts the generation rate until it converges to the optimal.

%To the best of our knowledge, this is the first work to experimentally evaluate AoI-based resource allocation in a real-world time-sensitive application.


% AoI system papers:
% Systems for measuring AoI in real networks
% -	C. S ̈onmez, S. Baghaee, A. Ergis ̧i, and E. Uysal-Biyikoglu, “Age-of- information in practice: Status age measured over TCP/IP connections through WiFi, ethernet and LTE,” in Proc. of IEEE BlackSeaCom, 2018
% -	H. B. Beytur, S. Baghaee, and E. Uysal, “Towards AoI-aware Smart IoTsystems,” in International Conference on Computing, Networking and Communications, 2020.
% -	Measuring Age of Information on Real-Life
% -	Connections
% -	B. Barakat, H. Yassine, S. Keates, I. Wassell, and K. Arshad, “How to measure the average and peak age of information in real networks?” in 25th European Wireless Conference, 2019.
% Systems that optimize freshness
% -	Age of Information in Random Access Networks with Stochastic Arrivals
% -	WiFresh: Age-of-Information from Theory to Implementation
% -	I. Kadota, M. S. Rahman, and E. Modiano, “Poster: Age of information in wireless networks: from theory to implementation,” in Proc. of ACM MobiCom, 2020.
% -	T. Shreedhar, S. Kaul, and R. D. Yates, “An age control transport protocol for delivering fresh updates in the internet-of-things,” in Proc. of IEEE WoWMoM, 2019.
% -	S. Kaul, M. Gruteser, V. Rai, and J. Kenney, “Minimizing age of information in vehicular networks,” in Proc. of IEEE SECON, 2011.
% -	Trading Throughput for Freshness: Freshness-Aware Traffic Engineering and In-Network Freshness Control
% -	Software-Defined Radio Implementation of Age-of-Information-Oriented Random Access
% Systems that use freshness to optimize time-sensitive applications
% -	An Experimental Framework for Age of Information and Networked Control via Software-Defined Radios. Emulates the problem of control of an inverted pendulum.
