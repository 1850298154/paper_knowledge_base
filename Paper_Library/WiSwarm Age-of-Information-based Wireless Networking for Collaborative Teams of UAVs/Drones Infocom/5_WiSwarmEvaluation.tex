\section{Evaluation}\label{sec.Evaluation}
In this section, we evaluate the performance of both WiFi and WiSwarm for the mobility tracking application. We perform our experiments in a \emph{dynamic indoor campus space with multiple external sources of interference} such as WiFi base stations, mobile phones, and laptops. Throughout this section when we refer to WiFi, we mean 2.4 GHz WiFi. %This is due to three reasons - 1) RasPis (Zero W) only support 2.4 GHz WiFi, 2) 2.4 GHz WiFi has longer range and 3) 2.4 GHz WiFi is more reliable for mobile nodes.

In our evaluation, we consider two experimental setups:
%\begin{enumerate}
 (i) \textbf{Stationary experiments}, which involve up to fourteen RasPis running an emulated version of the mobility tracking application and sending video-frames to a central Compute Node. These experiments  involved hardware-in-the-loop and allowed us to test a variety of network sizes, update generation rates, scheduling policies, frame resolutions, packet sizes and interference conditions. %Over the course of three months, we collected nearly 400 hours worth of data from these experiments.
 (ii) \textbf{Flight experiments}, which involve interfacing the RasPis with UAVs and conducting real mobility tracking experiments. These allowed us to test how WiSwarm performs with mobile agents, at longer distances, and in the presence of significant interference. They also illustrate the drawbacks of using WiFi more clearly. %Over the course of two months, we collected nearly 4 hours worth of flight data.
%\end{enumerate}

\textbf{Baseline}. To demonstrate the performance improvement of WiSwarm, we compare it with two baseline WiFi systems, namely WiFi-TCP and WiFi-UDP. Both systems collect video frames from the application layer at a fixed rate, packetize them, store them in FIFO queues, and send these packets over standard WiFi to the Compute Node. TCP uses its congestion control mechanism to adjust the number of packets in flight, while UDP simply forwards packets. In all of our stationary experiments, we found that accommodating the entire video frame within a single UDP packet (i.e., with no fragmentation) was the best choice in terms of tracking error. 

For flight experiments, we consider an \emph{optimized version of WiFi-UDP} as the baseline. Our flight tests showed that mobility tracking with WiFi-UDP and WiFi-TCP with fixed video frame rate (e.g., 50 fps) was not possible for more than a single sensing-UAV. To get mobility tracking to work with two sensing-UAVs, we had to carefully tune the frame generation rate (to 5 fps) and the UDP packet size (to 6 kB per fragment). This is due to the high congestion and unreliability caused by high generation rates and large packets, which caused tracking failures. Further, we also had to tune RTS/CTS thresholds. Despite all of this optimization, WiFi-UDP was only able to enable tracking for \textit{at most two UAVs} at a time, as we show in the discussion on flight experiments.

\subsection{Stationary Experiments}\label{sec.HIL}

\begin{figure}
	%\captionsetup{justification=centering}
	\centering
	\includegraphics[width=\linewidth]{Images/HIL_Simulation_Setup_w_frames.png}
	%\vspace{-0.4cm}
	\caption{Screenshot from the videos used to simulate car movement during stationary experiments. The tags are programmed to perform random walks with time-varying velocities. The virtual UAVs need to keep track of the tags. On the right, two examples of 224x224 frames sent to the Compute Node by the RasPis based on their current virtual UAV locations.} %Pis running the virtual UAV application send 224x224 sized frames to WiSwarm or WiFi. These frames represent their local FoV, based on their current location. Virtual UAV locations are regularly updated based on control commands received from the Compute Node.}
	\label{fig.vid_tags}
	%\vspace{-0.2cm}
\end{figure}

% \begin{figure}
% 	%\captionsetup{justification=centering}
% 	\centering
% 	\includegraphics[width=\linewidth]{Images/Sent_Frames.png}
% 	\caption{224x224 frames sent to the base station for processing by the RasPis based on the current virtual UAV locations.}
% 	\label{fig.simulated_frames}
% \end{figure}

In this section, we discuss the performance improvements of WiSwarm over WiFi for three different metrics: (i) AoI, (ii) throughput, and, most importantly, (iii) tracking error. Each data-point in the following discussion represents 16 minutes worth of experiments, split into 4 batches of 4 minutes each. We calculate the time-average of the performance metric over the entire 4 minutes of each batch and then the mean and standard deviation across batches.
%For each, compare WiSwarm and WiFi (TCP + UDP) over different network configurations in HIL -> show mean plots with errorbars -> discuss performance improvement
%Briefly discuss flight tests and confirm that they behave somewhat similarly.
\begin{figure}[t]
\centering
\subfloat[]
{\includegraphics[width=0.5\columnwidth]{Images/plot_wifi_N6_a-eps-converted-to.pdf}}%\label{fig.plot_wifi_1_a}
%\hspace{0.5cm}
\subfloat[]
{\includegraphics[width=0.5\columnwidth]{Images/plot_wifi_N6_b-eps-converted-to.pdf}}%\label{hist_1a} 
%\vspace{-0.4cm}
\caption{(a) AoI and (b) tracking error of baseline WiFi-TCP and WiFi-UDP plotted against the update generation rate of each of the $N=6$ emulated UAVs.} 
\label{fig.plot_wifi_1}
%\vspace{-0.4cm}
\end{figure}
% \begin{figure}
% 	%\captionsetup{justification=centering}
% 	\centering
% 	\includegraphics[width=\linewidth]{Images/plot_wifi_N6.eps}
% 	\vspace{-0.4cm}
% 	\caption{AoI and tracking error of the baseline WiFi system with TCP and UDP plotted against the update generation rate at the UAVs. There are $N=6$ transmitting RasPis.}
% 	\label{fig.plot_wifi_1}
% 	\vspace{-0.4cm}
% \end{figure}

%We start by first describing the experimental setup for the stationary experiments.
\textbf{Experimental Setup}. The experiments involve multiple RasPis running an emulated virtual UAV application. This application does two things. First, each RasPi has a video simulating the movement of cars stored on it. Using this video, the RasPis create cropped frames of size 224x224, based on the current location of the virtual UAV, which capture the local Field-of-View (FoV). These video frames are generated at a specified rate that can be set using the rate control mechanism, and are forwarded to WiFi or WiSwarm for delivery. The frames are stored as unencoded grayscale yuv images (1 byte per pixel), so each video frame is 49 kB in size. Second, the application decodes the control packets received from the Compute Node and updates the virtual UAV's location by moving between control waypoints at a specified speed. Figure~\ref{fig.vid_tags} shows a frame from the video used for simulating movement of the car tags, along with two examples of 224x224 frames that the RasPis send to the Compute Node for processing.   

Figure~\ref{fig.plot_wifi_1} plots the mean AoI and tracking error per UAV for both WiFi-TCP and WiFi-UDP as the frame generation rate at the RasPis increases. This plot is for a system with $6$ transmitting RasPis. Note that lower AoI and lower tracking error are preferred in terms of performance.

We make two important observations from Fig.~\ref{fig.plot_wifi_1}. First, the performance of both WiFi-TCP and WiFi-UDP degrades when the generation rate is high, since the network becomes congested. Second, \textit{WiFi needs optimization of the generation rate at the application layer} to be anywhere close to working in practice. This optimization is challenging since it needs to be at the application layer and also adjust quickly to changes in the traffic load and link reliability, which can vary due to external interference. This is true for both TCP and UDP, i.e. \textit{TCP congestion control was unable to adjust to the optimal rate on its own}.

%We also observe that for tracking error, UDP is slightly better than TCP for nearly all update rates. This is as we expect, since UDP is the protocol of choice in real-time applications in practice. 
Next, we compare the performance of WiSwarm with both fixed-rate versions of WiFi and rate-optimized versions of WiFi. We choose the frame generation rates from the set $\{ 1,3,5,7,10,15,20,25,50,100\}$ fps and the number of RasPis from the set $N\in\{ 2,4,6,8,10,12,$ $14\}$. We find the best performing rates for each value of $N$ from the rate set (based on tracking error).% to be $\{25,7,5,5,3,3,$ $1\}$ fps for $N \in \{ 2,4,6,8,10,12,14\}$, respectively. 

To the best of our knowledge, there are no general purpose systems that can do application layer rate control for a wide variety of real-time applications, so the \textbf{rate-optimized WiFi systems are overly optimistic baselines}. Despite this, WiSwarm achieves significant performance gains over both fixed-rate and optimized rate versions of WiFi-TCP and WiFi-UDP.  
%.

\begin{figure}[t]
\centering
\subfloat[]
{\includegraphics[width=0.5\columnwidth]{Images/plot_mean_AoI_comparison_a-eps-converted-to.pdf}}%\label{demofig}
%\hspace{0.5cm}
\subfloat[]
{\includegraphics[width=0.5\columnwidth]{Images/plot_mean_AoI_comparison_b-eps-converted-to.pdf}}%\label{hist_1a} 
%\vspace{-0.4cm}
\caption{Mean AoI per UAV plotted for (a) fixed-rate (50 fps) and (b) optimized rate WiFi, as well as WiSwarm, as the number of UAVs increases.} 
\label{fig.plot_mean_AoI}
%\vspace{-0.5cm}
\end{figure}
% \begin{figure}
% 	%\captionsetup{justification=centering}
% 	\centering
% 	\includegraphics[width=\linewidth]{Images/plot_mean_AoI_comparison.eps}
% 	\vspace{-0.4cm}
% 	\caption{Mean AoI (per UAV) plotted for WiFi and WiSwarm as the number of UAVs increases.}
% 	\label{fig.plot_mean_AoI}
% 	\vspace{-0.4cm}
% \end{figure}
\begin{figure}[t]
\centering
\subfloat[]
{\includegraphics[width=0.5\columnwidth]{Images/plot_tail_AoI_comparison_a-eps-converted-to.pdf}}%\label{demofig}
%\hspace{0.5cm}
\subfloat[]
{\includegraphics[width=0.5\columnwidth]{Images/plot_tail_AoI_comparison_b-eps-converted-to.pdf}}%\label{hist_1a} 
%\vspace{-0.4cm}
\caption{Tail ($95^{th}$ percentile) AoI per UAV plotted for (a) fixed-rate (50 fps) and (b) optimized rate WiFi, as well as WiSwarm, as the number of UAVs increases.} 
\label{fig.plot_tail_AoI}
%\vspace{-0.4cm}
\end{figure}
% \begin{figure}
% 	%\captionsetup{justification=centering}
% 	\centering
% 	\includegraphics[width=\linewidth]{Images/plot_tail_AoI_comparison.eps}
% 	\vspace{-0.4cm}
% 	\caption{Tail ($95^{th}$ percentile) AoI per UAV plotted for WiFi and WiSwarm as the number of UAVs increases.}
% 	\label{fig.plot_tail_AoI}
% 	\vspace{-0.4cm}
% \end{figure}
% \begin{figure}
% 	%\captionsetup{justification=centering}
% 	\centering
% 	\includegraphics[width=0.9\linewidth]{Images/AoI_histogram.eps}
% 	\vspace{-0.4cm}
% 	\caption{AoI histogram for $N=8$ sources.}
% 	\label{fig.AoI_histogram}
% 	\vspace{-0.4cm}
% \end{figure}
\textbf{AoI}. Figure~\ref{fig.plot_mean_AoI} plots the mean AoI per UAV as the system size $N$ increases. More sources in the system means more congestion, more packet collisions (in WiFi) and hence poor performance and scalability. We see this clearly in Fig.~\ref{fig.plot_mean_AoI}(a), where we compare the baseline versions of WiFi-UDP and WiFi-TCP to WiSwarm. The baseline versions of WiFi have fixed update generation rate of 50 fps at each source while WiSwarm uses the maximum generation rate of 100 fps. Mean AoI improves by 16x for $N=8$ and by almost 50x for $N=14$ compared to fixed-rate WiFi. A major cause of the poor performance of WiFi is buildup of FIFO queues once the network becomes congested. Fixed-rate TCP eventually starts outperforming fixed-rate UDP for larger $N$, due to its congestion control mechanism. WiSwarm does not suffer from the congestion problem due to the LIFO queues.

Figure~\ref{fig.plot_mean_AoI}(b) compares rate-optimized versions of WiFi-TCP and WiFi-UDP with WiSwarm. We observe that mean AoI still improves by 1.5x for $N=8$ and 2.2x for $N=14$. While the FIFO queues in WiFi are no longer congested due to careful tuning of the frame generation rates, there are still packet collisions due to the distributed nature of the CSMA protocol and external interference sources. WiSwarm avoids these collisions by centralizing medium access scheduling decisions and prioritizing sources with higher AoI.%This can be thought of as the additional improvement by due to our centralized application-specific scheduling.

Since AoI combines the idea of service regularity with latency, we are also interested in the tail of information freshness. Figure~\ref{fig.plot_tail_AoI} plots the performance of baseline WiFi systems and WiSwarm for the $95th$ percentile of AoI, i.e. the value of AoI which is only exceeded $5\%$ of the time during an entire experiment. We observe very similar gains as mean AoI. For fixed rate, we observe an 18x reduction at $N=8$ and 36x reduction at $N=14$. For rate-optimized, we observe a 1.2x reduction at $N=8$ and a $1.7x$ reduction for $N=14$. Note that the tail AoI is important for our tracking application in addition to mean AoI, since a worse tail suggests a higher probability of the car going out of the UAV's Field-of-View leading to lost tracking. %We further illustrate the difference in performance between WiSwarm, WiFi-UDP and WiFi-TCP by plotting the empirical histogram of AoI for each system in an experiment involving $N=8$ RasPis. 

\begin{figure}[t]
\centering
\subfloat[]
{\includegraphics[width=0.49\columnwidth]{Images/plot_throughput_comparison_a-eps-converted-to.pdf}}%\label{demofig}
%\hspace{0.5cm}
\subfloat[]
{\includegraphics[width=0.49\columnwidth]{Images/plot_throughput_comparison_b-eps-converted-to.pdf}}%\label{hist_1a} 
%\vspace{-0.4cm}
\caption{Mean Throughput per UAV plotted for (a) fixed-rate (50 fps) and (b) optimized rate WiFi, as well as WiSwarm, as the number of UAVs increases.} 
\label{fig.plot_throughput}
%\vspace{-0.7cm}
\end{figure}
% \begin{figure}
% 	%\captionsetup{justification=centering}
% 	\centering
% 	\includegraphics[width=\linewidth]{Images/plot_throughput_comparison.eps}
% 	\vspace{-0.4cm}
% 	\caption{Mean Throughput per UAV plotted for WiFi and WiSwarm as the number of UAVs increases.}
% 	\label{fig.plot_throughput}
% 	\vspace{-0.4cm}
% \end{figure}

\textbf{Throughput}. Figure~\ref{fig.plot_throughput} plots the mean throughput per UAV for each of the considered systems as the number of UAVs increases. From Fig.~\ref{fig.plot_throughput}(a), we observe that both fixed-rate WiFi-TCP and WiFi-UDP have higher per UAV throughput than WiSwarm. However, this doesn't help in getting better AoI (as we saw earlier) or tracking performance (as we will see later). \textbf{This supports the idea that high throughput alone is not sufficient and AoI is the right metric to optimize for in such real-time applications}. For the rate-optimized versions of WiFi, we see a performance improvement in mean throughput per-UAV since WiSwarm can avoid packet collisions and deliver higher rates than the distributed CSMA mechanism while also ensuring lower AoIs. For $N=8$, WiSwarm achieves 1.2x higher throughput and for $N=14$, it achieves 2.7x higher throughput. 

\textbf{Tracking Error}. This is where we see how all the pieces of our system design come together to deliver better application performance. Figure~\ref{fig.plot_error} plots the mean tracking error (in pixels) per UAV as the number of UAVs increases, for WiSwarm and WiFi implementations. From Fig.~\ref{fig.plot_error}(a), which shows the fixed-rate baselines, we observe that tracking performance improves by 12x for $N=8$ and 4x for $N=14$. From Fig.~\ref{fig.plot_error}(b), with rate-optimized WiFi versions, we observe that tracking error is reduced by 2x at $N=10$ and 4x at $N=14$ with WiSwarm. We also note that the gap in performance between WiSwarm and the WiFi baselines \textit{increases} with the system size. In other words, the performance of WiFi-TCP and WiFi-UDP degrades much more quickly with $N$ leading to poor scalability.  
%From Fig.~\ref{fig.plot_tail_error}, we conclude that the improvements are similar in the tail ($95^{th}$ percentile) of the tracking error. For systems with $N=14$ UAVs, tracking error per UAV is reduced by 4x. 
\begin{figure}[t]
\centering
\subfloat[]
{\includegraphics[width=0.5\columnwidth]{Images/plot_error_comparison_a-eps-converted-to.pdf}}%\label{demofig}
%\hspace{0.5cm}
\subfloat[]
{\includegraphics[width=0.5\columnwidth]{Images/plot_error_comparison_b-eps-converted-to.pdf}}%\label{hist_1a} 
%\vspace{-0.4cm}
\caption{Mean Tracking Error per UAV plotted for (a) fixed-rate (50 fps) and (b) optimized rate WiFi, as well as WiSwarm, as the number of UAVs increases.} 
\label{fig.plot_error}
%\vspace{-0.5cm}
\end{figure}
% \begin{figure}
% 	%\captionsetup{justification=centering}
% 	\centering
% 	\includegraphics[width=0.95\linewidth]{Images/plot_error_comparison.eps}
% 	\vspace{-0.4cm}
% 	\caption{Mean Tracking Error per UAV plotted for WiFi and WiSwarm as the number of UAVs increases.}
% 	\label{fig.plot_error}
% 	\vspace{-0.4cm}
% \end{figure}
% \begin{figure}
% 	%\captionsetup{justification=centering}
% 	\centering
% 	\includegraphics[width=\linewidth]{Images/plot_tail_error_comparison.eps}
% 	\vspace{-0.4cm}
% 	\caption{Tail ($95^{th}$ percentile) Tracking Error per UAV plotted for WiFi and WiSwarm as the number of UAVs increases.}
% 	\label{fig.plot_tail_error}
% 	\vspace{-0.4cm}
% \end{figure}

\subsection{Flight Experiments}\label{sec.Flight}
\begin{figure}
	%\captionsetup{justification=centering}
	\centering
	\includegraphics[width=\linewidth]{Images/flight_frames.png}
	%\vspace{-0.4cm}
	\caption{Two Examples of 160x160 grayscale video frames sent by the RasPis during flight experiments.}%to the Compute Node for processing 
	\label{fig.flight_frames}
	%\vspace{-0.5cm}
\end{figure}
While the stationary experiments allowed us to test our system in great detail and provide extensive comparisons, they did not involve implementing the application on real UAVs tracking actual mobile targets in a dynamic environment. Our flight experiments address exactly this setting. Broadly, we will observe that the mobility of UAVs and higher degree of interference leads to worse wireless connectivity and, in turn, more congestion and packet collisions for WiFi. This allows us to bring the robustness of WiSwarm into focus. % and how well it performs in challenging wireless conditions as compared to WiFi.
We provide a video describing the setup and results from the flight experiments at \cite{Video}.

%The flight experiments involve interfacing the RasPis with quadcopter drones to build the sensing-UAVs as seen in Fig.~\ref{fig.sensor-uav}.
\textbf{Experimental Setup}. In the flight tests, we replace the internal antenna of the RasPis with an external high-gain (8 dBi) antenna to improve range and reliability when the UAVs fly. We fly up to 5 UAVs at a time in our experiment space which is roughly 20 meters x 10 meters in size. The mobile objects are autonomous cars with RasPi 3Bs shown in Fig.~\ref{fig.sensor-uav-and-car}(b). We program these cars to move in different polygonal trajectories over time and also stop occasionally at random for a few seconds. These trajectories are unknown to the UAVs and the Compute Node, and the job of the UAVs is to track the cars as closely as possible. Figure~\ref{fig.flight_setup} depicts the setup for an experiment involving 5 UAVs tracking the corresponding cars.

We configure the Pi-Cameras at the UAVs to generate video frames at the maximum possible rate, which is 90 frames per second. For WiSwarm, we utilize this full rate, while for WiFi, we choose the optimized rate by using rate control. The video frames are 160x160 unencoded grayscale images in the yuv format (1 byte per pixel), with a total size of 25 kB per frame. %Notice the lower resolution of the frames as compared to the stationary experiments. This is because of the lower transmission rates and unreliability of the wireless channel due to mobility and longer distances.
Figure~\ref{fig.flight_frames} shows two examples of frames sent to the Compute Node by RasPis from the flying UAVs during different experiments.

%Motion Capture (MoCap) plays an important role in our flight experiments. %We use the MoCap system to obtain position and orientation measurements of the autonomous cars and sensing UAVs. 
The sensing-UAVs implement a controller that requires knowledge of their own global position and orientation to be able to plan desired trajectories. A Motion Capture (MoCap) system provides this information to the UAVs (also via 2.4 GHz WiFi). 
%In our experiments, MoCap effectively take the place of the GNSS inertial navigation system (GNSS-INS) that is often used to obtain position and orientation estimates in real-world applications. 
These MoCap messages are sent to the UAVs in UDP messages at 30 messages/second and each message contains timestamp, position, and orientation of a single vehicle in 45 bytes. So the MoCap network usage is approximately 1.3 kB/s (or 11 kb/s) per UAV. Importantly, the MoCap system runs completely independently from the WiSwarm and WiFi systems and causes a low level of persistent interference in the channel. Thus, results from our flight experiments are a good measure of robustness of WiSwarm and WiFi to external interference. 

%Finally, the MoCap data corresponding to the cars is also recorded, but not used during the experiments. We use this in post-processing for calculating the tracking error between the UAVs and the target cars, creating videos and plotting time-lapse trajectories of cars and UAVs.

%\textbf{Baseline}. 



%Vishrant: discuss all the specific parameter choices we made - fragment sizes, video resolution, source generation rates, etc. %\\
%also discuss any hardware setup that hasn't been covered yet \\
\textbf{Results}. Figure~\ref{fig.coord_wiswarm} plots the coordinates of the sensing-UAVs and the target cars over time, for a two drone WiSwarm experiment, in both 2-D and 3-D. Similarly, Fig.~\ref{fig.coord_wiFi} plots the coordinates of the sensing-UAVs and the target cars over time, for a two drone WiFi experiment. It is easy to see that WiSwarm allows for far better tracking than WiFi even for just two UAVs. This is further supported by the histograms of AoI and tracking error plotted in Fig.~\ref{fig.flight_histogram}. The lower tracking error for WiSwarm is \textit{due to the fact that it can achieve lower AoI}, and hence deliver fresher information.


\begin{figure}
	%\captionsetup{justification=centering}
	\centering
	\includegraphics[width=\linewidth]{Images/tracking_2_wifresh-eps-converted-to.pdf}
	%\vspace{-0.4cm}
	\caption{Coordinates of sensing-UAVs and target cars in 2-D and 3-D, for a two drone flight experiment running WiSwarm.}
	\label{fig.coord_wiswarm}
	%\vspace{-0.5cm}
\end{figure}

\begin{figure}
	%\captionsetup{justification=centering}
	\centering
	\includegraphics[width=\linewidth]{Images/tracking_2_wifi-eps-converted-to.pdf}
	%\vspace{-0.4cm}
	\caption{Coordinates of sensing-UAVs and target cars in 2-D and 3-D, for a two drone flight experiment running optimized WiFi-UDP.}
	\label{fig.coord_wiFi}
	%\vspace{-0.5cm}
\end{figure}

\begin{figure}[t]
\centering
\subfloat[]
{\includegraphics[width=0.49\columnwidth]{Images/histogram_flight_a-eps-converted-to.pdf}}%\label{demofig}
%\hspace{0.5cm}
\subfloat[]
{\includegraphics[width=0.49\columnwidth]{Images/histogram_flight_b-eps-converted-to.pdf}}%\label{hist_1a} 
%\vspace{-0.4cm}
\caption{Histograms of (a) AoI and (b) tracking error for flight experiments with two UAVs, comparing WiSwarm with WiFi.} 
\label{fig.flight_histogram}
%\vspace{-0.5cm}
\end{figure}
% \begin{figure}
% 	%\captionsetup{justification=centering}
% 	\centering
% 	\includegraphics[width=\linewidth]{Images/histogram_flight.eps}
% 	\vspace{-0.4cm}
% 	\caption{Histograms of AoI and tracking error for two drone flight experiments comparing WiSwarm with WiFi.}
% 	\label{fig.flight_histogram}
% 	\vspace{-0.2cm}
% \end{figure}

We summarize the results of all of our flight experiments in Tables \ref{table:error} and \ref{table:AoI}. We average over 4 minutes of flight data for each experiment. Our main observation is as follows: \textbf{while WiFi allows tracking for up to two UAVs at a time, WiSwarm can easily allow tracking for up to five UAVs at a time}. In fact, when there are more than two sources in the system, WiFi is unable to deliver more than a handful of packets and essentially no UAV control is possible. The main reason for this is the high level of packet collisions for WiFi. WiSwarm is relatively robust to the unreliable wireless channels, interference and mobility issues encountered in flight experiments, due to our scheduler design that avoids packet collisions and prioritizes AoI.

\begin{table}[h!]
\centering
\begin{tabular}{|c|c|c|c|c|c|} 
 \hline
 Number of Drones & 1 & 2 & 3 & 4 & 5 \\ [0.3ex] 
 \hline\hline
 WiFi-UDP (Optimized) & 0.43 & 1.85 & - & - & - \\ 
 \hline
 WiSwarm & 0.39 & 0.30 & 0.39 & 0.35 & 0.36\\[0.3ex] 
 \hline
\end{tabular}
\caption{Average tracking error per sensing-UAV (in meters).}
%\vspace{-0.7cm}
\label{table:error}
\end{table}

\begin{table}[h!]
\centering
\begin{tabular}{|c|c|c|c|c|c|} 
 \hline
 Number of Drones & 1 & 2 & 3 & 4 & 5 \\ [0.3ex] 
 \hline\hline
 WiFi-UDP (Optimized) & 0.10 & 0.19 & - & - & - \\ 
 \hline
 WiSwarm & 0.08 & 0.09 & 0.11 & 0.12 & 0.16\\[0.3ex] 
 \hline
\end{tabular}
\caption{Average AoI per sensing-UAV (in seconds).}
\label{table:AoI}
%\vspace{-0.7cm}
\end{table}