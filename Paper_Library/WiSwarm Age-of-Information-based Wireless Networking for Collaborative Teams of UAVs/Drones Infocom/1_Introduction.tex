\section{Introduction}\label{sec.Intro}

%\igor{Key points in this section:
%\begin{itemize}
%\item Done: Emerging time-sensitive applications have the potential to benefit society. Emerging applications have stringent performance requirements, specially in terms of information freshness. Developing such networks is challenging.
%\item Done: Vishrant: Might be worthwhile to discuss computational offloading at the wireless edge as an idea that is becoming increasingly popular in robotics to increase scalability and use that as motivation as well. will also cite some papers for this.
%
%\item Done: Describe an object tracking mission, with emphasis on the exchange of messages and the important of freshness. If possible, discuss the difference of freshness and latency and throughput. Show a table with results of the flight test. Conclude that WiFi only works for small scale and under loaded systems. Point out that the same fact can be observed in the literature that also use Wi-Fi.
%
%\item Note that our current networking technology is not enough. Two common technologies that are considered are WiFi and Cellular. Both have strong points and drawbacks. WiFi is inexpensive, tried-and-true, and has an open standard, but it has poor performance when the network is large-scale and overloaded. Vishrant: (WiFi performs poorly due to distributed random access scheduling which works well for small systems and for ensuring robustness but not for large-scale systems with stringent latency or freshness requirements. Maybe discuss how collisions affect throughput and latency). Cellular (e.g., 4G or 5G) can have better performance, but the technology is closed, vendor specific (look at O-RAN for a citation), and can be costly, especially when dedicated resources or custom resource allocation are needed for the application. 
%
%\item Ideally, the wireless network underlying such applications should satisfy the application-specific performance requirements and it should be inexpensive. 
%
%\item Prior work (?). Cite papers from the literature. Perhaps from FANET. Conclude that they are theoretical and have little to none practical implementation. [Perhaps this is not the best place to have this].
%
%\item One approach is to develop a virtualized, standardized, and open communication network that allows users to easily deploy networking solutions tailored to their application needs. This is the approach taken by O-RAN for cellular communication. However, this is a complex long-term approach that requires negotiation with vendors, hardware developers, and government. Moreover, it is limited to environments with cellular infrastructure. Search and Rescue after a disaster and Automated exploration of unknown subterranean environments may not be able to use such infrastructure.  
%
%\item Our contributions: (i) We propose a simple and inexpensive approach for customizing wireless networks to specific time-sensitive applications. In particular, we propose a networking middleware written in Python that allows a standard WiFi network to support large-scale time-sensitive applications by driving the WiFi network to behave as a polling-based network with customizable application layer transmission scheduling and queueing. (ii) To demonstrate the superior performance of our approach, specially for time-sensitive applications, we develop WiSwarm: a monitoring and control system that allows a cooperative UAV swarm to perform distributed object tracking using an Age-of-Information-based transmission scheduling. To the best of our knowledge, this is the first work to experimentally validate
%
%\item Challenges...
%
%\item The remainder of this work is organized as follows.
% \begin{itemize}
%     \item Sec. 2 Motivating examples: we describe four real-world time-sensitive applications that could benefit from the proposed system. Here we can (again) emphasize the importance of information freshness.
%     \item Sec. 3 Background: we describe age of information and relate it to position uncertainty + Networking Middleware for Information Freshness: we describe the architecture design of the middleware and its components
%     \item Sec. 4 WiSwarm: Design and Implementation: we describe the design and implementation of WiSwarm
%     \item Sec. 5 WiSwarm: Evaluation: we describe the evaluation of WiSwarm in both HIL simulations and flight tests
%     \item Sec. 6 Related Works (? - Not sure we need this here)
%     \item Sec. 7 Discussion: we discuss our design choices (e.g., WiFi and polling) and limitations of our system (centralization and flexibility and coexistance).
%     \item Sec. 8 Conclusion
% \end{itemize}
%\end{itemize}}

%{\color{red}[BELOW NEEDS TO BE ADAPTED to the OUTLINE ABOVE]}

Emerging time-sensitive applications increasingly rely on collaborative multi-agent systems. Examples are abundant: search and rescue missions using a team of unmanned aerial vehicles (UAVs), smart factories with connected automated machinery, and smart city intersections with connected self-driving cars. In such application domains, it is essential that agents communicate in a timely manner about changes in the environment and adapt their behavior accordingly. A major roadblock in deploying these applications in the real-world is that traditional communication networks were not designed to support large-scale multi-agent system that need to share time-sensitive information to collaborate effectively.
% Emerging time-sensitive applications such as search and rescue missions using a team of UAVs, smart factories with connected automated machinery, and smart city intersections with connected self-driving cars have the potential to significantly benefit society. For example, a smart city intersection can improve transportation efficiency and safety~\cite{NYC_project}, reducing costs, energy consumption, pollution, and accidents. 
% %
% %There has also been significant recent interest in computational offloading to enhance the scale of multi-agent robotics applications by performing intensive inference tasks at the wireless edge or in the cloud \cite{chinchali2021network}. 
% However, a major roadblock in deploying these large-scale time-sensitive applications in the real-world is that traditional communication networks were not designed to satisfy stringent application-dependent performance requirements. %\color{red}One might argue that cellular standards do have application specific performance requirements, we might want to rephrase\color{black}. 

WiFi is a common choice for deploying time-sensitive applications. Some examples include: automated fulfilment warehouses at Amazon \cite{KivaWiFi}, vehicle-to-everything communication in New York City \cite{VehicleToInfrastructureDSRC,AppLevelDSRC,NYCDOT}, and various multi-agent systems using teams of UAVs and ground robots \cite{CERBERUS,SAR,DistributedRobotFormation,CooperativeSLAM,ASTRO,DroneCinema,DOOR-SLAM,BeeCluster,MultiRobotSlam,MultiRobotMapping,hu2020hivemind}.  %\cite{RobotFormation,MultiRobotSlam,MultiRobotMapping,CERBERUS}. 
WiFi is attractive for deploying such systems because it is inexpensive, tried-and-true, and readily available in sensors, cameras, UAVs, and robotic platforms. However, it is well-known that WiFi's performance degrades sharply as the network size scales and traffic load increases. This is due to WiFi's Carrier-Sense Multiple Access (CSMA) distributed random access mechanism that works well for small-scale underloaded networks, but not for large-scale systems with stringent latency or freshness requirements. When a larger number of sources attempt to transmit using distributed random access, the higher probability of packet collisions leads to lower throughput and higher latency, which can result in degraded performance (or even failure) of the time-sensitive application.

% Websites that might be good to cite:
% - https://www.fiercewireless.com/tech/nokia-looks-to-ran-programmability-openness-new-platform
% - https://cloud.google.com/blog/topics/telecommunications/google-cloud-joins-o-ran-alliance
% - https://blogs.vmware.com/telco/oran-defines-path-to-innovation/
% - https://static1.squarespace.com/static/5ad774cce74940d7115044b0/t/5e95a0a306c6ab2d1cbca4d3/1586864301196/O-RAN+Use+Cases+and+Deployment+Scenarios+Whitepaper+February+2020.pdf
% - https://www.o-ran.org/

%An alternative to WiFi are cellular networks. Cellular networks employ a centralized resource allocation mechanism \cite{standard} that prevents packet collisions and prioritizes traffic according to its preassigned Quality-of-Service (QoS) level. 
%%provide differentiated levels of performance to different applications, depending on their Quality-of-Service (QoS) profile. Each QoS profile can be associated with different performance requirements in terms of throughput, latency, and reliability.  
%Major drawbacks of cellular networks include their high cost and the fact that the technology is proprietary and vendor specific, meaning that customizing resource allocation to the needs of distinct applications is prohibitively complex. %%An effort to make cellular networks open, virtualized, and programmable is underway by the O-RAN Alliance \cite{ORAN_1,ORAN_2}. Programmability is seen as key to enabling the deployment of \emph{custom solutions that satisfy application-specific performance requirements} \cite{Programmability}. However, this is a long-term approach that requires substantial efforts and negotiations with vendors, network operators, and hardware/software developers. Moreover, it is limited to environments with cellular infrastructure. 
%Moreover, applications such as search and rescue following a disaster and automated exploration of remote environments may not be able to use cellular infrastructure.

%Two common technologies that are considered in practice for such applications are WiFi and Cellular. While WiFi is inexpensive, tried-and-true, and has an open standard, it has poor performance when the network sizes or data rates increase. This is due to distributed random access scheduling for medium access, which works well for small systems and for ensuring robustness, but not for large-scale systems with stringent latency or freshness requirements. When a large number of sources attempt to access the wireless medium to send updates, there are packet collisions which lead to congestion at the transport layer, causing large delays and hence poor performance. Cellular networks (e.g., 4G or 5G) can have better performance, but the technology is closed, vendor specific (look at O-RAN for a citation), and can be costly, especially when dedicated resources or custom resource allocation are needed for the application. Ideally, the wireless network underlying such systems should satisfy application-specific performance requirements, be easily modifiable depending on the use-case and should also be inexpensive. 

%One approach is to develop a virtualized, standardized, and open communication network that allows users to easily deploy networking solutions tailored to their application needs. This is the approach taken by O-RAN for cellular communication. However, this is a complex long-term approach that requires negotiation with vendors, hardware developers, and government. Moreover, it is limited to environments with cellular infrastructure. Applications such as search-and-rescue after a disaster and automated exploration of unknown subterranean environments may not be able to use such infrastructure. \igor{I can talk about O-RAN and V2X. They are ongoing network customizations that are complex and expensive.}
%Consider an object tracking application composed of a team of several small and inexpensive UAVs and a central controller with high compute power. Each UAV takes pictures of the environment and transmits this contextual information to the central controller. The central controller consolidates the information from the UAVs and transmits trajectory updates that allow the UAVs to track the moving objects. Clearly, it is essential to keep the contextual information at the controller and the trajectory updates in every UAV as fresh as possible, as outdated information loses its value and can lead to system failures and safety risks (e.g., UAV collisions). To achieve information freshness, which is formally defined in Sec.~\ref{sec.AoI}, it is necessary to provide high throughput, low latency, and service regularity, simultaneously, to every UAV, which is often infeasible in large-scale networks due to their limited resources. 

%\textbf{Challenge.} Traditional networking technology can support time-sensitive applications \emph{only when the network is underloaded}. There are numerous works that showcase successful \emph{small-scale} WiFi-based implementations of different time-sensitive applications \cite{ManyPapers}. Table~\ref{tab.FlightTest} shows flight test results of the object tracking application with an increasing number of UAVs. Specifically, the first row displays the tracking error, i.e. the average distance between the UAV and the associated moving object, when WiFi is utilized to connect the UAVs to the central controller. Details about the experiment and measurements are provided in Sec.~\ref{sec.Flight}. The results show that the tracking error increases sharply with the number of UAVs, causing the system to frequently lose track of objects when the network has three or more UAVs. As the number of UAVs sharing the network increases, the traffic load increases and, when the traffic load exceeds the network capacity, the network becomes congested, leading to a rapid growth in the number of backlogged packets at the sources. A large backlog in First-Come First-Served queues (which are typical in traditional networks including WiFi) leads to high packet delay and, thus, to outdated information at the central controller. In Sec.~\ref{sec.Evaluation}, we discuss how we adjusted the image generation rate at the UAVs for each different network size $N$, aiming to minimize congestion. The results in the first row of Table~\ref{tab.FlightTest} are after the adjustments.

%The reason for this system failure is that the traffic load in the WiFi network leads to congestion\footnote{As discussed in Sec.~\ref{sec.Evaluation}, congestion occurs even after adjusting the image generation rate at the UAVs. Notice that if the image generation rate is too high, it leads to added traffic and congestion. In contrast, if the rate is too low, it leads to outdated information at the central controller due to the lack of images updates. As the number of UAVs increases, the maximum image generation rate that allows the network to operate without congestion becomes too low to keep the central controller updated.}, which sharply degrades information freshness and, thus, the tracking capabilities of the system. 

%While two UAVs may be enough to track a few objects, this scalability limitation severely undermines the application. Similar arguments can be made for search and rescue missions, smart factories, and smart cities, all of which are large-scale applications. \color{blue} Vishrant: discuss collisions and random-access as the major cause of poor performance and scalability of 802.11 WiFi.\color{black}


%The results show that, as expected, the tracking error increases from $N=1$ to $N=2$ and the system fails beyond $N=3$ due to data traffic congestion in the WiFi network, which occurs despite our best efforts to adjust the image generation rate at the UAVs. Notice that if the image generation rate is too high, the WiFi network becomes congested while is the rate is too low, the central controller does not receive enough

%Table~\ref{} compares flight test results of this object tracking application with an increasing number of UAVs. Specifically, the first row displays the tracking error, i.e. the time-average distance between the UAV and the associated moving object, when a WiFi network is utilized to connect the UAVs to the central controller. The tracking error increases from $N=1$ to $N=2$, as expected, and the system fails beyond $N=3$ due to data traffic congestion in the WiFi network, which occurs despite our best efforts to control the data generation rate at the UAVs. The takeaway is that traditional
%

%Traditional networks are designed to support a wide range of applications. 

\textbf{Our contributions: (1) Middleware design.} We develop a networking middleware that makes WiFi networks customizable, allowing system designers to easily tailor WiFi to the needs of specific time-sensitive applications. %, enabling the deployment of networking solutions that are tailored to the needs of time-sensitive applications. 
Our middleware drives the underlying distributed WiFi network to behave as a network with centralized resource allocation and with custom queues at the sources. By controlling the storage and flow of information in the WiFi network, the middleware: (i) prevents packet collisions; (ii) dynamically prioritizes the transmissions that are most valuable to the application; and (iii) discards stale packets that are no longer useful to the application before they are ever transmitted, thus alleviating congestion.

The networking middleware has two distinct features. First, it is \emph{implemented at the application layer}, without (any) modifications to lower layers of the networking protocol stack. The middleware runs over UDP/IP and standard 802.11 WiFi, making it easy to customize and integrate to existing time-sensitive applications that are already implemented using WiFi, such as \cite{KivaWiFi,VehicleToInfrastructureDSRC,AppLevelDSRC,NYCDOT,CERBERUS,SAR,DistributedRobotFormation,CooperativeSLAM,ASTRO,DroneCinema,BeeCluster,DOOR-SLAM,MultiRobotSlam,MultiRobotMapping,hu2020hivemind}. Second, the middleware is \emph{designed around the idea of information freshness}, specifically the \textbf{Age-of-Information (AoI) metric}. The AoI captures the freshness of the information \emph{from the perspective of the destination}, in contrast to the long-established packet delay that represents the latency of a \emph{particular packet}. The networking middleware can leverage AoI to prioritize transmissions to destinations with stale information. %The resource allocation mechanism can leverage this metric to prioritize transmission updates for the pieces of information that are most relevant. 
Keeping information fresh is critical for time-sensitive applications, especially those that rely on cooperative multi-agent systems. %\textbf{To the best of our knowledge, this is the first work to experimentally evaluate the impact of an AoI-based networking solution in a real-world time-sensitive application.} %\textbf{To the best of our knowledge, this is the first work to experimentally evaluate AoI-based resource allocation in a real-world deployment of a time-sensitive application.}

%there has been no prior work on the experimental evaluation of real-world time-sensitive applications running over an AoI-based communication network.

%\begin{itemize}
    %\item the design of our networking middleware incorporates the concept of information freshness - formally defined in Sec.~\ref{}. Information freshness has been receiving increasing attention in the literature \cite{AOIPaper} for its application in networked systems that carry time-sensitive data. Keeping track of information freshness enables the development of resource allocation mechanisms that aim at keeping 
    %\item the networking middleware is implemented at the application layer, without modifications to lower layers of the networking protocol stack. The networking middleware runs over UDP/IP and standard 802.11 WiFi, making it easy to integrate into existing applications that are implemented using WiFi such as \cite{ManyPapers}. 
%\end{itemize}
 

%\textbf{Our contributions.} To support large-scale time-sensitive applications, the underlying communication network must be designed to satisfy its application-specific requirements. In this paper, we propose a simple approach for customizing wireless networks to different time-sensitive applications. In particular, we propose a networking middleware design that tightly controls the storage and flow of information in the underlying WiFi network, dynamically prioritizing the transmissions that are most valuable to the specific application, eliminating packets that are no longer useful to the application before they are ever transmitted, and preventing packet collisions. The networking middleware drives WiFi to behave like a polling-based network with easily customizable centralized scheduling policies, choice of update generation rates and queuing disciplines. A distinct feature of our middleware is that it is implemented at the application layer, without modifications to lower layers of the networking protocol stack. The networking middleware runs over UDP/IP and standard 802.11 WiFi, making it easy to integrate into existing applications that are implemented using WiFi such as \cite{ManyPapers}. 

\textbf{Our contribution: (2) WiSwarm implementation.} To demonstrate the performance improvement that can be achie\-ved by customizing the WiFi network, we implement Wi\-Swarm: an instantiation of our networking middleware for a mobility tracking application that relies on a collaborative UAV swarm. Following the recent growing interest in computational offloading to enhance the scale of multi-agent robotics applications \cite{chinchali2021network}, we implement a mobility tracking %by performing intensive inference tasks at the wireless edge \cite{chinchali2021network}, we consider 
%A representative example of a time-sensitive application is mobility tracking using a team of UAVs. Consider a mobility tracking application composed 
application composed of several small and inexpensive UAVs and one leader node with high compute power. Each UAV senses the environment (e.g., collects video) and transmits this contextual information to the leader node. The leader node consolidates the information from the UAVs and transmits trajectory updates that allow the UAVs to track the moving objects. Clearly, \emph{it is essential to keep the contextual information at the leader node and the trajectory updates at every UAV as fresh as possible}, since outdated information loses its value and can lead to system failures (e.g., UAV losing track of an object) and safety risks (e.g., UAV collisions). 
%To keep the information fresh, the underlying communication network must continuously provide high throughput, low latency, and service regularity to every UAV in the network. This is a challenging requirement, especially as the number of UAVs increases. 

%\textbf{Our contribution: (3) Middleware evaluation.} 
\begin{figure}
	%\captionsetup{justification=centering}
	\centering
	\includegraphics[width=\linewidth]{Images/Flight_Setup_5drone_new_v2.jpg}
		%\vspace{-0.35cm}
	\caption{Flight experiment with five UAVs.}
	\label{fig.flight_setup}
	%\vspace{-0.6cm}
\end{figure}
We evaluate WiSwarm in \textbf{flight experiments}
with up to five sensing-UAVs (see Fig.~\ref{fig.flight_setup}), and in \textbf{stationary experiments} with up to fourteen  Raspberry Pis (RasPis) emulating UAVs. We collect data from nearly 4 hours of flight tests and $400$ hours of stationary tests. We also provide a video \cite{Video} summarizing the setup and results from our flight experiments. Our experimental results show that WiSwarm significantly outperforms WiFi in terms of throughput, information freshness, tracking performance, and scalability. %even when we consider the optimal configuration choices for WiFi. 
The stationary experiments with fourteen sources %(Raspberry Pi's emulating drones) and a leader node (PC with WiFi router ) 
shows that WiSwarm improves information freshness by a factor of $109$, and tracking error by a factor of $4$. The flight tests show that mobility tracking with WiFi can support \emph{at most two} sensing-UAVs while WiSwarm can support \emph{at least five} sensing-UAVs under similar conditions. 

\textbf{To the best of our knowledge, this is the first work to develop and implement an application-layer solution that optimizes information freshness in the wireless network without requiring modifications to lower layers of the networking protocol stack and, also, the first work to experimentally evaluate the impact of an AoI-based solution in a real-world time-sensitive application.}

%We implement WiSwarm in a Raspberry Pi's using Python. To evaluate WiSwarm in large-scale systems, we employed \textbf{Hardware-in-the-loop (HIL) simulations} with fourteen Raspberry Pi's that run WiSwarm together with a program that simulates the movement of UAVs and objects being tracked. To evaluate WiSwarm in a full-fledged mobility tracking application, we integrate five Rasbperry Pi's to UAVs that track five autonomous cars on the ground.

%In this setting, we are able to demonstrate that WiSwarm significantly outperforms WiFi in terms of information freshness, tracking performance and network scalability. WiSwarm... \igor{Talk about our measurements on the second row of Table~\ref{tab.FlightTest}} To the best of our knowledge, this is the first work to experimentally validate the use of information freshness metrics and scheduling policies to gain drastic performance and scalability improvements in real-time monitoring and control applications. 

%The remainder of this work is organized as follows. In Sec.~\ref{sec.Examples}, we describe four real-world time-sensitive applications that could benefit from our proposed networking middleware. In Sec.~\ref{sec.Middleware}, we introduce the Age of Information metric and discuss the design choices of the networking middleware. In Sec.~\ref{sec.WiSwarm}, we describe the design and implementation of WiSwarm for a mobile tracking application. In Sec.~\ref{sec.Evaluation}, we evaluate the performance of WiSwarm in both HIL simulations and UAV flight tests. In Sec.~\ref{sec.Discussion}, we discuss our design choices, explain how they lead to performance gains and also limitations of our design approach. Finally, in Sec.~\ref{sec.RelatedWork} we discuss related works and conclude in Sec.~\ref{sec.Conclusion}.  

The remainder of this work is organized as follows. 
%In Sec.~\ref{sec.Examples}, we describe real-world applications that can benefit from the networking middleware. 
%In Sec.~\ref{sec.Middleware}, we introduce the Age of Information metric, develop key ideas underlying our design for information freshness and go over the components of our networking middleware approach.
In Sec.~\ref{sec.Middleware}, we introduce the AoI metric and describe the middleware. 
%In Sec.~\ref{sec.WiSwarm}, we discuss the mobility tracking application and describe the design and implementation of WiSwarm, which is a specific instantiation of our middleware design for monitoring using UAVs. 
In Sec.~\ref{sec.WiSwarm}, we describe the design and implementation of WiSwarm. 
%In Sec.~\ref{sec.Evaluation}, we evaluate in detail the performance of WiSwarm in both hardware-in-loop simulations and actual UAV flight tests. We also demonstrate the performance improvements of WiSwarm over WiFi. 
In Sec.~\ref{sec.Evaluation}, we evaluate the performance of WiSwarm in flight tests and stationary experiments. 
%In Sec.~\ref{sec.Discussion}, we discuss our design choices. 
%In Sec.~\ref{sec.RelatedWork}, we discuss related works. 
Finally, in Sec.~\ref{sec.Conclusion}, we conclude and discuss future work.

%\textbf{Existing network customization.} \igor{Here, I can talk about O-RAN and V2X. They are ongoing network customizations that are complex and expensive.} A noteworthy example of specialization of networking technology to a specific time-sensitive applications is the ongoing development of Dedicated Short-Range Communications (DSRC) or Cellular Vehicle-to-Everything (C-V2X) for vehicular communication, where DSRC is based on the WiFi standard \cite{DSRC} and C-V2X is based on LTE \cite{Qualcomm}. One approach to developing application-aware networks is to create an open, standardized, and virtualized architecture that allows users to customize (at least parts of) the network to the application needs. This flexibility is a main goal of the O-RAN alliance \cite{ORAN}. However, this is a complex long-term approach that involves negotiations with a community of operators, vendors, research institutions, and government. Moreover, network customization is not a simple task and will most likely be limited to network operators and researchers. In this paper, we propose a simple and readily available solution

%For example, suppose that the central controller knows the position of object A two seconds ago and the position of object B five seconds ago. Intuitively, the central controller would benefit the most from receiving a fresh picture from object B since it is associated with the least fresh information. In this simplistic example, the optimal resource allocation was easily determined. In the real object tracking implementation described in Sec.~\ref{sec.WiSwarm}, the network control algorithm has access to information about the instantaneous tracking error, the velocities of the objects, and the reliabilities of the communication links, all of which may affect the optimal resource allocation. 

%However, if the central controller also knew that the velocity of object A is higher than the velocity of object B, then the optimal resource allocation becomes less evident. Additional information that, if available, should also be considered when optimizing the resource allocation for the object tracking application are the reliability of the communication links and the distances between UAVs and moving objects. 

%Traditional networking technologies such as WiFi and Cellular are to a large extent unaware of the concept of information freshness or the objectives of the application. Nonetheless, due to its low cost, open technology, and immediate availability in sensors, UAVs, and robots, WiFi is often the go-to choice for deploying time-sensitive applications. Examples are abundant: multi-UAV system for exploration of subterranean environments [6]; path planning, localization and motion control for multi-robot formations using UAVs [4] and using ground-robots [5]; real-time surveillance system using a fleet of ground-robots [9]; and data collection from sensors, UAVs and cameras for agriculture [10]. These various implementations \cite{all_citations} showcase that WiFi is capable of supporting time-sensitive applications as long as the network is small-scale and underloaded. Two shortcomings of WiFi are scalability and congestion. To

%Two networking technologies that are often used to support time-sensitive applications are Cellular and WiFi. In general, Cellular networks can achieve larger coverage area and provide added protection from interference, but the technology and architecture are proprietary and vendor-specific \cite{ORAN}. On the other hand, WiFi is cheaper, its technology is open, and it is widely available in sensors, UAVs, and computing platforms, making WiFi the usual choice in the literature.

%Two networking technologies that are often used to support time-sensitive applications are Cellular and WiFi. A noteworthy example is the ongoing discussion about the adoption of one or the other for vehicle-to-everything (V2X) communication \cite{DSRC_vs_C-V2X}. In general, Cellular networks can achieve larger coverage area and provide added protection from interference, but the technology and architecture are proprietary and vendor-specific \cite{ORAN}. On the other hand, WiFi is cheaper, its technology is open, and it is widely available in sensors, UAVs, and computing platforms.
% %
% Obs.: WiFi was the first technology to be tested and deployed in vehicles.

%\textbf{Our contributions}. (i) We propose a simple and inexpensive approach for customizing wireless networks to time-sensitive applications. In particular, we propose a networking middleware written in Python that allows a standard WiFi network to support large-scale time-sensitive applications by driving the WiFi network to behave as a polling-based network with flexible application layer transmission scheduling and queueing. (ii) To demonstrate the superior performance of our approach, we develop WiSwarm: a monitoring and control system that allows a cooperative UAV swarm to perform distributed object tracking using an Age-of-Information-based transmission scheduling. To the best of our knowledge, this is the first work to experimentally validate


%Collaborative multi-agent systems play a fundamental role in emerging time-sensitive applications such as: search and rescue missions, smart-city intersections with cloud assisted vehicles and traffic lights, automated fulfillment warehouses, automated streaming of live sport events, exploration of unknown subterranean environments, and many others. In such application domains, the underlying communication network should satisfy the performance requirements of the specific application while also supporting the multi-agent system. For example, in an autonomous search and rescue mission using a UAV swarm, the team of UAVs should collaboratively explore a geographical location in order to find people in distress as fast as possible. It is essential that the underlying communication network allows the team of UAVs to share information about the environment, e.g. images of the ground, and about themselves, e.g. position, attitude, and heading of the UAVs, in a timely manner. Notice that outdated information loses its value and can lead to system failures. Designing communication networks that can support time-sensitive applications that rely on collaborative multi-agent systems is clearly a challenging task with a high societal impact.
%The underlying communication network should... [Time is at the essence... \url{https://www.huffpost.com/entry/11-robotic-applications-for-search-and-rescue_b_5a173c9ae4b0bf1467a845c4}]. 

%In such application domains, it is essential that agents adapt their behavior (in a timely manner) to changes in the environment. To detect events of interest in an uncertain dynamic environment, agents may process information from their own sensors (e.g. LiDAR, camera, GPS, and/or IMU) and may exchange information with other agents or infrastructure. Naturally, the system design and, in particular, the choices of which information to process locally, which computational task to offload, and what information to share depend on the application, the performance requirements, and the available computation and communication resources.

%\textbf{Prior work.} The literature on networking solutions for mobile networks is vast, dating almost $X$ decades. Most relevant to this work are Flying Ad-Hoc Networks (FANETs). For surveys on FANETs, we refer the readers to \cite{ }. Related work is discussed in detail in \textsection~\ref{sec.RelatedWork}. The literature includes numerous interesting works that develop novel networking solutions employing Non-Orthogonal Multiple Access (NOMA) mechanisms \cite{ }, customized priority queueing \cite{ }, full-duplex communication \cite{ }, directional antennas \cite{ }, and others. The works in \cite{ } are all evaluated using simulations and/or emulations. 

%In practice, the implementation of collaborative multi-agent systems often employs WiFi. For example… [describe a few works]. WiFi is an attractive choice for it is low-cost, well-established, and immediately available in UAVs, ground-robots, sensors, and computing platforms such as Raspberry Pis. Moreover, as showcased by the various implementations in \cite{ }, underloaded WiFi networks are capable of supporting small-scale time-sensitive applications that rely on collaborative multi-agent systems. However, two well-known shortcomings of WiFi are scalability and congestion. These shortcomings must be addressed to allow real-world deployment...

%In this work, we propose \emph{WiSwarm}: a networking middleware that allows a standard WiFi network to support large-scale time-sensitive applications that rely on collaborative multi-agent systems. WiSwarm transforms a standard WiFi network into a polling-based system with priority queues that supports both the collaborative multi-agent system and the time-sensitive application. A distinct feature of WiSwarm is that it is implemented at the Application layer, without modifications to lower layers of the networking protocol stack. WiSwarm runs over UDP/IP and standard WiFi, making it easy to integrate into existing applications that are implemented using WiFi such as \cite{}.

%\textbf{Challenges.} Polling prevents packet collisions, especially in large-scale overloaded wireless networks, and it allows for a tight control on the time instant in which data is transmitted, making it well-suited for time-sensitive applications, despite the fact that it is rarely implemented in practice. Two important challenges associated with polling-based networks are the control overhead and the scheduler design. \emph{Control overhead}: before sending data packets, agents must wait for a poll packet. The poll packet and the waiting time reduce the communication efficiency. In WiSwarm, polling is implemented at the Application layer, as opposed to the Medium Access Control (MAC) layer, which significantly increases this overhead. \emph{Scheduler Design}: the scheduler decides which agent should transmit. The optimal sequence of scheduling decisions depends on the state of the agents (e.g. if the agent has data to transmit or not), the state of the wireless link, and the specific application. Some of these information is now known. For example, .... Thus, we seek to develop... [talk about scaling too]

%\textbf{Opportunity.} Age of Information (AoI), which is formally defined in \textsection\ref{}, has been receiving increasing attention in the literature, particularly for applications that generate time-sensitive data such as position, command and control, or sensor data. An interesting feature of this performance metric is that it captures the freshness of the information from the perspective of the destination, in contrast to the long-established packet delay, that represents the freshness of the information with respect to individual packets. In particular, AoI measures the time elapsed since the generation of the packet that was most recently delivered to the destination, while packet delay measures the time elapsed from the generation of a packet to its delivery. In recent years, several theory-oriented papers have proposed novel network control mechanisms that could be leveraged in real applications, but (as discussed in \textsection~\ref{sec.RelatedWork}) only a hand-full \cite{} have considered system implementation. For example, the authors of \cite{} proposed a MAC layer architecture based on Polling Multiple Access mechanism and Last-Come First-Served queues, implemented this architecture in a Software Defined Radio testbed, and showed that it achieves near optimal information freshness. 

%The two parameters that influence AoI are packet delay and packet interdelivery time \cite{}. In general, controlling only one is insufficient for achieving good AoI performance. For example, consider the object tracking application in Fig.~\ref{} with sensing-UAVs generating images at a high rate. In this setting, the Wi-Fi network may become congested, resulting in high packet delay, which leads to outdated information at the leader-UAV (i.e. poor AoI performance). In contrast, in the setting with sensing-UAVs generating images at a low rate, the Wi-Fi network is often idle, resulting in low packet delay. Nonetheless, the AoI may still be poor since the infrequent image generation may lead to stale information at the leader-UAV. A good AoI performance is achieved when fresh images are frequently delivered to the leader-UAV. %Table~\ref{} provides measurements of the AoI for different image generation rates at the sensing-UAVs. 

%\textbf{Our contribution.} 
%In this paper, we propose WiSwarm: a hierarchical multi-UAV system architecture that scales gracefully both in terms of cost and information freshness. In WiSwarm, we have one \emph{leader-UAV} with high on-board computational power (Jetson TX2) and $N$ uncostly \emph{follower-UAVs} with low on-board computational power (Raspberry Pi Zero W). The follower-UAVs take pictures of the ground objects and send these pictures (without any pre-processing) to the leader-UAV using Wi-Fi. The leader-UAV has two intertwined roles: i) coordinating the trajectory of the follower-UAVs; ii) coordinating the communication in the Wi-Fi network. The leader-UAV processes the images received from the follower-UAVs, infers the position of the objects, and sends trajectory updates. Notice that keeping the information at the leader-UAV as fresh as possible is central for this closed-loop control system to operate satisfactorily. To achieve information freshness, WiSwarm leverages the 

%WiSwarm has three main components: (i) a predictive agent control mechanism that sends sequences of control updates to the agents, assisting them to adapt to changes in the environment; (ii) a multi-agent polling mechanism that reduces packet collisions, especially when the large-scale wireless network is congested; (iii) a custom queueing discipline. 

%Our contributions: 1) we model this object monitoring problem using the Age-of-Information metric; 2) we propose a simple network architecture called WiSwarm that solves both the scaling and congestion challenges associated with WiFi; 3) despite the apparent mismatch between the proposed architecture and standard WiFi, we propose a strategy to implement WiSwarm at the application layer, over UDP and standard WiFi. In essence, our implementation drives WiFi to behave as WiSwarm, without modification to lower layers of the communication system; 4) Our experimental results show that WiSwarm improves.. by 100 times. To the best of our knowledge, this is the first work to propose and experimentally evaluate Wireless Networking for Real-Time Simultaneous Monitoring and Control Using Heterogeneous Swarms of UAVs.


%\textbf{Object tracking system.} Introduce the big picture of our system: 1 leader-UAV that controls N micro-UAVs that are monitoring N moving objects on the ground. Describe the underlying WiFi network. Clearly point out that scaling and congestion are two major challenges. Here we can already add some measurement result that shows that the swarm won’t be able to monitor the cars when we add too many UAVs.

%Consider a object tracking application with multiple-UAVs following numerous moving objects on the ground. Clearly, outdated information about the position of the objects has a direct impact on the capability of the UAVs of tracking the objects. Ideally, the UAVs would like to receive fresh information about the objects continuously. One simple system design that achieves this goal consists of UAVs with high on-board computational power that are able to process images taken from their cameras to detect and track objects. The continuous stream of images is processed locally, adding almost no delay, which keeps the UAVs updated about the position of the objects. A critical drawback of this approach is cost. In particular, the cost of deploying numerous UAVs with high on-board computational power may grow prohibitively high. 

%The leader-UAV has two intertwined roles: i) coordinating the trajectory of the follower-UAVs; ii) coordinating the communication in the Wi-Fi network. The leader-UAV processes the images received from the follower-UAVs, infers the position of the objects, and sends trajectory updates. Notice that keeping the information at the leader-UAV as fresh as possible is central for this closed-loop control system to operate satisfactorily. To achieve information freshness, WiSwarm leverages the 

%The remainder of this paper is organized as follows. In \textsection\ref{sec.Background}...

%Applications: managing wildlife, traffic monitoring, remote sensing, military, event streaming.