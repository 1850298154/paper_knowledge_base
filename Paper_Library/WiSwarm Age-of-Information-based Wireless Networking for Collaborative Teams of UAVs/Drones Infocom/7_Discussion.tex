\section{Discussion}\label{sec.Discussion}
We briefly discuss centralization and coexistence which are important aspects of the networking middleware.

\textbf{Centralization.} Our networking middleware is designed with the leader at its center. This centralization allows the leader to leverage information about both the application and the communication network to optimize resource allocation. Specifically, the leader takes into account the application-defined priorities $w_i(t)$, the state of the WiFi network $p_i(t)$, and the freshness of the information $A_i(t)$ to prioritize the transmissions that are most valuable to the application. Centralization is key to achieve the results discussed in Sec.~\ref{sec.Evaluation}. Another advantage of centralization is that it allows for computational offloading. The main disadvantage is that the leader becomes a single point of failure. To improve robustness, the system designer can deploy additional leaders behaving as followers. It is important to emphasize that the networking middleware does not require the underlying WiFi network to have a star topology. The middleware can also be deployed in mesh WiFi networks as long as there is only one leader middleware.% as discussed in Sec.~\ref{sec.Examples}.

\textbf{Coexistence.} Communication networks supporting multi-agent systems running time-sensitive applications should have priority in accessing networking resources in order to reduce the effects of interference. Our networking middleware runs over standard WiFi and, thus, it is prone to interference. In Sec.~\ref{sec.Evaluation}, WiSwarm is evaluated and validated in a campus space with multiple sources of interference such as WiFi base stations, mobile phones, and laptops. In practice, we expect the system designer to attempt to reduce external interference as much as possible. Ideally, the middleware should not coexist with other networks, which is justified by the fact that the middleware is designed to support a time-sensitive multi-agent system.

%Some results to discuss:
%\begin{itemize}
    %\item The fragment size plays an important role in the performance of UDP. For images of 53k bytes, if the entire image is sent at once using an UDP socket, then the unreliability of the wireless channel causes the transmission to fail. This happens despite the MAC layer re-transmissions. To overcome this challenge, we split the image into fragments of 10k (or less) and implement an acknowledgement system similar to TCP. 
    %\item When we turn on the system, TCP behaves badly at first, since it is still figuring out its optimal window. To improve the performance of TCP, we added 1 minute of warm up in which all tags are stationary.
    %\item Number of waypoints
    %\item Sleep between poll packets
%\end{itemize}


%\color{blue} V: moved. Naturally, the larger the number of transmissions per fragment needed, the lower the probability of successfully delivering the fragment. To reduce the number of transmissions per fragment, we can reduce the UDP payload size set by the application. However, in that case, we are increasing the number of fragments and, thus, the end-to-end delay of delivering information updates. Both the lower reliability and the higher delay affect information freshness. This trade-off is more prominent when the WiFi network is subject to external interference. We show experimental results and discuss this trade-off in Section~\ref{}.\color{black}