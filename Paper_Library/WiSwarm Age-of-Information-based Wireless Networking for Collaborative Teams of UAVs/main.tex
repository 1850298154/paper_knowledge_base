%\documentclass[12pt,draftcls,peerreview, onecolumn]{IEEEtran}
\documentclass[conference,letterpaper,10pt]{IEEEtran}

%\usepackage{svg}
\usepackage{framed}
\usepackage{tcolorbox}
\usepackage{amsmath}
%\usepackage{algorithm}
\usepackage{mathtools}
\usepackage{bbm}
%\usepackage[noend]{algpseudocode}
\usepackage[ruled,linesnumbered]{algorithm2e}
\usepackage{bm}
\usepackage{stackrel}
\usepackage{subfig}
\usepackage{epsf}
\usepackage{amsmath,amssymb}
\DeclareMathOperator*{\argmax}{arg\,max}
\DeclareMathOperator*{\argmin}{arg\,min}
\usepackage{graphicx}
\usepackage{color}
\usepackage{cite}
\usepackage{multirow,tabularx}
%\usepackage[bookmarks = false]{hyperref}
\usepackage{ifthen}
\usepackage{epstopdf}
\input{epstopdf.sty}
% Add these after the document class declaration

\usepackage{hyperref}
\hypersetup{
	colorlinks=true, % false: boxed links; true: colored links
    linkcolor=blue,  % color of internal links (change box color with linkbordercolor)
    citecolor=blue,  % color of links to bibliography
    filecolor=blue,  % color of file links
    urlcolor=blue    % color of external links
}
\usepackage{times}
\setlength{\parskip}{-0.07em}
\setlength{\belowcaptionskip}{-12pt}
%\setlength{\abovecaptionskip}{-3pt}
\input{epsf.sty}
%-------------------------------

\makeatletter
\def\BState{\State\hskip-\ALG@thistlm}
\makeatother

\newcommand{\showfigure}{\boolean{true}}
%\newcommand{\showfigure}{\boolean{false}}

\newcommand{\hide}[1]{\ifthenelse{\boolean{false}}{#1}{}}

%\include{../../commonHeader}
%%%%%%%%%%%%%%%%%%%%%%
% Theorems, etc.

\newtheorem{theorem}{{\bf Theorem}}
\newtheorem{conjecture}{{\bf Conjecture}}
\newtheorem{lemma}{{\bf Lemma}}
\newtheorem{proposition}{{\bf Proposition}}
\newtheorem{corollary}{{\bf Corollary}}
\newtheorem{observation}{{\bf Observation}}
\newtheorem{remark}{\textit{Remark}}
\newtheorem{myrule}{{\textit Rule}}
\newenvironment{proof}[1][Proof]{\begin{trivlist}
\item[\hskip \labelsep {\bfseries #1}]}{\end{trivlist}}

%\newtheorem{defn}{Definition}
\newenvironment{definition}[1][Definition]{\begin{trivlist}
\item[\hskip \labelsep {\bfseries #1}]}{\end{trivlist}}

\newenvironment{example}[1][Example]{\begin{trivlist}
\item[\hskip \labelsep {\bfseries #1}]}{\end{trivlist}}

%\newenvironment{remark}[1][Remark]{\begin{trivlist}
%\item[\hskip \labelsep {\bfseries #1}]}{\end{trivlist}}

\newcommand{\qed}{\nobreak \ifvmode \relax \else
      \ifdim\lastskip<1.5em \hskip-\lastskip
      \hskip1.5em plus0em minus0.5em \fi \nobreak
      \vrule height0.75em width0.5em depth0.25em\fi}


\newtheorem{fact}{Fact}
\newtheorem{obs}{Observation}

%%%%%%%%%%%%%%%%%%%%%%
% Environments

\newcommand{\beq}{\begin{equation}}
\newcommand{\eeq}{\end{equation}}
\newcommand{\barr}{\begin{array}}
\newcommand{\earr}{\end{array}}

\newcommand{\benum}{\begin{enumerate}}
\newcommand{\eenum}{\end{enumerate}}

\newcommand{\bit}{\begin{itemize}}
\newcommand{\eit}{\end{itemize}}

\newcommand{\bc}{\begin{center}}
\newcommand{\ec}{\end{center}}

\newcommand{\bdes}{\begin{description}}
\newcommand{\edes}{\end{description}}

\newcommand{\bfig}{\begin{figure}}
\newcommand{\efig}{\end{figure}}

\newcommand{\bemq}{\begin{quote} \begin{em}}
\newcommand{\eemq}{\end{em} \end{quote}}

\newcommand{\bmp}{\begin{minipage}}
\newcommand{\emp}{\end{minipage}}


%%%%%%%%%%%%%%%%%%%%%%
% References

\newcommand{\eqn}[1]{(\eqref{#1})}
\newcommand{\fgr}[1]{Fig.~\ref{#1}}
\newcommand{\tbl}[1]{Table~\ref{#1}}
\newcommand{\secref}[1]{Section~\ref{#1}}
\newcommand{\apndx}[1]{Appendix~\ref{#1}}
\newcommand{\app}[1]{Appendix~\ref{#1}}
\newcommand{\chpref}[1]{Chapter~\ref{#1}}
\newcommand{\lemref}[1]{Lemma~\ref{#1}}
\newcommand{\thmref}[1]{Theorem~\ref{#1}}

%%%%%%%%%%%%%%%%%%%%%%
% Brackets

\newcommand{\brac}[1]{\left({#1}\right)}
\newcommand{\sbrac}[1]{\left[{#1}\right]}
\newcommand{\cbrac}[1]{\left\{{#1}\right\}}

\newcommand{\floor}[1]{\left\lfloor{#1}\right\rfloor}
\newcommand{\ceil}[1]{\left\lceil{#1}\right\rceil}
%%%%%%%%%%%%%%%%%%%%%%

% Indicator function

\newcommand{\indic}[1]{I\left[{#1}\right]}

%%%%%%%%%%%%%%%%%%%%%%
% Superscripts
\newcommand{\supth}{^{{\mathrm{th}}}}
\newcommand{\kth}{^{{\mathrm{th}}}}
\newcommand{\supnd}{^{{\mathrm{nd}}}}
\newcommand{\suprd}{^{{\mathrm{rd}}}}


%%%%%%%%%%%%%%%%%%%%%%
% Combinatorics

\newcommand{\comb}[2]{\left(\begin{array}{c}
                                {\!\!#1\!\!}  {\!\!#2\!\!}
                             \end{array}
                        \right)}
\newcommand{\nchoosek}{\comb}

%%%%%%%%%%%%%%%%%%%%%%
% Symbols

\newcommand{\db}{{\mathrm dB}}
\newcommand{\Om}{\hat{\Omega}}
\newcommand{\define}{\triangleq}
%\newcommand{\define}{\stackrel{\triangle}{=}}
%\newcommand{\implies}{\Rightarrow}
%\newcommand{\tendsto}{\rightarrow}
\newcommand{\tendsto}{\to}
\newcommand{\degree}{^{\circ}}
\newcommand{\kroneck}{\otimes}

%%%%%%%%%%%%%%%%%%%%%%
% Special phrases

\newcommand{\ie}{{\it i.e.}}
\newcommand{\eg}{{\it e.g.}}
\newcommand{\etal}{{\it et al.}}
\newcommand{\wrt}{w.r.t.}
\newcommand{\vs}{{\it vs.}}

%%%%%%%%%%%%%%%%%%%%%%
% Matrix related

\newcommand{\tvec}{\text{vec}} % \vec is already defined
\newcommand{\mtx}[1]{{\bf #1}} % matrix
\newcommand{\mtxg}[1]{{\boldsymbol #1}} % matrix for greek symbol

\newcommand{\trp}{\text{T}} % transpose
\newcommand{\trc}[1]{\text{Tr}\left\{{#1}\right\}} % trace

%%%%%%%%%%%%%%%%%%%%%%
% Special matrices

\newcommand{\mi}[1]{\mtx{I}_{#1}}
\newcommand{\mzero}{\mtx{0}}

%%%%%%%%%%%%%%%%%%%%%%
% Principal sub-matrix

\newcommand{\princ}[2]{{#2}^{({#1})}}
\newcommand{\princsup}[3]{{{#2}^{({#1})^{#3}}}}

%%%%%%%%%%%%%%%%%%%%%%
% Probability related

\newcommand{\expect}[1]{{\bf E}\left[{#1}\right]}
\newcommand{\expectpow}[2]{{\bf E}^{#2}\left[{#1}\right]}
\newcommand{\prob}[1]{\text{Pr}\brac{#1}}
\newcommand{\pdf}{PDF}
\newcommand{\cdf}{CDF}
%%%%%%%%%%%%%%%%%%%%%%
% Derivatives

\newcommand{\deriv}[2]{\frac{d{#1}}{d{#2}}}
\newcommand{\pderiv}[2]{\frac{\partial{#1}}{\partial{#2}}}

%%%%%%%%%%%%%%%%%%%%%%
% Misc

\newcommand{\expc}[2][]{E^{#1}\left[{#2}\right]}
\newcommand{\ave}[1]{\overline{#1}}
\newcommand{\set}[1]{\{#1\}}
\newcommand{\ul}{\underline}
\newcommand{\smq}[1]{{\it #1}}
\newcommand{\recip}[1]{\frac{1}{{#1}}}

%%%%%%%%%%%%%%%%%%%%%%
% Slides

\newcommand{\bsp}{\begin{slide*}}
\newcommand{\esp}{\end{slide*}}
\newcommand{\bsl}{\begin{slide}}
\newcommand{\esl}{\end{slide}}
\newcommand{\vsp}[1]{\vspace{#1}}

%%%%%%%%%%%%%%%%%%%%%%%%%%
% Theorem
\newcommand{\blem}{\begin{lemma}}
\newcommand{\elem}{\end{lemma}}
\newcommand{\bthm}{\begin{theorem}}
\newcommand{\ethm}{\end{theorem}}





%%%%%%%%%%%%%%%%%%%%%%%%%%%%
% Notations
\newcommand{\iid}{{i.i.d.}}
\newcommand{\EX}[1]{\mathbb{E}\left[ #1 \right]} % expectation operator
\newcommand{\pr}[1]{\mathbb{P}\left[ #1 \right]}
\newcommand{\rt}[1]{{\bf RT: {#1}}}

\newcommand{\SetAgeRoutes}{\mathcal{R}}
% Title Page

\IEEEoverridecommandlockouts

\begin{document}


\title{%WiSwarm: Time-Sensitive Wireless Networking for a Collaborative Team of UAVs
WiSwarm: Age-of-Information-based Wireless Networking for Collaborative Teams of UAVs\color{black}}
\author{Vishrant Tripathi*, Igor Kadota*, Ezra Tal*, Muhammad Shahir Rahman, Alexander Warren,\\Sertac Karaman, and Eytan Modiano% <-this % stops a space
\IEEEcompsocitemizethanks{\IEEEcompsocthanksitem Vishrant Tripathi, Ezra Tal, Muhammad Shahir Rahman, Alexander Warren, Sertac Karaman, and Eytan Modiano are with the Laboratory for Information and Decision Systems (LIDS), Massachusetts Institute of Technology, Cambridge, MA, 02139. Igor Kadota is with the Department of Electrical Engineering, Columbia University, New York, NY, 10027. This work was supported in part by the NSF under grant CNS-1713725 and the ARO under grant W911NF1910322. A version of this work will be presented at IEEE INFOCOM 2023.\protect\\
E-mail: \{vishrant, eatal, shahir, warrena, sertac, modiano\}@mit.edu and igor.kadota@columbia.edu.\protect\\
*These authors contributed equally to this work
}% <-this % stops an unwanted space
%\thanks{This work was supported by NSF Grants AST-1547331, CNS-1713725, and CNS-1701964, and by Army Research Office (ARO) grant number W911NF-17-1-0508.}% <-this % stops a space
}
%\author{Vishrant Tripathi, and Eytan Modiano\\
%Laboratory for Information \& Decision Systems, MIT\\
%vishrant@mit.edu, modiano@mit.edu
%\thanks{This work was supported by NSF Grants AST-1547331, CNS-1713725, and CNS-1701964, and by Army Research Office (ARO) grant number W911NF-17-1-0508.}
%\vspace{-3mm}
%}


\IEEEaftertitletext{\vspace{-0.6\baselineskip}}

\maketitle
\begin{abstract}
%A major challenge in deploying collaborative multi-agent systems for time-sensitive applications such as search and rescue, aerial sensing, and smart factories is to manage limited wireless resources efficiently, especially as the network size scales and traffic load increases.
%
%Leveraging extensive theoretical research on Age of Information, 
The Age-of-Information (AoI) metric has been widely studied in the theoretical communication networks and queuing systems literature. However, experimental evaluation of its applicability to complex real-world time-sensitive systems is largely lacking. In this work, we develop, implement, and evaluate an AoI-based application layer middleware that enables the customization of WiFi networks to the needs of time-sensitive applications. By controlling the storage and flow of information in the underlying WiFi network, our middleware can: (i) prevent packet collisions; (ii) discard stale packets that are no longer useful; and (iii) dynamically prioritize the transmission of the most relevant information. To demonstrate the benefits of our middleware, we implement a mobility tracking application using a swarm of UAVs communicating with a central controller via WiFi. Our experimental results show that, when compared to WiFi-UDP/WiFi-TCP, the middleware can improve information freshness by a factor of 109x/48x and tracking accuracy by a factor of 4x/6x, respectively. Most importantly, our results also show that the performance gains of our approach increase as the system scales and/or the traffic load increases.  %demonstrating its applicability to real-world time-sensitive applications. %To the best of our knowledge, this is the first work to experimentally evaluate AoI-based resource allocation in a real-world deployment of a time-sensitive application.
\end{abstract}




%\vspace{-0.1cm}
\section{Introduction}
Time-optimal point-to-point motion planning, which involves transitioning the system from its current state to a desired terminal state in the shortest time while continuously satisfying stage constraints, finds wide applications in various fields such as robot manipulation, cranes, and autonomous navigation.
This problem can be approached in a two-level manner, comprising high-level geometric path planning and lower-level path following considering system dynamics (\cite{Verscheure09}). 
Alternatively, it can be directly solved in the system's state space, known as the direct approach (\cite{8264024}).
The direct approach commonly formulates the time-optimal motion planning problem as a receding horizon Optimal Control Problem (OCP) and is solved either offline or repeatedly in a Nonlinear Model Predictive Control (NMPC) implementation (\cite{Zhao04}).

In this paper, our focus is on direct approaches. 
One approach proposed in \cite{7331052} involves scaling the continuous-time system with a temporal factor before discretizing it.
The OCP formulated in this time scaling approach has a variable time grid with a fixed horizon length.
The temporal factor, treated as an additional decision variable, is minimized in the objective to achieve time-optimality.
% A work implementing this time scaling approach in the context of truck-trailer autonomous mobile robot parking maneuver planning is demonstrated in \cite{BOS20234877}.
When applied with a relatively small horizon length, the computational complexity remains manageable.
Yet, the time grid is typically coarse when the time needed to reach the terminal state is long, leading to a correspondingly coarse motion trajectory.
In accordance with the discrete-time control system, the resulting motion trajectory must undergo interpolation to align appropriately. 
Regrettably, this refinement process may give rise to infeasibility concerns attributed to interpolation errors.
An alternative approach proposed by \cite{8264024} chooses to use a fixed time grid, i.e., discretizing the system with the control sampling time.
This approach, referred to as the exponential weighting approach, opts for a fixed but much larger horizon length and minimizes the L1-norm of the deviation from the desired terminal state, weighted by exponentially increasing weights, to achieve time-optimality.
A limitation of the exponential weighting approach arises when the system transitions to a distant terminal state. 
The considerable horizon length poses computational challenges for real-time implementation and introduces numerical ill-conditions due to the exponentially increasing weights.

We propose a two-stage approach to formulate the time-optimal OCP. 
This approach involves a first stage with a fixed time grid corresponding to the control sampling time and a second stage with a variable time grid derived from the time-scaled system.
It leverages the advantages of both aforementioned approaches by formulating the time-optimal OCP with a fixed and low number of control steps for computational manageability and preempting interpolation errors in the first stage.
Solving the OCP formulated through this two-stage approach using the classical NMPC implementation scheme — specifically, solving the OCP within a single control sampling time and applying the first optimal control from stage 1 to the system — can still be challenging, particularly in scenarios characterized by complex system models and stage constraints.
To ensure complete convergence in every NMPC iteration, we employ the ASAP-MPC update strategy, an asynchronous NMPC implementation scheme (\cite{ASAP-MPC}) that is designed to handle the fluctuating computational delays.

\textit{Paper structure:} In Section \ref{Sec_II}, we introduce the time-optimal point-to-point motion planning problem and discuss two approaches to formulate it: the time scaling approach and the exponential weighing approach.
In Section \ref{Sec_III}, we state the proposed two-stage approach and subsequently present a scheme that integrates this two-stage approach with the ASAP-MPC update strategy to handle fluctuating computation delays.
In Section \ref{Sec_IV}, we compare the three approaches and showcase the presented NMPC scheme through numerical examples of autonomous navigation while avoiding collisions with obstacles.
Section \ref{Sec_V} concludes the paper.

\textit{Notation:} The set of positive integer numbers is denoted by $\mathbb{N}_+$.
The L1-norm of a variable $s$ is denoted by $\|s\|_1$.
The sequence of a variable $s$ is denoted by 
$\{s\}_{n=0}^N:=s_0,s_1,...,s_N$. 

%\vspace{-0.1cm}
%\section{Motivating Applications}\label{sec.Examples}

%We describe two representative examples of time-sensitive applications and how they could benefit from the networking middleware proposed in this work. 
%We describe how the proposed networking middleware could benefit two representative time-sensitive applications. %and how they could benefit from the networking middleware proposed in this work. 

We use two representative examples of time-sensitive applications to discuss the simplicity of integration and performance gains of the networking middleware.

%Two examples of this important class of applications are \emph{automated search and rescue} (SAR) missions and \emph{automated exploration of unknown environments} using collaborative teams of UAVs and ground robots. 

\textbf{Search and Rescue (SAR).} Consider the SAR mission described in \cite{SAR} in which a team of autonomous UAVs are supervised by a human rescue operator. The operator provides sparse high-level inputs to the UAVs via a Human-Robot Interaction (HRI) system. The HRI system displays real-time information about the mission and can be used to define regions of high/low interest, switch between search patterns (e.g., from expanding square search to parallel track search~\cite[Sec.~3]{SARManual}), modify UAV trajectories that were autonomously planned, and, if necessary, directly teleoperate the UAVs. The authors of \cite{SAR} propose an HRI system that allows the operator to interact with the UAVs in an efficient way. The HRI system is tested using two UAVs equipped with on-board cameras and connected to the operator via a WiFi base station.

%NATO, ATP-10 (C). Manual on Search and Rescue. Annex H of Chapter 6, 1988.
% Information freshness is clearly central to the automated SAR mission in \cite{SAR}, as outdated mission information at the HRI system may mislead the rescue operator, and outdated trajectory information at the UAVs may degrade group coordination, increasing the risk of UAV collisions. In addition to information freshness, operator-defined priorities are also important for the success of the SAR mission. For example, images from UAVs within high interest regions should have high priority and trajectory updates to teleoperated UAVs should also have high priority.
% %
% %In a SAR mission with a small-scale underloaded WiFi network (e.g., an HRI system with one or two UAVs), most data packets from all types of traffic should be delivered without delay, irrespective of their priority, thus, satis. 
% As we add more UAVs to the HRI system, the traffic load in the network increases, leading to a sharp degradation of WiFi's performance, which  can rapidly become insufficient to support the SAR mission. 

% Our proposed networking middleware can be easily integrated into the SAR mission in \cite{SAR} to improve its scalability by alleviating congestion in the WiFi network and by controlling the information flow. Specifically, the centralized resource allocation mechanism prevents packet collisions and can be programmed to adaptively prioritize transmissions, taking into account both information freshness and operator-defined priorities, in a similar way as the description in Sec.~\ref{sec.Scheduler}.


\textbf{Exploration of underground environments.} The DAR\-PA Subterranean Challenge is an international competition in which research teams develop multi-robot systems that can \emph{rapidly} search for, detect, and geolocalize artifacts (e.g., manikins, cellphones, and backpacks) placed in environments such as tunnels, mines, and caves. The CERBERUS team (who won the competition in 2021) deployed a system \cite{CERBERUS} with five robots: (i) two aerial robots that can explore spaces inaccessible to ground robots; (ii) two legged robots with long endurance and the ability to deploy WiFi breadcrumbs; and (iii) one wheeled roving robot with an on-board directional WiFi antenna and a $300$\thinspace{m}-long optical fiber reel connected to the WiFi base station located outside of the underground environment. Each robot is equipped with multiple cameras (regular, FLIR, and/or LIDAR) and powerful compute (CPUs, GPUs, and/or VPUs) that enable on-board multi-modal terrain perception, object detection, and path planning for exploration. The base station, the wheeled roving robot, and the WiFi breadcrumbs create a mesh WiFi network that allows robots to share maps and artifact detections with the human supervisor at the base station. The supervisor can teleoperate the roving robot (via the fiber connection) and send high-level controls to override the autonomously planned robots' behavior. Notice from the experimental evaluation in \cite[Sec.~10]{CERBERUS} that not all five robots were deployed in every mission run. The performance of the WiFi network was identified as a critical challenge. %in the experimental evaluation. %described in \cite[Sec.~10]{CERBERUS}.

Automated SAR \cite{SAR} and automated exploration \cite{CERBERUS} are clear examples of how information freshness can be important for time-sensitive applications. In both applications, outdated contextual information at the WiFi base station may mislead the human supervisor, decreasing the chances of finding survivors or artifacts, and outdated trajectory/control information at the robots may degrade group coordination, increasing the chances of collisions. In addition to information freshness, application-defined priorities also play an important role. For example, transmission of control updates should have high priority, and images from robots within high interest regions should have higher priority than images from low interest regions. 
%
In small-scale applications with an underloaded WiFi network (e.g., a SAR mission with one or two UAVs), most data packets from every source should be delivered promptly. However, as the application size scales, the sharp degradation of WiFi's performance drives the need for a networking solution that can alleviate congestion and prioritize the transmissions that are most valuable to the application, taking into account both information freshness and application-defined priorities.

The networking middleware proposed in this paper can be directly used to tailor WiFi to the needs of automated SAR \cite{SAR} and automated exploration \cite{CERBERUS}, without requiring hardware modifications. Specifically, the middleware for the leader compute node (described in Sec.~\ref{sec.Scheduler}) can be deployed at the operator's terminal, next to the WiFi base station, and the middleware for the followers (described in Sec.~\ref{sec.Queueing}) can be deployed at the autonomous robots. Similarly to the results obtained in Sec.~\ref{sec.Evaluation} for the mobility tracking application, we expect that our networking middleware can significantly improve throughput, information freshness, and scalability of the underlying WiFi networks in \cite{SAR} and \cite{CERBERUS}. %, similar to the results obtained in Sec.~\ref{sec.Evaluation} for the mobility tracking application.

%Our proposed networking middleware can be easily integrated into the automated SAR \cite{SAR} and the automated exploration \cite{CERBERUS} systems without hardware modifications. In particular, the programmable resource allocation mechanism described in Sec.~\ref{sec.Scheduler} can be deployed at the terminal used by the human operator (next to the WiFi base station) and the customizable information queues described in Sec.~\ref{sec.Queueing} can be deployed at the autonomous robots. Similarly to the results achieved for the mobility tracking application in Sec.~\ref{sec.Evaluation}, we believe that our networking middleware can improve information freshness and customize the underlying WiFi network to the needs of the specific time-sensitive application, thus, enabling the wireless network to support larger-scale time-sensitive systems.

%Add citations from Search and Rescue literature and from the DARPA SubT Challenge. Describe the automated search example in detail. Discuss the \emph{direct} relationship between automated search and object tracking. Discuss the importance of information freshness. Our system could have been directly used for this purpose. \color{blue} Insight from Luca: even if communication is not a major constraint, human validator becomes a constraint and requires scheduling. Not crucial, but might be worthwhile to add.\color{black}

%For example, in an autonomous search and rescue mission using a UAV swarm, the team of UAVs should collaboratively explore a geographical location in order to find people in distress as fast as possible. It is essential that the underlying communication network allows the team of UAVs to share information about the environment, e.g. images of the ground, and about themselves, e.g. position, attitude, and heading of the UAVs, in a timely manner. Notice that outdated information loses its value and can lead to system failures. Designing communication networks that can support time-sensitive applications that rely on collaborative multi-agent systems is clearly a challenging task with a high societal impact.
%The underlying communication network should... [Time is at the essence... \url{https://www.huffpost.com/entry/11-robotic-applications-for-search-and-rescue_b_5a173c9ae4b0bf1467a845c4}]. 

%\subsection{Smart Factories and Smart Cities}\label{sec.SmartApplications}
%\igor{Describe these two examples briefly. Show that our system could be used to improve the performance. Some adaptations may be necessary.} 
%\subsection{Computational Offloading}

%\textbf{Automated Fulfillment Warehouses.} Describe Amazon/Alibaba Warehouses. Add citations showing that they are using WiFi. Discuss how our system could be used to improve information freshness and allow for a better rumba/router ratio.

%\textbf{Smart-City Intersection.} Describe Smart-City intersection. Discuss how improved information freshness is needed. Add citations showing that US DoT and NY DoT are planing to use WiFi for V2X communication. Say that in a Smart-City intersection using WiFi-like technology (i.e. DSRC) could benefit from this type of system.
%\vspace{-0.1cm}
\section{Networking Middleware for Information Freshness}\label{sec.Middleware}

%% Óutline in comments

%\igor{In this section, we discuss the design principles of the middleware. In particular, AoI, LIFO queueing, Max-Weight, and Dynamic Weights. Then, we describe the basic functions and information flow in Fig.~\ref{fig.overview}. In the next section, we will discuss the design and implementation of WiSwarm which is an instantiation of the networking middleware for an object tracking application.}
\begin{figure}
	%\captionsetup{justification=centering}
	\centering
	\includegraphics[width=\linewidth]{Images/overview_drones_v2.png}
	%\vspace{-0.4cm}
	\caption{AoI-based application layer Networking Middleware.}% that provides information freshness for applications with a team of followers and a single leader.}%\igor{Add sections, store data at queue, and time-stamp.}}
	\label{fig.overview}
	%\vspace{-0.4cm}
\end{figure}

%In this section, we formally define the Age-of-Information metric and discuss the design principles of the networking middleware. Figure~\ref{fig.overview} provides an overview of its architecture, which is motivated by the need for information freshness in large-scale time-sensitive applications. %The networking middleware is designed to enable standard WiFi networks to support large-scale applications that rely on a team of agents, called followers, collecting and transmitting time-sensitive information to a central compute controller which is responsible for centralized computation and coordination among the team. In this important class of applications, it is imperative to keep the leader as updated as possible, since outdated information loses its value and can lead to poor control and coordination. To achieve this design goal, the key idea is to dynamically control the flow of information in the WiFi network in order to: (i) avoid network congestion and packet collisions; (ii) prioritize transmissions from the most relevant followers/sources; and (iii) at each source, prioritize the transmission of the most relevant pieces of information. 

%In this section, we describe the networking middleware 
In this section, we describe a networking middleware (illustrated in Fig.~\ref{fig.overview}) that customizes WiFi to the needs of the important and broad class of time-sensitive applications that rely on multi-agent systems. In these applications, agents (also called followers) collect and transmit time-sensitive information to a central compute node (also called the leader). 
%composed of multiple agents, called \emph{followers}, and a central compute node, called \emph{leader}. Each follower collects and transmits time-sensitive information to the leader. 
The leader consolidates the received information and coordinates the followers' behavior in a timely manner. Naturally, it is critical to keep information in the network as fresh as possible. 
%
%Figure~\ref{fig.overview} provides an overview of the middleware architecture, which is motivated by the need for information freshness. 
We formally define the Age-of-Information metric in Sec.~\ref{sec.AoI}. In Sec.~\ref{sec.MiddlewareDesign}, we describe the middleware design based on the considerations in Secs.~\ref{sec.AoI}, \ref{sec.Queueing}, and~\ref{sec.Scheduler}.

%To customize WiFi to the needs of the application, we propose a middleware that can control the storage and flow of information in the WiFi network in order to: (i) avoid network congestion and packet collisions; (ii) prioritize transmissions from the most relevant followers, taking into account both information freshness and application-defined priorities; and (iii) within each follower, eliminate packets that are no longer useful to the application and prioritize transmissions of the most relevant packets. Figure~\ref{fig.overview} provides an overview of the middleware's architecture.

% %In this section, we discuss the design principles of the middleware. Figure~\ref{fig.overview} provides an overview of its architecture. %, which is motivated by the need for information freshness in large-scale time-sensitive applications. 
% The networking middleware is designed to enable standard WiFi networks to support large-scale time-sensitive applications that rely on multi-agent systems composed of \emph{several followers} and a central compute node, called \emph{leader}. Each follower collects and transmits time-sensitive information to the leader. The leader consolidates the received information and coordinates the followers' behavior in a timely manner. 
% Naturally, in this important class of time-sensitive applications, it is critical to keep the information as fresh as possible. %leader as updated as possible, since outdated information loses its value and can lead to poor control and coordination. 
% To achieve this goal, we propose a programmable middleware that can control the storage and flow of information in the WiFi network in order to: (i) avoid network congestion and packet collisions; (ii) prioritize transmissions from the most relevant followers, taking into account both information freshness and application-defined priorities; and (iii) within each follower, eliminate packets that are no longer useful to the application and prioritize transmissions of the most relevant packets. Figure~\ref{fig.overview} provides an overview of the middleware's architecture.

%Specifically, to avoid network congestion that can lead to excessive packet collisions in the WiFi network, the middleware drives the CSMA-based network to behave like a \emph{polling-based network}. To dynamically prioritize the most relevant followers, the middleware employs an \emph{application-centric transmission scheduler}. This scheduler uses Age-of-Information (a metric that measures information freshness) and application specific scheduling weights that describe the relative importance of each follower to design an efficient scheduling policy. We formally define the Age-of-Information metric in Section~\ref{sec.AoI} and discuss the scheduler in detail in Section~\ref{sec.Scheduler}. Finally, to prioritize what information to transmit from each follower, we eliminate stale pieces of information before they are ever transmitted. Our middleware achieves this by employing a tail drop information queue with an \emph{application-centric queueing discipline} (typically Last-In-First-Out). An important feature of our networking middleware is that it controls the information flow in the network from the application layer, without requiring modifications to lower layers of the networking protocol stack.

%Prior to delving into the details of the design of the networking middleware for information freshness,
%We present in Section~\ref{sec.AoI} the definition and intuition behind the Age-of-Information metric. Then, in Sec.~\ref{sec.Queueing} we discuss the queueing discipline employed at the followers/sources. Next, in Sec.~\ref{sec.Scheduler}, we discuss the application-centric transmission scheduler that runs at the leader/controller. Finally, in Sec.~\ref{sec.MiddlewareDesign}, we provide an overview of the middleware design, describing its basic functions and the flow of information over the WiFi network.

\subsection{Age-of-Information Metric}
\label{sec.AoI}
\begin{figure}
	%\captionsetup{justification=centering}
	\centering
	\includegraphics[width=0.95\linewidth]{Images/AoI_continuous.png}
	%\vspace{-0.4cm}
	\caption{AoI evolution.} %The first update is generated at the source at time $t_1$ and is delivered to the destination at time $t'_1$. The destination now has information about the source that is $t'_1 - t_1$ old, so AoI drops to $A(t'_1) = t'_1 - t_1$.} %The second update is generated at time $t_2$ and delivered at $t'_2$. So, $A(t)$ increases linearly until time $t'_2$ and then drops to $A(t'_2) = t'_2 - t_2$.}
	\label{fig.AoI}
	%\vspace{-0.5cm}
\end{figure}
%\igor{outline in comments}
%\igor{Introduce the concept of Age-of-Information (continuous time) and the “Network Age-of-Information”. Connect Network Age-of-Information to monitoring objects on the ground. High Network Age-of-Information translates to stale information at the leader which results in outdated trajectory updates at the followers. Minimizing Network Age-of-Information yields good automated object tracking. Discuss AoI literature. Few system papers. No real-world applications. Here we are using AoI-based design to improve feedback control. Define time stamp. Define Age of Information. Discuss the intuition. Define Time-Average Age-of-Information. Relate Time-Average Age-of-Information and position uncertainty by giving one concrete example that can build intuition. Discuss that each AoI is associated with one type of information.}

AoI is an end-to-end metric that characterizes \emph{how old the information is from the perspective of the destination} \cite{kaul2012real}. Consider a multi-agent system in which updates from followers are time-stamped upon generation. Let $\tau_i(t)$ be the \emph{largest time-stamp} of an update from follower~$i$ received by the leader by time~$t$. The AoI associated with follower~$i$ is defined as $A_i(t):=t-\tau_i(t)$. The AoI increases linearly with time when no updates are delivered, representing the information getting older. At the moment a \emph{fresher} update from follower~$i$ is received by the leader, the value of $\tau_i(t)$ increases and the AoI reduces to the delay of the received update. This evolution of the AoI metric with time is illustrated in Fig.~\ref{fig.AoI}. 

%\textbf{Illustrative example:} Consider the mobility tracking application described in Sec.~\ref{sec.Intro}. If the AoI associated with the last known position of a moving object is $A(t)=1.5$\thinspace{seconds}, this means that the object has been moving around for $1.5$\thinspace{seconds} without the leader knowing about it. Clearly, a larger AoI corresponds to the leader having a higher uncertainty about the current position of the object. Similarly, a larger average object velocity also corresponds to the leader having a higher position uncertainty. Therefore, to support mobility tracking applications, the underlying communication network should strive to keep the AoI associated with the position of every moving object as low as possible, further prioritizing objects with higher velocities. The networking middleware proposed in this work allows for the design of resource allocation mechanisms based on AoI and estimated velocities, as described in Secs.~\ref{sec.Scheduler} and \ref{sec.LeaderNode}.

%AoI has been receiving increasing attention in the literature for its application in communication systems that carry time-sensitive data such as position, command and control, or sensor information. In recent years, several theory-oriented papers have analyzed AoI in queuing systems \cite{kaul2012real, kam2013age, huang2015optimizing, inoue2018general, yin17_tit_update_or_wait, bedewy2019minimizing, AoI_management} and proposed novel network control mechanisms \cite{kadota2018scheduling2, tripathi2017age, talak2018optimizing, jhun2018age, farazi2018age, tripathi2019whittle, tripathi2021online} that could potentially be leveraged in real-world applications. A related line of work has looked at AoI as a metric for monitoring and control over networks \cite{sun2017remote,ornee2019sampling, champati2019performance, klugel2019aoi}. However, as we discuss in Sec.~\ref{sec.RelatedWork}, only a few \cite{AoI_measure_1,AoI_measure_3,kadota2021age,kadota2021wifresh,shreedhar2018acp,AoI_Wierman,AoI_SDR,ayan2021experimental} have considered system implementations. 

Over the past decade there has been a rapidly growing body of works analyzing AoI in different contexts (see surveys in \cite{yates2021age,kosta2017age,sun2019age_book}). Several theory-oriented papers have analyzed AoI in queuing systems \cite{kaul2012real, kam2013age, huang2015optimizing, inoue2018general, yin17_tit_update_or_wait, bedewy2019minimizing} and proposed novel network control mechanisms \cite{kadota2018scheduling2, talak2018optimizing, maatouk2020optimality, tripathi2019whittle, tripathi2021online,jhun2018age,farazi2018age} that could potentially be leveraged in real-world applications. %A related line of work has looked at AoI as a metric for monitoring and control over networks \cite{sun2017remote,ornee2019sampling, champati2019performance, klugel2019aoi}. 

More recently, a few works \cite{AoI_measure_1,AoI_measure_3,shreedhar2018acp,AoI_Wierman,AoI_SDR,ayan2021experimental,kadota2021age,kadota2021wifresh} have considered system implementations. These system-oriented works can be separated into two categories: (i) measurement of AoI in real networks %using devices connected via Ethernet, WiFi, or LTE
\cite{AoI_measure_1,AoI_measure_3}; and (ii) evaluation of communication networks that attempt to minimize AoI by looking at congestion control \cite{shreedhar2018acp}, traffic engineering \cite{AoI_Wierman}, and medium access using Software Defined Radios (SDRs) \cite{ayan2021experimental,kadota2021age,kadota2021wifresh,AoI_SDR}. However, there has been no prior work on the experimental evaluation of the impact of an AoI-based networking solution in a real-world time-sensitive application, which is the focus of this work.
%there has been no prior work on minimizing AoI at the application layer or making readily deployable systems for real applications. 
%\textbf{To the best of our knowledge, this is the first work to (i) develop an AoI-based solution that can be easily deployed on standard WiFi networks and (ii) to evaluate its performance using a real-world time-sensitive application.}

%\igor{Add example from mobility tracking?}
%An important feature of the AoI metric is that it captures the freshness of information from the perspective of the application, in contrast to the long-established packet delay metric, that represents the freshness of the information with respect to individual packets. AoI effectively tries to represents a measure of distortion between the state of the source that is expected at the destination based on past updates and the actual current state of the source. Thus, a larger AoI corresponds to the monitor having a higher uncertainty about the current state of the source being observed. This, in turn, means that ensuring a low average AoI can lead to higher monitoring accuracy and better control performance.
%\autoref{fig.AoI} shows how AoI evolves for an example update generation and delivery process. Consider the first update that is generated at the source at time $t_1$. At the instant that this update is generated, it is ``fresh". This update is eventually delivered to the destination at time $t'_1$. At the time of delivery, the destination now has information about the source that is $t'_1 - t_1$ old, so AoI drops to $A(t'_1) = t'_1 - t_1$. The second update is generated at time $t_2$ and delivered at $t'_2$. So, AoI $A(t)$ increases linearly up till time $t'_2$ and then drops to $A(t'_2) = t'_2 - t_2$. This evolution leads to the saw-tooth style plot depicted in \autoref{fig.AoI}.  

%An interesting feature of the AoI metric is that it captures the freshness of information from the perspective of the application, in contrast to the long-established packet delay metric, that represents the freshness of the information with respect to individual packets. AoI effectively tries to represents a measure of distortion between the state of the source that is expected at the destination based on past updates and the actual current state of the source. Thus, a larger AoI corresponds to the monitor having a higher uncertainty about the current state of the source being observed. This, in turn, means that ensuring a low average AoI can lead to higher monitoring accuracy and better control performance.

%In particular, AoI measures the time elapsed since the generation of the packet that was most recently delivered to the destination, while packet delay measures the time elapsed from the generation of a packet to its delivery.

%For the setting of our interest, there are $N$ followers sending status updates to the central leader, which in turn, performs control and coordination. The process $A_i(t)$ denotes the AoI of the $i^{th}$ follower at the central node. The followers send updates over a shared wireless channel, thus requiring scheduling to prevent collisions and deliver fresh information. \color{blue} Mention AoI scheduling literatures

%\color{blue} \textit{Vishrant: move WiFresh discussion to sec 6.}\color{black} %For example, the authors of \cite{} proposed a MAC layer architecture based on Polling Multiple Access mechanism and Last-Come First-Served queues, implemented this architecture in a Software Defined Radio testbed, and showed that it achieves near optimal information freshness.\color{black}

\subsection{Customizable Queueing at the Followers}\label{sec.Queueing}
%\igor{Discuss the effect of queueing discipline on the AoI. FIFO versus LIFO. Effect of dropping packets. Discuss issues with fragmentation: (i) AoI reduction when a subset of fragments is received and (ii) fragmentation and LIFO that replaces packets while transmitting their fragments. An implementation of fragmentation will be discussed in Sec.~\ref{sec.MiddlewareDesign}]. Recall that the queue is implemented at the Application layer. Hence, it is easy to implement queueing disciplines tailored to the specific application.}
Data generation and queueing have significant impact on information freshness. The \emph{follower middleware} architecture illustrated on the left in Fig.~\ref{fig.overview} receives updates at rates that are determined by sensors/applications, then it time-stamps and enqueues these updates. Upon receiving a polling request from the leader, the follower middleware releases a \emph{single} update via UDP/IP to lower layers of the network protocol stack. Our middleware incorporates two key ideas from the AoI literature to enable information freshness - a mechanism to control the update generation rate, and an implementation of Last-In First-Out (LIFO) queues.  %allows the system designer to employ different queueing disciplines and rate control to store/drop/prioritize updates depending on the application. %Next, we discuss potential queueing disciplines and their impact on information freshness. 

%Two important factors that control the AoI and consequently information freshness are the rate at which a source generates updates and the queuing discipline employed in sending these updates. 
First-In First-Out (FIFO) queues are the default queuing implementation in most communication networks. %Consider a middleware implementation utilizing FIFO queues to store information updates at the followers before transmitting them to the leader.
However, to manage AoI, they require careful control of the arrival rate. If updates are generated at a low rate, then the information updates are too infrequent. On the other hand, if updates are generated at a very high rate, then the FIFO queue will often be backlogged and fresh updates will have to face large queueing delays. To address this problem, we implement a rate control mechanism at the followers that can be used when applications use FIFO queues.
%This queueing delay leads to outdated information and, thus, high AoI. When FIFO queues are employed, it is imperative that the generation rate is carefully controlled to minimize AoI. 

\textbf{Rate Control}: To adjust the update generation rate, the rate control mechanism %that selectively discards updates. 
only updates its queue at fixed intervals of time, dropping any updates generated in between. This mechanism ensures that the middleware only accepts new updates at the desired rate. 
Note that finding the optimal generation rate for a given network setup is a nontrivial task, as the optimal rate depends on the network's topology, traffic load, link reliability, and Medium Access Control (MAC) mechanism. To illustrate the impact of the generation rate on information freshness, we plot in Fig.~\ref{fig.plot_wifi_1}(a)  the AoI of a standard WiFi system (that uses FIFO queues at the MAC layer) with different update generation rates, including the optimal rate which is obtained by grid search.  

%Thus, for FIFO queues, the followers need to carefully select their update generation rate to keep information fresh at the destination. For example, in a FIFO M/M/1 queue: to minimize queuing delay, the rate must be as small as possible (close to zero) while to optimize throughput the rate must be as close to the service rate of the queue $\mu$. On the other hand, to minimize AoI, the optimal generation rate turns out to be approximately $0.53\mu$ where $\mu$ is the service rate of the queue \cite{kaul2012real}. 

%Consider a source utilizing a FIFO queue to store generated updates before transmitting them. Very low generation rates would mean too few update deliveries to the destination leading to high inter-update times and high AoI, while very high generation rates would mean excessive queuing leading to large delays and high AoI. Thus, for FIFO queues, the sources need to carefully select their update generation rate to keep information fresh at the destination. For example, in a FIFO M/M/1 queue: to minimize queuing delay, the rate must be as small as possible (close to zero) while to optimize throughput the rate must be as close to the service rate of the queue $\mu$. On the other hand, to minimize AoI, the optimal generation rate turns out to be approximately $0.53\mu$ where $\mu$ is the service rate of the queue \cite{kaul2012real}. 

%Note that high generation rates in networks with many sources will also lead to frequent packet collisions and congestion making performance even worse, which further highlights the need for rate control. Our framework allows the application designer to specify rates of generation and queueing discipline for each source to control information freshness and reduce congestion in the network.

%Last-In First-Out (LIFO) queues transmit the most recently generated packet first, making them ideal for applications that rely on the knowledge of the \emph{current state} of the system, such as mobility tracking. LIFO queues were shown to be \emph{optimal} for minimizing AoI in a wide variety of network settings \cite{bedewy2019minimizing}. However, LIFO queues are rarely implemented at the transport, MAC, or physical layers in practice. Consider a follower middleware utilizing a LIFO queue. If updates are generated at a high rate, the LIFO queue would frequently replace its head-of-line packet with fresh packets. It follows that, the higher the update generation rate at the followers, the fresher the information received by the leader and, thus, the lower the AoI. For this reason, LIFO queues eliminate the need for adjusting the update generation rate.

\textbf{LIFO Queues:} Last-In First-Out (LIFO) queues transmit the most recently generated update first, making them ideal for applications that rely on the knowledge of the \emph{current state} of the system, such as mobility tracking. When an update is generated, the LIFO queue simply replaces the old head-of-line update with the fresh update. %whenever they arrive. 
A higher update generation rate at the followers can only lead to fresher updates at  the leader and, hence, a lower AoI. %For this reason, LIFO queues eliminate the need for adjusting the update generation rate. 
LIFO queues have been shown to be \emph{optimal} for minimizing AoI in a wide variety of network settings \cite{bedewy2019minimizing,AoI_management}. However, LIFO queues are rarely implemented at the transport, MAC, or physical layers in practice. Our middleware supports both FIFO and LIFO queues at the application layer, while also supporting rate control, providing the system designer with two important tools to manage AoI. %For example, encoded video that needs packets to be delivered sequentially can be stored in FIFO queues with appropriate rate control, while raw video frames that don't need sequential updates can be stored in a LIFO queues with the highest update rates.

%In general, delivering \emph{older} information updates to the leader after a \emph{fresher} update was successfully received does not improve information freshness. Hence, discarding older updates when a fresh update arrives at the follower middleware %i.e. before the older update is ever transmitted, 
%could save network resources, alleviating congestion. On the other hand, older information may still be useful to the application. For example, in a mobility tracking application, older position information can be useful in predicting future movement. This trade-off should be considered by the system designer when deciding whether or not to discard older updates at the follower middleware.

%The follower middleware receives updates from different sensors/applications and enqueues them according to the rules set forth by the system designer. Specifically, the system designer can choose between different queueing disciplines, e.g., LIFO, FIFO, Priority Queueing, and Fair Queueing, and set the rate at which this queue accepts new updates from the application. %Moreover, the designer can choose to implement different strategies for updates from different sensors/applications. 
%For example, encoded video that needs frames to be delivered sequentially can be stored in a FIFO queue, while raw video frames can be stored in a LIFO queue that keeps only the freshest update. 

%For applications where consecutive updates cannot be dropped easily such as encoded video data, FIFO queues can be used with appropriately chosen generation rates. In applications involving transmission of event-triggered data or safety information, a priority queue with appropriately chosen priorities for events can be implemented.

%It is also well known in the AoI community that the pre-emptive Last-Come-First-Serve queuing discipline is \textit{optimal} for minimizing AoI in general networks \cite{bedewy2019minimizing}. However, LIFO queues are rarely implemented at the transport or MAC layers in practice. So, our framework allows the application designer to choose between three different queueing discplines at the application layer - FIFO, LIFO and priority queues. We also allow control of the buffer size of the queues. We avoid implementing pre-emption since that would involve stopping a transmission that has already started at the lower layers, which is not feasible due to signaling overheads. \igor{Perhaps, most importantly, it would envolve changing lower layers of the networkiung protocol stack.} Thus, for applications in which older updates can be dropped for newer updates, \textit{the choice of LIFO queues with a single update buffer ensures the best performance}. This is because whenever a newer update arrives, it replaces an older buffered update if there was one waiting, without interrupting the current update being transmitted. For applications where consecutive updates cannot be dropped easily such as encoded video data, FIFO queues can be used with appropriately chosen generation rates. In applications involving transmission of event-triggered data or safety information, a priority queue with appropriately chosen priorities for events can be implemented.



%\color{red} Vishrant: \textit{maybe avoid discussion on fragments here} and keep it to 3.4. cite paper - pre-emptive LCFS minimizes AoI. Possibly also discuss application layer control of the rate at which sources send updates to the queue. In case of applications which must have FIFO + WiFi, rate optimization becomes crucial to control congestion.\color{black}

%\igor{Great point, Vishrant!}
%\vspace{-0.1cm}
\subsection{Customizable Transmission Scheduling at the Leader}\label{sec.Scheduler}

%Consider a time-sensitive multi-agent system with several followers and a leader. % while the leader coordinates the followers' behavior by transmitting control updates in a timely manner.
The multiple access mechanism controls the method by which followers and leader share information using the limited communication resources. WiFi employs a distributed random access mechanism that works well for small-scale underloaded networks. However, for large-scale congested networks, it leads to excess packet collisions that in turn lead to lower throughput and higher latency, and ultimately poor performance for real-time applications.%which can result in degraded information freshness.

%One approach to reducing congestion and packet drops is to implement congestion control at the application \cite{kadota2021age} or transport layers \cite{shreedhar2018acp}. However, these approaches do not address the issue of medium access. Distributed congestion control cannot eliminate packet collisions at the MAC layer, which are a major cause of high latency and poor scalability of 802.11 WiFi. Further, it is not trivial to find the optimum rate to generate information updates or to develop a scheme that adaptively finds the optimum operation point.  

We design the \emph{leader middleware}, illustrated on the right in Fig.~\ref{fig.overview}, to: (i) prevent packet collisions; (ii) enable dynamic prioritization of the transmissions that are most valuable to the application; and (iii) facilitate integration with existing multi-agent systems that use WiFi. %To achieve these goals, we propose the \emph{leader middleware} illustrated on the right in Fig.~\ref{fig.overview}.
The middleware drives the underlying distributed WiFi network to behave like a centralized network with support for polling. Specifically, the leader middleware coordinates the flow of information in the network by sending polling packets to the followers selected for transmission. At every decision time $t$, the leader selects the next follower to poll based on an application-centric \emph{transmission scheduling policy} $\pi$, which can be a function of the current AoI of the followers $A_i(t)$, the reliability of the WiFi links $p_i(t)$, where $p_i(t)\in(0,1]$ represents the probability of a successful transmission from follower $i$ to the leader, and the application-defined priority weights $w_i(t)\geq 0$, which represent the relative importance of each follower's information to the overall application goal. For example, in a mobility tracking application, the estimated velocities of the moving objects can be assigned as application weights $w_i(t)$, since faster objects may require more updates than slower objects in order to achieve the same tracking performance. 

%The approach that we take in this paper to address the issue of medium access is to implement a centralized scheduling scheme over WiFi. Specifically, our application-centric scheduler allows for the specification of priority weights $w_i$ for each follower $i$ in the network. These weights denote the relative importance of each follower's information to the overall application goal. For example, in a mobility tracking application, the priority weights can be the velocities of the objects being tracked: since faster objects need more updates than slower objects to get the same tracking performance. 

To capture application priorities and information freshness, we define the expected time-average of the weighted sum of AoIs across the entire network as
\begin{equation}\label{eq:AoI_opt}
\frac{1}{T} \mathbb{E} \left[ \sum_{i=1}^{N} \left(\int_{t=0}^{T} w_i(t)A_i(t) dt \right) \right] \; ,
\end{equation}
where $N$ is the number of followers, $T$ is the time-horizon, and the expectation is with respect to the randomness in the link's reliability $p_i(t)$ and the policy $\pi$. %To minimize \eqref{eq:AoI_opt}, the scheduling policy $\pi$ should attempt to improve information freshness, i.e. reduce $A_i(t)$, where the application needs it the most, i.e. where $w_i(t)$ is higher.
%
%Given the values of $\{A_i(t),p_i(t),w_i(t), \forall i \in [N]\}$, the goal of the transmission scheduling policy $\pi$ is to select followers $i$ that minimizes the expected time average of the weighted sum of AoIs across the entire network:
% \begin{equation}\label{eq:AoI_opt}
% \min_{\pi}~~~\frac{1}{T} \mathbb{E} \Bigg[ \sum_{i=1}^{N} \bigg(w_i \int_{t=0}^{T} A_i(t) dt \bigg) \Bigg].
% \end{equation}
%
% Plenty of theoretical work \cite{kadota2018scheduling, kadota2018scheduling2, tripathi2017age, talak2018optimizing, farazi2018age} has gone into studying the structure of scheduling policies that solve the optimization problem of the form described by \eqref{eq:AoI_opt} under interference constraints. The key result from these works is that given the scheduling weight $w_i$ and link reliability $p_i$ for each source $i$, a Whittle index style policy (described below) is near optimal for achieving information freshness. At every transmission opportunity, the Whittle index policy chooses the source that satisfies:
% \begin{equation}
%     \label{eq:AoI_whittle}
% 	i^{*} = \underset{i}{\operatorname{argmax}} \biggl\{ w_i p_i A^2_i(t) \biggr\}.
% \end{equation}

Many theoretical works \cite{kadota2018scheduling2, talak2018optimizing, maatouk2020optimality,tripathi2019whittle} have studied the structure of scheduling policies that attempt to minimize objective functions of the form \eqref{eq:AoI_opt}. A key take away from these works is that, given the knowledge of the application weights $w_i(t)$, link reliabilities $p_i(t)$, and information freshness $A_i(t)$ of every follower $i$, \emph{the Whittle's Index Policy is a near-optimal solution to the problem of minimizing} \eqref{eq:AoI_opt}. The \textbf{Whittle's Index Policy} selects, at every decision time $t$, the follower $i^*$ that satisfies
\begin{equation}
    \label{eq:AoI_whittle}
	%i^{*} \in \textstyle\operatorname{argmax}_{i\in\{1,\cdots,N\}} \left\{ w_i(t) p_i(t) A^2_i(t) \right\} \; ,
	i^{*} \in \textstyle\operatorname{argmax}_i \left\{ w_i(t) p_i(t) A^2_i(t) \right\} \; ,
\end{equation}
with ties being broken arbitrarily. 
%
Intuitively, the Whittle's Index Policy is polling the followers associated with high application weights, reliable WiFi links, and outdated information at the leader. %It is important to emphasize that the Whittle's Index Policy has low computational complexity. It only requires the computation in \eqref{eq:AoI_whittle} and estimates of $w_i(t)$, $p_i(t)$, and $A_i(t)$. Algorithms that estimate these parameters are described in Sec.~\ref{sec.MiddlewareDesign}. 

Recent works have developed similar Whittle's Index Policies to address generalizations of \eqref{eq:AoI_opt}. Specifically, \cite{tripathi2019whittle} addressed the problem of minimizing general non-decreasing cost functions of AoI, $f_i(A_i(t))$, as opposed to simply minimizing $A_i(t)$, and \cite{tripathi2021online} considered network settings with time-varying, unknown and even adversarial application weights $w_i(t)$. This suggests that Whittle's Index Policies are remarkably robust and can be applied to a wide variety of applications. Moreover, the Whittle's Index Policy has low computational complexity: it only requires solving the maximization in \eqref{eq:AoI_whittle} and computing estimates of $w_i(t)$, $p_i(t)$, and $A_i(t)$. %Algorithms that estimate these parameters are described in Sec.~\ref{sec.MiddlewareDesign}. 

%Recent works have further extended the performance guarantees for Whittle's Index Policies to (i) more general information freshness costs where the cost of AoI increases non-linearly \cite{tripathi2019whittle} and (ii) to settings with time-varying, unknown and even adversarial weights \cite{tripathi2021online}. This suggests that such index based policies are remarkably robust and can be applied to a wide variety of networked monitoring and control applications. 

%\autoref{eq:AoI_whittle} is pivotal to our application-centric transmission scheduler. It uses the relative importance of each follower (weights), the quality of connection to the leader (link reliabilities), and the information freshness at the leader (AoIs) to decide which follower currently has the most pressing or ``valuable'' update to be sent. Importantly, the centralized polling mechanism ensures that no two followers ever transmit at the same time, thus avoiding collisions between sources and removing a major cause of congestion and delay.
%Once the leader decides which follower to poll, it sends a polling packet to the corresponding follower and receives a new update (or a failed transmission). Once the application processes the new update and generates monitoring or control information, the scheduler decides which follower to poll next and the cycle repeats.

%Since the AoIs and weights are updated and maintained at the application layer, it makes it much easier to implement the transmission scheduler at the application layer as well, without worrying about cross-layer signaling and overheads. It also allows the scheduler to be easily tailored for different applications.

%Our middleware design allows for non-linear functions mapping AoI to a monitoring or control cost that can also be used to specify application specific latency requirements and design a scheduling policy. 


%Polling is performed using a Transmission Scheduling policy with dynamic weights and controls when each follower gets to transmit fresh updates to the leader. 

% \color{red} Vishrant: Discuss application specific weights and cost functions that describe the latency requirements of each source/follower. The specific parameters can be modified based on the application at hand. Convert the problem to an AoI scheduling problem and use the Whittle index solution. When weights and/or cost functions change with time, use an online adaptive scheme to update weights. Also keep track of channel reliabilities using polling acks and use for making scheduling decisions. Weights, probabilities, etc. are updated after reception of every complete information update.\color{black}

% \color{blue}
% %Igor: Describe multi-polling (V: better to discuss in 3.4). %We can also discuss functions of AoI (V: will confuse reader).%Recall that the Transmission Scheduler is implemented at the Application layer. Hence, it is easy to implement Transmission Schedulers with dynamic weights tailored to the specific application.
% \color{black}

\subsection{Middleware Design}\label{sec.MiddlewareDesign}


We describe the networking middleware illustrated in Fig.~\ref{fig.overview}, which incorporates both the application-centric queueing at the followers and transmission scheduling at the leader.
%
% %\igor{Describe the basic functions and information flow in Fig.~\ref{fig.overview}.}
% Now, we discuss the design of our middleware, which incorporates the two ideas introduced above: 1) application-centric queuing at the followers and 2) centralized application-centric transmission scheduling at the leader. %\color{blue}V: removed repeated line\color{black}
% % This design approach enables standard WiFi networks to support large-scale applications in which a team of followers collects and transmit time-sensitive information to a leader which coordinates the team. 

%First, we describe the role of the middleware at the followers: how information updates are fragmented, queued and transmitted in response to polling packets. Then we describe the role of the middleware at the leader: keeping track of AoI and scheduling weights, running the scheduler, sending polling packets and control information.

%\subsubsection{Followers} 

\textbf{Followers} collect information updates about their immediate environment (e.g., video, pictures, laser scans, and temperature) and about their own platforms (e.g., position, attitude, velocity, and battery level). These updates are sent to the \emph{follower middleware} to be prepared for transmission.

The rate control mechanism decides whether each update is discarded or enqueued. The follower middleware time-stamps each update that is not discarded at the time of collection and enqueues them. These time-stamps are used to compute $A_i(t)=t-\tau_i(t)$ at the leader upon delivery. %After time-stamping, updates are enqueued.
%We also provide a rate control mechanism that decides how often to send a newly collected update to the application-layer queue. 
The queuing discipline, update rates, and queue buffer sizes can be controlled by the middleware to satisfy the requirements of the application. %For example, in applications in which older updates can be dropped for newer updates, we set up the sources to have LIFO queues with  buffers that can accommodate only one update at a time.
%Our design allows the user to specify the queuing discipline at the application layer queue: FIFO, LIFO, or any other priority queue. The best choice of queueing discipline and queue size depend on the specific application and on the type of information. For example, ... 

%\textbf{Data fragmentation and transmission.} 
%Note that our communication architecture is polling based. 

When the follower receives a polling packet, %if its queue is empty, then it transmits an \emph{empty update} to the leader. The empty update is used by the leader to distinguish between not receiving an update due to WiFi transmission errors and not receiving an update due to an empty queue, which impacts the estimation of the reliability of the links $p_i(t)$. If the follower's queue is not empty, then 
it releases a single information update from its queue. Assuming that the update does not exceed the maximum length of the UDP payload (or any threshold set by the system designer), the released update can be simply forwarded via UDP/IP to lower layers of the networking protocol stack. However, if the update is too large, then the middleware divides the update into \emph{fragments}. %, which are used to implement acknowledgements and error control at the application layer, as well as to calculate estimates of channel reliabilities $p_i(t)$. 
%
Fragments are stored in a separate FIFO queue and then transmitted one-by-one to the leader. Each fragment is transmitted via UDP/IP over standard WiFi. Since the maximum WiFi frame length can be smaller than the UDP payload size, it is possible that WiFi will require multiple successful over-the-air transmissions to deliver a single fragment to the leader. If WiFi fails to deliver a fragment, the middleware attempts to re-transmit the fragment using an error-control mechanism based on acknowledgements at the fragment level.% until either successful delivery or when a timeout (of 300 milliseconds) is hit, stopping new polls from the leader. 

%When the polling packet acknowledging the final fragment is received, the follower's queue releases the next information update. 

%When a follower receives a polling packet from the leader, the follower's queue releases a single information update to be sent to the leader. Assuming that this information update fits into a single UDP packet, upon being released from the queue, the update can be simply forwarded to lower layers of the networking protocol stack. However, if the size of the information update exceeds the maximum length of the UDP payload (chosen based on the application), then upon being released from the application layer queue, our middleware \textit{fragments} the information update into separate UDP packets. These fragments are stored in a separate FIFO queue and then transmitted one-by-one to the leader until the entire update is sent. Our middleware also implements an acknowledgement mechanism at the fragment level. This allows the leader to keep track of link reliabilities for each follower. %Fragments are stored in a FIFO queue which is separate from the follower's queue containing information updates. 
%Upon receiving a packet acknowledging the previous fragment, the follower forwards the next fragment, until all fragments are successfully delivered to the leader. 
%When the polling packet acknowledging the final fragment is received, the follower's queue releases the next information update, which is then fragmented, stored in the FIFO queue and then transmitted following the same procedure. 

%If the follower's queue is empty when a polling packet arrives, the follower transmits a small \emph{empty packet} to the leader. The \emph{empty packet} is used by the leader to differentiate between not receiving data due to a transmission error or due to an empty queue, which impacts its estimate of the condition of the network. 

%Each fragment is transmitted to the leader via UDP/IP over standard WiFi. Since the maximum WiFi packet length can be smaller than the chosen UDP payload size, it is possible that WiFi will require multiple successful over-the-air transmissions to deliver a single fragment to the leader. 

%\subsubsection{Leader} 
%At the leader, our middleware implements the centralized application-centric transmission scheduler discussed in \autoref{sec.Scheduler}. 
%\textbf{Leader.} The leader is responsible for coordinating the followers' behavior and for controlling the information flow in the WiFi network. To do so, the leader controls the generation of polling packets. Since a new update is only transmitted upon reception of a polling packet from the leader, the leader has almost full control of the flow of information in the WiFi network.

\textbf{The Leader's} responsibilities include coordinating both the flow of information in the WiFi network and the followers' behavior. To do so, the leader manages the generation and transmission of \emph{polling packets} and \emph{control information}. Since follower's updates are transmitted only upon reception of a polling packet, the leader has almost full control of the flow of information in the WiFi network, irrespective of the number of followers and the amount of data they generate. %This control allows the leader to alleviate congestion and prevent excessive packet collisions in the WiFi network. 

%By transmitting control information to the followers, the leader can coordinate their 

%\igor{New scheduling decisions are made after each fragment or after each complete image?} 
The leader uses the Whittle's Index Policy \eqref{eq:AoI_whittle} to decide the next follower to poll. After transmitting a polling packet, the leader waits for the reception of a fragment. If this waiting period exceeds a timeout interval (e.g., $300$\thinspace{milliseconds}), the attempt is assumed to have failed. Upon receiving a fragment or after a timeout, the leader prepares for the transmission of the next polling packet.

Prior to transmitting the next polling packet, the leader takes a series of steps that depend on whether the received fragment was the final fragment of an information update or not. If the received fragment from follower $i$ was not the final one, then the leader middleware simply updates $p_i(t)$. On the other hand, if the received fragment was the final, then the leader: 
%Specifically, the reception of a fragment of an information update triggers the update of the AoI, link reliability, application weight, and control information. Upon receiving the final fragment, the leader should: 
(i) updates $p_i(t)$; (ii) combines fragments to obtain the original information update; (iii) extracts the associated time-stamp and updates $A_i(t)$; (iv) sends the information update to the application for processing; and (v) updates both $w_i(t)$ and the \emph{control information} based on the results of this processing.

%If the last fragment of an update was delivered in the previous polling cycle, the leader updates its application weights $w_i(t)$, its estimates of $p_i(t)$ and $A_i(t)$ accordingly. %The algorithms used to estimate $p_i(t)$ and $A_i(t)$ are described next. 
To estimate $p_i(t)$, the leader computes $\hat{p}_i(t)=D_i(t)/W,$ where $D_i(t)$ is the number of polling packets which received a successful response from follower $i$ out of the last $W$ polling packets sent to it. To accurately compute $A_i(t)=t-\tau_i(t)$, where $t$ is the current time measured by the leader and $\tau_i(t)$ is the largest time-stamp received from follower $i$, the clock at follower $i$ should be synchronized with the leader's clock. The middleware performs periodic clock synchronization across all followers and the leader, at every $120$ seconds using NTP \cite{NTP}. %Note that $A_i(t)$ is typically on the order of tens or hundreds of milliseconds. Thus the synchronization accuracy of NTP, which is around 1 millisecond for local area networks, is sufficient for our experiments. %a built-in algorithm based on the \emph{on-wire protocol} that is part of

% The reception of the final fragment of an information update triggers the update of the AoI, application weight, and control information. Upon receiving the final fragment, the leader should: (i) combine fragments to obtain the original information update; (ii) extract the associated time-stamp and update $A_i(t)$ accordingly; (iii) send the information update to the application for processing; and (iv) update both $w_i(t)$ and the control information based on the results of this processing. 
After performing the necessary updates, the leader middleware transmits a new polling packet to the selected follower. The latest control information is broadcast to all followers along with every polling packet. %This repeated broadcast ensures redundancy in the delivery of control information.% which is critical to multi-agent systems. 

%When the leader receives the final fragment, the middleware takes a series of steps. First, it combines all the received fragments to recreate the original information update. Then, it extracts the corresponding time-stamp and updates the AoI associated with the transmitting follower if all fragments are delivered without error. Finally, it sends the update to the application for processing. Based on the results of this processing, the application might set new scheduling weights and link reliabilities for the scheduler. Using this updated configuration, the middleware then runs the scheduler described in Sec.~\ref{sec.Scheduler} to generate a polling packet which contains the address of the follower that should respond with a new update next. It also appends the control information generated from processing to this polling packet, which is then broadcast to all followers.

%\igor{Importantly, the centralized polling mechanism ensures that no two followers ever transmit at the same time, thus avoiding collisions between sources and removing a major cause of congestion and delay.}
%\igor{empty packet}
%\igor{discuss multi-polling (depends on space)}
%\igor{clock sync}
%\igor{esitmation of p}
%\igor{transmit control packets}

%\textbf{Polling-based channel access.} To control the flow of information in the WiFi network, the leader controls the generation of polling packets. The Application-centric Transmission Scheduler determines the generation time and the destination of each polling packet. Since fragments are only transmitted upon reception of a polling packet from the leader, the leader has almost full control of the flow of information in the WiFi network.



%To demonstrate the performance improvement of our middleware in a real-world application, we 


% \igor{Draft of the leader part:}

% The reception of a fragment of an information update triggers the update of the AoI, link reliability, application weight, and control information. Upon receiving the final fragment, the leader middleware: (i) combines fragments to obtain the original information update; (ii) extracts the associated time-stamp and updates $A_i(t)$ accordingly; (iii) sends the information update to the application for processing; and (iv) update both $w_i(t)$ and the control information based on the results of this processing. %The latest control information is stored by the leader middleware. and broadcast to all followers right before every polling packet. The repeated broadcast ensures timely delivery of the latest control information which is critical to multi-agent systems. 

% To estimate $p_i(t)$, the leader computes $\hat{p}_i(t)=D_i(t)/W,$ where $D_i(t)$ is the number of polling packets which received a successful response from follower $i$ out of the last $W$ polling packets sent to it. To accurately compute $A_i(t)=t-\tau_i(t)$, where $t$ is the current time measured by the leader and $\tau_i(t)$ is the largest time-stamp received from follower $i$, the clock at follower $i$ should be synchronized with the leader's clock. The middleware performs periodic clock synchronization across all followers and the leader, at every $120$ seconds using NTP \cite{NTP}. Note that AoIs $A_i(t)$ are typically of the order of tens or hundreds of milliseconds. Thus the synchronization accuracy of NTP, which is around 1 millisecond for local area networks, is sufficient for our experiments.




%\vspace{-0.1cm}
\section{WiSwarm: Design and Implementation}\label{sec.WiSwarm}

In this section, we describe the design and implementation of WiSwarm which is an instantiation of the networking middleware for information freshness discussed in Sec.~\ref{sec.Middleware} tailored to a mobility tracking application. %An almost identical middleware can be used to enable the search and rescue and the automated exploration applications described in Sec.~\ref{sec.Examples}. %In the following, we describe the mobility tracking application and the design of the WiSwarm system.
%\igor{First, describe the application: the UAVs, the moving objects on the ground, the goal, and the need for information freshness.
%Then, design of WiSwarm. Start with the follower UAV. Then, describe the Leader Compute Node.}
\subsection{Mobility Tracking Application}\label{sec.MobilityTracking}
Consider a setting where multiple UAVs are tracking moving objects on the ground. Clearly, outdated information about the position of the objects has a direct impact on the tracking capability of the UAVs. Ideally, the UAVs would like to receive fresh information about the objects continuously. One simple system design that achieves this goal consists of UAVs with high on-board computational power that are able to process video frames acquired from their cameras to detect and track objects. The continuous stream of images is processed locally, adding almost no delay, which keeps the UAVs updated about the position of the objects. A critical drawback of this approach is the prohibitively high cost of deploying numerous UAVs with high on-board computational power. % may grow prohibitively high. 
\begin{figure}
	%\captionsetup{justification=centering}
	\centering
	\includegraphics[width=1.0\linewidth]{Images/Drones_Tracking.jpg}
	%\vspace{-0.4cm}
	\caption{Mobility tracking application implemented using multiple sensing-UAVs and a leader compute node.}
	\label{figure.tracking-model}
	%\vspace{-0.4cm}
\end{figure}

The separation of computing and sensing allows for more scalable system design - with one \emph{leader-node} that has plenty of on-board computational power, and numerous low-cost \emph{sensing-UAVs} that have little computational power but can effectively collect sensor data and communicate over a wireless network. Figure~\ref{figure.tracking-model} illustrates an example of this system design approach. In general, the leader node could be an UAV with a powerful on-board computer such as a Jetson TX2, a compute node located at the wireless edge, or even a cloud server performing high-speed inference and sending back control commands. %Computational offloading techniques to enhance the scale of multi-agent robotics applications have also been receiving recent interest in the robotics community \cite{chinchali2021network}.%The sensing UAVs could be equipped with different kinds of sensors, say cameras, LIDAR, GPS, etc. and a wireless communication system. 

In our specific implementation of the mobility tracking application, the sensing-UAVs capture video of the immediate environment below them and send the captured video frames (without any pre-processing) to the leader compute node. The leader processes the received frames, infers the position of the objects, and sends trajectory updates to the sensing-UAVs via WiFi. %WiFi is an attractive choice since it is low-cost, well-established, and readily available in platforms such as Raspberry Pis and the Jetson TX2. %Notice that WiFi is employed in the various time-sensitive applications described in \cite{ }. 
\emph{The main challenge of this design approach is to manage the limited wireless resources efficiently in order to keep information at the UAVs as fresh as possible.} WiSwarm, an instantiation of our networking middleware, ensures information freshness and scalable tracking performance by carefully controlling the flow of information over the network. 

Next, we describe the different individual components involved in our application - the mobile objects to be tracked, the sensing-UAVs, the leader compute node. We also discuss how WiSwarm is implemented at the sensing-UAVs and the leader compute node. %Then, we discuss the implementation of WiSwarm.

\subsection{Mobile Objects}\label{sec.MobileObjects}
% \begin{figure}
% 	%\captionsetup{justification=centering}
% 	\centering
% 	\includegraphics[width=0.95\linewidth]{Images/Target_w_and_without_tag.png}
% 	\vspace{-0.4cm}
% 	\caption{An autonomous car with and without the identifying ArUco marker on top.}
% 	\label{fig.target}
% 	\vspace{-0.4cm}
% \end{figure}

We use small autonomous cars equipped with RasPis (3B) as the moving objects whose mobility is tracked by the UAVs. Figure~\ref{fig.sensor-uav-and-car}(b) shows one such car, with the ArUco marker tag on top, which is used for uniquely identifying and tracking the position of the cars by the leader compute node. 

%Sensing-UAVs capture video of the environment below them and send these frames to the central node for processing.  The central node attempts to locate the ArUco marker tags in the received video frames and send future trajectory updates. For our tracking application, we assign each tag/car to be followed by a unique UAV beforehand.

\subsection{Follower Sensing-UAVs}\label{sec.FollowerUAVs}
% \begin{figure}
% 	%\captionsetup{justification=centering}
% 	\centering
% 	\includegraphics[width=0.95\linewidth]{Images/drone_labeled.png}
% 	\vspace{-0.4cm}
% 	\caption{Sensing-UAV}
% 	\label{fig.sensor-uav}
% 	\vspace{-0.4cm}
% \end{figure}
\begin{figure}[t]
\centering
\subfloat[]
{\includegraphics[width=0.59\columnwidth]{Images/drone_labeled_2.png}}%\label{demofig}
%\hspace{0.5cm}
\subfloat[]
{\includegraphics[width=0.39\columnwidth]{Images/Target_with_tag.jpg}}%\label{hist_1a} 
%\vspace{-0.2cm}
\caption{(a) Sensing-UAV. (b) Autonomous car with an identifying ArUco marker on top.} 
\label{fig.sensor-uav-and-car}
%\vspace{-0.5cm}
\end{figure}
%\subsubsection{Subsystems}
The sensing-UAV consists of two subsystems: a quadcopter drone and a RasPi (Zero W). Figure~\ref{fig.sensor-uav-and-car}(a) shows a sensing-UAV with a RasPi on board the quadcopter drone, along with its sensing and communication peripherals. %Figure~\ref{fig.follower} provides a system level overview of the sensing-UAV along with WiSwarm.

RasPi (Zero Ws) have very little computation capability (1 GHz single-core CPU and 512 MB RAM), but can effectively interact with multiple sensors and also communicate over WiFi. They are also extremely cost-efficient (\$10), making them ideal for use in the sensing-UAVs. Each UAV is also equipped with a micro-controller unit (MCU) that runs state estimation and flight control algorithms. The state estimator combines measurements from an on-board inertial measurement unit (IMU) with global position and orientation measurements.
These global measurements are obtained from a motion capture system and received by an Xbee WiFi module mounted on the UAV.
When motion capture data is not available, the Xbee module can be replaced by an alternative data source, such as a global navigation satellite system (GNSS) receiver.
%When prompted, the position and orientation estimate is sent to the Pi Zero W over an asynchronous serial connection that is also used to send time-stamped waypoints (defined in global coordinates) from the Pi Zero W to the vehicle MCU.
%The waypoints are interpolated to obtain a continuous trajectory that is tracked using the flight control algorithm described in \cite{tal2020accurate}.
%\color{black}

The RasPi is connected to a camera that captures video of the area below the UAV. Along with each frame, the RasPi also collects the position and orientation at which the frame was collected by asking for this information from the MCU using an asynchronous serial connection. Following the discussion in Sec.~\ref{sec.Queueing}, we know that fresh frames are the most useful for tracking, so we set the queuing discipline at the sensing-UAVs to be LIFO and the buffer size to be such that it can accommodate only one frame at a time. %Thus, WiSwarm stores the captured video frames, time-stamps, position and orientation in an application layer LIFO queue. %ready to be fragmented and sent to the leader compute node for inference upon request. 

The RasPi is connected to the leader compute node over 2.4 GHz WiFi using a high gain (8 dBi) antenna. Whenever the RasPi receives a polling packet, it transmits the most recent update in its LIFO queue to the compute node. %The application at the compute node locates the target to be tracked in the captured frame and uses the associated location information to compute a list of future waypoints and time-stamps. WiSwarm at the compute node then broadcasts this control information in the next polling packet. The control information denotes the times and locations (in global coordinates) where the drone should be in the future to follow the target.
The RasPi also collects the control information transmitted by the compute node which contains the times and locations (in global coordinates) where the UAV should be in the future in order to track the moving object. The RasPi sends these waypoints over the serial connection to the UAV MCU. The UAV MCU then plans and executes a trajectory that reaches the specified waypoints at the specified future time instants. It does this by interpolating the waypoints to obtain a continuous trajectory that is followed using the flight control algorithm described in \cite{tal2020accurate}. This completes the control loop. 


%\igor{Describe the functions and information flow. In Section~\ref{sec.MiddlewareDesign}, we described the networking middleware and its mechanisms in general. Here, we describe the details of the middleware implementation as WiSwarm. We should describe the specific functions and information flow with the contents of the messages. When appropriate, we might want to "refresh" the reader's memory or point to descriptions from Section~\ref{sec.MiddlewareDesign}. Near the end of this subsection, we describe the hardware in which the functions were implemented.}
% \begin{figure}
% 	%\captionsetup{justification=centering}
% 	\centering
% 	\includegraphics[width=0.98\linewidth]{Images/sensing-UAV-systems.png}
% 	\vspace{-0.4cm}
% 	\caption{System level overview of the sensing-UAV with WiSwarm.}
% 	\label{fig.follower}
% 	\vspace{-0.4cm}
% \end{figure}
 
%Videos frames captures by the camera are stored in a Last-Come-First-Serve (LCFS) queue. A rate parameter controls how frequently the WiSwarm application reads fresh data from the camera stream and updates this LCFS queue. For our application, we use a video frame resolution of 160x160 pixels. Importantly, we do not perform any compression or encoding of the video frames, since the Pi Zero W is not capable of doing so at a fast enough rate. Thus, when communicating over the WiFi network, we send unencoded 160x160 pixels of grayscale images in the YUV format (25 kB per image). 




\subsection{Leader Compute Node}\label{sec.LeaderNode}
%\igor{when control packets are sent, weight update mechanism}
%The leader compute node receives video frames from all the sensing UAVs and is responsible for processing them and sending back control commands. %It also implements the dynamic application centric scheduler that decides which sensing-UAV to poll for a new update at every transmission opportunity. %\autoref{fig.leader} provides an overview of the application layer and WiSwarm at the leader compute node.
% \begin{figure}
% 	\captionsetup{justification=centering}
% 	\centering
% 	\includegraphics[width=0.98\linewidth]{Images/central-overview.png}
% 	\vspace{-0.4cm}
% 	\caption{System level overview of the leader compute node with WiSwarm.}
% 	\label{fig.leader}
% 	\vspace{-0.4cm}
% \end{figure}
The compute node collects video-frames received from sensing-UAVs in response to polling requests. These video-frames are stored in separate LIFO queues - one for each sensing-UAV. The compute node runs an image processing thread which goes over the queues maintained by WiSwarm in a round-robin manner and processes the received video-frames whenever it finds a non-empty queue.

For each video-frame, the image processing thread attempts to locate the car that the UAV was assigned to track. If the car is found, it uses the relative location of the tag in the frame and the absolute position and orientation at which the frame was captured to compute the global coordinates of the car. The thread also keeps a record of the last known locations of the car. Using the current and previous locations, the image processing thread obtains: (i) the relative velocity between the car and the sensing UAV; and (ii) a list of future waypoints and the time-stamps at which it expects the car to reach these coordinates. In our implementation, we use a simple linear extrapolation scheme to predict future waypoints.

%Upon completion of processing, t
The image processing thread sends the  waypoints and time-stamps to WiSwarm along with information about the relative velocity between the car and the sensing-UAV. WiSwarm uses the relative velocity information to update its  application-defined priority weights %according to %the following equation %Intuitively, objects with higher relative velocities need urgent scheduling and so should have higher weight.
\begin{equation}
   \textstyle w_i(t) \leftarrow \alpha w_i(t^-) + (1-\alpha) \hat{v}_i(t),
\end{equation}  
where $\hat{v}_i(t)$ is the estimate of relative velocity between the car and the associated sensing UAV, and $\alpha = 0.8$. Since velocity estimates are noisy and car velocities are time-varying, we use an exponential moving average motivated by the adaptive AoI-based scheduling algorithms proposed in \cite{tripathi2021online}. WiSwarm updates link reliabilities $p_i(t)$ by using the number of successful fragment deliveries, as described in Sec.~\ref{sec.MiddlewareDesign}. 

With updated application weights $w_i(t)$ and link reliabilities $p_i(t)$, WiSwarm uses Whittle's Index Policy \eqref{eq:AoI_whittle} to select the sensing-UAVs that need to be scheduled for transmission most urgently. %using the Whittle's Index policy described earlier in \eqref{eq:AoI_whittle}. 
Together with the unicast transmission of a polling packet, WiSwarm broadcasts the \emph{most recent} list of future waypoints and time-stamps for every sensing-UAV. This repeated broadcast ensures redundancy in the delivery of control information. %Note that control information is broadcast to every UAV in every polling cycle allowing us to have redundancy.% in case a previous polling packet was lost.%from \autoref{sec.Scheduler}:
% \begin{equation}
% 	\text{poll UAV id} = \underset{i}{\operatorname{argmax}} \biggl\{ w_i p_i A^2_i(t) \biggr\}.
% \end{equation}
% Here, $w_i$ is the scheduling weight, $p_i$ is the link reliability, and $A_i(t)$ is the current AoI of the $i^{th}$ sensing-UAV,

% Upon deciding which UAV to poll, WiSwarm broadcasts a polling packet which contains the id of the UAV. At the sensing-UAV, WiSwarm keeps hearing for polling packets and whenever it receives packet beginning with its own id, it transmits a fresh update from its LIFO queue. Appended with the polling packet is the most recent list of future waypoints and time-stamps computed for each sensing-UAV. Since control information is broadcast to every UAV in every polling cycle it allows us to have redundancy, in case a previous polling packet with relevant control information was lost.
%\color{blue}
%Vishrant - comment on queues for processing maybe - ARtags can be detected efficiently - in experiments central node capable of processing 120 frames per second, Pi Zero only 5-6 frames per second - control information sent with polling command (might be better to include this in Sec 4.4)
%\color{black}

%\subsection{WiSwarm}
%\vspace{-0.1cm}
\section{Evaluation}\label{sec.Evaluation}
In this section, we evaluate the performance of both WiFi and WiSwarm for the mobility tracking application. We perform our experiments in a \emph{dynamic indoor campus space with multiple external sources of interference} such as WiFi base stations, mobile phones, and laptops. Throughout this section when we refer to WiFi, we mean 2.4 GHz WiFi. %This is due to three reasons - 1) RasPis (Zero W) only support 2.4 GHz WiFi, 2) 2.4 GHz WiFi has longer range and 3) 2.4 GHz WiFi is more reliable for mobile nodes.

In our evaluation, we consider two experimental setups:
%\begin{enumerate}
 (i) \textbf{Stationary experiments}, which involve up to fourteen RasPis running an emulated version of the mobility tracking application and sending video-frames to a central Compute Node. These experiments  involved hardware-in-the-loop and allowed us to test a variety of network sizes, update generation rates, scheduling policies, frame resolutions, packet sizes and interference conditions. %Over the course of three months, we collected nearly 400 hours worth of data from these experiments.
 (ii) \textbf{Flight experiments}, which involve interfacing the RasPis with UAVs and conducting real mobility tracking experiments. These allowed us to test how WiSwarm performs with mobile agents, at longer distances, and in the presence of significant interference. They also illustrate the drawbacks of using WiFi more clearly. %Over the course of two months, we collected nearly 4 hours worth of flight data.
%\end{enumerate}

\textbf{Baseline}. To demonstrate the performance improvement of WiSwarm, we compare it with two baseline WiFi systems, namely WiFi-TCP and WiFi-UDP. Both systems collect video frames from the application layer at a fixed rate, packetize them, store them in FIFO queues, and send these packets over standard WiFi to the Compute Node. TCP uses its congestion control mechanism to adjust the number of packets in flight, while UDP simply forwards packets. In all of our stationary experiments, we found that accommodating the entire video frame within a single UDP packet (i.e., with no fragmentation) was the best choice in terms of tracking error. 

For flight experiments, we consider an \emph{optimized version of WiFi-UDP} as the baseline. Our flight tests showed that mobility tracking with WiFi-UDP and WiFi-TCP with fixed video frame rate (e.g., 50 fps) was not possible for more than a single sensing-UAV. To get mobility tracking to work with two sensing-UAVs, we had to carefully tune the frame generation rate (to 5 fps) and the UDP packet size (to 6 kB per fragment). This is due to the high congestion and unreliability caused by high generation rates and large packets, which caused tracking failures. Further, we also had to tune RTS/CTS thresholds. Despite all of this optimization, WiFi-UDP was only able to enable tracking for \textit{at most two UAVs} at a time, as we show in the discussion on flight experiments.

\subsection{Stationary Experiments}\label{sec.HIL}

\begin{figure}
	%\captionsetup{justification=centering}
	\centering
	\includegraphics[width=\linewidth]{Images/HIL_Simulation_Setup_w_frames.png}
	%\vspace{-0.4cm}
	\caption{Screenshot from the videos used to simulate car movement during stationary experiments. The tags are programmed to perform random walks with time-varying velocities. The virtual UAVs need to keep track of the tags. On the right, two examples of 224x224 frames sent to the Compute Node by the RasPis based on their current virtual UAV locations.} %Pis running the virtual UAV application send 224x224 sized frames to WiSwarm or WiFi. These frames represent their local FoV, based on their current location. Virtual UAV locations are regularly updated based on control commands received from the Compute Node.}
	\label{fig.vid_tags}
	%\vspace{-0.2cm}
\end{figure}

% \begin{figure}
% 	%\captionsetup{justification=centering}
% 	\centering
% 	\includegraphics[width=\linewidth]{Images/Sent_Frames.png}
% 	\caption{224x224 frames sent to the base station for processing by the RasPis based on the current virtual UAV locations.}
% 	\label{fig.simulated_frames}
% \end{figure}

In this section, we discuss the performance improvements of WiSwarm over WiFi for three different metrics: (i) AoI, (ii) throughput, and, most importantly, (iii) tracking error. Each data-point in the following discussion represents 16 minutes worth of experiments, split into 4 batches of 4 minutes each. We calculate the time-average of the performance metric over the entire 4 minutes of each batch and then the mean and standard deviation across batches.
%For each, compare WiSwarm and WiFi (TCP + UDP) over different network configurations in HIL -> show mean plots with errorbars -> discuss performance improvement
%Briefly discuss flight tests and confirm that they behave somewhat similarly.
\begin{figure}[t]
\centering
\subfloat[]
{\includegraphics[width=0.5\columnwidth]{Images/plot_wifi_N6_a-eps-converted-to.pdf}}%\label{fig.plot_wifi_1_a}
%\hspace{0.5cm}
\subfloat[]
{\includegraphics[width=0.5\columnwidth]{Images/plot_wifi_N6_b-eps-converted-to.pdf}}%\label{hist_1a} 
%\vspace{-0.4cm}
\caption{(a) AoI and (b) tracking error of baseline WiFi-TCP and WiFi-UDP plotted against the update generation rate of each of the $N=6$ emulated UAVs.} 
\label{fig.plot_wifi_1}
%\vspace{-0.4cm}
\end{figure}
% \begin{figure}
% 	%\captionsetup{justification=centering}
% 	\centering
% 	\includegraphics[width=\linewidth]{Images/plot_wifi_N6.eps}
% 	\vspace{-0.4cm}
% 	\caption{AoI and tracking error of the baseline WiFi system with TCP and UDP plotted against the update generation rate at the UAVs. There are $N=6$ transmitting RasPis.}
% 	\label{fig.plot_wifi_1}
% 	\vspace{-0.4cm}
% \end{figure}

%We start by first describing the experimental setup for the stationary experiments.
\textbf{Experimental Setup}. The experiments involve multiple RasPis running an emulated virtual UAV application. This application does two things. First, each RasPi has a video simulating the movement of cars stored on it. Using this video, the RasPis create cropped frames of size 224x224, based on the current location of the virtual UAV, which capture the local Field-of-View (FoV). These video frames are generated at a specified rate that can be set using the rate control mechanism, and are forwarded to WiFi or WiSwarm for delivery. The frames are stored as unencoded grayscale yuv images (1 byte per pixel), so each video frame is 49 kB in size. Second, the application decodes the control packets received from the Compute Node and updates the virtual UAV's location by moving between control waypoints at a specified speed. Figure~\ref{fig.vid_tags} shows a frame from the video used for simulating movement of the car tags, along with two examples of 224x224 frames that the RasPis send to the Compute Node for processing.   

Figure~\ref{fig.plot_wifi_1} plots the mean AoI and tracking error per UAV for both WiFi-TCP and WiFi-UDP as the frame generation rate at the RasPis increases. This plot is for a system with $6$ transmitting RasPis. Note that lower AoI and lower tracking error are preferred in terms of performance.

We make two important observations from Fig.~\ref{fig.plot_wifi_1}. First, the performance of both WiFi-TCP and WiFi-UDP degrades when the generation rate is high, since the network becomes congested. Second, \textit{WiFi needs optimization of the generation rate at the application layer} to be anywhere close to working in practice. This optimization is challenging since it needs to be at the application layer and also adjust quickly to changes in the traffic load and link reliability, which can vary due to external interference. This is true for both TCP and UDP, i.e. \textit{TCP congestion control was unable to adjust to the optimal rate on its own}.

%We also observe that for tracking error, UDP is slightly better than TCP for nearly all update rates. This is as we expect, since UDP is the protocol of choice in real-time applications in practice. 
Next, we compare the performance of WiSwarm with both fixed-rate versions of WiFi and rate-optimized versions of WiFi. We choose the frame generation rates from the set $\{ 1,3,5,7,10,15,20,25,50,100\}$ fps and the number of RasPis from the set $N\in\{ 2,4,6,8,10,12,$ $14\}$. We find the best performing rates for each value of $N$ from the rate set (based on tracking error).% to be $\{25,7,5,5,3,3,$ $1\}$ fps for $N \in \{ 2,4,6,8,10,12,14\}$, respectively. 

To the best of our knowledge, there are no general purpose systems that can do application layer rate control for a wide variety of real-time applications, so the \textbf{rate-optimized WiFi systems are overly optimistic baselines}. Despite this, WiSwarm achieves significant performance gains over both fixed-rate and optimized rate versions of WiFi-TCP and WiFi-UDP.  
%.

\begin{figure}[t]
\centering
\subfloat[]
{\includegraphics[width=0.5\columnwidth]{Images/plot_mean_AoI_comparison_a-eps-converted-to.pdf}}%\label{demofig}
%\hspace{0.5cm}
\subfloat[]
{\includegraphics[width=0.5\columnwidth]{Images/plot_mean_AoI_comparison_b-eps-converted-to.pdf}}%\label{hist_1a} 
%\vspace{-0.4cm}
\caption{Mean AoI per UAV plotted for (a) fixed-rate (50 fps) and (b) optimized rate WiFi, as well as WiSwarm, as the number of UAVs increases.} 
\label{fig.plot_mean_AoI}
%\vspace{-0.5cm}
\end{figure}
% \begin{figure}
% 	%\captionsetup{justification=centering}
% 	\centering
% 	\includegraphics[width=\linewidth]{Images/plot_mean_AoI_comparison.eps}
% 	\vspace{-0.4cm}
% 	\caption{Mean AoI (per UAV) plotted for WiFi and WiSwarm as the number of UAVs increases.}
% 	\label{fig.plot_mean_AoI}
% 	\vspace{-0.4cm}
% \end{figure}
\begin{figure}[t]
\centering
\subfloat[]
{\includegraphics[width=0.5\columnwidth]{Images/plot_tail_AoI_comparison_a-eps-converted-to.pdf}}%\label{demofig}
%\hspace{0.5cm}
\subfloat[]
{\includegraphics[width=0.5\columnwidth]{Images/plot_tail_AoI_comparison_b-eps-converted-to.pdf}}%\label{hist_1a} 
%\vspace{-0.4cm}
\caption{Tail ($95^{th}$ percentile) AoI per UAV plotted for (a) fixed-rate (50 fps) and (b) optimized rate WiFi, as well as WiSwarm, as the number of UAVs increases.} 
\label{fig.plot_tail_AoI}
%\vspace{-0.4cm}
\end{figure}
% \begin{figure}
% 	%\captionsetup{justification=centering}
% 	\centering
% 	\includegraphics[width=\linewidth]{Images/plot_tail_AoI_comparison.eps}
% 	\vspace{-0.4cm}
% 	\caption{Tail ($95^{th}$ percentile) AoI per UAV plotted for WiFi and WiSwarm as the number of UAVs increases.}
% 	\label{fig.plot_tail_AoI}
% 	\vspace{-0.4cm}
% \end{figure}
% \begin{figure}
% 	%\captionsetup{justification=centering}
% 	\centering
% 	\includegraphics[width=0.9\linewidth]{Images/AoI_histogram.eps}
% 	\vspace{-0.4cm}
% 	\caption{AoI histogram for $N=8$ sources.}
% 	\label{fig.AoI_histogram}
% 	\vspace{-0.4cm}
% \end{figure}
\textbf{AoI}. Figure~\ref{fig.plot_mean_AoI} plots the mean AoI per UAV as the system size $N$ increases. More sources in the system means more congestion, more packet collisions (in WiFi) and hence poor performance and scalability. We see this clearly in Fig.~\ref{fig.plot_mean_AoI}(a), where we compare the baseline versions of WiFi-UDP and WiFi-TCP to WiSwarm. The baseline versions of WiFi have fixed update generation rate of 50 fps at each source while WiSwarm uses the maximum generation rate of 100 fps. Mean AoI improves by 16x for $N=8$ and by almost 50x for $N=14$ compared to fixed-rate WiFi. A major cause of the poor performance of WiFi is buildup of FIFO queues once the network becomes congested. Fixed-rate TCP eventually starts outperforming fixed-rate UDP for larger $N$, due to its congestion control mechanism. WiSwarm does not suffer from the congestion problem due to the LIFO queues.

Figure~\ref{fig.plot_mean_AoI}(b) compares rate-optimized versions of WiFi-TCP and WiFi-UDP with WiSwarm. We observe that mean AoI still improves by 1.5x for $N=8$ and 2.2x for $N=14$. While the FIFO queues in WiFi are no longer congested due to careful tuning of the frame generation rates, there are still packet collisions due to the distributed nature of the CSMA protocol and external interference sources. WiSwarm avoids these collisions by centralizing medium access scheduling decisions and prioritizing sources with higher AoI.%This can be thought of as the additional improvement by due to our centralized application-specific scheduling.

Since AoI combines the idea of service regularity with latency, we are also interested in the tail of information freshness. Figure~\ref{fig.plot_tail_AoI} plots the performance of baseline WiFi systems and WiSwarm for the $95th$ percentile of AoI, i.e. the value of AoI which is only exceeded $5\%$ of the time during an entire experiment. We observe very similar gains as mean AoI. For fixed rate, we observe an 18x reduction at $N=8$ and 36x reduction at $N=14$. For rate-optimized, we observe a 1.2x reduction at $N=8$ and a $1.7x$ reduction for $N=14$. Note that the tail AoI is important for our tracking application in addition to mean AoI, since a worse tail suggests a higher probability of the car going out of the UAV's Field-of-View leading to lost tracking. %We further illustrate the difference in performance between WiSwarm, WiFi-UDP and WiFi-TCP by plotting the empirical histogram of AoI for each system in an experiment involving $N=8$ RasPis. 

\begin{figure}[t]
\centering
\subfloat[]
{\includegraphics[width=0.49\columnwidth]{Images/plot_throughput_comparison_a-eps-converted-to.pdf}}%\label{demofig}
%\hspace{0.5cm}
\subfloat[]
{\includegraphics[width=0.49\columnwidth]{Images/plot_throughput_comparison_b-eps-converted-to.pdf}}%\label{hist_1a} 
%\vspace{-0.4cm}
\caption{Mean Throughput per UAV plotted for (a) fixed-rate (50 fps) and (b) optimized rate WiFi, as well as WiSwarm, as the number of UAVs increases.} 
\label{fig.plot_throughput}
%\vspace{-0.7cm}
\end{figure}
% \begin{figure}
% 	%\captionsetup{justification=centering}
% 	\centering
% 	\includegraphics[width=\linewidth]{Images/plot_throughput_comparison.eps}
% 	\vspace{-0.4cm}
% 	\caption{Mean Throughput per UAV plotted for WiFi and WiSwarm as the number of UAVs increases.}
% 	\label{fig.plot_throughput}
% 	\vspace{-0.4cm}
% \end{figure}

\textbf{Throughput}. Figure~\ref{fig.plot_throughput} plots the mean throughput per UAV for each of the considered systems as the number of UAVs increases. From Fig.~\ref{fig.plot_throughput}(a), we observe that both fixed-rate WiFi-TCP and WiFi-UDP have higher per UAV throughput than WiSwarm. However, this doesn't help in getting better AoI (as we saw earlier) or tracking performance (as we will see later). \textbf{This supports the idea that high throughput alone is not sufficient and AoI is the right metric to optimize for in such real-time applications}. For the rate-optimized versions of WiFi, we see a performance improvement in mean throughput per-UAV since WiSwarm can avoid packet collisions and deliver higher rates than the distributed CSMA mechanism while also ensuring lower AoIs. For $N=8$, WiSwarm achieves 1.2x higher throughput and for $N=14$, it achieves 2.7x higher throughput. 

\textbf{Tracking Error}. This is where we see how all the pieces of our system design come together to deliver better application performance. Figure~\ref{fig.plot_error} plots the mean tracking error (in pixels) per UAV as the number of UAVs increases, for WiSwarm and WiFi implementations. From Fig.~\ref{fig.plot_error}(a), which shows the fixed-rate baselines, we observe that tracking performance improves by 12x for $N=8$ and 4x for $N=14$. From Fig.~\ref{fig.plot_error}(b), with rate-optimized WiFi versions, we observe that tracking error is reduced by 2x at $N=10$ and 4x at $N=14$ with WiSwarm. We also note that the gap in performance between WiSwarm and the WiFi baselines \textit{increases} with the system size. In other words, the performance of WiFi-TCP and WiFi-UDP degrades much more quickly with $N$ leading to poor scalability.  
%From Fig.~\ref{fig.plot_tail_error}, we conclude that the improvements are similar in the tail ($95^{th}$ percentile) of the tracking error. For systems with $N=14$ UAVs, tracking error per UAV is reduced by 4x. 
\begin{figure}[t]
\centering
\subfloat[]
{\includegraphics[width=0.5\columnwidth]{Images/plot_error_comparison_a-eps-converted-to.pdf}}%\label{demofig}
%\hspace{0.5cm}
\subfloat[]
{\includegraphics[width=0.5\columnwidth]{Images/plot_error_comparison_b-eps-converted-to.pdf}}%\label{hist_1a} 
%\vspace{-0.4cm}
\caption{Mean Tracking Error per UAV plotted for (a) fixed-rate (50 fps) and (b) optimized rate WiFi, as well as WiSwarm, as the number of UAVs increases.} 
\label{fig.plot_error}
%\vspace{-0.5cm}
\end{figure}
% \begin{figure}
% 	%\captionsetup{justification=centering}
% 	\centering
% 	\includegraphics[width=0.95\linewidth]{Images/plot_error_comparison.eps}
% 	\vspace{-0.4cm}
% 	\caption{Mean Tracking Error per UAV plotted for WiFi and WiSwarm as the number of UAVs increases.}
% 	\label{fig.plot_error}
% 	\vspace{-0.4cm}
% \end{figure}
% \begin{figure}
% 	%\captionsetup{justification=centering}
% 	\centering
% 	\includegraphics[width=\linewidth]{Images/plot_tail_error_comparison.eps}
% 	\vspace{-0.4cm}
% 	\caption{Tail ($95^{th}$ percentile) Tracking Error per UAV plotted for WiFi and WiSwarm as the number of UAVs increases.}
% 	\label{fig.plot_tail_error}
% 	\vspace{-0.4cm}
% \end{figure}

\subsection{Flight Experiments}\label{sec.Flight}
\begin{figure}
	%\captionsetup{justification=centering}
	\centering
	\includegraphics[width=\linewidth]{Images/flight_frames.png}
	%\vspace{-0.4cm}
	\caption{Two Examples of 160x160 grayscale video frames sent by the RasPis during flight experiments.}%to the Compute Node for processing 
	\label{fig.flight_frames}
	%\vspace{-0.5cm}
\end{figure}
While the stationary experiments allowed us to test our system in great detail and provide extensive comparisons, they did not involve implementing the application on real UAVs tracking actual mobile targets in a dynamic environment. Our flight experiments address exactly this setting. Broadly, we will observe that the mobility of UAVs and higher degree of interference leads to worse wireless connectivity and, in turn, more congestion and packet collisions for WiFi. This allows us to bring the robustness of WiSwarm into focus. % and how well it performs in challenging wireless conditions as compared to WiFi.
We provide a video describing the setup and results from the flight experiments at \cite{Video}.

%The flight experiments involve interfacing the RasPis with quadcopter drones to build the sensing-UAVs as seen in Fig.~\ref{fig.sensor-uav}.
\textbf{Experimental Setup}. In the flight tests, we replace the internal antenna of the RasPis with an external high-gain (8 dBi) antenna to improve range and reliability when the UAVs fly. We fly up to 5 UAVs at a time in our experiment space which is roughly 20 meters x 10 meters in size. The mobile objects are autonomous cars with RasPi 3Bs shown in Fig.~\ref{fig.sensor-uav-and-car}(b). We program these cars to move in different polygonal trajectories over time and also stop occasionally at random for a few seconds. These trajectories are unknown to the UAVs and the Compute Node, and the job of the UAVs is to track the cars as closely as possible. Figure~\ref{fig.flight_setup} depicts the setup for an experiment involving 5 UAVs tracking the corresponding cars.

We configure the Pi-Cameras at the UAVs to generate video frames at the maximum possible rate, which is 90 frames per second. For WiSwarm, we utilize this full rate, while for WiFi, we choose the optimized rate by using rate control. The video frames are 160x160 unencoded grayscale images in the yuv format (1 byte per pixel), with a total size of 25 kB per frame. %Notice the lower resolution of the frames as compared to the stationary experiments. This is because of the lower transmission rates and unreliability of the wireless channel due to mobility and longer distances.
Figure~\ref{fig.flight_frames} shows two examples of frames sent to the Compute Node by RasPis from the flying UAVs during different experiments.

%Motion Capture (MoCap) plays an important role in our flight experiments. %We use the MoCap system to obtain position and orientation measurements of the autonomous cars and sensing UAVs. 
The sensing-UAVs implement a controller that requires knowledge of their own global position and orientation to be able to plan desired trajectories. A Motion Capture (MoCap) system provides this information to the UAVs (also via 2.4 GHz WiFi). 
%In our experiments, MoCap effectively take the place of the GNSS inertial navigation system (GNSS-INS) that is often used to obtain position and orientation estimates in real-world applications. 
These MoCap messages are sent to the UAVs in UDP messages at 30 messages/second and each message contains timestamp, position, and orientation of a single vehicle in 45 bytes. So the MoCap network usage is approximately 1.3 kB/s (or 11 kb/s) per UAV. Importantly, the MoCap system runs completely independently from the WiSwarm and WiFi systems and causes a low level of persistent interference in the channel. Thus, results from our flight experiments are a good measure of robustness of WiSwarm and WiFi to external interference. 

%Finally, the MoCap data corresponding to the cars is also recorded, but not used during the experiments. We use this in post-processing for calculating the tracking error between the UAVs and the target cars, creating videos and plotting time-lapse trajectories of cars and UAVs.

%\textbf{Baseline}. 



%Vishrant: discuss all the specific parameter choices we made - fragment sizes, video resolution, source generation rates, etc. %\\
%also discuss any hardware setup that hasn't been covered yet \\
\textbf{Results}. Figure~\ref{fig.coord_wiswarm} plots the coordinates of the sensing-UAVs and the target cars over time, for a two drone WiSwarm experiment, in both 2-D and 3-D. Similarly, Fig.~\ref{fig.coord_wiFi} plots the coordinates of the sensing-UAVs and the target cars over time, for a two drone WiFi experiment. It is easy to see that WiSwarm allows for far better tracking than WiFi even for just two UAVs. This is further supported by the histograms of AoI and tracking error plotted in Fig.~\ref{fig.flight_histogram}. The lower tracking error for WiSwarm is \textit{due to the fact that it can achieve lower AoI}, and hence deliver fresher information.


\begin{figure}
	%\captionsetup{justification=centering}
	\centering
	\includegraphics[width=\linewidth]{Images/tracking_2_wifresh-eps-converted-to.pdf}
	%\vspace{-0.4cm}
	\caption{Coordinates of sensing-UAVs and target cars in 2-D and 3-D, for a two drone flight experiment running WiSwarm.}
	\label{fig.coord_wiswarm}
	%\vspace{-0.5cm}
\end{figure}

\begin{figure}
	%\captionsetup{justification=centering}
	\centering
	\includegraphics[width=\linewidth]{Images/tracking_2_wifi-eps-converted-to.pdf}
	%\vspace{-0.4cm}
	\caption{Coordinates of sensing-UAVs and target cars in 2-D and 3-D, for a two drone flight experiment running optimized WiFi-UDP.}
	\label{fig.coord_wiFi}
	%\vspace{-0.5cm}
\end{figure}

\begin{figure}[t]
\centering
\subfloat[]
{\includegraphics[width=0.49\columnwidth]{Images/histogram_flight_a-eps-converted-to.pdf}}%\label{demofig}
%\hspace{0.5cm}
\subfloat[]
{\includegraphics[width=0.49\columnwidth]{Images/histogram_flight_b-eps-converted-to.pdf}}%\label{hist_1a} 
%\vspace{-0.4cm}
\caption{Histograms of (a) AoI and (b) tracking error for flight experiments with two UAVs, comparing WiSwarm with WiFi.} 
\label{fig.flight_histogram}
%\vspace{-0.5cm}
\end{figure}
% \begin{figure}
% 	%\captionsetup{justification=centering}
% 	\centering
% 	\includegraphics[width=\linewidth]{Images/histogram_flight.eps}
% 	\vspace{-0.4cm}
% 	\caption{Histograms of AoI and tracking error for two drone flight experiments comparing WiSwarm with WiFi.}
% 	\label{fig.flight_histogram}
% 	\vspace{-0.2cm}
% \end{figure}

We summarize the results of all of our flight experiments in Tables \ref{table:error} and \ref{table:AoI}. We average over 4 minutes of flight data for each experiment. Our main observation is as follows: \textbf{while WiFi allows tracking for up to two UAVs at a time, WiSwarm can easily allow tracking for up to five UAVs at a time}. In fact, when there are more than two sources in the system, WiFi is unable to deliver more than a handful of packets and essentially no UAV control is possible. The main reason for this is the high level of packet collisions for WiFi. WiSwarm is relatively robust to the unreliable wireless channels, interference and mobility issues encountered in flight experiments, due to our scheduler design that avoids packet collisions and prioritizes AoI.

\begin{table}[h!]
\centering
\begin{tabular}{|c|c|c|c|c|c|} 
 \hline
 Number of Drones & 1 & 2 & 3 & 4 & 5 \\ [0.3ex] 
 \hline\hline
 WiFi-UDP (Optimized) & 0.43 & 1.85 & - & - & - \\ 
 \hline
 WiSwarm & 0.39 & 0.30 & 0.39 & 0.35 & 0.36\\[0.3ex] 
 \hline
\end{tabular}
\caption{Average tracking error per sensing-UAV (in meters).}
%\vspace{-0.7cm}
\label{table:error}
\end{table}

\begin{table}[h!]
\centering
\begin{tabular}{|c|c|c|c|c|c|} 
 \hline
 Number of Drones & 1 & 2 & 3 & 4 & 5 \\ [0.3ex] 
 \hline\hline
 WiFi-UDP (Optimized) & 0.10 & 0.19 & - & - & - \\ 
 \hline
 WiSwarm & 0.08 & 0.09 & 0.11 & 0.12 & 0.16\\[0.3ex] 
 \hline
\end{tabular}
\caption{Average AoI per sensing-UAV (in seconds).}
\label{table:AoI}
%\vspace{-0.7cm}
\end{table}
%\vspace{-0.1cm}
%\section{Discussion}\label{sec.Discussion}
We briefly discuss centralization and coexistence which are important aspects of the networking middleware.

\textbf{Centralization.} Our networking middleware is designed with the leader at its center. This centralization allows the leader to leverage information about both the application and the communication network to optimize resource allocation. Specifically, the leader takes into account the application-defined priorities $w_i(t)$, the state of the WiFi network $p_i(t)$, and the freshness of the information $A_i(t)$ to prioritize the transmissions that are most valuable to the application. Centralization is key to achieve the results discussed in Sec.~\ref{sec.Evaluation}. Another advantage of centralization is that it allows for computational offloading. The main disadvantage is that the leader becomes a single point of failure. To improve robustness, the system designer can deploy additional leaders behaving as followers. It is important to emphasize that the networking middleware does not require the underlying WiFi network to have a star topology. The middleware can also be deployed in mesh WiFi networks as long as there is only one leader middleware.% as discussed in Sec.~\ref{sec.Examples}.

\textbf{Coexistence.} Communication networks supporting multi-agent systems running time-sensitive applications should have priority in accessing networking resources in order to reduce the effects of interference. Our networking middleware runs over standard WiFi and, thus, it is prone to interference. In Sec.~\ref{sec.Evaluation}, WiSwarm is evaluated and validated in a campus space with multiple sources of interference such as WiFi base stations, mobile phones, and laptops. In practice, we expect the system designer to attempt to reduce external interference as much as possible. Ideally, the middleware should not coexist with other networks, which is justified by the fact that the middleware is designed to support a time-sensitive multi-agent system.

%Some results to discuss:
%\begin{itemize}
    %\item The fragment size plays an important role in the performance of UDP. For images of 53k bytes, if the entire image is sent at once using an UDP socket, then the unreliability of the wireless channel causes the transmission to fail. This happens despite the MAC layer re-transmissions. To overcome this challenge, we split the image into fragments of 10k (or less) and implement an acknowledgement system similar to TCP. 
    %\item When we turn on the system, TCP behaves badly at first, since it is still figuring out its optimal window. To improve the performance of TCP, we added 1 minute of warm up in which all tags are stationary.
    %\item Number of waypoints
    %\item Sleep between poll packets
%\end{itemize}


%\color{blue} V: moved. Naturally, the larger the number of transmissions per fragment needed, the lower the probability of successfully delivering the fragment. To reduce the number of transmissions per fragment, we can reduce the UDP payload size set by the application. However, in that case, we are increasing the number of fragments and, thus, the end-to-end delay of delivering information updates. Both the lower reliability and the higher delay affect information freshness. This trade-off is more prominent when the WiFi network is subject to external interference. We show experimental results and discuss this trade-off in Section~\ref{}.\color{black}
%\vspace{-0.1cm}
%\section{Related Work}\label{sec.RelatedWork}

%\textbf{UAV Networks.} The literature on communication protocols for UAV networks is vast (see surveys in \cite{Drone_Surv_1,Drone_Surv_2,Drone_Surv_3,Drone_Surv_4}) and mostly theory-oriented (e.g., \cite{Drone_Theory_1,Drone_Theory_2,Drone_Theory_3,Drone_Theory_4}). Practical deployments of communication networks with multiple UAVs often use: (i) radios for unidirectional broadcast of control information to the UAVs \cite{Crazyswarm}; (ii) commercial off-the-shelf networking solutions such as WiFi and/or ZigBee \cite{CERBERUS,SAR,DistributedRobotFormation,CooperativeSLAM,ASTRO,DroneCinema,DOOR-SLAM,BeeCluster,Vijay_1,Vijay_2,hu2020hivemind}, both of which use CSMA and FIFO queues; or (iii) commercial LTE base-stations \cite{yuan2018ultra, moradi2018skycore, van2019androne, ahmad2021ares}. Communication is often mentioned as a bottleneck for scaling the multi-UAV system \cite{Drone_Surv_1,Drone_Surv_4,Vijay_2, chinchali2021network, honig2017flying}. To improve scalability, a common approach is to avoid/limit communication whenever possible \cite{Drone_Surv_4,Vijay_1,SLAM_review,Vijay_2} either by performing on-board computation; by compressing the data before transmitting \cite{mohanarajah2015cloud, choudhary2017distributed,chang2021kimera}, and/or by separating the UAV swarm into smaller communication groups \cite{hu2018centralize}. In this paper, we provide a novel solution to improve scalability that could be easily integrated to existing UAV networks.

\textbf{Age-of-Information.} 
Over the past decade there has been a rapidly growing body of theoretical works analyzing AoI in different contexts (see surveys in \cite{yates2021age,sun2019age_book,kosta2017age}). More recently, a few works \cite{AoI_measure_1,AoI_measure_3,kadota2021age,kadota2021wifresh,shreedhar2018acp,AoI_Wierman,AoI_SDR,ayan2021experimental} have considered system implementations. These system-oriented works can be separated into two categories: (i) measurement of AoI in real networks %using devices connected via Ethernet, WiFi, or LTE
\cite{AoI_measure_1,AoI_measure_3}; and (ii) systems that attempt to minimize AoI; by looking at congestion control \cite{shreedhar2018acp}, traffic engineering \cite{AoI_Wierman} and medium access via Software Defined Radios (SDRs) \cite{ayan2021experimental,kadota2021age,kadota2021wifresh,AoI_SDR}. However, there has been no prior work on minimizing AoI at the application layer or making readily deployable systems for real applications.
% removed AoI_measure_2, AoI_measure_4, AoI_vehicular
%and (iii) simulation of a time-sensitive application running over a real communication network that attempts to minimize AoI \cite{ayan2021experimental}. 
To the best of our knowledge, this is the first work to 
(i) build a system that can be easily deployed on current WiFi networks to minimize AoI and (ii) evaluate the performance of an AoI-based system using a time-sensitive application running over a real communication network. 

%We can discuss paper on AoI (theory and systems). Hardest part will be to compare this paper with the ICCCN paper. WiSwarm is not simply an extension of WiFresh. We should emphasize the many important differences. For example, the middleware is general and can be implemented for different applications. Its goal is to provide information freshness and application-defined dynamic prioritization in order to improve coordination. WiSwarm is the first system that optimizes freshness in a real application. Other system papers simply measure AoI or optimize AoI. They do not measure the impact of AoI-based design on the end application.

%A few theory-oriented papers \cite{Papers} have addressed the problem of finding the optimal rate in simple and idealized settings (e.g., M/M/1 queue \cite{kaul2012real} and ALOHA network \cite{ }) and a system-oriented paper \cite{shreedhar2018acp} developed a mechanism that iteratively adapts the generation rate until it converges to the optimal.

%To the best of our knowledge, this is the first work to experimentally evaluate AoI-based resource allocation in a real-world time-sensitive application.


% AoI system papers:
% Systems for measuring AoI in real networks
% -	C. S ̈onmez, S. Baghaee, A. Ergis ̧i, and E. Uysal-Biyikoglu, “Age-of- information in practice: Status age measured over TCP/IP connections through WiFi, ethernet and LTE,” in Proc. of IEEE BlackSeaCom, 2018
% -	H. B. Beytur, S. Baghaee, and E. Uysal, “Towards AoI-aware Smart IoTsystems,” in International Conference on Computing, Networking and Communications, 2020.
% -	Measuring Age of Information on Real-Life
% -	Connections
% -	B. Barakat, H. Yassine, S. Keates, I. Wassell, and K. Arshad, “How to measure the average and peak age of information in real networks?” in 25th European Wireless Conference, 2019.
% Systems that optimize freshness
% -	Age of Information in Random Access Networks with Stochastic Arrivals
% -	WiFresh: Age-of-Information from Theory to Implementation
% -	I. Kadota, M. S. Rahman, and E. Modiano, “Poster: Age of information in wireless networks: from theory to implementation,” in Proc. of ACM MobiCom, 2020.
% -	T. Shreedhar, S. Kaul, and R. D. Yates, “An age control transport protocol for delivering fresh updates in the internet-of-things,” in Proc. of IEEE WoWMoM, 2019.
% -	S. Kaul, M. Gruteser, V. Rai, and J. Kenney, “Minimizing age of information in vehicular networks,” in Proc. of IEEE SECON, 2011.
% -	Trading Throughput for Freshness: Freshness-Aware Traffic Engineering and In-Network Freshness Control
% -	Software-Defined Radio Implementation of Age-of-Information-Oriented Random Access
% Systems that use freshness to optimize time-sensitive applications
% -	An Experimental Framework for Age of Information and Networked Control via Software-Defined Radios. Emulates the problem of control of an inverted pendulum.

%\vspace{-0.1cm}
\section{Conclusion}\label{sec.Conclusion}
In this paper, we propose an AoI-based networking middleware that enables the customization of WiFi networks to the needs of time-sensitive applications that rely on multi-agent systems. By controlling the storage and flow of information in the underlying WiFi network, the middleware can prevent packet collisions and dynamically prioritize transmissions aiming to optimize information freshness. The middleware is implemented at the application layer, facilitating customization and integration to existing systems %such as \cite{KivaWiFi,VehicleToInfrastructureDSRC,AppLevelDSRC,NYCDOT,CERBERUS,SAR,DistributedRobotFormation,CooperativeSLAM,ASTRO,DroneCinema,DOOR-SLAM,RobotFormation,MultiRobotSlam,MultiRobotMapping}. 
To demonstrate the benefits of our middleware, we implement a mobility tracking application using a swarm of sensing-UAVs communicating with a central controller via WiFi. Our experimental results show that our middleware can improve information freshness and, as a result, tracking accuracy by \emph{more than one order of magnitude} when compared to an equivalent system that uses plain WiFi. Our flight tests also show that the middleware improves scalability of the mobility tracking application. Interesting extensions of this work include consideration of a distributed middleware architecture. 

\bibliographystyle{ieeetr}
\bibliography{bibliography_2}



\end{document}



