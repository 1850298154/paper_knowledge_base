
\section{Discussion}
\label{sec:discussion}

We have constructed and analyzed mathematical models of dynamic
task allocation in a multi-robot system. The models are general
and can be easily extended to other systems in which robots use a
history of local observations of the environment as a basis for
making decisions about future actions. These models are based on
theory of stochastic processes. In order to study a robot's
behavior, we do not need to know its exact trajectory or the
trajectories of other robots; instead, we derive a probabilistic
model that governs how a robot's behavior changes in time. In some
simple cases these models can be solved analytically. However,
stochastic models are usually too complex for exact analytic
treatment. Thus, in the scenario described in
\secref{sec:pucksonly} in which only observations of tasks are
made, though the individual model is tractable, the stochastic
model of the collective behavior is not. Instead, we use averaging
and approximation techniques to quantitatively study the dynamics
of the collective behavior. Such models, therefore, do not
describe the robots' behavior in a single experiment, but rather
the behavior that has been averaged over many experimental or
simulations runs. Fortunately, results of experiments and
simulations are usually presented as an average over many runs;
therefore, mathematical models of average collective behavior can
be used to describe experimental results. In fact, the stochastic
model produces excellent agreement with experimental results under
all experimental conditions and without using any adjustable
parameters.


Phenomenological models are more straightforward to construct and
analyze than exact stochastic models --- in fact, they can be
easily constructed from details of the individual robot
controller~\cite{Lerman04sab}. The ease of use comes at a price,
namely, the number of simplifying assumptions that were made in
order to produce a mathematically tractable model. First, we
assume that the robots are functioning in a dilute limit, where
they are sufficiently separated that their actions are largely
independent of one another. Second, we assume that the transition
rates can be represented by aggregate quantities that are
spatially uniform and independent of the details of the individual
robot's actions or history. We also assume the system is
homogeneous, with modeled robots characterized by a set of
parameters, each of them representing the mean value of some real
robot feature: mean speed, mean duration for performing a certain
maneuver, and so on. Real robot systems are heterogeneous: even if
the robots are executing the same controller, there will always be
variations due to inherent differences in hardware. We do not
consider parameter distributions in our models as would be
necessary to describe such heterogeneous systems. Finally,
phenomenological models more reliably describe systems where
fluctuations (deviations from the mean behavior) can be neglected,
as happens in large systems or when many experimental runs are
aggregated. However, even if phenomenological models don't agree
with experiments exactly, as we saw in \secref{sec:results2}, they
can still reliably predict most behaviors of interest even in
not-so-large systems. They are, therefore, a useful tool for
modeling and analyzing multi-robot systems.
