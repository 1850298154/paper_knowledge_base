\section{Conclusion}
\label{sec:conclusion}

 Mathematical analysis can be a useful tool
for the study and design of MRS and  a viable alternative to
experiments and simulations. It can be applied to large systems
that are too costly to build or take too long to run in
simulation. Mathematical analysis can be used to study the
behavior of an MRS, select parameters that optimize its
performance, prevent instabilities, \emph{etc}. In conjunction
with the design process, mathematical analysis can help understand
the effect individual robot characteristics have on the collective
behavior \emph{before} a system is implemented in hardware or in
simulation. Unlike experiments and simulations, where exhaustive
search of the design parameter space is often required to reach
any conclusion, analysis can often produce exact analytic results,
or scaling relationships, for the quantities of interest. If these
are not possible, exhaustive search of the parameter space is much
more practical and efficient. Finally, results of analysis can be
used as feedback to guide performance-enhancing modifications of
the robot controller.

In this paper we have described an dynamic task allocation
mechanism where robots use local observations of the environment
to decide their task assignments. We have presented a mathematical
model of this task allocation mechanism and studied it in the
context of a multi-foraging task scenario. We compared predictions
of the model with results of embodied simulations and found
excellent quantitative agreement. In this application,
mathematical analysis could help the designer choose robot
properties, such as the form of the transition probability used by
robots to switch their task state, or decide how many observations
the robot ought to consider.

Mathematical analysis of MRS is a new field, but its success in
explaining experimental results shows it to be a promising tool
for the design and analysis of robotic systems. The field is open
to new research directions, from applying analysis to new robotic
systems to developing increasingly sophisticated mathematical
models that, for example, account for heterogeneities in robot
population that are due to differences in their sensors and
actuators.
