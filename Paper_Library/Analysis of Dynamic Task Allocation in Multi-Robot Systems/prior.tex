\newcommand{\ea}{{\em et al.}}

\section{Related Work}
\label{sec:prior}

% modeling of collective behavior
Mathematical modeling and analysis of the collective behavior of
MRS is a relatively new field with approaches and methodologies
borrowed from other fields, including mathematics, physics, and
biology. Recently, a number of researchers attempted to
mathematically analyze  multi-robot systems by using
phenomenological models of the type present here. Sugawara
\ea~\cite{Sugawara97,SugSanYosAbe98} developed a simple model of
cooperative foraging in groups of communicating  and
non-co\-mmu\-ni\-cating robots. Kazadi \ea~\cite{Kazadi02} studied
the general properties of multi-robot aggregation using
phenomenological macroscopic models. Agassounon and
Martinoli~\cite{Agassounon02} presented a model of aggregation in
which the number of robots taking part in the clustering task is
based on the division of labor mechanism in ants. These models are
\emph{ad-hoc} and domain specific, and the authors give no
explanation as to how to apply such models to other domain. In
earlier works we have developed a general framework for creating
phenomenological models of collective behavior in groups of
robots~\cite{Lerman02nasa,Lerman04sab}. We applied this framework
to study collaborative stick-pulling in a group of reactive
robots~\cite{Lerman01} and foraging in robots~\cite{Lerman02}.

Most of the approaches listed above are implicitly or explicitly
based on stochastic processes theory. Another example of the
stochastic approach is the probabilistic microscopic model
developed by Martinoli and
coworkers~\cite{Martinoli99,MarIjsGam99,IMB2001} to study
collective behavior of a group of robots. Rather than compute the
exact trajectories and sensory information of individual robots,
Martinoli {\em et al.} model each robot's interactions with other
robots and the environment as a series of stochastic events, with
probabilities determined by simple geometric considerations.
Running several series of stochastic events in parallel, one for
each robot, allowed them to study the group behavior of the
multi-robot system.


\comment{ Application-level studies of adaptation and learning in
multi-robot systems have recently been carried out
~\cite{Kaelbling90,Mataric97,Martinoli02learn,Jones03icra,Dahl03icra}.
The RoboCup robot soccer domain provided a fruitful framework for
introducing learning in the context of multi-agent and multi-robot
systems. Several authors examined the use of reinforcement
learning to learn basic soccer skills, coordination
techniques~\cite{Riedmiller01} and game strategies~\cite{Stone01}.
Li {\em et al.}~\cite{Martinoli02learn} introduced learning into
collaborative stick pulling robots and showed in simulation that
learning does improve system performance by allowing robots to
specialize. No analysis of the collective behavior or performance
of the system have been attempted in any of these studies. }


So far very little work has been done on mathematical analysis of
multi-robot systems in dynamic environments. We have recently
extended~\cite{Lerman03aamas} the stochastic processes framework
developed in earlier work to robots that change their behavior
based on history of local observations of the (possibly changing)
environment~\cite{Lerman03iros}. In the current paper we develop
these ideas further, and present the exact stochastic model of the
system, in addition to the phenomenological model.


%%%%%%
Closest to ours is the work of Huberman and
Hogg~\cite{HubermanHogg88}, who mathematically studied collective
behavior of a system of adaptive agents using game dynamics as a
mechanism for adaptation.  In game dynamical systems, winning
strategies are rewarded, and agents use the best performing
strategies to decide their next move. Although their adaptation
mechanism is different from our dynamic task allocation mechanism,
their analytic approach is similar to ours, in that it is based on
the theory of stochastic processes. Others have mathematically
studied collective behavior of systems composed of large numbers
of concurrent learners~\cite{Wolpert99,Sato03}. These are
microscopic models, which only allow one to study collective
behavior of relatively small systems of a few robots. We are
interested in macroscopic approaches that enable us to directly
study collective behavior in large systems. Our work differs from
earlier ones in another important way:  we systematically compare
theoretical predictions of mathematical models with results of
experiments carried out in a sensor-based simulator.
