\section{Introduction}

In the 1980's it was considered ground-breaking for a mobile robot
to move around an unstructured environment at reasonable speeds.
In the years since, advancements in both hardware mechanisms and
software architectures and algorithms have resulted in quite
capable mobile robot systems.  Provided with this baseline
competency of individual robots, increasing attention has been
paid to the study of Multi-Robot Systems (MRS), and in particular
{\it distributed} MRS with which the remainder of this paper is
concerned.  In a distributed MRS there is no centralized control
mechanism -- instead, each robot operates independently under
local sensing and control, with coordinated system-level behavior
arising from local interactions among the robots and between the
robots and the task environment. The effective design of
coordinated MRS is restricted by the lack of formal design tools
and methodologies.  The design of single robot systems (SRS) has
greatly benefited from the formalisms provided by control theory
-- the design of MRS is in need of analogous formalisms.

For a group of robots to effectively perform a given system-level
task, the designer must address the question of which robot should
do which task and when \cite{Gerkey04}.  The process of assigning
individual robots to sub-tasks of a given system-level task is
called \textit{task allocation}, and it is a key functionality
required of any MRS.  {\it Dynamic} task allocation is a class of
task allocation in which the assignment of robots to sub-tasks is
a dynamic process and may need to be continuously adjusted in
response to changes in the task environment or group performance.
The problem of task allocation in a {\it distributed} MRS is
further compounded by the fact that task allocation must occur as
a result of a distributed process as there is no central
coordinator available to make task assignments.  This increases
the problem's complexity because, due to the local sensing of each
robot, no robot has a complete view of the world state.  Given
this incomplete and often noisy information, each robot must make
local control decisions about which actions to perform and when,
without complete knowledge of what other robots have done in the
past, are doing now, or will do in the future.

There are a number of task allocation models and philosophies.
Historically, the most popular approaches rely on intentional
coordination to achieve task allocation \cite{Parker98}.  In
those, the robots coordinate their respective actions explicitly
through deliberate communications and negotiations. Due to scaling
issues, such approaches are primarily used in MRS consisting of a
relatively small number of robots (i.e., fewer than 10). Task
allocation through intentional coordination remains the preferred
approach because it is better understood, easier to design and
implement, and more amenable to formal analysis \cite{Gerkey04}.

As the size of the MRS grows, the complexity of the design of
intentional approaches increases due to increased demands in
communication bandwidth and computational abilities of individual
robots. Furthermore, complexity introduced by increased robot
interactions makes such systems much more difficult to analyze and
design. This leads to the alternative to intentional coordination,
namely, task allocation through utilizing emergent coordination.
In systems using emergent coordination, individual robots
coordinate their actions based solely on local sensing information
and local interactions. Typically, there is very little or no
direct communication or explicit negotiations between robots. They
are, therefore, more scalable to larger numbers of robots and are
more able to take advantage of the robustness and parallelism
provided by the aggregation of large numbers of coordinated
robots.  The drawback of task allocation as achieved through
emergent coordination mechanisms is that such systems can be
difficult to design, solutions are commonly sub-optimal, and since
coordination is achieved through many simultaneous local
interactions between various subsets of robots, predictive
analysis of expected system performance is difficult.

As MRS composed of ever-larger numbers of robots become available,
the need for task allocation through emergent coordination will
increase.   To address the lack of formalisms in the design of
such MRS, in this article we present and experimentally verify a
predictive mathematical model of dynamic task allocation for MRS
using emergent coordination.  Such a formal model of task
allocation is a positive step in the direction of placing the
design of MRS on a formal footing.

The rest of the paper is organized as follows. \secref{sec:prior}
provides a summary of related work. In
\secref{sec:task-allocation} we describe a general mechanism for
task allocation in dynamic environments. This is a distributed
mechanism based on local sensing. In \secref{sec:analysis} we
present a mathematical model of the collective behavior of an MRS
using this mechanism and study its performance under a variety of
conditions. We validate the model in a multi-foraging domain.  In
\secref{sec:multi-foraging} we define the experimental task domain
of multi-foraging, robot controllers and the simulation
environment. In \secref{sec:results} we compare the predictions of
mathematical models with the results from sensor-based
simulations. We conclude the paper in \secref{sec:discussion},
with a discussion of the approach and the results.
