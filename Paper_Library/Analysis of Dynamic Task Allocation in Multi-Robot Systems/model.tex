\section{Dynamic Task Allocation Mechanism\label{sec:task-allocation}}

The dynamic task allocation scenario we study considers a world
populated with tasks of $T$ different types and robots that are
equally capable of performing each task but can only be assigned
to one type at any given time. For example, the tasks could be
targets of different priority that have to be tracked, different
types of explosives that need to be located, etc. Additionally, a
robot cannot be idle --- each robot is always performing a task at
any given time. We introduce the notion of a robot state as a
shorthand for the type of task the robot is assigned to service. A
robot may switch its state according to its control policy when it
determines it is appropriate to do so. However, needlessly
switching tasks is to be avoided, since in physical robot systems,
this can involve complex physical movement that requires time to
perform.


The purpose of task allocation is to assign robots to tasks in a
way that will enhance the performance of the system, which
typically means reducing the overall execution time. Thus, if all
tasks take an equal amount of time to complete, in the best
allocation, the fraction of robots in state $i$ will be equal to
the fraction of tasks of type $i$. In general, however, the
desired allocation could take other forms ---  for example, it
could be related to the relative reward or cost of completing each
task type --- without change to our approach. In the dynamic task
allocation scenario, the number of tasks and the number of
available robots are allowed to change over time, for example, by
adding new tasks, deploying new robots, or removing malfunctioning
robots.


The challenge faced by the designer is to devise a mechanism that
will lead to a desired task allocation in a distributed MRS even
as the environment changes. The challenge is made even more
difficult by the fact that robots have limited sensing
capabilities, do not directly communicate with other robots, and
therefore, cannot acquire global information about the state of
the world, the initial or current number of tasks (total or by
type), or the initial or current number of robots (total or by
assigned type). Instead, robots can sample the world (assumed to
be finite) --- for example, by moving around and making local
observations of the environment. We assume that robots are able to
observe tasks and discriminate their types. They may also be able
to observe and discriminate the task states of other robots.

One way to give the robot an ability to respond to environmental
changes (including actions of other robots) is to give a robot an
internal state where it can store its knowledge of the environment
as captured by its observations~\cite{Jones03iros,Lerman03aamas}.
The observations are stored in a rolling history window of finite
length, with new observations replacing the oldest ones. The robot
consults these observations periodically and updates its task
state according to some transition function specified by the
designer. In an earlier work we
showed~\cite{Jones03iros,Lerman03iros} that this simple dynamic
task allocation mechanism leads to the desired task allocation in
a multi-foraging scenario.


In the following sections we present a mathematical model of
dynamic task allocation and study the role that transition
function and the number of observations (history length) play in
the performance of a multi-foraging MRS. In
\secref{sec:pucksonly}, we present a model of a simple scenario in
which robots base their decisions to change state solely on
observations of tasks in the environment. We study the simplest
form of the transition function, in which the probability to
change state to some type is proportional to the fraction of
existing tasks of that type. In \secref{sec:results1} we compare
theoretical predictions with no adjustable parameters to
experimental data and find excellent agreement. In
\secref{sec:phenomenological} we examine the more complex scenario
where the robots base their decisions to change task state on the
observations of both existing task types and task states of other
robots. In \secref{sec:results2} we study the consequences of the
choice of the transition function and history length on the system
behavior and find good agreement with the experimental data.
