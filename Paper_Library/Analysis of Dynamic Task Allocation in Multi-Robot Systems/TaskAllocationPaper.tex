\documentclass[]{article}
\usepackage{graphicx}
\usepackage{subfigure}

\long\def\comment#1{}
\newcommand{\commentout}[1]{}

\newcommand{\etc}{{\em etc.}}
\newcommand{\eg}{{\em e.g.}}
\newcommand{\ie}{{\em i.e.}}
\newcommand{\noi}{\noindent}

\newcommand{\secref}[1]{Section~\ref{#1}}
\newcommand{\eqref}[1]{Equation~\ref{#1}}
\newcommand{\figref}[1]{Figure~\ref{#1}}
\newcommand{\tabref}[1]{Table~\ref{#1}}

\pagestyle{empty}

\begin{document}


%\title{Analysis of Dynamic Task Allocation Experiments in Robots}

%\author{Kristina Lerman, Chris V. Jones, Aram Galstyan, Maja J. Matari{\'c}}

%\maketitle


\begin{center}
{\LARGE\bf Analysis of Dynamic Task Allocation in Multi-Robot Systems}\\[0.25cm]
{\bf Kristina Lerman$^1$, Chris Jones$^2$, Aram Galstyan$^1$ and Maja J Matari\'{c}$^2$}\\
\end{center}

\begin{center}
1. Information Sciences Institute \\
2. Computer Science Department \\
University of Southern California, Los Angeles, CA
90089-0781, USA \\
\{lerman\(|\)galstyan\}@isi.edu, \{cvjones\(|\)maja\}@robotics.usc.edu
\end{center}

\begin{abstract}

Dynamic task allocation is an essential requirement for
multi-robot systems operating in unknown dynamic environments. It
allows robots to change their behavior in response to
environmental changes or actions of other robots in order to
improve overall system performance. Emergent coordination
algorithms for task allocation that use only local sensing and no
direct communication between robots are attractive because they
are robust and scalable. However, a lack of formal analysis tools
makes emergent coordination algorithms difficult to design. In
this paper we present a mathematical model of a general dynamic
task allocation mechanism. Robots using this mechanism have to
choose between two types of task, and the goal is to achieve a
desired task division in the absence of explicit communication and
global knowledge. Robots estimate the state of the environment
from repeated local observations and decide which task to choose
based on these observations. We model the robots and observations
as stochastic processes and study the dynamics of the collective
behavior. Specifically, we analyze the effect that the number of
observations and the choice of the decision function have on the
performance of the system. The mathematical models are validated
in a multi-robot multi-foraging scenario. The model's predictions
agree very closely with experimental results from sensor-based
simulations.


\end{abstract}

\section{Introduction}

In the 1980's it was considered ground-breaking for a mobile robot
to move around an unstructured environment at reasonable speeds.
In the years since, advancements in both hardware mechanisms and
software architectures and algorithms have resulted in quite
capable mobile robot systems.  Provided with this baseline
competency of individual robots, increasing attention has been
paid to the study of Multi-Robot Systems (MRS), and in particular
{\it distributed} MRS with which the remainder of this paper is
concerned.  In a distributed MRS there is no centralized control
mechanism -- instead, each robot operates independently under
local sensing and control, with coordinated system-level behavior
arising from local interactions among the robots and between the
robots and the task environment. The effective design of
coordinated MRS is restricted by the lack of formal design tools
and methodologies.  The design of single robot systems (SRS) has
greatly benefited from the formalisms provided by control theory
-- the design of MRS is in need of analogous formalisms.

For a group of robots to effectively perform a given system-level
task, the designer must address the question of which robot should
do which task and when \cite{Gerkey04}.  The process of assigning
individual robots to sub-tasks of a given system-level task is
called \textit{task allocation}, and it is a key functionality
required of any MRS.  {\it Dynamic} task allocation is a class of
task allocation in which the assignment of robots to sub-tasks is
a dynamic process and may need to be continuously adjusted in
response to changes in the task environment or group performance.
The problem of task allocation in a {\it distributed} MRS is
further compounded by the fact that task allocation must occur as
a result of a distributed process as there is no central
coordinator available to make task assignments.  This increases
the problem's complexity because, due to the local sensing of each
robot, no robot has a complete view of the world state.  Given
this incomplete and often noisy information, each robot must make
local control decisions about which actions to perform and when,
without complete knowledge of what other robots have done in the
past, are doing now, or will do in the future.

There are a number of task allocation models and philosophies.
Historically, the most popular approaches rely on intentional
coordination to achieve task allocation \cite{Parker98}.  In
those, the robots coordinate their respective actions explicitly
through deliberate communications and negotiations. Due to scaling
issues, such approaches are primarily used in MRS consisting of a
relatively small number of robots (i.e., fewer than 10). Task
allocation through intentional coordination remains the preferred
approach because it is better understood, easier to design and
implement, and more amenable to formal analysis \cite{Gerkey04}.

As the size of the MRS grows, the complexity of the design of
intentional approaches increases due to increased demands in
communication bandwidth and computational abilities of individual
robots. Furthermore, complexity introduced by increased robot
interactions makes such systems much more difficult to analyze and
design. This leads to the alternative to intentional coordination,
namely, task allocation through utilizing emergent coordination.
In systems using emergent coordination, individual robots
coordinate their actions based solely on local sensing information
and local interactions. Typically, there is very little or no
direct communication or explicit negotiations between robots. They
are, therefore, more scalable to larger numbers of robots and are
more able to take advantage of the robustness and parallelism
provided by the aggregation of large numbers of coordinated
robots.  The drawback of task allocation as achieved through
emergent coordination mechanisms is that such systems can be
difficult to design, solutions are commonly sub-optimal, and since
coordination is achieved through many simultaneous local
interactions between various subsets of robots, predictive
analysis of expected system performance is difficult.

As MRS composed of ever-larger numbers of robots become available,
the need for task allocation through emergent coordination will
increase.   To address the lack of formalisms in the design of
such MRS, in this article we present and experimentally verify a
predictive mathematical model of dynamic task allocation for MRS
using emergent coordination.  Such a formal model of task
allocation is a positive step in the direction of placing the
design of MRS on a formal footing.

The rest of the paper is organized as follows. \secref{sec:prior}
provides a summary of related work. In
\secref{sec:task-allocation} we describe a general mechanism for
task allocation in dynamic environments. This is a distributed
mechanism based on local sensing. In \secref{sec:analysis} we
present a mathematical model of the collective behavior of an MRS
using this mechanism and study its performance under a variety of
conditions. We validate the model in a multi-foraging domain.  In
\secref{sec:multi-foraging} we define the experimental task domain
of multi-foraging, robot controllers and the simulation
environment. In \secref{sec:results} we compare the predictions of
mathematical models with the results from sensor-based
simulations. We conclude the paper in \secref{sec:discussion},
with a discussion of the approach and the results.


\newcommand{\ea}{{\em et al.}}

\section{Related Work}
\label{sec:prior}

% modeling of collective behavior
Mathematical modeling and analysis of the collective behavior of
MRS is a relatively new field with approaches and methodologies
borrowed from other fields, including mathematics, physics, and
biology. Recently, a number of researchers attempted to
mathematically analyze  multi-robot systems by using
phenomenological models of the type present here. Sugawara
\ea~\cite{Sugawara97,SugSanYosAbe98} developed a simple model of
cooperative foraging in groups of communicating  and
non-co\-mmu\-ni\-cating robots. Kazadi \ea~\cite{Kazadi02} studied
the general properties of multi-robot aggregation using
phenomenological macroscopic models. Agassounon and
Martinoli~\cite{Agassounon02} presented a model of aggregation in
which the number of robots taking part in the clustering task is
based on the division of labor mechanism in ants. These models are
\emph{ad-hoc} and domain specific, and the authors give no
explanation as to how to apply such models to other domain. In
earlier works we have developed a general framework for creating
phenomenological models of collective behavior in groups of
robots~\cite{Lerman02nasa,Lerman04sab}. We applied this framework
to study collaborative stick-pulling in a group of reactive
robots~\cite{Lerman01} and foraging in robots~\cite{Lerman02}.

Most of the approaches listed above are implicitly or explicitly
based on stochastic processes theory. Another example of the
stochastic approach is the probabilistic microscopic model
developed by Martinoli and
coworkers~\cite{Martinoli99,MarIjsGam99,IMB2001} to study
collective behavior of a group of robots. Rather than compute the
exact trajectories and sensory information of individual robots,
Martinoli {\em et al.} model each robot's interactions with other
robots and the environment as a series of stochastic events, with
probabilities determined by simple geometric considerations.
Running several series of stochastic events in parallel, one for
each robot, allowed them to study the group behavior of the
multi-robot system.


\comment{ Application-level studies of adaptation and learning in
multi-robot systems have recently been carried out
~\cite{Kaelbling90,Mataric97,Martinoli02learn,Jones03icra,Dahl03icra}.
The RoboCup robot soccer domain provided a fruitful framework for
introducing learning in the context of multi-agent and multi-robot
systems. Several authors examined the use of reinforcement
learning to learn basic soccer skills, coordination
techniques~\cite{Riedmiller01} and game strategies~\cite{Stone01}.
Li {\em et al.}~\cite{Martinoli02learn} introduced learning into
collaborative stick pulling robots and showed in simulation that
learning does improve system performance by allowing robots to
specialize. No analysis of the collective behavior or performance
of the system have been attempted in any of these studies. }


So far very little work has been done on mathematical analysis of
multi-robot systems in dynamic environments. We have recently
extended~\cite{Lerman03aamas} the stochastic processes framework
developed in earlier work to robots that change their behavior
based on history of local observations of the (possibly changing)
environment~\cite{Lerman03iros}. In the current paper we develop
these ideas further, and present the exact stochastic model of the
system, in addition to the phenomenological model.


%%%%%%
Closest to ours is the work of Huberman and
Hogg~\cite{HubermanHogg88}, who mathematically studied collective
behavior of a system of adaptive agents using game dynamics as a
mechanism for adaptation.  In game dynamical systems, winning
strategies are rewarded, and agents use the best performing
strategies to decide their next move. Although their adaptation
mechanism is different from our dynamic task allocation mechanism,
their analytic approach is similar to ours, in that it is based on
the theory of stochastic processes. Others have mathematically
studied collective behavior of systems composed of large numbers
of concurrent learners~\cite{Wolpert99,Sato03}. These are
microscopic models, which only allow one to study collective
behavior of relatively small systems of a few robots. We are
interested in macroscopic approaches that enable us to directly
study collective behavior in large systems. Our work differs from
earlier ones in another important way:  we systematically compare
theoretical predictions of mathematical models with results of
experiments carried out in a sensor-based simulator.


\section{Dynamic Task Allocation Mechanism\label{sec:task-allocation}}

The dynamic task allocation scenario we study considers a world
populated with tasks of $T$ different types and robots that are
equally capable of performing each task but can only be assigned
to one type at any given time. For example, the tasks could be
targets of different priority that have to be tracked, different
types of explosives that need to be located, etc. Additionally, a
robot cannot be idle --- each robot is always performing a task at
any given time. We introduce the notion of a robot state as a
shorthand for the type of task the robot is assigned to service. A
robot may switch its state according to its control policy when it
determines it is appropriate to do so. However, needlessly
switching tasks is to be avoided, since in physical robot systems,
this can involve complex physical movement that requires time to
perform.


The purpose of task allocation is to assign robots to tasks in a
way that will enhance the performance of the system, which
typically means reducing the overall execution time. Thus, if all
tasks take an equal amount of time to complete, in the best
allocation, the fraction of robots in state $i$ will be equal to
the fraction of tasks of type $i$. In general, however, the
desired allocation could take other forms ---  for example, it
could be related to the relative reward or cost of completing each
task type --- without change to our approach. In the dynamic task
allocation scenario, the number of tasks and the number of
available robots are allowed to change over time, for example, by
adding new tasks, deploying new robots, or removing malfunctioning
robots.


The challenge faced by the designer is to devise a mechanism that
will lead to a desired task allocation in a distributed MRS even
as the environment changes. The challenge is made even more
difficult by the fact that robots have limited sensing
capabilities, do not directly communicate with other robots, and
therefore, cannot acquire global information about the state of
the world, the initial or current number of tasks (total or by
type), or the initial or current number of robots (total or by
assigned type). Instead, robots can sample the world (assumed to
be finite) --- for example, by moving around and making local
observations of the environment. We assume that robots are able to
observe tasks and discriminate their types. They may also be able
to observe and discriminate the task states of other robots.

One way to give the robot an ability to respond to environmental
changes (including actions of other robots) is to give a robot an
internal state where it can store its knowledge of the environment
as captured by its observations~\cite{Jones03iros,Lerman03aamas}.
The observations are stored in a rolling history window of finite
length, with new observations replacing the oldest ones. The robot
consults these observations periodically and updates its task
state according to some transition function specified by the
designer. In an earlier work we
showed~\cite{Jones03iros,Lerman03iros} that this simple dynamic
task allocation mechanism leads to the desired task allocation in
a multi-foraging scenario.


In the following sections we present a mathematical model of
dynamic task allocation and study the role that transition
function and the number of observations (history length) play in
the performance of a multi-foraging MRS. In
\secref{sec:pucksonly}, we present a model of a simple scenario in
which robots base their decisions to change state solely on
observations of tasks in the environment. We study the simplest
form of the transition function, in which the probability to
change state to some type is proportional to the fraction of
existing tasks of that type. In \secref{sec:results1} we compare
theoretical predictions with no adjustable parameters to
experimental data and find excellent agreement. In
\secref{sec:phenomenological} we examine the more complex scenario
where the robots base their decisions to change task state on the
observations of both existing task types and task states of other
robots. In \secref{sec:results2} we study the consequences of the
choice of the transition function and history length on the system
behavior and find good agreement with the experimental data.


\section{Analysis of Dynamic Task Allocation}
\label{sec:analysis}

As proposed in the previous section,  a robot may be able to adapt
to a changing environment in the absence of complete global
knowledge if it is  able to make and remember local observations
of the environment. In the treatment below we assume that there
are two types of tasks
---  arbitrarily referred to as $Red$ and $Green$. This simplification is for pedagogical reason only; the model
can be extended to a greater number of task types.


During a sufficiently short time interval, each robot can be
considered to belong to the $Green$ or $Red$ task state. This is a
very high level, coarse-grained description.  In reality, each
state is composed of several robot actions and behaviors, for
example, searching for new tasks, detecting and executing them,
avoiding obstacles, {\em etc}. However, since we want the model to
capture how the fraction of robots in each task state evolves in
time, it is a sufficient level of abstraction to consider only
these two states. If we find that additional levels of detail are
required to explain system behavior, we can elaborate the model by
breaking each of the high level states into its underlying
components.

\subsection{Observations of Tasks Only}
\label{sec:pucksonly}

In this section we study dynamic task allocation mechanism in
which robots make decisions to switch task states based solely on
observations of available tasks. Let $m_r$ and $m_g$ be  the
numbers of the observed $Red$ and $Green$ tasks, respectively, in
a robot's memory or history window.  The robot chooses to change
its state, or the type of task it is assigned to execute, with
probabilities given by transition functions $f_{g \rightarrow
r}(m_r,m_g)$ (probability of switching to $Red$ from $Green$) and
$f_{r \rightarrow g}(m_r,m_g)$ (probability of switching to
$Green$ from $Red$). We would like to define transition rules so
that the fraction of time the robot spends in the $Red$ ($Green$)
state be equal to the fraction of $Red$ ($Green$) tasks. This will
assure that on average the number of $Red$  and $Green$ robots
reflect the desired task distribution. Clearly, if the robots have
global knowledge about the numbers of $Red$ and $Green$ tasks
$M_r$ and $M_g$, then each robot could choose each state with
probability equal to the fraction of the tasks of corresponding
type. Such global knowledge is not available; hence, we want to
investigate how incomplete knowledge of the environment (through
local observations), as well as the dynamically changing
environment (e.g., changing ratio of $Red$ and $Green$ tasks),
affects task allocation.

\subsubsection{Modelling Robot Observations}

As explained above, the transition rate between task execution
states depends on robot's observations stored in its history. In
our model we assume that a robot makes an observation of a task
with a time period $\tau$. For simplicity, by an observation we
mean here detecting a task, such as a target to be monitored, mine
to be cleared or an object to be gathered. Therefore, observation
history of length $h$ comprises of the number of $Red$ and $Green$
tasks a robot has observed during a time interval ${h}\tau$. We
assume that $\tau$ has unit length and drop it. The process of
observing a task is given by a Poisson distribution with rate
$\lambda = \alpha M^0$, where $\alpha$ is a constant
characterizing the physical parameters of the robot such as its
speed, view angles, etc., and $M^0$ is the number of tasks in the
environment. This simplification is based on the idea that robot's
interactions with other robots and the environment are independent
of the robot's actual trajectory and are  governed by
probabilities determined by simple geometric considerations. This
simplification has been shown to produce remarkably good
agreements with experiments~\cite{MarIjsGam99,IMB2001}.

Let $M_r(t)$ and $M_g(t)$ be the number of $Red$ and $Green$ tasks
respectively (can be time dependent), and let
$M(t)=M_r(t) + M_g(t)$ be the total number of tasks. The
probability that in the time interval $[t-{h}, t]$ the robot has
observed exactly $m_r$ and $m_g$ tasks is the product of two
Poisson distributions:
\begin{equation}
\label{eq:poisson} P(m_r,m_g) = \frac{\lambda_r^{m_r}
\lambda_g^{m^g}}{m_r! m_g!}e^{- \lambda_r-\lambda_g}
\end{equation}
where $\lambda_i$~, $i=r,g$, are the means of the respective
distributions. If the task distribution does not change in time,
$\lambda_i = \alpha M_i {h}$. For time dependent task
distributions, $\lambda_i = \alpha \int_{t-{h}}^tdt'M_i(t')$.


\subsubsection{Individual Dynamics: The Stochastic Master Equation}

Let us consider a single robot that has to decide between
executing  $Red$ and $Green$ tasks in a closed arena and makes a
transition to $Red$ and $Green$ states according to its
observations. Let $p_r(t)$ be the probability that a robot is in
the $Red$ state at time $t$. The equation governing its evolution
is
\begin{equation}
\label{eq:inddyn} \frac{dp_r}{dt} = \varepsilon (1-p_r) f_{g
 \rightarrow r} - \varepsilon p_r f_{r \rightarrow
g}
\end{equation}
where $\varepsilon$ is the rate at which the robot makes decisions
to switch its state, and $f_{g  \rightarrow r}$ and $f_{r
\rightarrow g}$ are the corresponding transitions probabilities
between the states. As explained above, these probabilities depend
on the robot's history --- the number of tasks of either type it
has observed during the time interval ${h}$ preceding the
transition. If the robots have global knowledge about the numbers
of $Red$ and $Green$ tasks $M_r$ and $M_g$, one could choose the
transition probabilities as the fraction of tasks of corresponding
type, $f_{g \rightarrow r}\propto M_r/(M_r+M_g)$ and $f_{r
\rightarrow g}\propto M_g/(M_r+M_g)$. In the case when the global
information is not available, it is natural to use similar
transition probabilities using robots' local estimates:
\begin{eqnarray}
\label{eq:rates} f_{g \rightarrow r}(m_r,m_g) =
\frac{m_r}{m_r+m_g}\equiv
\gamma_r(m_r,m_g) \\
\label{eq:ratesg} f_{r \rightarrow g}(m_r,m_g) =
\frac{m_g}{m_r+m_g}\equiv
 \gamma_g(m_r,m_g)
\end{eqnarray}
Note that $\gamma_r(m_r,m_g) + \gamma_g(m_r,m_g) =1$ whenever
$m_r+m_g>0$, \eg, whenever there is at least one observation in
the history window. In the case when there are no observations in
history, $m_r=m_g=0$, robots will choose either state with
probability $1/2$ as it follows from taking the appropriate limits
in Equations \ref{eq:rates} and \ref{eq:ratesg}. Hence, we
supplement \eqref{eq:rates} with $f_{g \rightarrow r}(0,0)=f_{r
\rightarrow g}(0,0)=0$ (and similarly for \eqref{eq:ratesg}) to
assure that robots do not change their state when the history
window does not contain any observations.

\eqref{eq:inddyn}, together with the transition rates shown in
Equations \ref{eq:rates}--\ref{eq:ratesg}, determines the
evolution of the probability density of a robot's state. It is a
stochastic equation since the coefficients (transition rates)
depend on random variables $m_r$ and $m_g$. Moreover, since the
robot's history changes gradually, the values of the coefficients
at different times are correlated, hence making the exact
treatment very difficult. We propose, instead, to study the
problem within the $annealed$ approximation: we neglect
time--correlation between robot's histories at different times,
assuming instead that at any time the real history $\{m_r,m_g\}$
can be replaced by a random one drawn from the Poisson
distribution \eqref{eq:poisson}. Next, we average
\eqref{eq:inddyn} over all histories to obtain
\begin{equation}
\label{eq:inddyn2} \frac{dp_r}{dt} = \varepsilon
\overline{\gamma}_r (1-n_r)  - \varepsilon \overline{\gamma}_g n_r
\end{equation}
Here $\overline{\gamma}_r$ and $\overline{\gamma}_g$ are given by
\begin{equation}
\label{eq:avg_gamma} \overline{\gamma}_r =
\sum_{r,g}P(r,g)\frac{r}{r+g},  \overline{\gamma}_g =
\sum_{r,g}P(r,g)\frac{g}{r+g}
\end{equation}
where $P(m_r,m_g)$ is the Poisson distribution \eqref{eq:poisson}
and the summation excludes the term $r=g=0$. Note that if the
distribution of tasks changes in time, then
$\overline{\gamma}_{r,g}$ are time-dependent,
$\overline{\gamma}=\overline{\gamma}_{r,g}(t)$.

To proceed further, we need to evaluate the summations in
\eqref{eq:avg_gamma}. Let us  define an auxiliary function
\begin{eqnarray}
\label{eq:aux} F(x) =
\sum_{m_r=0}^{\infty}\sum_{m_g=0}^{\infty}{x^{m_r+m_g} \frac
{\lambda_r^{m_r} \lambda_g^{m_g}} {m_r! m_g!}
e^{-\lambda_r}e^{-\lambda_g}\frac {m_r} {m_r+m_g}}
\end{eqnarray}
It is easy to check that $\overline{\gamma}_{r,g}$ are given by
\begin{eqnarray}
\label{eq:aux1} \overline{\gamma}_r &=& F(1) - \frac{1}{2}P(0,0)
= F(1) - \frac{1}{2} e^{\alpha {h} M_0} \nonumber \\
\overline{\gamma}_g &=& 1-F(1) -\frac{1}{2}e^{\alpha {h} M_0}
\end{eqnarray}
Differentiating \eqref{eq:aux} with respect to $x$ yields
\begin{equation}
\label{eq:aux2} \frac{dF}{dx} =
\sum_{m_r=1}^{\infty}\sum_{m_g=0}^{\infty}x^{m_r+m_g-1}\frac{\lambda_r^{m_r}
\lambda_g^{m_g}}{m_r! m_g!}e^{- \lambda_r}e^{-\lambda_g}m_r
\end{equation}
Note that the summation over $m_r$ starts from $m_r=1$. Clearly,
 the sums over $m_r$ and $m_g$ are de--coupled thanks to the
cancellation of the denominator $(m_r+m_g)$:
\begin{equation}
\label{eq:aux3} \frac{dF}{dx} = \biggl( e^{-
\lambda_r}\sum_{m_r=1}^{\infty}x^{m_r-1}\frac{\lambda_r^{m_r}
}{m_r!}m_r \biggr )\biggl( e^{-
\lambda_g}\sum_{m_g=0}^{\infty}\frac{(x \lambda_g)^{m_g} }{m_g!}
\biggr )
\end{equation}
The resulting sums are evaluated easily (as the Taylor expansion
of corresponding exponential functions), and the results is
\begin{equation}
\label{eq:diff} \frac{dF}{dx} = \lambda_r e^{- \lambda_0(1-x)}
\end{equation}
where $\lambda_0 = \lambda_r+\lambda_g$. After elementary
integration of \eqref{eq:diff} (subject to the condition $F(0) =
1/2$),  we obtain using \eqref{eq:aux2} and the expressions for
$\lambda_r$, $\lambda_0$:
\begin{equation}
\label{eq:gamma} \overline{\gamma}_{r,g}(t) = \frac{1-e^{\alpha
{h} M_0}}{{h} }\int_{t-{h}}^t dt^{\prime}\mu_{r,g}(t^{\prime})
\end{equation}
Here $\mu_{r,g}(t)=M_{r,g}(t)/M_0$ are the fraction of $Red$ and
$Green$ tasks respectively.

Let us first consider the case when the task distribution does not
change with time, \ie, $\mu_r(t)=\mu_0$. Then we have
\begin{equation}
\label{eq:gamma-constant} \overline{\gamma}_{r,g}(t)
=(1-e^{-\alpha {h} M_0})\mu_{r,g}^0
\end{equation}
The solution of \eqref{eq:inddyn2} subject to the initial
condition $p_r(t=0)=p_0$ is readily obtained:
\begin{equation}
\label{eq:solution} p_r(t) = \mu_r^0 + \biggl ( p_0 -
\frac{\overline{\gamma}_r}{\overline{\gamma}_r +
\overline{\gamma}_g}\biggr )e^{ - \varepsilon (\overline{\gamma}_r
+ \overline{\gamma}_g)t}
\end{equation}
One can see that the probability distribution approaches  the
desired steady state value $p_r^s = \mu_r^0$ exponentially. Also,
the coefficient of the exponent depends on the density of tasks
and the length of the history window. Indeed, it is easy to check
that $\overline{\gamma}_r + \overline{\gamma}_g = 1-e^{-\alpha
{h} M_0}$. Hence,   for large enough $M_0$  and ${h}$, $\alpha
{h} M_0 \gg 1$, the convergence rate is determined solely by
$\varepsilon$. For a small task density or short history
length, on the other hand, the convergence rate is proportional to
the number of tasks, $ \varepsilon ( 1-e^{-\alpha {h} M_0})\sim
\varepsilon \alpha {h} M_0$. Note that this is a direct
consequence of the rule that robots do not change their state
whenever there are no observation in the history window.


Now let us consider the case where the task distribution changes
suddenly at time $t_0$, $\mu_r(t)= \mu_r^0 + \Delta \mu
\theta(t-t_0)$, where $\theta(t-t_0)$ is the step function. For
simplicity, let us assume that $\alpha {h} M_0 \gg 1$ so that the
exponential term in \eqref{eq:gamma} can be neglected,
\begin{equation}
\label{eq:gamma-timedep} \overline{\gamma}_{r,g}(t) =
\frac{1}{{h} }\int_{t-{h}}^t dt^{\prime}\mu_{r,g}(t^{\prime}),
\overline{\gamma}_{r}(t) + \overline{\gamma}_{g} =1
\end{equation}


Replacing \eqref{eq:gamma-timedep} into \eqref{eq:inddyn2}, and
solving the resulting  differential equation yields
\begin{eqnarray}
\label{eq:jump} p_r(t)& =& \mu_r^0 + \frac{\Delta\mu}{{h}}t
-\frac{\Delta\mu}{\varepsilon {h}}(1-e^{-\varepsilon
t}),~~~~~~~~~~t\leq {h} \nonumber \\
p_r(t)& =& \mu_r^0 + \Delta\mu -\frac{\Delta\mu}{\varepsilon
{h}}(e^{-\varepsilon (t-{h})} -e^{-\varepsilon t} ),~~~t>{h}
\,.
\end{eqnarray}
\eqref{eq:jump} describes how the robot distribution converges to the
new steady state value after the change in task distribution.
Clearly, the convergence properties of the solutions depend on
${h}$ and $\varepsilon$. It is easy to see that in the limiting
case $\varepsilon {h} \gg 1$ the new steady state is attained
after time ${h}$, $|p_r({h})-(\mu_0 + \Delta \mu)| \sim \Delta
\mu/(\varepsilon {h})\ll 1$, so the convergence time is
$t_{conv}\sim {h}$. In the other limiting case $\varepsilon {h}
\ll 1$, on the other hand, the situation is different. A simple
analysis of \eqref{eq:jump} for $t>{h}$ yields $|p_r(t)-(\mu_0
+ \Delta \mu)| \sim \Delta \mu e^{-\varepsilon t}$ so the
convergence is exponential with characteristic time $t_{conv} \sim
1/\varepsilon$.



\subsubsection{Collective Behavior}
In order to make predictions about the behavior of an MRS using a
dynamic task allocation mechanism, we need to develop a
mathematical model of the collective behavior of the system. In
the previous section we derived a model of how an individual
robot's behavior changes in time. In this section we extend it to
model the behavior of a MRS. In particular, we study the
collective behavior of a homogenous system consisting of $N$
robots with identical controllers. Mathematically, the MRS is
described by a probability density function that includes the
states of all $N$ robots. However, in most cases we are interested
in studying the evolution of global, or average, quantities, such
as the average number of robots in the $Red$ state, rather than
the exact probability density function. This applies when
comparing theoretical predictions with results of experiments,
which are usually quoted as an average over many experiments.
Since the robots in either state are independent of each other,
$p_r(t)$, is now the fraction of robots in the $Red$ state, and
consequently $Np_r(t)$ is the average number of robots in that
state. The results of the previous section, namely solutions for
$p_r(t)$ for constant task distribution (\eqref{eq:solution}) and
for changing task distribution (\eqref{eq:jump}), can be used to
study the average collective behavior. \secref{sec:results1}
presents results of analysis of the mathematical model.


\subsubsection{Stochastic Effects}
\label{sec:stochastic1}
In some cases it is useful to know the probability distribution of
robot task states over the entire MRS. This probability function describes
the exact collective behavior from which one could derive the
average behavior as well as the fluctuations around the average.
Knowing the strength of fluctuations is necessary for assessing how
the probabilistic nature of robot's observations and actions affects the global
properties of the system. Below we consider the problem of finding
the probability distribution of the collective state of the
system.

Let $P_n(t)$ be the probability that there are exactly $n$ robots
in the $Red$ state at time $t$. For a sufficiently short time
interval $\Delta t$ we can write~\cite{Lerman03iros}
\begin{equation}
\label{eq:master1} P_n(t+\Delta t) = \sum_{n^{\prime}}
W_{n^{\prime} n}(t;\Delta t)P_{n^{\prime}}(t) -\sum_{n^{\prime}}
W_{n n^{\prime}}(t;\Delta t)P_{n}(t)
\end{equation}
where $W_{i j}(t;\Delta t)$ is the transition probability between
the states $i$ and $j$ during the time interval $(t, t+\Delta t)$.
In our MRS, this transitions correspond to robots changing their
state from $Red$ to $Green$ or vice versa. Since the probability
that more than one robot will have a transition during a time
interval $\Delta t$ is $O(\Delta t)$, then, in the limit $\Delta t
\rightarrow 0$ only transitions between neighboring states are
allowed in \eqref{eq:master1}, $n \rightarrow n \pm 1$. Hence, we
obtain

\begin{equation}
\label{eq:master2} \frac{dP_n}{dt} = r_{n+1}P_{n+1}(t) +
g_{n-1}P_{n-1}(t) - (r_n+g_n)P_n(t) \,.
\end{equation}
Here $r_k$ is the probability density of having one of the $k$
$Red$ robots change its state to $Green$, and $g_k$ is the
probability density of having one of the $N-k$ $Green$ robots
change its state to $Red$. Let us assume again that $\alpha {h}
M_0 \gg 1$ so that $\overline{\gamma}_g = 1-\overline{\gamma}_r$.
Then one has
\begin{equation}
r_k = k  (1 - \overline{\gamma}_r ) \ , \ g_k = (N-k)
\overline{\gamma}_r
\end{equation}
with $r_0=g_{-1}=0$, $r_{N+1}=g_N = 0$. $\overline{\gamma}_r$ is
history-averaged transition rate to $Red$ states.

The steady state solution of \eqref{eq:master2} is given by
\cite{VanKampen}
\begin{equation}
P_n^s = \frac{g_{n-1}g_{n-2}...g_1g_0}{r_nr_{n-1}...r_2r_1}P_0^s
\end{equation}
where $P_0^s$ is determined by the normalization:
\begin{equation}
P_0^s = \left[ 1+ \sum_{n=1}^{N}
\frac{g_{n-1}g_{n-2}...g_1g_0}{r_nr_{n-1}...r_2r_1} \right]^{-1}
\end{equation}
Using the expression for $\overline{\gamma}$,  after
some algebra we obtain
\begin{equation}
\label{eq:Pn} P_n^s = \frac{N!}{(N-n)!n!}
\overline{\gamma}_r^n(1-\overline{\gamma}_r)^{N-n}
\end{equation}
e.g., the steady state is a binomial distribution with parameter
$\overline{\gamma}$. Note again that this is a direct consequence
of the independence of the robots' dynamics. Indeed, since the
robots act independently, in the steady state each robot has the
same probability of being in either state. Moreover, using this
argument it becomes clear that the time-dependent probability
distribution $P_n(t)$ is given by \eqref{eq:Pn} with
$\overline{\gamma}$ replaced by  $p_r(t)$, \eqref{eq:solution}.


\subsection{Observations of Tasks and Robots}
\label{sec:phenomenological} In this section we study the more
complex dynamic task allocation mechanism in which robots make
decisions to change their state based on the observations of not
only available tasks but also on the observed task states of other
robots. Specifically, each robot now records the numbers and types
of task as well as the numbers and task types of robots it has
encountered. Again, we let $m_r$ and $m_g$ be the number of tasks
of $Red$ and $Green$ type, and $n_r$ and $n_g$ be the number of
robots in $Red$ and $Green$ task state in a robot's history
window. The probabilities for changing a robot's state are again
given by transition functions that now depend on the fractions of
observed tasks and robots of each type: $\hat{m}_r=m_r/(m_r+m_g)$,
$\hat{m}_g=m_g/(m_r+m_g)$, $\hat{n}_r=n_r/(n_r+n_g)$, and
$\hat{n}_g=n_g/(n_r+n_g)$. In our previous
work~\cite{Lerman03iros} we showed that in order to achieve the
desired long term behavior for task allocation (\ie, in the steady
state the average fraction of $Red$ and $Green$ robots is equal to
the fraction of $Red$ and $Green$ tasks respectively), the
transition rates must have the following functional form:
\begin{eqnarray}
\label{eq:fR}
f_{g \rightarrow r} (\hat{m}_r,\hat{n}_r) &=&\hat{m}_rg(\hat{m}_r-\hat{n}_r),\\
\label{eq:fG} f_{r \rightarrow g} (\hat{m}_r,\hat{n}_r)
&=&\hat{m}_g g(\hat{m}_g-\hat{n}_g) \equiv (1-\hat{m}_r)
g(-\hat{m}_r+\hat{n}_r) .
\end{eqnarray}
Here $g(z)$ is a continuous, monotonically increasing function of
its argument defined on an interval $[-1,1]$. In this paper we
consider the following forms for $g(z)$:
\begin{itemize}
\item {\em Power:}
$g(z)=100^{z}/100$
\item  {\em Stepwise linear:} $g(z)=z \Theta(z)$.\footnote{The step function $\Theta$ is defined as
$\Theta(z)=1$ if $z \geq 0$; otherwise, it is $0$. The step
function guarantees that no transitions to $Red$ state occur when
$ {m}_r < {n}_r$.}
\end{itemize}

To analyze this task allocation model, let us again consider a
single robot that searches for tasks to perform and makes a
transition to $Red$ and $Green$ states according to transition
functions defined above. Let $p_r(t)$ be the probability that the
robot is in the $Red$ state at time $t$, with \eqref{eq:inddyn}
governing its time evolution. Note that $p_r(t)$ is also the
average fraction of $Red$ robots, $p_r(t) = N_r(t)/N$.
%
\comment{
The equation governing the evolution of $p_r(t)$ reads
%\begin{equation}
%\label{eq:inddyn3} \frac{dp_r}{dt} = \varepsilon (1-p_r) f_{g
%\rightarrow r} - \varepsilon p_r f_{r \rightarrow g}
%\end{equation}
Here, again, $\varepsilon$ is the rate at which the robot makes
decisions to switch its state}

As in the previous case, the next step of the analysis is
averaging over the the robot's histories, \ie, $\hat{m}_r$ and
$\hat{n}_r$. Note that a robot's observations of available tasks
can still be modeled by a Poisson distribution similar to
\eqref{eq:poisson}. However, since the number of robots of each
task state changes stochastically in time, the statistics of $n_r$
and $n_g$
 should be modeled as a doubly stochastic
Poisson process (also called Cox process) with stochastic rates.
This would complicate the calculation of the average over $\hat{n}_r =
n_r/(n_r+n_g)$ and require mathematical details that go well
beyond the scope of this paper. Fortunately, as we demonstrated in
the previous section, if a robot's observation window contains
many readings, then the estimated fraction of task types is
exponentially close to the average of the Poisson distribution.
This suggests that for sufficiently high densities of tasks and
robots we can neglect the stochastic effects of modeling
observations for the purpose of our analysis, and replace the
robot's observation by their average (expected) values. In other
words, we use the following approximation:
\begin{eqnarray}
\label{eq:nr2}
\hat{n}_r &\approx& \frac{1} {h} \int_{t-h}^{t}{p_r(t')}dt'\\
\label{eq:mr2} \hat{m}_r &\approx& \frac{1} {h}
\int_{t-h}^{t}{\mu_r(t')}dt' .
\end{eqnarray}

The  Equations~\ref{eq:inddyn}, ~\ref{eq:nr2}, and~\ref{eq:mr2}
are a system of integro--differential equations that uniquely
determine the dynamics of $p_r(t)$. In the most general case it is
not possible to obtain solutions by analytical means, hence one
has to solve the system numerically. However, if the task density
does not change in time, we can still perform steady state
analysis. Steady state analysis looks for long-term solutions that
do not change in time, \ie, $d p_r/dt=0$. Let $\mu_r^0$ be the
density of $Red$ tasks, and $p_0=p_r(t \rightarrow \infty)$ be the
steady state value, so that $\hat{m}_r = \mu_r^0$, $\hat{n}_r =
p_r^0$. Then, by setting left hand side of \eqref{eq:inddyn} to
zero, we get

\begin{equation}
\label{eq:steady} (1-p_0) \mu_r^0 g(\mu_r^0-p_0) = p_0 (1-\mu_r^0)
g(-\mu_r^0+p_0)
\end{equation}

Note that $p_0=\mu_r^0$ is a solution to \eqref{eq:steady} so that
in the steady state the fraction of $Red$ robots equals the fraction
of red tasks as desired. To show that this is the only solution,
we note that for a fixed $\mu_r^0$ the right- and left-hand sides
of the equation are monotonically increasing and decreasing
functions of $p_0$ respectively, due to the monotonicity of $g(z)$.
Consequently, the two curves can meet only once and that proves the
uniqueness of the solution.




\subsubsection{Phenomenological Model}
\label{sec:stochastic}

Exact stochastic models of task allocation can quickly become
analytically intractable, as we saw above. Instead of exact
models, it is often more convenient to work with the so-called
Rate Equations model. These equations can be derived from the
exact stochastic model by appropriately averaging
it~\cite{Lerman03iros}; however, they are often (see, for example,
population dynamics~\cite{Haberman}) phenomenological, or \emph{ad
hoc}, in nature --- constructed by taking into account the
system's salient processes. This approach makes a number of
simplifying assumptions: namely, that the system is uniform and
dilute (not too dense), that actions of individual entities are
independent of one another, that parameters can be represented by
their mean values and that system behavior can be described by its
average value. Despite these simplifications, resulting models
have been shown to correctly describe dynamics of collective
behavior of robotic systems~\cite{Lerman04sab}. Phenomenological
models are useful for answering many important questions about the
performance of a MRS, such as, does the steady state exist, how
long does it take to reach it, and so on. Below we present a
phenomenological model of dynamic task allocation.


Individual robots are making their decisions to change task state
probabilistically and independently of one another. A robot will
change state from $Green$ to $Red$ with probability $f_{g
\rightarrow r}$ and with probability $1-f_{g \rightarrow r}$ it
will remain in the $Green$ state. We can succinctly write $\Delta
N_{g \rightarrow r}$  and $\Delta N_{r \rightarrow g}$, the number
of robots that switch from $Green$ to $Red$ and \emph{vice versa}
during a sufficiently small time interval $\Delta t$, as
\begin{eqnarray}
\Delta N_{g \rightarrow r}& =&\sum_{i=1}^{N_g}{x_i \big(f_{g
\rightarrow r} \delta(x_i-1)+(1- f_{g \rightarrow r})
\delta(x_i)\big)}
\nonumber\\
\Delta N_{r \rightarrow g}&=&\sum_{i=1}^{N_r}{(1-x_i) \big(f_{r
\rightarrow g} \delta(x_i)+(1-f_{r \rightarrow g})
\delta(x_i-1)\big)}\,. \nonumber
\end{eqnarray}
Here we introduced a state variable $x_i$, such that $x_i=1$ when a robot is in the $Green$ state,
and $x_i=0$ when a robot is in the $Red$ state. $\delta(x)$ is Kronecker delta, defined as $\delta(x)=1$ when $x=0$ and $\delta(x)=0$ otherwise.  Therefore, $\Delta N_{g \rightarrow r}$
is a random variable from a binomial distribution specified by a mean $\mu=f_{g \rightarrow r} N_g$ and variance
$\sigma^2=f_{g \rightarrow r} (1-f_{g \rightarrow r}) N_g$.
Similarly, the distribution of the random variable $\Delta N_{r
\rightarrow g}$ is specified by mean $\mu=f_{r \rightarrow g} N_r$
and variance $\sigma^2=f_{r \rightarrow g} (1-f_{r \rightarrow g})
N_r$.


During a time interval $\Delta t$ the total number of robots in
$Red$ and $Green$ task states will change as individual
robots make decisions to change states. The following finite
difference equation specifies how the number of $Red$ will change on average:
\begin{equation}
\label{eq:stochastic} N_r(t+\Delta t)  = N_r(t) + \varepsilon
\Delta N_{g \rightarrow r} \Delta t - \varepsilon  \Delta N_{r
\rightarrow g} \Delta t \nonumber
\end{equation}
Rearranging the equation and taking the continuous time limit
($\Delta t \rightarrow 0$) yields a differential Rate Equation
that describes time evolution of the number of $Red$ robots. By
taking the means of $\Delta N$'s as their values, we recover
\eqref{eq:inddyn}.


Keeping $\Delta N$'s as random variables allows us to study the
effect the probabilistic nature of the robots' decisions have on
the collective behavior.\footnote{Note that we do not model here
the effect of observation noise due to uncertainty in sensor
readings and fluctuations in the distribution of tasks.} We solve
\eqref{eq:stochastic} by iterating it in time and drawing $\Delta
N$'s at random from their respective distributions. The solutions
are subject to the initial condition $N_r(t \leq 0)=N$ and specify
the dynamics of task allocation in robots.



%%%%%%%%%%%%%%%%%%%%%%%

Functions $f_{g \rightarrow r}$ and $f_{r \rightarrow g}$ are
calculated using
estimates of the densities of $Red$ tasks ($m_r$) and robots in $Red$ state ($n_r$)
from the observed counts stored in the robot's history window.

% collective transition rates
Transition rates  $f_{g \rightarrow r}$ and $f_{r \rightarrow g}$ in the model are mean values, averaged over all histories and all robots. In order to compute them, we need to aggregate observations of all robots. Suppose each robot has a history  window of length $h$.
For a particular robot $i$, the values in the most recent observational slot
are $N^0_{i,r}$, $N^0_{i,g}$, $M^0_{i,r}$ and
$M^0_{i,g}$, the observed numbers of $Red$ and $Green$ robots
and tasks respectively at time $t$. In the next latest slot, the
values are $N^1_{i,r}$, $N^1_{i,g}$, $M^1_{i,r}$ and
$M^1_{i,g}$, the observed numbers at time $t-\Delta$, and so
on. Each robot estimates the densities of $Red$ robots and tasks using the following calculation:
\begin{eqnarray}
{n}_{i,r} & = & \frac{1}{h}\sum^{h-1}_{j=0}{\frac{N^j_{i,r}}{N^j_{i,r}+N^j_{i,g}}}= \frac{1}{h}\sum^{h-1}_{j=0}{n^j_{i,r}} \\
{m}_{i,r} & = & \frac {1}{h}
\sum^{h-1}_{j=0}{\frac{M^j_{i,r}}{M^j_{i,r}+M^j_{i,g}}}=
\frac{1}{h}\sum^{h-1}_{j=0}{m^j_{i,r}}.
\end{eqnarray}


When observations of all robots are taken into account, the
mean of the observed densities of $Red$ robots at time $t$ --- $\frac{1} {N} \sum^N_{i=1}{n^0_{i,r}} $  --- will fluctuate due to observation noise, but on average it will be proportional to $N_r(t)/N$, which is the actual density of $Red$
robots at time $t$. The proportionality factor is related to
physical robot parameters, such as speed and observation area (see \secref{sec:results1}).
Likewise, the average of the observed densities at time  $t-j \Delta$ is $\frac{1} {N}
\sum^N_{i=1}{n^j_{i,r}} \propto N_r(t-j\Delta)/N$, the
 density of robots at time $t-j\Delta$.  Thus, the aggregate
estimates of the fractions of $Red$ robots and tasks are:
\begin{eqnarray}
\label{eq:2} \hat{n}_r & = & \frac{1}{N}
\sum^{N}_{i=1}{{n}_{i,r}}=
\frac{1}{Nh}\sum^{h-1}_{j=0}{N_r(t-j\Delta)} \\
\label{eq:2b}\hat{m}_r & = & \frac{1}{N}
\sum^{N}_{i=1}{{m}_{i,r}}=
\frac{1}{Mh}\sum^{h-1}_{j=0}{M_r(t-j\Delta)}
\end{eqnarray}
Robots are making their decisions asynchronously, {\em i.e.}, at
slightly different times. Therefore, the last terms in the above
equations are best expressed in continuous form: {\em e.g.}, $1/Nh
\int^0_{h}N_r(t-\tau)d\tau$ (see \eqref{eq:nr2} and
\eqref{eq:mr2}).

Estimates \eqref{eq:2} and \ref{eq:2b} can be plugged into
\eqref{eq:fR} and \eqref{eq:fG} to compute the values of
transition probabilities for any choice of the transition function
(power or linear). Once we know $f_{r \rightarrow g}$ and $f_{g
\rightarrow r}$, we can solve \eqref{eq:stochastic} to study the
dynamics of task allocation in robots. Note that
\eqref{eq:stochastic} is now a time-delay finite difference
equation, and solutions will show typical oscillations.

We solve the models presented in this section and validate their predictions in context of the multi-foraging task described next.



\section{Multi-Robot Multi-Foraging Task\label{sec:multi-foraging}}

In this section we describe the multi-foraging task domain in
which we experimentally tested our dynamic task allocation
mechanism, including the simulation environment used and robot
sensing and control characteristics. In \secref{sec:results1} we
use this application to validate the models presented above, solve
them and compare their solutions to the results of embodied
simulations.


\subsection{Task Description \label{sec:task_description}}

The traditional foraging task is defined by having an individual robot or group of robots collect a set of objects from an environment and either consume on the spot or return them to a common location \cite{Goldberg02}.  Multi-foraging, a variation on traditional foraging, is defined in \cite{Balch99} and consists of an arena populated by multiple types of objects to be concurrently collected.

In our multi-foraging domain, there are two types of objects (e.g., pucks) randomly dispersed throughout the arena: Puck\(_{Red}\) and Puck\(_{Green}\) pucks that are distinguishable by their color.  Each robot is equally capable of foraging both puck types, but can only be allocated to foraging for one type at any given time.  Additionally, all robots are engaged in foraging at all times; a robot cannot be idle.  A robot may switch the puck type for which it is foraging according to its control policy, when it determines it is appropriate to do so. This is an instantiation of the general task allocation problem described earlier in this paper, with puck colors representing different task types.


In the multi-foraging task, the robots move in an enclosed arena
and pick up encountered pucks.  When a robot picks up a puck, the
puck is consumed (i.e., it is immediately removed from the
environment, not transported to another region) and the robot
carries on foraging for other pucks.  Immediately after a puck is
consumed, another puck of the same type is placed in the arena at
a random location.  This is done so as to maintain a constant puck
density in the arena throughout the course of an experiment.  In
some situations, the density of pucks can impact the accuracy or
speed of convergence to the desired task allocation.  This is an
important consideration in dynamic task allocation mechanisms for
many domains; however, in this work we want to limit the number of
experimental variables impacting system performance.  Therefore,
we reserve the investigation on the impact of varying puck
densities for future work.

The role of dynamic task allocation in this domain requires the
robots to split their numbers by having some forage for
Puck\(_{Red}\) pucks and others for Puck\(_{Green}\) pucks.  For
the purpose of our experiments, we desire an allocation of robots
to converge to a situation in which the proportion of robots
foraging for Puck\(_{Red}\) pucks is equal to the proportion of
Puck\(_{Red}\) pucks present in the foraging arena (e.g., if
Puck\(_{Red}\) pucks make up 30\% of the pucks present in the
foraging arena, then 30\% of the robots should be foraging for
Puck\(_{Red}\) pucks).  In general, the desired allocation could
take other forms.  For example, it could be related to the
relative reward or cost of foraging each puck type without change
to our approach.

We note that the limited sensing capabilities and lack of direct communication of the individual robots in the implementation of our task domain prohibits them from acquiring global information such as the size and shape of the foraging arena, the initial or current number of pucks to be foraged (total or by type), or the initial or current number of foraging robots (total or by foraging type).

\subsection{Simulation Environment \label{simulation}}

In order to experimentally demonstrate the dynamic task allocation
mechanism we made use of a physically-realistic simulation
environment.  Our simulation trials were performed using Player
and Gazebo simulation environments.  Player~\cite{Player} is a
server that connects robots, sensors, and control programs over a
network.  Gazebo \cite{Gazebo} simulates a set of Player devices
in a 3-D physically-realistic world with full dynamics.  Together,
the two represent a high-fidelity simulation tool for individual
robots and teams that has been validated on a collection of
real-robot robot experiments using Player control programs
transferred directly to physical mobile robots.
%
\begin{figure}
\begin{center}
\includegraphics[height=1.7in]{gazebo_snapshot1.eps}
\includegraphics[height=1.7in]{gazebo_snapshot2.eps}
\caption{Snapshots from the simulation environment used. (left) An overhead view of foraging arena and robots.  (right) A closeup of robots and pucks.} \label{fig:gazebo_snapshot}
\end{center}
\end{figure}
%
Figure~\ref{fig:gazebo_snapshot} provides snapshots of the simulation environment used.  All experiments involved 20 robots foraging in a 400m\(^2\) arena.

The robots used in the experimental simulations are realistic models of the ActivMedia Pioneer 2DX mobile robot.  Each robot, approximately 30 cm in diameter, is equipped with a differential drive, an odometry system using wheel rotation encoders, and 180 degree forward-facing laser rangefinder used for obstacle avoidance and as a fiducial detector/reader.  Each puck is marked with a fiducial that marks the puck type and each robot is equipped with a fiducial that marks the active foraging state of the robot.  Note that the fiducials do not contain unique identities of the pucks or robots but only mark the type of the puck or the puck type a given robot is engaged in foraging.  Each robot is also equipped with a 2-DOF gripper on the front, capable of picking up a single 8 cm diameter puck at a time.  There is no capability available for explicit, direct communication between robots nor can pucks and other robots be uniquely identified.

\subsection{Behavior-Based Robot Controller\label{controller}}

All robots have identical behavior-based controllers consisting of the following mutually exclusive behaviors: Avoiding, Wandering, Puck Servoing, Grasping, and Observing.  Descriptions of robot behaviors are provided below.

\begin{list}{-}{}
\item The {\bf Avoiding} behavior causes the robot to turn to avoid obstacles in its path.
\item The {\bf Wandering} behavior causes the robot to move forward and, after a random length of elapsed time, to turn left or right through a random arc for a random period of time.
\item The {\bf Puck Servoing} behavior causes the robot to move toward a detected puck of the desired type.  If the robot's current foraging state is Robot\(_{Red}\), the desired puck type is Puck\(_{Red}\), and if the robots current foraging state is Robot\(_{Green}\), the desired puck type is Puck\(_{Green}\).
\item The {\bf Grasping} behavior causes the robot to use its gripper to pick up and consume a puck within the gripper's grasp.
\item The {\bf Observing} behavior causes the robot to take the current fiducial information returned by the laser rangefinder and record the detected pucks and robots to their respective histories.  The robot then updates its foraging state based on those histories.  A description of the histories is given in \secref{state-info} and a description of the foraging state update procedure is given in \secref{transition-functions}.
\end{list}

\begin{table*}[t]
\begin{center}
\begin{tabular}{|c|c|c|c|c|}
\hline Obstacle & Puck\(_{Det}\) & Gripper Break- & Observation & Active \\
       Detected & Detected & Beam On & Signal & Behavior \\
\hline
\hline X & X & X & 1 & Observing \\
\hline 1 & X & X & X & Avoiding \\
\hline 0 & 1 & 0 & 0 & Puck Servoing \\
\hline 0 & X & 1 & 0 & Grasping \\
\hline 0 & X & X & X & Wandering \\ \hline
\end{tabular}
\end{center}
\caption{Behavior Activation Conditions.  Behaviors are listed in order of decreasing rank.  Higher ranking behaviors preempt lower ranking behaviors in the event multiple are active.  X denotes the activation condition is irrelevant for the behavior.  \label{beh-table}}
\end{table*}

Each behavior listed above has a set of activation conditions based on relevant sensor inputs and state values.  When met, the conditions cause the behavior to be become active.  A description of when each activation condition is active is given below.  The activation conditions of all behaviors are shown in Table~\ref{beh-table}.

\begin{list}{-}{}
\item The {\bf Obstacle Detected} activation condition is true when an obstacle is detected by the laser rangefinder within a distance of 1 meter.  Other robots, pucks, and the arena walls are considered obstacles.
\item The {\bf Puck\(_{Det}\) Detected} activation condition is true if the robot's current foraging state is Robot\(_{Det}\) and a puck of type Puck\(_{Det}\) (where Det is {\it Red} or {\it Green}) is detected within a distance of 5 meters and within \(\pm\) 30 degrees of the robot's direction of travel.
\item The {\bf Gripper Break-Beam On} activation condition is true if the break-beam sensor between the gripper jaws detects an object.
\item The {\bf Observation Signal} activation condition is true if the distance traveled by the robot according to odometry since the last time the {\bf Observing} behavior was activated is greater than 2 meters.
\end{list}

\subsubsection{Robot State Information \label{state-info}}

All robots maintain three types of state information: foraging state, observed puck history, and observed robot history.  The foraging state identifies the type of puck the robot is currently involved in foraging.  A robot with a foraging state of Robot\(_{Red}\) refers to a robot engaged in foraging Puck\(_{Red}\) pucks and a foraging state of Robot\(_{Green}\) refers to a robot engaged in foraging Puck\(_{Green}\) pucks.  For simplicity, we will refer to both robot foraging states and puck types as $Red$ and $Green$. The exact meaning will be clear in context.

Each robot is outfitted with a colored beacon passively observable by nearby robots which indicates the robot's current foraging state.  The color of the beacon changes to reflect the current state -- a red beacon for a foraging state of $Red$ and a green beacon for foraging state $Green$.  Thus, the colored beacon acts as a form of local, passive communication conveying the robot's current foraging state.  All robots maintain a limited, constant-sized history storing the most recently observed puck types and another constant-sized history storing the foraging state of the most recently observed robots.  Neither of these histories contains a unique identity or location of detected pucks or robots, nor does it store a time stamp of when any given observation was made.  The history of observed pucks is limited to the last {\tt MAX-PUCK-HISTORY} pucks observed and the history of the foraging states of observed robots is limited to the last {\tt MAX-ROBOT-HISTORY} robots observed.

While moving about the arena, each robot keeps track of the approximate distance it has traveled by using odometry measurements.  At every interval of 2 meters traveled, the robot makes an observation performed by the {\it Observing} behavior.  This procedure is nearly instantaneous; therefore, the robot's behavior is not outwardly affected.  The area in which pucks and other robots are visible is within 5 meters and \(\pm\) 30 degrees in the robot's direction of travel.  Observations are only made after traveling 2 meters because updating too frequently leads to over-convergence of the estimated puck and robot type proportions due to repeated observations of the same pucks and/or robots.  On average, during our experiments, a robot detected 2 pucks and robots  per observation.

\subsubsection{Foraging State Transition Function \label{transition-functions}}

After a robot makes an observation, it re-evaluates and probabilistically changes its current foraging state given the newly updated puck and robot histories. The probability by which the robot changes its foraging state is defined by the transition function.  We experimentally studied transition functions given by \eqref{eq:rates}, \eqref{eq:fR} and \eqref{eq:fG} with both $power$ and $linear$ forms. Below we present results of analysis and simulations and discuss the consequences the choice of the transition function has on system level behavior.


% section: results

% dev set decision table
% manually created from 25-official-html.tgz
% with custom editing

\begin{table}
\centering
\begin{tabular}{l|rr}
\toprule
 & \multicolumn{2}{c}{\textbf{ELAS F1}} \\
\textbf{Treebank} & \textbf{sem-frag}
 & \textbf{heuristic}\\
%\hline
\midrule
% e7 elmo udpf task ar padt	allennlp 090 dm lbert luxfb ar padt 20200424 080759
% e7 elmo udpf task ar padt	copy2e arcase mark rel
ar\_padt & \textbf{70.99} & 59.74 \\
% e5 elmo udpf task bg btb	allennlp 090 dm pbert u bg btb 20200312 003243	
% e7 elmo udpf task bg btb	copy2e encase mark rel	
bg\_btb & \textbf{88.09} & 86.19 \\
%\hline
\midrule
% 	e3 elmo udpf task cs cac	allennlp 090 dm pbert luxfb cs cac 20200424 002226
%	e5 elmo udpf task cs cac	copy2e arcase mark rel	
cs\_cac & \textbf{86.51} & 74.41 \\
% e3 elmo udpf task cs fictree	allennlp 090 dm pbert luxfb cs cac 20200424 002226
% e5 elmo udpf task cs fictree	copy2e arcase mark rel	77.37
cs\_fictree & \textbf{83.23} & 77.37  \\
% e3 elmo udpf task cs fictree	allennlp 090 dm pbert u cs cac 20200419 171603
% e3 elmo udpf task cs cac	copy2e arcase mark
cs\_pdt & \textbf{79.58} & 	71.19 \\
%\hline
\midrule
% e7 elmo udpf task en ewt + ud25 en gum + ud25 en lines + ud25 en partut	allennlp 090 dm lbert u en ewt 20200312 051351
% e7 elmo udpf task en ewt + ud25 en gum + ud25 en lines + ud25 en partut	copy2e encase mark cc rel
en\_ewt & \textbf{84.71} & 	82.86 \\
% e3 elmo udpf task et edt + task et ewt	allennlp 090 dm mbert u et 20200419 234001
% e3 elmo udpf task et edt + task et ewt	copy2e arcase mark
et\_edt & 62.74 & \textbf{69.35} \\
% e5 elmo udpf task fi tdt fasttext udpf task fi tdt	allennlp 090 dm lbert u fi tdt 20200420 050020
% e7 elmo udpf task fi tdt	copy2e arcase mark rel
fi\_tdt & \textbf{83.64} & 71.84 \\ 
% e5 elmo udpf task fr sequoia + ud25 fr gsd + ud25 fr partut + ud25 fr spoken	allennlp 090 dm mbert u fr sequoia 20200312 072651
% e3 elmo udpf task fr sequoia + ud25 fr gsd + ud25 fr partut + ud25 fr spoken	copy2e	
fr\_sequoia & \textbf{88.65} &  87.53 \\
% e3 elmo udpf task it isdt + ud25 it partut + ud25 it postwita + ud25 it twittiro + ud25 it vit	allennlp 090 dm lbert u it isdt 20200419 172143	
% e7 elmo udpf task it isdt	copy2e encase mark cc rel	
it\_isdt & \textbf{90.13} & 88.28 \\
% e3 plain udpf task lt alksnis	allennlp 090 dm mbert u lt alksnis 20200420 014618	
% e3 plain udpf task lt alksnis	copy2e arcase mark	
lt\_alksnis & \textbf{73.63} & 57.84 \\
% e7 elmo udpf task lv lvtb	allennlp 090 dm mbert luxfb lv lvtb 20200423 191418	
% e5 elmo udpf task lv lvtb fasttext udpf task lv lvtb	copy2e encase rel	
lv\_lvtb & \textbf{81.82} & 71.29 \\
%\hline
\midrule
% 	e7 elmo udpf task nl alpino + task nl lassysmall	allennlp 090 dm lbert u nl alpino 20200312 025649	
% e7 elmo udpf task nl alpino + task nl lassysmall	copy2e encase mark cc rel	
nl\_alpino & \textbf{89.93} & 89.00 \\
% e7 elmo udpf task nl alpino + task nl lassysmall	allennlp 090 dm lbert u nl alpino 20200312 025649	
% e7 elmo udpf task nl alpino + task nl lassysmall	copy2e encase mark cc rel	
nl\_lassysmall & 79.00 & \textbf{81.24} \\
%\hline
\midrule
% e5 elmo udpf task pl lfg + task pl pdb	allennlp 090 dm mbert luxfb pl lfg 20200423 222537	
% e5 elmo udpf task pl lfg + task pl pdb	copy2e encase mark rel	
pl\_lfg & \textbf{94.12} & 93.84 \\
% e3 fasttext udpf task pl lfg + task pl pdb	allennlp dev dm lbert luxf pl 20200416 194726	
% e5 elmo udpf task pl lfg + task pl pdb	copy2e arcase mark rel	
pl\_pdb & \textbf{82.25} & 78.27 \\
%\hline
\midrule
% e7 elmo udpf task ru syntagrus	allennlp 090 dm lbert lufb ru syntagrus 20200423 210055	
% e7 elmo udpf task ru syntagrus	copy2e arcase mark	
ru\_syntagrus & \textbf{88.48} & 80.03 \\
% e3 elmo udpf task sk snk plain udpf task sk snk	allennlp 090 dm mbert u sk snk 20200420 020636	
% e7 elmo udpf task sk snk	copy2e arcase mark	75.98
sk\_snk  & \textbf{81.30} & 75.98 \\
% e3 elmo udpf task sv talbanken	allennlp 090 dm lbert u sv talbanken 20200419 195336	
% e7 elmo udpf task sv talbanken	copy2e encase mark cc rel	
sv\_talbanken & \textbf{84.54} & 81.32 \\
% e3 plain udpf task ta ttb	allennlp 090 dm mbert u ta ttb 20200419 232103	
%	e3 plain udpf task ta ttb	copy2e arcase	
ta\_ttb & \textbf{55.68} & 43.94 \\
% e3 elmo udpf task uk iu	allennlp 090 dm mbert u uk iu 20200420 004219	
% e7 elmo udpf task uk iu	copy2e arcase mark	
uk\_iu & \textbf{82.41} & 76.88 \\
%\hline
\bottomrule
\end{tabular}
\caption{Development set ELAS F1 score %f-score
        for the best semantic parser evaluated without connecting
            fragmented graphs (sem-frag)
        and
        for the best combination of heuristic rules
            (heuristic)
}
\label{devresults:decision_custom}
\end{table}

% eof

Table~\ref{devresults:decision_custom} compares the semantic parser against the heuristic approach on the ELAS F1 metric.
The evaluation script was run without connecting fragmented graphs and format validation.
For all but two treebanks, the semantic parser performs better than the
best
heuristic approach.
For some languages, the difference in performance is large.
For \texttt{et\_ewt}, which does not have a development set,
we suspect that we overfitted our semantic parser on the
\texttt{et\_ewt} training data
by allowing it to train for 75 epochs.

% test set results table
% manually created from eval pages linked on
% https://quest.ms.mff.cuni.cz/sharedtask/cgi-bin/overview.pl

% main body generated by copy and pasting the qualitative tables,
% then using `cut -f1,16` to get the right columns, pasting them
% together with `paste` and tabs converted to `&` and \\ added to
% lines in `vim`

\begin{table}
\centering
\begin{tabular}{l|rrr}
\toprule
 & \multicolumn{3}{c}{\textbf{ELAS F1}} \\
\textbf{Treebank} & \textbf{subm}
 & \textbf{frag fix} & \textbf{re-run}\\
\midrule
Arabic-PADT         &  57.19  &  70.08  &  \bf 70.40  \\
Bulgarian-BTB       &  77.29  &  89.58  &  \bf 89.60  \\
Czech-FicTree       &  70.04  &  80.77  &  \bf 81.63  \\
Czech-CAC           &  71.72  &  86.00  &  \bf 86.38  \\
Czech-PDT           &  65.94  &  79.03  &  \bf 79.84  \\
Czech-PUD           &  64.34  &  77.37  &  \bf 78.08  \\
Dutch-Alpino        &  71.44  &  87.61  &  \bf 87.77  \\
Dutch-L.Small       &  64.03  &  77.39  &  \bf 77.24  \\
English-EWT         &  70.61  &  \bf 83.56  & \bf 83.56  \\
English-PUD         &  70.25  &  86.88  & \bf 87.03  \\
Estonian-EDT        &  62.29  &  68.20  &  \bf 68.37  \\
Estonian-EWT        &  55.70  &  \bf 61.19  &  60.67  \\
Finnish-TDT         &  73.02  &  \bf 84.36  &  84.33  \\
Finnish-PUD         &  71.58  & \bf 84.62  & \bf 84.62  \\
French-Sequoia      &  77.44  &  87.58  & \bf 88.60  \\
French-FQB          &  74.30  &  82.68  & \bf 83.26  \\
Italian-ISDT        &  71.98  &  \bf 90.24  &  90.23  \\
Latvian-LVTB        &  72.41  &  81.81  &  \bf 82.40  \\
Lithuanian-AL.      &  58.36  &  68.76  &  \bf 68.84  \\
Polish-LFG          &  61.23  &  \bf 70.89  &  70.71  \\
Polish-PDB          &  67.68  &  80.93  &  \bf 82.43  \\
Polish-PUD          &  65.64  &  79.77  & \bf 80.79  \\
Russian-SynT.       &  75.27  &  89.21  & \bf 89.47  \\
Slovak-SNK          &  68.43  &  81.63  &  \bf 81.97  \\
Swedish-Talb.       &  71.86  &  86.78  & \bf 86.72  \\
Swedish-PUD         &  64.70  &  79.35  & \bf 79.37  \\
Tamil-TTB           &  48.47  &  \bf 57.28  &  57.10  \\
Ukrainian-IU        &  66.43  &  79.81  & \bf 82.92  \\
\midrule
Average             &  67.49  &  79.76  & \bf 80.15  \\
\bottomrule
\end{tabular}
\caption{Test set results:
    subm = submitted,
    frag fix = using our own fragment connector and quick-fix.pl without connect-to-root,
    re-run = a re-run with bug fixes, no new models but new model selection
}
\label{testresults_custom}
\end{table}

% eof

Table~\ref{testresults_custom} shows test set ELAS obtained on the shared task
submission site for
\textit{(a)} our submission fully relying on the organiser's
             \texttt{quick-fix} tool to fix issues in the output of
             our system,
\textit{(b)} the same predictions post-processed by our own
             fragment connector that aims to minimise the
             number of root edges added, and
\textit{(c)} a re-run of our pipeline using the same models
             for system components as before but with all
             bugs fixed during development applied to all
             predictions and new decisions which models
             to apply to the test sets.
While the \texttt{quick-fix} tool enabled us to make a valid submission
in time, its
approach of adding edges from the root node to
all unreachable tokens
has a strong negative impact on 
precision, \eg 62.26 ELAS precision on the Czech CAC development set
\vs 87.37 without post-processing.
Our own post-competition fix avoids this
and would have brought us to the top half of the competition.

% eof


\section{Discussion}
\label{sec:discussion}

We have constructed and analyzed mathematical models of dynamic
task allocation in a multi-robot system. The models are general
and can be easily extended to other systems in which robots use a
history of local observations of the environment as a basis for
making decisions about future actions. These models are based on
theory of stochastic processes. In order to study a robot's
behavior, we do not need to know its exact trajectory or the
trajectories of other robots; instead, we derive a probabilistic
model that governs how a robot's behavior changes in time. In some
simple cases these models can be solved analytically. However,
stochastic models are usually too complex for exact analytic
treatment. Thus, in the scenario described in
\secref{sec:pucksonly} in which only observations of tasks are
made, though the individual model is tractable, the stochastic
model of the collective behavior is not. Instead, we use averaging
and approximation techniques to quantitatively study the dynamics
of the collective behavior. Such models, therefore, do not
describe the robots' behavior in a single experiment, but rather
the behavior that has been averaged over many experimental or
simulations runs. Fortunately, results of experiments and
simulations are usually presented as an average over many runs;
therefore, mathematical models of average collective behavior can
be used to describe experimental results. In fact, the stochastic
model produces excellent agreement with experimental results under
all experimental conditions and without using any adjustable
parameters.


Phenomenological models are more straightforward to construct and
analyze than exact stochastic models --- in fact, they can be
easily constructed from details of the individual robot
controller~\cite{Lerman04sab}. The ease of use comes at a price,
namely, the number of simplifying assumptions that were made in
order to produce a mathematically tractable model. First, we
assume that the robots are functioning in a dilute limit, where
they are sufficiently separated that their actions are largely
independent of one another. Second, we assume that the transition
rates can be represented by aggregate quantities that are
spatially uniform and independent of the details of the individual
robot's actions or history. We also assume the system is
homogeneous, with modeled robots characterized by a set of
parameters, each of them representing the mean value of some real
robot feature: mean speed, mean duration for performing a certain
maneuver, and so on. Real robot systems are heterogeneous: even if
the robots are executing the same controller, there will always be
variations due to inherent differences in hardware. We do not
consider parameter distributions in our models as would be
necessary to describe such heterogeneous systems. Finally,
phenomenological models more reliably describe systems where
fluctuations (deviations from the mean behavior) can be neglected,
as happens in large systems or when many experimental runs are
aggregated. However, even if phenomenological models don't agree
with experiments exactly, as we saw in \secref{sec:results2}, they
can still reliably predict most behaviors of interest even in
not-so-large systems. They are, therefore, a useful tool for
modeling and analyzing multi-robot systems.


\section{Conclusions}
In this paper, we set out to address the problem of multi-tasking robots in multi-robot tasks. 
%A fundamental limitation of existing multi-robot systems was addressed by the removal of a restrictive assumption that was often made--robots are single-tasking.
%Our method allowed coalitions to overlap thus enabling multi-tasking robots. 
We observed that the key underlying challenge was to reason about the physical constraints that could be synergistically satisfied.
%which directly affected the feasibility of multi-tasking.
To address the challenge, we developed our method based on the information invariant theory and modeled constraints as information instances. 
%This allowed us to reason about the relationships between constraints by reasoning about those between information requirements. 
Thereby, a formal and general framework to achieve multi-tasking robots was developed. 
We showed that our algorithm was sound and complete under our problem settings. 
%Our method was integrated with a simple greedy heuristic for task allocation.
Simulation  results  were  provided  to  show  the  effectiveness  of  our approach under resource-constrained situations and in handling challenging situations. % in a multi-UAV simulator. 

% The idea of multi-tasking is attractive in many ways. 
% Humans are living in multi-tasking environments--at any point of time, 
% we are optimizing for more than one task. 
% Multi-task often leads to more efficient task performance since it allows us to exploit task synergies. 
% The work presented in this paper takes us one step forward in realizing multi-tasking robots. 
% In particular, we started looking at the feasibility of multi-tasking. 
% There are many potential directions to pursue along this direction. First, several limitations are present in the current approach. 
% For example, although our method guarantees that there exists a physical configuration that satisfies all the constraints, it does not explicitly take the environmental influence into account. For example, a narrow corridor may prevent a robot formation from passing through, even though all the constraints for the formation do not introduce any conflicts. In this sense, our work should better be characterized as establishing a necessary condition for multi-tasking. Also, our method is mainly focused on the ``{\it planning}'' phase and hence does not address how the robots reach the desired configuration and maintain the constraints. These issues are assumed to be handled by the execution layer.

% More generally, the question of how to execute the tasks with overlapping coalitions is not addressed in this work. 
% As we already discussed, executing individual tasks with non-overlapping coalitions is straightforward but task synergies impose additional requirements on the task execution: how should the robots that are assigned multiple tasks execute them? Should they consider them in a prioritized strategy~\cite{van2005prioritized}? Or should they combine the different tasks in a way that is similar to motor schemas~\cite{arkin2}. 
% Communication requirements for maintaining the constraints must also be taken into account. How should the robots optimize their communication to improve the task performance? 

% The stringency of the physical constraints is another interesting question. It may be desirable to relax the constraints in certain situations (e.g., due to environmental influences). In such cases, it may be important to consider the problem where the constraints are least violated~\cite{kim2012revision}, or specify task constraints in different ways to increase the diversity of the configurations~\cite{srivastava2007domain} so as to make it robust to different environments. 


\section*{Acknowledgment}
The research reported here was supported in part by the Defense Advanced Research Projects Agency (DARPA) under contract number F30602-00-2-0573.

\bibliographystyle{plain}
% argument is your BibTeX string definitions and bibliography database(s)
%\bibliography{IEEEabrv,../bib/paper}
\bibliography{../../../tex/bib/agents,../../../tex/bib/robots,../../../tex/bib/lerman,../../../tex/bib/physics}


\end{document}
