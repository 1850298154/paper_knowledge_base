
\section{Multi-Robot Multi-Foraging Task\label{sec:multi-foraging}}

In this section we describe the multi-foraging task domain in
which we experimentally tested our dynamic task allocation
mechanism, including the simulation environment used and robot
sensing and control characteristics. In \secref{sec:results1} we
use this application to validate the models presented above, solve
them and compare their solutions to the results of embodied
simulations.


\subsection{Task Description \label{sec:task_description}}

The traditional foraging task is defined by having an individual robot or group of robots collect a set of objects from an environment and either consume on the spot or return them to a common location \cite{Goldberg02}.  Multi-foraging, a variation on traditional foraging, is defined in \cite{Balch99} and consists of an arena populated by multiple types of objects to be concurrently collected.

In our multi-foraging domain, there are two types of objects (e.g., pucks) randomly dispersed throughout the arena: Puck\(_{Red}\) and Puck\(_{Green}\) pucks that are distinguishable by their color.  Each robot is equally capable of foraging both puck types, but can only be allocated to foraging for one type at any given time.  Additionally, all robots are engaged in foraging at all times; a robot cannot be idle.  A robot may switch the puck type for which it is foraging according to its control policy, when it determines it is appropriate to do so. This is an instantiation of the general task allocation problem described earlier in this paper, with puck colors representing different task types.


In the multi-foraging task, the robots move in an enclosed arena
and pick up encountered pucks.  When a robot picks up a puck, the
puck is consumed (i.e., it is immediately removed from the
environment, not transported to another region) and the robot
carries on foraging for other pucks.  Immediately after a puck is
consumed, another puck of the same type is placed in the arena at
a random location.  This is done so as to maintain a constant puck
density in the arena throughout the course of an experiment.  In
some situations, the density of pucks can impact the accuracy or
speed of convergence to the desired task allocation.  This is an
important consideration in dynamic task allocation mechanisms for
many domains; however, in this work we want to limit the number of
experimental variables impacting system performance.  Therefore,
we reserve the investigation on the impact of varying puck
densities for future work.

The role of dynamic task allocation in this domain requires the
robots to split their numbers by having some forage for
Puck\(_{Red}\) pucks and others for Puck\(_{Green}\) pucks.  For
the purpose of our experiments, we desire an allocation of robots
to converge to a situation in which the proportion of robots
foraging for Puck\(_{Red}\) pucks is equal to the proportion of
Puck\(_{Red}\) pucks present in the foraging arena (e.g., if
Puck\(_{Red}\) pucks make up 30\% of the pucks present in the
foraging arena, then 30\% of the robots should be foraging for
Puck\(_{Red}\) pucks).  In general, the desired allocation could
take other forms.  For example, it could be related to the
relative reward or cost of foraging each puck type without change
to our approach.

We note that the limited sensing capabilities and lack of direct communication of the individual robots in the implementation of our task domain prohibits them from acquiring global information such as the size and shape of the foraging arena, the initial or current number of pucks to be foraged (total or by type), or the initial or current number of foraging robots (total or by foraging type).

\subsection{Simulation Environment \label{simulation}}

In order to experimentally demonstrate the dynamic task allocation
mechanism we made use of a physically-realistic simulation
environment.  Our simulation trials were performed using Player
and Gazebo simulation environments.  Player~\cite{Player} is a
server that connects robots, sensors, and control programs over a
network.  Gazebo \cite{Gazebo} simulates a set of Player devices
in a 3-D physically-realistic world with full dynamics.  Together,
the two represent a high-fidelity simulation tool for individual
robots and teams that has been validated on a collection of
real-robot robot experiments using Player control programs
transferred directly to physical mobile robots.
%
\begin{figure}
\begin{center}
\includegraphics[height=1.7in]{gazebo_snapshot1.eps}
\includegraphics[height=1.7in]{gazebo_snapshot2.eps}
\caption{Snapshots from the simulation environment used. (left) An overhead view of foraging arena and robots.  (right) A closeup of robots and pucks.} \label{fig:gazebo_snapshot}
\end{center}
\end{figure}
%
Figure~\ref{fig:gazebo_snapshot} provides snapshots of the simulation environment used.  All experiments involved 20 robots foraging in a 400m\(^2\) arena.

The robots used in the experimental simulations are realistic models of the ActivMedia Pioneer 2DX mobile robot.  Each robot, approximately 30 cm in diameter, is equipped with a differential drive, an odometry system using wheel rotation encoders, and 180 degree forward-facing laser rangefinder used for obstacle avoidance and as a fiducial detector/reader.  Each puck is marked with a fiducial that marks the puck type and each robot is equipped with a fiducial that marks the active foraging state of the robot.  Note that the fiducials do not contain unique identities of the pucks or robots but only mark the type of the puck or the puck type a given robot is engaged in foraging.  Each robot is also equipped with a 2-DOF gripper on the front, capable of picking up a single 8 cm diameter puck at a time.  There is no capability available for explicit, direct communication between robots nor can pucks and other robots be uniquely identified.

\subsection{Behavior-Based Robot Controller\label{controller}}

All robots have identical behavior-based controllers consisting of the following mutually exclusive behaviors: Avoiding, Wandering, Puck Servoing, Grasping, and Observing.  Descriptions of robot behaviors are provided below.

\begin{list}{-}{}
\item The {\bf Avoiding} behavior causes the robot to turn to avoid obstacles in its path.
\item The {\bf Wandering} behavior causes the robot to move forward and, after a random length of elapsed time, to turn left or right through a random arc for a random period of time.
\item The {\bf Puck Servoing} behavior causes the robot to move toward a detected puck of the desired type.  If the robot's current foraging state is Robot\(_{Red}\), the desired puck type is Puck\(_{Red}\), and if the robots current foraging state is Robot\(_{Green}\), the desired puck type is Puck\(_{Green}\).
\item The {\bf Grasping} behavior causes the robot to use its gripper to pick up and consume a puck within the gripper's grasp.
\item The {\bf Observing} behavior causes the robot to take the current fiducial information returned by the laser rangefinder and record the detected pucks and robots to their respective histories.  The robot then updates its foraging state based on those histories.  A description of the histories is given in \secref{state-info} and a description of the foraging state update procedure is given in \secref{transition-functions}.
\end{list}

\begin{table*}[t]
\begin{center}
\begin{tabular}{|c|c|c|c|c|}
\hline Obstacle & Puck\(_{Det}\) & Gripper Break- & Observation & Active \\
       Detected & Detected & Beam On & Signal & Behavior \\
\hline
\hline X & X & X & 1 & Observing \\
\hline 1 & X & X & X & Avoiding \\
\hline 0 & 1 & 0 & 0 & Puck Servoing \\
\hline 0 & X & 1 & 0 & Grasping \\
\hline 0 & X & X & X & Wandering \\ \hline
\end{tabular}
\end{center}
\caption{Behavior Activation Conditions.  Behaviors are listed in order of decreasing rank.  Higher ranking behaviors preempt lower ranking behaviors in the event multiple are active.  X denotes the activation condition is irrelevant for the behavior.  \label{beh-table}}
\end{table*}

Each behavior listed above has a set of activation conditions based on relevant sensor inputs and state values.  When met, the conditions cause the behavior to be become active.  A description of when each activation condition is active is given below.  The activation conditions of all behaviors are shown in Table~\ref{beh-table}.

\begin{list}{-}{}
\item The {\bf Obstacle Detected} activation condition is true when an obstacle is detected by the laser rangefinder within a distance of 1 meter.  Other robots, pucks, and the arena walls are considered obstacles.
\item The {\bf Puck\(_{Det}\) Detected} activation condition is true if the robot's current foraging state is Robot\(_{Det}\) and a puck of type Puck\(_{Det}\) (where Det is {\it Red} or {\it Green}) is detected within a distance of 5 meters and within \(\pm\) 30 degrees of the robot's direction of travel.
\item The {\bf Gripper Break-Beam On} activation condition is true if the break-beam sensor between the gripper jaws detects an object.
\item The {\bf Observation Signal} activation condition is true if the distance traveled by the robot according to odometry since the last time the {\bf Observing} behavior was activated is greater than 2 meters.
\end{list}

\subsubsection{Robot State Information \label{state-info}}

All robots maintain three types of state information: foraging state, observed puck history, and observed robot history.  The foraging state identifies the type of puck the robot is currently involved in foraging.  A robot with a foraging state of Robot\(_{Red}\) refers to a robot engaged in foraging Puck\(_{Red}\) pucks and a foraging state of Robot\(_{Green}\) refers to a robot engaged in foraging Puck\(_{Green}\) pucks.  For simplicity, we will refer to both robot foraging states and puck types as $Red$ and $Green$. The exact meaning will be clear in context.

Each robot is outfitted with a colored beacon passively observable by nearby robots which indicates the robot's current foraging state.  The color of the beacon changes to reflect the current state -- a red beacon for a foraging state of $Red$ and a green beacon for foraging state $Green$.  Thus, the colored beacon acts as a form of local, passive communication conveying the robot's current foraging state.  All robots maintain a limited, constant-sized history storing the most recently observed puck types and another constant-sized history storing the foraging state of the most recently observed robots.  Neither of these histories contains a unique identity or location of detected pucks or robots, nor does it store a time stamp of when any given observation was made.  The history of observed pucks is limited to the last {\tt MAX-PUCK-HISTORY} pucks observed and the history of the foraging states of observed robots is limited to the last {\tt MAX-ROBOT-HISTORY} robots observed.

While moving about the arena, each robot keeps track of the approximate distance it has traveled by using odometry measurements.  At every interval of 2 meters traveled, the robot makes an observation performed by the {\it Observing} behavior.  This procedure is nearly instantaneous; therefore, the robot's behavior is not outwardly affected.  The area in which pucks and other robots are visible is within 5 meters and \(\pm\) 30 degrees in the robot's direction of travel.  Observations are only made after traveling 2 meters because updating too frequently leads to over-convergence of the estimated puck and robot type proportions due to repeated observations of the same pucks and/or robots.  On average, during our experiments, a robot detected 2 pucks and robots  per observation.

\subsubsection{Foraging State Transition Function \label{transition-functions}}

After a robot makes an observation, it re-evaluates and probabilistically changes its current foraging state given the newly updated puck and robot histories. The probability by which the robot changes its foraging state is defined by the transition function.  We experimentally studied transition functions given by \eqref{eq:rates}, \eqref{eq:fR} and \eqref{eq:fG} with both $power$ and $linear$ forms. Below we present results of analysis and simulations and discuss the consequences the choice of the transition function has on system level behavior.
