\section{A Deadlock Avoidance Algorithm for ATS}
\label{sec:algorithm}
This section describes basic elements of the modelled algorithm, which was defined in our previous works~\cite{nfm14,mazzanti2014deadlock}.
Fig.~\ref{fig:layout} shows the structure of the railway layout considered in this study. Nodes in the yard correspond to itinerary endpoints, and the connecting lines correspond to the entry/exit itineraries to/from those endpoints. Eight trains are placed in the layout. Each train has its own mission to execute, defined as a sequence of itinerary endpoints.
For example, the mission of \texttt{train0}, which traverses the layout from left to right along top side of the yard, is defined by the mission vector: $T_0=[1,9,10,$ $13,15,20,23]$ (the numbers in the vector refer to the sequence of traversed endpoints in the diagram of Fig.~\ref{fig:layout}).
%The mission of train1, which also t verses the layout from left to right, is defined by the vector: T1=[3,9,10,13,15,20,24]. 
The mission of \texttt{train7}, which instead traverses the layout from right to left, is defined by the vector: $T_7=[26,22,17,18,12,27, 8]$.
The progress status of each train is represented by the index, pointing to a position in the mission vector, which allows the identification of the endpoint in which the train is at a certain moment.
We will have 8 variables $P_0, \ldots, P_7$, one for each train, which store the current index for the train. For example, at the beginning, we have
$P_0 = 0, \ldots, P_7 = 0$, since all the trains occupy the initial endpoints of their missions -- at index $0$ in the vector.
% The status of progresses for all the 8 trains can be represented by a vector P of 8 positions, where P[i] denoted the current progress of train i.
% The initial value for such vector is P=[0,0,0,0,0,0,0,0], i.e. all trains are in the first endpoint of their mission. 
%E.g., train0 is in endpoint 1 (T0[P0]) and train1 is in endpoint 3 (T1([P1]). 

\begin{figure*}[!htp]\vspace{-3 mm}
\centering
\includegraphics[width=1.00\textwidth]{img/Figure1.eps}
\caption{A fragment of the yard layout and the 8 missions of the trains}
\label{fig:layout}
\vspace{-5 mm}
\end{figure*}

If the 8 trains are allowed to move freely, i.e., if their next endpoint is free, there is the possibility of creating deadlocks, i.e., a situation in which the 8 trains block each other in their expected progression. To solve this problem the scheduling algorithm of the ATS must take into consideration two \textit{critical sections} A and B -- i.e., zones of the layout in which a deadlock might occur -- which have the form of a ring of length 8 (see Fig.~\ref{fig:criticalregion}), and guarantee that these rings are never saturated with 8 trains -- further information on how critical sections are identified can be found in our previous work~\cite{mazzanti2014deadlock}. %, because these are precisely the sources of possible deadlocks. 
This can be modelled by using two global counters $RA$ and $RB$, which record the current number of trains inside these critical sections,
and by updating them whenever a train enters or exits these sections. For this purpose, each train mission $T_i$, with $i = 0 \ldots \text{MISSION\_LEN}$ (in our case MISSION\_LEN = 7)
, is associated with: a vector of increments/decrements $A_i$ to be applied to counter $RA$ at each step of progression; a vector $B_i$ of increments/decrements to be applied to counter $RB$.

For example, given $T_0=[1,9,10,13,15,20,23]$, and $A_0=[0, 0, 0, 1, 0,-1, 0]$, when \texttt{train0} moves from endpoint 10 to endpoint 13 ($P_0 = 3$) we must check that the +1 increment of $RA$ does not saturate the critical section A, i.e.,  $RA + A_0[P_0] \leq LA$ (in our case, $LA$ = 7); if the check passes then the train can proceed and safely update the counter $RA := RA + A_0[P_0]$. The maximum number of trains allowed in each critical section (i.e., $7$), will be expressed as $LA$ and $LB$ in the following. 

\begin{figure*}[!htp]\vspace{-3 mm}
\centering
\includegraphics[width=1.00\textwidth]{img/Figure2.eps}
\caption{\label{fig:criticalregion} The critical section A and B which must not be saturated by 8 trains}
\vspace{-5 mm}
\end{figure*}

The models presented in Appendix A, which implement the algorithm described above, are deadlock-free, since the verification is being carried on as a final validation of a correct design. The actual possibility of having deadlocks, if the critical sections management were not supported or incorrectly implemented, can easily be observed by raising from 7 to 8 the values of the variables $LA$ or $LB$. 
%, i.e. allowing any train to freely advance if just its next endpoint is not occupied.

The case study presented here is actually a fragment of the complete TRACE-IT case study. In the original model the railway layout is much larger and the trains continually repeat cycling round missions. In that configuration further deadlocks situations may occur and further critical sections have to be defined and managed. The model considered in this case study represents just one of the three fragments in which the complete TRACE-IT layout has been decomposed to render the complexity of the problem amenable for formal verification. This is a typical procedure in the verification of real-world railway problems~\cite{winter2003modelling}.

The current design, in which each system state logically corresponds to a set of train progresses and each train movement logically corresponds to an atomic system evolution step, leads to a state-space of 1,636,535 configurations. This data is useful because it allows the user to cross-check the correctness of the encoding of this logical design in the various frameworks.
