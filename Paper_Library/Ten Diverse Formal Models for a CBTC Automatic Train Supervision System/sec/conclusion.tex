\section{Conclusion}
\label{sec:conclusion}
The availability of CBTC systems relies on the existence of smart ATS systems that prevent the occurrence of deadlock situations in the metro network. 
In this paper, we present different models of a scheduling algorithm for an ATS, which was designed and verified to avoid deadlocks. 
Ten different formal frameworks are used, and different variants of system design structure are presented, according to the 
features made available by the frameworks. Differences in terms of language style, allowed data types, and treatments of the system evolution are observed, 
based on the developed models. In our future work, we plan to adapt our design to tools for model-based development such as Simulink/Stateflow, 
and SCADE, to explore their potential in terms of modelling styles and verification capabilities, and compare them with the other frameworks. 
Furthermore, in the context of the EU ASTRail project\footnote{\url{http://www.astrail.eu}} we are involved in a comparative analysis of formal and semi-formal 
tools in the railway domain. The experience gained with the different frameworks will be applied to provide diverse models for ERTMS/ETCS (European Rail Traffic Management System/European Train Control System) Level 3,
the next evolution of ERTMS/ETCS. This will allow us to further stress the capability of the frameworks with a different design, including time and probabilistic aspects. It shall be noticed that, in the current work, we did not discuss aspects related to the usability of the various frameworks. This issue is of paramount importance, as highlighted, among others, by Sirjani~\cite{rebeca}, and is also going to be considered in the context of the ASTRail project.  

\smallskip
\noindent
\textbf{Acknowledgements} 
This work has been partially funded by the ASTRail project. This project received funding from the Shift2Rail Joint Undertaking under the European Union’s Horizon 2020 research and innovation programme under grant agreement No 777561. The content of this paper reflects only the authors’ view and the Shift2Rail Joint Undertaking is not responsible for any use that may be made of the included information.