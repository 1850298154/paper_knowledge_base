\section{Introduction}
\label{sec:intro}
Communications-based Train Control (CBTC) systems are the \textit{de-facto} standard for metro signalling and control, 
including several interacting wayside and onboard components that ensure safety and availability of trains within the metro network. 
In the context of a technology transfer project named TRACE-IT, the authors of the current paper, together with representatives of a large railway company, designed 
one of the main components of a CBTC system prototype, namely the Automatic Train Supervision (ATS) system~\cite{ferrari2014commercial}. 
This is a wayside system that dispatches and monitor trains along the metro network, according to a set of predefined missions. 
The ATS includes a scheduling kernel, which shall ensure that, regardless of train delays, no deadlock situation occurs, i.e., the missions
are designed in such a way that it never happens that two or more trains block each other from completing their missions. 
In the context of the project, we applied formal methods to design and verify a scheduling algorithm that addresses the deadlock avoidance 
problem~\cite{mazzanti2014deadlock}. 
The application of the algorithm to the TRACE-IT case study was initially modelled and verified by means of the UMC tool~\cite{ter2011state,kandi,umcsite}. 
Then, the design of the case study was replicated with other six different formal frameworks -- i.e., SPIN~\cite{spin,spinsite}, NuSMV/nuXmv~\cite{nuxmv,smvsite}, mCRL2~\cite{mcrl2,mcrl2site}, CPN Tools~\cite{cpn,cpnsite}, FDR4~\cite{fdr3,fdrsite} and CADP~\cite{cadp,cadpsite} --  to explore the potential of  formal methods diversity~\cite{mazzanti2018towards}. 
This is the usage of different formal tools to validate the same design, to increase the confidence on the verification results~\cite{kuismin}. 
In the current paper, we present the models discussed in~\cite{mazzanti2018towards}, 
focusing on the differences between the modelling languages, rather than on formal verification diversity. Furthermore, we provide three additional models, using TLA+~\cite{tla,tlasite}, ProB~\cite{eventb,probsite} and UPPAAL~\cite{uppaalsmc,uppaalsite}. Within the context of this paper, our goal is to provide some feedback on the differences and traps that should be tackled when changing the reference frameworks, and the commonalities that would allow a simple translation from one framework to another. 
The models are made available in Appendix A and in attachment to this paper.

The remainder of the paper is structured as follows. In Sect.~\ref{sec:algorithm} we provide an overview of the modelled algorithm. 
In Sect.~\ref{sec:comparison} we present the different models, discussing commonalities and differences with a focus on syntactic and semantics discrepancies. 
Sect.~\ref{sec:conclusion} concludes the paper. In Appendix A, we report the different models presented. 
