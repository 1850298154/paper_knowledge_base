In this work, we introduced \href{https://github.com/proroklab/VectorizedMultiAgentParticleSimulator}{VMAS}, an open-source vectorized simulator for multi-robot learning. VMAS uses PyTorch and is composed of a core vectorized 2D physics simulator and a set of multi-robot scenarios, which encode hard collective robotic tasks. The focus of this framework is to act as a platform for MARL benchmarking. Therefore, to incentivize contributions from the community, we made implementing new scenarios as simple and modular as possible. We showed the computational benefits of vectorization with up to 30,000 parallel simulations executed in under 10s on a GPU. We benchmarked the performance of MARL algorithms on our scenarios. During our training experiments, we were able to collect 60,000 environment steps and perform a training iteration in under 25s. Experiments also showed how VMAS scenarios prove difficult in orthogonal ways for state-of-the-art MARL algorithms. In the future, we plan to extend the features of VMAS to widen its adoption, continuing to implement new scenarios and benchmarks. We are also interested in modularizing the physics engine, enabling users to swap vectorized engines with different fidelities and computational demands.