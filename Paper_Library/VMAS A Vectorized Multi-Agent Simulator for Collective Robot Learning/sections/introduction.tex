%\begin{figure}[t]
%\centering
%\sidecaption[t]
%\includegraphics[width=0.64\textwidth]{author/figures/vmaps_hero_2.pdf}
%\caption{A suite of multi-robot scenarios included in VMAS. The simulator is vectorized, enabling multiple simulation instances to run in parallel.}
%\label{fig:vmaps_hero}
%\end{figure}



Many real-world problems require coordination of multiple robots to be solved. However, coordination problems are commonly computationally hard. Examples include path-planning~\cite{li2020graph}, task assignment~\cite{prorok2020robust}, and area coverage~\cite{zheng2010multirobot}. While exact solutions exist, their complexity grows exponentially in the number of robots~\cite{bernstein2002complexity}. Metaheuristics~\cite{braysy2005vehicle} provide a fast and scalable solutions, but lack in optimality. Multi-Agent Reinforcement Learning (MARL) can be used as a scalable approach to find near-optimal solutions to these problems~\cite{wang2020mobile}. In MARL, agents trained in simulation collect experiences by interacting with the environment, and train their policies (typically represented with deep neural networks) through a reward signal.

However, current MARL approaches present several issues. Firstly, the training phase can require significant time to converge to optimal behavior. This is partially due to the sample efficiency of the algorithm, and partially to the computational complexity of the simulator. Secondly, current benchmarks are specific to a predefined task and mostly tackle unrealistic videogame-like scenarios~\cite{samvelyan19smac,suarez2019neural}, far from real-world multi-robot problems. This makes research in this area fragmented, with a new simulation framework being implemented for each new task introduced. Multi-robot simulators, on the other hand, prove to be more general, but their high fidelity and full-stack simulation results in slow performance, preventing their applicability to MARL. Full-stack learning can significantly hinder training performance. Learning can be made more sample-efficient if simulation is used to solve high-level multi-robot coordination problems, while leaving low-level robotic control to first-principles-based methods.

Motivated by these reasons, we introduce VMAS, a vectorized multi-agent simulator. VMAS is a vectorized 2D physics simulator written in PyTorch~\cite{paszke2019pytorch}, designed for efficient MARL benchmarking. It simulates agents and landmarks of different shapes and supports torque, elastic collisions and custom gravity. Holonomic motion models are used for the agents to simplify simulation. Vectorization in PyTorch allows VMAS to perform simulations in a batch, seamlessly scaling to tens of thousands of parallel environments on accelerated hardware. With the term \textit{GPU vectorization} we refer to the Single Instruction Multiple Data (SIMD) execution paradigm available inside a GPU warp. This paradigm permits to execute the same instruction on a set of parallel simulations in a batch. VMAS has an interface compatible with OpenAI Gym~\cite{brockman2016openai} and with the RLlib library~\cite{liang2018rllib}, enabling out-of-the-box integration with a wide range of RL algorithms. VMAS also provides a framework to easily implement custom multi-robot scenarios. Using this framework, we introduce a set of 12 multi-robot scenarios representing difficult learning problems. Additional scenarios can be implemented through a simple and modular interface. We vectorize and port all scenarios from OpenAI MPE~\cite{lowe2017multi} in VMAS. We benchmark four of VMAS's new scenarios using three MARL algorithms based on Proximal Policy Optimization (PPO)~\cite{schulman2017proximal}. We show the benefits of vectorization by benchmarking our scenarios in the RLlib~\cite{liang2018rllib} library. Our scenarios prove to challenge state-of-the-art MARL algorithms in complementary ways.

\textbf{Contributions}. We now list the main contributions of this work:
\begin{itemize}
    \item We introduce the \textit{VMAS framework}. A vectorized multi-agent simulator which enables MARL training at scale. VMAS supports inter-agent communication and customizable sensors, such as LIDARs.
    \item We implement a set of \textit{twelve multi-robot scenarios} in VMAS, which focus on testing different collective learning challenges including: behavioural heterogeneity, coordination through communication, and adversarial interaction. 
    \item We port and vectorize all scenarios from OpenAI MPE~\cite{lowe2017multi} into VMAS and run a performance comparison between the two simulators. We demonstrate the benefits of vectorization in terms of simulation speed, showing that VMAS is up to 100$\times$ faster than MPE.
\end{itemize}
\noindent The VMAS codebase is available \href{https://github.com/proroklab/VectorizedMultiAgentParticleSimulator}{here}\footnote{\url{https://github.com/proroklab/VectorizedMultiAgentSimulator}\label{foot:vmas_url}}.