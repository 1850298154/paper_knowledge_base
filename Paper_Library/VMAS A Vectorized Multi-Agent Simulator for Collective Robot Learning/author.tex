%%%%%%%%%%%%%%%%%%%% author.tex %%%%%%%%%%%%%%%%%%%%%%%%%%%%%%%%%%%
%
% sample root file for your "contribution" to a proceedings volume
%
% Use this file as a template for your own input.
%
%%%%%%%%%%%%%%%% Springer %%%%%%%%%%%%%%%%%%%%%%%%%%%%%%%%%%


\documentclass{svproc}
%
% RECOMMENDED %%%%%%%%%%%%%%%%%%%%%%%%%%%%%%%%%%%%%%%%%%%%%%%%%%%
%

% to typeset URLs, URIs, and DOIs
\usepackage{url}
\def\UrlFont{\rmfamily}
\usepackage{newtxtext}       % 
\usepackage[varvw]{newtxmath}       % selects Times Roman as basic font

\usepackage[font=small,labelfont=bf]{caption}
\usepackage{subcaption}

%\hypersetup{hidelinks}
\usepackage[export]{adjustbox}
\usepackage{booktabs}
\usepackage[utf8]{inputenc}
\usepackage{pgfplots}
\usepackage{tikz,tikzscale}
\usepackage{adjustbox}
\usepackage{svg}
\usepackage{pifont}% http://ctan.org/pkg/pifont
\usepackage{footmisc}
\usepackage{hyperref}

\newcommand{\R}{\mathbb{R}}

\newcommand{\fixit}[1]{\textcolor{red}{\textbf{#1}}}
\newcommand{\removable}[1]{\textcolor{blue}{\textbf{#1}}}
\newcommand{\matteo}[1]{\sethlcolor{green}\hl{[Matteo: #1]}}
\newcommand{\jan}[1]{\textcolor{olive}{#1}}
\newcommand{\amanda}[1]{\textcolor{blue}{\textbf{#1}}}

\DeclareUnicodeCharacter{2212}{−}
\usepgfplotslibrary{groupplots,dateplot}
\usetikzlibrary{patterns,shapes.arrows}
\pgfplotsset{compat=newest}
\newcommand{\lilskip}[0]{\vspace{0.15cm}}
\newcommand{\mylittleskip}{\vspace{0.2cm}}
\newcommand{\cmark}{\ding{51}}%
\newcommand{\xmark}{\ding{55}}%

%\linespread{1.0}

\renewcommand{\tableautorefname}{Tab.}
\renewcommand{\equationautorefname}{Eq.}
\renewcommand{\figureautorefname}{Fig.}
\renewcommand{\sectionautorefname}{Sec.}
\renewcommand{\subsectionautorefname}{Sec.}


\begin{document}
\mainmatter              % start of a contribution
%
\title{VMAS: A Vectorized Multi-Agent Simulator for Collective Robot Learning}
%
\titlerunning{VMAS: A Vectorized Multi-Agent Simulator for Collective Robot Learning}  % abbreviated title (for running head)
%                                     also used for the TOC unless
%                                     \toctitle is used
%
\author{Matteo Bettini \and Ryan Kortvelesy \and Jan Blumenkamp \and Amanda Prorok}
%
\authorrunning{Matteo Bettini et al.} % abbreviated author list (for running head)
%
%%%% list of authors for the TOC (use if author list has to be modified)
\tocauthor{Matteo Bettini, Ryan Kortvelesy, Jan Blumenkamp and Amanda Prorok}
%
\institute{Department of
Computer Science and Technology, University of Cambridge, Cambridge, UK, \email{\{mb2389,rk627,jb2270,asp45\}@cl.cam.ac.uk}}

\maketitle              % typeset the title of the contribution

\begin{abstract}
While many multi-robot coordination problems can be solved optimally by exact algorithms, solutions are often not scalable in the number of robots. Multi-Agent Reinforcement Learning (MARL) is gaining increasing attention in the robotics community as a promising solution to tackle such problems. Nevertheless, we still lack the tools that allow us to \textit{quickly} and \textit{efficiently} find solutions to large-scale collective learning tasks. In this work, we introduce the Vectorized Multi-Agent Simulator (VMAS). VMAS is an open-source framework designed for efficient MARL benchmarking. It is comprised of a vectorized 2D physics engine written in PyTorch and a set of twelve challenging multi-robot scenarios. Additional scenarios can be implemented through a simple and modular interface. We demonstrate how vectorization enables parallel simulation on accelerated hardware without added complexity. When comparing VMAS to OpenAI MPE, we show how MPE's execution time increases linearly in the number of simulations while VMAS is able to execute 30,000 parallel simulations in under 10s, proving more than 100$\times$ faster. Using VMAS's RLlib interface, we benchmark our multi-robot scenarios using various Proximal Policy Optimization (PPO)-based MARL algorithms. VMAS's scenarios prove challenging in orthogonal ways for state-of-the-art MARL algorithms. The VMAS framework is available at: \url{https://github.com/proroklab/VectorizedMultiAgentSimulator}. A video of VMAS scenarios and experiments is available \href{https://youtu.be/aaDRYfiesAY}{here}\footnote{\url{https://youtu.be/aaDRYfiesAY} \label{foot:video}}
\keywords{simulator, multi-robot learning, vectorization}
\end{abstract}

\begin{figure}[ht]
    \newcommand{\subfigsize}{0.25}
     \centering
       \begin{subfigure}{\subfigsize\linewidth}
         \centering
         \includegraphics[width=\linewidth,frame]{figures/scenarios/transport.pdf}
         \caption{Transport}
         \label{fig:transport}
     \end{subfigure}%
          \begin{subfigure}{\subfigsize\linewidth}
         \centering
         \includegraphics[width=\linewidth,frame]{figures/scenarios/wheel.pdf}
         \caption{Wheel}
         \label{fig:wheel}
     \end{subfigure}%
     \begin{subfigure}{\subfigsize\linewidth}
         \centering
         \includegraphics[width=\linewidth,frame]{figures/scenarios/balance.pdf}
         \caption{Balance}
         \label{fig:balance}
     \end{subfigure}%
       \begin{subfigure}{\subfigsize\linewidth}
         \centering
         \includegraphics[width=\linewidth,frame]{figures/scenarios/give_way.pdf}
         \caption{Give way}
         \label{fig:give_way}
     \end{subfigure}
    \begin{subfigure}{\subfigsize\linewidth}
         \centering
         \includegraphics[width=\linewidth,frame]{figures/scenarios/football.pdf}
         \caption{Football}
         \label{fig:football}
     \end{subfigure}%
     \begin{subfigure}{\subfigsize\linewidth}
         \centering
         \includegraphics[width=\linewidth,frame]{figures/scenarios/passage.pdf}
         \caption{Passage}
         \label{fig:passage}
     \end{subfigure}%
     \begin{subfigure}{\subfigsize\linewidth}
         \centering
         \includegraphics[width=\linewidth,frame]{figures/scenarios/reverse_transport.pdf}
         \caption{Reverse Transport}
         \label{fig:reverse_transport}
     \end{subfigure}%
     \begin{subfigure}{\subfigsize\linewidth}
         \centering
         \includegraphics[width=\linewidth,frame]{figures/scenarios/dispersion.pdf}
         \caption{Dispersion}
         \label{fig:dispersion}
     \end{subfigure}
    
     \begin{subfigure}{\subfigsize\linewidth}
         \centering
         \includegraphics[width=\linewidth,frame]{figures/scenarios/dropout.pdf}
         \caption{Dropout}
         \label{fig:dropout}
     \end{subfigure}%
     \begin{subfigure}{\subfigsize\linewidth}
         \centering
         \includegraphics[width=\linewidth,frame]{figures/scenarios/flocking.pdf}
         \caption{Flocking}
         \label{fig:flocking}
     \end{subfigure}%
     \begin{subfigure}{\subfigsize\linewidth}
         \centering
         \includegraphics[width=\linewidth,frame]{figures/scenarios/discovery.pdf}
         \caption{Discovery}
         \label{fig:discovery}
     \end{subfigure}%
     \begin{subfigure}{\subfigsize\linewidth}
         \centering
         \includegraphics[width=\linewidth,frame]{figures/scenarios/waterfall.pdf}
         \caption{Waterfall}
         \label{fig:waterfall}
     \end{subfigure}%
    \caption{Multi-robot scenarios introduced in VMAS. Robots (blue shapes) interact among each other and with landmarks (green, red, and black shapes) to solve a task.}
    \label{fig:vmaps_scenarios}
\end{figure}

\section{Introduction}
\label{sec:introduction}


\IEEEPARstart{T}{wo} %
main challenges in the deployment of large-scale swarms are the localization and coordination of vehicles.
Localization methods that rely on external infrastructure 
(e.g., GPS) 
are prone to systematic errors (e.g., multipath effect)
and may not always be available.
Coordination strategies that are centralized can deconflict motion plans to prevent collisions and gridlock, but introduce a single point of failure and are difficult to scale in swarm size due to communication bandwidth limitations.

This paper presents a unified formation flying pipeline for unmanned aerial vehicles (UAVs).
Our pipeline uses \textit{onboard} sensors for localization, which eliminate the need for external positioning systems, and \textit{distributed} techniques for coordination, which enable each vehicle to make decisions independently while communicating their state to a subset of the team.
For \textit{localization}, we use an off-the-shelf commercial visual inertial odometry (VIO) package \cite{VIO}
that fuses inertial measurement unit (IMU) and downward-facing monocular camera measurements to estimate changes in the vehicle pose.
\edit{For \textit{coordination}, we present distributed formation control and task assignment strategies that run onboard the vehicles, do not rely on a common reference frame, and use vehicle-to-vehicle communication.} 
Key features of our formation control strategy include scalability to a large number of vehicles and robustness to disturbances.
The latter is crucial for reaching the desired formations with sensing imperfections.
Our task assignment strategy uses an auction-based algorithm to guarantee conflict-free assignments.
This algorithm can deconflict vehicle gridlocks resulting from distributed collision avoidance (type 3 deadlock~\cite{Wang2017}) and is well-suited for vehicles with limited computational capability and low-bandwidth communication. 


\begin{figure}[t!]
	\begin{center}
		\includegraphics[trim =0mm 10mm 0mm 0mm, clip, width=\columnwidth]{Figs/slanted_plane.png}	
		\caption{
		Six multirotors in a slanted plane formation.
		Vehicles communicate with each other, make distributed decisions onboard, and use VIO for localization.}
		\label{fig:slantedplane}
	\end{center}
\end{figure}


\subsection{Contributions}

This research extends our previous work on UAV formations~\cite{Fathian2019} and presents a unified pipeline consisting of \textit{onboard localization} and \textit{distributed coordination}.
The three main contributions of this work are:
\begin{enumerate}
    \item \edit{scalable formulation of control design suitable for
    onboard sensing without a common reference frame;}
    \item algorithms for deconfliction via \edit{distributed} task assignment of vehicles to desired formation points;    
    \item simulation- and hardware-ready open-source pipeline.
\end{enumerate}
\edit{Our pipeline is tested in hardware with six multirotors (see Fig.~\ref{fig:slantedplane}), and 
to our knowledge is the first demonstration of formation flying that does not rely on external sensing, fiducial markers for localization, a common reference frame, or a centralized base station for coordination.}
The only requirements for the presented pipeline are that the vehicles can communicate, can find the transformation between their VIO start frames, and the environment is sufficiently textured---a standard assumption for VIO systems.
As such, this framework paves the way for future, real-world deployments of aerial vehicle swarms in large numbers and without requiring external localization infrastructure.


\begin{figure} [t!]
\centering
	\begin{subfigure}[b]{0.32\columnwidth}
	   %
	    \includegraphics[width=0.8\textwidth,left]{Figs/Frames2_full.pdf}
	    \caption{\scriptsize full alignment}
	    \label{fig:frame-a}
	\end{subfigure}
	\begin{subfigure}[b]{0.32\columnwidth}
	    \includegraphics[width=0.8\textwidth,center]{Figs/Frames2_orientation.pdf}
	    \caption{\scriptsize orientation alignment}
	    \label{fig:frame-b}
	\end{subfigure}
	\begin{subfigure}[b]{0.32\columnwidth}
	    \includegraphics[width=0.8\textwidth,right]{Figs/Frames2_none.pdf}
	    \caption{\scriptsize no alignment}
        \label{fig:frame-c}
	\end{subfigure}
\caption{\edit{Required alignment of UAV frames in existing swarm strategies: (a) the most restrictive case requiring a common reference frame, i.e., orientation and origin of the frames must be aligned; (b) only the orientation of the frames must be aligned; (c) no alignment restrictions (this work).}}
	\label{fig:Frames}
\end{figure}




\subsection{Related Work}

Existing aerial swarms can be grouped based on the coordination (centralized vs.\ distributed) and localization (external vs.\ onboard) methods used. 
\edit{It is further crucial to distinguish these methods based on the level of alignment required for the vehicle coordinate frames; see Fig.~\ref{fig:Frames}.} 
 
\edit{
Works with \textit{centralized} coordination and \textit{external} localization include~\cite{Preiss2017, Honig2018, Du2019}, which are based on lightweight UAVs with limited onboard computational capability and therefore rely on an external motion capture system and a base station.
Works with \textit{distributed} coordination and \textit{external} localization include \cite{wilson2020robotarium}, \cite{enright2004spheres}, where robots execute distributed controls  based on external localization by motion capture and ultrasonic beacons, respectively.
Works with \textit{centralized} coordination and \textit{onboard} localization include~\cite{Forster2013}, \cite{Loianno2016}, which use a ground station for task assignment among vehicles.
In \cite{Weinstein2018}, formation flying based on VIO is demonstrated, where motion planning and assignment are run on a base station to ensure collision-free trajectories.
The coordination strategies used in aforementioned works require a \textit{common reference frame} (Fig.~\ref{fig:frame-a}).
}


\edit{
Despite the large body of work on formation control~\cite{Oh2015}, and the variety of onboard sensing solutions for localization (e.g., VIO~\cite{Delmerico2018}), few frameworks demonstrated formation flying with \textit{distributed} coordination and \textit{onboard} localization.
A key reason is reliance of many distributed control and assignment algorithms on aligned frames (Fig.~\ref{fig:frame-a}, \ref{fig:frame-b}), which require computation-expensive and/or communication-intensive synchronization/consensus steps for frame alignment.
Equally important, dependence on alignment in existing methods \cite{Wang2017,Turpin2014, van2011reciprocal, morgan2016swarm} diminishes robustness to inherent noise and unobservable errors that cannot be corrected (e.g., disparities between the actual and estimated body frame \textit{orientation} caused by VIO drift).
Leveraging coordination methods that are \textit{robust to misaligned frames} is hence crucial and a focus of this work. 
}






\edit{
Examples of other pipelines with distributed coordination and onboard localization include \cite{Montijano2016,Tron2016}.
Both works demonstrated formation flying on three UAVs, required information from an external motion capture system due to hardware limitations, did not incorporate collision avoidance, and required frame alignment.
}
\edittwo{Note that while~\cite{Montijano2016,Tron2016} can achieve formations with arbitrary headings as illustrated in Fig.~\ref{fig:frame-c}, knowledge of relative orientations is still required; therefore, they belong to the category of Fig.~\ref{fig:frame-b}.}






\if 0

\r{
decentralized coordination setting combined with VIO:
D-CAPT [26]~\cite{}:
ORCA ~\cite{}: 
CBF [2]~\cite{} :
[A]
}

\r{Robusteness in coordination,  with compounded noise/latency, which would eventually break (b).\\


some existing algorithm might as well
work in a similar fully decentralized setting, when combined with VIO
as proposed here. For example, D-CAPT [26], ORCA, CBF [2] might also be
useful for such a task and are computationally even more efficient than
the proposed approach. \\

R2:  onboard sensing for localization ->
 Finally, the related work section only
focuses on this aspect of the pipeline, discussing how many formation papers include
onboard localization but barely sells the advantages of the coordination module (the actual
proposal of the paper) against other competitors such as [26] or [A] or to mention similar
coordination pipelines. \\


Given a solution to this problem, the controller in Section III seems unnecessary, each drone
has a target position and can use a local controller with collision avoidance that drives it to
that position. Note that such controllers exists in the literature (e.g., RVO in any of its
multi-agent variantes), they are distributed in nature and only require local sensing.


}

\fi
\section{Related work}
\label{sec:related_work}

In this section, we review the related literature in the fields of multi-agent and multi-robot simulation, highlighting the core gaps of each field. Furthermore, we compare the most relevant simulation frameworks with VMAS in \autoref{tab:simulator_comparison}. 

\textbf{Multi-agent reinforcement learning environments}. A significant amount of work exists in the context of MARL to address the issues of multi-robot simulation for learning hard coordination strategies. Realistic GPU-accelerated simulators and engines have been proposed. Isaac~\cite{makoviychuk2021isaac} is a proprietary NVIDIA simulator used for realistic robotic simulation in reinforcement learning. Instead of using environment vectorization to accelerate learning, it uses concurrent execution of multiple training environments in the same simulation instance. Despite of this, its high-fidelity simulation makes it computationally expensive for high-level MARL problems. Brax~\cite{brax2021github} is a vectorized 3D physics engine introduced by Google. It uses the Jax~\cite{jax2018github} library to achieve environment batching and full-differentiability. However, computational issues occur when scaling the number of simulated agents, leading to stalled environments with just 20 agents. There also exist projects for single-agent vectorized environments~\cite{gymnax2022github,envpool}, but the complexity of extending these to the multi-agent domain is non-trivial.

The core benchmark environments of the MARL literature focus on high-level inter-robot learning. Multiagent Particle Environments (MPE)~\cite{lowe2017multi} are a set of enviroments created by OpenAI. They share VMAS's principles of modularity and ease of new scenario creation, without providing environment vectorization. \text{MAgent}~\cite{zheng2018magent} is a discrete-world environment supporting a high number of agents. Multi-Agent-Learning-Environments~\cite{jiang2021multi} is another simplified discrete-world set of environments with a range of different multi-robot tasks. Multi-Agent-Emergence-Environments~\cite{baker2019emergent} is a customizable OpenAI 3D simulator for hide-and-seek style games. Pommerman~\cite{DBLP:journals/corr/abs-1809-07124} is a discretized playground for learning multi-agent competitive strategies.
SMAC~\cite{samvelyan19smac} is a very popular MARL benchmark based on the Starcraft 2 videogame. Neural-MMO~\cite{suarez2019neural} is another videogame-like set of environments where agents learn to survive in large populations. Google Research Football~\cite{kurach2020google} is a football simulation with a suite of scenarios that test different aspects of the game.
Gym-pybullet-drones~\cite{panerati2021learning} is a realistic PyBullet simulator for multi-quadricopters control. Particle Robots Simulator~\cite{shen2022deep} is a simulator for particle robots, which require high coordination strategies to overcome actuation limitations and achieve high-level tasks. Multi-Agent Mujoco~\cite{peng2021facmac} consists in multiple agents controlling different body parts of a single Mujoco~\cite{todorov2012mujoco} agent. While all these environments provide interesting MARL benchmarks, most of them focus on specific tasks. Furthermore, none of these environments provide GPU vectorization, which is key for efficient MARL training. We present a comparison between VMAS and all the aforementioned environments in \autoref{tab:simulator_comparison}.

\textbf{Multi-robot simulators}. Video-game physics engines such as Unity and Unreal Engine grant realistic simulation that can be leveraged for multi-agent robotics. Both make use of the GPU-accelerated NVIDIA PhysX. However, their generality causes high overheads when using them for robotics research.
% their use in research environments comes with a high overhead due to their generality---the tools themselves are not tailored towards robotics research. Furthermore, they incorporate complex simulations which require significant computational resources.
Other popular physics engines are Bullet, Chipmunk, Box2D, and ODE. These engines are all similar in their capabilities and prove easier to adopt due to the availability of Python APIs. Thus, they are often the tool of choice for realistic robotic simulation. However, because they do not leverage GPU-accelerated batched simulation, these tools lead to performance bottlenecks in MARL training.

The most widely known robotic simulators are Gazebo~\cite{koenig2004design} and Webots~\cite{michel2004cyberbotics}. Their engines are based on the ODE 3D dynamics library. These simulators support a wide range of robot models, sensors, and actuators, but suffer from significant performance loss when scaling in the number of agents. Complete simulation stall is shown to occur with as few as 12 robots~\cite{8088134}. For this reason, Argos~\cite{Pinciroli:SI2012} has been proposed as a scalable multi-robot simulator. It is able to simulate swarms in the thousands of agents by assigning parts of the simulation space to different physics engines with different simulation goals and fidelity. Furthermore, it uses CPU parallelization through multi-threading. Despite these features, none of the simulators described are fast enough to be usable in MARL training. This is because they prioritize realistic full-stack multi-robot simulation over speed, and they do not leverage GPU acceleration for parallel simulations. 
This focus on realism is not always necessary in MARL. In fact, most collective coordination problems can be decoupled from low-level problems relating to sensing and control. When these problems can be efficiently solved independently without loss of generality, fast high-level simulation provides an important tool. This insight is the key factor motivating the holonomicity assumption in VMAS.



\begin{table}[t]
\caption{Comparison of multi-agent and multi-robot simulators and environments.}
\label{tab:simulator_comparison} 
\resizebox{\linewidth}{!}{%
\begin{tabular}{r c c c c c c c c c c c}
\hline\noalign{\smallskip}
 & \rotatebox{0}{Vector$^a$} & \rotatebox{0}{State$^b$} & \rotatebox{0}{Comm$^c$} & \rotatebox{0}{Action$^d$} & \rotatebox{0}{PhysEng$^e$} & \rotatebox{0}{\#Agents$^f$} & \rotatebox{0}{Gen$^g$} & \rotatebox{0}{Ext$^h$} & \rotatebox{0}{MRob$^i$} & \rotatebox{0}{MARL$^j$} & \rotatebox{0}{RLlib$^k$} \\
\toprule

Brax~\cite{brax2021github} & \cmark  & C & \xmark & C & 3D & $<10$ & \cmark & \cmark & \xmark & \xmark &\xmark\\
MPE~\cite{lowe2017multi}  & \xmark  & C & C+D & C+D & 2D & $<100$ & \cmark & \cmark & \xmark & \cmark &\cmark\\
MAgent~\cite{zheng2018magent} & \xmark  & D & \xmark & D & \xmark & $>1000$ & \xmark & \xmark & \xmark & \cmark &\cmark\\
MA-Learning-Environments~\cite{jiang2021multi} & \xmark  & D & \xmark & D & \xmark & $<10$ & \cmark & \xmark & \cmark & \cmark &\xmark\\
MA-Emergence-Environments~\cite{baker2019emergent} & \xmark  & C & \xmark & C+D & 3D & $<10$ & \xmark & \xmark & \xmark & \cmark &\xmark\\
Pommerman~\cite{DBLP:journals/corr/abs-1809-07124} & \xmark  & D & \xmark & D & \xmark & $<10$ & \xmark & \xmark & \xmark & \cmark &\xmark\\
SMAC~\cite{samvelyan19smac}  & \xmark  & C & \xmark & D & \xmark & $<100$ & \xmark & \cmark & \xmark & \cmark & \cmark\\
Neural-MMO~\cite{suarez2019neural} & \xmark  & C & \xmark & C+D & \xmark & $<1000$ & \xmark & \cmark & \xmark & \cmark & \cmark \\
Google research football~\cite{kurach2020google} & \xmark  & C & \xmark & D & 2D & $<100$ & \xmark & \cmark & \xmark & \cmark &\cmark\\
gym-pybullet-drones~\cite{panerati2021learning} & \xmark & C & \xmark & C & 3D & $<100$ & \xmark & \cmark &\cmark & \cmark &\cmark \\
Particle robots simulator~\cite{shen2022deep} & \xmark  & C & \xmark & C+D & 2D & $<100$ & \xmark & \cmark & \cmark & \cmark &\xmark\\
MAMujoco~\cite{peng2021facmac} & \xmark  & C & \xmark & C & 3D & $<10$ & \xmark & \xmark & \xmark & \cmark &\xmark\\

\midrule

Gazebo~\cite{koenig2004design} &  \xmark & C & C+D & C+D & 3D & $<10$  & \cmark & \cmark & \cmark & \xmark & \xmark\\
Webots~\cite{michel2004cyberbotics} & \xmark  & C & C+D  & C+D & 3D & $<10$  & \cmark & \cmark & \cmark &\xmark  & \xmark \\
ARGOS~\cite{Pinciroli:SI2012} & \xmark & C & C+D & C+D & 2D\&3D & $<1000$ & \cmark & \cmark &\cmark & \xmark &\xmark \\

\midrule

VMAS & \cmark  & C & C+D & C+D & 2D & $<100$ & \cmark & \cmark & \cmark & \cmark &\cmark\\

\bottomrule
\end{tabular}}\\

$^a$ Vectorized\\
$^b$ Continuous state (C) or discrete state/grid world (D)\\
$^c$ Continuous communication (C) or discrete communication (D) inside the simulator\\
$^d$ Continuous actions (C) or discrete actions (D)\\
$^e$ Type of physics engine\\
$^f$ Number of agents supported\\
$^g$ General purpose simulator: any type of task can be created\\
$^h$ Extensibility (API for creating new scenarios)\\
$^i$ Contains multi-robot tasks\\
$^j$ Made for Multi-Agent Reinforcement Learning (MARL)\\
$^k$ Compatible with RLlib framework~\cite{liang2018rllib}
\end{table}
\section{The VMAS platform}
\label{sec:simulator}
The unique characteristic that makes VMAS different from the related works compared in \autoref{tab:simulator_comparison} is the fact that our platform brings together multi-agent learning and environment vectorization. Vectorization is a key component to speed-up MARL training. In fact, an on-policy training iteration\footnote{Here we illustrate an on-policy training iteration, but simulation is a key component of any type of MARL algorithm} is comprised of simulated team rollouts and a policy update. During the rollout phase of iteration $k$, simulations are performed to collect experiences from the agents' interactions with the environment according to their policy $\pi_k$. The collected experiences are then used to update the team policy. The new policy $\pi_{k+1}$ will be employed in the rollout phase of the next training iteration. The rollout phase usually constitutes the bottleneck of this process. Vectorization allows parallel simulation and helps alleviate this issue.


Inspired by the modularity of some existing solutions, like MPE~\cite{lowe2017multi}, we created our framework as a new scalable platform for running and creating MARL benchmarks. With this goal in mind, we developed VMAS following a set of tenets:

\begin{itemize}
    \item \textbf{Vectorized}. VMAS vectorization can step any number of environments in parallel. This significantly reduces the time needed to collect rollouts for training in MARL.
    \item \textbf{Simple}. Complex vectorized physics engines exist (e.g., Brax~\cite{brax2021github}), but they do not scale efficiently when dealing with multiple agents. This defeats the computational speed goal set by vectorization. VMAS uses a simple custom 2D dynamics engine written in PyTorch to provide fast simulation. 
    \item \textbf{General}. The core of VMAS is structured so that it can be used to implement general high-level multi-robot problems in 2D. It can support adversarial as well as cooperative scenarios. Holonomic robot simulation shifts focus to high-level coordination, obviating the need to learn low-level controls using MARL.
    \item \textbf{Extensible}. VMAS is not just a simulator with a set of environments. It is a framework that can be used to create new multi-agent scenarios in a format that is usable by the whole MARL community. For this purpose, we have modularized our framework to enable new task creation and introduced interactive rendering to debug scenarios.
    \item \textbf{Compatible}. VMAS has multiple wrappers which make it directly compatible with different MARL interfaces, including RLlib~\cite{liang2018rllib} and Gym~\cite{brockman2016openai}. RLlib has a large number of already implemented RL algorithms.

\end{itemize}
Let us break down VMAS's structure in depth.

\begin{figure}[ht]
\centering
\includegraphics[width=0.85\textwidth]{figures/vmas.pdf}
\caption{VMAS structure. VMAS has a vectorized MARL interface (left) with wrappers for compatibility with OpenAI Gym~\cite{brockman2016openai} and the RLlib RL library~\cite{liang2018rllib}. The default VMAS interface uses PyTorch~\cite{paszke2019pytorch} and can be used for feeding input already on the GPU. Multi-agent tasks in VMAS are defined as scenarios (center). To define a scenario, it is sufficient to implement the listed functions. Scenarios access the VMAS core (right), where agents and landmarks are simulated in the world using a 2D custom written physics module.}
\label{fig:vmaps_structure}
\end{figure}

\textbf{Interface}. The structure of VMAS is illustrated in \autoref{fig:vmaps_structure}. It has a vectorized interface, which means that an arbitrary number of environments can be stepped in parallel in a batch. In \autoref{sec:mpe_comparison}, we demonstrate how vectorization grants important speed-ups on the CPU and seamless scaling on the GPU. While the standard simulator interface uses PyTorch~\cite{paszke2019pytorch} to enable feeding tensors directly as input/output, we provide wrappers for the standard non-vectorized OpenAI Gym~\cite{brockman2016openai} interface and for the vectorized interface of the RLlib~\cite{liang2018rllib} framework. This enables  users to effortlessly access the range of RL training algorithms already available in RLlib. Actions for all environments and agents are fed to VMAS for every simulation step. VMAS supports movement and inter-agent communication actions, both of which can be either continuous or discrete. The interface of VMAS provides rendering through Pyglet~\cite{pyglet}. 

\textbf{Scenario}. Scenarios encode the multi-agent task that the team is trying to solve. Custom scenarios can be implemented in a few hours and debugged using interactive rendering. Interactive rendering is a feature where agents in scenarios can be controlled by users in a videogame-like fashion and all environment-related data is printed on screen. To implement a scenario, it is sufficient to define a few functions: \verb|make_world| creates the agents and landmarks for the scenario and spawns them in the world, \verb|reset_world_at| resets a specific environment in the batch or all environments at the same time, \verb|reward| returns the reward for one agent for all environments, \verb|observation| returns the agent's observations for all environments. Optionally, \verb|done| and \verb|info| can be implemented to provide an ending condition and extra information. Further documentation on how to create new scenarios is available in the \href{https://github.com/proroklab/VectorizedMultiAgentParticleSimulator}{repository}\footref{foot:vmas_url} and in the code.

\textbf{Core}. Scenarios interact with the core. This is where the world simulation is stepped. The world contains $n$ entities, which can be agents or landmarks. Entities have a shape (sphere, box, or line) and a vectorized state $(\mathbf{x}_i,\dot{\mathbf{x}}_i,\theta_i,\dot{\theta}_i ),\, \forall i \in [1..n] \equiv N$, which contains their position $\mathbf{x}_i\in\R^2$, velocity $\dot{\mathbf{x}}_i\in\R^2$, rotation $\theta_i\in\R$, and angular velocity $\dot{\theta}_i \in\R$ for all environments. Entities have a mass $m_i\in\R$ and a maximum speed and can be customized to be movable, rotatable, and collidable. Agents’ actions consist of physical actions, represented as forces $\mathbf{f}^a_i \in \R^2$, and optional communication actions. In the current state of the simulator, agents cannot control their orientation. Agents can either be controlled from the interface or by an “action script” defined in the scenario. Optionally, the simulator can introduce noise to the actions and observations. Custom sensors can be added to agents. We currently support LIDARs.
The world has a simulation step $\delta t$, velocity damping coefficient $\zeta$, and customizable gravity $\mathbf{g} \in \R^2$.

VMAS has a force-based physics engine. Therefore, the simulation step uses the forces at time $t$ to integrate the state by using a semi-implicit Euler method~\cite{niiranen1999fast}:

\begin{equation}
    \begin{cases}
      \mathbf{f}_i(t) = \mathbf{f}^a_i(t) + \mathbf{f}_i^g + \sum_{j \in N \setminus \{i\}}\mathbf{f}_{ij}^e(t) \\
      \dot{\mathbf{x}}_i(t + 1) = (1-\zeta)\dot{\mathbf{x}}_i(t) + \frac{\mathbf{f}_i(t)}{m_i}\delta t\\
      \mathbf{x}_i(t + 1) = \mathbf{x}_i(t) + \dot{\mathbf{x}}_i(t + 1)\delta t 
    \end{cases}\,,
\end{equation}
where $\mathbf{f}^a_i$ is the agent action force, $\mathbf{f}_i^g = m_i\mathbf{g}$ is the force deriving from gravity and $\mathbf{f}_{ij}^e$ is the environmental force used to simulate collisions between entities $i$ and $j$. It has the following form:

\begin{equation}
\mathbf{f}^e_{ij}(t) = 
\begin{cases}
    c \frac{\mathbf{x}_{ij}(t)}{\left \| \mathbf{ x}_{ij}(t)\right \|}  k\log{\left(1 + e^{\frac{-\left(\left \| \mathbf{ x}_{ij}(t)\right \|-d_{\textrm{min}}\right)}{k}}\right )} & \quad\text{if }\left \| \mathbf{ x}_{ij}(t)\right \| \leqslant d_{\textrm{min}} \\
    0  & \quad\text{otherwise}\

    \end{cases}\, .
\end{equation}
 Here, $c$ is a parameter regulating the force intensity. $\mathbf{x}_{ij}$ is the relative position between the closest points on the shapes of the two entities. $d_{\textrm{min}}$ is the minimum distance allowable between them. The term inside the logarithm computes a scalar proportional to the penetration of the two entities, parameterized by a coefficient $k$. This term is then multiplied by the normalized relative position vector. Collision intensity and penetration can be tuned by regulating $c$ and $k$. This is the same collision system used in OpenAI MPE~\cite{lowe2017multi}. 
 
 The simulation step used for the linear state is also applied to the angular state:

\begin{equation}
    \begin{cases}
      \tau_i(t) =  \sum_{j \in N \setminus \{i\}}\left \| \mathbf{r}_{ij}(t) \times \mathbf{f}^e_{ij}(t) \right \| \\
      \dot{\theta}_i(t + 1) = (1-\zeta)\dot{\theta}_i(t) + \frac{\tau_i(t)}{I_i}\delta t\\
      \theta_i(t+1) = \theta_i(t) + \dot{\theta}_i(t+1)\delta t 
    \end{cases}\,.
\end{equation}
Here, $\mathbf{r}_{ij}\in\R^2$ is the vector from the center of the entity to the colliding point, $\tau_i$ is the torque, and $I_i$ is the moment of inertia of the entity. The rules regulating the physics simulation in the  core are basic 2D dynamics implemented in a vectorized manner using PyTorch. They simulate holonomic (unconstrained motion) entities only. 

%Given the current rate of development of vectorized physics simulators, we have also made it possible to eventually swap the core of VMAS with a more complex and realistic engine.

%Holonomic simulation has been chosen in order keep the research focus on using MARL to solve high-level NP-Hard multi-robot coordination problems. 

%The low lever control translation from holonomic to custom robot dynamics can then be tackled by using first-principles-based exact solutions and is not the focus of this project.




\section{Multi-robot scenarios}
\label{sec:scenarios}
Alongside VMAS, we introduce a set of 12 multi-robot scenarios. These scenarios contain various multi-robot problems, which require complex coordination---like leveraging heterogeneous behaviour and inter-agent communication---to be solved. While the ability to send communication actions is not used in these scenarios, communication can be used in the policy to improve performance. For example, Graph Neural Networks (GNNs) can be used to overcome partial observability through information sharing~\cite{blumenkamp2021framework}. 

Each scenario delimits the agents' input by defining the set of their observations. This set typically contains the minimum observation needed to solve the task (e.g., position, velocity, sensory input, goal position).
Scenarios can be made arbitrarily harder or easier by modifying these observations. For example, if the agents are trying to transport a package, the precise relative distance to the package can be removed from the agent inputs and replaced with LIDAR measurements. Removing global observations from a scenario is a good incentive for inter-agent communication.

All tasks contain numerous parametrizable components. 
Every scenario comes with a set of tests, which run a local heuristic on all agents. Furthermore, we vectorize and port all 9 scenarios from MPE~\cite{lowe2017multi} to VMAS. In this section, we give a brief overview of our new scenarios. For more details (e.g., observation space, reward, etc.) you can find in-depth descriptions in the \href{https://github.com/proroklab/VectorizedMultiAgentParticleSimulator}{VMAS repository}\footref{foot:vmas_url}. 

\vspace{8pt}

\noindent\textbf{Transport (\autoref{fig:transport})}. $N$ agents have to push $M$ packages to a goal. Packages have a customizable mass and shape. Single agents are not able to move a high-mass package by themselves. Cooperation with teammates is thus needed to solve the task.

\vspace{8pt}

\noindent\textbf{Wheel (\autoref{fig:wheel})}. $N$ agents have to collectively rotate a line. The line is anchored to the origin and has a parametrizable mass and length. The team's goal is to bring the line to a desired angular velocity. Lines with a high mass are impossible to push for single agents. Therefore, the team has to organize with agents on both sides to increase and reduce the line's velocity. 

\vspace{8pt}


\noindent\textbf{Balance (\autoref{fig:balance})}. $N$ agents are spawned at the bottom of a world with vertical gravity. A line is spawned on top of them. The agents have to transport a spherical package, positioned randomly on top of the line, to a given goal at the top. The package has a parametrizable mass and the line can rotate.

\vspace{8pt}

\noindent\textbf{Give Way (\autoref{fig:give_way})}. Two agents start in front of each other's goals in a symmetric environment. To solve the task, one agent has to give way to the other by using a narrow space in the middle of the environment.

\vspace{8pt}

\noindent\textbf{Football (\autoref{fig:football})}. A team of $N$ blue agents competes against a team of $M$ red agents to score a goal. By default, red agents are controlled by a heuristic AI, but self-play is also possible. Cooperation among teammates is required to coordinate attacking and defensive maneuvers. Agents need to communicate and assume different behavioural roles in order to solve the task.

\vspace{8pt}

\noindent\textbf{Passage (\autoref{fig:passage})}. 5 agents, starting in a cross formation, have to reproduce the same formation on the other side of a barrier. The barrier has $M$ passages ($M=1$ in the figure). Agents are penalized for colliding amongst each other and with the barrier. This scenario is a generalization of the one considered in~\cite{blumenkamp2021framework}.

\vspace{8pt}

\noindent\textbf{Reverse transport (\autoref{fig:reverse_transport})}. This task is the same as Transport, except only one package is present. Agents are spawned \textit{inside} of it and need to push it to the goal.

\vspace{8pt}

\noindent\textbf{Dispersion (\autoref{fig:dispersion})}. There are $N$ agents and $N$ food particles. Agents start in the same position and need to cooperatively eat all food. Most MARL algorithms cannot solve this task (without communication or observations from other agents) as they are constrained by behavioural homogeneity deriving from parameter sharing. Heterogeneous behaviour is thus needed for each agent to tackle a different food particle.

\vspace{8pt}

\noindent\textbf{Dropout (\autoref{fig:dropout})}. $N$ agents have to collectively reach one goal. To complete the task, it is enough for only one agent to reach the goal. The team receives an energy penalty proportional to the sum of all the agents' controls. Therefore, agents need to organize themselves to send only the closest robot to the goal, saving as much energy as possible. 

\vspace{8pt}

\noindent\textbf{Flocking (\autoref{fig:flocking})}. $N$ agents have to flock around a target without colliding among each other and $M$ obstacles. Flocking has been an important benchmark in multi-robot coordination for years, with first solutions simulating behaviour according to local rules~\cite{reynolds1987flocks}, and more recent work using learning-based approaches~\cite{tolstaya2020flocking}. In contrast to related work, our flocking environment contains static obstacles.

\vspace{8pt}

\noindent\textbf{Discovery (\autoref{fig:discovery})}. $N$ agents have to coordinate to cover $M$ targets as quickly as possible while avoiding collisions. A target is considered covered if $K$ agents have approached a target at a distance of at least $D$. After a target is covered, the $K$ covering agents each receive a reward and the target is re-spawned at a random position. This scenario is a variation of the Stick Pulling Experiment~\cite{ijspeert2001collaboration} and while it can be solved without communication, it has been shown that communication significantly improves performance for $N$ < $M$.

\vspace{8pt}

\noindent\textbf{Waterfall (\autoref{fig:waterfall})}. $N$ agents move from top to bottom through a series of obstacles. This is a testing scenario that can be used to discover VMAS's functionalities.

%VMAS contains 8 other scenarios, excluded for brevity. These scenarios range from adversarial and hierarchical tasks, such as football, to hard multi-robot problems such as flocking and the passage scenario introduced in~\cite{blumenkamp2021framework}. We invite the reader to refer to the \href{https://github.com/proroklab/VectorizedMultiAgentParticleSimulator}{VMAS codebase}\footref{foot:vmas_url} for a through description and visualization of all scenarios.
\section{Comparison with MPE}
\label{sec:mpe_comparison}
In this section, we compare the scalability of VMAS and MPE~\cite{lowe2017multi}. Given that we vectorize and port all the MPE scenarios in VMAS, we can compare the two simulators on the same MPE task. The task chosen is ``simple\_spread'', as it contains multiple collidable agents in the same environment. VMAS and MPE use two completely different execution paradigms: VMAS, being vectorized, leverages the Single Instruction Multiple Data (SIMD) paradigm, while MPE uses the Single Instruction Single Data (SISD) paradigm. Therefore, it is sufficient to report the benefits of this paradigm shift on only one task, as the benefits are task-independent.

In \autoref{fig:mpe_comparison}, we can see the growth in execution time with respect to the number of environments stepped in parallel for the two simulators. MPE runs only on the CPU, while VMAS, using PyTorch, runs both on the CPU and on the GPU. In this experiment, we compare the two simulators on an Intel(R) Xeon(R) Gold 6248R CPU @ 3.00GHz and we also run VMAS on an NVIDIA GeForce RTX 2080 Ti. The results show the impact of vectorization on simulation speed. On the CPU, VMAS is up to 5x faster than MPE. On the GPU, the simulation time for VMAS is independent of the number of environments, and runs up to 100$\times$ faster. The same results can be reproduced on different hardware. In the \href{https://github.com/proroklab/VectorizedMultiAgentParticleSimulator}{VMAS's repository}\footref{foot:vmas_url} we provide a script to repeat this experiment.

 
\begin{figure}[h!]
    \centering
    \scalebox{0.6}{%
    In this section, we compare the scalability of VMAS and MPE~\cite{lowe2017multi}. Given that we vectorize and port all the MPE scenarios in VMAS, we can compare the two simulators on the same MPE task. The task chosen is ``simple\_spread'', as it contains multiple collidable agents in the same environment. VMAS and MPE use two completely different execution paradigms: VMAS, being vectorized, leverages the Single Instruction Multiple Data (SIMD) paradigm, while MPE uses the Single Instruction Single Data (SISD) paradigm. Therefore, it is sufficient to report the benefits of this paradigm shift on only one task, as the benefits are task-independent.

In \autoref{fig:mpe_comparison}, we can see the growth in execution time with respect to the number of environments stepped in parallel for the two simulators. MPE runs only on the CPU, while VMAS, using PyTorch, runs both on the CPU and on the GPU. In this experiment, we compare the two simulators on an Intel(R) Xeon(R) Gold 6248R CPU @ 3.00GHz and we also run VMAS on an NVIDIA GeForce RTX 2080 Ti. The results show the impact of vectorization on simulation speed. On the CPU, VMAS is up to 5x faster than MPE. On the GPU, the simulation time for VMAS is independent of the number of environments, and runs up to 100$\times$ faster. The same results can be reproduced on different hardware. In the \href{https://github.com/proroklab/VectorizedMultiAgentParticleSimulator}{VMAS's repository}\footref{foot:vmas_url} we provide a script to repeat this experiment.

 
\begin{figure}[h!]
    \centering
    \scalebox{0.6}{%
    In this section, we compare the scalability of VMAS and MPE~\cite{lowe2017multi}. Given that we vectorize and port all the MPE scenarios in VMAS, we can compare the two simulators on the same MPE task. The task chosen is ``simple\_spread'', as it contains multiple collidable agents in the same environment. VMAS and MPE use two completely different execution paradigms: VMAS, being vectorized, leverages the Single Instruction Multiple Data (SIMD) paradigm, while MPE uses the Single Instruction Single Data (SISD) paradigm. Therefore, it is sufficient to report the benefits of this paradigm shift on only one task, as the benefits are task-independent.

In \autoref{fig:mpe_comparison}, we can see the growth in execution time with respect to the number of environments stepped in parallel for the two simulators. MPE runs only on the CPU, while VMAS, using PyTorch, runs both on the CPU and on the GPU. In this experiment, we compare the two simulators on an Intel(R) Xeon(R) Gold 6248R CPU @ 3.00GHz and we also run VMAS on an NVIDIA GeForce RTX 2080 Ti. The results show the impact of vectorization on simulation speed. On the CPU, VMAS is up to 5x faster than MPE. On the GPU, the simulation time for VMAS is independent of the number of environments, and runs up to 100$\times$ faster. The same results can be reproduced on different hardware. In the \href{https://github.com/proroklab/VectorizedMultiAgentParticleSimulator}{VMAS's repository}\footref{foot:vmas_url} we provide a script to repeat this experiment.

 
\begin{figure}[h!]
    \centering
    \scalebox{0.6}{%
    \input{figures/mpe_comparison/mpe_comparison}}
    \caption{Comparison of the scalability of VMAS and MPE~\cite{lowe2017multi} in the number of parallel environments. In this plot, we show the execution time of the ``simple\_spread'' scenario for 100 steps. MPE does not support vectorization and thus cannot be run on a GPU.}
    \label{fig:mpe_comparison}
\end{figure}
}
    \caption{Comparison of the scalability of VMAS and MPE~\cite{lowe2017multi} in the number of parallel environments. In this plot, we show the execution time of the ``simple\_spread'' scenario for 100 steps. MPE does not support vectorization and thus cannot be run on a GPU.}
    \label{fig:mpe_comparison}
\end{figure}
}
    \caption{Comparison of the scalability of VMAS and MPE~\cite{lowe2017multi} in the number of parallel environments. In this plot, we show the execution time of the ``simple\_spread'' scenario for 100 steps. MPE does not support vectorization and thus cannot be run on a GPU.}
    \label{fig:mpe_comparison}
\end{figure}

\section{Experiments and benchmarks}
\label{sec:experiments}
%!TEX root = main.tex

\section{Experimental Evaluation}
\seclabel{experiments}

We first evaluated our algorithms
in an offline setting~(\secref{offline-expr}), where we record execution traces and evaluate different approaches on the \emph{same} input.
This eliminates biases due to non-deterministic thread scheduling.
Next, we consider an online setting~(\secref{online-expr}),
where we instrument programs and perform the analyses during runtime.
We conducted all our experiments on a standard laptop with \SI{1.8}{GHz} Intel Core i7 processor and \SI{16}{GB} RAM.

% We evaluated our algorithms in two experimental settings. 
% The first setting is offline experiments~(\secref{offline-expr}),
% in which we record execution traces and evaluate different approaches.
% This has the benefit that different approaches can be compared on the \emph{same} input,
% thereby eliminating biases due to non-deterministic thread scheduling.
% The second setting is online experiments~(\secref{online-expr}),
% in which we instrument programs and perform the analyses during runtime.
% We conducted all our experiments on a standard laptop with 1.8GHz Intel Core i7 processor and 16GB RAM.

%We compare our predictive algorithm with the \dlfuzzer tool. 
%This setting enables us to assess the applicability of our technique in a runtime monitoring system.
%The state-of-the-art deadlock predictors \dirk and \seqc work offline by design.
%Hence, they are not applicable for a comparison in this setting.
%As discussed in~\secref{otf}, offline methods are not directly applicable in runtime monitoring systems.

%Talk about implementation - name of tool + prog lang + traces are logged + filtering phase + cycle detection phase + online vs offline + trace conversion for seqc
%
%Setup - benchmarks + log traces using so-and-so-tools + 1-trace-per-benchmark  + cluster details + how many runs per trace + Timeout (if any)
%
%benchmark details: number of shared locks
%\subsection{Evaluation}
%
%Some suggested experiments:
%
%\begin{itemize}
%	\item Comparison with other tool(s) - \dirk and \seqc
%	\item Comparison of online vs online algorithms
%	\item Scalability with number of events
%	\item Scalability with number of threads in the trace
%	\item Scalability with number of threads in the deadlock pattern
%	\item Time spent in each kinds of events (this will be useful to guide future research)
%\end{itemize}
%
%\Andreas{For matching reports, we slightly adapt the algorithm to report all sets of program locations that contain a deadlock pattern (hence we might have more than one reports per abstract deadlock pattern, corresponding to different program locations)}


% Here we report on an implementation and experimental evaluation of our algorithms.

% \subsection{Experimental Setup}



\subsection{Offline Experiments}
\seclabel{offline-expr}


\Paragraph{Experimental setup}
The goal of the first set of experiments is to evaluate 
$\SyncPDOffline$, and compare it
against prior algorithms for dynamic deadlock prediction.
In order for our evaluation to be precise we evaluate all algorithms on the \emph{same} execution trace.
We implemented $\SyncPDOffline$ in Java inside the \toolname analysis tool~\cite{rapid}, 
following closely the pseudocode in \algoref{offline}.
\toolname takes as input execution traces, as defined in \secref{prelim}.
These also include fork, join, and lock-request events.
We compare $\SyncPDOffline$ with two state-of-the-art, 
theoretically-sound albeit computationally more expensive, deadlock predictors,
\seqc~\cite{Cai2021} and \dirk~\cite{Kalhauge2018}, both of which also work on execution traces.




On the theoretical side, the complexity of \seqc is $\Otilde(\NumEvents^4)$, 
as opposed to the $\Otilde(\NumEvents)$ complexity of $\SyncPDOffline$. 
Moreover, \seqc only predicts deadlocks of size $2$, and though it could be extended to handle deadlocks of any size, this would degrade performance further.
\seqc may miss sync-preserving deadlocks even of size $2$, 
but can  detect deadlocks that are not sync-preserving.
Thus \seqc and $\SyncPDOffline$ are theoretically incomparable in their detection capability.
We refer to\begin{pldi}~\cite{arxiv}\end{pldi}\begin{arxiv}~\appref{incomp}\end{arxiv} for examples.
We noticed that \seqc fails on traces with non-well-nested locks --- we encountered one such case in our dataset.
\dirk's algorithm is theoretically complete, i.e., it can find all predictable deadlocks in a trace.
In addition, it can find deadlocks beyond the predictable ones, by reasoning about event values.
However, \dirk relies on heavyweight SMT-solving and
employs windowing techniques to scale to large traces. 
Due to windowing, it can miss deadlocks between events that are outside the given window. 
%\hunkar{
As with previous works~\cite{Cai2021, Kalhauge2018}, we set a window size of $10$K for \dirk.
%}

Our dataset consists of several benchmarks 
from standard benchmark suites --- IBM Contest suite~\cite{Farchi03}, Java Grande suite~\cite{Smith01},
DaCapo~\cite{Blackburn06}, and
SIR~\cite{doESE05} ---
and recent literature~\cite{Kalhauge2018, Cai2021, jula2008deadlock, Joshi2009}.
Each benchmark was instrumented with RV-Predict~\cite{rvpredict} or Wiretap~\cite{Kalhauge2018} and
executed in order to log a single execution trace.

%!TEX root = ../main.tex

\begin{table}[h!]
\caption{
%\hcomment{Should we remove some not very important rows from this table?}
Trace characteristics, abstract lock graph statistics and performance comparison.
%Column 1 denotes the name of the benchmark.
Columns 2-6 show the number of events, threads, variables, locks
and total number of lock acquire and request events.
Columns 7-9 show the number of cycles, abstract and concrete deadlock patterns in the abstract lock graph.
Columns 10 - 15 show the number of deadlocks reported and the times (in seconds) taken. 
by \dirk, \seqc, and \SyncPDOffline.
%Statistics on the lock graph $\lkevgraph{\tr}$ of each trace $\tr$.
% We denote by $\mathcal{N}$, $\mathcal{T}$, $\mathcal{M}$, $\mathcal{L}$ and $\NumAcquires + \mathcal{R}$ the total number of events, number of threads, number of memory locations, number of locks and number of acquire and request events in the benchmarks, respectively. 
%All times are in seconds. 
Time out (T.O) was set to $3$h.
F stands for technical failure.
\label{tab:expr-results}
}
\vspace{-0.17cm}
\setlength\tabcolsep{3pt}
\renewcommand{\arraystretch}{0.91}
\centering
\scalebox{0.86}{
\begin{tabular}{|r|c|c|c|c|c||c|c|c||c|c||c|c||c|c|}
\hline
1 & 2 & 3 & 4 & 5 & 6 & 7 & 8 & 9 & 10 & 11 & 12 & 13 & 14 & 15 \\
\hline
\multirow{2}{*}{\textbf{Benchmark}}& 
\multirow{2}{*}{$\mathcal{N}$} & \multirow{2}{*}{$\mathcal{T}$} & \multirow{2}{*}{$\mathcal{V}$} & \multirow{2}{*}{$\mathcal{L}$} & \multirow{2}{*}{$\NumAcquires/\mathcal{R}$} 
& \multicolumn{3}{c||}{ \textsf{A. Lock Graph }}
& \multicolumn{2}{c||}{ {\dirk} } 
& \multicolumn{2}{c||}{ {\seqc}} 
& \multicolumn{2}{c|}{ {\textsf{$\SyncPDOffline$}}} \\
\cline{7-15}
& & & & & 
& \textsf{|$\textsf{Cyc}$|}
& \textsf{\textsf{A. P.}}
& \textsf{\textsf{C. P.}}
& \textbf{Dlk} 
& {\textbf{Time}} 
&{\textbf{Dlk}} 
& {\textbf{Time}}
&{\textbf{Dlk}} 
&{\textbf{Time}} \\
\hline
Deadlock & 39 & 3 & 4 & 3 & 8 & 1 & 1 & 1 & 1 & 0.02 & 0 & 0.09 & 0 & 0.16\\
NotADeadlock & 60 & 3 & 4 & 5 & 16 & 1 & 1 & 1 & 0 & 0.02 & 0 & 0.09 & 0 & 0.16\\
Picklock & 66 & 3 & 6 & 6 & 20 & 2 & 2 & 2 & 1 & 0.02 & 1 & 0.10 & 1 & 0.18\\
Bensalem & 68 & 4 & 5 & 5 & 22 & 2 & 2 & 2 & 1 & 0.02 & 1 & 0.12 & 1 & 0.16\\
Transfer & 72 & 3 & 11 & 4 & 12 & 1 & 1 & 1 & 1 & 0.02 & 0 & 0.09 & 0 & 0.15\\
Test-Dimmunix & 73 & 3 & 9 & 7 & 26 & 2 & 2 & 2 & 2 & 0.02 & 2 & 0.10 & 2 & 0.17\\
StringBuffer & 74 & 3 & 14 & 4 & 16 & 1 & 3 & 6 & 2 & 0.02 & 2 & 0.12 & 2 & 0.19\\
Test-Calfuzzer & 168 & 5 & 16 & 6 & 48 & 2 & 1 & 1 & 1 & 0.02 & 1 & 0.12 & 1 & 0.17\\
DiningPhil & 277 & 6 & 21 & 6 & 100 & 1 & 1 & 3K & 1 & 1.60 & 0 & 0.09 & 1 & 0.17\\
HashTable & 318 & 3 & 5 & 3 & 174 & 1 & 2 & 43 & 2 & 0.19 & 2 & 0.12 & 2 & 0.19\\
Account & 706 & 6 & 47 & 7 & 134 & 3 & 1 & 12 & 0 & 0.19 & 0 & 0.09 & 0 & 0.18\\
Log4j2 & 1K & 4 & 334 & 11 & 43 & 1 & 1 & 1 & 1 & 0.65 & 1 & 0.11 & 1 & 0.20\\
Dbcp1 & 2K & 3 & 768 & 5 & 56 & 2 & 2 & 3 & - & F & 2 & 0.11 & 2 & 0.19\\
Dbcp2 & 2K & 3 & 592 & 10 & 76 & 1 & 2 & 4 & - & F & 0 & 0.10 & 0 & 0.18\\
Derby2 & 3K & 3 & 1K & 4 & 16 & 1 & 1 & 1 & 1 & 0.23 & 1 & 0.10 & 1 & 0.17\\
RayTracer & 31K & 5 & 5K & 15 & 976 & 0 & 0 & 0 & - & F & 0 & 0.15 & 0 & 0.19\\
jigsaw & 143K & 21 & 8K & 2K & 67K & 172 & 12 & 70 & - & F & 2 & 0.36 & 1 & 1.55\\
elevator & 246K & 5 & 727 & 52 & 48K & 0 & 0 & 0 & 0 & 1.65 & 0 & 0.33 & 0 & 0.27\\
hedc & 410K & 7 & 109K & 8 & 32 & 0 & 0 & 0 & 0 & 2.09 & 0 & 0.50 & 0 & 0.24\\
JDBCMySQL-1 & 442K & 3 & 73K & 11 & 13K & 2 & 4 & 6 & 2 & 28.45 & 2 & 0.24 & 2 & 0.48\\
JDBCMySQL-2 & 442K & 3 & 73K & 11 & 13K & 4 & 4 & 9 & 1 & 3.37 & 1 & 0.22 & 1 & 0.33\\
JDBCMySQL-3 & 443K & 3 & 73K & 13 & 13K & 5 & 8 & 16 & 1 & 31.23 & 1 & 0.25 & 1 & 0.45\\
JDBCMySQL-4 & 443K & 3 & 73K & 14 & 13K & 5 & 10 & 18 & 2 & 5.51 & 2 & 0.28 & 2 & 0.49\\
cache4j & 775K & 2 & 46K & 20 & 35K & 0 & 0 & 0 & 0 & 5.86 & 0 & 0.46 & 0 & 0.39\\
ArrayList & 3M & 801 & 121K & 802 & 176K & 9 & 3 & 672 & 3 & 8.7K & 3 & 21.98 & 3 & 1.68\\
IdentityHashMap & 3M & 801 & 496K & 802 & 162K & 1 & 3 & 4 & 1 & 443.93 & 1 & 8.51 & 1 & 1.45\\
Stack & 3M & 801 & 118K & 2K & 405K & 9 & 3 & 481 & 1 & T.O & 3 & 25.34 & 3 & 2.94\\
Sor & 3M & 301 & 2K & 3 & 719K & 0 & 0 & 0 & 0 & 15.89 & 0 & 44.12 & 0 & 0.61\\
LinkedList & 3M & 801 & 290K & 802 & 176K & 9 & 3 & 10K & 3 & 4.7K & 3 & 48.02 & 3 & 2.06\\
HashMap & 3M & 801 & 555K & 802 & 169K & 1 & 3 & 10K & 3 & 4.4K & 2 & 504.36 & 2 & 1.65\\
WeakHashMap & 3M & 801 & 540K & 802 & 169K & 1 & 3 & 10K & - & T.O & 2 & 499.68 & 2 & 1.70\\
Swing & 4M & 8 & 31K & 739 & 2M & 0 & 0 & 0 & - & F & 0 & 0.72 & 0 & 0.88\\
Vector & 4M & 3 & 15 & 4 & 800K & 1 & 1 & 1B & - & T.O & 1 & 1.52 & 1 & 1.90\\
LinkedHashMap & 4M & 801 & 617K & 802 & 169K & 1 & 3 & 10K & 2 & 40.74 & 2 & 492.87 & 2 & 1.69\\
montecarlo & 8M & 3 & 850K & 3 & 26 & 0 & 0 & 0 & 0 & 2.6K & 0 & 1.81 & 0 & 0.79\\
TreeMap & 9M & 801 & 493K & 802 & 169K & 1 & 3 & 10K & 2 & 105.45 & 2 & 480.11 & 2 & 1.92\\
hsqldb & 20M & 46 & 945K & 403 & 419K & 0 & 0 & 0 & - & F & - & - & 0 & 2.38\\
sunflow & 21M & 16 & 2M & 12 & 1K & 0 & 0 & 0 & - & F & 0 & 8.35 & 0 & 1.62\\
jspider & 22M & 11 & 5M & 15 & 10K & 0 & 0 & 0 & - & F & 0 & 8.49 & 0 & 1.95\\
tradesoap & 42M & 236 & 3M & 6K & 245K & 2 & 1 & 4 & - & F & 0 & 108.16 & 0 & 7.06\\
tradebeans & 42M & 236 & 3M & 6K & 245K & 2 & 1 & 4 & - & F & 0 & 116.23 & 0 & 7.26\\
eclipse & 64M & 15 & 10M & 5K & 377K & 9 & 5 & 280 & - & F & 0 & 26.67 & 0 & 9.90\\
TestPerf & 80M & 50 & 599 & 9 & 197K & 0 & 0 & 0 & 0 & 795.04 & 0 & 47.56 & 0 & 4.30\\
Groovy2 & 120M & 13 & 13M & 10K & 69K & 0 & 0 & 0 & 0 & 1.7K & 0 & 38.06 & 0 & 8.92\\
Tsp & 200M & 6 & 24K & 3 & 882 & 0 & 0 & 0 & 0 & 7.6K & 0 & 72.62 & 0 & 12.70\\
lusearch & 203M & 7 & 3M & 98 & 273K & 0 & 0 & 0 & 0 & 1.3K & 0 & 75.88 & 0 & 14.44\\
biojava & 221M & 6 & 121K & 79 & 16K & 0 & 0 & 0 & - & F & 0 & 63.79 & 0 & 12.65\\
graphchi & 241M & 20 & 25M & 61 & 1K & 0 & 0 & 0 & - & F & 0 & 102.05 & 0 & 25.25\\
\hline\hline\textbf{Totals} & \textbf{1B} & \textbf{7K} & \textbf{70M} & \textbf{37K} & \textbf{8M} & \textbf{256} & \textbf{93} & \textbf{1B} & \textbf{35} & \textbf{>18h} & \textbf{40} & \textbf{2801} & \textbf{40} & \textbf{135}\\\hline
\end{tabular}
}
\end{table}

\Paragraph{Evaluation}
\cref{tab:expr-results} presents our results.
A bug identifies a unique tuple of source
code locations corresponding to events participating in the deadlock.
%Columns 2-6 present the characteristics of our benchmark traces.
Trace lengths vary vastly from $39$ to about $241$M, while the number of threads ranges from $3$ to about $800$,
which are representative features of real-world settings.
\texttt{Hsqldb} contains critical sections that are not well nested, 
and \seqc was not able to handle this benchmark;
our algorithm does not have such a restriction.
%We were unable to run \dirk on certain benchmarks due to \dirk's technical issues, which are marked with F.


\vspace{-0.1cm}
\SubParagraph{\underline{Abstract vs Concrete Patterns}}
Columns 7-9 present statistics on the 
abstract lock graph $\lkevgraph{\tr}$ of each trace $\tr$.
Many traces have a large number of concrete deadlock patterns 
but much fewer abstract deadlock patterns;
a single abstract deadlock pattern can 
comprise up to an order of $10^4$ more concrete patterns (Column $8$ v/s Column $9$).
Unlike all prior sound techniques, 
our algorithms analyze 
abstract deadlock patterns, instead of concrete ones. 
We thus expect our algorithms to be much more scalable in practice.


\SubParagraph{\underline{Deadlock-detection capability}}
In total, both \seqc and $\SyncPDOffline$ reported 40 deadlocks.
\seqc misses a deadlock of size $5$ in \texttt{DiningPhil},
which is detected by $\SyncPDOffline$,
and $\SyncPDOffline$ misses a deadlock in \texttt{jigsaw} which is detected by \seqc.
As $\SyncPDOffline$ is complete for sync-preserving deadlocks, we conclude that there are no more such deadlocks in our dataset.
Overall, $\SyncPDOffline$ and \seqc miss only three deadlocks reported by \dirk. 
On closer inspection, we found that these deadlocks are not witnessed by correct reorderings, and require reasoning about event values.
On the other hand, \dirk struggles to analyze even moderately-sized benchmarks and times out in $3$ of them. %(timeout is 3h).
This results in \dirk failing to report 5 deadlocks after $9$ hours, all of which are reported by $\SyncPDOffline$ in under a minute.
Similar conclusions were recently made in~\cite{Cai2021}.  
Overall, our results strongly indicate that the notion of sync-preservation characterizes most deadlocks that other tools are able to predict.


\SubParagraph{\underline{Unsoundness of \dirk}}
In our evaluation, we discovered that the soundness guarantee 
underlying \dirk~\cite{Kalhauge2018} is broken, resulting in it reporting false positives.
% Although \dirk is portrayed as sound, we have found two independent sources of unsoundness 
% resulting in it reporting false positives.
First, its constraint formulation~\cite{Kalhauge2018} 
does not rule out deadlock patterns when acquire events in the pattern hold common locks, 
in which case mutual exclusion disallows such a pattern to be a real predictable deadlock.
Second, \dirk also models conditional statements, allowing it to reason about witnesses beyond correct reorderings.
While this relaxation allows \dirk to predict additional deadlocks in \texttt{Transfer}, \texttt{Deadlock} and \texttt{HashMap}, 
its formalization is not precise and its implementation is erroneous.
We elaborate these aspects further in\begin{pldi}~\cite{arxiv}. \end{pldi}\begin{arxiv}~\appref{unsound-dirk}.\end{arxiv}


%We noticed that \dirk reports false positives in certain cases, which contradicts its theoretical soundness guarantee.
%We provide two such cases in~\appref{unsound-dirk}.
%The first case is a modified version of \texttt{Transfer}.
%Here, \dirk falsely reports a deadlock because it inadequately models conditional statements, which are used to allow more trace reorderings that can expose a deadlock.
%This relaxation enables \dirk to predict deadlocks in the benchmarks \texttt{Transfer} and \texttt{Deadlock}, which are missed by $\SyncPDOffline$ and \seqc. 
%However, this relaxation is not formalized and its implementation is erroneous.
%In the second case, \dirk falsely reports a deadlock due to missing that a common lock protects an otherwise deadlock pattern.

\SubParagraph{\underline{Running time}}
Our experimental results indicate that \dirk, backed by SMT solving, 
is the least efficient technique in terms of running time ---
it takes considerably longer or times out on large benchmark instances.
$\SyncPDOffline$ analyzed the entire set of traces $\sim\!\!\!21\times$ faster than \seqc.
On the most demanding benchmarks, such as 
HashMap and TreeMap, $\SyncPDOffline$ is more than $200\times$ faster than \seqc.
Although \seqc employs a polynomial-time algorithm for deadlock prediction, 
and thus significantly faster than the SMT-based \dirk,
the large polynomial complexity in its running time hinders scalability on 
execution traces coming from benchmarks that are more representative of realistic workloads.
In contrast, the linear time guarantees of $\SyncPDOffline$ are realized in practice, 
allowing it to scale on even the most challenging inputs.
More importantly, the improved performance comes while preserving essentially the same precision.


%\begin{figure}[!h]
\def\scatterscale{0.192}
\def\scatterwidth{0.23\textwidth}

\centering
\begin{subfigure}[b]{\scatterwidth}
\includegraphics[scale=\scatterscale]{figures/LinkedListDeadlockTest.pdf}
\label{subfig:sca-one-lock-clique}
\caption{LinkedList}
\end{subfigure}
\begin{subfigure}[b]{\scatterwidth}
\includegraphics[scale=\scatterscale]{figures/StackDeadlockTest.pdf}
\caption{Stack}
\label{subfig:sca-skewed}
\end{subfigure}


\caption{
Scalability experiments.
}
\label{fig:scalability}
\end{figure}


%\Andreas{We might revisit/remove the following}

%\textcolor{red}{check the numbers again.}
\SubParagraph{\underline{False negatives}}
Our benchmark set contains $93$ abstract deadlock patterns, $40$ of which are confirmed sync-preserving deadlocks.
We inspected the remaining $53$ abstract patterns to see if any of them are predictable deadlocks
missed by our sync-preserving criterion, independently of the compared tools.
$48$ of these $53$ patterns are in fact not predictable deadlocks ---
for every such pattern $D$, 
the set $S_D$ of events in the downward-closure of $\prev{}(D)$ with respect to $\tho{}$ and $\rf{}$,
already contains an event from $D$, disallowing any correct reordering
(sync-preserving or not) in which $D$ can be enabled.
Of the remaining, $4$ deadlock patterns obey the following scheme:
there are two acquire events $\acq_1, \acq_2$ participating in the deadlock pattern, 
each $\acq_i$ is preceded by a critical section on a lock that appears in 
$\lheld{}(\acq_{3-i})$, again disallowing a correct reordering that witnesses the pattern.
Thus, \emph{only one} predictable deadlock is not sync-preserving in our whole dataset.
This analysis supports that the notion of sync-preservation is not overly conservative in practice.
%\hunkar{I think here we should clarify that this analysis is not overly conservative as far as predictable deadlocks go.}

%\hunkar{
The above analysis concerns false negatives wrt. predictable deadlocks.
Some deadlocks are beyond the common notion of predictability we have adopted here, as they can only be exposed by reasoning about event values and control-flow dependencies, a problem that is $\NP$-hard even for 3 threads~\cite{Gibbons1997}.
We noticed $3$ such deadlocks in our dataset, found by \dirk,
though, as mentioned above, \dirk's reasoning for capturing such deadlocks is unsound in practice.
%}

%\hunkar{Maybe stress again that this reasoning about event values is non-trivial as it relies on heavyweight SMT solving. Also, maybe have the section "Unsoundness of Dirk" after this one and say that next we show that it is also tricky to implement in practice.}
%sync-preserving deadlocks are likely to be permissive and not lead to a high false negative rate.
%We conduct an analyses based on our benchmark set and characterize potential false negatives of our technique.
%Recall the conditions imposed on correct reorderings (see \secref{prelim}).
%One of the main conditions imposed by the definition of a correct reordering is that
%every read event reads from the same write as in the original trace. 
%We investigated the effect of this condition on the potential false negatives.
%Our benchmark set contains $93$ abstract deadlock patterns and our technique is able to find $42$ deadlocks.
%The remaining $51$ deadlock patterns constitute potential false negatives.
%If we only impose the above condition, then $46$ of the deadlock patterns cannot be realized to a real deadlock.
%The additional conditions that are specific to sync-preserving deadlocks (e.g., the order of critical sections on the same lock cannot be reversed) prevents the remaining $5$ deadlock patterns from being realizable.
%We remark that the definition of correct reorderings adopted in this work originate from the standard model that is widely used in this domain~\cite{Smaragdakis12,serbanuta2013,Koushik05}.
%Hence, this analyses further supports that given this standard program model, the additional restrictions introduced with the
%sync-preserving deadlocks are likely to be permissive and not lead to a high false negative rate.


\subsection{Online Experiments}
\seclabel{online-expr}

\Paragraph{Experimental setup}
%Dynamic analyses can incur high runtime overheads.
The objective of our second set of experiments is to evaluate 
the performance 
of our proposed algorithms in an \emph{online} setting.
For this, we implemented our $\SyncPDOnline$ algorithm inside the
framework of \dlfuzzer~\cite{Joshi2009} following closely the pseudocode in \algoref{online}. 
This framework instruments a concurrent program so that it can
perform analysis on-the-fly while executing it.
If a deadlock occurs during execution, it is reported and the execution halts.
However, if a deadlock is predicted in an alternate interleaving, 
then this deadlock is reported and the execution continues to search further deadlocks.
We used the same dataset as in \secref{offline-expr}, 
after discarding some benchmarks that could not be instrumented by \dlfuzzer.

To the best of our knowledge, all prior deadlock prediction techniques work offline.
For this reason, we only compared our online tool with the randomized 
scheduling technique of~\cite{Joshi2009} already
implemented inside the same \dlfuzzer framework.	
At a high level, this random scheduling technique works as follows.
Initially, it
(i)~executes the input program with a random scheduler, 
(ii)~constructs a \emph{lock dependency relation}, and 
(iii)~runs a cycle detection algorithm to discover deadlock patterns. 
For each deadlock pattern thus found, it spawns new executions that attempt 
to realize it as an actual deadlock.
To increase the likelihood of hitting the deadlock,
\dlfuzzer biases the random scheduler by pausing threads at specific locations.
%(e.g., before acquiring a certain lock).

The second, confirmation phase of~\cite{Joshi2009}
acts as a best-effort proxy for sound deadlock prediction.
On the other hand, $\SyncPDOnline$ is already sound and predictive, and thus does not require
additional confirmation runs, making it more efficient.
Towards effective prediction, we also implemented a simple bias to the scheduler.
If a thread $t$ attempts to write on a shared variable $x$ while holding a lock, then 
our procedure randomly decides to pause this operation for a short duration.
This effectively explores race conditions in different orders.
Overall, implementing $\SyncPDOnline$ inside \dlfuzzer provided the added advantage of supplementing a powerful prediction technique with a biased randomized scheduler.
%As an added advantage of implementing our algorithms in the
% framework, we are able to achieve
%the complementary objective of evaluating how well prediction supplements
%the effectiveness of an otherwise naive
%controlled concurrency testing
%technique like randomized scheduling.
To our knowledge, our work is the first to effectively 
combine these two orthogonal techniques.
We also remark that such a bias is of no benefit to \dlfuzzer itself
since it does not employ any predictive reasoning.

% a randomized testing procedure, even though the latter is rather simple.

For this experiment, we run \dlfuzzer on each benchmark, and for each deadlock pattern found in the initial execution, 
we let it spawn $3$ new executions trying to realize the deadlock, 
as per standard (\href{https://github.com/ksen007/calfuzzer}{https://github.com/ksen007/calfuzzer}).
We repeated this process $50$ times and recorded the total time taken.
Then, we allocated the same time for $\SyncPDOnline$ to repeatedly execute the same program and perform deadlock prediction.
We measured all deadlocks found by each technique.
%We also noticed that \dlfuzzer fails to work properly if the given input program deadlocks in the initial run.
%This results in \dlfuzzer itself to go into deadlock without producing any deadlock reports.  
%Hence, we made a modest modification on \dlfuzzer allowing it to work properly and report deadlocks in such cases.
%We used this modified version of \dlfuzzer in our evaluation.





%!TEX root = ../main.tex

\begin{table*}
\caption{
Performance comparison of $\SyncPDOnline$ ($\syncpdshort$) and $\dlfuzzer$ ($\dlfshort$).
%%Column 1 denotes the name of the benchmark.
Columns 2-3 show the total number of bug reports. 
Columns 4-6 show the total number of unique bugs found by each tool, and their union. 
Columns 7-12 show the hit rate on each bug.
Columns 13-16 show the runtime overhead of the tools.
%We refer to $\dlfuzzer$ as $\dlfshort$, to $\SyncPDOnline$ as $\syncpdshort$, and the instrumentation phases using the suffix $\mathsf{\tt -I}$.
%Performance comparison of $\SyncPDOnline$ ($\syncpdshort$) and $\dlfuzzer$ ($\dlfshort$).
%Column 1 denotes the name of the benchmark.
%Columns 2-3 show the total number of bug reports. 
%Columns 4-6 show the total number of unique bugs found by each tool, and their union. 
%Columns 7-12 show the hit rate on each bug.
%Columns 13-17 show the runtime overhead of the tools.
%We use the suffix $\mathsf{\tt -I}$ for the instrumentation phases.
\label{tab:expr-dlf-results}
}
\vspace{-0.15cm}
\setlength\tabcolsep{3pt}
\renewcommand{\arraystretch}{0.9}
\small
\centering
\scalebox{1}{
\begin{tabular}{|r|c|c|c|c|c|c|c|c|c|c|c|c|c|c|c|c|c|c|c|c|c|c|c|c|c|c|c|}
\hline
1 & 2 & 3 & 4 & 5 & 6 & 7 & 8 & 9 & 10 & 11 & 12 & 13 & 14 & 15 & 16 \\
\hline
\multirow{2}{*}{\textbf{Benchmark}}& 
\multicolumn{2}{c|}{ {\textsf{Bug Hits}}} & 
\multicolumn{3}{c|}{ {\textsf{Unique Bugs}}} 
& \multicolumn{2}{c|}{ \textsf{Bug 1}} &
\multicolumn{2}{c|}{ \textsf{Bug 2}} &
\multicolumn{2}{c|}{ \textsf{Bug 3}} & \multicolumn{4}{c|}{ {\textsf{Runtime Overhead}}}  \\
\cline{7-16}
\cline{2-6}
& \textsf{$\syncpdshort$} & \textsf{$\dlfshort$} & \textsf{$\syncpdshort$} & \textsf{$\dlfshort$}  & All 
& \textsf{\textsf{$\syncpdshort$}} 
& \textbf{$\dlfshort$} 
& {\textbf{$\syncpdshort$}} 
&{\textbf{$\dlfshort$}} 
& {\textbf{$\syncpdshort$}}
&{\textbf{$\dlfshort$}} & {\textbf{$\syncpdshortinstr$}}  & {\textbf{$\syncpdshort$}}  & 
{\textbf{$\dlfshortinstr$}}  &
 {\textbf{$\dlfshort$}} \\
\hline
%\multirow{2}{*}{\textbf{Benchmark}}& \multirow{2}{*}{$\mathcal{N}$} & \multirow{2}{*}{$\mathcal{T}$} & \multirow{2}{*}{$\mathcal{M}$}
%& \multirow{2}{*}{$\mathcal{L}$} &
%\multirow{2}{*}{$\NumAcquires+\mathcal{R}$} &\multicolumn{2}{c||}{ {\seqc}} & \multicolumn{2}{c||}{ {\textsf{$\SyncPDOffline$}}} & \multicolumn{2}{c|}{ {\textsf{$\SyncPDOnline$}}} \\
%\cline{7-12}
%& & & & & &  \textbf{$\#$Deadlocks} & { \textbf{Time (s)}} &{ \textbf{$\#$Deadlocks}} & { \textbf{Time (s)}} &{ \textbf{$\#$Deadlocks}} &{ \textbf{Time (s)}} \\
\hline
Deadlock & 50 & 50 & 1 & 1 & 1 & 50 & 50 & - & - & - & - & 2$\times$ & 3$\times$ & 2$\times$ & 4$\times$\\
Picklock & 227 & 97 & 2 & 1 & 2 & 226 & 97 & 1 & 0 & - & - & 2$\times$ & 2$\times$ & 2$\times$ & 5$\times$\\
Bensalem & 355 & 32 & 2 & 1 & 2 & 8 & 0 & 347 & 32 & - & - & 2$\times$ & 2$\times$ & 2$\times$ & 6$\times$\\
Transfer & 54 & 50 & 1 & 1 & 1 & 54 & 50 & - & - & - & - & 2$\times$ & 2$\times$ & 1$\times$ & 4$\times$\\
Test-Dimmunix & 702 & 0 & 2 & 0 & 2 & 351 & 0 & 351 & 0 & - & - & 2$\times$ & 2$\times$ & 2$\times$ & 4$\times$\\
StringBuffer & 153 & 131 & 2 & 2 & 2 & 128 & 118 & 25 & 13 & - & - & - & - & - & -\\
Test-Calfuzzer & 177 & 44 & 1 & 1 & 1 & 177 & 44 & - & - & - & - & 2$\times$ & 2$\times$ & 2$\times$ & 4$\times$\\
DiningPhil & 162 & 100 & 1 & 1 & 1 & 162 & 100 & - & - & - & - & - & - & - & -\\
HashTable & 169 & 120 & 2 & 2 & 2 & 82 & 21 & 87 & 99 & - & - & - & - & - & -\\
Account & 19 & 188 & 1 & 1 & 1 & 19 & 188 & - & - & - & - & 2$\times$ & 8$\times$ & 2$\times$ & 16$\times$\\
Log4j2 & 290 & 100 & 2 & 1 & 2 & 145 & 100 & 145 & 0 & - & - & - & - & - & -\\
Dbcp1 & 265 & 138 & 2 & 2 & 2 & 264 & 61 & 1 & 77 & - & - & - & - & - & -\\
Dbcp2 & 129 & 126 & 2 & 2 & 2 & 86 & 99 & 43 & 27 & - & - & - & - & - & -\\
RayTracer & 0 & 0 & 0 & 0 & 0 & - & - & - & - & - & - & 122$\times$ & 124$\times$ & 109$\times$ & 111$\times$\\
Tsp & 0 & 0 & 0 & 0 & 0 & - & - & - & - & - & - & 47$\times$ & 60$\times$ & 37$\times$ & 40$\times$\\
jigsaw & 1189 & 1 & 1 & 1 & 2 & 1189 & 0 & 0 & 1 & - & - & - & - & - & -\\
elevator & 0 & 0 & 0 & 0 & 0 & - & - & - & - & - & - & 2$\times$ & 2$\times$ & 2$\times$ & 2$\times$\\
JDBCMySQL-1 & 349 & 117 & 2 & 3 & 3 & 1 & 21 & 0 & 4 & 348 & 92 & 3$\times$ & 4$\times$ & 2$\times$ & 13$\times$\\
JDBCMySQL-2 & 559 & 73 & 1 & 1 & 1 & 559 & 73 & - & - & - & - & 2$\times$ & 4$\times$ & 2$\times$ & 18$\times$\\
JDBCMySQL-3 & 560 & 224 & 1 & 1 & 1 & 560 & 224 & - & - & - & - & 2$\times$ & 5$\times$ & 2$\times$ & 24$\times$\\
JDBCMySQL-4 & 1717 & 101 & 3 & 1 & 3 & 95 & 0 & 834 & 0 & 788 & 101 & 3$\times$ & 5$\times$ & 2$\times$ & 31$\times$\\
hedc & 0 & 0 & 0 & 0 & 0 & - & - & - & - & - & - & 2$\times$ & 2$\times$ & 1$\times$ & 2$\times$\\
cache4j & 0 & 0 & 0 & 0 & 0 & - & - & - & - & - & - & 2$\times$ & 2$\times$ & 2$\times$ & 2$\times$\\
lusearch & 0 & 0 & 0 & 0 & 0 & - & - & - & - & - & - & 16$\times$ & 17$\times$ & 13$\times$ & 16$\times$\\
ArrayList & 47 & 45 & 3 & 3 & 3 & 20 & 22 & 3 & 5 & 24 & 18 & 50$\times$ & 69$\times$ & 18$\times$ & 79$\times$\\
Stack & 44 & 27 & 3 & 3 & 3 & 18 & 13 & 8 & 4 & 18 & 10 & 69$\times$ & 91$\times$ & 64$\times$ & 86$\times$\\
IdentityHashMap & 68 & 62 & 2 & 2 & 2 & 13 & 47 & 55 & 15 & - & - & 4$\times$ & 8$\times$ & 3$\times$ & 10$\times$\\
LinkedList & 48 & 26 & 3 & 2 & 3 & 21 & 17 & 7 & 0 & 20 & 9 & 16$\times$ & 28$\times$ & 14$\times$ & 32$\times$\\
Swing & 0 & 0 & 0 & 0 & 0 & - & - & - & - & - & - & 5$\times$ & 6$\times$ & 4$\times$ & 6$\times$\\
Sor & 0 & 0 & 0 & 0 & 0 & - & - & - & - & - & - & 2$\times$ & 7$\times$ & 2$\times$ & 2$\times$\\
HashMap & 46 & 44 & 2 & 2 & 2 & 18 & 11 & 28 & 33 & - & - & 7$\times$ & 11$\times$ & 4$\times$ & 8$\times$\\
Vector & 126 & 50 & 1 & 1 & 1 & 126 & 50 & - & - & - & - & 2$\times$ & 2$\times$ & 2$\times$ & 3$\times$\\
LinkedHashMap & 57 & 43 & 2 & 2 & 2 & 22 & 10 & 35 & 33 & - & - & 10$\times$ & 10$\times$ & 4$\times$ & 8$\times$\\
WeakHashMap & 29 & 40 & 2 & 2 & 2 & 6 & 11 & 23 & 29 & - & - & 7$\times$ & 12$\times$ & 4$\times$ & 8$\times$\\
montecarlo & 0 & 0 & 0 & 0 & 0 & - & - & - & - & - & - & 16$\times$ & 100$\times$ & 13$\times$ & 126$\times$\\
TreeMap & 42 & 47 & 2 & 2 & 2 & 16 & 15 & 26 & 32 & - & - & 9$\times$ & 12$\times$ & 5$\times$ & 9$\times$\\
eclipse & 0 & 0 & 0 & 0 & 0 & - & - & - & - & - & - & 2$\times$ & 2$\times$ & 2$\times$ & 2$\times$\\
TestPerf & 0 & 0 & 0 & 0 & 0 & - & - & - & - & - & - & 2$\times$ & 2$\times$ & 2$\times$ & 2$\times$\\
\hline\hline\textbf{Total} & \textbf{7633} & \textbf{2076} & \textbf{49} & \textbf{42} & \textbf{51} & - & - & - & - & - & - & - & - & - & - \\ 
\hline
\end{tabular}
}
\end{table*}


\Paragraph{Evaluation}
\cref{tab:expr-dlf-results} presents our experimental results.
%A bug identifies a unique tuple of source code locations corresponding to events
%participating in the deadlock.
Columns $2$-$3$ of the table display the total number of bug hits,
which is the total number of times a bug was predicted by $\SyncPDOnline$ in the entire duration,
or was confirmed in any trial of \dlfuzzer.
Columns $4$-$6$ display the unique bugs (i.e., unique tuples of source code locations leading to a deadlock) 
found by the techniques.
The employed techniques are able to find a maximum of $3$ unique bugs for each benchmark
in our benchmark set. 
Respectively, columns $7$-$12$ display the detailed information on the number 
of times a particular bug was found by each technique.
Runtime overheads are displayed in the columns $13$-$16$, with $\mathsf{\tt -I}$ denoting the instrumentation phase only.


\SubParagraph{\underline{Deadlock-detection capability}}
\dlfuzzer had $2076$ bug reports in total, and it found $42$ unique bugs.
In contrast, $\SyncPDOnline$ flagged $7633$ bug reports, corresponding to $49$ unique bugs.
In more detail, \dlfuzzer missed $9$ bugs reported by \SyncPDOnline whereas 
$\SyncPDOnline$ missed $2$ bugs reported by \dlfuzzer.
Also, \SyncPDOnline significantly outperformed \dlfuzzer in total number of bugs hits.
Our experiments again support that the notion of sync-preservation  captures most deadlocks that occur in practice, to the extent that other state-of-the-art techniques can capture.
%\hunkar{
A further observation is that in the offline experiments, \SyncPDOffline  was not able to find deadlocks in \texttt{Transfer} and \texttt{Deadlock}. 
However, the random scheduling procedure allowed \SyncPDOnline 
to navigate to executions from which deadlocks can be predicted.
This demonstrates the potential of combining predictive dynamic
techniques with controlled concurrency testing.
%; a direction we find promising to be pursued further by the community.


\SubParagraph{\underline{Runtime overhead}}
We have also measured the runtime overhead of both \SyncPDOnline and \dlfuzzer,
both as incurred by instrumentation, as well as by the deadlock analysis.
The latter is the time taken by \algoref{online} for the case of $\SyncPDOnline$,
and the overhead introduced due to the new executions in the second confirmation phase for the case of \dlfuzzer.
Our results show that the instrumentation overhead of \SyncPDOnline is, in fact, comparable to that of \dlfuzzer, though somewhat larger. 
This is expected, as \SyncPDOnline needs to also instrument memory access events, while \dlfuzzer only instruments lock events, but at the same time surprising because the number of memory access events
is typically much larger than the number of lock events.
On the other hand, the analysis overhead is often larger for \dlfuzzer, 
even though it reports fewer bugs.
It was not possible to measure the runtime overhead in certain benchmarks as 
either they were always deadlocking or the computation was running indefinitely.



\section{Conclusion}
\label{sec:conclusion}
\section{Conclusions}
In this paper, we set out to address the problem of multi-tasking robots in multi-robot tasks. 
%A fundamental limitation of existing multi-robot systems was addressed by the removal of a restrictive assumption that was often made--robots are single-tasking.
%Our method allowed coalitions to overlap thus enabling multi-tasking robots. 
We observed that the key underlying challenge was to reason about the physical constraints that could be synergistically satisfied.
%which directly affected the feasibility of multi-tasking.
To address the challenge, we developed our method based on the information invariant theory and modeled constraints as information instances. 
%This allowed us to reason about the relationships between constraints by reasoning about those between information requirements. 
Thereby, a formal and general framework to achieve multi-tasking robots was developed. 
We showed that our algorithm was sound and complete under our problem settings. 
%Our method was integrated with a simple greedy heuristic for task allocation.
Simulation  results  were  provided  to  show  the  effectiveness  of  our approach under resource-constrained situations and in handling challenging situations. % in a multi-UAV simulator. 

% The idea of multi-tasking is attractive in many ways. 
% Humans are living in multi-tasking environments--at any point of time, 
% we are optimizing for more than one task. 
% Multi-task often leads to more efficient task performance since it allows us to exploit task synergies. 
% The work presented in this paper takes us one step forward in realizing multi-tasking robots. 
% In particular, we started looking at the feasibility of multi-tasking. 
% There are many potential directions to pursue along this direction. First, several limitations are present in the current approach. 
% For example, although our method guarantees that there exists a physical configuration that satisfies all the constraints, it does not explicitly take the environmental influence into account. For example, a narrow corridor may prevent a robot formation from passing through, even though all the constraints for the formation do not introduce any conflicts. In this sense, our work should better be characterized as establishing a necessary condition for multi-tasking. Also, our method is mainly focused on the ``{\it planning}'' phase and hence does not address how the robots reach the desired configuration and maintain the constraints. These issues are assumed to be handled by the execution layer.

% More generally, the question of how to execute the tasks with overlapping coalitions is not addressed in this work. 
% As we already discussed, executing individual tasks with non-overlapping coalitions is straightforward but task synergies impose additional requirements on the task execution: how should the robots that are assigned multiple tasks execute them? Should they consider them in a prioritized strategy~\cite{van2005prioritized}? Or should they combine the different tasks in a way that is similar to motor schemas~\cite{arkin2}. 
% Communication requirements for maintaining the constraints must also be taken into account. How should the robots optimize their communication to improve the task performance? 

% The stringency of the physical constraints is another interesting question. It may be desirable to relax the constraints in certain situations (e.g., due to environmental influences). In such cases, it may be important to consider the problem where the constraints are least violated~\cite{kim2012revision}, or specify task constraints in different ways to increase the diversity of the configurations~\cite{srivastava2007domain} so as to make it robust to different environments. 
\section*{Acknowledgements}
This work was supported by ARL DCIST CRA W911NF-17-2-0181 and European Research Council (ERC) Project 949940 (gAIa). R. Kortvelesy was supported by Nokia Bell Labs through their donation for the Centre of Mobile, Wearable Systems and Augmented Intelligence to the University of Cambridge. J. Blumenkamp acknowledges the support of the ‘Studienstiftung des deutschen Volkes’ and an EPSRC tuition fee grant.


\bibliographystyle{spmpsci} 
\bibliography{bibliography}

\end{document}
