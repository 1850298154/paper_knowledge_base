\section{Conclusion}
In this paper, we formalize the \emph{stochastic task and motion planning (STAMP)} problem and introduce an \emph{anytime}, \emph{sound}, and \emph{probabilistically complete} \emph{HPlan} algorithm that uses \emph{entity} abstractions to compute contingent task and motion solutions for the STAMP problems using off-the-shelf task and motion planners. The \emph{HPlan} algorithm interleaves search for concretizations of the actions in the current model with computing refinements of the current abstract model. Policies generated through \emph{HPlan} are complete in the sense that it provides an action for each scenario that may arise while executing the policy, unlike previous works that provide partial policies for most likely scenarios only. The approach uses greedy selection to prioritize more likely actions over less likely ones to allow the robot to start executing actions rather than waiting for a complete solution. 

The approach assumes access to accurate descriptions for the robot's action models and an abstraction function that can be used to formulate the STAMP problem. As part of the future work, we intend to relax these assumptions by learning the abstraction function. \citet{shah2020learning} present a framework to learn critical regions for arbitrary motion planning problem. We plan to extend this work to learn the abstract representation for the STAMP problems. The current approach requires the robot to have full observability over its state. We also aim to relax this requirement and support partially observable setups as part of our future work. 

