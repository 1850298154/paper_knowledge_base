% \documentclass[Afour,sageh,times]{sagej}

\documentclass[jair,twoside,11pt,theapa]{article}
\usepackage{jair, theapa, rawfonts}

\jairheading{1}{1993}{1-15}{6/91}{9/91}
\ShortHeadings{Anytime Stochastic Task and Motion Policies}
{Shah and Srivastava}


\usepackage{moreverb,url}
\usepackage{amsmath}

\usepackage{graphicx}
\usepackage{stmaryrd}

% \usepackage[colorlinks,bookmarksopen,bookmarksnumbered,citecolor=blue,urlcolor=red,draft]{hyperref}
% \usepackage[colorlinks,bookmarksopen,bookmarksnumbered,citecolor=blue,urlcolor=red]{hyperref}
\usepackage{url}
\usepackage{cleveref}
\usepackage[inline,shortlabels]{enumitem}
\usepackage[ruled,vlined,linesnumbered]{algorithm2e}
\usepackage{makecell}
\usepackage{tikz}
% \usepackage{natbib}
\usepackage{amssymb}
\usepackage{siunitx}

\newcommand{\U}{\mathcal{U}\xspace}
\newcommand\M{{\cal M}}        % caligraphic Mß
\newcommand{\mysssection}[1]{\noindent\textbf{#1}\hspace{8pt}}
\newcommand{\set}[1]{\{#1 \}}
\newcommand{\?}{\raisebox{.5pt}{\textcircled{\raisebox{-.9pt} {{\small
          ?}}}}}
\newcommand{\abs}[1]{\left[ #1 \right] }
\newcommand{\Tau}{\mathcal{T}}
\newcommand{\Psym}{\mathcal{P}_\text{sym}}
\newcommand{\Ph}{\mathcal{P}_\text{h}}
\newcommand{\psym}{p_\text{sym}}
\newcommand{\ph}{p_\text{h}}
\newcommand{\Asym}{\mathcal{A}_\text{sym}}
\newcommand{\Ah}{\mathcal{A}_\text{h}}
\newcommand{\Am}{\mathcal{A}} 
\newcommand{\asym}{a_\text{sym}}
\newcommand{\ah}{a_\text{h}}
\newcommand{\Pm}{\mathcal{P}}
\newcommand{\Sm}{\mathcal{S}}
\newcommand{\X}{\mathcal{X}}
\newcommand{\C}{\mathcal{C}}
\newcommand{\citet}[1]{\citeauthor{#1}~\citeyear{#1}}
\newcommand{\inter}[1]{\llbracket #1 \rrbracket}


\usepackage{lipsum}

\newcommand\blfootnote[1]{%
  \begingroup
  \renewcommand\thefootnote{}\footnote{#1}%
  \addtocounter{footnote}{-1}%
  \endgroup
}

\newtheorem{example}{Example}
\newtheorem{proposition}{Proposition}
\newtheorem{definition}{Definition}
\newtheorem{lemma}{Lemma}
\newtheorem{theorem}{Theorem}
\newtheorem{proof}{Proof}
\usepackage{xcolor}
\usepackage{etoolbox}
\usepackage{caption}
% \captionsetup[figure]{font=footnotesize}
\usepackage{subcaption}
\providetoggle{full}


\newcommand\BibTeX{{\rmfamily B\kern-.05em \textsc{i\kern-.025em b}\kern-.08em
T\kern-.1667em\lower.7ex\hbox{E}\kern-.125emX}}

\def\volumeyear{2021}


\setcounter{secnumdepth}{3}
% \hypersetup{draft}
\begin{document}

\title{An Anytime Hierarchical Approach for \\ Stochastic Task and Motion Planning}

% \author{Naman Shah\affilnum{1} and Siddharth Srivastava\affilnum{1}}
% \author{Naman Shah\affilnum{1} and Siddharth Srivastava\affilnum{1}}
% \affiliation{\affilnum{1}Arizona State University, Tempe, AZ}

% \corrauth{Naman Shah, 
% School of Computing and Augmented Intelligence (SCAI),
% Arizona State University,
% Tempe, AZ, 85281, USA.}
\author{\name Naman Shah \email namanshah@asu.edu \\
    \addr Arizona State University, \\ 699 S Mill Ave, 
    Tempe, AZ, USA, 85281 
    \AND 
    \name Siddharth Srivastava \email siddharths@asu.edu \\ 
    \addr Arizona State University, \\ 699 S Mill Ave,
    Tempe, AZ, USA, 85281 
}


\maketitle

\begin{abstract}
    In order to solve complex, long-horizon tasks, intelligent robots
    need to carry out high level, abstract planning and reasoning in
    conjunction with motion planning. However, abstract models are
    typically lossy and plans or policies computed using them can be
    inexecutable. These problems are exacerbated in 
    stochastic situations where the robot needs to reason about, and
    plan for multiple  contingencies. 
      
    We present a new approach for integrated task and
    motion planning in stochastic settings. In contrast to prior work in this
    direction, we show that our approach can effectively compute
    integrated task and motion policies whose branching structures
    encode agent behaviors that handle multiple execution-time contingencies. We
    prove that our algorithm is probabilistically complete and can
    compute feasible solution policies in an anytime fashion so that the
    probability of encountering an unresolved contingency decreases over
    time. Empirical results on a set of challenging  problems show the
    utility and scope of our method.
\end{abstract}

% \keywords{ Stochastic task and motion planning, State and entity abstraction, Mobile manipulation, Hierarchical planning }

\blfootnote{In submission}

% \footnote{In submission}



\IEEEPARstart{T}{wo} %
main challenges in the deployment of large-scale swarms are the localization and coordination of vehicles.
Localization methods that rely on external infrastructure 
(e.g., GPS) 
are prone to systematic errors (e.g., multipath effect)
and may not always be available.
Coordination strategies that are centralized can deconflict motion plans to prevent collisions and gridlock, but introduce a single point of failure and are difficult to scale in swarm size due to communication bandwidth limitations.

This paper presents a unified formation flying pipeline for unmanned aerial vehicles (UAVs).
Our pipeline uses \textit{onboard} sensors for localization, which eliminate the need for external positioning systems, and \textit{distributed} techniques for coordination, which enable each vehicle to make decisions independently while communicating their state to a subset of the team.
For \textit{localization}, we use an off-the-shelf commercial visual inertial odometry (VIO) package \cite{VIO}
that fuses inertial measurement unit (IMU) and downward-facing monocular camera measurements to estimate changes in the vehicle pose.
\edit{For \textit{coordination}, we present distributed formation control and task assignment strategies that run onboard the vehicles, do not rely on a common reference frame, and use vehicle-to-vehicle communication.} 
Key features of our formation control strategy include scalability to a large number of vehicles and robustness to disturbances.
The latter is crucial for reaching the desired formations with sensing imperfections.
Our task assignment strategy uses an auction-based algorithm to guarantee conflict-free assignments.
This algorithm can deconflict vehicle gridlocks resulting from distributed collision avoidance (type 3 deadlock~\cite{Wang2017}) and is well-suited for vehicles with limited computational capability and low-bandwidth communication. 


\begin{figure}[t!]
	\begin{center}
		\includegraphics[trim =0mm 10mm 0mm 0mm, clip, width=\columnwidth]{Figs/slanted_plane.png}	
		\caption{
		Six multirotors in a slanted plane formation.
		Vehicles communicate with each other, make distributed decisions onboard, and use VIO for localization.}
		\label{fig:slantedplane}
	\end{center}
\end{figure}


\subsection{Contributions}

This research extends our previous work on UAV formations~\cite{Fathian2019} and presents a unified pipeline consisting of \textit{onboard localization} and \textit{distributed coordination}.
The three main contributions of this work are:
\begin{enumerate}
    \item \edit{scalable formulation of control design suitable for
    onboard sensing without a common reference frame;}
    \item algorithms for deconfliction via \edit{distributed} task assignment of vehicles to desired formation points;    
    \item simulation- and hardware-ready open-source pipeline.
\end{enumerate}
\edit{Our pipeline is tested in hardware with six multirotors (see Fig.~\ref{fig:slantedplane}), and 
to our knowledge is the first demonstration of formation flying that does not rely on external sensing, fiducial markers for localization, a common reference frame, or a centralized base station for coordination.}
The only requirements for the presented pipeline are that the vehicles can communicate, can find the transformation between their VIO start frames, and the environment is sufficiently textured---a standard assumption for VIO systems.
As such, this framework paves the way for future, real-world deployments of aerial vehicle swarms in large numbers and without requiring external localization infrastructure.


\begin{figure} [t!]
\centering
	\begin{subfigure}[b]{0.32\columnwidth}
	   %
	    \includegraphics[width=0.8\textwidth,left]{Figs/Frames2_full.pdf}
	    \caption{\scriptsize full alignment}
	    \label{fig:frame-a}
	\end{subfigure}
	\begin{subfigure}[b]{0.32\columnwidth}
	    \includegraphics[width=0.8\textwidth,center]{Figs/Frames2_orientation.pdf}
	    \caption{\scriptsize orientation alignment}
	    \label{fig:frame-b}
	\end{subfigure}
	\begin{subfigure}[b]{0.32\columnwidth}
	    \includegraphics[width=0.8\textwidth,right]{Figs/Frames2_none.pdf}
	    \caption{\scriptsize no alignment}
        \label{fig:frame-c}
	\end{subfigure}
\caption{\edit{Required alignment of UAV frames in existing swarm strategies: (a) the most restrictive case requiring a common reference frame, i.e., orientation and origin of the frames must be aligned; (b) only the orientation of the frames must be aligned; (c) no alignment restrictions (this work).}}
	\label{fig:Frames}
\end{figure}




\subsection{Related Work}

Existing aerial swarms can be grouped based on the coordination (centralized vs.\ distributed) and localization (external vs.\ onboard) methods used. 
\edit{It is further crucial to distinguish these methods based on the level of alignment required for the vehicle coordinate frames; see Fig.~\ref{fig:Frames}.} 
 
\edit{
Works with \textit{centralized} coordination and \textit{external} localization include~\cite{Preiss2017, Honig2018, Du2019}, which are based on lightweight UAVs with limited onboard computational capability and therefore rely on an external motion capture system and a base station.
Works with \textit{distributed} coordination and \textit{external} localization include \cite{wilson2020robotarium}, \cite{enright2004spheres}, where robots execute distributed controls  based on external localization by motion capture and ultrasonic beacons, respectively.
Works with \textit{centralized} coordination and \textit{onboard} localization include~\cite{Forster2013}, \cite{Loianno2016}, which use a ground station for task assignment among vehicles.
In \cite{Weinstein2018}, formation flying based on VIO is demonstrated, where motion planning and assignment are run on a base station to ensure collision-free trajectories.
The coordination strategies used in aforementioned works require a \textit{common reference frame} (Fig.~\ref{fig:frame-a}).
}


\edit{
Despite the large body of work on formation control~\cite{Oh2015}, and the variety of onboard sensing solutions for localization (e.g., VIO~\cite{Delmerico2018}), few frameworks demonstrated formation flying with \textit{distributed} coordination and \textit{onboard} localization.
A key reason is reliance of many distributed control and assignment algorithms on aligned frames (Fig.~\ref{fig:frame-a}, \ref{fig:frame-b}), which require computation-expensive and/or communication-intensive synchronization/consensus steps for frame alignment.
Equally important, dependence on alignment in existing methods \cite{Wang2017,Turpin2014, van2011reciprocal, morgan2016swarm} diminishes robustness to inherent noise and unobservable errors that cannot be corrected (e.g., disparities between the actual and estimated body frame \textit{orientation} caused by VIO drift).
Leveraging coordination methods that are \textit{robust to misaligned frames} is hence crucial and a focus of this work. 
}






\edit{
Examples of other pipelines with distributed coordination and onboard localization include \cite{Montijano2016,Tron2016}.
Both works demonstrated formation flying on three UAVs, required information from an external motion capture system due to hardware limitations, did not incorporate collision avoidance, and required frame alignment.
}
\edittwo{Note that while~\cite{Montijano2016,Tron2016} can achieve formations with arbitrary headings as illustrated in Fig.~\ref{fig:frame-c}, knowledge of relative orientations is still required; therefore, they belong to the category of Fig.~\ref{fig:frame-b}.}






\if 0

\r{
decentralized coordination setting combined with VIO:
D-CAPT [26]~\cite{}:
ORCA ~\cite{}: 
CBF [2]~\cite{} :
[A]
}

\r{Robusteness in coordination,  with compounded noise/latency, which would eventually break (b).\\


some existing algorithm might as well
work in a similar fully decentralized setting, when combined with VIO
as proposed here. For example, D-CAPT [26], ORCA, CBF [2] might also be
useful for such a task and are computationally even more efficient than
the proposed approach. \\

R2:  onboard sensing for localization ->
 Finally, the related work section only
focuses on this aspect of the pipeline, discussing how many formation papers include
onboard localization but barely sells the advantages of the coordination module (the actual
proposal of the paper) against other competitors such as [26] or [A] or to mention similar
coordination pipelines. \\


Given a solution to this problem, the controller in Section III seems unnecessary, each drone
has a target position and can use a local controller with collision avoidance that drives it to
that position. Note that such controllers exists in the literature (e.g., RVO in any of its
multi-agent variantes), they are distributed in nature and only require local sensing.


}

\fi

\section{Related work}\label{background} 
Task allocation is a well-studied problem, posing ongoing challenges in various computing environments \cite{Stavrinides2019, Genez2020, Jayanetti2022, Kanbar2022, Kritikakou2022, Peixoto2022, Mo2023}.
However, previous related research efforts do not consider the edge/hub/cloud architecture, nor all of the parameters investigated in this work. This is demonstrated in \cref{table:comparison}, which summarizes our qualitative comparison with relevant state-of-the-art approaches. 
The comparison is made with respect to the objectives and parameters considered in this work, the applicability of each approach to applications comprising multiple tasks with precedence relationships among them (i.e., applications with a task flow graph structure), as well as the optimality of the solution provided by each method.
An overview of the related literature, as well as a comparison with our preliminary research, are provided in the remainder of this section.


\begin{table*}[!ht]
\centering
\caption{Qualitative comparison of this work with relevant research efforts.}
\label{table:comparison}
\footnotesize
\resizebox{0.85\textwidth}{!}{
    \begin{tabular}{@{\extracolsep{4pt}}lcccccccccc@{}} 
        \toprule
        \multirow{3}{*}{Reference} & \multicolumn{2}{c}{Objectives} & \multicolumn{6}{c}{Considered Parameters} & \multirow{2}{*}{Task Flow} & \multirow{2}{*}{Optimal}\\
         \cline{2-3}   \cline{4-9} 
        & Latency & Energy & Comp. & Comp. & Comm. &  Comm. &  \multirow{2}{*}{Memory} & \multirow{2}{*}{Storage} & \multirow{2}{*}{Graph} & \multirow{2}{*}{Solution}\\
        & Min. & Min. & Latency & Energy & Latency & Energy & & & & \\
        
        \hline
        %-- Latency Minimization References
        %                               Latency         Energy       Comp.          Comp.        Comm.        Comm.        Memory        Storage       TFG           Optimal Solution
        \cite{Alfakih2021}              & \checkmark    & -           & \checkmark  & -           & -          & -          & \checkmark  & \checkmark & -          & -           \\
        \cite{Guevara2022}              & \checkmark    & -           & \checkmark  & -           & \checkmark & -          & \checkmark  & \checkmark & \checkmark & -           \\
        \cite{Weikert2022}              & \checkmark    & -           & \checkmark  & -           & \checkmark & \checkmark & \checkmark  & -          & \checkmark & -           \\
        \cite{Lai2022}                  & \checkmark    & -           & \checkmark  & -           & \checkmark & -          & \checkmark  & \checkmark & -          & -           \\
        \cite{Barijough2019}            & \checkmark    & -           & \checkmark   & -          & \checkmark & \checkmark & -           & -          & \checkmark & \checkmark  \\
        \cite{Tang2022}                 & \checkmark    & -           & \checkmark   & -          & \checkmark & -          & -           & \checkmark & -          & \checkmark  \\
        \cite{Kuang2021}                & \checkmark    & -           & \checkmark   & \checkmark & \checkmark & \checkmark & -           & -          & -          & -           \\
        
        %-- Energy Minimization References
        %                               Latency         Energy       Comp.          Comp.        Comm.        Comm.        Memory      Storage         TFG           Optimal Solution                  
        \cite{Avgeris2022}              & -             & \checkmark  & \checkmark  & \checkmark  &\checkmark & -           & -           & -          & -          & \checkmark \\
        \cite{Khalil2018}               & -             & \checkmark  & -           & \checkmark  & -         & \checkmark  & -           & -          & -          & -          \\
        \cite{Kritikakou2023}           & -             & \checkmark  & \checkmark  & \checkmark  & -         & -           & -           & -          & \checkmark & -          \\
        \cite{Hu2020}                   & -             & \checkmark & \checkmark   & \checkmark & \checkmark & \checkmark & -          & -            & -          & -          \\
        \cite{Azizi2022}                & -             & \checkmark & \checkmark   & \checkmark & \checkmark & -          & -          & -            & -          & -          \\
        \cite{Li2022}                   & -             & \checkmark & \checkmark   & \checkmark & \checkmark & \checkmark & -          & -            & -          & -          \\

        %-- Latency and Energy Minimization References
        %                               Latency         Energy       Comp.          Comp.        Comm.        Comm.        Memory      Storage         TFG          Optimal Solution          
        \cite{Zhang2021}                & \checkmark    & \checkmark & \checkmark   & \checkmark & \checkmark & \checkmark & -          & -            & -          & -          \\
        \cite{Dinh2017}                 & \checkmark    & \checkmark & \checkmark   & \checkmark & \checkmark & \checkmark & -          & -            & -          & -          \\        
        \cite{Tong2023}                 & \checkmark    & \checkmark & \checkmark   & \checkmark & \checkmark & \checkmark & -          & -            & -          & -          \\

        
        This work & \checkmark & \checkmark & \checkmark & \checkmark & \checkmark & \checkmark & \checkmark & \checkmark & \checkmark  & \checkmark \\
        \bottomrule
    \end{tabular}
}
%\vspace{-3mm}
\end{table*}



\subsection{Latency minimization}
A number of works on task allocation in edge computing and multi-tier environments have a primary focus on latency minimization.
For instance, Alfakih et al. \cite{Alfakih2021} explore the minimization of the computational latency of task execution in an edge computing system, based on an accelerated particle swarm optimization algorithm combined with a dynamic programming approach.
Guevara et al. \cite{Guevara2022} present a reinforcement learning-based resource allocation technique for minimizing the total execution time of tasks in a fog-cloud environment.
On the other hand, Weikert et al. \cite{Weikert2022} propose an algorithm for task allocation in an IoT platform, aiming to optimize the overall latency.
Furthermore, Lai et al. \cite{Lai2022} propose an online Lyapunov optimization-based method to tackle the problem of allocating user tasks in an edge computing environment, utilizing a stochastic approach. 
Barijough et al. \cite{Barijough2019} introduce a technique for allocating real-time streaming applications under latency and quality constraints.  
Tang et al. \cite{Tang2022} propose a framework for managing the physical resources of the edge and cloud layers, so that the response time is minimized and the system throughput is improved.
Moreover, Kuang et al. \cite{Kuang2021} present an iterative algorithm based on Lagrangian dual decomposition in order to minimize latency in an edge computing system. 


\subsection{Energy consumption minimization}
Several studies are focused on task allocation strategies aiming to reduce the total energy consumption.
Specifically, Avgeris et al. \cite{Avgeris2022} propose a resource allocation technique based on mixed integer linear programming in order to minimize the energy consumption of edge servers.
Within this context, Khalil et al. \cite{Khalil2018} present a framework for energy-efficient task allocation in an IoT environment, utilizing evolutionary-based meta-heuristics. 
Cui et al. \cite{Kritikakou2023} propose a heuristic algorithm for minimizing the total energy consumption of a platform comprising homogeneous processors, utilizing dynamic voltage and frequency scaling (DVFS).  
On the other hand, Hu et al. \cite{Hu2020} introduce a game-theoretic approach for task allocation in an edge computing environment to minimize the system energy consumption within an acceptable delay range.
Similarly, Azizi et al. \cite{Azizi2022} propose two priority-aware semi-greedy algorithms for allocating  IoT tasks in a heterogeneous fog platform, so that the total energy consumption is optimized, while meeting the deadline of each task.
Furthermore, Li et al. \cite{Li2022} examine a two-stage iterative algorithm, in which the resource allocation problem is decomposed into two sub-problems to obtain a suboptimal solution. 




\subsection{Latency and energy consumption minimization}
On the other hand, certain related works consider both optimization objectives, the minimization of latency and energy consumption.
For instance, Zhang et al. \cite{Zhang2021} present a game theory-based scheme for task allocation in a UAV-assisted edge computing environment. The goal of the proposed approach is to minimize the weighted latency and energy consumption of the system, considering resource allocation constraints.
Dinh et al. \cite{Dinh2017} propose a semi-definite relaxation-based optimization framework for allocating tasks in an edge architecture. The particular framework aims to minimize the total latency of the tasks, as well as the total energy consumption of the system.
On the other hand, Tong et al. \cite{Tong2023} present a latency and energy-aware Stackelberg game-based task allocation strategy, considering an edge device with limited computational resources.


\subsection{Our approach vs. state-of-the-art}
Overall, none of the aforementioned research efforts considers the specific edge/hub/cloud architecture examined in this work. 
Furthermore, some approaches do not take into account the energy required for the execution of the tasks \cite{Alfakih2021, Guevara2022, Weikert2022, Lai2022, Barijough2019, Tang2022} or the energy consumed for inter-task communication \cite{Alfakih2021, Guevara2022, Lai2022, Tang2022, Avgeris2022, Kritikakou2023, Azizi2022}. 
The majority of the related studies consider devices with unlimited resources, such as memory \cite{Barijough2019, Tang2022, Kuang2021, Avgeris2022, Khalil2018, Kritikakou2023, Hu2020, Azizi2022, Li2022} and storage \cite{Weikert2022, Barijough2019, Kuang2021, Avgeris2022, Khalil2018, Kritikakou2023, Hu2020, Azizi2022, Li2022}, an assumption that is not realistic, especially in the case of resource-limited devices at the edge of the network.     
Moreover, several approaches are only applicable to single-task applications \cite{Alfakih2021, Lai2022, Tang2022, Kuang2021, Avgeris2022, Khalil2018, Hu2020, Azizi2022, Li2022} or cannot provide an optimal solution to each of the objectives considered in this work \cite{Alfakih2021, Guevara2022, Weikert2022, Lai2022, Kuang2021, Khalil2018, Kritikakou2023, Hu2020, Azizi2022, Li2022}.

Related studies that are closer to ours \cite{Zhang2021, Dinh2017, Tong2023}, even though they consider both the latency and energy aspects of the problem, do not take into account the memory and storage limitations of the devices. Furthermore, they cannot be applied to applications with precedence relationships among their tasks, and can only provide suboptimal solutions.
Hence, our proposed approach aims to fill these gaps, by incorporating all of the important parameters that characterize an edge/hub/cloud environment, providing an optimal allocation for a task flow graph. 


\begin{figure*}[t]
    \centering
    \includegraphics[width=.85\textwidth]{coins_journal_tfg_to_etfg_v6.1.1.pdf}
    \caption{Overview of proposed optimization framework. The task flow graph transformation, including the encapsulated energy model, is described in \cref{extended,subsec:energyModel}. The formulation of the optimization problem is presented in \cref{subsec:optimization}.}
    \label{flow}
    %\vspace{-3mm}
\end{figure*}


\subsection{Comparison with our preliminary research}
The foundational concepts of this work were first presented in a preliminary form in \cite{Kouloumpris2019}.
Below, we outline the main differences and contributions of the current study with respect to our preliminary research:
\begin{enumerate}
    \item We streamlined and enhanced the mathematical representation of all aspects of the proposed approach, from the description of the task flow graph transformation to the modeling of the optimization problem.
    
    \item We extended our optimization framework to consider a new objective for the minimization of overall energy consumption (in addition to the latency objective), based on an improved energy model.
     
    \item We developed suitable synthetic benchmarks to further validate and evaluate the efficiency and scalability of our framework, by extending our transformation method to randomly generated task flow graphs.
    
    \item We conducted extensive experimentation with alternative configurations of different devices, for both the real-world use-case scenario and the synthetic benchmarks.  
\end{enumerate}


\section{Other Related Work}
\label{sec:related}
There has been a renewed interest in integrated task and motion planning algorithms. Most research in this direction has been focused on deterministic environments~\citep{cambon09_asymov,plaku10_sampling,dornhege12_semantic,kaelbling11_hierarchical,garrett15_ffrob,dantam16_incremental}. \cite{kaelbling13_hpnPOMDP} consider  a partially observable formulation of the problem. Their approach utilizes regression modules on belief fluents to develop a regression-based solution algorithm. \cite{sucan12_tmp_mdp} use an explicit multigraph to represent the plan or policy for which motion planning refinements are desired.  \cite{hadfield15_modular} address problems where the high-level formulation is deterministic and the low-level is determinized using most likely observations. In contrast, our approach employs abstraction to bridge MDP solvers and motion planners to solve problems where the high-level model is stochastic. In addition, the transitions in our MDP formulation depend on properties of the refined motion planning trajectories (e.g., battery usage). 

Principles of abstraction in MDPs have been well studied~\citep{hostetler14_state,bai16_markovian,li06_abstractMDP,singh95_abstractRL}. However, these directions of work assume that the full, unabstracted MDP can be efficiently expressed as a discrete MDP. \cite{marecki06_cmdp} consider continuous time MDPs with finite sets of states and actions. In contrast, our focus is on MDPs with high-dimensional, uncountable state and action spaces. Recent work on deep reinforcement learning  (e.g., \citep{hausknecht16_iclr,mnih15_drl}) presents  approaches for using deep neural networks in conjunction  with reinforcement learning to solve MDPs with continuous state spaces. We believe that these approaches can be used in a complementary fashion with our proposed approach. They could be used to learn maneuvers spanning shorter-time horizons, while our approach could be used to efficiently abstract their representations and to use them as actions or macros in longer-horizon tasks. 

Efforts towards improved representation languages are orthogonal to our contributions~\citep{fox02_pddl+}. The fundamental computational complexity results indicating growth in complexity with increasing sizes of state spaces, branching factors, and time horizons remain true regardless of the solution approach taken. It is unlikely that a uniformly precise model, a simulator at the level of precision of individual atoms, or even circuit diagrams of every component used by the agent will help it solve the kind of complex tasks on which humans would appreciate assistance. On the other hand, not using any model at all would result in dangerous agents that would not be able to safely evaluate the possible outcomes of their actions. Our results show that these divides can be bridged using hierarchical modeling and solution approaches that simplify the representational requirements and offer computational advantages that could make autonomous robots feasible in the real world. 



% ~\newpage ~ \newpage

\section{Formal Framework}
\label{sec:formal}


% \subsection{Stoachstic Task and Motion Planning Problem}
% \subsection{Problem Statement}

% The main contribution of the paper is a probabilistically complete approach that computes task and motion policies for stochastic task and motion planning problems. We define the stochastic task and motion planning problem (STAMPP) as follows: 
% \begin{definition}
%     A stochastic task and motion planning problem (STAMPP) is defined as triplet $\langle \M, \alpha, \abs{\M}   \rangle$ where $\M$ is a low-level continuous stochastic shortest path (SSP) problem, $\alpha$ is an abstraction function, and $\abs{\M}$ is an abstract stochastic shortest path problem computed by applying the abstraction function $\alpha$ on the low-level SSP problem $\M$. 
% \end{definition}


% A solution of stochastic task and motion planning problem is a policy with actions from the concrete low-level continuous $SSP$ $\M$. Now, we define each component in the STAMPP triplet.


\subsection{Stochastic Task and Motion Planning Problem}
\label{sec:definition}

The main contribution of the paper is a probabilistically complete approach that computes task and motion policies for stochastic task and motion planning problems. We define the stochastic task and motion planning (STAMP) problem  as follows: 
\begin{definition}
    A stochastic task and motion planning (STAMP) problem  is defined as triplet $\langle \M, \alpha, \abs{\M}   \rangle$ where $\M$ is a low-level continuous stochastic shortest path (SSP) problem with $|\Am_\text{mp}| > 0$, $\alpha$ is an abstraction function, and $\abs{\M}$ is an abstract stochastic shortest path problem computed by applying the abstraction function $\alpha$ on the low-level SSP problem $\M$. 
\end{definition}

A solution for a stochastic task and motion planning (STAMP) problem  is a policy with actions from the concrete model $\M$. In this work, we consider solutions in the form of  a policy tree where each node $u_p$ in the tree represents a  state $s_{u_p}$ and an edge $e_p$ represents an action  $a_{e_p}$. The child of a node-edge pair $(u_p,e_p)$ in the policy tree refers to a possible outcome of executing the action $a_{e_p}$ at the state $s_{u_p}$. In the case of all deterministic actions, the tree would have a single branch. Now, we define the specific  \emph{entity abstraction} that we use to define the STAMP problem.



\subsection{Entity Abstraction}
\label{sec:entity}
In this paper, we use entity abstraction to define a stochastic task and motion planning problem. We define entity abstraction by extending the notion of abstractions introduced in Sec.~\ref{subsec:abstraction} as follows: Let $\mathcal{U}_l\;(\mathcal{U}_h)$ be the universe of $V_l\;(V_h)$ such that $|\mathcal{U}_h| \leq |\mathcal{U}_l|$. Let $\rho \; : \; \mathcal{U}_h \rightarrow 2^{\mathcal{U}_l}$ be a collection function that maps elements in $\mathcal{U}_h$ to the collection of $\mathcal{U}_l$ elements that they represent, e.g., $\rho(Table) = \{ loc \; : \; \land_{i}\; loc \cdot BoundaryVector_i < 0 \}$. Here $\rho$ binds Table $\in \mathcal{U}_h$ to a set of locations in $\mathcal{U}_l$ that are bounded by some polygonal boundary. Here $\mathcal{U}_l$ and $\mathcal{V}_l$ are low-level concrete universe and vocabulary and $\mathcal{U}_h$ and $\mathcal{V}_h$ are their abstract counterparts.


We define \emph{entity abstraction} $\alpha_\rho$ using the collection function $\rho$ as $\llbracket r \rrbracket _{\alpha_\rho(V_l)}(\tilde{o_1},\dots,\tilde{o_n}) = True $ iff $\exists\,o_1,\dots,o_n$ such that $o_i \in \rho(\tilde{o_i})$ and $\llbracket \psi_{r}^{\alpha_{\rho}}(o_1,\dots,o_n) \rrbracket _{S_l} = True$. We omit the subscript $\rho$ when it is clear from the context. Entity abstractions define the truth values of predicates over abstracted entities as disjunction of the corresponding concrete predicate instantiations. E.g., an object is in the abstract region ``kitchen'' if it is at one of the any locations in that region and an object is on ``table'' if it is at any location on the table-top. Such abstractions have been used for efficient generalized planning~\cite{srivastava2008AAAI} as well as answer set programming~\cite{zeynep18_aspocp}. These type of abstractions introduce terms that may not be identifiable at high level which makes these abstractions lossy and high-level models obtained by these abstractions inaccurate. E.g., the exact location of the table, the trajectory used to reach a configuration from current configuration.

Now, we use entity abstraction to define an abstract hybrid predicate for each hybrid predicate in our vocabulary by replacing each continuous argument in the hybrid predicate with its symbolic reference. E.g., \emph{$\abs{at}$($o_1$,$\overline{loc}$)} is an abstract hybrid predicate corresponding to a hybrid predicate \emph{at($o_1$,{loc})} where, \emph{$\overline{loc} \in \mathcal{U}$} is a symbolic reference of type $\tau_O$ for the continuous vector \emph{loc}. 

To formally define an abstract hybrid predicate, let $\alpha$ be a composition of entity abstraction and function abstraction. The abstract version of a concrete predicate $p_h$ is denoted as $\abs{p_h}_{\alpha}$. We omit the subscript $\alpha$ when it is clear from the context. We define $\abs{p_h}$ as follows:  
\begin{definition}
    A predicate $\abs{p_h}_\alpha(y_1,\dots,y_k,\bar{\theta}_1,\dots,\bar{\theta}_m)$ is an \textbf{abstract hybrid predicate} corresponding to a concrete hybrid predicate $p_h(y_1,\dots,y_k,\theta_1,\dots,\theta_m)$ iff all of its arguments $y_1,\dots,y_k,\bar{\theta}_1,\dots,\bar{\theta}_m$ are variables of type $\tau_O$ and $\forall\,\bar{\theta}_i \in \emph{arg}(\abs{p_h}_{\alpha})\, \theta_i \in \rho(\overline{\theta_i})$. $\abs{\mathcal{P}_h}_{\alpha}$ is a set of all abstract hybrid predicates.  
\end{definition}


We also define an abstract hybrid action for each hybrid action in the model using the abstraction $\alpha$. The abstraction $\alpha$ replaces each action argument of type $\tau_R$ with its symbolic reference of type $\tau_O$ and each concrete hybrid predicate in its precondition and effect with its abstract counterpart. Finally, we use these concepts to define an abstract SSP as follows: 



\begin{definition}
    Given a concrete planning problem $\mathcal{M}$, an \textbf{abstract planning} problem $\abs{\mathcal{M}} = \langle \mathcal{O}, \abs{\mathcal{P}}, \abs{\mathcal{\Sm}}, \abs{\mathcal{A}}, T, C \abs{s_0}, \abs{S_g}, \gamma, H \rangle$, where, 

    \begin{itemize}
        \item $O$ is a set of names for the objects in the environment and symbolic references for entities in the environment, 
        \item $\abs{\mathcal{P}} = \mathcal{P}_{sym} \cup \abs{\mathcal{P}_h}$ is a set of abstract predicates,
        \item $\abs{\Sm}$ is a set of abstract states,
        \item $\abs{\mathcal{A}} = \mathcal{A}_{sym} \cup \abs{\mathcal{A}_{h}}$ is a set of abstract actions available to the robot, 
        \item $T: \abs{\Sm} \times \abs{\Am} \times \abs{\Sm} \rightarrow [0,1]$ is a transition function,
        \item $C: \abs{\Sm} \times \abs{\mathcal{A}} \rightarrow \mathbb{R}$ is a cost function,
        \item $\abs{s_0} \in \abs{\Sm}$ is the initial state,
        \item $\abs{S_g} \subset \abs{\Sm}$ is the set of goal states,
        \item $\gamma = 1$ is a discount factor (fixed),
        \item $H$ is a fixed horizon. 
    \end{itemize}

\end{definition}

A solution to an abstract planning problem is a valid sequence of actions $ \abs{\pi}_{\alpha} = \langle \abs{a_0},\dots, \abs{a_n} \rangle$ such that each action in $\abs{\pi}$, when applied sequentially from the initial state $\abs{s_0}$, the system reaches one of the goal states in $\abs{S_g}$.



\begin{figure}[t!]
  % \begin{small}
      \noindent \emph{Place($obj_1$, $config_1$, $config_2$, $target\_pose$, $traj_1$)} \\
      \begin{tabular}{rl}
          \emph{precon} & \emph{RobotAt($config_1$)} , \emph{holding($obj_1$)}, \\ 
                          & \emph{IsValidMP($traj_1$, $config_1$, $config_2$)}, \\
                          & \emph{IsCollisionFree($traj_1$)}, \\ 
                          & \emph{IsPlacementConfig($obj_1$,$config_2$,$target\_pose$)} \\ \\ 
          \emph{Concrete} & $\lnot$\emph{holding($obj_1$)}, \\
          \emph{effect}  & $\forall$ \emph{traj intersects(vol(obj, target\_pose))}, \\
                          & \emph{sweptVol(robot,traj)} $\rightarrow$ \emph{Collision($obj_1$,traj)}, \\
                          & \emph{RobotAt($config_2$)}, \emph{at($obj_1$,target\_pose)}  \\ \\ 
          \emph{Abstract} & $\lnot$ \emph{holding($obj_1$)}, \\
          \emph{effect}   & $\forall \; traj$ \? \emph{Collision($obj_1$,$traj_1$)}, \\
                          & $\lnot$\emph{RobotAt($config_1$)}, \emph{RobotAt($config_2$)}, \\
                          & \emph{at($obj_1$,target\_pose)} \\ 
      \end{tabular}
  % \end{small}
  \caption{Specification of concrete (above) and abstract(below) effects of a one-handed robot's action for placing an object}
  \label[fig]{abs_example1}
\end{figure}



Concretization operation is performed by replacing abstract symbolic references with concrete objects from their low-level domains. For instance, let $\Sm_h$ be the set of abstract states generated when an abstraction
$\alpha$ is applied on a set of concrete states $\Sm_l$. For any
$s_h \in \Sm_h$, the \emph{concretization function}
$\Gamma_\alpha(s_h) = \set{s_l \in \Sm_l: \alpha(s_l)=s_h}$ denotes the set of
concrete states represented by the abstract state $s$. Similarly, abstract hybrid actions are refined by grounding abstract entities using values from their low-level domains. But, generating the complete concretization of an
abstract state can be computationally intractable, especially in cases
where the concrete state space is continuous. In such situations, the
concretization operation can be implemented as a \emph{generator} that
incrementally samples elements from an abstract argument's concrete
domain. A generator can also be designed in way that it validates the generated values while generating them and only yield valid instantiations for the symbolic arguments.



\paragraph{\textbf{Example}}  Consider the specification of a robot's action of placing an item as a part of an SSP. In practice, low-level accurate models of such actions may be expressed as generative models or simulators. Fig. \ref{abs_example1} helps to identify the nature of abstract representations needed for expressing such actions. For readability, we use a convention where preconditions are comma-separated conjunctive lists and universal quantifiers represent conjunctions over the quantified variables. 

Fig. \ref{abs_example1} shows the specification of an action that places an object at the specified pose. Concrete description of the action requires action arguments representing object to be placed (\emph{obj}$_1$), the initial and final configuration of the robot (\emph{config}$_1$, \emph{config}$_2$), target pose for the object (\emph{target\_pose}), and the motion trajectory that takes the robot from its initial configuration to final configuration (\emph{traj}$_1$). Here \emph{obj$_1$} is an argument of type $\tau_O$ and \emph{config}$_1$, \emph{config}$_2$, \emph{target\_pose}, and \emph{traj}$_1$ are continuous 0arguments of type $\tau_R$. The abstract counterpart of this action is computed by replacing the continuous arguments of type $\tau_R$ in the concrete version with symbolic arguments representing abstract entities as mentioned earlier. E.g., \emph{target\_pose} in the abstract specification is a symbolic reference for all valid target poses for the object and \emph{traj}$_1$ is a reference for all valid motion trajectories that take the robot from \emph{config}$_1$ to \emph{config}$_2$. Values of these arguments can not be determined precisely in the abstracted space and thus a subset of preconditions and effects can not be evaluated while planning with abstract models.  E.g., it is not possible to determine whether a trajectory is collision-free as part of the precondition. Similarly, it is also not possible to determine what trajectories will be in a collision when an object is placed at a certain pose in the abstract model. Such predicates are annotated in the set of effects with the symbol \?. While computing abstractions in such a way loses important information, the abstract model is still sound~\cite{srivastava14_tmp,srivastava2016metaphysics}. 

\begin{figure}[t!]
  \begin{center}
    \includegraphics[width=0.5\columnwidth]{./new_prn.png}
  \end{center}
  \caption{Plan refinement graph (PRG) used to maintain
separate abstract models. Each plan refinement node (PRN) contains an abstract
model, partially refined policy, and current state of refinement. Each edge contains refinement for a partial policy ($\sigma_{ij}$) and a failure reason ($p_k$). }
  \label{fig:prg}
\end{figure}




\begin{algorithm}[t!]
  % \begin{small}
    \KwIn{model $\M$, abstraction function $\alpha$, concretization function $\gamma$,  abstract model $\abs{\M}_{\alpha}$, symbolic planner $P$}
    \KwOut{anytime, contingent policy that is executable in $\M$ }
    Initialize PRG with a node with an abstract policy $\abs{\pi}$ for $\mathcal{G}$
    computed using \emph{P}\;
    \While{solution of desired quality not found}
    {
      $u$ $\gets$ GetPRNode()\;
      $\abs{\M}_u$ $\gets$ GetAbstractModel($u$)\;
      $\abs{\pi}_u$ $\gets$ GetAbstractPolicy($\abs{\M}_u$, $\mathcal{G}$, $P$, $u$)\;
      Choice $\gets$ NDChoice\{\emph{RefinePolicy}, \emph{RefineAbstraction}\}\;
      \If{Choice = \emph{RefinePolicy}}{
        \While{$\abs{\pi}_u$ has an unrefined RTL path and resource limit is not
          reached}{
            $path$ $\gets$ GetUnrefinedRTLPath($\abs{\pi}_u$)\;
          \If{explore\tcp{non-deterministic}}{
            replace a suffix of refined partial $path$ with a random action\;
          }
          Search for a feasible concretization of $path$\;
        }
      }
      \If{Choice = \emph{RefineAbstraction}}{
        $path$ $\gets$ GetUnrefinedRTLPath($\abs{\pi}_u$)\;
        $\sigma \gets$ ConcretizeFirstUnrefinedAction($path$)\;
        failure\_reason $\gets$ GetFailedPrecondition($\sigma$
        )\;
        $\abs{\M'}$ $\gets$ UpdateAbstraction($\abs{\M}$, failure\_reason) \;
        % random\_action $\gets$ choose\_random\_action(${\cal D}$)\;
        % next\_state $\gets$ apply\_action(random\_action)\;
        $\abs{\pi'}$ $\gets$ merge($\abs{\pi}$, GetAbstractPolicy($\abs{\M'}$, $\mathcal{G}$, solver))\;
        generate\_new\_pr\_node($\abs{\pi'}$, $\abs{\M'}$)\;
      }
      recompute  $p/c$ ratio for unrefined RTL paths\;
    }
  % \end{small}
\caption{\small HPlan Algorithm}
\label{alg:atam}
\end{algorithm}



Refining (instantiating) the abstract \emph{place} action sampling concrete values for each symbolic abstract entity in its arguments from their low-level domain. E.g., refining the symbolic entity \emph{target\_pose} would require using a generative model such as a simulator to sample a valid pose for the object being placed and computing a valid motion plan that takes the robot from its current configuration to a configuration that places the object at the sampled pose. This can be implemented using a backtracking search that tries to instantiate abstract entities in a sequential order while evaluating concrete preconditions for the instantiations. 



% \subsection{Stochastic Task and Motion Planning Problem}
% \label{sec:definition}


% The robot may require to change its pose as part of executing some actions which which indeed requires it to have an explicit motion plan. E.g., to execute the action \emph{Place(obj$_1$, pd\_pose, traj$_1$)}, the robot must have a valid trajectory that changes robot's pose to \emph{pd\_pose} in order to place the object \emph{obj}$_1$. We define such actions as \emph{motion planning actions}. Action arguments for such motion planning actions specify trajectories required to execute these actions and preconditions can be used to specify constraints on these motion planning trajectories. Values for these motion planning arguments can be \emph{``sampled''} using a motion planner. E.g, the action \emph{place} (Fig. \ref{abs_example1}) contains a motion planning argument \emph{traj}$_1$ and its precondition specifies a constraint that it should be a valid collision-free trajectory (\emph{IsCollisionFree(traj$_1$)}). We formally define motion planning actions as follows:
% \begin{definition}
% A \textbf{motion planning action} $a_{mp}(o_1,\dots,o_k,\theta_1,\dots,\theta_j,t_1,\dots,t_n)$ is a hybrid action where $o_1,\dots,o_k$ are of type $\tau_o$, $\theta_1,\dots,\theta_j$ are of type $\tau_R$, and $t_1,\dots,t_n$ are motion planning trajectories. \emph{pre(a$_{mp}$)} contains constraints on $t_1,\dots,t_n$ and \emph{eff(a$_{mp})$} represents the effective pose of the robot after executing action $a_{mp}$. $\mathcal{A}_{mp} \subset \mathcal{A}_h$ is the set of all motion planning actions.
% \end{definition}  

% We use these components to define concrete and abstract planning problems as follows:


% \begin{definition}
%     A \textbf{concrete planning} problem $P$ is defined as a $6$-tuple $ \mathcal{M} = \langle \mathcal{O}^\mathcal{M}, \mathcal{P}^{\mathcal{M}}, \mathcal{X}^\mathcal{M},  \mathcal{A}^\mathcal{M}, T^{\mathcal{M}}, C, x_0, X_g, \gamma, H \rangle$, where, 


% \begin{itemize}
%     \item $\mathcal{O}^\mathcal{M}$ is a set of names for the objects in the environment,
%     \item $\mathcal{P}^\mathcal{M} = \mathcal{P}_{sym} \cup \mathcal{P}_{h}$ is a set of predicates,
%     \item $\mathcal{X}^\mathcal{M}$ is a set of states defined using predicates in $\mathcal{P}^\mathcal{M}$,
%     \item $\mathcal{A}^\mathcal{M} = \mathcal{A}_{sym} \cup \mathcal{A}_{h}$ is a set of actions available to the robot, where $\mathcal{A}_{mp} \subset \mathcal{A}_{h}$ is a set of motion planning actions,
%     \item $T: \mathcal{X} \times \mathcal{A} \times \mathcal{X} \rightarrow [0,1]$ is a transition function,
%     \item $C: \mathcal{X} \times \mathcal{A} \rightarrow \mathbb{R}$ is a cost function,
%     \item $x_0 \in \mathcal{X}^\mathcal{M}$ is the initial state,
%     \item $X_g \subset \mathcal{X}^\mathcal{M}$ is the set of goal or terminal states,
%     \item $\gamma = 1$ is the discount factor,
%     \item $H$ is the horizon. 
% \end{itemize}
% \end{definition}

% For ease of reading, we omit the superscript when it is clear from the context. The solution to a concrete planning problem is a valid sequence of actions $\pi = \langle a_0,\dots,a_n \rangle$ such that every action when applied sequentially from the initial state $x_0$, the system reaches one of the goal states in $X_g$. 




% The main contribution of the paper is a probabilistically complete approach that computes task and motion policies for stochastic task and motion planning problems. We define the stochastic task and motion planning problem (STAMPP) as follows: 
% \begin{definition}
%     A stochastic task and motion planning problem (STAMPP) is defined as triplet $\langle \M, \alpha, \abs{\M}   \rangle$ where $\M$ is a low-level continuous stochastic shortest path (SSP) problem with $|\Am_\text{mp}| > 0$, $\alpha$ is an abstraction function, and $\abs{\M}$ is an abstract stochastic shortest path problem computed by applyting the abstraction function $\alpha$ on the low-level SSP problem $\M$. 
% \end{definition}


  




% With this, we define stochastic task and motion planning problems as follows: 
% \begin{definition}

%     A \textbf{stochastic task and motion planning problem}  (\emph{STAMPP}) can be defined as a $3$-tuple $\langle \M$, $\alpha$, $\abs{\M}_{\alpha} \rangle$, where $\M$ is a concrete \emph{SSP} with $|\mathcal{A}^{\M}_{mp}| > 0$, $\alpha$ is an abstraction function that is a composition of entity abstraction and function abstraction, and $\abs{\M}_{\alpha}$ is an abstract model computed by applying $\alpha$ on the concrete model $\M$.
% \end{definition}





% A solution for a stochastic task and motion planning problem (STAMPP) is a policy with actions from the concrete model $\M$. In this work, we consider solutions in the form of  a policy tree where each node $u_p$ in the tree represents a  state $s_{u_p}$ and an edge $e_p$ represents an action  $a_{e_p}$. The child of a node-edge pair $(u_p,e_p)$ in the policy tree refers to a possible outcome of executing the action $a_{e_p}$ at the state $s_{u_p}$. In the case of all deterministic actions (\emph{TAMPP}), the tree would have a single branch. 



% \newpage
% ~ \newpage
\section{Computing Task and Motion Policies}
\label{sec:algo}


\subsection{HPlan Algorithm}
\label{sec:atm}
We extend the idea of planning with abstractions briefly discussed by \cite{srivastava2016metaphysics} to perform task and motion planning in stochastic environments using abstraction hierarchies. The goal is to find a valid high-level policy that also has valid low-level refinements for each of its actions. We propose the HPlan algorithm (Alg. \ref{alg:atam}) that performs hierarchical planning with arbitrary abstraction and concretization function.






% In this section, we describe our approach for computing task and motion policies as defined above. Remember that every abstract action $[a] \in [\mathcal{M}]$  (e.g., \emph{Place($obj_1$,$config_1$,$config_2$,$pose_1$,$traj_1$)}) has symbolic arguments that has to be instantiated to generate a concrete action $a$, that can be executed in the concrete model $\mathcal{M}$. The goal is to compute instantiations for each action in the high-level solution. 

HPlan (Alg.~\ref{alg:atam}) uses a policy refinement graph (PRG) to keep track of different abstract models and their corresponding policies. As shown in Fig.~\ref{fig:prg}, each node $u$ in a PRG contains an abstract model $\abs{\M}_{u}$, an abstract policy $\abs{\pi}_u$, and the current state of refinement for each action $\abs{a_j} \in \abs{\pi}_u$. An edge $(u,v)$ in a PRG from a node $u$ to a node $v$ consists of a partial refinement of the policy ($\sigma_{uv}$) and a failed precondition of the first action from $\abs{\pi}_u$ that does not have a valid motion planning refinement. Our approach combines two processes: $1)$ Concretizing the abstract policy, and $2)$ refining the abstract model.



HPlan (Alg.~\ref{alg:atam}) performs the above-mentioned two steps in an interleaved manner. The algorithm starts by 
% a single policy refinement node (PRN) in the PRG with an initially provided abstract model. Line $1$ 
initializing the PRG with a node containing this abstract model $\abs{\M}$, and an abstract policy $\abs{\pi}$ computed using an off-the-shelf symbolic solver that achieves the goal $\mathcal{G}$ (line $1$). Each iteration of the main loop (line $2$) selects a policy refinement node (PRN) $u$ from the PRG using a defined strategy (line $3$). 
%  and extracts a root-to-leaf (RTL) path from the current PRG node's policy $\abs{\pi}_u$ such that the path has at least one action that has not been instantiated (line $3$-$5$). 
 Arbitrary strategies can be used to make this selection. HPlan uses an off-the-shelf task planner to compute a high-level policy for the current abstract model if the selected PRN does not already have a high-level policy (line $5$). Once a policy is computed (or obtained), HPlan non-deterministically decides (line $6$) to either refine the high-level policy in the selected PRN by instantiating abstract arguments  of actions in the policy (lines $7$-$13$) or to update the high-level abstractions to compute accurate high-level policies (lines $14$-$20$). The algorithm carries out these interleaved steps in as follows: 

\paragraph{\textbf{a) Concretizing the Abstract Policy}}  
Lines \emph{8-13} search for a valid concretization (refinement) of the high-level policy selected/computed on line $5$ by concretizing the abstract actions with actions from the concrete domain $\mathcal{M}$ using the concretization function $\Tau_{\alpha}$ as explained in Sec.~\ref{sec:entity}. To refine a high-level policy, a root-to-leaf (RTL) path is selected that has at least one action that has not been refined. Each unrefined action is concretized using a local backtracking search (line $13$)~\cite{srivastava14_tmp}. A concretization $c_0, a_1, c_1, \ldots, a_k, c_k$ is a valid concretization of an RTL path $\abs{s_0}, \abs{a_1}, \abs{s_1}, \ldots, \abs{a_k}, \abs{s_k}$ is valid iff $c_{i+1} \in a_{i+1}(c_i)$ and $c_i \models precon(a_i+1)$ for $i = 0, \ldots, k -1$. A policy is refined when concretization for each action in every RTL path in the policy is computed. However, due to lossy nature of the abstraction, it may be possible that no valid concretization exists for the policy $\abs{\pi}_u$. For example, consider an abstraction which drops \emph{InCollision} predicate that checks whether a trajectory is in collision with some object or not from an action that places an object at a desired pose. Such high-level actions would not have any valid concretization if all the trajectories are being obstructed by some object in the low level. 




\paragraph{\textbf{b) Refining the Abstract Model}} 
Lines \emph{15-20} fix a concretization for the partially refined policy selected on line $5$ and identify the earliest abstract state in the selected policy whose subsequent action's concretization is infeasible. The abstract model is refined by adding the true form of the violated precondition at the low level. Continuing the same example, if all the trajectories from the current state to the state that has the object at the desired pose are in a collision with some other object $obj_x$, the concrete precondition \emph{InCollision(traj, $obj_x$)} is violated at the concrete level and is added to the current abstract model. The rest of the policy after this abstract state is discarded.  Lines \emph{19-20} use the new model to compute a new policy. The symbolic planner is invoked to compute a new policy from the updated state; its solution policy is unrolled as a tree of bounded depth and appended to the partially refined path. This allows the time horizon of the policy to be increased dynamically.


  






   \begin{theorem}
    \label{thm:pc}
    If there exists a proper policy that reaches the goal within
    horizon $h$ -{}- i.e. the probability of reaching the goal is $1.0$ -{}- and has feasible
    low-level concretization for each of its actions,  and measure of these refinements under the probability density of the generators is non-zero, then Alg.~\ref{alg:atam} will find it with
    probability $1.0$ in the limit of infinite samples.
  \end{theorem}

  \begin{proof}
    Let $\pi_p$ be the proper policy that achieves the goal with horizon $h$ and has valid low-level concretization for each of its actions. Consider a policy $\pi_i$ inside a PRN $i$ at an intermediate step of Alg.~\ref{alg:atam}; let $k$ denote the minimum depth up to which $\pi_p$ and $\pi_i$ match. Here, $k$ denotes a measure of correctness. When PRN $i$ is selected for refinement, eventually Alg.~\ref{alg:atam} would try to compute low-level concretization for an action at depth $k+1$ that does not match with the proper policy $\pi_p$. In this case, there is a chance that Alg.~\ref{alg:atam} would select the correct action (that matches with $\pi_p$ at depth $k+1$) under the \emph{explore} condition (lines $10$-$12$) of Alg.~\ref{alg:atam} and then generates a plan that reaches the goal state. Finite number of discrete actions in the abstract model and the fixed horizon ensures that in time bounded in expectation, HPLan will generate a policy with the measure of correctness $k+1$ and eventually with the measure of correctness $h$. Once the algorithm finds the policy with the measure of correctness $h$, it stores it in the PRG and is guaranteed to find feasible refinements with probability one if the measure of these refinements under the probability-density of the generators is non-zero.
  \end{proof}
  
  % \begin{proof} 
  %   \textcolor{red}{\emph{(Sketch)} Let $\pi_p$ be a proper policy that reaches the goal withing horizon. Consider a policy
  %   $\pi$ in a \emph{PRG}; let $k$ denote the minimum depth up to which
  %   $\pi_p$ and $\pi$ match. $k$ will be used as a \emph{measure of correctness}. When the plan refinement node containing $\pi$ is selected, suppose we try to refine one of the child nodes of depth $k+1$ in the partial path that had the $k$-length prefix consistent with the correct solution $\pi_p$.  The algorithm selects the correct child action with non-zero probability under the \emph{explore} step (line $11$) and then generates a plan to reach the goal from the resultant state. The finite number of discrete actions and the fixed horizon ensures that in time bounded in expectation, \emph{HPlan} will generate a policy with the measure of correctness $k+1$. Once the algorithm finds the policy with the measure of correctness \emph{h}, it stores it in the PRG and is guaranteed to find feasible refinements with probability one if the measure of these refinements under the probability-density of the generators is non-zero.}
  % \end{proof}


  \begin{figure}[t!]
    \begin{center}
      \includegraphics[width=0.6\columnwidth]{./ptree.eps}
    \end{center}
    \caption{Left: Backtracking from node $B$ invalidates the concretization of subtree rooted at $A$. Right: Replanning from node $B$}
    \label{fig:ptree}
  \end{figure}



 



  %     \hspace{0.6cm}
  %     \begin{subfigure}{0.46\textwidth}
  %       \includegraphics[height=2in,width=1.5in]{./clutter_1.png}
  %       % \hspace{1em}
  %       \includegraphics[height=2in,width=1.5in]{./clutter_2.png}
  %       \caption{Fetch sorts a cluttered table. All the blue cans have to be placed on the left table and all the green cans have to be placed on right table. Red cans act as obstacles.}
  %     \end{subfigure} \\ 
  %     \begin{subfigure}{1\textwidth}
  %       \includegraphics[height=1.5in,width=1.5in]{./kitchen_1.png}
  %       % \hspace{1em}
  %       \includegraphics[height=1.5in,width=1.5in]{./kitchen_3.png}
  %       \caption{Fetch uses STAMP policy to set up a dining table. A tray is available to carry multiple items at a time but carrying more than two items on the tray may break the items.}
  %     \end{subfigure}
  %     \hspace{0.6cm}
  %     \begin{subfigure}{0.46\textwidth}
  %       \includegraphics[height=2in,width=1.5in]{./drawer_1.png}
  %       % \hspace{1em}
  %       \includegraphics[height=2in,width=1.5in]{./drawer_2.png}
  %       \caption{Fetch searches for a can in drawers. The can can be placed in one of the drawers stochastically.}
  %     \end{subfigure} \\ 
  
  %     \caption{Photos from our evaluation using the pysical and simulated robots. Videos are available at: \url{https://aair-lab.github.io/stamp.html}. \textcolor{red}{images are as placeholders. trying to get better images and alignment}}
  %     \label{fig:keva_strucutres}
  %     \label{fig:fetch_exp}
  %     \label{fig:sim_fig}
  %     % \vspace{-10pt}
  % \end{figure*}

% \subsection{HPlan with entity abstraction for TAMP}

% We use \emph{entity abstraction} as defined in section \ref{sec:entity} to define the TAMP problem (section \ref{sec:definition}). As defined in the section \ref{sec:entity}, \emph{entity abstraction} symbolizes the continuous arguments in the action descriptions. All the symbolic arguments need to be instantiated with continuous values from the concrete domain to compute the solution task and motion policies. We use the \emph{HPlan} algorithm to concretize the abstracted entities in the symbolic actions. 

% \emph{HPlan} uses a \emph{generator} for each abstracted entity to instantiate the symbolic arguments as the concretization function $\gamma$. A generator samples a value for the corresponding abstracted argument from its concrete domain.  Line $13$ in Algorithm \ref{alg:atam} performs a local backtracking search to instantiate the arguments using such generators. 
% If it fails to assign a valid concrete value to one of the arguments, the algorithm backtracks to the previous action in the high-level solution or generates a new \emph{PRN} depending on the non-deterministic choice made on line $7$.





\subsection{HPlan for STAMP}
We enhance the basic Alg.~\ref{alg:atam} in two primary directions to facilitate STAMP problems. These optimizations allow Alg.~\ref{alg:atam} to compute anytime solutions for STAMP problems and improve the search of concretization of abstract policies. 



\paragraph{\textbf{Search for Concretizations}} 
Sampling-based backtracking search performed by Alg.~\ref{alg:atam} (line $13$) to concretize the abstract actions suffers from a few limitations in stochastic settings that are not present in the deterministic settings. Fig.~\ref{fig:ptree} illustrates the problem. The gray nodes in the image show the actions which are concretized. White nodes represent actions that are yet to be concretized. Sibling nodes represent the non-deterministic action outcomes. Now, if the action in the node $B$ does not accept any valid concretization, backtracking to node $A$ and changing its action's concretization would invalidate concretization for the entire subtree rooted at node $A$. Alg.~\ref{alg:atam} handles such scenarios by non-deterministically selecting whether to perform backtracking searching or not (line $6$) and by maintaining different abstract models through \emph{PRG} and employing a resource limit (line $8$) to explore them simultaneously. 



\paragraph{\textbf{Anytime Computation for Task and Motion Policies}} 
The main computational challenge for Alg.~\ref{alg:atam} in stochastic settings is that the number
of root-to-leaf (RTL) branches grows exponentially with the time
horizon and the number of contingencies in the domain. In most scenarios, not all contingencies are equally probable. Each RTL path has a certain
probability of being encountered; refining it incurs a computational
cost. Waiting for a complete refinement of the policy tree results in wasting a lot of
time as most of the situations have a very low probability of being encountered.  The optimal selection of the paths to refine within a fixed
computational budget can be reduced to the knapsack
problem. Unfortunately, we do not know the precise
computational costs required to refine an RTL path. However, we can approximate this cost depending on the number of actions in an RTL path and the size of the domains of the arguments of those actions. Furthermore, the knapsack problem is NP-hard. However, we can compute provably good approximate solutions to this problem using a greedy approach: we prioritize the selection of a path to refine based on the probability
of encountering that path \emph{p} and the estimated cost of
refining that path \emph{c}. We compute $p/c$ ratio for all the paths and select the unrefined path with the largest ratio for refinement (line $9$ and $15$). $p/c$ ratio for each path is updated after each iteration of the main loop (line $21$). Intuitively, our approach works as follows:

\begin{figure*}[t]
  \centering
  % \includegraphics[width=\textwidth]{./example1.png}
  \includegraphics[width=0.8\textwidth]{./stamp_image.pdf}
  % \includegraphics[width=0.7\textwidth]{./example1_2.png}
  \caption{A working example for Alg.~\ref{alg:atam}. (a) shows initial environment configuration. Goal for the robot is to pick up the ``Red'' object which is surrounded by ``Blue'', ``Green'', ``Orange'', and ``Black'' objects. G is the end-effector of a robot. (b) shows a high-level, abstract task specification of the ``pick'' action. (c) shows the policy refinement graph (PRG) which is generated incrementally by Alg.~\ref{alg:atam}. Each green box represents a policy refinement node (PRN). Tree in each PRN represents a high-level policy. Each node in a high-level policy is a state-action pair. For brevity, we only show high-level action in the node. Trees with dotted lines are partial policies. Red number represents $p/c$ ratio for each RTL path in a policy. }
  \label{fig:working_example}
  
\end{figure*}



\begin{figure*}[t!]
  \begin{subfigure}{\textwidth}
    \centering
    \vspace{1em}
      % \includegraphics[width=1.5in,height=1.5in]{./fetch_replace.eps}
      \includegraphics[width=1.5in,height=1.5in]{./fetch_new_1.png}
      % \includegraphics[width=1.5in,height=1.5in]{./fetch_crushed.eps}
      \includegraphics[width=1.5in,height=1.5in]{./fetch_new_2.png}
      % \includegraphics[width=1.5in,height=1.5in]{./fetch_target.eps}
      \includegraphics[width=1.5in,height=1.5in]{./fetch_new_3.png}
      % \caption{\footnotesize The Fetch mobile manipulator uses a STAMP policy to
      %   pickup a target bottle while avoiding those that are likely to
      %   be crushed. It replaces a bottle that wasn't crushed (left), 
      %   discards a bottle that was crushed (center) and picks up the target
      %   bottle (right).  }
    \end{subfigure}
    \begin{subfigure}{\textwidth}
      \centering
    \includegraphics[width=1.5in,height=1.5in]{./tower_12.eps}
    \includegraphics[width=1.5in,height=1.5in]{./twisted_12.eps}
    \includegraphics[width=1.5in,height=1.5in]{./3Towers.eps}
    \end{subfigure}  \\
    \caption{
    Top: Cluttered Table: The Fetch mobile manipulator uses a STAMP policy to
      pick up a target bottle while avoiding those that are likely to
      be crushed. It replaces a bottle that wasn't crushed (left), 
      discards a bottle that was crushed (center) and picks up the target
      bottle (right).    
    Bottom: Building Structures with Keva Planks: ABB YuMi builds Keva structures using a STAMP policy:
        12-level tower (left), twisted 12-level tower (center), and
        $3$-towers (right).}
    \label{fig:exp_1_2}
  \end{figure*}
  
  \begin{figure*}

  
    \centering
    \includegraphics[height=0.24\columnwidth]{./hanger_1.png}
    \hspace{-0.5em}
    \includegraphics[height=0.24\columnwidth]{./hangar_2.png}
    \hspace{0.3em}
    \includegraphics[height=0.24\columnwidth]{./drawer_1_new.png}
    \hspace{-0.7em}
    \includegraphics[height=0.24\columnwidth]{./drawer_2_new.png}


  % \begin{subfigure}{0.45\textwidth}
  %   \centering
  %   \includegraphics[width=1.5in]{./hanger_1.png}
  %   % \hspace{1em}
  %   \includegraphics[width=1.5in]{./hangar_2.png}
   
  % \end{subfigure}
  % \begin{subfigure}{0.45\textwidth}
  %   \centering
  %   \includegraphics[width=1in]{./drawer_1_new.png}
  %   % \hspace{1em}
  %   \includegraphics[width=1in]{./drawer_2_new.png}
  % \end{subfigure} \\
  \caption{Top: Aircraft Inspection: UAV inspects faulty parts of an aircraft in an airplane hangar and alerts the human about the location of the fault. UAV's movements and sensors are noisy, so it may drift from its location or fail to locate the fault. Bottom: Find the can: Fetch searches for a can in drawers. The can can be placed in one of the drawers stochastically.}
  \label{fig:exp_3_6}
\end{figure*}


\paragraph{\textbf{Example}} Fig.~\ref{fig:working_example} illustrates our approach for solving a STAMP problem using Alg.~\ref{alg:atam}. Fig.~\ref{fig:working_example}(a) shows a low-level configuration of an environment. Here, a robot with an effector G is asked to pick up the red object which is surrounded by green, blue, orange, and black objects. Fig.~\ref{fig:working_example}(b) shows a high-level specification of the pick action in the PPDDL format. Fig.~\ref{fig:working_example}(c) shows the policy refinement graph (PRG) that is generated incrementally by Alg.~\ref{alg:atam}.

As explained in Sec.~\ref{sec:atm}, Alg.~\ref{alg:atam} starts with a single node in the PRG -{}- in this case, PRN$1$. Initially, PRN$1$ does not have a high-level policy. Alg.~\ref{alg:atam} uses the abstract action descriptions (abstract model $\abs{\M}$) and an off-the-shelf high-level SSP solver to compute a high-level symbolic policy that reaches the abstract goal (line $5$) and computes $p/c$ ratios for each RTL path in this abstract policy. To compute this ratio, we estimate the cost of refining each high-level action as follows: Suppose that the generators used to concretize the pick actions samples four grasp up poses in four cardinal directions to pick up the object and five motion planning trajectories between robot's current configuration to the grasp pose, then the approximate cost of refining this action would $4\times5 = 20$. We use this approximate cost to compute $p/c$ ratios (red numbers in Fig.~\ref{fig:working_example}). The next step for Alg.~\ref{alg:atam} is to non-deterministically decide between refining the computed high-level policy and refining the abstraction. 

Assume Alg.~\ref{alg:atam}  non-deterministically decides to refine the high-level policy (line $6$). After deciding to refine the high-level policy, Alg.~\ref{alg:atam} selects an RTL path using the $p/c$ ratio and tries to refine each action on this path by instantiating each symbolic argument. Here in this example, the first RTL path would only have a single high-level action \emph{pick(Red, gp$_1$, traj$_1$)} that needs refinement. To instantiate the high-level pick action, it first uses a generator to sample one of the possible grasp poses for the red object and then uses a low-level motion planner to generate a trajectory that would take the robot end-effector G to the selected grasp pose from its current pose. As the red object is surrounded by other objects, all the trajectories that take the end-effector to the grasp pose, are in collision with at least one object. This violates the precondition of the pick action making the refinement infeasible. Alg~\ref{alg:atam} continues trying to refine this action using the local and global backtracking search for a fixed amount of time before again making a non-deterministic choice between refining the high-level policy or the high-level abstraction. 




Suppose this time Alg.~\ref{alg:atam} decides to refine the high-level abstraction. To do so, it would identify the failing precondition preventing a valid refinement for the high-level policy and generate a set of child nodes in the PRG -{}- PRN$2$ and PRN$3$ in this case corresponding to failing preconditions \emph{Obstructs(traj$_1$, Blue)} and \emph{Obstructs(traj$_1$, Green)}. Once these nodes are generated, Alg.~\ref{alg:atam} would move on to the next iteration of the approach where it would select one of these newly generated plan refinement nodes and repeat the entire process until a complete task and motion policy is computed. 

\begin{theorem} \label{thm:knapsack} Let $t$ be the time since the start of the algorithm
  at which the refinement of any \emph{RTL} path is completed. If
  path costs are accurate and constant then the total probability of
  unrefined paths at time $t$ is at most $1 - opt(t)/2$, where
  $opt(t)$ is the best possible refinement (in terms of the
  probability of outcomes covered) that could have been achieved in
  time $t$.
\end{theorem}

% \begin{figure*}
  % \begin{subfigure}{\textwidth}
  %   \centering
  %     % \includegraphics[width=3in]{./clutter_1_1.png} 
  %     \includegraphics[width=3in]{./sort_top_1.eps} 
  %     % \hspace{1em}
  %   %  \hspace{1em} \includegraphics[width=3in]{./clutter_2_1.png}
  %    \hspace{1em} \includegraphics[width=3in]{./sort_top_2.eps}
  %   \caption{Fetch sorts a cluttered table. All the blue cans have to be placed on the left table and all the green cans have to be placed on right table. Red cans act as obstacles. Left: The initial state for a problem. Right: The goal state.}
  %   \label{fig:sort}
  %   \end{subfigure} 
 







\begin{proof}
  (\emph{Sketch})
   The proof follows from the fact that the greedy algorithm achieves a
   2-approximation for the knapsack problem. In practice, we estimate the
   cost as $\hat{c}$, the product of measures of the true domains of each
   symbolic argument in the given \emph{RTL}. Since, $\hat{c}\ge c$ modulo
   constant factors, the priority queue never can only underestimate the relative value of refining a path, and the algorithm's coverage of
   high-probability contingencies will be closer to optimal than the
   bound suggested in the theorem above. This optimization gives a user
   the option of starting execution when the desired value of the probability
   of covered contingencies has been reached. 
   \end{proof}














\section{Empirical Evaluation}
\label{sec:empirical}
\begin{figure*}[t]
  \centering
  \includegraphics[width=2.2in]{figures/n05-h10}
  \includegraphics[width=2.2in]{figures/n10-h10}
  \includegraphics[width=2.2in]{figures/n20-h10}
\caption{\small Performance of our anytime algorithm for solving MDPs using
  dynamic abstractions. The plots from left to right corresponds to
  formulation of the problem with 5\%, 10\%, and 20\% rates of failure
  of the abstract actions described in the text. The blue lines (red
  lines) plot the probability mass of possible outcomes (proportion of
  nodes in the policy graph) that is covered by the partially computed
  policy as computation time (x axis, in seconds) evolves. }\label{fig:results_mdp}
\end{figure*}

We implemented the algorithms presented in Sec.\,\ref{sec:alg} using an implementation of LAO*~\citep{hansen01_lao} as the SSP solver. We used the OpenRAVE~\citep{diankov10_openrave} system for modeling and visualizing test environments and its collision checkers and RRT~\citep{lavalle2000rapidly} implementation for motion planning. Since there has been very little research on the task and motion planning problem in stochastic settings, there are no standardized benchmarks.
%with discount factor $\gamma=1$
%Variations among robot platforms and substantial diversity in research approaches and their corresponding input requirements have made it difficult to create benchmarks for task and motion planning even in deterministic settings~\cite{lagriffoul15_benchmarks}.  
We evaluated our algorithms by creating a hangar model in OpenRAVE for the aircraft inspection problem (Fig.\,\ref{fig:scenario}). UAV actions in this domain include actions for moving to various components of the aircraft, such as the left and right wings, nacelles, fuselage, etc. Each such action could result in the UAV reaching the specified component or a region around the component. The inspection action for a component had the stochastic effect of localizing a fault's location. The environment included docking stations that the UAV could reach and recharge on reserve battery power. Generators for concretizing all actions except the inspect action uniformly sampled  poses in the target regions. Some of these poses naturally lead to shorter trajectories and therefore lower battery usage, depending on the UAV's current pose. However, we used uniform-random samples to evaluate the performance of the algorithm while avoiding domain-specific enhancements. The generator for \emph{inspect($s$)} simulated an inspection pattern by randomly sampling five waypoint poses in an envelope around $s$ and ordering them along the medial axis of the component. We used a linear function of the trajectories to keep track of battery usage at the low level and to report insufficient battery as the \emph{failureReason} when infeasibility was detected. This function was used to provide failure reasons to the high-level when the battery level was found to be insufficient.


 Fig.\,\ref{fig:results_mdp} shows the performance of our
 approach for producing execution strategies with motion planning
 refinements as a function of the time for which the algorithm is
 allowed to run. The red lines show the number of nodes in the
 high-level policy that have been evaluated, refined, and potentially
 replaced with updated policies that permit low-level plans. The blue
 lines show the probability with which the policy available at any
 time during the algorithm's computation will be able to handle all
 possible execution-time outcomes. The different plots show how these relations change as we increase the level of uncertainty in the domain. The horizon is fixed at ten high-level decision epochs (each of which can involve arbitrarily long movements) and the number of parts with faults is fixed at two. The policy generated by LAO* is unrolled into a tree prior to the start of refinement. The reported times include the time taken for unrolling.


Our main result is that that our anytime algorithm balances complexity of computing task and motion policies with time very well and produces desirable concave anytime peformance profiles. Fig.\,\ref{fig:results_mdp} shows that when noise in the agent's actuators
and sensors is set at $5\%$, with $10\%$ of computation our algorithm computes an executable
policy that misses only the least likely $10\%$ of the possible execution outcomes. This policy is computed in less
than 10 seconds. In the worst case, with a 20\% error rate in actuators and sensors (sensors used in practice are much more reliable), we miss only about 20\% of the execution trajectories with 40\% of the computation.

 





% ~ \newpage~ \newpage


% \section*{Problem Statement}

Let $\mathcal{O}$ be the set of symbolic references to the objects in the environment. For convenience, we will refer to these references directly as objects. 

Let $\mathcal{P}$ be a set of predicates. In this work, we consider two kinds predicates: \emph{symbolic} and \emph{hybrid}. We define each of them as follows: 
\begin{definition}
    Let $\mathcal{P}_{sym}$ be a set of \textbf{symbolic predicates} defined using a set of objects $o \subseteq \mathcal{O}$ as their arguments. 
\end{definition}

\begin{definition}
    Let $\mathcal{P}_{h}$ be a set of \textbf{concrete hybrid predicates}. Each $p_h(o,\Theta) \in \mathcal{P}_h$ uses a set of vectors $\Theta$ each of which is a vector of continuous values, along with the set of objects $o \subseteq \mathcal{O}$ as its arguments.
\end{definition}

A predicate $p \in \mathcal{P} = \mathcal{P}_{sym} \cup \mathcal{P}_h$ represents either a property of the object or a relation between two or more objects. For e.g., \emph{at($o_1$, loc)} specifies the location of the object $o_1$ where, $\{o_1\} \subset \mathcal{O}$ and \emph{loc} is a \emph{6-D} vector representing the pose of the object $o_1$ in the $3$-D environment, and \emph{on($o_x$, $o_y$)} specifies that object $o_x$ is placed on $o_y$.

Let $\alpha$ be a composition of entity abstraction (sec. \ref{sec:entity}) and function abstraction. For each concrete hybrid predicate $p_h \in \mathcal{P}_h$ we derive an abstract hybrid predicate $\abs{p_h}_{\alpha}$ as follows:  
\begin{definition}
    Let $ \llbracket \mathcal{P}_{h} \rrbracket_{\alpha}$ be a set of \textbf{abstract hybrid predicates}. Each $ \llbracket p_h \rrbracket_{\alpha} (o,\bar{\Theta}) \in \llbracket \mathcal{P}_{h} \rrbracket $ is an abstraction of its corresponding concrete predicate $p_h(o,\Theta) \in \mathcal{P}_h$, defined using a set of objects $o \subset \mathcal{O}$ and $\bar{\Theta}$ such that $\Theta \in \rho(\bar{\Theta})$.   
\end{definition}


We use predicates from $\mathcal{P}$ to define the concrete and abstract states as follows:

\begin{definition}
    A \textbf{concrete state} $x$ is defined as a subset of $\mathcal{P} = \mathcal{P}_{sym} \cup \mathcal{P}_h$. 
    $\mathcal{X}$ is a set of all possible concrete states. 
\end{definition}

\begin{definition}
    An \textbf{abstract state} $\abs{x}_{\alpha}$ is defined as a  subset of $ \llbracket \mathcal{P} \rrbracket_{\alpha} = \mathcal{P}_{sym} \cup \llbracket \mathcal{P}_h \rrbracket_{\alpha}$.  $\abs{\mathcal{X}}_{\alpha}$ is a set of all possible abstract states. 
\end{definition}

We classify actions available to the robot as \emph{symbolic} and \emph{hybrid} actions depending on the type predicates that appear in its actions. We define them as follows:

% \begin{definition}
%     Let $\llbracket \mathcal{X} \rrbracket_{\alpha}$ be the \textbf{abstract state space}. Each state $\llbracket x \rrbracket_{\alpha}$ is defined as a subset of $ \llbracket \mathcal{P} \rrbracket_{\alpha} = \mathcal{P}_{sym} \cup \llbracket \mathcal{P}_h \rrbracket_{\alpha}$.
% \end{definition}


\begin{definition}
    Let $\mathcal{A}_{sym}$ be a set of \textbf{symbolic actions} defined using predicates from $\mathcal{P}_{sym}$ such that $\forall a_{sym} \in \mathcal{A}_{sym},\, \emph{eff}_{a_{sym}} \subset \mathcal{P}_{sym}$.
\end{definition}


\begin{definition}
    Let $\mathcal{A}_h$ be a set of \textbf{concrete hybrid actions} defined using predicates from $\mathcal{P}_{sym} \cup \mathcal{P}_h$ such that $\forall a_{h} \in \mathcal{A}_{h},\, \emph{eff}_{a_{h}} \subset \mathcal{P}_{sym} \cup \mathcal{P}_h$.
\end{definition}
Each hybrid action $a_h \in \mathcal{A}_h$ is indeed 
% Each hybrid action $a_h \in \mathcal{A}_h$ is defined as 
a temporal abstraction of primitive actions that enable use of low-level controllers to manipulate the agent, and can be refined using a motion planner. For simplicity and without the loss of generality, we embed these refinements as parts of action arguments (see Fig. \ref{abs_example1}).


\begin{definition}
    Let $\llbracket \mathcal{A}_h \rrbracket_{\alpha}$ be a set of \textbf{abstract hybrid actions} defined using predicates from $\mathcal{P}_{sym} \cup \llbracket \mathcal{P}_{h} \rrbracket_{\alpha}$ such that $\forall \llbracket a_{h} \rrbracket_{\alpha} \in \llbracket \mathcal{A}_h \rrbracket_{\alpha},\, \emph{eff}_{\llbracket a_{h} \rrbracket_{\alpha}} \subset \mathcal{P}_{sym} \cup \llbracket \mathcal{P}_{h} \rrbracket_{\alpha}$.
\end{definition}

\begin{definition}
    A \textbf{concrete planning} problem $P$ is defined as a $6$-tuple $ P = \langle \mathcal{O}, \mathcal{P} = \mathcal{P}_{sym} \cup \mathcal{P}_h, \mathcal{X},  \mathcal{A} = \mathcal{A}_{sym} \cup \mathcal{A}_h, x_0, X_g \rangle$ where, $\mathcal{O}$ is a set of symbolic references to the objects in the environment, $\mathcal{P}$ is a set of predicates, $\mathcal{X}$ is a set of states, $\mathcal{A}$ is a set of actions available to the agent, $x_0 \in \mathcal{X}$ is an initial state, and $X_g \subset \mathcal{X}$ is a set of goal states.
    
\end{definition}

The solution to a concrete planning problem is a valid sequence of actions $\pi = \langle a_0,\dots,a_n \rangle$ such that $\forall a_i \in \pi,\, a_i \in  \mathcal{A}$ and when applied to $x_0$, the resultant state is $x_n \in X_g$.


\begin{definition}
    An \textbf{abstract planning} problem $\llbracket P \rrbracket_{\alpha}$ is defined as a $6$-tuple $ P = \langle \mathcal{O}, \llbracket \mathcal{P} \rrbracket_{\alpha} = \mathcal{P}_{sym} \cup \llbracket \mathcal{P}_h \rrbracket_{\alpha}, \llbracket \mathcal{X} \rrbracket_{\alpha},  \llbracket \mathcal{A} \rrbracket_{\alpha} = \mathcal{A}_{sym} \cup  \llbracket \mathcal{A}_h \rrbracket_{\alpha}, \llbracket x_0 \rrbracket_{\alpha}, \llbracket X_g \rrbracket_{\alpha} \rangle$ where, $\mathcal{O}$ is a set of symbolic references to the objects in the environment, $ \llbracket \mathcal{P} \rrbracket_{\alpha}$ is a set of abstract predicates, $\llbracket \mathcal{X} \rrbracket_{\alpha}$ is a set of abstract states, $\llbracket \mathcal{A} \rrbracket_{\alpha}$ is a set of abstract actions available to the agent, $\llbracket x_0 \rrbracket_{\alpha} \in \llbracket \mathcal{X} \rrbracket_{\alpha}$ is an initial state, and $\llbracket X_g \rrbracket_{\alpha} \subset \llbracket \mathcal{X} \rrbracket_{\alpha}$ is a set of goal states.
    
\end{definition}

The solution to an abstract planning problem is a valid sequence of actions $ \llbracket \pi \rrbracket_{\alpha} = \langle a_0,\dots,a_n \rangle$ such that  $\forall \llbracket a_i \rrbracket_{\alpha} \in \llbracket \pi \rrbracket_{\alpha},\, \llbracket a_i \rrbracket_{\alpha} \in  \llbracket \mathcal{A} \rrbracket_{\alpha}$ and when applied to $\llbracket x_0 \rrbracket_{\alpha}$, the resultant state is $\llbracket x_n \rrbracket_{\alpha} \in \llbracket X_g \rrbracket_{\alpha}$.

A \emph{concrete SSP} can be defined in a similar fashion with stochastic action descriptions for actions in $\mathcal{A}$.


% A \emph{concrete planning} problem $P$ is defined as a $5$-tuple $ P = \langle \mathcal{P}, \mathcal{X}, \mathcal{O}, \mathcal{A}, X_g \rangle$ where,  $\mathcal{P}$ is a set of predicates, $\mathcal{X}$ define a concrete state space that consists of states defined using the set of predicates $\mathcal{P}$, $\mathcal{O}$ defines a set of symbolic names for the objects in the environment, $\mathcal{A}$ is a set of actions available to the agent, and $X_g$ defines a set of goal states. Each state $x \in \mathcal{X}$ is made up of a subset of the set of predicates $ \mathcal{P}$. $\mathcal{P}$ is a union of sets $\mathcal{P}_{sym}$ and $\mathcal{P}_{h}$ where, $\mathcal{P}_{sym}$ is a set of symbolic predicates that use objects ffrom $\mathcal{O}$ as their arguments, and $\mathcal{P}_h$ is a set of hybrid predicates that use a set of real valued vectors $\theta$ along with symbolic arguments.  $\mathcal{A}$ defines a set of symbolic and hybrid actions defined using predicates from $\mathcal{P}$ and objects from $\mathcal{O}$. While every action $a \in \mathcal{A}$ affects the symbolic predicates in the state, hybrid actions also affects hybrid predicates. Hybrid actions are defined as a temporal abstraction of primitive actions that enable use of low-level controllers to manipulate the agent, and refined using a motion planner. For simplicity and without lose of generality, we embed these refinements as parts of action arguments (see Fig. \ref{abs_example1}). The solution to a concrete planning problem is a valid sequence of actions $[a_0,\dots,a_n]$ such that when applied to $x_0$, the resultant state is $x_n \in X_g$. 
% We use entity abstraction defined in the section \ref{sec:entity} to compute an abstract planning problem $[P]$ by abstracting the predicates with continuous vector $\theta$ using a set of symbolic arguments $[\theta]$. We then define set of abstract actions $[\mathcal{A}]$ with abstracted predicates. Fig. \ref{abs_example1} provides an example of one of the abstract actions where continuous parameters such as \emph{target\_pose} and robot configurations \emph{config$_1$} and \emph{config$_2$} are replaced by corresponding symbolic arguments. 

We define  task and motion planning problems as follows: 
\begin{definition}

    A \textbf{task and motion planning problem}  (\emph{TAMPP}) can be defined as a $5$-tuple $\langle \M$, $\alpha$, $\gamma$, $\abs{\M}_{\alpha} \rangle$ where, $\M$ is a concrete planning problem, $\alpha$ is an abstraction function that is composition of entity abstraction and function abstraction, $\gamma$ is a concretization function in the form of generators, and $\abs{\M}_{\alpha}$ is an abstract model computed by applying $\alpha$ on the concrete model $\M$.
\end{definition}

\begin{definition}

    A \textbf{stochastic task and motion planning problem}  (\emph{STAMPP}) can be defined as a $5$-tuple $\langle \M$, $\alpha$, $\gamma$, $\abs{\M}_{\alpha} \rangle$ where, $\M$ is a concrete \emph{SSP}, $\alpha$ is an abstraction function that is composition of entity abstraction and function abstraction, $\gamma$ is a concretization function in the form of generators, and $\abs{\M}_{\alpha}$ is an abstract model computed by applying $\alpha$ on the concrete model $\M$.
\end{definition}



Solutions to \emph{TAMPPs} or \emph{STAMPPs} are policies with actions from the concrete model $\M$. In this work, We consider solutions in the form of  a policy tree where each node $u_p$ in the tree represents an  state $s_{u_p}$ and edge $e_p$ represents an $a_{e_p}$. The child of a node-edge pair $(u_p,e_p)$ in the policy tree refers to a possible outcome of executing the action $a_{e_p}$ at the state $s_{u_p}$. In the case of all deterministic actions (\emph{TAMPP}), the tree would have a single branch. 


% \section{Conclusions}
In this paper, we set out to address the problem of multi-tasking robots in multi-robot tasks. 
%A fundamental limitation of existing multi-robot systems was addressed by the removal of a restrictive assumption that was often made--robots are single-tasking.
%Our method allowed coalitions to overlap thus enabling multi-tasking robots. 
We observed that the key underlying challenge was to reason about the physical constraints that could be synergistically satisfied.
%which directly affected the feasibility of multi-tasking.
To address the challenge, we developed our method based on the information invariant theory and modeled constraints as information instances. 
%This allowed us to reason about the relationships between constraints by reasoning about those between information requirements. 
Thereby, a formal and general framework to achieve multi-tasking robots was developed. 
We showed that our algorithm was sound and complete under our problem settings. 
%Our method was integrated with a simple greedy heuristic for task allocation.
Simulation  results  were  provided  to  show  the  effectiveness  of  our approach under resource-constrained situations and in handling challenging situations. % in a multi-UAV simulator. 

% The idea of multi-tasking is attractive in many ways. 
% Humans are living in multi-tasking environments--at any point of time, 
% we are optimizing for more than one task. 
% Multi-task often leads to more efficient task performance since it allows us to exploit task synergies. 
% The work presented in this paper takes us one step forward in realizing multi-tasking robots. 
% In particular, we started looking at the feasibility of multi-tasking. 
% There are many potential directions to pursue along this direction. First, several limitations are present in the current approach. 
% For example, although our method guarantees that there exists a physical configuration that satisfies all the constraints, it does not explicitly take the environmental influence into account. For example, a narrow corridor may prevent a robot formation from passing through, even though all the constraints for the formation do not introduce any conflicts. In this sense, our work should better be characterized as establishing a necessary condition for multi-tasking. Also, our method is mainly focused on the ``{\it planning}'' phase and hence does not address how the robots reach the desired configuration and maintain the constraints. These issues are assumed to be handled by the execution layer.

% More generally, the question of how to execute the tasks with overlapping coalitions is not addressed in this work. 
% As we already discussed, executing individual tasks with non-overlapping coalitions is straightforward but task synergies impose additional requirements on the task execution: how should the robots that are assigned multiple tasks execute them? Should they consider them in a prioritized strategy~\cite{van2005prioritized}? Or should they combine the different tasks in a way that is similar to motor schemas~\cite{arkin2}. 
% Communication requirements for maintaining the constraints must also be taken into account. How should the robots optimize their communication to improve the task performance? 

% The stringency of the physical constraints is another interesting question. It may be desirable to relax the constraints in certain situations (e.g., due to environmental influences). In such cases, it may be important to consider the problem where the constraints are least violated~\cite{kim2012revision}, or specify task constraints in different ways to increase the diversity of the configurations~\cite{srivastava2007domain} so as to make it robust to different environments. 

\section*{Acknowledgements}
This work was supported in parts by the NSF under grants IIS 1844325, IIS 1909370, and OIA 1936997.

% ~ \newpage
% ~ \newpage
%  \bibliographystyle{SageH}
\bibliography{ref}
\bibliographystyle{theapa}


z

\end{document}
