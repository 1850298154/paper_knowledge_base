\hypertarget{the-cmp-algorithm}{%
\section{The CMP Algorithm}\label{the-cmp-algorithm}}

Cyclic Memory Protection (CMP) is a lock-free MPMC queue that achieves
strict FIFO ordering and unbounded capacity with bounded reclamation
requiring no inter-thread coordination. CMP eliminates use-after-free
hazards and uses a configurable protection window to bound reclamation
and prevent ABA within the window. Building on the Michael \& Scott
linking discipline, CMP removes stale helping and enables fully
concurrent reclamation without per-thread announcements (hazard
pointers, epochs, etc.). A scan cursor and a single claim CAS make
enqueue and dequeue O(1) in the common case.

\hypertarget{dual-protection-mechanism-1}{%
\subsection[Dual Protection Mechanism ]{\texorpdfstring{Dual Protection
Mechanism \footnote{\textbf{Memory Ordering Convention:} \texttt{LOAD}
  uses acquire ordering for reads requiring visibility of prior writes;
  \texttt{STORE} uses relaxed ordering except for publishing operations
  which use release; \texttt{FETCH\_MIN} and \texttt{INCREMENT} use
  sequential consistency or relaxed; CAS operations use acquire-release
  semantics. Non-atomic operations are safe due to immutability (e.g.,
  \texttt{current.cycle}) or single-writer guarantees (e.g., recovery
  state updates).}}{Dual Protection Mechanism }}\label{dual-protection-mechanism-1}}

The key idea is that protection is established before any thread
dereferences a node. A sliding protection window defines which nodes are
temporally safe; within this window, nodes cannot be reclaimed, so
threads dereference without handshakes or coordination. After accessing
a node, a thread unilaterally publishes its observation point (the
node's cycle), sliding the window forward for future nodes. Safety
against ABA and use-after-free (UAF) follows from mathematical
invariants, not consensus.

This preemptive model contrasts with traditional reactive approaches
(hazard pointers, epochs) that attempt protection after dereferencing,
often leading to additional complex race conditions and retries. CMP
eliminates this by establishing temporal safety boundaries first,
accessing once, and moving on, resulting in scalable performance with
fewer atomic operations and lower contention.

CMP ensures safety through two independent but complementary mechanisms:

\textbf{1. State-based protection.} Each node follows a two-state
lifecycle: \texttt{AVAILABLE\ →\ CLAIMED}. \texttt{AVAILABLE} nodes are
protected from reclamation and eligible for dequeue operations.
\texttt{CLAIMED} nodes become reclamation candidates once they satisfy
the cycle-based protection condition.

\textbf{2. Cycle-based sliding window protection.} Each enqueued node is
assigned an immutable monotonically increasing \texttt{cycle} at
creation, establishing a temporal identity that persists throughout its
lifetime. Dequeues publish the cycle of the node they claim as
\texttt{deque\_cycle}, which establishes an active sliding protection
window, \texttt{P\ =\ {[}deque\_cycle\ -\ W,\ deque\_cycle{]}}

where \texttt{W} is the protection window size, tuned to balances memory
usage with tolerance to thread delays,
\texttt{W\ =\ max\ (MIN\_WINDOW,\ OPS} \(\times\) \texttt{R)} where
\texttt{OPS} is expected dequeue rate (ops/s) and \texttt{R} is
resilience in seconds (maximum acceptable thread delay). The window
should accommodate the worst-case delay between dequeues under normal
and stressed conditions. Larger windows increases resilience to
scheduling delays and preemption but increase memory use, which is
bounded by \texttt{window\_size} \(\times\) \texttt{node\_size}
regardless of total queue capacity. \texttt{W} is configured per queue
instance at initialization, allowing different queues in the same
deployment to use different window sizes based on their workload
characteristics.

The key insight is that dequeue operations collectively maintain a
protection boundary without coordination, guaranteeing safe access to
future nodes and even to \texttt{CLAIMED} nodes that stalled threads may
still observe. When a new dequeue operation begins, protection from
previous dequeues already exists, enabling immediate safe access without
per-operation coordination. This preemptive approach eliminates the
coordination overhead, complex retry logic, and timing dependencies
common in traditional memory reclamation schemes, making dequeue latency
predictable and scalable regardless of thread count or timing anomalies.

While state-based protection is common in lock-free algorithms, the
sliding protection window approach eliminates the need for inter-thread
coordination entirely. Each node receives a cycle timestamp at enqueue,
and garbage collection maintains a protection window \texttt{P} based on
the highest cycle accessed by any dequeue operation. Nodes can only be
reclaimed when they are both \texttt{CLAIMED} and outside this sliding
window. This cycle-based protection provides bounded memory usage with
configurable parameters, automatic advancement without coordination, and
resilience to thread failures - unlike traditional approaches that
require hazard pointer scanning, epoch coordination, or other complex
synchronization protocols.

These two mechanisms together provide safety that is both mathematically
robust and operationally simple. State protection prevents freeing any
node that may still be accessed, while the cycle window adds a temporal
guard that covers stalls and failures. In combination, they eliminate
UAF and ABA by construction, without auxiliary validation protocols.
This redundancy does not add overhead: the mechanism relies only on the
atomic operations already required for correctness.

The dual protection also improves performance. Each operation is
self-contained and coordination-free, which improves cache locality and
delivers linear scalability across cores. It additionally provides
automatic recovery: coordinated schemes can let a stalled thread block
reclamation and trigger unbounded memory growth, whereas the cycle
window naturally expires old protection so a stalled thread cannot hold
the system hostage. This behavior is crucial in production, where long
preemptions and failures are inevitable.

\hypertarget{data-structure-and-rationale}{%
\subsection{Data Structure and
Rationale}\label{data-structure-and-rationale}}

The data structure design implements a lock-free FIFO queue with state-
and cycle-based protection for memory safety. It comprises two
components: a lock-free queue structure that handles FIFO linking and
node state transitions, and a cycle management structure that enforces
temporal ordering and controls reclamation.

\hypertarget{lock-free-queue-structure}{%
\subsubsection{Lock-free Queue
Structure}\label{lock-free-queue-structure}}

Each queue node contains four critical fields that together provide dual
protection. The \texttt{cycle} field is an immutable temporal
identifier; it is set at enqueue and never changes. The \texttt{next}
field maintains the linked list structure required for FIFO ordering,
with atomic updates ensuring that modifications are visible consistently
across threads and set to \texttt{NULL} on reclaim to prevent stale
traversal. The \texttt{data} field stores the user payload and is only
accessed once the node has been safely claimed. The \texttt{state} field
implements the state-based protection mechanism, transitioning through
\texttt{AVAILABLE} → \texttt{CLAIMED} to guarantee that no node is
reclaimed while still available for dequeue operations.

All linked-list nodes (not payloads) are allocated and recycled from a
type-stable memory pool - nodes reside in a persistent pool, recycled
exclusively as Node objects, and never freed to the OS. This type
stability ensures any pointer to pool memory always references a valid
Node structure with an accessible cycle field, enabling safe cycle-based
protection checks even on recycled addresses.

The \texttt{head} pointer always references a dummy node, simplifying
insertion and deletion. The \texttt{tail} pointer optimizes concurrent
enqueue operations under contention.

The \texttt{scan\_cursor} serves as a dequeue optimization, pointing to
the first likely available node to reduce traversal to \texttt{O(1)} in
the common case. Each successful dequeue advances the cursor atomically
for the benefit of subsequent operations. The
\texttt{protection\_window} parameter bounds memory usage by defining
the temporal window for reclamation safety.

\hypertarget{cycle-management-structure}{%
\subsubsection{Cycle Management
Structure}\label{cycle-management-structure}}

The cycle mechanism enforces temporal safety through invariants rather
than coordination, using two counters to define ordering and reclamation
boundaries.

The global \texttt{cycle} establishes ordering across all enqueue
operations. Each enqueued node is tagged with the current cycle,
creating an immutable temporal identity that enables safe reclamation
decisions.

The \texttt{deque\_cycle} counter tracks the highest cycle value of any
node successfully claimed and processed, representing the current
frontier of dequeue progress. Every dequeue operation unilaterally
announces its progress by updating the global \texttt{deque\_cycle} to
the cycle of the node it has claimed. This announcement requires no
coordination or acknowledgment and always advances monotonically.

Safe reclamation is determined by computing protection window
\texttt{P}. Only nodes outside this window and in state \texttt{CLAIMED}
are reclaimed. This creates a temporal barrier: any node inside
protection window \texttt{P} is guaranteed to be protected. Since
\texttt{deque\_cycle} monotonically tracks the highest cycle accessed by
any dequeue, and the reclaim process maintains a trailing protection
window of size \texttt{W}, this ensures that nodes cannot be freed while
active and cannot be freed if their cycle lies within the protection
window, even if retired.

The 64-bit cycle counter provides a practically infinite sequence space.
At one billion operations per second, wraparound would require more than
584 years, making it irrelevant for practical systems.

\hypertarget{lock-free-enqueue-algorithm}{%
\subsection{Lock-Free Enqueue
Algorithm}\label{lock-free-enqueue-algorithm}}

The enqueue operation implements a modified Michael \& Scott linked-list
insertion algorithm with cycle-based versioning for memory safety.
Multiple producer threads can enqueue concurrently without coordination,
maintaining strict FIFO ordering through atomic linked-list operations.

\begin{algorithm}
\caption{Lock-Free Enqueue}
\begin{algorithmic}[1]
\Function{enqueue}{queue, data}
   \State \textit{// Phase 1: Node allocation and cycle assignment}
   \State $new\_node \gets$ allocate\_node$()$
   \State $new\_node.data \gets data$
   \State $new\_node.next \gets$ NULL
   \State $new\_node.state \gets$ AVAILABLE
   \State $cycle \gets$ INCREMENT$(queue.cycle)$
   \State $new\_node.cycle \gets cycle$
   
   \State \textit{// Phase 2: Lock-free Insertion}
   \State $retry\_count \gets 0$
   \Repeat
       \State $tail \gets$ LOAD$(queue.tail)$
       \State $next \gets$ LOAD$(tail.next)$
       \State \textit{// Pre-check to avoid expensive CAS (OPTIONAL)}
       \If{$next \neq$ NULL}
           \State \textit{// Tail has advanced, retry with fresh state}
           \State $retry\_count \gets retry\_count + 1$
           \If{$retry\_count > 3$}
               \State CPU\_PAUSE$()$
           \EndIf
           \State \textbf{continue}
       \EndIf
       \State \textit{// Attempt to link new node}
       \If{CAS$(tail.next,$ NULL$, new\_node)$}
           \State \textit{// Optional tail advancement}
           \State CAS$(queue.tail, tail, new\_node)$
           \State \textbf{break}
       \EndIf
   \Until{false}
   
   \State \textit{// Phase 3: Conditional Reclamation}
   \If{$(cycle \bmod N) = 0$}
       \State trigger\_reclamation$(queue)$
   \EndIf
   \State \Return SUCCESS
\EndFunction
\end{algorithmic}
\end{algorithm}


\textbf{Phase 1: Node Allocation and Cycle Assignment.} A new node is
allocated from the available pool, and initialized with user data. The
global cycle counter is atomically incremented and assigned to the node,
establishing its unique temporal identity within the 64-bit space. This
ordering enables cycle-based protection, where garbage collection
retains a trailing protection window relative to dequeue progress. If
allocation fails, the algorithm triggers immediate reclamation and
retries, providing automatic memory pressure relief without external
intervention.

\textbf{Phase 2: Lock-Free Insertion.} The insertion phase uses a
streamlined version of the Michael \& Scott algorithm that eliminates
helping mechanisms. The algorithm loads the current \texttt{tail}
pointer and its \texttt{next} field using ACQUIRE ordering to ensure
visibility of concurrent updates. The \texttt{tail-\textgreater{}next}
pointer is expected to be \texttt{null} for normal operation. However,
if the tail's next pointer is non-null, indicating that another thread
has inserted a node, the algorithm retries with fresh state rather than
attempting complex coordination. It uses cpu pause when necessary. When
the tail's next is null, the algorithm attempts to link the new node
using CAS with RELEASE ordering, ensuring all node writes are visible
before publication. After successful insertion, the algorithm advances
the tail pointer to the new node.

\textbf{Phase 3: Conditional Reclamation.} After every N insertions, the
enqueue thread triggers reclamation (cycle \% N == 0), distributing
reclamation work across producers and avoiding dedicated GC threads
while maintaining bounded memory use. The algorithm is agnostic to the
triggering policy - implementations may use deterministic modulo (as
shown), randomized triggers (Bernoulli p = 1/N), or hybrid approaches,
depending on workload fairness requirements. Reclamation is
non-blocking. If another thread is already reclaiming, enqueue proceeds
without reclamation.

\textbf{Performance.} In the common case, enqueue requires 3--5 atomic
operations: one increment for cycle assignment, one or two for loads,
one CAS to link the node, and optionally one CAS to advance the tail.
This matches or improves upon Michael \& Scott's performance while
adding stronger correctness guarantees through cycle-based protection.

\hypertarget{modifications-to-michael-scotts-algorithm}{%
\subsection{Modifications to Michael \& Scott's
Algorithm}\label{modifications-to-michael-scotts-algorithm}}

CMP modifies the original Michael \& Scott (M\&S) algorithm to address
specific inefficiencies, particularly by reducing unnecessary atomic
operations.

\begin{algorithm}
\caption{Original M\&S Helping Mechanism}
\begin{algorithmic}[1]
\Repeat
   \State $tail \gets$ LOAD$(queue.tail)$
   \State $next \gets$ LOAD$(tail.next)$
   \State \textit{// Revalidate tail}
   \If{$tail =$ LOAD$(queue.tail)$}
       \If{$next \neq$ NULL}
           \State \textit{// Help advance tail using 'next' (may be stale)}
           \State CAS$(queue.tail, tail, next)$
           \State \textbf{continue}
       \EndIf
   \EndIf
\Until{false}
\end{algorithmic}
\end{algorithm}


In the original M\&S algorithm, threads attempt to advance the tail
pointer when it is stale (i.e., when next is not null). However, this
helping based on stale data is redundant and often counterproductive.
Under high load, it can increase contention when multiple threads
attempt to act on outdated observations. A more effective approach is to
simply retry with fresh state whenever stale conditions are detected
(Algorithm 1, line 15), ensuring that operations always proceed on
current information. Eliminating ``helping'' reduces both the number of
atomic operations and cache line bouncing in contended scenarios, while
preserving the correctness guarantees of the original algorithm.

Removing this helping mechanism also helps with removing an extra
validation (Algorithm 2 line 5) which was originally meant to avoid
expensive CAS. Since the \texttt{next} non-null value is now eliminated
in earlier condition, the extra atomic validation is no longer required.

These modifications do not compromise any correctness properties of the
original algorithm. The helping mechanism was intended as a performance
optimization, not as a correctness requirement. By removing it, CMP
strengthens correctness by avoiding potential races between stale
observations and helping actions.

It is worth noting that production implementations have taken different
approaches: Java's ConcurrentLinkedQueue employs opportunistic helping,
Boost.Lockfree implements the full helping mechanism as in the original
Michael \& Scott algorithm, while .NET's ConcurrentQueue omits helping
entirely.

\hypertarget{lock-free-dequeue-algorithm}{%
\subsection{Lock-Free Dequeue
Algorithm}\label{lock-free-dequeue-algorithm}}

The dequeue operation follows a coordination-free model where temporal
protection mechanisms are preemptively established, enabling seamless
lock-free execution without requiring explicit safety coordination. It
uses the \texttt{scan\_cursor} optimization to achieve \texttt{O(1)}
common case performance while allowing multiple threads to improve
shared state without explicit coordination.

\begin{algorithm}
\caption{Lock-Free Dequeue}
\begin{algorithmic}[1]
\Function{dequeue}{queue}
    \State $current \gets$ LOAD$(queue.head)$ \Comment{Non-NULL value}
    \State $last\_deque\_cycle \gets 0$
    
    \While{$current \neq$ NULL}
        \State \textit{// Phase 1: Load scan cursor - starting position}
        \State $deque\_cycle \gets$ LOAD$(queue.deque\_cycle)$
        \If{$deque\_cycle \neq last\_deque\_cycle$}
            \State $last\_deque\_cycle \gets deque\_cycle$
            \State $current \gets$ LOAD$(queue.scan\_cursor)$
            \State $last\_cursor \gets current$
            \State $cursor\_cycle \gets last\_cursor.cycle$
        \EndIf
        
        \State \textit{// Phase 2: Atomic node claiming}
        \State $expected \gets$ AVAILABLE
        \If{\textbf CAS$(current.state, expected,$ CLAIMED$)$}
            \State \textbf{break}
        \EndIf
            
	\State $current \gets current.next$
    \EndWhile
    \State if {$current =$ NULL} \textbf{then} \Return NULL
    \State \textit{// Phase 3: Claim Data with CAS}
    \If{$current.state =$ AVAILABLE}
        \State \Return NULL  \Comment{Detect ABA/reassignment}
    \EndIf
    
    \State $data \gets current.data$
    \If{$data =$ NULL \textbf{or not} CAS$(current.data, data,$ NULL$)$}
        \State \Return NULL
    \EndIf
    
    \State $advance\_boundary \gets$ true
    
    \State \textit{// Phase 4: Scan\_cursor Advance \& Cycle Announce}
    \If{$last\_cursor =$ LOAD$(queue.scan\_cursor)$ \textbf{and} 
    \Statex \hspace{2em} $cursor\_cycle = queue.scan\_cursor.cycle$}
        \State $next \gets$ LOAD$(current.next)$
        \State $advance\_boundary \gets$ false
        
        \If{$next =$ NULL \textbf{or} ($next \neq$ NULL \textbf{and}
        \Statex \hspace{2em} CAS$(queue.scan\_cursor, last\_cursor, next))$}
            \State $advance\_boundary \gets$ true
        \EndIf
    \EndIf
    
    \State \textit{// Phase 5: Protection Boundary Update}
    \If{$advance\_boundary$}
        \State $cycle \gets$ LOAD$(queue.deque\_cycle)$
        \While{$cycle < current.cycle$}
        \If{CAS$(queue.deque\_cycle, cycle, current.cycle)$}
            \State \textbf{break}
        \EndIf
        \State $cycle \gets$ LOAD$(queue.deque\_cycle)$
        \EndWhile
    \EndIf
    
    \State \Return $data$
\EndFunction
\end{algorithmic}
\end{algorithm}


\textbf{Phase 1: Scan Cursor Load.} The dequeue loads
\texttt{scan\_cursor}, which points to the first potentially available
node, avoiding re-traversal of already-processed nodes and providing
\texttt{O(1)} performance in most cases without requiring coordination
protocols. The cursor is initialized to the dummy node at setup and is
never set to NULL. During traversal, if \texttt{deque\_cycle} advances
(indicating other threads' progress), the algorithm reloads
\texttt{scan\_cursor} to accelerate convergence under high contention to
reduce traversal.

\textbf{Phase 2: Atomic Node Claiming.} Starting at
\texttt{scan\_cursor}, the thread linearly probes and attempts
\texttt{CAS\ (state,\ AVAILABLE\ →\ CLAIMED)}. This atomic operation
ensures that exactly one thread claims a node; others automatically
continue scanning forward to subsequent nodes. If a CAS fails, another
thread claimed it; continue scanning. On success, the thread has
exclusive access and can read the payload.

\textbf{Phase 3: Atomic Data Claiming.} After claiming, the thread
revalidates that the node state remains \texttt{CLAIMED} (detecting
potential ABA/reassignment scenarios), and extracts data using
CAS(current.data, data, NULL) to atomically claim ownership of the
payload. This CAS prevents duplicate data extraction when multiple
threads might contest for the same data if threads were preempted and
stalled in the \texttt{CLAIMED} state. The atomic nullification ensures
data will be accessed by only one thread, preventing duplicate access
and potential UAF by clearing the data pointer immediately after
extraction.

\textbf{Phase 4: Scan Cursor Advance \& Cycle Announcement.} After
claiming, the thread opportunistically advances \texttt{scan\_cursor} if
and only if both the pointer value and cycle still equal the last loaded
cursor, and the claimed node has a non-null next. This dual condition
mathematically eliminates ABA problems. Since cycles are monotonic and
never wrap around, even if \texttt{scan\_cursor} is reassigned to the
same pointer value, the different cycle will prevent this condition from
being satisfied, providing mathematical certainty against ABA hazards.
This conditional advance prevents redundant atomic operations when
another thread has already advanced the cursor, and if the advancement
CAS fails, correctness is unchanged with only performance impact.

\textbf{Phase 5: Protection Boundary Update.} Upon successful
advancement, it announces temporal progress by writing
\texttt{deque\_cycle\ =\ claimed.cycle}. This maintains the invariant
\texttt{scan\_cursor.cycle} \(\geq\) \texttt{deque\_cycle} and tightens
the protection boundary.

\textbf{Correctness Guarantees.} The Boundary Update (Phase 5) maintains
a trailing protection window that prevents freeing nodes within
protection window of any completed dequeue. The \texttt{scan\_cursor}
optimization maintains correctness even under high concurrency because
all updates move the pointer forward monotonically, and race conditions
only affect performance, never safety. Dual protection ensures nodes
cannot be freed in the state \texttt{AVAILABLE} (state protection) and
cannot be freed while within the temporal window (cycle protection).

\textbf{Performance.} The dequeue requires 4-9 atomic operations in the
common case: one for \texttt{deque\_cycle} loading, one for
\texttt{scan\_cursor} loading, one for node claiming, one for data
claiming, and optionally 3-5 additional operations for
\texttt{scan\_cursor} advancement and \texttt{deque\_cycle} updates. The
\texttt{scan\_cursor} optimization typically reduces scanning to a
single iteration, achieving near-constant time performance regardless of
queue history. Advancement failures are benign---they simply indicate
another thread already optimized the cursor position.

\hypertarget{coordination-free-memory-reclamation}{%
\subsection{Coordination-Free Memory
Reclamation}\label{coordination-free-memory-reclamation}}

\textbf{Safety Predicate.} A node is reclaimed if and only if: \[
(state \neq AVAILABLE) \land (node.cycle < safe\_cycle)
\]

\begin{itemize}
\tightlist
\item
  \texttt{state} \(\neq\) \texttt{AVAILABLE} ensures the node has been
  removed from the abstract queue.
\item
  \texttt{node.cycle\ \textless{}\ safe\_cycle} ensures no active
  consumer thread can access this node or any older cycle.
\end{itemize}

Both conditions are jointly necessary; omitting either risks freeing
memory still reachable by some thread.

The reclamation operates without coordination and is lock-free, allowing
normal queue operations to proceed unimpeded, and uses batched
collection to amortize atomic costs while maintaining predictable
latency. Dual safety conditions ensure nodes are reclaimed only when
both state-based and cycle-based protection mechanisms confirm safety.

Unlike traditional approaches where stalled threads can block garbage
collection indefinitely, CMP allows reclamation to continue past nodes
in \texttt{CLAIMED} state from failed or stalled threads. Once a thread
successfully claims a node, that node becomes a reclamation candidate
after a minimum of \texttt{W} dequeue cycles have passed, enabling
automatic recovery while maintaining all safety guarantees.

\begin{algorithm}
\caption{Coordination-Free Memory Reclamation}
\begin{algorithmic}[1]
\Function{reclaim}{queue}
   \State \textit{// Phase 1: Calculate current protection boundary}
   \State $cycle \gets$ LOAD$(queue.deque\_cycle)$
   \State $window \gets queue.protection.window\_size$
   \State $safe\_cycle \gets \max(0, cycle - window)$
   \State $head \gets$ LOAD$(queue.head)$
   \State $current \gets$ LOAD$(head.next,)$
   
   \While{$current \neq$ NULL}
       \State $original\_next \gets current$
       \State $new\_next \gets current$
       
       \State \textit{// Collect batch of safely reclaimable nodes}
       \While{$current \neq$ NULL}
           \State \textit{// Phase 2: Cycle-based protection check}
           \If{$current.cycle \geq safe\_cycle$}
               \State \textbf{break}
           \EndIf
           \State \textit{// Phase 3: State-based protection check}
           \If{LOAD$(current.state)$ = AVAILABLE}
               \State \textbf{break}
           \EndIf
           
           \State \textit{// Phase 4: Add to reclamation batch}
	   \State $batch.add(current)$
           \State $next \gets$ LOAD$(current.next)$
           \State $new\_next \gets next$
           \State $current \gets next$
       \EndWhile
       
       \State \textit{// Enforce minimum batch size for efficiency}
       \If{$batch.count < MIN\_BATCH\_SIZE$}
           \State \textbf{break}
       \EndIf
       
       \State \textit{// Phase 5: Atomic head pointer advancement}
       \If{CAS$(head.next, original\_next, new\_next)$}
	   \State \textit{// Free and reset nodes (set next = NULL)}
           \State deallocate\_nodes$(batch)$
       \Else
           \State \textit{// Concurrent modification detected}
           \State \textbf{break}
       \EndIf
   \EndWhile
\EndFunction
\end{algorithmic}
\end{algorithm}


\textbf{Phase 1: Protection Boundary Calculation.} Reclamation begins by
calculating the \texttt{safe\_cycle} value that is safe for reclamation
based on the current \texttt{deque\_cycle} value and the configured
protection window size \texttt{W}. \[
safe\_cycle = deque\_cycle - W
\] Nodes with cycles below this \texttt{safe\_cycle} threshold are
candidates for reclamation, while nodes inside the protection window are
preserved to prevent ABA hazards. Only nodes in \texttt{CLAIMED} state
are considered; \texttt{AVAILABLE} nodes are absolutely protected
regardless of cycle value.

\textbf{Phase 2: Cycle-Based Safety Boundary.} Starting from
\texttt{head.next}, the algorithm first checks whether a node's cycle is
below the threshold. Since the \texttt{cycle} field is immutable, this
check is a fast, non-atomic read. Only nodes passing this test proceed
to state verification. Dequeue operations maintain the invariant
\texttt{scan\_cursor.cycle} \(\geq\) \texttt{deque\_cycle}, and the
\texttt{tail} always holds the latest cycle value, so active queue
boundaries remain protected without explicit pointer checks.

\textbf{Phase 3: State-Based Safety.} The algorithm confirms that the
node is not in \texttt{AVAILABLE} state. Nodes in \texttt{AVAILABLE}
state are never reclaimed, regardless of their cycle values, as they
represent nodes waiting to be dequeued. Reclamation halts at the first
node that is in \texttt{AVAILABLE} state to maintain FIFO order and
prevent premature reclamation.

\textbf{Phase 4: Batch Collection.} Eligible nodes are collected in a
batch rather than reclaimed one by one. Batching avoids repeated head
updates, reduces atomic operations, and improves cache efficiency. The
algorithm records the original head and the new head after all
reclaimable nodes, performing a single atomic CAS to advance across the
batch.

\textbf{Phase 5: Atomic Head Update and Memory Management.} When the
batch reaches the minimum threshold, reclamation attempts a CAS update
of the head pointer. If another thread interferes, the attempt is
abandoned to avoid consistency issues. Only after a successful CAS are
the collected nodes processed for deallocation. To ensure any dequeue
thread with a stale pointer safely terminate traversal, the
\texttt{next} and \texttt{data} pointers set to NULL before being
returning free node to the memory pool.

\textbf{Mathematical Safety Guarantees.} Temporal progression is
enforced by monotonic updates of \texttt{deque\_cycle}. Reclamation lags
by at least \texttt{W} cycles, ensuring that any active or stalled
thread remains protected against reclamation. This eliminates UAF and
ABA hazards without coordination, provided the protection window
\texttt{W} exceeds the maximum possible delay in dequeue progress.

\hypertarget{correctness-and-progress-guarantees-2}{%
\subsection[Correctness and Progress Guarantees
]{\texorpdfstring{Correctness and Progress Guarantees \footnote{Complete
  formal proofs of linearizability, FIFO ordering, memory safety,
  lock-free progress, and bounded reclamation properties available upon
  request.}}{Correctness and Progress Guarantees }}\label{correctness-and-progress-guarantees-2}}

CMP achieves strict FIFO semantics, lock-free progress, unbounded
capacity, and bounded reclamation, with common-case \texttt{O(1)}
latency and stable performance under contention. The design simplifies
the enqueue path by eliminating ``helping'' and redundant validations
from M\&S, and ensures automatic recovery from failed or stalled threads
through its bounded temporal protection window.

\textbf{Linearizability.} Enqueue operations linearize at the successful
\texttt{CAS(tail.next,\ null,\ new\_node)} that links a node, while
dequeue operations linearize at the successful
\texttt{CAS\ (current.state,\ AVAILABLE,\ CLAIMED)} that claims
ownership. Empty dequeues linearize when the scan cursor reaches null.
These points yield a total order consistent with real-time precedence
and sequential queue semantics.

\textbf{FIFO Ordering.} Strict FIFO ordering is maintained through three
key invariants: (1) \textbf{Append-only linking:} Enqueues link new
nodes strictly at the physical tail via
\texttt{CAS\ (tail.next,\ null,\ new\_node)}, establishing chronological
order. After linking, list order is immutable. (2) \textbf{Cursor
minimality:} The \texttt{scan\_cursor} never advances past an
\texttt{AVAILABLE} node---every node strictly before
\texttt{scan\_cursor} is in \texttt{CLAIMED} state. (3)
\textbf{Earliest-claim property:} Dequeues linearly probe from
\texttt{scan\_cursor}; a node can be claimed
\texttt{(AVAILABLE\ →\ CLAIMED)} only after all predecessors are
non-AVAILABLE. Together, (1)--(3) ensure the first successful claim
always targets the earliest available node in list order, guaranteeing
removals respect enqueue order regardless of thread scheduling.

\textbf{Lock-Free Progress.} All operations complete in bounded time
with system-wide progress guarantees. Enqueues use simplified M\&S
insertion where each failed CAS indicates another thread's successful
progress. Dequeues scan forward from optimized positions, with each
failed claim CAS representing another thread's advancement. The
\texttt{scan\_cursor} optimization achieves \texttt{O(1)} amortized
dequeue performance by tracking the earliest viable position.

\textbf{Bounded Reclamation.} Nodes are reclaimed within at most
\texttt{W} dequeue cycles plus GC delay cycles after reaching
\texttt{CLAIMED} state, independent of thread failures, scheduling
stalls, or queue size. This provides predictable memory usage, stable
latency, and resilience under crashes.

\textbf{Memory Safety Guarantees.} The state-based protection prevents
reclamation of nodes in the \texttt{AVAILABLE} state, while the
cycle-based protection retains processed nodes within a bounded temporal
window \texttt{P}. Together, these mechanisms ensure absolute safety
against UAF, and safety against ABA within the configured window size
\texttt{W}, without relying on complex retries, validation, or
coordination overhead, achieving both safety and performance through
algorithmic invariants rather than runtime detection.

These properties are established through rigorous mathematical analysis
including invariant preservation, progress arguments, and safety proofs.
The formal treatment demonstrates that CMP achieves the challenging
combination of strict FIFO semantics, high performance,
coordination-free operation, and automatic fault recovery without
sacrificing correctness guarantees.
