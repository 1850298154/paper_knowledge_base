\hypertarget{performance-evaluation}{%
\section{Performance Evaluation}\label{performance-evaluation}}

We evaluate CMP against production-ready lock-free queue implementations
deployed at scale to assess performance and scalability under real-world
conditions. CMP was originally designed to meet production requirements
in mesibo (real-time communication platform) and PatANN
(high-performance vector search engine), both requiring unbounded,
coordination-free queues to sustain high-throughput, low-latency
workloads.

Numerous academic prototypes (LCRQ \citep{morrison2013fast}, FAA-based
queues \citep{yang2016wait}, and others) were excluded, as they either
rely on non-portable primitives (e.g., DCAS or architecture-specific
atomics unavailable on ARM and in managed languages) or remain research
prototypes with no production adoption or large-scale deployment, making
them unsuitable for representing real-world industry scenarios.

\textbf{CMP Queue:} The CMP implementation with strict FIFO ordering and
unbounded queue.

\textbf{Moodycamel (MC) Concurrent Queue \citep{moodycamel_queue} }:
Industry-standard lock-free queue implementation employing a relaxed
ordering model that sacrifices strict FIFO semantics to maximize
throughput, with optimized fast paths for single-producer,
single-consumer settings.

\textbf{Boost Lockfree Queue \citep{boost_lockfree}:} Based on the
Michael \& Scott (M\&S) algorithm, using hazard pointers for memory
safety and CAS for synchronization. It provides strict FIFO and serves
as a baseline. CMP also builds on M\&S but replaces hazard pointers with
cyclic memory protection, showing that modern reclamation schemes can
improve both performance and simplicity.

All experiments were conducted on a controlled testbed with round-robin
sequencing of implementations to eliminate bias from CPU thermal
throttling and dynamic frequency scaling. To reduce noise from OS
preemption, interrupts, and memory traffic, we applied 3-sigma filtering
uniformly across all implementations (CMP, Moodycamel, Boost): samples
beyond \(\mu\) \(\pm\) 3\(\sigma\) were discarded, removing
\textasciitilde{}0.3\% of anomalies and yielding stable, reproducible
results. This standard practice \citep{georges2007statistically} ensures
P99 values reflect queue performance under stable conditions rather than
random OS or hardware artifacts. We report two regimes: baseline,
measuring raw queue throughput; and synthetic load, where threads
perform additional computation between operations to emulate realistic
workloads.

\hypertarget{performance-results}{%
\subsection{Performance Results}\label{performance-results}}

\hypertarget{throughput-performance}{%
\paragraph{Throughput Performance}\label{throughput-performance}}

Throughput under different producer-consumer configurations is shown in
Table 1.

\begin{figure}[htbp]
    \centering
    \vspace{-\baselineskip}  % Remove one line of space at top
    \caption{Throughput comparison across thread configurations. CMP maintains superior performance across all configurations, with particularly dramatic advantages at high contention (64P64C shows 425\% improvement over alternatives).}
    \label{fig:cmp_performance}
    \includegraphics[width=\columnwidth]{cmp_performance.pdf}
\end{figure}



In the single-producer/single-consumer case (1P1C), CMP sustains 6.49M
items/second, 72\% higher than Moodycamel and 188\% higher than Boost,
reflecting CMP's reduced synchronization overhead when contention is
minimal. At the largest configuration (64P64C), CMP reaches 1.19M
items/second, 325\% higher than Boost and 892\% higher than Moodycamel,
indicating exceptional scaling under extreme contention. Notably, at
this extreme contention level, Boost outperforms Moodycamel by 2.3x in
throughput, suggesting that Moodycamel's coordinated design suffers
badly under maximum stress---a pattern that underscores the importance
of CMP's coordination-free approach.

For intermediate contention (2P2C---16P16C), CMP achieves 5---23\%
higher throughput than Moodycamel while preserving strict FIFO ordering.
This advantage demonstrates that CMP's coordination-free design not only
maintains ordering but actually improves performance under moderate
contention.

Across all configurations, CMP outperforms Boost by 77---325\%,
demonstrating efficient lock-free design while retaining correctness
properties.

\hypertarget{latency-analysis}{%
\paragraph{Latency Analysis}\label{latency-analysis}}

The following analysis examines operation costs under varying contention
levels, reporting latency in nanoseconds for no contention (1P1C),
balanced contention (4P4C), high contention (32P32C), and extreme
contention (64P64C).

\begin{table}[htbp]
	\caption{Latency with no contention. CMP achieves 40\% lower enqueue and 50\% lower dequeue latencies than Moodycamel (MC).}
\label{tab:latency_performance}
\centering
\begin{tabular}{lcccc}
\toprule
\textbf{Impl} & \textbf{Avg Enq} & \textbf{P99 Enq} & \textbf{Avg Deq} & \textbf{P99 Deq} \\
\midrule
CMP    & \textbf{63.9} & \textbf{111} & \textbf{70.6} & \textbf{74} \\
Moodycamel & 106.0         & 146          & 141.9         & 172          \\
Boost      & 258.5         & 314          & 164.0         & 193          \\
\bottomrule
\end{tabular}
\end{table}



In the single-producer, single-consumer case, CMP achieves the lowest
enqueue and dequeue latencies. Compared to Moodycamel, enqueue
operations are 40\% faster on average, and dequeue operations are 50\%
faster. These results indicate that CMP's coordination-free enqueue and
dequeue mechanisms incur low overhead and latency in the baseline
configuration.

\begin{table}[htbp]
\caption{Under balanced contention, CMP shows higher enqueue latency but 38\% lower dequeue latency than MC.}
\label{tab:contention_latency}
\centering
\begin{tabular}{lcccc}
\toprule
\textbf{Impl} & \textbf{Avg Enq} & \textbf{P99 Enq} & \textbf{Avg Deq} & \textbf{P99 Deq} \\
\midrule
CMP        & 217.1         & 314          & \textbf{87.5} & \textbf{116} \\
Moodycamel & \textbf{144.3} & \textbf{213} & 170.4          & 211          \\
Boost      & 401.8         & 489          & 233.0          & 307          \\
\bottomrule
\end{tabular}
\end{table}



With balanced producer---consumer load, CMP shows higher enqueue latency
(50\% higher than Moodycamel) due to strict FIFO enforcement, but
achieves 49\% lower dequeue latency. This trade-off results in overall
performance advantages: despite slower enqueues, faster dequeues allow
CMP to maintain competitive throughput while providing strict ordering
guarantees.

\begin{table}[htbp]
\caption{Under high contention, CMP reduces enqueue latency by 10\% and dequeue latency by 70\% relative to MC.}
\label{tab:high_contention_latency}
\centering
\begin{tabular}{lcccc}
\toprule
\textbf{Impl} & \textbf{Avg Enq} & \textbf{P99 Enq} & \textbf{Avg Deq} & \textbf{P99 Deq} \\
\midrule
CMP        & \textbf{359.3} & \textbf{350} & \textbf{66.4} & \textbf{70} \\
Moodycamel & 398.7          & 387          & 223.1          & 289          \\
Boost      & 445.2          & 542          & 267.8          & 354          \\
\bottomrule
\end{tabular}
\end{table}


Under high contention (32P32C and 64P64C), CMP consistently reduces both
enqueue and dequeue latencies compared to Moodycamel. At 32P32C, CMP
achieves 10\% lower enqueue latency and 70\% lower dequeue latency,
while at 64P64C it maintains 14\% lower enqueue and 30\% lower dequeue
latencies. CMP also demonstrates superior P99 characteristics across
both configurations, underscoring its robustness under maximum stress
conditions.

\hypertarget{synthetic-workload-resilience-analysis}{%
\paragraph{Synthetic Workload Resilience
Analysis}\label{synthetic-workload-resilience-analysis}}

To assess performance under realistic workloads, we applied a synthetic
mixed workload by interleaving queue operations with additional
computation, inducing memory pressure, cache contention, and scheduling
interference. Retention is reported as the fraction of baseline
throughput sustained under this load.

\begin{figure}[htbp]
    \centering
    \vspace{-\baselineskip}  % Remove one line of space at top
    \caption{Performance retention under synthetic load. CMP maintains 75-92\% throughput across all configurations, demonstrating superior resilience under contention.}
    \label{fig:phantom_load}
    \includegraphics[width=\columnwidth]{synthetic_load_performance.pdf}
\end{figure}


CMP maintains 92\% of baseline performance at 8P8C, compared to
Moodycamel's 76.9\%, a +15.1 percentage-point advantage. At 1P1C, CMP
sustains 91.8\% retention with a +6.7 pp advantage.

The strong retention across configurations indicates that CMP's
coordination-free design is less affected by cache pressure and memory
subsystem interference. By eliminating synchronization between producers
and consumers, CMP reduces waiting and scheduling overhead. In contrast,
Boost shows the weakest resilience (69--78\% retention), reflecting the
sensitivity of coordination-based schemes to memory pressure.

\textbf{Performance Summary.} CMP's coordination-free enqueue and
dequeue mechanisms deliver high throughput while enforcing strict FIFO
ordering. In the baseline case (1P1C), CMP achieves 6.49M items/sec,
which is 72\% higher than Moodycamel and 188\% higher than Boost,
showing that FIFO enforcement incurs negligible overhead without
contention. At maximum concurrency (64P64C), CMP sustains 1.19M
items/sec, representing a 325\% improvement over Boost and an 892\%
improvement over Moodycamel, demonstrating scalability under extreme
load.

Across balanced producer--consumer scenarios, CMP remains competitive or
superior across all metrics while preserving deterministic FIFO
guarantees. Relative to the academic baseline, CMP outperforms Boost by
77--325\% across configurations, with the largest advantages at high
concurrency levels. Under synthetic load, CMP retains up to 92\% of
baseline throughput, including a +15.1 percentage-point advantage at
8P8C over Moodycamel, underscoring resilience to cache pressure, memory
interference, and scheduling effects.

At 64P64C, Moodycamel's throughput falls below Boost, illustrating that
complex optimization strategies can exhibit scalability limitations
under extreme contention. These results reinforce the benefits of CMP's
coordination-free approach, which combines scalability with strict
correctness guarantees.
