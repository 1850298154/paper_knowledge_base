 \begin{abstract}
Multi-robot systems are emerging as a promising solution to the growing demand for productivity, safety, and adaptability across industrial sectors. However, effectively coordinating multiple robots in dynamic and uncertain environments, such as construction sites, remains a challenge, particularly due to unpredictable factors like material delays, unexpected site conditions, and weather-induced disruptions. To address these challenges, this study proposes an adaptive task allocation framework that strategically leverages the synergistic potential of Digital Twins, Integer Programming (IP), and Large Language Models (LLMs). The multi-robot task allocation problem is formally defined and solved using an IP model that accounts for task dependencies, robot heterogeneity, scheduling constraints, and re-planning requirements. A mechanism for narrative-driven schedule adaptation is introduced, in which unstructured natural language inputs are interpreted by an LLM, and optimization constraints are autonomously updated, enabling human-in-the-loop flexibility without manual coding. A digital twin–based system has been developed to enable real-time synchronization between physical operations and their digital representations. This closed-loop feedback framework ensures that the system remains dynamic and responsive to ongoing changes on site. A case study demonstrates both the computational efficiency of the optimization algorithm and the reasoning performance of several LLMs, with top-performing models achieving over 97\% accuracy in constraint and parameter extraction. The results confirm the practicality, adaptability, and cross-domain applicability of the proposed methods.
\end{abstract}



\begin{IEEEkeywords}
Construction digital twin, large language model (LLM), multi-robot task allocation, adaptive scheduling
\end{IEEEkeywords}
