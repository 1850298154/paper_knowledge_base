

\section{Introduction}
\IEEEPARstart{W}ith rising demands for faster project delivery and improved efficiency, automation is becoming an essential solution for the construction industry 
\cite{begic2022digitalization, chen2018construction, klarin2024automation}. Robotics, particularly the use of coordinated teams of robots, offers a promising approach that could revolutionize traditional construction practices. Robotic systems are being employed on construction sites to assist with tasks such as material delivery \cite{saidi2016robotics}, assembly \cite{liu2022digital, wang2023automatic, fu2025digital}, and installation \cite{m2023robotics, xu2024adaptive}, with the potential to significantly improve efficiency \cite{gharbia2020robotic, attalla2023construction} and safety \cite{xiao2022recent}. However, effectively coordinating multiple robots in dynamic and unpredictable construction environments remains a considerable challenge. Ensuring smooth interoperability and avoiding operational conflicts require sophisticated algorithms and management strategies \cite{fu2022robust, valero2023multi, lei2023convex}. Furthermore, the limited number of available robots and their diverse capabilities add complexity to task assignment. This necessitates task allocations that are not only efficient but also aligned with each robot’s specialized skills and functional strengths, thereby making the problem NP-hard \cite{abdzadeh2022simultaneous}. %However, existing studies have not formulated optimization methods specifically tailored for multi-robot task allocation in dynamic construction environments characterized by their inherent uncertainties.

Although existing solvers and algorithms based on standard Integer Programming (IP) \cite{wolsey2020integer} and Genetic Algorithms (GA) \cite{omara2010genetic} can address task allocation problems in multiple scenarios, they lack the flexibility to adapt to changes in construction sites. These methods are generally designed to find solutions under static constraints \cite{wang2020stochastic, perreault2025stochastic, abreu2020genetic} and are not inherently equipped to accommodate dynamic factors such as material supply shortages or changes in project priorities. Uncertainties in construction environments may arise from various sources, such as human decision-making, abrupt weather shifts, delays in material supply, and operational disruptions, all of which can influence task priorities, durations, or dependencies. For example, sudden weather changes, such as an unexpected rainstorm, might force the suspension of outdoor tasks \cite{deng2020research, schuldt2021weather}. Additionally, operational events like unforeseen robot malfunctions or the deployment of new robots require frequent recalculations to maintain optimal task assignments. However, purely mathematical algorithms lack the ability to reason about real-world construction site dynamics. They struggle to adapt task assignments in response to changing conditions, such as reassigning robots to indoor tasks like electrical wiring or interior finishing when outdoor work is halted due to weather. Recalculating task schedules in response to these changes often requires extensive manual intervention and algorithmic reconfiguration, which is time-consuming and impractical. 

The lack of adaptability of the existing task allocation approaches can result in project delays \cite{schuldt2021weather}, underutilized labor \cite{yu2022multiobjective, lee2017workflow}, and resource inefficiencies \cite{jin2020improving}. In addition, there is a notable gap in current research regarding the capability to perform real-time adjustments of task requirements based on actual construction progress. These limitations and gaps highlight the need for flexible approaches capable of adapting to the dynamic and ever-changing conditions of construction sites, leading to the following research questions:
\begin{itemize}
    \item \textbf{RQ1.} What task allocation approaches can effectively handle dynamic characteristics and uncertainties inherent in multi-robot operations on construction sites?
    \item \textbf{RQ2.} How can real-time task information from construction progress be used to dynamically adjust task assignments for multiple robots?
\end{itemize}

To address these research questions, this study introduces an optimization framework for multi-robot task allocation that integrates Large Language Models (LLMs)\cite{naveed2023comprehensive} with IP, enabling effective management of dynamic uncertainties typical of construction environments. The key contributions of this study are as follows. First, we develop a digital twin-based framework that enables communications between real-time robotic operations and digital building models (e.g., Building Information Models (BIMs)). This integration supports the dynamic updating of task statuses and material supplies for task management and robot control. Second, we formulate the multi-robot task allocation problem as an IP model specifically designed for the complexities of dynamic construction tasks, accounting for task interdependencies, heterogeneous robot capabilities, and evolving constraints. Third, we propose a novel mechanism for narrative-driven task allocation adaptation, whereby an LLM interprets unstructured natural language inputs and autonomously modifies optimization constraints. This approach facilitates human-in-the-loop decision-making while eliminating the need for manual reconfiguration of the optimization code. Lastly, we validate the proposed framework through extensive simulation-based experiments. They demonstrate both the computational efficiency of the optimization methods and the reasoning accuracy of multiple LLM models. The results demonstrate the framework’s effectiveness, adaptability, and potential for practical deployment in real-world construction scenarios.

The rest of this paper is organized as follows. Section II reviews the related literature and identifies key research gaps. Section III presents the detailed research methodology, including the design of the closed-loop digital twin system, the formulation of the optimization function, and the development of the LLM-driven adaptive decision-making mechanism. Section IV demonstrates and validates the proposed system and algorithms through a case study, with further discussion provided in Section V. Section VI concludes the paper.



%% Use \subsubsection, \paragraph, \subparagraph commands to 
%% start 3rd, 4th and 5th level sections.
%% Refer following link for more details.
%% https://en.wikibooks.org/wiki/LaTeX/Document_Structure#Sectioning_commands




%% Refer following link for more details.
%% https://en.wikibooks.org/wiki/LaTeX/Mathematics
%% https://en.wikibooks.org/wiki/LaTeX/Advanced_Mathematics

%% Use a table environment to create tables.
%% Refer following link for more details.
%% https://en.wikibooks.org/wiki/LaTeX/Tables




