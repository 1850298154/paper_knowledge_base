\section{Related Work}

\subsection{Robotic automation in construction}
Recent studies have explored the potential of integrating robots into construction tasks. Exploratory studies such as \cite{gautam2020collaborative} conducted by Gautam et al. examined the application of robots to assist with the repetitive task of screwing gypsum boards onto ceilings. During initial testing with two rows of screws, the robots demonstrated a success rate of about 78\%. Hard 'wooden eye' spots were also avoided through fine-tuning in the final phase, resulting in an average screwing time of 6.07 seconds per screw. Similarly, Kunic et al. \cite{kunic2021design} demonstrated a robotic workflow for assembling reversible timber structures through a multi-stage process. Using a kit of CNC-milled timber elements, the robots could handle the assembly with reversible joinery, allowing the structure to be disassembled and reassembled. A prototype showed that robots could build, disassemble, and reassemble complex structures with only a small amount of human assistance.

To further improve the efficiency, precision, and adaptability of robots, many researchers have also started to implement the concept of digital twins\cite{deng2021bim}. For example, a BIM-integrated system that enabled task planning and simulation for construction robots in indoor wall painting was developed by Kim et al. \cite{kim2021development}. The created IFC-SDF converter linked BIM data with the ROS and facilitated robot behavior simulations based on construction schedules and site conditions. The system showcased potential advancements in allowing robots to autonomously perform tasks in construction sectors. Likewise, Wang et al. \cite{wang2024enabling} presented a BIM-integrated framework that can dynamically update an interactive digital twin from BIM data, enabling robots to adjust work plans in response to site deviations with human oversight. The system combined human flexibility with robotic precision to improve construction robustness by demonstrating drywall and block placement tests.  Moreover, a BIM-enhanced robotic system was proposed by Follini et al. \cite{follini2020bim}, which could provide pre-existing geometric and semantic data. They used two applications to demonstrate that integrating BIM with robotic systems could improve environmental awareness and operational progress tracking, laying the groundwork for easier implementation of robots in construction. The percentage of improvement reached up to 53\%, with a mean value of 20\%.

While these studies have demonstrated the potential of implementing robots to assist with construction tasks, they primarily focus on single-task or narrowly scoped applications, without considering the scenarios when a variety of tasks and multi-robots exist. Additionally, existing approaches often require manual adjustments to accommodate evolving site conditions, highlighting a gap in automation for adaptive task management.

\subsection{Algorithms for multi-robot task allocation (MRTA)}
Driven by industrial demands and future trends, multi-robot task allocation (MRTA) has emerged as a significant research area in robotics, focusing on the efficient coordination and assignment of tasks among multiple robots to accomplish complex objectives \cite{chakraa2023optimization}. Existing studies have extensively explored various optimization methods to enhance performance metrics such as task completion time \cite{bai2022group, patil2022algorithm, guo2024effective} and resource utilization \cite{mayya2021resilient, lee2018resource}. For example,  game theory was often used for solving the MRTA, such as the study conducted by Martin et al. \cite{martin2023multi}, which employed the Shapley value to measure the individual contributions of robots and tasks. The authors partitioned complex MRTA problems into smaller subproblems by clustering robots and tasks based on their Shapley values, effectively balancing robot capabilities and task demands. The simulations demonstrated that this framework outperformed conventional metaheuristic algorithms, such as Genetic Algorithms (GA), in both solution quality and computational efficiency. Similarly, a game-theoretic decision-making algorithm for MRTA in changing environments was developed by Park et al. \cite{park2021multi}. The proposed approach enabled robots to iteratively adapt their task selections, guaranteeing convergence to optimal allocations. Experimental validation demonstrated the algorithm's effectiveness in responding to environmental changes and maintaining resilience against disturbances or robot failures. In addition, Yan and Di \cite{yan2023solving} introduced a hyper-heuristic algorithm for MRTA problems involving compulsory and optional (“functional") tasks. The developed algorithm assigned functional tasks using an influence diffusion model optimized by Particle Swarm Optimization (PSO). Simulation results demonstrated superior performance compared to existing benchmarks, particularly with increasing numbers of functional tasks. With more complex scenarios, a stochastic programming framework was proposed by Fu et al. \cite{fu2022robust} to simultaneously optimize task decomposition, assignment, and scheduling for heterogeneous robot teams. Unlike conventional approaches that assumed fixed task decompositions, this framework captured uncertainties in task requirements and agent capabilities using stochastic representations. It enhanced robustness and flexibility by minimizing risk through Conditional Value at Risk (CVaR). The results demonstrated scalability, low optimality gaps, and effective trade-offs among energy, time, and success probabilities. 

Despite considerable advancements, existing MRTA approaches often fail to adequately address dynamic uncertainties and unforeseen events prevalent in real-world scenarios, such as construction environments. Such uncertainties often necessitate human-like reasoning to manage effectively, as they can alter task dependencies, initiation times, and task durations.


\subsection{Applications of LLMs in construction}
The integration of LLMs into the construction industry has emerged as a transformative area of research, reshaping traditional approaches to data management, decision-making, and communication. Recent studies have explored the potential of LLMs to enhance construction information exchange \cite{wang2025integrated}, planning \cite{qian2025large}, and inspection \cite{pu2024autorepo} through their advanced information processing capabilities. For example, Chen et al. \cite{chen2025meet2mitigate} introduced a Meet2Mitigate (M2M) framework to automate the capture, transcription, and analysis of construction meetings by integrating speaker diarization, automatic speech recognition (ASR), and LLMs with retrieval-based summarization. The M2M prototype showed significant accuracy, promising a seamless, automated solution for meeting recaps that enhances project management and decision alignment across teams. The Word Error Rate (WER) and Character Error Rate (CER) were only 4.68\% and 4.31\%, respectively. This study confirmed LLMs' ability to extract essential information from fragmented content and demonstrated their potential to obtain real-time construction site information through communicating with human.  In addition, the capabilities of LLMs in information integration were investigated by Forth and Borrmann \cite{forth2024semantic}, who proposed a methodology to automate the enrichment of BIM models with missing data. By leveraging Semantic Textual Similarity (STS) and fine-tuned LLMs, the approach enabled efficient mapping of thermal properties to construction elements, such as walls and spaces, based on architectural and mechanical, electrical, and plumbing (MEP) data. Tested on real-world case studies, the framework demonstrated high accuracy in construction-specific matching tasks. Moreover, Prieto et al. \cite{prieto2023investigating} explored the potential of LLM for generating construction schedules, finding that it can produce coherent and logically organized schedules that fulfill project requirements. Participants reported a positive interaction experience with the tool. One step further, Kim et al. \cite{kim4827728context} developed a framework integrating LLMs, BIM, and robotic systems to enhance autonomous task planning in construction, aiming to address challenges with dynamic site conditions. The prototype achieved high accuracy in managing task responses and adapting to environmental changes, demonstrating the potential for improving task adaptability and efficiency in complex construction environments. In terms of construction site inspection, a framework named AutoRepo, which could use multimodal LLMs and unmanned vehicles to automate report generation, was presented by Du et al. \cite{pu2024autorepo}. It aimed at improving efficiency and reducing human error in traditional inspection practices. AutoRepo had shown the potential to deliver high-quality, standardized reports with greater accuracy and speed. The reports generated by the framework were all scored higher than 80 (considered qualified) based on expert reviews.

Although existing studies demonstrated promising applications of LLMs in construction-relevant tasks, there was a lack of a comprehensive framework for dynamic, multi-robot task management in rapidly changing construction environments. They did not leverage the potential of LLMs to autonomously reassign and adapt tasks in response to real-time environmental factors. Additionally, while some frameworks incorporated structured data sources (e.g., BIM), they did not integrate continuous, real-time feedback from digital twins to support responsive task reallocation across multiple robots.

