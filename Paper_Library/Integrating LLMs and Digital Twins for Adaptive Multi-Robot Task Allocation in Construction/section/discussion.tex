\section{Discussion}
The findings of this study highlight the potential of integrating LLMs, optimization algorithms, and digital twins to address  challenges in multi-robot coordination on dynamic construction sites. The results have demonstrated the effectiveness of the adaptive system architecture. By separating the reasoning capabilities of the LLM from the task optimization algorithm itself, the framework maintains computational rigor while maintaining flexibility. The results indicate that the optimization model remains interpretable and robust while benefiting from the LLM’s ability to interpret unstructured input and dynamically generate task-specific modifications. Although the accuracy of LLMs in interpreting construction-relevant narratives varies across models, the results underscore their practical viability. In particular, advanced models such as GPT-4.1 consistently extracted constraints and parameters with high accuracy, even as task complexity increased. This suggests strong potential for LLMs to serve as intelligent intermediaries between human planners and autonomous systems, especially in time-sensitive, information-rich environments. The integration of a digital twin enabled a closed-loop information flow between the physical construction site and its virtual representation. Real-time updates of task progress, robot states, and environmental changes are synchronized through the digital twin, which allows for continuous refinement of task allocations and site-level decision-making. This capability is especially valuable in dynamic construction settings, where conditions shift frequently and adaptability is paramount.
The simulation results have supported the practicability of the proposed framework.

Beyond the context of construction, the proposed framework exhibits strong potential for scalability and cross-domain applications. The generalizable principles of optimization, real-time feedback, and natural language reasoning, makes it highly adaptable to other sectors such as manufacturing, logistics, warehousing, and industrial assembly. In these domains, similar to construction, tasks often follow strict sequences, depend on heterogeneous agent capabilities, and may subject to unexpected disruptions. Additionally, the digital twin integration offers significant benefits for sectors that rely heavily on spatial coordination and progress tracking, such as infrastructure development and maintenance.

Despite some promising outcomes, there are several limitations in this study. For example, the current simulation setup simplifies certain aspects of real-world construction scenarios. In particular, the collaboration between robots and human workers is not explicitly modeled, resulting in idealized task transitions that may differ slightly from real-world conditions. Additionally, while LLMs show strong performance in structured prompt settings, their reliability in processing more ambiguous or informal construction narratives need further investigations.
