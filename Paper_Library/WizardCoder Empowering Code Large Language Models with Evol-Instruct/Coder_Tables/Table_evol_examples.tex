\begin{table}[!htb]
    \centering
    \caption{Examples of interaction with our \modelname{}.}
    \begin{tabular}{p{2.5cm}|p{10cm}}
        \toprule
        \textbf{Instruction} & \textbf{Response}\\
        \midrule
        Write a Python function to tell me what the date is today. &
        \begin{minipage}{29em}
            Here's a Python function that tells you what the date is today:

\begin{lstlisting}[language=Python]
import datetime

def get_date():
    now = datetime.datetime.now()
    return now.strftime("%A, %B %d, %Y")

print(get_date())
\end{lstlisting}

This function uses the datetime module to get the current date and time. It then uses the strftime() method to format the date as a string in the format "Day, Month Date, Year". Finally, it returns the formatted string.\\

You can call this function from your Python code to get the current date and time.
        \end{minipage}\\
        \midrule
        I do not know how to integrate a Python function numerically. Please write a code for me. &
        \begin{minipage}{29em}
        Sure, here's an example code for integrating a Python function numerically using the trapezoidal rule:

\begin{lstlisting}[language=Python]
def integrate(f, a, b, n):
    h = (b - a) / n
    s = (f(a) + f(b)) / 2
    for i in range(1, n):
        s += f(a + i * h)
    return h * s
\end{lstlisting}

This code takes in a function f, the lower bound a, the upper bound b, and the number of subintervals n. It then calculates the width of each subinterval h, and uses the trapezoidal rule to approximate the integral of f between a and b. The result is returned as a float.

To use this function, you would call it with the function you want to integrate, the lower bound, the upper bound, and the number of subintervals you want to use. For example:

\begin{lstlisting}[language=Python]
def f(x):
    return x**2

result = integrate(f, 0, 1, 1000)
print(result)
\end{lstlisting}

This would integrate the function f(x) = x**2 between 0 and 1 using 1000 subintervals, and print the result.
        \end{minipage}\\
        \midrule
        List the prime numbers between 20 and 30 with Java.&
        \begin{minipage}{29em}
Here's the Java code to list the prime numbers between 20 and 30:

\begin{lstlisting}[language=Java]
public class PrimeNumbers {
    public static void main(String[] args) {
        for (int i = 20; i <= 30; i++) {
            boolean isPrime = true;
            for (int j = 2; j < i; j++) {
                if (i % j == 0) {
                    isPrime = false;
                    break;
                }
            }
            if (isPrime) {
                System.out.print(i + " ");
            }
        }
    }
}
\end{lstlisting}

Output:

\begin{lstlisting}[language=Java]
23 29
\end{lstlisting}

Explanation:

- We use a nested for loop to iterate through all the numbers between 20 and 30.

- For each number, we use another for loop to check if it is prime or not.

- If a number is divisible by any number other than 1 and itself, then it is not prime.

- If a number is not divisible by any number other than 1 and itself, then it is prime.

- We print the prime numbers to the console.
        \end{minipage}\\
        \bottomrule
    \end{tabular}
    \label{tab:example}
\end{table}