% section: introduction

The 
2020 IWPT Shared Task on enhanced dependency parsing
\cite{bouma-etal-2020-overview}
requires participants
to predict the enhanced dependencies
(DEPS column in the CoNLL-U format)
in addition to sentence boundaries, tokenisation, lemmata, POS tags,
morphological features and the basic dependency tree.
We take a pipeline approach using
\begin{enumerate}
    \item UDPipe for sentence splitting and tokenisation,
    \item ensembles of UDPipe-future basic parsers, that also
          predict lemmata, POS tags and morphological features,
          with added support for multi-treebank models
          \cite{stymne-etal-2018-parser}, and
    \item two types of enhancers: \textit{(a)} copying the basic tree and applying a small set of heuristics 
    (baseline system), and \textit{(b)} a graph-based semantic dependency parser \citep{dozat-manning-2018-simpler}.
\end{enumerate}

To enable reproduction of our results, we make available our helper scripts
and modifications of the semantic
parser.\footnote{\url{https://github.com/jbrry/Enhanced-UD-Parsing}}

Our approach to the task does not guarantee a connected graph -- something that we did not account for.
Thus, on submission day, we did not have an appropriate solution ready to fix our outputs
but
were able to provide a valid submission due to some
functionality that was added to the
\texttt{quick-fix}
tool provided by the
organisers\footnote{\url{https://github.com/UniversalDependencies/tools}}
to alter the enhanced graph.
The solution was designed 
primarily
to make the files 
pass validation
but in doing so,
harms F1-score.
In a post-competition run, we addressed the connected graph issue with an alternative solution which increased our macro-averaged ELAS F1-score from 67.23 to 79.53 and
the treebank average from 67.49 to 79.76.


% eof