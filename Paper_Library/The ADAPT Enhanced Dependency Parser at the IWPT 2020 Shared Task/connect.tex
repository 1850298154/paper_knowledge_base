% section: connect the graph

We had no solution ready to connect fragmented graphs
produced by our semantic parser\footnote{Between
    90.18\% (Lithuanian) and 99.51\% (Russian) 
    of test sentences in our official submission are not affected,
    \ie all nodes are reachable from a root node.
    This observation excludes Estonian, for which we submitted
    predictions using our heuristic system.
}
on the system submission day
and resorted to using the
``connect-to-root'' option of the \texttt{quick-fix} tool provided 
by the shared task organisers, who warned that it
had not been thoroughly tested.

After the system submission deadline, we investigated the fragmentation
issue.
The task is to make all nodes reachable from 
the notional ROOT\footnote{UD distinguished between
    the notional ROOT (ID 0) and root nodes. The latter are any nodes
    that have `0' as a head.
    %In the enhanced graph, there can be multiple root nodes.
}, where reachability is directional.
Adding more edges than necessary harms precision and thus F1-score.
We found that the \texttt{quick-fix} tool with the
``connect-to-root'' option adds edges to every unreachable node.
We also noticed a bug in the implementation where certain reachable nodes were
being reported as unreachable.

We then
implemented an improved tool to connect fragmented graphs trying to minimise
the number of edges added to the graph.
We repeatedly check for each unreachable node how many unreachable nodes
can be reached from it.
Among the nodes that maximise this number we
pick the first node in surface order and make it a child of
the notional ROOT, \ie it becomes an additional root node.
This is a rather naive approach which does not try to connect fragments in a sensible manner but, rather, mimics the behaviour of the ``connect-to-root'' option.
Future work could try to show whether our above algorithm adds the minimal
number of edges necessary to connect the graph or if a lower optimum exists.

% eof