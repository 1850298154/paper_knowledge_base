\documentclass[conference]{IEEEtran}
\IEEEoverridecommandlockouts
% The preceding line is only needed to identify funding in the first footnote. If that is unneeded, please comment it out.
\usepackage{cite}
\usepackage{amsmath,amssymb,amsfonts}
\usepackage{algorithmic}
\usepackage{graphicx}
\usepackage{textcomp}
\usepackage{xcolor}
\usepackage{float}
\usepackage{soul} % for the command \hl

\graphicspath{ {./Figures/} }
\def\BibTeX{{\rm B\kern-.05em{\sc i\kern-.025em b}\kern-.08em
    T\kern-.1667em\lower.7ex\hbox{E}\kern-.125emX}}


\begin{document}

\title{A Method for Multi-Robot Asynchronous Trajectory Execution in MoveIt2 (Extended Abstract)}

\author{\IEEEauthorblockN{Pascal Stoop}
\IEEEauthorblockA{\textit{ILT,} \textit{OST}\\
%\textit{name of organization (of Aff.)}\\
Rapperswil SG, Switzerland \\
pascal.stoop@ost.ch}
\and
\IEEEauthorblockN{Tharaka Ratnayake}
\IEEEauthorblockA{\textit{InIT,} \textit{ZHAW}\\
Winterthur, Switzerland \\
raty@zhaw.ch}
\and
\IEEEauthorblockN{Giovanni Toffetti}
\IEEEauthorblockA{\textit{InIT,} \textit{ZHAW}\\
Winterthur, Switzerland \\
toff@zhaw.ch}
}

\maketitle

\begin{abstract}
This work presents an extension to the MoveIt2 planning library supporting asynchronous 
%(i.e., uncoordinated) 
execution for multi-robot/multi-arm robotic setups. The proposed method introduces a unified way for the execution of both synchronous and asynchronous trajectories by implementing a simple scheduler and guarantees collision-free operation by continuous collision checking while the robots are moving. 
%We also propose a suitable test apparatus to benchmark the proposed system w.r.t the state-of-the-art synchronous execution method.

\end{abstract}

\begin{IEEEkeywords}
Robotics, multi-arm, multi-robot, motion planning, pick and place, moveit
\end{IEEEkeywords}

\section{Introduction}


\IEEEPARstart{T}{wo} %
main challenges in the deployment of large-scale swarms are the localization and coordination of vehicles.
Localization methods that rely on external infrastructure 
(e.g., GPS) 
are prone to systematic errors (e.g., multipath effect)
and may not always be available.
Coordination strategies that are centralized can deconflict motion plans to prevent collisions and gridlock, but introduce a single point of failure and are difficult to scale in swarm size due to communication bandwidth limitations.

This paper presents a unified formation flying pipeline for unmanned aerial vehicles (UAVs).
Our pipeline uses \textit{onboard} sensors for localization, which eliminate the need for external positioning systems, and \textit{distributed} techniques for coordination, which enable each vehicle to make decisions independently while communicating their state to a subset of the team.
For \textit{localization}, we use an off-the-shelf commercial visual inertial odometry (VIO) package \cite{VIO}
that fuses inertial measurement unit (IMU) and downward-facing monocular camera measurements to estimate changes in the vehicle pose.
\edit{For \textit{coordination}, we present distributed formation control and task assignment strategies that run onboard the vehicles, do not rely on a common reference frame, and use vehicle-to-vehicle communication.} 
Key features of our formation control strategy include scalability to a large number of vehicles and robustness to disturbances.
The latter is crucial for reaching the desired formations with sensing imperfections.
Our task assignment strategy uses an auction-based algorithm to guarantee conflict-free assignments.
This algorithm can deconflict vehicle gridlocks resulting from distributed collision avoidance (type 3 deadlock~\cite{Wang2017}) and is well-suited for vehicles with limited computational capability and low-bandwidth communication. 


\begin{figure}[t!]
	\begin{center}
		\includegraphics[trim =0mm 10mm 0mm 0mm, clip, width=\columnwidth]{Figs/slanted_plane.png}	
		\caption{
		Six multirotors in a slanted plane formation.
		Vehicles communicate with each other, make distributed decisions onboard, and use VIO for localization.}
		\label{fig:slantedplane}
	\end{center}
\end{figure}


\subsection{Contributions}

This research extends our previous work on UAV formations~\cite{Fathian2019} and presents a unified pipeline consisting of \textit{onboard localization} and \textit{distributed coordination}.
The three main contributions of this work are:
\begin{enumerate}
    \item \edit{scalable formulation of control design suitable for
    onboard sensing without a common reference frame;}
    \item algorithms for deconfliction via \edit{distributed} task assignment of vehicles to desired formation points;    
    \item simulation- and hardware-ready open-source pipeline.
\end{enumerate}
\edit{Our pipeline is tested in hardware with six multirotors (see Fig.~\ref{fig:slantedplane}), and 
to our knowledge is the first demonstration of formation flying that does not rely on external sensing, fiducial markers for localization, a common reference frame, or a centralized base station for coordination.}
The only requirements for the presented pipeline are that the vehicles can communicate, can find the transformation between their VIO start frames, and the environment is sufficiently textured---a standard assumption for VIO systems.
As such, this framework paves the way for future, real-world deployments of aerial vehicle swarms in large numbers and without requiring external localization infrastructure.


\begin{figure} [t!]
\centering
	\begin{subfigure}[b]{0.32\columnwidth}
	   %
	    \includegraphics[width=0.8\textwidth,left]{Figs/Frames2_full.pdf}
	    \caption{\scriptsize full alignment}
	    \label{fig:frame-a}
	\end{subfigure}
	\begin{subfigure}[b]{0.32\columnwidth}
	    \includegraphics[width=0.8\textwidth,center]{Figs/Frames2_orientation.pdf}
	    \caption{\scriptsize orientation alignment}
	    \label{fig:frame-b}
	\end{subfigure}
	\begin{subfigure}[b]{0.32\columnwidth}
	    \includegraphics[width=0.8\textwidth,right]{Figs/Frames2_none.pdf}
	    \caption{\scriptsize no alignment}
        \label{fig:frame-c}
	\end{subfigure}
\caption{\edit{Required alignment of UAV frames in existing swarm strategies: (a) the most restrictive case requiring a common reference frame, i.e., orientation and origin of the frames must be aligned; (b) only the orientation of the frames must be aligned; (c) no alignment restrictions (this work).}}
	\label{fig:Frames}
\end{figure}




\subsection{Related Work}

Existing aerial swarms can be grouped based on the coordination (centralized vs.\ distributed) and localization (external vs.\ onboard) methods used. 
\edit{It is further crucial to distinguish these methods based on the level of alignment required for the vehicle coordinate frames; see Fig.~\ref{fig:Frames}.} 
 
\edit{
Works with \textit{centralized} coordination and \textit{external} localization include~\cite{Preiss2017, Honig2018, Du2019}, which are based on lightweight UAVs with limited onboard computational capability and therefore rely on an external motion capture system and a base station.
Works with \textit{distributed} coordination and \textit{external} localization include \cite{wilson2020robotarium}, \cite{enright2004spheres}, where robots execute distributed controls  based on external localization by motion capture and ultrasonic beacons, respectively.
Works with \textit{centralized} coordination and \textit{onboard} localization include~\cite{Forster2013}, \cite{Loianno2016}, which use a ground station for task assignment among vehicles.
In \cite{Weinstein2018}, formation flying based on VIO is demonstrated, where motion planning and assignment are run on a base station to ensure collision-free trajectories.
The coordination strategies used in aforementioned works require a \textit{common reference frame} (Fig.~\ref{fig:frame-a}).
}


\edit{
Despite the large body of work on formation control~\cite{Oh2015}, and the variety of onboard sensing solutions for localization (e.g., VIO~\cite{Delmerico2018}), few frameworks demonstrated formation flying with \textit{distributed} coordination and \textit{onboard} localization.
A key reason is reliance of many distributed control and assignment algorithms on aligned frames (Fig.~\ref{fig:frame-a}, \ref{fig:frame-b}), which require computation-expensive and/or communication-intensive synchronization/consensus steps for frame alignment.
Equally important, dependence on alignment in existing methods \cite{Wang2017,Turpin2014, van2011reciprocal, morgan2016swarm} diminishes robustness to inherent noise and unobservable errors that cannot be corrected (e.g., disparities between the actual and estimated body frame \textit{orientation} caused by VIO drift).
Leveraging coordination methods that are \textit{robust to misaligned frames} is hence crucial and a focus of this work. 
}






\edit{
Examples of other pipelines with distributed coordination and onboard localization include \cite{Montijano2016,Tron2016}.
Both works demonstrated formation flying on three UAVs, required information from an external motion capture system due to hardware limitations, did not incorporate collision avoidance, and required frame alignment.
}
\edittwo{Note that while~\cite{Montijano2016,Tron2016} can achieve formations with arbitrary headings as illustrated in Fig.~\ref{fig:frame-c}, knowledge of relative orientations is still required; therefore, they belong to the category of Fig.~\ref{fig:frame-b}.}






\if 0

\r{
decentralized coordination setting combined with VIO:
D-CAPT [26]~\cite{}:
ORCA ~\cite{}: 
CBF [2]~\cite{} :
[A]
}

\r{Robusteness in coordination,  with compounded noise/latency, which would eventually break (b).\\


some existing algorithm might as well
work in a similar fully decentralized setting, when combined with VIO
as proposed here. For example, D-CAPT [26], ORCA, CBF [2] might also be
useful for such a task and are computationally even more efficient than
the proposed approach. \\

R2:  onboard sensing for localization ->
 Finally, the related work section only
focuses on this aspect of the pipeline, discussing how many formation papers include
onboard localization but barely sells the advantages of the coordination module (the actual
proposal of the paper) against other competitors such as [26] or [A] or to mention similar
coordination pipelines. \\


Given a solution to this problem, the controller in Section III seems unnecessary, each drone
has a target position and can use a local controller with collision avoidance that drives it to
that position. Note that such controllers exists in the literature (e.g., RVO in any of its
multi-agent variantes), they are distributed in nature and only require local sensing.


}

\fi

\section{Approach}
\section{Approach}
\label{sec:approach}

We first describe how we extended the \shop planner to emit dependency information to support provenance.
We then describe our approach with respect to relevant questions from the planning literature \cite{MariaFoxExplainablePlanning2017} and information analysis literature \cite{icd_203,icd_206,zelik2010measuring} that have been proposed as primary targets for integrity and explainability.
We describe relevant representations and algorithms in our approach as they apply to these questions.

\subsection{\shop and Provenance Tracking}
\label{sec:planner-extensions}

In related work on plan repair~\cite{Goldman:StablePlanRepair:2020}, we have augmented \shop so that, when planning, it builds a plan tree that has dependency information (causal links).  These links allow the plan repair system to identify the minimally compromised subtree of the plan, as a way to provide \emph{stable}, minimal-perturbation plan repairs.  This extension provides much of the provenance information that we need for explainability, because it allows us to trace the choice of methods and primitive actions back to other choices that enabled them.
The present approach extends the scope and semantics of these links to
\begin{enumerate*}[label=(\arabic*)]
\item trace decisions back to the model components that justify them and
\item trace preconditions back to actions that establish them and information sources that provided them.
\end{enumerate*}

\textbf{Tracing decisions back to model components} is straightforward: the \shop planner takes as input \texttt{domain} and \texttt{problem} data structures, and the \texttt{domain} data structures contain the model components, specifically the primitive operator and method definitions.  For the moment, we do not track the provenance of components of the planner's model.
However, since the domain descriptions are typically maintained in a revision control system, such as subversion or git, it would be relatively simple to extend our provenance tracing back to the person or persons who wrote these model components.  For a more sophisticated development environment, one could imagine a traceback that reaches into an integrated development environment or a machine learning system.

\textbf{Tracing decisions back to information sources} is somewhat more difficult. In the base case, a proposition is established in the \texttt{problem} data structure -- that is, in the initial state.  In a larger system that incorporates \shop, there is generally a component that builds these \texttt{problem} data structures.  For example, in a robot planning system, we generally have a component that builds problems programmatically from user input (tasks to achieve) and some source of external information (\eg{}, a map database, telemetry from robotic platforms, \etc).  These components can annotate the initial state (and potentially the tasks \shop is asked to plan) with provenance information, using PROV-DM in a way that is appropriate to the application domain.  This provenance information can then be propagated through the causal links in the plan tree.

There is one remaining complication: in the interests of modeling efficiency and expressivity, \shop incorporates a theorem-prover -- a backward-chaining engine inspired by Prolog.
This is necessary because \shop{}'s expressive power is not limited to propositional logic, the way most planners are: it permits state axioms, and non-finite domains of quantification.\footnote{Note that though critical to \shop{} applications, these features must be used with care, because they can compromise soundness and completeness.} Thus some preconditions may be established not just causally, but inferentially, through Horn clause (``axiom'') deduction.  Accordingly, we must extend our theorem-prover so that it also provides traceability. Provenance annotations that traced provenance through axioms back to actions that established antecedents for the axioms were already in place for plan repair. These will now automatically incorporate information source provenance, as well as causal provenance.  At the moment, we do not trace the axioms themselves, but this would be a trivial extension.

\subsection{Mapping \shop plans to PROV}

Our system converts the extended \shop plans into the PROV data model, using the PROV-O ontology to represent the elements and relationships between them.
\figref{prov-to-plan} illustrates the SHOP-to-PROV mapping in a screenshot of our system displaying \shop planner output (some elements removed for simplicity).
The plan content in \figref{prov-to-plan} displays a single goal (at right) to transmit image data of \textbf{objective1} in high-resolution, and this goal is supported by two paths of tasks, performed by two separate agents (the aerial unit \textbf{flier1} and the land unit \textbf{rover0}), with foundational beliefs derived from a \textbf{Terrain Map} and an \textbf{Elevation Map}.
We use the following mapping:
\begin{itemize}
  \item \textbf{Planned Tasks} are specializations of \textbf{prov:Activity}.  Unlike traditional uses of provenance for tracking \textit{past} events, the PROV Activities from the plan may not yet have--or may never actually--occur.
  \item \textbf{Plan Actors} are specializations of \textbf{prov:Agent}.
  They are the performers of the PROV Activities, related via \textbf{prov:wasAssociatedWith} (see \figref{prov}).
  \item \textbf{Plan Beliefs} are specializations of \textbf{prov:Entity}.  They support tasks with \textbf{prov:used} and they are realized by tasks with \textbf{prov:wasGeneratedBy}.
  \item \textbf{Information Sources} are specializations of \textbf{prov:Entity}.  They represent sensors and repositories that emit information to derive beliefs and measurements used in the plan, and support beliefs via \textbf{prov:wasDerivedFrom}.
\end{itemize}

As shown in the \figref{prov-to-plan} screenshot, the resulting provenance graph incorporates the information sources (at left) with the goals of the plan (at right), and the dependency network between them.
We use this \shop plan to acquire \textbf{objective1} imagery--with one or more possible planned courses of action--to articulate our approach below.

\begin{figure}
  \begin{center}
  \frame{
  \includegraphics[width=\linewidth]{figures/ss-index.png}}
  \caption{The index of agents, sources, source classes, and operation classes used in a plan.}
  \label{fig:ss-index}
  \end{center}
\end{figure}

\subsection{Indexing the Dimensions of a Plan}

Given a plan to assess, our provenance system automatically identifies and catalogs the following dimensions of the plan.  These are displayed for user assessment and dynamic interaction, as shown in \figref{ss-index}.
\begin{enumerate}
  \item \textbf{Contributing Agents}: Actors in the plan.
  \item \textbf{Source Entities}: Individual devices or informational resources from which plan-relevant beliefs are derived, such as geolocation, visibility, inventory, and more.
  \item \textbf{Source Classes}: General categories of information across beliefs and information sources.
  These may include information sources % (\eg{}, GEOINT, IMINT, etc.)
  or belief predicates, as shown in \figref{ss-index}.
  % I dropped the IC jargon that the reader might not understand, and that we don't need to explain.
  \item \textbf{Operation Classes}: General categories of activities, spanning potentially many planned activities.
  In \figref{ss-index}, we catalog classes of actions.
\end{enumerate}
\noindent
Cataloging plan nodes along these dimensions allows our approach to instantaneously identify, emphasize, or refute nodes along these dimensions to support explanation.
These elements are identified by mining the predicates and sources of the plan, but could also be informed by the planner's model, in future work.

We use an algorithm similar to assumption-based truth-maintenance and explanation-maintenance systems \cite{forbus1993building,friedman2018csj} to compute the \emph{environment} of all nodes (\ie{}, planned action or belief) in the provenance graph.
The algorithm traverses backward exactly once from all sink nodes, so it reaches each node $m$ in the provenance graph and computes its environment $E(m) = \{S_1, ..., S_n\}$, a disjunction of sets ($S_i$) of \emph{assumptions}, where any $S_i \in E(m)$ is sufficient to derive (\ie{}, believe, achieve, or enact) $m$, and where the assumptions correspond to root nodes in the provenance graph.
The algorithm attends to the AND- and OR-like links listed in \figref{prov-to-plan} to properly encode disjunctive derivation trees.
This compact index answers questions of necessity and sufficiency in constant time.

The joint indexing of plan nodes by the four above dimensions and by their environments allows the provenance analysis system to identify abstract classes of sources and operations that contribute to it, and that it contributes to.
We leverage these indices to help explain the plan in context, as we describe below.

\begin{figure*}[t]
  \begin{center}
  \frame{
  \includegraphics[width=\textwidth]{figures/ss-nocross.png}}
  \caption{The \figref{prov-to-plan} plan to acquire imagery of \textbf{objective1}, with moderately low (0.20) confidence ascribed to the \textbf{Terrain Map} and moderately high (0.80) confidence ascribed to the \textbf{Elevation Map}.}
  \label{fig:ss-compare}
  \frame{
  \includegraphics[width=\textwidth]{figures/ss-impact-take-image.pdf}}
  \caption{Viewing the relevance and reachability of the \textbf{take-image} action, including the sources it relies on, and the impact on downstream actions.}
  \label{fig:ss-impact-take-image}
  \end{center}
\end{figure*}

\subsection{Visualization Environment}
\label{sec:viz}

Our visualization environment is a graphical display within a larger web-based platform for human-machine collaborative intelligence analysis.
At any time, the user may select one or more elements from diagrams or listings and peruse its full provenance.

A web service traverses the knowledge graph to retrieve the full provenance for the desired belief(s) and all relevant appraisals, and then sends it to the client.
The client's provenance visualizer uses D3.js, as shown in the \figref{prov-to-plan} and \figref{ss-index} screenshots, to implement the rendering, refutation, emphasis, and propagation effects described below, operating over the PROV and DIVE representations.


%%% Local Variables:
%%% mode: latex
%%% TeX-master: "main"
%%% End:


% \section{Selected Experiments}
% %!TEX root = main.tex

\section{Experimental Evaluation}
\seclabel{experiments}

We first evaluated our algorithms
in an offline setting~(\secref{offline-expr}), where we record execution traces and evaluate different approaches on the \emph{same} input.
This eliminates biases due to non-deterministic thread scheduling.
Next, we consider an online setting~(\secref{online-expr}),
where we instrument programs and perform the analyses during runtime.
We conducted all our experiments on a standard laptop with \SI{1.8}{GHz} Intel Core i7 processor and \SI{16}{GB} RAM.

% We evaluated our algorithms in two experimental settings. 
% The first setting is offline experiments~(\secref{offline-expr}),
% in which we record execution traces and evaluate different approaches.
% This has the benefit that different approaches can be compared on the \emph{same} input,
% thereby eliminating biases due to non-deterministic thread scheduling.
% The second setting is online experiments~(\secref{online-expr}),
% in which we instrument programs and perform the analyses during runtime.
% We conducted all our experiments on a standard laptop with 1.8GHz Intel Core i7 processor and 16GB RAM.

%We compare our predictive algorithm with the \dlfuzzer tool. 
%This setting enables us to assess the applicability of our technique in a runtime monitoring system.
%The state-of-the-art deadlock predictors \dirk and \seqc work offline by design.
%Hence, they are not applicable for a comparison in this setting.
%As discussed in~\secref{otf}, offline methods are not directly applicable in runtime monitoring systems.

%Talk about implementation - name of tool + prog lang + traces are logged + filtering phase + cycle detection phase + online vs offline + trace conversion for seqc
%
%Setup - benchmarks + log traces using so-and-so-tools + 1-trace-per-benchmark  + cluster details + how many runs per trace + Timeout (if any)
%
%benchmark details: number of shared locks
%\subsection{Evaluation}
%
%Some suggested experiments:
%
%\begin{itemize}
%	\item Comparison with other tool(s) - \dirk and \seqc
%	\item Comparison of online vs online algorithms
%	\item Scalability with number of events
%	\item Scalability with number of threads in the trace
%	\item Scalability with number of threads in the deadlock pattern
%	\item Time spent in each kinds of events (this will be useful to guide future research)
%\end{itemize}
%
%\Andreas{For matching reports, we slightly adapt the algorithm to report all sets of program locations that contain a deadlock pattern (hence we might have more than one reports per abstract deadlock pattern, corresponding to different program locations)}


% Here we report on an implementation and experimental evaluation of our algorithms.

% \subsection{Experimental Setup}



\subsection{Offline Experiments}
\seclabel{offline-expr}


\Paragraph{Experimental setup}
The goal of the first set of experiments is to evaluate 
$\SyncPDOffline$, and compare it
against prior algorithms for dynamic deadlock prediction.
In order for our evaluation to be precise we evaluate all algorithms on the \emph{same} execution trace.
We implemented $\SyncPDOffline$ in Java inside the \toolname analysis tool~\cite{rapid}, 
following closely the pseudocode in \algoref{offline}.
\toolname takes as input execution traces, as defined in \secref{prelim}.
These also include fork, join, and lock-request events.
We compare $\SyncPDOffline$ with two state-of-the-art, 
theoretically-sound albeit computationally more expensive, deadlock predictors,
\seqc~\cite{Cai2021} and \dirk~\cite{Kalhauge2018}, both of which also work on execution traces.




On the theoretical side, the complexity of \seqc is $\Otilde(\NumEvents^4)$, 
as opposed to the $\Otilde(\NumEvents)$ complexity of $\SyncPDOffline$. 
Moreover, \seqc only predicts deadlocks of size $2$, and though it could be extended to handle deadlocks of any size, this would degrade performance further.
\seqc may miss sync-preserving deadlocks even of size $2$, 
but can  detect deadlocks that are not sync-preserving.
Thus \seqc and $\SyncPDOffline$ are theoretically incomparable in their detection capability.
We refer to\begin{pldi}~\cite{arxiv}\end{pldi}\begin{arxiv}~\appref{incomp}\end{arxiv} for examples.
We noticed that \seqc fails on traces with non-well-nested locks --- we encountered one such case in our dataset.
\dirk's algorithm is theoretically complete, i.e., it can find all predictable deadlocks in a trace.
In addition, it can find deadlocks beyond the predictable ones, by reasoning about event values.
However, \dirk relies on heavyweight SMT-solving and
employs windowing techniques to scale to large traces. 
Due to windowing, it can miss deadlocks between events that are outside the given window. 
%\hunkar{
As with previous works~\cite{Cai2021, Kalhauge2018}, we set a window size of $10$K for \dirk.
%}

Our dataset consists of several benchmarks 
from standard benchmark suites --- IBM Contest suite~\cite{Farchi03}, Java Grande suite~\cite{Smith01},
DaCapo~\cite{Blackburn06}, and
SIR~\cite{doESE05} ---
and recent literature~\cite{Kalhauge2018, Cai2021, jula2008deadlock, Joshi2009}.
Each benchmark was instrumented with RV-Predict~\cite{rvpredict} or Wiretap~\cite{Kalhauge2018} and
executed in order to log a single execution trace.

%!TEX root = ../main.tex

\begin{table}[h!]
\caption{
%\hcomment{Should we remove some not very important rows from this table?}
Trace characteristics, abstract lock graph statistics and performance comparison.
%Column 1 denotes the name of the benchmark.
Columns 2-6 show the number of events, threads, variables, locks
and total number of lock acquire and request events.
Columns 7-9 show the number of cycles, abstract and concrete deadlock patterns in the abstract lock graph.
Columns 10 - 15 show the number of deadlocks reported and the times (in seconds) taken. 
by \dirk, \seqc, and \SyncPDOffline.
%Statistics on the lock graph $\lkevgraph{\tr}$ of each trace $\tr$.
% We denote by $\mathcal{N}$, $\mathcal{T}$, $\mathcal{M}$, $\mathcal{L}$ and $\NumAcquires + \mathcal{R}$ the total number of events, number of threads, number of memory locations, number of locks and number of acquire and request events in the benchmarks, respectively. 
%All times are in seconds. 
Time out (T.O) was set to $3$h.
F stands for technical failure.
\label{tab:expr-results}
}
\vspace{-0.17cm}
\setlength\tabcolsep{3pt}
\renewcommand{\arraystretch}{0.91}
\centering
\scalebox{0.86}{
\begin{tabular}{|r|c|c|c|c|c||c|c|c||c|c||c|c||c|c|}
\hline
1 & 2 & 3 & 4 & 5 & 6 & 7 & 8 & 9 & 10 & 11 & 12 & 13 & 14 & 15 \\
\hline
\multirow{2}{*}{\textbf{Benchmark}}& 
\multirow{2}{*}{$\mathcal{N}$} & \multirow{2}{*}{$\mathcal{T}$} & \multirow{2}{*}{$\mathcal{V}$} & \multirow{2}{*}{$\mathcal{L}$} & \multirow{2}{*}{$\NumAcquires/\mathcal{R}$} 
& \multicolumn{3}{c||}{ \textsf{A. Lock Graph }}
& \multicolumn{2}{c||}{ {\dirk} } 
& \multicolumn{2}{c||}{ {\seqc}} 
& \multicolumn{2}{c|}{ {\textsf{$\SyncPDOffline$}}} \\
\cline{7-15}
& & & & & 
& \textsf{|$\textsf{Cyc}$|}
& \textsf{\textsf{A. P.}}
& \textsf{\textsf{C. P.}}
& \textbf{Dlk} 
& {\textbf{Time}} 
&{\textbf{Dlk}} 
& {\textbf{Time}}
&{\textbf{Dlk}} 
&{\textbf{Time}} \\
\hline
Deadlock & 39 & 3 & 4 & 3 & 8 & 1 & 1 & 1 & 1 & 0.02 & 0 & 0.09 & 0 & 0.16\\
NotADeadlock & 60 & 3 & 4 & 5 & 16 & 1 & 1 & 1 & 0 & 0.02 & 0 & 0.09 & 0 & 0.16\\
Picklock & 66 & 3 & 6 & 6 & 20 & 2 & 2 & 2 & 1 & 0.02 & 1 & 0.10 & 1 & 0.18\\
Bensalem & 68 & 4 & 5 & 5 & 22 & 2 & 2 & 2 & 1 & 0.02 & 1 & 0.12 & 1 & 0.16\\
Transfer & 72 & 3 & 11 & 4 & 12 & 1 & 1 & 1 & 1 & 0.02 & 0 & 0.09 & 0 & 0.15\\
Test-Dimmunix & 73 & 3 & 9 & 7 & 26 & 2 & 2 & 2 & 2 & 0.02 & 2 & 0.10 & 2 & 0.17\\
StringBuffer & 74 & 3 & 14 & 4 & 16 & 1 & 3 & 6 & 2 & 0.02 & 2 & 0.12 & 2 & 0.19\\
Test-Calfuzzer & 168 & 5 & 16 & 6 & 48 & 2 & 1 & 1 & 1 & 0.02 & 1 & 0.12 & 1 & 0.17\\
DiningPhil & 277 & 6 & 21 & 6 & 100 & 1 & 1 & 3K & 1 & 1.60 & 0 & 0.09 & 1 & 0.17\\
HashTable & 318 & 3 & 5 & 3 & 174 & 1 & 2 & 43 & 2 & 0.19 & 2 & 0.12 & 2 & 0.19\\
Account & 706 & 6 & 47 & 7 & 134 & 3 & 1 & 12 & 0 & 0.19 & 0 & 0.09 & 0 & 0.18\\
Log4j2 & 1K & 4 & 334 & 11 & 43 & 1 & 1 & 1 & 1 & 0.65 & 1 & 0.11 & 1 & 0.20\\
Dbcp1 & 2K & 3 & 768 & 5 & 56 & 2 & 2 & 3 & - & F & 2 & 0.11 & 2 & 0.19\\
Dbcp2 & 2K & 3 & 592 & 10 & 76 & 1 & 2 & 4 & - & F & 0 & 0.10 & 0 & 0.18\\
Derby2 & 3K & 3 & 1K & 4 & 16 & 1 & 1 & 1 & 1 & 0.23 & 1 & 0.10 & 1 & 0.17\\
RayTracer & 31K & 5 & 5K & 15 & 976 & 0 & 0 & 0 & - & F & 0 & 0.15 & 0 & 0.19\\
jigsaw & 143K & 21 & 8K & 2K & 67K & 172 & 12 & 70 & - & F & 2 & 0.36 & 1 & 1.55\\
elevator & 246K & 5 & 727 & 52 & 48K & 0 & 0 & 0 & 0 & 1.65 & 0 & 0.33 & 0 & 0.27\\
hedc & 410K & 7 & 109K & 8 & 32 & 0 & 0 & 0 & 0 & 2.09 & 0 & 0.50 & 0 & 0.24\\
JDBCMySQL-1 & 442K & 3 & 73K & 11 & 13K & 2 & 4 & 6 & 2 & 28.45 & 2 & 0.24 & 2 & 0.48\\
JDBCMySQL-2 & 442K & 3 & 73K & 11 & 13K & 4 & 4 & 9 & 1 & 3.37 & 1 & 0.22 & 1 & 0.33\\
JDBCMySQL-3 & 443K & 3 & 73K & 13 & 13K & 5 & 8 & 16 & 1 & 31.23 & 1 & 0.25 & 1 & 0.45\\
JDBCMySQL-4 & 443K & 3 & 73K & 14 & 13K & 5 & 10 & 18 & 2 & 5.51 & 2 & 0.28 & 2 & 0.49\\
cache4j & 775K & 2 & 46K & 20 & 35K & 0 & 0 & 0 & 0 & 5.86 & 0 & 0.46 & 0 & 0.39\\
ArrayList & 3M & 801 & 121K & 802 & 176K & 9 & 3 & 672 & 3 & 8.7K & 3 & 21.98 & 3 & 1.68\\
IdentityHashMap & 3M & 801 & 496K & 802 & 162K & 1 & 3 & 4 & 1 & 443.93 & 1 & 8.51 & 1 & 1.45\\
Stack & 3M & 801 & 118K & 2K & 405K & 9 & 3 & 481 & 1 & T.O & 3 & 25.34 & 3 & 2.94\\
Sor & 3M & 301 & 2K & 3 & 719K & 0 & 0 & 0 & 0 & 15.89 & 0 & 44.12 & 0 & 0.61\\
LinkedList & 3M & 801 & 290K & 802 & 176K & 9 & 3 & 10K & 3 & 4.7K & 3 & 48.02 & 3 & 2.06\\
HashMap & 3M & 801 & 555K & 802 & 169K & 1 & 3 & 10K & 3 & 4.4K & 2 & 504.36 & 2 & 1.65\\
WeakHashMap & 3M & 801 & 540K & 802 & 169K & 1 & 3 & 10K & - & T.O & 2 & 499.68 & 2 & 1.70\\
Swing & 4M & 8 & 31K & 739 & 2M & 0 & 0 & 0 & - & F & 0 & 0.72 & 0 & 0.88\\
Vector & 4M & 3 & 15 & 4 & 800K & 1 & 1 & 1B & - & T.O & 1 & 1.52 & 1 & 1.90\\
LinkedHashMap & 4M & 801 & 617K & 802 & 169K & 1 & 3 & 10K & 2 & 40.74 & 2 & 492.87 & 2 & 1.69\\
montecarlo & 8M & 3 & 850K & 3 & 26 & 0 & 0 & 0 & 0 & 2.6K & 0 & 1.81 & 0 & 0.79\\
TreeMap & 9M & 801 & 493K & 802 & 169K & 1 & 3 & 10K & 2 & 105.45 & 2 & 480.11 & 2 & 1.92\\
hsqldb & 20M & 46 & 945K & 403 & 419K & 0 & 0 & 0 & - & F & - & - & 0 & 2.38\\
sunflow & 21M & 16 & 2M & 12 & 1K & 0 & 0 & 0 & - & F & 0 & 8.35 & 0 & 1.62\\
jspider & 22M & 11 & 5M & 15 & 10K & 0 & 0 & 0 & - & F & 0 & 8.49 & 0 & 1.95\\
tradesoap & 42M & 236 & 3M & 6K & 245K & 2 & 1 & 4 & - & F & 0 & 108.16 & 0 & 7.06\\
tradebeans & 42M & 236 & 3M & 6K & 245K & 2 & 1 & 4 & - & F & 0 & 116.23 & 0 & 7.26\\
eclipse & 64M & 15 & 10M & 5K & 377K & 9 & 5 & 280 & - & F & 0 & 26.67 & 0 & 9.90\\
TestPerf & 80M & 50 & 599 & 9 & 197K & 0 & 0 & 0 & 0 & 795.04 & 0 & 47.56 & 0 & 4.30\\
Groovy2 & 120M & 13 & 13M & 10K & 69K & 0 & 0 & 0 & 0 & 1.7K & 0 & 38.06 & 0 & 8.92\\
Tsp & 200M & 6 & 24K & 3 & 882 & 0 & 0 & 0 & 0 & 7.6K & 0 & 72.62 & 0 & 12.70\\
lusearch & 203M & 7 & 3M & 98 & 273K & 0 & 0 & 0 & 0 & 1.3K & 0 & 75.88 & 0 & 14.44\\
biojava & 221M & 6 & 121K & 79 & 16K & 0 & 0 & 0 & - & F & 0 & 63.79 & 0 & 12.65\\
graphchi & 241M & 20 & 25M & 61 & 1K & 0 & 0 & 0 & - & F & 0 & 102.05 & 0 & 25.25\\
\hline\hline\textbf{Totals} & \textbf{1B} & \textbf{7K} & \textbf{70M} & \textbf{37K} & \textbf{8M} & \textbf{256} & \textbf{93} & \textbf{1B} & \textbf{35} & \textbf{>18h} & \textbf{40} & \textbf{2801} & \textbf{40} & \textbf{135}\\\hline
\end{tabular}
}
\end{table}

\Paragraph{Evaluation}
\cref{tab:expr-results} presents our results.
A bug identifies a unique tuple of source
code locations corresponding to events participating in the deadlock.
%Columns 2-6 present the characteristics of our benchmark traces.
Trace lengths vary vastly from $39$ to about $241$M, while the number of threads ranges from $3$ to about $800$,
which are representative features of real-world settings.
\texttt{Hsqldb} contains critical sections that are not well nested, 
and \seqc was not able to handle this benchmark;
our algorithm does not have such a restriction.
%We were unable to run \dirk on certain benchmarks due to \dirk's technical issues, which are marked with F.


\vspace{-0.1cm}
\SubParagraph{\underline{Abstract vs Concrete Patterns}}
Columns 7-9 present statistics on the 
abstract lock graph $\lkevgraph{\tr}$ of each trace $\tr$.
Many traces have a large number of concrete deadlock patterns 
but much fewer abstract deadlock patterns;
a single abstract deadlock pattern can 
comprise up to an order of $10^4$ more concrete patterns (Column $8$ v/s Column $9$).
Unlike all prior sound techniques, 
our algorithms analyze 
abstract deadlock patterns, instead of concrete ones. 
We thus expect our algorithms to be much more scalable in practice.


\SubParagraph{\underline{Deadlock-detection capability}}
In total, both \seqc and $\SyncPDOffline$ reported 40 deadlocks.
\seqc misses a deadlock of size $5$ in \texttt{DiningPhil},
which is detected by $\SyncPDOffline$,
and $\SyncPDOffline$ misses a deadlock in \texttt{jigsaw} which is detected by \seqc.
As $\SyncPDOffline$ is complete for sync-preserving deadlocks, we conclude that there are no more such deadlocks in our dataset.
Overall, $\SyncPDOffline$ and \seqc miss only three deadlocks reported by \dirk. 
On closer inspection, we found that these deadlocks are not witnessed by correct reorderings, and require reasoning about event values.
On the other hand, \dirk struggles to analyze even moderately-sized benchmarks and times out in $3$ of them. %(timeout is 3h).
This results in \dirk failing to report 5 deadlocks after $9$ hours, all of which are reported by $\SyncPDOffline$ in under a minute.
Similar conclusions were recently made in~\cite{Cai2021}.  
Overall, our results strongly indicate that the notion of sync-preservation characterizes most deadlocks that other tools are able to predict.


\SubParagraph{\underline{Unsoundness of \dirk}}
In our evaluation, we discovered that the soundness guarantee 
underlying \dirk~\cite{Kalhauge2018} is broken, resulting in it reporting false positives.
% Although \dirk is portrayed as sound, we have found two independent sources of unsoundness 
% resulting in it reporting false positives.
First, its constraint formulation~\cite{Kalhauge2018} 
does not rule out deadlock patterns when acquire events in the pattern hold common locks, 
in which case mutual exclusion disallows such a pattern to be a real predictable deadlock.
Second, \dirk also models conditional statements, allowing it to reason about witnesses beyond correct reorderings.
While this relaxation allows \dirk to predict additional deadlocks in \texttt{Transfer}, \texttt{Deadlock} and \texttt{HashMap}, 
its formalization is not precise and its implementation is erroneous.
We elaborate these aspects further in\begin{pldi}~\cite{arxiv}. \end{pldi}\begin{arxiv}~\appref{unsound-dirk}.\end{arxiv}


%We noticed that \dirk reports false positives in certain cases, which contradicts its theoretical soundness guarantee.
%We provide two such cases in~\appref{unsound-dirk}.
%The first case is a modified version of \texttt{Transfer}.
%Here, \dirk falsely reports a deadlock because it inadequately models conditional statements, which are used to allow more trace reorderings that can expose a deadlock.
%This relaxation enables \dirk to predict deadlocks in the benchmarks \texttt{Transfer} and \texttt{Deadlock}, which are missed by $\SyncPDOffline$ and \seqc. 
%However, this relaxation is not formalized and its implementation is erroneous.
%In the second case, \dirk falsely reports a deadlock due to missing that a common lock protects an otherwise deadlock pattern.

\SubParagraph{\underline{Running time}}
Our experimental results indicate that \dirk, backed by SMT solving, 
is the least efficient technique in terms of running time ---
it takes considerably longer or times out on large benchmark instances.
$\SyncPDOffline$ analyzed the entire set of traces $\sim\!\!\!21\times$ faster than \seqc.
On the most demanding benchmarks, such as 
HashMap and TreeMap, $\SyncPDOffline$ is more than $200\times$ faster than \seqc.
Although \seqc employs a polynomial-time algorithm for deadlock prediction, 
and thus significantly faster than the SMT-based \dirk,
the large polynomial complexity in its running time hinders scalability on 
execution traces coming from benchmarks that are more representative of realistic workloads.
In contrast, the linear time guarantees of $\SyncPDOffline$ are realized in practice, 
allowing it to scale on even the most challenging inputs.
More importantly, the improved performance comes while preserving essentially the same precision.


%\begin{figure}[!h]
\def\scatterscale{0.192}
\def\scatterwidth{0.23\textwidth}

\centering
\begin{subfigure}[b]{\scatterwidth}
\includegraphics[scale=\scatterscale]{figures/LinkedListDeadlockTest.pdf}
\label{subfig:sca-one-lock-clique}
\caption{LinkedList}
\end{subfigure}
\begin{subfigure}[b]{\scatterwidth}
\includegraphics[scale=\scatterscale]{figures/StackDeadlockTest.pdf}
\caption{Stack}
\label{subfig:sca-skewed}
\end{subfigure}


\caption{
Scalability experiments.
}
\label{fig:scalability}
\end{figure}


%\Andreas{We might revisit/remove the following}

%\textcolor{red}{check the numbers again.}
\SubParagraph{\underline{False negatives}}
Our benchmark set contains $93$ abstract deadlock patterns, $40$ of which are confirmed sync-preserving deadlocks.
We inspected the remaining $53$ abstract patterns to see if any of them are predictable deadlocks
missed by our sync-preserving criterion, independently of the compared tools.
$48$ of these $53$ patterns are in fact not predictable deadlocks ---
for every such pattern $D$, 
the set $S_D$ of events in the downward-closure of $\prev{}(D)$ with respect to $\tho{}$ and $\rf{}$,
already contains an event from $D$, disallowing any correct reordering
(sync-preserving or not) in which $D$ can be enabled.
Of the remaining, $4$ deadlock patterns obey the following scheme:
there are two acquire events $\acq_1, \acq_2$ participating in the deadlock pattern, 
each $\acq_i$ is preceded by a critical section on a lock that appears in 
$\lheld{}(\acq_{3-i})$, again disallowing a correct reordering that witnesses the pattern.
Thus, \emph{only one} predictable deadlock is not sync-preserving in our whole dataset.
This analysis supports that the notion of sync-preservation is not overly conservative in practice.
%\hunkar{I think here we should clarify that this analysis is not overly conservative as far as predictable deadlocks go.}

%\hunkar{
The above analysis concerns false negatives wrt. predictable deadlocks.
Some deadlocks are beyond the common notion of predictability we have adopted here, as they can only be exposed by reasoning about event values and control-flow dependencies, a problem that is $\NP$-hard even for 3 threads~\cite{Gibbons1997}.
We noticed $3$ such deadlocks in our dataset, found by \dirk,
though, as mentioned above, \dirk's reasoning for capturing such deadlocks is unsound in practice.
%}

%\hunkar{Maybe stress again that this reasoning about event values is non-trivial as it relies on heavyweight SMT solving. Also, maybe have the section "Unsoundness of Dirk" after this one and say that next we show that it is also tricky to implement in practice.}
%sync-preserving deadlocks are likely to be permissive and not lead to a high false negative rate.
%We conduct an analyses based on our benchmark set and characterize potential false negatives of our technique.
%Recall the conditions imposed on correct reorderings (see \secref{prelim}).
%One of the main conditions imposed by the definition of a correct reordering is that
%every read event reads from the same write as in the original trace. 
%We investigated the effect of this condition on the potential false negatives.
%Our benchmark set contains $93$ abstract deadlock patterns and our technique is able to find $42$ deadlocks.
%The remaining $51$ deadlock patterns constitute potential false negatives.
%If we only impose the above condition, then $46$ of the deadlock patterns cannot be realized to a real deadlock.
%The additional conditions that are specific to sync-preserving deadlocks (e.g., the order of critical sections on the same lock cannot be reversed) prevents the remaining $5$ deadlock patterns from being realizable.
%We remark that the definition of correct reorderings adopted in this work originate from the standard model that is widely used in this domain~\cite{Smaragdakis12,serbanuta2013,Koushik05}.
%Hence, this analyses further supports that given this standard program model, the additional restrictions introduced with the
%sync-preserving deadlocks are likely to be permissive and not lead to a high false negative rate.


\subsection{Online Experiments}
\seclabel{online-expr}

\Paragraph{Experimental setup}
%Dynamic analyses can incur high runtime overheads.
The objective of our second set of experiments is to evaluate 
the performance 
of our proposed algorithms in an \emph{online} setting.
For this, we implemented our $\SyncPDOnline$ algorithm inside the
framework of \dlfuzzer~\cite{Joshi2009} following closely the pseudocode in \algoref{online}. 
This framework instruments a concurrent program so that it can
perform analysis on-the-fly while executing it.
If a deadlock occurs during execution, it is reported and the execution halts.
However, if a deadlock is predicted in an alternate interleaving, 
then this deadlock is reported and the execution continues to search further deadlocks.
We used the same dataset as in \secref{offline-expr}, 
after discarding some benchmarks that could not be instrumented by \dlfuzzer.

To the best of our knowledge, all prior deadlock prediction techniques work offline.
For this reason, we only compared our online tool with the randomized 
scheduling technique of~\cite{Joshi2009} already
implemented inside the same \dlfuzzer framework.	
At a high level, this random scheduling technique works as follows.
Initially, it
(i)~executes the input program with a random scheduler, 
(ii)~constructs a \emph{lock dependency relation}, and 
(iii)~runs a cycle detection algorithm to discover deadlock patterns. 
For each deadlock pattern thus found, it spawns new executions that attempt 
to realize it as an actual deadlock.
To increase the likelihood of hitting the deadlock,
\dlfuzzer biases the random scheduler by pausing threads at specific locations.
%(e.g., before acquiring a certain lock).

The second, confirmation phase of~\cite{Joshi2009}
acts as a best-effort proxy for sound deadlock prediction.
On the other hand, $\SyncPDOnline$ is already sound and predictive, and thus does not require
additional confirmation runs, making it more efficient.
Towards effective prediction, we also implemented a simple bias to the scheduler.
If a thread $t$ attempts to write on a shared variable $x$ while holding a lock, then 
our procedure randomly decides to pause this operation for a short duration.
This effectively explores race conditions in different orders.
Overall, implementing $\SyncPDOnline$ inside \dlfuzzer provided the added advantage of supplementing a powerful prediction technique with a biased randomized scheduler.
%As an added advantage of implementing our algorithms in the
% framework, we are able to achieve
%the complementary objective of evaluating how well prediction supplements
%the effectiveness of an otherwise naive
%controlled concurrency testing
%technique like randomized scheduling.
To our knowledge, our work is the first to effectively 
combine these two orthogonal techniques.
We also remark that such a bias is of no benefit to \dlfuzzer itself
since it does not employ any predictive reasoning.

% a randomized testing procedure, even though the latter is rather simple.

For this experiment, we run \dlfuzzer on each benchmark, and for each deadlock pattern found in the initial execution, 
we let it spawn $3$ new executions trying to realize the deadlock, 
as per standard (\href{https://github.com/ksen007/calfuzzer}{https://github.com/ksen007/calfuzzer}).
We repeated this process $50$ times and recorded the total time taken.
Then, we allocated the same time for $\SyncPDOnline$ to repeatedly execute the same program and perform deadlock prediction.
We measured all deadlocks found by each technique.
%We also noticed that \dlfuzzer fails to work properly if the given input program deadlocks in the initial run.
%This results in \dlfuzzer itself to go into deadlock without producing any deadlock reports.  
%Hence, we made a modest modification on \dlfuzzer allowing it to work properly and report deadlocks in such cases.
%We used this modified version of \dlfuzzer in our evaluation.





%!TEX root = ../main.tex

\begin{table*}
\caption{
Performance comparison of $\SyncPDOnline$ ($\syncpdshort$) and $\dlfuzzer$ ($\dlfshort$).
%%Column 1 denotes the name of the benchmark.
Columns 2-3 show the total number of bug reports. 
Columns 4-6 show the total number of unique bugs found by each tool, and their union. 
Columns 7-12 show the hit rate on each bug.
Columns 13-16 show the runtime overhead of the tools.
%We refer to $\dlfuzzer$ as $\dlfshort$, to $\SyncPDOnline$ as $\syncpdshort$, and the instrumentation phases using the suffix $\mathsf{\tt -I}$.
%Performance comparison of $\SyncPDOnline$ ($\syncpdshort$) and $\dlfuzzer$ ($\dlfshort$).
%Column 1 denotes the name of the benchmark.
%Columns 2-3 show the total number of bug reports. 
%Columns 4-6 show the total number of unique bugs found by each tool, and their union. 
%Columns 7-12 show the hit rate on each bug.
%Columns 13-17 show the runtime overhead of the tools.
%We use the suffix $\mathsf{\tt -I}$ for the instrumentation phases.
\label{tab:expr-dlf-results}
}
\vspace{-0.15cm}
\setlength\tabcolsep{3pt}
\renewcommand{\arraystretch}{0.9}
\small
\centering
\scalebox{1}{
\begin{tabular}{|r|c|c|c|c|c|c|c|c|c|c|c|c|c|c|c|c|c|c|c|c|c|c|c|c|c|c|c|}
\hline
1 & 2 & 3 & 4 & 5 & 6 & 7 & 8 & 9 & 10 & 11 & 12 & 13 & 14 & 15 & 16 \\
\hline
\multirow{2}{*}{\textbf{Benchmark}}& 
\multicolumn{2}{c|}{ {\textsf{Bug Hits}}} & 
\multicolumn{3}{c|}{ {\textsf{Unique Bugs}}} 
& \multicolumn{2}{c|}{ \textsf{Bug 1}} &
\multicolumn{2}{c|}{ \textsf{Bug 2}} &
\multicolumn{2}{c|}{ \textsf{Bug 3}} & \multicolumn{4}{c|}{ {\textsf{Runtime Overhead}}}  \\
\cline{7-16}
\cline{2-6}
& \textsf{$\syncpdshort$} & \textsf{$\dlfshort$} & \textsf{$\syncpdshort$} & \textsf{$\dlfshort$}  & All 
& \textsf{\textsf{$\syncpdshort$}} 
& \textbf{$\dlfshort$} 
& {\textbf{$\syncpdshort$}} 
&{\textbf{$\dlfshort$}} 
& {\textbf{$\syncpdshort$}}
&{\textbf{$\dlfshort$}} & {\textbf{$\syncpdshortinstr$}}  & {\textbf{$\syncpdshort$}}  & 
{\textbf{$\dlfshortinstr$}}  &
 {\textbf{$\dlfshort$}} \\
\hline
%\multirow{2}{*}{\textbf{Benchmark}}& \multirow{2}{*}{$\mathcal{N}$} & \multirow{2}{*}{$\mathcal{T}$} & \multirow{2}{*}{$\mathcal{M}$}
%& \multirow{2}{*}{$\mathcal{L}$} &
%\multirow{2}{*}{$\NumAcquires+\mathcal{R}$} &\multicolumn{2}{c||}{ {\seqc}} & \multicolumn{2}{c||}{ {\textsf{$\SyncPDOffline$}}} & \multicolumn{2}{c|}{ {\textsf{$\SyncPDOnline$}}} \\
%\cline{7-12}
%& & & & & &  \textbf{$\#$Deadlocks} & { \textbf{Time (s)}} &{ \textbf{$\#$Deadlocks}} & { \textbf{Time (s)}} &{ \textbf{$\#$Deadlocks}} &{ \textbf{Time (s)}} \\
\hline
Deadlock & 50 & 50 & 1 & 1 & 1 & 50 & 50 & - & - & - & - & 2$\times$ & 3$\times$ & 2$\times$ & 4$\times$\\
Picklock & 227 & 97 & 2 & 1 & 2 & 226 & 97 & 1 & 0 & - & - & 2$\times$ & 2$\times$ & 2$\times$ & 5$\times$\\
Bensalem & 355 & 32 & 2 & 1 & 2 & 8 & 0 & 347 & 32 & - & - & 2$\times$ & 2$\times$ & 2$\times$ & 6$\times$\\
Transfer & 54 & 50 & 1 & 1 & 1 & 54 & 50 & - & - & - & - & 2$\times$ & 2$\times$ & 1$\times$ & 4$\times$\\
Test-Dimmunix & 702 & 0 & 2 & 0 & 2 & 351 & 0 & 351 & 0 & - & - & 2$\times$ & 2$\times$ & 2$\times$ & 4$\times$\\
StringBuffer & 153 & 131 & 2 & 2 & 2 & 128 & 118 & 25 & 13 & - & - & - & - & - & -\\
Test-Calfuzzer & 177 & 44 & 1 & 1 & 1 & 177 & 44 & - & - & - & - & 2$\times$ & 2$\times$ & 2$\times$ & 4$\times$\\
DiningPhil & 162 & 100 & 1 & 1 & 1 & 162 & 100 & - & - & - & - & - & - & - & -\\
HashTable & 169 & 120 & 2 & 2 & 2 & 82 & 21 & 87 & 99 & - & - & - & - & - & -\\
Account & 19 & 188 & 1 & 1 & 1 & 19 & 188 & - & - & - & - & 2$\times$ & 8$\times$ & 2$\times$ & 16$\times$\\
Log4j2 & 290 & 100 & 2 & 1 & 2 & 145 & 100 & 145 & 0 & - & - & - & - & - & -\\
Dbcp1 & 265 & 138 & 2 & 2 & 2 & 264 & 61 & 1 & 77 & - & - & - & - & - & -\\
Dbcp2 & 129 & 126 & 2 & 2 & 2 & 86 & 99 & 43 & 27 & - & - & - & - & - & -\\
RayTracer & 0 & 0 & 0 & 0 & 0 & - & - & - & - & - & - & 122$\times$ & 124$\times$ & 109$\times$ & 111$\times$\\
Tsp & 0 & 0 & 0 & 0 & 0 & - & - & - & - & - & - & 47$\times$ & 60$\times$ & 37$\times$ & 40$\times$\\
jigsaw & 1189 & 1 & 1 & 1 & 2 & 1189 & 0 & 0 & 1 & - & - & - & - & - & -\\
elevator & 0 & 0 & 0 & 0 & 0 & - & - & - & - & - & - & 2$\times$ & 2$\times$ & 2$\times$ & 2$\times$\\
JDBCMySQL-1 & 349 & 117 & 2 & 3 & 3 & 1 & 21 & 0 & 4 & 348 & 92 & 3$\times$ & 4$\times$ & 2$\times$ & 13$\times$\\
JDBCMySQL-2 & 559 & 73 & 1 & 1 & 1 & 559 & 73 & - & - & - & - & 2$\times$ & 4$\times$ & 2$\times$ & 18$\times$\\
JDBCMySQL-3 & 560 & 224 & 1 & 1 & 1 & 560 & 224 & - & - & - & - & 2$\times$ & 5$\times$ & 2$\times$ & 24$\times$\\
JDBCMySQL-4 & 1717 & 101 & 3 & 1 & 3 & 95 & 0 & 834 & 0 & 788 & 101 & 3$\times$ & 5$\times$ & 2$\times$ & 31$\times$\\
hedc & 0 & 0 & 0 & 0 & 0 & - & - & - & - & - & - & 2$\times$ & 2$\times$ & 1$\times$ & 2$\times$\\
cache4j & 0 & 0 & 0 & 0 & 0 & - & - & - & - & - & - & 2$\times$ & 2$\times$ & 2$\times$ & 2$\times$\\
lusearch & 0 & 0 & 0 & 0 & 0 & - & - & - & - & - & - & 16$\times$ & 17$\times$ & 13$\times$ & 16$\times$\\
ArrayList & 47 & 45 & 3 & 3 & 3 & 20 & 22 & 3 & 5 & 24 & 18 & 50$\times$ & 69$\times$ & 18$\times$ & 79$\times$\\
Stack & 44 & 27 & 3 & 3 & 3 & 18 & 13 & 8 & 4 & 18 & 10 & 69$\times$ & 91$\times$ & 64$\times$ & 86$\times$\\
IdentityHashMap & 68 & 62 & 2 & 2 & 2 & 13 & 47 & 55 & 15 & - & - & 4$\times$ & 8$\times$ & 3$\times$ & 10$\times$\\
LinkedList & 48 & 26 & 3 & 2 & 3 & 21 & 17 & 7 & 0 & 20 & 9 & 16$\times$ & 28$\times$ & 14$\times$ & 32$\times$\\
Swing & 0 & 0 & 0 & 0 & 0 & - & - & - & - & - & - & 5$\times$ & 6$\times$ & 4$\times$ & 6$\times$\\
Sor & 0 & 0 & 0 & 0 & 0 & - & - & - & - & - & - & 2$\times$ & 7$\times$ & 2$\times$ & 2$\times$\\
HashMap & 46 & 44 & 2 & 2 & 2 & 18 & 11 & 28 & 33 & - & - & 7$\times$ & 11$\times$ & 4$\times$ & 8$\times$\\
Vector & 126 & 50 & 1 & 1 & 1 & 126 & 50 & - & - & - & - & 2$\times$ & 2$\times$ & 2$\times$ & 3$\times$\\
LinkedHashMap & 57 & 43 & 2 & 2 & 2 & 22 & 10 & 35 & 33 & - & - & 10$\times$ & 10$\times$ & 4$\times$ & 8$\times$\\
WeakHashMap & 29 & 40 & 2 & 2 & 2 & 6 & 11 & 23 & 29 & - & - & 7$\times$ & 12$\times$ & 4$\times$ & 8$\times$\\
montecarlo & 0 & 0 & 0 & 0 & 0 & - & - & - & - & - & - & 16$\times$ & 100$\times$ & 13$\times$ & 126$\times$\\
TreeMap & 42 & 47 & 2 & 2 & 2 & 16 & 15 & 26 & 32 & - & - & 9$\times$ & 12$\times$ & 5$\times$ & 9$\times$\\
eclipse & 0 & 0 & 0 & 0 & 0 & - & - & - & - & - & - & 2$\times$ & 2$\times$ & 2$\times$ & 2$\times$\\
TestPerf & 0 & 0 & 0 & 0 & 0 & - & - & - & - & - & - & 2$\times$ & 2$\times$ & 2$\times$ & 2$\times$\\
\hline\hline\textbf{Total} & \textbf{7633} & \textbf{2076} & \textbf{49} & \textbf{42} & \textbf{51} & - & - & - & - & - & - & - & - & - & - \\ 
\hline
\end{tabular}
}
\end{table*}


\Paragraph{Evaluation}
\cref{tab:expr-dlf-results} presents our experimental results.
%A bug identifies a unique tuple of source code locations corresponding to events
%participating in the deadlock.
Columns $2$-$3$ of the table display the total number of bug hits,
which is the total number of times a bug was predicted by $\SyncPDOnline$ in the entire duration,
or was confirmed in any trial of \dlfuzzer.
Columns $4$-$6$ display the unique bugs (i.e., unique tuples of source code locations leading to a deadlock) 
found by the techniques.
The employed techniques are able to find a maximum of $3$ unique bugs for each benchmark
in our benchmark set. 
Respectively, columns $7$-$12$ display the detailed information on the number 
of times a particular bug was found by each technique.
Runtime overheads are displayed in the columns $13$-$16$, with $\mathsf{\tt -I}$ denoting the instrumentation phase only.


\SubParagraph{\underline{Deadlock-detection capability}}
\dlfuzzer had $2076$ bug reports in total, and it found $42$ unique bugs.
In contrast, $\SyncPDOnline$ flagged $7633$ bug reports, corresponding to $49$ unique bugs.
In more detail, \dlfuzzer missed $9$ bugs reported by \SyncPDOnline whereas 
$\SyncPDOnline$ missed $2$ bugs reported by \dlfuzzer.
Also, \SyncPDOnline significantly outperformed \dlfuzzer in total number of bugs hits.
Our experiments again support that the notion of sync-preservation  captures most deadlocks that occur in practice, to the extent that other state-of-the-art techniques can capture.
%\hunkar{
A further observation is that in the offline experiments, \SyncPDOffline  was not able to find deadlocks in \texttt{Transfer} and \texttt{Deadlock}. 
However, the random scheduling procedure allowed \SyncPDOnline 
to navigate to executions from which deadlocks can be predicted.
This demonstrates the potential of combining predictive dynamic
techniques with controlled concurrency testing.
%; a direction we find promising to be pursued further by the community.


\SubParagraph{\underline{Runtime overhead}}
We have also measured the runtime overhead of both \SyncPDOnline and \dlfuzzer,
both as incurred by instrumentation, as well as by the deadlock analysis.
The latter is the time taken by \algoref{online} for the case of $\SyncPDOnline$,
and the overhead introduced due to the new executions in the second confirmation phase for the case of \dlfuzzer.
Our results show that the instrumentation overhead of \SyncPDOnline is, in fact, comparable to that of \dlfuzzer, though somewhat larger. 
This is expected, as \SyncPDOnline needs to also instrument memory access events, while \dlfuzzer only instruments lock events, but at the same time surprising because the number of memory access events
is typically much larger than the number of lock events.
On the other hand, the analysis overhead is often larger for \dlfuzzer, 
even though it reports fewer bugs.
It was not possible to measure the runtime overhead in certain benchmarks as 
either they were always deadlocking or the computation was running indefinitely.




\section{Conclusions and Future Work}
In this paper we introduced an extension to allow MoveIt2 to support multi-arm / multi-robot asynchronous trajectory execution.
Through a simulated pick and place scenario of two 7 DoF Panda arms in a shared workspace (see Figure \ref{fig:platforms}), we are currently running experiments that are 1) demonstrating the correct functioning of the implementation and collision avoidance logic; 2) proving the advantage of this approach with respect to synchronous execution for specific workloads; and 3) measuring the overhead induced by the backlog and re-planning logic.

% Ongoing work is collecting experimental results to: 1) validate the implementation and collision avoidance logic, 2) demonstrate advantages with respect to multi-arm synchronous execution for specific workloads; and 3) measure the overhead induced by the backlog and re-planning logic.
% Preliminary results show that collision avoidance works as expected in a simulated scenario with two 7 DoF Panda arms sharing the entire workspace, incurred overhead is limited and not significant when compared to synchronous arm execution delay for longer combinations of robotic tasks.

\begin{figure}[!htb]\vspace*{-0.3cm}
\centerline{
\includegraphics[scale=0.063]{Figures/baseline.png}
\includegraphics[scale=0.063]{Figures/synchronous.png}
\includegraphics[scale=0.063]{Figures/asynchronous.png}
}
\caption{baseline with only one arm \textit{[left]}, two arms synchronous movements \textit{[middle]} and two arms asynchronous movements \textit{[right]} }
\label{fig:platforms}
\end{figure}
\vspace{-1em}

%\section*{References}
\bibliographystyle{IEEEtran}
\bibliography{reference.bib}


\end{document}
