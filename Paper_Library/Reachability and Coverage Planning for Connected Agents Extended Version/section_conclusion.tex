% In this work, we extended the study of multi agent connected path planning 
% by considering the coverage problem. We showed that in the general case the complexity of the
% decision of the coverage matches the complexity of the reachability problem.
% Furthermore, this complexity still holds for the neighbor-communicable
% subclass.
% We identified an important subclass of topological graphs on which the complexity
% of the problem is as low as LOGSPACE.
% Unfortunately, the bounded
% versions of both problems stays NP-complete.
% A LOGSPACE algorithm can be obtained by ignoring communication constraints,
% that is, for the complete-communication subclass.
%However, an even more restrictive
%subclass admits a logarithmic-space algorithm for the bounded reachability
%problem.

%We can observe that the
Sight-moveable topological graphs we introduced in this work only constrain
the communication graph. One can be interested to constrain the movement graph
to a planar graph or a 2D grid given the common usage of grid modelling of the
environment. Given the intractability of MAPP on planar graphs \cite{Yu:2015}
and on general 2D grid graphs \cite{Banfi:2017}, it is likely that this
problem is intractable as well. Furthermore, in
\cite{dblp:conf/aaai/tateobrab18}, the decision is proved to stay
PSPACE-complete on planar graphs and grids as well. However, one can study this
problem on solid grid graphs, given that the Hamiltonian cycle is tractable on
such graphs \cite{Umans:1997}.

One can note that our NP lower bound reductions hold without the
anonymity of the agents. Indeed, the \pbBCoverage{} case is straightforward and for \pbBReach{} case, each agent can be associated to a clause or variable, so the reduction would still hold.
%Thus, the lower bound holds without anonymity.
%\todo{Ocan: Do we claim in the sense of conjecture or do we state it? Please avoid using claim in the latter sense}

We do not know  if \pbCoverage{} remains hard when the
$\rightarrow$-relations become symmetric, depicted in Figure~\ref{fig:results}
as a question mark. We think this open issue is important since symmetric
{$\rightarrow$-relations} (if UAVs can go from $v$ to $v'$, they can also come
back from $v'$ to $v$) are relevant for practical applications. We plan to
study the \emph{parametrized complexity} \cite{DBLP:series/mcs/DowneyF99} of our
problems - parameters could be the treewidth of the topological graph, the
number of UAVs.

% \todo{Can we remove following two paragraphs?}
% We attend to develop efficient parallel algorithms for the problems proven in
% LOGSPACE. Those algorithms can yield feasible solution in a short amount of time
% in order to be improved by heuristic-based algorithms. The sight-moveable
% topological graphs being a ``simpler'' class than the undirected ones, we can
% hope to obtain a polynomial-time approximation algorithm. Furthermore, this
% class seems to admit an intuitive resolution process which involve finding the
% relay positions and optimizing the coverage of multiple small areas.

% Interestingly, we plan to generalize to decentralized versions of our problems
% and to dynamic environments. Instead of generating sequences of actions, we will
% have to generate strategies as in ATL (alternating-time temporal logic)
% \cite{DBLP:journals/corr/abs-1006-1414}. As UAVs stay connected, we may suppose
% that when information is gained, it is common knowledge and that all actions,
% especially sensing actions, are public \cite{DBLP:conf/atal/BelardinelliLMR17}.
% We also aim at using a high-level dedicated formal logic to express objectives,
% such as the language proposed in \cite{DBLP:conf/atal/Rubin15} and
% \cite{DBLP:conf/atal/AminofMRZ16}.

%%% Local Variables:
%%% mode: latex
%%% TeX-master: "main"
%%% End:
