We start with the following upper bound which is obtained
by a straightforward guess and check algorithm:
\begin{proposition} \label{prop:dir:ub:bcover-breach}
	\pbBCoverageDir and \pbBReachDir are in NP.
\end{proposition}

In both cases, we can guess and check a path of bounded length in NP since the
input is encoded in unary.

We furthermore establish the following results:
%In this subsection, we show that \pbReachDir and \pbCoverageDir are
%PSPACE-complete and \pbBCoverageDir is in NP.

\begin{theorem}
  \pbCoverageDir and \pbReachDir are PSPACE-complete.
  \label{prop:dir:ub:cover-reach}
  \label{th:dir:compl:reach}
\end{theorem}

The upper bounds are obtained by a straightforward NPSPACE algorithm
that guesses an execution by keeping in memory the last configuration,
and, for \pbCoverageDir, the set of visited regions. We conclude with
Savitch's Theorem (NPSPACE=PSPACE)\cite{Savitch:70}.



% \begin{proposition} \label{prop:dir:ub:cover-reach}
% 	\pbCoverageDir and \pbReachDir are in PSPACE.
% \end{proposition}

% \begin{proof}
%   In both cases, the straightforward algorithm which guesses an execution runs in
%   non-deterministic polynomial space. For \pbCoverageDir, we only keep in memory
%   the last configuration and the set of already visited regions. For
%   \pbReachDir, we only keep in memory the last configuration. By Savitch's
%   theorem (NPSPACE = PSPACE) \cite{Savitch:70}, the proposition is proven. 
% \end{proof}

% \begin{proposition} \label{prop:dir:ub:bcover-breach}
% 	\pbBCoverageDir is in NP.
% \end{proposition}

% \begin{proof}
%   We define the same algorithms given in the Proof of Proposition
%   \ref{prop:dir:ub:cover-reach} except that we stop the execution when the
%   length is exceeded. Thus, the algorithms are non-deterministic and run in
%   polynomial time.
% \end{proof}

%%% Local Variables:
%%% mode: latex
%%% TeX-master: "main"
%%% End:
