%From the work of Tateo et al. \cite{dblp:conf/aaai/tateobrab18}, in Section 3.6,
%we get that \pbReach{} is PSPACE-complete on undirected graph even with
%anonymous agents.
% The PSPACE-hardness of \pbReach{} for undirected graphs stated in
% Theorem~\ref{th:dir:lb:reach} implies the PSPACE-hardness of this problem for
% directed graphs as well.
%Thus, we obtain a PSPACE-hardness result for \pbReachDir and,
%with
% By Proposition~\ref{prop:dir:ub:cover-reach} we get the PSPACE-completeness, as stated
% in the next theorem.

% \begin{theorem}
%     \label{th:dir:compl:reach}
%   \pbReachDir is PSPACE-complete.
% \end{theorem}

% We now establish that the computational complexity of the directed coverage problem is also PSPACE-complete.
% The proof of the following problem gives a reduction from the directed reachability problem.
The lower bound on \pbReachDir was proven in Theorem~\ref{th:dir:lb:reach}. We
now concentrate on \pbCoverageDir.

\begin{lemma}
  \label{lemma:dir:lb:cover}
  \pbCoverageDir is PSPACE-hard.
\end{lemma}
\begin{proof}
  The proof is by reduction from \pbReachDir in which the base node has a
  self-loop. As noted in the remark following Theorem~\ref{th:dir:lb:reach},
  this problem remains PSPACE-hard.
  We map an instance $(G, c)$ of \pbReachDir
  to the instance $G'$ of \pbCoverageDir where $G'$ is depicted in
  Fig.~\ref{figure:topologicgraphproblemcoverage}. Let~$k$ denote the number
  of agents in the instance~$(G,c)$. $G'$ contains $G$ as a subgraph, plus fresh
  nodes $v_1, \dots, v_k$ and $s_1, \dots, s_k$. An agent can move from any node
  of $G$ to $v_1$ and back.

    \begin{figure}[t]
        \begin{center}
            \newcommand{\ymax}{5.3}
            \newcommand{\ybase}{2.8}
            \scalebox{0.8}
            {
                \begin{tikzpicture}[scale=0.9]
                    \draw[copyofG] (-0.2,\ybase-0.3) rectangle (8.2,3.5);
                    \draw (1.5,\ybase) node (nb) {B};
                    \draw (1,4) node (s1) {$s_1$};
                    \draw (3,4) node (s2) {$s_2$};
                    \draw (7,4) node (sk) {$s_k$};
                    \draw (1,3.2) node (t1) {$c_1$};
                    \draw (3,3.2) node (t2) {$c_2$};
                    \draw (7,3.2) node (tk) {$c_k$};
                    \draw (1,4.8) node (v1) {$v_1$};
                    \draw (3,4.8) node (v2) {$v_2$};
                    \draw (7,4.8) node (vk) {$v_k$};
                    \draw (8,5.3) node (all) {all};
                    \draw[-stealth] (t1) -- (s1);
                    \draw[-stealth] (t2) -- (s2);
                    \draw[-stealth] (tk) -- (sk);
                    \draw[tobehere,communication] (s1) -- (s2);
                    \draw[tobehere,communication] (s2) -- (4.5,4);
                    \draw[dotted,color=blue!50!white] (4.5,4) -- (5.5,4);
                    \draw[tobehere,communication] (sk) -- (5.5,4);
                    \draw[tobehere,communication] (t1) -- (t2);
                    \draw[tobehere,communication] (t2) -- (4.5,3.2);
                    \draw[dotted,color=blue!50!white] (4.5,3.2) -- (5.5,3.2);
                    \draw[tobehere,communication] (tk) -- (5.5,3.2);
                    \path[tobehere,communication] (s1) edge[bend right=70] (nb);
                    %\addCommunicationB{s1}
                    \draw[-stealth] (s1) -- (v1);
                    \draw[-stealth] (s2) -- (v2);
                    \draw[-stealth] (sk) -- (vk);
                    \draw[stealth-stealth] (v1) to[bend right] (0.1,3.4);
                    \path (vk) edge [loop right] (vk);
                    \draw[-stealth] (nb) to [loop right] (nb);
                    \draw [-stealth, rounded corners = 8pt] (vk) -- (7,\ymax) -- (-1,\ymax) -- (-1, \ybase)  -- (nb);
                    \draw (v1) -- (1,\ymax);
                    \draw (v2) -- (3,\ymax);
                    \draw[tobehere,communication] (vk) -- (all);
                    \node at (4, 2.7) {copy of $G$};
                \end{tikzpicture}
            }
            \caption{Topological graph $G'$ constructed from the \pbReachDir-instance.}
            \label{figure:topologicgraphproblemcoverage}
        \end{center}
    \end{figure}
    Node $s_1$ can communicate with the base $B$, and node $v_k$ can communicate
    with all nodes of $G'$. Furthermore, we have the communication edges
    $(s_i,s_{i+1})$ and~$(v_i,v_{i+1})$ for all~$1\leq i\leq k-1$.
    Now we prove that the $k$ agents can progress to the
    configuration ($c_1,\dots,c_k$) in $G$  if and only if there exists a
    covering execution in~$G'$.

    ($\Rightarrow$) If the agents are in the configuration ($c_1,\dots,c_k$)
    then they can progress in one step to configuration ($s_1,\dots,s_k$). Then,
    they have no choice but progress to the configuration ($v_1,\dots,v_k$).
    Once in this configuration, the agent placed on the node $v_k$
    communicates with the base and with all other agents. %with any agent, placed on any node, and to the base $B$.
    This agent stays at $v_k$. Meanwhile the agent placed on the
    node $v_1$ will visit all unvisited nodes of $G$ and come back to $v_1$
    while keeping communication to the base through the agent placed on $v_k$.
    Meanwhile, agents placed on $v_2,\dots,v_{k-1}$ come back to $B$. Finally,
    when all the nodes have been visited, both agents on $v_1$ and $v_k$ come
    back to $B$.

    \newcommand{\timesk}{t_{s_k}}

    ($\Leftarrow$) If there exists a covering execution of the whole graph $G'$,
    it means all nodes have been visited. In particular, node $s_k$ has been
    visited and let us consider the first time $\timesk$ when $s_k$ is visited.
    Time $\timesk-1$ denotes the time just before $\timesk$.


    \begin{fact} \label{fact:nodesoutsideGprimeunvisited}
      At time $\timesk-1$, no node $v_i$ and no node~$s_i$ were visited.
    \end{fact}

    \begin{proof}
        Suppose by contradiction that a node $v_i$ was visited by some agent
        before $\timesk$, then the only possibility such an agent to communicate
        to the base is that there is also an agent at $v_k$ at time $\timesk$.
        But then, it means that $s_k$ was visited strictly before $\timesk$,
        leading to a contradiction. Thus, no node $v_i$ were visited at time
        $\timesk$ (thus at time $\timesk-1$).

        As no node $v_i$ are visited before $\timesk$, no node $s_i$ are visited
        before $\timesk-1$.
    \end{proof}


    \begin{fact} \label{fact:reachc1ck}
        At time $\timesk-1$, the configuration is $\langle c_1, \dots, c_k\rangle$.
    \end{fact}

    \begin{proof}
        At time $\timesk$, as the agent at $s_k$ needs to communicate with the base,
        the only
        possibility is that the configuration is $\langle s_1, \dots, s_k\rangle$. Thus, the
        only possibility is that configuration is $\langle c_1, \dots, c_k\rangle$.
    \end{proof}

    Facts~\ref{fact:nodesoutsideGprimeunvisited} implies that the prefix from
    time 0 to time $\timesk-1$ of the covering execution is an execution in $G$.
    Fact~\ref{fact:reachc1ck} implies that sub-execution reaches $\langle
    c_1,\dots,c_k\rangle$.
\end{proof}


% By Proposition~\ref{prop:dir:ub:cover-reach} and Theorem~\ref{th:dir:lb:cover}:

% \begin{theorem} \label{th:dir:compl:cover}
%     \pbCoverageDir is PSPACE-complete.
% \end{theorem}
% \begin{theorem}
%     \label{th:dir:compl:cover}
%   \pbCoverageDir is PSPACE-complete.
% \end{theorem}

%%% Local Variables:
%%% mode: latex
%%% TeX-master: "main"
%%% End:
