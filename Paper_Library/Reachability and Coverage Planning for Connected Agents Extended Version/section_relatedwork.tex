The coverage planning is an interesting approach to path planning. Indeed, a
covering plan can be used for fields such as floor cleaning, lawn mowing, etc. A
survey of this field appears in \cite{Choset:01}. This multi-agent extension has
the ability to reduce the length of the overall mission and also reach parts of
the area a single agent would not able to. This problem was studied in
\cite{Rekleitis:97} for two agents. As shown in the survey by Chen et al.
\cite{ChenSurvey}, many coverage problems have been addressed by using analytic
techniques. For instance, in \cite{DBLP:conf/icc/Yanmaz12} and
\cite{teacy2010maintaining}, they consider UAVs that should cover an area
while staying connected to the base, but only empirically study some path planning
algorithms without proving their algorithms formally.
%planning algorithms and their algorithms are not proven formally but only
%tested experimentally.

We advocate formal methods that give formal guarantees and have already been applied to
generate plans for robots and UAVs. Model checking has been
applied to robot planning (see \cite{DBLP:conf/iros/LacerdaPH14}) and to UAVs~\cite{webster2011formal}.
Humphrey \cite{Humphrey2013} shows how to use LTL (linear-temporal logic) model
checking for capturing response and fairness properties in cooperation (for
instance, if a task is requested then it is eventually performed). 
%Model
%checking has also been used to verify pre-programmed UAVs
%\cite{webster2011formal}.

% In \cite{DBLP:journals/corr/abs-1003-0381}, they discuss CTL model checking
% for checking properties. CTL is not suitable for our purpose because we need
% to express the existence of \emph{one} path along which UAVs stay connected
% and eventually have covered all the locations and have come back to the base
% location.

Bodin et al. \cite{IJCAI2018demodrones} treat a similar problem except that the
UAVs cover the graph without returning to the base. Without the return-to-the-base
constraint, we claim that all our hardness results still hold, except
for \pbBCoverageCC. They provide an implementation by describing the problem in
Planning Domain Description Language and then run the planner
Functional Strips \cite{DBLP:conf/ijcai/FrancesRLG17}.

% Both \pbReach{} and \pbCoverage{} may be expressed in MA-STRIPS
% \cite{DBLP:conf/aips/BrafmanD08}, that is a multi-agent variant of STRIPS
% (Stanford Research Institute Problem Solver) in which actions for each agent
% can be described independently. The representation in multi-agent planning
% languages is especially efficient when actions of the different agents are
% independent and when they required to coordinate not so often. However, as the
% agents should maintain connection, it requires a lot of coordination.

Murano et al. \cite{DBLP:conf/prima/MuranoPR15} advocate for a
graph-theoretic representations of states, that is, by assigning locations to
agents as in Definition~\ref{def:config}. In
\cite{DBLP:conf/atal/AminofMRZ16,DBLP:conf/atal/Rubin15}, a general formalism is given to specify LTL and monadic second-order logic properties,
which are expressive enough to describe the connectivity constraint.
%Indeed, linear temporal operators enable to express that any vertex
%should be visited in the future and the connectivity invariant. MSO on the
%topological graph enables to express the connectivity as a fix point (the
%subgraph made up of the UAVs and the base is connected).
They provide an
algorithm for parametrized verification in the sense that they check a temporal
property in a class of graphs. This is relevant for partially-known
environments. The algorithm described is
non-elementary %\footnote{\todo[inline]{explain non-elementary}}
(\textit{i.e.} the running time cannot bounded by any tower of exponentials)
and therefore not
usable in practice. We believe that this is an important problem
and our paper identifies an efficient and relevant fragment.
%studying fragments of this is
%relevant, and our paper identifies a relevant fragment.

The multiple traveling salesman problem (mTSP) is a generalization of the
traveling salesman problem (TSP) in which multiple salesmen are located at a
depot \cite{Anbuudayasankar:2016}. mTSP asks for the coverage of all cities so as to minimize the total plan cost by visiting each city exactly once.
%This generalization of TSP is considered as a
%relaxation of the vehicle routing problem (VRP) in which the capacity constraint
%is removed.
An overview of TSP and its extensions are presented
\cite{Matai:2010}. The \pbCoverage{} problem is related to mTSP, since we use
results on Hamiltonian cycle to prove the NP-hardness of \pbBCoverageCC.
However, we wish to minimize the length of the execution and not the cost of the
execution. Those problems are equivalent on unit graphs, but it is not trivial
to use general results on mTSP in order to solve \pbCoverage{}. Furthermore, to
the best of our knowledge, connected versions of mTSP and VRP have not been
studied.

%%% Local Variables:
%%% mode: latex
%%% TeX-master: "main"
%%% End:
