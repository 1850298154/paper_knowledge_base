A number of use cases of planning rose in information-gathering missions from
the development of unmanned autonomous vehicles (UAVs). For instance, in search
and rescue missions, a fleet of drones can cover a lot of ground in a short
amount of time and report any finding to a mission supervisor to narrow the
search for the rescue team. Other examples are the analysis of terrain for smart
farms and in hazardous locations. For this kind of missions, the information
gathered is used for decision making at a supervising station. Thus, the drones
need to be constantly in communication with the station to report the gathered information
during the mission. The use of multiple UAVs to cover an area not only reduces the time
required to complete the mission but can also enable reaching locations which would
not be reachable with a single drone due to connection constraints.
%be used to reach some location,
%where a single one cannot, without losing the connection with the supervisor.
\iffull
\else
\blfootnote{$^1$\href{https://drive.google.com/open?id=1dmvAWjN6vNwxGIBw9TmXCOjw3gKCZWoh}{https://drive.google.com/open?id=1dmvAWjN6vNwxGIBw9T\newline mXCOjw3gKCZWoh}}
\fi
The original multi-agent path planning problem asks for a plan to reach a
configuration of agents in a graph. However, an important problem for search and
rescue missions or terrain analysis is the coverage of an area. We thus study
both the coverage problem and the reachability problem under a connection
constraint over the agents which requires them to be connected to the base
either directly or via another agent, who can relay its data. We establish the
computational complexity of the connected coverage in its general case and for a
practical subclass introduced recently~\cite{dblp:conf/aaai/tateobrab18} in
which the UAVs can communicate with others located within one step, called the
\emph{neighbor-communicable} topological graphs. We show that the coverage is
PSPACE-complete in the general case, and remains so for neighbor-communicable
topological graphs. Thus, restricting to neighbor-communicable graphs does not
render the problem feasible, and the relatively high complexity unfortunately
remains. Note that this is in line with the PSPACE-completeness of the
reachability problem recently reported in~\cite{dblp:conf/aaai/tateobrab18}.
%Note that the PSPACE-completeness of the reachability problem was recently shown~


% We show that, in the
% general case, the complexity of deciding whether the coverage is achievable is
% PSPACE-complete and that this complexity result holds on the
% neighbor-communicable topological graphs.
%\textbf{\textcolor{red}{TODO: CITE RETINA}}

Our main result in this paper is the definition of a class of topological graphs
which is well adapted and realistic for UAV missions, and for which the coverage
and reachability problems admit efficient solutions. Our subclass, called
\emph{sight-moveable} graphs, is defined assuming that the UAVs cannot
communicate through obstacles and are restricted to line-of-sight communication.
This class emerged from an ongoing case study for a drone assisted search and
rescue project in which the authors take part. For this class, we prove that
both the reachability and coverage problems are in LOGSPACE while the existence
of a bounded execution is in NP. This drastically changes the status of this
problem since by LOGSPACE $\subseteq$ NC (this is the class of problems solvable
in polylogarithmic time in a parallel machine with a polynomial number of
processors),
%\footnote{\todo[inline]{Explain NC !}}, 
%the LOGSPACE
%membership means that
one can build an efficient parallel algorithm~\cite{Cook:1979}.
The NP upper bound is also useful since this means efficient SAT solvers can be used directly
to compute bounded executions.
We prove that our algorithms for the bounded variants are optimal by showing NP-hardness
in each case.

%The motivation behind our subclass is the following.
%In some applications, the UAVs gather a large amount of data for a rigorous
%analysis. Hence, the connection between the UAVs and the supervising station
%needs to be reliable.
%the sight-moveable topological graphs.
%We show that deciding the existence of a
%feasible plan can be achieved by a logarithmic-space algorithm and that deciding
%of the existence of a bounded execution is NP-complete. 
%In addition to the
%reachability of a configuration, we study the problem of bounded and unbounded
%coverage on sight-moveable topological graphs. We claim that the unbounded
%version is in LOGSPACE, and the bounded version is NP-complete. 

\begin{figure}[t]
    \vspace{0.4cm}
    \begin{center}
      \hspace{-1.3cm}
      \begin{tikzpicture}[scale=0.5]
      \tikzstyle{node} = [draw, circle, fill=black, inner sep=0.5mm];\tikzstyle{basenode} = [draw, circle, red, fill=red, inner sep=0.5mm];
      \tikzstyle{basenode} = [draw, circle, red, fill=red, inner sep=0.5mm];
      \tikzstyle{movetransition} = [draw, gray!50!black];
      \tikzstyle{communicationtransition} = [communication, bend right=20, tobehere];
      \node[basenode] (node0) at (0.5, 2.5){};
      \node[node] (node1) at (0.5, 0.5){};
      \node[node] (node2) at (2.5, 0.5){};
      \node[node] (node3) at (2.5, 2.5){};
      \node[node] (node4) at (1.5, 1.5){};
      \node[node] (node5) at (3.5, 1.5){};
      \draw[use as bounding box] (node5);
      \node[node] (node6) at (5.5, 1.5){};
      \node[node] (node7) at (4.5, 2.5){};
      \node[node] (node8) at (4.5, 0.5){};
      \node[node] (node9) at (6.5, 0.5){};
      \node[node] (node10) at (6.5, 2.5){};
      \draw[movetransition] (node0) edge (node0);
      \draw[movetransition] (node0) edge (node1);
      \draw[movetransition] (node0) edge (node3);
      \draw[movetransition] (node0) edge (node4);
      \draw[movetransition] (node1) edge (node0);
      \draw[movetransition] (node1) edge (node1);
      \draw[movetransition] (node1) edge (node2);
      \draw[movetransition] (node1) edge (node4);
      \draw[movetransition] (node2) edge (node1);
      \draw[movetransition] (node2) edge (node2);
      \draw[movetransition] (node2) edge (node4);
      \draw[movetransition] (node3) edge (node0);
      \draw[movetransition] (node3) edge (node3);
      \draw[movetransition] (node3) edge (node4);
      \draw[movetransition] (node4) edge (node0);
      \draw[movetransition] (node4) edge (node1);
      \draw[movetransition] (node4) edge (node2);
      \draw[movetransition] (node4) edge (node3);
      \draw[movetransition] (node4) edge (node4);
      \draw[movetransition] (node4) edge (node5);
      \draw[movetransition] (node5) edge (node4);
      \draw[movetransition] (node5) edge (node5);
      \draw[movetransition] (node5) edge (node6);
      \draw[movetransition] (node6) edge (node5);
      \draw[movetransition] (node6) edge (node6);
      \draw[movetransition] (node6) edge (node7);
      \draw[movetransition] (node6) edge (node8);
      \draw[movetransition] (node6) edge (node9);
      \draw[movetransition] (node6) edge (node10);
      \draw[movetransition] (node7) edge (node6);
      \draw[movetransition] (node7) edge (node7);
      \draw[movetransition] (node7) edge (node10);
      \draw[movetransition] (node8) edge (node6);
      \draw[movetransition] (node8) edge (node8);
      \draw[movetransition] (node8) edge (node9);
      \draw[movetransition] (node9) edge (node6);
      \draw[movetransition] (node9) edge (node8);
      \draw[movetransition] (node9) edge (node9);
      \draw[movetransition] (node9) edge (node10);
      \draw[movetransition] (node10) edge (node6);
      \draw[movetransition] (node10) edge (node7);
      \draw[movetransition] (node10) edge (node9);
      \draw[movetransition] (node10) edge (node10);
      \draw[communicationtransition] (node0) edge (node1);
      \draw[communicationtransition] (node0) edge (node2);
      \draw[communicationtransition] (node0) edge (node3);
      \draw[communicationtransition] (node0) edge (node4);
      \draw[communicationtransition] (node1) edge (node2);
      \draw[communicationtransition] (node1) edge (node3);
      \draw[communicationtransition] (node1) edge (node4);
      \draw[communicationtransition] (node2) edge (node4);
      \draw[communicationtransition] (node2) edge (node8);
      \draw[communicationtransition] (node3) edge (node4);
      \draw[communicationtransition] (node3) edge (node7);
      \draw[communicationtransition] (node4) edge (node5);
      \draw[communicationtransition] (node4) edge (node6);
      \draw[communicationtransition] (node5) edge (node6);
      \draw[communicationtransition] (node6) edge (node7);
      \draw[communicationtransition] (node6) edge (node8);
      \draw[communicationtransition] (node6) edge (node9);
      \draw[communicationtransition] (node6) edge (node10);
      \draw[communicationtransition] (node7) edge (node9);
      \draw[communicationtransition] (node7) edge (node10);
      \draw[communicationtransition] (node8) edge (node9);
      \draw[communicationtransition] (node8) edge (node10);
      \draw[communicationtransition] (node9) edge (node10);
      \node at (0.5, 2.9) {\imgDrone};
    %   \node at (0.46, 2.86) {\imgDrone};
    %   \node at (0.42, 2.82) {\imgDrone};
      \end{tikzpicture}
      \vspace{0.4cm}
      %
      \hspace{2.3cm}
      \begin{tikzpicture}[scale=0.5]
      \tikzstyle{node} = [draw, circle, fill=black, inner sep=0.5mm];\tikzstyle{basenode} = [draw, circle, red, fill=red, inner sep=0.5mm];
      \tikzstyle{movetransition} = [draw, gray!50!black];
      \tikzstyle{communicationtransition} = [communication, bend right=20, tobehere];
      \node[basenode] (node0) at (0.5, 2.5){};
      \node[node] (node1) at (0.5, 0.5){};
      \node[node] (node2) at (2.5, 0.5){};
      \node[node] (node3) at (2.5, 2.5){};
      \node[node] (node4) at (1.5, 1.5){};
      \node[node] (node5) at (3.5, 1.5){};
      \draw[use as bounding box] (node5);
      \node[node] (node6) at (5.5, 1.5){};
      \node[node] (node7) at (4.5, 2.5){};
      \node[node] (node8) at (4.5, 0.5){};
      \node[node] (node9) at (6.5, 0.5){};
      \node[node] (node10) at (6.5, 2.5){};
      \draw[movetransition] (node0) edge (node0);
      \draw[movetransition] (node0) edge (node1);
      \draw[movetransition] (node0) edge (node3);
      \draw[movetransition] (node0) edge (node4);
      \draw[movetransition] (node1) edge (node0);
      \draw[movetransition] (node1) edge (node1);
      \draw[movetransition] (node1) edge (node2);
      \draw[movetransition] (node1) edge (node4);
      \draw[movetransition] (node2) edge (node1);
      \draw[movetransition] (node2) edge (node2);
      \draw[movetransition] (node2) edge (node4);
      \draw[movetransition] (node3) edge (node0);
      \draw[movetransition] (node3) edge (node3);
      \draw[movetransition] (node3) edge (node4);
      \draw[movetransition] (node4) edge (node0);
      \draw[movetransition] (node4) edge (node1);
      \draw[movetransition] (node4) edge (node2);
      \draw[movetransition] (node4) edge (node3);
      \draw[movetransition] (node4) edge (node4);
      \draw[movetransition] (node4) edge (node5);
      \draw[movetransition] (node5) edge (node4);
      \draw[movetransition] (node5) edge (node5);
      \draw[movetransition] (node5) edge (node6);
      \draw[movetransition] (node6) edge (node5);
      \draw[movetransition] (node6) edge (node6);
      \draw[movetransition] (node6) edge (node7);
      \draw[movetransition] (node6) edge (node8);
      \draw[movetransition] (node6) edge (node9);
      \draw[movetransition] (node6) edge (node10);
      \draw[movetransition] (node7) edge (node6);
      \draw[movetransition] (node7) edge (node7);
      \draw[movetransition] (node7) edge (node10);
      \draw[movetransition] (node8) edge (node6);
      \draw[movetransition] (node8) edge (node8);
      \draw[movetransition] (node8) edge (node9);
      \draw[movetransition] (node9) edge (node6);
      \draw[movetransition] (node9) edge (node8);
      \draw[movetransition] (node9) edge (node9);
      \draw[movetransition] (node9) edge (node10);
      \draw[movetransition] (node10) edge (node6);
      \draw[movetransition] (node10) edge (node7);
      \draw[movetransition] (node10) edge (node9);
      \draw[movetransition] (node10) edge (node10);
      \draw[communicationtransition] (node0) edge (node1);
      \draw[communicationtransition] (node0) edge (node2);
      \draw[communicationtransition] (node0) edge (node3);
      \draw[communicationtransition] (node0) edge (node4);
      \draw[communicationtransition] (node1) edge (node2);
      \draw[communicationtransition] (node1) edge (node3);
      \draw[communicationtransition] (node1) edge (node4);
      \draw[communicationtransition] (node2) edge (node4);
      \draw[communicationtransition] (node2) edge (node8);
      \draw[communicationtransition] (node3) edge (node4);
      \draw[communicationtransition] (node3) edge (node7);
      \draw[communicationtransition] (node4) edge (node5);
      \draw[communicationtransition] (node4) edge (node6);
      \draw[communicationtransition] (node5) edge (node6);
      \draw[communicationtransition] (node6) edge (node7);
      \draw[communicationtransition] (node6) edge (node8);
      \draw[communicationtransition] (node6) edge (node9);
      \draw[communicationtransition] (node6) edge (node10);
      \draw[communicationtransition] (node7) edge (node9);
      \draw[communicationtransition] (node7) edge (node10);
      \draw[communicationtransition] (node8) edge (node9);
      \draw[communicationtransition] (node8) edge (node10);
      \draw[communicationtransition] (node9) edge (node10);
      \draw[communicationtransition, orange] (node0) edge (node4);
      \draw[communicationtransition, orange] (node0) edge (node4);
      \draw[communicationtransition, orange] (node0) edge (node4);
      \node at (0.5, 2.9) {\imgDrone};
    %   \node at (0.46, 2.86) {\imgDrone};
      \node at (1.5, 1.9) {\imgDrone};
      \end{tikzpicture}
      %
      
      \hspace{-1.3cm}
      \begin{tikzpicture}[scale=0.5]
      \tikzstyle{node} = [draw, circle, fill=black, inner sep=0.5mm];\tikzstyle{basenode} = [draw, circle, red, fill=red, inner sep=0.5mm];
      \tikzstyle{movetransition} = [draw, gray!50!black];
      \tikzstyle{communicationtransition} = [communication, bend right=20, tobehere];
      \node[basenode] (node0) at (0.5, 2.5){};
      \node[node] (node1) at (0.5, 0.5){};
      \node[node] (node2) at (2.5, 0.5){};
      \node[node] (node3) at (2.5, 2.5){};
      \node[node] (node4) at (1.5, 1.5){};
      \node[node] (node5) at (3.5, 1.5){};
      \draw[use as bounding box] (node5);
      \node[node] (node6) at (5.5, 1.5){};
      \node[node] (node7) at (4.5, 2.5){};
      \node[node] (node8) at (4.5, 0.5){};
      \node[node] (node9) at (6.5, 0.5){};
      \node[node] (node10) at (6.5, 2.5){};
      \draw[movetransition] (node0) edge (node0);
      \draw[movetransition] (node0) edge (node1);
      \draw[movetransition] (node0) edge (node3);
      \draw[movetransition] (node0) edge (node4);
      \draw[movetransition] (node1) edge (node0);
      \draw[movetransition] (node1) edge (node1);
      \draw[movetransition] (node1) edge (node2);
      \draw[movetransition] (node1) edge (node4);
      \draw[movetransition] (node2) edge (node1);
      \draw[movetransition] (node2) edge (node2);
      \draw[movetransition] (node2) edge (node4);
      \draw[movetransition] (node3) edge (node0);
      \draw[movetransition] (node3) edge (node3);
      \draw[movetransition] (node3) edge (node4);
      \draw[movetransition] (node4) edge (node0);
      \draw[movetransition] (node4) edge (node1);
      \draw[movetransition] (node4) edge (node2);
      \draw[movetransition] (node4) edge (node3);
      \draw[movetransition] (node4) edge (node4);
      \draw[movetransition] (node4) edge (node5);
      \draw[movetransition] (node5) edge (node4);
      \draw[movetransition] (node5) edge (node5);
      \draw[movetransition] (node5) edge (node6);
      \draw[movetransition] (node6) edge (node5);
      \draw[movetransition] (node6) edge (node6);
      \draw[movetransition] (node6) edge (node7);
      \draw[movetransition] (node6) edge (node8);
      \draw[movetransition] (node6) edge (node9);
      \draw[movetransition] (node6) edge (node10);
      \draw[movetransition] (node7) edge (node6);
      \draw[movetransition] (node7) edge (node7);
      \draw[movetransition] (node7) edge (node10);
      \draw[movetransition] (node8) edge (node6);
      \draw[movetransition] (node8) edge (node8);
      \draw[movetransition] (node8) edge (node9);
      \draw[movetransition] (node9) edge (node6);
      \draw[movetransition] (node9) edge (node8);
      \draw[movetransition] (node9) edge (node9);
      \draw[movetransition] (node9) edge (node10);
      \draw[movetransition] (node10) edge (node6);
      \draw[movetransition] (node10) edge (node7);
      \draw[movetransition] (node10) edge (node9);
      \draw[movetransition] (node10) edge (node10);
      \draw[communicationtransition] (node0) edge (node1);
      \draw[communicationtransition] (node0) edge (node2);
      \draw[communicationtransition] (node0) edge (node3);
      \draw[communicationtransition] (node0) edge (node4);
      \draw[communicationtransition] (node1) edge (node2);
      \draw[communicationtransition] (node1) edge (node3);
      \draw[communicationtransition] (node1) edge (node4);
      \draw[communicationtransition] (node2) edge (node4);
      \draw[communicationtransition] (node2) edge (node8);
      \draw[communicationtransition] (node3) edge (node4);
      \draw[communicationtransition] (node3) edge (node7);
      \draw[communicationtransition] (node4) edge (node5);
      \draw[communicationtransition] (node4) edge (node6);
      \draw[communicationtransition] (node5) edge (node6);
      \draw[communicationtransition] (node6) edge (node7);
      \draw[communicationtransition] (node6) edge (node8);
      \draw[communicationtransition] (node6) edge (node9);
      \draw[communicationtransition] (node6) edge (node10);
      \draw[communicationtransition] (node7) edge (node9);
      \draw[communicationtransition] (node7) edge (node10);
      \draw[communicationtransition] (node8) edge (node9);
      \draw[communicationtransition] (node8) edge (node10);
      \draw[communicationtransition] (node9) edge (node10);
      \draw[communicationtransition, orange] (node0) edge (node4);
      \draw[communicationtransition, orange] (node4) edge (node5);
      \draw[communicationtransition, orange] (node0) edge (node4);
      \node at (1.5, 1.5) {\ok};
      \node at (0.5, 2.9) {\imgDrone};
      \node at (1.5, 1.9) {\imgDrone};
      \node at (3.5, 1.9) {\imgDrone};
      \end{tikzpicture}
      \vspace{0.4cm}
      %
      \hspace{2.3cm}
      \begin{tikzpicture}[scale=0.5]
      \tikzstyle{node} = [draw, circle, fill=black, inner sep=0.5mm];\tikzstyle{basenode} = [draw, circle, red, fill=red, inner sep=0.5mm];
      \tikzstyle{movetransition} = [draw, gray!50!black];
      \tikzstyle{communicationtransition} = [communication, bend right=20, tobehere];
      \node[basenode] (node0) at (0.5, 2.5){};
      \node[node] (node1) at (0.5, 0.5){};
      \node[node] (node2) at (2.5, 0.5){};
      \node[node] (node3) at (2.5, 2.5){};
      \node[node] (node4) at (1.5, 1.5){};
      \node[node] (node5) at (3.5, 1.5){};
      \draw[use as bounding box] (node5);
      \node[node] (node6) at (5.5, 1.5){};
      \node[node] (node7) at (4.5, 2.5){};
      \node[node] (node8) at (4.5, 0.5){};
      \node[node] (node9) at (6.5, 0.5){};
      \node[node] (node10) at (6.5, 2.5){};
      \draw[movetransition] (node0) edge (node0);
      \draw[movetransition] (node0) edge (node1);
      \draw[movetransition] (node0) edge (node3);
      \draw[movetransition] (node0) edge (node4);
      \draw[movetransition] (node1) edge (node0);
      \draw[movetransition] (node1) edge (node1);
      \draw[movetransition] (node1) edge (node2);
      \draw[movetransition] (node1) edge (node4);
      \draw[movetransition] (node2) edge (node1);
      \draw[movetransition] (node2) edge (node2);
      \draw[movetransition] (node2) edge (node4);
      \draw[movetransition] (node3) edge (node0);
      \draw[movetransition] (node3) edge (node3);
      \draw[movetransition] (node3) edge (node4);
      \draw[movetransition] (node4) edge (node0);
      \draw[movetransition] (node4) edge (node1);
      \draw[movetransition] (node4) edge (node2);
      \draw[movetransition] (node4) edge (node3);
      \draw[movetransition] (node4) edge (node4);
      \draw[movetransition] (node4) edge (node5);
      \draw[movetransition] (node5) edge (node4);
      \draw[movetransition] (node5) edge (node5);
      \draw[movetransition] (node5) edge (node6);
      \draw[movetransition] (node6) edge (node5);
      \draw[movetransition] (node6) edge (node6);
      \draw[movetransition] (node6) edge (node7);
      \draw[movetransition] (node6) edge (node8);
      \draw[movetransition] (node6) edge (node9);
      \draw[movetransition] (node6) edge (node10);
      \draw[movetransition] (node7) edge (node6);
      \draw[movetransition] (node7) edge (node7);
      \draw[movetransition] (node7) edge (node10);
      \draw[movetransition] (node8) edge (node6);
      \draw[movetransition] (node8) edge (node8);
      \draw[movetransition] (node8) edge (node9);
      \draw[movetransition] (node9) edge (node6);
      \draw[movetransition] (node9) edge (node8);
      \draw[movetransition] (node9) edge (node9);
      \draw[movetransition] (node9) edge (node10);
      \draw[movetransition] (node10) edge (node6);
      \draw[movetransition] (node10) edge (node7);
      \draw[movetransition] (node10) edge (node9);
      \draw[movetransition] (node10) edge (node10);
      \draw[communicationtransition] (node0) edge (node1);
      \draw[communicationtransition] (node0) edge (node2);
      \draw[communicationtransition] (node0) edge (node3);
      \draw[communicationtransition] (node0) edge (node4);
      \draw[communicationtransition] (node1) edge (node2);
      \draw[communicationtransition] (node1) edge (node3);
      \draw[communicationtransition] (node1) edge (node4);
      \draw[communicationtransition] (node2) edge (node4);
      \draw[communicationtransition] (node2) edge (node8);
      \draw[communicationtransition] (node3) edge (node4);
      \draw[communicationtransition] (node3) edge (node7);
      \draw[communicationtransition] (node4) edge (node5);
      \draw[communicationtransition] (node4) edge (node6);
      \draw[communicationtransition] (node5) edge (node6);
      \draw[communicationtransition] (node6) edge (node7);
      \draw[communicationtransition] (node6) edge (node8);
      \draw[communicationtransition] (node6) edge (node9);
      \draw[communicationtransition] (node6) edge (node10);
      \draw[communicationtransition] (node7) edge (node9);
      \draw[communicationtransition] (node7) edge (node10);
      \draw[communicationtransition] (node8) edge (node9);
      \draw[communicationtransition] (node8) edge (node10);
      \draw[communicationtransition] (node9) edge (node10);
      \draw[communicationtransition, orange] (node4) edge (node5);
      \draw[communicationtransition, orange] (node4) edge (node6);
      \draw[communicationtransition, orange] (node5) edge (node6);
      \draw[communicationtransition, orange] (node0) edge (node4);
      \node at (1.5, 1.5) {\ok};
      \node at (3.5, 1.5) {\ok};
      \node at (1.5, 1.9) {\imgDrone};
      \node at (3.5, 1.9) {\imgDrone};
      \node at (5.5, 1.9) {\imgDrone};
      \end{tikzpicture}
      %
      
      \hspace{-1.3cm}
      \begin{tikzpicture}[scale=0.5]
      \tikzstyle{node} = [draw, circle, fill=black, inner sep=0.5mm];\tikzstyle{basenode} = [draw, circle, red, fill=red, inner sep=0.5mm];
      \tikzstyle{movetransition} = [draw, gray!50!black];
      \tikzstyle{communicationtransition} = [communication, bend right=20, tobehere];
      \node[basenode] (node0) at (0.5, 2.5){};
      \node[node] (node1) at (0.5, 0.5){};
      \node[node] (node2) at (2.5, 0.5){};
      \node[node] (node3) at (2.5, 2.5){};
      \node[node] (node4) at (1.5, 1.5){};
      \node[node] (node5) at (3.5, 1.5){};
      \draw[use as bounding box] (node5);
      \node[node] (node6) at (5.5, 1.5){};
      \node[node] (node7) at (4.5, 2.5){};
      \node[node] (node8) at (4.5, 0.5){};
      \node[node] (node9) at (6.5, 0.5){};
      \node[node] (node10) at (6.5, 2.5){};
      \draw[movetransition] (node0) edge (node0);
      \draw[movetransition] (node0) edge (node1);
      \draw[movetransition] (node0) edge (node3);
      \draw[movetransition] (node0) edge (node4);
      \draw[movetransition] (node1) edge (node0);
      \draw[movetransition] (node1) edge (node1);
      \draw[movetransition] (node1) edge (node2);
      \draw[movetransition] (node1) edge (node4);
      \draw[movetransition] (node2) edge (node1);
      \draw[movetransition] (node2) edge (node2);
      \draw[movetransition] (node2) edge (node4);
      \draw[movetransition] (node3) edge (node0);
      \draw[movetransition] (node3) edge (node3);
      \draw[movetransition] (node3) edge (node4);
      \draw[movetransition] (node4) edge (node0);
      \draw[movetransition] (node4) edge (node1);
      \draw[movetransition] (node4) edge (node2);
      \draw[movetransition] (node4) edge (node3);
      \draw[movetransition] (node4) edge (node4);
      \draw[movetransition] (node4) edge (node5);
      \draw[movetransition] (node5) edge (node4);
      \draw[movetransition] (node5) edge (node5);
      \draw[movetransition] (node5) edge (node6);
      \draw[movetransition] (node6) edge (node5);
      \draw[movetransition] (node6) edge (node6);
      \draw[movetransition] (node6) edge (node7);
      \draw[movetransition] (node6) edge (node8);
      \draw[movetransition] (node6) edge (node9);
      \draw[movetransition] (node6) edge (node10);
      \draw[movetransition] (node7) edge (node6);
      \draw[movetransition] (node7) edge (node7);
      \draw[movetransition] (node7) edge (node10);
      \draw[movetransition] (node8) edge (node6);
      \draw[movetransition] (node8) edge (node8);
      \draw[movetransition] (node8) edge (node9);
      \draw[movetransition] (node9) edge (node6);
      \draw[movetransition] (node9) edge (node8);
      \draw[movetransition] (node9) edge (node9);
      \draw[movetransition] (node9) edge (node10);
      \draw[movetransition] (node10) edge (node6);
      \draw[movetransition] (node10) edge (node7);
      \draw[movetransition] (node10) edge (node9);
      \draw[movetransition] (node10) edge (node10);
      \draw[communicationtransition] (node0) edge (node1);
      \draw[communicationtransition] (node0) edge (node2);
      \draw[communicationtransition] (node0) edge (node3);
      \draw[communicationtransition] (node0) edge (node4);
      \draw[communicationtransition] (node1) edge (node2);
      \draw[communicationtransition] (node1) edge (node3);
      \draw[communicationtransition] (node1) edge (node4);
      \draw[communicationtransition] (node2) edge (node4);
      \draw[communicationtransition] (node2) edge (node8);
      \draw[communicationtransition] (node3) edge (node4);
      \draw[communicationtransition] (node3) edge (node7);
      \draw[communicationtransition] (node4) edge (node5);
      \draw[communicationtransition] (node4) edge (node6);
      \draw[communicationtransition] (node5) edge (node6);
      \draw[communicationtransition] (node6) edge (node7);
      \draw[communicationtransition] (node6) edge (node8);
      \draw[communicationtransition] (node6) edge (node9);
      \draw[communicationtransition] (node6) edge (node10);
      \draw[communicationtransition] (node7) edge (node9);
      \draw[communicationtransition] (node7) edge (node10);
      \draw[communicationtransition] (node8) edge (node9);
      \draw[communicationtransition] (node8) edge (node10);
      \draw[communicationtransition] (node9) edge (node10);
      \draw[communicationtransition, orange] (node4) edge (node6);
      \draw[communicationtransition, orange] (node6) edge (node10);
      \draw[communicationtransition, orange] (node0) edge (node4);
      \node at (1.5, 1.5) {\ok};
      \node at (3.5, 1.5) {\ok};
      \node at (5.5, 1.5) {\ok};
      \node at (1.5, 1.9) {\imgDrone};
      \node at (5.5, 1.9) {\imgDrone};
      \node at (6.5, 2.9) {\imgDrone};
      \end{tikzpicture}
      \vspace{0.4cm}
      %
      \hspace{2.3cm}
      \begin{tikzpicture}[scale=0.5]
      \tikzstyle{node} = [draw, circle, fill=black, inner sep=0.5mm];\tikzstyle{basenode} = [draw, circle, red, fill=red, inner sep=0.5mm];
      \tikzstyle{movetransition} = [draw, gray!50!black];
      \tikzstyle{communicationtransition} = [communication, bend right=20, tobehere];
      \node[basenode] (node0) at (0.5, 2.5){};
      \node[node] (node1) at (0.5, 0.5){};
      \node[node] (node2) at (2.5, 0.5){};
      \node[node] (node3) at (2.5, 2.5){};
      \node[node] (node4) at (1.5, 1.5){};
      \node[node] (node5) at (3.5, 1.5){};
      \draw[use as bounding box] (node5);
      \node[node] (node6) at (5.5, 1.5){};
      \node[node] (node7) at (4.5, 2.5){};
      \node[node] (node8) at (4.5, 0.5){};
      \node[node] (node9) at (6.5, 0.5){};
      \node[node] (node10) at (6.5, 2.5){};
      \draw[movetransition] (node0) edge (node0);
      \draw[movetransition] (node0) edge (node1);
      \draw[movetransition] (node0) edge (node3);
      \draw[movetransition] (node0) edge (node4);
      \draw[movetransition] (node1) edge (node0);
      \draw[movetransition] (node1) edge (node1);
      \draw[movetransition] (node1) edge (node2);
      \draw[movetransition] (node1) edge (node4);
      \draw[movetransition] (node2) edge (node1);
      \draw[movetransition] (node2) edge (node2);
      \draw[movetransition] (node2) edge (node4);
      \draw[movetransition] (node3) edge (node0);
      \draw[movetransition] (node3) edge (node3);
      \draw[movetransition] (node3) edge (node4);
      \draw[movetransition] (node4) edge (node0);
      \draw[movetransition] (node4) edge (node1);
      \draw[movetransition] (node4) edge (node2);
      \draw[movetransition] (node4) edge (node3);
      \draw[movetransition] (node4) edge (node4);
      \draw[movetransition] (node4) edge (node5);
      \draw[movetransition] (node5) edge (node4);
      \draw[movetransition] (node5) edge (node5);
      \draw[movetransition] (node5) edge (node6);
      \draw[movetransition] (node6) edge (node5);
      \draw[movetransition] (node6) edge (node6);
      \draw[movetransition] (node6) edge (node7);
      \draw[movetransition] (node6) edge (node8);
      \draw[movetransition] (node6) edge (node9);
      \draw[movetransition] (node6) edge (node10);
      \draw[movetransition] (node7) edge (node6);
      \draw[movetransition] (node7) edge (node7);
      \draw[movetransition] (node7) edge (node10);
      \draw[movetransition] (node8) edge (node6);
      \draw[movetransition] (node8) edge (node8);
      \draw[movetransition] (node8) edge (node9);
      \draw[movetransition] (node9) edge (node6);
      \draw[movetransition] (node9) edge (node8);
      \draw[movetransition] (node9) edge (node9);
      \draw[movetransition] (node9) edge (node10);
      \draw[movetransition] (node10) edge (node6);
      \draw[movetransition] (node10) edge (node7);
      \draw[movetransition] (node10) edge (node9);
      \draw[movetransition] (node10) edge (node10);
      \draw[communicationtransition] (node0) edge (node1);
      \draw[communicationtransition] (node0) edge (node2);
      \draw[communicationtransition] (node0) edge (node3);
      \draw[communicationtransition] (node0) edge (node4);
      \draw[communicationtransition] (node1) edge (node2);
      \draw[communicationtransition] (node1) edge (node3);
      \draw[communicationtransition] (node1) edge (node4);
      \draw[communicationtransition] (node2) edge (node4);
      \draw[communicationtransition] (node2) edge (node8);
      \draw[communicationtransition] (node3) edge (node4);
      \draw[communicationtransition] (node3) edge (node7);
      \draw[communicationtransition] (node4) edge (node5);
      \draw[communicationtransition] (node4) edge (node6);
      \draw[communicationtransition] (node5) edge (node6);
      \draw[communicationtransition] (node6) edge (node7);
      \draw[communicationtransition] (node6) edge (node8);
      \draw[communicationtransition] (node6) edge (node9);
      \draw[communicationtransition] (node6) edge (node10);
      \draw[communicationtransition] (node7) edge (node9);
      \draw[communicationtransition] (node7) edge (node10);
      \draw[communicationtransition] (node8) edge (node9);
      \draw[communicationtransition] (node8) edge (node10);
      \draw[communicationtransition] (node9) edge (node10);
      \draw[communicationtransition, orange] (node3) edge (node7);
      \draw[communicationtransition, orange] (node7) edge (node9);
      \draw[communicationtransition, orange] (node0) edge (node3);
      \node at (1.5, 1.5) {\ok};
      \node at (3.5, 1.5) {\ok};
      \node at (5.5, 1.5) {\ok};
      \node at (6.5, 2.5) {\ok};
      \node at (2.5, 2.9) {\imgDrone};
      \node at (6.5, 0.9) {\imgDrone};
      \node at (4.5, 2.9) {\imgDrone};
      \end{tikzpicture}
      %
      
      \hspace{-1.3cm}
      \begin{tikzpicture}[scale=0.5]
      \tikzstyle{node} = [draw, circle, fill=black, inner sep=0.5mm];\tikzstyle{basenode} = [draw, circle, red, fill=red, inner sep=0.5mm];
      \tikzstyle{movetransition} = [draw, gray!50!black];
      \tikzstyle{communicationtransition} = [communication, bend right=20, tobehere];
      \node[basenode] (node0) at (0.5, 2.5){};
      \node[node] (node1) at (0.5, 0.5){};
      \node[node] (node2) at (2.5, 0.5){};
      \node[node] (node3) at (2.5, 2.5){};
      \node[node] (node4) at (1.5, 1.5){};
      \node[node] (node5) at (3.5, 1.5){};
      \draw[use as bounding box] (node5);
      \node[node] (node6) at (5.5, 1.5){};
      \node[node] (node7) at (4.5, 2.5){};
      \node[node] (node8) at (4.5, 0.5){};
      \node[node] (node9) at (6.5, 0.5){};
      \node[node] (node10) at (6.5, 2.5){};
      \draw[movetransition] (node0) edge (node0);
      \draw[movetransition] (node0) edge (node1);
      \draw[movetransition] (node0) edge (node3);
      \draw[movetransition] (node0) edge (node4);
      \draw[movetransition] (node1) edge (node0);
      \draw[movetransition] (node1) edge (node1);
      \draw[movetransition] (node1) edge (node2);
      \draw[movetransition] (node1) edge (node4);
      \draw[movetransition] (node2) edge (node1);
      \draw[movetransition] (node2) edge (node2);
      \draw[movetransition] (node2) edge (node4);
      \draw[movetransition] (node3) edge (node0);
      \draw[movetransition] (node3) edge (node3);
      \draw[movetransition] (node3) edge (node4);
      \draw[movetransition] (node4) edge (node0);
      \draw[movetransition] (node4) edge (node1);
      \draw[movetransition] (node4) edge (node2);
      \draw[movetransition] (node4) edge (node3);
      \draw[movetransition] (node4) edge (node4);
      \draw[movetransition] (node4) edge (node5);
      \draw[movetransition] (node5) edge (node4);
      \draw[movetransition] (node5) edge (node5);
      \draw[movetransition] (node5) edge (node6);
      \draw[movetransition] (node6) edge (node5);
      \draw[movetransition] (node6) edge (node6);
      \draw[movetransition] (node6) edge (node7);
      \draw[movetransition] (node6) edge (node8);
      \draw[movetransition] (node6) edge (node9);
      \draw[movetransition] (node6) edge (node10);
      \draw[movetransition] (node7) edge (node6);
      \draw[movetransition] (node7) edge (node7);
      \draw[movetransition] (node7) edge (node10);
      \draw[movetransition] (node8) edge (node6);
      \draw[movetransition] (node8) edge (node8);
      \draw[movetransition] (node8) edge (node9);
      \draw[movetransition] (node9) edge (node6);
      \draw[movetransition] (node9) edge (node8);
      \draw[movetransition] (node9) edge (node9);
      \draw[movetransition] (node9) edge (node10);
      \draw[movetransition] (node10) edge (node6);
      \draw[movetransition] (node10) edge (node7);
      \draw[movetransition] (node10) edge (node9);
      \draw[movetransition] (node10) edge (node10);
      \draw[communicationtransition] (node0) edge (node1);
      \draw[communicationtransition] (node0) edge (node2);
      \draw[communicationtransition] (node0) edge (node3);
      \draw[communicationtransition] (node0) edge (node4);
      \draw[communicationtransition] (node1) edge (node2);
      \draw[communicationtransition] (node1) edge (node3);
      \draw[communicationtransition] (node1) edge (node4);
      \draw[communicationtransition] (node2) edge (node4);
      \draw[communicationtransition] (node2) edge (node8);
      \draw[communicationtransition] (node3) edge (node4);
      \draw[communicationtransition] (node3) edge (node7);
      \draw[communicationtransition] (node4) edge (node5);
      \draw[communicationtransition] (node4) edge (node6);
      \draw[communicationtransition] (node5) edge (node6);
      \draw[communicationtransition] (node6) edge (node7);
      \draw[communicationtransition] (node6) edge (node8);
      \draw[communicationtransition] (node6) edge (node9);
      \draw[communicationtransition] (node6) edge (node10);
      \draw[communicationtransition] (node7) edge (node9);
      \draw[communicationtransition] (node7) edge (node10);
      \draw[communicationtransition] (node8) edge (node9);
      \draw[communicationtransition] (node8) edge (node10);
      \draw[communicationtransition] (node9) edge (node10);
      \draw[communicationtransition, orange] (node4) edge (node6);
      \draw[communicationtransition, orange] (node6) edge (node8);
      \draw[communicationtransition, orange] (node0) edge (node4);
      \node at (2.5, 2.5) {\ok};
      \node at (1.5, 1.5) {\ok};
      \node at (3.5, 1.5) {\ok};
      \node at (5.5, 1.5) {\ok};
      \node at (4.5, 2.5) {\ok};
      \node at (6.5, 0.5) {\ok};
      \node at (6.5, 2.5) {\ok};
      \node at (1.5, 1.9) {\imgDrone};
      \node at (4.5, 0.9) {\imgDrone};
      \node at (5.5, 1.9) {\imgDrone};
      \end{tikzpicture}
      \vspace{0.4cm}
      %
      \hspace{2.3cm}
      \begin{tikzpicture}[scale=0.5]
      \tikzstyle{node} = [draw, circle, fill=black, inner sep=0.5mm];
      \tikzstyle{basenode} = [draw, circle, red, fill=red, inner sep=0.5mm];
      \tikzstyle{movetransition} = [draw, gray!50!black];
      \tikzstyle{communicationtransition} = [communication, bend right=20, tobehere];
      \node[basenode] (node0) at (0.5, 2.5){};
      \node[node] (node1) at (0.5, 0.5){};
      \node[node] (node2) at (2.5, 0.5){};
      \node[node] (node3) at (2.5, 2.5){};
      \node[node] (node4) at (1.5, 1.5){};
      \node[node] (node5) at (3.5, 1.5){};
      \draw[use as bounding box] (node5);
      \node[node] (node6) at (5.5, 1.5){};
      \node[node] (node7) at (4.5, 2.5){};
      \node[node] (node8) at (4.5, 0.5){};
      \node[node] (node9) at (6.5, 0.5){};
      \node[node] (node10) at (6.5, 2.5){};
      \draw[movetransition] (node0) edge (node0);
      \draw[movetransition] (node0) edge (node1);
      \draw[movetransition] (node0) edge (node3);
      \draw[movetransition] (node0) edge (node4);
      \draw[movetransition] (node1) edge (node0);
      \draw[movetransition] (node1) edge (node1);
      \draw[movetransition] (node1) edge (node2);
      \draw[movetransition] (node1) edge (node4);
      \draw[movetransition] (node2) edge (node1);
      \draw[movetransition] (node2) edge (node2);
      \draw[movetransition] (node2) edge (node4);
      \draw[movetransition] (node3) edge (node0);
      \draw[movetransition] (node3) edge (node3);
      \draw[movetransition] (node3) edge (node4);
      \draw[movetransition] (node4) edge (node0);
      \draw[movetransition] (node4) edge (node1);
      \draw[movetransition] (node4) edge (node2);
      \draw[movetransition] (node4) edge (node3);
      \draw[movetransition] (node4) edge (node4);
      \draw[movetransition] (node4) edge (node5);
      \draw[movetransition] (node5) edge (node4);
      \draw[movetransition] (node5) edge (node5);
      \draw[movetransition] (node5) edge (node6);
      \draw[movetransition] (node6) edge (node5);
      \draw[movetransition] (node6) edge (node6);
      \draw[movetransition] (node6) edge (node7);
      \draw[movetransition] (node6) edge (node8);
      \draw[movetransition] (node6) edge (node9);
      \draw[movetransition] (node6) edge (node10);
      \draw[movetransition] (node7) edge (node6);
      \draw[movetransition] (node7) edge (node7);
      \draw[movetransition] (node7) edge (node10);
      \draw[movetransition] (node8) edge (node6);
      \draw[movetransition] (node8) edge (node8);
      \draw[movetransition] (node8) edge (node9);
      \draw[movetransition] (node9) edge (node6);
      \draw[movetransition] (node9) edge (node8);
      \draw[movetransition] (node9) edge (node9);
      \draw[movetransition] (node9) edge (node10);
      \draw[movetransition] (node10) edge (node6);
      \draw[movetransition] (node10) edge (node7);
      \draw[movetransition] (node10) edge (node9);
      \draw[movetransition] (node10) edge (node10);
      \draw[communicationtransition] (node0) edge (node1);
      \draw[communicationtransition] (node0) edge (node2);
      \draw[communicationtransition] (node0) edge (node3);
      \draw[communicationtransition] (node0) edge (node4);
      \draw[communicationtransition] (node1) edge (node2);
      \draw[communicationtransition] (node1) edge (node3);
      \draw[communicationtransition] (node1) edge (node4);
      \draw[communicationtransition] (node2) edge (node4);
      \draw[communicationtransition] (node2) edge (node8);
      \draw[communicationtransition] (node3) edge (node4);
      \draw[communicationtransition] (node3) edge (node7);
      \draw[communicationtransition] (node4) edge (node5);
      \draw[communicationtransition] (node4) edge (node6);
      \draw[communicationtransition] (node5) edge (node6);
      \draw[communicationtransition] (node6) edge (node7);
      \draw[communicationtransition] (node6) edge (node8);
      \draw[communicationtransition] (node6) edge (node9);
      \draw[communicationtransition] (node6) edge (node10);
      \draw[communicationtransition] (node7) edge (node9);
      \draw[communicationtransition] (node7) edge (node10);
      \draw[communicationtransition] (node8) edge (node9);
      \draw[communicationtransition] (node8) edge (node10);
      \draw[communicationtransition] (node9) edge (node10);
      \draw[communicationtransition, orange] (node4) edge (node6);
      \draw[communicationtransition, orange] (node4) edge (node5);
      \draw[communicationtransition, orange] (node5) edge (node6);
      \draw[communicationtransition, orange] (node0) edge (node4);
      \node at (2.5, 2.5) {\ok};
      \node at (1.5, 1.5) {\ok};
      \node at (3.5, 1.5) {\ok};
      \node at (5.5, 1.5) {\ok};
      \node at (4.5, 2.5) {\ok};
      \node at (4.5, 0.5) {\ok};
      \node at (6.5, 0.5) {\ok};
      \node at (6.5, 2.5) {\ok};
      \node at (1.5, 1.9) {\imgDrone};
      \node at (5.5, 1.9) {\imgDrone};
      \node at (3.5, 1.9) {\imgDrone};
      \end{tikzpicture}
      %
      
      \hspace{-1.3cm}
      \begin{tikzpicture}[scale=0.5]
      \tikzstyle{node} = [draw, circle, fill=black, inner sep=0.5mm];\tikzstyle{basenode} = [draw, circle, red, fill=red, inner sep=0.5mm];
      \tikzstyle{movetransition} = [draw, gray!50!black];
      \tikzstyle{communicationtransition} = [communication, bend right=20, tobehere];
      \node[basenode] (node0) at (0.5, 2.5){};
      \node[node] (node1) at (0.5, 0.5){};
      \node[node] (node2) at (2.5, 0.5){};
      \node[node] (node3) at (2.5, 2.5){};
      \node[node] (node4) at (1.5, 1.5){};
      \node[node] (node5) at (3.5, 1.5){};
      \draw[use as bounding box] (node5);
      \node[node] (node6) at (5.5, 1.5){};
      \node[node] (node7) at (4.5, 2.5){};
      \node[node] (node8) at (4.5, 0.5){};
      \node[node] (node9) at (6.5, 0.5){};
      \node[node] (node10) at (6.5, 2.5){};
      \draw[movetransition] (node0) edge (node0);
      \draw[movetransition] (node0) edge (node1);
      \draw[movetransition] (node0) edge (node3);
      \draw[movetransition] (node0) edge (node4);
      \draw[movetransition] (node1) edge (node0);
      \draw[movetransition] (node1) edge (node1);
      \draw[movetransition] (node1) edge (node2);
      \draw[movetransition] (node1) edge (node4);
      \draw[movetransition] (node2) edge (node1);
      \draw[movetransition] (node2) edge (node2);
      \draw[movetransition] (node2) edge (node4);
      \draw[movetransition] (node3) edge (node0);
      \draw[movetransition] (node3) edge (node3);
      \draw[movetransition] (node3) edge (node4);
      \draw[movetransition] (node4) edge (node0);
      \draw[movetransition] (node4) edge (node1);
      \draw[movetransition] (node4) edge (node2);
      \draw[movetransition] (node4) edge (node3);
      \draw[movetransition] (node4) edge (node4);
      \draw[movetransition] (node4) edge (node5);
      \draw[movetransition] (node5) edge (node4);
      \draw[movetransition] (node5) edge (node5);
      \draw[movetransition] (node5) edge (node6);
      \draw[movetransition] (node6) edge (node5);
      \draw[movetransition] (node6) edge (node6);
      \draw[movetransition] (node6) edge (node7);
      \draw[movetransition] (node6) edge (node8);
      \draw[movetransition] (node6) edge (node9);
      \draw[movetransition] (node6) edge (node10);
      \draw[movetransition] (node7) edge (node6);
      \draw[movetransition] (node7) edge (node7);
      \draw[movetransition] (node7) edge (node10);
      \draw[movetransition] (node8) edge (node6);
      \draw[movetransition] (node8) edge (node8);
      \draw[movetransition] (node8) edge (node9);
      \draw[movetransition] (node9) edge (node6);
      \draw[movetransition] (node9) edge (node8);
      \draw[movetransition] (node9) edge (node9);
      \draw[movetransition] (node9) edge (node10);
      \draw[movetransition] (node10) edge (node6);
      \draw[movetransition] (node10) edge (node7);
      \draw[movetransition] (node10) edge (node9);
      \draw[movetransition] (node10) edge (node10);
      \draw[communicationtransition] (node0) edge (node1);
      \draw[communicationtransition] (node0) edge (node2);
      \draw[communicationtransition] (node0) edge (node3);
      \draw[communicationtransition] (node0) edge (node4);
      \draw[communicationtransition] (node1) edge (node2);
      \draw[communicationtransition] (node1) edge (node3);
      \draw[communicationtransition] (node1) edge (node4);
      \draw[communicationtransition] (node2) edge (node4);
      \draw[communicationtransition] (node2) edge (node8);
      \draw[communicationtransition] (node3) edge (node4);
      \draw[communicationtransition] (node3) edge (node7);
      \draw[communicationtransition] (node4) edge (node5);
      \draw[communicationtransition] (node4) edge (node6);
      \draw[communicationtransition] (node5) edge (node6);
      \draw[communicationtransition] (node6) edge (node7);
      \draw[communicationtransition] (node6) edge (node8);
      \draw[communicationtransition] (node6) edge (node9);
      \draw[communicationtransition] (node6) edge (node10);
      \draw[communicationtransition] (node7) edge (node9);
      \draw[communicationtransition] (node7) edge (node10);
      \draw[communicationtransition] (node8) edge (node9);
      \draw[communicationtransition] (node8) edge (node10);
      \draw[communicationtransition] (node9) edge (node10);
      \draw[communicationtransition, orange] (node2) edge (node4);
      \draw[communicationtransition, orange] (node4) edge (node5);
      \draw[communicationtransition, orange] (node0) edge (node2);
      \draw[communicationtransition, orange] (node0) edge (node4);
      \node at (2.5, 2.5) {\ok};
      \node at (1.5, 1.5) {\ok};
      \node at (3.5, 1.5) {\ok};
      \node at (5.5, 1.5) {\ok};
      \node at (4.5, 2.5) {\ok};
      \node at (4.5, 0.5) {\ok};
      \node at (6.5, 0.5) {\ok};
      \node at (6.5, 2.5) {\ok};
      \node at (2.5, 0.9) {\imgDrone};
      \node at (3.5, 1.9) {\imgDrone};
      \node at (1.5, 1.9) {\imgDrone};
      \end{tikzpicture}
      \vspace{0.4cm}
      %
      \hspace{2.3cm}
      \begin{tikzpicture}[scale=0.5]
      \tikzstyle{node} = [draw, circle, fill=black, inner sep=0.5mm];\tikzstyle{basenode} = [draw, circle, red, fill=red, inner sep=0.5mm];
      \tikzstyle{movetransition} = [draw, gray!50!black];
      \tikzstyle{communicationtransition} = [communication, bend right=20, tobehere];
      \node[basenode] (node0) at (0.5, 2.5){};
      \node[node] (node1) at (0.5, 0.5){};
      \node[node] (node2) at (2.5, 0.5){};
      \node[node] (node3) at (2.5, 2.5){};
      \node[node] (node4) at (1.5, 1.5){};
      \node[node] (node5) at (3.5, 1.5){};
      \draw[use as bounding box] (node5);
      \node[node] (node6) at (5.5, 1.5){};
      \node[node] (node7) at (4.5, 2.5){};
      \node[node] (node8) at (4.5, 0.5){};
      \node[node] (node9) at (6.5, 0.5){};
      \node[node] (node10) at (6.5, 2.5){};
      \draw[movetransition] (node0) edge (node0);
      \draw[movetransition] (node0) edge (node1);
      \draw[movetransition] (node0) edge (node3);
      \draw[movetransition] (node0) edge (node4);
      \draw[movetransition] (node1) edge (node0);
      \draw[movetransition] (node1) edge (node1);
      \draw[movetransition] (node1) edge (node2);
      \draw[movetransition] (node1) edge (node4);
      \draw[movetransition] (node2) edge (node1);
      \draw[movetransition] (node2) edge (node2);
      \draw[movetransition] (node2) edge (node4);
      \draw[movetransition] (node3) edge (node0);
      \draw[movetransition] (node3) edge (node3);
      \draw[movetransition] (node3) edge (node4);
      \draw[movetransition] (node4) edge (node0);
      \draw[movetransition] (node4) edge (node1);
      \draw[movetransition] (node4) edge (node2);
      \draw[movetransition] (node4) edge (node3);
      \draw[movetransition] (node4) edge (node4);
      \draw[movetransition] (node4) edge (node5);
      \draw[movetransition] (node5) edge (node4);
      \draw[movetransition] (node5) edge (node5);
      \draw[movetransition] (node5) edge (node6);
      \draw[movetransition] (node6) edge (node5);
      \draw[movetransition] (node6) edge (node6);
      \draw[movetransition] (node6) edge (node7);
      \draw[movetransition] (node6) edge (node8);
      \draw[movetransition] (node6) edge (node9);
      \draw[movetransition] (node6) edge (node10);
      \draw[movetransition] (node7) edge (node6);
      \draw[movetransition] (node7) edge (node7);
      \draw[movetransition] (node7) edge (node10);
      \draw[movetransition] (node8) edge (node6);
      \draw[movetransition] (node8) edge (node8);
      \draw[movetransition] (node8) edge (node9);
      \draw[movetransition] (node9) edge (node6);
      \draw[movetransition] (node9) edge (node8);
      \draw[movetransition] (node9) edge (node9);
      \draw[movetransition] (node9) edge (node10);
      \draw[movetransition] (node10) edge (node6);
      \draw[movetransition] (node10) edge (node7);
      \draw[movetransition] (node10) edge (node9);
      \draw[movetransition] (node10) edge (node10);
      \draw[communicationtransition] (node0) edge (node1);
      \draw[communicationtransition] (node0) edge (node2);
      \draw[communicationtransition] (node0) edge (node3);
      \draw[communicationtransition] (node0) edge (node4);
      \draw[communicationtransition] (node1) edge (node2);
      \draw[communicationtransition] (node1) edge (node3);
      \draw[communicationtransition] (node1) edge (node4);
      \draw[communicationtransition] (node2) edge (node4);
      \draw[communicationtransition] (node2) edge (node8);
      \draw[communicationtransition] (node3) edge (node4);
      \draw[communicationtransition] (node3) edge (node7);
      \draw[communicationtransition] (node4) edge (node5);
      \draw[communicationtransition] (node4) edge (node6);
      \draw[communicationtransition] (node5) edge (node6);
      \draw[communicationtransition] (node6) edge (node7);
      \draw[communicationtransition] (node6) edge (node8);
      \draw[communicationtransition] (node6) edge (node9);
      \draw[communicationtransition] (node6) edge (node10);
      \draw[communicationtransition] (node7) edge (node9);
      \draw[communicationtransition] (node7) edge (node10);
      \draw[communicationtransition] (node8) edge (node9);
      \draw[communicationtransition] (node8) edge (node10);
      \draw[communicationtransition] (node9) edge (node10);
      \draw[communicationtransition, orange] (node1) edge (node4);
      \draw[communicationtransition, orange] (node1) edge (node3);
      \draw[communicationtransition, orange] (node3) edge (node4);
      \draw[communicationtransition, orange] (node0) edge (node1);
      \draw[communicationtransition, orange] (node0) edge (node4);
      \draw[communicationtransition, orange] (node0) edge (node3);
      \node at (2.5, 0.5) {\ok};
      \node at (2.5, 2.5) {\ok};
      \node at (1.5, 1.5) {\ok};
      \node at (3.5, 1.5) {\ok};
      \node at (5.5, 1.5) {\ok};
      \node at (4.5, 2.5) {\ok};
      \node at (4.5, 0.5) {\ok};
      \node at (6.5, 0.5) {\ok};
      \node at (6.5, 2.5) {\ok};
      \node at (0.5, 0.9) {\imgDrone};
      \node at (1.5, 1.9) {\imgDrone};
      \node at (2.5, 2.9) {\imgDrone};
      \end{tikzpicture}
      %
      
      \hspace{-5.5cm}
      \begin{tikzpicture}[scale=0.5]
      \tikzstyle{node} = [draw, circle, fill=black, inner sep=0.5mm];\tikzstyle{basenode} = [draw, circle, red, fill=red, inner sep=0.5mm];
      \tikzstyle{movetransition} = [draw, gray!50!black];
      \tikzstyle{communicationtransition} = [communication, bend right=20, tobehere];
      \node[basenode] (node0) at (0.5, 2.5){};
      \node[node] (node1) at (0.5, 0.5){};
      \node[node] (node2) at (2.5, 0.5){};
      \node[node] (node3) at (2.5, 2.5){};
      \node[node] (node4) at (1.5, 1.5){};
      \node[node] (node5) at (3.5, 1.5){};
      \draw[use as bounding box] (node5);
      \node[node] (node6) at (5.5, 1.5){};
      \node[node] (node7) at (4.5, 2.5){};
      \node[node] (node8) at (4.5, 0.5){};
      \node[node] (node9) at (6.5, 0.5){};
      \node[node] (node10) at (6.5, 2.5){};
      \draw[movetransition] (node0) edge (node0);
      \draw[movetransition] (node0) edge (node1);
      \draw[movetransition] (node0) edge (node3);
      \draw[movetransition] (node0) edge (node4);
      \draw[movetransition] (node1) edge (node0);
      \draw[movetransition] (node1) edge (node1);
      \draw[movetransition] (node1) edge (node2);
      \draw[movetransition] (node1) edge (node4);
      \draw[movetransition] (node2) edge (node1);
      \draw[movetransition] (node2) edge (node2);
      \draw[movetransition] (node2) edge (node4);
      \draw[movetransition] (node3) edge (node0);
      \draw[movetransition] (node3) edge (node3);
      \draw[movetransition] (node3) edge (node4);
      \draw[movetransition] (node4) edge (node0);
      \draw[movetransition] (node4) edge (node1);
      \draw[movetransition] (node4) edge (node2);
      \draw[movetransition] (node4) edge (node3);
      \draw[movetransition] (node4) edge (node4);
      \draw[movetransition] (node4) edge (node5);
      \draw[movetransition] (node5) edge (node4);
      \draw[movetransition] (node5) edge (node5);
      \draw[movetransition] (node5) edge (node6);
      \draw[movetransition] (node6) edge (node5);
      \draw[movetransition] (node6) edge (node6);
      \draw[movetransition] (node6) edge (node7);
      \draw[movetransition] (node6) edge (node8);
      \draw[movetransition] (node6) edge (node9);
      \draw[movetransition] (node6) edge (node10);
      \draw[movetransition] (node7) edge (node6);
      \draw[movetransition] (node7) edge (node7);
      \draw[movetransition] (node7) edge (node10);
      \draw[movetransition] (node8) edge (node6);
      \draw[movetransition] (node8) edge (node8);
      \draw[movetransition] (node8) edge (node9);
      \draw[movetransition] (node9) edge (node6);
      \draw[movetransition] (node9) edge (node8);
      \draw[movetransition] (node9) edge (node9);
      \draw[movetransition] (node9) edge (node10);
      \draw[movetransition] (node10) edge (node6);
      \draw[movetransition] (node10) edge (node7);
      \draw[movetransition] (node10) edge (node9);
      \draw[movetransition] (node10) edge (node10);
      \draw[communicationtransition] (node0) edge (node1);
      \draw[communicationtransition] (node0) edge (node2);
      \draw[communicationtransition] (node0) edge (node3);
      \draw[communicationtransition] (node0) edge (node4);
      \draw[communicationtransition] (node1) edge (node2);
      \draw[communicationtransition] (node1) edge (node3);
      \draw[communicationtransition] (node1) edge (node4);
      \draw[communicationtransition] (node2) edge (node4);
      \draw[communicationtransition] (node2) edge (node8);
      \draw[communicationtransition] (node3) edge (node4);
      \draw[communicationtransition] (node3) edge (node7);
      \draw[communicationtransition] (node4) edge (node5);
      \draw[communicationtransition] (node4) edge (node6);
      \draw[communicationtransition] (node5) edge (node6);
      \draw[communicationtransition] (node6) edge (node7);
      \draw[communicationtransition] (node6) edge (node8);
      \draw[communicationtransition] (node6) edge (node9);
      \draw[communicationtransition] (node6) edge (node10);
      \draw[communicationtransition] (node7) edge (node9);
      \draw[communicationtransition] (node7) edge (node10);
      \draw[communicationtransition] (node8) edge (node9);
      \draw[communicationtransition] (node8) edge (node10);
      \draw[communicationtransition] (node9) edge (node10);
      \node at (0.5, 0.5) {\ok};
      \node at (2.5, 0.5) {\ok};
      \node at (2.5, 2.5) {\ok};
      \node at (1.5, 1.5) {\ok};
      \node at (3.5, 1.5) {\ok};
      \node at (5.5, 1.5) {\ok};
      \node at (4.5, 2.5) {\ok};
      \node at (4.5, 0.5) {\ok};
      \node at (6.5, 0.5) {\ok};
      \node at (6.5, 2.5) {\ok};
      \node at (0.5, 2.9) {\imgDrone};
    %   \node at (0.46, 2.86) {\imgDrone};
    %   \node at (0.42, 2.82) {\imgDrone};
  \end{tikzpicture}
  
  \vspace{-1.35cm}
  \hspace{3.9cm}		
  % \includegraphics[scale=0.165]{example_example_resized.png}
  \raisebox{1cm}{
    \fbox{
      \tiny
      \begin{tabular}{ll}
        \imgDrone & UAVs \\
        \tikz{\draw (0, 0) edge (0.4, 0);} & Movement \\
        \tikz{\draw[communication, tobehere] (0, 0) edge (0.4, 0);} & Communication \\
        \tikz{\draw[communication, orange] (0, 0) edge (0.4, 0);} & Active communications \\
        \ok & Visited regions
      \end{tabular}
      }
  }
  \end{center}
  \vspace{-3mm}
  \caption{Example of a mission execution
  %{\color{red}Ocan: Can we make a smaller example?}
}
  \label{figure:exampleexecution}
\end{figure}

%%% Local Variables:
%%% mode: latex
%%% TeX-master: "main"
%%% End:


In this work, we consider all agents to be anonymous in both cases (reachability
and coverage). Furthermore, we consider the collisions to be handled by the
agents themselves, hence are not considered along the results of this paper. We
depicted a covering execution of a topological graph by 3 UAVs in Figure
\ref{figure:exampleexecution}. In this example, the UAVs need to gather
information at each node of the graph while staying connected to the base (red
node) during the whole mission.

%The rest of the paper is structured as follows. 
In Section
\ref{sec:preli}, we present the typical notions used in MAPP and their extension
for our case and the known results in connected planning. In Sections
\ref{sec:dir} to \ref{sec:compl}, we study the complexity of our problems from the general case
to the most restrictive one. We describe the related works in Section \ref{sec:relate}.
%Finally, 
We conclude in Section \ref{sec:concl}.

\iffull
\else
\mbox{The full version of the paper\footnotemark[1] is available.}
\fi

%%% Local Variables:
%%% mode: latex
%%% TeX-master: "main"
%%% End:
