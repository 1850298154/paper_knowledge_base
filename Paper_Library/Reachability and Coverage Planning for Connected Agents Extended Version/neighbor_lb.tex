%The \pbReach{} on the neighbor-communicable topological graphs was proven
%to be PSPACE-complete in \cite{dblp:conf/aaai/tateobrab18}. 
%\todo{Ocan: Removed ref to Tateo}
%We study the \pbCoverage{}problem for this subclass.

\begin{theorem} 
  \label{thm:nc:lb:cover}
  \pbCoverageNC is PSPACE-complete.
\end{theorem}

\begin{proof}
  The upper bound is given by Theorem~\ref{prop:dir:ub:cover-reach}.

  For the lower bound on \pbCoverageNC, the reduction given in
  Figure~\ref{figure:topologicgraphproblemcoverage} is not adapted 
  for neighbor communicable graphs.
  Indeed, all nodes may be visited  although $c_1, \dots, c_k$
  is not reached: $v_1$ and $v_k$ can be reached by two lines of agents
  connected to the base, making the coverage of the full graph possible.
  We nevertheless give a similar reduction by adapting the previous reduction.
  \iffull

  \begin{figure}[t]
        \begin{center}
            \newcommand{\xretourgauche}{-1}
            \tikzstyle{circlenode} =  [circle, draw=black, fill=black, inner sep=0.8mm]
            \newcommand{\ligne}[1]{
            \node[circlenode]  (s1) at (0,#1)  {};
            \node[circlenode]  (s1) at (0.3,#1)  {};
            \node[circlenode]  (s1) at (1,#1)  {};
            \node[circlenode] (s2) at (3,#1)   {};
            \node[circlenode]  (sk) at (7,#1)  {};
            \draw[-stealth] (1, #1-0.5) -- (s1);
            \draw[-stealth] (3, #1-0.5) -- (s2);
            \draw[-stealth] (7, #1-0.5) -- (sk);
            \draw[tobehere,communication] (s1) -- (s2);
            \draw[tobehere,communication] (s2) -- (4.5,#1);
            \draw[dotted,color=green] (4.5,#1) -- (5.5,#1);
            \draw[tobehere,communication] (sk) -- (5.5,#1);}

            \newcommand{\lignevdots}[1]{
            \node[]  (s1) at (1,#1)  {$\vdots$};
            \node[] (s2) at (3,#1)   {$\vdots$};
            \node[]  (sk) at (7,#1)  {$\vdots$};
            \draw[-stealth] (1, #1-0.7) -- (s1);
            \draw[-stealth] (3, #1-0.7) -- (s2);
            \draw[-stealth] (7, #1-0.7) -- (sk);
            }

            \newcommand{\vnodeY}{8.7}
            \newcommand{\posBaseY}{2.75}
            \scalebox{1}
            {
                \begin{tikzpicture}[xscale=0.75,yscale=0.9]
                \draw[copyofG] (-0.2,\posBaseY-0.3) rectangle (8.2,4);
%                \draw (4,\posBaseY	) node[draw] (nb) {B};
                \draw (4,\posBaseY) node (nb) {B};
                \draw (1,3.75) node (t1) {$c_1$};
                \draw (3,3.75) node (t2) {$c_2$};
                \draw (7,3.75) node (tk) {$c_k$};

                \draw[tobehere,communication] (t1) -- (t2);
                \draw[tobehere,communication] (t2) -- (4.5,3.75);
                \draw[dotted,color=green] (4.5,3.75) -- (5.5,3.75);
                \draw[tobehere,communication] (tk) -- (5.5,3.75);


                \ligne {4.5}
                \addCommunicationB {s1}
                \lignevdots {5.25}
                \ligne {5.9}
                \addCommunicationB {s1}

                \node at (7.5, 5.8) {$s_k$};

                \draw[decoration={brace,mirror,raise=5pt},decorate]
                (8,4.3) -- node[right=10pt] {$k+1$} (8, 6);



                \ligne {6.5}
                \addCommunicationB {sk}
                %			\ligne {10.5}
                %			\addCommunicationB {sk}
                \lignevdots {7.25}
                \ligne {7.9}
                \addCommunicationB {sk}


                \draw[decoration={brace,mirror,raise=5pt},decorate]
                (8,6.3) -- node[right=10pt] {$k+1$} (8,8);



                \draw (1,\vnodeY) node (v1) {$v_1$};
                \draw (3,\vnodeY) node (v2) {$v_2$};
                \draw (7,\vnodeY) node (vk) {$v_k$};
                \draw (8,\vnodeY+0.5) node (all) {all};
                \draw[-stealth] (s1) -- (v1);
                \draw[-stealth] (s2) -- (v2);
                \draw[-stealth] (sk) -- (vk);
                \draw[-stealth, rounded corners = 8pt] (v1) -- (0.3, \vnodeY-0.3) -- (0.3,4);
                \draw[stealth-, rounded corners = 8pt] (v1) -- (0, \vnodeY) -- (0,4);

                \path (vk) edge [loop right] (vk);
                \draw[-stealth] (nb) to [loop right] (nb);
                \draw [-stealth, rounded corners = 8pt] (vk) -- (7,\vnodeY+0.5) -- (\xretourgauche,\vnodeY+0.5) -- (\xretourgauche, \posBaseY)  -- (nb);
                \draw (v1) -- (1,\vnodeY+0.5);
                \draw (v2) -- (3,\vnodeY+0.5);

                \draw[tobehere,communication] (vk) -- (all);
                \node at (4, 3.25) {copy of $G$};
                \end{tikzpicture}
            }
            \caption{Topological graph of the \pbCoverageNC-instance constructed from the \pbReachNC-instance for the case of neighbor-communicable topological graph.}
            \label{figure:topologicgraphproblemcoveragefixedforneighborcommunicable}
        \end{center}
    \end{figure}
  
  
	The corrected construction is given in Figure
	\ref{figure:topologicgraphproblemcoveragefixedforneighborcommunicable}. When
	configuration $\langle c_1,...,c_k\rangle$ is reached, the agents go through
	a first layer of length $k+1$ in which the first agent can communicate with
	$B$. Then they go through another layer of length $k+1$ in which the
	$k^{th}$ agent can communicate with $B$. This way, it is mandatory that all
	agents move at the same time to visit $\langle v_1,\dots,v_k\rangle$. Once
	the $k^{th}$ agent is at $v_k$, all agents can communicate with $B$ wherever
	they are, so they can visit remaining states in the copy of $G$. Now let us
	prove that $\langle c_1, \dots, c_k\rangle$ is reachable in $G$ iff it is
	possible to cover all nodes in $G'$.
	
	
	($\Rightarrow$) If $\langle c_1, \dots, c_k\rangle$ is reachable in $G$,
	then we extend the execution to reach $\langle v_1, \dots, v_k\rangle)$ and
	by the same trick as in Figure \ref{figure:topologicgraphproblemcoverage},
	the agent that reaches $v_1$ visits all the remaining unvisited nodes in
	$G$. Thus, we extend the execution for covering all nodes in $G'$.
	
	$(\Leftarrow)$ Suppose all nodes are visited in $G'$. In particular, $v_1$
	and $v_k$ are visited. Let us consider the first moment $t_{v_i}$ when a
	node $v_i$ is visited.
	
	\begin{fact} \label{fact:nodesv1vk}
		At the first moment, the configuration of the agents is $\langle v_1,
		\dots, v_k\rangle$.
	\end{fact}
	
	\begin{proof}
		Let us prove that there is an agent at $v_k$. Suppose that at that
		moment there is no agent at $v_k$. Due to the topological graph $G'$, the
		agent at $v_i$ is disconnected from the base since  nodes that
		communicate directly to $B$ are too far from $v_i$: indeed, the top
		$k+1$-grid is too long and, for $i=1$, the path on left between $v_1$
		and the copy of $G$ is too long. Contradiction.
		
		The agent at $v_k$ came from the unique $2k+2$-long path from $c_k$ to
		$v_k$. Actually, $k+1$ steps before - let us call this moment $t_{s_k}$,
		she was on $s_k$. But at that time, due to the topological graph, there
		are $k$ agents on the row containing $s_k$, otherwise the agent at $s_k$
		would have been disconnected from the base (the bottom $k+1$-grid is too
		long).
		
		So $k+1$ times later $t_{s_k}$, all the $k$ agents are at $\langle v_1,
		\dots, v_k\rangle$.
	\end{proof}
	
	Taking Fact \ref{fact:nodesv1vk} as granted, we consider time $t$ that is
	$2k+2$ steps before and we clearly have the following fact.
	
	\begin{fact}
		At time $t$, the configuration is $\langle c_1, \dots, c_k\rangle$.
	\end{fact}

	Moreover, the following fact holds.
	
	\begin{fact}
		At time $t$, no node outside $G$ were visited.
	\end{fact}
	
	\begin{proof}
		By contradiction, if some node outside $G$ were visited, it means that
		some agent went out the copy of $G$. By definition of $G'$, it would
		mean that a node $v_i$ would have been visited, before time $t$, hence
		strictly before $t_{v_i}$. Contradiction.
	\end{proof}

	To sum up, the prefix of the execution from $\langle B, \dots, B\rangle$ to
	$\langle c_1, \dots, c_k\rangle$ is fully inside the copy of $G$. So
  $\langle c_1, \dots, c_k\rangle$ is reachable in $G$.
  
  \else

  The details are given in the long version.

  \fi

\end{proof}

%By Proposition~\ref{prop:nc:lb:cover} and


% \begin{theorem}
% \label{th:nc:compl:cover}
% \pbCoverageNC is PSPACE-complete.
% \end{theorem}

%%% Local Variables:
%%% mode: latex
%%% TeX-master: "main"
%%% End:
