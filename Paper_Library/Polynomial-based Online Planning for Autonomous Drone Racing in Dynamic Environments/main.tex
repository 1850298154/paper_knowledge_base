% \pdfminorversion=4
\documentclass[letterpaper, 10 pt, conference]{support/ieeeconf}
\usepackage{mathtools}
\usepackage{times}
\usepackage[pdftex]{graphicx}
\usepackage{subfigure}
\usepackage{amsmath,amssymb,amsopn,amstext,amsfonts}
\usepackage{cancel}
\usepackage[space]{cite}
\usepackage{pdfsync}
\usepackage{balance}
\usepackage{color}
\usepackage[ruled,vlined]{algorithm2e}
\usepackage{bm}
\newtheorem{theorem}{Theorem}
\usepackage{diagbox}
\usepackage{float}
\usepackage{epstopdf}
\usepackage{pifont}
\usepackage{fixltx2e}
\usepackage{amsmath}
\usepackage{multirow}
\usepackage{booktabs}
\usepackage{url}
\usepackage{svg}
\usepackage{bm}
\usepackage{threeparttable}
\usepackage[linkcolor=black,citecolor=black,urlcolor=black,colorlinks=true]{hyperref}
%\epstopdfsetup{update}
\usepackage{siunitx,array}
\usepackage{lipsum}
\usepackage{makecell}
\usepackage{caption}
\usepackage{mathrsfs}


\bibliographystyle{support/IEEEtran}

\newcommand{\todo}[1]{\textcolor{red}{\emph{\bf#1}}}
\newcommand{\transpose}{\mbox{${}^{\text{T}}$}}
\newcommand{\eye}[1][]{\ensuremath{\mathbb{I}_{#1}}}
\DeclareMathOperator*{\argmax}{arg\,max}
\DeclareMathOperator*{\argmin}{arg\,min}
\DeclareMathOperator*{\trace}{Tr}
\newcommand{\tp}{^{\mathrm{T}}}
\newcommand{\invtp}{^{-\mathrm{T}}}
\newcommand{\grad}{\nabla}
\newcommand{\hessian}{\nabla^2}
\newcommand{\df}[1]{\mathrm{d}{#1}}
\newcommand{\rbrac}[1]{({#1})}
\newcommand{\rBrac}[1]{\left({#1}\right)}
\newcommand{\RBrac}[1]{\Big({#1}\Big)}
\newcommand{\cbrac}[1]{\{{#1}\}}
\newcommand{\cBrac}[1]{\left\{{#1}\right\}}
\newcommand{\sbrac}[1]{[{#1}]}
\newcommand{\sBrac}[1]{\left[{#1}\right]}
\newcommand{\norm}[1]{\Vert{#1}\Vert}
\newcommand{\Norm}[1]{\left\Vert{#1}\right\Vert}
\newcommand{\abs}[1]{\vert{#1}\vert}
\newcommand{\Abs}[1]{\left\vert{#1}\right\vert}
\newcommand{\ceil}[1]{\lceil{#1}\rceil}
\newcommand{\Ceil}[1]{\left\lceil{#1}\right\rceil}
\newcommand{\up}[2]{\overset{#2}{#1}}
\newcommand{\ld}[2]{{}_{#2}{#1}}
\newcommand{\lu}[2]{{}^{#2}{#1}}
\newcommand{\down}[2]{\underset{#2}{#1}}

\newcommand{\tabincell}[2]{\begin{tabular}{@{}#1@{}}#2\end{tabular}}

\makeatletter
\newcommand{\setword}[2]{%
  \phantomsection
  #1\def\@currentlabel{\unexpanded{#1}}\label{#2}%
}
\makeatother

\graphicspath{{./figures/}}
\DeclareGraphicsExtensions{.png,.jpg,.eps,.pdf}
%\DeclareGraphicsExtensions{.pdf}
\IEEEoverridecommandlockouts
\overrideIEEEmargins

\makeatletter
\def\endthebibliography{%
	\def\@noitemerr{\@latex@warning{Empty `thebibliography' environment}}%
	\endlist
}

%\title{\textbf{A Linear and Exact Algorithm for Whole-Body Collision Evaluation via Scale Optimization}}
\title{\textbf{Polynomial-based Online Planning for\\
		Autonomous Drone Racing in Dynamic Environments}}
\author{
        Qianhao Wang $^{\dag}$, 
        Dong Wang $^{\dag}$, 
        Chao Xu,
        Alan Gao,
        and Fei Gao
	    \thanks{\textbf{${\dag}$ Equal contribution.}}        
        \thanks{All authors are with the College of Control Science and Engineering, Zhejiang University, Hangzhou, 310027, China, and also with the Huzhou Institute of Zhejiang University, Huzhou, 313000, China. }
	    \thanks{Email:{\tt\small \{qhwangaa, fgaoaa\}@zju.edu.cn}} 
	    \thanks{Corresponding Author: Fei Gao.}}  
 %        \thanks{1 State Key Laboratory of Industrial Control Technology, Institute of Cyber-Systems and Control, Zhejiang University, Hangzhou, 310027, China.} 
	% \thanks{2 Huzhou Institute of Zhejiang University, Huzhou, 313000, China.}
	% \thanks{Email:{\tt\small \{qhwangaa, fgaoaa\}@zju.edu.cn}}}
	  % TODO
 

\begin{document}

\maketitle
\thispagestyle{empty}
\pagestyle{empty}

\begin{abstract}
%	In recent years, significant progress has been made in autonomous drone racing. 
%	However, most of the focus has been on achieving fast execution times, with little attention paid to dynamic environments. 
%	Meanwhile, unpredictable environmental changes and the high-speed racing scenarios place strict requirements on online replanning and its timeliness.
    In recent years, there is a noteworthy advancement in autonomous drone racing. However, the primary focus is on attaining execution times, while scant attention is given to the challenges of dynamic environments. The high-speed nature of racing scenarios, coupled with the potential for unforeseeable environmental alterations, present stringent requirements for online replanning and its timeliness.
    For racing in dynamic environments, we propose an online replanning framework with an efficient polynomial trajectory representation.
    We trade off between aggressive speed and flexible obstacle avoidance based on an optimization approach.
    Additionally, to ensure safety and precision when crossing intermediate racing waypoints, we formulate the demand as hard constraints during planning.
    For dynamic obstacles, parallel multi-topology trajectory planning is designed based on engineering considerations to prevent racing time loss due to local optimums.
    The framework is integrated into a quadrotor system and successfully demonstrated at the DJI Robomaster Intelligent UAV Championship, where it successfully complete the racing track and placed first,  finishing in less than half the time of the second-place\footnote{\label{foot_rankings}https://pro-robomasters-hz-n5i3.oss-cn-hangzhou.aliyuncs.com/sass/event-list.html}.
\end{abstract}


\section{Introduction}
\label{sec:Introduction}
Quadrotors are gaining popularity in various industrial and commercial scenarios due to their versatility and exceptional performance.
In recent years, autonomous drone racing, a research field focusing on planning trajectories for quadrotors to follow an aggressive reference routine while precisely crossing some intermediate landmarks, receives considerable attentions~\cite{romero2022time,han2021fast,foehn2021time,hanover2023autonomous} and sparkes an international competition craze, such as the AlphaPilot Challenge \cite{foehn2022alphapilot,guerra2019flightgoggles} and the Autonomous Drone Race~\cite{moon2019challenges,cocoma2019towards} in IEEE IROS.
The ultimate pursuit of minimizing the execution time in drone racing increasingly ignites the fire for quadrotors to be applied in several emergencies, such as post-disaster communications and urgent transportation of essential supplies.

% In such scenarios, it is inevitable that dynamic obstacles or moving landmarks that require traversal will arise due to the environmental changes. 
% How to plan a minimum-time trajectory through a series of waypoints in the dynamic environments remains an intractable problem that can not be completely solved by previous works.
% Despite existing works~\cite{foehn2021time,romero2022time} demonstrate the ability to compete with skilled human pilots in racing, minite-level computational demands make them difficult to respond to real-time changes.
% % they do not take into account the real-time changes. 
% % In addition, not all the variations of the surroundings can be preemptively observed and accurately predicted, these unforeseeable circumstances necessitate online replanning. 
% % Especially, during the high-speed drone racing, ensuring the timeliness of the trajectories presents demanding requirement on the efficiency of replan.
% In addition, unforeseeable circumstances in dynamic environments require online replanning, as not all variations in the surroundings can be preemptively observed and accurately predicted. 
% Particularly, during high-speed drone racing, the efficiency of replan is crucial for ensuring the timeliness of trajectories. 
% But most online works~\cite{wang2021autonomous,chen2022rast} that consider dynamic environments are far from achieving the shortest execution time as they cannot fully exploit the potential of quadrotor's actuator. 
% In detail, this problem requires that the planning method satisfies the following conditions simultaneously: 
% \textbf{(1)}~shortest possible time of trajectory replan;
% \textbf{(2)}~fastest possible speed of completing the track;
% \textbf{(3)}~safe traversal of the waypoints, whether stationary or in motion;
% \textbf{(4)}~agile avoidance of the dynamic obstacles.

In such scenarios, dynamic obstacles or moving landmarks that require investigation and traversal will inevitably arise due to environmental changes.
How to plan a minimum-time trajectory through a series of waypoints in dynamic environments remains a challenging problem that can not be completely solved by previous works.
In detail, this problem requires that the planning method satisfies the conditions simultaneously: 
\setword{\textbf{(1)}}{requirement1}~highest possible speed of completing the track;
\setword{\textbf{(2)}}{requirement2}~agile avoidance of the dynamic obstacles;
\setword{\textbf{(3)}}{requirement3}~shortest possible computational time of trajectory replan;
\setword{\textbf{(4)}}{requirement4}~precise traversal of the waypoints.
% \textbf{(1)}~highest possible speed of completing the track;
% \textbf{(2)}~agile avoidance of the dynamic obstacles;
% \textbf{(3)}~shortest possible computational time of trajectory replan;
% \textbf{(4)}~precise traversal of the waypoints.


\begin{figure}[!t]
	\centering
    \vspace{0.2cm}
	\includegraphics[width=1\linewidth]{figures/toutu.pdf}
    \captionsetup{font={footnotesize}}
	\caption{
        The snapshot of our online planning method applied in a challenging dynamic environment, where the quadrotor is required to precisely
        pass through two dynamic gates in order.
        In \textbf{(a)}, the quadrotor passes through the first gate and flies towards the second one.
        \textbf{(b)} depicts a close-up of crossing the second gate.
	}
	\label{fig:toutu}
    \vspace{-1.0cm}
\end{figure}

The first two points \ref{requirement1} and \ref{requirement2} are intuitive for safe racing in dynamic environments.
As for \ref{requirement3}, since not all variations in the dynamic surroundings can be preemptively observed and accurately predicted, these unforeseeable circumstances require online replanning.
Particularly during high-speed drone racing, the efficiency of replanning is crucial for ensuring the timeliness of trajectories. 
Although existing works~\cite{foehn2021time,romero2022time} demonstrate the ability to compete with skilled human pilots in racing, minute-level computational demands make them prohibitive to respond to unforeseeable changes.
For \ref{requirement4}, as shown in Fig.~\ref{fig:toutu}, some waypoints may have low spatial tolerance even in motion.
Inaccurate traversal can cause collision or mission failure.
However, most racing works~\cite{han2021fast,romero2022model} do not guarantee stable and precise traversal in planning, because of the use of soft constraint approach that relies on parameters.


% When it comes to online replanning in the least amount of time possible, polynomial trajectory is a widely used option~\cite{zhou2019robust,zhou2020ego,wang2022geometrically} due to its high computational efficiency. Utilizing the differential flatness property of quadrotors~\cite{mellinger2011minimum}, polynomial can constrain and calculate the full state with only the position and yaw angle trajectory. 
% % Although some works~\cite{hanover2023autonomous,foehn2021time,romero2022model} argue that polynomial trajectories are unsuitable for racing due to their inherent smoothness and the need for aggressive controller input, a smooth trajectory is more friendly to the controller, actuator and onboard perception, which is vital for the stability of flight.
% Additionally, the inherent smoothness of polynomial trajectory is more friendly to the controller, actuator and onboard perception, which is vital for the stability of flight.
% To achieves fast online replanning for racing, we employ an optimization method to trade off the smoothness and aggressiveness of the polynomial trajectory, which takes advantage of above properties of polynomial while fulfilling the need for high speed.


In order to address the aforementioned requirements, we propose a strong polynomial-based online planning framework for racing in dynamic environments by incorporating careful engineering considerations with our previous works.
We obtain the trajectory by an optimization method.
For online replanning of \ref{requirement3}, we adopt MINCO~\cite{wang2022geometrically} as the trajectory representation and improve it to a time-uniform version.
This implementation is more lightweight yet still maintains the capacity to allow for spatial-temporal deformations, improving computational efficiency. 
For rapid speed of \ref{requirement1}, we boost the aggressiveness of quadrotors by minimizing execution time in the optimization based on the temporal freedom of trajectory.
For precise waypoint traversal of \ref{requirement4}, we formulate this waypoint-through requirement into a hard constraint to ensure safety and stability, regardless of whether the waypoint is static or in motion.
For obstacle avoidance of \ref{requirement2}, we build upon our previous work~\cite{wang2021autonomous} by calculating multiple trajectories in parallel under different topologies segmented by dynamic obstacles and then selecting the optimal one.

% Concerning safe waypoint traversal, as opposed to parameter-dependent soft constraints~\cite{romero2022time,han2021fast}, we formulate this requirement into a hard constraint to ensure the safety and stability, regardless of whether the waypoint is static or moving.
% Regarding obstacle avoidance, based on the our previous work~\cite{wang2021autonomous}, we calculate multiple trajectories in parallel under different topologies, and then select the optimal one.


Finally, we integrate the proposed planning framework into a customized quadrotor system, combining state estimation, control, and real-time vision-based detection modules. 
This system was deployed at the 2022 \textbf{DJI Robomaster Intelligent UAV Championship}\footnote{\label{dji_bisai_web}https://www.robomaster.com/zh-CN/robo/drone}, where quadrotors are tasked with navigating a track with dynamic obstacles, narrow gaps that require SE(3) planning, and gates.
The gates must be traversed in a specific order as fast as possible, even though they are either in motion or have a random location within a range.
In this competition, our system succeeded in completing the racing track and placed first, which proves our method's comprehensive capabilities of performing high-speed flight in challenging dynamic environment.

To summarize, the contributions of this paper are:
\begin{itemize}
    \item We implement time-uniform MINCO by improving our previous work, which boosts computational efficiency to enhance the timeliness of replan for high-speed flight. 
    % 把这个time uniform给写出来,具体一些
    % \item To ensure safety during traversal of waypoints, both static and dynamic, we addressed the requirement by formulating it as a hard constraint.
    % \item We model the position of waypoints as trajectories about time and use them as boundary conditions for joint trajectory optimization, which achieves hard constraint on the position when crossing waypoints, both static and moving, to ensure precision of crossing.
    \item We achieve a hard constraint on the position of drone during crossing both static and moving waypoints, to ensure precision of crossing, by modeling the position of waypoints as trajectories about time and formulating them as boundary conditions for joint optimization.
    % 硬约束具体一些,但是表述上抽象一些,这样可以用在很多不同的地方
    % 用移动boundary condition
    % 我们将门建模为关于时间的轨迹,将其作为轨迹边界状态进行联合优化实现硬约束来保证穿环的安全性
    % \item We propose a replan framework by combining careful engineering considerations, such as evaluating different topologies segmented by dynamic obstacles and large attitude flight.
    \item We propose a replanning framework, combined with evaluating different topologies segmented by dynamic obstacles and large attitude flight, to deal with challenging environments.
    Ablation experiments and competition prove our method's effectiveness for racing in dynamic environments.
\end{itemize}


\section{Related Work}
\label{sec:relatedworks}
We divide racing works online and offline depending on whether the method can perform in real-time.

In general, offline methods are more comprehensive in problem construction and can serve as the baseline for online methods.
Recently, there is a remarkable work\cite{foehn2021time} of Foehn et al. to generate time-optimal trajectories for drones to outperform professional pilots in racing.
They use discrete state points to represent the trajectory and solve it using an optimization method while formulating the gate-through requirement as a complementary constraint.
Additionally, they adopt a full-state quadrotor model with the thrust of single rotor as control input and impose constraint on each rotor, which saturates the actuator to achieve optimal time.
% Compared to simple model\cite{van2013time}, this method saturates the actuator to achieve optimal time.
Different from traditional racing, Han et al.\cite{han2021fast} focus on planning SE(3) trajectories to cross narrow gaps.
They opt for polynomial trajectory and use the differential flatness of the quadrotor~\cite{mellinger2011minimum} to deliver spatial constraints to SE(3) state.
Compared to solving a large-scale problem in Wang's work\cite{wang2022geometrically}, this method develops parallel computing for this planning problem, significantly enhancing efficiency.
As for cluttered environments, Penicka et al.\cite{penicka2022minimum} extend on Li's work\cite{li2016asymptotically} using a hierarchical sampling-based framework guided with an incrementally more complex quadrotor model.
However, above methods do not have a real-time performance, making them inadequate to adapt to unpredictable changes, such as disturbances in the position of gate.
% However, similar to Foehn's work\cite{foehn2021time}, both of these methods\cite{han2021fast,penicka2022minimum} do not have a real-time performance, making it prohibitive to adapt to changing environments.

To enable the drone to handle environmental changes, online replanning is essential.
Some search-based methods\cite{liu2017search,liu2018search} add the need to bring time to a minimum in the cost function to achieve real-time replan for rapid flight.
But they can merely limit the acceleration of every axis while failing to explore the boundaries of the actuator, leading to conservative practices.
For the 2019 AlphaPliot Challenge, Foehn\cite{foehn2022alphapilot} perform an onboard detection of the gates as a reference for replanning.
They generate motion primitives based on maximum acceleration and obtain the time-optimal trajectory by sampling velocity states.
Finally, they use a polynomial to fit the trajectory to track.
Afterward, Romero~\cite{romero2022time} replace the polynomial parameterization of the above work\cite{foehn2022alphapilot} and use the Model Predictive Contouring Control (MPCC)\cite{romero2022model} considering an accurate full-state model which is extended to include a linear drag model\cite{faessler2017differential} to execute the trajectory obtained from sampling.
Due to the penalty on progress item and constraining the dynamics of each individual rotor in MPCC, this approach can better exploit the drone's performance limits.
Nevertheless, both methods of leaving it to the controller to track the trajectory that ignores the dynamic feasibility constraints can only ensure the safety of the gate-through by adding a certain cost weight at the controller side.
This is rarely applicable to dynamic environments.


\begin{table*}[!t]
    \centering
    % \captionsetup{font={small}}
    \caption{Comparison of Different Trajectory Representations}
    \label{tab:parameter}
    \small
    % \begin{tabular*}{\textwidth}{
    \begin{tabular*}{\textwidth}{
      @{\extracolsep{\fill}}
      m{1.4cm}
      >{$\displaystyle}l<{\vphantom{\sum_{1}{N}}$}
      >{\refstepcounter{equation}(\theequation)}r
      >{$\displaystyle}l<{\vphantom{\sum_{1}{N}}$}
      >{\refstepcounter{equation}(\theequation)}r
      >{$\displaystyle}l<{\vphantom{\sum_{1}{N}}$}
      >{\refstepcounter{equation}(\theequation)}r
      @{}
    }
    \toprule
    % meaning & \multicolumn{2}{l}{MINCO} & \multicolumn{2}{l}{Time-uniform MINCO}  & \multicolumn{2}{l}{\makecell[l]{Normalized Time-\\uniform MINCO}} \\
      & \multicolumn{2}{l}{MINCO} & \multicolumn{2}{l}{Time-uniform MINCO}  & \multicolumn{2}{l}{Normalized Time-uniform MINCO} \\
    \midrule
        \makecell[l]{time \\allocation} &
        \mathbf{T} = (T_1,\dots,T_{M})\tp, &
        \label{eq:minco_T} & 
        \hat{\mathbf{T}} = (T/M)\mathbf1, &
        % \hat{\mathbf{T}} = (T/M,...,T/M)\tp \in \mathbb{R}^{M\times 1}, &
        \label{eq:u_minco_T} & 
        \bar{\mathbf{T}} = \mathbf1, &
        \label{eq:n_u_minco_T}
        \\

        boundary conditions &
        \mathbf{z}^o, \mathbf{z}^f,&
        \label{eq:minco_b} & 
        \mathbf{z}^o, \mathbf{z}^f,&
        \label{eq:u_minco_b} & 
        \makecell[l]{\bar{\mathbf{z}}^o = \mathbf{S}_{s}(T/M){\mathbf{z}}^o,\\\bar{\mathbf{z}}^f = \mathbf{S}_{s}(T/M){\mathbf{z}}^f,} &
        \label{eq:n_u_minco_b}
        \\

        mapping equation &
        \mathbf{M}({\mathbf{T}}){\mathbf{C}} = \mathbf{b}(\mathbf{Q},~\mathbf{z}^o,\mathbf{z}^f), & 
        \label{eq:minco_mapping} & 
        % \multicolumn{2}{l}{MINCO} &
        \mathbf{M}(\hat{\mathbf{T}})\hat{\mathbf{C}} = \mathbf{b}(\mathbf{Q},~\mathbf{z}^o,\mathbf{z}^f), &
        \label{eq:u_minco_mapping} & 
        \mathbf{M}(\mathbf{1})\bar{\mathbf{C}} = \mathbf{b}(\mathbf{Q},~\bar{\mathbf{z}}^o, \bar{\mathbf{z}}^f), &
        \label{eq:n_u_minco_mapping}
        \\
        
        coefficients solving &
        \multicolumn{2}{l}{
            \makecell[l]{by online PLU factorization and \\
            solving linear systems of equations}    
        } &
        % \hat{\mathbf{c}} = \mathbf{M}^{-1}(\hat{\mathbf{T}})\mathbf{b}(\mathbf{Q},~\mathbf{z}^o,\mathbf{z}^f),&
        % \label{eq:minco_c} & 
        \hat{\mathbf{c}}_i = \mathbf{S}_{2s}(M/T)\bar{\mathbf{c}}_i,&
        \label{eq:u_minco_c} & 
        \bar{\mathbf{C}} = \mathbf{M}^{-1}(\mathbf{1})\mathbf{b}(\mathbf{Q},~\bar{\mathbf{z}}^o, \bar{\mathbf{z}}^f).&
        \label{eq:n_u_minco_c}
        \\

    % Reference yield &
    %   Y_{(R)}=\frac{H_{t}}{G}\quad \frac{(\si{kWh/m^2})}{(\si{kW/m^2})} &
    % \\
    \bottomrule
    \end{tabular*}
    \vspace*{-0.4cm}
\end{table*}


As for the navigation works mentioned for collision-free flight, they concentrate on handling the information about the static or dynamic environment\cite{wang2021autonomous,chen2022rast} obtained from perception to serve subsequent trajectory generation and on constructing constraints to avoid obstacles\cite{zhou2020ego,zhou2021raptor}.
However, they have a shortage of adaptability for drone racing tasks, which require the quadrotor to traverse waypoints in order with a focus on execution time.

% Consequently, to cope with the unpredictable changes and to ensure the safety when piercing gates, we propose an online replanning framework which formulate the gate-through demand by hard constraints.
% Then by penalizing the execution time and constraining the dynamics of the full-state quadrotor model in the trajectory optimization, we are capable of obtaining racing level trajectory in the challenging environments.



\section{Trajectory Representation}
\label{sec:traj}

% 参考一下小伦的论文,简单介绍为什么用minco和一些优势
% 说明这一节我们会通过对比的方式介绍均匀minco的定义和优势
% 为了避免混乱,我们使用了一个表格来将minco中重要的公式进行了一一对比排列,下面介绍一下表中的内容
% 然后说独有的会带上符号
% 下面统一说明一下这些公式中符号的定义

% 我们通过表格对比的方式来介绍其定义和优势
% 本质上后面两个只是特殊的minco而已,所以为了区别不同轨迹表示中的不同部分,我们分别用带有hat的符号和带有横线的符号表示在miinco和归一化轨迹中特殊部分
% 接下来我们将总体上介绍一遍表中出现的变量含义

In this section, for enhancing the computational efficiency for online replanning, we implement time-uniform MINCO based on our previous work~\cite{wang2022geometrically} to conduct spatial-temporal deformation of the flat-output trajectory.
We present the definitions and comparison in Tab.~\ref{tab:parameter}.
Essentially time-uniform MINCO and the normalized one are just two special kinds of MINCO.
Therefore, to distinguish the special parts of the different trajectory representations, we use the symbols with hat $\hat{\cdot}$ and the symbols with horizontal lines $\bar{\cdot}$, such as Eq.(\ref{eq:u_minco_T}) and Eq.(\ref{eq:n_u_minco_T}), to denote the special parts in the time-uniform MINCO and normalized one, respectively.
In the following, we go over the meaning of the variables that appear in Tab.~\ref{tab:parameter}.


% 几阶minco对应几个piece的几阶多项式,给出多项式定义。
% 2. 开始介绍上面定义的公式里面的参数是啥意思
% 3. 时间分配、边界条件、中间的位置点、b表示了由边界条件组成的矩阵,定义如下、c表示了系数
% 4. 接下来会解释公式
An $s$-order MINCO is defined as an $m$-dimensional $M$-piece polynomial trajectory. The $i$-th piece trajectory is defined by an $\mathcal{D}=2s-1$ degree polynomial as
\begin{equation}
    \label{eq:minco_def}
        p_i(t) = \mathbf{c}_i\tp\beta(t),~~\forall t \in [0, T_i],
\end{equation}
where $\beta(x):=(1,x,\dots,x^{\mathcal{D}})\tp$ is the natural basis.
$\mathbf{c}_i$  and $T_i$ means the coefficients and duration of the $i$-th piece respectively.
For the $M$-piece trajectory, the total duration is $T=\sum_{i=1}^{M}T_i$, and its coefficients $\mathbf{C}$, time allocation $\mathbf{T}$ and intermediate points $\mathbf{Q}$ can be written as
\begin{equation}
\label{eq:seg_parameter}
\begin{aligned}
    \mathbf{C} &= (\mathbf{c}_1\tp,\dots,\mathbf{c}_i\tp,\dots,\mathbf{c}_{M}\tp)\tp &&\in \mathbb{R}^{2Ms \times m},\\
    \mathbf{T} &= (T_1,\dots,{T}_i,\dots,T_{M})\tp &&\in \mathbb{R}_{>0}^{M},\\
    \mathbf{Q} &= (\mathbf{q}_1,\dots,\mathbf{q}_i,\dots,\mathbf{q}_{M-1}) &&\in \mathbb{R}^{m\times(M-1)},
\end{aligned}
\end{equation}
where $\mathbf{q}_i$ means the $0$-order derivative of $p_i(t)$ at $T_i$.
And in Tab.~\ref{tab:parameter}, $\mathbf{z}^o, \mathbf{z}^f \in \mathbb{R}^{m\times s}$ is the boundary conditions containing high order derivative at $p_1(0)$ and $p_M(T_M)$.
$ \mathbf{M} \in \mathbb{R}^{2Ms\times 2Ms}$ and $ \mathbf{b} \in \mathbb{R}^{2Ms\times m}$ are matrixes with $\mathbf{T}$ and $\mathbf{Q},~\mathbf{z}^o,\mathbf{z}^f$ as variables, respectively, which refer to the Eq.(54,~55) in \cite{wang2022geometrically} for the detailed definition. 
In Eq.(\ref{eq:u_minco_T},~\ref{eq:n_u_minco_T}), $\mathbf{1}=\left(1,1,...,1\right)\tp \in \mathbb{R}^{M}$.
In Eq.(\ref{eq:n_u_minco_b},~\ref{eq:u_minco_c}), $\mathbf{S}_{y}(x):=$ Diag$(1,x,\dots,x^{y-1})$.

% where $\beta(x):=(1,x,\dots,x^N)\tp$ is the natural basis.
% $\mathbf{C} = (\mathbf{c}_1\tp,\dots,\mathbf{c}_{M}\tp)\tp \in \mathbb{R}^{2Ms \times m}$ is the polynomial trajectory coefficients and $T_i$ means the duration of the $i$-th piece.
% $\mathbf{T} = (T_1,\dots,T_{M})\tp$ is the time allocation for all pieces and $T=\sum_{i=1}^{M}T_i$ is the total duration.


% $\mathbf{Q} = (\mathbf{q}_1\tp,\dots,\mathbf{q}_{M-1}\tp)\tp \in \mathbb{R}^{m\times(M-1)}$ is the intermediate points, where $\mathbf{q}_i$ means the $0$-order derivative of $p_i(t)$ at $T_i$.
% And $\mathbf{z}^o, \mathbf{z}^f \in \mathbb{R}^{m\times s}$ is the boundary conditions containing high order derivative at $p_1(0)$ and $p_M(T_M)$.
% $ M \in \mathbb{R}^{2Ms\times 2Ms}$ and $ b \in \mathbb{R}^{2Ms\times m}$ are matrixes with $\mathbf{T}$ and $\mathbf{Q},~\mathbf{z}^o,\mathbf{z}^f$ as variables, respectively, which refer to the Eq.~54 and 55 in \cite{wang2022geometrically} for the detailed definition. 
% In Eq.~\ref*{eq:u_minco_T} and \ref*{eq:n_u_minco_T}, $\mathbf{1}=\left(1,1,...,1\right)\tp \in \mathbb{R}^{M}$.
% In Eq.~\ref{eq:n_u_minco_b} and \ref{eq:u_minco_c}, $\mathbf{S}_{y}(x):=$ Diag$(1,x,\dots,x^{y-1})$.


\subsection{MINCO Trajectory}
\label{subsec:minco}
% minco
% 1. 然后说minco怎么求解c的,带状矩阵
% 2. 说明一下minco的工作原理

For an $s$-order MINCO, given boundary conditions Eq.(\ref{eq:minco_b}), intermediate points $\mathbf{Q}$ and time allocation Eq.(\ref{eq:minco_T}), the coefficients ${\mathbf{C}}$ can be uniquely determined by $\mathbf{z}^o,\mathbf{z}^f,\mathbf{Q}$ and ${\mathbf{T}}$ based on the linear mapping equation Eq.(\ref{eq:minco_mapping}), which is defined as Theorem 2 in \cite{wang2022geometrically}.
Because $\mathbf{M}$ is a nonsingular banded matrix, based on its banded PLU factorization, the coefficients can be computed by solving two linear systems of equations with linear time and space complexity~\cite{golub2013matrix}, which prevents the need to explicitly calculate $\mathbf{M}^{-1}$.

For user-defined penalty function $F({\mathbf{C}},{\mathbf{T}})$ with available gradients, MINCO serves as a linear-complexity differentiable layer $H(\mathbf{z}^o,\mathbf{z}^f,\mathbf{Q},{\mathbf{T}})=F({\mathbf{C}},{\mathbf{T}})$.
To accomplish the deformation of MINCO, we need to obtain the gradients of $H$ w.r.t. the trajectory's variable $\mathbf{z}^o,\mathbf{z}^f,\mathbf{Q}$ and ${\mathbf{T}}$ from the given gradients $\partial{F}/\partial{{\mathbf{C}}}$ and $\partial{F}/\partial{{\mathbf{T}}}$ as  Eq.(60,~68) in \cite{wang2022geometrically}.
During the process, results of right multiplying $\mathbf{M}^{-1}$ by a vector are needed several times. 
Similar to coefficients solving, these results can be get by using the PLU factorization of $\mathbf{M}$.

\subsection{Time-uniform MINCO Trajectory}
It should be emphasized that, in Tab.~\ref{tab:parameter} and this section,  the role of normalized time-uniform MINCO is to serve as an intermediate state of the time-uniform MINCO to aid in the calculation of coefficients and gradients.

% 均匀minco
% 1. 均匀minco开头开始说,minco的问题,然后引出我们想提高计算效率实现了均匀minco(原第1、2句)
% 2. 然后说一下公式258,表示这些内容除了T之外还是和minco一样,但是T的特性,让我们想到是否可以骚一把
% 3. 把均匀minco极其scale的思想介绍清楚,并在这里按照公式16定义归一化轨迹,再说明边界条件的变化
% 4. 然后说公式9和10,可以看到不同于7或8,带状矩阵M在piece定下来的时候就确定了,可以离线计算好,因此有公式12,这个c矩阵可以在确定了边界条件后很高效的获得
% 5. 回到公式8,之前我们说这个公式求解c还是需要动态求解M矩阵的PLU分解来处理逆,但是现在我们可以从归一化的minco出发,计算归一化的c,然后再通过公式11求解c。避免了在线plu分解
  
We refer to a special MINCO trajectory with uniform time allocation Eq.(\ref{eq:u_minco_T}) as time-uniform MINCO, since each piece has the same duration.
The time-uniform MINCO still allows for spatial-temporal deformation, which is accomplished through the freedom of the total time $T$.

As Eq.(\ref{eq:u_minco_mapping}) states, even with a uniform time allocation, we still need to online deal with $\mathbf{M}^{-1}$ whenever $T$ changes.
To avoid PLU factorization of $\mathbf{M}$ every time, using temporal scaling, we define normalized time-uniform MINCO $\bar{p}(t)$ of a time-uniform MINCO $\hat{p}(t)$, and the $i$-th piece is
\begin{equation}
    \label{eq:time_uniform_minco}
    \bar{p}_i(t) = \hat{p}_i({T}/{M} \cdot t),~~\forall t \in [0, 1],
\end{equation}
where $\bar{p}_i(t)$ takes $1$ as piece duration.
Since the original trajectory is time-uniform,
As defined in Eq.(\ref{eq:time_uniform_minco}), the intermediate waypoints $\mathbf{Q}$ do not change because the spatial shape of the trajectory is kept constant.
The high order derivatives of $\bar{p}_i(t)$ can be written as
\begin{equation}
    \bar{p}_i^{(s)}(t) = (T/M)^s \hat{p}_i^{(s)}({T}/{M} \cdot t),~~\forall t \in [0, 1].
\end{equation}
Then the boundary condition is deflated by a factor of $(T/M)^s$ in the $s$-th order derivative due to the temporal scaling, as written in Eq.(\ref{eq:n_u_minco_b}).

For the normalized $\bar{p}(t)$, the mapping is written by Eq.(\ref{eq:n_u_minco_mapping}), where the banded matrix $\mathbf{M}(\mathbf{1})$ becomes constant because the quantity of the pieces $M$ is fixed during trajectory optimization.
Therefore, $\mathbf{M}(\mathbf{1})^{-1}$ can be computed explicitly or performed PLU factorization offline, instead of online factorization every time $T$ changes in the optimization.
Then the coefficients $\bar{\mathbf{C}}$ can be obtained from Eq.(\ref{eq:n_u_minco_c}) and the coefficients $\hat{\mathbf{C}}$ of the original trajectory can be obtained by scaling the normalized trajectory, as written in Eq.(\ref{eq:u_minco_c}).
Moreover, the gradient calculation mentioned in Sec.~\ref{subsec:minco} can be performed without online factorization either.
These have an improvement in computational efficiency, as shown in Sec~.\ref{sec:Evaluations:Ablation:minco}.

To summarize, to use time-uniform MINCO yet avoid online PLU factorization, we first normalize the trajectory in time. 
Then we use the property that $\mathbf{M}(\mathbf{1})$ of the normalized trajectory can be processed offline to quickly obtain the normalized parameters $\bar{\mathbf{C}}$ and gradients. 
Finally, we get the parameters $\hat{\mathbf{C}}$ and gradients of the time-uniform MINCO which is truly desired by time deflating as shown in Eq.(\ref{eq:u_minco_c}).

\section{Polynomial-based Online Planning}
\label{sec:Online_Replan}
% In this section, we proposed a general optimization-based replan-framework for drone racing, which can handle  dynamic environment and unpredictable noise.
% The basic idea is to transform the constrained trajectory optimization problem to an unconstrained optimization problem, then solve it using numerical optimization techniques.

% \todo cope with intro.
% For efficiently solve trajectory optimization problem, we adopt MINCO~\cite{wang2022geometrically} representation.
% Thanks to the spatial-temporal property of MINCO, we can efficiently handle both spatial constrain and temporal constrain.
% In order to reduce computation time, we propose time-uniform MINCO with ablation study in Sec. \ref{sec:Simulation_Result}.
% In order to generate time-optimal trajectory without violating the dynamic constraints, we add the trajectory duration to our optimization problem and penalize the dynamical overconstraint.
% To ensure that the trajectory passes through dynamic gate, we utilize gate trajectory with MINCO representation to eliminate the hard constraint.
% To handle unpredictable noise, such as uncertainty on gate positions and dynamic obstacles, we proposed a warm-up replan framework(Sec. \ref{subsec:Implement_Details}).


In this section, we use time-uniform MINCO as the trajectory representation for online planning, requiring the shortest possible execution time while satisfying some environmental and actuator constraints.
First in Sec.~\ref{sec:Problem_Formulation} we construct an optimization problem considering above requirements.
Then based on our dealing with its inequality and equation constraints respectively in Sec.~\ref{sec:Inequality_Constraints_Transcription} and \ref{sec:Equality_Constrains_Elimination}, the problem is reformulated into an unconstrained optimization problem in Sec.~\ref{sec:Problem_Reformulation}.
Additionally, in Sec.~\ref{subsec:Implement_Details}, we state some engineering considerations that are effective in improving racing performance.
Note that the symbol definitions of time-uniform MINCO in this section are inherited from Sec.~\ref{sec:traj}.
% 主要总结一下做什么
% 本章我们将使用前一章介绍的均匀minco作为轨迹表征,进行在线轨迹重规划,要求尽可能短执行时间的同时满足一些环境和执行器约束
% 首先在第一段将考虑所有约束构建一个优化问题,后面两段将分别处理不等式和等式约束,最终在第四段将其重构成为一个无约束优化问题。
% 值得注意的是,在第五段中,我们将一些有效提升成绩的工程处理进行了陈述

\subsection{Problem Formulation}
\label{sec:Problem_Formulation}
% 1. 表示多段MINCO轨迹图
% 2. 给定N个门,轨迹被分成了N段,第i段表示成具有Mn段的时间均匀MINCO轨迹,强调不一样的颜色表示了不一样的segment,都是一段均匀minco(总)
% 3. 对于第n个segment轨迹,他的参数是怎么定义的,然后轨迹定义为什么符号,由piece组成,Tn还是得单独定义一下的,为了引出Tn(不加粗)(分)
% 4. 然后定义整条多段minco的轨迹由多个segment组成,定义为什么 (总)

In this paper, we use segmented polynomials to represent the replan trajectory.
Given the next $N$ gates, we plan an $N$-segment trajectory.
As shown in Fig.~\ref{fig:traj}, each colored curve represents one segment.
The $n$-th segment $\sigma_n(t)$ is an $s$-order $m$-dimensional $M_n$-piece time-uniform MINCO, 
whose coefficients and intermediate points are defined as $\hat{\mathbf{C}}_n$ and $\mathbf{Q}_n$ respectively, detailed in Eq.(\ref{eq:seg_parameter}).
Its time allocation is defined as $\hat{\mathbf{T}}_n=(T_n/M_n)\mathbf{1}$, where $T_n$ is the trajectory duration of $n$-th segment.
% 定义总的轨迹得定义一下符号还有时间timestamp
The whole segmented trajectory $\sigma(t):[t_0, t_N]\mapsto\mathbb{R}^m$ is formulated as
\begin{equation}
    \begin{gathered}
        \sigma(t_{n-1}+t) = \sigma_n(t), \\
        \forall n \in \{1,2,\dots,N\}, ~\forall t \in [0,T_n],
    \end{gathered}
\end{equation}
where $t_n=t_0+\sum_{j=1}^{n}T_j$ is the timestamp and $t_0$ is the start time of the trajectory.
The coefficients, time allocation, and intermediate points of the whole trajectory can be written as
\begin{equation}
    \begin{aligned}
        \boldsymbol{\mathcal{C}} &= (\hat{\mathbf{C}}_1\tp,\dots,\hat{\mathbf{C}}_n\tp,\dots,\hat{\mathbf{C}}_{N}\tp)\tp &&\in \mathbb{R}^{(\sum_{n=1}^{N}2 M_n s) \times m},\\
        \boldsymbol{\mathcal{T}} &= (\hat{\mathbf{T}}_1\tp,\dots,\hat{\mathbf{T}}_n\tp,\dots,\hat{\mathbf{T}}_{N}\tp)\tp &&\in \mathbb{R}_{>0}^{(\sum_{n=1}^{N}M_n)},\\
        \boldsymbol{\mathcal{Q}} &= (\mathbf{Q}_1,\dots,\mathbf{Q}_n,\dots,\mathbf{Q}_{N}) &&\in \mathbb{R}^{m \times \left(\sum_{n=1}^{N}(M_n-1)\right)}.
    \end{aligned}
\end{equation}

% 基于我们定义的轨迹,考虑什么什么,最终我们将问题构建为一个如下的优化问题
% 和sec首部的文字联动
We require the trajectory to pass through a series of gates, both static and moving, as quickly as possible, with constraints of dynamical feasibility, narrow gap crossing, and dynamic obstacle avoidance.
Taking all requirements into account, our problem takes the following form:
\begin{subequations}
    \label{eq:problem_formulation}
    \begin{align}
        \underset{\boldsymbol{\mathcal{C}},\boldsymbol{\mathcal{T}}}{min}~~
                 & \label{eq:problem_formulation:cost} J_o=\int_{t_0}^{t_N} \| \sigma^{(s)} (t) \|^2 dt + \rho \cdot ||\boldsymbol{\mathcal{T}}||_1,                                 \\
        s.t.~~
                & \label{eq:problem_formulation:boundary_1} \sigma_1^{[0,s-1]}(0)=\mathbf{s}^o,                                                                       \\
                 & \label{eq:problem_formulation:boundary_2} \sigma_N^{[0,s-1]}(T_N)=\mathbf{s}^f,                                                                     \\
                 & \label{eq:problem_formulation:boundary_3} \sigma_{n}^{[0,s-1]}(T_{n})=\sigma_{n+1}^{[0,s-1]}(0),~\forall n \in \{1,\dots,N-1\},                         \\
                 & \label{eq:problem_formulation:gate} \sigma(t_{n})=g_n(t_{n}), ~\forall n \in \{1,\dots,N\},                                             \\
                 & \label{eq:problem_formulation:inequal} \mathcal{G}_x\left(\sigma(t),...,\sigma^{(s)}(t),t\right)\preceq \mathbf{0}, ~\forall x \in \mathcal{X},\forall t \in [t_0,t_N],
    \end{align}
\end{subequations}
where we define two costs in Eq.(\ref{eq:problem_formulation:cost}) for smoothness and short execution time, which are weighed by parameter $\rho$.
% []的含义
Eq.(\ref{eq:problem_formulation:boundary_1}--\ref{eq:problem_formulation:boundary_2}) and Eq.(\ref{eq:problem_formulation:boundary_3}) are the boundary conditions and the continuity constraint up to degree $s-1$.
$\mathbf{s}^o$ and $\mathbf{s}^f$ are boundary states. 
We denote $\sigma^{[x,y]} \in \mathbb{R}^{m \times \left(y-x+1\right)}$ as
\begin{equation}
    \sigma^{[x,y]}=\left(\sigma^{\left(x\right)},\sigma^{\left(x+1\right)},\dots,\sigma^{\left(y\right)} \right),~~x < y.
\end{equation}
Moreover, Eq.(\ref{eq:problem_formulation:gate}) is the gate-through constraint, where $g_i(t)$ is the predicted trajectory for the $i$-th gate.
% 函数怎么描述
Eq.(\ref{eq:problem_formulation:inequal}) is continuous-time constraints, the set $\mathcal{X}=\{t,b,g,d\}$ include actuator physical limits on thrust $(t)$ and body rate $(b)$, narrow gap crossing $(g)$ and dynamic obstacle avoidance~$(d)$.

\begin{figure}[!t]
	\centering
    \vspace{0.2cm}
	\includegraphics[width=1\linewidth]{figures/traj.pdf}
    \captionsetup{font={footnotesize}}
	\caption{
        Illustration of the segmented trajectory for replan to traversal the four gates in order.
	}
	\label{fig:traj}
    \vspace{-1cm}
\end{figure}

\subsection{Inequality Constraints Transcription}
\label{sec:Inequality_Constraints_Transcription}
% 先把大的讲的,就是离散采样来代替连续函数的方式,然后说明优化问题变成什么形式
% 然后针对一个采样点,梯度是怎么传递的,整个函数写成一个通用形式
% 然后说下面介绍的是每个采样点的情况,把梯度传递稍微写一下
% 参考一下小伦和韩巨佳林的,写到那一层
For the inequality constraints Eq.(\ref{eq:problem_formulation:inequal}), which are required to be fulfilled along the whole trajectory.
We use the thought behind penalty function method~\cite{jennings1990computational} to deal with these constraints becoming time integral of constraint violations, which is then evaluated by a finite sum of sample points.
We define these points sampled on $n$-th segment by $\mathring{\mathbf
p}_{n,j}=\sigma_n((j/\kappa_n)T_n), j\in \{0,1,2,\dots,\kappa_n\}$, where $\kappa_n$ is the sample quantity and we name $\mathring{\mathbf
p}_{n,j}$ as \textbf{constraint points}.
We denote the penalty function of the constraint points as
\begin{equation}
    \mathcal{G}_x({n,j})=\mathcal{G}_x\left(\mathring{\mathbf
    p}_{n,j},\mathring{\mathbf p}_{n,j}^{(1)},...,\mathring{\mathbf p}_{n,j}^{(s)},t_{n-1}+\frac{j}{\kappa_n}T_n\right).
\end{equation}
Then these inequality constraints Eq.(\ref{eq:problem_formulation:inequal}) can be transformed into a weighted sum of the sampled penalty as
\begin{equation}
\begin{gathered}
    \label{eq:inequality_cost}
    \underset{\boldsymbol{\mathcal{C}},\boldsymbol{\mathcal{T}}}{min}~~\sum\nolimits_{x}^{\mathcal{X}}\lambda_x J_x,\\
    J_x=\sum\nolimits_{n=1}^{N}\frac{T_n}{\kappa_n}
    \sum\nolimits_{j=0}^{\kappa_n}\bar\omega_j\underset{}{max}
    (\mathcal{G}_x(n,j),0)^3,
\end{gathered}
\end{equation}
where $\lambda_x$ is the weight for each cost $J_x$, we follow the trapezoidal rule~\cite{press2007numerical} $(\bar{\omega}_0,\bar{\omega}_1,\dots,\bar{\omega}_{\kappa_i})=(1/2,1,\cdots,1,1/2)$.
Then for the inequality constraints Eq.(\ref{eq:problem_formulation:inequal}), with Eq.(\ref{eq:minco_def},~\ref{eq:inequality_cost}), 
once the gradients of $\mathcal{G}_x(n,j)$ w.r.t. $\mathring{\mathbf p}_{n,j}^{(k)}, k \in \{1,\dots,s\}$ and $\{T_1,\dots,T_n\}$ are given, we can derive the gradients $\partial J_x/ \partial \boldsymbol{\mathcal{C}}$ and $\partial J_x/ \partial \boldsymbol{\mathcal{T}}$ which are required in the optimization.
Then we introduce each penalty function and its gradient.
% 接下来我们给出各个定义和对应的梯度说明

\subsubsection{Actuator Limits with Drag Effects $\mathcal{G}_t$ and $\mathcal{G}_b$}
\label{sec:limits}
% 为了保证轨迹在物理上可以执行,我们将对通过无人机的微分平坦特性,对轨迹状态对应的推力和旋转进行约束.同时由于无人机高速飞行,空气动力学效应不能被忽视,因此我们用之前的工作的drag模型拓展了无人机的动力学:公式
% 其中参数是什么什么,基于工作中的公式什么,可以得知,采样点上对应的推力 转速和旋转可以被获得
% 我们定义约束等于(推力和角速度)
% 通过这些公式,给一个梯度好说明
To ensure that the trajectory is physically feasible, we constrain the thrust~$f$ and body rate $\omega$ through the differential flatness.
Meanwhile, due to the high speed of racing, aerodynamic effects cannot be ignored, then we extended the quadrotor's dynamics with a drag model of our previous work~\cite{wang2022robust}.
% \begin{equation}
%     \left\{
%     \begin{aligned}
%         m\ddot{r} & = -mge_3 - RDR^T\sigma(\|\dot{r}\|)\dot{r}+Rfe_3, \\
%         \dot{R}   & = R\hat{ \omega},
%     \end{aligned}
%     \right.
% \end{equation}
% where $r \in \mathbb{R}^3$ and $R \in SO(3)$ are translation and rotation, $f \in \mathbb{R}_{\geq 0}$ is the thrust and $\omega \in \mathbb{R}^3$ is the body rates, $m$ is the vehicle mass, $g$ the gravitational acceleration, $e_3=(0,0,1)\tp, D=diag\{d_h,d_h,d_v\}$ a horizontally symmetric drag coefficients matrix, $\sigma : \mathbb{R}_{\geq 0} \mapsto \mathbb{R}_{\geq 0}$ a nonlinear term, and $\hat{\omega}$ a skew-symmetric matrix.
As shown in the Eq.(17-21) of \cite{wang2022robust}, given the trajectory, thrust, body rate and rotation $R \in SO(3)$ in the world frame of the constraint point $\mathring{\mathbf p}_{n,j}$ can be denoted as
\begin{equation}
\label{eq:flatness}
    f,\omega,R = \mathcal{F}\left(\mathring{\mathbf p}_{n,j}^{(1)}, \mathring{\mathbf p}_{n,j}^{(2)},\mathring{\mathbf p}_{n,j}^{(3)}\right),
\end{equation}

We define the penalty of actuator physical limits as
\begin{equation}
\label{eq:physical_limits}
\begin{aligned}
    \mathcal{G}_t(n,j)&=
     (f-f_m)^2-f_r^2,\\
    \mathcal{G}_b(n,j) &= ||\omega||^2-\omega_{max}^2,
\end{aligned}
\end{equation}
where $f_m=(f_{max}+f_{min})/2$ and $f_r=(f_{max}-f_{min})/2$, $f_{max}$ and $f_{min}$ are the maximum and minimum values of thrust.
$\omega_{max}$ is the maximum body rate. With Eq.(\ref{eq:flatness},~\ref{eq:physical_limits}), the gradients $\partial \mathcal{G}_t(n,j)/\partial \mathring{\mathbf p}_{n,j}^{(k)},~ \partial \mathcal{G}_b(n,j)/\partial \mathring{\mathbf p}_{n,j}^{(k)},~k={1,2,3}$ can be calculated easily using the chain rule.

% \textbf{\textcolor{red}{TODO,gradient}}

% \begin{figure}[!t]
% 	\centering
%     \vspace{0.2cm}
% 	\includegraphics[width=1\linewidth]{figures/se3.pdf}
%     \captionsetup{font={footnotesize}}
% 	\caption{
%         Illustration of the desired crossing position and direction of z-axis of the quadrotor for traversing the narrow gap.
% 	}
% 	\label{fig:se3}
%     \vspace{-1cm}
% \end{figure}

\subsubsection{Narrow Gap Crossing $\mathcal{G}_g$}
\label{sec:Narrow_Gap}
% 在规划中我们假设,通过别的模块比如在线识别,我们获得了在某个gap穿越的最佳姿态和位置.
% 同时为了保证穿越gap时候的安全性,为了不失通用性,我们将gap的最佳穿越位置设置为第n个门gn,其位置上的约束将在后面详细说明如何处理,这里只说明姿态上的约束
% 在前面一章通过平摊(公式)我们已经能够得到约束点对应的R了
% 这里得写出我们的约束函数啥形式,但是我不知道
In this framework, we assume that we can obtain the optimal attitude and position for crossing a narrow gap from other modules such as online detection.
As shown in Fig.~\ref{fig:traj}, we set the desired direction of the z-axis of the quadrotor when traversing the gate as a normalized vector $\mathbf{z}_{des} \in \mathbb{R}^m$. 
To avoid loss of generality, we set the optimal crossing position as a gate, denoted as the $n_g$-th gate to be passed through.
The position constraint will be detailed in Sec.~\ref{sec:gate_con}, and here we only describe the direction constraint during gate crossing.
To make it safer to traverse the narrow gate with a large attitude, we impose this constraint on the trajectory at a certain sample range $n_{ran}$ in front of and behind the $n_g$-th gate:
\begin{equation}
\label{eq:narrow_gap}
\begin{gathered}
    \mathcal{G}_g(n,j)=|| R\mathbf{e}_3 - \mathbf{z}_{des} ||^2-\theta_{tol},\\
    \forall j \in 
    \left\{
    \begin{aligned}
        &\{0,1,\dots,n_{ran}\}, &&n=n_g\\
        &\{\kappa_n-n_{ran},\dots,\kappa_n\}, &&n=n_g-1
    \end{aligned}
    \right.,
\end{gathered}
\end{equation}
where $\mathbf{e}_3=(0,0,1)\tp$, $\theta_{tol} \in (0,1)$ is the tolerance, $R$ is the rotation of $\mathring{\mathbf p}_{n,j}$ obtained from Eq.(\ref{eq:flatness}).
Then the gradien $\partial \mathcal{G}_g(n,j)/\partial \mathring{\mathbf p}_{n,j}^{(k)}$ can be computed with Eq.(\ref{eq:flatness},~\ref{eq:narrow_gap}).


\subsubsection{Dynamic Obstacle Avoidance $\mathcal{G}_d$}
% 把拓扑的事情讲讲清楚
% 给出参考轨迹pd
% 给出马氏距离定义,然后再给出椭球定义
% 然后给出cost和对应的梯度
Based on our previous work~\cite{wang2021autonomous}, we model a dynamic obstacle as ellipsoid:
\begin{equation}
    \mathbb{E}=\{\mathbf{x}|E(\mathbf{x},\mathbf{y})<1\},~~
    E(\mathbf{x},\mathbf{y})=(\mathbf{x}-\mathbf{y})\tp\mathbf{H}(\mathbf{x}-\mathbf{y}),
\end{equation}
where $\mathbf{H}=R_E\tp diag(1/a^2,1/b^2,1/c^2)R_E$, $a,b,c$ is the axis-length, $R_E$ is its rotation and $\mathbf{y}$ is the center of the ellipsoid.
Then we define the obstacle avoidance penalty as
\begin{equation}
    \label{eq:dynamic}
    \mathcal{G}_d(n,j)=d_{thr}-E\left(\mathring{\mathbf p}_{n,j},~p_d(t_{n-1}+\frac{j}{\kappa_n}T_n)\right),
\end{equation}
where $p_d(t)$ is the predicted trajectory of the dynamic obstacle, $d_{thr}$ is security threshold.
Similar to Sec.~\ref{sec:limits} and \ref{sec:Narrow_Gap}, by the chain rule, we can utilize Eq.(\ref{eq:dynamic}) to obtain gradient of $\mathcal{G}_d(n,j)$ w.r.t. $\mathring{\mathbf p}_{n,j}$ and $\{T_1,\dots,T_n\}$, which are used to derive $\partial J_d/ \partial \boldsymbol{\mathcal{C}}$ and $\partial J_d/ \partial \boldsymbol{\mathcal{T}}$ for optimization, as stated in Sec.~\ref{sec:Inequality_Constraints_Transcription}.

\subsection{Equality constraints Elimination}
\label{sec:Equality_Constrains_Elimination}
% 总起一些
% 前面我们将不等式约束处理成了什么,这里我们将对等式约束进行处理,最终把问题进行reformulation
% 前一章我们通过惩罚函数的方法将不等式约束进行了转化,本节我们将通过更换决策变量的方式把等式约束消除,将问题转化成无约束的
In Sec.~\ref{sec:Inequality_Constraints_Transcription} we transform the inequality constraints by the penalty function method, in this section we eliminate the equation constraints by replacing the decision variables.
\subsubsection{Boundary Conditions and Continuity Constraint}
\label{sec:Boundary_Conditions}
% 定义中间状态点,然后定义这个一个set,其中z0和zN=s0和sf是固定的
% 那么针对第n segment轨迹,在时间分配和初末状态都确定的时候,我们可以通过公式8-11来快去的唯一确认系数cn,我们将这个求解过程记为 c=m 巴拉巴拉的
% 此时,我们遍可以替换变量,从什么什么到什么什么,消除了约束b-d
% 目的就是简单说明一下这段是要做什么
We define $\mathbf{z}_0$ as the start state of the first segment and $\mathbf{z}_n $ as the end state of the $n$-th segment:
\begin{equation}
    \begin{aligned}
    \mathbf{z}_0&=\sigma_{1}^{[0,s-1]}(0),\\
    \mathbf{z}_n&=\sigma_{n}^{[0,s-1]}(T_{n}),~\forall n \in \{1,\dots,N\}.
    \end{aligned}
\end{equation}
For the $n$-th segment, given the time allocation $\hat{\mathbf{T}}_n$, intermediate points $\mathbf{Q}_n$ and the boundary conditions $\mathbf{z}_{n-1}$,~$\mathbf{z}_n$, its coefficients $\hat{\mathbf{C}}_n$ can be uniquely determined by a serie of calculations Eq.(\ref{eq:u_minco_mapping}--\ref{eq:n_u_minco_c}).
We denote this calculation process as
\begin{equation}
\label{eq:mapping}
        \mathbf{C}_n =\mathcal{M}(\mathbf{Q}_n,\hat{\mathbf{T}}_n,\mathbf{z}_{n},\mathbf{z}_{n-1}).
\end{equation}

We set $\boldsymbol{\mathcal{Z}} = (\mathbf{z}_1, \dots, \mathbf{z}_{N-1})$, and fix $\mathbf{z}_0=\mathbf{s}^o$ and $\mathbf{z}_N=\mathbf{s}^f$.
% \in \mathbb{R}^{m \times s(N-1)}$.
Based on Eq.(\ref{eq:mapping}), $\boldsymbol{\mathcal{C}}$ can be determined with variables $\boldsymbol{\mathcal{Z}}$, $\boldsymbol{\mathcal{T}}$ and $\boldsymbol{\mathcal{Q}}$.
Therefore, for our original problem Eq.(\ref{eq:problem_formulation}), we replace the decision variables from $(\boldsymbol{\mathcal{C}}$, $\boldsymbol{\mathcal{T}})$ to $(\boldsymbol{\mathcal{Z}}$, $\boldsymbol{\mathcal{T}}$, $\boldsymbol{\mathcal{Q}})$, eliminating the constraints of Eq.(\ref{eq:problem_formulation:boundary_1}--\ref{eq:problem_formulation:boundary_2}).




\subsubsection{Gate-through Constraint}
\label{sec:gate_con}
% 对于动态环的gate constrain,其他方法都是使用软约束来实现的,稍微引一下别的论文
% 正如图中所示,当门比较小的时候,空间容错率很低,使用软约束并不能保证穿环的安全性,因此我们将公式e以硬约束的方式进行构建

% 首先我们将上面定义的变量z分割为,则公式e可以写为什么,把时间展开写
% 即这个变量可以由T唯一确认,因此可以从自变量中移除
% 此时已经将变量全部换完,优化问题变成了这个形式
% 而前一章中,对于我们定义的约束,我们都只把梯度传导到了c和T,现在给出传到新变量的写法


The requirement of crossing gate is usually handled with soft constraints, for instance, penalty function $\|\sigma(t_{n})-g_n(t_{n})\|^2$ used in \cite{romero2022model,romero2022time}.
However, As shown in Fig.~\ref{fig:traj}, when the gate is small, the space tolerance at the moment of crossing is very low.
The use of soft constraints does not guarantee that trajectory passes through the center of the gate, which leads to more pressure on safety being placed on other modules, such as the controller.
In this paper, we construct hard constraints for gate-through to ensure that the trajectory can traverse the center of the gate.

We split the state matrix $\mathbf{z}_n$ as $\left((\mathbf{z}_n)_{0}, (\mathbf{z}_n)_{*}\right)$, where $(\mathbf{z}_n)_{0} \in \mathbb{R}^{m \times 1}$ is the $0$-order derivative state, and $(\mathbf{z}_n)_{*} \in \mathbb{R}^{m \times (s-1)}$ means the derivatives whose order from $1$ to $s$.
Then the gate-through constraints Eq.(\ref{eq:problem_formulation:gate}) can be written as
\begin{equation}
    \begin{aligned}
        (\mathbf{z}_n)_{0}
            &=g_n(t_{n})\\
            &=g_n(t_0+\sum\nolimits_{j=1}^{n}T_j),
    \end{aligned}
\end{equation}
which demonstrates that $(\mathbf{z}_n)_{0}$ can determined by $\boldsymbol{\mathcal{T}}$.
Therefore, the mapping function Eq.(\ref{eq:mapping}) can be formulated as
\begin{equation}
    \begin{aligned}
        \label{eq:new_mapping}
        \mathbf{C}_n=\mathcal{M}\Bigg(\mathbf{Q}_n,\hat{\mathbf{T}}_n,
        \Big(g_{n}(t_0+\sum\nolimits_{j=1}^{n}T_j),~(\mathbf{z}_{n})_{*}&\Big),\\
        \Big(g_{n-1}(t_0+\sum\nolimits_{j=1}^{n-1}T_j),~(\mathbf{z}_{n-1})_{*}&\Big)\Bigg).
    \end{aligned}
\end{equation}
To eliminate the constraints of Eq.(\ref{eq:problem_formulation:gate}), We replace the variables in Sec.~\ref{sec:Boundary_Conditions} from $(\boldsymbol{\mathcal{Z}}$, $\boldsymbol{\mathcal{T}}$, $\boldsymbol{\mathcal{Q}})$ to $(\boldsymbol{\mathcal{Z}}_*$,~$\boldsymbol{\mathcal{T}}$,~$\boldsymbol{\mathcal{Q}})$, where ${\boldsymbol{\mathcal{Z}}_*}=((\mathbf{z}_1)_{*},\dots, (\mathbf{z}_{N-1})_{*})$.

\subsection{Problem Reformulation}
\label{sec:Problem_Reformulation}
Combining the transcription of inequality constraints in Sec.~\ref{sec:Inequality_Constraints_Transcription} and the elimination of equality constraints in Sec.~\ref{sec:Equality_Constrains_Elimination}, the whole original problem Eq.(\ref{eq:problem_formulation}) is finally reformulated as an unconstrained optimization problem:
\begin{equation}
    \label{eq:unconstrained_optimization_problem}
    \underset{\boldsymbol{\mathcal{Z}}_*,\boldsymbol{\mathcal{T}},\boldsymbol{\mathcal{Q}}}{min}
    J=(J_o+
    \sum\nolimits_{x}^{\mathcal{X}}\lambda_x J_x ).
\end{equation}

Sec.~\ref{sec:Inequality_Constraints_Transcription} gives the gradients of $J$ w.r.t. $(\boldsymbol{\mathcal{C}},\boldsymbol{\mathcal{T}})$,  but we change the decision variables in Eq.(\ref{eq:unconstrained_optimization_problem}).
Based on Eq.(\ref{eq:mapping}), we can obtain the gradients of $\boldsymbol{\mathcal{C}}$ w.r.t. $(\boldsymbol{\mathcal{Z}}$,~$\boldsymbol{\mathcal{T}}$,~$\boldsymbol{\mathcal{Q}})$, which are detailed in the Eq.(60,~68) in \cite{wang2022geometrically}.
However, in the new mapping function Eq.(\ref{eq:new_mapping}), we set $\mathbf{z}_n$ a matrix with $(\mathbf{z}_n)_{0}$ and $(\mathbf{z}_n)_*$ as variables, and $(\mathbf{z}_n)_{0}$ is a vector with $\boldsymbol{\mathcal{T}}$ as decision variable.
Therefore, to obtain the gradient of $J$ w.r.t. the new decision variables $(\boldsymbol{\mathcal{Z}}_*$,~$\boldsymbol{\mathcal{T}}$,~$\boldsymbol{\mathcal{Q}})$, which are required when solving the final problem Eq.(\ref{eq:unconstrained_optimization_problem}), we derive the gradient 
% 在这一章的新函数中,我们设计了z是T和z*的函数,
% 为了算到新变量的梯度,我们给出z到t的梯度
% 至此所有的梯度就都完成了
\begin{equation}
    \begin{aligned}
        % \frac{\partial{(\mathbf{z}_n)_0}}{\partial{T_n}}&=\frac{\partial{\mathcal{M}}}{\partial \hat{\mathbf{T}}_n}\frac{\partial \hat{\mathbf{T}}_n}{\partial{T_n}}+\frac{\partial \mathcal{M}}{\partial g_{n-1}}\frac{\partial g_{n-1}}{\partial T_n}+\frac{\partial \mathcal{M}}{\partial g_{n}}\frac{\partial g_{n}}{\partial T_n},\\
        % \frac{\partial{\mathbf{C}_n}}{\partial{T_{k}}}&=\frac{\partial \mathcal{M}}{\partial g_{n-1}}\frac{\partial g_{n-1}}{\partial T_k}+\frac{\partial \mathcal{M}}{\partial g_{n}}\frac{\partial g_{n}}{\partial T_k},~\forall k \in \{1,\dots,n-1\},\\
        \frac{\partial{(\mathbf{z}_n)_0}}{\partial T_k}=\frac{\partial g_{n}(t_n)}{\partial t_n}\frac{\partial (t_0+\sum\nolimits_{j=1}^{n}T_j)}{\partial T_k}=\left\{
        \begin{aligned}
            &0,&n < k\\
            &\dot{g}_{n}(t_n),&n \geq k
        \end{aligned}
        \right. .
    \end{aligned}
\end{equation}

% \begin{equation}
% \begin{aligned}
%     \frac{\partial{\mathbf{C}_n}}{\partial{T_n}}&=\frac{\partial{\mathcal{M}}}{\partial \hat{\mathbf{T}}_n}\frac{\partial \hat{\mathbf{T}}_n}{\partial{T_n}}+\frac{\partial \mathcal{M}}{\partial g_{n-1}}\frac{\partial g_{n-1}}{\partial T_n}+\frac{\partial \mathcal{M}}{\partial g_{n}}\frac{\partial g_{n}}{\partial T_n},\\
%     \frac{\partial{\mathbf{C}_n}}{\partial{T_{k}}}&=\frac{\partial \mathcal{M}}{\partial g_{n-1}}\frac{\partial g_{n-1}}{\partial T_k}+\frac{\partial \mathcal{M}}{\partial g_{n}}\frac{\partial g_{n}}{\partial T_k},~\forall k \in \{1,\dots,n-1\},\\
%     \frac{\partial g_{x}}{\partial T_y}&=\frac{\partial g_{x}(t_x)}{\partial t_x}\frac{\partial (t_0+\sum\nolimits_{j=1}^{x}T_j)}{\partial T_y}=\left\{
%     \begin{aligned}
%         &0,&x < y\\
%         &\dot{g}_{x}(t_x),&x \geq y
%     \end{aligned}
% \right. ,
% \end{aligned}
% \end{equation}
% \begin{equation}
% \frac{\partial g_{x}}{\partial T_y}=\frac{\partial g_{x}(t_x)}{\partial t_x}\frac{\partial (t_0+\sum\nolimits_{j=1}^{x}T_j)}{\partial T_y}=\left\{
%     \begin{aligned}
%         &0,&x < y\\
%         &\dot{g}_{x}(t_x),&x \geq y
%     \end{aligned}
% \right. .
% \end{equation}

\subsection{Implementation Details}
\label{subsec:Implement_Details}
\subsubsection{Parallel Optimization for Different Topologies}
% 虽然动态障碍物在很多工作中被处理,但是他们忽略了动态障碍物带来的空间拓扑分割
% 不同拓扑的选择对racing的执行速度结果有着明显的影响
% 如图所示,我们将基于动态障碍物的速度方向,在其周围构建几条不同拓扑的轨迹作为初值,让他们并行优化,最终选择效果最好的一条
Although dynamic obstacles are addressed in many works~\cite{wang2021autonomous,chen2022rast}, they only care about obstacle avoidance, ignoring the spatial topology segmentation brought by dynamic obstacles.
The choice of different topologies has a significant impact on the execution time of racing, as demonstrated in Sec.~\ref{sec:Evaluations:Ablation:topo}.
Therefore, in this paper, as shown in Fig.~\ref{fig:xiaorong3}, we generate several trajectories with different topologies split by the dynamic obstacle.
Then we use them in parallel as initial values for trajectory optimization and finally choose the one with the shortest flight time for the drone to execute.


\subsubsection{Numerical Optimization}
We adopt L-BFGS\footnote{https://github.com/ZJU-FAST-Lab/LBFGS-Lite}~\cite{liu1989limited} to solve the unconstrained optimization problem Eq.(\ref{eq:unconstrained_optimization_problem}).

\subsubsection{Global Planning}
% 对于一些较大的场景,如实验中提到的dji和benchmark,我们会先使用一段minco轨迹生成一条全局参考轨迹,然后在无人机飞行的时候以全局轨迹为初值进行局部的在线重规划
For some large-scale scenarios, such as Sec.~\ref{sec:Evaluations:dji}, we first generate a global reference trajectory using a one-segment MINCO trajectory, and then use part of the global trajectory as the initial value for the optimization of the local online replanning during flight.


\section{Evaluations}
\label{sec:Evaluations}
% 本节中,我们分别设计了三个消融实验来验证我们我们contrition的有效性
% 同时,由于现在缺乏在这样动态挑战性场景中的工作有能拿到的代码,因此我们通过一个比赛来验证这个方法的实用性
% 所有的实验都在什么电脑上完成
In this section, to verify the effectiveness of our contributions summarized in Sec.~\ref{sec:Introduction}, we design three ablation experiments.
Then to validate the practicality of our planning method, we integrate the method into a complete quadrotor system, which is applied to 2022 DJI Robomaster Intelligent UAV Championship\textsuperscript{\ref{dji_bisai_web}}.
All the simulation experiments are run on a desktop equipped with
an Intel Core  i7-10700 CPU.

\subsection{Ablation Experiments}
\label{sec:Evaluations:Ablation}

\subsubsection{Evaluation for Efficiency Improvement of Time-uniform MINCO}
\label{sec:Evaluations:Ablation:minco}
% 我们对比minco和tnminco的计算效率
% 由于minco没有段数,为了公平对比,对比NsegMpiece的均匀minco,我们用MNpiece的minco
% 对两个方法,我们分别测试了segment数量N={2,10},同时varypiece数量从,M从2到5变化。
% 每次对比时两个方法给相同的轨迹初值。
% 最终结果如图所示,我们将单次迭代所化的时间均匀所花的时间用不同颜色的柱状图表示,minco所花的时间用带透明度的柱状图表示。需要注意的是,minco所画的时候是从灰色0平面开始,透明柱子的高度
% 同时我们在每个柱状图上写出了均匀minco相比minco的时间提升比例
% 可以看出使用均匀minco有效的提升了时间效率

We compare the computational efficiency of MINCO and time-uniform MINCO with the same segment and piece number.
For both of them, we test the number of segments $N$ varied from $2$ to $5$, while the number of pieces $M=\{2,4,\dots,10\}$.
Both methods are given the same initial trajectory for each comparison.
The results are illustrated in Fig.~\ref{fig:xiaorong1}, where the time of a single iteration of the trajectory optimization is visualized.
We represent the time spent by time-uniform MINCO with different colored bars.
The transparent part indicates the additional time required for MINCO compared to time-uniform MINCO.
Additionally, we show the percentage of time reduction of time-uniform MINCO over MINCO on each bar.

As shown in Fig.~\ref{fig:xiaorong1}, using time-uniform MINCO effectively improves the calculation efficiency, even if the quantity of segments or pieces varies.

\begin{figure}[!t]
	\centering
    \vspace{0.0cm}
	\includegraphics[width=1\linewidth]{figures/xiaorong1.pdf}
    \captionsetup{font={footnotesize}}
	\caption{
        Illustration of the time of a single iteration in the trajectory optimization.
        The transparent area represents the more computation time that MINCO takes than the time-uniform MINCO.
	}
	\label{fig:xiaorong1}
    \vspace{-1.4cm}
\end{figure}

\begin{figure}[htbp]
	\centering
    % \vspace{-0.2cm}
	\includegraphics[width=1\linewidth]{figures/xiaorong2_traj.pdf}
    \captionsetup{font={footnotesize}}
	\caption{
        Illustration of the top view of the results in Sec.~\ref{sec:Evaluations:Ablation:hard}.
        The short thick black lines represent the static gates and the short thick red line represents the dynamic gate.
        The circle on the line represents the center of the gate.
        The red bar on the red line indicates the movement range of the dynamic gate.
        \textbf{(a)} demonstrates the trajectories of the quadrotor completing the track under different soft constraint weights and the same time weight $\rho=10^5$.
        \textbf{(b)}-\textbf{(e)} represent the moments for the quadrotor to pass through the dynamic gate with different parameters.
	}
	\label{fig:xiaorong2_traj}
    % \vspace{-0.8cm}
\end{figure}

\subsubsection{Evaluation for constraints for Precise Gate Crossing}
\label{sec:Evaluations:Ablation:hard}
% 我们对比了常用的软约束方法,和我们在这一章提出的方法
% 设置了如图所示的场景,每次轨迹规划给相同的初值
% 我们分别给了不同的时间权重系数和软约束的权重系数进行实验,实验结果如图所示
% 可以看到软约束方法挣扎于找到比较好的参数来权衡尽可能短的执行时间和精准的穿门。
% 换句话说,软约束方法在左图中选择绿色的能够精准穿越的参数组时就会导致其在右图飞行时间上效果很差,反之,当其在右图中选择绿色的尽可能短的飞行时间时,对应左图的穿门需求就很难被精准执行
% 相比之下,我们的硬约束方法,只需要关注在调整时间参数来缩短执行时间即可,能够保持精准穿越
We compare our method proposed in Sec.~\ref{sec:gate_con} with a commonly used soft constraint method which uses penalty function $\lambda_{gate}\|\sigma(t_{n})-g_n(t_{n})\|^2$, where $\lambda_{gate}$ is the weight.
The experiment scenario is set up as shown in Fig.~\ref{fig:xiaorong2_traj}.\textbf{(a)}, where the quadrotor is required to pass through $5$ gates in order.
The red dynamic gate moves at $2$ m/s horizontally back and forth from side to side within a certain range, which is indicated by the red bar in Fig.~\ref{fig:xiaorong2_traj}.\textbf{(a)}.
To compare fairly, each trajectory optimization is given the same initial value and we set other parameters that are not about this experiment the same.
We set different time weight $\rho$ and soft constraint weights $\lambda_{gate}$ for the experiments.


The results are illustrated in Fig.~\ref{fig:xiaorong2}, soft constraint method struggles to find suitable parameters to trade off the shortest possible execution time with precise gate crossing.
To elaborate, when choosing the parameters that achieve a small distance to gate as shown in the green area of Fig.~\ref{fig:xiaorong2}.\textbf{(a)}, the soft constraint method causes poor results in terms of the time of flight as shown in the red area of Fig.~\ref{fig:xiaorong2}.\textbf{(b)}.
Conversely, when choosing the parameters that aim for the shortest possible flight time in the green or yellow area of Fig.~\ref{fig:xiaorong2}.\textbf{(b)}, the soft constraint method is difficult to precisely crossing the gate, which corresponds to the red area of Fig.~\ref{fig:xiaorong2}.\textbf{(a)}.
In contrast, our hard constraint approach only needs to focus on adjusting the time weight $\rho$ to shorten the execution time, while maintaining precise traversal.

\begin{figure}[tp]
	\centering
    % \vspace{0cm}
	\includegraphics[width=1\linewidth]{figures/xiaorong2.pdf}
    \captionsetup{font={footnotesize}}
    % \vspace{0.0cm}
	\caption{
        Results of experiments in Sec.~\ref{sec:Evaluations:Ablation:hard}, where time weight $\rho$ and soft constraint weight $\lambda_{gate}$ vary from $10^3$ to $10^5$ and $10^2$ to $10^4$, respectively.
        \textbf{(a)} shows the average distance from the trajectory to the corresponding $n$-th gate at the moment $t_n$, from far to near, the color changes from red to green.
        \textbf{(b)} displays the flight time with different parameters, from less to more, and the color varies from green to red.
        % 到门的距离按照从远离到精准,颜色从红到绿
        % 飞行时间从快速到缓慢,颜色从绿到红
	}
	\label{fig:xiaorong2}
    \vspace{-1cm}
\end{figure}

To more visually demonstrate the importance of accurate traversal, in Fig.~\ref{fig:xiaorong2_traj}, we visualize the trajectories of our method and the soft constraint method when the time weight $\rho=10^5$.
The results show that, compared to the soft constraint method, our method guarantees accurate traversal for both dynamic and static gates.
% 解释一下,可以看到什么什么的咋样咋样的

\subsubsection{Evaluation for Choosing Different Topologies Segmented by Dynamic Obstacles}
\label{sec:Evaluations:Ablation:topo}
% 我们使用了一个在比较典型的场景去验证这个工程考虑的作用
% 如图所示,图a表示了测试场景,无人机被要求从起点出发,按顺序穿过两个门,然后达到终点。
% 两个门之间有一个动态障碍物,我们将其建模成椭球。其中心位置关于时间的变化我们用一条带颜色的轨迹来表示
% 我们将从障碍物四周经过不同拓扑的轨迹作为初值进行优化,最终结果如图所示。
% 可以看到,加粗的轨迹明显快于剩下拓扑的轨迹
% 同时我们还将轨迹时间展示在了下表中,可以看到选择不同的拓扑对执行时间有着很大的影响,这也应征了我们这里的工程处理是有意义的
We use a typical scenario to verify the usefulness of this engineering consideration.
As shown in Fig.~\ref{fig:xiaorong3}.\textbf{(a)}, the quadrotor is required to take off from the start point, pass through two gates in order, and then reach the end point.
Between the two gates, there is a dynamic obstacle which we model as an ellipsoid. 
The movement of its central position with respect to time we represent by a colored trajectory in Fig.~\ref{fig:xiaorong3}.\textbf{(a)}.
We set the trajectories through different topologies past the obstacle as initial values.
The final trajectories after optimization are illustrated in Fig.~\ref{fig:xiaorong3}.\textbf{(b)},  the bolded trajectory is significantly faster than the trajectories of other topologies.
Meanwhile, we present the execution time of each trajectory in Tab.~\ref{tab:topo}, where the choice of different topologies has a considerable impact on the execution time, which confirms the validity of our evaluating and choosing different topologies segmented by the dynamic obstacle.

\begin{figure}[!t]
	\centering
    \vspace{0.0cm}
	\includegraphics[width=1\linewidth]{figures/xiaorong3.pdf}
    \captionsetup{font={footnotesize}}
	\caption{
        Illustration of the results in Sec.~\ref{sec:Evaluations:Ablation:topo}.
        \textbf{(a)} demonstrates the experiment setup.
        \textbf{(b)} shows the trajectories optimized based on initial values from different topological.
	}
	\label{fig:xiaorong3}
    % \vspace{-0.8cm}
\end{figure}


\begin{table}[!t]
	\vspace{-0.2cm}
	\centering
	\caption{Trajectories of Different Topologies} 
    \tabcolsep=0.35cm 
	\begin{tabular}{c|cccc}
		\toprule
		Traj number & 1 & 2 & 3 & 4  \\
		\midrule
		execution time  (s) & 3.82 & \bf 3.79 & 5.60 & 4.38   \\
		trajectory length (m) &  20.41 & \bf 20.05 & 22.97 & 20.64  \\
		\bottomrule
	\end{tabular}
	\label{tab:topo}
	\vspace{-1.2cm}
\end{table} 

\subsection{2022 DJI Robomaster Intelligent UAV Championship}
\label{sec:Evaluations:dji}
% 比赛规则可以这里引用一下,然后详细介绍一下比赛规则,并且把图一一对应,说明每个阶段是在干什么
We take part in the second event Autonomous Racing of this competition, where the quadrotor is required to fly in a dynamic and challenging environment as fast as possible.
The racing track is about $160$ m long,  which contains a series of static gates (as shown in Fig.~\ref{fig:bisai}.\textbf{(a)}), moving obstacles, dynamic gates with different movement patterns (as illustrated in Fig.~\ref{fig:toutu}), and a shaped gate which demands to plan SE(3) trajectory (as shown in Fig.~\ref{fig:bisai}.\textbf{(b)}).
The radius of the gates is $1$ m.
The positions of the gates are provided ahead of each race up to approximately $2$ m uncertainty.
This requires the drone to be able to detect gates and make trajectory adjustments in real time, such as online replanning based on the detection results.

\begin{table}[H]
	% \vspace{-0.2cm}
	\centering
	\caption{Parameters of Planning} 
    \tabcolsep=0.25cm 
	\begin{tabular}{cccccccc}
		\toprule
		$N$ & $M$ &  $m$ & $\lambda_{t}$ & $\lambda_{b}$ & $\lambda_{g}$ & $\lambda_{d}$  \\
		\midrule
		~~2~~ & ~~4~~ & ~~3~~ & ~100~ & 100 & 10000 & 10000   \\
		\bottomrule
	\end{tabular}
	\label{tab:planning}
	\vspace{-0.1cm}
\end{table} 

In the competition, we integrate the proposed framework into a customized quadrotor system, combining localization, detection, and control modules.
For the planning module, the parameters defined in Sec.~\ref{sec:Online_Replan} of trajectory optimization are shown in Tab.~\ref{tab:planning}.
The average time overhead of replan is $16.6$ ms when considering SE(3) for narrow gaps and $7.0$ ms when not considering.
We opt for VINS-MONO~\cite{qin2018vins} as the localization module.
Then we use the point cloud from the depth image for the detection of gates and dynamic obstacles.
Finally, we adopt MPC~\cite{faessler2017differential} as the control method to track the planned aggressive trajectory.


We visualize the flight of our system in the competition in Fig.~\ref{fig:toutu} and \ref{fig:bisai}.
Readers can get a better understanding of the experiment from the attached video.
Additionally, we show the final results and rankings in Tab.~\ref{tab:dji}, where our system is significantly faster than the other teams, demonstrating the great performance of our method for racing in dynamic environments.
% 可以看到我们的方法能够快速穿越一系列门,灵巧躲避动态障碍物,精准穿越动态门,最后以大姿态穿越了窄缝门
% It is worth mentioning that in Fig.~\ref{fig:bisai} the drone does not pass through the center of the gate.
% The reason is that there is a certain error in detection and control modules.
% 最终结果


% \begin{table}[!t]
% 	\vspace{0.1cm}
% 	\centering
% 	\caption{Ranking of Competition Results} 
%     \tabcolsep=0.35cm 
% 	\begin{tabular}{c|c}
% 		\toprule
% 		rankings & completion time (s)  \\
% 		\midrule
% 		\bf our team &  \bf 22.0  \\
% 		2nd place &  50.3   \\
% 		3rd place &  76.9   \\
% 		\vdots &  \vdots   \\
% 		\bottomrule
% 	\end{tabular}
% 	\vspace{-0.6cm}
% 	\label{tab:dji}
% \end{table} 

\begin{figure}[!t]
	\centering
    \vspace{0.2cm}
	\includegraphics[width=1\linewidth]{figures/bisai.pdf}
    \captionsetup{font={footnotesize}}
	\caption{
        Illustration of the flight of our system in the 2022 DJI Robomaster Intelligent UAV Championship.
	}
	\label{fig:bisai}
    \vspace{-0.9cm}
\end{figure}

% 1. dji是一个什么什么比赛(参考一下drone challenge)
% 比赛的规则,特别是环会有2m的随机距离,飞机的大小和环的大小半径m
% 2. 把规则需要做的内容总分的方式写。
% 说一下我们系统方面做了啥,两三句话就行。用啥模块干了啥之类的
%   比赛中,我们使用本文提出的轨迹规划方法,同时结合了vins作为实时定位模块,以颜色结合深度图实现对门和动态障碍物的识别,同时搭建高效的mpc控制器
% 每次replan的时间
% 说明一下每次我们规划多长的轨迹,还有一个就是说明一下飞起来的误差是什么导致的
% 然后依照每个图的顺序说明一下情况,最后贴一下我们的成绩很好,验证了什么


\begin{table}[H]
	% \vspace{-0.2cm}
	\centering
	\caption{Ranking of Competition Results\textsuperscript{\ref{foot_rankings}}} 
    \tabcolsep=0.25cm 
	\begin{tabular}{c|cccc}
		\toprule
		rankings & \bf our team & 2nd place & 3rd place & \bf \dots  \\
		\midrule
		completion time (s) & \bf 22.0 &  50.3 &  76.9 & \bf \dots  \\
		\bottomrule
	\end{tabular}
	\vspace{-0.2cm}
	\label{tab:dji}
\end{table} 

\section{Conclusion}
% 这篇文章中我们针对intro中提出的在动态环境下规划的四点需求,设计了一套基于多项式的轨迹规划方法
% 从轨迹表征、穿环硬约束设计和并行评估不同拓扑下的轨迹再选择三个方面做出贡献,有效的提升了规划效率、实现了精准穿环和面对动态障碍物时的飞行时间
% 最后将方法应用到dji比赛上,极佳的成绩印证了方法的优越新能

In this paper, we propose a polynomial-based trajectory planning method to address the $4$ requirements for racing in dynamic environments presented in Sec.~\ref{sec:Introduction}.
Efforts in trajectory representation, hard constraint designed for crossing waypoints, and parallel evaluation of trajectory under different topologies, effectively improve replan efficiency, the accuracy of waypoint traversal, and flight time when facing dynamic obstacles.
Finally, the method is applied to the DJI competition, and the outstanding result proves the good performance of our method.

\bibliography{references}
\end{document}