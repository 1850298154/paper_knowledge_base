\documentclass{article}


% if you need to pass options to natbib, use, e.g.:
%     \PassOptionsToPackage{numbers, compress}{natbib}
% before loading neurips_2025


% ready for submission
\usepackage[preprint]{neurips_2025}


% to compile a preprint version, e.g., for submission to arXiv, add add the
% [preprint] option:
%     \usepackage[preprint]{neurips_2025}


% to compile a camera-ready version, add the [final] option, e.g.:
%     \usepackage[final]{neurips_2025}


% to avoid loading the natbib package, add option nonatbib:
%    \usepackage[nonatbib]{neurips_2025}


\usepackage[utf8]{inputenc} % allow utf-8 input
\usepackage[T1]{fontenc}    % use 8-bit T1 fonts
\usepackage{hyperref}       % hyperlinks
\usepackage{url}            % simple URL typesetting
\usepackage{booktabs}       % professional-quality tables
\usepackage{amsfonts}       % blackboard math symbols
\usepackage{nicefrac}       % compact symbols for 1/2, etc.
\usepackage{microtype}      % microtypography
\usepackage{xcolor}         % colors

\usepackage{amsmath}
\usepackage{array}
\usepackage{graphicx}



\title{PestMA: LLM-based Multi-Agent System for Informed Pest Management}


% The \author macro works with any number of authors. There are two commands
% used to separate the names and addresses of multiple authors: \And and \AND.
%
% Using \And between authors leaves it to LaTeX to determine where to break the
% lines. Using \AND forces a line break at that point. So, if LaTeX puts 3 of 4
% authors names on the first line, and the last on the second line, try using
% \AND instead of \And before the third author name.


\author{%
  Hongrui Shi \\
  % \thanks{Use footnote for providing further information
  %   about author (webpage, alternative address)---\emph{not} for acknowledging
  %   funding agencies.} \\
  School of Computer Science\\
  University of Sheffield\\
  Sheffield S10 2TN, UK \\
  \texttt{alex.yalishanda@gmail.com} \\
\And
  Shunbao Li \\
  School of Computer Science\\
  University of Sheffield\\
  Sheffield S10 2TN, UK \\
  \texttt{leeshunbao@gmail.com} \\
\And
  Zhipeng Yuan \\
  School of Computer Science\\
  University of Sheffield\\
  Western Bank, Sheffield S10 2TN, UK \\
  \texttt{zhipeng.yuan@sheffield.ac.uk} \\
\And
  Po Yang \\
  School of Computer Science\\
  University of Sheffield\\
  Sheffield S10 2TN, UK \\
  \texttt{po.yang@sheffield.ac.uk} \\
  % examples of more authors
  % \And
  % Coauthor \\
  % Affiliation \\
  % Address \\
  % \texttt{email} \\
  % \AND
  % Coauthor \\
  % Affiliation \\
  % Address \\
  % \texttt{email} \\
  % \And
  % Coauthor \\
  % Affiliation \\
  % Address \\
  % \texttt{email} \\
  % \And
  % Coauthor \\
  % Affiliation \\
  % Address \\
  % \texttt{email} \\
}


\begin{document}


\maketitle


\begin{abstract}
  Effective pest management is complex due to the need for accurate, context-specific decisions. Recent advancements in large language models (LLMs) open new possibilities for addressing these challenges by providing sophisticated, adaptive knowledge acquisition and reasoning. However, existing LLM-based pest management approaches often rely on a single-agent paradigm, which can limit their capacity to incorporate diverse external information, engage in systematic validation, and address complex, threshold-driven decisions. To overcome these limitations, we introduce PestMA, an LLM-based multi-agent system (MAS) designed to generate reliable and evidence-based pest management advice. Building on an editorial paradigm, PestMA features three specialized agents—an Editor for synthesizing pest management recommendations, a Retriever for gathering relevant external data, and a Validator for ensuring correctness. Evaluations on real-world pest scenarios demonstrate that PestMA achieves an initial accuracy of 86.8\% for pest management decisions, which increases to 92.6\% after validation. These results underscore the value of collaborative agent-based workflows in refining and validating decisions, highlighting the potential of LLM-based multi-agent systems to automate and enhance pest management processes. 
\end{abstract}


\section{Introduction}

Effective pest management is vital for maintaining agricultural productivity and safeguarding food security worldwide. Pests significantly reduce crop yields, pose health and environmental risks, and can lead to economic losses if not addressed promptly. Moreover, the escalating impacts of climate change, coupled with growing global demands for sustainable agriculture, call for adaptable and evidence-based strategies to prevent and mitigate pest infestations. Although various pest management approaches exist—including chemical controls, biological agents, and integrated pest management (IPM) programs—they often rely on specialized domain knowledge that is difficult to disseminate broadly or update regularly.

% What we do in this work and contributions
Recent advancements in large language models (LLMs) offer a novel means to consolidate and apply domain expertise more effectively. By leveraging these models, stakeholders can access comprehensive, up-to-date recommendations for informed pest management practices. In this paper, we introduce PestMA, a large language model (LLM)-based multi-agent system for generating reliable pest management advice. Our design draws on an editorial paradigm in which three agents—an Editor for synthesizing pest management advice (PMA), a Retriever for gathering relevant external knowledge to address the knowledge gaps in the Editor, and a Validator for checking PMA correctness—collaborate to address pest scenarios accurately. Evaluations on real-world pest data show that PestMA achieves 86.8\% accuracy for the synthesised pest management decision (PMD) and improves to 92.6\% following the validation from the Validator, demonstrating both the practical applicability of our approach and the enhancements gained from agent-based collaboration.

Our contributions are threefold:
\begin{itemize}
    \item Applying a Multi-Agent System to Pest Management: We propose PestMA, an LLM-based MAS explicitly tailored to address real-world pest management scenarios.
    \item Collaborative Agent Design for Pest Management: Our system features a unique Editor–Retriever–Validator workflow, illustrating how role specialization and collaboration enhance accuracy and reliability.
    \item Comprehensive PMD-Based Evaluation: We adopt the accuracy of the pest management decision (PMD) as a central metric to assess PestMA’s workflow and agent functionality, demonstrating that rigorous integration of external knowledge and logical validation significantly bolsters performance.
\end{itemize}


\section{Literature Review}

\subsection{LLM-based Multi-Agent Systems}

Recently, Large Language Models (LLMs)—such as GPT-4 \citep{achiam2023gpt}, Gemini-1.5 \citep{team2024gemini}, Llama-3 \citep{grattafiori2024llama}, Qwen-2.5 \citep{yang2024qwen2}, DeepSeek-R1 \citep{guo2025deepseek}—pretrained on substantial amounts of text have emerged as a universal foundation for addressing a broad range of AI tasks. To adapt these general-purpose LLMs to domain-specific contexts, Parameter-Efficient-Fine-Tuning (PEFT) techniques (e.g., LoRA \citep{hu2022lora}), along with prompt engineering \citep{sahoo2024systematic} and retrieval-augmented generation (RAG) \citep{lewis2020retrieval}, have achieved significant success. Driven by the remarkable capabilities demonstrated by LLMs, researchers have begun exploring their potential for undertaking more complex real-world applications, such as automated online purchases and research proposal generation.

This need has spurred the development of Agentic AI, referring to autonomous systems capable of accomplishing intricate tasks with minimal human intervention~\citep{bousetouane2025agentic, acharya2025agentic}. Owing to their strong language understanding and generative abilities, LLMs are increasingly viewed as pivotal performers of agentic AI. In particular, LLM-based multi-agent systems (MAS) have attracted attention for their capacity to exceed the capabilities of single-agent approaches~\citep{guo2024large, tran2025multi}. By enabling role specialization, supporting flexible interaction paradigms (e.g., collaboration or debate), and integrating diverse tools and skills, MAS can tackle more complex tasks than individual agents. This approach has proven effective in innovative applications such as Deep Research~\citep{openai2025deepresearch} by OpenAI—designed to generate entire research proposals—and Google’s AI co-scientist~\citep{gottweis2025ai}, which assists researchers by proposing novel scientific hypotheses. Hence, LLM-based MAS stands at the forefront of advanced agentic AI, demonstrating robust potential for tackling complex real-world tasks.

% Recently, Large Language Models (LLMs), e.g. GPT-4 \citep{achiam2023gpt}, Gemini-1.5 \citep{team2024gemini}, Llama-3 \citep{grattafiori2024llama}, Qwen-2.5 \citep{yang2024qwen2}, DeepSeek-R1 \citep{guo2025deepseek}, pretrained on a large amount of textual data has risen has a universal ground to approach a range of AI applications. To adapt LLMs pretrained on a general purpose to the domain-specific applications, fine-tuning, parameter-efficient fine-tuning (PEFT) techniques such as LoRA, prompt engineering, RAG, are proposed and have achieved massive success. With the amazing capability demonstrated by LLMs, frontiers are seeking to exploit LLMs to assist human in more complex scenarios with real-world impacts, e.g. automatic online purchases, research proposals. 

% This needs and incentives have pushed the emergence of Agentic AI, referring to an autonomous system that can achieve complex task with minimal human intervention \citep{bousetouane2025agentic, acharya2025agentic}. With the capability of LLMs, agnetic AI has been largely considered as the next big advancement in the approach to artificial general intelligence. Inside this realm, the LLM-based multi-agent systems (MAS) exhibits more potentials that single-agent systems~\citep{guo2024large, tran2025multi}. Compared to a single agent, multi-agent AI can tackle more complicated tasks by allowing versatile role specifications, flexible interaction paradigms, such as collaboration and debate, as well as diverse tools and skills. To this end, MAS has been put into applications in many fantastic cases, such as Deep Research \citep{openai2025deepresearch} developed by OpenAI that is capable to write an entire research proposal for researchers, and Google's AI co-scientist \citep{gottweis2025ai} that can assist scientists to generate novel hypotheses. 


\subsection{LLM-based Pest Management}
Despite the growing success of LLMs in various fields, efforts to leverage these models for informed pest management have been relatively limited \citep{shaikh2024role}. Notably, \cite{yang2024gpt} employs GPT-based models to produce pest management advice. Conversely, \cite{yuan2024pestgpt} mitigates inherent knowledge gaps by infusing domain-specific information from reliable sources when prompting LLMs. However, none of the existing works extend to an LLM-based multi-agent system (MAS) for pest management. It is precisely this gap that the present research aims to address, by introducing a multi-agent framework where LLMs collaborate to provide comprehensive, validated pest management solutions.

% Though LLMs have achieved significant success in various domains, the efforts of leveraging LLMs to achieve informed pest management remain limited~\citep{shaikh2024role}. Notably, the \citep{yang2024gpt} applies GPT models to generate pest management advice and \citep{yang2024gpt} overcomes the knowledge limitation on pest management of LLMs by prompting LLMs with pest management knowledge extracted from reliable sources. Nevertheless, none of these works have advanced the pest management into LLM-based MAS, which we will address in this work.  


\section{Methodology}
\label{sec: method}

We introduce PestMA (Pest Multi-Agents), an LLM-based multi-agent system structured around an editorial workflow, inspired by the processes typically found in news publication agencies. PestMA incorporates three distinct agents analogous to journalism roles:

\begin{itemize}
\item \textbf{Retriever (Journalist)}: Responsible for gathering relevant external data and documentation related to pest management.
\item \textbf{Editor}: Synthesizes the information provided by the Retriever into coherent, practical pest management advice (PMA).
\item \textbf{Validator (Managing Editor)}: Reviews and validates the synthesized advice to ensure credibility, reliability, and compliance with established standards and policies.
\end{itemize}

This structured editorial workflow ensures the integration of credible external knowledge and thorough validation processes, resulting in accurate, reliable, and actionable pest management recommendations. Figure~\ref{fig: workflow of pestma} illustrates the workflow of PestMA. The three agents mainly conduct the following tasks:

\paragraph{Initialising PMA.} Given an identified pest scenario, the Editor is tasked with generating an initial pest management advice (PMA). To ensure logical coherence and consistency, the Editor receives an initial prompt containing a PMA template generated by a reasoning model trained with Chain-of-Thought~\citep{wei2022chain} methods (e.g., GPT-o1~\citep{openai2024o1}). This template serves as structured guidance, facilitating logical analysis and eliciting the intrinsic knowledge of the Editor to produce a coherent initial PMA.

\paragraph{Customising PMA.} Once the initial PMA is generated, the Retriever identifies knowledge gaps that cannot be sufficiently addressed by the Editor’s intrinsic knowledge, such as local pesticide resistance levels. To address these gaps, the Retriever searches external sources, including relevant documentation and credible websites, to gather supplementary information tailored specifically to the pest scenario. The Retriever then summarizes these findings and returns them to the Editor, who subsequently integrates this external knowledge with the initial PMA. This integration results in a customised PMA that combines pretrained knowledge from the LLM with precise external insights, providing tailored advice specific to the user's scenario.

\paragraph{Validating PMA.} The Validator assesses the customised PMA to ensure its accuracy, credibility, and reliability. The Validator critically reviews the customised PMA produced by the Editor, identifying potential errors or inconsistencies, such as incorrect pest management decisions. Unlike the Retriever, who seeks external information to address knowledge gaps, the Validator utilizes external sources primarily for cross-validation and verification purposes. Thus, the Validator ensures robust validation of the customised PMA, enhancing the overall quality and reliability of the generated pest management advice.


\begin{figure*}[th]
% \captionsetup{width=.9\linewidth}
  \begin{center}
    \includegraphics[width=0.9\linewidth]{pestma_workflow.png}
  \end{center}
  \caption{The workflow of PestMA. Three agents--Editor, Retriever, and Validator--collaboratively generate the pest management advice given a pest management scenario requested by the user. Editor is responsible for synthesising the pest management advice, Retriever is tasked to find knowledge gaps in the PMA and search relevant information to fill the knowledge gaps, and the Validator accesses the PMA to ensure its accuracy and reliability.}
  \label{fig: workflow of pestma}
\end{figure*}

% -------------------------------------------------------------------------------------
% Original contents
% -------------------------------------------------------------------------------------

% We construct the multi-agent system for the pest management with an editorial structure. We draw inspiration from the publication of an article at a news agency, where typically involves three roles. An editor who is responsible to write the article, a journalists who is responsible to gather relevant data or documents, and a managing editor provides final approval to ensure the article aligns with the newspaper's standards and policies. We define the Editor, Retriever, and Validator in our MAS to alike the roles in the news publication. The designed MAS ensures credible external knowledge and are integrated into the generation process of the pest management advice. 

% We designed a multi-agent system (MAS) for pest management advice structured around an editorial workflow, inspired by the processes typically found in news publication agencies. Our MAS includes three distinct roles analogous to those in journalism: 1) Retriever: his agent is responsible for gathering relevant external data and documentation pertinent to pest management. 2) Editor: Utilizing information provided by the Retriever, the Editor agent synthesizes the data into coherent, practical pest management advice. 3) Validator: Ensuring credibility and reliability, the Validator agent reviews and approves the advice, confirming alignment with established standards and policies. This structured editorial workflow in our MAS ensures that credible external knowledge and comprehensive validations are effectively integrated into generating accurate, reliable, and actionable pest management recommendations.

% \paragraph{Initial PMA.} Specifically, the Editor is responsible to provide the pest management advice (PMA) given an identified pest scenario. To ensure a logical and consistent PMA, in the initial prompt to the Editor, we introduce a PMA template generated by a reasoning model that is trained with Chain-of-Thought, e.g. GPT-o1 model, and ask the Editor to follow the pattern of the template to generate an initial PMA with its intrinsic knowledge. The PMA template acts as a guidance for the Editor to perform logical analysis in the initial PMA.

% \paragraph{Customised PMA.} With the initial PMA, the Retriever kicks in to examine the knowledge gaps in the initial PMA that cannot be filed by the intrinsic knowledge from the LLM, e.g. the resistance level to the proposed pesticide in the region. Therefore, the Retriever will seek external knowledge, such as relevant documentations and websites to tailor the initial PMA to be specific on the request pest management scenario. Then the Retriever will return the Editor with a summary of the findings from external search in accordance with the identified knowledge gaps. The Editor then will integrate the summary from the Retriever to customise the initial PMA by combining pretrained knowledge from LLM and knowledge grabs from external sources. The customsied PMA will provide the user with tailored advice on the given request.

% \paragraph{PMA validation.} The Validator checks the customised PMA to ensure its credibility and reliability. Notably, the Validator will review the customised PMA generated by the Editor, aiming to find its flaws, e.g. the correctness of the pest management decision. Furthermore, Validator is allowed to refer to external knowledge to enhance its effectiveness of validation rather than relying on its internal knowledge, the difference between the Retriever and the Validator when using external knowledge is the context. The Validator uses the external knowledge to verify the customised PMA, whereas the Retriever uses the external knowledge to fill the knowledge gaps identified in the initial PMA. 


\section{Experiments and Results}
\subsection{Dataset and Evaluation}
% -------------------------------------------------------------------------------------
% Dataset description, Shunbao (start)
% -------------------------------------------------------------------------------------
We evaluate the proposed PestMA on a dataset comprising 68 distinct pest management scenarios, covering 39 different pest species prevalent in the United Kingdom. Each scenario in the dataset consists of attributes of an identified pest scenario. In addition, a pest management decision (PMD) indicating if this scenario needs immediate pest management actions is attached to every scenario. The attached PMD is calculated by an expert system of pest management. Table \ref{tab: pest management scenario} depicts one sample of this dataset.


\begin{table}
  \caption{Attributes of an exemplary pest management scenario used in the PestMA evaluation}
  \label{tab: pest management scenario}
  \centering
  \begin{tabular}{ll}
    \toprule
    Attribute     & Value     \\
    \midrule
    Pest & Beet Cyst Nematode      \\
    Infestation Severity     & 1 egg and larvae per gram of soil     \\
    Crop Name     & Sugar Beet        \\
    Crop Growth Stage     & Seedling        \\
    Temperature & $15^\circ\text{C}$ \\ 
    Weather & Overcast \\ 
    Humidity & 75\% \\
    Precipitation & 20mm \\
    Time & April \\ 
    Location & Lincolnshire \\
    Pest Management Decision &  False \\

    \bottomrule
  \end{tabular}
\end{table}
% -------------------------------------------------------------------------------------
% Dataset description, Shunbao (end)
% -------------------------------------------------------------------------------------

As described in Section~\ref{sec: method}, we additionally introduce a PMA template to prompt PestMA. The PMA template used in our experiments is generated by the GPT-o1 model, a reasoning-oriented model trained using the CoT prompt combined with reinforcement learning. Guided by this PMA template, the initial PMA provided by the Editor agent addresses multiple aspects essential to effective pest management, including the pest management decision (PMD), integrated pest management (IPM) strategies, economic considerations, application timing, and post-treatment monitoring.

To evaluate PestMA systematically and efficiently, we adopt an ablation study approach by instructing the PestMA, specifically the Retriever and Validator, to focus on a single aspect in the initial PMA for customisation and validation. Focusing on one aspect prevents overwhelming the agents with excessive contextual complexity, thus reducing potential errors or hallucinations in agent responses. Additionally, concentrating on a single aspect significantly decreases the costs associated with extensive external information retrieval, such as frequent API calls on online search services provided by third parties.

% Rather than asking the MA to enhance and safeguard all the aspects, we follow an ablation study approach, which means we will particularly ask the MA to focus on one of these aspects. This approach, on one hand, avoids complicating the internal context among the agents, e.g. a lengthy summary on the retrieved information covering all knowledge gaps in the entire PMA from the Retriever, therefore reducing the difficulties of the agent for managing and responding on a massive context, reducing possible errors and hallucinations. On the other hand, this approach is cost effective. More aspects mean more possible knowledge gaps, resulting in substantial external knowledge search, e.g. multiple API calls of the online search tool. By asking MA to focus on one aspect, we can significantly reduce the cost on money and time. 

Notably, we choose the pest management decision (PMD) as the focused aspect. PMD specifically involves a binary decision (true or false) indicating whether immediate action is necessary in response to the identified pest scenario requested by the user. Table~\ref{tab: pmd} describes how PMD is determined within PestMA. The decision is based on comparing infestation severity to an established threshold value. If the infestation severity exceeds this threshold, immediate management actions are advised, resulting in a true PMD; otherwise, the PMD is false. Accurate PMD decisions are crucial because inappropriate actions could negatively impact pest management effectiveness and cost-efficiency, thereby affecting PestMA’s credibility with users.

% Specifically, PestMA compares the infestation severity with a threshold that serves as an oracle. When the infestation severity exceeds the threshold, immediate actions are needed, therefore PMD is true, vice verse. PMD is a critical advice in the PMA as the immediate actions affect the effectiveness and cost of pest management. Incorrect PMD could significantly harm the credibility of PestMA to users.  

Evaluating PestMA based on the PMD allows us to assess both the collaboration among agents and the individual performance of each agent. Importantly, the threshold value is intentionally omitted from the initial prompts. Consequently, PestMA must independently retrieve this threshold from reliable external knowledge sources, such as data provided by the Agriculture and Horticulture Development Board (AHDB) or the British Crop Production Council (BCPC). The Editor leverages intrinsic knowledge to initially estimate PMD, while the Retriever locates and retrieves the exact threshold from external databases. Finally, the Validator ensures the correctness of the comparison made by the Editor, thus verifying the reliability of the final PMD.

% Evaluating PestMA on PMD can evaluate both the collaboration of agents and functionality of individual agent. The threshold for deciding PMD is not provided in the prompts of PestMA, instead we let PestMA to find this information by itself. The threshold is an established criterion for pest management that can be found on credible pest management database, e.g. Agriculture and Horticulture Development Board (AHDB), British Crop Production Council (BCPC). Therefore, PestMA needs to exercise both its internal and external knowledge to acquire the threshold, and then making reasonable judgment based on the requested scenario. Particularly, the Editor will be elicited its internal knowledge to provide an initial judgment on PMD in the initial PMA, the Retriever will seek the threshold through external sources and provide it to the Editor for integration of the customised PMA, the Validator will further check whether the Editor has compared the threshold with the user-provided scenario correctly. Therefore, PMD is an effective aspect to evaluate PestMA. 

Given that PMD is a binary classification task, we utilize accuracy as our evaluation metric. Accuracy is calculated using the following formula:

\begin{equation}
    \text{Accuracy} = \frac{\text{Number of Correct Predictions}}{\text{Total Number of Predictions}}
    \label{eq: accuracy metrics}
\end{equation}

\begin{table}
  \caption{Illustration of determining Pest Management Decision (PMD) based on infestation severity relative to an established threshold}
  \label{tab: pmd}
  \centering
  \begin{tabular}{ll}
    \toprule
    Description     & Value     \\
    \midrule
    Pest & Beet Cyst Nematode      \\
    Crop Name     & Sugar Beet        \\
    Crop Growth Stage     & Seedling        \\
    Infestation Severity     & 1 egg and larvae per gram of soil     \\
    Threshold & 2 eggs and larvae per relevant soil volume (source: AHDB) \\ \midrule
    PMD & False \\

    \bottomrule
  \end{tabular}
\end{table}


\subsection{Agent and Task Profiling}

We developed PestMA using the CrewAI framework, which enables explicit agent and task profiling. Table \ref{tab: agent profiling} presents an overview of each agent’s profile and the tasks assigned to them. Tables \ref{tab: task profiling 1} and \ref{tab: task profiling 2} outline the specific tasks within PestMA and detail the tools that support these tasks. The profiling of the agents and tasks define the collaborative workflow for PestMA described in Section~\ref{sec: method}, with a particular focus on the PMD aspect in the PMA. 


\begin{table}
\caption{Agent profiles and their assigned tasks in PestMA}
\label{tab: agent profiling}
\centering
\begin{tabular}{
    >{\centering}m{1.5cm}  % Centered and bold header for "Arch."
    >{}m{6cm}    % Centered and bold header for "Definition"
    >{\centering\arraybackslash}m{5cm}    % Centered and bold header for "Advantages"
}
\toprule
Agent & \centering Profile & Related Tasks \\
\midrule
Editor & You are a agronomist with exception knowledge about pest management. Coordinate and synthesize information to generate both initial and final pest management advice (PMA) to the queried pest identification scenario.  & 
\begin{itemize}
  \item Generate initial PMA.
  \item Generate customised PMA. 
\end{itemize} \\
\addlinespace
Retriver & Review the initial PMA created by the Editor and suggest a customisation plan to enhance the initial PMA. With the customisation plan, retrieve detailed online information, using the provided search queries and recommended online sources. & 
\begin{itemize}
  \item Make customisation plan. 
  \item Knowledge retrieval. 
\end{itemize} \\
\addlinespace
Validator & Critically review and validate the customised PMA, with a specific emphasis on verifying that the threshold conclusion is accurately derived from the diagnostic data. Leverage both your intrinsic expertise and external online search tool to ensure that the decision about whether the pest infestation exceeds the action threshold is scientifically sound and consistent with industry guidelines. & 
\begin{itemize}
  \item Validate threshold. 
\end{itemize} \\
\bottomrule
\end{tabular}
\end{table}


\begin{table}
\caption{Task definitions and the tools used by each agent (Part 1)}
\label{tab: task profiling 1}
\centering
\begin{tabular}{
    >{\centering}m{2cm}  % Centered and bold header for "Arch."
    >{}m{6cm}    % Centered and bold header for "Definition"
    >{\centering\arraybackslash}m{3cm}    % Centered and bold header for "Advantages"
}
\toprule
Agent & \centering Profile & Related Tools \\
\midrule
Generate initial PMA & Generate comprehensive pest management advice (PMA) on a pest identification scenario queried by the user. The request scenario is detailed in a JSON file at \{query\_path\}.
    To generate the PMA to the queried scenario, an exemplary scenario and a PMA template to this exemplary scenario are provided for references.
    Particularly, the exemplary scenario is detailed in the same format to the query scenario and can be read at \{example\_path\}. Read it first.
    The PMA template to the exemplary scenario is a markdown file and can be read at \{example\_pma\_path\}. Read this PMA after reading the exemplary scenario.
    The generated PMA to the queried scenario should follow the pattern of the exemplary PMA template.
    The task involves reading JSON files with JSON reader and the markdown file using MDReader, parsing their sections (pest identification, control methods, preventative measures, and environmental considerations), and populating them with detailed, queried PMA.  & 
\begin{itemize}
  \item JSON reader.
  \item Markdown reader.
\end{itemize} \\
\addlinespace
    Make customisation plan & Review the initial PMA saved as a Markdown file at \{initial\_pma\_path\} and identify the gaps in the pest management decision, which could be filled by searching online knowledge,
    therefore leading to a customised PMA based on the queried scenario.
    Compile the identified gaps into sections. For each section, list the necessary search queries and recommended online sources,
    along with a justification for the recommendation.
    For the online sources, they must be trustworthy and relevant to the region queried.
    For instances, sources from Agriculture and Horticulture Development Board (AHDB), EU-FarmBook, and British Crop Production Council (BCPC). & 
\begin{itemize}
  \item Markdown reader.
\end{itemize} \\
\bottomrule
\end{tabular}
\end{table}
% -------------------------------------------------------------------------------------

\begin{table}
\caption{Task definitions and the tools used by each agent (Part 2)}
\label{tab: task profiling 2}
\centering
\begin{tabular}{
    >{\centering}m{2cm}  % Centered and bold header for "Arch."
    >{}m{6cm}    % Centered and bold header for "Definition"
    >{\centering\arraybackslash}m{3cm}    % Centered and bold header for "Advantages"
}
\toprule
Agent & \centering Profile & Related Tools \\
\midrule
Knowledge retrieval & 
    Retrieve detailed online information for each enhancement section specified in the customization plan.
    The plan is a JSON file that can be read at \{custom\_plan\_path\}.
    For each section in the plan, use the provided search queries and recommended online sources
    to gather relevant information. Compile a comprehensive report that analyses and summary the information searched. The report should be composed with enhancement sections in accordance with the customization plan. Each enhancement section should include: 1. The section name. 2. A summary of findings for each search query. 3. Relevant citations\&links from the recommended online sources. 4. An analysis that highlights how the new data can enhance the initial PMA. & 
\begin{itemize}
  \item JSON reader.
  \item Online search.
\end{itemize} \\
\addlinespace
    Generate customised PMA &  
    Using the initial PMA generated in \{initial\_pma\_path\} and the report summarised from online information by the Retriever in \{retrieved\_info\_path\},
    synthesize a final, customised pest management advice (PMA) for the queried scenario.
    This task involves integrating the additional insights (addressing the sections identified in the customization plan in \{custom\_plan\_path\}) into the initial PMA. Ensure that the final document adheres to the structure and style of the initial PMA.& 
\begin{itemize}
    \item JSON reader.
  \item Markdown reader.
\end{itemize} \\
\addlinespace
    Validate threshold &  
    Analyze the customised PMA (located at \{custom\_pma\_path\}) with a focus on the threshold conclusion and management (or action) decision.
    Examine whether the diagnostic data—such as the pest identification details and infestation levels—logically supports the conclusion stated under "Threshold Exceeded."
    Identify any discrepancies or areas that may require further external validation. In cases where the threshold evaluation reveals uncertainty or potential misalignment with standard practices,
    initiate searches for updated industry guidelines, scientific literature, or official thresholds related to pest infestation levels. Compare the external findings with the customised PMA’s threshold decision to validate or recommend corrections. & 
\begin{itemize}
  \item JSON reader.
  \item Markdown reader.
  \item Online search.
\end{itemize} \\
\bottomrule
\end{tabular}
\end{table}


\subsection{Results}
\label{sec: results}

We evaluated the accuracy of PestMA’s pest management decision (PMD) at two distinct stages. First, we measured the accuracy of the PMD proposed by the Editor after integrating information retrieved by the Retriever. This stage reflects the collaboration and effectiveness of both the Editor and Retriever agents. Second, we assessed the PMD validated by the Validator, indicating whether the Validator can accurately confirm (or correct) the logic used by the Editor when comparing infestation severity to the threshold.

Table \ref{tab: test performance} reports the accuracy for both stages. Notably, the Editor’s PMD in the customised PMA achieves an accuracy of 86.8\%, which is already a strong baseline performance. Once validated, the PMD accuracy further increases to 92.6\%, demonstrating that the Validator can detect and correct threshold comparison errors, thereby enhancing PestMA’s overall reliability.

% On the test dataset, we report the accuracy of the PMD at two different stages in the PestMA. The first is the PMD in the customised PMA suggested by the Editor after the integration of the information retrieved by the Retriever. This accuracy reflects the functionality of the Editor and Retriever, and their collaboration. The second is the PMD that is checked with Validator, expressing the functionality of the Validator, if it can examine the logical correctness of the comparison of the threshold with the given scenario in the Editor. Table~\ref{tab: test performance} reports the results. We show that the PMD in the customised PMA achieves a fairly good performance, with 86.8\% accuracy. Furthermore, with the Validator in place, we find that a proportion of the wrongly comparisons can be further corrected, improving the PMD accuracy to 92.6\%, demonstrating the effectiveness of validation. 

Table \ref{tab: a test example} presents a representative test example illustrating how the Validator safeguards the PMD. In this scenario, the Validator identifies that the Editor’s decision, while reasonably justified, does not explicitly reference the threshold established by regional guidelines. By incorporating guidance from the AHDB, the Validator clarifies that immediate action is still recommended when infestation closely approaching to the threshold of a nematode count of 1,000 per litre of soil. This process highlights the Validator’s role in reinforcing evidence-based recommendations, ultimately resulting in improved decision accuracy and greater confidence in the response of PestMA.


\begin{table}
\caption{PMD accuracy at different stages of PestMA’s workflow}
\label{tab: test performance}
\centering
\begin{tabular}{@{}ll@{}}
\toprule
Response stage in PestMA & Acc. (\%)  \\ \midrule
PMD (Editor + Retriever) &  86.8  \\ 
PMD (Validated by Validator) &  92.6 \\

\bottomrule
\end{tabular}
\label{tab: PMD accuracy}
\end{table}


\begin{table}
\caption{Example scenario demonstrating how the Validator refines the PMD}
\label{tab: a test example}
\centering
\begin{tabular}{
    >{}m{3cm}  % Centered and bold header for "Arch."
    >{}m{3cm}    % Centered and bold header for "Definition"
    >{}m{6cm}    % Centered and bold header for "Advantages"
}
\toprule
 Scenario & PMD in customised PMA & Validated PMD \\
\midrule
    \textit{Pest}: Free-Living Nematodes,
    \textit{InfestationSeverity}: \textcolor{red}{800} Trichodorus nematodes per litre of soil,
    \textit{CropName}: Sugar Beet,
    \textit{CropGrowthStage}: Early root development,
    \textit{Temperature}: 12°C,
    \textit{Weather}: Partly cloudy,
    \textit{Humidity}: 75\%,
    \textit{Precipitation}: 30 mm,
    \textit{Time}: May,
    \textit{Location}: Norfolk, 
    & With \textcolor{red}{800} nematodes per litre of soil, consider intervention due to potential root damage and yield loss during this vital growth phase.  
    & AHDB Guidance: According to the AHDB, severe symptoms in sugar beet can occur when Trichodorus populations exceed \textcolor{red}{1,000} nematodes per litre of soil. Hence, \textcolor{red}{interventions often are recommended when levels approach this threshold} (source: [AHDB Knowledge Library](https://ahdb.org.uk/knowledge-library/free-living-nematodes-and-their-impact-on-the-yield-and-quality-of-field-crops)).
    The inital PMD asserts an infestation of 800 nematodes per litre of soil, aligning closely with regional guidelines suggesting severe symptoms develop above 1,000 nematodes per litre. \textcolor{red}{While the PMD emphasizes timely intervention, it would strengthen its recommendations by explicitly correlating with the 1,000 nematode threshold mentioned in regional studies.} \\
\bottomrule
\end{tabular}
\end{table}

\section{Conclusion}

In this paper, we introduced PestMA, a multi-agent system (MAS) underpinned by large language models (LLMs) for delivering precise and reliable pest management advice. By structuring the MAS with an editorial paradigm—an Editor synthesizing pest management advice, a Retriever gathering external domain knowledge, and a Validator verifying correctness—we demonstrated how specialized roles and agent-based collaboration can enhance decision accuracy. Our empirical results on real-world pest data showed that PestMA’s customised pest management decision (PMD) attains 86.8\% accuracy, which is further improved to 92.6\% following the Validator’s intervention.

Despite these promising outcomes, two limitations remain. First, PestMA currently relies exclusively on an online search tool for external knowledge retrieval; future work can integrate more advanced approaches, such as retrieval-augmented generation (RAG) and other domain-specific tools. Second, our evaluation focused specifically on the PMD dimension, thus providing a narrower scope of assessment. In forthcoming research, PestMA will be extended to encompass additional aspects of pest management, ensuring a more holistic evaluation and further strengthening the system’s practical utility.

\bibliographystyle{plainnat}
\bibliography{references}

% -------------------------------------------
% Format Instructions
% -------------------------------------------

% \subsection{Headings: second level}


% Second-level headings should be in 10-point type.


% \subsubsection{Headings: third level}


% Third-level headings should be in 10-point type.


% \paragraph{Paragraphs}


% There is also a \verb+\paragraph+ command available, which sets the heading in
% bold, flush left, and inline with the text, with the heading followed by 1\,em
% of space.


% \section{Citations, figures, tables, references}
% \label{others}


% These instructions apply to everyone.


% \subsection{Citations within the text}


% The \verb+natbib+ package will be loaded for you by default.  Citations may be
% author/year or numeric, as long as you maintain internal consistency.  As to the
% format of the references themselves, any style is acceptable as long as it is
% used consistently.


% The documentation for \verb+natbib+ may be found at
% \begin{center}
%   \url{http://mirrors.ctan.org/macros/latex/contrib/natbib/natnotes.pdf}
% \end{center}
% Of note is the command \verb+\citet+, which produces citations appropriate for
% use in inline text.  For example,
% \begin{verbatim}
%    \citet{hasselmo} investigated\dots
% \end{verbatim}
% produces
% \begin{quote}
%   Hasselmo, et al.\ (1995) investigated\dots
% \end{quote}


% If you wish to load the \verb+natbib+ package with options, you may add the
% following before loading the \verb+neurips_2025+ package:
% \begin{verbatim}
%    \PassOptionsToPackage{options}{natbib}
% \end{verbatim}


% If \verb+natbib+ clashes with another package you load, you can add the optional
% argument \verb+nonatbib+ when loading the style file:
% \begin{verbatim}
%    \usepackage[nonatbib]{neurips_2025}
% \end{verbatim}


% As submission is double blind, refer to your own published work in the third
% person. That is, use ``In the previous work of Jones et al.\ [4],'' not ``In our
% previous work [4].'' If you cite your other papers that are not widely available
% (e.g., a journal paper under review), use anonymous author names in the
% citation, e.g., an author of the form ``A.\ Anonymous'' and include a copy of the anonymized paper in the supplementary material.


% \subsection{Footnotes}


% Footnotes should be used sparingly.  If you do require a footnote, indicate
% footnotes with a number\footnote{Sample of the first footnote.} in the
% text. Place the footnotes at the bottom of the page on which they appear.
% Precede the footnote with a horizontal rule of 2~inches (12~picas).


% Note that footnotes are properly typeset \emph{after} punctuation
% marks.\footnote{As in this example.}


% \subsection{Figures}


% \begin{figure}
%   \centering
%   \fbox{\rule[-.5cm]{0cm}{4cm} \rule[-.5cm]{4cm}{0cm}}
%   \caption{Sample figure caption.}
% \end{figure}


% All artwork must be neat, clean, and legible. Lines should be dark enough for
% purposes of reproduction. The figure number and caption always appear after the
% figure. Place one line space before the figure caption and one line space after
% the figure. The figure caption should be lower case (except for first word and
% proper nouns); figures are numbered consecutively.


% You may use color figures.  However, it is best for the figure captions and the
% paper body to be legible if the paper is printed in either black/white or in
% color.


% \subsection{Tables}


% All tables must be centered, neat, clean and legible.  The table number and
% title always appear before the table.  See Table~\ref{sample-table}.


% Place one line space before the table title, one line space after the
% table title, and one line space after the table. The table title must
% be lower case (except for first word and proper nouns); tables are
% numbered consecutively.


% Note that publication-quality tables \emph{do not contain vertical rules.} We
% strongly suggest the use of the \verb+booktabs+ package, which allows for
% typesetting high-quality, professional tables:
% \begin{center}
%   \url{https://www.ctan.org/pkg/booktabs}
% \end{center}
% This package was used to typeset Table~\ref{sample-table}.


% \begin{table}
%   \caption{Sample table title}
%   \label{sample-table}
%   \centering
%   \begin{tabular}{lll}
%     \toprule
%     \multicolumn{2}{c}{Part}                   \\
%     \cmidrule(r){1-2}
%     Name     & Description     & Size ($\mu$m) \\
%     \midrule
%     Dendrite & Input terminal  & $\sim$100     \\
%     Axon     & Output terminal & $\sim$10      \\
%     Soma     & Cell body       & up to $10^6$  \\
%     \bottomrule
%   \end{tabular}
% \end{table}

% \subsection{Math}
% Note that display math in bare TeX commands will not create correct line numbers for submission. Please use LaTeX (or AMSTeX) commands for unnumbered display math. (You really shouldn't be using \$\$ anyway; see \url{https://tex.stackexchange.com/questions/503/why-is-preferable-to} and \url{https://tex.stackexchange.com/questions/40492/what-are-the-differences-between-align-equation-and-displaymath} for more information.)

% \subsection{Final instructions}

% Do not change any aspects of the formatting parameters in the style files.  In
% particular, do not modify the width or length of the rectangle the text should
% fit into, and do not change font sizes (except perhaps in the
% \textbf{References} section; see below). Please note that pages should be
% numbered.


% \section{Preparing PDF files}


% Please prepare submission files with paper size ``US Letter,'' and not, for
% example, ``A4.''


% Fonts were the main cause of problems in the past years. Your PDF file must only
% contain Type 1 or Embedded TrueType fonts. Here are a few instructions to
% achieve this.


% \begin{itemize}


% \item You should directly generate PDF files using \verb+pdflatex+.


% \item You can check which fonts a PDF files uses.  In Acrobat Reader, select the
%   menu Files$>$Document Properties$>$Fonts and select Show All Fonts. You can
%   also use the program \verb+pdffonts+ which comes with \verb+xpdf+ and is
%   available out-of-the-box on most Linux machines.


% \item \verb+xfig+ "patterned" shapes are implemented with bitmap fonts.  Use
%   "solid" shapes instead.


% \item The \verb+\bbold+ package almost always uses bitmap fonts.  You should use
%   the equivalent AMS Fonts:
% \begin{verbatim}
%    \usepackage{amsfonts}
% \end{verbatim}
% followed by, e.g., \verb+\mathbb{R}+, \verb+\mathbb{N}+, or \verb+\mathbb{C}+
% for $\mathbb{R}$, $\mathbb{N}$ or $\mathbb{C}$.  You can also use the following
% workaround for reals, natural and complex:
% \begin{verbatim}
%    \newcommand{\RR}{I\!\!R} %real numbers
%    \newcommand{\Nat}{I\!\!N} %natural numbers
%    \newcommand{\CC}{I\!\!\!\!C} %complex numbers
% \end{verbatim}
% Note that \verb+amsfonts+ is automatically loaded by the \verb+amssymb+ package.


% \end{itemize}


% If your file contains type 3 fonts or non embedded TrueType fonts, we will ask
% you to fix it.


% \subsection{Margins in \LaTeX{}}


% Most of the margin problems come from figures positioned by hand using
% \verb+\special+ or other commands. We suggest using the command
% \verb+\includegraphics+ from the \verb+graphicx+ package. Always specify the
% figure width as a multiple of the line width as in the example below:
% \begin{verbatim}
%    \usepackage[pdftex]{graphicx} ...
%    \includegraphics[width=0.8\linewidth]{myfile.pdf}
% \end{verbatim}
% See Section 4.4 in the graphics bundle documentation
% (\url{http://mirrors.ctan.org/macros/latex/required/graphics/grfguide.pdf})


% A number of width problems arise when \LaTeX{} cannot properly hyphenate a
% line. Please give LaTeX hyphenation hints using the \verb+\-+ command when
% necessary.

% \begin{ack}
% Use unnumbered first level headings for the acknowledgments. All acknowledgments
% go at the end of the paper before the list of references. Moreover, you are required to declare
% funding (financial activities supporting the submitted work) and competing interests (related financial activities outside the submitted work).
% More information about this disclosure can be found at: \url{https://neurips.cc/Conferences/2025/PaperInformation/FundingDisclosure}.


% Do {\bf not} include this section in the anonymized submission, only in the final paper. You can use the \texttt{ack} environment provided in the style file to automatically hide this section in the anonymized submission.
% \end{ack}

% \section*{References}


% References follow the acknowledgments in the camera-ready paper. Use unnumbered first-level heading for
% the references. Any choice of citation style is acceptable as long as you are
% consistent. It is permissible to reduce the font size to \verb+small+ (9 point)
% when listing the references.
% Note that the Reference section does not count towards the page limit.
% \medskip


% {
% \small


% [1] Alexander, J.A.\ \& Mozer, M.C.\ (1995) Template-based algorithms for
% connectionist rule extraction. In G.\ Tesauro, D.S.\ Touretzky and T.K.\ Leen
% (eds.), {\it Advances in Neural Information Processing Systems 7},
% pp.\ 609--616. Cambridge, MA: MIT Press.


% [2] Bower, J.M.\ \& Beeman, D.\ (1995) {\it The Book of GENESIS: Exploring
%   Realistic Neural Models with the GEneral NEural SImulation System.}  New York:
% TELOS/Springer--Verlag.


% [3] Hasselmo, M.E., Schnell, E.\ \& Barkai, E.\ (1995) Dynamics of learning and
% recall at excitatory recurrent synapses and cholinergic modulation in rat
% hippocampal region CA3. {\it Journal of Neuroscience} {\bf 15}(7):5249-5262.
% }


% %%%%%%%%%%%%%%%%%%%%%%%%%%%%%%%%%%%%%%%%%%%%%%%%%%%%%%%%%%%%

% \appendix

% \section{Technical Appendices and Supplementary Material}
% Technical appendices with additional results, figures, graphs and proofs may be submitted with the paper submission before the full submission deadline (see above), or as a separate PDF in the ZIP file below before the supplementary material deadline. There is no page limit for the technical appendices.

% %%%%%%%%%%%%%%%%%%%%%%%%%%%%%%%%%%%%%%%%%%%%%%%%%%%%%%%%%%%%

% \newpage
% \section*{NeurIPS Paper Checklist}

% %%% BEGIN INSTRUCTIONS %%%
% The checklist is designed to encourage best practices for responsible machine learning research, addressing issues of reproducibility, transparency, research ethics, and societal impact. Do not remove the checklist: {\bf The papers not including the checklist will be desk rejected.} The checklist should follow the references and follow the (optional) supplemental material.  The checklist does NOT count towards the page
% limit. 

% Please read the checklist guidelines carefully for information on how to answer these questions. For each question in the checklist:
% \begin{itemize}
%     \item You should answer \answerYes{}, \answerNo{}, or \answerNA{}.
%     \item \answerNA{} means either that the question is Not Applicable for that particular paper or the relevant information is Not Available.
%     \item Please provide a short (1–2 sentence) justification right after your answer (even for NA). 
%    % \item {\bf The papers not including the checklist will be desk rejected.}
% \end{itemize}

% {\bf The checklist answers are an integral part of your paper submission.} They are visible to the reviewers, area chairs, senior area chairs, and ethics reviewers. You will be asked to also include it (after eventual revisions) with the final version of your paper, and its final version will be published with the paper.

% The reviewers of your paper will be asked to use the checklist as one of the factors in their evaluation. While "\answerYes{}" is generally preferable to "\answerNo{}", it is perfectly acceptable to answer "\answerNo{}" provided a proper justification is given (e.g., "error bars are not reported because it would be too computationally expensive" or "we were unable to find the license for the dataset we used"). In general, answering "\answerNo{}" or "\answerNA{}" is not grounds for rejection. While the questions are phrased in a binary way, we acknowledge that the true answer is often more nuanced, so please just use your best judgment and write a justification to elaborate. All supporting evidence can appear either in the main paper or the supplemental material, provided in appendix. If you answer \answerYes{} to a question, in the justification please point to the section(s) where related material for the question can be found.

% IMPORTANT, please:
% \begin{itemize}
%     \item {\bf Delete this instruction block, but keep the section heading ``NeurIPS Paper Checklist"},
%     \item  {\bf Keep the checklist subsection headings, questions/answers and guidelines below.}
%     \item {\bf Do not modify the questions and only use the provided macros for your answers}.
% \end{itemize} 
 

% %%% END INSTRUCTIONS %%%


% \begin{enumerate}

% \item {\bf Claims}
%     \item[] Question: Do the main claims made in the abstract and introduction accurately reflect the paper's contributions and scope?
%     \item[] Answer: \answerTODO{} % Replace by \answerYes{}, \answerNo{}, or \answerNA{}.
%     \item[] Justification: \justificationTODO{}
%     \item[] Guidelines:
%     \begin{itemize}
%         \item The answer NA means that the abstract and introduction do not include the claims made in the paper.
%         \item The abstract and/or introduction should clearly state the claims made, including the contributions made in the paper and important assumptions and limitations. A No or NA answer to this question will not be perceived well by the reviewers. 
%         \item The claims made should match theoretical and experimental results, and reflect how much the results can be expected to generalize to other settings. 
%         \item It is fine to include aspirational goals as motivation as long as it is clear that these goals are not attained by the paper. 
%     \end{itemize}

% \item {\bf Limitations}
%     \item[] Question: Does the paper discuss the limitations of the work performed by the authors?
%     \item[] Answer: \answerTODO{} % Replace by \answerYes{}, \answerNo{}, or \answerNA{}.
%     \item[] Justification: \justificationTODO{}
%     \item[] Guidelines:
%     \begin{itemize}
%         \item The answer NA means that the paper has no limitation while the answer No means that the paper has limitations, but those are not discussed in the paper. 
%         \item The authors are encouraged to create a separate "Limitations" section in their paper.
%         \item The paper should point out any strong assumptions and how robust the results are to violations of these assumptions (e.g., independence assumptions, noiseless settings, model well-specification, asymptotic approximations only holding locally). The authors should reflect on how these assumptions might be violated in practice and what the implications would be.
%         \item The authors should reflect on the scope of the claims made, e.g., if the approach was only tested on a few datasets or with a few runs. In general, empirical results often depend on implicit assumptions, which should be articulated.
%         \item The authors should reflect on the factors that influence the performance of the approach. For example, a facial recognition algorithm may perform poorly when image resolution is low or images are taken in low lighting. Or a speech-to-text system might not be used reliably to provide closed captions for online lectures because it fails to handle technical jargon.
%         \item The authors should discuss the computational efficiency of the proposed algorithms and how they scale with dataset size.
%         \item If applicable, the authors should discuss possible limitations of their approach to address problems of privacy and fairness.
%         \item While the authors might fear that complete honesty about limitations might be used by reviewers as grounds for rejection, a worse outcome might be that reviewers discover limitations that aren't acknowledged in the paper. The authors should use their best judgment and recognize that individual actions in favor of transparency play an important role in developing norms that preserve the integrity of the community. Reviewers will be specifically instructed to not penalize honesty concerning limitations.
%     \end{itemize}

% \item {\bf Theory assumptions and proofs}
%     \item[] Question: For each theoretical result, does the paper provide the full set of assumptions and a complete (and correct) proof?
%     \item[] Answer: \answerTODO{} % Replace by \answerYes{}, \answerNo{}, or \answerNA{}.
%     \item[] Justification: \justificationTODO{}
%     \item[] Guidelines:
%     \begin{itemize}
%         \item The answer NA means that the paper does not include theoretical results. 
%         \item All the theorems, formulas, and proofs in the paper should be numbered and cross-referenced.
%         \item All assumptions should be clearly stated or referenced in the statement of any theorems.
%         \item The proofs can either appear in the main paper or the supplemental material, but if they appear in the supplemental material, the authors are encouraged to provide a short proof sketch to provide intuition. 
%         \item Inversely, any informal proof provided in the core of the paper should be complemented by formal proofs provided in appendix or supplemental material.
%         \item Theorems and Lemmas that the proof relies upon should be properly referenced. 
%     \end{itemize}

%     \item {\bf Experimental result reproducibility}
%     \item[] Question: Does the paper fully disclose all the information needed to reproduce the main experimental results of the paper to the extent that it affects the main claims and/or conclusions of the paper (regardless of whether the code and data are provided or not)?
%     \item[] Answer: \answerTODO{} % Replace by \answerYes{}, \answerNo{}, or \answerNA{}.
%     \item[] Justification: \justificationTODO{}
%     \item[] Guidelines:
%     \begin{itemize}
%         \item The answer NA means that the paper does not include experiments.
%         \item If the paper includes experiments, a No answer to this question will not be perceived well by the reviewers: Making the paper reproducible is important, regardless of whether the code and data are provided or not.
%         \item If the contribution is a dataset and/or model, the authors should describe the steps taken to make their results reproducible or verifiable. 
%         \item Depending on the contribution, reproducibility can be accomplished in various ways. For example, if the contribution is a novel architecture, describing the architecture fully might suffice, or if the contribution is a specific model and empirical evaluation, it may be necessary to either make it possible for others to replicate the model with the same dataset, or provide access to the model. In general. releasing code and data is often one good way to accomplish this, but reproducibility can also be provided via detailed instructions for how to replicate the results, access to a hosted model (e.g., in the case of a large language model), releasing of a model checkpoint, or other means that are appropriate to the research performed.
%         \item While NeurIPS does not require releasing code, the conference does require all submissions to provide some reasonable avenue for reproducibility, which may depend on the nature of the contribution. For example
%         \begin{enumerate}
%             \item If the contribution is primarily a new algorithm, the paper should make it clear how to reproduce that algorithm.
%             \item If the contribution is primarily a new model architecture, the paper should describe the architecture clearly and fully.
%             \item If the contribution is a new model (e.g., a large language model), then there should either be a way to access this model for reproducing the results or a way to reproduce the model (e.g., with an open-source dataset or instructions for how to construct the dataset).
%             \item We recognize that reproducibility may be tricky in some cases, in which case authors are welcome to describe the particular way they provide for reproducibility. In the case of closed-source models, it may be that access to the model is limited in some way (e.g., to registered users), but it should be possible for other researchers to have some path to reproducing or verifying the results.
%         \end{enumerate}
%     \end{itemize}


% \item {\bf Open access to data and code}
%     \item[] Question: Does the paper provide open access to the data and code, with sufficient instructions to faithfully reproduce the main experimental results, as described in supplemental material?
%     \item[] Answer: \answerTODO{} % Replace by \answerYes{}, \answerNo{}, or \answerNA{}.
%     \item[] Justification: \justificationTODO{}
%     \item[] Guidelines:
%     \begin{itemize}
%         \item The answer NA means that paper does not include experiments requiring code.
%         \item Please see the NeurIPS code and data submission guidelines (\url{https://nips.cc/public/guides/CodeSubmissionPolicy}) for more details.
%         \item While we encourage the release of code and data, we understand that this might not be possible, so “No” is an acceptable answer. Papers cannot be rejected simply for not including code, unless this is central to the contribution (e.g., for a new open-source benchmark).
%         \item The instructions should contain the exact command and environment needed to run to reproduce the results. See the NeurIPS code and data submission guidelines (\url{https://nips.cc/public/guides/CodeSubmissionPolicy}) for more details.
%         \item The authors should provide instructions on data access and preparation, including how to access the raw data, preprocessed data, intermediate data, and generated data, etc.
%         \item The authors should provide scripts to reproduce all experimental results for the new proposed method and baselines. If only a subset of experiments are reproducible, they should state which ones are omitted from the script and why.
%         \item At submission time, to preserve anonymity, the authors should release anonymized versions (if applicable).
%         \item Providing as much information as possible in supplemental material (appended to the paper) is recommended, but including URLs to data and code is permitted.
%     \end{itemize}


% \item {\bf Experimental setting/details}
%     \item[] Question: Does the paper specify all the training and test details (e.g., data splits, hyperparameters, how they were chosen, type of optimizer, etc.) necessary to understand the results?
%     \item[] Answer: \answerTODO{} % Replace by \answerYes{}, \answerNo{}, or \answerNA{}.
%     \item[] Justification: \justificationTODO{}
%     \item[] Guidelines:
%     \begin{itemize}
%         \item The answer NA means that the paper does not include experiments.
%         \item The experimental setting should be presented in the core of the paper to a level of detail that is necessary to appreciate the results and make sense of them.
%         \item The full details can be provided either with the code, in appendix, or as supplemental material.
%     \end{itemize}

% \item {\bf Experiment statistical significance}
%     \item[] Question: Does the paper report error bars suitably and correctly defined or other appropriate information about the statistical significance of the experiments?
%     \item[] Answer: \answerTODO{} % Replace by \answerYes{}, \answerNo{}, or \answerNA{}.
%     \item[] Justification: \justificationTODO{}
%     \item[] Guidelines:
%     \begin{itemize}
%         \item The answer NA means that the paper does not include experiments.
%         \item The authors should answer "Yes" if the results are accompanied by error bars, confidence intervals, or statistical significance tests, at least for the experiments that support the main claims of the paper.
%         \item The factors of variability that the error bars are capturing should be clearly stated (for example, train/test split, initialization, random drawing of some parameter, or overall run with given experimental conditions).
%         \item The method for calculating the error bars should be explained (closed form formula, call to a library function, bootstrap, etc.)
%         \item The assumptions made should be given (e.g., Normally distributed errors).
%         \item It should be clear whether the error bar is the standard deviation or the standard error of the mean.
%         \item It is OK to report 1-sigma error bars, but one should state it. The authors should preferably report a 2-sigma error bar than state that they have a 96\% CI, if the hypothesis of Normality of errors is not verified.
%         \item For asymmetric distributions, the authors should be careful not to show in tables or figures symmetric error bars that would yield results that are out of range (e.g. negative error rates).
%         \item If error bars are reported in tables or plots, The authors should explain in the text how they were calculated and reference the corresponding figures or tables in the text.
%     \end{itemize}

% \item {\bf Experiments compute resources}
%     \item[] Question: For each experiment, does the paper provide sufficient information on the computer resources (type of compute workers, memory, time of execution) needed to reproduce the experiments?
%     \item[] Answer: \answerTODO{} % Replace by \answerYes{}, \answerNo{}, or \answerNA{}.
%     \item[] Justification: \justificationTODO{}
%     \item[] Guidelines:
%     \begin{itemize}
%         \item The answer NA means that the paper does not include experiments.
%         \item The paper should indicate the type of compute workers CPU or GPU, internal cluster, or cloud provider, including relevant memory and storage.
%         \item The paper should provide the amount of compute required for each of the individual experimental runs as well as estimate the total compute. 
%         \item The paper should disclose whether the full research project required more compute than the experiments reported in the paper (e.g., preliminary or failed experiments that didn't make it into the paper). 
%     \end{itemize}
    
% \item {\bf Code of ethics}
%     \item[] Question: Does the research conducted in the paper conform, in every respect, with the NeurIPS Code of Ethics \url{https://neurips.cc/public/EthicsGuidelines}?
%     \item[] Answer: \answerTODO{} % Replace by \answerYes{}, \answerNo{}, or \answerNA{}.
%     \item[] Justification: \justificationTODO{}
%     \item[] Guidelines:
%     \begin{itemize}
%         \item The answer NA means that the authors have not reviewed the NeurIPS Code of Ethics.
%         \item If the authors answer No, they should explain the special circumstances that require a deviation from the Code of Ethics.
%         \item The authors should make sure to preserve anonymity (e.g., if there is a special consideration due to laws or regulations in their jurisdiction).
%     \end{itemize}


% \item {\bf Broader impacts}
%     \item[] Question: Does the paper discuss both potential positive societal impacts and negative societal impacts of the work performed?
%     \item[] Answer: \answerTODO{} % Replace by \answerYes{}, \answerNo{}, or \answerNA{}.
%     \item[] Justification: \justificationTODO{}
%     \item[] Guidelines:
%     \begin{itemize}
%         \item The answer NA means that there is no societal impact of the work performed.
%         \item If the authors answer NA or No, they should explain why their work has no societal impact or why the paper does not address societal impact.
%         \item Examples of negative societal impacts include potential malicious or unintended uses (e.g., disinformation, generating fake profiles, surveillance), fairness considerations (e.g., deployment of technologies that could make decisions that unfairly impact specific groups), privacy considerations, and security considerations.
%         \item The conference expects that many papers will be foundational research and not tied to particular applications, let alone deployments. However, if there is a direct path to any negative applications, the authors should point it out. For example, it is legitimate to point out that an improvement in the quality of generative models could be used to generate deepfakes for disinformation. On the other hand, it is not needed to point out that a generic algorithm for optimizing neural networks could enable people to train models that generate Deepfakes faster.
%         \item The authors should consider possible harms that could arise when the technology is being used as intended and functioning correctly, harms that could arise when the technology is being used as intended but gives incorrect results, and harms following from (intentional or unintentional) misuse of the technology.
%         \item If there are negative societal impacts, the authors could also discuss possible mitigation strategies (e.g., gated release of models, providing defenses in addition to attacks, mechanisms for monitoring misuse, mechanisms to monitor how a system learns from feedback over time, improving the efficiency and accessibility of ML).
%     \end{itemize}
    
% \item {\bf Safeguards}
%     \item[] Question: Does the paper describe safeguards that have been put in place for responsible release of data or models that have a high risk for misuse (e.g., pretrained language models, image generators, or scraped datasets)?
%     \item[] Answer: \answerTODO{} % Replace by \answerYes{}, \answerNo{}, or \answerNA{}.
%     \item[] Justification: \justificationTODO{}
%     \item[] Guidelines:
%     \begin{itemize}
%         \item The answer NA means that the paper poses no such risks.
%         \item Released models that have a high risk for misuse or dual-use should be released with necessary safeguards to allow for controlled use of the model, for example by requiring that users adhere to usage guidelines or restrictions to access the model or implementing safety filters. 
%         \item Datasets that have been scraped from the Internet could pose safety risks. The authors should describe how they avoided releasing unsafe images.
%         \item We recognize that providing effective safeguards is challenging, and many papers do not require this, but we encourage authors to take this into account and make a best faith effort.
%     \end{itemize}

% \item {\bf Licenses for existing assets}
%     \item[] Question: Are the creators or original owners of assets (e.g., code, data, models), used in the paper, properly credited and are the license and terms of use explicitly mentioned and properly respected?
%     \item[] Answer: \answerTODO{} % Replace by \answerYes{}, \answerNo{}, or \answerNA{}.
%     \item[] Justification: \justificationTODO{}
%     \item[] Guidelines:
%     \begin{itemize}
%         \item The answer NA means that the paper does not use existing assets.
%         \item The authors should cite the original paper that produced the code package or dataset.
%         \item The authors should state which version of the asset is used and, if possible, include a URL.
%         \item The name of the license (e.g., CC-BY 4.0) should be included for each asset.
%         \item For scraped data from a particular source (e.g., website), the copyright and terms of service of that source should be provided.
%         \item If assets are released, the license, copyright information, and terms of use in the package should be provided. For popular datasets, \url{paperswithcode.com/datasets} has curated licenses for some datasets. Their licensing guide can help determine the license of a dataset.
%         \item For existing datasets that are re-packaged, both the original license and the license of the derived asset (if it has changed) should be provided.
%         \item If this information is not available online, the authors are encouraged to reach out to the asset's creators.
%     \end{itemize}

% \item {\bf New assets}
%     \item[] Question: Are new assets introduced in the paper well documented and is the documentation provided alongside the assets?
%     \item[] Answer: \answerTODO{} % Replace by \answerYes{}, \answerNo{}, or \answerNA{}.
%     \item[] Justification: \justificationTODO{}
%     \item[] Guidelines:
%     \begin{itemize}
%         \item The answer NA means that the paper does not release new assets.
%         \item Researchers should communicate the details of the dataset/code/model as part of their submissions via structured templates. This includes details about training, license, limitations, etc. 
%         \item The paper should discuss whether and how consent was obtained from people whose asset is used.
%         \item At submission time, remember to anonymize your assets (if applicable). You can either create an anonymized URL or include an anonymized zip file.
%     \end{itemize}

% \item {\bf Crowdsourcing and research with human subjects}
%     \item[] Question: For crowdsourcing experiments and research with human subjects, does the paper include the full text of instructions given to participants and screenshots, if applicable, as well as details about compensation (if any)? 
%     \item[] Answer: \answerTODO{} % Replace by \answerYes{}, \answerNo{}, or \answerNA{}.
%     \item[] Justification: \justificationTODO{}
%     \item[] Guidelines:
%     \begin{itemize}
%         \item The answer NA means that the paper does not involve crowdsourcing nor research with human subjects.
%         \item Including this information in the supplemental material is fine, but if the main contribution of the paper involves human subjects, then as much detail as possible should be included in the main paper. 
%         \item According to the NeurIPS Code of Ethics, workers involved in data collection, curation, or other labor should be paid at least the minimum wage in the country of the data collector. 
%     \end{itemize}

% \item {\bf Institutional review board (IRB) approvals or equivalent for research with human subjects}
%     \item[] Question: Does the paper describe potential risks incurred by study participants, whether such risks were disclosed to the subjects, and whether Institutional Review Board (IRB) approvals (or an equivalent approval/review based on the requirements of your country or institution) were obtained?
%     \item[] Answer: \answerTODO{} % Replace by \answerYes{}, \answerNo{}, or \answerNA{}.
%     \item[] Justification: \justificationTODO{}
%     \item[] Guidelines:
%     \begin{itemize}
%         \item The answer NA means that the paper does not involve crowdsourcing nor research with human subjects.
%         \item Depending on the country in which research is conducted, IRB approval (or equivalent) may be required for any human subjects research. If you obtained IRB approval, you should clearly state this in the paper. 
%         \item We recognize that the procedures for this may vary significantly between institutions and locations, and we expect authors to adhere to the NeurIPS Code of Ethics and the guidelines for their institution. 
%         \item For initial submissions, do not include any information that would break anonymity (if applicable), such as the institution conducting the review.
%     \end{itemize}

% \item {\bf Declaration of LLM usage}
%     \item[] Question: Does the paper describe the usage of LLMs if it is an important, original, or non-standard component of the core methods in this research? Note that if the LLM is used only for writing, editing, or formatting purposes and does not impact the core methodology, scientific rigorousness, or originality of the research, declaration is not required.
%     %this research? 
%     \item[] Answer: \answerTODO{} % Replace by \answerYes{}, \answerNo{}, or \answerNA{}.
%     \item[] Justification: \justificationTODO{}
%     \item[] Guidelines:
%     \begin{itemize}
%         \item The answer NA means that the core method development in this research does not involve LLMs as any important, original, or non-standard components.
%         \item Please refer to our LLM policy (\url{https://neurips.cc/Conferences/2025/LLM}) for what should or should not be described.
%     \end{itemize}

% \end{enumerate}


\end{document}