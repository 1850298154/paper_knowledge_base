\section{Introduction}

%1st paragraph on how the two central issues with UTM is mission planning with safety and performance guarantees and collision avoidance. Describe each and say that the problem with UTM is harder than ATC because there is no central authority to monitor and enforce drone safety and the density is at least an order of magnitude higher than what TCAS was designed for.
%[TCAS II was designed to operate in traffic densities of up to 0.3 aircraft per square nautical mile (nmi), i.e., 24 aircraft within a 5 nmi radius, which was the highest traffic density envisioned over the next 20 years. REF- https://www.faa.gov/documentlibrary/media/advisory_circular/tcas%20ii%20v7.1%20intro%20booklet.pdf]

The development of safe and reliable UAS Traffic Management (UTM) is necessary to enable Urban Air Mobility (UAM) \cite{NAP25646}. The two fundamental issues here are: a) mission planning for UAS fleets with guarantees on 
safety and performance, and b) real-time airborne collision avoidance (CA) methods so UAS run by different operators can share the airspace without a priori approval of all flight plans. Tackling the planning and inter-UAS collision avoidance jointly yields a computationally intractable problem as the number of UAS in the airspace increase \cite{pant2018fly, SahaRSJ14}. So we separate these two aspects in a manner where individual UAS (or those in the same fleet) plan independently, which in turn requires an approach for runtime collision avoidance. The scalability of this will be essential in UAM applications as there will be no central authority to monitor and enforce UAS safety for hundreds of drones per square mile. This stands in contrast to the existing Air-Traffic Control and collision avoidance methods for commercial aviation, like TCAS-II, which was designed to operate in traffic densities of up to 0.3 aircraft per square nautical mile (nmi), i.e., 24 aircraft within a 5 nautical mile radius, which was the highest traffic density envisioned over the next 20 years \cite{TCAS}.

%2nd paragraph on the challenge with airborne collision avoidance - complexity, runtime, etc. Not specific to MILP. Start with "Airborne collision avoidance is a complex problem."
Airborne collision avoidance however is a complex problem. With high-speed UAS operating at low altitudes in cluttered urban airspace, decisions for collision avoidance need to be made within fractions of a second. The CA system must also be able to take into account the environment (e.g. buildings and other infrastructure, altitude limits, geofenced areas etc.) around it, making the problem harder than simply avoiding inter-UAS collisions. 
%For this reason, even the state-of-art collision avoidance method for UAS, ACAS-Xu \cite{ACASXu}, is only being envisioned for use in high-flying large UAS like the MQ4C Triton \cite{MQ4C}.
%, although a variant for small UAS (ACAS-sXu) is also under development.

\begin{figure}[t]
	\centering
\includegraphics[width=0.35\textwidth]{figures/CADiagram2.pdf}
\caption{\small Two UAS communicating their planned trajectories, and cooperatively maneuvering within their \textit{robustness tubes} to avoid a potential collision in the future.}
\vspace{-10pt}
\label{fig:CAdiagram}
\end{figure}

%3rd paragraph - the two problems you solve here - take this from the start of Section II. It belongs here.

%In this paper, we aim to develop a framework for UAS traffic management (UTM) that overcomes the limitations outlined above and solves the following two problems (which will be formalized in later sections):

To overcome the limitations outlined above, we aim to solve the following problems: %our primary focus is on solving:

\begin{myprob}
\label{prob:traj_STL_highlevel}
\textbf{Independent trajectory planning} for an individual UAS (or fleets run by the same operator) to satisfy spatial, temporal and reactive mission objectives specified using Signal Temporal Logic (STL), independently of other UAS that could be in the airspace.
\end{myprob}
%\textcolor{green}{ Independent or decentralized?}

\begin{myprob}
\label{prob:CA_highlevel}
\textbf{Real-time pairwise predictive airborne collision avoidance} such that UAS mission requirements satisfied by the trajectories obtained by solving problem \ref{prob:traj_STL_highlevel} are not violated, e.g. UAS make it to their destinations in time while avoiding collisions with each other.
\end{myprob}

%We build upon existing work \cite{pant2018fly} to also solve:


%4th paragraph - further details on MILP-based solution and why your approach is needed.

%Executing collision avoidance on-demand, instead of pre-flight, allows us to develop a trajectory generation approach to solve problem \ref{prob:traj_STL_highlevel}. 

%We use a variant of the method in \cite{pant2018fly} which employs Signal Temporal Logic (STL) to capture mission specifications and synthesize UAS trajectories with safety and performance guarantees. 
The airborne collision avoidance (Problem \ref{prob:CA_highlevel}) poses a larger challenge, and is the primary focus here. 
%We start by formulating a Mixed Integer Program to solve the two UAS collision problem in a cooperative manner. 
%While fundamentally sound, the solution cannot run in real-time (see Section \ref{sec:experiments}), and also requires the optimization to be solved centrally.
% Consequently, as the primary contribution of this work, we develop a two-stage decentralized collision avoidance method to overcome these limitations.

%Executing collision avoidance on-demand, instead of pre-flight, allows us to develop a trajectory generation approach to solve problem \ref{prob:traj_STL_highlevel}. We use a variant of the method in \cite{pant2018fly} which employs Signal Temporal Logic (STL) to capture mission specifications and synthesize UAS trajectories with safety and performance guarantees. The airborne collision avoidance (problem \ref{prob:CA_highlevel}) however poses a larger challenge. We start by formulating a Mixed Integer Program to solve the two UAS collision problem in a cooperative manner. While fundamentally sound, the solution cannot run in real-time (see Section \ref{sec:experiments}), and also requires the optimization to be solved centrally. Consequently, as the primary contribution of this work, we develop a two-stage decentralized collision avoidance method to overcome these limitations.


%This method consists of a learning-based conflict resolution scheme, and a cooperative decentralized Model Predictive Controller (MPC) to carry out the collision avoidance by actuating the UAS.
%Section \ref{sec:overview} gives an overview of the approach. Sections \ref{sec:learning} presents the two learning-based collision resolution schemes developed in this paper, and section \ref{sec:CA_control} covers the distributed Collision Avoidance MPC (CA-MPC) controllers that run on board the UAS. Extensive simulations are carried out in section \ref{sec:experiments} to show the 

%Contributions
%It will really help to give the two approaches easy names. This will be good for the abstract and intro too.You have three contributions
%1. and 2. two learning-based collision avoidance schemes - Clearly state the benefits of your approaches here.
%2. CA-MPC
%You need to define Robustness here. You cannot use that word without defining it.
\begin{figure*}[t]
\includegraphics[width=0.99\textwidth]{figures/Concept.pdf}
\vspace{-10pt}
\caption{\small Step-wise explanation and visualization of the framework. Each UAS generates its own trajectories to satisfy a mission expressed as a Signal Temporal Logic (STL) specification, e.g. regions in green are regions of interest for the UAS to visit, and the obstacle corresponds to infrastructure that all the UAS must avoid. When executing these trajectories, UAS communicate their trajectories to others in range to detect any collisions that may happen in the near future. If a collision is detected, the two UAS execute a conflict resolution scheme that generates a set of additional constraints that the UAS must satisfy in order to avoid the collision. A co-operative CA-MPC controls the UAS in order to best satisfy these constraints while ensuring each UAS's STL specification is still satisfied. This results in new trajectories (in solid blue and pink) that will avoid the conflict and still stay within the pre-defined robustness tubes.}
\vspace{-15pt}
\label{fig:concept}
\end{figure*}


\subsection{Contributions of this work}

Our main contribution is \textit{Learning-to-Fly} (L2F)\footnote{Videos of the simulations and demonstrations in this paper can be viewed at \url{https://tinyurl.com/vvvuukh}}, a scheme 
for real-time, on-the-fly collision avoidance between two UAS whose main features are:

\begin{enumerate}
\item \textit{Systematic composition of machine learning and control theory:} We combine learning-based decision-making, and linear programming-based control to solve the problem in a decentralized manner. Unlike many other ad-hoc Machine Learning-based solutions, we provide a sound theoretical justification for our approach in Theorem \ref{th:MILP_CAMPC_relation}. We also provide a sufficient condition for the scheme to work successfully (Theorem \ref{th:CAMPC_success}).

\item  \textit{A notion of priority among the UAS} can be encoded naturally in L2F, where the UAS with higher priority does not have to deviate from its originally planned trajectory until absolutely necessary.

\item \textit{Computationally lightweight enough for real-time implementation:} Experimental results show that L2F, with a computation time in milliseconds can be used in a real-time implementation at a high-rate ($10$ Hz). 

\item \textit{High performance:} In the best case, L2F successfully results in 2-UAS collision avoidance $100\%$ of the test cases, gracefully degrading to $90\%$ for the worst case. Comparisons with other methods also show the superior performance of L2F. 

\item \textit{Enabling fast, independent planning for UAS with temporal logic objectives}, as individual UAS, or fleets of UAS run by the same operator, can plan for themselves without considering other UAS in the airspace while calling upon L2F for on-the-fly collision avoidance. For a 4-UAS case study, we demonstrate a speed up of $3.5\times$ over the centralized planning method of \cite{pant2018fly}.

\item \textit{Proof-of-concept demonstration} on Crazyflie quad-rotor robots to show feasibility on real UAS.

\end{enumerate}

%consisting of:
%\textcolor{green}{Too much repetition, shorten this.
%}
%\begin{enumerate}
%\item \textit{Learning-based conflict resolution:} Given the planned trajectories of two UAS that have a conflict in the near future, the scheme chooses a sequence of maneuvers such that the two UAS avoid colliding with each other without violating their higher level mission objective. We develop a supervised learning-based method for conflict resolution, \textit{\crLSTM} (Section~\ref{sec:learning_supervised}).
%
%\item \textit{CA-MPC: A distributed, cooperative convex Model Predictive 
%Control (MPC) algorithm for collision avoidance:} that takes in the desired 
%maneuvers from the conflict resolution scheme as constraints for each UAS to 
%satisfy such that the resulting trajectories are collision-free and also 
%satisfy the mission objective (Section \ref{sec:CA_MPC}).
%\end{enumerate}
%
%This collision avoidance scheme allows us to build a UAS planning and traffic management framework where we have:
%
%\begin{enumerate}
%\item  \textit{Fast, independent planning for UAS with temporal logic objectives}, as individual UAS, or fleets of UAS run by the same operator, can plan for themselves without having to consider the other UAS in the airspace. 
%\item  \textit{A notion of priority among the UAS}, where the UAS with higher priority does not have to deviate from its originally planned trajectory until absolutely necessary.
%\item  \textit{Real-time collision avoidance} as the CA framework relies on executing one learning based step and two convex optimizations. Experimental evaluations of our implementation show their low computation times.
%\end{enumerate}

%Through extensive simulations, we evaluate the performance of this collision avoidance scheme and show that this framework for planning and collision avoidance allows for faster and more scalable planning for UAS fleets with Signal Temporal Logic specifications than the centralized approach of \cite{pant2018fly}. Finally, through a proof-of-concept implementation on real quadrotor drones, we show this framework can be used in real-world settings. While our method of relying on a learning-based component does not result in collision-free trajectories for the UAS 1\% of the time on average, it outperforms other baseline approaches, and can be used for real-time implementations. \textcolor{green}{Shrink this and make consistent w/ final results}.
\begin{figure*}
  \centering
    \includegraphics[width=\textwidth]{images/overview.png}
  \caption{
  Overview of our approach. (a) We aim to find a collision-free path for a rigid body from its current configuration $q_t$ to the goal configuration $q_g$. (b) We assume no prior knowledge about the scene and represent obstacles by points and normals sampled on object surfaces. (c) Our neural network learns the PointNet encoding of observed points and normals together with the motion policy. (d) The learned network generates actions that move the body towards the goal configuration along a collision-free path.   
 }
  \label{fig:overview}
\vspace{-0.4cm}
\end{figure*}
\subsection{Related work}
\label{sec:relatedwork}
\subsubsection{UTM and Automatic Collision Avoidance approaches}
The UAS Traffic Management (UTM) problem has been studied in various contexts. 
In the NASA/FAA Concept of Operations document \cite{FAA2018UTM}, an airspace allocation scheme is outlined where individual UAS reserve airspace in the form of 4D polygons (space and time), and the polygons of different UAS are not allowed to overlap. Similarly, \cite{maxetal} presents a voxel-based airspace allocation approach. Our approach is less restrictive and allows overlaps in the 4D polygons, but performs maneuvers for collision avoidance when two UAS are on track to a collision (see Fig. \ref{fig:CAdiagram}). TCAS \cite{TCAS} and ACAS \cite{kochenderfer2012next} systems for collision avoidance in commercial aircrafts rely on transponders in the two aircrafts to communicate information for the collision avoidance modules. 
These generate recommendations for the pilots to follow and create vertical separation between aircrafts \cite{ACASX}. 
In the context of UAS, \cite{UTMTCL4} uses vehicle-to-vehicle communication and tree-search based planning to achieve collision avoidance. 
ACAS-Xu \cite{ACASXu}, an automatic collision avoidance scheme for UAS relies on a look-up table to provide high-level recommendations to two UAS that have potentially colliding trajectories. 
%ACAS-Xu is already finding implementation on real-world UAS \cite{MQ4C}. 
%It provides recommendations only and needs an underlying control algorithm to carry out its recommendations. 
%ACAS-Xu restricts maneuvering to a single axis of motion based, vertical or horizontal based on if traffic is cooperative or not
 It restricts desired maneuvers for CA to the vertical axis for cooperative traffic, and the horizontal axis for uncooperative traffic. 
While we consider only the cooperative case in this work, our method does not restrict CA maneuvers to any single axis of motion. Finally, in its current form, ACAS-Xu also does not take into account any higher-level mission objectives, unlike our approach. 
This excludes its application to low-level flights in urban settings, e.g. it can result in situations where ACAS-Xu recommends an action that avoids a nearby UAS but results in the primary UAS going close to a static obstacle. 
Our method avoids this as CA maneuvers are restricted to keeping UAS inside \textit{robustness tubes} (see Fig. \ref{fig:concept}) such that mission requirements are not violated. 
For this reason, ACAS-Xu is currently only being explored for large, high-flying UAS \cite{ACASXu} and is not directly applicable to the problem we study here.

\subsubsection{Multi-agent planning with temporal logic objectives}
Many approaches exist for the problem of planning for multiple robotic agents with temporal logic specifications. 
Most rely on abstract grid-based representations of the workspace \cite{SahaRSJ14, DeCastro17}, or abstract dynamics of the agents \cite{Drona,AksarayCDC16}.
\cite{MaICUAS16} combines a discrete planner with a continuous trajectory generator. 
Some methods \cite{4459804, 1582935, 1641832}   work for subsets of Linear Temporal Logic (LTL) that do not allow for explicit timing bounds on the mission requirements.
While \cite{SahaRSJ14} uses a subset of LTL, safe-LTL$_f$ that allows them to express reach-avoid specifications with explicit timing constraints. However, in addition to a discretization of the workspace, they also restrict motion to a simple, discrete set of motion primitives. 
The predictive control method of \cite{Raman14_MPCSTL} allows for using the expressiveness of the complete grammar STL for mission specifications. 
It handles a continuous workspace and linear dynamics of robots, however its reliance on mixed-integer encoding (similar to \cite{Saha_acc16,KaramanF11_LTLrouting}) for the STL specification limit use in planning/control for multiple agents in 3D workspaces as seen in \cite{pant2017smooth}. 
The approach of \cite{pant2018fly} instead relies on optimizing a smooth (non-convex) function for generating trajectories for fleets of multi-rotor UAS with STL specifications. 
%However, performance degrades as the number of UAS is increased, as it accounts for pairwise separation in a centralized manner. 
In our framework, we use the planning method of \cite{pant2018fly}, but we let each UAS plan independently of each other. 
We ensure the safe operation of all UAS in the airspace through the use of our predictive collision avoidance scheme.
%In our framework, we use the planning method \cite{pant2018fly}, but instead of jointly planning for all UAS in the airspace, we let each UAS plan individually and independently of each other. We ensure the safe operation of all UAS in the airspace through the use of our predictive collision avoidance scheme.

 



%CA stuff: D'Andrea non-cvx, Hadas et al LTL reactive, Fly-by-Logic,
%Kochenderfer and ACAS, ACAS-Xu
%Reachability based from the UMich guy (AR knows)


%TL-based planning: LTLx,f stuff, BluSTL, FbL/SoP, 

%\ypcomment{Do we need full TL section? Can we boil it down to a single section with theorem on robustness and smooth robustness.}

%\subsection{Outline of the paper}

The following section (Section \ref{sec:overview}) presents an overview of the 
framework we develop to solve problems \ref{prob:traj_STL_highlevel} and 
\ref{prob:CA_highlevel}. Section \ref{sec:STL_intro} is an introduction to 
Signal Temporal Logic (STL), which we use for formally representing mission 
requirements. 
Section \ref{sec:problem_planning} presents the variant of the 
algorithm from \cite{pant2018fly} that we use for UAS trajectory generation 
for satisfying STL specifications. 
Section \ref{sec:CA} formalizes the two UAS 
collision-avoidance problem and formulates the solution to it as a 
Mixed-Integer Linear Program (MILP). 
Sections \ref{sec:CA_MPC} and \ref{sec:learning} outlines our approach to the problem, where instead of solving the computationally expensive MILP, we first solve a discrete 
conflict-resolution problem, and then a decentralized and co-operative 
collision avoidance MPC controller. 
These two components form the basis of our framework. 
We present a learning-based approach for the conflict resolution, as well as a greedy, computationally lightweight approach to the problem that serves as a baseline for comparison. 
In Section \ref{sec:experiments} we evaluate our framework for 2 UAS collision avoidance through extensive simulations and compare its performance to the MILP approach and the greedy algorithm. 
Section \ref{sec:casestudy} demonstrates how we combine the trajectory generation method of Section \ref{sec:problem_planning} and the collision avoidance schemes developed in this paper into a UTM framework. 
Finally in Section \ref{sec:future} we discuss potential future directions to overcome the limitations of the current approach.



