%\begin{figure*}[t]
%\includegraphics[width=0.99\textwidth]{Concept.pdf}
%\caption{Step-wise explanation and visualization of the framework. Each UAS generates its own trajectories to satisfy a mission expressed as a STL specification, e.g. regions in green are regions of interest for the UAS to visit, and the obstacle corresponds to infrastructure that all the UAS must avoid. When executing these trajectories, UAS that are within communication range of each other will communicate their trajectories to each other to detect ahead of time any collisions that may happen in the future. If a collision is detected, the two UAS execute a conflict resolution scheme that generates a set of additional constraints that the UAS must satisfy in order to avoid the collision. A co-operative CA-MPC controls the UAS in order to best satisfy these constraints while ensuring each UAS's STL specification is still satisfied. This results in new trajectories (in solid blue and pink) that will avoid the conflict and still stay within the pre-defined robustness tubes.}
%\label{fig:concept}
%\end{figure*}

\subsection{Overview of approach and paper outline}
\label{sec:overview}
%\textcolor{green}{Edit figure to have sections in the fig themselves so outline is not mentioned in the text}.

In this paper, we aim to develop a framework for UAS traffic management (UTM) that solves problems \ref{prob:traj_STL_highlevel} and \ref{prob:CA_highlevel}.  Figure~\ref{fig:concept} depicts the proposed planning and control process and indicates the relevant sections in the paper.


\begin{enumerate}
\item 
\textit{Trajectory planning with Signal Temporal Logic (STL) specifications:} 
Each UAS, $j$, given the mission as an STL specification $\varphi_j$, generates a trajectory that robustly satisfies $\varphi_j$. The \emph{robustness value} $\rho_{\varphi_j}$, associated with this trajectory, corresponds to the maximum deviation from the planned trajectory such that the UAS $j$ still satisfies its mission $\varphi_j$.
%see Corollary~\ref{cor:rob_tube}. 
%see Section~\ref{sec:stl_rob_short}.
\end{enumerate}

Two UAS within communication range share a look-ahead of planned trajectories and if a future collision is detected, new trajectories %with deviations 
are needed that still satisfy their original mission specifications. 
%We formalize this problem of predictive collision avoidance (problem \ref{prob:CA_highlevel}) in Section \ref{sec:CA}, and Section \ref{sec:subsec_milp} presents a centralized MILP encoding for the problem. 
%Instead of solving this centralized MILP, 
For this, we develop our decentralized approach L2F, which consists of two stages:

\begin{enumerate}
  \setcounter{enumi}{1}
\item \textit{Collision detection} and \textit{Conflict resolution:} 
When a potential collision is detected, a supervised-learning based conflict resolution policy (CR-S), with pre-defined priority among the two UAS, generates a sequence of discrete decisions corresponding to maneuvers to avoid the collision. 

\item \textit{Distributed and co-operative Collision Avoidance MPC (CA-MPC):} The CA-MPC for each UAS takes as input the conflicting trajectories and the output of the conflict resolution policy, and controls the UAS to avoid collision. 
\end{enumerate}



%Formalization of the two UAS collision avoidance problem is given in Section~\ref{sec:CA}. 
%Section~\ref{sec:subsec_milp} presents a centralized solution to the problem as a 
%Mixed-Integer Linear Program (MILP). 
%Sections \ref{sec:CA_MPC} and \ref{sec:learning_supervised} outline our proposed decentralized approach to the problem, where instead of solving computationally expensive MILP, we first solve a discrete 
%conflict-resolution problem using the learning-based scheme $\crLSTM$, and then a %distributed and co-operative collision avoidance MPC controller. 
In Section \ref{sec:experiments} we evaluate our framework for 2 UAS collision avoidance through extensive simulations and compare its performance to other approaches. 
%Section \ref{sec:casestudy} demonstrates how we combine the trajectory generation method of Section \ref{sec:problem_planning} and the collision avoidance schemes developed in this paper into a UTM framework. 
Section \ref{sec:casestudy} demonstrates a particular UTM framework case study.  
Finally, in Section \ref{sec:future} we discuss potential future directions.% to overcome the limitations of the current approach.

%Proposed learning-based conflict resolution scheme is presented in Section~\ref{sec:learning_supervised}.

