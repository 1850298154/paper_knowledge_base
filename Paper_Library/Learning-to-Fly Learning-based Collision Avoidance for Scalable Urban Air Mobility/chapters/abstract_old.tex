\begin{abstract}
\todo{Abstract is $\leq$200 words! The paper is 6 for free, 8 for gold.}
With increasing urban population densities, there is a global interest in Urban Air Mobility (UAM), where hundreds of autonomous Unmanned Aircraft Systems (UAS) execute missions in the airspace above major cities for moving people and goods. 
The two fundamental safety challenges in UAM are with urban air traffic management and airborne collision avoidance. 
Unlike traditional human-in-the-loop air traffic management, UAM requires decentralized autonomous approaches that scale for an order of magnitude higher aircraft densities and are applicable to urban settings. %and operate along unconstrained paths with free flight.  
We present Learning-to-Fly, a decentralized mission planning and on-demand airborne collision avoidance framework for multiple UAS to execute missions with spatial, temporal and reactive mission objectives while sharing the same airspace for collision-free flight. 
Using Signal Temporal Logic to capture mission specifications, our approach synthesizes trajectories with safety and performance guarantees. 
For on-demand airborne collision avoidance, we develop a two-stage method that consists of: 
1) Learning-based conflict resolution to decide a sequence of maneuvers for the two UAS to execute 
cooperatively avoid potential conflicts arising in the near future, and 
2) CA-MPC, a distributed Model Predictive Control scheme for the two UAS to execute these maneuvers for collision avoidance.
Through extensive simulations, we show the real-time applicability of the supervised learning method for conflict resolution. We also compare its performance to a greedy approach, a randomization-based approach and a Mixed Integer Programming (MIP) based approach. 
Our method is successful in resolving collision over $99\%$ of the time, and with a run-time in milliseconds, which is orders of magnitude faster than the MIP-based approach. 
Learning-to-Fly is demonstrated on real quadrotor drones. 
	
%	With increasing urban population densities, there is a global interest in Urban Air Mobility (UAM), where hundreds of autonomous Unmanned Aircraft Systems (UAS) execute missions in the airspace above major cities for moving people and goods. 
%The two fundamental safety challenges in UAM are with urban air traffic management and airborne collision avoidance. 
%Unlike traditional air traffic management which operates in a centralized manner, UAM requires decentralized approaches that scale for an order of magnitude higher aircraft densities and operate along unconstrained paths with free flight.  
%We present Learning-to-Fly, a decentralized mission planning and on-demand airborne collision avoidance framework for multiple UAS to execute missions with spatial, temporal and reactive mission objectives while sharing the same airspace for collision-free flight. 
%Using Signal Temporal Logic to capture mission specifications, our approach synthesizes trajectories with safety and performance guarantees. 
%For on-demand airborne collision avoidance, we develop a two-stage method that consists of: 
%1) Learning-based conflict resolution to decide a sequence of maneuvers for the two UAS to execute 
%cooperatively avoid potential conflicts arising in the near future, and 
%2) CA-MPC, a distributed Model Predictive Control scheme for the two UAS to execute these maneuvers for collision avoidance.
%Through extensive simulations, we show the real-time applicability of the supervised learning method for conflict resolution. We also compare its performance to a greedy approach, a randomization-based approach and a Mixed Integer Programming (MIP) based approach. 
%Our method is successful in resolving collision over $99\%$ of the time, and with a run-time in milliseconds, which is orders of magnitude faster than the MIP-based approach. 
%Learning-to-Fly is demonstrated on real quadrotor drones. 
\end{abstract}
