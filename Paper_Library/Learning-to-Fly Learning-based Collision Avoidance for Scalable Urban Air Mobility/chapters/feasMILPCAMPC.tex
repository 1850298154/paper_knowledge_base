%feasMILPCAMPC
%\subsection{Results on feas of MILP and CA-MPC}

%%Let $\mathcal{M} = \{x_1^M,x_2^M,u_1^M,u_2^M,\gamma^M\}$ be the (appropriate dimension-ed) solutions of the MILP \eqref{eq:CentralMILP} for given (conflicting) trajectories of the 2 drones. 

%%The following result is now established: The $H_i$, $g_i$ selected at time steps $k$ by the boolean variables $\gamma$ are such that the distributed two drone CA-MPC optimizations \eqref{eq:drone1mpc} and \eqref{eq:drone2mpc} with these $H_{i[k]}$ and $g_{i[k]}$ are feasible and result in new trajectories that do not conflict. This will be captured in the following lemmas and theorem (to be formally written later):

%\begin{enumerate}
%\item Lemma: The feasibility of optimization of \eqref{eq:drone1mpc}, and the existence of optimal solution such that $\sum_k \lambda_2[k] = 0$ for \eqref{eq:drone2mpc} implies that the resulting two drone trajectories are non-conflicting and within the robustness tubes of the initial trajectories.
%
%\item Lemma: The components of $\mathcal{M}$ result in a feasible solution for optimization \eqref{eq:drone1mpc} and solution for \eqref{eq:drone2mpc} such that $\lambda_2[k]=0\,\forall k$. 
%
%\item Theorem: feasible MILP $\Rightarrow$ two drone non-conflicting CA-MPC with same maneuvers. 
%
%\item Theoem: IFF result if needed.
%
%\end{enumerate}

%The CA-MPC optimization formulations have the following main result:
%
%\begin{theorem}
%\label{th:CAMPC_success}
%The feasibility of optimization of \eqref{eq:drone1mpc}, and the existence of optimal solution such that $\sum_k \lambda_{2,k} = 0$ for \eqref{eq:drone2mpc} implies that the resulting two drone trajectories are non-conflicting and within the robustness tubes of the initial trajectories.
%\end{theorem}


%Next, the following result captures the relation between the 
%existence of a feasible solution for the MILP approach and the 
%success of CA-MPC optimizations for the two UAS:

\begin{theorem}[Existence of the zero-slack solution]
	\label{th:MILP_CAMPC_relation}
	Feasibility of the MILP problem~\eqref{eq:CentralMILP} implies the existence of the zero-slack solution of CA-MPC optimization \eqref{eq:campc}.
\end{theorem}

The Theorem~\ref{th:MILP_CAMPC_relation} states that the binary decision variables $b^i_k$ selected by the feasible solution of the MILP problem \eqref{eq:CentralMILP}, when used to select the constraints (defined by $H,\,g$) for the CA-MPC formulations for UAS 1 and 2, imply the existence of a zero-slack solution of \eqref{eq:campc}.
%We use this result as a motivation for the learning-based conflict resolution scheme of the next section. 

%Next, the following result captures the relation between the 
%existence of a feasible solution for the MILP approach and the 
%success of CA-MPC optimizations for the two UAS:
%
%\begin{theorem}
%\label{th:MILP_CAMPC_relation}
%The non-zero binary variables $b^i_k$ selected by a feasible MILP formulation of eq. \ref{eq:CentralMILP}, when used to select the constraints (defined by $H,\,g$) for the CA-MPC formulations for UAS 1 and 2, imply the existence of: 1) a feasible solution for the optimizations of \eqref{eq:drone1mpc} and 2) a solution for \eqref{eq:drone2mpc} such that $\lambda_{2,k}=0\,\forall k$.  
%\end{theorem}

%\textit{Proof sketch:} The proof is through construction. Plugging in the $\mathbf{u}'_1$ and $\mathbf{u}'_2$ obtained through the MILP (along with the constraints $H^i_k,\, g^i_k$ selected by the binary variables in the MILP) into eqs. \eqref{eq:drone1mpc} and \eqref{eq:drone2mpc} respectively will result in the constraints of eq. \eqref{eq:drone1mpc} being feasible (with some $\mathbf{\lambda_1} \geq 0$) as well as the constraints of eq. \ref{eq:drone2mpc} being feasible with $\mathbf{\lambda_2} = 0$.

