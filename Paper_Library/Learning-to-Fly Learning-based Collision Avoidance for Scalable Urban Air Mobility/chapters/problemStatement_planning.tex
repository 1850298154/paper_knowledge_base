\subsection{UAS planning with STL specifications}
\label{sec:problem_planning}

Fly-by-logic \cite{pant2018fly} generates trajectories by centrally planning for fleets of UAS with STL specifications, e.g. the specification $\varphi_{\text{reach-avoid}}$ of example \ref{ex:reach_avoid_exmp}. 
It maximizes a smooth approximation $\srob_\formula$ of the robustness function \cite{pant2017smooth} by picking waypoints (connected via jerk-minimzing splines \cite{MuellerTRO15}) for all UAS through a centralized, non-convex optimization.

While successful in planning for multiple multi-rotor UAS, performance degrades as the number of UAS being planned for increases. The non-convex optimization involving the variables of all the UAS becomes harder as the number of variables increase \cite{pant2018fly} in particular because for $J$ UAS, $J \choose 2$ terms for pair-wise separation between the UAS are needed. 
%For these reasons, the method cannot be used for real-time planning. 
%In addition, for larger airspaces with a smaller number of UAS, centrally considering all pairwise trajectories may not be neccessary as conflicts could be infrequent.
% \footnote{KJ: I think the reasoning is counter to the setting for the problem and should either be changed or removed}
%In addition, in a larger airspace
Taking this into account, we use the underlying optimization of  \cite{pant2018fly} to generate trajectories%\footnote{we could also have used the method of \cite{Raman14_MPCSTL}, but instead use \cite{pant2018fly} because of it being computationally faster and tailored for multi-rotor robots.}
, but ignore the mutual separation requirement, allowing each UAS to independently (and in parallel) solve for their own STL specification. For the timed reach-avoid specification \eqref{eq:timed_RA} in example \ref{ex:reach_avoid_exmp}, this is equivalent to each UAS generating its own trajectory to satisfy $\varphi_j = \varphi_{reach, j} \land \varphi_{avoid, j}$, independently of the other UAS. Associated with these trajectories, $\sstraj_j$ is a robustness values $\rho_{\varphi_j}$. 
%The robustness value of the generated trajectory for $\sstraj_j$, $\rho_{\varphi_j}$, defines the robustness tube around the trajectory of UAS $j$.
Ignoring the collision avoidance requirement ($\varphi_\text{separation}$) in the planning stage allows for the specification of \eqref{eq:timed_RA} to be decoupled across UAS, but now requires online pairwise UAS collision avoidance if the planned trajectories are in conflict. This is covered in the following section.



 
%This results in the following problem:
%\begin{myprob}
%\label{prob:STLtraj}
%For UAS $j$, given STL specification $\varphi_j$, generate trajectory 
%$\sstraj_j$ such that $\sstraj_j \models \varphi_j$. 
%\end{myprob}
%
%Each UAS solves this problem (to generate a trajectory of duration 
%$hrz(\varphi_j)$) by solving an optimization of the form:
%
%\begin{equation}
%\label{eq:single_FbL}
%\text{max}_{\mathbf{w}_j} \,\srob_{\varphi_j}(L\mathbf{w}_j)\, \text{subject to} \, F\mathbf{w_j} \leq l
%\end{equation}

%\begin{equation}
%\label{eq:single_FbL}
%\begin{split}
%\text{max}_{\mathbf{w}_j} \,\srob_{\varphi_j}(L\mathbf{w}_j) \\
%F\mathbf{w_j} \leq l
%\end{split}
%\end{equation}
%Here, $\srob_{\varphi_j}$ is the smooth robustness, and $L$ is a linear map that takes a sequence of waypoints $\mathbf{w}_j$ (vectors in 3D position and velocity space) to generate a discrete-time trajectory $\sstraj_j=L(\mathbf{w}_j)$ of duration $hrz(\varphi_j)$. $F,\,l$ define linear constrains on the waypoints.
%This trajectory consists of spline segments that minimize the integral of jerk over time going from one waypoint to the next \cite{MuellerTRO15}. 
%The linear constraints on the waypoints are such that they ensure kinematic feasibility (Theorem 3.2 in \cite{pant2018fly}) of the resulting trajectories, i.e. the velocities and accelerations of the UAS are within predefined bounds. 
%For brevity, we omit the details here, but the full formulation can 


%For the timed reach-avoid specification \eqref{eq:timed_RA} in example \ref{ex:reach_avoid_exmp}, this is equivalent to each UAS generating its own trajectory to satisfy $\varphi_j = \varphi_{reach, j} \land \varphi_{avoid, j}$, independently of the other UAS. Ignoring the collision avoidance requirement ($\varphi_\text{separation}$) in the planning stage allows for the specification of \eqref{eq:timed_RA} to be decoupled across UAS, but now requires online pairwise UAS collision avoidance if the planned trajectories are in conflict. This is covered in the following section.


%We simplify the problem by allowing each UAS to plan for itself independently of the others and rely on online collision avoidance to ensure pairwise that UAS do not crash with each other. 


\textbf{Note:} 
In the following sections, $\mathbf{x}$ will refer to a full-state (discrete-time, finite duration) trajectory for a UAS. We will also use $\mathbf{p}$ to refer to the position components in that trajectory, the position trajectory. $x_k$ (and $p_k$) refer to the components of the trajectory at time step $k$.

%The robustness value $\rho_d*$ achieved through solving this optimization can be interpreted as a discrete tube in the position space $P_d$ such that as long as the discrete time positions of the UAS remain inside the robustness cubes of size $2\rho_d$ centered around the planned positions, $\phi_d$ is satisfied (see Theorem \ref{thm:rob objective}). See fig \ref{fig:concept} for a visualization of this interpretation. It should be noted that such a trajectory (and its robustness tube) can be generated by other methods as well, e.g. that of \cite{Raman14_MPCSTL}, but we use \cite{pant2018fly} because of it being computationally faster and tailored for multi-rotor robots.