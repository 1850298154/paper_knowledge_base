\subsection{Experimental setup}

All the simulations were performed on a computer with an AMD Ryzen 7 2700 8-core processor and 16GB RAM, running Ubuntu 18.04. 
%\footnote{Videos for the simulations and demonstrations can be viewed at \url{https://www.youtube.com/channel/UCs-5zrF1oNrgCfntydgA39g/playlists}}. 
The MILP formulation was implemented in MATLAB using Yalmip \cite{lofberg2004yalmip} with MOSEK v8 as the solver. The learning-based approach was implemented in Python 3 with Tensorflow 1.14 and Keras API and Casadi with QPOASES as the solver. 
We implemented the CA-MPC using CVXGEN for a measurement of 
computation times and real-time implementation for experiments of actual hardware.

%\begin{figure}[tb]
%	\begin{center}
%		\includegraphics[width=0.42\textwidth,trim={0 0.2cm 0 3.5cm},clip]{figures/scenario1_w_ca_traj_plot.png}
%	\end{center}
%	\vspace{-10pt}
%	{\footnotesize
%		\caption{\small Trajectories for 2 UAS. The dotted (planned) trajectories have a collision at the halfway point. The solid ones, generated through L2F method, avoid the collision while remaining within the robustness tube of the original trajectories. Playback of scenario at: \url{https://tinyurl.com/sk5kyvl}}
%		\label{fig:scen1_w_ca}}
%\end{figure} 

%\begin{figure}[tb]
%	\begin{center}
%		\includegraphics[width=0.42\textwidth]{figures/non_parabola2.png}
%	\end{center}
%	\vspace{-10pt}
%	{\footnotesize
%		\caption{\small Trajectories for 2 UAS from different angles. The dashed (planned) trajectories have a collision at the halfway point. The solid ones, generated through L2F method, avoid the collision while remaining within the robustness tube of the original trajectories. Initial UAS positions marked as stars. Playback of scenario at: \url{https://tinyurl.com/y8cm65ya}}
%		\label{fig:scen1_w_ca}\vspace{-10pt}}
%\end{figure} 

For the experiments, we set minimum separation to $\delta = 0.1$m. 
The learning-based CR scheme was trained for $\rho = 0.055$ which is close to the lower bound in assumption \ref{assumption1}.

We have generated the data set of 14K training and 10K test conflicting trajectories using the minimum-jerk trajectory generation algorithm from~\cite{pant2018fly}.
The time horizon was set to $T=4$s and $dt=0.1$s. 
%Thus, each trajectory consists of $N+1=41$ time-steps. 
The initial and final waypoints were sampled uniformly at random from two 3D cubes close to the fixed collision point, initial velocities were set to zero.
%An example trajectories are presented in Fig.~\ref{fig:scen1_w_ca}.

%We have generated the data set of $27K$ conflicting trajectories $\mathbf{x}_1$, $\mathbf{x}_2$ with initial and final waypoints uniformly sampled from two 3D cubes close to the fixed collision point $(1,1,1)$. Duration was
%By solving the centralized MILP problem~\ref{eq:CentralMILP} for such conflicting trajectories, the sequence of binary decision variables $\mathbf{b}\in\{0,1\}^{6\times 41}$, and therefore, the deconfliction sequence $\mathbf{d}\in\{1,\ldots,6\}^{41}$ was obtained.
%For each pair of such trajectories the sequence $\mathbf{z}$ was defined using~\eqref{eq:noconf}.

We have trained and ran experiments for various network 
configurations. 
For each model, the number of training epochs was set to 2K with a batch size of 2K. Each network was trained to minimize categorical cross-entropy loss using Adam optimizer with training rate of $0.001$.
%and moment exponential decay rates of
%$\beta_1= 0.9$ and $\beta_2=0.999$.
The model with 3 LSTM layers with 128 neurons each has 
been chosen as the default learning-based CR model.