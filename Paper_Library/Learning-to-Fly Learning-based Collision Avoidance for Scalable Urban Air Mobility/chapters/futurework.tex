%\vspace{-5pt}
\section{Conclusion}
\label{sec:future}
%\vspace{-2pt}
%Further test and evaluate the scheme for some common scenarios. Extend algorithm and analysis for more than 2 UAS colliding in close succession. 

\textbf{Summary.}
We developed \textit{Learning-to-Fly} (L2F), a two-stage, on-the-fly and predictive collision avoidance approach that combines learning-based decision-making for conflict resolution with decentralized linear programming-based UAS control. %consists of: 
%1) Learning-based conflict resolution to decide a sequence of maneuvers for the two UAS to execute 
%cooperatively avoid potential conflicts arising in the near future, and 
%2) CA-MPC, a distributed Model Predictive Control scheme for the two UAS to execute these maneuvers for collision avoidance.
Through extensive simulations and demonstrations on real quadrotor drones we show that L2F, with a run-time $<10$ms is computationally fast enough for real-time implementation. It is successful in resolving $100\%$ of collisions in most cases, with a graceful degradation to the worst-case performance of $90\%$ when there is little room for the UAS to maneuver. %Simulations show that L2F outperforms ot
L2F also enables independent UAS planning, %to plan out their missions independently of each other, 
 speeding up the process compared to centrally planning for all the UAS in the airspace. A 4-UAS case study shows that the independent planning is $3.5\times$-faster. 
% , and with a run-time in milliseconds, which is orders of magnitude faster than the MIP-based approach. 


\textbf{Limitations and Future Work.}
While pairwise collision avoidance is sufficient when the airspace density is low, in the future we will extend the approach to cases where more than two UAS could be in conflict with each other. 
As L2F does not always succeed, we plan to investigate this further and use the failure cases as counterexamples to make the learning-based models better. In general, we expect L2F to be realized within a larger UTM system with additional contingencies (e.g. FAA Lost Link procedures \cite{pastakia_faa_2015}), including the possibility of online re-planning of missions when L2F cannot guarantee collision avoidance.
%Here, we provide an example of a contingency protocol for when L2F fails which is inspired from the LOST-LINK protocol used by the FAA\cite{james_won}.
%The protocol consists of four stages addressing failure modes of increasing severity. 
%In the case that L2F is unable to resolves the conflict, the UAS with lower priority will:
%\begin{enumerate}
%	\item (LOITER) Move to a designated safe loiter zone until the conflict is resolved. 
%	\item (RETURN TO ORIGIN) If continuing the mission is impossible and on-board resources are sufficient.
%	\item (EMERGENCY LANDING IN ZONE) If near a pre-designated safe landing area 
%	\item (EMERGENCY CRASH) If all else fails, the UAS attempts to land by causing minimal damage.
%\end{enumerate}
%Such protocols and regulations will be necessary to ensure safe, reliable traffic management in urban airspaces.

