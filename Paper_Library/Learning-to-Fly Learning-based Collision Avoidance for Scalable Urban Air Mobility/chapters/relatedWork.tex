\subsection{Related work}
\label{sec:relatedwork}
\subsubsection{UTM and Automatic Collision Avoidance approaches}
The UAS Traffic Management (UTM) problem has been studied in various contexts. 
In the NASA/FAA Concept of Operations document \cite{FAA2018UTM}, an airspace allocation scheme is outlined where individual UAS reserve airspace in the form of 4D polygons (space and time), and the polygons of different UAS are not allowed to overlap. Similarly, \cite{maxetal} presents a voxel-based airspace allocation approach. Our approach is less restrictive and allows overlaps in the 4D polygons, but performs maneuvers for collision avoidance when two UAS are on track to a collision (see Fig. \ref{fig:CAdiagram}). TCAS \cite{TCAS} and ACAS \cite{kochenderfer2012next} systems for collision avoidance in commercial aircrafts rely on transponders in the two aircrafts to communicate information for the collision avoidance modules. 
These generate recommendations for the pilots to follow and create vertical separation between aircrafts \cite{ACASX}. 
In the context of UAS, \cite{UTMTCL4} uses vehicle-to-vehicle communication and tree-search based planning to achieve collision avoidance. 
ACAS-Xu \cite{ACASXu}, an automatic collision avoidance scheme for UAS relies on a look-up table to provide high-level recommendations to two UAS that have potentially colliding trajectories. 
%ACAS-Xu is already finding implementation on real-world UAS \cite{MQ4C}. 
%It provides recommendations only and needs an underlying control algorithm to carry out its recommendations. 
%ACAS-Xu restricts maneuvering to a single axis of motion based, vertical or horizontal based on if traffic is cooperative or not
 It restricts desired maneuvers for CA to the vertical axis for cooperative traffic, and the horizontal axis for uncooperative traffic. 
While we consider only the cooperative case in this work, our method does not restrict CA maneuvers to any single axis of motion. Finally, in its current form, ACAS-Xu also does not take into account any higher-level mission objectives, unlike our approach. 
This excludes its application to low-level flights in urban settings, e.g. it can result in situations where ACAS-Xu recommends an action that avoids a nearby UAS but results in the primary UAS going close to a static obstacle. 
Our method avoids this as CA maneuvers are restricted to keeping UAS inside \textit{robustness tubes} (see Fig. \ref{fig:concept}) such that mission requirements are not violated. 
For this reason, ACAS-Xu is currently only being explored for large, high-flying UAS \cite{ACASXu} and is not directly applicable to the problem we study here.

\subsubsection{Multi-agent planning with temporal logic objectives}
Many approaches exist for the problem of planning for multiple robotic agents with temporal logic specifications. 
Most rely on abstract grid-based representations of the workspace \cite{SahaRSJ14, DeCastro17}, or abstract dynamics of the agents \cite{Drona,AksarayCDC16}.
\cite{MaICUAS16} combines a discrete planner with a continuous trajectory generator. 
Some methods \cite{4459804, 1582935, 1641832}   work for subsets of Linear Temporal Logic (LTL) that do not allow for explicit timing bounds on the mission requirements.
While \cite{SahaRSJ14} uses a subset of LTL, safe-LTL$_f$ that allows them to express reach-avoid specifications with explicit timing constraints. However, in addition to a discretization of the workspace, they also restrict motion to a simple, discrete set of motion primitives. 
The predictive control method of \cite{Raman14_MPCSTL} allows for using the expressiveness of the complete grammar STL for mission specifications. 
It handles a continuous workspace and linear dynamics of robots, however its reliance on mixed-integer encoding (similar to \cite{Saha_acc16,KaramanF11_LTLrouting}) for the STL specification limit use in planning/control for multiple agents in 3D workspaces as seen in \cite{pant2017smooth}. 
The approach of \cite{pant2018fly} instead relies on optimizing a smooth (non-convex) function for generating trajectories for fleets of multi-rotor UAS with STL specifications. 
%However, performance degrades as the number of UAS is increased, as it accounts for pairwise separation in a centralized manner. 
In our framework, we use the planning method of \cite{pant2018fly}, but we let each UAS plan independently of each other. 
We ensure the safe operation of all UAS in the airspace through the use of our predictive collision avoidance scheme.
%In our framework, we use the planning method \cite{pant2018fly}, but instead of jointly planning for all UAS in the airspace, we let each UAS plan individually and independently of each other. We ensure the safe operation of all UAS in the airspace through the use of our predictive collision avoidance scheme.

 



%CA stuff: D'Andrea non-cvx, Hadas et al LTL reactive, Fly-by-Logic,
%Kochenderfer and ACAS, ACAS-Xu
%Reachability based from the UMich guy (AR knows)


%TL-based planning: LTLx,f stuff, BluSTL, FbL/SoP, 

%\ypcomment{Do we need full TL section? Can we boil it down to a single section with theorem on robustness and smooth robustness.}
