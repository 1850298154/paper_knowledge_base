\section{Case study: Independent planning and L2F for a 4-UAS example}
\label{sec:casestudy}

Figure~\ref{fig:scenario} depicts a UAS case-study with a 
reach-avoid mission. Scenario consists of four UAS which must 
reach desired goal states within 4 seconds while avoiding the wall 
obstacle and each other. Each UAS $j\in\{1,\ldots,4\}$ specification can be defined as:
\begin{equation}
\label{eq:scenario_spec_d}
\varphi_j =  
\eventually_{[0,4]} (p_j \in \text{Goal})
\ \wedge\ 
\always_{[0,4]} \neg (p_j \in \text{Wall})
\vspace{-2pt}
\end{equation}
A pairwise separations requirement of $0.1$ meters is enforced for 
all UAS, therefore, the overall mission specification is:
\vspace{-2pt}
\begin{equation}
\label{eq:case_mission}
\varphi_{\text{mission}} = \bigwedge_{j=1}^4 \varphi_j\ \wedge\ 
\bigwedge_{j\not=j'} \always_{[0,4]}||p_j-p_{j'}||\geq 0.1
\vspace{-3pt}
\end{equation}

\begin{figure}[h]
	\includegraphics[width=0.49\textwidth,trim={0 0cm 0.2cm 4.9cm},clip]{figures/case_4drones.png}
	\vspace{-10pt}
	\caption{\small Workspace for the case study scenario. 
		Trajectories for 4 UAS (magenta stars) reaching their goal sets 
		(green boxes) within 4 seconds, while not crashing into the vertical 
		wall (in red). A pairwise separation requirement 
		of $0.1$m is enforced. Simulations are available at 
		\url{https://tinyurl.com/t8bwwqk}.}
	\label{fig:scenario}
	\vspace{-10pt}
\end{figure}

%\begin{table}[h!]
%	\vspace{-15pt}
%	\renewcommand{\arraystretch}{1.3}
%	\caption{Computation times for the centralized versus decentralized 
%		UAS planning for the case study scenario
%	}
%	\label{tbl:runtimes}
%	
%	\centering
%	\begin{tabular}{c|c|c|c|c|c|}
%		\cline{2-6}
%		\multicolumn{1}{l|}{}                   & 
%		\textbf{Centralized} & 
%		\multicolumn{4}{c|}{\textbf{Independent 
%				planning}}                   
%		\\ \cline{2-6} 
%		& \textbf{4 UAS}    & \textbf{UAS 1} & \textbf{UAS 2} & 
%		\textbf{UAS 3} & \textbf{UAS 4} \\
%		\hline\hline
%		\multicolumn{1}{|c|}{\textbf{Mean (s)}} & 
%		0.346               & 0.139          & 0.102           & 
%		0.086           & 0.098           \\ \hline
%%		\multicolumn{1}{|c|}{\textbf{SD (s)}} & 
%%		0.087            & 0.063   & 0.061       & 
%%		0.046           & 0.055            \\ \hline
%	\end{tabular}
%\vspace{-15pt}
%\end{table}

First, we solved the planning problem for all four UAS in a centralized manner following 
approach from~\cite{pant2018fly} 
Next, we solved the planning problem for each UAS $j$ and its specification $\varphi_j$ independently, with calling L2F on-the-fly, after planning is complete. This way, independent planning with the online collision avoidance scheme guarantees the satisfaction of the overall mission specification \eqref{eq:case_mission}.
%Experiments for the L2F approach over a set of random initial conditions showed a separation rate of $85\%$.

\textbf{Simulation results.}
We have simulated the scenario for 100 different initial conditions.
The average computation time to generate trajectories in a centralized manner was $0.35$ seconds. The average time per UAS when planning independently (and in parallel) was $0.1$ seconds. 
These results demonstrate a speed up of $3.5\times$ for the individual UAS planning versus centralized \cite{pant2018fly}. % in comparison with a centralized approach from \cite{pant2018fly}.


%\begin{table}[]
%	\renewcommand{\arraystretch}{1.3}
%	\caption{Success rate for the centralized planning vs LSTM-CA-MPC 
%	over random initial conditions}
%	\label{tbl:success_casestudy}
%	\centering
%	\begin{tabular}{l|l|l|}
%		\cline{2-3}
%		& LSTM                      & Centralized Planning       \\ 
%		\hline\hline
%		\multicolumn{1}{|l||}{Success Rate} & 
%		\multicolumn{1}{c|}{84.6\%} & \multicolumn{1}{c|}{100\%} \\ 
%		\hline
%	\end{tabular}
%\end{table}

%\arcomment{Table on CAMPC robustness/success rate/computation time?}

%\begin{table}[]
%	\begin{tabular}{lll}
%		\textbf{Algorithm} & \textbf{Robustness} & 
%\textbf{Computation Time} \\
%		Centralized        &                     
%&                           \\
%		CAMPC              &                     
%&                          
%	\end{tabular}
%\end{table}


%Interesting example where UAS plan individually and still don't 
%crash.
%We want to show that FbL for single drone/single spec + CAMPC is 
%computationally much faster than centralized FbL over all drones. 
%
%\begin{itemize}
%\item Robustness of the centralized FbL vs robustness of the single 
%drone FbL (but w.r.t the overall spec over all drones, we can get 
%this easily by looking at min of robustness over drones and min 
%separation achieved, so you don't need to call anything to compute) 
%+  CAMPC (we expect to take a hit in terms of robustness, unless the 
%goal sets and workspace are large).
%\item Success rate if not 100 percent (ideally it is 100 percent for 
%both?)
%\end{itemize}
