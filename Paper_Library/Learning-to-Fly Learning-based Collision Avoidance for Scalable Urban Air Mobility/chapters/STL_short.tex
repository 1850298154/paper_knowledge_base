\subsection{Introduction to Signal Temporal Logic}
\label{sec:STL_intro}
Signal Temporal Logic (STL) \cite{MalerN2004STL} is a behavioral specification language that can be used to encode requirements on signals. 
%Signal Temporal Logic (STL) \cite{MalerN2004STL} is a behavioral specification language that can be used to encode logical relationships between signals, along with the time intervals over which these relationships should hold. 
%An STL specification evaluates to either \textit{True} or \textit{False} given the signals it is defined over. 
The grammar of STL~\cite{Raman14_MPCSTL} allows for capturing a rich set of 
behavioral requirements using temporal operators, such as \textit{Always} ($\always$) and \textit{Eventually} ($\eventually$), % and \textit{Until} ($\until$) operators, 
 as well as logical operators like \textit{And} ($\land$), \textit{Or} ($\lor$), and \textit{negation} ($\neg$). 
With these operators, an STL specification $\varphi$ is defined over a signal, e.g. over the trajectories of quad-rotor robots, and evaluates to either \textit{True} or \textit{False}. 
The following example demonstrates STL to capture operational requirements for two UAS:
%These operators are used to define an STL specification $\varphi$, defined over a signal, e.g. over the trajectories of quad-rotor robots. 
%The following example shows the use of STL to capture operational requirements for two UAS:

\begin{exmp}
\label{ex:reach_avoid_exmp}
\textit{(A two UAS timed reach-avoid problem)} 
Two quad-rotor UAS are tasked with a mission with spatial and temporal requirements in the workspace shown in Fig. \ref{fig:dronetube}:
%We develop a specification where two quad-rotor UAS are tasked with a mission with spatial and temporal requirements in the workspace shown in Fig. \ref{fig:dronetube}:

\begin{enumerate}

\item The two UAS have to reach a $\text{Goal}$ set (shown in green), 
or a region of interest, within a time of $6$ seconds after starting. 
UAS $j$ (where $j\in \{1,2\}$), with position denoted by $p_j$, has to satisfy: $\varphi_{reach, j} = 
\eventually_{[0,6]} (p_j \in \text{Goal})$.
The \textit{Eventually} operator over the time interval $[0,6]$ requires UAS $j$ to be inside the set $\text{Goal}$ at some point within $6$ seconds. 

\item In addition, the two UAS also have an $\text{Unsafe}$ (in red) set to avoid, e.g. a no-fly zone. For each UAS $j$, this is encoded with \textit{Always} and \textit{Negation} operators:

$\varphi_{\text{avoid},j} = \always_{[0,6]} \neg (p_j \in 
\text{Unsafe})$

\item Finally, the two UAS should also be separated by at least $\delta$ meters along every axis of motion:

$\varphi_{\text{separation}} = \always_{[0,6]} ||p_1 - p_2||_{\infty} 
\geq \delta$

\end{enumerate}

The 2-UAS timed reach-avoid specification is thus:
\begin{equation}
\label{eq:timed_RA}
\varphi_{\text{reach-avoid}} = \land_{j=1}^2 ( \varphi_{\text{reach},j} \land 
\varphi_{\text{avoid},j}) \land \varphi_\text{separation}
\end{equation}
\end{exmp}

In order to satisfy $\varphi$, a planning method generates trajectories $\mathbf{p}_1$ and $\mathbf{p}_2$ of a duration at least $hrz(\varphi)= 6$s, where $hrz(\varphi)$ is the time \textit{horizon} of $\varphi$. 
If the trajectories satisfy the specification, i.e. $(\mathbf{p}_1,\, \mathbf{p}_2) \models \varphi$, then the specification $\varphi$ evaluates to \textit{True}, otherwise it is \textit{False}. 
In general, an upper bound for the time horizon can be computed as shown in \cite{Raman14_MPCSTL}. 
In this work, we consider specifications such that the horizon is bounded. More details on STL can be found in \cite{MalerN2004STL} or \cite{Raman14_MPCSTL}. 
In this paper, we consider discrete-time STL semantics which are defined over discrete-time trajectories.

%In order to satisfy $\varphi$, our planning method needs to generate trajectories $\mathbf{p}_1$ and $\mathbf{p}_2$ of a duration of at least $hrz(\varphi)=6$s, where $hrz(\varphi)$ is the time \textit{horizon} associated with $\varphi$. If the two UAS trajectories satisfy the STL specification $(\mathbf{p}_1,\, \mathbf{p}_2) \models \varphi$, then the specification $\varphi$ evaluates to \text{True}, otherwise it is false. In general, an upper bound for the time horizon for any STL specification can be computed as shown in \cite{Raman14_MPCSTL}. In this work, we consider STL specifications such that the horizon is bounded. More details on the grammar and semantics of STL can be found in \cite{MalerN2004STL} or \cite{Raman14_MPCSTL}. In this paper, we consider the discrete-time STL semantics, where specifications are defined over a discrete-time trajectories.

%The grammar of STL also allows for nesting of operators.
%
%\begin{exmp}
%\textit{Patrolling using nested STL operators:} If we want a UAS, with its position denoted by $p$ to alternately visit regions $\text{Goal}_1$ and then $\text{Goal}_2$ with no more than $2$ seconds between them, we can encode it in STL as:
%
%$\varphi_\text{patrol} = \always_{[0,8]}(\eventually_{[0,2]} (p \in \text{Goal}_1) \land \eventually_{[2,4]} (p \in \text{Goal}_2))$
%
%\end{exmp}
%
%Here, the specification horizon is $hrz(\varphi_\text{patrol}) = 8+2+2 = 12$ seconds. 
 



