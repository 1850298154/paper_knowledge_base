\subsection{Contributions of this work}

Our main contribution is \textit{Learning-to-Fly} (L2F)\footnote{Videos of the simulations and demonstrations in this paper can be viewed at \url{https://tinyurl.com/vvvuukh}}, a scheme 
for real-time, on-the-fly collision avoidance between two UAS whose main features are:

\begin{enumerate}
\item \textit{Systematic composition of machine learning and control theory:} We combine learning-based decision-making, and linear programming-based control to solve the problem in a decentralized manner. Unlike many other ad-hoc Machine Learning-based solutions, we provide a sound theoretical justification for our approach in Theorem \ref{th:MILP_CAMPC_relation}. We also provide a sufficient condition for the scheme to work successfully (Theorem \ref{th:CAMPC_success}).

\item  \textit{A notion of priority among the UAS} can be encoded naturally in L2F, where the UAS with higher priority does not have to deviate from its originally planned trajectory until absolutely necessary.

\item \textit{Computationally lightweight enough for real-time implementation:} Experimental results show that L2F, with a computation time in milliseconds can be used in a real-time implementation at a high-rate ($10$ Hz). 

\item \textit{High performance:} In the best case, L2F successfully results in 2-UAS collision avoidance $100\%$ of the test cases, gracefully degrading to $90\%$ for the worst case. Comparisons with other methods also show the superior performance of L2F. 

\item \textit{Enabling fast, independent planning for UAS with temporal logic objectives}, as individual UAS, or fleets of UAS run by the same operator, can plan for themselves without considering other UAS in the airspace while calling upon L2F for on-the-fly collision avoidance. For a 4-UAS case study, we demonstrate a speed up of $3.5\times$ over the centralized planning method of \cite{pant2018fly}.

\item \textit{Proof-of-concept demonstration} on Crazyflie quad-rotor robots to show feasibility on real UAS.

\end{enumerate}

%consisting of:
%\textcolor{green}{Too much repetition, shorten this.
%}
%\begin{enumerate}
%\item \textit{Learning-based conflict resolution:} Given the planned trajectories of two UAS that have a conflict in the near future, the scheme chooses a sequence of maneuvers such that the two UAS avoid colliding with each other without violating their higher level mission objective. We develop a supervised learning-based method for conflict resolution, \textit{\crLSTM} (Section~\ref{sec:learning_supervised}).
%
%\item \textit{CA-MPC: A distributed, cooperative convex Model Predictive 
%Control (MPC) algorithm for collision avoidance:} that takes in the desired 
%maneuvers from the conflict resolution scheme as constraints for each UAS to 
%satisfy such that the resulting trajectories are collision-free and also 
%satisfy the mission objective (Section \ref{sec:CA_MPC}).
%\end{enumerate}
%
%This collision avoidance scheme allows us to build a UAS planning and traffic management framework where we have:
%
%\begin{enumerate}
%\item  \textit{Fast, independent planning for UAS with temporal logic objectives}, as individual UAS, or fleets of UAS run by the same operator, can plan for themselves without having to consider the other UAS in the airspace. 
%\item  \textit{A notion of priority among the UAS}, where the UAS with higher priority does not have to deviate from its originally planned trajectory until absolutely necessary.
%\item  \textit{Real-time collision avoidance} as the CA framework relies on executing one learning based step and two convex optimizations. Experimental evaluations of our implementation show their low computation times.
%\end{enumerate}

%Through extensive simulations, we evaluate the performance of this collision avoidance scheme and show that this framework for planning and collision avoidance allows for faster and more scalable planning for UAS fleets with Signal Temporal Logic specifications than the centralized approach of \cite{pant2018fly}. Finally, through a proof-of-concept implementation on real quadrotor drones, we show this framework can be used in real-world settings. While our method of relying on a learning-based component does not result in collision-free trajectories for the UAS 1\% of the time on average, it outperforms other baseline approaches, and can be used for real-time implementations. \textcolor{green}{Shrink this and make consistent w/ final results}.