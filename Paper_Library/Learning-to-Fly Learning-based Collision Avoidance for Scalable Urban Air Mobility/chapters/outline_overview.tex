\subsection{Outline of the paper}

The following section (Section \ref{sec:overview}) presents an overview of the 
framework we develop to solve problems \ref{prob:traj_STL_highlevel} and 
\ref{prob:CA_highlevel}. Section \ref{sec:STL_intro} is an introduction to 
Signal Temporal Logic (STL), which we use for formally representing mission 
requirements. 
Section \ref{sec:problem_planning} presents the variant of the 
algorithm from \cite{pant2018fly} that we use for UAS trajectory generation 
for satisfying STL specifications. 
Section \ref{sec:CA} formalizes the two UAS 
collision-avoidance problem and formulates the solution to it as a 
Mixed-Integer Linear Program (MILP). 
Sections \ref{sec:CA_MPC} and \ref{sec:learning} outlines our approach to the problem, where instead of solving the computationally expensive MILP, we first solve a discrete 
conflict-resolution problem, and then a decentralized and co-operative 
collision avoidance MPC controller. 
These two components form the basis of our framework. 
We present a learning-based approach for the conflict resolution, as well as a greedy, computationally lightweight approach to the problem that serves as a baseline for comparison. 
In Section \ref{sec:experiments} we evaluate our framework for 2 UAS collision avoidance through extensive simulations and compare its performance to the MILP approach and the greedy algorithm. 
Section \ref{sec:casestudy} demonstrates how we combine the trajectory generation method of Section \ref{sec:problem_planning} and the collision avoidance schemes developed in this paper into a UTM framework. 
Finally in Section \ref{sec:future} we discuss potential future directions to overcome the limitations of the current approach.