In order to evaluate the feasibility of the deconflicted trajectories, the trajectories were flown using two Crazyflie quad-rotor robots.  
The two robots representing UAS 1 and UAS 2, respectively, flew 5 trajectories generated using Learning-to-Fly.
The UAS were programmed to execute both the initial conflicting 
trajectories and the deconflicted trajectories.
Pose estimates for the UAS were obtained from using the Vicon motion 
capture system, and recorded. 
Fig. \ref{fig:scen1_w_ca} shows one such trajectory (before and after 
collision avoidance). Learning-to-Fly de-conflicts the trajectories 
in a manner such that the actual quad-rotors can execute them 
successfully. 
Videos of additional experimental recordings and simulations can be found at \url{https://tinyurl.com/yxttq7l5}.


 %Overall, the drones executed the deconflicted trajectories 
 %successfully.
%The two drones representing UAS1 and UAS2, respectively, executed the planned trajectories with waypoints input at $10\,Hz$.
%The drones were programmed to execute both the initial conflicting trajectories and the deconflicted trajectories. 
%The position of the drones were recorded through a Vicon system at a rate of 100Hz and assessed for collision.

%An example of a scenario after deconfliction is shown in  
%Fig. \ref{fig:scen1_w_ca}. 
%As can be seen in Fig. \ref{fig:scen1_w_ca}, CA-MPC deconflicts the intial trajectories and the drone successfully complete the deconflicted trajectories.

 


