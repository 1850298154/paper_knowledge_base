\subsection{Results and comparison to other methods}
\label{sec:exp_results}

We analyzed three other methods alongside the proposed learning-based approach for L2F.
\begin{enumerate}
	\item A \textbf{random} decision approach which outputs a %decision 
	sequence sampled from the discrete uniform distribution.
	\item A \textbf{greedy} approach that selects the discrete decisions for which the most distance between the two UAS available at each time step. 
	\item A centralized \textbf{MILP} solution that picks a decision corresponding to a binary decision variable in \eqref{eq:CentralMILP}.
\end{enumerate}

%For the evaluation, we measured and compared: (i) the \textbf{success rate}, 
%(ii) the \textbf{separation rate} and (iii) the \textbf{computation time}.
%\textit{Success rate} defines the fraction of given conflicting trajectories, 
%for which the conflict resolution
%discrete decision sequence leads to the 
%zero-slack solution, see Definition~\ref{def:zero_slack}.
%%lead to zero slack variables for UAS 2: $\mathbf{\lambda_2}=0$. 
%In this case, as Theorem~\ref{th:CAMPC_success} states, 
%UAS achieved minimum separation.
%\textit{Separation rate} considers the 
%minimum separation requirement only with no condition on the slack 
%variables. Note that by the definition separation rate is always 
%higher than the success rate. Moreover, since MILP formulation does not have slack variables, we cannot compute success rate for the MILP approach.
For the evaluation, we measured and compared the \textbf{separation rate} and the \textbf{computation time} over 10K test trajectories.
\textit{Separation rate} defines the fraction of given initially conflicting trajectories for which UAS managed to achieve minimum separation.

\begin{figure}[t!]
	\begin{center}
		\includegraphics[width=0.49\textwidth, trim={0 0.5cm 0 0.25cm}]{figures/rho_rate_uniform.pdf}
	\end{center}
	\vspace{-5pt}
	{\footnotesize
		\caption{\small Model sensitivity analysis with respect to variations of fraction $\rho/\delta$, which connects the minimum allowable robustness tube radius $\rho$ to the minimum allowable separation between two UAS $\delta$, see Assumption~\ref{assumption1}. A higher $\rho/\delta$ implies there is more room within the robustness tubes to maneuver within for CA.}
		\label{fig:rho_rate_33}}
	\vspace{-10pt}
\end{figure} 

Figure~\ref{fig:rho_rate_33} shows the trade-off between performance in terms of separation rate and $\rho/\delta$ fraction, which defines the connection between the robustness tube $\rho$ and the minimum separation $\delta$. Higher $\rho/\delta$ implies wider robustness tubes for the UAS to maneuver within, which should make the CA task easier. In the case of $\rho/\delta=0.5$, where the robustness tubes are just wide enough to fit two UAS (see assumption~\ref{assumption1}), we see the L2F significantly outperforms the methods (excluding the MILP). As the ratio grows, the performance of all methods improve with L2F still outperforming the others, topping out to achieve a best case separation of $1$. The worst-case performance for L2F is $0.9$ which is again significantly better than the other approaches. 

Table \ref{tbl:success-time} %depicts the results of evaluation. 
shows the separation rates for three different $\rho/\delta$ values %where e.g. for $\rho\/delta=0.95$, the proposed \crLSTM{} algorithm reaches an overall 
%separation rate of $0.992$ over the $10$K trajectories, which is significantly higher than the random and greedy approaches (also see Fig. \ref{fig:rho_rate_33}
as well as the computation times for conflict resolution schemes plus the CA-MPC optimizations. In terms of separation rate, L2F outperformed the random and the greedy approaches. The centralized MILP outperformed the L2F, however, the computation time for the centralized approach was orders of magnitude higher than L2F. 
These shows the benefits of L2F compared to 
other approaches, especially when considering the success-computation time 
trade-off. 

%We also performed a sensitivity analysis for the different schemes, see Figure~\ref{fig:rho_rate_33}. 
%Figure~\ref{fig:rho_rate_33} shows the trade-off between performance in terms of separation rate and $\rho/\delta$ fraction, which defines the connection between the robustness tube $\rho$ and the minimum separation $\delta$. Higher $\rho/\delta$ implies wider robustness tubes for the UAS to maneuver within, which should make the CA task easier. In the case of $\rho/\delta=0.55$, where the robustness tubes are just wide enough to fit two UAS (see assumption~\ref{assumption1}), we see the L2F significantly outperforms the methods (excluding the MILP). As the ratio grows, the performance of all methods improve with L2F still outperforming the others, topping out to achieve a best case separation of \textcolor{green}{$x\%$}.
%One can see {\color{red} BLA. increasing trend. L2F achieves 100\%, random sucks.}

%The values have been chosen in the range to still satisfy assumption \ref{assumption1}.

\begin{table}[h]
	\vspace{-10pt}
	\renewcommand{\arraystretch}{1.3}
	\setlength{\tabcolsep}{2.5pt}
	\centering
	%	\begin{tabular}{|l||c|c|c|}
	%		\hline
	%		\multirow{2}{*}{\textbf{CA Scheme}} & \multirow{2}{*}{\textbf{Separation Rate}} & \multicolumn{2}{c|}{\textbf{Computation time}} \\ \cline{3-4} 
	%		&                                           & \textbf{Mean}           & \textbf{Std}          \\ \hline\hline
	%		\textbf{Random}                     & 0.64                                      & 2.02ms                  & 0.17ms               \\ \hline
	%		\textbf{Greedy}                     & 0.847                                     & 3.82ms                  & 0.25ms               \\ \hline
	%		\textbf{L2F}                        & 0.992                                     & 9.36ms                  & 1.75ms               \\ \hline
	%		\textbf{MILP}                       & 1                                         & 68.5s                   & 87.3s                \\ \hline
	%	\end{tabular}
	\begin{tabular}{|l||c|c|c||c|c|}
		\hline
		\multirow{2}{*}{\textbf{CA Scheme}} & \multicolumn{3}{c||}{\textbf{Separation Rate}} & \multicolumn{2}{c|}{\textbf{Computation time}} \\ \cline{2-6} 
		& \textbf{$\rho/\delta$ = 0.5}           & \textbf{$\rho/\delta$ = 0.95} & \textbf{$\rho/\delta$ = 1.15}     & \textbf{Mean}           & \textbf{Std}          \\ \hline\hline
		\textbf{Random}                     & 0.311                          & 0.609  &0.661           & 2.02ms                  & 0.17ms               \\ \hline
		\textbf{Greedy}                     & 0.529                         & 0.836  &0.994           & 3.82ms                  & 0.25ms               \\ \hline
		\textbf{L2F}                        & 0.901                         & 0.999   &1          & 9.36ms                  & 1.75ms               \\ \hline
		\textbf{MILP}                       & 1                             & 1   &1          & 68.5s                   & 87.3s                \\ \hline
	\end{tabular}
	\caption{{\small Separation rates and computation times (mean and standard deviation) comparison of different CA schemes. \textit{Separation rate} is the fraction of conflicting trajectories for which separation requirement
			\eqref{eq:msep} is satisfied after CA. 
			%	 $||p_{1,k}'-p_{2,k}'||\geq \delta \, \forall k = 0,\dotsc,N$ achieved satisfaction after collision avoidance.
	}}
	\label{tbl:success-time}
	\vspace{-5pt}
\end{table}

\begin{figure}[tb]
	\begin{center}
		\includegraphics[width=0.49\textwidth,trim={0 1cm 0 0.25cm}]{figures/non_parabola2.pdf}
	\end{center}
	\vspace{-2pt}
	{\footnotesize
		\caption{\small Trajectories for 2 UAS from different angles. The dashed (planned) trajectories have a collision at the halfway point. The solid ones, generated through L2F method, avoid the collision while remaining within the robustness tube of the original trajectories. Initial UAS positions marked as stars. Playback of the scenario is at \url{https://tinyurl.com/y8cm65ya}.}
		\label{fig:scen1_w_ca}\vspace{-10pt}}
\end{figure} 

Figure~\ref{fig:scen1_w_ca} shows an example of two UAS trajectories before and after collision avoidance through L2F method.
In addition, in order to evaluate the feasibility of the deconflicted trajectories, 
we have also ran experiments using two Crazyflie quad-rotor robots.   
%Crazyflie quad-rotor robots, running the Robot Operating System (ROS), were used for actual hardware.
Video recordings of the actual flights
%, tracking of the trajectories from Figure~\ref{fig:scen1_w_ca} 
%can be found at \url{https://youtu.be/Llu-zGALx9s} and \url{https://youtu.be/QM6VsDykDkQ}.
and additional simulations can be found at {\footnotesize\url{https://tinyurl.com/yxttq7l5}}.
