In this Section, we provide the proof of \cref{th:main}.

\paragraph{Notation}
We follow the standard notion of the differential $\D^{\gamma} f$ of a function $f:\inputSpace\to\bbR$, for a multi-index $\gamma\in\inputSpace$.

Following the setup of \cite{feffermanExtensionLinearOperators2007}, we denote by $\Cm(\inputSpace)$ the space of real-valued functions on $\inputSpace$ with continuous and bounded derivatives through order $\smoothDeg$; and
\begin{equation}
  \|F\|_{\Cm(\inputSpace)} = \max_{|\gamma|\le m} \sup_{x\in\bbR^{n}} |\D^{\gamma} F(x)|.
\end{equation}
Furthermore, for a subset $E$ of $\inputSpace$, we denote by $\Cm(E)$ the Banach space of all real-valued functions $\varphi$ on $E$ such that $\varphi=F$ on $E$ for some $F\in\Cm(\inputSpace)$.
The natural norm on $\Cm(E)$ is given by
\begin{align}
  \|F\|_{\Cm(E)} &= \inf_{\varphi\in\Cm(\inputSpace) \text{ and } \varphi = F \text{ on } E} \|\varphi\|_{\Cm(\inputSpace)}.
\end{align}
We collect here a simple remark:
\begin{lemma}\label{lmm:banachnorm}
  Consider two subsets $E$ and $F$ of $\inputSpace$.
  If $E\subset F$, then for any $f\in\Cm(F)$, there holds
  \begin{equation}
    \|f\|_{\Cm(E)} \le \| f \|_{\Cm(F)}.
  \end{equation}
\end{lemma}
\begin{proof}
The property follows from the fact that, since $E\subset F$, the set of functions $\varphi\in\Cm(\inputSpace)$ such that $\varphi=f$ on $F$ is contained in the set of functions $\varphi\in\Cm(\inputSpace)$ such that $\varphi = f$ on $E$.
\end{proof}

Finally, we need the following notion of distance between a pair of manifolds:
\begin{definition}
    The Frechet distance $d_F(\cdot,\cdot)$ between two homeomorphic manifolds is defined as
    \begin{equation*}
        d_F(\mathcal{M}_a,\mathcal{M}_b)=\inf_{h\in \mathcal{H}}\sup_{x\in\mathcal{M}_a}d(x,h(x)),
    \end{equation*}
    where $\mathcal{H}$ is the set of all homemorphisms from $\mathcal{M}_a$ to $\mathcal{M}_b$. 
\end{definition}

\subsection{A Jackson and extension theorem for subsets of $\cancube$}
We provide a variant of the Jackson theorem that allows for fast approximation of smooth functions.
The main difference relative to classical statements \eg{} \cite{plesniakMultivariateJacksonInequality2009,bagbyMultivariateSimultaneousApproximation2002}, is that the constant $\Cnm$ does not depend on the domain $E\subset\cancube$ where the function is approximated, only on the working domain $\cancube$.

\begin{lemma}\label{lmm:Jackson}
  Fix $\ndim$ and $\smoothDeg$ positive integers and assume that $\appDom$ is a connected compact subset of $\inputSpace$, such that any two points $x$ and $y$ in $\appDom$ can be joined by a rectifiable arc in $\appDom$ with length no greater that $\sigma\|x-y\|$, where $\sigma$ is a positive constant.
  
  Then, there exists a constant $\Cfeff$ that depends only on $\ndim$ and $\smoothDeg$, and a constant $\Cnm$ that depends only on $\ndim$, $\smoothDeg$, and $\cancube$ such that the following holds: for any subset $E$ of $\cancube$,
  \begin{enumerate}[i.]
      \item there exists a linear operator $T:\Cm(E)\to\Cm(\inputSpace)$ such that $T$ extends functions of $E$ to $\inputSpace$, and the operator norm of $T$ is bounded by $\Cfeff$, so that for any $f\in\Cm(E)$,
          \begin{equation}\label{eq:Feffext}
            \|T(f)\|_{\Cm(\inputSpace)} \le \Cfeff \|f\|_{\Cm(E)}.
          \end{equation}
      \item for any function $f\in\C^{\smoothDeg}(E)$, and integer $\polyDeg$, there exists a polynomial $p_{\polyDeg}$ of degree at most $\polyDeg$ such that
  \begin{equation}
    \| f - p_{\polyDeg}\|_{\infty, E} \le \Cnm \frac{1}{\polyDeg^{\smoothDeg}} \|f\|_{\Cm(E)}.
  \end{equation}
  \end{enumerate}
\end{lemma}



\begin{proof}
  The first item is exactly \Citet[Th. 1]{feffermanExtensionLinearOperators2007}.
  % , there exists a constant $\Cfeff$ such that, for any set $E\subset \cancube$, there exists a linear operator $T:\Cm(E)\to\Cm(\inputSpace)$ such that \emph{(i)} $T$ extends functions of $E$ to $\inputSpace$, and \emph{(ii)} the operator norm of $T$ is bounded by $\Cfeff$, so that for any $f\in\Cm(E)$,
  % \begin{equation}\label{eq:Feffext}
  %   \|T(f)\|_{\Cm(\inputSpace)} \le \Cfeff \|f\|_{\Cm(E)}.
  % \end{equation}
  For the second item, applying Jackson's theorem \cite[Th. 2]{bagbyMultivariateSimultaneousApproximation2002} to $T(f)$ on $\cancube$ provides the existence of a polynomial $p_{\polyDeg}$ of degree up to $\polyDeg$ such that:
  \begin{equation}\label{eq:Jackson}
    \|T(f) - p_{\polyDeg}\|_{\infty, \cancube} \le \Cjack \frac{1}{\polyDeg^{\smoothDeg}} \sum_{|\gamma|\le \smoothDeg}|\D^{\gamma} T(f)|_{\infty, \cancube}.
  \end{equation}
  Note that the assumptions of Theorem 2 \cite{bagbyMultivariateSimultaneousApproximation2002} are verified: $T(f)$ is indeed of class $\C^{\smoothDeg}$ on a neighborhood of $\cancube$.%, and $\cancube$ is a connected compact set, and such that any two points $a$ and $b$ of $\cancube$ can be joined by a rectifiable arc in $\cancube$ with length no greater that $\sigma\|a-b\|$, where $\sigma$ is a positive constant.

Finally, note that
  \begin{multline}\label{eq:normbound}
    \sum_{|\gamma|\le \smoothDeg}|\D^{\gamma} T(f)|_{\infty, \cancube} \le \sum_{|\gamma|\le \smoothDeg}|\D^{\gamma} T(f)|_{\infty, \inputSpace}  \\ \le \binom{\smoothDeg+\ndim}{\ndim} \max_{|\gamma|\le \smoothDeg}|\D^{\gamma} T(f)|_{\infty, \inputSpace} = \binom{\smoothDeg+\ndim}{\ndim} \| T(f) \|_{\Cm(\inputSpace)}.
  \end{multline}

  The above elements combine as follows,
  \begin{align*}
    \| f - p_{\polyDeg}\|_{\infty, E}
    &= \| T(f) - p_{\polyDeg}\|_{\infty, E} \\
    &\le \| T(f) - p_{\polyDeg}\|_{\infty, \cancube} \\
    &\overset{\eqref{eq:Jackson}}{\le} \Cjack \frac{1}{\polyDeg^{\smoothDeg}} \sum_{|\gamma|\le \smoothDeg}|\D^{\gamma} T(f)|_{\infty, \cancube} \\
    &\overset{\eqref{eq:normbound}}{\le} \Cjack \binom{\smoothDeg+\ndim}{\ndim} \frac{1}{\polyDeg^{\smoothDeg}} \|T(f)\|_{\Cm(\inputSpace)}  \\
    &\overset{\eqref{eq:Feffext}}{\le} \Cjack \binom{\smoothDeg+\ndim}{\ndim}  \Cfeff \frac{1}{\polyDeg^{\smoothDeg}} \|f\|_{\Cm(E)},
  \end{align*}
  which shows the claim.
\end{proof}

\subsection{Piecewise-linear approximation of the cell decomposition}
% \begin{definition}
%     A Freudenthal-Kuhn (FK) triangulation of $\inputSpace$ is given by the hyperplane arrangement $H_E=\{ x\in\inputSpace:\langle x,u\rangle=k,\,u\in E,\,k\in\mathbb{Z}\}$ where $E$ is the set of vectors $E\coloneqq \{e_1,\dots, e_\ndim\}\cup\{u_{ij}=e_j-e_i:1\leq i<j\leq \ndim\}$. 
% \end{definition}

% The statement from \cite{boissonnatTracingIsomanifoldsTime2021}:
% \begin{lemma}
%   Consider a $p$-dimensional manifold $\mathcal{M}$ defined as the zero set of a smooth function $f$, and such that it is a submanifold of $\inputSpace$ contained in $\cancube$. Assume that $f$ has bounded complexity, and that any $k$-flat intersects $\mathcal{M}$ at most $K$ times. There exists an algorithm (from \cite{boissonnatTracingIsomanifoldsTime2021,boissonnat2022topological}) which provides a piecewise linear approximation $\hat{\mathcal{M}}$ of $\mathcal{M}$, with FK triangulation given by $\mathcal{T}$, such that
%   \begin{enumerate}
%       \item $d_F(\mathcal{M},\hat{\mathcal{M}})\leq C_f D^2$, where the constant $C_f$ depends on the gradient and Hessian of the function $f$, and $D$ is the longest edge of a simplex in $\mathcal{T}$;
%       \item the number of simplices, $N$, is bounded by $N\leq \frac{K}{p!}\left(\frac{\ndim^3}{\sqrt{2}D}\right)^p$. 
%   \end{enumerate}

%   Furthermore, if the triangulation satisfies $D\leq C_f \ndim^{-5/2}$, for some constant $C_f$ depending again on the gradient and Hessian of $f$, then $\hat{\mathcal{M}}$ is a manifold isotopic to $\mathcal{M}$. 
% \end{lemma}

% Here, saying that $f$ has bounded complexity means that three quantities related to gradient and Hessian of $f$ are bounded and positive (see Sec. 2.2 of \cite{boissonnatTracingIsomanifoldsTime2021}). 

% \begin{remark}
% For us, we are interested in the case $p=\ndim-1$ which would give
% \begin{equation}
%     N\leq \frac{K}{(\sqrt{2}D)^{\ndim-1}}\frac{\ndim^{3(\ndim-1)}}{(\ndim-1)!}= C D^{-(\ndim-1)},
% \end{equation}
% where the constant $C$ depends on $K$ and  $\ndim$. This means that, if we want the algorithm to give us an approximation that is $\epsilon$ away from the actual manifold we need
% \begin{equation}
%     N\sim \epsilon^{-\frac{\ndim-1}{2}}
% \end{equation}
% hyperplanes. \jamargincomment{Include constants}. 
% \end{remark}

% \begin{remark}
%     They also have the other type of triangulation, the Coxeter one, which gives better bounds on the number of pieces, but is slightly more complicated perhaps. 
% \end{remark}

% \jacomment{A $k$-flat is simply an $(\ndim-p)\eqqcolon k$-dimensional affine subspace of $\inputSpace$. See second sentence of abstract here: \cite{de1996point}. So it seems to me that this simply says that you shouldnt have manifolds that sort of oscillates ad infinitum. Which we get from tame things. Do you agree?}

% [What distance $\dist$?]

\begin{assumption}[On the cell-decomposition]\label{assump:stratification}
    We consider a cell decomposition $\Mcol$, as per \cref{def:celldecomp}, such that each cell $\M$ is a manifold with boundaries. This is typically referred to as a stratification, see \cref{appx:def}. The stratification $\Mcol$ of $\cancube$ is such that for every codimension $1$ strata $\M$ there exists an application $\Phi_{\M}:\inputSpace\to\bbR$ such that:
    \begin{itemize}
        \item zero is a regular point of $\Phi_\M$, and the zero-set of $\Phi_\M$ defines $\M$: $\M = \Phi_\M^{-1}(\{0\})$; 
        \item $\Phi_\M$ is $\C^2(\cancube)$ and its gradient and Hessian are non-zero and bounded away from infinity.
        % \item $\Phi_\M^{-1}(0)$ is compact. 
    \end{itemize}
\end{assumption}

\begin{proposition}\label{prop:strattobintree}
  % Fix an o-minimal expansion of $\bbR$. 
  % definable function $f:\appDom\to\bbR$ together with a 
  Consider a stratification $\Mcol$ definable in an o-minimal expansion of $\bbR$ that meets \cref{assump:stratification}. There exists a binary tree partition of $\cancube$ with depth $\nCuts$, as defined in \cref{def:piecepolycuts}, such that the distance between each codimension 1 strata, $\M$, and the corresponding partition boundary, $\widehat{\M}$, is bounded by 
  \begin{equation}
      d_F(\M, \widehat{\M}) \le \Cstrat \nCuts^{-\frac{2}{\ndim-1}},
  \end{equation}
  where the constant $\Cstrat$ depends only on the dimension $\ndim$ and the geometry of the strata $\M$. 
\end{proposition}
\begin{proof}
We apply the algorithm of \cite{boissonnatTracingIsomanifoldsTime2021} to the closure of the strata $\M$.
From \cite[Thm. 3.4]{boissonnatTracingIsomanifoldsTime2021} we have that $\dist(\M,\widehat{\M}) \le \Cdist D^2 $, where $D$ is the maximal diameter of a linear piece. Here the constant depends on the magnitude of the gradient and Hessian of the mapping $\Phi_\M$ from Assumption \ref{assump:stratification}. From \cite[Prop. 3.6]{boissonnatTracingIsomanifoldsTime2021} we have $\nCuts\leq \Cerr D^{-(\ndim-1)}$, where now the constant $\Cerr$ depends on the space dimension $\ndim$, and the number of times any straight line intersects $\M$. Note that this number is finite by the definability assumption, as is well-known in the case of algebraic manifolds. Putting this together gives the result. 
\end{proof}

% \old{We can consider the function $\eta_{\M}:\bbR^\ndim\to\bbR$, defined by $\eta_{\M}(x)\coloneqq \tfrac{1}{2}(\distM)^2$. The zeros of this function defines the manifold $\M$. Using the algorithm of \cite{boissonnat2022topological,boissonnatTracingIsomanifoldsTime2021} we can then construct a piecewise linear}
%     \old{
%   Note that [the above theorem] tells us that, $\dist_{F}(\M, \hat{\M}) \le \epsilon$ as soon as $C_{f}D^{2}\le \epsilon$, or equivalently $D \le \sqrt{\epsilon / C_{f}}$.
%   Noticing that
%   \begin{equation}
%     \nCuts \le C^{\circ} / D^{\ndim-1} \Leftrightarrow D \le (C^{\circ} / \nCuts)^{\frac{1}{\ndim-1}},
%   \end{equation}
%   it follows that a sufficient condition for $\dist_{F}(\M, \hat{\M})\le\epsilon$ is
%   \begin{equation}
%     \left(\frac{C^{\circ}}{\nCuts}\right)^{\frac{1}{\ndim-1}} \le \sqrt{\frac{\epsilon }{C_{f}}} \Leftrightarrow \nCuts \ge C^{\circ} \left(\frac{C_{f}}{\epsilon}\right)^{\frac{\ndim-1}{2}}.
%   \end{equation}
%     }
% [Take $\distM^2$ for Boissonat's statement?]

\subsection{Proof of the main result}

We are now ready to prove the main result \cref{th:main}, which we restate here for convenience.
\mainres*

% \gbcomment{[Make claim that this rate only works for tame functions.]
% [The best rate for Lipschitz is $...$, we can only do better because we have tame
% ]}

\begin{proof}
  The proof consists in constructing a piecewise polynomial function in $\piecePolySpaceCuts$ that has the claimed distance to $f$.

  We first construct the depth-$\nCuts$ binary tree that defines the pieces of the piecewise polynomial.
  Since $f$ is definable in an o-minimal structure, \cref{prop:celldecomp} yields a $\Cm$-decomposition of $\cancube$ such that
  \begin{itemize}
      \item each cell is a connected $\Cm$-manifold,
      \item the restriction of $f$ to each cell $\M$ is $\Cm(\M)$.
  \end{itemize}
  If the decomposition meets \cref{assump:stratification}, \cref{prop:strattobintree} provides a way to recursively split the space along affine hyperplanes such that, for any cell, or stratum, $\M$ that is not full-dimensional and any point $x$ in $\M$, there exists a point of the boundary between the pieces that is at most $\epsmarg$ away from $x$.
  This specifies exactly the geometry of the pieces of the piecewise polynomial function we construct.

  We now turn to define the polynomial on an arbitrary piece $\piece$ on which the piecewise polynomial function is a polynomial.
  Note that by construction of the geometry of the pieces, there can be at most one (full-dimensional) cell such that there exists a point both in $\piece$ and the cell that are more that $\varepsilon$ away from the boundaries of the cell, that is, the points where $f$ is not $\Cm$.
  We let $\Mbig$ denote such a cell, if it exists, or otherwise, an arbitrary cell that intersects with the piece.
  
  % \Cref{lmm:Jackson} with $E = \piece \cap \Mbig$ provides
  % \begin{equation}\label{eq:Jacksonpiece}
  %   \| f - p_{\polyDeg}\|_{\infty, \piece \cap \Mbig} \le \Cnm \frac{1}{\polyDeg^{\smoothDeg}} \|f\|_{\Cm(\piece\cap\Mbig)}.
  % \end{equation}
  % We now consider the situation at $x\in\piece \setminus (\piece\cap\Mbig)$.
  % By construction of the piecewise linear approximation of the strata of positive codimension of \cref{prop:strattobintree}, $x$ is at most $\epsilon$ away from $\M$, with $\varepsilon \le \epsmarg$.
  % Let $\projM(x)$ denote the orthogonal projection of $x$ on $\M$, uniquely defined for $\epsilon$ small enough.
  % Then, a simple triangular inequality yields
  % \begin{align*}
  %   |p(x) - f(x)| &\le |p(x) - p(\projM(x))| + |p(\projM(x)) - f(\projM(x))| + |f(\projM(x)) - f(x)|.
  % \end{align*}
  % By \eqref{eq:Jacksonpiece},
  % \begin{equation}
  %   |p(\projM(x)) - f(\projM(x))| \le \Cnm \frac{1}{\polyDeg^{\smoothDeg}} \|f\|_{\Cm(\piece\cap \M)}
  % \end{equation}
  % Since $f$ is $K$-Lipschitz and $x$ is at most $\epsilon$ away from $\M$,
  % \begin{equation}
  %   |f(\projM(x)) - f(x)| \le K \|\projM(x) - x\| \le K \epsilon
  % \end{equation}
  % and, [xxx]
  % \begin{equation}
  %   |p(x) - p(\projM(x))| \le [???].
  % \end{equation}
  % Therefore, for all $x \in \piece \setminus (\piece \cap \Mbig)$,
  % \begin{equation}
  %   |p(x) - f(x)| \le \Cnm \frac{1}{\polyDeg^{\smoothDeg}} \|f\|_{\Cm(\piece\cap \M)} + K \epsilon + ??
  % \end{equation}

  % \paragraph{An other way.}

% \gbcomment{
%   % [expression of $\epsmarg$ in terms of $\nCuts$?]
%   [there is one full-dim strata at most with points more than $\epsmarg$ away from the non-diff points contained in the piece $\piece$]
%   % [macro for Lipschitz constant of $f$]
% }
  
  Consider a point $x$ in $\piece$.
  \Cref{lmm:Jackson} provides \emph{(i)} $T(f)$, a smooth extension of $f$ from $\piece\cap\Mbig$ to a neighborhood of $\cancube$, the norm of which is bounded by $\Cfeff$ according to \eqref{eq:Feffext}, and \emph{(ii)} $p_{\polyDeg}^\piece$, the polynomial of degree at most $\polyDeg$ such that, for all $x\in \cancube$,
  % The Jackson error bound \eqref{eq:Jacksonpiece} implies
  \begin{equation}\label{eq:bounda}
    |T(f)(x) - p^{\piece}_\polyDeg(x)| \le \Cnm \frac{1}{\polyDeg^{\smoothDeg}} \|f\|_{\Cm(\piece\cap \Mbig)}.
  \end{equation}

  The triangular inequality yields
  \begin{equation}\label{eq:trianga}
      |f(x) - p^{\piece}_\polyDeg(x)| \le |f(x) - T(f)(x)| + |T(f)(x) - p_\polyDeg(x)|.
  \end{equation}
  The second term of \eqref{eq:trianga} is readily bounded by \eqref{eq:bounda}.

  We now turn to bound the first term of \eqref{eq:trianga}.
  We have, by the triangular inequality
  \begin{multline}
      |f(x) - T(f)(x)| \le |f(x) - f(\projM(x))| \\+ |f(\projM(x))-T(f)(\projM(x))| + |T(f)(\projM(x)) - T(f)(x)|.
  \end{multline}
  First, since $f$ is $K$-Lipschitz,
  \begin{equation}\label{eq:boundb}
    |f(x) - f(\projM(x))| \le K \|x - \projM(x)\|.
  \end{equation}
  Second, since $T(f)$ is an extension of $f$ on $\piece\cap\Mbig$, and $\projM(x)$ belongs to the closure of that space, there holds $f(\projM(x))=T(f)(\projM(x))$.
  Third, since $T(f)$ is $\smoothDeg$-times continuously differentiable on $\inputSpace$,
  \begin{equation}\label{eq:boundc}
    |T(f)(\projM(x)) - T(f)(x)| \le \sup_{y\in\inputSpace} \|\nabla T(f)(y)\| \|x - \projM(x)\|.
  \end{equation}
  Combining the inequality $\|\cdot\|_{2} \le \ndim\|\cdot\|_{\infty}$ and the definition of the norm on the Banach space $\Cm(\inputSpace)$,
  \begin{equation}\label{eq:boundd}
    \sup_{y\in\inputSpace} \|\nabla T(f)(y)\| \le  \ndim \|T(f)\|_{\Cm(\inputSpace)} |.
  \end{equation}

  Combining the above bounds \cref{eq:bounda,eq:boundb,eq:boundc,eq:boundd} yields
  \begin{equation}
      |f(x) - p_{\polyDeg}^\piece(x)| \le (K + \ndim\|T(f)\|_{\Cm(\inputSpace)}) \|x - \projM(x)\| + \Cnm \frac{1}{\polyDeg^{\smoothDeg}} \|f\|_{\Cm(\piece\cap \Mbig)}.
  \end{equation}

  Now, since $\|x-\projM(x)\| \le \epsmarg$ for $x\in\piece$, and using the bound on the norm of $T$ \eqref{eq:Feffext} yields, for all $x\in\piece$,
  \begin{equation}\label{eq:abovething}
      |f(x) - p_{\polyDeg}^\piece(x)| \le (K + \ndim \Cfeff \|f\|_{\Cm(\piece\cap\Mbig)}) \epsmarg + \Cnm \frac{1}{\polyDeg^{\smoothDeg}} \|f\|_{\Cm(\piece\cap \Mbig)}.
  \end{equation}
  We can now use \cref{lmm:banachnorm} to deduce that
  \begin{equation}
    \|f\|_{\Cm(\piece\cap\Mbig)} \le \|f\|_{\Cm(\Mbig)} \le \max_{\M\in\Mcol \text{ such that } \dim(\M) = \ndim} \|f\|_{\Cm(\M)} \eqdef  \Cfbanach,
  \end{equation}
  so that the bound of \eqref{eq:abovething} is now independent of $\piece$:
  \begin{equation}
    |f(x) - p_{\polyDeg}^\piece(x)| \le (K + \ndim \Cfeff \Cfbanach) \epsmarg + \Cnm \frac{1}{\polyDeg^{\smoothDeg}} \Cfbanach.
  \end{equation}

  Taking the supremum over all $x$ in $\piece$, and then over all pieces $\piece$ of $\appDom$ yields the bound for the piecewise polynomial function $p$
  \begin{equation}
  %     \| f - p \|_{\cancube, \infty} \le (K + \Cfeff \|f\|_{\Cm(\piece\cap\Mbig)}) \epsmarg + \Cnm \frac{1}{\polyDeg^{\smoothDeg}} \|f\|_{\Cm(\piece\cap \Mbig)}.
  % \end{equation}
  % This approximation is valid for any piece $\piece$, which yields the result
  % \begin{equation}
      \| f - p \|_{\cancube, \infty} \le C_1 \polyDeg^{-\smoothDeg} + C_2 \nCuts^{-\frac{2}{\ndim-1}}.
  \end{equation}
  where $C_1 = \Cnm \Cfbanach$ depends only on $\ndim$, $\smoothDeg$, $\cancube$, and $f$, and $C_2 =(K + \ndim \Cfeff \Cfbanach) \Cstrat$ depends only on $\ndim$, $\smoothDeg$, and $f$.
\end{proof}


% \old{
%   Assume that
%   for any strata $\M$ such that $\dim(\M)=\ndim-1$ and any $x\in\M$, there exists a point on the boundary of the pieces $p_{bound}\in\polyboundary$ such that $\|x - p_{bound}\| < \epsilon$.

%   \gbcomment{Rewrite this to match theorem statement?}
%   Consider a piecewise polynomial function such that the cuts are defined to minimize the distance between the union of the not-full-dimensional strata and the boundaries of the polynomial pieces.
%   % We now define the polynomial on each piece of the approximant.
%   % Consider a piece $\piece$.
%   Then, there is a distinguished manifold $\M$ among the full-dimensional manifolds intersecting $\piece$.
% }



%%% Local Variables:
%%% mode: latex
%%% TeX-master: "main_neurips"
%%% End:
