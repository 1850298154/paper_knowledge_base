We recall \cref{th:main}:
\mainres*

\Cref{th:main} will follow from a slightly more general result.
Let $\piecePolySpaceCuts(\appDom)$ denote the set of piecewise polynomial functions such that
\emph{(i)} the cells of the function form a hierarchical partition of $\appDom$ of depth $\nCuts$, and \emph{(ii)} the polynomial on each piece has degree at most $\polyDeg$.
% with a hierarchical partition of depth $\nCuts$ (\ie{} the function modeled by the MIP with depth $D = \nCuts$).
This is exactly the class of piecewise polynomial functions modelled by the mixed-integer formulation of \cref{sec:mip-form}.
This result gives an upper bound on the approximation error of a tame function by polynomials in that function class.
% the best piecewise polynomial function such that \emph{(i)} the cells of the polynomial form a hierarchical partition of $\appDom$ of depth $\nCuts$, and \emph{(ii)} the polynomial on each piece has degree at most $\polyDeg$.
\begin{theorem}\label{th:appx}
  Consider a cubic set $\appDom$, a function $f:\appDom\to\bbR$, and a constant $K>0$ such that:
  \begin{itemize}
    \item $f$ is definable in an o-minimal structure, and
    \item $f$ is $K$-Lipschitz on $\appDom$: for all $x$, $y\in\appDom$, $|f(x) - f(y)| \le K \|x-y\|$.
  \end{itemize}
  Then $f$ is piecewise approximable by polynomials: for any integers $\nCuts\ge 1$, and $\polyDeg \ge \smoothDeg > 1$, there holds
  \begin{equation}
    \inf_{p \in \piecePolySpaceCuts(\appDom)} \| f - p\|_{\infty, \appDom} \le \max\left( \frac{\constalt_1}{\nCuts^{\smoothDeg} \polyDeg^{\smoothDeg-1}}, \frac{\constalt_2}{\nCuts}\right),
  \end{equation}
  where $\constalt_{1} = \ndim^{\smoothDeg} \diam(\appDom)^\smoothDeg \const_{\ndim,\smoothDeg} \sum_{j=1}^{\ndim} \left \| \frac{\partial^{\smoothDeg}f}{\partial x_j^{\smoothDeg}} \right\|_{\infty,\diff(f)}$, $\constalt_{2} = \ndim K \diam(\appDom)$, and $\const_{d,r}$ is a constant that depends only on $d$ and $r$.
\end{theorem}

%\gbcomment{There should be some dimensionality curse in the second term of (24): a $\sqrt{d}$ term.}

% \gbcomment{def of hierarchical partition. Adapt proof.}

Before proceeding to the proof of \cref{th:appx}, we recall a result from approximation of $\C^{\smoothDeg}$ functions on a cube mentionned in \citet{plesniakMultivariateJacksonInequality2009}.
\begin{proposition}[Multivariate Jackson theorem]%
  \label{prop:jackson}
  Let $X$ be a compact cube in $\bbR^{\ndim}$, $f:\,X\to\bbR$ a $\mathcal{C}^{\smoothDeg}(X)$ function, and $\polyDeg\ge\smoothDeg$ two integers. Then
  there exists a constant $\const_{\ndim,\smoothDeg}$ depending only on $\ndim$, and $\smoothDeg$ such that:
  \begin{equation}
    \inf_{p\in \polySpace(X)} \|f - p\|_{\infty, X} \leq \rCube(X)^{\smoothDeg} \frac{\const_{\ndim,\smoothDeg}}{\polyDeg^{\smoothDeg-1}}\sum_{j=1}^{\ndim} \left\| \frac{\partial^{\smoothDeg}f}{\partial x_{j}^{\smoothDeg}} \right\|_{\infty, X},
  \end{equation}
  where $\rCube(X)$ denotes the edge length of $X$.
\end{proposition}
\begin{proof}[Proof of \cref{prop:jackson}]
    This result is obtained by reproducing the steps of Proposition 2.4 of \citet{plesniakExtensionPolynomialApproximation1994}, specializing to a cube which edge have same length $\rCube(X)$, and tracking the dependency in the size of the cube when combining Theorem 2.1 and Lemma 2.2 in \citet{plesniakExtensionPolynomialApproximation1994}.
\end{proof}
\begin{remark}
    We are not aware of a similar results in prominent approximation theory literature \cite{ditzian2012moduli,totik2020polynomial,dai2023polynomial} that propose an approximation result on a cubic domain, in which the constant $\const$ is independent of the geometry of the domain.
    This last point is essential for our purposes, since this result will be applied to small domains the size of which depends on parameters $\polyDeg$ and $\nPieces$ of \Cref{th:main}.
    % Having a constant $\const$ in the proof that does not depend on the geometry of the approximation domain $X$ is essential for our purposes.
\end{remark}

We now proceed with the proof of \cref{th:appx}.
\begin{proof}[Proof of \cref{th:appx}]
  % Consider the smallest hypercube enclosing the approximation set $\appDom$.
  Consider the set $\cubes$ of closed cubes obtained by regularly slicing the approximation domain $\nslice \eqdef \lfloor \nCuts / \ndim \rfloor + 1$ times along each cartesian axis.
  % Its radius is $\diam(\appDom)$, the diameter of $\appDom$.
  The domain $\appDom$ is thus split in $\nPieces = \nslice^{\ndim}$ pieces, each of which is a cube of edge length $\diam(\appDom) / \nslice$.
  Consider the set $\mathcal{P}_{\nslice}$ of functions from $\appDom$ to $\bbR$ such that the restriction to any cube of $\cubes$ is a polynomial of degree up to $\polyDeg$.
  Since $\mathcal{P}_{\nslice} \subset \piecePolySpaceCuts$, there holds
  \begin{equation}
    \inf_{p\in\piecePolySpaceCuts} \|f - p\|_{\infty, \appDom} \le \inf_{p \in \mathcal{P}_{\nslice}} \|f - p\|_{\infty, \appDom}.
  \end{equation}

  Since $f$ is definable, \cref{prop:celldecomp} yields a finite collection $\Mcol$ of cells $\M\subset\bbR^{\ndim}$, such that each cell is open, definable, full-dimensional, any two cells are disjoint, and the union of the closure of the cells is $\appDom$.
  Then, define the following subsets of $\cubes$:
  \begin{equation}
    \cubesU = \{ c \in \cubes : \exists \M \in \Mcol, c \subset \M \}, \qquad \cubesV = \cubes \setminus \cubesU.
  \end{equation}
  % $\cubesU$ as the set of cubes of $\cubes$ such that is either contained in the closure of an element of $\Mcol$.
  % define the partition of $\cubes$ in $\cubesU$ and $\cubesV$
  % Consider the subset $\cubesU$ of $\cubes$ that contains all the cubes that are contained in the closure of a full-dimensional cell.
  % Denote $\cubesV$ the set of all the cubes that are not contained in the closure of a full-dimensional strata.
  We have the following:
  \begin{align}
    \inf_{p \in \mathcal{P}_{\nslice}} \|f - p\|_{\infty, \appDom} &= \inf_{p\in\mathcal{P}_{\nslice}} \max\left(\max_{c \in \cubesU} \| f - p \|_{\infty, c}, \max_{c \in \cubesV} \| f - p \|_{\infty, c}\right) \\
                                          &= \max\left(\max_{c \in \cubesU} \inf_{p \in \polySpace(c)} \| f - p \|_{\infty, c}, \max_{c \in \cubesV} \inf_{p \in \polySpace(c)} \| f - p \|_{\infty, c}\right).
  \end{align}
  The first equality follows from the definition of the infinity norm and that the union of the cubes in $\cubesU$ and $\cubesV$ is $\appDom$.
  The second equality follow from the independence of the expression of any $p \in \mathcal{P}_{\nslice}$ on any two distinct cubes of $\cubes$.

  Let $c$ denote a cube in $\cubesU$.
  Since $c$ is contained in a cell $\M$ on which $f$ is $\C^{\smoothDeg}$, the restriction of $f$ to $c$ is $\C^{r}$.
  % [
  % $f$ is $\C^{r}$ on $\M_{c}$.
  % $f$ admits a natural expansion to $\cl(\M_{c})$ that is continuous. [Is it $\C^{r}$].
  % Does $f$ admits a $\C^{r}$ expansion from $\M_{c}$ to $\cl(\M_{c})$? If so, any will do.
  % ]
  % \gbcomment{
  %   Handle:
  %   [$f$ is $\C^{r}$ on the open set $\M$. What about on $\cl(\M)$?]
  %   [($\polyDeg > r-1$)]
  % }
  Therefore, \cref{prop:jackson} applies and yield
  \begin{equation}\label{eq:boundint}
    \inf_{p \in \polySpace(c)} \| f - p \|_{\infty, c} \le \left(\frac{\rCube(\appDom)}{\nslice}\right)^{\smoothDeg} \frac{\const_{\ndim,\smoothDeg}}{\polyDeg^{\smoothDeg-1}}\sum_{j=1}^{\ndim} \left\| \frac{\partial^{\smoothDeg}f}{\partial x_j^{\smoothDeg}}\right\|_{\infty, c}%\sup_{x\in c}\left|\frac{\partial^{\smoothDeg}f}{\partial x_j^{\smoothDeg}}(x)\right|.
  \end{equation}
  for some constant $\const_{\ndim,\smoothDeg}$ depending only on $\ndim$ and $\smoothDeg$.

  Let $c$ denote a cube in $\cubesV$, and $x_{0}$ denote an arbitray point in $c$.
  Since $f$ is $K$-Lipschitz, for any $x\in c$,
  \begin{equation}
    | f(x) - f(x_{0}) | \le K \|x - x_{0}\|.
  \end{equation}
  Thus,
  \begin{equation}\label{eq:boundbound}
    \| f(x) - f(x_{0}) \|_{\infty, c} \le K \sup_{x\in c}\|x - x_{0}\| \le K \diam(c) = K \diam(\appDom) \nslice^{-1}.
  \end{equation}

  Combining the two bounds \eqref{eq:boundint} and \eqref{eq:boundbound}, and using $\nslice = \lfloor \nCuts / \ndim \rfloor + 1 \ge \nCuts / \ndim$ yields
  \begin{align}
    \inf_{p \in \piecePolySpaceCuts(\appDom)} \| f - p\|_{\infty, \appDom} &\le \max \left(\frac{\diam(\appDom)^\smoothDeg}{\nslice^{\smoothDeg}}\frac{\const_{\ndim,\smoothDeg}}{\polyDeg^{\smoothDeg-1}} \sum_{j=1}^{\ndim} \left \| \frac{\partial^{\smoothDeg}f}{\partial x_j^{\smoothDeg}} \right\|_{\infty,\diff(f)}, \frac{K \diam(\appDom)}{\nslice} \right) \\
    & \le \max \left(\frac{\ndim^{\smoothDeg}\diam(\appDom)^\smoothDeg}{\nCuts^{\smoothDeg}}\frac{\const_{\ndim,\smoothDeg}}{\polyDeg^{\smoothDeg-1}} \sum_{j=1}^{\ndim} \left \| \frac{\partial^{\smoothDeg}f}{\partial x_j^{\smoothDeg}} \right\|_{\infty,\diff(f)}, \frac{\ndim K \diam(\appDom)}{\nCuts} \right) \\
    % & = \max\left( \bigoh\left(\frac{1}{\nPieces \polyDeg^{\smoothDeg-1}}\right), \bigoh\left(\frac{1}{\nPieces^{1/\ndim}}\right) \right). \\
      & = \max\left( \frac{\constalt_1}{\nCuts^{\smoothDeg} \polyDeg^{\smoothDeg-1}}, \frac{\constalt_2}{\nCuts}\right).
  \end{align}
  where $\constalt_{1} = \ndim^{\smoothDeg} \diam(\appDom)^\smoothDeg \const_{\ndim,\smoothDeg} \sum_{j=1}^{\ndim} \left \| \frac{\partial^{\smoothDeg}f}{\partial x_j^{\smoothDeg}} \right\|_{\infty,\diff(f)}$ and $\constalt_{2} = \ndim K \diam(\appDom)$.
  % where $\const = \max_{c \in \cubesU} \const_{c}$.
  % \gbcomment{Fix rad vs diam}
  % \gbcomment{Adapt main theorem statement to this.}
\end{proof}

We can now deduce the result of \cref{th:main}.
\begin{proof}[Proof of \cref{th:main}]
  Consider a number of pieces $\nPieces$ and the highest number of cuts that allow splitting a cube in $\bbR^{\ndim}$ in $\nPieces$, that is the highest $\nCuts$ such that $\nPieces \ge (\lfloor \nCuts / \ndim \rfloor + 1)^{\ndim}$.
  The bound of \cref{th:appx} applies to this situation and yields
  \begin{equation}\label{eq:thm1Proof}
    \inf_{p \in \piecePolySpace[\nslice^{\ndim}](\appDom)} \| f - p\|_{\infty, \appDom} \le \max \left(\frac{\diam(\appDom)^\smoothDeg}{\nPieces^{\frac{\smoothDeg}{\ndim}}}\frac{\const_{\ndim,\smoothDeg}}{\polyDeg^{\smoothDeg-1}} \sum_{j=1}^{\ndim} \left \| \frac{\partial^{\smoothDeg}f}{\partial x_j^{\smoothDeg}} \right\|_{\infty,\diff(f)}, \frac{K \diam(\appDom)}{\nPieces^{1/\ndim}} \right),
    % & = \max\left( \bigoh\left(\frac{1}{\nPieces \polyDeg^{\smoothDeg-1}}\right), \bigoh\left(\frac{1}{\nPieces^{1/\ndim}}\right) \right). \\
      % & = \max\left( \frac{\const_1}{\nPieces \polyDeg^{\smoothDeg-1}}, \frac{\const_2}{\lfloor \nPieces^{1/\ndim} \rfloor}\right).
  \end{equation}
  since $\nPieces^{1/\ndim} \ge \nCuts / \ndim$, which is the claimed bound.
  % where $\const_{1} = \diam(\appDom)^\ndim \const_{\ndim,\smoothDeg} \sum_{j=1}^{\ndim} \left \| \frac{\partial^{\smoothDeg}f}{\partial x_j^{\smoothDeg}} \right\|_{\infty,\diff(f)}$ and $\const_{2} = K \diam(\appDom)$.
\end{proof}

% \textcolor{red}{TODO: Main issue is how to deal with the openess of the cells/stratifications. The SW or Jackson theorems are for closed sets. \cite{aschenbrenner2019whitney} could have some answers, see also below.}

% \textcolor{red}{TODO (Johannes): Work this theorem out, what happens with Jackson when the cells are open boxes and not closed as in the ordinary Jackson? Because this is the case for the cell decomposition. IT DOES NOT WORK IN GENERAL!!! What is the rate of convergence? }

%%% Local Variables:
%%% mode: latex
%%% TeX-master: "main_icml"
%%% End:
