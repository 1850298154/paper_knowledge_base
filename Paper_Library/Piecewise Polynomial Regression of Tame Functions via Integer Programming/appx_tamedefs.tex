%\jacomment{add some notes on stratifications and cell decompositions}

We give a more formal but brief review of the results needed from tame, or o-minimal, geometry.
We refer the interested reader to \citet{van1998tame,costeIntroductionOminimalGeometry2000} for extensive expositions.

Let us start with the definition of an o-minimal structure.
For simplicity, we consider only structures over the real field $\bbR$.
\begin{definition}[o-minimal structure]\label{def:omin}
  An o-minimal structure on $\bbR$ is a sequence $\CS=(\CS_m)_{m\in\bbN}$ such that for each $m\geq 1$:
  \begin{enumerate}[i.]%\setlength\itemsep{0.5em}
    \item $\CS_m$ is a boolean algebra of subsets of $\bbR^m$;
    \item if $A\in\CS_m$, then $\bbR\times A$ and $A\times \bbR$ belongs to $\CS_{m+1}$;
    \item $\{(x_1,\dots,x_m)\in \bbR^m\,:\, x_1=x_m\}\in \CS_m$;
    \item if $A\in\CS_{m+1}$, and $\pi:\,\bbR^{m+1}\to\bbR^m$ is the projection map on the first $m$ coordinates, then $\pi(A)\in\CS_m$;
    \item the sets in $\CS_1$ are exactly the finite unions of intervals and points.
  \end{enumerate}
\end{definition}

A set $A\subseteq \bbR^m$ is said to be \emph{definable} in $\CS$ if $A$ belongs to $\CS_m$.
Similarly, a map $f:\, A\to B$, with $A\subseteq \bbR^m$, $B\subseteq \bbR^n$, is said to be definable in $\CS$ if its graph $\Gamma(f) \eqdef \{(x,f(x))\in\bbR^{m+n}: x\in A\}$ belongs to $\CS_{m+n}$.

A set or function definable in some o-minimal structure is often simply referred to as \emph{tame} when the specific o-minimal structure is not important.

The most fundamental o-minimal structure is that containing the semialgebraic sets:
\begin{example}[Semialgebraic structure]
  A \emph{semialgebraic set} in $\bbR^{\ndim}$ defines as a finite union of sets of the form
  \begin{equation}\label{eq:semi_alg}
    \{ x \in \bbR^{\ndim} : f_1(x) = 0,\ldots, f_k(x)=0, g_{1}(x) > 0, \ldots, g_{l}(x) > 0\}
  \end{equation}
  where the $f_i$ and the $g_i$ are all real polynomials in $\ndim$ variables.
  % In turn, a \emph{semialgebraic mapping} $f:\bbR^{\ndim}\to\bbR^{p}$ is a mapping whose graph is a semialgebraic set in $\bbR^{\ndim}\times\bbR^{p}$.
  % Other viewpoint: compositionality.

  The collection of all semialgebraic sets forms an o-minimal structure. In particular, any function built from polynomials, boolean rules, and coordinate projections is definable in that structure. Examples include affine applications, piecewise polynomial functions such as linear layers, (pointwise) ReLU activation functions, and their composition, or functions such as $(x, y) \mapsto \sqrt{x^{2} + y^{3}}$.
  % The semialgebraic sets verify all requirements of \Cref{def:omin}.
  % The most delicate item is the stability by projection, which corresponds to Tarski-Seidenberg's theorem.
\end{example}


As stated in \cref{prop:celldecomp}, tame sets can always be partitioned into smaller subsets.
For completeness, we state the formal cell decomposition theorem.
To this end, we first need the following.

\begin{definition}[Cells, decompositions]\label{def:celldecomp}
  We define both cells and decompositions inductively. A \emph{cell} in $\bbR$ is a point $\{a\}$, for $a\in\bbR$ or an (open) interval $(a,b)$, for $a,b\in\bbR\cup\{\pm\infty\}$.
  \begin{itemize}
    \item Let $C$ be a cell in $\bbR^m$ and let $f:\, C\to\bbR$ be tame and continuous, then $\{(x,f(x))\,:\,x\in C\}$ is a cell in $\bbR^{m+1}$.
    \item Let $C$ be a cell in $\bbR^m$ and let $f,g:\, C\to\bbR\cup\{\pm\infty\}$ be tame and continuous such that for all $x\in C:\,f(x)<g(x)$, then $\{(x,y)\in C\times \bbR\,:\, f(x)<y<g(x)\}$ is a cell in $\bbR^{m+1}$.
  \end{itemize}

  A \emph{decomposition} of $\bbR$ is a finite partition of the form
  \begin{equation}
    \{(-\infty,a_1),(a_1,a_2),\dots, (a_n,+\infty),\{a_1\},\dots,\{a_n\}\},
  \end{equation}
  and a decomposition of $\bbR^{m+1}$ is a finite partition of $\bbR^{m+1}$ into cells $C_1,\dots,C_n$ such that the set of projections $\pi(C_j)$ gives a partition of $\bbR^m$.
\end{definition}
Cells are connected sets.
We further say that a cell is $\mathcal{C}^\smoothDeg$ if all functions used to construct the cell are $\mathcal{C}^\smoothDeg$. We now have the fundamental result:

\begin{theorem}[Cell decomposition, cf. Thm. 7.3.2 \cite{van1998tame}]
For any definable sets $A,\, A_1,\dots, A_n\subseteq \bbR^m$ and definable function $f:A\to\bbR$, there is
    \begin{itemize}
        \item a decomposition of $\bbR^m$ into $\C^\smoothDeg$-cells partitioning the sets $A_j$.
        \item a decomposition of $\bbR^m$ into $\C^\smoothDeg$-cells partitioning $A$, such that each restriction $f|_C:C\to\bbR$ is $\C^\smoothDeg$ for each cell $C\subseteq A$ of the decomposition.
    \end{itemize}
\end{theorem}

\begin{example}[\cref{ex:conecelldecomp} continued]
  We return on the cone function, defined in \cref{eq:cone2dintro} as
  \begin{equation}
    f(x) =
    \begin{cases}
      -\sCone x_{1} + \frac{1+\sCone}{\rCone}x_{2} & \text{ if } x_{1} > 0 \text{ and } 0 < x_{2} < \rCone x_{1} \\
      -\sCone x_{1} - \frac{1+\sCone}{\rCone}x_{2} & \text{ if } x_{1} > 0 \text{ and } -\rCone x_1 < x_{2} < 0 \\
      \|x\|_{\infty} & \text{ else}
    \end{cases}.
  \end{equation}
  The above cell decomposition theorem provides the cells described in \cref{table:conecells}; see \cref{fig:conelevelsjointappx} for an illustration.
  % , we give the cell decomposition of the graph of the ``cone'' function, defined in \cref{eq:cone2dintro}, in .
   % shows the graphical illustration of this corresponding.
\end{example}

\begin{figure}[t]
  % \vskip 0.2in
  \begin{center}
    \includegraphics[width=0.48\textwidth]{figures/figab.pdf}
    \caption{Illustration of the ``cone'' function \eqref{eq:cone2dintro}, with $\sCone=\rCone=0.5$, showing \emph{(i)} the level lines of the function, and \emph{(ii)} the decomposition of the domain into cells on which the function is smooth, as provided by \cref{prop:celldecomp}; see \cref{table:conecells} for details.}
    \label{fig:conelevelsjointappx}
  \end{center}
  \vskip -0.2in
\end{figure}
\begin{table}[t]
  \caption{Cell decomposition of the ``cone'' function \eqref{eq:cone2dintro} with $\rCone=\sCone=0.5$: the space decomposes in $7$ open full-dimensional sets on which the function is smooth; see \cref{fig:conelevelsjoint} for an illustration.\label{table:conecells}}
  % \resizebox{0.48\textwidth}{!}{
  \centering
  \begin{tabular}{lll}
    \toprule
    Cell & cell expression & $f$ restricted to cell \\ \midrule
    $\M_{1}$ & $\{x\in\bbR^{2}: x_{1}+x_{2}<0, \; x_{1}-x_{2}<0\}$ & $f|_{\M_{1}} = -x_{1}$ \\
    $\M_{2}$ & $\{x\in\bbR^{2}: x_{1}+x_{2}<0, \; x_{1}-x_{2}<0\}$ & $f|_{\M_{2}} = \phantom{-}x_{2}$ \\
    $\M_{3}$ & $\{x\in\bbR^{2}: x_{1}+x_{2}>0, \; x_{1}-x_{2}>0\}$ & $f|_{\M_{3}} = \phantom{-}x_{1}$ \\
    $\M_{4}$ & $\{x\in\bbR^{2}: x_{1}>0, \; 0 < \phantom{-}x_{2}< 0.5x_{1}\}$ & $f|_{\M_{4}} = -0.5x_{1} + 3x_{2}$ \\
    $\M_{5}$ & $\{x\in\bbR^{2}: x_{1}>0, \; 0 < -x_{2}< 0.5x_{1}\}$ & $f|_{\M_{5}} = -0.5x_{1} - 3x_{2}$ \\
    $\M_{6}$ & $\{x\in\bbR^{2}: x_{1}+x_{2}>0, \; x_{1}-x_{2}>0\}$ & $f|_{\M_{6}} = \phantom{-}x_{1}$ \\
    $\M_{7}$ & $\{x\in\bbR^{2}: x_{1}+x_{2}<0, \; x_{1}-x_{2}<0\}$ & $f|_{\M_{7}} = -x_{2}$ \\
    \bottomrule
  \end{tabular}
  % }
\end{table}

A related notion to that of cell decomposition is that of stratifications. A stratification is slightly stronger than a cell decomposition, in that it gives further conditions on how the different pieces fit together. The basic idea is that the set is partitioned into a finite number of manifolds, called the \emph{strata}, with some additional conditions on how the different strata glue together. Various types of stratifications, with differing conditions on the gluing of the strata, exist in the literature. Some examples are the Whitney, Verdier and Lipschitz conditions; see \eg{} \citet{le1998verdier}. For o-minimal structures, we can again require that the function is $\C^\smoothDeg$, for some $\smoothDeg<\infty$, on each strata. The additional gluing conditions have played a vital role in the recent optimization literature when addressing \eg{} questions of convergence for gradient descent algorithms to Clarke critical points \cite{davisStochasticSubgradientMethod2020, bolte2021conservative, davis2021active}.

%\begin{definition}[Stratification]
%    A \emph{stratification} of a set $A\subseteq \bbR^d$ is a partition of $A$ into finitely many non-empty manifolds, typically called \emph{strata}, satisfying:
%    \begin{itemize}
%        \item \textbf{Frontier condition}: For any two strata $M_1$, $M_2$, the following implication holds:
%        \begin{equation*}
%            \text{cl}(M_1)\cap M_2\neq \emptyset\implies M_2\subset \text{cl}(M_1).
%        \end{equation*}
%    \end{itemize}
%\end{definition}

%%% Local Variables:
%%% mode: latex
%%% TeX-master: "main_icml"
%%% End:
