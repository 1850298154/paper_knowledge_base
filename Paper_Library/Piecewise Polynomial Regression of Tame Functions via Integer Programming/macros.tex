%**********************************************************************
%*****	EDITING
%**********************************************************************

\newcommand{\needref}{{\color{red}\upshape\textbf{[??]}}\xspace}	% for missing refs
\newcommand{\attn}{{\color{red}\upshape\textbf{[!!]}}\xspace}		% for attention

% \newcommand{\debug}[1]{{\color{purple}#1}}		% for macro coloring
\newcommand{\debug}[1]{#1}		% for removing macro coloring
\newcommand{\lookout}[1]{{\bfseries\color{red}[#1]}}		% for attention
\newcommand{\nlookout}[1]{\note{\bfseries\color{red}[#1]}}		% for attention
\newcommand{\old}[1]{{\color{gray}#1}}		% for removal markup
\newcommand{\revise}[1]{{\color{blue}#1}}		% for revision markup


\usepackage{color-edits}
% \usepackage{color-edits}
\addauthor{gb}{blue}
\addauthor{jn}{Orange}
\addauthor{ja}{pink}


\usepackage{siunitx}
\newcommand{\scinum}[1]{\num[round-precision=2,round-mode=figures,
     scientific-notation=true]{#1}}

\newcommand{\minor}[1]{\textcolor{gray}{#1}}


\usepackage{thmtools}
\usepackage{thm-restate} % see https://tex.stackexchange.com/questions/51286/recalling-a-theorem


%**********************************************************************
%*****	MACROS: GENERAL PURPOSE
%**********************************************************************
\newcommand{\newmacro}[2]{\newcommand{#1}{\debug{#2}}}		% for shorthand definitions
\newcommand{\newop}[2]{\DeclareMathOperator{#1}{\debug{#2}}}		% for shorthand definitions


\newmacro{\defeq}{\triangleq}
\newmacro{\eqdef}{\triangleq}

%----------------------------------------------------------------------
%% Delimiters
%----------------------------------------------------------------------
% \DeclarePairedDelimiter{\braces}{\{}{\}}		% for braces
% \DeclarePairedDelimiter{\bracks}{[}{]}		% for brackets
% \DeclarePairedDelimiter{\parens}{(}{)}		% for parentheses

% \DeclarePairedDelimiter{\abs}{\lvert}{\rvert}		% for absolute value
% \DeclarePairedDelimiter{\ceil}{\lceil}{\rceil}		% for ceiling
% \DeclarePairedDelimiter{\floor}{\lfloor}{\rfloor}		% for floor
% \DeclarePairedDelimiter{\clip}{[}{]}		% for clipping
% \DeclarePairedDelimiter{\negpart}{[}{]_{-}}		% for negative part
% \DeclarePairedDelimiter{\pospart}{[}{]_{+}}		% for positive part

% \DeclarePairedDelimiter{\bra}{\langle}{\rvert}		% for bras
% \DeclarePairedDelimiter{\ket}{\lvert}{\rangle}		% for kets

% \DeclarePairedDelimiterX{\setdef}[2]{\{}{\}}{#1:#2}		% for set builder notation
% \DeclarePairedDelimiterXPP{\exclude}[1]{\mathopen{}\,\backslash}{\{}{\}}{}{#1}


%----------------------------------------------------------------------
%% Text and formatting
%----------------------------------------------------------------------
\newcommand{\spc}[1]{\spacedallcaps{#1}}
\newcommand{\splc}[1]{\spacedlowsmallcaps{#1}}

\newcommand{\cf}{cf.}		% for consistency
\newcommand{\eg}{e.g.,}		% for consistency
\newcommand{\ie}{i.e.,}		% for consistency
\newcommand{\vs}{vs.}		% for consistency

\newcommand{\textbrac}[1]{\textup[#1\textup]}		% for upshape brackets
\newcommand{\textpar}[1]{\textup(#1\textup)}		% for upshape parentheses

\newcommand{\dis}{\displaystyle}		% for forcing display style
\newcommand{\txs}{\textstyle}		% for forcing inline style


%----------------------------------------------------------------------
%% Modifiers
%----------------------------------------------------------------------
\newcommand{\alt}[1]{#1'}		% for alternates


%----------------------------------------------------------------------
%% Number fields
%----------------------------------------------------------------------
\newcommand{\F}{\mathbb{F}}		% generic field
\newcommand{\N}{\mathbb{N}}		% for naturals
\newcommand{\Z}{\mathbb{Z}}		% for integers
\newcommand{\Q}{\mathbb{Q}}		% for rationals
% \newcommand{\R}{\mathbb{R}}		% for reals
\newcommand{\bbR}{\mathbb{R}}		% for reals, julia style
\newcommand{\BR}{\mathbb{R}}
\newcommand{\bbN}{\mathbb{N}}		% for integers, julia style
\newcommand{\bbZ}{\mathbb{Z}}		% for integers
\newcommand{\Rbar}{\bar{\bbR}}		% for reals, julia style

\ifdef{\C}{                          % Command \C is already defined on my machine...
    \renewcommand{\C}{\mathcal{C}}
}{
    \newcommand{\C}{\mathcal{C}}
}

\newcommand{\CS}{\mathcal{S}}
%----------------------------------------------------------------------
%% Operators
%----------------------------------------------------------------------
\DeclareMathOperator{\bigoh}{\mathcal O}		% for Landau O
\DeclareMathOperator{\ess}{ess}		% for essential
% \DeclareMathOperator{\grad}{\nabla}		% for gradient
% \DeclareMathOperator{\Hess}{Hess}		% for Hessian
\DeclareMathOperator{\ind}{ind}		% for index
\DeclareMathOperator{\rank}{rank}		% for rank
\DeclareMathOperator{\sign}{sgn}		% for sign
% \DeclareMathOperator{\Sym}{Sym}		% for symmetric


%----------------------------------------------------------------------
%% Sundries
%----------------------------------------------------------------------
\newmacro{\dd}{\:d}		% for integrators
\newcommand{\ddt}[1]{\frac{d#1}{dt}}		% for Leibniz
\newcommand{\del}{\partial}		% for derivatives
\newcommand{\eps}{\varepsilon}		% for better epsilon
\renewcommand{\epsilon}{\varepsilon}		% for better epsilon
\newcommand{\pd}{\partial}		% for derivatives
\newcommand{\wilde}{\widetilde}		% for wide tildes

\newcommand{\insum}{\sum\nolimits}		% for compact sums
\newcommand{\inprod}{\prod\nolimits}		% for compact products



%**********************************************************************
%*****	MACROS: SET THEORY
%**********************************************************************

%----------------------------------------------------------------------
%% Points and sets
%----------------------------------------------------------------------
\DeclareMathOperator*{\intersect}{\bigcap}		% for intersections
\DeclareMathOperator*{\union}{\bigcup}		% for unions

\DeclareMathOperator{\card}{card}		% for cardinality
\newop{\dom}{dom}		% for domain
\DeclareMathOperator{\im}{im}		% for image
\DeclareMathOperator{\one}{\mathds{1}}		% for indicator

\newcommand{\from}{\colon}		% for function definition
\newcommand{\too}{\rightrightarrows}		% for correspondences
\newcommand{\injects}{\hookrightarrow}		% for injections
\newcommand{\surjects}{\twoheadrightarrow}		% for surjections

\newmacro{\set}{\mathcal{S}}		% for generic set

\newmacro{\points}{\mathcal{K}}		% for point set
\newmacro{\point}{x}		% for generic point
\newmacro{\pointalt}{\alt\point}		% for alternate point

\newmacro{\dpoints}{\mathcal{Y}}		% for second point set (duals, etc.)
\newmacro{\dpoint}{y}		% for second generic point
\newmacro{\dpointalt}{\alt\dpoint}		% for second alternate variable

\newmacro{\base}{p}		% for base point
\newmacro{\basealt}{q}		% for alternate base point

\newcommand{\test}[1][\point]{\hat{#1}}		% for test point (\point by default)
\newcommand{\tests}{\test[\points]}		% for set of test points


%----------------------------------------------------------------------
%% Point set topology
%----------------------------------------------------------------------
\DeclareMathOperator{\bd}{bd}		% for boundary
\DeclareMathOperator{\cl}{cl}		% for closure
\DeclareMathOperator{\diam}{diam}		% for diameter
\DeclareMathOperator{\dist}{dist}		% for distance
\DeclareMathOperator{\intr}{int}		% for interior

\newmacro{\open}{\mathcal{U}}           % for open sets
\newmacro{\cpt}{\mathcal{K}}            % for compacts
\newcommand{\nhd}[1]{\mathcal{N}_{#1}}  % for neighborhoods
\newmacro{\U}{U}                        % for neighborhoods


%**********************************************************************
%*****	MACROS: SEQUENCES AND RECURSIONS
%**********************************************************************

%----------------------------------------------------------------------
%% Indexing
%----------------------------------------------------------------------
\newmacro{\start}{1}		% for start index
\newmacro{\running}{1,2,\dotsc}		% for running index
\newmacro{\run}{t}		% for main sequence index
\newmacro{\runalt}{s}		% for alternate sequence index
\newmacro{\nRuns}{T}		% for total number of runs
\newmacro{\runtime}{S}		% for runtime

\newcommand{\inrun}{\runalt}		% for dummy counter
\newmacro{\runs}{\mathcal{\nRuns}}		% for set of indices



%**********************************************************************
%*****	MACROS: LINEAR ALGEBRA
%**********************************************************************

%----------------------------------------------------------------------
%% Vector spaces
%----------------------------------------------------------------------
\newmacro{\vecspace}{\mathcal{X}}		% for generic vector space
\newmacro{\vdim}{n}		% for dimension
\newmacro{\vvec}{v}		% for generic vector
\newmacro{\bvec}{e}		% for basis vectors
\newmacro{\unitvec}{z}		% for unit vectors

\newmacro{\subspace}{\mathcal{W}}		% for subspace
\newmacro{\subdim}{m}		% for dimension

\newmacro{\tanspace}{\mathcal{Z}}		% for tangent space
\newmacro{\tvec}{z}		% for tangent vectors

\newcommand{\norm}[1]{\left\lVert#1\right\rVert}
%----------------------------------------------------------------------
%% Duality
%----------------------------------------------------------------------
%\DeclarePairedDelimiterX{\braket}[2]{\langle}{\rangle}{#1\mathopen{}\delimsize\vert\mathopen{}#2}
% \DeclarePairedDelimiterX{\braket}[2]{\langle}{\rangle}{#1,#2}		% for duality pairing
\newcommand{\dual}[1]{#1^{\ast}}		% for dual variables

\newmacro{\dspace}{\vecspace^{\ast}}		% for dual space
\newmacro{\dvec}{w}		% for dual basis vectors
\newmacro{\dbvec}{\eps}		% for dual basis vectors


%----------------------------------------------------------------------
%% Matrices and vectors
%----------------------------------------------------------------------
% \DeclareMathOperator{\diag}{diag}              % for diagonal matrices
\DeclareMathOperator{\eig}{eig}                  % for eigenvalues
\DeclareMathOperator{\tr}{tr}                    % for trace

\newmacro{\ones}{\mathbf{1}}                     % for vector of ones
\newmacro{\mat}{A}                               % for generic matrix
\newmacro{\eye}{I}                               % for identity matrix

\newcommand{\mg}{\succ}                          % for positive-definite
\newcommand{\mgeq}{\succcurlyeq}                 % for positive-semidefinite
\newcommand{\ml}{\prec}                          % for negative-definite
\newcommand{\mleq}{\preccurlyeq}                 % for negative-semidefinite

\newop{\Tr}{Tr}
\newcommand{\Sym}[1][m]{\debug{\mathbb{S}_{#1}}} % for symmetric matrices

%**********************************************************************
%*****	MACROS: OPTIMIZATION
%**********************************************************************

%----------------------------------------------------------------------
%% Convex analysis
%----------------------------------------------------------------------
\DeclareMathOperator{\aff}{aff}   % for affine hull
\DeclareMathOperator{\conv}{conv} % for convex hull (but see also \simplex)
\DeclareMathOperator{\relint}{ri} % for relative interior

\newmacro{\cvx}{\mathcal{C}}      % for generic convex set
\newmacro{\subd}{\partial}        % for subdifferential

\newmacro{\strong}{\alpha}        % for convexity modulus
\newmacro{\smooth}{\beta}         % for smoothness modulus


%----------------------------------------------------------------------
%% Optimization
%----------------------------------------------------------------------
\DeclareMathOperator*{\argmax}{arg\,max}		% for argmax
\DeclareMathOperator*{\argmin}{arg\,min}		% for argmin
\DeclareMathOperator{\Opt}{Opt}		% for value of problem
\DeclareMathOperator{\Sol}{Sol}		% for solution of problem
\DeclareMathOperator{\gap}{Gap}		% for gap function

\newmacro{\obj}{f}		% for objective function
\newmacro{\objalt}{g}		% for alternative objective function
\newmacro{\sobj}{F}		% for stochastic objective
\newmacro{\gvec}{g}		% for gradient vector
\newmacro{\gbound}{G}		% for gradient bound

\newmacro{\param}{\theta}		% for parameter
\newmacro{\params}{\Theta}		% for parameter space

\newmacro{\oper}{A}		% for operator
\newmacro{\vecfield}{V}		% for vector field
\newmacro{\vbound}{L}		% for field bound

\newcommand{\sol}[1][\point]{#1^{\ast}}		% for solutions (x by default)
\newcommand{\sols}{\sol[\points]}		% for set of solutions

\newcommand{\pmat}[1]{
  \begin{pmatrix}
    #1
  \end{pmatrix}
}

%----------------------------------------------------------------------
%% Manifolds
%----------------------------------------------------------------------
\newmacro{\M}{\mathcal{M}}		% for manifold
\newmacro{\Mcol}{\mathcal{W}}		% for manifold collection / stratification

\newmacro{\vx}{x}
\newcommand{\vxup}{\debug{x}_{\debug +}}
\newmacro{\vy}{y}
\newmacro{\vxalt}{x'}
\newmacro{\vyalt}{y'}

\newmacro{\vxman}{\vx^{\M}}
\newcommand{\Mproj}[2][\M]{#2^{#1}}

\newmacro{\vn}{v_n}
\newmacro{\vnalt}{\hat{v}_n}

\newmacro{\vsg}{v} % for subgradients


\newmacro{\tangentBundle}{T\mathcal{B}}

\newcommand{\tangent}[2]{\debug{T}_{#1} #2}
\newcommand{\tangentM}[1][\vx]{\debug{T}_{#1}\M}
\newcommand{\normal}[2]{\debug{N}_{#1} #2}
\newcommand{\normalM}[1][\vx]{\debug{N}_{#1}\M}

\newcommand{\projN}[2]{\proj_{\normal{#1}{#2}}}
\newcommand{\projT}[2]{\proj_{\tangent{#1}{#2}}}

\newmacro{\maneq}{h}
\newcommand{\maneqinter}{\maneq^\funns}

\newmacro{\smoothcurve}{  c}
\newmacro{\geocurve}{  \gamma}

\newmacro{\grad}{  \operatorname{grad}}
\newmacro{\Hess}{  \operatorname{Hess}}

\newmacro{\R}{  \operatorname{R}}           % retraction
\newmacro{\D}{  \operatorname{D}}           % Differential
\newmacro{\Jac}{  \operatorname{Jac}}           % Differential

\newcommand{\distMgeo}[1][\M]{\dist^{\text{geo}}_{#1}}
\newcommand{\distM}[1][\M]{\dist_{#1}}
\newcommand{\projM}[1][\M]{\proj_{#1}}

\newmacro{\fun}{  F}   % smooth fun
\newmacro{\funalt}{ \tilde{F}}   % smooth fun
\newmacro{\funman}{  f} % Manifolds
\newmacro{\funs}{  f}   % smooth fun
\newmacro{\funns}{  g}  % non smooth fun
\newmacro{\funcomp}{  F}  % composition
\newmacro{\mapping}{ c}


\newop{\proj}{\pi}
\newop{\prox}{prox}
\newop{\ri}{ri}
\newop{\Conv}{Conv}
\newop{\Aff}{Aff}
\newop{\Par}{Par}
\newop{\Diag}{Diag}
\newop{\diag}{diag}
\newop{\trace}{trace}
\newop{\bnd}{bnd}
\newop{\rbd}{rbd}

\newcommand{\PGop}{\mathsf{T}} % prox grad.

% BUG: to choose between the two
\newcommand{\B}{\mathcal{B}}
\newmacro{\ball}{\mathcal{B}}

\newcommand{\funtot}{F}     % Total
\newcommand{\funspart}{f}   % smooth part
\newcommand{\funnspart}{g}  % nonsmooth part


%----------------------------------------------------------------------
%% Composite setting
%----------------------------------------------------------------------
\newcommand{\Fsext}[1][\funcomp]{\debug{\tilde{#1}}}   % smooth extension of $F$
\newcommand{\partialext}[1][\M]{\debug{\partial^{#1}}}        % smooth extension of $F$
\newcommand{\Mext}[2][\M]{\debug{#1_{#2}}}             % smooth extension of manifold
\newcommand{\criext}[1][\M]{\debug{\cri^{#1}}}         % smooth extension of manifold


\newmacro{\ndim}{n}
\newmacro{\inputSpace}{\bbR^{\ndim}}

% \newmacro{\ndimalt}{m}
% \newmacro{\Rnalt}{\bbR^{\ndimalt}}
% \newmacro{\interSpace}{\Rnalt}

\newmacro{\vxinter}{\vy}
\newmacro{\Minter}{\M^{\funns}}
\newmacro{\Mopt}{\opt[\M]}
\newmacro{\Mr}{\M^{\lambda_{\max}}_{\mult}}
\newmacro{\MI}{\M^{\max}_I}
\newmacro{\Mell}{\M_{I}^{\ell_{1}}}


\newmacro{\lammax}{\lambda_{\max{}}}
\newmacro{\mult}{r}

\newcommand{\opt}[1][\vx]{\debug{{#1}^\star}}		% for optimal (x*)
\newcommand{\crit}[1][\vx]{\debug{\bar{#1}}}				% for current iterate (X by default)

\DeclareMathOperator{\Ima}{Im}
\renewcommand{\Im}{\Ima}
\DeclareMathOperator{\Kera}{Ker}
\renewcommand{\ker}{\Kera}

\newmacro{\cm}{ e}
\newcommand{\cri}{\debug{c}_{\debug{\text{ri}}}}
\newcommand{\cmap}{\debug{c_{\text{map}}}}

\newmacro{\stepLowBnd}{\varphi}
\newmacro{\stepUpBnd}{\Gamma}
\usepackage{accents}                                  % for low bar
\newmacro{\stepLow}{\underaccent{\bar}{\step}}
\newmacro{\stepUp}{\bar{\step}}

\newcommand{\currinter}[1][\vx]{\debug{#1^{\funns}_{\ite}}}
\newcommand{\currFsext}[1][\vx]{\debug{\tilde{#1}_{\ite}}}

\newmacro{\Mcodim}{p}
\newmacro{\maneqSpace}{\bbR^{\Mcodim}}

\newcommand{\VU}{$\mathcal{VU}$}

%----------------------------------------------------------------------
%% SQP notations
%----------------------------------------------------------------------
\newmacro{\dSQP}{d^{\mathrm{SQP}}}
% \newcommand{\dSQP}[1][\vx]{d^{\mathrm{SQP}}(#1)}
\newmacro{\currdSQP}{d_{\ite}^{\mathrm{SQP}}(\curr[\vx])}

\newmacro{\dSQPt}{d^{\mathrm{SQP}}_{t}}
\newmacro{\dSQPn}{d^{\mathrm{SQP}}_{n}}
\newmacro{\dSQPr}{d^{\mathrm{SQP}}_{r}}

\newmacro{\gred}{\grad_{\M} \funcomp}
\newmacro{\redg}{\grad_{\M} \funcomp}
\newmacro{\redH}{\Hess_{\M} \funcomp}
% \newmacro{\Z}{Z^{-}}
% \newmacro{\Zt}{Z^{-\top}}


\newmacro{\hessLag}{\nabla^2_{xx} L}

\newmacro{\dcorr}{d^{\mathrm{corr}}}
% \newcommand{\dcorr}[1][\vx]{d^{\mathrm{corr}}(#1)}
% \newcommand{\Lagmult}[1][\vx]{\lambda(#1)}
\newmacro{\Lagmult}{\lambda}

\newmacro{\lsstep}{\alpha}
\newmacro{\lsArmijoParam}{m}

\newcommand{\IdentifProc}{\mathrm{IdentifProc}}

\newmacro{\Msqp}{M}

\newmacro{\stepinit}{\step_{\text{init}}}



%%%%%%%%%%%%%%%%%%%%%%%%%%%%%%%%%%%%%%%%%%%%%%%%

%----------------------------------------------------------------------
%% Saddle Point
%----------------------------------------------------------------------
\newmacro{\minmax}{\Phi}		% for minmax objective
\newmacro{\sadobj}{\minmax}		% for saddle objective

\newmacro{\minvar}{\point_{1}}		% for minimization variable
\newmacro{\minvaralt}{\alt\minvar}		% for alternate minvar
\newmacro{\minvars}{\points_{1}}		% for minvar space

\newmacro{\maxvar}{\point_{2}}		% for maximization variable
\newmacro{\maxvars}{\points_{2}}		% for maxvar space
\newmacro{\maxvaralt}{\alt\maxvar}		% for alternate maxvar


%----------------------------------------------------------------------
%% Game theory
%----------------------------------------------------------------------
\DeclareMathOperator{\Nash}{NE}		% for Nash set
\DeclareMathOperator{\CE}{CE}		% for CE set
\DeclareMathOperator{\CCE}{CCE}		% for Hannan set

\DeclareMathOperator{\brep}{br}		% for best responses
\DeclareMathOperator{\reg}{Reg}		% for regret
\DeclareMathOperator{\preg}{\overline{Reg}}		% for pseudo-regret
\DeclareMathOperator{\val}{val}		% for value of game
\DeclareMathOperator{\NI}{NI}		% for Nikaido-Isoda function

\newcommand{\strat}{\point}		% for mixed strategy
\newcommand{\strats}{\points}		% for set of mixed strategies
\newcommand{\intstrats}{\strats^{\circ}}		% for set of interior strategies
\newcommand{\eq}{\sol}		% for equilibria
\newcommand{\eqs}{\sols}		% for equilibria

\newmacro{\play}{i}		% for main player index
\newmacro{\playalt}{j}		% for alternate player index
\newmacro{\nPlayers}{N}		% for number of players
\newmacro{\players}{\mathcal{\nPlayers}}		% for set of players

\newmacro{\pure}{a}		% for main strategy index
\newmacro{\purealt}{a'}		% for alternate strategy index
\newmacro{\nPures}{n}		% for number of strategies
\newmacro{\pures}{\mathcal{A}}		% for set of strategies

\newmacro{\cost}{c}		% for cost function
\newmacro{\loss}{\ell}		% for loss function
\newmacro{\pay}{u}		% for payoff function
\newmacro{\payv}{v}		% for payoff vector
\newmacro{\pot}{F}		% for potential function

\newmacro{\game}{\mathcal{G}}		% for game
\newmacro{\gamefull}{\game(\players,\points,\pay)}		% for full game

\newmacro{\fingame}{\Gamma}		% for finite game
\newmacro{\fingamefull}{\Gamma(\players,\pures,\pay)}		% for full finite game

\newmacro{\mixgame}{\simplex(\fingame)}		% for finite game

\newmacro{\corstrat}{\pi}		% for mixed strategy
\newmacro{\corprob}{\chi}		% for mixed strategy
\newmacro{\cormarg}{\point}		% for mixed strategy
\newmacro{\corprobs}{\points_{\mathrm{c}}}


%**********************************************************************
%*****	MACROS: PROBABILITY AND STATISTICS
%**********************************************************************

%----------------------------------------------------------------------
%% Probability
%----------------------------------------------------------------------
\DeclareMathOperator{\ex}{\mathbb{E}}		% for expectations
\DeclareMathOperator{\prob}{\mathbb{P}}		% for probability
\DeclareMathOperator{\Var}{Var}		% for variance
\DeclareMathOperator{\supp}{supp}		% for support
\DeclareMathOperator{\simplex}{\Delta}		% for simplices
\DeclareMathOperator{\spectraplex}{Sp}		% for simplices
\DeclareMathOperator{\unif}{unif}		% for uniform distribution

\newmacro{\sample}{\omega}		% for samples
\newmacro{\samples}{\Omega}		% for sample space
\newmacro{\filter}{\mathcal{F}}		% for filtrations
\newmacro{\probspace}{(\samples,\filter,\prob)}		% for probability space

\newmacro{\mean}{\mu}		% for mean of distribution
\newmacro{\sdev}{\sigma}		% for mean of distribution
\newmacro{\variance}{\sdev^{2}}		% for mean of distribution

\newmacro{\dkl}{D_{\mathrm{KL}}}		% for Kullback Leibler
\newmacro{\as}{\textpar{a.s.}\xspace}		% for almost surely

\providecommand\given{}		% empty command for conditionals

% \DeclarePairedDelimiterXPP{\exof}[1]{\ex}{[}{]}{}{%		% for conditional expectations
% \renewcommand\given{\nonscript\:\delimsize\vert\nonscript\:\mathopen{}} #1}

% \DeclarePairedDelimiterXPP{\probof}[1]{\prob}{(}{)}{}{%		% for conditional probabilities
% \renewcommand\given{\nonscript\:\delimsize\vert\nonscript\:\mathopen{}} #1}



%**********************************************************************
%*****	MACROS: VARIOUS
%**********************************************************************

%----------------------------------------------------------------------
%% Bregman functions
%----------------------------------------------------------------------
\DeclareMathOperator{\Eucl}{\Pi}		% for Euclidean projection
\DeclareMathOperator{\logit}{\Lambda}		% for logit map

\newmacro{\hreg}{h}		% for regularizer
\newmacro{\breg}{D}		% for Bregman divergence
\newmacro{\proxmap}{P}		% for mirror map
\newmacro{\mirror}{Q}		% for mirror map
\newmacro{\fench}{F}		% for Fenchel coupling
\newmacro{\hstr}{K}		% for strong convexity constant
\newmacro{\depth}{H}		% for regularizer depth
\newmacro{\radius}{R}		% for Bregman divergence
\newmacro{\zone}{\mathbb{D}}		% for Bregman zone
\newmacro{\subpoints}{\points^{\circ}}		% for Bregman interior

%----------------------------------------------------------------------
%% Sequences and recursions
%----------------------------------------------------------------------
\newcommand{\new}[1]{#1^{+}}		% for new iterate

\newmacro{\state}{X}		% for main iterate
\newmacro{\dstate}{Y}		% for other iterate
\newmacro{\signal}{V}		% for input signal

\newmacro{\step}{\gamma}		% for step-size
\newmacro{\learn}{\eta}		% for learning rate

\newcommand{\lazy}[1]{#1^{\textup{lazy}}}		% for lazy iterate
\newcommand{\eager}[1]{#1^{\textup{eager}}}		% for eager iterate

\newmacro{\ite}{k}
\newmacro{\initite}{0}
\newmacro{\afterinitite}{1}
\newmacro{\sumite}{n}
\newmacro{\Afterite}{K}

\newcommand{\prev}[1][\vx]{\debug{#1_{\ite\debug{-1}}}}		% for previous iterate (X by default)
\newcommand{\curr}[1][\vx]{\debug{#1_{\ite}}}				% for current iterate (X by default)
\renewcommand{\next}[1][\vx]{\debug{#1_{\ite\debug{+1}}}}		% for next iterate (X by default)
\newcommand{\afternext}[1][\vx]{\debug{#1_{\ite\debug{+2}}}}		% for next iterate (X by default)

\newcommand{\init}[1][\vx]{\debug{#1_{\initite}}}			% for init iterate (X by default)

%----------------------------------------------------------------------
%% Graph theory
%----------------------------------------------------------------------
\newmacro{\graph}{\mathcal{G}}
\newmacro{\vertices}{\mathcal{V}}
\newmacro{\edges}{\mathcal{E}}


%----------------------------------------------------------------------
%% Learning
%----------------------------------------------------------------------
\DeclareMathOperator{\ReLU}{ReLU}		% for operator
\newcommand{\pred}{{\tt pred}}		% for operator


%----------------------------------------------------------------------
%% O-minimality
%----------------------------------------------------------------------
\newcommand{\Ostr}{\mathcal{O}}                               % o-min structure
\newcommand{\Rsemialg}{$\mathcal{R_{\text{alg.}}}$} % o-min structure
\newcommand{\Ran}{$\bbR_{\text{an.}}$}              % o-min structure
\newcommand{\Rexp}{$\bbR_{\text{exp.}}$}            % o-min structure
\newcommand{\Ranexp}{$\bbR_{\text{an,exp.}}$}       % o-min structure

%----------------------------------------------------------------------
%% Current paper
%----------------------------------------------------------------------
\newmacro{\nPieces}{k}
\newmacro{\polyDeg}{N}
\newmacro{\smoothDeg}{m}

\newmacro{\Cm}{\C^\smoothDeg}
\newmacro{\cancube}{\appDom}

\newmacro{\nslice}{L}
\newmacro{\nCuts}{l}
\newmacro{\appDom}{A} % We may set this to $A$ again.
\newmacro{\appDomRadius}{r_{\appDom}}

\newcommand{\polySpace}[1][\polyDeg]{\debug{\mathcal{P}}_{#1}}
\newcommand{\piecePolySpace}[1][\nPieces]{\debug{\mathcal{P}}_{\polyDeg}^{#1}}
\newcommand{\piecePolySpaceCuts}[1][\nCuts]{\debug{\mathcal{\widetilde{P}}}_{\polyDeg}^{#1}}

\DeclareMathOperator{\diffSet}{diff}
\newcommand{\diff}[1][\smoothDeg]{\debug{\diffSet}_{#1}}

\newmacro{\cube}{c}
\newmacro{\cubes}{\mathcal{C}_p}
\newmacro{\cubesU}{\mathcal{C}^{U}_p}
\newmacro{\cubesV}{\mathcal{C}^{V}_p}

\newmacro{\const}{C}
\newmacro{\consta}{C'}
\newmacro{\constalt}{C'}
\newmacro{\constb}{C^{\circ}}

\newmacro{\Cnm}{C_{\ndim,\smoothDeg,\cancube}}
\newmacro{\Cjack}{C^{2}_{\ndim,\smoothDeg,\cancube}}
\newmacro{\Cfeff}{C^{1}_{\ndim,\smoothDeg}}
\newmacro{\Cstrat}{C^{3}}
\newmacro{\Cdist}{C^{4}}
\newmacro{\Cerr}{C^{5}}
\newmacro{\Cfbanach}{\bar{C}_f}


\newmacro{\bNode}{\mathcal{T}_{B}}
\newmacro{\lNode}{\mathcal{T}_{L}}

\newmacro{\nSample}{n_{\text{samp}}}

\newmacro{\Mvol}{{v}}
\newmacro{\Mrad}{\mathrm{rad}}

\newmacro{\fCone}{f^\text{cone}}
\newmacro{\sCone}{s^\text{cone}}
\newmacro{\rCone}{r^\text{cone}}

\newmacro{\pmid}{p^\mathrm{mid}}
\newmacro{\rCube}{\diam}

\newmacro{\polyboundary}{p^{boundary}}
\newmacro{\piece}{s}


\newmacro{\epsmarg}{\Cstrat \nCuts^{-\frac{2}{\ndim-1}}}
\newmacro{\Mbig}{\M_0^\piece}

%%% Local Variables:
%%% mode: latex
%%% TeX-master: "main_neurips"
%%% End:
