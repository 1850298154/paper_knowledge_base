
\section{\edit{Distributed} Task Assignment} \label{sec:assign}


The goal of task assignment is to uniquely allocate each UAV to a point in the desired formation.
A natural objective for this task is to minimize the overall distance from the UAVs to their assignments in the desired formation.
We are interested in the final 3D geometric shape rather than the exact location and yaw of the end formation.
To this effect, we allow a rotation $R$ around the $z$-axis, and translation $t$ of the desired formation coordinates that minimize the overall distance from the UAVs to the rotated and translated desired formation.
More precisely, our objective is to find the assignment map $\sigma$ that solves
\begin{equation}\label{eq:assign_prob}
\begin{aligned}
\underset{\substack{R\in\mathcal{R}_z,\;t\in\br^3\\\sigma\in\mathrm{S}_n}}{\text{minimize}} && \sum_{i=1}^{n} \| q_i - (R \, p_{\sigma(i)}+t) \|^2,
\end{aligned}
\end{equation}
where $q_i$ denotes the UAV positions, 
$\mathrm{S}_n$ is the symmetric group of all permutations from the set $\{1,\dots, n\}$ to itself,
and $\mathcal{R}_z$ is the set of rotation matrices around the $z$-axis. Recall that the 
$z$-axes of UAV body frames are assumed to be aligned, as per Section~\ref{sec:motionplanning}.


\edit{Three challenges arise in finding a distributed solution for~\eqref{eq:assign_prob}.
First, the objective of \eqref{eq:assign_prob} includes the positions of all UAVs, whereas each UAV only obtains the positions of its neighbors.
Second, the UAV body frames are not aligned and the UAVs only know the transformations between their body frames and their neighbors.  
Lastly, \eqref{eq:assign_prob} is a non-convex mixed integer program, for which finding the global optimizer becomes intractable for large $n$.} 
As computational efficiency and scalability are of utmost concern for UAV platforms, 
we settle with obtaining a suboptimal answer via a coordinate descent approach and an auction assignment strategy.
This approach is inspired by \cite{macdonald2011multi}, which in contrast uses a centralized Hungarian algorithm for assignment.
Every iteration of our algorithm consists of an \textit{alignment} stage and an \textit{assignment} stage, where the assignment is fixed as we solve for an alignment, and vice versa.


\subsection{Alignment}

Given an assignment $\sigma^*$ (e.g., identity assignment ${\sigma^*(i)=i}$ for every new formation, or prior assignment computed for the same formation), UAV $i$ solves a \edit{distributed} formulation of \eqref{eq:assign_prob} given by
\begin{equation}\label{eq:assign_prob_fix_sigma}
\begin{aligned}
\underset{\substack{R_i\in \mathcal{R}_z,\;t_i\in\br^3}}{\text{minimize}} && \sum_{j\in\mathcal{N}'_i} \| q_j - (R_i \, p_{\sigma^*(j)} + t_i) \|^2,
\end{aligned}
\end{equation}
where ${\mathcal{N}}'_i \eqdef \mathcal{N}_i\cup \{i\}$, and positions $q_j$ are in UAV $i$'s start frame.
In \eqref{eq:assign_prob_fix_sigma}, UAV $i$ aims to align the desired formation to minimize the distance to its own and neighbors' positions based on the given assignment $\sigma^*$.   
Fig.~\ref{fig:alignment} gives an illustrative example of this stage.
Problem \eqref{eq:assign_prob_fix_sigma} is the well-known point cloud alignment problem, for which the optimal solution $\left(R^*_i,t^*_i\right)$ 
is obtained from Arun's method~\cite{arun1987least}  using the projection of $q_j$ and $p_{\sigma^*(j)}$ on the {$x$-$y$ plane} for the rotation. %



\begin{figure} [t!]
	\centering
	\includegraphics[width=\columnwidth]{Figs/alignment4.pdf}    
		\caption{
		Illustrative 2D alignment example with four vehicles from UAV~1's perspective.
        UAV $4$ is not shown because it does not communicate with UAV $1$.
		\textbf{(left)}~New formation graph, UAV $1$ and its neighbors before the alignment.
		\textbf{(right)}~Aligned formation 
		based on UAV $1$, its neighbors and their corresponding formation points. 
		The formation point associated to UAV 4 is faded to indicate that UAV 1 does not have information about it.
		}	
	\label{fig:alignment}
\end{figure}






\subsection{Assignment}

In this stage, the UAVs aim to collaboratively find an assignment based on the results obtained from their alignment stage. The assignment problem is formulated as
\begin{equation}\label{eq:assign_prob_fix_Rt}
\begin{aligned}
\underset{\substack{\sigma\in\mathrm{S}_n}}{\text{minimize}} && \sum_{i=1}^{n} \| q_i - (R_i^* \, p_{\sigma(i)}+t_i^*) \|^2.
\end{aligned}
\end{equation}
Problem~\eqref{eq:assign_prob_fix_Rt} is a linear sum assignment problem that can be solved optimally by methods such as the distributed Hungarian algorithm~\cite{giordani2010distributed, chopra2017distributed}, which converges in $\text{O}(n^3)$ iterations. 
Due to onboard resource constraints, we trade optimality for computational efficiency by using our prior work, the consensus-based auction algorithm (CBAA)~\cite{choi2009consensus}. 
CBAA is guaranteed to converge in at most $n d$ iterations, where $d$ is the diameter of the formation graph, $\mathcal{G}$.




To bring \eqref{eq:assign_prob_fix_Rt} into the standard form for applying CBAA, let binary variables $x_{ij}$ represent the assignment $\sigma$, where ${x_{ij} = 1}$ if ${\sigma(i) = j}$ and $0$ otherwise.
Further, let ${c_{ij} \eqdef 1\,/\,\| q_i-(R_i^* \, p_j + t_i^*) \|^2}$ denote the positive score for assigning UAV $i$ to formation point $j$. In practice, a small positive number can be added to the denominator of $c_{ij}$ to avoid division by zero.
It is straightforward to show that~\eqref{eq:assign_prob_fix_Rt} can be expressed as the integer program
\begin{equation}\label{eq:cbaa_problem}
\begin{aligned}
\underset{x_{ij} \in \{0,1\}}{\text{maximize}} &
&&\sum_{i,j=1}^{n} c_{ij} \,x_{ij}, \\
\text{subject to}
&&& \textstyle\sum_{i=1}^{n} x_{ij} = 1, \; \forall_{j} \\
&&& \textstyle\sum_{j=1}^{n} x_{ij} = 1,  \; \forall_{i} 
\end{aligned}
\end{equation}
where the constraints on $x_{ij}$ enforce conflict-free and one-to-one assignment captured by $\sigma \in S_n$ in~\eqref{eq:assign_prob_fix_Rt}, and maximizing~\eqref{eq:cbaa_problem} is equivalent to minimizing the overall distance from the UAVs to the rotated and translated formation in~\eqref{eq:assign_prob_fix_Rt}.





    



In executing CBAA, UAV $i$ stores and updates its own assignment and a list of winning bids (initialized as zeros) for all formation points. Each iteration of CBAA consists of an auction phase and a consensus phase. 
In the auction phase, UAV $i$ determines which formation point it would like to be assigned to in three steps: (1) check if any formation point $p_j$ produces a score $c_{ij}$ higher than its current winning bid; (2) of those formation points, set $x_{ij}=1$ for the $p_j$ that produces the highest score; (3) update the bid for the winning $p_{j}$ with new score $c_{ij}$.
In the consensus phase, vehicles converge on a common winning bid list. UAV $i$ exchanges its winning bid list with its neighbors and updates its list with the highest values from its own and all received lists. It sets $x_{ij}=0$ if the new winning bid for $p_j$ is higher than $c_{ij}$, implying that a different vehicle has been assigned to $p_j$.

Since CBAA is distributed, no central authority exists to affirm convergence.
Therefore, we enforce a synchronous execution to terminate the algorithm in $n d$ iterations, which is the maximum number of iterations required to guarantee convergence. 
The final assignment is recovered by letting $\sigma^*(i)=j$ for each $x_{ij}=1$.
CBAA guarantees a conflict-free assignment even though UAVs \edittwo{do not have a common reference frame} and may have inconsistent position estimates or different $R^*_i$ and $t^*_i$ for alignment.
\edittwo{We emphasize that although $c_{ij}$ is calculated by each UAV using only local knowledge, CBAA assigns UAVs to formation points without conflict.}
Further, it retains at least $50\%$ of the optimal performance; that is, given the optimal overall score $C^*$ of \eqref{eq:cbaa_problem} and the $C$ resulted from CBAA, $C/C^* \geq 0.5$. 
