
\section{System Overview}\label{sec:overview}

A schematic representation of our pipeline is depicted in Fig.~\ref{fig:System} for a swarm of $n$ multirotor UAVs.
The key components of this pipeline include modules for localization and coordination of the vehicles, which require exchanging information between a subset of UAV peers referred to as \textit{neighbors}.
\edit{The main goal of the pipeline is formation flying.
We assume a desired formation shape is specified by an operator.
This desired formation is used to design the required gains for formation control and both are given as input to the vehicles.
}




\begin{figure}[t!]
	\begin{center}
		\includegraphics[trim =0mm 0mm 0mm 0mm, clip, width=\columnwidth]{Figs/System.pdf}	
		\caption{\edit{Modules of our formation flying pipeline. The desired formation is used to design required gains for formation control. The localization framework provides self and relative pose measurements. The coordination framework assigns each UAV to a formation point and plans its motion to attain formation flying.}}
		\label{fig:System}
	\end{center}
\end{figure}



\edit{
With onboard localization,
the pose of a UAV with respect to its own start frame, which is fixed at its initial pose, is estimated using VIO.}
These self pose estimates provide feedback to the low-level controller, which stabilizes the UAV and tracks a reference velocity specified by the high-level formation control strategy.
Through inter-vehicle pose updates, a UAV acquires relative pose estimates of its neighbors by transforming their poses into its own start frame.
This process requires knowledge of the transformations that relate the UAVs' start frames.
Several methods can be used to obtain these transformations. 
For instance, if two vehicles have a common field of view, once the correspondence among the 3D landmarks reconstructed by VIO is determined, the relative pose between the UAVs' start frames can be found using Arun's method~\cite{arun1987least}.
For simplicity, the transformations are obtained in our experiments by initializing the UAVs at pre-specified locations.
\edit{
Note that the UAVs do \textit{not} require the transformations to \textit{non-neighboring} vehicles.
}





\edit{The coordination framework handles formation flying of the UAV swarm.
This framework consists of the task assignment and formation control modules, as depicted in Fig.~\ref{fig:System}}.
Formation control is concerned with finding collision-free trajectories that bring the UAVs to a desired formation.
A \textit{desired formation} is defined by a graph $\sg$ with vertices located at 3D points $p_1, \dots, p_n$ and edges connecting the vertices to indicate neighbors.
\edit{Graph $\sg$ is used in designing the formation control and is also 
broadcasted to the UAVs from the base station.}
The vehicles aim to achieve the overall geometric shape specified by the points $p_i$ (rather than the exact location and orientation of this point configuration in the space).
Throughout this paper, we assume that $\sg$ is undirected, connected, and \textit{universally rigid} \cite{Gortler2014}.
Informally, rigidity implies that $\sg$ cannot be deformed without violating the desired distances between formation points. 


The goal of task assignment is to uniquely allocate each UAV to a point in the desired formation.
Each UAV $i$ is assigned to a formation point $p_j$ using a one-to-one assignment map $\sigma$ with $\sigma(i) = j$ (see Section~\ref{sec:assign}). 
The set of neighbors of UAV $i$, denoted by $\mathcal{N}_i$, is defined as the set of UAVs $j$ such that $p_{\sigma(j)}$ is connected to $p_{\sigma(i)}$ by an edge in $\sg$.
UAV $i$ and its neighbors communicate to attain relative pose measurements using the localization framework.
\edittwo{We emphasize that $\mathcal{N}_i$ is defined according to the current assignment map $\sigma$ and formation graph $\sg$, and that these neighbors are used for both formation control and communication.
As tasks are reassigned, the set of neighbors, and therefore communication links, may change.}
