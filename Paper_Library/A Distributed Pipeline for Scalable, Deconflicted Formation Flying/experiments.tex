

\begin{table}[t!]	
    \ra{1.1}
	\centering
	\caption{Execution time of the CVX solver used for \eqref{eq:OptimCVX} vs. our ADMM solver \eqref{eq:ADMM} for obtaining formation gains for different number of vehicles. Reported times are in seconds and rounded to two decimals.}
	\begin{tabularx}{\columnwidth}{@{} l rrrrr @{}} %
		\toprule
		Algorithm  &  \multicolumn{5}{c @{}}{Number of Vehicles}\\ 
		\cmidrule(l){2-6}
		& {5} & {20} & {50} & {100} &  {200}  \\
		\midrule
		CVX-SDP time         & 0.54 & 32.48 & 8684.24 &  \texttt{OOM} & \texttt{OOM}  \\
		ADMM time (ours) & 0.01 & 0.03 & 1.31 & 12.26 & 134.67  \\
		\bottomrule
	\end{tabularx}
	\\ [0.2em]
	\begin{flushleft}
	{\scriptsize \texttt{OOM}: Out of memory}
	\end{flushleft}
	\label{tbl:gaintimes}
	\vskip-0.05in
\end{table}



\section{Experimental Results} \label{sec:exper}

This section shows that our distributed formation control and \edit{distributed} task assignment solutions scale with the number of UAVs, resolve gridlocks resulting from collision avoidance, and reduce the total distance traveled.

First, we investigate scalability by comparing the runtime of our ADMM-based solver~\eqref{eq:ADMM} with the interior-point method used in CVX (\href{http://cvxr.com/cvx}{\texttt{http://cvxr.com/cvx}}) to solve the SDP formulation~\eqref{eq:OptimCVX}.
These results are shown in Table~\ref{tbl:gaintimes}, and are generated in MATLAB using an Intel Core i7-7700K with \SI{32}{\giga\byte} RAM.
While the interior-point method becomes intractable for formations with more than 50 vehicles, our ADMM approach can solve for the control gains in seconds.

Second, we use software-in-the-loop simulations and hardware demonstrations to highlight how task assignment leads to quicker formation convergence with nearly \SI{100}{\percent} success.
Our pipeline is implemented in \texttt{C++} using Robot Operating System (ROS)~\cite{Quigley2009}.
Hardware demonstrations use a team of custom-built hexarotors, each with a diameter of \SI{0.5}{\meter} and an all-up-weight of \SI{1.1}{\kilo\gram}.
Code runs onboard the Qualcomm Snapdragon Flight board that includes a platform-optimized VIO package that outputs odometry at \SI{30}{\hertz} \cite{VIO}.
For simplicity of the implementation and the safety of the vehicles, we use our localization module (see Fig.~\ref{fig:System}) to inform each vehicle of every other vehicle's position.
However, the information about non-neighbors is only used for collision avoidance and could alternatively be found using, for example, onboard cameras.

\begin{table}[t]
\centering
\ra{1.1}
\caption{
Simulation results for 30 vehicles over 100 Monte Carlo trials.
Using our distributed assignment algorithm, we obtain results closer to the optimal, but centralized, Hungarian approach.
}
\label{tbl:simresults}
\begin{tabular}{l@{\hskip3pt}r x{1cm}x{0.6cm}x{1cm}x{0.6cm} r}
\toprule
&& \multicolumn{2}{c}{Distance Traveled (m)} & \multicolumn{2}{c}{Convergence Time (s)} & Success \\ \cmidrule{3-6}
&& mean & std & mean & std & \\ \midrule
\multirow{2}{*}{\normalsize{NA}}
& nc &
28.2 & 4.1 & 131.0 & 30.0 & \SI{58}{\percent} \\
& c    &
28.6 & 3.6 & 134.0 & 25.2 & \SI{66}{\percent} \\ [0.2em]
\multirow{2}{*}{\normalsize{A}}
& nc &
10.9 & 2.2 & 64.7 & 38.6 & \SI{98}{\percent} \\
& c    &
9.9 & 2.0 & 68.1 & 43.7 & \SI{96}{\percent} \\ [0.2em]
\multirow{1}{*}{\normalsize{H}}
& c    &
5.1 & 1.0 & 40.6 & 53.5 & \SI{100}{\percent} \\
\bottomrule
\end{tabular}
\\ [0.2em]
\begin{flushleft}
{\scriptsize NA: no assignment\quad A: \edit{distributed} assignment (ours)\quad H: centralized Hungarian} \\
{\scriptsize c: complete graph\quad nc: non-complete graph}
\end{flushleft}
\vskip-0.05in
\end{table}


\subsection{Simulations}

We perform Monte Carlo trials to measure the impact of \edit{distributed} task assignment on a large team of vehicles.
A trial consists of randomly initializing 30 UAVs in a \SI[product-units = single]{20 x 20}{\meter} area, where the minimum distance between initial positions is \SI{1.5}{\meter}.
A random formation is generated for each trial within a \SI[product-units = single]{15 x 15 x 2}{\meter} volume, with a minimum distance between formation points of \SI{2}{\meter}.
A trial is completed once the swarm has successfully reached the formation from the random initialization.
If the swarm is trapped in a gridlock for more than \SI{90}{\second}, the trial is considered indefinitely gridlocked and is aborted.

For each trial, our pipeline is tested in three main configurations: with centralized assignment, with distributed assignment, and without assignment.
Except for centralized assignment, each configuration is further tested with both a complete and randomly generated non-complete formation graph.
Centralized assignment provides an optimal baseline for comparison and is performed using the Hungarian algorithm with a complete graph.
The assignment algorithms are executed at a period of \SI{2}{\second}, allowing the swarm to resolve gridlocks by enabling new collision-free motion directions.


\begin{figure}[t]
    \centering
    \includegraphics[width=1\columnwidth]{Figs/mitacl100.png}
    \caption{Large-scale simulation with 100 UAVs. The last \SI{40}{\second} of motion are shown and UAVs are depicted at 2x scale for better visibility.}
	\label{fig:mitacl100}
\end{figure}

Table~\ref{tbl:simresults} shows the comparison results, where for successful trials the average distance traveled and average flying time to converge to the desired formation are reported.
As expected, the centralized Hungarian approach obtains \SI{100}{\percent} success rate with the shortest distance traveled \edit{and only an average of 2.0 reassignments} to converge to the formation.
\edit{However, this approach has a computational cost of $O(n^3)$ in the number of vehicles and relies on a centralized coordinator in a common reference frame with complete knowledge of the swarm (Fig.~\ref{fig:frame-a}).
On the other hand, our CBAA-based assignment algorithm is a more scalable deconfliction strategy that is executed in the non-aligned frames of the UAVs (Fig.~\ref{fig:frame-c}) and is nearly optimal in practice as confirmed by the \SI{97}{\percent} average convergence rate in Table~\ref{tbl:simresults}, with an average of 11.6 reassignments}.
Compared to formation control without assignment, our algorithm allows the swarm to achieve formation convergence in nearly every case and in half the amount of time, on average.
The results also show that, on average, there is no significant performance decrease between complete and non-complete formation graphs.
Thus, non-complete formation graphs can be used to reduce communication overhead without sacrificing the convergence rate or ability to reach the desired formation.

\edit{We remark that symmetric formations may lead our task assignment strategy~\eqref{eq:assign_prob} to exhibit momentary swapping behavior.
However, noise in each UAV's sensing is included in the simulation and we have not observed convergence failure of~\eqref{eq:assign_prob} in simulation or hardware.
We believe this swapping behavior is caused by ignoring the vehicle dynamics in~\eqref{eq:assign_prob} and consider this in future work.
}

To demonstrate scalability, we performed a large-scale simulation with 100 UAVs randomly initialized in a \SI[product-units = single]{60 x 30}{\meter} area.
This simulation was performed using Amazon Web Services.
Vehicles achieve the \texttt{MIT ACL} formation using a sparse formation graph with only \SI{24}{\percent} of the edges in a complete graph, which is beneficial for bandwidth-limited communication.
The last \SI{40}{\second} are shown in Fig.~\ref{fig:mitacl100}, where the motion traces indicate deconfliction due to reassignment.

\begin{figure}[t!]
	\centering
	\begin{subfigure}[b]{0.32\columnwidth}
	    \centering
	    \includegraphics[width=\textwidth, trim={3cm 0.5cm 3cm 2cm},clip]{Figs/formation/nonfc_pyramid_grid.png}
	    \caption{\scriptsize Pentagonal pyramid}
	\end{subfigure}
	\begin{subfigure}[b]{0.32\columnwidth}
	    \centering
	    \includegraphics[width=\textwidth, trim={3cm 0.5cm 3cm 2cm},clip]{Figs/formation/nonfc_prism_grid.png} 
	    \caption{\scriptsize Triangular prism}
	\end{subfigure}
	\begin{subfigure}[b]{0.32\columnwidth}
	    \centering
	    \includegraphics[width=\textwidth, trim={3cm 0.5cm 3cm 2cm},clip]{Figs/formation/nonfc_plane_grid.png} 
	    \caption{\scriptsize Slanted plane}
	\end{subfigure}
	\caption{Non-complete formation graphs used in the hardware experiments.}
	\label{fig:exp_formations}
\end{figure}

\begin{figure}[t]
    \centering
    \includegraphics[trim=20 85 20 120, clip, width=1\columnwidth]{Figs/gridlock_annotated.png}
    \caption{Without assignment, UAVs attempting to achieve the pyramid formation are gridlocked due to collision avoidance.}
	\label{fig:hwgridlock}
\end{figure}

\subsection{Hardware Demonstrations}

We demonstrate formation flight with six UAVs by cycling through the three formations illustrated in Fig.~\ref{fig:exp_formations}.
The minimum distance between desired formation points is \SI{2}{\meter} for each formation.
\edit{Because the time required to calculate the formation gains from~\eqref{eq:ADMM} is small, in our experiments each UAV independently calculates the formation gains onboard in \SI{20}{\milli\second}.
In this case, the base station is used only to dispatch the desired formation graph to the UAVs.}


The UAVs are initialized at pre-specified locations so that the transforms between vehicles' VIO start frames are known.
After taking off and hovering, an operator dispatches each desired formation to the swarm.
Four configurations are tested by cycling through the formations twice.
For each configuration, a total of six trials are recorded and averaged over, where a trial is the transition from the current swarm state to the next desired formation.
When assignment is used, the period of reassignment is \SI{1.2}{\second}.

Consistent with the simulations, the results in Table~\ref{tbl:hwresults} indicate that without our assignment strategy, the vehicles fail to achieve the desired formation in up to \SI{50}{\percent} of the trials, while every trial using assignment was successful.
An example of convergence failure is shown in Fig.~\ref{fig:hwgridlock}.

The supplementary video provides insights into the qualitative behavior of our system.
Note that the achieved formations in the video are occasionally inverted from the desired formations shown in Fig.~\ref{fig:exp_formations}.
Recall that our formation control aims to achieve the desired \textit{shape}.
The inverted formations seen in experiments are due to negative scaling in the $z$-axis.
We also point out that the formations shown in Fig.~\ref{fig:exp_formations} are not universally rigid. 
Universal rigidity is a sufficient condition for our gain design, but not necessary.
In practice, formations with sparser graphs can be used,
so long as the recovered gain matrix leads to a negative objective~\eqref{eq:OptimX}.
This helps to alleviate communication load across the swarm.

\edit{The transmission requirements for localization and assignment are $5.2$ kbps per neighbor and $0.064nd(n+1)$ kb per neighbor at the reassignment period, respectively.
For example, in our experiments with non-complete graphs, the theoretical bandwidth between each vehicle is approximately $9$ kbps.
Using a sparse graph, a reassignment period of \SI{30}{\second}, and mid-grade WiFi connectivity, the expected upper bound before channel saturation is 800 UAVs.
These numbers are supported by our simulation of 30 UAVs in a non-complete formation graph, where we measured 2161 kbps of communication between a UAV and its neighbors.
}




\begin{table}[t]
\centering
\ra{1.1}
\caption{
Hardware results.
Our distributed assignment algorithm successfully breaks gridlock and converges to every desired formation.
}
\label{tbl:hwresults}
\begin{tabular}{l@{\hskip3pt}r x{1cm}x{0.6cm}x{1cm}x{0.6cm} r}
\toprule
&& \multicolumn{2}{c}{Distance Traveled (m)} & \multicolumn{2}{c}{Convergence Time (s)} & Success \\ \cmidrule{3-7}
&& mean & std & mean & std & \\ \midrule
\multirow{2}{*}{\normalsize{NA}}
& nc &
0.8 & 0.5 & 15.5 & 9.2 & \SI{50}{\percent} \\
& c    &
0.9 & 1.0 & 11.3 & 3.0 & \SI{67}{\percent} \\ [0.2em]
\multirow{2}{*}{\normalsize{A}}
& nc &
1.4 & 0.8 & 14.3 & 8.7 & \SI{100}{\percent} \\
& c    &
0.8 & 0.6 & 10.1 & 4.8 & \SI{100}{\percent} \\
\bottomrule
\end{tabular}
\\ [0.2em]
\begin{flushleft}
{\scriptsize NA: no assignment\quad A: \edit{distributed} assignment (ours)} \\
{\scriptsize c: complete graph\quad nc: non-complete graph}
\end{flushleft}
\vskip-0.1in
\end{table}






