



\begin{proof}[\textbf{Proof of Proposition~\ref{prop:SDP}}]
Consider problem \eqref{eq:OptimZ}. 
The facts that $B\, M = 0$ and $R \in \br^{n\times (n-2)}$ is the orthogonal complement of $M$ imply that $B$ can be factored as ${B = - R \, Z \, R^\top}$, where $Z \in \bs^+_{n-2}$. Substituting $B$ with ${- R \, Z \, R^\top}$ in \eqref{eq:OptimZ} and simplifying yields 
\begin{equation} \label{eq:OptimSDP1}
\begin{aligned}
& \underset{Z \in \bs^+_{n-2}}{\text{maximize}}
& & \lambda_{\text{min}}(Z) & \\
& \text{subject to}
& & \left[R \, Z \, R^\top \right]_{ij} = 0 \;   & \forall_{i} \ \forall_{j \notin \mathcal{N}_i} \\
&&& \tr(Z) = \text{constant} & \\
\end{aligned}
\end{equation}
which reduces the dimension of the optimization variable from $n$ to $n-2$.
Note that the constraint $B\, M = 0$ in \eqref{eq:OptimZ} is automatically satisfied in \eqref{eq:OptimSDP1} as $R^\top M = 0$ by orthogonality.
The objective of \eqref{eq:OptimSDP1}, i.e., maximizing the smallest eigenvalue of the positive semidefinite matrix $Z$, can be expressed equivalently as finding $Z$ and the smallest $\gamma \geq 0$ such that $Z - \, \gamma^{-1} I$ remains positive semidefinite (this statement can be proved by diagonalizing $Z$). Hence, \eqref{eq:OptimSDP1} can be expressed as
\begin{equation} \label{eq:OptimSDP2}
\begin{aligned}
& \underset{Z \in \bs^+_{n-2}}{\text{minimize}} 
& & \gamma  \\
& \text{subject to}
& & \gamma \geq 0, \quad
Z - \, \gamma^{-1} I \succeq 0  \\
&&& \left[R \, Z \, R^\top \right]_{ij} = 0 \;   & \forall_{i} \ \forall_{j \notin \mathcal{N}_i} \\
&&& \tr(Z) = \text{constant} & \\
\end{aligned}
\end{equation}
Let $C \eqdef \begin{bsmallmatrix}
I & 0 \\
0 & 0
\end{bsmallmatrix}$ 
and 
$X \eqdef \begin{bsmallmatrix}
\gamma\, I && I \\
I && Z 
\end{bsmallmatrix}$,
where the size of the identity matrix $I$ is the same as $Z$. The Schur complement condition for positive semidefinite matrices states that ${X \succeq 0}$ if and only if $\gamma\, I \succeq 0$ and $Z - I \, (\gamma \, I)^{-1} I \succeq 0$. The latter implies that $\gamma \geq 0$ and $Z - \gamma^{-1} I \succeq 0$, which are the constraints in \eqref{eq:OptimSDP2}. Consequently, \eqref{eq:OptimSDP2} can be written concisely as 
\begin{equation} \label{eq:OptimSDP3}
\begin{aligned}
& \underset{X \in \bs^+_{2n-4} }{\text{minimize}} 
& & \langle C,\, X \rangle \\
& \text{subject to}
& & \sa(X) = b
\end{aligned}
\end{equation}
where $\langle C,\, X \rangle$ is the Frobenius inner product, and $\sa(X) = b$ captures the linear constraints on both the structure of $X$, i.e., the identity blocks and the last two constraints in~\eqref{eq:OptimSDP2}.
\end{proof}




