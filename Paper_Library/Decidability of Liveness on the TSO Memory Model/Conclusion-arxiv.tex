\section{Conclusion}
\label{sec:conclusion}

Liveness is an important property of programs, and using objects %libraries
with
incorrect liveness assumptions can cause problematic behaviors.
In this paper, we prove that lock-freedom, wait-freedom, deadlock-freedom and
starvation-freedom are undecidable on TSO for a bounded number of processes by reducing a known undecidable problem of lossy channel machines to checking liveness properties of specific libraries.
%specific libraries.
This library simulates the lossy channel machine $\textit{CM}'_{(A,B)}$ and is designed to contain at most two kinds of executions: If %all five
methods collaborate in a fair way and the lossy channel machine execution being simulated visits state $s_1$ infinitely many times, then the library executions violate all four liveness properties; otherwise, the library executions satisfy all four liveness properties.
Therefore, one library is sufficient for the undecidability proof of four liveness properties.
%Since the library used to show undecidability of lock-freedom (resp., wait-freedom) satisfies that all unfair executions satisfy both lock-freedom and deadlock-freedom (resp., both wait-freedom and starvation freedom), we use a same library to prove the undecidability of both lock-freedom and deadlock-freedom (resp., both wait-freedom and starvation-freedom).
%The possible executions of our library that violations of progress properties are in certain form and satisfies the fairness requirement.
Our undecidability proof reveals the intrinsic difference in liveness
verification between TSO and SC, resulting from the unbounded size of store buffers
in the TSO memory model.

Perhaps unexpectedly, we show that obstruction-freedom is decidable. %We prove that obstruction-freedom and bounded wait-freedom (given a bound) is decidable on TSO for a bounded number of processes by reducing it to a control state reachability problem of a lossy channel machine, which is known to be decidable.
%\redt{We reduce checking obstruction-freedom (resp., bounded wait-freedom with a bound) into a control state reachability problem of lossy channel machines (resp., state reachability problem of TSO concurrent systems), which is known decidable.}
Since each violation of obstruction-freedom eventually runs in isolation,
from some time point the violation, running on TSO, has the same behavior as on SC.
Therefore, checking whether a configuration contains a potential violation can be done by checking only this configuration itself, instead of checking all infinite executions from this configuration.
Checking obstruction-freedom is thus reduced to a known decidable reachability problem.
%Bounded wait-freedom is also checked by reducing to another known decidable reachability problem, since each violation must have a specific prefix that can be detected by remembering actions taken by each process.
\forget{Our decidability results for liveness are obtained by reducing liveness to some decidable reachability problem. For the case of obstruction-freedom, each violation is of infinite length.
However, such executions run in isolation eventually, which causes to eventually
run as on the SC memory model.
Thus, whether a configuration contains a potential violation is computable.
The case of bounded wait-freedom is easier since we only need to check a finite prefix to find the violation.
%The reason of decidability is that these two problem can be reduced into a finite trace problem on TSO.
We also prove that wait-freedom for libraries on TSO may not have a bound on steps for a bounded number of processes, while a wait-free library on SC must have bound on steps for a bounded number of processes.}
%We also prove that wait-freedom for libraries on TSO may not have a bound on steps for a bounded number of processes, while a wait-free library on SC must have bound on steps for a bounded number of processes. This reveals that wait-free libraries on TSO have subtle differences in terms of liveness when compared to SC.

%There are variants of liveness properties, such as $k$-bounded lock-freedom, bounded lock-freedom, $k$-bounded wait-freedom and bounded wait-freedom \cite{DBLP:conf/pldi/PetrankMS09}.
Other relaxed memory models, such as the memory models of ARM and POWER, %also
%use write buffers %\gpnote{$\leftarrow$ This is not necessarily true. There are behaviors similar to those
%  resulting from write-buffers, but they could be due to other artifacts.},
%and
are much weaker than TSO.
We conjecture that the undecidable liveness properties on TSO are still undecidable on ARM and POWER.
As future work, we would like to investigate the decidability of
obstruction-freedom %and bounded wait-freedom (given a bound)
on more relaxed memory models.
There are variants of liveness properties, such as $k$-bounded lock-freedom, bounded lock-freedom, $k$-bounded wait-freedom and bounded wait-freedom \cite{DBLP:conf/pldi/PetrankMS09}.
We would also like to investigate the decidability of bounded version of liveness properties on TSO and more relaxed memory models.

% \gpnote{$\leftarrow$ This doesn't sound very realistic to me.}

%%% Local Variables:
%%% mode: latex
%%% TeX-master: "CONCUR2021.tex"
%%% End: 