%\documentclass[a4paper,UKenglish]{lipics-v2016}
\documentclass{llncs}
%\usepackage{mathbbold}


%\newtheorem{theorem}{Theorem}[section]
%\newtheorem{conjecture}[theorem]{Conjecture}
%\newtheorem{corollary}[theorem]{Corollary}
%\newtheorem{proposition}[theorem]{Proposition}
%\newtheorem{lemma}[theorem]{Lemma}
%\newdef{definition}[theorem]{Definition}
%\newdef{remark}[theorem]{Remark}p
%\newdef{example}[theorem]{Example}



\usepackage{epsfig}
\usepackage{amsmath}
\usepackage{color}
%\usepackage{txfonts}
%\usepackage{pxfonts}
%\usepackage{mathabx}
%\usepackage{makeidx}  % allows for indexgeneration
\usepackage{verbatim}
\usepackage{url}
%\usepackage{hyperref}
\usepackage{paralist}
\usepackage{times}
\usepackage{ulem}    %(\emph{....} in the texfile) is underlined instead of italic
\normalem
%\usepackage{enumerate}
\usepackage{enumitem}
%\usepackage[square, comma, sort&compress, numbers]{natbib}

%added by Yi Lv%
%\usepackage[linesnumbered,ruled,procnumbered,noend]{algorithm2e}
%\usepackage[linesnumbered,ruled,procnumbered,vlined]{algorithm2e}
\usepackage[linesnumbered,ruled,procnumbered,noend]{algorithm2e}
\usepackage{stmaryrd}
%\usepackage{hyperref}
%added by Wang Chao%
%\usepackage{mathrsfs}
%\usepackage{extarrows}
\usepackage{thmtools}
\usepackage{thm-restate}

\newcommand{\gpnote}[1]{{\color{green!40!black} GP: #1}}


%\declaretheorem[name=Claim]{clm}
%\declaretheorem[name=Lemma]{lem}
%\declaretheorem[name=Theorem]{thm}


\DeclareSymbolFont{largesymbolsA}{U}{txexa}{m}{n}
\SetSymbolFont{largesymbolsA}{bold}{U}{txexa}{bx}{n}
\DeclareFontSubstitution{U}{txexa}{m}{n}
\DeclareMathSymbol{\bigsqcupplus}{\mathop}{largesymbolsA}{"02}

% "Import" \bigsqcupplus from txfonts without loading txfonts (since
% it changes the default font and replaces many symbols)

% Commands by Catuscia

%\DeclareMathOperator*{\minarg}{{\rm minarg}}

%Previous notation for the probabilistic choice
%\newcommand{\probsum}{\bigoplus}
%\newcommand{\smallprobsum}[1]{\mkern4mu{\textstyle\circ\mkern-15.5mu\sum_{#1}\:}}
%\newcommand{\bigprobsum}[1]{{\;\displaystyle\odot\mkern-21mu\sum_{#1}}}
%\newcommand{\probsum}[1]{\mathchoice{\bigprobsum{#1}}{\smallprobsum{#1}}{\smallprobsum{#1}}{\smallprobsum{#1}}}
%For the uniform distribution
%\newcommand{\usmallprobsum}[1]{\mkern4mu{\textstyle\circ\mkern-15.5mu\sum^\mathcal{U}_{#1}\:}}
%\newcommand{\ubigprobsum}[1]{{\;\displaystyle\odot\mkern-21mu\sum^\mathcal{U}_{#1}}}
%\newcommand{\uprobsum}[1]{\mathchoice{\ubigprobsum{#1}}{\usmallprobsum{#1}}{\usmallprobsum{#1}}{\usmallprobsum{#1}}}

%\newcommand{\nondsum}{\bigbox}
\newcommand{\nondsum}{\bigsqcupplus}
\newcommand{\probplus}[1]{\oplus_{#1}}
%\newcommand{\nondplus}{\square}
\newcommand{\bang}{!\,}
\newcommand{\nondplus}{{\textstyle\bigsqcupplus}}
\newcommand{\partmap}{\rightharpoonup}
\newcommand{\map}{\rightarrow}
%\newcommand{\exc}{\phi}
\newcommand{\exc}{\alpha}
\newcommand{\exec}{\mathit{exec}}
\newcommand{\execp}{\mathit{execp}}
\newcommand{\Act}{\mathit{Act}}
\newcommand{\Sec}{\mathit{Sec}}
\newcommand{\Obs}{\mathit{Obs}}
\newcommand{\etree}{\mathit{etree}}
\newcommand{\lstate}{\mathit{lst}}
\newcommand{\fstate}{\mathit{fst}}
%\newcommand{\STATE}{\mathcal{P}r}  $ original marked
%\newcommand{\STATE}{\mathcal{P}}   $ marked by me
%\newcommand{\st}{P}
\newcommand{\trans}{\mathcal{T}}
\newcommand{\Aut}{\mathcal{M}}
\newcommand{\init}{\mathit{init}}
\newcommand{\perr}{\mathcal{P}}

\newcommand{\calo}{\mathcal{O}}
\newcommand{\cals}{\mathcal{S}}
\newcommand{\sseq}{\vec s}
\newcommand{\oseq}{\vec o}
\newcommand{\ccsp}{CCS$_p$}

\newcommand{\bigfrac}[2]{\frac{\raisebox{1ex}{$#1$}}{\raisebox{-1.5ex}{$#2$}}}
\newcommand{\nondarr}[1]{\overset{#1}{\longrightarrow}}
\newcommand{\Nondarr}[1]{\overset{#1}{\Longrightarrow}}
\newcommand{\vectorArrow}[1]{\stackrel{\longrightarrow}{\mbox{#1}}}
\newcommand{\probarr}[1]{\overset{#1}{\dashrightarrow}}
\newcommand{\paral}{\,|\,}
\newcommand{\outp}[1]{\overline{#1}}
\renewcommand{\Pr}{{\rm Pr}}

%Commands by Chao Wang

%memory models%
\newcommand{\TSO}{\textrm{TSO}}
\newcommand{\PSO}{\textrm{PSO}}

%correctness conditions%
\newcommand{\lin}{\textrm{linearizability}}
\newcommand{\slin}{\textrm{static linearizability}}
\newcommand{\qlin}{\textrm{quasi linearizability}}
\newcommand{\TTlin}{\textrm{TSO-to-TSO linearizability}}

\newcommand{\pair}[2]{\langle #1 , #2 \rangle}% pairs
\newcommand{\setof}[2]{\{ \, #1 \mid #2 \, \}}% Sets
\newcommand{\set}[1]{\{ {#1}  \}  }
%\newcommand{\map}[3]{{#1} \colon {#2} \longmapsto {#3}} %functions
\newcommand{\den}[1]{[\![#1]\!]}% Denotation of
\newcommand{\mean}[1]{|\!|#1|\!|}
\newcommand{\forget}[1]{}

%%%%%%%%%%%%%GENERAL%%%%%%%%%%%%%%%%%%%%%%%%%%%
%%%%%%%%%%%%%%%%%%%%%%%%%%%%%%%%%%%%%%%%%%%%%%%%%%%%%%%%%%%

\newcommand{\itbox}[1]{{\it #1\/}}
\newcommand{\un}[1]{\uline{#1}}%\underline{#1}}
\newcommand{\ov}[1]{\overline{#1}}
\newcommand{\smallspace}{\vspace{10mm}}
\newcommand{\is}{\mbox{$\Longleftarrow\ $}}
\newcommand{\pright}[1]{\hfill{#1}}
\newcommand{\bnfor}{\;\;\mid\;\;}

\newcommand{\ar}[1]{\stackrel{\scriptstyle #1}{\longrightarrow}}

%%%%%%%%%italics in math mode
%%%%%%%%%%%%%%%%%%%%%%%%%%%%%%%%%%%%%%%%%

%\newcommand{\true}{{\it true}}
%\newcommand{\false}{{\it false}}
\newcommand{\calB}{{\cal B}}
\newcommand{\calF}{{\cal F}}
\newcommand{\calP}{{\cal P}}
\newcommand{\order}{{\cal O}}
\newcommand{\size}[1]{|#1|}

%other notations%
\newcommand{\LTS}{\textit{LTS}}
\newcommand{\lyedt}[1]{{\color{blue}#1}}
\newcommand{\bedt}[1]{{\color{blue}#1}}
\newcommand{\redt}[1]{{\color{red}#1}}
%\pagestyle{empty}
%\pagestyle{plain}


%\def\lastname{Xu,Palamidessi}

\title{Decidability of Liveness on the TSO Memory Model}




\forget{
\author[1]{Chao Wang}
\author[2]{Gustavo Petri}
\author[3]{Yi Lv}
\author[4]{Teng Long}
%\author[1]{Zhiming Liu}
\author[1,5]{Zhiming Liu}
\affil[1]{Southwest University, China}
\affil[2]{Arm Research}
\affil[3]{Institute of Software, Chinese Academy of Sciences}
\affil[4]{China University of Geosciences}
\affil[5]{Northwest Ploytechnical University, China}
}

\author {Chao Wang\inst{1} \and Gustavo Petri\inst{2} \and Yi Lv\inst{3} \and Teng Long \inst{4}
\and Zhiming Liu\inst{1,5}}
\institute{
  Southwest University, China
  \and
  Arm Research
  \and
  Institute of Software, Chinese Academy of Sciences
  \and
  China University of Geosciences
  \and
  Northwest Ploytechnical University, China
}



\begin{document}

\maketitle

%\vspace{-10pt}

\begin{abstract}
  An important property of concurrent objects is whether they support
  progress -- a special case of liveness -- guarantees, which ensure the
  termination of individual method calls under system fairness assumptions.
  %
  Liveness properties have been proposed for concurrent objects. Typical
  liveness properties include \emph{lock-freedom}, \emph{wait-freedom},
  \emph{deadlock-freedom}, \emph{starvation-freedom} and
  \emph{obstruction-freedom}.
  %
  It is known that the five liveness properties above are decidable on the
  Sequential Consistency (SC) memory model for a bounded number of processes.
  However, the problem of decidability of liveness for finite state concurrent
  programs running on relaxed memory models remains open.
  %
  In this paper we address this problem for the Total Store Order (TSO) memory
  model, as found in the x86 architecture.
  %
  We prove that lock-freedom, wait-freedom, deadlock-freedom and
  starvation-freedom are undecidable on TSO for a bounded number of processes,
  while obstruction-freedom is decidable.
  % The undecidability result is obtained by reducing a known undecidable
  % problem, post correspondence problem (PCP), into checking if a lossy channel
  % machine has a specific infinite executions, and further reduce the latter
  % problem into lock-freedom (resp., other progress properties) of a specific
  % concurrent data structure.
  % We prove that obstruction-freedom is decidable on TSO memory model for bounded
  % number of processes.
  % The decidability result is obtained by reducing obstruction-freedom into
  % checking whether there exists some finite execution that goes through a
  % specific configuration.

%{\color{red}We investigate the verification of \emph{$k$-bounded wait-freedom}, a bounded version of wait-freedom, and prove that, given a fixed $k$, the problem of checking $k$-bounded wait-freedom is decidable on TSO for bounded number of processes.}

  \forget{
  We investigate the verification of \emph{bounded wait-freedom},
  % and \emph{population-oblivious wait-freedom} are
  a bounded version of wait-freedom,
  % Bounded versions of wait-freedom, such as bounded wait-freedom and
  % population-oblivious wait-freedom, are proposed to {\color{red} TODO: what
  % is the purpose of bounded versions of liveness properties.} are also typical
  % progress properties.
  and prove that, given a fixed bound $k$, the problem of checking bounded
  wait-freedom is decidable on TSO for bounded number of processes.
  Since wait-freedom is undecidable on TSO, but for each bound $k$
    bounded wait-freedom is decidable, a conjecture is that there exists
    a wait-free concurrent library on TSO for which there is no bound on the
    number steps of one or more of its methods.
    %
    We demonstrate this conjecture by means of an example library.
    By contrast, we prove that such property does not hold on SC.
    }

  %Finally, we prove that given a wait-freedom library running on SC with a
  %bounded number of processes, in each of its execution, each method returns
  %within bounded number of commands. By contrast, we prove that such property
  %does not hold on TSO by means of a counterexample data structure.

%We consider three related decidability/existence problems of bounded versions of wait-freedom.
%First, we prove that, on TSO memory model, there exists a wait-freedom library that has no bound.
%Second, we prove that, there does not exists a computable function that takes a wait-freedom library, and returns the bound of the library (or $\infty$ if it has no bound).
%Third, we prove that, given a fixed number $k$, the problem of checking whether a library is $k$-bounded wait-freedom is decidable on TSO memory model.
%As a contrast, we prove that the answers is ``not exists'', ``there exists'' and decidable for above three problems on SC memory model, respectively.


%We prove that, given a fixed number $k$, the problem of checking whether a library is $k$-bounded wait-freedom is decidable on TSO memory model.
%One question related to bounded versions of wait-freedom is, if the processes number is fixed, does a wait-free library must have a bound?
%We find that the answer is ``no'' on TSO memory model, with one example which is the library used in our undecidability proof of wait-freedom.
%We also show that the answer is ``yes'' on SC memory model.


%On one hand, we prove given a fixed number $k$, the problem of checking whether a library is $k$-bounded wait-freedom is decidable on TSO memory model. One the other hand, we prove that the attempt to calculate bound for wait-freedom is undecidable even on sequential consistent (SC) memory model. Or we can say, there is no computable function that takes a library $\mathcal{L}$ and its running processes number $n$ as argument, and returns a number $k$ which means that each execution of $\mathcal{L}$ on $n$ processes on SC (and TSO) memory model satisfies $k$-bounded wait-freedom.

\forget{
Various progress properties have been proposed for concurrent data structures.
Typical progress properties include lock-freedom, wait-freedom, and obstruction-freedom.
We address the problem of liveness verification for finite-state concurrent program running on TSO memory model, which is the memory model of intel X86 architecture.

We prove that lock-freedom and wait-freedom is undecidable on TSO memory model for bounded number of processes. The undecidability result is obtained by reducing a known undecidable problem, post correspondence problem (PCP), into checking if a lossy channel machine has a specific infinite executions, and further reduce this problem into lock-freedom (resp., wait-freedom) of a specific concurrent data structure. The reduction is based on simulating lossy channel machine with two collaborative processes.

We also prove that obstruction-freedom is decidable on TSO memory model. The decidability result is obtained by reducing obstruction-freedom into whether a finite path contains a configuration of specific control state and memory valuation.}
\end{abstract}

\forget{
\noindent Keywords: weak memory model, $\textit{linearizability}$,
$\textit{TSO-to-TSO linearizability}$
}

%%% Local Variables:
%%% mode: latex
%%% TeX-master: "CAV2021.tex"
%%% End: 

\forget{
\noindent Keywords: weak memory model, $\textit{linearizability}$,
$\textit{TSO-to-TSO linearizability}$
}

\section{Introduction}
\label{sec:introduction}

A concurrent object provides a set of methods for client programs to access the object.
Given the complexity of writing efficient concurrent code, it is recommended to
use mature libraries of concurrent objects such as
\emph{java.util.concurrent} for Java and \emph{std::thread} for C++11.
The verification of these concurrent libraries is obviously important but intrinsically hard, since they
are highly optimized to avoid blocking -- thus exploiting more parallelism -- by
using optimistic concurrency in combination with atomic instructions like
compare-and-set.
%

%Moreover, different implementations of concurrent data structures offer different progress guarantees.
Various liveness properties (progress conditons) have been proposed for concurrent objects.
%concurrent data structures.
Lock-freedom, wait-freedom, %lock-freedom,
deadlock-freedom, starvation-freedom and obstruction-freedom \cite{DBLP:books/daglib/0020056,DBLP:conf/concur/LiangHFS13} are five typical liveness properties.
A liveness property describes conditions under which method calls are guaranteed to successfully complete in an execution.
Intuitively, clients of a wait-free library can expect each method call to
return in finite number of steps.
%
Clients using a lock-free library can expect that at any time, at least one
library method call will return after a sufficient number of steps, while it
is possible for methods on other processes to never return.
%
Deadlock-freedom and starvation-freedom require each fair execution to satisfy
lock-freedom and wait-freedom, respectively.
%
Clients using an obstruction-free library can expect each method call to
return in a finite number of steps when executed in isolation.
%
\forget{ For
  example, %wait-freedom requires that each call to a library method should return if it is scheduled for a sufficient number of steps, while
  lock-freedom requires that %some
  a call to a library method should return if the system executes for a
  sufficient number of steps, while wait-freedom requires that each call to a
  library method should return if it is scheduled for a sufficient number of
  steps.
%A liveness property guarantees that in certain executions, some method should return in finite number of steps.
%For example, wait-freedom requires that each call to a library method returns in a finite number of steps, while lock-freedom requires that %in
%at each point in time there is at least one method that should return in a finite number of steps.
%
%Deadlock-freedom and starvation-freedom can be obtained from lock-freedom and
%wait-freedom respectively by adding assumptions about the fair scheduling of
%executions.
%Obstruction-freedom guarantees that each method can return in a finite number of
%steps if it executes in isolation.
%There are also bounded versions of progress properties, such as
%
%{\color {red}$k$-bounded wait-freedom \cite{DBLP:conf/pldi/PetrankMS09} is a bounded version of wait-freedom. It requires that each call to a library method should return within $k$ steps.}
%Bounded wait-freedom \cite{DBLP:books/daglib/0020056} %and population-oblivious wait-freedom
%is a bounded version of wait-freedom. It requires that each call to a library method returns in a number of steps lower or equal to a fixed bound.
%
%They require progress properties as well as each method should return within
%bounded number of steps.
Deadlock-freedom and starvation-freedom assume fair schedulers. Deadlock-freedom
(resp., starvation-freedom) requires that each fair execution must satisfy
lock-freedom (resp., wait-freedom). Obstruction-freedom requires that calls to
library methods which execute in isolation should return in finite number of
steps.
}

% In some cases, a method can not block any other processes. In some cases, a method can block some processes but not all processes. In some cases, the whole systems may be blocked, while from any time point if we choose to execute only one process, that process always proceed. %In a bad implementation, the whole systems may be kept block, even when we select a process in the block configuration and let it run itself.
%Wait-freedom, lock-freedom and obstruction-freedom \cite{DBLP:books/daglib/0020056,DBLP:conf/concur/LiangHFS13} are three typical liveness properties.
%They correspond to above first three cases.
%For example, lock-freedom guarantees that in a infinite execution, there is always a process that will return in finite steps.

%Often times
It is often that programmers assume that all accesses to the shared memory are
performed instantaneously and atomically, which is guaranteed only by the
sequential consistency (SC) memory model \cite{DBLP:journals/tc/Lamport79}.
However, modern multiprocessors (e.g., x86 \cite{Intel2021}, ARM \cite{ARMv8}
), and programming languages (e.g., C/C++
\cite{DBLP:conf/popl/BattyOSSW11}, Java \cite{DBLP:conf/popl/MansonPA05}) do not
implement the SC memory model. Instead they provide {\em relaxed memory models},
which allow subtle behaviors due to hardware and compiler optimizations.
%For instance, in a multiprocessor system implementing the Total Store Order (TSO) memory model \cite{DBLP:conf/tphol/OwensSS09}, each processor is equipped with an FIFO store buffer.
For instance, in a multiprocessor system implementing the Total Store Order (TSO) memory model \cite{DBLP:conf/popl/AtigBBM10}, each processor is equipped with a FIFO store buffer.
In this paper we follow the %theoretical
TSO memory model of~\cite{DBLP:conf/popl/AtigBBM10}
(similarly
to~\cite{DBLP:conf/tphol/OwensSS09,DBLP:conf/esop/BouajjaniDM13,DBLP:conf/esop/BurckhardtGMY12}). Although in every realistic
multiprocessor system implementing the TSO memory model, the buffer is of bounded size, to describe
the semantics of \emph{any} TSO implementing system it is necessary to consider unbounded size FIFO
store buffers associated with each process, as in the semantics of~\cite{DBLP:conf/popl/AtigBBM10}. Otherwise, TSO implementations with larger store buffer will not be captured by this theoretical TSO memory model.
%associates a unbounded size FIFO store buffer with each process. This is clearly a safe over-approximation of any TSO system.
%{\color {red}of unbounded size}.
%{\color{orange} GP: This is certainly no true. There can be no unbounded resources in any real system. The model is unbounded, the real system isn't. This is the main criticism that the paper suffered in ESOP.}
Any write action %operation
performed by a processor is put into its local store buffer
first and can then be flushed into the main memory at any time.
%{\color {red}Although the FIFO store buffer is of bounded size in each detailed TSO chips, since the memory model of \cite{DBLP:conf/tphol/OwensSS09} intends to model all TSO chips, it associates a unbounded size FIFO store buffer with each process.}
% A read operation first attempts to check the buffer for item of the same
% memory location, and reads the main memory if failed.
%{\color {red}Since TSO memory model is a theoretical model and is used to model all multiprocessor system supporting this memory model, TSO memory model of \cite{DBLP:conf/tphol/OwensSS09} associates a unbounded size FIFO store buffer with each process. Many research on TSO assume such unbounded buffer \cite{DBLP:conf/popl/AtigBBM10}.}
Some libraries %data structure implementations
are optimized for relaxed memory models.
For example, some work-stealing queue implementations
\cite{DBLP:conf/oopsla/LeijenSB09,DBLP:conf/ppopp/MichaelVS09} are specifically
written to perform well on TSO.

%\lyedt{Even on SC, most decidability problems of concurrent system with unbounded data domains trivially become undecidable. Thus, we restrict data domains to be bounded.}
To address the problem of decidability of liveness properties, we remark that
concurrent systems with a bounded number of processes on SC can be expressed
as %fair discrete system,
finite state $\textit{labelled transition systems}$ (LTS).
% All above liveness properties can be expressed as LTL formulas, as shown in \cite{DBLP:conf/pldi/PetrankMS09}.
Lock-freedom, wait-freedom and obstruction-freedom can be expressed as LTL
formulas, as shown in \cite{DBLP:conf/pldi/PetrankMS09}. We show that
deadlock-freedom and starvation-freedom can %also
be expressed as CTL$^*$ %LTL
formulas.
%and all above liveness properties can be expressed as LTL formulas.
Given that LTL and CTL$^*$ model checking is decidable %in polynomial space
\cite{DBLP:reference/mc/2018},
it is known that lock-freedom, wait-freedom, deadlock-freedom,
starvation-freedom and obstruction-freedom %and {\color {red}$k$-bounded wait-freedom}
are decidable in this case.
However, their decidability problem on TSO memory model for a bounded number of
processes remains open. %, since concurrent system with a bounded number of processes on TSO is a infinite state LTS.
\forget{
{\color {red}
Liveness is an important property of programs, and {\color {blue} using
  libraries with unsuitable liveness property may cause unintended problematic
  behaviors.}
  } }
%not being able to reason
%about progress guarantees of programs running under relaxed memory is limits
%programmers from reasoning about liveness.

% Unaware of liveness may lead programmers to use unsuitable libraries and cause
% unintended trouble in some situation.
%Libraries with different progress guarantees can be used in different situations, for example, bounded wait-freedom libraries can be used in real-time situations where each method has a strong progress requirements {\color {red} TODO: add a reference}.
%However, the decidability of liveness of libraries on TSO memory model for bounded number of processes is still open.
%Unaware of liveness may lead programmers to use unsuitable libraries and cause unintended trouble in some situation.
%As a contrast, the problem for the five progress properties is decidable on SC memory model, since its operational is a finite state labelled transition system (LTS), and checking liveness can be done by model checking {\color {red} TODO: add a reference}.

In this paper, we study the decision problem of liveness properties on TSO.
% and answer related problems of bounded versions of wait-freedom on TSO memory
% model, and provide a decidability hierarchy for liveness on TSO memory model.
Our work covers the five typical liveness properties of
\cite{DBLP:books/daglib/0020056}, which are commonly used in practice.
Our main findings are:
%The result of our work is shown in the table below.
\begin{itemize}
%\setlength{\itemsep}{0.5pt}
\item[-] Lock-freedom, wait-freedom, deadlock-freedom and starvation-freedom are
  \emph{undecidable on TSO}, which reveals that the verification of liveness properties
  on TSO is intrinsically harder when compared to their verification on SC.
  %We prove the undecidability of lock-freedom, wait-freedom, deadlock-freedom and starvation-freedom with using the same library.
  %We use a specific library %,
  %which is ``immune'' to unfair scheduling,
  %to prove the undecidability of %both lock-freedom and deadlock-freedom.
  %lock-freedom, wait-freedom, deadlock-freedom and starvation-freedom.}
  %Similarly, we use another library to prove the undecidability of both wait-freedom and starvation-freedom.

\item[-] Obstruction-freedom is \emph{decidable on TSO}.

%\item[-] %Given a wait-freedom data structures running on SC memory model for bounded number of processes,
%  For wait-freedom on TSO with a bounded number of processes, there is no bound for steps of a method from its call to its return. This doesn't hold on SC.% {\color {red}\cite{DBLP:journals/jucs/BritM96}.}
  %Wait-freedom on SC with a bounded number of processes requires each method
  %to return within bounded number of commands.
  %This doesn't hold on TSO.
\end{itemize}

To the best of our knowledge, ours are the first decidability and
undecidability results for liveness properties %on TSO.
on relaxed memory models.
Let us %know
now present a sketch of the techniques used in the paper to justify our
findings.
\forget{
{\color{orange} GP: I think the text that follows is too detailed for an
  introduction: While the machines haven't been defined, the reader is required
  to imagine all these executions with $M_1$ and $M_2$ without even knowing what
  they represent. I would postpone this to following sections. Perhaps a couple
  of paragraphs summarizing what's here could work, but there is far too much now.}
}
%

\forget{
\begin{tabular}{|l|l|l|}
\hline
properties & decidability/answer on TSO & decidability/answer on SC \\
\hline
lock-freedom & $\times$ & $\checkmark$ {\color {red} TODO:add reference} \\
\hline
wait-freedom & $\times$ & $\checkmark$ {\color {red} TODO:add reference} \\
\hline
deadlock-freedom & $\times$ & $\checkmark$ {\color {red} TODO:add reference} \\
\hline
starvation-freedom & $\times$ & $\checkmark$ {\color {red} TODO:add reference} \\
\hline
obstruction-freedom & $\checkmark$ & $\checkmark$ {\color {red} TODO:add reference} \\
\hline
$k$-bounded wait-freedom & $\checkmark$ & $\checkmark$ \\
\hline
Does each wait-freedom library has \\ a bound ($n$ processes)? & $\times$ & $\checkmark$ \\
%\hline
%Is bound computable for wait-free libraries? & $\times$ & $\checkmark$ \\
\hline
\end{tabular}
}

\vspace{5pt}
\noindent \textbf{Undecidability Proof.}
%Let us explain our undecidability proof of lock-freedom, wait-freedom, deadlock-freedom and starvation-freedom on TSO memory model for bounded number of processes.
%Let us take wait-freedom and starvation-freedom as an example to explain our undecidable proof.
%We choose the formalization of starvation-freedom of \cite{DBLP:conf/concur/LiangHFS13}, which requires that in each fair execution, every method call can finish its execution and return in finite number of steps.
%Wait-freedom is similarly defined, except that it consider all executions instead of only fair executions.
%There are three challenges for proving undecidability of these progress properties. %wait-freedom and starvation-freedom.
%First, they all consider executions of infinite length.
%Second, some of them, such as starvation-freedom, consider only fair executions.
%Third, some of them, such as wait-freedom, requires that each method should finish in finite number of steps, irrespective of scheduling.
%Or we can say, a method can not block any other processes.
%We use the case of lock-freedom to explain how to solve the first challenge,
%and this approach can be used similarly for other three progress properties.
Abdulla \emph{et al.} \cite{DBLP:journals/iandc/AbdullaJ96a} reduce the cyclic
post correspondence problem (CPCP) \cite{DBLP:journals/acta/Ruohonen83}, a known
undecidable problem, into checking whether a specific lossy channel machine has
an infinite execution that visits a specific state infinitely often.
Our undecidability proof of %lock-freedom
lock-freedom and wait-freedom is obtained by reducing the checking of
the lossy channel machine problem into checking lock-freedom and wait-freedom for a specific library,
based on a close connection of concurrent programs on TSO and lossy
channel machines \cite{DBLP:conf/popl/AtigBBM10,DBLP:conf/atva/WangLW15}.
%More precisely,
%{\color {red}This lossy channel machine contains two phases: a guess phase and a check phase.} %, a guess phase
%that guesses a solution of CPCP and inserts it into channel, and a check phase
%that repeatedly checks if the channel content contains a solution of CPCP.


We %then
generate a library template that can be instantiated as a specific library for each instance of CPCP. %This library contains five methods $M_1$, $M_2$, $M_3$, $M_4$ and $M_5$. The collaboration of $M_1$ and $M_2$ simulates one channel of the lossy channel machine, while the collaboration of $M_3$ and $M_4$ simulates the other channel. Thus, the executions of this library simulate the executions of the lossy channel machine.
The collaboration between methods simulates lossy channel machine transitions.
Each execution of the lossy channel machine contains (at most) two
phases. %: a \emph{guess phase} and a \emph{check phase}.
Each \emph{accepting} infinite execution of the lossy channel machine loops
infinitely in %check phase.
the second phase. Thus, we make each method of the library to work differently
depending on the phase.
%
Our library has the following features: if an infinite library execution simulates an accepting infinite
execution of the lossy channel machine, then it violates lock-freedom, and thus,
it also violates wait-freedom; if an infinite library execution does not simulate an
accepting infinite execution of the lossy channel machine, then it satisfies
wait-freedom, and thus, it also satisfies lock-freedom. This is because any
execution that satisfies wait-freedom also satisfies lock-freedom.
Therefore, we reduce checking whether the lossy channel machine has an
accepting infinite execution, or more precisely CPCP, into checking lock-freedom
and wait-freedom of the library.


\forget{
We then generate a library template that can be instantiated as a specific library for each CPCP.
This library contains two methods $M_1$ and $M_2$, and the collaboration of $M_1$ and $M_2$ simulates the lossy channel machine transitions.
Each execution of the lossy channel machine contains (at most) two phases: a \emph{guess phase} and a \emph{check phase}.
In a library execution that successfully simulates a lossy channel machine transition, there should be two processes. Process $1$ (resp., $2$) runs $M_1$ (resp., $M_2$).
$M_1$ does not return until the simulation procedure ends, while
$M_2$ returns several times (to simulate several receive operations) during the guess phase, but does not return until the simulation procedure ends in the check phase.
%$M_2$ returns several times during the guess phase, but does not return until the simulation procedure ends in the check phase.
%{\color{red} In the guess phase, the simulation of channel receive operations requires at least that one method $M_2$ returns.}
%{\color{orange} GP: I tried to fix this, but the sentence misses something}
}

\forget{
Each accepting infinite execution of the lossy channel machine loops infinitely in check phase. %Therefore, the
The lossy channel machine has an accepting infinite execution
$t_1$, if and only if the library has an accepting infinite execution
$t_2$ which simulates $t_1$. Hence, $t_2$ violates lock-freedom since $M_1$
never returns and the $M_2$ invoked in the check phase does not return either.
On the other hand, if a library execution fails to simulate lossy channel
machine transitions, then $M_1$ and $M_2$ will go to trap state.
% {and a flag will be set.}
From that point onward,  $M_1$ and $M_2$ both return trivially, and hence such executions
satisfy lock-freedom.
If a library execution finishes simulating a finite channel machine execution,
% {\color{red} GP: removing this as it is TMI at this point }such flag will also be set and the execution thus
it immediately satisfies lock-freedom.
}
\forget{
{\color {blue}This library contains two methods $M_1$ and $M_2$, and the collaboration of $M_1$ and $M_2$ simulates the lossy channel machine transitions.
In a library execution that successfully simulates lossy channel machine transitions, $M_1$ runs on process $P_1$, and it does not return until the lossy channel machine execution ends, while $M_2$ runs on process $P_2$, and it behaviors differently in two phase.
In guess phase, to simulate one lossy channel transition, $M_2$ returns several times.
In check phase, $M_2$ does not return until the lossy channel machine execution ends.
Therefore, the lossy channel machine has an infinite execution $t_1$ that visits a specific state $s_1$ infinite often, if and only if the library has an infinite length execution $t_2$ which simulates $t_1$. In $t_2$, $M_1$ never return, and $M_2$ does not return after it goes to check phase, and thus, $t_2$ violates lock-freedom.
On the other hand, if a library execution fails to simulate lossy channel machine transitions, then $M_1$ and $M_2$ will go to trap state, and from then on $M_1$ and $M_2$ becomes a method that returns trivially, and such executions satisfy lock-freedom.}
}
%
% We use a result of Abdulla \emph{et al.}
% \cite{DBLP:journals/iandc/AbdullaJ96a}, which reduces the problem of whether a
% specific cyclic post correspondence problem (CPCP)
% \cite{DBLP:journals/acta/Ruohonen83} %of sequences $A$ and $B$
% has a solution into checking whether a specific lossy channel
% machine %$M_{A,B}$
% has an infinite execution that visits a specific state infinite often. Then,
% we reduce the latter problem into whether a specific
% library %$\mathcal{L}_{(A,B)}$
% satisfies lock-freedom, based on a close connection of concurrent programs on
% TSO and lossy channel machines. More precisely, such lossy channel machine
% contains two phase, a guess phase that guess solution of CPCP and insert them
% into channel, and a check phase that repeatedly checks if the channel contains
% a solution of CPCP. Although the channel is lossy, since the channel content
% is finite in the beginning of the check phase, if the channel is checked
% correct for infinite times, the channel content contains a solution of PCP.
%
%We generate a library template that can be instantiated as a specific library
%for each CPCP.
%This library contains two methods $M_1$ and $M_2$, designed to run repeatedly on
%two process $P_1$ and $P_2$, respectively.
%The channel is simulated by requiring $M_1$ (resp., $M_2$) to read from a memory
%location $y$ (resp., $x$) and write to memory location $x$ (resp., $y$), and the
%loss of channel content is simulated by the case that $M_1$ (resp., $M_2$) may
%ignore updates made by $M_1$ (resp., $M_2$).
%In this way, we simulate the executions of the lossy channel machine.
%To deal with lock-freedom,
%the two phases of the lossy channel are simulated in
%a different way.
%In simulating the guess phase, method $M_2$ returns after it reads content from
%$x$ and writes it to $y$,
% as long as simulating one lossy channel machine transition,
%while in simulating the check phase, $M_2$ does not return unless the simulation
%procedure of lossy channel machine fails.
%if the check is still correct in current round.
%This reduces the existence of specific infinite execution of the lossy channel
%machine of \cite{DBLP:journals/iandc/AbdullaJ96a} to the existence of an infinite
%length execution of the library where $M_2$ never returns, a lock-freedom
%violation.
%Since CPCP is known undecidable, our reduction of lossy channel machine (generated from CPCP) into lock-freedom lock-freedom reflects that lock-freedom is undecidable on TSO.
%This reduce the existence of infinite execution of that lossy channel machine (represents a solution of PCP) to existence of a lock-freedom violation, which also consider infinite executions.

Perhaps surprisingly, the same library can be used to show the
undecidability of deadlock-freedom (resp., starvation-freedom), which requires
that each infinite fair execution satisfies lock-freedom (resp.,
wait-freedom). This is because whenever a library execution simulates an
accepting infinite execution of the lossy channel machine, we require %$M_1$, $M_2$, $M_3$, $M_4$ and $M_5$
library methods to collaborate and work alternatingly, and thus, such
library execution must be fair and violates deadlock-freedom, and thus,
violates starvation-freedom. Therefore, checking the existence of accepting
infinite executions of the lossy channel machines is reduced into checking
violations of deadlock-freedom and starvation-freedom.


\forget{
Perhaps surprisingly, the same library can be used to show the undecidability of
deadlock-freedom, which requires that each fair execution satisfies lock-freedom.
The reason is that whenever a library execution simulates an infinite lossy
channel machine execution, we require $M_1$ and $M_2$ to collaborate and work
alternatingly, and thus, such library execution must be fair. Therefore, the
accepting infinite executions of the library (if any) are fair and violate deadlock-freedom, and checking their existence is reduced to checking deadlock-freedom.
}


\forget{
When compared to lock-freedom, deadlock-freedom additionally requires fair
sche-duling.
Perhaps surprisingly, the same library can be used to show the undecidability of
deadlock-freedom.
It is known that lock-freedom implies deadlock-freedom, and to show that deadlock-freedom of this library implies lock-freedom, we need to show that every lock-freedom violation must be fair. Each lock-freedom violation needs to simulate an infinite execution of the lossy channel machine. %, and to simulate lossy channel machine transitions,
  %we require $M_1$ (resp., $M_2$) to read from a memory location $y$ (resp., $x$) and write to memory location $x$ (resp., $y$).
  %To simulate one lossy channel machine transition,
  To simulate the transitions of this lossy channel machine, %we bind $M_1$ and $M_2$ in process $P_1$ and $P_2$, respectively, and we require $M_1$ and $M_2$ %to execute alternately, and
  %repeatedly exchange information of buffer content. %by reading from each other's buffer and then writing it to buffer.
  we require $M_1$ (resp., $M_2$) to repeatedly read from a memory location $y$ (resp., $x$) and write to memory location $x$ (resp., $y$). In this procedure, the processes of $M_1$ and $M_2$ execute alternatingly, and such procedure must be fair.
%The reason is that this library simulates the lossy channel machine transitions
%only when $M_1$ and $M_2$ collaborate in a certain manner, and such
%manner satisfies fair scheduling.
%For any other cases, the library will finish simulating, and make every method
%trivially return, and the executions in such cases satisfy both lock-freedom and
%wait-freedom.
}

%The second and third challenge is solved by the construction of our library.
%Our library simulates lossy channel machine only if when $M_1$ and $M_2$ collaborate in a certain manner.
%For any other scheduling, the library will finish simulating, and change the concurrent system into a situation that only has wait-freedom executions: From that time point, every method will trivially return.
%Any executions that violates the conditions of second or third challenges do not satisfy this manner, and they will not influence progress properties checking.
%Or we can say, our construction is unaware of the second and third challenges.

\forget{
Wait-freedom imposes stronger requirements than lock-freedom. %and requires that each method returns in a finite number of steps.
For the library above, in most cases, a library execution that satisfies
lock-freedom implies that it also satisfies wait-freedom. The only exception are library executions that simulate an infinite execution of the lossy channel machine that always loops in the guess phase, and such executions satisfy lock-freedom but violate wait-freedom.
To make executions in this case satisfy wait-freedom, we modify $M_1$ by making it
return as long as it simulates one lossy channel machine transition.
Then, when simulating a lossy channel machine execution, an
execution of the new library satisfies wait-freedom, if and only if an
execution of the previous library satisfies lock-freedom.
Hence, we can prove the undecidability of wait-freedom and starvation-freedom following the argument used to prove lock-freedom and deadlock-freedom.
}


\forget{
Wait-freedom requires that each method returns in a finite number of steps.
In the library above, the library execution that simulates an
  infinite execution of the lossy channel machine that always loops in
  the guess phase satisfies lock-freedom but violates wait-freedom.
  To make such an execution satisfy wait-freedom,
%Method $M_1$ above does not return if the library execution simulates an infinite length execution of lossy channel machine that always loop in guess phase.
%Such execution should still be wait-freedom.
%To deal with this situation,
  {we modify $M_1$ by making it
%while method $M_1$ above does not return unless the simulation procedure of
%lossy channel machine fails.
%To deal with wait-freedom, such that $M_1$
return as long as it simulates one lossy channel machine transition.
}
With this new library we can prove the undecidability of wait-freedom and
starvation-freedom following the argument used to prove
  lock-freedom and deadlock-freedom.\\
}


\vspace{5pt}
\noindent \textbf{Decidability Proof.}
We introduce a notion called \emph{blocking pair}, coupling a process
control state and a memory valuation, and capturing a time point from which
we can generate an infinite execution on the SC memory model, for
which eventually one process runs in isolation and does not perform any
return. We reduce checking obstruction-freedom into %finite number of
the state reachability problem, a known decidable problem
\cite{DBLP:conf/popl/AtigBBM10}, for configurations that ``contain a blocking
pair'' and have an empty buffer for each process.
There are two difficulties here:
\begin{inparaitem}
\item[] firstly, the TSO concurrent systems of \cite{DBLP:conf/popl/AtigBBM10} do
  not use libraries; and
\item[] secondly, the state reachability problem requires the buffer of each
  process to be empty for the destination configuration, while such
  configuration may not exist in a obstruction-freedom violation.
\end{inparaitem}

The first difficulty is addressed by making each process repeatedly call
an arbitrary method with an arbitrary arguments for an arbitrarily number of
times, while transforming each call and return action into internal actions.
To solve the second difficulty, we show that each obstruction-freedom
violation has a prefix reaching a configuration that ``contains a blocking
pair''. By discarding specific actions of the prefix execution and forcing
some flush actions to happen, we obtain another prefix execution reaching a
configuration that both ``contains a blocking pair'' and has an empty buffer
for each process.


\forget{
For this proof we introduce a notion called \emph{blocking pair}, coupling a process
control state and a memory valuation, and capturing a time point from which
we can generate an infinite execution on the SC memory model, for
which eventually one process runs in isolation and does not perform any
return. With this notion, we reduce checking obstruction-freedom into a
reachability problem of configurations that ``contain a blocking pair''.
%an execution such that this process, even when executing in isolation, can't ever return on SC.



% Let us explain our decidability proof of obstruction-freedom. On SC memory
% model, a control state $q$ and a memory valuation $d$ is called a blocking
% pair, if some executions (of one process) from this control state and memory
% valuation can never return.

Atig \emph{et al.} \cite{DBLP:conf/popl/AtigBBM10} prove that the
state reachability problem of TSO concurrent systems is decidable. They
consider systems where several processes run concurrently on TSO, however they
do no consider the case of processes accessing library methods. Our
decidability result of obstruction-freedom is obtained by reducing checking
blocking pairs into the state reachability problem of
\cite{DBLP:conf/popl/AtigBBM10}. There are two difficulties here:
\begin{inparaitem}
\item[] firstly, the TSO concurrent systems of \cite{DBLP:conf/popl/AtigBBM10} do
  not use libraries; and
\item[] secondly, checking blocking pairs requires atomically reading the whole
  memory valuation, while the commands can only atomically read one memory
  location.
\end{inparaitem}


The first difficulty is addressed by making each process repeatedly
call an arbitrary method with an arbitrary argument for arbitrarily many
times, while transforming each call and return action into internal actions.
To solve the second difficulty, a process can non-deterministically start to
read the memory valuation atomically in two phases. In the first phase, it
marks each memory location with special values. In the second phase, it
ensures the value of each memory location has not been overwritten by other
processes or memory model actions, and then checks the value of memory
locations. When a process finishes this check procedure, the result is stored in
a %new
memory location %$result$
and the whole system stops. Therefore, checking
blocking pairs is reduced into checking if a specific configuration %with $result=1$
is
reachable.
}

%\begin{inparaenum}[(1)]

\forget{
We slightly extend the TSO semantics to capture the first
  occurrence of a blocking pair. To that end, we need to be able to read %the control states of the processes,
  the current memory valuation and the current store buffer atomically.
Our approach for checking obstruction-freedom on TSO for a bounded number of
processes is divided into two steps:
\begin{inparaenum}[(1)]
\item in the first step, {we reduce checking obstruction-freedom on the original TSO semantics into checking
  if some configuration that ``contains a blocking pair'' is reachable in the extended TSO semantics.}
  %, such that the control state of some process and the memory valuation %of this configuration obtained by updating the current memory valuation with this process's buffer content is a blocking pair.
  %Note that %{\color {red}there are only a finite number of candidate blocking pairs, and the set of blocking pairs are computable by reducing into model checking problem of finite state LTS}.
  %there are only a finite number of blocking pairs.
  Since the number of blocking pairs is finite, we observe that there are
    only a finite number of finite execution problems of this form.
\item in the second step, we reduce the finite execution problem above into a control
state reachability problem of a lossy channel machine, which is known decidable.
}

\forget{
{A lossy channel machine \redt{$\textit{CM}_i$} is then constructed, which simulates the behavior
of process $P_i$ in the extended TSO semantics.}
\redt{$\textit{CM}_i$} contains only one channel to store the pending write actions according to
the total store orders {in the extended TSO semantics.} %under the extended %original
%concurrent system.
%Importantly, call and return actions are considered as internal transitions.
Then, the finite execution problem can be reduced to a control state
reachability problem of the product of lossy channel %the
machines \redt{$\textit{CM}_1^w,\ldots,\textit{CM}
_n^w$}.
Each \redt{$\textit{CM}_i^w$} is obtained from \redt{$\textit{CM}_i$} by replacing all of its transitions except
for write and \textit{cas} with internal transitions.
We require that each write in a channel contains a run-time snapshot of
the memory, while always keeping bounded the amount of information that needs to
be stored as in a perfect channel.
}

\forget{
With these specialized lossy channels, missing some intermediate channel
contents would not break the reachability between control states under perfect
channels. The reason is that when an item is flushed, its whole snapshot is used to update the memory valuation of \redt{$\textit{CM}_i$}. The result of this flush action will not be influenced if several previous items are missing.
}
%\end{inparaenum}



%The $k$-bounded wait-freedom {\color {red} TODO: add a reference} condition
%requires wait-freedom and that each method returns within $k$ steps.
%Let us explain our verification of bounded versions wait-freedom.
%We investigate the decidability of $k$-bounded wait-freedom {\color {red} TODO: add a reference}, which extends wait-freedom by requiring each method terminates in $k$ steps of the calling process, irrelative with the scheduler.

\forget{
Given a bound %value
$k$, we reduce checking bounded wait-freedom on TSO into the state reachability problem of the concurrent system of \cite{DBLP:conf/popl/AtigBBM10}.
Since the concurrent system of \cite{DBLP:conf/popl/AtigBBM10} does not consider call and return actions, in the reduced concurrent system we consider call and return actions as internal %transitions.
actions.
To capture bounded wait-freedom violation, for each process we record if some method of this process has executed more than $k$ steps without return. Since the state reachability problem of the concurrent system of \cite{DBLP:conf/popl/AtigBBM10} is decidable, bounded wait-freedom is decidable.
}

\forget{
Given a bound %value
$k$, our verification of bounded wait-freedom is divided into two steps.
We reduce checking bounded wait-freedom on TSO into checking whether there
exists a finite execution, on which a method runs $k+1$ steps without return.
We then reduce the latter problem into a control state reachability problem of a
lossy channel machine, similarly to that of obstruction-freedom.\\
}

\forget{
\noindent \textbf{Bound in Wait-Freedom.}
Since wait-freedom is undecidable on TSO while bounded wait-freedom is decidable
on TSO for each bound $k$, one direct conjecture is that for some wait-free
library on TSO with a bounded number of processes, there
is no bound on the numbers of steps of a method from its call to its return. We demonstrate this conjecture by providing an example
library of a concurrent data structure, which is the library used in undecidability proof of wait-freedom on TSO.

In the guess phase the lossy channel machine guesses a solution of CPCP by nondeterministically inserting elements into channel,
and in the check phase the lossy channel machine repeatedly checks whether the current channel content is a solution to CPCP.
Given a bound $k$, we can generate a specific CPCP that has no solution, and its
lossy channel machine contains an execution of the following form: in the guess
phase, the lossy channel machine guesses a long enough candidate solution. Then,
in the check phase, the first round of check is passed by losing channel
content, while the second round of check fails. Additionally, we require that at
the end of the first round of check, there are more than $k$ elements in the channels.
We can generate a library for this lossy channel machine, which is wait-free
since this CPCP has no solution. Since TSO uses unbounded buffers, there is a
library execution for the lossy channel machine above, and %the first $M_1$ of the check phase will run for more than $k$ steps.
the invocation of $M_1$ that simulates the first lossy channel machine
transition in the check phase will run for more than $k$ steps.
Such library execution violates bounded wait-freedom.
By contrast, we prove that each wait-free library on SC has a bound for bounded number of processes.
}


\forget{The reason is as follows:
during the guess phase the lossy channel machine guesses a solution of CPCP and inserts it into channel,
and in the check phase the lossy channel machine repeatedly checks if the channel content contains a solution to CPCP.
Given a library for CPCP that has no solution, and given a bound value $k$, we could guess a long enough candidate solution.
Then, in the check phase we could make the check pass for one round by losing
some ``channel content'', and fail in the next round, while in the end of the first round, there should be more than $k$ elements in channel.
The library for such CPCP is wait-free, and the first $M_1$ of the check phase could run for more than $k$ steps.
We should note that we assume the TSO buffer size to be unbounded.
By contrast, we prove that each wait-free library on SC has a bound for bounded number of processes. \\
}


%Assume that some people is given a wait-freedom library and want to obtain the detailed bound of it.
%With the decidability result of $k$-bounded wait-freedom, one seems-work approach of computing bound for wait-freedom libraries is that, we could tests $1,\ldots$, until some test returns true.
%However, we prove that this is unrealistic on TSO, and there is no computable function that returns the bound of a library and process number of concurrent system.


%We starts from a known undecidable problem, post correspondence problem (PCP) \cite{Post1946A}.
%Given a Turing machine and its input, the halting problem can be reduced into a specific PCP problem, which uses intervals embraced by $\sharp$ to represent one snapshot of Turing machine computing.
%Similar to that of CPCP, we reduce this specific PCP into checking whether a specific lossy channel machine has an infinite execution that visits a specific state infinite often, and generate a specific library for this specific lossy channel machine.
%Here are the differences: in the new library,


\forget{
Two important properties of concurrent data structures are correctness and liveness.
The correctness property concern if the behaviors of a concurrent object conform to a better understandable sequential specification in some manner.
$\textit{Linearizability}$ \cite{Herlihy:1990} is accepted as a \emph{de facto} correctness condition %for a concurrent library with respect to its sequential specification
on the sequential consistency (SC) memory model \cite{Lamport:1979}.
%Intuitively, linearizability asks whether every individual operation appears to take place instantaneously at some point between its invocation and its return.
The liveness property concern the condition under which method calls are guaranteed to successfully complete in an execution.
Wait-freedom, lock-freedom and obstruction-freedom \cite{Herilihy:2008,DBLP:Liang2013concur}are three typical liveness properties.
For example, lock-freedom guarantees that in a infinite execution, there is always a process that will return in finite steps.

Programmers usually assume that all accesses to the shared memory are performed instantaneously and atomically, which is guaranteed only by the SC memory model.
However, modern multiprocessors (e.g., x86 \cite{Owens:2009}, POWER \cite{Sarkar:2011}) and programming languages (e.g., C/C++ \cite{Batty:2011}, Java \cite{Manson:2005}) do not comply with the SC memory model.
As a matter of fact, they provide {\em relaxed memory models}, which allow subtle behaviors due to hardware or compiler optimization.
For instance, in a multiprocessor system implementing the total store order (TSO) memory model \cite{Owens:2009}, each processor is equipped with an FIFO store buffer.
Any write operation performed by a processor is put into its local store buffer first and can then be flushed into the main memory at any time.
%A read operation first attempts to check the buffer for item of the same memory location, and reads the main memory if failed.

Linearizability has been extensively studied for decidability \cite{Alur:1996,Bouajjani:2013,DBLP:Jad2015} and verification approach \cite{DBLP:Liang2010POPL,DBLP:Liang2013PLDI} on SC memory model, as well as variants of linearizability on relaxed memory model \cite{Sebastian:2012,Alexey:2012,Mark:2013} and their verifications \cite{Wang:2015,Wang:2015a,Derrick2014}. It is known that linearizability is decidable on SC for bounded number of processes \cite{Alur:1996}, but undecidable on TSO for bounded number of processes \cite{Wang:2015}.
The liveness are decidable on SC for bounded number of processes, however, the decidability of liveness on relaxed memory model remains open.
}





% \smallskip
\vspace{5pt}
\noindent {\bf Related work.}
%{\color {red}\noindent Lossy counter machine \cite{DBLP:journals/tcs/Mayr03} and lossy channel machine \cite{DBLP:journals/iandc/AbdullaJ96a} are used to prove decidability and undecidability result for concurrent systems with unreliable channels or store buffers \cite{DBLP:conf/popl/AtigBBM10,DBLP:conf/atva/WangLW15,DBLP:conf/sofsem/WangLW16}.}
\noindent There are several works on the decidability of verification on TSO. Atig \emph{et al.} \cite{DBLP:conf/popl/AtigBBM10} prove that the state reachability problem is decidable on TSO while the repeated state reachability problem is undecidable on TSO. Bouajjani \emph{et al.} \cite{DBLP:conf/esop/BouajjaniDM13} prove that robustness is decidable on TSO.
%Wang \emph{et al.}
Our previous work \cite{DBLP:conf/atva/WangLW15} proves that TSO-to-TSO linearizability \cite{DBLP:conf/esop/BurckhardtGMY12}, a correctness condition of concurrent libraries on TSO, is undecidable on TSO.
%Wang \emph{et al.}
Our previous work \cite{DBLP:conf/sofsem/WangLW16} proves that a bounded version of TSO-to-SC linearizability
\cite{DBLP:conf/wdag/GotsmanMY12} is decidable on TSO.
None of these works address the decidability of concurrent library liveness on TSO.
%\label{sec:related work}

Our approach for simulating executions of lossy channel machines with libraries is partly inspired by %Wang \emph{et al.} \cite{DBLP:conf/atva/WangLW15} and
Atig \emph{et al.} \cite{DBLP:conf/popl/AtigBBM10}. %and %our previous work
%Wang \emph{et al.} \cite{DBLP:conf/atva/WangLW15}.
%However, Arig \emph{et al.} use such idea to reduce the reachability problem of lossy channel machine to control state reachability problem on TSO memory model.
%Their work consider only finite executions, and their TSO concurrent programs have no call or return actions.
%Our previous work simulates an execution of lossy channel machine with a history of call and return actions on TSO memory model.
%Our previous work consider only finite executions, and it is mainly used to output transition labels, instead of concerning liveness.
However, Atig \emph{et al.} do not consider libraries, and their concurrent
programs do not have call or return actions. Our library needs to ensure that,
in each infinite library execution simulating an infinite execution of the lossy
channel machine, methods are ``fixed to process'', in other words, the same
method must run on the same process. The TSO concurrent systems of
\cite{DBLP:conf/popl/AtigBBM10} %\redt{state the task}
do not need to ``fix methods to processes'' since they record the control states
and transitions of each process.
%for each process and thus do not need to ``fix methods to processes''. %, while in our paper we need to fix method to process when simulating an accepting infinite execution of the lossy channel machine.
Both Atig \emph{et al.} and our work store the lossy channel content in
the store buffer. However, when simulating one lossy channel machine
transition, methods of Atig \emph{et al.} only require to do one read or write
action, while methods of our paper require to read the whole channel content.
It appears that we can not ``fix methods to processes'' with methods in the
style of \cite{DBLP:conf/popl/AtigBBM10}, unless we use specific command to
directly obtain the process identifier. %The reason is that methods in the style of \cite{DBLP:conf/popl/AtigBBM10} permit items of one memory location to be stored in store buffers of multiple processes.

Our previous work \cite{DBLP:conf/atva/WangLW15}
% Wang \emph{et al.}
considers safety properties of libraries. %, and %they
Both \cite{DBLP:conf/atva/WangLW15} and this paper %uses
use the collaboration of two methods to simulate one lossy channel machine transition. %Each method read updates of one memory location from the other process's buffer, and write them into its own buffer as items of another memory location.
% Method $1$ reads updates from $x$ and writes to $y$, while method $2$ reads updates from $y$ and writes to $x$.
Our idea for simulating lossy channel machine transitions with libraries extends that of \cite{DBLP:conf/atva/WangLW15}, since each library constructed using the latter contains executions violating liveness, which makes such libraries not suitable for their reduction to liveness.
\forget{
The methods in \cite{DBLP:conf/atva/WangLW15}
% This approach
induce infinite loops when methods do not observe some key update, and this may occur in simulating each lossy channel machine transition. Although this does not influence safety, when simulating the lossy channel machine of \cite{DBLP:journals/iandc/AbdullaJ96a}, this introduces ``false negatives'' to four liveness properties. In our paper, %we require each method to exhaust the updates of one memory location and do not directly write them back as items of the other memory location, but via a longer procedure, which contains four times of reading updates and writing into buffer, instead of two times in \cite{DBLP:conf/atva/WangLW15}.
we modify the methods of \cite{DBLP:conf/atva/WangLW15} by exhausting the
buffered items in each method. This eliminates the ``false negatives''. %We also need to prove that in each infinite library execution simulating an accepting infinite execution of the lossy channel machine, each method is fixed to some process.
}
The library of %Wang \emph{et al.}
\cite{DBLP:conf/atva/WangLW15} contains a method that never returns and thus, do not need to consider ``fixing methods to processes''.



\forget{
Wang \emph{et al.} simulate lossy channel machines with libraries in a slightly different manner, and they consider safety properties and not liveness properties.
Atig \emph{et al.} do not consider libraries, and their concurrent
programs do not have call or return actions. %Atig \emph{et al.} uses
They use a TSO
concurrent program with two processes to simulate a lossy
channel machine with one channel, while we use libraries running on five processes to simulate
a lossy channel machine with two channels. Interestingly, it appears that
one can not prove undecidability of liveness on TSO using lossy channel machine with one channel. We can transform a lossy channel machine $\textit{CM}$ with
multiple channels into a lossy channel machine $\textit{CM}'$ with one channel, %by using delimiters,
and use several transitions of $\textit{CM}'$ to simulate one transition of
$\textit{CM}$. Such $\textit{CM}$ and $\textit{CM}'$ are even trace equivalent. However, it seems that $\textit{CM}$ and $\textit{CM}'$ can be distinguished by liveness properties.
$\textit{CM}'$ introduces additional infinite executions, say $t$, which repeatedly try to simulate one transition of $\textit{CM}$ but never succeed. When $\textit{CM}$ is the lossy channel machine with two channels in \cite{DBLP:journals/iandc/AbdullaJ96a}, %such additional infinite execution $t$ of $\textit{CM}'$
$t$ introduces ``false negatives'' to liveness checking, since $t$ does not
simulate an accepting infinite execution of $\textit{CM}$, and the library
executions that simulate $t$ violate four of the liveness properties we consider. %lock-freedom, wait-freedom, deadlock-freedom and starvation-freedom.
% $\textit{CM}'$ introduces ``false negatives'' $t$, which are infinite lossy channel machine executions that repeatedly try to simulate one transition of $\textit{CM}$ but never success, and the library executions that simulate $t$ violate four liveness properties.
For this reason, %This is the reason why
we do not transform the lossy channel machine with two channels in \cite{DBLP:journals/iandc/AbdullaJ96a} into lossy channel machine with one channel in our undecidability proof.
%
%{\color{orange} GP: I think this sentence is hard to read for someone that hasn't read section 4.}
%If all delimiters are lost during transition,
%In some case, $\textit{CM'}$ falls into infinite loop for simulating one transition of $\textit{CM}$. %Such infinite executions do not influence trace equivalence. However,
%The library executions that simulate such lossy channel machine executions violate all four liveness properties.
%We discuss in Section \ref{sec:undecidability of lock-freedom and wait-freedom} that why we do not transform such lossy channel machine with two channels into a single channel lossy channel machine.
%Wang \emph{et al.} simulate lossy channel machines with libraries in a slightly different manner, and they consider safety properties and not liveness properties.
%Our previous work simulates lossy channel machine executions with call and return actions on TSO, but consider only finite length executions and do not consider liveness properties.
}


%Wang \emph{et al.}
Our previous work \cite{DBLP:conf/sofsem/WangLW16} verifies bounded TSO-to-SC linearizability by reducing it into another known decidable reachability problem, the control state reachability problem of lossy channel machines.
That work focuses on dealing with call and return actions across multiple processes, while our verification approach for obstruction-freedom %in this paper
considers call and return action as internal actions.


\forget{
Our approach of reducing checking %properties (
obstruction-freedom %and bounded wait-freedom)
into control state reachability of a lossy channel
machine is inspired by Atig \emph{et al.} \cite{DBLP:conf/popl/AtigBBM10}, which
reduces the state reachability of %TSO programs
TSO concurrent system into control state reachability of
a lossy channel machine.
However, Atig \emph{et al.} do not mention libraries, and their lossy channel
machine %transitions
cannot directly deal with blocking pairs, since capturing blocking pairs requires reading the memory valuation and buffers atomically. Our previous work
\cite{DBLP:conf/sofsem/WangLW16} reduces bounded TSO-to-SC linearizability
\cite{DBLP:conf/wdag/GotsmanMY12} into control state reachability of a lossy
channel machine.
Our previous work \cite{DBLP:conf/sofsem/WangLW16} focuses on dealing with call and return actions across multiple
processes, while in this paper the call and return actions are considered as internal %transitions.
actions.}

%Brit \emph{et al.} \cite{DBLP:journals/jucs/BritM96} shows that on SC a library for an unbounded number of processes can be wait-free but not bounded wait-free. We prove that on SC any wait-free library for bounded number of processes must have a bound.

\forget{
{\color{orange} GP: I would leave the acknowledgements for the final version,
  and not the submission.}
{\color {red}\noindent \textbf{Acknowledgements:} We are grateful to an anonymous reviewer for his insightful suggestion, which greatly helped to simplify the proof of obstruction-freedom checking.}
}

%For reducing PCP into problem of lossy channels: Our idea of reducing PCP to an infinite execution problem of lossy channel machine is inspired by Abdulla \emph{et al.} \cite{DBLP:journals/iandc/AbdullaJ96a}, which reduces the cyclic post correspondence problem (CPCP) \cite{DBLP:journals/acta/Ruohonen83} into checking if the lossy channel machine has an execution that visits a specific state infinite times (the recurrent state problem).
%Our approach generate uses PCP and generates a lossy channel machine that is more easier to understand.

%%% Local Variables:
%%% mode: latex
%%% TeX-master: "FSTTCS2021.tex"
%%% End: 



\section{Concurrent Systems}
\label{sec:concurrent systems}

%In this section, we present the notations of concurrent objects and concurrent systems. We then introduce their operational semantics on TSO.

\subsection{Notations}

In general, a finite sequence on an alphabet $\Sigma$ is denoted $l=a_1 \cdot a_2 \cdot \ldots \cdot a_k$, where $\cdot$ is the concatenation symbol and $a_i\in\Sigma$ for each $1 \! \leq \! i \! \leq \! k$. Let $|l|$ and $l(i)$ denote the length and the $i$-th element of $l$, respectively, i.e., $|l|=k$ and $l(i)=a_i$ for $1 \! \leq \! i \! \leq \! k$.
Let $l(i,j)$ denote the string $l(i) \cdot \ldots \cdot l(j)$. %, and $l(i,\infty)$ denote the string $l(i) \cdot \ldots$, where the size of $l$ is infinite.
Let $l \uparrow_{\Sigma'}$ denote the projection of $l$ on the alphabet $\Sigma'$. Given a function $f$, let $f[x:y]$ be the function that is the same as $f$ everywhere, except for $x$, where it has the value $y$. Let $\_$ denote an item, of which the value is irrelevant, and $\epsilon$ the empty word.

A $\textit{labelled transition system}$ (LTS) is a tuple $\mathcal{A}=(Q,\Sigma,\rightarrow,q_0)$, where $Q$ is a set of states, $\Sigma$ is an alphabet of transition labels, $\rightarrow\subseteq Q\times\Sigma\times Q$ is a transition relation and $q_0$ is the initial state.
A finite path of $\mathcal{A}$ is a finite  sequence of transitions
$q_0\xrightarrow{a_1}q_1\overset{a_2}{\longrightarrow}\ldots\overset{a_k}{\longrightarrow}q_k$ with
$k \! \geq \! 0$, and a finite trace of $\mathcal{A}$ is a finite sequence $t= a_1 \cdot a_2 \cdot
\ldots \cdot a_k$, with $k \! \geq \! 0$ if there exists a finite path
$q_0\overset{a_1}{\longrightarrow}q_1\overset{a_2}{\longrightarrow}\ldots\overset{a_k}{\longrightarrow}q_k$ of $\mathcal{A}$.
An infinite path of $\mathcal{A}$ is an infinite sequence of transitions
$q_0\xrightarrow{a_1}q_1\overset{a_2}{\longrightarrow}\ldots$, and correspondingly an infinite trace of $\mathcal{A}$ is an infinite sequence $t= a_1 \cdot a_2 \cdot \ldots$ if there exists an infinite path $q_0\overset{a_1}{\longrightarrow}q_1\overset{a_2}{\longrightarrow}\ldots$ of $\mathcal{A}$.

\subsection{Concurrent Objects %Libraries
and The Most General Client}

%A concurrent data structure provides a number of methods for accessing the data structure. %internal state representation.
%\redt{A concurrent object provides a set of methods for client programs to access the object.}
%Here we assume, as customary, that
Concurrent objects %concurrent data structures
are implemented as well-encapsulated libraries. %A client program is a program that interacts with libraries.
The \emph{most general client} of a concurrent object %a library
is a program that interacts with the object, %library,
and is designed to exhibit all the possible behaviors of the object. %a library.
A simple instance of the %
most general
client is a client %with a number of threads
that repeatedly makes non-deterministic method calls with non-deterministic arguments. %{\color {red}while respecting methods preconditions.}
Libraries may contain private memory locations for their own uses. For simplicity, and without loss
of generality, we assume that methods have only one argument and one return value (when %it returns).
they return).

Given a finite set $\mathcal{X}$ of memory locations, a finite set $\mathcal{M}$
of method names and a finite data domain $\mathcal{D}$, the set $\textit{PCom}$
of primitive commands
is defined by the following grammar:
\vspace{-3pt}
\[
  \begin{array}{lcl}
    \textit{PCom} & ::= & \tau \ | \ %\textit{checkPID}\_\textit{suc}(i)  \ | \
                          %\textit{checkPID}\_\textit{fail}(i) \ | \
                          \textit{read}(x,a) \  |\  \textit{write}(x,a) \ | \ %\\
                  %& | &
                        \textit{cas}\_\textit{suc}(x,a,b) \ | \
                  %& | &
                         \textit{cas}\_\textit{fail}(x,a,b)\\
    & | & \textit{call}(m,a) \ |\ \textit{return}(m,a)
  \end{array}
\]

% \begin{equation*}
% \begin{split}
%   \textit{PCom} ::= \tau \ | \  %\textit{getPID}\Rightarrow j
%     \textit{checkPID}(i)  \ | \ \textit{read}(x,a) \  |\  \textit{write}(x,a)\  | \\
%   % \textit{checkPID}\_\textit{cas}(i) \ | \textit{checkPID}\_\textit{fail}(i)}  \ | \ \textit{read}(x,a) \  |\  \textit{write}(x,a)\  | \\
%     \textit{cas}\_\textit{suc}(x,a,b) \ |\ \textit{cas}\_\textit{fail}(x,a,b) \ | \  \textit{call}(m,a) \ |\ \textit{return}(m,a)
% \end{split}
% \end{equation*}
%{\color{orange} GP: I don't really like the $\Rightarrow j$ notation for the getPID command. Could we rename it $checkPID(j)$ or something like that if it is a conditional?}

\noindent where $a, b \in \mathcal{D}, x \in \mathcal{X}$ %,
and $m\in\mathcal{M}$. %,
%and $i \in \mathbb{N}$.
Here $\tau$ represents an internal command.
To use the commands as labels in an LTS we assume that they encode the
expected values that they return (an oracle of sorts). Hence, for instance the
read command $\textit{read}(x, a)$ encodes the value read $a$.
%
%Similarly, the $\textit{checkPID}\_\textit{suc}(i)$ command is the successful case of the command used to check if the process identifier of the process executing it is $i$. If the process identifier does not match the command cannot execute, but $\textit{checkPID}\_\textit{fail}(i)$ can execute instead.
In general \textit{cas} (compare-and-set) commands execute a read
and a conditional write (or no write at all) in a single atomic step. In our
case a successful $\textit{cas}$ is represented with the command
$\textit{cas}\_\textit{suc}(x,a,b)$, and it is enabled when the initial value of
$x$ is $a$, upon which the command %it
updates it with value $b$, while a failed
$\textit{cas}$ command, represented with the command
$\textit{cas}\_\textit{fail}(x,a,$ $b)$ does not update the state, and can only
happen when the value of $x$ is not $a$.


A library $\mathcal{L}$ is a tuple
\mbox{$\mathcal{L}$ = $(\mathcal{X}_{\mathcal{L}},\mathcal{M}_{\mathcal{L}}, \mathcal{D}_{\mathcal{L}}, Q_\mathcal{L},$ $\rightarrow_\mathcal{L})$}, where $\mathcal{X}_{\mathcal{L}}$, $\mathcal{M}_{\mathcal{L}}$ and $\mathcal{D}_{\mathcal{L}}$ are a finite memory location set, a finite method name set and a finite data domain of $\mathcal{L}$, respectively.
$Q_\mathcal{L} = \bigcup_{m \in \mathcal{M_\mathcal{L}}} Q_m $ is the union of disjoint finite sets $Q_m$ of program positions of each method $m\in\mathcal{M}_\mathcal{L}$.
Each program position represents the current program counter value and local register value of a
process and can be considered as a state.
$\rightarrow_\mathcal{L} = \bigcup_{m \in \mathcal{M}_\mathcal{L}} \rightarrow_m$ is the union of disjoint transition relations of each method $m\in\mathcal{M}_\mathcal{L}$. Let $\textit{PCom}_{\mathcal{L}}$ be the set of primitive commands (except call and return commands) upon $\mathcal{X}_{\mathcal{L}}$, $\mathcal{M}_{\mathcal{L}}$ and $\mathcal{D}_{\mathcal{L}}$. Then, for each $m \in \mathcal{M}_{\mathcal{L}}$, $\rightarrow_m \subseteq Q_m \times \textit{PCom}_{\mathcal{L}} \times Q_m$. For each $m \in \mathcal{M}_{\mathcal{L}}$ and $a\in\mathcal{D}_\mathcal{L}$, $Q$ contains an initial %state
program position $\textit{is}_{(\textit{m,a})}$, which represents that library begins to execute method $m$ with argument $a$, and a final %state
program position $\textit{fs}_{(\textit{m,a})}$ which represents that method $m$ has finished its execution and then a return action with return value $a$ can occur. There are neither incoming transitions to $\textit{is}_{(\textit{m,a})}$ nor outgoing transitions from $\textit{fs}_{(\textit{m,a})}$ in $\rightarrow_m$.
% ; while for each $a\in\mathcal{D}_\mathcal{L}$ there exists an initial state $\textit{is}_{(\textit{m,a})}$ and a final state $\textit{fs}_{(\textit{m,a})}$ in $Q_m$ such that there are neither incoming transitions to $\textit{is}_{(\textit{m,a})}$ nor outgoing transitions from $\textit{fs}_{(\textit{m,a})}$ in $\rightarrow_m$. $\textit{is}_{(\textit{m,a})}$ represents that concurrent data structure begins to execute method $m$ with argument $a$, and $\textit{fs}_{(\textit{m,a})}$ represents that method $m$ has finished its execution and then a return action with return value $a$ can occur.

% \gpnote{I think we need to explain why $\{q_c,q'_c\}$ here. This is unusual.}

% \color {red}A most general client is a special program that repeatedly calls an arbitrary method with an arbitrary argument for arbitrarily many times.
%Formally, the %a
The most general client $\mathcal{MGC}$ is defined as a tuple $( \mathcal{M}_{\mathcal{C}}, \mathcal{D}_{\mathcal{C}}, Q_{\mathcal{C}},\rightarrow_{\textit{mgc}})$, where $\mathcal{M}_{\mathcal{C}}$ is a finite method name set, $\mathcal{D}_{\mathcal{C}}$ is a finite data domain, $Q_{\mathcal{C}}=\{\textit{in}_{\textit{clt}},\textit{in}_{\textit{lib}}\}$ %$Q_{\mathcal{C}}=\{q_c,q'_c\}$
is the state set, %$q_c$ and $q'_c$ are two states,
and $\rightarrow_{\textit{mgc}}=  \{ (\textit{in}_{\textit{clt}},\textit{call}(m,a),\textit{in}_{\textit{lib}}), (\textit{in}_{\textit{lib}},\textit{return}(m,b), \textit{in}_{\textit{clt}}), \vert m \in \mathcal{M}_{\mathcal{C}}, a,b \in \mathcal{D}_{\mathcal{C}} \} $ %and $\rightarrow_{\textit{mgc}}=  \{ (q_c,\textit{call}(m,a),$ $q'_c),(q'_c,\textit{return}(m,b)$, $q_c), \vert m \in \mathcal{M}_{\mathcal{C}}, a,b \in \mathcal{D}_{\mathcal{C}} \} $
is a transition relation. State $\textit{in}_{\textit{clt}}$ %$q_c$
represents that currently %there is
no method of library is running, and $\textit{in}_{\textit{lib}}$ %$q'_c$
represents that some method of library is running.
%A client program $\mathcal{C}$ can then be defined as a tuple $\mathcal{C}$ = $(\mathcal{X}_{\mathcal{C}},\mathcal{M}_{\mathcal{C}}, \mathcal{D}_{\mathcal{C}}, Q_\mathcal{C}, $ $\rightarrow_\mathcal{C})$ where $\mathcal{X}_{\mathcal{C}}$, $\mathcal{M}_{\mathcal{C}}$,  $\mathcal{D}_{\mathcal{C}}$ and $Q_\mathcal{C}$ are a finite memory location set, a finite method name set, a finite data domain and a finite program position set of $\mathcal{C}$, respectively. Let $\textit{PCom}_{\mathcal{C}}$ be the set of primitive commands upon $\mathcal{X}_{\mathcal{C}}$, $\mathcal{M}_{\mathcal{C}}$ and $\mathcal{D}_{\mathcal{C}}$. Then, $\rightarrow_{\mathcal{C}} \subseteq Q_{\mathcal{C}} \times \textit{PCom}_{\mathcal{C}} \times Q_{\mathcal{C}}$ is a transition relation of $\mathcal{C}$. A most general client is a special client program that is designed to exhibit all the possible behaviors of a library. Formally, a most general client $\mathcal{MGC}$ is defined as a client $( \{\}, \mathcal{M}_{\mathcal{C}}, \mathcal{D}_{\mathcal{C}}, \{q_c,q'_c\},\rightarrow_{\textit{mgc}})$ with a transition relation $\rightarrow_{\textit{mgc}}=  \{ (q_c,\textit{call}(m,a),$ $q'_c),(q'_c,\textit{return}(m,b),q_c), \vert m \in \mathcal{M}_{\mathcal{C}}, a,b \in \mathcal{D}_{\mathcal{C}} \} $. $q_c$ represents that currently there is no method of library running on this most general client program, and $q'_c$ represents that some method of library is running on this most general client program. Intuitively, a most general client %simply
%repeatedly calls an arbitrary method with an arbitrary argument for arbitrarily many times.






\subsection{TSO Operational Semantics}
\label{sec:operational semantics}

A concurrent system consists of $n$ processes, each of which runs the %a
most general client $\mathcal{MGC}$ = $( \mathcal{M}, \mathcal{D}, \{\textit{in}_{\textit{clt}},\textit{in}_{\textit{lib}}\},\rightarrow_{\textit{mgc}})$, and all the most general clients interact with a same library $\mathcal{L}$ = $(\mathcal{X}_{\mathcal{L}},\mathcal{M}, \mathcal{D}, Q_\mathcal{L},\rightarrow_\mathcal{L})$.
%{\color {red}On TSO memory model \cite{DBLP:conf/tphol/OwensSS09}, each processor is equipped with an FIFO store buffer. Although in each detailed chip of TSO memory model the buffer size is bounded, since the TSO memory model is proposed to model all chips with such mechanism, the buffer size is unbounded in TSO memory model of \cite{DBLP:conf/tphol/OwensSS09}.}
In this paper we follow the %theoretical
TSO memory model of~\cite{DBLP:conf/popl/AtigBBM10}
(similarly
to~\cite{DBLP:conf/tphol/OwensSS09,DBLP:conf/esop/BouajjaniDM13,DBLP:conf/esop/BurckhardtGMY12}),
where each processor is equipped with a FIFO store buffer.
As explained in the introduction, in this %theoretical
TSO memory model, each process is associated with an unbounded FIFO store buffer.
%{\color {red}Although in every realistic multiprocessor system implementing the TSO memory model, the buffer is of bounded size, to describe the semantics of \emph{any} TSO implementing system it is necessary to consider unbounded size FIFO store buffers associated with each process, as in the semantics of~\cite{DBLP:conf/popl/AtigBBM10}.}
Fence commands are used to ensure order between commands before fence and commands after fence. The TSO memory model of \cite{DBLP:conf/popl/AtigBBM10} does not include fence commands, since fence commands can be simulated with $\textit{cas}$ commands.
%Although in every realistic multiprocessor system implementing the TSO memory model, the buffer is of bounded size, to describe the semantics of \emph{any} TSO implementing system it is necessary to consider unbounded size FIFO store buffers associated with each process, as in the semantics of~\cite{DBLP:conf/tphol/OwensSS09}. associates a unbounded size FIFO store buffer with each process. This is clearly a safe over-approximation of any TSO system.




The operational semantics of a concurrent system (with library $\mathcal{L}$ and $n$ processes) on TSO is defined
as an LTS $\llbracket \mathcal{L}, n \rrbracket$ = $(\textit{Conf}, \Sigma,
\rightarrow,\textit{InitConf})$, %where
with $\textit{Conf}$, $\Sigma$, $\rightarrow$
and $\textit{InitConf}$ described below. %as follows.

Configuration of $\textit{Conf}$ are tuples $(p,d,u)$, where $p: \{ 1, \ldots, n \} \rightarrow \{\textit{in}_{\textit{clt}}\} \cup (Q_{\mathcal{L}} \times \{\textit{in}_{\textit{lib}}\})$ %$p: \{ 1, \ldots, n \} \rightarrow \{q_c\} \cup (Q_{\mathcal{L}} \times \{q'_c\})$
represents the control state of each process, $d: \mathcal{X}_{\mathcal{L}} \rightarrow \mathcal{D}$ is the valuation of memory locations, and $u: \{ 1, \ldots, n\} \rightarrow (\{ (x,a)\ \vert\ x \in \mathcal{X}_{\mathcal{L}}, a \in \mathcal{D} \})^*$ is the content of each process's %processor's
store buffer. The initial configuration $\textit{InitConf} \in \textit{Conf}$ is $(p_{\textit{init}}, d_{\textit{init}}, u_{\textit{init}})$. Here $p_{\textit{init}}$ maps each process id to $\textit{in}_{\textit{clt}}$, %$q_c$,
$d_{\textit{init}}$ is a valuation for memory locations in $\mathcal{X}_{\mathcal{L}}$, and $u_{\textit{init}}$ initializes each process with an empty buffer.

%{\color{orange} Indeed there seems to be an  inconsistent use of $getPID$ below.}

We denote with $\Sigma$ the set of actions defined by the following grammar:\\[-5pt]
\[
  \begin{array}{lcl}
    \Sigma & ::= & \tau(i) \  | \ %{\color {red}getPID(i)}
    \textit{read}(i,x,a) \  |\  \textit{write}(i,x,a)\  | \
    \textit{cas}(i,x,a,b) \ | \
    \textit{flush}(i,x,a) \\
    & | & \textit{call}(i,m,a) \ | \  \textit{return}(i,m,a)
  \end{array}
\]


\noindent where $1 \! \leq \! i \! \leq \! n, m \in \mathcal{M}$, $x \in
\mathcal{X}_{\mathcal{L}}$ and $a,b \in \mathcal{D}$. The transition relation $\rightarrow$ is the least relation satisfying the
transition rules shown in \figurename~\ref{fig:transition relation te} for each
$1 \leq i \leq n$. The rules are explained below:
\begin{itemize}
%\setlength{\itemsep}{0.5pt}
\item[-] $\textit{Tau}$ rule: A $\tau$ transition only influences the control state of one process.

%\item[-] {\color {red}$\textit{GetPID}$ rule: A $\textit{getPID}$ transition can occur if the process id matches the argument of the command.}
%\item[-] $\textit{CheckPID}\_\textit{Suc}$ and $\textit{CheckPID}\_\textit{Fail}$ rules: A $\textit{checkPID}\_\textit{suc}$ transition can occur if the current process ID matches the process ID of the $\textit{checkPID}\_\textit{suc}$ command, and a $\textit{checkPID}\_\textit{fail}$ transition can occur if the current process ID does not match the process ID of the $\textit{checkPID}\_\textit{fail}$ command.

\item[-] $\textit{Read}$ rule: A function $\textit{lookup}(u,d,i,x)$ is used to search for the latest value of $x$ %from its %processor-local
    %store buffer
    in the buffer or the main memory, i.e.,
\begin{displaymath}
\textit{lookup(u,d,i,x)} = \left \{ \begin{array}{ll}
                      a & \textrm{if } u(i) %\uparrow_{\Sigma_x}
                          \uparrow_{ \{ (x,b)\vert b \in \mathcal{D} \} }  =(x,a) \cdot l, \ \textit{for some sequence} \ l %\in \Sigma_x^*
                          \\
                      d(x) & \textrm{otherwise }
                      \end{array} \right.
\end{displaymath}
where %$\Sigma_x$ = $\{(x,a) \vert x \in \mathcal{X}_{\mathcal{L}}, a \in \mathcal{D} \}$
$\{ (x,b)\vert b \in \mathcal{D} \}$ is the set of items of $x$ in buffer. %pending write actions for $x$.
%Read actions return the latest value of $x$ from the processor-local store buffer if present, and return the value in memory if there isn't any.
A $\textit{read}(i,x,a)$ action returns the latest value of $x$ in the buffer
if present, or returns the value in memory if the buffer contains no stores on
$x$.

\item[-] $\textit{Write}$ rule: %A write action will insert a pair of memory location and value at the %end
    %tail of its processor-local store buffer.
    An $\textit{write}(i,x,a)$ action puts an item $(x,a)$ into
    the tail of its %processor-local %the
    store buffer.

\item[-] $\textit{Cas}\_\textit{Suc}$ and $\textit{Cas}\_\textit{Fail}$ rules: %A $\textit{cas}$ command can only be executed when the  processor-local store buffer is empty and thus forces the current process to clear its store buffer in advance. %A successful $\textit{cas}$ command will change the value of memory location $x$ immediately while a failed $\textit{cas}$ command does not change memory.
  A $\textit{cas}$ action atomically executes a read and a conditional write (or no write at all) if and only if the process's %processor-local
  store buffer is empty.

\item[-] $\textit{Flush}$ rule: %The memory model may decide to flush the entry at the head of processor-local store buffer to memory at any time.
    An $\textit{flush}$ action is carried out by the memory model to flush the item at the head of the process's %processor-local
    store buffer to memory at any time.

\item[-] $\textit{Call}$ and $\textit{Return}$ rules: After a $\textit{call}$ action, the current
  process %transitions
  transits to $\textit{is}_{(\textit{m,a})}$. When the current process %comes to
  %transitions
  transits to $\textit{fs}_{(\textit{m,a})}$ it can launch a $\textit{return}$ action and %return to
  move to %$q_c$
  $\textit{in}_{\textit{clt}}$ of the most general client. %program.
\end{itemize}
%\gpnote{Where are \emph{is} and \emph{fs} defined?}
\begin{figure}[tbp]
%tau%
\[
\begin{array}{l c}
  \bigfrac{ p(i)=(q_i,\textit{in}_{\textit{lib}} %q'_c
  ) \quad q_i\ {\xrightarrow{\tau}}_{\mathcal{L}}\ q'_i } { ( p,d,u)\ {\xrightarrow{\tau(i)}}\ ( p[i:(q'_i,\textit{in}_{\textit{lib}} %q'_c
)],d,u )} {\textit{Tau}}
\end{array}
\]

\vspace{-2pt}

%checkPID_suc
%\[
%  \begin{array}{l c}
%    \bigfrac{ p(i)=(q_i,q'_c) \quad q_i\ {\xrightarrow{\textit{checkPID}\_\textit{suc}(i)}}_{\mathcal{L}}\ q'_i } { ( p,d,u)\ {\xrightarrow{\textit{checkPID}(i)}}\ ( p[i:(q'_i,q'_c)],d,u )} {\textit{CheckPID}\_\textit{Suc}}
%  \end{array}
%\]

%\vspace{-2pt}

%checkPID_fail
%\[
%  \begin{array}{l c}
%    \bigfrac{ p(i)=(q_i,q'_c) \quad j \neq i \quad q_i\ {\xrightarrow{\textit{checkPID}\_\textit{fail}(j)}}_{\mathcal{L}}\ q'_i } { ( p,d,u)\ {\xrightarrow{\textit{checkPID}(i)}}\ ( p[i:(q'_i,q'_c)],d,u )} {\textit{CheckPID}\_\textit{Fail}}
%  \end{array}
%\]

%\vspace{-2pt}


%getPID%
%{\color {red}\[
%\begin{array}{l c}
%\bigfrac{ p(i)=(q_i,q'_c) \quad q_i\ {\xrightarrow{\textit{getPID}\Rightarrow i}}_{\mathcal{L}}\ q'_i } { %( p,d,u)\ {\xrightarrow{\textit{getPID}(i)}}\ ( p[i:(q'_i,q'_c)],d,u )} {\textit{GetPID}}
%\end{array}
%\]}

\vspace{-2pt}


%read%
\[
\begin{array}{l c}
  \bigfrac{ p(i)=(q_i,\textit{in}_{\textit{lib}} %q'_c
) \quad q_i\ {\xrightarrow{\textit{read}(x,a)}}_{\mathcal{L}}\ q'_i \quad \textit{lookup}(u,d,i,x)=a } { ( p,d,u)\ {\xrightarrow{\textit{read}(i,x,a)}}\
  ( p[i:(q'_i, \textit{in}_{\textit{lib}} %q'_c
)],d,u ] )} {\textit{Read}}
\end{array}
\]

\vspace{-2pt}



%write%
\[
\begin{array}{l c}
  \bigfrac{ p(i)=(q_i,\textit{in}_{\textit{lib}} %q'_c
) \quad q_i\
  {\xrightarrow{\textit{write}(x,a)}}_{\mathcal{L}}\ q'_i \quad u(i)=l } { ( p,d,u)\ {\xrightarrow{\textit{write}(i,x,a)}}\
  ( p[i:(q'_i, \textit{in}_{\textit{lib}} %q'_c
)],d,u[i:(x,a) \cdot l] )} {\textit{Write}}
\end{array}
\]

\vspace{-2pt}

%cas_suc%
\[
\begin{array}{l c}
  \bigfrac{ p(i)=(q_i, \textit{in}_{\textit{lib}} %q'_c
) \quad q_i\
  {\xrightarrow{\textit{cas}\_\textit{suc}(x,a,b)}}_{\mathcal{L}}\ q'_i \quad
  d(x)=a \quad u(i)=\epsilon } { ( p,d,u)\
{\xrightarrow{\textit{cas}(i,x,a,b)}}\
  ( p[i:(q'_i, \textit{in}_{\textit{lib}} %q'_c
)],d[x:b],u)} {\textit{Cas}\_\textit{Suc}}

\end{array}
\]

\vspace{-2pt}

%cas_fail%
\[
\begin{array}{l c}
  \bigfrac{ p(i)=(q_i, \textit{in}_{\textit{lib}} %q'_c
) \quad q_i\
  {\xrightarrow{\textit{cas}\_\textit{fail}(x,a,b)}}_{\mathcal{L}}\ q'_i \quad
  d(x) \neq a \quad u(i)=\epsilon } { (p,d,u)\
  {\xrightarrow{\textit{cas}(i,x,a,b)}}\ ( p[i:(q'_i, \textit{in}_{\textit{lib}} %q'_c
  )],d,u)} {\textit{Cas}\_\textit{Fail}}
\end{array}
\]

\vspace{-2pt}

%flush%
\[
\begin{array}{l c}
\bigfrac{ u(i)=l \cdot (x,a) } { ( p,d,u)\
{\xrightarrow{\textit{flush}(i,x,a)}}\
( p,d[x:a],u[i:l] )} {\textit{Flush}}
\end{array}
\]

\vspace{-2pt}


%call%
\[
\begin{array}{l c}
  \bigfrac{ p(i)= \textit{in}_{\textit{clt}} %q_c
} { (p,d,u)\
{\xrightarrow{\textit{call}(i,m,a)}}\
  ( p[ i: (\textit{is}_{(\textit{m,a})}, \textit{in}_{\textit{lib}} %q'_c
) ],d,u)} {\textit{Call}}
\end{array}
\]

\vspace{-2pt}

%return%
\[
\begin{array}{l c}
  \bigfrac{ p(i)=(\textit{fs}_{(\textit{m,a})}, \textit{in}_{\textit{lib}} %q'_c
) } { (p,d,u)\
{\xrightarrow{\textit{return}(i,m,a)}}\
  ( p[i: \textit{in}_{\textit{clt}} %q_c
],d,u )} {\textit{Return}}
\end{array}
\]

%\vspace{-5pt}
\caption{Transition Relation $\rightarrow$}\label{fig:transition
relation te}

\vspace{-15pt}

\end{figure}




\forget{The detailed
definition of the transition relation can be found in Appendix
\ref{sec:appendix proof of section sec:concurrent systems}. Intuitively, an
$\textit{write}(i,x,a)$ action puts an item $(x,a)$ into the store buffer, and
a $\textit{read}(i,x,a)$ action returns the latest value of $x$ in the buffer
if present, or returns the value in memory if the buffer contains no stores on
$x$. Actions $\tau$, $\textit{call}$ and $\textit{return}$ only influence the
control state. An $\textit{flush}$ action is carried out by the memory model
to flush the item at the head of the process's %processor-local
store buffer to memory at
any time. A $\textit{cas}$ action atomically executes a read and a
conditional write (or no write at all) if and only if the process's %processor-local
store buffer is empty.
}


\forget{
The transition relation $\rightarrow$ is the least relation satisfying the
transition rules shown in \figurename~\ref{fig:transition relation te} for each
$1 \leq i \leq n$. The rules are explained below:
\begin{itemize}
%\setlength{\itemsep}{0.5pt}
\item[-] $\textit{Tau}$ rule: A $\tau$ transition only influences the control state of one process.

%\item[-] {\color {red}$\textit{GetPID}$ rule: A $\textit{getPID}$ transition can occur if the process id matches the argument of the command.}
%\item[-] $\textit{CheckPID}\_\textit{Suc}$ and $\textit{CheckPID}\_\textit{Fail}$ rules: A $\textit{checkPID}\_\textit{suc}$ transition can occur if the current process ID matches the process ID of the $\textit{checkPID}\_\textit{suc}$ command, and a $\textit{checkPID}\_\textit{fail}$ transition can occur if the current process ID does not match the process ID of the $\textit{checkPID}\_\textit{fail}$ command.

\item[-] $\textit{Read}$ rule: A function $\textit{lookup}(u,d,i,x)$ is used to search for the latest value of $x$ from its  processor-local store buffer or the main memory, i.e.,
\begin{displaymath}
\textit{lookup(u,d,i,x)} = \left \{ \begin{array}{ll}
                      a & \textrm{if } u(i) %\uparrow_{\Sigma_x}
                      \redt{ \uparrow_{ \{ (x,b)\vert b \in \mathcal{D} \} } } =(x,a) \cdot l, \ \textit{for some} \ l \in \Sigma_x^*  \\
                      d(x) & \textrm{otherwise }
                      \end{array} \right.
\end{displaymath}
where %$\Sigma_x$ = $\{(x,a) \vert x \in \mathcal{X}_{\mathcal{L}}, a \in \mathcal{D} \}$
\redt{$\{ (x,b)\vert b \in \mathcal{D} \}$} is the set of \redt{items of $x$ in buffer.} %pending write actions for $x$.
Read actions return
the latest value of $x$ from the processor-local store buffer if present, and
return the value in memory if there isn't any.

\item[-] $\textit{Write}$ rule: A write action will insert a pair of memory location and value at the %end
    tail of its processor-local store buffer.

\item[-] $\textit{Cas}\_\textit{Suc}$ and $\textit{Cas}\_\textit{Fail}$ rules: A $\textit{cas}$
  command can only be executed when the  processor-local store buffer is empty
  and thus forces the current process to clear its store buffer in advance. %A successful $\textit{cas}$ command will change the value of memory location $x$ immediately while a failed $\textit{cas}$ command does not change memory.

\item[-] $\textit{Flush}$ rule: The memory model may decide to flush the entry at the head of processor-local store buffer to memory at any time.

\item[-] $\textit{Call}$ and $\textit{Return}$ rules: After a $\textit{call}$ action, the current
  process %transitions
  transits to $\textit{is}_{(\textit{m,a})}$. When the current process %comes to
  %transitions
  transits to $\textit{fs}_{(\textit{m,a})}$ it can launch a $\textit{return}$ action and %return to
  move to %$q_c$
  \redt{$\textit{in}_{\textit{clt}}$} of the most general client. %program.
\end{itemize}
%\gpnote{Where are \emph{is} and \emph{fs} defined?}
\begin{figure}[tbp]
%tau%
\[
\begin{array}{l c}
\bigfrac{ p(i)=(q_i,\redt{\textit{in}_{\textit{lib}}} %q'_c
) \quad q_i\ {\xrightarrow{\tau}}_{\mathcal{L}}\ q'_i } { ( p,d,u)\ {\xrightarrow{\tau(i)}}\ ( p[i:(q'_i,\redt{\textit{in}_{\textit{lib}}} %q'_c
)],d,u )} {\textit{Tau}}
\end{array}
\]

\vspace{-2pt}

%checkPID_suc
%\[
%  \begin{array}{l c}
%    \bigfrac{ p(i)=(q_i,q'_c) \quad q_i\ {\xrightarrow{\textit{checkPID}\_\textit{suc}(i)}}_{\mathcal{L}}\ q'_i } { ( p,d,u)\ {\xrightarrow{\textit{checkPID}(i)}}\ ( p[i:(q'_i,q'_c)],d,u )} {\textit{CheckPID}\_\textit{Suc}}
%  \end{array}
%\]

%\vspace{-2pt}

%checkPID_fail
%\[
%  \begin{array}{l c}
%    \bigfrac{ p(i)=(q_i,q'_c) \quad j \neq i \quad q_i\ {\xrightarrow{\textit{checkPID}\_\textit{fail}(j)}}_{\mathcal{L}}\ q'_i } { ( p,d,u)\ {\xrightarrow{\textit{checkPID}(i)}}\ ( p[i:(q'_i,q'_c)],d,u )} {\textit{CheckPID}\_\textit{Fail}}
%  \end{array}
%\]

%\vspace{-2pt}


%getPID%
%{\color {red}\[
%\begin{array}{l c}
%\bigfrac{ p(i)=(q_i,q'_c) \quad q_i\ {\xrightarrow{\textit{getPID}\Rightarrow i}}_{\mathcal{L}}\ q'_i } { %( p,d,u)\ {\xrightarrow{\textit{getPID}(i)}}\ ( p[i:(q'_i,q'_c)],d,u )} {\textit{GetPID}}
%\end{array}
%\]}

\vspace{-2pt}


%read%
\[
\begin{array}{l c}
\bigfrac{ p(i)=(q_i,\redt{\textit{in}_{\textit{lib}}} %q'_c
) \quad q_i\ {\xrightarrow{\textit{read}(x,a)}}_{\mathcal{L}}\ q'_i \quad \textit{lookup}(u,d,i,x)=a } { ( p,d,u)\ {\xrightarrow{\textit{read}(i,x,a)}}\
( p[i:(q'_i, \redt{\textit{in}_{\textit{lib}}} %q'_c
)],d,u ] )} {\textit{Read}}
\end{array}
\]

\vspace{-2pt}



%write%
\[
\begin{array}{l c}
\bigfrac{ p(i)=(q_i,\redt{\textit{in}_{\textit{lib}}} %q'_c
) \quad q_i\
  {\xrightarrow{\textit{write}(x,a)}}_{\mathcal{L}}\ q'_i \quad u(i)=l } { ( p,d,u)\ {\xrightarrow{\textit{write}(i,x,a)}}\
( p[i:(q'_i, \redt{\textit{in}_{\textit{lib}}} %q'_c
)],d,u[i:(x,a) \cdot l] )} {\textit{Write}}
\end{array}
\]

\vspace{-2pt}

%cas_suc%
\[
\begin{array}{l c}
\bigfrac{ p(i)=(q_i, \redt{\textit{in}_{\textit{lib}}} %q'_c
) \quad q_i\
  {\xrightarrow{\textit{cas}\_\textit{suc}(x,a,b)}}_{\mathcal{L}}\ q'_i \quad
  d(x)=a \quad u(i)=\epsilon } { ( p,d,u)\
{\xrightarrow{\textit{cas}(i,x,a,b)}}\
( p[i:(q'_i, \redt{\textit{in}_{\textit{lib}}} %q'_c
)],d[x:b],u)} {\textit{Cas}\_\textit{Suc}}

\end{array}
\]

\vspace{-2pt}

%cas_fail%
\[
\begin{array}{l c}
\bigfrac{ p(i)=(q_i, \redt{\textit{in}_{\textit{lib}}} %q'_c
) \quad q_i\
  {\xrightarrow{\textit{cas}\_\textit{fail}(x,a,b)}}_{\mathcal{L}}\ q'_i \quad
  d(x) \neq a \quad u(i)=\epsilon } { (p,d,u)\
  {\xrightarrow{\textit{cas}(i,x,a,b)}}\ ( p[i:(q'_i, \redt{\textit{in}_{\textit{lib}}} %q'_c
  )],d,u)} {\textit{Cas}\_\textit{Fail}}
\end{array}
\]

\vspace{-2pt}

%flush%
\[
\begin{array}{l c}
\bigfrac{ u(i)=l \cdot (x,a) } { ( p,d,u)\
{\xrightarrow{\textit{flush}(i,x,a)}}\
( p,d[x:a],u[i:l] )} {\textit{Flush}}
\end{array}
\]

\vspace{-2pt}


%call%
\[
\begin{array}{l c}
\bigfrac{ p(i)= \redt{\textit{in}_{\textit{clt}}} %q_c
} { (p,d,u)\
{\xrightarrow{\textit{call}(i,m,a)}}\
( p[ i: (\textit{is}_{(\textit{m,a})}, \redt{\textit{in}_{\textit{lib}}} %q'_c
) ],d,u)} {\textit{Call}}
\end{array}
\]

\vspace{-2pt}

%return%
\[
\begin{array}{l c}
\bigfrac{ p(i)=(\textit{fs}_{(\textit{m,a})}, \redt{\textit{in}_{\textit{lib}}} %q'_c
) } { (p,d,u)\
{\xrightarrow{\textit{return}(i,m,a)}}\
( p[i: \redt{\textit{in}_{\textit{clt}}} %q_c
],d,u )} {\textit{Return}}
\end{array}
\]

\vspace{-5pt}
\caption{Transition Relation $\rightarrow$}\label{fig:transition
relation te}

\vspace{-15pt}

\end{figure}

%\vspace{-10pt}

The initial configuration $\textit{InitConf} \in \textit{Conf}$ is a tuple $(p_{\textit{init}}, d_{\textit{init}}, u_{\textit{init}})$. Here $p_{\textit{init}}$ maps each process id to \redt{$\textit{in}_{\textit{clt}}$}. %$q_c$,
$d_{\textit{init}}$ is a valuation for memory locations in $\mathcal{X}_{\mathcal{L}}$, and $u_{\textit{init}}$ initializes each process with an empty buffer.
}



%%% Local Variables:
%%% mode: latex
%%% TeX-master: "CONCUR2021.tex"
%%% End: 



\section{Liveness}
\label{sec:liveness}

%In this section, we introduce definitions of lock-freedom, wait-freedom, deadlock-freedom, starvation-freedom and obstruction-freedom \cite{DBLP:books/daglib/0020056,DBLP:conf/concur/LiangHFS13}. %, as well as the
%definition of %bounded wait-freedom \cite{DBLP:journals/jucs/BritM96,DBLP:books/daglib/0020056}.
%{\color {red}$k$-bounded wait-freedom \cite{DBLP:conf/pldi/PetrankMS09}.}

%Intuitively, lock-freedom requires that at each point in time there is at least one method that should return in a finite number of steps. Wait-freedom requires that each call to a library method returns in a finite number of steps. Deadlock-freedom and starvation-freedom can be obtained from lock-freedom and wait-freedom respectively by adding assumptions about the fair scheduling of executions. Obstruction-freedom guarantees that each method can return in a finite number of steps if it executes in isolation. Given a bound $k$, bounded wait-freedom requires that each call to a library method returns in no more than $k$ steps.
%\redt{Intuitively, clients using a wait-free library find that each method call returns in finite steps. Clients using a lock-free library find that there will always be some method return after the whole system executes for sufficient number of steps, while it is possible that method on some process never return. Deadlock-freedom and Starvation-freedom requires each fair execution satisfies lock-freedom and wait-freedom, respectively. Clients using a obstruction-free library find that each method call is guaranteed to return in finite number of steps when executed in isolation.}
\forget{
Below we give an intuitive descriptions of the properties listed above:
\begin{itemize}
\item Lock-freedom requires that some library method call returns whenever the concurrent system
  executes for a sufficient number of steps.
\item Wait-freedom requires that each library method call returns whenever the thread executing it
  is scheduled for a sufficient number of steps.
\item Deadlock-freedom requires that every fair execution satisfies lock-freedom.
\item Starvation-freedom requires that every fair execution satisfies wait-freedom.
\item Obstruction-freedom requires that each library method call returns in a finite number of steps
  when executed in isolation.
\end{itemize}
}


% {\color {red}The intuitive explanation of above liveness properties are as follows: Lock-freedom requires that some call to library method should return if the concurrent system executes for a sufficient number of steps, while wait-freedom requires that each call to a library method should return if it is scheduled for a sufficient number of steps. Deadlock-freedom (resp., starvation-freedom) %can be obtained from lock-freedom (resp., wait-freedom) by considering only executions of fair scheduling.
% requires each fair execution to satisfy lock-freedom (resp., wait-freedom).
% Obstruction-freedom requires that each call to library method should return in finite steps if it executes in isolation.} %$k$-bounded wait-freedom requires that each call to a library method should return within $k$ steps of the calling process.

%In this section, we introduce definitions of lock-freedom, wait-freedom, deadlock-freedom, starvation-freedom and obstruction-freedom \cite{DBLP:books/daglib/0020056,DBLP:conf/concur/LiangHFS13}.
%Let us now introduce their formal definitions.
We use $T_{\omega}(\llbracket \mathcal{L}, n \rrbracket)$ to denote all %finite or
infinite traces of the concurrent system $\llbracket \mathcal{L}, n \rrbracket$.
Given an execution $t \in T_{\omega}(\llbracket \mathcal{L}, n \rrbracket)$, %let $t(i,j)$ denote the substring $t(i) \cdot \ldots \cdot t(j)$ of $t$.
we say a call action $t(i)$ matches a return action $t(j)$ with $i<j$, if the two actions are by the
same process, and there are no call or return actions by the same process in-between.
%{\color{orange}
Here we assume that %there are no recursive methods, and also we assume that
methods do not call other methods. %}
%\gpnote{Shouldn't we require that there are not call or return actions of the
%  same process in-between?}
Let $\textit{pend}\_\textit{inv}(t)$ denote the set of pending call actions of
$t$, in other words, call actions of $t$ with no matching return action in $t$.

We define the following predicates borrowed from
\cite{DBLP:conf/concur/LiangHFS13}. Since we do not consider aborts, and we do
not consider termination markers, %with slight modifications.
we slightly modify the predicates definition of
\cite{DBLP:conf/concur/LiangHFS13} by consider only infinite executions. Given
an infinite execution $t \in T_{\omega}(\llbracket \mathcal{L},n \rrbracket)$:
\begin{itemize}
\item[-] $\textit{prog-t}(t)$: This predicate holds when every method call in $t$ eventually
  returns. Formally, for each index $i$ and action $e$, if $e \in
  \textit{pend}\_\textit{inv}(t(1,i))$, %then
  there exists $j>i$, such that $t(j)$ matches $e$. %in $t$.

\item[-] $\textit{prog-s}(t)$: This predicate holds when %it is always true that if there are pending method calls, at least one method returns in the future.
    there is always some method return action happens in the future if the system executes for a sufficient number of steps. Formally, for each index $i$ and action
  $e$, if $e \in \textit{pend}\_\textit{inv}(t(1,i))$ holds, then there exists $j>i$, such that
  $t(j)$ is a return action.

\item[-] $\textit{sched}(t)$: This predicate holds if %there are no pending invocations of library methods in $t$ (e.g. if $t$ is finite), or if
  $t$ is an infinite trace with pending call actions, %pending method
  %invocations,
  and at least one of the processes with a pending call action %pending invocations
  is scheduled infinitely many
  times. Formally, if $\vert t \vert = \omega$ and $\textit{pend}\_\textit{inv}(t) \neq \emptyset$,
  then there exists $e \in \textit{pend}\_\textit{inv}(t)$, such that $\vert t \uparrow_{pid(e)}
  \vert = \omega$. Here $t \uparrow_{pid(e)}$ represents the projection of $t$ into the actions of the process of $e$.

\item[-] $\textit{fair}(t)$: This predicate describing fair interleavings requires that if $t$ is an
  infinite execution, then each %non-terminated
  process is scheduled infinitely many times. Formally,
  if $\vert t \vert = \omega$, then for each process $proc$ %which is not ruminated
  in $t$,
  \mbox{$\vert t \uparrow_{proc} \vert$} \mbox{$= \omega$}. Here $t \uparrow_{proc}$ represents the
  projection of $t$ into actions of process $proc$.

\item[-] $\textit{iso}(t)$: This predicate requires that if $t$ is an infinite execution, eventually
  only one process is scheduled. Formally, if $\vert t \vert = \omega$, then there exists index $i$
  and process $proc$, such that for each $j>i$, $t(j)$ is an action of process $proc$.
\end{itemize}

With these predicates, we can present the formal notions of lock-freedom, wait-freedom, deadlock-freedom, starvation-freedom and obstruction-freedom of \cite{DBLP:conf/concur/LiangHFS13}. %{\color {red}and $k$-bounded wait-freedom of \cite{DBLP:conf/pldi/PetrankMS09}}.

\begin{definition}\label{def:lock-free, wait-free, deadlock-free, starvation-free and obstruction-free}
Given an execution $t \in T_{\omega}(\llbracket \mathcal{L},n \rrbracket)$:
\begin{itemize}
\item[-] $t$ satisfies lock-freedom whenever: $\textit{sched}(t) \Rightarrow \textit{prog-s}(t)$,

\item[-] $t$ satisfies wait-freedom whenever: $\textit{sched}(t) \Rightarrow \textit{prog-t}(t)$,

\item[-] $t$ satisfies deadlock-freedom whenever: $\textit{fair}(t) \Rightarrow \textit{prog-s}(t)$,

\item[-] $t$ satisfies starvation-freedom whenever: $\textit{fair}(t) \Rightarrow \textit{prog-t}(t)$,

\item[-] $t$ satisfies obstruction-freedom whenever: $\textit{sched}(t) \wedge \textit{iso}(t) \Rightarrow \textit{prog-s}(t)$
\end{itemize}

For library $\mathcal{L}$, we parameterize the definitions above over $n$ processes, and we define
their satisfaction requiring that each execution of $\llbracket \mathcal{L}, n \rrbracket$ satisfies
the corresponding liveness property.
\end{definition}

\forget{
The definition of bounded wait-freedom is as follows:

\begin{definition}\label{def:bounded wait-freedom}
Given an execution $t \in T(\llbracket \mathcal{L},n \rrbracket)$ and a bound $k$, $t$ satisfies bounded wait-freedom, if $t$ satisfies wait-freedom, and for each call action $t(i)$ of some process $proc$, there exists indexes $j \leq  k  +  1$ and $i'$, such that the index of $t(i)$ in $t \uparrow_{proc}$ is $i'$, and $(t \uparrow_{proc})(i'+j)$ is a matching return action (if $\vert t  \uparrow_{proc}  \vert \geq i'  +  k  +  1$).
Given a bound $k$, $\mathcal{L}$ satisfies bounded wait-freedom for $n$ processes, if each execution of $\llbracket \mathcal{L}, n \rrbracket$ satisfies bounded wait-freedom.
\end{definition}
}

%Let us introduce the notion of bounded wait-freedom and population-oblivious wait-freedom of \cite{DBLP:books/daglib/0020056}.
%An execution $t$ satisfies bounded wait-freedom \cite{DBLP:books/daglib/0020056}, if there is a bound on the number of steps a method call can take.
%An execution $t$ satisfies population-oblivious wait-freedom \cite{DBLP:books/daglib/0020056}, if it satisfies bounded wait-freedom, and the bound does not rely on the processes number.
%{\color {red}We define a notion called $k$-bounded wait-freedom,} which requires wait-freedom as well as each method should return within $k$ steps of the calling process.
%To verify bounded versions of wait-freedom, we define a notion called $k$-bounded wait-freedom.
%Let $t \uparrow_{pid}$ denotes the projection of execution $t$ into actions of process $pid$.
%An execution $t$ satisfies $k$-bounded wait-freedom, if it satisfies wait-freedom, and each method will return in $k$ steps of the calling process.
%Formally, $t$ satisfies $k$-bounded wait-freedom, if it satisfies wait-freedom, and for each call action $t(i)$ of process $pid$, there exists $j<k$ and $i'$, such that the index of $t(i)$ in $t \uparrow_{pid}$ is $i'$, and $l(i'+j)$ is a matching return action.
%$\llbracket \mathcal{L}, n \rrbracket$ satisfies $k$-bounded wait-freedom, if each of its execution satisfies $k$-bounded wait-freedom.


%Note that lock-freedom and wait-freedom requires fair scheduling, while obstruction-freedom does not require fair scheduling.

%TSO memory model associate with each process an FIFO buffer. One reasonable assumption is that in an infinite execution, if process $p$ do not launch new command after some time point, then its buffer should be eventually cleared. Based on this assumption, we propose a sub-notion of obstruction-freedom called fair-obstruction-freedom. An infinite execution $t$ is fair-obstruction-freedom, if it satisfies $isolate(t,p) \Rightarrow prog(t,p) \wedge fairBuf(t)$. Here predicate $fairBuf(t)$ holds, if for each process $p'$, either $\vert t \uparrow_{write(p')} \vert$ is infinite, or $\vert t \uparrow_{write(p')} \vert$ is finite and $\vert t \uparrow_{write(p')} \vert = \vert t \uparrow_{flush(p')} \vert$.

\forget{
Petrank \emph{et al.} \cite{DBLP:conf/pldi/PetrankMS09} demonstrate how to formalize lock-freedom, wait-freedom and obstruction-freedom as LTL formulas.
In Appendix \ref{sec:appendix discussion of section sec:liveness} we state how to formalize deadlock-freedom and starvation-freedom for $n$ processes as CTL$^*$ formulas.
As explained in Section \ref{subsec:equivlant characterization of obstruction-freedom}, %Since
concurrent system with $n$ processes on SC can be expressed as finite state LTS, and LTL and CTL$^*$ model checking is decidable %in polynomial space
\cite{DBLP:reference/mc/2018}, we thus obtain that the above five liveness properties are decidable for SC.
}

Petrank \emph{et al.} \cite{DBLP:conf/pldi/PetrankMS09} demonstrate how to formalize lock-freedom, wait-freedom and obstruction-freedom as LTL formulas.
It remains to show that deadlock-freedom and starvation-freedom for $n$ processes can be formalized
as %LTL
CTL$^*$ formulas.
Let $A$, $F$ and $G$ be the standard modalities in CTL$^*$.
%\redt{Let $A$ be the path quantifier of ``for all computation paths'', $F$ be the temporal operator of ``in the future'', and $G$ be the temporal operator of ``globally'' in CTL$^*$.}
Let $P_{\textit{ret}}$ be a predicate identifying return actions, %;
$P_{proc}$ be a
predicate identifying actions of process $proc$, and $P_{(r,proc)}$ %a the
the predicate identifying process $proc$'s
return actions. We define $\textit{fair}=(GF\ P_1) \wedge \ldots \wedge (GF\ P_n)$ to describe fair
executions of $n$ processes. Then, deadlock-freedom can be defined as the %LTL
CTL$^*$ formula $A (\textit{fair}
\rightarrow GF\ P_{\textit{ret}})$, and starvation-freedom can be defined as the %LTL
CTL$^*$ formula $A (\textit{fair} \rightarrow GF\ P_{(r,1)} \wedge \ldots \wedge GF\ P_{(r,n)})$.

As explained in Section \ref{subsec:equivlant characterization of obstruction-freedom}, %Since
concurrent system with $n$ processes on SC can be expressed as finite state LTS, and LTL and CTL$^*$ model checking is decidable %in polynomial space
\cite{DBLP:reference/mc/2018}, we can obtain that the above five liveness properties are decidable for SC.


%In Appendix \ref{sec:encoding liveness as temporal logic formaulas} we show that deadlock-freedom and starvation-freedom for $n$ processes can also be %expressed
%formalized as LTL formulas.}

%In Appendix \ref{sec:encoding liveness as temporal logic formaulas} we show that all the six liveness properties of this section for $n$ processes can be expressed as LTL formulas.

%%% Local Variables:
%%% mode: latex
%%% TeX-master: "FSTTCS2021.tex"
%%% End: 


\section{Undecidability of Four Liveness Properties} %Lock-Freedom and Wait-Freedom}
\label{sec:undecidability of lock-freedom and wait-freedom}

In this section we propose our undecidability proof of lock-freedom, wait-freedom, deadlock-freedom and starvation-freedom on TSO.

\forget{
In this section we propose our undecidability proof of lock-freedom,
wait-freedom, deadlock-freedom and starvation-freedom %for
on TSO.
We first introduce the notion of lossy channel machines, the cyclic post
correspondence problem (CPCP) \cite{DBLP:journals/acta/Ruohonen83}, and a known
result of Abdulla \emph{et al.} \cite{DBLP:journals/iandc/AbdullaJ96a} which
reduces checking CPCP into checking if a specific lossy channel machine has
infinite executions of a particular form.
%
Then, we generate a specific library for this lossy channel machine, and further
reduce checking %acceptance on
the existence of such executions of the lossy channel machines into checking
lock-freedom, deadlock-freedom, wait-freedom and starvation-freedom for this
library.
%Then, we generate two specific libraries for this lossy channel machine, and further reduce checking this lossy channel machine into checking lock-freedom and deadlock-freedom for one library, as well as wait-freedom and starvation-freedom for the other library.
}

\subsection{Perfect/Lossy Channel %system
Machines}
\label{subsec:perfect/lossy channel system}

A channel machine
\cite{DBLP:journals/iandc/AbdullaJ96a,DBLP:conf/popl/AtigBBM10} is a finite
control machine equipped with channels of unbounded size. It can perform send
and receive operations on its channels. A lossy channel machine is a channel
machine where arbitrarily many items in its channels may be lost
non-deterministically at any time and without any notification.

Let $\mathcal{CH}$ be the finite set of channel names and $\Sigma_{\mathcal{CH}}$ be a finite alphabet of channel contents.
The content of a channel is a finite sequence over $\Sigma_{\mathcal{CH}}$. A
channel operation is either a send operation $c!a$ sending the value $a$ over
channel $c$, a receive operation $c?a$ receiving $a$ over $c$, or a silent operation $nop$.
We associate with each channel operation a relation over words as follows:
%{\color{orange} GP: I think there is a type problem in the following sentence.}
Given $u \in \Sigma_{\mathcal{CH}}^*$, we have $\llbracket nop \rrbracket(u,u)$, $\llbracket c!a \rrbracket(u,a \cdot u)$ and $\llbracket c?a \rrbracket(u \cdot a,u)$.
%$\llbracket nop \rrbracket(u,u)$, $\llbracket c!a \rrbracket(u,a \cdot u(c))$ and $\llbracket c?a \rrbracket(u(c) \cdot a,u)$.
%\[
%  \llbracket nop \rrbracket(u,u)\qquad
%  \llbracket c!a \rrbracket(u,a \cdot u(c)) \qquad
%  \llbracket c?a \rrbracket(u(c) \cdot a,u)
%\]
\forget{
Let $\textit{Op}(\mathcal{CH})$ be the set of channel operations over $\mathcal{CH}$.
Given $u,u' \in \mathcal{CH} \rightarrow \Sigma_{\mathcal{CH}}^*$ two functions
that record the contents of each channel, we define the channel operation
relation, relating a channel before and after, as follows:\\[-5pt]
\[
  \llbracket nop \rrbracket(u,u)\qquad
  \llbracket c!a \rrbracket(u,u[c: a \cdot u(c)]) \qquad
  \llbracket c?a \rrbracket(u'[c: u'(c) \cdot a],u')
\]
}
% \begin{centerline}
%   \(
%   \begin{array}{ccc}
%     \llbracket nop \rrbracket(u,u') & \mathtt{if} & u=u'\\
%     \llbracket c!a \rrbracket(u,u') & \mathtt{if} & u'= u[c: a \cdot u(c)]\\
%     \llbracket c?a \rrbracket(u,u') & \mathtt{if} & u=u'[c: u'(c) \cdot a]
%   \end{array}
%   \)
% \end{centerline}\\
% we have $\llbracket nop \rrbracket(u,u')$ if $u=u'$, $\llbracket c!a \rrbracket(u,u')$ if $u'= u[c: a \cdot u(c)]$, and $\llbracket c?a \rrbracket(u,u')$ if $u=u'[c: u'(c) \cdot a]$.
A channel operation over a finite channel name set $\mathcal{CH}$ is a
mapping that associates, with each channel of $\mathcal{CH}$, a channel
operation. Let $\textit{Op}(\mathcal{CH})$ be the set of channel operations
over $\mathcal{CH}$. The relation of channel operations is extended to channel
operations over $\mathcal{CH}$ as follows: given a channel operation
$\textit{op}$ over $\mathcal{CH}$ and two functions $u,u' \in \mathcal{CH}
\rightarrow \Sigma_{\mathcal{CH}}^*$, we have $\llbracket \textit{op} \rrbracket(u,u')$, if $\llbracket \textit{op}(c) \rrbracket(u(c),u'(c))$ holds for each $c \in \mathcal{CH}$.

A $\textit{channel machine}$ is formally defined as a tuple $\textit{CM} = (Q,\mathcal{CH},\Sigma_{\mathcal{CH}},\Lambda,\Delta)$, where (1) $Q$ is a finite set of states, (2) $\mathcal{CH}$ is a finite set of channel names, (3) $\Sigma_{\mathcal{CH}}$ is a finite alphabet for channel contents, (4) $\Lambda$ is a finite set of transition labels, and (5) $\Delta \subseteq Q \times (\Lambda\cup\{\epsilon\}) \times \textit{Op}(\mathcal{CH}) \times Q$ is a finite set of transitions.
When $\textit{CM}$ is considered as a perfect channel machine, its semantics is defined as an LTS $(\textit{Conf}, \Lambda,\rightarrow,\textit{initConf})$.
A configuration of $\textit{Conf}$ is a pair $(q,u)$ where $q \in Q$ and $u:\mathcal{CH} \rightarrow \Sigma_{\mathcal{CH}}^*$. $\textit{initConf}$ is the initial configuration and all its channels are empty.
The transition relation $\rightarrow$ is defined as follows: given $q,q' \in Q$ and $u,u' \in \mathcal{CH} \rightarrow \Sigma_{\mathcal{CH}}^*$,
$(q,u) \overset{\alpha}{\longrightarrow} (q',u')$,
if there exists $op$, such that $(q,\alpha,\textit{op},q') \in \Delta$ and $\llbracket \textit{op} \rrbracket (u,u')$.
When $\textit{CM}$ is considered as a lossy channel machine, its semantics is defined as another LTS $(\textit{Conf}, \Lambda,\rightarrow',\textit{initConf})$, with transition relation $\rightarrow'$ defined as follows:
$(q,u) \overset{\alpha}{\longrightarrow}' (q',u')$,
if there exists $v,v' \in \mathcal{CH} \rightarrow \Sigma_{\mathcal{CH}}^*$,
such that (1) for each $c \in \mathcal{CH}$, $v(c)$ is a sub-word of $u(c)$, (2) $(q,v) \overset{\alpha}{\longrightarrow} (q',v')$ and (3) for each $c \in \mathcal{CH}$, $u'(c)$ is a sub-word of $v'(c)$.
Here a sequence $l_1 = a_1 \cdot \ldots \cdot a_u$ is a sub-word of another sequence $l_2=b_1 \cdot \ldots \cdot b_v$, if there exists $i_1 < \ldots < i_u$, such that $a_j=b_{i_j}$ for each $1 \leq j \leq u$.




\subsection{The Lossy Channel Machine for CPCP of Abdulla et al. \cite{DBLP:journals/iandc/AbdullaJ96a}}
\label{subsec:the lossy channel machine of CPCP in Abbdulla}

Given two sequences $l$ and $l'$, let $l =_c l'$ denote that there exists
sequences $l_1$ and $l_2$, such that $l=l_1 \cdot l_2$ and $l'=l_2 \cdot l_1$.
Given two finite sequences $\alpha_1,\ldots,\alpha_m$ and $\beta_1,\ldots,\beta_m$, where each
$\alpha_i$ and $\beta_i$ is a finite sequence over a finite alphabet, a solution of $\alpha_1,\ldots,\alpha_m$ and $\beta_1,\ldots,\beta_m$ is a nonempty sequence of
indices $i_1 \cdot \ldots \cdot i_k$, such that $\alpha_{i_1} \cdot \ldots \cdot \alpha_{i_k}$ $=_c$ $\beta_{i_1} \cdot \ldots \cdot \beta_{i_k}$.
The cyclic post correspondence problem (CPCP)
\cite{DBLP:journals/acta/Ruohonen83}, known to be undecidable, requires to answer given
$\alpha_1,\ldots,\alpha_m$ and $\beta_1,\ldots,\beta_m$, whether there exists one such solution.

Given two finite sequences $A = \alpha_1,\ldots,\alpha_m$ and $B = \beta_1,\ldots,\beta_m$ of finite sequences,
Abdulla \emph{et al.} \cite{DBLP:journals/iandc/AbdullaJ96a} generate the lossy channel machine $\textit{CM}_{(A,B)}$ %that is
shown in \figurename \ref{fig:M_{(A,B)}, original}.
Moreover, they prove that CPCP has a solution for $A$ and $B$, if and only if
$\textit{CM}_{(A,B)}$ has an infinite execution that visits state $s_1$ infinite times.
We point the readers to \cite{DBLP:journals/iandc/AbdullaJ96a} for
  an explanation on how %this channel machine
  $\textit{CM}_{(A,B)}$ solves CPCP.


\begin{figure}[tbp]
  \centering
  \includegraphics[width=0.5\textwidth]{PIC_M-CPCP.pdf}
%\vspace{-5pt}
  \caption{The lossy channel machine $\textit{CM}_{(A,B)}$.}
  \label{fig:M_{(A,B)}, original}
%\vspace{-10pt}
\end{figure}


%\gpnote{Either explain how the machine solves the CPCP or point the reader
%  to~\cite{DBLP:journals/iandc/AbdullaJ96a} for an explanation.}
$\textit{CM}_{(A,B)}$ contains two channels $c_1$ and $c_2$.
We use $c_1 ! \alpha_1$ to represent inserting the contents of $\alpha_1$ into $c_1$ one by one, and use $c_1 ? \beta_1$ to represent receiving the content of $\beta_1$ from $c_1$ one by one.
We use $c_1 ! \alpha_1 \cdot c_1 ? \beta_1$ to represent first do $c_1 ! \alpha_1$ and then do $c_1 ? \beta_1$.
Each execution of $\textit{CM}_{(A,B)}$ can be divided into (at most) two phases.
The first phase, called the guess phase, is a self-loop of state $s_0$, and is used to guess a solution of CPCP.
The second phase, called the check phase, goes from $s_0$ to $s_1$ and then repeatedly ``checks the content of $c_1$ and $c_2$''.
%When an execution comes to the second phase, its channel content is finite.
%{\color {red}Abdulla \emph{et al.} prove that, from each solution of CPCP, we could generate an execution of $M_{(A,B)}$ that visits $s_1$ infinite times.
%They also prove that, if there is an execution that visits $s_1$ infinitely many times, then according to Higman's Theorem \cite{Graham1952Ordering}, one could find a solution of CPCP.}
%To make the construction of the library of next subsection more clear, we assume that there is a special state $s_{\textit{trap}}$ in $\textit{CM}_{(A,B)}$.
%When the transitions of the second phase
%When any receive transition fails to receive the intended content, the execution goes to $s_{\textit{trap}}$ and stops there.
%Therefore, any possible cases has a transition rule.
%For clarity we do not draw $s_{\textit{trap}}$ and such transitions in \figurename \ref{fig:M_{(A,B)}, original}.

Based on $\textit{CM}_{(A,B)}$ we generate the lossy channel machine $\textit{CM}'_{(A,B)}$ which uses only one channel $c$ and works in a similar way.
To simulate one transition of $\textit{CM}_{(A,B)}$, $\textit{CM}'_{(A,B)}$ stores the content of $c_1$ followed by the content of $c_2$ (as well as new delimiter symbols) in its channel. Then it scans each symbol in its channel, modifies it (if necessary) and puts it back into its buffer, until the contents of $c_1$ and $c_2$ have all been dealt with.

\forget{
Based on $\textit{CM}_{(A,B)}$ we generate a lossy channel machine $\textit{CM}'_{(A,B)}$ which uses only one channel $c$ and works in a similar way as $\textit{CM}_{(A,B)}$. We introduce additional symbols $\bot_1$ and $\bot_2$. One transition of $\textit{CM}_{(A,B)}$ from a configuration $(q,u)$ corresponds to a sequence of transitions in $\textit{CM}'_{(A,B)}$ from configuration $(q,u')$, where $u'$ maps $c$ into $\bot_2 \cdot u(c_2) \cdot \bot_2 \cdot \bot_1 \cdot u(c_1) \cdot \bot_1$. In detail:
\begin{itemize}
\item[-] $c_1 ? a$ of $\textit{CM}_{(A,B)}$ corresponds to (1) do $c ? \bot_1 \cdot c ! \bot_1 \cdot c ? a$, then repeatedly receive non-$\bot_1$ data from channel $c$ and put it back into channel $c$, until receiving a $\bot_1$ from channel $c$ and put it back into channel $c$, and (2) do $c ? \bot_2 \cdot c ! \bot_2$, then repeatedly receive a non-$\bot_2$ data from channel $c$ and put it back into channel $c$, until receiving a $\bot_2$ and put it back into channel $c$.

\item[-] $c_1 ! a$ of $\textit{CM}_{(A,B)}$ corresponds to (1) do $c ? \bot_1 \cdot c ! \bot_1$, then repeatedly receive non-$\bot_1$ data from channel $c$ and put it back into channel $c$, until receiving a $\bot_1$ from channel $c$, and then do $c ! a \cdot c ! \bot_1$, the remaining part is the same as step (2) of $c_1 ? a$.

\item[-] The case of $c_2 ? a$ and $c_2 ! a$ is similar and we omit its detailed description here.
\end{itemize}
}

% The channel content of the initial configuration of $\textit{CM}'_{(A,B)}$ is $\bot_2 \cdot \bot_2 \cdot \bot_1 \cdot \bot_1$.
We could depict $\textit{CM}'_{(A,B)}$ similarly to \figurename~\ref{fig:M_{(A,B)},
  original}, and each transition of $\textit{CM}'_{(A,B)}$ is now a ``extended version transition''
as we discussed above. Therefore, there are ``$\textit{CM}'_{(A,B)}$'s versions'' of $s_0$, $s_1$ and
$s_{\textit{trap}}$, and when no confusion is possible we also call them $s_0$, $s_1$ and
$s_{\textit{trap}}$, respectively.
Note that if some new delimiter symbols are lost during transition, then such paths can not complete the simulation of one transition of $\textit{CM}_{(A,B)}$, and thus, do not influence the proof of the following lemma.
\forget{Note that items, including $\bot_1$ and
$\bot_2$, may be lost during transitions of $\textit{CM}'_{(A,B)}$. However,
such paths of $\textit{CM}'_{(A,B)}$ can not reach $s_1$, and thus they do not influence the proof of the following lemma.}
Based on above discussion, we reduce CPCP of $A$ and $B$ into an infinite execution problem of the lossy channel machine $\textit{CM}'_{(A,B)}$, as stated by the following lemma.

\begin{lemma}
  \label{lemma:reducing CPCP into a infinite execution problem of lossy channel machine M'{(A,B)}}
  There is a CPCP solution %between
  for sequences $A$ and $B$ of finite sequences, if and only if there is an infinite execution of $\textit{CM}'_{(A,B)}$ that visits $s_1$ infinitely often.
\end{lemma}

\forget{
We can generate a lossy channel machine $\textit{CM}'_{(A,B)}$ from $\textit{CM}_{(A,B)}$, such that $\textit{CM}'_{(A,B)}$ works in a similar way as $\textit{CM}_{(A,B)}$ and has only one channel $c$. $\textit{CM}'_{(A,B)}$ is generated from $\textit{CM}_{(A,B)}$ as follows: We introduce additional symbols $\bot_1$ and $\bot_2$, such that if the channel content of $c_1$ and $c_2$ is $l_1$ and $l_2$ in $\textit{CM}_{(A,B)}$, respectively, then the channel content of $c$ is $\bot_2 \cdot l_2 \cdot \bot_2 \cdot \bot_1 \cdot l_1 \cdot \bot_1$ in $\textit{CM}'_{(A,B)}$. Each execution of $\textit{CM}_{(A,B)}$ is transformed into an execution of $\textit{CM}'_{(A,B)}$ as follows:
In the beginning, $\textit{CM}'_{(A,B)}$ first puts %$\bot_2 \cdot \bot_2 \cdot \bot_1 \cdot \bot_1$ into $c$.
{\color {red}two $\bot_1$ and then two $\bot_2$ into $c$.}
Then, it translates each transition of $\textit{CM}_{(A,B)}$ as follows:

\begin{itemize}
\item[-] $c_1 ? a$ is translated into (1) do $c ? \bot_1 \cdot c ! \bot_1 \cdot c ? a$, then repeatedly receive non-$\bot_1$ data from channel $c$ and put it back into channel $c$, until receiving a $\bot_1$ from channel $c$ and put it back into channel $c$, and (2) do $c ? \bot_2 \cdot c ! \bot_2$, then repeatedly receive a non-$\bot_2$ data from channel $c$ and put it back into channel $c$, until receiving a $\bot_2$ and put it back into channel $c$.

\item[-] $c_1 ! a$ is translated into (1) do $c ? \bot_1 \cdot c ! \bot_1$, then repeatedly receive non-$\bot_1$ data from channel $c$ and put it back into channel $c$, until receiving a $\bot_1$ from channel $c$, and then do $c ! a \cdot c ! \bot_1$, the remaining part is the same as step (2) of $c_1 ? a$.

\item[-] The case of $c_2 ? a$ and $c_2 ! a$ is similar and we omit its detailed description here.
\end{itemize}


Therefore, we could draw $\textit{CM}'_{(A,B)}$ similarly as \figurename \ref{fig:M_{(A,B)},
  original}, and each transition of $\textit{CM}'_{(A,B)}$ is now a ``extended version transition''
as we discussed above. Therefore, there are ``$\textit{CM}'_{(A,B)}$'s version of $s_0$, $s_1$ and
$s_{\textit{trap}}$'', and when no confusion is possible we also call them $s_0$, $s_1$ and
$s_{\textit{trap}}$, respectively.
%Therefore, we can also assume that there are states $s_0$, $s_1$ and $s_{\textit{trap}}$ in $\textit{CM}'_{(A,B)}$.
The following lemma states that, the CPCP of %sequences
$A$ and $B$ can be reduced into an infinite execution problem of the lossy channel
machine $\textit{CM}'_{(A,B)}$. The proof is direct from
\cite{DBLP:journals/iandc/AbdullaJ96a} jointly with $\textit{CM}'_{(A,B)}$ and therefore
we omit it here.


\begin{lemma}
\label{lemma:reducing CPCP into a infinite execution problem of lossy channel machine M'{(A,B)}}
There is a CPCP solution %between
for sequences $A$ and $B$ of finite sequences, if and only if there is an infinite execution of $\textit{CM}'_{(A,B)}$ that visits $s_1$ infinitely often.
\end{lemma}
}

\forget{
The following lemma states that the CPCP of $A$ and $B$ can be reduced into an infinite execution problem of the lossy channel machine $\textit{CM}_{(A,B)}$. This lemma is directly obtained from %Theorem 3.6 of
\cite{DBLP:journals/iandc/AbdullaJ96a}.


\begin{lemma}
\label{lemma:reducing CPCP into a infinite execution problem of lossy channel machine M'{(A,B)}}
There is a CPCP solution for sequences $A$ and $B$ of finite sequences, if and only if there is an infinite execution of $\textit{CM}_{(A,B)}$ that visits $s_1$ infinitely often.
\end{lemma}
}



\subsection{Libraries for Four Liveness Properties}%Lock-Freedom and Deadlock-Freedom}
\label{subsec:concurrent data structures for lock-freedom and deadlock-freedom}

In this subsection, we propose our library $\mathcal{L}(A,B)$ that is generated from $\textit{CM}'_{(A,B)}$ %$\textit{CM}_{(A,B)}$
and simulates the executions of $\textit{CM}'_{(A,B)}$. %$\textit{CM}_{(A,B)}$.
This library contains two methods $M_1$ and $M_2$. Similarly to \cite{DBLP:conf/popl/AtigBBM10,DBLP:conf/atva/WangLW15}, we use the collaboration of two methods to simulate a lossy channel. %Our concurrent system contains two processes.
% The library executions simulating infinite executions of $\textit{CM}'_{(A,B)}$ require one process only run $M_1$ while another process only run $M_2$. Therefore, our library requires to ensure each method be fixed to some process.
Our library requires that each method be fixed to a single process when simulating infinite execution of $\textit{CM}'_{(A,B)}$.
Methods of our library work differently when simulating lossy channel machine transitions of different phases. %In the next subsection we will show that this reduces CPCP into violation of liveness properties.

Let us now explain in detail the construction of $\mathcal{L}(A,B)$.
$\mathcal{L}(A,B)$ uses %five
the following memory locations: %$x$, $y$,
$x_1$, $y_1$, $x_2$, $y_2$, $\textit{phase}$, $\textit{failSimu}$ and $\textit{firstM1}$.
$\textit{phase}$ stores the phase of $\textit{CM}'_{(A,B)}$, and its initial value is $\textit{guess}$.
$\textit{failSimu}$ is a flag indicating the failure of the simulation of $\textit{CM}'_{(A,B)}$, and its initial value is $\textit{false}$.
$\textit{firstM1}$ is used to indicate the first execution of $M_1$, and its initial value is $\textit{true}$.

The pseudo-code of $M_1$ and $M_2$ are shown in Algorithms \ref{Method1OfLockFreedom} and \ref{Method2OfLockFreedom}, respectively.
$\bot_s$ and $\bot_e$ are two new symbols not contained in $\textit{CM}'_{(A,B)}$.
For brevity, we use the following notations.
We use $\textit{writeOne}(x,a)$ to represent the sequence of commands writing $a$ followed by $\sharp$ into $x$.
We use $\textit{writeSeq}(x,a_1 \cdot \ldots \cdot a_k)$ to represent the sequence of commands writing $a_1 \cdot \sharp \cdot \ldots \cdot a_k \cdot \sharp$ into $x$.
$\sharp$ is a delimiter that ensures one update of a memory location will not be read twice.
\forget{For example, channel content $a \cdot b$ will be transformed into $(x,a) \cdot
(x,\sharp) \cdot (x,b) \cdot (x,\sharp)$ for some memory location $x$ in the store buffer.}
We use $v:=\textit{readOne}(x)$ to represent the sequence of commands reading $e$ followed by $\sharp$ from $x$ for some $e \neq \sharp$ and then assigning $e$ to $v$.
Moreover, if the values read do not correspond with $e$ followed by $\sharp$ we set $\textit{failSimu}$ to $\textit{true}$ and then let the current method return. This will terminate the simulation procedure.
Similarly, $v:=\textit{readRule}(x)$ reads a transition rule followed by $\sharp$ from $x$, and assign the rule to $v$.
We use $\textit{transportData}(z_1,z_2)$ to represent repeatedly using
$v=\textit{readOne}(z_1)$ to read an update of $z_1$ and using $\textit{writeOne}(z_2,v)$ to write it to $z_2$, until reading $\bot_e$ from $z_1$ and writing $\bot_e$ to $z_2$.
Given a transition rule $r$, let $valueRead(r)$ and $valueWritten(r)$ be the value received and sent by $r$, respectively. The symbols $s_0$ and $s_1$ in the pseudo-code of $M_1$ represent the corresponding state of $\textit{CM}_{(A,B)}$.

\redt{
\begin{algorithm}[t]
\KwIn {an arbitrary argument}
\While {$\textit{true}$} {
If $\textit{failSimu}$, then \KwRet; \\
\If {$\textit{firstM1}$}{
%if $\textit{cas}(\textit{exlM1},0,1)$ fails, then set $\textit{failSimu}$ to $\textit{true}$ and \KwRet; \\
guess a transition rule $r_1$ that starts from $s_0$; \\
$\textit{writeSeq}(x_1,r_1 \cdot \bot_s \cdot \bot_e)$; \\
$\textit{firstM1}=\textit{false}$; \\
}
\Else {
$r_1:=\textit{readRule}(y_2)$; \\
let $z_1:=valueRead(r_1)$ and $z_2:=valueWritten(r_1)$; \\
$\textit{readOne}(y_2,\bot_s)$; \\
if $z_1 \neq \epsilon$, then $\textit{readOne}(y_2,z_1)$; \\
guess a transition rule $r_2$ starts from the destination state of $r_1$; \\
$\textit{writeSeq}(x_1,r_2 \cdot \bot_s)$; \\
\While{$\textit{true}$}{
$tmp:=\textit{readOne}(y_2)$; \\
if $tmp=\bot_e$, then break; \\
$\textit{writeOne}(x_1,tmp)$; \\
}
$\textit{writeSeq}(x_1,z_2 \cdot \bot_e)$; \\
}
if $\textit{phase}=\textit{guess}$ and the destination state of $r_2$ is $s_1$, then set $\textit{phase}$ to $\textit{check}$; \\
$\textit{transportData}(y_1,x_2)$; \\
if $\textit{phase}=\textit{guess}$, then \KwRet; \\
}
\caption{$M_1$}
\label{Method1OfLockFreedom}
\end{algorithm}
}

\noindent\begin{algorithm}[!h]
\KwIn {an arbitrary argument}
%\If {$\textit{firstM2}$}{
%if $\textit{cas}(\textit{exlM2},0,1)$ fails, then set $\textit{failSimu}$ to $\textit{true}$ and \KwRet; %\\
%$\textit{firstM2}=\textit{false}$; \\
%}
\While {$\textit{true}$} {
if $\textit{failSimu}$, then \KwRet; \\
$\textit{transportData}(x_1,y_1)$; \\
$\textit{transportData}(x_2,y_2)$; \\
if $\textit{phase}=\textit{guess}$, then \KwRet; \\
}
\caption{$M_2$}
\label{Method2OfLockFreedom}
\end{algorithm}


\begin{figure}[tbp]
  \centering
  \includegraphics[width=1.0\textwidth]{PIC_SimluateFiniteTimesForLockFree.pdf}
%\vspace{-1pt}
  \caption{One execution of $\mathcal{L}(A,B)$.
  %Execution of $\mathcal{L}(A,B)$ for an execution of $M'_{(A,B)}$ that has one round of check successes and one round of check fails. Here each transition rule $r_i$ is of the form
  %$(q_{\textit{i-1}},\_,\_,q_i,\_)$.
  }
  \label{fig:execution for lock-freedom with one check success and one check fail}
%\vspace{-10pt}
\end{figure}

\figurename~\ref{fig:execution for lock-freedom with one check success and one
  check fail} illustrates a possible execution of $\mathcal{L}(A,B)$.
$M_1$ and $M_2$ work differently in the different phases. In the guess phase, $M_1$ and $M_2$ return after simulating one lossy channel machine transition, while in the check phase, $M_1$ and $M_2$ keep working until the simulation procedure fails.
% In this case the guess phase expands over several returns of $M_1$ and $M_2$, and then the check phase successfully simulates several lossy channel machine transition. This is the long spanning call to $M_1$ and $M_2$.

%\redt{Both $M_1$ and $M_2$ work differently in different phases. In the guess phase, $M_1$ (resp., $M_2$) returns after simulating one lossy channel machine transition, while in the check phase, $M_1$ (resp., $M_2$) keeps working until the simulation procedure fails.
%}





Assume that $M_1$ (resp., $M_2$) runs on process $P_1$ (resp., process $P_2$). %and $M_2$ runs on process $P_2$.
%Assume that the current channel content of $\textit{CM}'_{(A,B)}$ is $l$. Before simulating one lossy channel machine transition, $r_1 \cdot \bot_s \cdot l \cdot \bot_e$ are stored in $P_2$'s store buffer as buffered items of memory location $y_2$, where $r_1$ is a transition rule of $\textit{CM}'_{(A,B)}$, and $\bot_s$ and $\bot_e$ are additional symbols indicating the start and end of channel content of $\textit{CM}'_{(A,B)}$, respectively.
To simulate one lossy channel machine transition with channel content $l$, we first store $r_1 \cdot \bot_s \cdot l \cdot \bot_e$ in process $P_2$'s store buffer as buffered items of %memory location
$y_2$, where $r_1$ is the transition rule of this transition, and $\bot_s$ and $\bot_e$ are additional symbols indicating the start and end of channel content of $\textit{CM}'_{(A,B)}$, respectively. Then, the procedure for simulating one lossy channel machine transition is as follows:

%the channel content of $\textit{CM}'_{(A,B)}$ and a transition rule of the next transition are stored in $P_2$'s store buffer as buffered items of memory location $y_2$.
%The procedure for simulating one lossy channel machine transition is as follows:
\begin{itemize}
\item $M_1$ reads the transition rule $r_1$ and channel content of $\textit{CM}'_{(A,B)}$ by reading all the updates of $y_2$. After reading $r_1$, $M_2$ non-deterministically chooses a transition rule $r_2$ of the lossy channel machine. Such rule should begin from the destination state of $r_1$.

\item There are four points for information update and transfer
between $M_1$ and $M_2$: (1) According to transition rule $r_1$, $M_1$ modifies and writes all the updates of $y_2$ into %a memory location
    $x_1$, (2) then $M_2$ reads all the updates of $x_1$ and writes all the updates into %a memory location
    $y_1$, (3) then $M_1$ reads all the updates of $y_1$ and writes all the updates into %a memory location
    $x_2$, and (4) finally, $M_2$ reads all the updates of $x_2$ and writes all the updates into $y_2$. To read all the updates of a memory location, we need to repeatedly read until read $\bot_e$, which indicates the end of channel content.

    Since there is no item in the buffer at the beginning of the simulation procedure, to simulate the first lossy channel machine transition $M_1$ directly writes $r \cdot \sharp \cdot \bot_s \cdot \sharp \cdot \bot_e \cdot \sharp$ to $x_1$ and does not need to read updates from $y_2$, where $r$ is a transition rule from $s_0$. This is the reason for using $\textit{firstM1}$.

\item $M_1$ is also responsible for modifying the phase (stored in the memory location $\textit{phase}$). If the last lossy channel machine transition simulated belongs to the guess phase and the destination state of $r_1$ is $s_1$, $M_1$ changes the memory location $\textit{phase}$ to check.
\end{itemize}
%This completes the simulation procedure of one lossy channel machine transition.
%To ensure that one update of a memory location $x$ will not be read twice, we use a symbol $\sharp$ as delimiter. For example, channel content $a \cdot b$ will be transformed into $(x,a) \cdot (x,\sharp) \cdot (x,b) \cdot (x,\sharp)$ in buffer.
%Any failure of simulation procedure will set a memory location $\textit{failSimu}$ to $\textit{true}$ and finishes simulation. %At the beginning of our simulation procedure, there is no items in process $P_2$'s buffer. Therefore, to simulate the first lossy channel machine transition $M_1$ directly writes $r \cdot \sharp \cdot \bot_s \cdot \sharp \cdot \bot_e \cdot \sharp$ to $x_1$ and do not need to read updates from $y_2$, where $r$ is a transition rule from $s_0$.

%Let us explain how our library simulate one transition of lossy channel machine $\textit{CM}'_{(A,B)}$.
The reason why we need to update and transfer information between $M_1$ and $M_2$
is to deal with the case when an update of $\bot_e$ %for $y_2$
is not captured. %by $M_1$.
Let us first consider a simple but infeasible solution: $M_1$ reads updates from
$y_2$ (until reading $\bot_e$), and modifies and writes all the updates into
$x_1$; while $M_2$ repeatedly reads an update of $x_1$ and writes it into $y_2$,
until reading $\bot_e$. This solution can not deal with the case when updates of
$\bot_e$ for $y_2$ is not seen by $M_1$, and will make $M_1$ and $M_2$ fall into
infinite loop that violates liveness.
This may happen
in simulating each lossy channel machine transition, and thus, introduces ``false negatives'' to four liveness properties.
%This may happen in simulating each lossy channel machine transition. %, no matter whether there is a solution for CPCP or not.
To deal with this case, we need to break the infinite loop and avoid directly
writing the updates of $y_2$ back into $x_1$. Instead, in our update and
transfer points, we exhaust the updates of $y_2$, which are written to $x_1$,
and later written to $y_1$ instead of $y_2$. Therefore, there is no infinite
loop even if updates of $\bot_e$ for $y_2$ are lost.

%\redt{To fix each method to some process, we need to ensure that the executions of the following two cases can not be used to simulate infinite executions of the lossy channel machine $\textit{CM}'_{(A,B)}$. In the first case, the first method of the two processes are both $M_1$, or are both $M_2$. In the second case, some process previously runs $M_1$ (resp., $M_2$) and then runs $M_2$ (resp., $M_1$). There is a memory location $\textit{failSimu}$ as flag for failed simulation, and executions of both cases will set this flag to $\textit{true}$.
%}

%\redt{Let us explain how to deal with the first case. A memory location $\textit{firstM1}$ is used to indicate the first invocation of $M_1$. If $\textit{firstM1}$ is $\textit{true}$, then $M_1$ attempts to use $\textit{cas}$ commands to set a memory location $\textit{exlM1}$ from $0$ to $1$. If this attempt successes, then $\textit{firstM1}$ is set to $\textit{false}$; otherwise, $\textit{failSimu}$ is set to $\textit{false}$ and the simulation ends, since the fail of this attempt results from a concurrent $M_1$. $\textit{exlM1}$ is used to ensure the exclusive of the first $M_1$. The case for $M_2$ is similar.
%}

Assume that we can successfully simulate one transition of
$\textit{CM}'_{(A,B)}$ with one $M_1$ running on process $P_1$ and one $M_2$
running on process $P_2$. Then the most general client on process $P_1$
(resp., on process $P_2$) can call $M_1$ and $M_2$. Perhaps surprisingly, the
only possible way to simulate the second transition of $\textit{CM}'_{(A,B)}$
is to let $M_1$ and $M_2$ to continue to run on processes $P_1$ and $P_2$, respectively. Let us explain why other choices fail to simulate the second transition: (1) If both processes run method $M_1$, then they both %write $x_1$ and
require reading updates of $y_1$. Since there is no buffered item for $y_1$, and none of them write to $y_1$, both $M_1$ fail the simulation.
% the only possible buffered $y_2$ items are in process $P_2$'s buffer, according to the TSO memory model, $M_1$ on process $P_2$ always read a same value of $y_2$ and thus fails to do $\textit{readOne}(y_2,\_)$. Thus, this $M_1$ fails the simulation.
% , but neither of them writes to $y_2$. Thus, both $M_1$ fails simulation.
(2) If both processes run $M_2$, we arrive at a similar situation. (3) If $M_1$
and $M_2$ run on processes $P_2$ and $P_1$, respectively. $M_1$ requires reading
the updates on $y_2$, and the only possible buffered $y_2$ items are in process $P_2$'s buffer. According to the TSO memory model, $M_1$ always reads the same value for $y_2$ and thus fails to do $\textit{readOne}(y_2,\_)$. Thus, $M_1$ fails the simulation.
% Similarly, $M_1$ fails the simulation.
Therefore, we essentially ``fix methods to processes'' without adding specific commands for checking process id.

%Let us now explain how to associate each method to a single process. Two concurrent $M_1$s can not simulate lossy channel machine transitions, since they both write $x_1$ and require reading updates from $y_2$, but neither of them writes to $y_2$. Similarly, two concurrent $M_2$s can not simulate lossy channel machine transitions. Moreover, suppose that $M_1$ and $M_2$ run on processes $P_1$ and $P_2$, respectively; and later $M_1$ and $M_2$ run on processes $P_2$ and $P_1$, respectively; in this case they can not simulate lossy channel machine transitions.
%
%The reason is as follows: after the execution of the first $M_1$ and $M_2$, the buffered items of $y_2$ are in process $P_2$'s buffer. According to the TSO memory model, the second $M_1$ can only read the newest value of $y_2$ and can not read other items of $y_2$ in its buffer, and thus, fails to read the updates from $y_2$.










\forget{
\noindent\begin{algorithm}[!h]
\KwIn {two memory location $z_1$ and $z_2$}
$tmp:=\textit{readOne}(z_1)$; \\
\While{$tmp \neq \bot_e$}{
$\textit{writeOne}(z_2,tmp)$; \\
$tmp:=\textit{readOne}(z_1)$; \\
}
$\textit{writeOne}(z_2,tmp)$; \\
\caption{$\textit{TransportData}(y,u)$}
\label{Method5OfLockFreedom}
\end{algorithm}
}

\forget{
This library contains five methods $M_1$, $M_2$, $M_3$, $M_4$ and $M_5$.
Similarly to \cite{DBLP:conf/popl/AtigBBM10}, we use the collaboration of two
methods to simulate each lossy channel. Since $\mathcal{L}(A,B)$ has two
channels $c_1$ and $c_2$, we use the collaboration of $M_1$ and $M_2$ to
simulate $c_1$, and use the collaboration of $M_3$ and $M_4$ to simulate $c_2$.
$M_5$ is used to determine the following lossy channel machine transition rule
to be simulated, communicates to $M_1$, $M_2$, $M_3$ and $M_4$ the current phase of
$\textit{CM}_{(A,B)}$, and synchronizes with $M_1$ and $M_3$. We assume that
each $M_i$ ($1 \leq i \leq 5$) runs on its own process $i$, and use the
$\textit{checkPID}$ command to ensure that a $M_i$ runs on process $i$,
otherwise returning immediately. Hence, in the case there $\textit{checkPID}$
fails, on $M_i$, this method call does not influence the simulation of the lossy
channel machine.


The procedure for simulating one lossy channel machine transition is as
follows:
\begin{itemize}
\item First, $M_5$ non-deterministically chooses a transition rule of the
  lossy channel machine and stores it in the memory location $\textit{rule}$. Such
  rule should begin from the destination state of the rule of the last lossy
  channel machine transition simulated by our library. $M_5$ is also
  responsible for modifying the phase (stored in the memory location
  $\textit{phase}$). If the last lossy channel machine transition simulated
  belongs to the guess phase and the destination state of $\textit{rule}$ is
  $s_1$, $M_5$ changes the memory location $\textit{phase}$ to check.
\item Then, $M_1$ (resp., $M_3$) reads the updated transition rule. $M_1$ and
  $M_2$ (resp., $M_3$ and $M_4$) collaborate to simulate the channel operation
  of $c_1$ (resp., of $c_2$) of this transition rule (if any).
\item After that, $M_1$ (resp., $M_3$) writes $1$ into a memory location
  $\textit{ack1}$ (resp., $\textit{ack2}$), which notifies to $M_5$ the end of
  one channel operation. $M_5$ waits until receiving the updated
  $\textit{ack1}$ and $\textit{ack2}$.
\end{itemize}
This completes the simulation procedure of one lossy channel machine
transition.




\begin{algorithm}[t]
\KwIn {an arbitrary argument}
%If $\textit{getPID}() \neq 1$ then \KwRet; \\
If not $\textit{checkPID}(1)$ then \KwRet; \\
%If $tflag = true$ then \KwRet; \\
%guess a transition rule $(s_0,\_,\_,\_,\_) \in \Delta$ and assign it to $rule$; \\
%{\color {red}non-deterministically choose a transition rule $(s_0,\_,\_,\_,\_) \in \Delta$ and assign it to $rule$; \\}
\While {$\textit{tflag}=false$} {%true} {
%If $tflag = true$ then \KwRet; \\
do $r:=\textit{readOne}(\textit{rule})$ and reads a rule in $\Delta$ (Otherwise, \KwRet); \\
let $u_1:=valueRead(r,c_1)$ and $v_1:=valueWritten(r,c_1)$; \\
If $u_1 \neq \epsilon$ then $\textit{readOne}(y_1,u_1)$; \\
If $v_1 \neq \epsilon$ then $\textit{WriteOne}(x_1,v_1)$; \\
$\textit{writeOne}(\textit{ack1},1)$; \\
%If $dst(rule) = s_1$ and $phase=guess$ then set $phase$ to $check$; \\
%If $dst(rule) = s_{\textit{trap}}$ then set $tflag$ to $true$ and \KwRet; \\
%let $rule$ be a random transition rule $(dst(rule),\_,\_,\_,\_) \in \Delta$;\\
%{\color {red}non-deterministically let $rule$ be a transition rule $(dst(rule),\_,\_,\_,\_) \in \Delta$;\\}
If $\textit{phase}$ = guess then \KwRet; \\
}
\KwRet; \\
\caption{$M_1$}
\label{Method1OfLockFreedom}
\end{algorithm}


\noindent\begin{algorithm}[!h]
\KwIn {an arbitrary argument}
%If $\textit{getPID}() \neq 2$ then \KwRet; \\
If not $\textit{checkPID}(2)$ then \KwRet; \\
%If $tflag=true$ then \KwRet; \\
\While {$\textit{tflag}=false$} {%true} {
$tmp:=\textit{readOne}(x_1)$; \\
$\textit{writeOne}(y_1,tmp)$; \\
If $\textit{phase}$=guess then \KwRet; \\
}
%\If {$phase=guess$} {
%$tmp:=\textit{readOne}(x)$; \\
%$\textit{writeOne}(y,tmp)$; \\
%\KwRet;
%}
%\Else {
%\While {$tflag=false$} {
%$tmp:=\textit{readOne}(x)$; \\
%$\textit{writeOne}(y,tmp)$; \\
%}
%}
\KwRet;
\caption{$M_2$}
\label{Method2OfLockFreedom}
\end{algorithm}



\noindent\begin{algorithm}[!h]
\KwIn {an arbitrary argument}
If not $\textit{checkPID}(5)$ then \KwRet; \\
If $tflag=true$ then \KwRet; \\
\If {$\textit{rule}=\bot$} {
non-deterministically choose a transition rule $r=(s_0,\_,\_,\_,s',\_,\_) \in \Delta$ for some lossy channel machine state $s'$, and do $\textit{writeOne}(\textit{rule},r)$; \\
If $\textit{phase}=guess \wedge s'=s_1$ then set $phase$ to $check$; \\
%If $s'=s_{\textit{trap}}$ then set $tflag$ to $true$ and \KwRet; \\
$\textit{readOne}(\textit{ack1},1)$; \\
$\textit{readOne}(\textit{ack2},1)$; \\
\KwRet; \\
}
\Else {
\While{$\textit{tfalg}=false$}{
assume that $\textit{rule}=(s,u_1,u_2,\alpha,s',v_1,v_2)$ for some lossy channel machine states $s$ and $s'$; \\
non-deterministically choose a transition rule $r=(s',\_,\_,\_,s'',\_,\_) \in \Delta$ for some lossy channel machine state $s''$ and do $\textit{writeOne}(\textit{rule},r)$;\\
If $\textit{phase}=guess \wedge s'=s_1$ then set $phase$ to $check$; \\
%If $s'=s_{\textit{trap}}$ then set $tflag$ to $true$ and \KwRet; \\
$\textit{readOne}(\textit{ack1},1)$; \\
$\textit{readOne}(\textit{ack2},1)$; \\
If $\textit{phase}$=guess then  \KwRet; \\
}
\KwRet; \\
}
\caption{$M_5$}
\label{Method5OfLockFreedom}
\end{algorithm}




Throughout, $M_1$ reads values from memory location $y_1$, and
writes values into memory location $x_1$; while $M_2$ reads values from $x_1$
and writes values into $y_1$. We take the channel contents of $c_1$ to be the
buffered values of $x_1$ in the store buffer of process $1$ concatenated with the
buffered values of $y_1$ in the store buffer of process $2$.
% $M_1$ contains a loop, and it simulates one channel operation (and synchronizes with $M_5$) in each round of the loop.
To simulate a send operation $c_1!a$, $M_1$ writes $a$ to $x_1$, while to
simulate a receive operation $c_1?a$, $M_1$ reads $a$ from $y_1$. Both $M_1$
and $M_2$ work differently depending on the phase. In the guess phase, $M_1$
returns after it reads an updated rule, simulates one channel operation and
synchronizes with $M_5$; and $M_2$ returns after it reads a value from $x_1$
and writes it back into $y_1$. In the check phase, $M_1$ repeatedly reads an
updated rule, simulates one channel operation and synchronizes with $M_5$,
until the simulation of $\textit{CM}_{(A,B)}$ %finishes or
fails; while $M_2$ repeatedly reads values from $x_1$ and writes them into
$y_1$, until the simulation of $\textit{CM}_{(A,B)}$ %finishes or
fails.
%$M_2$ works in two phases as follows: In a first guess phase, $M_2$ returns after it reads a value from $x_1$ and writes it back into $y_1$. In a second check phase, $M_2$ repeatedly reads values from $x_1$ and writes them into $y_1$, until the simulation of $\textit{CM}_{(A,B)}$ finishes or fails.
%Thus, the buffered values of $y$ in the store buffer of process $2$ are the ``oldest contents of channel $c$''.
In this way, we simulate the non-lossy part of channel operations of $c_1$. The
case when $M_1$ (resp., $M_2$) misses one or more updates of $y_1$ (resp. of
$x_1$) simulates the loss of information in channel. In this way, we simulate
the channel operations on lossy channel $c_1$.
%
Similarly, $M_3$ and $M_4$ use memory locations $x_2$ and $y_2$ to simulate
channel operations on the lossy channel $c_2$, and work differently depending on the phase.
%
$M_5$ also operates differently depending on the phases. In the guess phase,
each call to $M_5$ returns after it updates $\textit{rule}$ (and possibly
$\textit{phase}$), and reads the updates from $\textit{ack1}$ and
$\textit{ack2}$. In the check phase, $M_5$ repeatedly updates $\textit{rule}$ %(and possibly $\textit{phase}$)
and reads the updates from $\textit{ack1}$ and
$\textit{ack2}$, until the simulation of $\textit{CM}_{(A,B)}$ %finishes or
fails.







\forget{
$M_1$ reads values from {\color {red}a memory location} %a global variable
$y$, and writes values into {\color {red}a memory location} %a global
%variable %to
$x$; while $M_2$ reads from $x$ and writes into $y$.
{\color {red}We take the channel contents of $\textit{CM}'_{(A,B)}$ to be %the values buffered in the store buffer of $x$ for process 1.
the buffered values of $x$ in the store buffer of process $1$.
}
%$M_1$ works as follows:
$M_1$ contains a loop, and {\color {red}it simulates one lossy channel machine transition %channel operation
in each
round of the loop.} {\color {red}The procedure of simulating one lossy channel machine transition is as follows: First, $M_1$ non-deterministically chooses a transition rule of the lossy channel machine and stores it in memory location $\textit{rule}$. Such rule should begin from the destination state of the rule of the last lossy channel machine transition simulated by $M_1$. $M_1$ then simulate channel operation of this transition rule.} %A silent operation is simulated by $M_1$ changing the control state of $\textit{CM}'_{(A,B)}$.
{\color {red}A send operation $c!a$ (resp., receive operation $c?a$) %additionally
require $M_1$ writing $a$ to $x$ (resp., reading $a$ from $y$).}
\forget{
$M_1$ stores {\color {red}the current transition rule that is being simulated in a variable $\textit{rule}$.} %the control state
{\color{orange} GP: what does this
  mean?} %{\color{red} (in the form of storing transition rule)} of $\textit{CM}'_{(A,B)}$.
  We take the channel contents of $\textit{CM}'_{(A,B)}$
to be {\color {red}the values buffered in the store buffer of $x$ for process
1}.\footnote{Notice that %since $M_1$ only writes to $x$, all buffer contents are to $x$.
{\color{red}$M_1$ writes to $x$, $\textit{rule}$, and possibly $\textit{tflag}$.}} $M_1$ contains a loop, and it simulates one channel operation in each
round of the loop. A silent operation is simulated by $M_1$ changing the control
state of $\textit{CM}'_{(A,B)}$. A send operation $c!a$ (resp., receive
operation $c?a$) additionally require $M_1$ writing $a$ to $x$ (resp., reading
$a$ from $y$).
}
%
$M_2$ works in two phases as follows:
In a first  guess phase, $M_2$ returns after it reads a value
from $x$ and writes it back into $y$.
In a second check phase, $M_2$ repeatedly reads values from $x$ and writes them into
$y$, until the simulation of $\textit{CM}'_{(A,B)}$ finishes or fails.
Thus, the buffered values of $y$ in the store buffer of process $2$ are the ``oldest
contents of channel $c$''.
%
In this way, we simulate the non-lossy part of $\textit{CM}'_{(A,B)}$. The
case when $M_1$ (resp., $M_2$) misses one or more updates of $y$ (resp. of $x$)
simulates the loss of information in channel.
%
In this way, we simulate the lossy channel machine $\textit{CM}'_{(A,B)}$.
}

\forget{
  $M_1$ works as follows: initially $M_1$ writes $s_0 \cdot \bot_s \cdot \bot_e$
  into the buffer and randomly guesses a transition to simulate %the first step of the lossy channel machine transition.
  the first lossy channel machine transition.
  This represents that the initial ``channel content'' of $\textit{CM}'_{(A,B)}$ is empty.
  Here $\bot_s$ and $\bot_e$ are additional symbols, where $\bot_s$ indicates the
  start of channel content of $\textit{CM}'_{(A,B)}$, and $\bot_e$ indicates the end of the
  channel content of the $\textit{CM}'_{(A,B)}$.
  The guessed transition rule is stored in a predefined memory location $rule$.
  Then, $M_1$ begins a loop, where each round of the loop simulates %one step of the lossy channel machine.
  one transition of the lossy channel machine.
  In each round, $M_1$ ``modifies the channel content'' according to the
  transition rule stored in $rule$. At the end of each round, $M_1$ finishes %simulating this step of the lossy channel machine
  simulating this lossy channel machine transition and randomly guesses a transition rule for the next %step of the lossy channel machine.
  transition of the lossy channel machine. $M_2$ works as follows: in the guess phase, $M_2$ returns after it reads an element from $x$ and writes
  it into $y$. In the check phase, $M_2$ repeatedly reads elements from $x$ and writes them into $y$.
}


\forget{
At the beginning of execution, $M_1$ writes $s_0 \cdot \bot_s \cdot \bot_e$ into
$x$. This represents that the initial ``channel content'' of $M'_{(A,B)}$ is empty.
Here $\bot_s$ and $\bot_e$ are additional symbols, where $\bot_s$ indicates the
start of channel content of $M'_{(A,B)}$, and $\bot_e$ indicates the end of
channel content of $M'_{(A,B)}$. Additionally, $M_1$ is responsible for choosing
a transition rule (stored in a predefined memory location $rule$) and changing
the channel content according to the transition rule stored therein.
}

\forget{
If the lossy channel machine execution that is simulated is of infinite length,
in a %successfully simulated library execution on this lossy channel machine,
library execution that successfully simulates this lossy channel machine execution,
$M_1$ does not return, while $M_2$ returns finitely many times.
This makes the library execution still lock-free if the lossy channel machine
execution infinitely loops in the guess phase, and makes the library execution
not lock-free %(that is locked)
if the lossy channel machine infinitely loops in
the check phase.
}

Distinguishing the different phases of $M_1$, $M_2$, $M_3$, $M_4$
and $M_5$ enables us to encode lock-freedom. Given
a library execution $t_1$ which successfully simulates an infinite lossy
channel machine execution $t_2$ that infinitely loops in the guess phase,
according to the behavior of $M_1$, $M_2$, $M_3$, $M_4$ and $M_5$ in the guess
phase, we can see that $t_1$ is lock-free and contains infinite number of
return actions. %of $M_1$, $M_2$, $M_3$, $M_4$ and $M_5$.
If such $t_2$ infinitely loops in the check phase, since $M_1$, $M_2$, $M_3$,
$M_4$ and $M_5$ never return in the check phase, if the simulation continues
indefinitely, we can see that $t_1$ violates lock-freedom since none of $M_1$,
$M_2$, $M_3$, $M_4$ or $M_5$ returns.
%{\color {orange} GP: the rest of this paragraph is not very clear to me.}
Thus, we relate the existence of solutions of CPCP with existence of lock-freedom violations. %Since $t_2$ visits $s_1$ infinitely many times if and only if $t_2$ loops infinitely in the check phase, we can relate the CPCP problem with lock-freedom when we successfully simulate infinite executions of the lossy channel machine.
%
On the other hand, we have to make every execution that fails to simulate the lossy
channel machine satisfy lock-freedom.
%
To that end, we set a flag $tflag$ (for terminate), upon which $M_1$, $M_2$, $M_3$, $M_4$ and
$M_5$ return trivially.

\forget{
Distinguishing the different phases of $M_2$ enables us to {\color {red}encode lock-freedom.}
{\color {red}Given a library execution $t_1$ which successfully simulates an infinite lossy
channel machine execution $t_2$, since (1) each transition of $\textit{CM}_{(A,B)}$ is simulated by $\textit{CM}'_{(A,B)}$ with several send and receive operations, and (2) in guess phase, each receive operation requires at least one $M_2$, we can see that $t_1$ is lock-free if $t_2$ infinitely loops in
the guess phase.} {\color {red}However}, $t_1$ is not lock-free if $t_2$ infinitely loops in the check phase {\color {red}and visits $s_1$ infinitely many times.}
%{\color {red}Since each transition of $\textit{CM}_{(A,B)}$ is simulated by $\textit{CM}'_{(A,B)}$ with several send and receive operations, and in check phase each receive operation requires a $M_2$},
%Given a library execution $t_1$ which successfully simulates an infinite lossy channel machine execution $t_2$, $t_1$ is lock-free if $t_2$ infinitely loops in the guess phase, and it is not lock-free if $t_2$ infinitely loops in the check phase {\color {red}and visits $s_1$ infinitely many times}.
%
%{\color {orange} GP: the rest of this paragraph is not very clear to me.}
{\color {red}Thus, we relate the existence of solutions of CPCP with existence of lock-freedom violations.}%Since $t_2$ visits $s_1$ infinitely many times if and only if $t_2$ loops infinitely in the check phase, we can relate the CPCP problem with lock-freedom when we successfully simulate infinite executions of the lossy channel machine.
%
On the other hand, we should make every execution that either %(1)
simulates a finite lossy channel machine execution {\color {red}end in $s_{\textit{trap}}$},
or %(2)
fails to simulate the lossy channel machine transitions to satisfy
lock-freedom. In any of these two cases we set a flag $tflag$ and subsequently
$M_1$ and $M_2$ return trivially.}

\figurename \ref{fig:execution for lock-freedom with one check success and one
  check fail} illustrates a possible execution of $\mathcal{L}(A,B)$. In this
case the guess phase expands over several returns of $M_1$, $M_2$, $M_3$, $M_4$
and $M_5$. We specifically draw how $M_1$, $M_3$ and $M_5$ communicate to
simulate the first lossy channel machine transition. Here we assume that the
first lossy channel machine transition does a send operation on channel $c_2$
and thus, $M_3$ writes $x_2$. Then, $M_4$ reads from $x_2$ and writes to $y_2$.
Note that since there is no write to $x_1$, $M_2$ does no work when simulating
the first lossy channel machine transition. The check phase is the long spanning
set of calls to $M_1$, $M_2$, $M_3$, $M_4$ and $M_5$. The check phase
successfully simulates one lossy channel machine transition, and fails when
simulating the second lossy channel machine transition. After this point $M_1$,
$M_2$, $M_3$, $M_4$ and $M_5$ return trivially, represented with the short
method calls at the end.

\forget{
\figurename \ref{fig:execution for lock-freedom with one check success and one
  check fail} illustrates a possible execution of $\mathcal{L}(A,B)$. In this
case the guess phase expands over several returns of $M_2$, and then the check
phase successfully simulates one lossy channel machine transition. This is the
long spanning call to $M_2$. The execution illustrates a later fail when
simulating the second lossy channel machine transition, and after this point both
$M_1$ and $M_2$ return trivially, represented with the short method calls at the
end.}

\forget{
If a library execution successfully simulates an infinite lossy channel machine execution, then the library execution is lock-free if the lossy channel machine execution infinitely loops in the guess phase, and is not lock-free if the lossy channel machine execution infinitely loops in the check phase.
Note that the set of infinite executions of $\textit{CM}'_{(A,B)}$
that visits $s_1$ infinitely many times is just the set of infinite executions
of $\textit{CM}'_{(A,B)}$ that loops infinitely in the check phase.
If the lossy channel machine execution that is simulated is an finite execution,
it will end in state $s_{\textit{trap}}$.
Any library execution which reaches $s_{\textit{trap}}$ should finish the
simulation.
To ensure such %library
execution is still lock-free, we make $M_1$ and $M_2$ only able to
trivially return after we reach $s_{\textit{trap}}$.

On the other hand, we should make every execution that fails to simulate the lossy
channel machine transitions satisfies lock-freedom.
When we find that the simulation procedure fails, we %force $M_1$ and $M_2$ to trivially return.
force $M_1$ and $M_2$ to trivially return after that point.
\figurename \ref{fig:execution for lock-freedom with one check success and one
  check fail} shows an execution of $\mathcal{L}(A,B)$. It finishes its guess
phase with several returns of $M_2$, and in the check phase it successfully
simulates one lossy channel machine transition, and it fails when simulating the
second lossy channel machine transition.
%it pass one round, and fails in the second round.
After this point, both $M_1$ and $M_2$ result in a method that trivially return.
}


%Similar to \cite{DBLP:journals/iandc/AbdullaJ96a} and our previous work \cite{DBLP:conf/atva/WangLW15}, we use the collaboration of two processes to simulate one lossy channel. Process $P_1$ repeatedly reads a value from $y$ and then writes it into $x$, while process $P_2$ repeatedly reads a value from $x$ and then writes it into $y$. The possible case when process $P_1$ (resp., process $P_2$) misses several updates of $y$ (resp., several updates of $x$) simulates the lose of information in channel.

%When the channel content of $M'_{(A,B)}$ is $l$, the ``channel content of $\mathcal{L}(A,B)$'' is $q \cdot \bot_s \cdot l \cdot \bot_e$, where $q$ is a state of $M'_{(A,B)}$ and represents the ``current state'' of $M'_{(A,B)}$, $\bot_s$ indicates the start of channel content of $M'_{(A,B)}$, and $\bot_e$ indicates the end of channel content of $M'_{(A,B)}$.

\forget{
We require $\mathcal{L}(A,B)$ to contain two methods, $M_1$ and $M_2$. We also require $M_1$ and $M_2$ to be fixed to process $P_1$ and $P_2$, respectively. Or we can say, if $M_1$ (resp., $M_2$) runs in a non-$P_1$ process (resp., a non-$P_2$ process), it will trivially return. This can be done by using $\textit{getPID}()$ commands.
%Methods $m_1$ and $m_2$ collaborate to simulate transitions of $M'_{(A,B)}$.
\figurename \ref{fig:execution for lock-freedom with one check success and one check fail} shows how $\mathcal{L}$ simulates one execution $t$ of $M'_{(A,B)}$.
%We draw a downward arrow and a upward arrow to represent $m_1$ and $m_2$ collaborate to simulate one transition of $M'_{(A,B)}$.
A downward arrow (resp., upward arrow) represents a modification of $x$ (resp., of $y$) done by $M_1$ (resp., done by $M_2$) is detected by $M_2$ (resp., by $M_1$).
%and a upward arrow to represent $m_1$ and $m_2$ collaborate to simulate one transition of $M'_{(A,B)}$.
$M_1$ additionally modify ``channel content'' according to transition rules.
To comply with lock-freedom and make $M_2$ not blocked when in guess phase, in the guess phase, $M_2$ returns as long as it modify $y$ one time.
In the check phase, $M_2$ keeps working until the simulation of lossy channel machine can not proceed and a flag $bflag$ is set to true.
In the check phase of $t$, we can see when $M_2$ reads $s_{\textit{trap}}$ from $x$, it returns.
%In the check phase of $t$, when $t$ does not go to $s_{\textit{trap}}$, $m_2$ never return;
%while when $t$ goes to $s_{\textit{trap}}$, $m_2$ set $z$ to $terminate$, which makes the afterwards $m_1$ and $m_2$ trivially return.
%When the simulation process fails, $z$ is also set to terminates, which makes the afterwards $m_1$ and $m_2$ trivially return.
%Note that $m_1$ never return.
}

\begin{figure}[tbp]
  \centering
  \includegraphics[width=0.738\textwidth]{PIC_SimluateFiniteTimesForLockFree.pdf}
\vspace{-10pt}
  \caption{One execution of $\mathcal{L}(A,B)$.
  %Execution of $\mathcal{L}(A,B)$ for an execution of $M'_{(A,B)}$ that has one round of check successes and one round of check fails. Here each transition rule $r_i$ is of the form
  %$(q_{\textit{i-1}},\_,\_,q_i,\_)$.
  }
  \label{fig:execution for lock-freedom with one check success and one check fail}
\vspace{-15pt}
\end{figure}

Let us now explain in detail the construction of $\mathcal{L}(A,B)$.
$\mathcal{L}(A,B)$ uses %five
the following memory locations: %$x$, $y$,
$x_1$, $y_1$, $x_2$, $y_2$, $\textit{ack1}$, $\textit{ack2}$, $\textit{phase}$,
$\textit{tflag}$ and $\textit{rule}$. Memory location $phase$ is used by $M_5$
to communicate to $M_1$, $M_2$, $M_3$ and $M_4$ whether the current phase is
guess or check, and its initial value is $guess$. Memory location
$\textit{tflag}$ is a flag indicating the failure of the simulation of the lossy
channel machine. Its initial value is $false$. Memory location $\textit{rule}$
is used by $M_5$ to stores the lossy channel machine transition rule that is
being simulated. The initial value of $\textit{rule}$ is a special value
$\bot$.

\forget{
%\gpnote{It would be useful to have variable names, and values use a different font.}
Let us now explain in detail the construction of $\mathcal{L}(A,B)$.
$\mathcal{L}(A,B)$ uses five memory locations: $x$, $y$, $phase$, $tflag$ and $rule$.
%$z$ is used to tell $m_2$ the current phase, and is a flag for $s_1$, $s_{\textit{trap}}$ or failed simulation.
%The initial value of $z$ is $phase1$;
%if the value of $z$ is $phase2$, then the simulated execution goes into check phase;
%if the value of $z$ is $terminate$, then $m_1$ and $m_2$ trivially return.
Location $phase$ is used to tell $M_2$ {whether the current phase is guess or
check,} and its initial value is $guess$.
Location $tflag$ stores a flag with initial value $false$. If the value is $true$, then the simulation of the lossy channel machine has already finished or failed.
Location $rule$ stores the current transition rule that is being simulated.
Note that, $x$ and $rule$ can only be written by $M_1$, while $y$ and $phase$
can only be written by $M_2$. $tflag$ can be written by both $M_1$ and $M_2$,
and the only possible update to $tflag$ is setting it to $true$.
}

%We now present the two methods in the pseudo-code, shown in Methods \ref{Method1OfLockFreedom} and \ref{Method2OfLockFreedom}.
We now present methods $M_1$, $M_2$ and $M_5$ in the pseudo-code, shown in Algorithms \ref{Method1OfLockFreedom}, \ref{Method2OfLockFreedom} and \ref{Method5OfLockFreedom}, respectively.
For brevity, we use the following notations.
We use $\textit{writeOne}(x,a)$ to represent the sequence of commands writing $\sharp$ followed by $a$ %followed by $\sharp$
into $x$. The symbol $\sharp$ is used as the delimiter to ensure that a value will
not be read twice.
We use $v:=\textit{readOne}(x)$ to represent the sequence of commands reading
%$e$ followed by$\sharp$
$\sharp$ followed by a value $e$ from $x$ for some $e \neq \sharp$ and then assigning $e$ to $v$.
Moreover, if the values read does not correspond with $\sharp$ followed by some
$e$, we set $\textit{tflag}$ to $true$ and then let the current method return. %, terminate the current method, and return.
We uniformly write a transition rule as $(q,u_1,u_2,\alpha,q',v_1,v_2)$, where $u_1$, $u_2,$ $v_1$ and $v_2$ can be either an value or $\epsilon$. % (but can not both be a value).
$(q,u_1,u_2,\alpha,q',v_1,v_2)$ represents a transition rule that
changes state from $q$ to $q'$ with transition label $\alpha$, receives $u_1$
from channel $c_1$ (if any), receives $u_2$ from channel $c_2$ (if any), sends
$v_1$ to channel $c_1$ (if any) and sends $v_2$ to channel $c_2$ (if any). We can see that $u_1$ and $v_1$ can not both be a value, and $u_2$ and $v_2$ can not both be a value. Given a transition rule $r=(q,u_1,u_2,\alpha,q',v_1,v_2)$, we define $valueRead(r,c_1)=u_1$, $valueRead(r,c_2)=u_2$, $valueWritten(r,c_1)=v_1$ and $valueWritten(r,c_2)=v_2$. %, and let $dst(r)$ denote the destination state $q'$ of $r$.
This library is designed to run on
five processes, and thus, the %$\textit{getPID}$ command
$\textit{checkPID}$ command only considers process
identifiers $1$ to $5$.
The symbols $s_0$ and $s_1$ in the pseudo-code of $M_5$ represent
the states of the lossy channel machine $\textit{CM}_{(A,B)}$.

The pseudo-code of $M_3$ (omitted) is obtained from $M_1$ by transforming
$\textit{checkPID}(1)$, $c_1$, $x_1$, $y_1$ and $\textit{ack1}$ into
$\textit{checkPID(3)}$, $c_2$, $x_2$, $y_2$ and $\textit{ack2}$, respectively.
The pseudo-code of $M_4$ (omitted) is obtained from $M_2$ by transforming
$\textit{checkPID}(2)$, $x_1$ and $y_1$ into $\textit{checkPID}(4)$, $x_2$ and
$y_2$, respectively.
}








%\vspace{-5pt}
\subsection{Undecidability of Four Liveness Properties}
\label{subsec:undecidability of four liveness properties}

The following theorem states that lock-freedom, wait-freedom,
deadlock-freedom and starvation-freedom are all undecidable on TSO for a
bounded number of processes. Perhaps surprisingly, we prove this theorem with
the same library $\mathcal{L}(A,B)$. %The detailed proof of this theorem can be found in Appendix \ref{sec:appendix proof of section sec:undecidability of lock-freedom and wait-freedom}.

\begin{theorem}
  \label{theorem:lock-freedom is undecidable}
  The problems of checking lock-freedom, wait-freedom, deadlock-freedom and starvation-freedom of a given library for a bounded number of processes %is
  are undecidable on TSO.
\end{theorem}
\begin {proof}(Sketch)
  For each infinite execution $t$ of $\llbracket \mathcal{L}(A,B),2 \rrbracket$, assume that it simulates an execution of $\textit{CM}'_{(A,B)}$, or it intends to do so. There are three possible cases for $t$ shown as follows:

  \begin{itemize}
  \item[-] Case $1$: The simulation fails because some $\textit{readOne}$ does not read
    the intended value.

  \item[-] Case $2$: The simulation procedure succeeds, and $t$ infinitely loops in the guess phase.

  \item[-] Case $3$: The simulation procedure succeeds, and $t$ infinitely loops in the check phase and visits $s_1$ infinitely many times.
  \end{itemize}

  In case $1$, since $\textit{failSimu}$ is set to $\textit{true}$, each method
  returns immediately. Therefore, $t$ satisfies wait-freedom and thus, satisfies
  lock-freedom. $t$ can be either fair or unfair. In case $2$, since each method
  returns after finite number of steps in the guess phase, $t$ satisfies wait-freedom and thus, satisfies lock-freedom. %Since $M_1$ and $M_2$ coordinate when simulating each transition of $\textit{CM}'_{(A,B)}$, $t$ must be fair.
  In case $3$, since $M_1$ and $M_2$ do not return in the check phase, $t$ violates lock-freedom and thus, violates wait-freedom. Since $M_1$ and $M_2$ coordinate when simulating each transition of $\textit{CM}'_{(A,B)}$, in case $2$ and $3$, $t$ must be fair.

  Therefore, we reduce the problem of checking whether $\textit{CM}'_{(A,B)}$ has an execution that visits $s_1$ infinitely often into the problem of checking whether $\mathcal{L}(A,B)$ has an infinite execution of case $3$ (which is fair and violates both wait-freedom and lock-freedom). By Lemma \ref{lemma:reducing CPCP into a infinite execution problem of lossy channel machine M'{(A,B)}}, we can see that the problems of checking lock-freedom, wait-freedom, deadlock-freedom and starvation-freedom of a given library for bounded number of processes are undecidable.
\end {proof}

% Our TSO memory model do not consider liveness condition on store buffers.
We remark here that~\cite{DBLP:journals/corr/abs-2012-01067} considers
imposing liveness condition on store buffers, and requires buffered items to be
eventually flushed. Our undecidability results on liveness properties on TSO
still hold when imposing such liveness condition on store buffers, since in case
$3$ of the proof of Theorem \ref{theorem:lock-freedom is undecidable}, each item
put into buffer will eventually be flushed.







%\vspace{-5pt}
\section{Checking Obstruction-Freedom}
\label{sec:checking obstruction-freedom}

%In this section, we prove that obstruction-freedom is decidable.

\forget{
In this section, we prove that obstruction-freedom is decidable.
We first introduce the basic TSO concurrent systems (the TSO concurrent systems of \cite{DBLP:conf/popl/AtigBBM10}) and the state reachability problem.
Then, we propose a notion called blocking pairs, and reduce checking obstruction-freedom on TSO, which considers infinite executions, into the state reachability problem of basic TSO concurrent systems, which is known to be decidable.
}

\forget{
In this section, we prove that obstruction-freedom is decidable.
We first introduce the basic TSO concurrent systems (the TSO concurrent systems of \cite{DBLP:conf/popl/AtigBBM10}) and the state reachability problem. Then, we propose a notion called blocking pairs, and reduce checking obstruction-freedom on TSO, which considers infinite executions, into checking if some
finite execution reaches a configuration containing a blocking pair.
Finally, we reduce checking blocking pairs into the state reachability problem of
basic TSO concurrent systems, which is known to be decidable.
}



\forget{
In this section, we prove that obstruction-freedom is decidable. We first
introduce the notion of $(S,k)$-(lossy) channel machines
\cite{DBLP:conf/popl/AtigBBM10}.
{Then, we propose a notion called blocking pairs, and slightly extend the TSO semantics to detect blocking pairs. We reduce checking obstruction-freedom on TSO,
which considers infinite executions, into checking if some
finite execution reaches a configuration containing a blocking pair on the extended TSO semantics.}
%Then, we reduce checking obstruction-freedom,
%{\color {blue}which considers infinite length executions,} into checking if some
%finite execution reaches some configuration that ``contains a blocking pair'',
%and then reduce the latter
The latter problem is further reduced into a control state reachability problem of lossy
channel machines, which is known to be decidable.
{We slightly extend the lossy channel machine of \cite{DBLP:conf/popl/AtigBBM10} to detect blocking pairs.}
%{\color {blue}We slightly
%  extend the operational semantics of libraries on TSO and the lossy channel
%  machines of \cite{DBLP:conf/popl/AtigBBM10} to detect blocking pairs.}
}


%\vspace{-5pt}
%\subsection{The TSO Concurrent System of \cite{DBLP:conf/popl/AtigBBM10}}
\subsection{The Basic TSO Concurrent Systems}
\label{subsec:the TSO concurrent system of DBLP:conf/popl/AtigBBM10}


Atig \emph{et al.} \cite{DBLP:conf/popl/AtigBBM10} considers the following concurrent systems on TSO: Each process runs a finite control state program that can do internal, %$\textit{checkPID}$,
read, write and $\textit{cas}$ actions, and different processes communicate via shared memory.
We use \emph{basic TSO concurrent systems} to denote such concurrent systems.
%We obtain the decidability of obstruction-freedom by reducing it to the state reachability problem of basic TSO concurrent systems \cite{DBLP:conf/popl/AtigBBM10}. %Our concurrent systems is similar to the concurrent system of \cite{DBLP:conf/popl/AtigBBM10}.
%Since their concurrent systems do not consider libraries and thus do not need to deal with call and return actions, we transform call and return actions into internal actions. %Since our concurrent systems additionally introduce $\textit{checkPID}$ command,
%We additionally introduce the $\textit{checkPID}$ command to their concurrent systems, and use \emph{basic TSO concurrent systems} to denote such concurrent systems.
% , except two points. First, they do not consider libraries, and thus do not need to deal with call and return actions; second, we additionally introduce $\textit{checkPID}$ command. The first difference can be solved by transforming call and return actions into internal actions. To deal with the second difference, we introduce $\textit{checkPID}$ command into their concurrent systems, and use \emph{basic TSO concurrent systems} do denote such concurrent systems.

%This is similar to our concurrent system, except that they do not consider libraries, and thus do not need to deal with call and return actions. {\color {red}Another difference is that we additionally introduce $\textit{checkPID}$ command.} %When no confusion occur we call the concurrent system of \cite{DBLP:conf/popl/AtigBBM10} a TSO concurrent system of \cite{DBLP:conf/popl/AtigBBM10}, or TSO concurrent system for short.
%{\color {red}We use \emph{basic TSO concurrent systems} to denote the extensions of their concurrent systems by introducing $\textit{checkPID}$ command.}
%To explicitly distinguish their concurrent systems from ours, we denote their concurrent systems as \emph{basic TSO concurrent systems}. %to call their concurrent systems.

Formally, let $\Sigma(proc,\mathcal{D},\mathcal{X})$ be the set containing the $\tau$
(internal) action, %command,
%the $\textit{checkPID}$ actions, %commands,
the read actions, %commands,
the write actions %commands
and the $\textit{cas}$ actions %commands
over memory locations $\mathcal{X}$ with data domain $\mathcal{D}$ and of process $proc$. A basic TSO concurrent system %of \cite{DBLP:conf/popl/AtigBBM10}
is a tuple $(P_1,\ldots,P_n)$, where each $P_i$ is a tuple $(Q_i,\Delta_i)$,
such that $Q_i$ is a finite control state set and $\Delta_i \subseteq Q_i \times
\Sigma(i,\mathcal{D},\mathcal{X}) \times Q_i$ is the transition relation. They
define an operational semantics similar to the one presented in Section
\ref{sec:concurrent systems}. Each configuration is also a tuple $(p,d,u)$,
where $p$ stores control state of each process, $d$ is a memory valuation and
$u$ stores buffer content of each process. We refer the reader to
\cite{DBLP:conf/popl/AtigBBM10} for a detailed description of the operational
semantics on TSO which is unsurprising, and hence omitted here.


Given a basic TSO concurrent system $(P_1,\ldots,P_n)$, two functions $p$ and
$p'$ that store control states of each process and two memory valuations $d$ and
$d'$, the state reachability problem requires to determine whether there is a
path from $(p,d,u_{\textit{init}})$ to $(p',d',u_{\textit{init}})$ in the
operational semantics, where $u_{\textit{init}}$ initializes each process with an empty buffer. Atig \emph{et al.} \cite{DBLP:conf/popl/AtigBBM10} prove
that the state reachability problem is decidable. %in this TSO semantics.



\forget{
\subsection{$(S,k)$-(Lossy) Channel Machines}
\label{subsec:(S,k)-(lossy) channel machines}

In this subsection we introduce the $(S,k)$-(lossy) channel machines of \cite{DBLP:conf/popl/AtigBBM10}, which extend classical (lossy) channel machines of Section \ref{sec:undecidability of lock-freedom and wait-freedom} in the following ways:

\begin{itemize}
\item[-] Each transition is guarded by a condition about whether the content of
  a channel belongs to a regular language.

\item[-] A substitution of the content of a channel may be performed before send operations.

\item[-] We introduce a set of symbols, called ``strong symbols'', which are not
  allowed to be lost. We ensure that the number of strong symbols in a channel
  is always bounded.
\end{itemize}


For a  given channel $c \in \mathcal{CH}$, a regular guard on channel $c$ is a constraint of the form $c \in L$, where
$L \subseteq \Sigma_{\mathcal{CH}}^*$ is a regular set of sequences. For a sequence $u \in \Sigma_{\mathcal{CH}}^*$ and a guard $c \in L$, we write $u \models c \in L$ if $u \in L$.
For notational convenience, we write $a \in c$ instead of $c \in \Sigma_{\mathcal{CH}}^* \cdot a \cdot \Sigma_{\mathcal{CH}}^*$,
$c = \epsilon$ instead of $c \in \{ \epsilon \}$ and $c:\Sigma'$ instead of $c \in \Sigma'^*$ for any subset $\Sigma'$ of $\Sigma_{\mathcal{CH}}$.
A regular guard over $\mathcal{CH}$ associates a regular guard for each channel of $\mathcal{CH}$.
Let $\textit{Guard}(\mathcal{CH})$ be the set of regular guards over $\mathcal{CH}$. The definition of
$\models$ can be extended as follows: for $g \in \textit{Guard}(\mathcal{CH})$ and $u \in \mathcal{CH} \rightarrow \Sigma_{\mathcal{CH}}^*$,
we write $ u \models g$, if $u(c) \models g(c)$ for each $c \in \mathcal{CH}$.

We extend channel operations and introduce substitutions in send operations.
A send operation is of the form $c[\sigma]!a$, where $\sigma$ is a substitution over $\Sigma_{\mathcal{CH}}$.
We write $c ! a$ instead of $c [\sigma] ! a$ when $\sigma$ is the identity substitution.
For every $u \in \Sigma_{\mathcal{CH}}^*$, we have %\\[-5pt]
%\[\llbracket c[\sigma]!a \rrbracket(u,a \cdot u(c)[\sigma])\]
$\llbracket c[\sigma]!a \rrbracket(u,a \cdot u(c)[\sigma])$. The cases $\llbracket c?a \rrbracket(u,u')$ and $\llbracket nop
  \rrbracket(u,u')$ remain unchanged from the previous lossy channel machines.


The notion of channel machine is now extended with guards and substitutions. A channel machine is formally defined as a tuple $\textit{CM} = (Q,\mathcal{CH},\Sigma_{\mathcal{CH}},\Lambda,\Delta)$, where (1) $Q$ is a finite set of states, (2) $\mathcal{CH}$ is a finite set of channel names, (3) $\Sigma_{\mathcal{CH}}$ is a finite alphabet for channel contents, (4) $\Lambda$ is a finite set of transition labels, and (5) $\Delta \subseteq Q \times (\Lambda\cup\{\epsilon\}) \times \textit{Guard}(\mathcal{CH}) \times \textit{Op}(\mathcal{CH}) \times Q$ is a finite set of transitions.


Let $S \subseteq \Sigma$ be a finite set of ``strong symbols'' that must be kept
in each transition (that is, cannot be lost), and $k$ be a positive integer
bounding the numbers of strong symbols in a channel.
For sequences $u,v \in \Sigma_{\mathcal{CH}}^*$, $u \preceq_S^k v$ holds if $u$ is a subword of $v$, $u \uparrow_{S} = v \uparrow_{S}$ and $\vert u \uparrow{S} \vert \leq k$.
This relation can be extended as follows: For every $u,v \in \mathcal{CH} \rightarrow \Sigma_{\mathcal{CH}}^*$, $u \preceq_S^k v$ holds,
if $u(c) \preceq_S^k v(c)$ holds for each $c \in \mathcal{CH}$.


A $\textit{(S,k)-(perfect) channel machine}$ (abbreviated as $\textit{(S,k)-CM}$) is a channel machine $\textit{CM} = (Q,\mathcal{CH},\Sigma_{\mathcal{CH}},\Lambda,\Delta)$ with the strong symbol restriction $(S,k)$. Its semantics is defined as an LTS $(\textit{Conf}_{\textit{CM}}, \Lambda,\rightarrow_{\textit{CM}},\textit{initConf}_{\textit{CM}})$.
A configuration of $\textit{Conf}_{\textit{CM}}$ is a pair $(q,u)$ where $q \in Q$, $u:\mathcal{CH} \rightarrow \Sigma_{\mathcal{CH}}^*$, and it satisfies the strong symbol restriction $(S,k)$, i.e., for each $c$, $\vert u(c) \uparrow{S} \vert \leq k$.
The transition relation $\rightarrow_{\textit{CM}}$ is defined as follows: given $q,q' \in Q$ and $u,u' \in \mathcal{CH} \rightarrow \Sigma_{\mathcal{CH}}^*$,
$(q,u) \overset{\alpha}{\longrightarrow}_{\textit{CM}} (q',u')$,
if there exists $g$ and $op$, such that $(q,\alpha,g,\textit{op},q') \in \Delta$, $u \models g$ and $\llbracket \textit{op} \rrbracket (u,u')$.
Similarly, a $\textit{(S,k)-lossy channel machine}$ (abbreviated as $\textit{(S,k)-LCM}$) is a channel machine $\textit{CM}$ with lossy channels and the strong symbol restriction $(S,k)$.
Its semantics is defined as an LTS $(\textit{Conf}_{\textit{CM}}, \Lambda,\rightarrow_{\textit{(\textit{CM},S,k)}},\textit{initConf}_{\textit{CM}})$.
The transition relation $\rightarrow_{\textit{(\textit{CM},S,k)}}$ is defined as follows:
$(q,u) \overset{\alpha}{\longrightarrow}_{\textit{(\textit{CM},S,k)}} (q',u')$,
if there exists $v,v' \in \mathcal{CH} \rightarrow \Sigma_{\mathcal{CH}}^*$,
such that $v \preceq_S^k u$, $(q,v) \overset{\alpha}{\longrightarrow}_{\textit{CM}} (q',v')$ and $u' \preceq_S^k v'$.
%Let $\rightarrow_M^*$ and $\rightarrow_{\textit{(M,S,K)}}^*$ be the transition closure of $\rightarrow_M$ and $\rightarrow_{\textit{(M,S,K)}}$.


Given a channel machine $\textit{CM}$, we say that $(q_0,u_0) \cdot \alpha_1 \cdot (q_1,u_1) \cdot \ldots \cdot \alpha_w \cdot (q_w,u_w)$ is a finite run of $\textit{CM}$ from $(q,u)$ to $(q',u')$,
if (1) $(q_0,u_0)=(q,u)$, (2) $(q_i,u_i) \overset{ \alpha_{\textit{i+1}} }{\longrightarrow}_{\textit{CM}} (q_{\textit{i+1}},u_{\textit{i+1}})$
for each $i$ and (3) $(q_w,u_w) = (q',u')$.
We say that $l$ is a trace of a finite run $\rho$ if $l = \rho \uparrow_{\Lambda}$.
Given $q,q' \in Q$, let $\textit{T}_{q,q'}^{S,k}(\textit{CM})$ denote the set of traces of all finite runs of a $(S,k)$-$\textit{CM}$ $\textit{CM}$ from the configuration $(q,c_{\textit{init}})$ to the configuration $(q',c_{\textit{init}})$. Here $c_{\textit{init}}$ is a function that maps each channel name to an empty channel $\epsilon$.
For $(S,k)-\textit{LCM}$ $\textit{CM}$, the notations of finite run and its trace are defined as in the non-lossy case by replacing $\rightarrow_{\textit{CM}}$ with $\rightarrow_{(\textit{CM},S,k)}$.
Let $\textit{LT}_{q,q'}^{S,k}(\textit{CM})$ denote the set of traces of all finite runs of $(S,K)\textit{-LCM}$ $\textit{CM}$ from the configuration $(q,c_{\textit{init}})$ to the configuration $(q',c_{\textit{init}})$.

For channel machines $\textit{CM}_1 =
(Q_1,\mathcal{CH}_1,\Sigma_{\mathcal{CH}},\Lambda,\Delta_1 )$ and $\textit{CM}_2 =
(Q_2,\mathcal{CH}_2,\Sigma_{\mathcal{CH}}$, $\Lambda,$ $\Delta_2)$ such that
$\mathcal{CH}_1 \cap \mathcal{CH}_2 = \emptyset$, the product of $\textit{CM}_1$ and $\textit{CM}_2$
is %also
a channel machine $\textit{CM}_1 \otimes \textit{CM}_2 = (Q_1 \times Q_2,\mathcal{CH}_1 \cup
\mathcal{CH}_2,\Sigma_{\mathcal{CH}},\Lambda,\Delta_{12} )$, where $\Delta_{12}$
is defined by synchronizing transitions sharing the same label in $\Lambda$
under the conjunction of their guards, and letting other transitions be
asynchronous. The following lemma holds as in \cite{DBLP:conf/popl/AtigBBM10}.

\begin{lemma}
\label{proposition:relation bewteen LT of M1 and M2 and (LT of M1 and LT of M2)}
Given channel machines $\textit{CM}_1 = (Q_1,\mathcal{CH}_1,\Sigma_{\mathcal{CH}},\Lambda,\Delta_1 )$ and $\textit{CM}_2 = (Q_2,$ $\mathcal{CH}_2,\Sigma_{\mathcal{CH}},\Lambda,\Delta_2 )$, let $q_1,q'_1 \in Q_1$, $q_2,q'_2 \in Q_2$, $q=(q_1,q_2)$, $q'=(q'_1,q'_2)$, then $\textit{LT}_{q,q'}^{S,K}(\textit{CM}_1 \otimes \textit{CM}_2)$ = $\textit{LT}_{q_1,q_1'}^{S,K}(\textit{CM}_1) \cap \textit{LT}_{q_2,q'_2}^{S,K}(\textit{CM}_2)$ and $T_{q,q'}^{S,K}(\textit{CM}_1 \otimes \textit{CM}_2)$ = $T_{q_1,q_1'}^{S,K}(\textit{CM}_1) \cap T_{q_2,q'_2}^{S,K}(\textit{CM}_2)$.
\end{lemma}


Given a $(S,k)\textit{-LCM}$ $\textit{CM}$ and two states $q,q' \in Q$, a control state reachability problem of $\textit{CM}$ is to determine whether $\textit{LT}_{q,q'}^{S,k}(\textit{CM}) \neq \emptyset$. Atig \emph{et al.} \cite{DBLP:conf/popl/AtigBBM10} prove that the control state reachability problem is decidable for $(S,k)\textit{-LCM}$.


%Given a $(S,k)\textit{-CM}$ (respectively, $(S,k)\textit{-LCM}$) $M$ and two states $q,q'\in Q$, a control state reachability problem of $M$ is to determine whether $\textit{T}_{q,q'}^{S,k}(M) \neq \emptyset$ (respectively, $\textit{LT}_{q,q'}^{S,k}(M) \neq \emptyset$). \cite{POPL2010} proves that the control state reachability problem is decidable for $(S,k)\textit{-LCM}$.
}


\subsection{Verification of Obstruction-Freedom}
\label{subsec:equivlant characterization of obstruction-freedom}
\forget{
\subsection{Equivalent Characterization of Obstruction-Freedom}
\label{subsec:equivlant characterization of obstruction-freedom}
}

\forget{
The definition of obstruction-freedom requires checking infinite executions. We propose an equivalent characterization of obstruction-freedom, which checks finite executions instead of infinite executions. This equivalent characterization is based on a notion called blocking pairs, which captures potential obstruction-freedom violations.
}
%It is hard to check obstruction-freedom directly using its definition, since
%each obstruction-freedom violation is of infinite length.
%We propose an equivalent characterization of obstruction-freedom, which
%considers executions of finite length, and is thus easier to check.
% that requires dealing with infinite executions of infinite state systems.
%To deal with that problem, we propose a equivalent characterization of obstruction-freedom, which consider only finite length executions.

\forget{
We propose an equivalent characterization of obstruction-freedom, which
checks finite executions instead of infinite executions, which are required in the definition of obstruction-freedom.
To reduce obstruction-freedom checking to control state reachability checking
for lossy channel machines, we slightly extend the TSO semantics and propose
a second equivalent characterization, also as a finite execution problem. Both
of our equivalent characterizations are based on a notion called blocking pairs,
which capture potential obstruction-freedom violation, and both of them require
detecting blocking pairs, which in turn requires reading a memory valuation and
the buffer contents atomically. This can not be done in the original TSO memory
model.}

The definition of obstruction-freedom requires checking infinite executions,
while the state reachability problem considers finite executions reaching
specific configurations. To bridge this gap, we propose a notion called
blocking pairs, which is defined on concurrent systems on the SC memory model
and captures potential obstruction-freedom violations.
%\redt{The notion of blocking pairs is defined on concurrent systems on SC memory model.}
Let $\llbracket \mathcal{L}, n \rrbracket_{sc}$ be the operational semantics of a concurrent system that runs on the SC memory model and contains $n$ processes.
The configurations of $\llbracket \mathcal{L}, n \rrbracket_{sc}$ coincide with the configurations of $\llbracket \mathcal{L}, n \rrbracket$ that preserve the buffer empty for each process.
When performing a write action $\llbracket \mathcal{L}, n \rrbracket_{sc}$ does
not put the item into the buffer, but directly updates the memory instead. $\llbracket \mathcal{L}, n \rrbracket_{sc}$ does not have flush actions, while other actions are unchanged from $\llbracket \mathcal{L}, n \rrbracket$. Since we use finite program positions, finite memory locations, a finite data domain, finite method names and a finite number of processes, and since we essentially do not use buffers, we observe that $\llbracket \mathcal{L}, n \rrbracket_{sc}$ is a finite state LTS.

\forget{
The notion blocking pair is defined on the operational semantics of the SC
memory model. Let $\llbracket \mathcal{L}, n \rrbracket_{sc}$ be the operational
semantics of a concurrent system that runs on the SC memory model and contains
$n$ processes. The configurations of $\llbracket \mathcal{L}, n \rrbracket_{sc}$
coincide with the configurations of $\llbracket \mathcal{L}, n \rrbracket$ that
preserve the buffer empty for each process. The transition relation of
$\llbracket \mathcal{L}, n \rrbracket_{sc}$ is generated from $\llbracket
\mathcal{L}, n \rrbracket$ by modifying the transition rule of the write action
as shown below, and by removing the transition rule for flush actions, while keeping the transition rules of all other actions unchanged:
\[
  \begin{array}{l c}
    \bigfrac{ p(i)=(q_i,q'_c) \quad q_i\
    {\xrightarrow{\textit{write}(x,a)}}_{\mathcal{L}}\ q'_i} { ( p,d,u)\ {\xrightarrow{\textit{write}(i,x,a)}}\
    ( p[i:(q'_i,q'_c)],d[x:a],u)}
  \end{array}
\]
Since we use finite program positions, finite memory locations, a finite data
domain, finite method names and a finite number of processes, and since we
  essentially do not use buffers, we observe that $\llbracket \mathcal{L}, n
\rrbracket_{sc}$ is a finite state LTS.
%It is easy to give the operational semantics $\llbracket \mathcal{L}, 1 \rrbracket_{sc}$ of concurrent system that runs on SC memory model and contains only one process. In each configuration of $\llbracket \mathcal{L},1 \rrbracket$, the only process must have a empty buffer. Therefore, $\llbracket \mathcal{L}, 1 \rrbracket_{sc}$ is a finite state LTS.
}



Let us now propose the notion of blocking pairs. Given a state $q \in \{\textit{in}_{\textit{clt}}\} \cup (Q_{\mathcal{L}} \times \{\textit{in}_{\textit{lib}}\})$ (recall that $Q_{\mathcal{L}}$ is the set of program positions of library, $\textit{in}_{\textit{clt}}$ and $\textit{in}_{\textit{lib}}$ are the states of the most general client) and a memory valuation $d$,
$(q,d)$ is a blocking pair, if in $\llbracket \mathcal{L}, 1 \rrbracket_{sc}$
there exists a configuration $(p,d,u)$, such that the state of process 1 of $p$ is $q$ ($p(1)=q$), %(recall that $p$ is a function and thus $q \in \{q_c\} \cup (Q_{\mathcal{L}} \times \{q'_c\})$),
and there exists an infinite execution from $(p,d,u)$ and such
execution does not have a return action. This property can be expressed by
the %CTL$^{*}$
%LTL
CTL$^*$ formula $E ( (G\ \neg P_{\textit{ret}}) \wedge (G\ X\ P_{\textit{any}})
)$, where $E$ is the usual modality of CTL$^*$,
%\redt{where $E$ is the path quantifier of ``for some computation paths'' of CTL$^*$,}
$P_{\textit{ret}}$ is a predicate that checks if the transition label is a
return action, %(defined in Section \ref{sec:liveness}),
and $P_{\textit{any}}$ returns
true %if the transition label is not empty.
{for any transition label.}
The following lemma reduces checking obstruction-freedom into the state reachability problem.

\forget{
{\color {blue}Given a library $\mathcal{L}$, we generate a library
  $\mathcal{L}_{in}$ as follows: Intuitively, $\mathcal{L}_{in}$ extends
  $\mathcal{L}$ by nondeterministically doing a $\textit{cas}(x,0,0)$ action
  between any two actions of $\mathcal{L}$. Formally, if $q_1
  {\xrightarrow{\alpha}} q_2$ is a transition of $\mathcal{L}$, then
  $\mathcal{L}_{in}$ adds state $q_{(1,2)}$ and transitions $q_1
  {\xrightarrow{\textit{cas}(x,0,0)}} q_{(1,2)}$ and $q_{(1,2)}
  {\xrightarrow{\alpha}} q_2$. We call the newly added states (e.g.,
  $q_{(1,2)}$) intermediate states. We can see that $\mathcal{L}_{in}$
  essentially has the ``same behavior'' as $\mathcal{L}$, since
  $\textit{cas}(x,0,0)$ does not change the memory valuation. The
  $\textit{cas}(x,0,0)$ transition of $\mathcal{L}_{in}$ can be understand as
  $\mathcal{L}$ doing several flush actions.}
  }



\forget{
  The following lemma provides an equivalent characterization of
  obstruction-freedom as a reachability problem of configurations that contain
  blocking pairs. Refer to Appendix \ref{subsec:appendix proof of lemma lemma:equivalent characterization of obstruction-freedom} %\ref{sec:appendix proof of section sec:checking obstruction-freedom}
  for the proof.
}

\forget{
%We say that $(p,d)$ is a blocking pair, if in $\llbracket \mathcal{L}, 1 \rrbracket_{sc}$, the configuration $(p,d,\epsilon^n)$ satisfies the LTL formula $E (G \neg \textit{isRet} \wedge G X \textit{true})$, where $\textit{isRet}$ is a predicate that checks if the transition label is a return action. By model checking techniques it is obvious that the set of blocking pairs are computable.
The following lemma provides an equivalent characterization of obstruction-freedom as a reachability problem of configurations that contain a blocking pair. Given a configuration $(p,d,u)$ and a process $proc$, let $d[u(proc)]$ be a memory valuation obtained from $d$ by using the buffered writes of $u(proc)$ to update $d$ one by one.
}

\begin{lemma}
\label{lemma:equivalent characterization of obstruction-freedom}
Given a library $\mathcal{L}$, there exists an infinite execution $t$ of $\llbracket \mathcal{L}, n \rrbracket$ that violates obstruction-freedom on TSO, if and only if there exists an finite execution $t'$ of $\llbracket \mathcal{L},n \rrbracket$ and a process $proc$, such that $t'$ leads to a
configuration $(p,d,u_{\textit{init}})$, where $(p(proc),d)$ is a blocking pair.
\end{lemma}

\forget{
The detailed proof can be found in Appendix \ref{subsec:appendix proof of
  lemma lemma:equivalent characterization of obstruction-freedom}. The
\emph{if} direction holds since we can obtain an obstruction-freedom violation
$t' \cdot t_1$, in which $t_1$ is the specific infinite execution in the
definition of blocking pairs. For the \emph{only if} direction, there exists
$t_1$, $t_2$ and process $proc$, such that $t=t_1 \cdot t_2$, and $t_2$
contains only actions of process $proc$. Let $\alpha_i$ be the last write
action of process $i \neq proc$ in $t_1$ that has not been flushed. Dropping
$\alpha_i$ in $t$ yields a legal execution, since $\alpha_i$ does not
influence the memory. Also, it is legal to clear buffer of process $proc$
before executing $t_2$. With the approach above, we can generate an new
execution that reaches a configuration, which has an empty buffer for each
process and contains a blocking pair, before the execution of $t_2$.
}



\begin {proof}(Sketch)
To prove the \emph{if} direction, consider the infinite executions which first act as $t'$, and then always run process $proc$ and disallowing other processes to do actions. The behaviors after $t'$ of these executions behave as the executions of $\llbracket \mathcal{L},1 \rrbracket_{sc}$ from the configuration  $(p',d,u_{\textit{init}})$, where $p'$ is a function that maps process $1$ to $p(proc)$. According to the definition of blocking pairs, there exists one such execution $t$ that violates obstruction-freedom.

The \emph{only if} direction is proved as follows: There exists $t_1$, $t_2$ and process $proc$, such that $t=t_1 \cdot t_2$, and $t_2$ contains only actions of process $proc$. Given process $i$, let $\alpha_i$ be the last write action of process $i$ in $t_1$ that has not been flushed. Let $t'_1$ be obtained from $t_1$ by removing the last non-flush action $\alpha$ of process $i$ after $\alpha_i$ for some process $i \neq proc$. $\alpha$ can not influence other process since it can not influence memory. Since the only possible actions of process $i$ after $\alpha$ is flush, the subsequent actions of process $i$ is not influenced. Therefore, $t'_1 \cdot t_2$ is an execution of $\llbracket \mathcal{L}, n \rrbracket$. By repeatedly apply above approach we obtain $t_3$ from $t_1$, such that $t_3 \cdot t_2$ violates obstruction-freedom, and for each process $i \neq proc$, all write of process $i$ has been flushed in $t_3$. Since process $proc$ runs as on SC in $t_2$ and read actions first try to read from buffer, we can see that $t_3 \cdot t_4 \cdot t_2$ is an execution of $\llbracket \mathcal{L}, n \rrbracket$, where $t_4$ flushes all the remaining items of process $proc$ in $t_1$. The configuration reached by execution $t_3 \cdot t_4$ has buffer empty for each process, and we can see that the control state and memory valuation is a blocking pair according to its definition. This completes the proof of this lemma.
\end {proof}




\forget{
\begin{lemma}
\label{lemma:equivalent characterization of obstruction-freedom}
Given a library $\mathcal{L}$, there exists an infinite execution $t$ of $\llbracket \mathcal{L}, n \rrbracket$ that violates obstruction-freedom, if and only if there exists an finite execution $t'$ of $\llbracket
%\mathcal{L}_{in},
\mathcal{L},n \rrbracket$ and a process $proc$, such that $t'$ leads to a
configuration $(p,d,u)$, where $u(proc)=\epsilon$, and $(p(proc),d)$ is a blocking pair. %Here $d[u(pro)]$ is a memory valuation obtained from $d$ by using the buffered write of $u(pro)$ to update $d$ one by one.
\end{lemma}
}
\forget{
\begin {proof}
  Let us prove the \emph{only if} direction first. Since $t$ violates
  obstruction-freedom, there exists a process $proc$ and a time $i_1$, such that
  from $i_1$ onward only the process $proc$ can launch actions on $t$. Since
  there is a finite number of write actions of $t$ from its beginning to time
  $i_1$, there exists a time $i_2$, such that after $i_2$, only the process
  $proc$ can do flush actions. Let time $i_3$ be the point of the last call
  action of process $proc$, and let $i_4 = \textit{max}(i_1,i_2,i_3)$. It is
  easy to see that from time $i_4$, process $proc$ is not influenced by any
  other processes, and the behavior of process $proc$ from then on is as on SC.
  Let us generate another execution $t_1$ as follows: $t_1$ works
  as $t$ until time $i_4$, then $t_1$ clears the store buffer of process $proc$
  (let $(p_5,d_5,u_5)$ be the configuration at this time point), and works as
  the behavior of $t$ after time $i_4$. It is easy to see that $t_1$ is also
  an execution of $\llbracket \mathcal{L}, n \rrbracket$. Since process $proc$
  is scheduled infinitely many times and does not return, and
  $d_5(proc)=\epsilon$, we can see that $(p_5(proc),d_5)$ is a blocking pair.

%Let $(p_4,d_4,u_4)$ be the configuration of time $t_4$. It is easy to see that from time $t_4$, process $proc$ is not influenced by any other process, and the behavior of process $proc$ from then on is as on SC. {\color {red}Let $d_4[u_4(pro)]$ be a memory valuation obtained from $d_4$ by using the buffered write of $u_4(pro)$ to update $d_4$ one by one. We can see that in $\llbracket \mathcal{L},1 \rrbracket_{sc}$, from $(p_4(proc),d_4[u_4(proc)],u_{\textit{init}})$, we can do the remaining transitions of $t$ from time point $t_4$. Therefore, $(p_4(proc),d_4[u_4(proc)])$ is a blocking pair.}

\forget{Let us generate an execution $t'$ as follows: $t'$ first does $t$ transitions
    from the initial configuration to $(p_4,d_4,u_4)$.
    Then, $t'$ flushes all buffered items in process $proc$'s buffer and then the
    process $proc$ goes to a intermediate state. Assume that $t'$ reaches a
    configuration $(p,d,u)$, with $p(proc)$ some intermediate state $q_{(u,v)}$.
    Finally, $t'$ does the remaining transitions of $t$. Since $t'$ also
    violates obstruction-freedom, and $t$'s behavior is as in SC after time
    $t_4$, we can see that $(q_v,d)$ is a blocking pair.}
%If $p_4(pro)$ is a intermediate state, then let $(p'_4,d'_4,u'_4)=(p_4,d_4,u_4)$. Otherwise, let $(p'_4,d'_4,u'_4)$ be obtained from $(p_4,d_4,u_4)$ by first doing several flush of process $pro$ and then let process $pro$ transit to intermediate state. The buffer of process $pro$ is empty in $(p'_4,d'_4,u'_4)$. Assume that $p'_4(pro)=(q_{(j,k)},q'_c)$.
%It is easy to see that in $\llbracket \mathcal{L}, 1 \rrbracket_{sc}$, we could generate transitions from $(\{(P_1:(q_k,q'_c))\},d'_4,\epsilon)$ that do the $t$ transitions after $(p_4,d_4,u_4)$ (except for the flush actions and into intermediate state transition). Since $t$ is a obstruction-freedom violation, we can see that $(q_k,d'_4)$ is a blocking pair.

%Let $d_4[u_4(pro)]$ represents a memory valuation obtained from $d_4$ by using the buffered write of $u_4(pro)$ to update $d_4$ one by one. It is easy to see that

%Then, it is easy to see that, after time point $t_3$, process $pro$ runs as in SC memory model (while the memory valuation is obtained from memory valuation of the system and buffered write of process $pro$).

%Let time point $t_4$ be the time point of the last call action of process $pro$, and let $t_5 = \textit{max}(t_3,t_4)$. Let $(p_5,d_5,u_5)$ be the configuration of time point $t_5$. Since $t$ is a obstruction-freedom violation, $t$ has no return action from $(p_5,d_5,u_5)$.
%The execution of $t$ from time point $t_5$ is the same as the execution from $(p_5, d_5[u_5(pro)], \epsilon)$ in $\llbracket \mathcal{L}, 1 \rrbracket_{sc}$. Therefore, $(p_5(pro), d_5[u_5(pro)])$ is a blocking pair.


%and let $(p_5,d_5,u_5)$ be the configuration of time point $t_5$. Since $t$ is a obstruction-freedom violation, $t$ contains finite number of return actions, and thus, from time point $t_5$, no method of $t$ returns. Since from time point $t_5$, $t$ runs as in SC memory model, only process $pro$ can run, and ``the memory valuation used at time point $t_5$'' is $d_5[u_5(pro)]$, we can see that $(p_5(pro),d_5[u_5(pro)])$ is a blocking pair.

%\gpnote{What is $i$ here? Should it be $t_1$?}
  To prove the \emph{if} direction, %given an execution $t'$, a process $proc$ and integer $i$ as above
  % exists. %from time $i$, we can generate another execution $t''$
  we generate an execution $t$ by first doing $t'$ transitions from the initial configuration to %time $i$,
  $(p,d,u)$ and then continuing to run process $proc$, and disallowing other processes to do actions (including flush actions). %and flush each buffered item as soon as it is inserted into buffer.
% {\color {blue}It is easy to see that from time point $i$, process $pro$ runs
% as in SC memory model (while the memory valuation is obtained from memory
% valuation of the system and buffered write of process $pro$).}
%Executions obtained by such manner will run as in SC memory model.
It is easy to see %this kind of executions behaves as on SC.
that such executions of $t$ from %time $i$
$(p,d,u)$ behaves as the executions of $\llbracket \mathcal{L},1 \rrbracket_{sc}$ from the configuration %$(p',d[u(proc)],u_{\textit{init}})$,
$(p',d,u_{\textit{init}})$, where $p'$ is a function that maps process $1$ to $p(proc)$.
Moreover, we require the execution of $t$ from %time $i$
$(p,d,u)$ to satisfy $(G\ \neg P_{\textit{ret}})
\wedge (G\ X\ P_{\textit{any}})$. This %is feasible
holds since %$(p(proc),d[u(proc)])$
$(p(proc),d)$ is a blocking
pair. Thus, we can see that $t$ has infinite length and does not contain
any return action after %its $i$-th transition,
it reaches configuration $(p,d,u)$, which implies that $t$ violates
obstruction-freedom. \qed
\end {proof}
}


\forget{
\begin{lemma}
\label{lemma:equivalent characterization of obstruction-freedom}
Given a library $\mathcal{L}$, there exists an infinite execution $t$ of
$\llbracket \mathcal{L}, n \rrbracket$ that violates obstruction-freedom, if and
only if there exists an infinite execution $t'$ of $\llbracket
%\mathcal{L}_{in},
\mathcal{L},n \rrbracket$, a process $proc$ and an integer $i$, such that
the $i$-th transition of $t'$ leads to a configuration $(p,d,u)$, such
that %$p(pro)=(q_{(u,v)},q'_c)$ is a intermediate state, and
% $(p(pro),d[u(pro)])$
%$(p(pro),d[u(pro)])$
%$(q_v,d)$
$(p(proc),d[u(proc)])$ is a blocking pair. %Here $d[u(pro)]$ is a memory valuation obtained from $d$ by using the buffered write of $u(pro)$ to update $d$ one by one.
\end{lemma}
\begin {proof}
  Let us prove the only if direction. Since $t$ violates
    obstruction-freedom, there exists a process $proc$ and a time $t_1$, such that
    from $t_1$ only the process $proc$ can launch actions on $t$.
    Since there is a finite number of write actions of $t$ from its beginning to
    time $t_1$, there exists a time $t_2$, such that after $t_2$, only the
    process $proc$ can do flush actions.
    Let time $t_3$ be the point of the last call action of process
    $proc$, and let $t_4 = \textit{max}(t_1,t_2,t_3)$.
    Let $(p_4,d_4,u_4)$ be the configuration of time $t_4$.
    {It is easy to see that from time $t_4$, process $proc$ is not
      influenced by any other process, and the behavior of process $proc$ from then on is as on SC.}
    Therefore, in $\llbracket \mathcal{L},1 \rrbracket_{sc}$, from $(p_4(proc),d_4[u_4(proc)],u_{\textit{init}})$, we can do the remaining transitions of $t$ from time point $t_4$. We can see that $t'=t$, $i=t_4$, and $(p_4(proc),d_4[u_4(proc)])$ is a blocking pair.

    \forget{Let us generate an execution $t'$ as follows: $t'$ first does $t$ transitions
    from the initial configuration to $(p_4,d_4,u_4)$.
    Then, $t'$ flushes all buffered items in process $proc$'s buffer and then the
    process $proc$ goes to a intermediate state. Assume that $t'$ reaches a
    configuration $(p,d,u)$, with $p(proc)$ some intermediate state $q_{(u,v)}$.
    Finally, $t'$ does the remaining transitions of $t$. Since $t'$ also
    violates obstruction-freedom, and $t$'s behavior is as in SC after time
    $t_4$, we can see that $(q_v,d)$ is a blocking pair.}
%If $p_4(pro)$ is a intermediate state, then let $(p'_4,d'_4,u'_4)=(p_4,d_4,u_4)$. Otherwise, let $(p'_4,d'_4,u'_4)$ be obtained from $(p_4,d_4,u_4)$ by first doing several flush of process $pro$ and then let process $pro$ transit to intermediate state. The buffer of process $pro$ is empty in $(p'_4,d'_4,u'_4)$. Assume that $p'_4(pro)=(q_{(j,k)},q'_c)$.
%It is easy to see that in $\llbracket \mathcal{L}, 1 \rrbracket_{sc}$, we could generate transitions from $(\{(P_1:(q_k,q'_c))\},d'_4,\epsilon)$ that do the $t$ transitions after $(p_4,d_4,u_4)$ (except for the flush actions and into intermediate state transition). Since $t$ is a obstruction-freedom violation, we can see that $(q_k,d'_4)$ is a blocking pair.

%Let $d_4[u_4(pro)]$ represents a memory valuation obtained from $d_4$ by using the buffered write of $u_4(pro)$ to update $d_4$ one by one. It is easy to see that

%Then, it is easy to see that, after time point $t_3$, process $pro$ runs as in SC memory model (while the memory valuation is obtained from memory valuation of the system and buffered write of process $pro$).

%Let time point $t_4$ be the time point of the last call action of process $pro$, and let $t_5 = \textit{max}(t_3,t_4)$. Let $(p_5,d_5,u_5)$ be the configuration of time point $t_5$. Since $t$ is a obstruction-freedom violation, $t$ has no return action from $(p_5,d_5,u_5)$.
%The execution of $t$ from time point $t_5$ is the same as the execution from $(p_5, d_5[u_5(pro)], \epsilon)$ in $\llbracket \mathcal{L}, 1 \rrbracket_{sc}$. Therefore, $(p_5(pro), d_5[u_5(pro)])$ is a blocking pair.


%and let $(p_5,d_5,u_5)$ be the configuration of time point $t_5$. Since $t$ is a obstruction-freedom violation, $t$ contains finite number of return actions, and thus, from time point $t_5$, no method of $t$ returns. Since from time point $t_5$, $t$ runs as in SC memory model, only process $pro$ can run, and ``the memory valuation used at time point $t_5$'' is $d_5[u_5(pro)]$, we can see that $(p_5(pro),d_5[u_5(pro)])$ is a blocking pair.

%\gpnote{What is $i$ here? Should it be $t_1$?}
To prove the if direction, given an execution $t'$, a process $proc$ and integer $i$ as above
exists. %from time $i$, we can generate another execution $t''$
We generate an execution $t$ by first doing $t'$ transitions from the initial configuration to time $i$, and then continuing to run process $proc$, and disallowing other processes to do actions
or flush. %and flush each buffered item as soon as it is inserted into buffer.
% {\color {blue}It is easy to see that from time point $i$, process $pro$ runs
% as in SC memory model (while the memory valuation is obtained from memory
% valuation of the system and buffered write of process $pro$).}
%Executions obtained by such manner will run as in SC memory model.
It is easy to see %this kind of executions behaves as on SC.
that such executions of $t$ from time $i$ behave as the executions of $\llbracket \mathcal{L},1 \rrbracket_{sc}$ from the configuration $(p',d[u(proc)],u_{\textit{init}})$, where $p'$ is a function that maps process $1$ to $p(proc)$.
Moreover, we require the execution of $t$ from time $i$ to satisfy $G \neg P_{\textit{ret}}
\wedge G X P_{\textit{any}}$. This is feasible since $(p(proc),d[u(proc)])$ is a blocking
pair. Thus, we can see that $t$ has infinite length and does not contain
any return action after its $i$-th transition, which implies that $t$ violates
obstruction-freedom. \qed
\end {proof}
}

\forget{
\begin {proof}
To prove the only if direction, since $t$ violates obstruction-freedom, there exists process $pro$ and time point $t_1$, such that from time point $t_1$, only process $pro$ can launch active on $t$. Since the total number of write actions of non-$pro$ process is finite, there exists a time point $t_2$, such that after time point $t_2$, only process $pro$ can do flush operation and all non-$pro$ process can not do flush operation. Let $t_3 = \textit{max}(t_1,t_2)$. Then, after time point $t_3$, we can see that in the concurrent system, only one process $pro$ can run while all other process can not either do action nor do flush. It is easy to see that, in this situation, process $pro$ runs as in SC memory model. Let time point $t_4$ be the time point of the last call action of process $pro$, let $t_5 = \textit{max}(t_3,t_4)$. Let $(p_5,d_5,u_5)$ be the configuration of time point $t_5$. Since $t$ is a obstruction-freedom violation, we can see that from time point $t_5$, (1) only process $pro$ can run, and its behavior is as one process run in SC memory model from control state $p_5(pro)$ and memory valuation $d_5[u_5(pro)]$, (2) there is infinite number of actions but no return actions. Here $d_5[u_5(pro)]$ represents a memory valuation obtained from $d_5$ by using the buffered write of $u_5(pro)$ to update $d_5$ one by one. Therefore, we can see that $(p_5(pro), d_5[u_5(pro)])$ is a blocking pair. We could generate another execution $t'$ from $t$, and the behavior of $t'$ is as follows: $t'$ first do behavior of $t$ till time point $t_5$, then $t'$ flush all buffered write of process $pro$, and then $t'$ do behavior of $t$ after time point $t_5$. It is easy to see that $t'$ belongs to $\llbracket \mathcal{L}, n \rrbracket$, and it reaches a configuration $(p_5,d_5[u_5(pro)],u_5[pro:\epsilon])$ with $(p_5(pro), d_5[u_5(pro)])$ being a blocking pair.


%and let $(p_5,d_5,u_5)$ be the configuration of time point $t_5$. Since $t$ is a obstruction-freedom violation, $t$ contains finite number of return actions, and thus, from time point $t_5$, no method of $t$ returns. Since from time point $t_5$, $t$ runs as in SC memory model, only process $pro$ can run, and ``the memory valuation used at time point $t_5$'' is $d_5[u_5(pro)]$, we can see that $(p_5(pro),d_5[u_5(pro)])$ is a blocking pair.

To prove the if direction, assume that such process $pro$ and integer $i$ exists. Then, from time point $i$, we can generate another execution $t'$ by continuing run process $pro$ and do not let other processes to do action or do flush. Executions obtained by such manner will run as in SC memory model. Moreover, we require the execution $t'$ to satisfy $E (G \neg \textit{isRet} \wedge G X \textit{true})$. This is feasible since $(p(pro),d[u(pro)])$ is a blocking pair. Therefore, we can see that $t'$ has infinite length and does not contain any return action after its $i$-th transition, which implies that $t'$ violates obstruction-freedom. \qed
\end {proof}
}

\forget{
Given a library $\mathcal{L}$, let $\llbracket \mathcal{L}, n \rrbracket_{of}$ be an LTS that extends $\llbracket \mathcal{L}, n \rrbracket$ by remembering if a configuration that contains a blocking pair has already been reached. Each configuration of $\llbracket \mathcal{L}, n \rrbracket_{of}$ is a tuple $(p,d,u,bpflag)$ where $(p,d,u)$ is a configuration of $\llbracket \mathcal{L}, n \rrbracket$ and $bpflag \in \{ 0,1 \}$ records if the execution already reach a configuration that contains a blocking pair. The transition rule of $\llbracket \mathcal{L}, n \rrbracket_{of}$ is generated from that of $\llbracket \mathcal{L}, n \rrbracket$ as follows: Let $\rightarrow_{of}$ be the transition relation of $\llbracket \mathcal{L}, n \rrbracket_{of}$ and $\rightarrow$ be the transition relation of $\llbracket \mathcal{L}, n \rrbracket$: If $(p,d,u) {\xrightarrow{ \alpha }} (p',d',u)$, then we have $(p,d,u,bpflag) {\xrightarrow{ \alpha }}_{of}$ $(p',d',u',bpflag')$. $bpflag'$ is obtained as follows: if $bpflag$ is $0$ and  %$p'(pro)=q_{(u,v)}$ and $(q_v,d')$
$(p'(proc),d'[u'(proc)])$ is a blocking pair for some process $proc$, %$q_u$ and $q_v$,
then $bpflag'$ is set to $1$; Otherwise, $bpflag'$ equals $bpflag$.
The initial configuration of $\llbracket \mathcal{L}, n \rrbracket_{of}$ is
$(p_{\textit{init}}, d_{\textit{init}}, u_{\textit{init}},0)$.
}

%\begin{itemize}
%\item[-] For $\tau$, read, write, $\textit{cas-fail}$, call and return actions, we have $(p,d,u,bflag) {\xrightarrow{ \alpha }}_o (p',d'$, $u',bflag)$, if we have $(p,d,u) {\xrightarrow{ \alpha }} (p',d',u)$.

%\item[-] For flush and $\textit{cas-suc}$ actions, if $(p,d,u) {\xrightarrow{ \alpha }} (p',d',u)$, then we have $(p,d,u,bflag) {\xrightarrow{ \alpha }}_o$ $(p',d',u',bflag')$. Here if $bflag=0$ and $(p'(pro),d'[u'(pro)])$ is a blocking pair for some process $pro$, then $bflag'=1$; Otherwise, $bflag'=bflag$.
%\end{itemize}

\forget{
Lemma \ref{lemma:equivalent characterization of obstruction-freedom} reduces checking obstruction-freedom into checking if any configuration contains a blocking pair.
Given a finite execution $t$ of $\llbracket \mathcal{L}, n \rrbracket_{of}$ from
the initial configuration to a configuration $(p,d,u,bpflag)$, some intermediate
configuration of $t$ contains a blocking pair,
if and only if $bpflag=1$. This implies the following lemma, which reduces
checking obstruction-freedom into the finite trace reachability problem for
$\llbracket \mathcal{L}, n \rrbracket_{of}$.
}

\forget{
We can see that
given an finite execution $t$ of $\llbracket \mathcal{L}, n
\rrbracket_{of}$ from the initial configuration to a configuration
$(p,d,u,$ $bpflag)$, some intermediate configuration of $t$ contains blocking pair,
if and only if $bpflag=1$. This implies the following lemma, which reduces
checking obstruction-freedom into the finite trace reachability problem for
$\llbracket \mathcal{L}, n \rrbracket_{of}$.
}

\forget{
\begin{lemma}
\label{lemma:reducing obstruction free to a reachability problem of ObsSem(L,n)}
Given a library $\mathcal{L}$, there exists an infinite execution of $\llbracket \mathcal{L}, n \rrbracket$ that violates obstruction-freedom, if and only if there is a finite trace of $\llbracket \mathcal{L}, n \rrbracket_{of}$ that reaches a configuration $(p,d,u_{\textit{init}},1)$ for some $p$ and $d$.
\end{lemma}
\begin {proof}
  % Directly consequence of %Lemma \ref{lemma:equivalent characterization of obstruction-freedom} and the construction of $\llbracket \mathcal{L}, n \rrbracket_{of}$.
  By Lemma \ref{lemma:equivalent characterization of obstruction-freedom} and the construction of $\llbracket \mathcal{L},n \rrbracket_{of}$, we reduce obstruction-freedom into reachability of $(p,d',u,1)$ in $\llbracket \mathcal{L},n \rrbracket_{of}$. Since flushing items does not influence $bpflag$, we further reduce it into reachability of $(p,d,u_{\textit{init}},1)$ in $\llbracket \mathcal{L},n \rrbracket_{of}$. \qed
\end {proof}
}



\forget{
\subsection{Construction of \redt{$\textit{CM}_i$}}
\label{lemma:channel Machines Mi}

%Let us new use the channel machine to simulate a single process of $\llbracket \mathcal{L}, n \rrbracket_{of}$.
We now construct a channel machine \redt{$\textit{CM}_i$} to simulate process $i$ of $\llbracket \mathcal{L}, n \rrbracket_{of}$. %is a channel machine that is used to simulate process $i$ on TSO.
Its construction is similar to the channel machines of Atig \emph{et al.} \cite{DBLP:conf/popl/AtigBBM10} and our previous work \cite{DBLP:conf/sofsem/WangLW16}.
Our work extends the channel machines of \cite{DBLP:conf/popl/AtigBBM10} and
\cite{DBLP:conf/sofsem/WangLW16} by allowing to read %the control state,
a memory valuation updated by process $i$'s buffer content atomically to detect blocking pairs.
In \redt{$\textit{CM}_i$}, call and return actions are treated as $\epsilon$ transitions.
%\redt{In the next subsection, we use $M_i$ to solve the equivalent characterization of obstruction-freedom of Lemma \ref{lemma:reducing obstruction free to a reachability problem of ObsSem(L,n)}.}
%With this extension, we are able to use $M_i$ to solve the equivalent characterization of obstruction-freedom of Lemma %\ref{lemma:equivalent characterization of obstruction-freedom}.
%\ref{lemma:reducing obstruction free to a reachability problem of ObsSem(L,n)}.

Let $\textit{Val}$ be the set of memory valuations, and each memory valuation is a function that maps a memory location in $\mathcal{X}_{\mathcal{L}}$ to a value in $\mathcal{D}_{\mathcal{L}}$. The $(S,k)$-channel machine \redt{$\textit{CM}_i$} ($1 \leq i \leq n$) is a tuple $(Q_i,\{c_i\}, \Sigma, \Lambda,\Delta_i)$, where $c_i$ is name of the single channel of \redt{$\textit{CM}_i$}. $Q_i$, $c_i$, $\Sigma$, $\Lambda$ and $\Delta_i$ are defined as follows: Let $Q_{\mathcal{L}}$ be the set of program positions of $\mathcal{L}$ and $\rightarrow_{\mathcal{L}}$ be the transition relation of $\mathcal{L}$,
$Q_i=( \{q_c\} \cup (Q_{\mathcal{L}} \times \{q'_c\}) ) \times \textit{Val} \times \textit{Val} \times \textit{Val} \times \{ 0,1 \}$ is the state set.
A state $(q,d_c,d_g,d_b,bpflag)\in Q_i$ consists of a control state $q$,
a valuation $d_c$ of the current memory, a valuation $d_g$ of the memory with
all the stored items in $c_i$ applied, a valuation $d_b$ of the memory with
all the stored items of process $i$ in $c_i$ applied, and a flag $bpflag$ that is used to
detect if the current execution %of process $i$
has already reached a
configuration that contains a blocking pair of process $i$. %, and a $tflag$ that is used to detect if the channel contains no strong symbols. Note that we can set $tflag$ at a random time, and when it is set, the execution is terminated.

$\Sigma=\Sigma_1 \cup \Sigma_2$ is the alphabet of channel contents with
$\Sigma_1=\{ (i,x,d) \vert 1 \! \leq \! i \! \leq \! n, x \in
\mathcal{X}_{\mathcal{L}}, d \in \textit{Val} \}$ and $\Sigma_2 = \{ (a,\sharp)
\vert a \in \Sigma_1 \}$. $\Sigma_1$ represents a buffered write and it stores the whole memory valuation. $\Sigma_2$ is used to represent the newest write of a variable. In case that \redt{$\textit{CM}_i$} is interpreted with a lossy channel, $\Sigma_2$ are the sets of strong symbols of \redt{$\textit{CM}_i$} and the number of strong symbols is less or equal to the size of $\mathcal{X}_{\mathcal{L}}$.
$\Lambda$ is the set of transition labels and is the union of $\{ \epsilon \}$ and $\{ \textit{write}(i,x,d), \textit{flush}(i,x,d)$, $\textit{cas}(i,x,d,d')  \vert 1 \leq i \leq n, x \in \mathcal{X}_{\mathcal{L}}, d,d' \in %\textit{Val}
\mathcal{D}_{\mathcal{L}} \}$. $\Lambda$ does not contain $\tau$, read, call or return transitions, which are seen as $\epsilon$ transition in \redt{$\textit{CM}_i$}.
$\Delta_i$ is the transition relation of \redt{$\textit{CM}_i$}, and it is the smallest set of transitions such that for each $q \in \{q_c\} \cup (Q_{\mathcal{L}} \times \{q'_c \}$, $q_1,q_2 \in Q_{\mathcal{L}}$ and $d_c,d_g,d_b \in \textit{Val}$,

\begin{itemize}[leftmargin=*]
\item[-] Nop: if $q_1 {\xrightarrow{ \tau }}_{\mathcal{L}} q_2$, then

$((q_1,q'_c),d_c,d_g,d_b,bpflag)
{\xrightarrow{ \epsilon, c_i:\Sigma,\textit{nop} }}_{\Delta_i}
((q_2,q'_c),d_c,d_g,d_b,bpflag')$

%If $q_2$ is the intermediate state for some $q_u$ and $q_v$, $(q_v,d_c)$ is a blocking pair, and $bpflag=0$, then $bpflag'=1$; Otherwise, $bpflag'=bpflag$. %The other cases of change from $bpflag$ to $bpflag'$ are similar and is omitted here.
In the destination state, if the pair of the first or fourth tuples is a blocking pair and $bpflag=0$, then $bpflag'=1$; Otherwise, $bpflag'=bpflag$. %The other cases of change from $bpflag$ to $bpflag'$ in other transition rules are the same and is omitted here.
The other transition rules use the same approach to modify $bpflag$ and we omit such approach when showing other transition rules.

\item[-] Library write: if $q_1 {\xrightarrow{ \textit{write}(x,a) }}_{\mathcal{L}} q_2$, then for each $d \in \textit{Val}$

\vspace{-12pt}
$$((q_1,q'_c),d_c,d_g,d_b,bpflag)
{\xrightarrow{ \textit{op}, (\alpha,\sharp) \in c_i, c_i[\alpha / (\alpha,\sharp)]!\alpha' }}_{\Delta_i}
((q_2,q'_c),d_c,d'_g,d'_b,bpflag')$$
$$((q_1,q'_c),d_c,d_g,d_b,bpflag)
{\xrightarrow{ \textit{op}, c_i:\Theta,c_i!\alpha' }}_{\Delta_i}
((q_2,q'_c),d_c,d'_g,d'_b,bpflag')$$

where $\alpha=(i,x,d)$, $d'_g = d_g[x:a]$, $d'_b=d_b[x:a]$, $\alpha' = ((i,x,d'_g),\sharp)$, $\Theta = \Sigma \backslash \{ ((i,x,d'),\sharp) \vert d' \in \textit{Val}\}$ and $\textit{op} = \textit{write}(i,x,a)$. %If $((q_2,q'_c),d_c)$ is a blocking pair and $bpflag=0$, then $bpflag'=1$; Otherwise, $bpflag'=bpflag$.
%If $q_2$ is the intermediate state for some $q_u$ and $q_v$, $(q_v,d_c)$ is a blocking pair, and $bpflag=0$, then $bpflag'=1$; Otherwise, $bpflag'=bpflag$.

\item[-] Guess write: if $1 \leq j \leq n$, $j \neq i$ and $x \in \mathcal{X}_{\mathcal{L}}$, then
$$(q,d_c,d_g,d_b,bpflag)
{\xrightarrow{ \textit{op}, c_i: \Sigma,c_i!\alpha }}_{\Delta_i}
(q,d_c,d'_g,d_b,bpflag')$$
where $d'_g=d_g[x:a]$, $\alpha = (j,x,d'_g)$ and  $\textit{op} = \textit{write}(j,x,a)$.

\item[-] Flush: for each $1 \leq j \leq n$ and $x \in \mathcal{D}_{\mathcal{L}}$,
$$(q,d_c,d_g,d_b,bpflag)
{\xrightarrow{ \textit{op}, c_i:\Sigma, c_i?(j,x,d) }}_{\Delta_i}
(q,d,d_g,d_b,bpflag')$$
$$(q,d_c,d_g,d_b,bpflag)
{\xrightarrow{ \textit{op}, c_i:\Sigma, c_i?((j,x,d),\sharp) }}_{\Delta_i}
(q,d,d_g,d_b,bpflag')$$
where $\textit{op}=\textit{flush}(j,x,d(x))$. %If $(q,d)$ is a blocking pair and $bpflag=0$, then $bpflag'=1$; Otherwise, $bpflag'=bpflag$.
%If $q=(q_2,q'_c)$, $q_2$ is the intermediate state for some $q_u$ and $q_v$, $(q_v,d)$ is a blocking pair, and $bpflag=0$, then $bpflag'=1$; Otherwise, $bpflag'=bpflag$.

%$$(q,d_c,d_g,bflag)
%{\xrightarrow{ \textit{op}, c_i:\Sigma, c_i?(j,x,d) }}_{\Delta_i}
%(q,d'_c,d_g,bflag')$$
%$$(q,d_c,d_g,bflag)
%{\xrightarrow{ \textit{op}, c_i:\Sigma, c_i?((j,x,d),\sharp) }}_{\Delta_i}
%(q,d'_c,d_g,bflag')$$
%where $d'_c = d_c[x:d(x)]$ and $\textit{op}=\textit{flush}(j,x,d(x))$. If $(q,d'_c)$ is a blocking pair and $bflag=0$, then $bflag'=1$; Otherwise, $bflag'=bflag$.

\item[-] Library read: if $q_1 {\xrightarrow{ \textit{read}(x,a) }}_{\mathcal{L}} q_2$, then for each $d \in \textit{Val}$ with $d(x)=a$,
$$((q_1,q'_c),d_c,d_g,d_b,bpflag)
{\xrightarrow{ \epsilon, (\beta,\sharp) \in c_i,\textit{nop} }}_{\Delta_i}
((q_2,q'_c),d_c,d_g,d_b,bpflag')$$
$$((q_1,q'_c),d,d_g,d_b,bpflag)
{\xrightarrow{ \epsilon, c_i:\Theta,\textit{nop} }}_{\Delta_i}
((q_2,q'_c),d,d_g,d_b,bpflag')$$
where $\beta=(i,x,d)$ and $\Theta=\Sigma \backslash \{ ((i,x,d'),\sharp) \vert d' \in \textit{Val} \}$. %If $((q_2,q'_c),d_c)$ of the first case (resp., $((q_2,q'_c),d)$ of the second case) is a blocking pair and $bpflag=0$, then $bpflag'=1$; Otherwise, $bpflag'=bpflag$.
%If $q_2$ is the intermediate state for some $q_u$ and $q_v$, $(q_v,d_c)$ of the first case (resp., $(q_v,d)$ of the second case) is a blocking pair, and $bpflag=0$, then $bpflag'=1$; Otherwise, $bpflag'=bpflag$.

\item[-] Library $\textit{cas}$: if $q_1 {\xrightarrow{ \textit{cas}\_\textit{suc}(x,a,b) }}_{\mathcal{L}} q_2$ , then for each $d \in \textit{Val}$ with $d(x)=a$,
$$((q_1,q'_c),d,d,d_b,bpflag)
{\xrightarrow{ \textit{cas}(i,x,a,b), c_i=\epsilon, \textit{nop} }}_{\Delta_i}
((q_2,q'_c),d[x:b],d[x:b],d_b[x:b],bpflag')$$


%Here if $((q_2,q'_c),d[x:b])$ is a blocking pair and $bpflag=0$, then $bpflag'=1$; Otherwise, $bpflag'=bpflag$.
%If $q_2$ is the intermediate state for some $q_u$ and $q_v$, $(q_v,d[x:b])$ is a blocking pair, and $bpflag=0$, then $bpflag'=1$; Otherwise, $bpflag'=bpflag$.

If $q_1 {\xrightarrow{ \textit{cas}\_\textit{fail}(x,a,b) }}_{\mathcal{L}} q_2$ , then for each $d \in \textit{Val}$ with $d(x) \neq a$,
$$((q_1,q'_c),d,d,d_b,bpflag)
{\xrightarrow{ \textit{cas}(i,x,a,a), c_i=\epsilon, \textit{nop} }}_{\Delta_i}
((q_2,q'_c),d,d,d_b,bpflag')$$

%Here if $((q_2,q'_c),d)$ is a blocking pair and $bpflag=0$, then $bpflag'=1$; Otherwise, $bpflag'=bpflag$.
%If $q_2$ is the intermediate state for some $q_u$ and $q_v$, $(q_v,d)$ is a blocking pair, and $bpflag=0$, then $bpflag'=1$; Otherwise, $bpflag'=bpflag$.

%\item[-] Set the terminate flag: $$(q,d,d,bflag,0)
%{\xrightarrow{ \epsilon, c_i:\Theta, \textit{nop} }}_{\Delta_i}
%(q,d,d,bflag,1)$$
%where $\Theta=\Sigma \backslash \{ ((i,x,d'),\sharp) \vert x \in \mathcal{X}_{\mathcal{L}}, d' \in \textit{Val} \}$.

\item[-] Call and return: the call and return actions are modelled as $\epsilon$ transitions of \redt{$\textit{CM}_i$}:

$(q_c,d_c,d_g,d_b,bpflag)
{\xrightarrow{ \epsilon, c_i:\Sigma,\textit{nop} }}_{\Delta_i}
((\textit{is}_{(\textit{m,a})},q'_c),d_c,d_g,d_b,bpflag')$.

$((\textit{fs}_{(\textit{m,a})},q'_c),d_c,d_g,d_b,bpflag)
{\xrightarrow{ \epsilon, c_i:\Sigma,\textit{nop} }}_{\Delta_i}
(q_c,d_c,d_g,d_b,bpflag')$.

%If $((\textit{is}_{(\textit{m,a})},q'_c),d_c)$ of the first case (resp., $(q_c,d_c)$ of the second case) is a blocking pair and $bpflag=0$, then $bpflag'=1$; Otherwise, $bpflag'=bpflag$.
\end{itemize}

%Note that only flush action and $\textit{cas}$ action can change the flag $z$.
}


\forget{
\subsection{Obstruction-Freedom is Decidable}
\label{lemma:obstruction-freedom is decidable}

Let \redt{$\textit{CM}_i^w$} (\redt{$\textit{CM}_i^f$}) be a channel machine that is obtained from \redt{$\textit{CM}_i$} by
replacing all of its transitions but write (flush) and $\textit{cas}$ with internal transitions, and the remaining $\textit{cas}$ actions as write (flush) actions.


%\gpnote{What does ``is the production of'' mean here?}
The following lemma reduces the finite trace reachability problem of Lemma \ref{lemma:reducing obstruction free to a reachability problem of ObsSem(L,n)} into checking emptiness of $\bigcap_{i=1}^n T_{( q_i,q'_i )}^{(S,k)} \redt{\textit{CM}_i^f}$. Such problem is equivalent to a control state reachability problem of a perfect channel machine that is the production of \redt{$\textit{CM}_1^f$} to \redt{$\textit{CM}_n^f$} by Lemma \ref{proposition:relation bewteen LT of M1 and M2 and (LT of M1 and LT of M2)}. The proof of this lemma can be found in Appendix \ref{subsec: proof of lemma lemma:reduce the existence of exeuction of ObsSem(L,n) to the reachability problem of production of M1f to Mnf}. %Given a sequence $l=a_1 \cdot a_2 \cdot \ldots \cdot a_k$, let $l(i,j)=a_i \cdot \ldots \cdot a_j$.

\begin{lemma}
\label{lemma:reduce the existence of exeuction of ObsSem(L,n) to the reachability problem of production of M1f to Mnf}
Given a library $\mathcal{L}$, $\llbracket \mathcal{L}, n \rrbracket_{of}$ has an execution from $(p_{\textit{init}}, d_{\textit{init}}, u_{\textit{init}}, 0)$ to $(p, d, u_{\textit{init}}, 1)$, if and only if $\bigcap_{i=1}^n T_{( q_i,q'_i )}^{(S,k)} \redt{\textit{CM}_i^f} \neq \emptyset$. Here for each process $1 \! \leq \!i \! \leq \! n$, $q_i=( p_{\textit{init}}(i), d_{\textit{init}}, d_{\textit{init}}, d_{\textit{init}}, 0)$, $q'_i=( p(i), d, d, d_{bi}, bpflag_i)$ for some $d_{bi}$ and $bpflag_i$. Moreover, $bpflag_j=1$ for some process $j$.
\end{lemma}


The following lemma reduces the control state reachability problem of the production of \redt{$\textit{CM}_1^f$} to \redt{$\textit{CM}_n^f$} as a perfect channel machine to the control state reachability problem of the production of \redt{$\textit{CM}_1^w$} to \redt{$\textit{CM}_n^w$} as a perfect channel machine. Its proof can be found in Appendix \ref{subsec:proof of lemma lemma:reduce the reachability problem of production of M1w to Mnw as perfect channel machine to that of M1f to Mnf}.

\begin{lemma}
\label{lemma:reduce the reachability problem of production of M1w to Mnw as perfect channel machine to that of M1f to Mnf}

$\bigcap_{i=1}^n T_{( q_i,q'_i )}^{(S,k)} \redt{\textit{CM}_i^f} \neq \emptyset$, if and only if $\bigcap_{i=1}^n T_{( q_i,q'_i )}^{(S,k)} \redt{\textit{CM}_i^w} \neq \emptyset$. Here for each process $1 \! \leq \!i \! \leq \! \textit{n+1}$, $q_i=( p_{\textit{init}}(i), d_{\textit{init}}, d_{\textit{init}}, d_{\textit{init}}, 0)$, $q'_i=( p(i), d, d, d_{bi}, bpflag_i)$ for some $d_{bi}$ and $bpflag_i$.
\end{lemma}
%\begin {proof}
%This is obvious, since the channel is perfect and FIFO, and each $M_i^w$ and $M_i^f$ have their channel empty in the end of the execution. \qed
%\end {proof}

The following lemma reduces the control state reachability problem of the
production of \redt{$\textit{CM}_1^w$} to \redt{$\textit{CM}_n^w$} as a perfect channel machine to the control
state reachability problem of the production of \redt{$\textit{CM}_1^w$} to \redt{$\textit{CM}_n^w$} as a lossy
channel machine. %Note that the latter is known to be decidable \cite{DBLP:conf/popl/AtigBBM10}.
Its proof can be found in Appendix \ref{subsec:proof of lemma lemma:reduce the reachability problem of production of M1w to Mnw as perfect channel machine to that of lossy channel machine}.

\begin{lemma}
\label{lemma:reduce the reachability problem of production of M1w to Mnw as perfect channel machine to that of lossy channel machine}

$\bigcap_{i=1}^n T_{( q_i,q'_i )}^{(S,k)} \redt{\textit{CM}_i^w} \neq \emptyset$, if and only if $\bigcap_{i=1}^n LT_{( q_i,q''_i )}^{(S,k)} \redt{\textit{CM}_i^w} \neq \emptyset$. Here for each process $1 \! \leq \!i \! \leq \! \textit{n+1}$, $q_i=( p_{\textit{init}}(i), d_{\textit{init}}, d_{\textit{init}}, d_{\textit{init}}, 0)$, $q'_i=( p(i), d, d, d_{bi}, bpflag'_i)$, $q''_i=( p(i), d, d, d_{bi}, bpflag''_i)$ for some $d_{bi}$, $bpflag'_i$ and $bpflag''_i$. Moreover, $bpflag'_j=1$ for some $j$, if and only if $bpflag''_k=1$ for some $k$.
\end{lemma}

The following theorem states that obstruction-freedom is decidable on TSO for $n$ processes. This is a direct consequence of Lemma \ref{lemma:equivalent characterization of obstruction-freedom} %, Lemma \ref{lemma:reducing obstruction free to a reachability problem of ObsSem(L,n)}, Lemma \ref{lemma:reduce the existence of exeuction of ObsSem(L,n) to the reachability problem of production of M1f to Mnf}, Lemma \ref{lemma:reduce the reachability problem of production of M1w to Mnw as perfect channel machine to that of M1f to Mnf} and
to Lemma \ref{lemma:reduce the reachability problem of production of M1w to Mnw
  as perfect channel machine to that of lossy channel machine}, as well as the
decidability result of the control state reachability problem of lossy channel
machines \cite{DBLP:conf/popl/AtigBBM10}, and the fact that there are only
a finite number of such $p$, $d$ and $d_{bi}$.

\begin{theorem}
\label{theorem:obstruction-freedom is decidable}
The problem of checking obstruction-freedom of a given library %on fixed
for bounded number of processes is decidable.
\end{theorem}
}

\forget{
Note that it seems hard to directly reduce checking blocking pairs into state
reachability, since the latter problem requires each process to have an empty
buffer. Thus, we need to force each process to clear their buffer and this may
change the memory valuation. To check blocking pairs we need to recover memory
valuations, and thus, we require each process to state its previous buffer
content, which seems infeasible on TSO.
}

\forget{
\subsection{Obstruction-Freedom is Decidable}
\label{lemma:obstruction-freedom is decidable}
}

\forget{
Checking the existence of blocking pairs requires atomically reading the whole memory valuation, while we only have commands to atomically read one memory location. Although it seems hard to atomically read the whole memory valuation, in this subsection we propose a method to generate a basic TSO concurrent system %of \cite{DBLP:conf/popl/AtigBBM10}
and reduce checking blocking pairs into the state reachability problem of this basic TSO concurrent system.
}

\forget{
Informally, the desired basic TSO concurrent system is obtained from $\llbracket
\mathcal{L}, n \rrbracket$ by transforming call and return actions into internal
actions, and additionally allowing each process to non-deterministically check
blocking pairs.
%
When a process finishes checking it records the result in a new memory location
and then terminates the whole basic TSO concurrent system.
  %
In the check procedure we need to read the whole memory
valuation, and this procedure %The read of whole memory valuation
is performed in two phases. In the first phase a process reads from each memory
location $x \in \mathcal{X}$ and writes new special values (depending on the
value of $x$ and process ID) to mark them. In the second phase the process
checks if the value of each memory location of $\mathcal{X}$ has not been
overwritten by actions (including flush actions) of other processes.
}

%another process or by flush actions.

\forget{
% Given a library $\mathcal{L}$ = $(\mathcal{X}_{\mathcal{L}},\mathcal{M}, \mathcal{D}, Q_\mathcal{L},\rightarrow_\mathcal{L})$ and $n$, we introduce the following value and new memory locatons:
Assume that the data domain $\mathcal{D}=\{a_1,\ldots,a_u\}$, then we introduce new values $\{b_{i,j}\vert 1\leq i\leq n, 1\leq j\leq u\}$ which are used to mark memory locations of $\mathcal{X}$. We introduce a new memory location $\textit{result}$ with initial value $0$ to store the result of checking blocking pairs. %We introduce new memory locations $\{s_{proc}\vert 1\leq proc\leq n\}$ to store control state of each process.
We introduce a new memory location $\textit{terFlag}$ with initial value $0$, and we use
it as a flag to denote termination of the check %phase,
procedure, as well as the whole
basic TSO concurrent system.
}

\forget{
%{\color{orange} GP: The text below is really hard to read. I recommend using displays and breaking lines so that at least formulae fall on the same line.}
Formally, the tuple $(Q_{proc}, \Delta_{proc})$ of process $proc$ of the basic
TSO concurrent system for a given library $\mathcal{L}$ =
$(\mathcal{X}_{\mathcal{L}},\mathcal{M}, \mathcal{D},
Q_\mathcal{L},\rightarrow_\mathcal{L})$ is obtained as follows. Each control
state of $Q_{proc}$ is either a tuple $(\textit{in}_{\textit{clt}},proc,x)$ or
$(q_{\mathcal{L}},\textit{in}_{\textit{lib}},proc,x)$ (with $q_{\mathcal{L}} \in Q_\mathcal{L}$ and $x
\in \{1,2\}$), or states of $\textit{CheckBP}$ (described below). States with
$x=1$ can first check $\textit{terFlag}$ (and set $x$ into 2 if success) and
then do transitions according to $\rightarrow_{\mathcal{L}}$. Here
$x$ is used to check $\textit{terFlag}$ before doing ``library
transitions''. %as a flag for such intermediate step.
}

%Assume a library $\mathcal{L}$ = $(\mathcal{X}_{\mathcal{L}},\mathcal{M}, \mathcal{D}, Q_\mathcal{L},\rightarrow_\mathcal{L})$. A control state of process $proc$ of the basic TSO concurrent system is a tuple $(q_c,proc,x)$, or $(q_{\mathcal{L}},q'_c,proc,x)$, or states of $\textit{CheckBP}$ (described below). Here $q_{\mathcal{L}} \in Q_{\mathcal{L}}$ is a state of $\mathcal{L}$ and $x \in \{1,2\}$. States with $x=1$ are ``normal states'' while states with $x=2$ are ``intermediate states''. %, and states with $x=0$ are states that is checking blocking pairs, as explained in construction of $\Delta_{proc}$.

\forget{
The transition relation $\Delta_{proc}$ %of process $proc$ of the basic TSO concurrent system
is defined as follows: Here we use $q {\xrightarrow{\alpha}}_{\Delta_{proc}} q'$ to denote that $(q,\alpha,q') \in \Delta_{proc}$.

\begin{itemize}
\item[-] %$(q_c,proc,1)$ and
$(q_{\mathcal{L}},\textit{in}_{\textit{lib}},proc,1)$ can first check $\textit{terFlag}$ value and then do transitions according to $\mathcal{L}$. Formally, if $\alpha$ is a $\tau$, %$\textit{checkPID}$,
read, write or $\textit{cas}$ action and $q_{\mathcal{L}} {\xrightarrow{\alpha}}_{\mathcal{L}} q'_{\mathcal{L}}$ in $\mathcal{L}$, then we have $(q_{\mathcal{L}},\textit{in}_{\textit{lib}},proc,1)$ ${\xrightarrow{\textit{read}(proc, \textit{terFlag}, 0)}}_{\Delta_{proc}}$ $(q_{\mathcal{L}},\textit{in}_{\textit{lib}},proc,2)$ and $(q_{\mathcal{L}},\textit{in}_{\textit{lib}},proc$, $2) {\xrightarrow{\alpha'}}_{\Delta_{proc}} (q'_{\mathcal{L}},\textit{in}_{\textit{lib}},proc,1)$ in $\Delta_{proc}$, where $\alpha'$ is obtained from $\alpha$ by adding process ID $proc$.

\item[-] To deal with call actions, we first check $\textit{terFlag}$ and then transform call actions into $\tau$ actions. Formally, we have $(\textit{in}_{\textit{clt}},proc,1) {\xrightarrow{\textit{read}(proc, \textit{terFlag}, 0)}}_{\Delta_{proc}} (\textit{in}_{\textit{clt}},proc,2)$ and $(\textit{in}_{\textit{clt}},proc,2)$ ${\xrightarrow{\tau(proc)}}_{\Delta_{proc}} (\textit{is}_{(\textit{m,a})},\textit{in}_{\textit{lib}},proc,1) )$ in $\Delta_{proc}$. The case for return actions is similar and we have $(\textit{fs}_{(\textit{m,a})}, \textit{in}_{\textit{lib}}, proc,1) {\xrightarrow{\textit{read}(proc, \textit{terFlag}, 0)}}_{\Delta_{proc}} (\textit{fs}_{(\textit{m,a})},\textit{in}_{\textit{lib}},proc,2)$ and $(\textit{fs}_{(\textit{m,a})},\textit{in}_{\textit{lib}},proc,2)$ \\ ${\xrightarrow{\tau(proc)}}_{\Delta_{proc}}$ $(\textit{in}_{\textit{clt}},proc, 1)$ in $\Delta_{proc}$.

    %{\color {red}Recall that $\textit{is}_{(\textit{m,a})}$ and $\textit{fs}_{(\textit{m,a})}$ are two special program positions of $\mathcal{L}$}.

    %$( (\textit{fs}_{(\textit{m,a})}, q'_c$, $proc,1), \textit{read}(terFlag,0), (\textit{fs}_{(\textit{m,a})},q'_c,proc,2) ), ( (\textit{fs}_{(\textit{m,a})},q'_c,proc,2), \tau$, $(q_c,proc$, $1) ) \in \Delta_{proc}$

%When $x=1$, $(q_c,proc,1)$ and $(q_{\mathcal{L}},q'_c,proc,1)$ can do transitions according to $\mathcal{L}$ if $terFlag=0$, or stop otherwise. Formally, if $\alpha$ is a $\tau$ action, $\textit{getPID}$ action, a read, a write or $\textit{cas}$, and we have $q_{\mathcal{L}} {\xrightarrow{\alpha}}_{\mathcal{L}} q'_{\mathcal{L}}$ in $\mathcal{L}$, then we have $( (q_{\mathcal{L}},q'_c,proc,1), \textit{read}($ $\textit{terFlag}$, $0), (q_{\mathcal{L}},q'_c,proc,2) ), ( (q_{\mathcal{L}},q'_c,proc,2), \alpha, (q'_{\mathcal{L}},q'_c,proc,1) ) \in \Delta_{proc}$.

    %Call and return actions are considered as internal actions with the following transition:
    %For each $m \in \mathcal{M}$ and $a \in \mathcal{D}$, we have $( (q_c,proc,1), \textit{read}(terFlag,0), (q_c, proc,2))$, $( (q_c,proc,2), \tau, (\textit{is}_{(\textit{m,a})},q'_c,proc,1) ) \in \Delta_{proc}$ for ``call actions'', and $( (\textit{fs}_{(\textit{m,a})}, q'_c$, $proc,1), \textit{read}(terFlag,0), (\textit{fs}_{(\textit{m,a})},q'_c,proc,2) ), ( (\textit{fs}_{(\textit{m,a})},q'_c,proc,2), \tau$, $(q_c,proc$, $1) ) \in \Delta_{proc}$ for ``return actions''.

\item[-] $(\textit{in}_{\textit{clt}},proc,1)$ and $(q_{\mathcal{L}},\textit{in}_{\textit{lib}},proc,1)$ can
  non-deterministically %change $x$ to $0$. Then, they operate as $\textit{CheckBP}$ (described below)
  decide to execute $\textit{CheckBP}$ (described below).
\end{itemize}

%The control state of a process records its process id and the ``current state'' $q_c$ or $(q,q'_c)$ in $\llbracket \mathcal{L}, n \rrbracket$. Call and return actions are transformed into internal actions. When process $proc$ do a ``$\llbracket \mathcal{L}, n \rrbracket$'' action, it checks the value of $terFlag$, and stop working if its value is $1$.
We denote by $\textit{CheckBP}$ the sequence of transitions that are used to check blocking pairs. %If it runs on process $proc$ and the current ``library state'' is $s$, then $\textit{CheckBP}$ works as follows:
If $\textit{CheckBP}$ is started from $(\textit{in}_{\textit{clt}},proc,1)$ (resp., $(q_{\mathcal{L}},\textit{in}_{\textit{lib}},proc,1)$), then $\textit{CheckBP}$ works as follows:
%Given $s\in \{q_c\} \cup (Q_{\mathcal{L}}\times \{q'_c\})$, if $\textit{CheckBP}$ is started by $(s,proc,1)$, then $\textit{CheckBP}$ works as follows: %the following is used to check blocking pairs, and a process can nondeterministically do $checkBP$ at any time. $checkBP$ works as follows:

\begin{itemize}
\item[-] %Obtain process id (assume it is $proc$), and then
  Perform a $\textit{cas}$ command to clear the buffer.

\item[-] For each $x \in \mathcal{X}$, read a value from $x$ (we assume the
  value is $a_j$), and then use $\textit{cas}\_\textit{suc}(%proc,
  x,a_j, $ $b_{proc,j})$ to write a special value to $x$.

\item[-] Read the value of each $x \in \mathcal{X}$ again and check if %the value of each memory location of $\mathcal{X}$
  these value belongs to $\{b_{proc,j}\vert 1\leq j\leq u\}$. If so, check if %$s$
  $\textit{in}_{\textit{clt}}$ (resp., $(q_{\mathcal{L}},\textit{in}_{\textit{lib}})$) and $f(x)$ for $x \in \mathcal{X}$ is a blocking pair. Here $f$ is a function that maps each $b_{proc,j}$ into $a_j$. If it is a blocking pair, then we set $\textit{result}$ to $1$.

\item[-] Set $\textit{terFlag}$ to $1$.
\end{itemize}
}

% Lemma \ref{lemma:equivalent characterization of obstruction-freedom} reduces checking obstruction-freedom into the state reachability problem of configurations containing blocking pairs.
Since the model checking problem for CTL$^*$ formulas is decidable for finite state LTSs \cite{DBLP:reference/mc/2018}, we could compute the set of blocking pairs by first enumerating all
configurations of $\llbracket \mathcal{L}, 1 \rrbracket_{sc}$, and then use
model checking to check each of them. Thus, the configurations of the state reachability problem of Lemma \ref{lemma:equivalent characterization of obstruction-freedom} is computable. Since the the state reachability problem is decidable, we conclude that obstruction-freedom is decidable, as stated by the following theorem.

\forget{
With this basic TSO concurrent system, we reduce checking blocking pairs into checking if some configuration with $\textit{result}=1$ is reachable on the basic TSO concurrent system, which is known decidable \cite{DBLP:conf/popl/AtigBBM10}. The following theorem states that obstruction-freedom is decidable on TSO for $n$ processes. %It is proved by reducing checking blocking pairs into the state reachability problem of the TSO concurrent system above, as well as Lemma \ref{lemma:equivalent characterization of obstruction-freedom}
The proof can be found in Appendix \ref{subsec:appendix proof of theorem theorem:obstruction-freedom is decidable}. %\ref{sec:appendix proof of section sec:checking obstruction-freedom}.
}
\begin{theorem}
  \label{theorem:obstruction-freedom is decidable}
  The problem of checking obstruction-freedom of a given library for bounded number of processes is decidable on TSO.
\end{theorem}







%\input{CheckingBoundedWaitFreedom}

\section{Conclusion}
\label{sec:conclusion}

Liveness is an important property of programs, and using objects %libraries
with
incorrect liveness assumptions can cause problematic behaviors.
In this paper, we prove that lock-freedom, wait-freedom, deadlock-freedom and
starvation-freedom are undecidable on TSO for a bounded number of processes by reducing a known undecidable problem of lossy channel machines to checking liveness properties of specific libraries.
%specific libraries.
This library simulates the lossy channel machine $\textit{CM}'_{(A,B)}$ and is designed to contain at most two kinds of executions: If %all five
methods collaborate in a fair way and the lossy channel machine execution being simulated visits state $s_1$ infinitely many times, then the library executions violate all four liveness properties; otherwise, the library executions satisfy all four liveness properties.
Therefore, one library is sufficient for the undecidability proof of four liveness properties.
%Since the library used to show undecidability of lock-freedom (resp., wait-freedom) satisfies that all unfair executions satisfy both lock-freedom and deadlock-freedom (resp., both wait-freedom and starvation freedom), we use a same library to prove the undecidability of both lock-freedom and deadlock-freedom (resp., both wait-freedom and starvation-freedom).
%The possible executions of our library that violations of progress properties are in certain form and satisfies the fairness requirement.
Our undecidability proof reveals the intrinsic difference in liveness
verification between TSO and SC, resulting from the unbounded size of store buffers
in the TSO memory model.

Perhaps unexpectedly, we show that obstruction-freedom is decidable. %We prove that obstruction-freedom and bounded wait-freedom (given a bound) is decidable on TSO for a bounded number of processes by reducing it to a control state reachability problem of a lossy channel machine, which is known to be decidable.
%\redt{We reduce checking obstruction-freedom (resp., bounded wait-freedom with a bound) into a control state reachability problem of lossy channel machines (resp., state reachability problem of TSO concurrent systems), which is known decidable.}
Since each violation of obstruction-freedom eventually runs in isolation,
from some time point the violation, running on TSO, has the same behavior as on SC.
Therefore, checking whether a configuration contains a potential violation can be done by checking only this configuration itself, instead of checking all infinite executions from this configuration.
Checking obstruction-freedom is thus reduced to a known decidable reachability problem.
%Bounded wait-freedom is also checked by reducing to another known decidable reachability problem, since each violation must have a specific prefix that can be detected by remembering actions taken by each process.
\forget{Our decidability results for liveness are obtained by reducing liveness to some decidable reachability problem. For the case of obstruction-freedom, each violation is of infinite length.
However, such executions run in isolation eventually, which causes to eventually
run as on the SC memory model.
Thus, whether a configuration contains a potential violation is computable.
The case of bounded wait-freedom is easier since we only need to check a finite prefix to find the violation.
%The reason of decidability is that these two problem can be reduced into a finite trace problem on TSO.
We also prove that wait-freedom for libraries on TSO may not have a bound on steps for a bounded number of processes, while a wait-free library on SC must have bound on steps for a bounded number of processes.}
%We also prove that wait-freedom for libraries on TSO may not have a bound on steps for a bounded number of processes, while a wait-free library on SC must have bound on steps for a bounded number of processes. This reveals that wait-free libraries on TSO have subtle differences in terms of liveness when compared to SC.

%There are variants of liveness properties, such as $k$-bounded lock-freedom, bounded lock-freedom, $k$-bounded wait-freedom and bounded wait-freedom \cite{DBLP:conf/pldi/PetrankMS09}.
Other relaxed memory models, such as the memory models of ARM and POWER, %also
%use write buffers %\gpnote{$\leftarrow$ This is not necessarily true. There are behaviors similar to those
%  resulting from write-buffers, but they could be due to other artifacts.},
%and
are much weaker than TSO.
We conjecture that the undecidable liveness properties on TSO are still undecidable on ARM and POWER.
As future work, we would like to investigate the decidability of
obstruction-freedom %and bounded wait-freedom (given a bound)
on more relaxed memory models.
There are variants of liveness properties, such as $k$-bounded lock-freedom, bounded lock-freedom, $k$-bounded wait-freedom and bounded wait-freedom \cite{DBLP:conf/pldi/PetrankMS09}.
We would also like to investigate the decidability of bounded version of liveness properties on TSO and more relaxed memory models.

% \gpnote{$\leftarrow$ This doesn't sound very realistic to me.}

%%% Local Variables:
%%% mode: latex
%%% TeX-master: "CONCUR2021.tex"
%%% End: 

%\section{Conclusion and Future Work}
%\label{sec:conclusion and future work}

%conclusion


\bibliography{reference}
\bibliographystyle{plain}

%\bibliography{short}
%\bibliographystyle{plain}
%\bibliographystyle{splncs}
%\bibliography{biblio_cat.bib}


%\begin{thebibliography}{50}
%\bibitem{POPL2010}
%Atig, M.F., Bouajjani, A., Burckhardt, S., Musuvathi, M.:
%\newblock On the verification problem for weak memory models.
%\newblock In: Hermenegildo, M. et al. (eds.) POPL 2010, pp. 7--18. ACM (2010)
%\end{thebibliography}


%\newpage

%\appendix

%%% The Appendices part is started with the command \appendix;
%% appendix sections are then done as normal sections
\appendix
% \onecolumn
\subsection{Prompt Template}
\label{app1}

Figs. \ref{appendix:prompt1}-\ref{appendix:prompt3} (due to the page length, we break it into three parts) show the prompt design for the information extraction in the context of construction project scheduling. 

\begin{figure}[t]
    \vspace{-4.5cm}
    \centering
    \begin{tcolorbox}[colback=gray!10!white, colframe=gray!50!gray, halign=left, boxrule=0.5pt, left=1mm, right=1mm, top=1mm, bottom=1mm]
    \fontsize{8pt}{8pt}\selectfont
    SYSTEM PROMPT: You are a project management assistant specializing in construction scheduling analysis. Your task is to analyze text descriptions of project changes and extract structured information about task relation changes in a construction project.
    \vspace{8pt}
    
    CONTEXT\\
    The project involves the following tasks and their relationships: \\
    \vspace{3pt}
    Task ID \textbar{} Predecessor \textbar{} Duration \textbar{} Description \textbar{} Robot Type
    \begin{itemize}
    \item T1 \textbar{} - \textbar{} 0.25 \textbar{} Move Electrical Conduit \textbar{} R1
    \item T2 \textbar{} - \textbar{} 0.25 \textbar{} Move Window Frame \textbar{} R1
    \item T3 \textbar{} - \textbar{} 0.25 \textbar{} Move Window \textbar{} R1
    \item T4 \textbar{} - \textbar{} 0.25 \textbar{} Move Duct Structural Materials \textbar{} R1
    \item T5 \textbar{} - \textbar{} 0.25 \textbar{} Move Duct \textbar{} R1
    \item T6 \textbar{} - \textbar{} 0.5 \textbar{} Drill Wall \textbar{} R4 or R2
    \item T7 \textbar{} T1, T6 \textbar{} 1 \textbar{} Install Electrical Conduit \textbar{} R5 or R2
    \item T8 \textbar{} T2 \textbar{} 1 \textbar{} Install Window Frame \textbar{} R4 or R2
    \item T9 \textbar{} T3, T8 \textbar{} 0.5 \textbar{} Install Window \textbar{} R3
    \item T10 \textbar{} T4 \textbar{} 2 \textbar{} Duct Structural Framing \textbar{} R4 or R2
    \item T11 \textbar{} T5, T10 \textbar{} 2 \textbar{} Install HVAC Duct \textbar{} R4 or R2
    \item T12 \textbar{} T7 \textbar{} 2 \textbar{} Install Wiring \textbar{} R5 or R2
    \item T13 \textbar{} T12 \textbar{} 1 \textbar{} Wall Painting \textbar{} R6
    \item T14 \textbar{} - \textbar{} 0.5 \textbar{} Construction Site Inspection \textbar{} R7
    \end{itemize}
    \vspace{8pt}
    
    The robot capabilities are listed below: \\
    \vspace{3pt}
    Robot ID \textbar{} Capabilities
    \begin{itemize}
    \item R1: Cargo container
    \item R2: High-payload, Precise parallel gripper, Normal parallel gripper
    \item R3: High-payload, Suction-based gripper
    \item R4: High-payload, Normal parallel gripper
    \item R5: Precise parallel gripper
    \item R6: Sprayer
    \item R7: Camera, IAQ sensors
    \end{itemize}
    \vspace{8pt}
    
    CONSTRAINT TYPES:
    \begin{enumerate}
    \item Task Dependency Adjustments
      \begin{itemize}
        \item Format: [task\_id, successor, +/-]
        \item task\_id: the target task
        \item successor: the successors of the target task
        \item +/-: ``+'' indicates a newly added successor, ``-'' means the dependency has been removed
      \end{itemize}
    \item Task Duration Variations
      \begin{itemize}
        \item Format: [task\_id, new\_duration]
        \item task\_id: the target task
        \item new\_duration: the new duration of the target task in hours
      \end{itemize}
    \item Task Starting Time Changes
      \begin{itemize}
        \item Format: [task\_id, start\_time\_change]
        \item task\_id: the target task
        \item start\_time\_change: the changes in start time of the target task (e.g., +2 means delayed by 2 hours; -2 means ahead by 2 hours)
      \end{itemize}
    \item Number of Robot Variations
      \begin{itemize}
        \item Format: [robot\_type\_id, robot\_number\_change]
        \item robot\_type\_id: the type of robot (e.g., R1, R2, R3, etc.)
        \item robot\_number\_change: the number changes of the robot (e.g., +1 means one more robot; -1 means one less robot)
      \end{itemize}
    \item Task Conflict Constraints
      \begin{itemize}
          \item Format: [task\_id1, task\_id2]
          \item task\_id1: the first task in the conflict
          \item task\_id2: the second task in the conflict
      \end{itemize}
    \end{enumerate}
    \end{tcolorbox}
    \caption{Prompt design for construction project scheduling - Part 1.}
    \label{appendix:prompt1}
\end{figure}


\begin{figure}[t]
    \centering
    \begin{tcolorbox}[colback=gray!10!white, colframe=gray!50!gray, halign=left, boxrule=0.5pt, left=1mm, right=1mm, top=1mm, bottom=1mm]
    \fontsize{8pt}{8pt}\selectfont
    STEP-BY-STEP INSTRUCTIONS:
    \begin{enumerate}
    \item Read through the entire description to understand the context.
    \item For each change mentioned in the description:
       \begin{itemize}
       \item[a.] Identify which task (T1-T14) or robot type (R1-R7) is being affected based on CONTEXT. 
        \begin{itemize}
          \item Be careful to distinguish between similar tasks, for example:
          \begin{itemize}
          \item T2 (Move Window Frame) vs. T3 (Move Window) vs. T8 (Install Window Frame) vs. T9 (Install Window) - These are different tasks.
          \item If text mentions ``window installation'', specifically, it refers to T9 (Install Window), not T3 or T8
          \item If text mentions "window frame installation," it refers to T8 (Install Window Frame), not T2
          \end{itemize}
        \item Be careful to distinguish between similar robots, for example:
          \begin{itemize}
          \item R2, R3, R4, and R5 are different robots. 
          \item Only R2 combines both high-payload and precise parallel gripper capabilities.
          \item If text only mentions ``high-payload and normal parallel gripper'', it refers to R4 not R2.
          \end{itemize}
       \end{itemize}
       \item[b.] Determine which constraint type (1-5) applies to the change based on CONSTRAINT TYPES.
       \item[c.] Extract the specific parameters needed for that constraint type.
       \item[d.] Format the parameters according to the required format for that constraint type.
       \end{itemize}
    \item Compile all identified changes into the JSON output format:
       \begin{itemize}
       \item[a.] Create a JSON object with a ``changes'' array.
       \item[b.] For each change, add an object with ``constraint\_type'' and ``parameters'' fields.
       \item[c.] Ensure numerical values (like durations and time changes) are formatted as numbers, not strings.
       \item[d.] Ensure task IDs, successors, and robot types are formatted as strings.
       \item[e.] For time-related values:
          \begin{itemize}
          \item Simplify all numerical values to their simplest form (e.g., 1.5 not 1.50, 2 not 2.0)
          \item Convert minutes to hours (e.g., 30 minutes = 0.5 hours, 45 minutes = 0.75 hours)
          \item Please be aware that if you identify the constraint as 3, the time change should be associated with ``+'' or ``-''. 
          \end{itemize}
       \item[g.] Please be aware that if you identify the constraint as 4, the robot change should be associated with ``+'' or ``-''.
       \end{itemize}
    \item Double-check your result to ensure all changes mentioned in the description have been captured.
       \begin{itemize}
       \item[a.] Please ensure that your output follows the required format; e.g., for constraint 1, the output should be [task\_id, successor, +/-] (do NOT nest successors in additional brackets) and the task\_id should be the predecessor of the successor. 
       \item[b.] Please ensure that if you identify the constraint as 1, you correctly identify the target task and the successor of the target task and put them in the right order [task\_id, successor, +/-].
       \item[c.] Please ensure that if you identify the constraint as 3, the time change should be associated with ``+'' or ``-''. 
       \item[d.] Please ensure that if you identify the constraint as 4, the robot change should be associated with ``+'' or ``-''. 
       \item[e.] Please ensure that the task description corresponds to the task\_id in the CONTEXT.
       \end{itemize}
    \end{enumerate}
    \end{tcolorbox}
    \caption{Prompt design for construction project scheduling - Part 2.}
    \label{appendix:prompt2}
\end{figure}


\begin{figure}[t]
    \vspace{-2cm}
    \centering
    \begin{tcolorbox}[colback=gray!10!white, colframe=gray!50!gray, halign=left, boxrule=0.5pt, left=1mm, right=1mm, top=1mm, bottom=1mm]
    \fontsize{8pt}{8pt}\selectfont
EXAMPLES:\\
    Example 1:\\
    Input: ``Due to how things are unfolding on-site, it's understood that the drilling machine is not functioning, so the wall will be drilled manually. The task is expected to take two hours, and in light of recent discussions, after coordinating with field staff, it seems that the original worker assigned to install the HVAC duct is no longer available; however, we have secured another worker who can arrive in 150 minutes.''\\
    Output:
    \begin{verbatim}
{"changes": [
{"constraint_type": 2, "parameters": [T6, 2]},
{"constraint_type": 3, "parameters": [T11, +2.5]}
]}
    \end{verbatim}

    Example 2:\\
    Input: ``Recent developments suggest that wall painting takes 1.5 hours instead of 1 hour due to the need for multiple coats, and in light of recent adjustments, a revised understanding across teams indicates that a specialist required for electrical conduit installation calls in sick, preventing work from starting for 2 hours., followed by further refinements as recent developments suggest that wall painting takes 1.5 hours instead of 1 hour due to the need for multiple coats.''\\
    Output:
    \begin{verbatim}
{"changes": [
{"constraint_type": 2, "parameters": [T13, 1.5]},
{"constraint_type": 3, "parameters": [T7, +2]},
{"constraint_type": 2, "parameters": [T13, 1.5]}
]}
    \end{verbatim}

    Example 3:\\
    Input: ``Task dependencies have shifted, and one of the robots capable of handling heavy loads and performing fine, precise tasks is currently out of service due to a mechanical failure. Additionally, in light of recent discussions and the evolving situation on-site, it appears that two robots with high-capacity arms and fine-movement grippers were not charged, and have now run out of power.''\\
    Output:
    \begin{verbatim}
{"changes": [
{"constraint_type": 4, "parameters": [R2, -1]},
{"constraint_type": 4, "parameters": [R2, -2]}
]}
    \end{verbatim}
    
    Now, analyze the following description and extract all task relation changes in the specified JSON format: \{description\}
    
    Please output your response in JSON format and do not output other things. 
    \begin{verbatim}
{"changes": [
{"constraint_type": <number>, 
 "parameters": [<value1>, <value2>, ...]},...
]}
    \end{verbatim}
    \end{tcolorbox}
    \caption{Prompt design for construction project scheduling - Part 3.}
    \label{appendix:prompt3}
\end{figure}

\end{document}

%%% Local Variables:
%%% mode: latex
%%% TeX-master: t
%%% End: 