\begin{abstract}
  An important property of concurrent objects is whether they support
  progress -- a special case of liveness -- guarantees, which ensure the
  termination of individual method calls under system fairness assumptions.
  %
  Liveness properties have been proposed for concurrent objects. Typical
  liveness properties include \emph{lock-freedom}, \emph{wait-freedom},
  \emph{deadlock-freedom}, \emph{starvation-freedom} and
  \emph{obstruction-freedom}.
  %
  It is known that the five liveness properties above are decidable on the
  Sequential Consistency (SC) memory model for a bounded number of processes.
  However, the problem of decidability of liveness for finite state concurrent
  programs running on relaxed memory models remains open.
  %
  In this paper we address this problem for the Total Store Order (TSO) memory
  model, as found in the x86 architecture.
  %
  We prove that lock-freedom, wait-freedom, deadlock-freedom and
  starvation-freedom are undecidable on TSO for a bounded number of processes,
  while obstruction-freedom is decidable.
  % The undecidability result is obtained by reducing a known undecidable
  % problem, post correspondence problem (PCP), into checking if a lossy channel
  % machine has a specific infinite executions, and further reduce the latter
  % problem into lock-freedom (resp., other progress properties) of a specific
  % concurrent data structure.
  % We prove that obstruction-freedom is decidable on TSO memory model for bounded
  % number of processes.
  % The decidability result is obtained by reducing obstruction-freedom into
  % checking whether there exists some finite execution that goes through a
  % specific configuration.

%{\color{red}We investigate the verification of \emph{$k$-bounded wait-freedom}, a bounded version of wait-freedom, and prove that, given a fixed $k$, the problem of checking $k$-bounded wait-freedom is decidable on TSO for bounded number of processes.}

  \forget{
  We investigate the verification of \emph{bounded wait-freedom},
  % and \emph{population-oblivious wait-freedom} are
  a bounded version of wait-freedom,
  % Bounded versions of wait-freedom, such as bounded wait-freedom and
  % population-oblivious wait-freedom, are proposed to {\color{red} TODO: what
  % is the purpose of bounded versions of liveness properties.} are also typical
  % progress properties.
  and prove that, given a fixed bound $k$, the problem of checking bounded
  wait-freedom is decidable on TSO for bounded number of processes.
  Since wait-freedom is undecidable on TSO, but for each bound $k$
    bounded wait-freedom is decidable, a conjecture is that there exists
    a wait-free concurrent library on TSO for which there is no bound on the
    number steps of one or more of its methods.
    %
    We demonstrate this conjecture by means of an example library.
    By contrast, we prove that such property does not hold on SC.
    }

  %Finally, we prove that given a wait-freedom library running on SC with a
  %bounded number of processes, in each of its execution, each method returns
  %within bounded number of commands. By contrast, we prove that such property
  %does not hold on TSO by means of a counterexample data structure.

%We consider three related decidability/existence problems of bounded versions of wait-freedom.
%First, we prove that, on TSO memory model, there exists a wait-freedom library that has no bound.
%Second, we prove that, there does not exists a computable function that takes a wait-freedom library, and returns the bound of the library (or $\infty$ if it has no bound).
%Third, we prove that, given a fixed number $k$, the problem of checking whether a library is $k$-bounded wait-freedom is decidable on TSO memory model.
%As a contrast, we prove that the answers is ``not exists'', ``there exists'' and decidable for above three problems on SC memory model, respectively.


%We prove that, given a fixed number $k$, the problem of checking whether a library is $k$-bounded wait-freedom is decidable on TSO memory model.
%One question related to bounded versions of wait-freedom is, if the processes number is fixed, does a wait-free library must have a bound?
%We find that the answer is ``no'' on TSO memory model, with one example which is the library used in our undecidability proof of wait-freedom.
%We also show that the answer is ``yes'' on SC memory model.


%On one hand, we prove given a fixed number $k$, the problem of checking whether a library is $k$-bounded wait-freedom is decidable on TSO memory model. One the other hand, we prove that the attempt to calculate bound for wait-freedom is undecidable even on sequential consistent (SC) memory model. Or we can say, there is no computable function that takes a library $\mathcal{L}$ and its running processes number $n$ as argument, and returns a number $k$ which means that each execution of $\mathcal{L}$ on $n$ processes on SC (and TSO) memory model satisfies $k$-bounded wait-freedom.

\forget{
Various progress properties have been proposed for concurrent data structures.
Typical progress properties include lock-freedom, wait-freedom, and obstruction-freedom.
We address the problem of liveness verification for finite-state concurrent program running on TSO memory model, which is the memory model of intel X86 architecture.

We prove that lock-freedom and wait-freedom is undecidable on TSO memory model for bounded number of processes. The undecidability result is obtained by reducing a known undecidable problem, post correspondence problem (PCP), into checking if a lossy channel machine has a specific infinite executions, and further reduce this problem into lock-freedom (resp., wait-freedom) of a specific concurrent data structure. The reduction is based on simulating lossy channel machine with two collaborative processes.

We also prove that obstruction-freedom is decidable on TSO memory model. The decidability result is obtained by reducing obstruction-freedom into whether a finite path contains a configuration of specific control state and memory valuation.}
\end{abstract}

\forget{
\noindent Keywords: weak memory model, $\textit{linearizability}$,
$\textit{TSO-to-TSO linearizability}$
}

%%% Local Variables:
%%% mode: latex
%%% TeX-master: "CAV2021.tex"
%%% End: 