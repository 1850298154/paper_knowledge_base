%%%%%%%%%%%%%%%%%%%%%%%%%%%%%%%%%%%%%%%%%%%%%%%%%%%%%%%%%%%%%%%%%%%%%%%%%%%%%%%%
%2345678901234567890123456789012345678901234567890123456789012345678901234567890
%        1         2         3         4         5         6         7         8
\documentclass[letterpaper, 10 pt, journal, twoside]{IEEEtran} 
%\documentclass[letterpaper, 10 pt, conference]{ieeeconf}  % Comment this line out if you need a4paper

%\documentclass[a4paper, 10pt, conference]{ieeeconf}      % Use this line for a4 paper

\IEEEoverridecommandlockouts                              % This command is only needed if 
                                                          % you want to use the \thanks command

% \overrideIEEEmargins                                      % Needed to meet printer requirements.

%In case you encounter the following error:
%Error 1010 The PDF file may be corrupt (unable to open PDF file) OR
%Error 1000 An error occurred while parsing a contents stream. Unable to analyze the PDF file.
%This is a known problem with pdfLaTeX conversion filter. The file cannot be opened with acrobat reader
%Please use one of the alternatives below to circumvent this error by uncommenting one or the other
%\pdfobjcompresslevel=0
%\pdfminorversion=4

% See the \addtolength command later in the file to balance the column lengths
% on the last page of the document

% The following packages can be found on http:\\www.ctan.org
\usepackage{graphics} % for pdf, bitmapped graphics files
\usepackage{graphicx}
%\usepackage{epsfig} % for postscript graphics files
%\usepackage{mathptmx} % assumes new font selection scheme installed
%\usepackage{times} % assumes new font selection scheme installed
\usepackage{amsmath} % assumes amsmath package installed
\DeclareMathOperator{\sech}{sech}
\DeclareMathOperator{\csch}{csch}
\DeclareMathOperator{\arcsec}{arcsec}
\DeclareMathOperator{\arccot}{arcCot}
\DeclareMathOperator{\arccsc}{arcCsc}
\DeclareMathOperator{\arccosh}{arcCosh}
\DeclareMathOperator{\arcsinh}{arcsinh}
\DeclareMathOperator{\arctanh}{arctanh}
\DeclareMathOperator{\arcsech}{arcsech}
\DeclareMathOperator{\arccsch}{arcCsch}
\DeclareMathOperator{\arccoth}{arcCoth} 
\usepackage{amssymb}  % assumes amsmath package installed
%% your usepackages here, for example:
\usepackage{booktabs}
\usepackage{algorithm}
\usepackage{algorithmic}
\usepackage{url}
\usepackage{color,soul}
\usepackage{tabularx}

\usepackage[font=footnotesize,labelfont=bf]{caption}
\setlength{\belowcaptionskip}{-15pt}


\newcommand{\argmax}[1]{\underset{#1}{\operatorname{\mathbf{arg}}\,\operatorname{\mathbf{max}}}\;}
\newcommand{\argmin}[1]{\underset{#1}{\operatorname{\mathbf{arg}}\,\operatorname{\mathbf{min}}}\;}
\newcommand{\x}{\mathbf{x}}
\newcommand{\xproj}{\mathbf{\check{x}}}
\newcommand{\vo}{\mathbf{v}}
\newcommand{\ao}{\mathbf{u}}
\newcommand{\X}{\mathbf{X}}
\newcommand{\p}{\mathbf{P}}
\newcommand{\z}{\mathbf{z}}
\newcommand{\e}{\mathbf{e}}
\newcommand{\la}{\mathbf{l}}
\newcommand{\ob}{\mathbf{o}}
\newcommand{\Ob}{\mathbf{O}}
\newcommand{\w}{\mathbf{w}}
\newcommand{\Wt}{\mathbf{W}}
\newcommand{\Om}{\mathbf{\Omega}}
\newcommand{\Sig}{\mathbf{\Sigma}}
% \newcommand{\vo}{\mathbf{v}}
\newcommand{\ur}{\mathbf{u}}
\newcommand{\zl}{\mathbf{z}}
\newcommand{\Xp}{\mathcal{X}}
\newcommand{\Px}{\mathbf{P}}
\newcommand{\Ze}{\mathbf{0}}
\newcommand{\rob}{\mathcal{L}}
\newcommand{\varthet}{\sigma_{\theta}^2}
\newcommand{\varph}{\sigma_{\phi}^2}
\newcommand{\varr}{\sigma_{r}^2}

\newcommand{\deriv}[1]{\frac{\partial}{\partial #1}}
\newcommand{\footnoteref}[1]{\textsuperscript{\ref{#1}}}
% \DeclareMathOperator{\Tr}{Tr}
\newcommand{\todoEric}[1]{\textcolor{red}{\bf{#1}}}
\newcommand{\todogeneral}[1]{\textcolor{red}{#1}}
\newcommand{\todoAamir}[1]{\textcolor{blue}{\bf{#1}}}
\newcommand{\todoRahul}[1]{\textcolor{green}{\bf{#1}}}
\newcommand{\R}{\mathbf{R}}

\begin{document}

\title{\LARGE \bf
AirCapRL: Autonomous Aerial Human Motion Capture \\ using Deep Reinforcement Learning
}

% \author{Rahul Tallamraju, Nitin Saini, Elia Bonetto, Michael Pabst, Yu Tang Liu, Michael J. Black and Aamir Ahmad
% \thanks{Authors are with the MPI for Intelligent Systems, T\"ubingen, Germany. \tt\small \{firstname.lastname\} @ tuebingen.mpg.de}%
% \thanks{The authors would like to thank Prof.\ Dr.\ Heinrich B\"ulthoff for his constant support and for providing us the access to the Vicon tracking hall in MPI for Biological Cybernetics. The authors also thank Igor Martinovi\'c and the anonymous reviewers for extremely helpful suggestions.}
% }



\author{Rahul Tallamraju$^{1}$, Nitin Saini$^1$, Elia Bonetto$^1$, Michael Pabst$^1$, Yu Tang Liu$^1$, \\ Michael J.\ Black$^1$ and Aamir Ahmad$^{1,2}$%
\thanks{Manuscript received: February, 24, 2020; Revised June, 1, 2020; Accepted July, 12, 2020.}%Use only for final RAL version
\thanks{This paper was recommended for publication by Editor Tamim Asfour upon evaluation of the Associate Editor and Reviewers' comments.} %Use only for final RAL version
\thanks{$^{1}$Max Planck Institute for Intelligent Systems, T\"ubingen, Germany.
        {\tt\footnotesize {firstname.lastname}@tuebingen.mpg.de}}%
\thanks{$^{2}$Department of Aerospace Engineering and Geodesy, University of Stuttgart, Germany.}
\thanks{Digital Object Identifier (DOI): see top of this page.}
}

\markboth{IEEE Robotics and Automation Letters. Preprint Version. Accepted July, 2020}
{Tallamraju \MakeLowercase{\textit{et al.}}: AirCapRL: Autonomous Aerial Human Motion Capture using Deep Reinforcement Learning} 



\maketitle
% \thispagestyle{empty}
% \pagestyle{empty}


%%%%%%%%%%%%%%%%%%%%%%%%%%%%%%%%%%%%%%%%%%%%%%%%%%%%%%%%%%%%%%%%%%%%%%%%%%%%%%%%

\begin{abstract}

Constraint Programming (CP) and Machine Learning (ML) face challenges in text generation due to CP's struggle with implementing ``meaning'' and ML's difficulty with structural constraints. This paper proposes a solution by combining both approaches and embedding a Large Language Model (LLM) in CP. The LLM handles word generation and meaning, while CP manages structural constraints. This approach builds on 
%
GenCP, an improved version of On-the-fly Constraint Programming Search (OTFS) using LLM-generated domains.
Compared to Beam Search (BS), a standard NLP method, this combined approach
%
(GenCP with LLM)
is faster and produces better results, ensuring all constraints are satisfied. This fusion of CP and ML presents new possibilities for enhancing text generation under constraints.


\end{abstract}
\begin{IEEEkeywords}
Reinforecment Learning; Aerial Systems: Perception and Autonomy; Multi-Robot Systems; Visual Tracking.
\end{IEEEkeywords}

\IEEEpeerreviewmaketitle

%%%%%%%%%%%%%%%%%%%%%%%%%%%%%%%%%%%%%%%%%%%%%%%%%%%%%%%%%%%%%%%%%%%%%%%%%%%%%%%%
%%%%%%%%%%%%%%%%%%%%%%%%%%%%%%%%%%%%%%%%%%%%%%%%%%%%%%%%%%%%%%%%%%%%%%%%%%%%%%%%%%%%%%%%%%%%%%%%%%%%%%%%%
%% start of main body of paper

\section{Introduction}

One of the most fundamental problems in combinatorial optimization is the traveling salesperson problem (TSP), formalized as early as 1832 (c.f. \cite[Ch 1]{ABCC07}).
In an instance of  TSP we are given a set of $n$ cities $V$ along with their pairwise symmetric distances, $c:V\times V \to\R_{\geq 0}$. The goal is to find a Hamiltonian cycle of minimum cost. In the metric TSP problem, which we study here, the distances satisfy the triangle inequality. Therefore, the problem is equivalent to finding a closed Eulerian connected walk of minimum cost.%\footnote{Given such an Eulerian cycle, we can use the triangle inequality to shortcut vertices visited more than once to get a Hamiltonian cycle.}

It is NP-hard to approximate TSP within a factor of $\frac{123}{122}$ \cite{KLS15}.  An algorithm of Christofides-Serdyukov~\cite{Chr76,Ser78} from four decades ago gives a $\frac32$-approximation for TSP.
Over the years there have been numerous attempts to improve the Christofides-Serdyukov algorithm and exciting progress has been made for various special cases of metric TSP, e.g., \cite{OSS11,MS11,Muc12,SV12,HNR21, KKO20, HN19, GLLM21}.
 Recently, ~\cite{KKO21} gave the first improvement for the general case by demonstrating that the so-called ``max entropy" algorithm of \cite{OSS11} gives a randomized $\frac{3}{2}-\epsilon$ approximation for some $\epsilon > 10^{-36}$.% (see \cite{VS20} for a historical note about TSP)

%After a long line of work %~\cite{Wol80,SW90,BP91,Goe95,CV00,GLS05,BM10,BC11,SWV12, HNR17,HN19, KKO20a} 
	%the best known approximation algorithm for the general case of the problem is $\frac{3}{2}-\epsilon$ for some $\epsilon > 10^{-36}$ due to ~\cite{KKO21}, a result that built upon the work of the third author, Saberi, and Singh ~\cite{OSS11}. 
	The method introduced in \cite{KKO21} exploits the optimum solution to the following linear programming relaxation of metric TSP studied by \cite{DFJ59,HK70,BG93}, also known as the subtour elimination LP:
\begin{equation}\label{eq:tsplp}
\begin{aligned}
	\min \quad& \sum_{u,v} x_{\{u,v\}} c(u,v)& \\
	\text{s.t.,} \quad &  \sum_{u} x_{\{u,v\}} = 2&\forall v\in V,\\
	& \sum_{u\in S, v\notin S} x_{\{u,v\}}\geq 2,&\forall S \subsetneq V, S\not= \emptyset\\
	& x_{\{u,v\}}\geq 0 &\forall u,v\in V.
\end{aligned}	
\end{equation} 
	
	 However, ~\cite{KKO21} did not show that the integrality gap of the subtour elimination polytope is bounded below $\frac{3}{2}$, and therefore did not make progress towards the ``4/3 conjecture" which posits that the integrality gap of LP \eqref{eq:tsplp} is $\frac{4}{3}$. In this work we remedy this discrepancy by proving the following theorem, improving upon the bound of $\frac{3}{2}$ from Wolsey~\cite{Wol80} in 1980:

\begin{theorem}\label{thm:main}
	Let $x$ be a solution to LP \eqref{eq:tsplp} for a TSP instance. For some absolute constant $\epsilon > 10^{-36}$, the \hyperlink{tar:alg}{max entropy algorithm} outputs a TSP tour with expected cost at most $\frac{3}{2}-\epsilon$ times the cost of $x$. Therefore the integrality gap of the subtour elimination LP is at most $\frac{3}{2} - \epsilon$. 
\end{theorem} 

To prove \cref{thm:main}, we amend Section 4 of \cite{KKO21} but keep the remainder of the analysis essentially the same. Unlike \cite{KKO21}, this argument now preserves the integrality gap by avoiding the use of the optimum solution in bounding the cost of the matching. See \cref{sec:overview} for a discussion of our new approach.
%We note that the analysis in this paper is not specialized to the max entropy algorithm (although we rely on many results from \cite{KKO21} to obtain \cref{thm:main} itself); instead, it is valid for any algorithm which samples a spanning tree from the support of a solution to LP \eqref{eq:tsplp} and then adds the minimum cost matching on the odd degree vertices of the tree.  
%Instead, we use the polygon representation of near minimum cuts \cite{Ben95,BG08} to bound  the cost of the matching (see the following section for an overview of our new findings). %An added benefit of avoiding the use of OPT in the analysis is  %We remark this makes the analysis constructive 
%We remark that this allows future analyses to explicitly compute and possibly utilize the relevant laminar family of near minimum cuts (whereas previously one needed to know OPT to find the laminar family used in the analysis in \cite{KKO21}).
%In particular, we show that to get a bound better than $\frac{3}{2}$ for this class of algorithm it is (essentially) sufficient to handle the case in which the near minimum cuts of $x$ are a laminar family.

\subsection{Other Consequences}
\paragraph{Path TSP} In recent exciting work, Traub, Vygen, Zenklusen \cite{TVZ20} showed that an $\alpha$-approximation algorithm for metric TSP can be used as a black box to get a $\alpha(1+\eps)$ approximation algorithm for Path TSP. This together with \cite{KKO21} implies that there is a $3/2-\eps$ approximation algorithm for Path TSP (for $\eps>10^{-36}$). On the other hand, it is a folklore result that the integrality gap of the natural LP relaxation of Path TSP is at least $3/2$.  Therefore, a consequence of the above theorem is that although the best possible approximation factors of the two problem are the same (up to polynomial reductions), the natural LP relaxation of metric TSP has a strictly smaller integrality gap.


\paragraph{2-ECSM} In the 2-edge-connected multi-subgraph problem, or 2-ECSM for short, we are given a weighted graph $G$ and we want to find a minimum cost 2-edge-connected spanning subgraph, where an edge can be chosen multiple times.
The classical Christofides-Serdyukov algorithm gives a 3/2-approximation for 2-ECSM and despite significant attempts \cite{CR98,BFS16,SV14,BCCGISW20} improved algorithms were designed only for special cases of the problem.
Since in \cite{BG93} it is shown that LP \eqref{eq:tsplp} is a valid relaxation for 2-ECSM, we obtain:

\begin{corollary}	
For some absolute constant $\epsilon > 10^{-36}$ the \hyperlink{tar:alg}{max entropy algorithm} is a randomized $\frac{3}{2}-\epsilon$ approximation for the 2-edge-connected multi-subgraph problem.
\end{corollary}
%Beyond these theorems, we believe the analysis in this paper will open new avenues to improve the arguments in ~\cite{KKO21}. The analysis in that work is by nature non-constructive because it uses information about the optimal solution. Here we remove this weakness and could in principle construct the proposed fractional matching in polynomial time. Although of course this has no practical benefit since the algorithm always finds the minimum cost matching, this may allow future works to manipulate the algorithm to better serve the analysis.

%We analyze the max-entropy rounding algorithm introduced in \cite{OSS11} and slightly modified in \cite{KKO20, KKO21}. 

%In other words, we design a feasible vector for the $O$-join polytope to ``satisfy'' all near min cuts ``crossed on both  sides'' 


%Whereas Section 4 of ~\cite{KKO21} only deals with the near minimum cuts of $x$ (where $x$ is a solution to LP \eqref{eq:tsplp}) which lie along the optimal Hamiltonian cycle, we deal with all near minimum cuts of $x$ using the so-called polygon representation of near minimum cuts ~\cite{Ben97,BG08}. %The results give new intuition for the structure of cuts that are within $\frac{6}{5}$ or less of the edge connectivity of the graph.

 %: we show that we can incur a cost of $O(\eta^2) \cdot c(x)$ to ensure that the set of cuts with $x(\delta(S)) \le 2+\eta$ is a laminar family.


\subsection{New techniques and contributions}\label{sub:newtechniques}

This paper can be seen as a case study on how to reason about and deal with {\em near} minimum cuts. One can deduce from the classical cactus representation of a graph $G$ \cite{DKL76} (i) the structure of {\em all} min cuts of $G$ and (ii) the structure of the edges of $G$ in the sense that every edge $\{u,v\}$ maps to a unique {\em path} in the cactus between the images of $u$ and $v$. Furthermore, such a path intersects every cycle of the cactus on at most one cactus edge. The theory has found many application from designing fast algorithms
\cite{Kar00,KP09} to the analysis of approximation algorithms for TSP \cite{KKO20} and connectivity augmentation \cite{BGJ20,CTZ21}.

Two decades later, the theory of min cuts was extended to near min cuts in works of Bencz\'ur and Goemans \cite{Ben95, BG08} where they introduced the polygon representation which represents all cuts of a graph with at most $\frac{6}{5}k$ edges, where $k$ is its edge connectivity. Although these works completely classify the structure of all near min cuts of a given graph $G$, they do not characterize the structure of the \textit{edges} of $G$ with respect to these cuts, which can be important in applications (for example, in many of the recent applications of min cuts,
 one also needs to exploit the structure of the edges in relation to the cactus).
The structure on the edges turns out to be highly relevant in this work as well, and as a byproduct of our analysis we make progress towards classifying the way in which the edges of $G$ relate to the structure of the polygon representation.
 
 % and (to some extent) a classification of the set of edges of $G$ with respect to the polygon representation of Bencz\'ur and Goemans.
 
  %i
 %s to give a better understanding of the structure of edges of $G$ with respect to its near min cuts.

  %One can partition the edges of $G$ into sets $F_1\dots,F_m$ such that the set of edges in every min cut $(S,\overline{S})$ of $G$ is the union of edges in a pair $F_i,F_j$ for $i\ neq j$.
%\Nathan{Shayan can add something} For example...

For motivation, consider a generic family of network design problems in which we want to construct a network such that every pair $u,v$ of vertices has connectivity at least $c_{u,v}$. A natural approach is to write an LP relaxation to find a (minimum cost) vector $x: E \to \R_{\ge 0}$ such that for every cut $S$ separating $u$ and $v$, $x(\delta(S))\geq c_{u,v}$. We can round this LP using independent rounding or a dependent rounding scheme such as sampling from max entropy distributions. Using classical concentration bounds one can show that if $x(\delta(S))\gg c_{u,v}$ then with high probability the rounded solution has at least $c_{u,v}$ edges across this cut. So the main challenge is to ``fix'' near tight cuts, i.e., cuts where $x(\delta(S))\approx c_{u,v}$.  For an explicit instantiation of this scheme see \cite{KKOZ22}. A better understanding of the global structure of the family of near tight cuts has the potential to significantly simplify or even improve the approximation factor of such rounding algorithms. A classical technique to design algorithms for such network design problems is to apply uncrossing to extreme point solutions of the LP. One can view our contribution as an approximate uncrossing technique that deals with all near tight cuts (instead of just tight cuts) as we explain next.
%Next, we explain how our main theorem can be used to give global structure for near tight cuts in the case that $c_{u,v}=2$ for all $u,v$ and we contrast it with the classical uncrossing technique which only deals with tight/min cuts. 


\paragraph{An Approximate Uncrossing Technique.} A fundamental technique in the field of approximation algorithms is the uncrossing technique\footnote{See e.g. \cite{LRS11} for a number of applications of this technique.} of Jain \cite{Jai01}. Given a graph $G=(V,E)$,  a weight vector $x:E\to\R_{\geq 0}$, and a  function $f:V\to\R$, suppose that $x(\delta(S))\geq f(S)$ for all $S\subseteq V$. Let $\cN$ be the family of sets $S$ such that $x(\delta(S)) = f(S)$, i.e., the family of {\em tight} sets with respect to $f$. The uncrossing technique says that if $f$ is (weakly) supermodular then we can refine $\cN$ to a laminar family of sets, $\cH$, such that if all sets of $\cH$ are tight, then all sets of $\cN$ are tight as well. For a concrete example, suppose $f$ is a constant function, say $f(S)=2$ for all $\emptyset\subsetneq S\subsetneq V$. Then, sets of $\cH$ can be constructed using the cactus representation \cite{DKL76} of cuts in $\cN$. The significance of this method is that if $x$ is a basic feasible solution to a LP with constraints $x(\delta(S))\geq f(S)$ for all $S$, one can use this machinery to argue that the support of $x$ has size $O(|V|)$.

Informally, we prove the following, which 
can be seen as  an {\em approximate uncrossing technique}: 
\begin{theorem}[Informal]\label{thm:uncrossing}Suppose we have a vector $x:E\to\R_{\geq 0}$ such that $x(\delta(S))\geq f(S)$ for all $S$; define $\cN$ to be sets $S$ where $x(\delta(S))\leq f(S)(1+\eps)$ for some fixed $\eps>0$. If $f(.)$ is constant, say $f(S)=2$ for all $S$, then there is a set $\cN^*\subseteq \cN$ and a collection of edge sets $F_1,\dots,F_m\subseteq E$ such that the following hold:
\begin{itemize}
	\item $|\cN^*|= O(|V|)$, $m= O(|V|)$.
	\item $x(F_i)\geq 1-\eps/2$ for all $1\leq i\leq m$.
	\item Every edge $e$ is in at most $O(1)$ of the $F_i$'s.
	\item For every set $S\in \cN\smallsetminus \cN^*$ there exists $1\leq i<j\leq m$ such that $F_i\cap F_j=\emptyset$ and $F_i\cup F_j\subseteq \delta(S)$ and for every $S\in \cN^*$, there exists $1\leq i\leq m$ such that $F_i\subseteq \delta(S)$. 
\end{itemize}
\end{theorem}
In words, although we cannot simply refine $\cN$ to a linear number of sets, we can refine the edges in cuts of $\cN$ to a linear number of sets $F_1,\dots, F_m$ such  that we can essentially capture the edges of $\delta(S)$ for any $S\in \cN\smallsetminus \cN^*$ by a pair of disjoint $F_i$'s. We give a slightly weaker condition for cuts in $\cN^*$; namely we only capture half of their edges by $F_i$'s.

\begin{example}For a simple example of the above theorem, suppose $\eps=0$, i.e. $\cN$ is the set of min cuts of a graph $G$. Furthermore, suppose that every proper  cut in $\cN$ is \hyperlink{tar:crossing}{crossed} (recall that $S$ is proper if $1<|S|<|V|-1$) and that $\cN$ has at least one proper cut. 
Then, one can use an uncrossing technique, namely that if $A,B\in \cN$ then $A\cap B\in \cN$, to prove that $G$ must be cycle, namely we can order vertices of $G$, $v_0,\dots,v_{n-1}$ such that $x_{\{v_i,v_{i+1\text{ mod n}}\}}=1$.
In such a case we let $\cN^*=\emptyset$ and $F_i=E(v_i,v_{i+1\text{ mod }n})$.
%partition $V$ into sets $a_0,\dots,a_{m-1}$ such that 
%Let $\C$ be a connected component of crossing cuts of $\cN$, namely, for any pair of sets $A,B\in \C$ there is a path of crossing cuts all from $\C$ that goes from $A$ to $B$.
% and further suppose that $\cN$ can be represented by a cycle $C$ in the sense every min cut of $\cN$ corresponds to a min cut of $C$ and vice versa. Here we assume $a_0,\dots,a_{m-1}$ are the nodes of $C$ where each $a_i$ is identified with a disjoint set of vertices where $V=\uplus_{i=1}^m a_i$. In such a case, we can simply let $\cN^*=\emptyset$ and $F_i=E(a_i,a_{i+1\text{ mod }m})$. 
\label{eg:cycle}\end{example}

\begin{example}\label{eg:laminar}
For a second example, suppose again $\eps=0$ and $\cN$ is the set of mincuts of a graph $G$ where $\cN$ forms a laminar family (no two cuts cross). It turns out that we cannot decompose edges of cuts of $\cN$ into a linear sized collection of sets where every edge appears only a constant number of times. The main reason is that some edges may appear in an unbounded number of cuts. In this case we let $\cN^*=\cN$ and for every $A\in \cN$ (with immediate parent $B\in \cN$ in the laminar family) we add a set $F_A=\delta(A)\smallsetminus \delta(B)$  to our collection.  It is straightforward to show, using the structure of min cuts, that $x(F_A)\geq 1$; furthermore, since the size of a laminar family is linear in $V$, this gives a valid decomposition in the sense of above theorem.
\end{example}
Lastly, if $\eps=0$ and $\cN$ is the set of min cuts of an arbitrary graph, one can represent all min cuts of $\cN$ by a cactus \cite{DKL76} which can be seen as a tree of cycles. In such a case, one can use a construction similar to \cref{eg:cycle} for each cycle where instead of a vertex $v_i$ we have a set $a_i \subseteq V$ and one similar to \cref{eg:laminar} for the tree part of the cactus. For a concrete application of such a decomposition of min cuts see \cite{KKO20}.
%More generally, if $\cN$ corresponds to the set of min cuts of an arbitrary graph, the cuts of $\cN$ can be represented by a {\em cactus graph}. In such a case we add one $F_i$ for every edge of a cycle of the cactus. 


%and further for simplicity assume that there is a single connected component of crossing cuts in $\cN$, namely we can go from any $A$ to $B$ for $A,B\in\cN$ simply following crossing cuts of $\cN$. Then, one can represent cuts in $\cN$ by the set of min cuts of a cycle, namely we can contract vertices of $G$ 

%For a concrete application , suppose we need at least two edges in every set in $\cN^*$, say in a network optimization problem. Then, if we make sure that we have at least one edge in each $F_i$, all typical cuts constraints, $\cN\smallsetminus \cN^*$,  are satisfied, so we  reduce the problem to cuts in $\cN^*$. 


One of the main challenges in dealing with near min cuts relative to min cuts is that if $x(\delta(A)),x(\delta(B))\leq 2+\eps$ then $x(\delta(A\cap B))\leq 2+2\eps$. Therefore, if $\eps=0$, then min cuts are closed under intersection, set difference and union, but this is no longer true when $\eps>0$. So, to employ the classical uncrossing machinery one should be very careful to "uncross" only a constant number of times (independent of $\eps$) to make sure that every cut remains within $2+O(\eps)$. This is the main reason that the polygon representation of near min cuts (see below) is more sophisticated, e.g., we can no longer argue $x(E(a_i, a_{i+1}))\approx 1$, see \cref{fig:nearmincutbadexample}.

Although we don't study it here, we believe it may be worthwhile to find generalizations of \cref{thm:uncrossing} which hold for any (weakly) supermodular function.% That could be helpful in many questions based on the network optimization framework of Jain \cite{Jai01}.

\begin{remark} 
 We do not explicitly prove \cref{thm:uncrossing} in this extended abstract, as it is not used to prove \cref{thm:main}. However it can be deduced from arguments in \cref{sec:twoside} and \cref{app:oneside}. 
%In \cref{sec:overview} we discuss the main ideas of the proof of \cref{thm:uncrossing}. Here, let us explain the main challenge: In principal one might try to simply extend the above decomposition for the case $\eps=0$. The main challenge is that if $x(\delta(A)),x(\delta(B))\leq 2+\eps$ then $x(\delta(A\cap B))\leq 2+2\eps$. Therefore, if $\eps=0$, then min cuts are closed under intersection, set difference and union, but this is no longer true when $\eps>0$. So, to employ the classical uncrossing machinery one should be very careful to "uncross" only a constant number of times (independent of $\eps$) to make sure that every cut remains within $2+O(\eps)$. This is the main reason that the polygon representation of near min cuts (see below) is more sophisticated, e.g., we can no longer argue $x(E(a_i, a_{i+1}))\approx 1$, see \cref{fig:nearmincutbadexample}.
\end{remark}





\paragraph{Extensions to the Polygon Representation} To obtain our uncrossing framework we prove new properties of the polygon representation.
Given a graph $G=(V,E)$, let $k$ be the edge-connectivity of $G$, i.e. the number of edges in a minimum cut of $G$. For $\eps>0$, consider the set of $(1+\eps)$-near minimum cuts of $G$: cuts $(S,\overline{S})$ where $|E(S,\overline{S})| < (1+\eps)k$.
Bencz\'ur \cite{Ben95} and Bencz\'ur, Goemans \cite{BG08} proved that if $\eps \le 1/5$ then the near minimum cuts of $G$ admit a {\em polygon representation}. Namely, every connected component $\cC$ of \hyperlink{tar:crossing}{crossing} $(1+\eps)$ near min cuts can be represented by the diagonals of a convex polygon. In this polygon, the vertices of $G$ are partitioned into sets called \textit{atoms}, and every atom is mapped to a cell of this polygon defined by the diagonals and the boundary of the polygon itself (see \cref{sec:polyrep} for more details). 

The polygon representation can be seen as a generalization of the well-known cactus representation \cite{DKL76} of minimum cuts where a cycle of the cactus is replaced by a convex polygon. Unlike a cycle, some vertices/atoms map to the interior of the polygon, which are called ``inside'' atoms. The inside atoms at first look like a mystery and one can ask many questions about them such as how many can exist and what structures they can exhibit.



 Here, we explain two lemmas we proved which might find further applications beyond TSP in the future. 
%Our results give new intuition and understanding about the structure of polygon representations. These guide our analysis of the integrality gap of the subtour LP.
 %For example, one of our new observations is a 
 First, we give a necessary condition for a cell of a polygon to contain an inside atom:
\begin{lemma}[Informal, see \cref{thm:halfplanes}]
	Consider a polygon $P$ for a connected component $\C$ of a family of $1+\eps$ near min cuts for $\eps \le 1/5$ (where representing diagonals correspond to cuts in $\C$). Any cell of $P$ that has an inside atom must have at least $\Omega(1/\eps)$ many sides. 
\end{lemma}
This can be seen as a generalization of \cite[Lem 22]{BG08} to the case in which the cell is allowed to be adjacent to vertices of the polygon $P$.

Now, we explain our second extension: it follows from the cactus representation of minimum cuts that for a graph $G$ and a min cut $S$ one can partition the set of all min cuts that cross $S$ into two groups ${\cal A}=\{A_1,\dots,A_k\}$ and ${\cal B}=\{B_1,\dots,B_l\}$ for some $k,l\geq 0$ such that $S\cap A_1\subseteq S\cap A_2 \subseteq \dots S\cap A_k$ and, similarly, $S\cap B_1\subseteq \dots\subseteq S\cap B_l$. We prove a generalization of this fact for near min cuts:
\begin{lemma}[Informal, see \cref{lem:crosschain}]
Consider the set of $1+\eps$ near min cuts of a graph $G$ for $\eps\leq 1/10$; for any such near min cut $S$, one can partition the $1+\eps$ near min cuts crossing $S$ into two groups ${\cal A}=\{A_1,\dots,A_k\}$ and ${\cal B}=\{B_1,\dots,B_l\}$ such that $S\cap A_1 \subseteq S\cap A_2\subseteq \dots \subseteq S\cap A_k$ and similarly for cuts in ${\cal B}$.
\end{lemma}

\subsection{Outline of rest of paper} After reviewing preliminaries in \cref{sec:prelims}, we give a high-level overview of our proof technique in \cref{sec:overview}. The main new technical contributions of this paper are in \cref{sec:polyrep} and  \cref{sec:twoside}. The remaining content of the paper essentially follows from ~\cite{KKO21}. %Therefore, the reader may want to skip \cref{sec:proof-of-main}. 



Distributed systems have been maintaining their importance for the last several decades due to the increase in the need for scalable and reliable distributed applications while preserving high performance. 
To analyze distributed systems comprehensively and compare them in terms of features and services, various surveys and evaluations have been published in the past. Surveys on cloud providers, data warehouses, distributed file systems, or metadata services can be counted among them. 

Cloud providers are analyzed and evaluated in terms of elasticity \cite{CMART}, computing power \cite{comperative-benchmarking}, and cost to performance efficiency \cite{fair-benchmarking} in previous efforts. Widely used distributed services are also analyzed in many works, such as a survey on stream processing \cite{stream-benchmarking} or performance and dependability evaluation of MapReduce systems \cite{MapReduce-benchmarking}. Similarly, different aspects of distributed systems are studied in several surveys, like reliability analysis on distributed systems \cite{reliability-survey} and load balancing characteristics of known systems \cite{load-balancing-survey}.

As a big part of distributed systems, data warehouses and file systems are studied for many specifications. Evaluation of distributed data warehouses for the cost-effectiveness of different hardware configurations \cite{ALOJA} and query performance of distinct design choices\cite{benchmarking-data-warehouse} are among the known efforts in these works. Distributed file systems are examined in many past works for general concepts \cite{file-systems-concepts,file-systems-gen1} or specific applications such as distributed access control \cite{access-control-file-systems}. Due to the differences in optimization, design techniques, and the complex interactions between the file systems and other system components like the kernel or operating system, benchmarking distributed file systems is not trivial. To identify the important metrics for the evaluation of distributed file systems, researchers also studied benchmarking file systems \cite{File-system-benchmarking,benchmarking-file-rocket}. 

Analysis of distributed coordination services in terms of general characteristics and importance of coordination \cite{importance-of-coordination} and the comparison of existing algorithms \cite{paxos-made-simple} are among the published works. However, to the best of our knowledge, there is no published work on the evaluation of distributed coordination systems. As mentioned in the Introduction, due to the lack of standard benchmarking tools for distributed coordination services, developers widely use their ad-hoc benchmarks, which are prone to unfair comparisons or limited results for the evaluation of the systems. This study is unique in identifying the metrics and parameters for the evaluation of distributed coordination systems, discussing how each system uses these metrics and parameters for its evaluation, pinpointing the deficiencies of well-known benchmarking suites in evaluating distributed computing systems, and finally discussing the features of an ideal distributed coordination benchmark. 
\section{Methodology}

\subsection{Problem Statement}
Let there be a team of K MAVs (with quadcopter-type dynamics) tracking a person P. The pose of the $k^{th}$ MAV in the  world frame at time $t$ is given by $\xi_t^{k} = [(\mathbf{x}_t^{k})^\top ~ (\Theta_t^{k})^\top] \in \mathbb{R}^6$, where $(\mathbf{x}_t^{k})^\top$ denotes the 3D position of the MAV's center in Cartesian coordinates and $(\Theta_t^{k})^\top$ denotes its orientation in Euler angles. Each MAV has an on-board, monocular, perspective camera. It is important to note that the camera is rigidly attached to the MAV's body frame, pitched down at an angle of $\theta_\textrm{cam}$. The global pose of the person is given by $\xi_t^P = [(\mathbf{x}_t^P)^\top ~ (\Theta_t^P)^\top ~(\mathbf{x}_{j,t}^P~\forall~j; j=1 \cdots 14)^\top ] \in \mathbb{R}^{48}$. $(\mathbf{x}_t^P)^\top$ and $(\Theta_t^P)^\top$ are the body's 3D center and global orientations, respectively. $\mathbf{x}_{j,t}^P$ denotes the 3-D position of a joint $j$ from a total of fourteen joints considered for the MoCap of the subject. Ground truth joints considered are visualized as circles in Fig.~\ref{fig:mesh}. The MAVs operate in an environment with neighboring MAVs as dynamic obstacles. Their task is to autonomously fly and record images of the person using their on-board camera. The formation control goal of the MAV team is to cooperatively navigate in a way such that the error in 3D pose estimates of the subject is minimized. 
%This error is obtained by running a state-of-the-art method like MultiViewHMR \cite{liang2019shape} on the images that the MAVs record. 


\subsection{Formulation as a Sequential Decision Making Problem}

Intuitively, the accuracy of aerial MoCap depends on the following two factors.
\begin{itemize}
 \item The subject should always remain completely in the FOV of every MAV's camera, occupying maximum possible area on the image plane.
 \item The subject is visually encapsulated from all possible directions (viewpoints).
\end{itemize}

Based on these intuitions and experimentally derived models for single and multiple camera-based observations, in our previous work \cite{ActiveTallamraju19} we approached this problem using a model predictive control (MPC) based formation controller. The MPC objective was to keep a threshold distance to the subject while satisfying constraints that enable uniform distribution of viewpoints around the subject. Additionally, a yaw controller ensured that the subject was always centered on the image plane. As discussed in the introduction, this method is hard to generalize because to i) it is agnostic to how the 3D pose and shape was estimated by the back end, and ii) it needs carefully derived observation models.

To address these issues in this work we take a deep reinforcement learning-based approach. We model this formation control problem as a sequential decision making problem for every MAV agent. Dropping the MAV superscript $k$, for each agent the problem is defined by the tuple $(S,O,A,T,R)$, where $S$ is the state-space, $O$ is the observation-space, $A$ is the action-space, $T$ is the environment transition model, and $R$ is the reward function. At each time instance $t$, an agent at state $s_t$ has access to an observation $o_t$ using its cameras and on-board sensors. The agent then chooses an action $a_t$, which is conditioned on $o_t$ using a stochastic policy $\pi_\theta(a_t|o_t)$, where $\theta$ represents parameters of a neural network. The agent experiences an instantaneous reward $r_t(s_t,a_t)$ from the environment indicating the goodness of the chosen action. We approach the problem without any underlying assumptions or knowledge about the environment transition model $T$. To this end, we leverage a model-free deep reinforcement learning method to train the agents. We will further describe the states, observations and actions in detail. Due to ease of notations and to keep the RL training computationally tractable, we will consider 2 MAV agents in this letter, i.e, $K=2$.  Rewards are described later when we discuss our proposed methodology in sub-section~\ref{subsec:proposed_method}.


\begin{figure}
	\centering
	\includegraphics[scale=0.60]{ethan_overlaid.pdf}
\caption{The ground truth body joints (left) and estimated pose and shape overlaid (right).}
\label{fig:mesh}
\end{figure}

% \begin{subequations}
%   \begin{tabularx}{\columnwidth}{Xp{5mm}X}
%   \begin{equation}
%     \label{eq-a}
%       a = b
%   \end{equation}
%   & &
%   \begin{equation}
%     \label{eq-b}
%     c = d
%   \end{equation}
%   \end{tabularx}
% \end{subequations}



\subsubsection{States and Observations}
Each agent's environment state, $s_t$, includes the MAV pose $\xi_t$, its neighboring MAV's pose $\bar{\xi}_t$ and the MoCap subject's pose $\xi_t^P$.
% \begin{equation}
%  s_t = [\xi_t ~~ \bar{\xi}_t ~~ \xi_t^P]
%  \label{eqn:state}
% \end{equation}
\begin{equation}
 s_t = [\xi_t ~~ \bar{\xi}_t ~~ \xi_t^P];  ~~ o_t = [\mathbf{y}_t^{P} ~~ \mathbf{\dot{y}}_t^{P} ~~ \psi_t^{P} ~~ \mathbf{y}_t^{N} ~~ \psi_t^{P,N}]
 \label{eqn:state_obs}
\end{equation}

The observation vector $o_t$ is given by (\ref{eqn:state_obs}). Its first two components are the measurements of the person $P$'s position and velocity made by the agent in its local Cartesian coordinates. This is given by $[\mathbf{y}_t^{P} ~~ \mathbf{\dot{y}}_t^{P}] \in\mathbb{R}^6$. The third component of the observation vector is the measurement of the relative yaw orientation of the person with respect to the robot's global yaw orientation, denoted by $\psi_t^{P}$.  Here we emphasize that we make no assumptions regarding the uncertainty model associated with these measurements. However, we assume that these measurements are available using a vision-based detector. In our synthetic training environment we directly use the available ground truth position and orientation of the person and the MAV to compute these measurement. In real robot scenarios we use Vicon readings to calculate it. The fourth component is the 3D position measurements to the neighboring MAV agent in the local Cartesian coordinates of the observing agent. This is given by $\mathbf{y}_t^{N} \in\mathbb{R}^3$. The fifth component is the measurement of the relative yaw angle orientation of the person with respect to the neighboring robot's global yaw orientation, denoted by $\psi_t^{P,N}$.


% \begin{equation}
%  o_t = [\mathbf{y}_t^{P} ~~ \mathbf{\dot{y}}_t^{P} ~~ \psi_t^{P} ~~ \mathbf{y}_t^{N} ~~ \psi_t^{P,N}]
%  \label{eqn:observations}
% \end{equation}


\subsubsection{Actions}
\label{subsec:actions}

Action $a_t$ is sampled from the control policy $\pi_\theta(a_t|o_t)$ for an input observation $o_t$. In our formulation, actions consist of egocentric 3-D linear translational velocity of the agent, given by $\mathbf{v}_t = [{vx}_t ~~ vy_t ~~ vz_t]$ and a rotational velocity $\omega_t$ about its z-axis. The chosen action defines a way-point $\{\mathbf{x}^w_t,~\phi^w_t\}$, which is obtained as $~\mathbf{x}^w_t = \mathbf{x}_t + \mathbf{R}(\phi_t)\mathbf{v}_t\Delta, ~\phi^w_t=\phi_t+\omega_t\Delta$, for the agent in the world frame. $\{\mathbf{x}^w_t,~\phi^w_t\}$ is provided to low-level geometric tracking controller (Lee controller) \cite{lee2010geometric} of the agent. $\mathbf{x}_t$, as defined before, denotes the current 3D position of the agent. $\mathbf{R}(\phi_t)\in SO(3)$ is a rotation matrix. Thus,
\begin{equation}
 a_t= [\mathbf{v}_t ~~ \omega_t] \in\mathbb{R}^4
 \label{eqn:actions}
\end{equation}

% In an end-to-end learning-based approach, where agents learn optimal actions, given only the image input and the self-pose using a self-localization system, the observations would be the on-board camera images and the self-pose obtained by a self-localization method.

\subsection{Proposed Methodology}
\label{subsec:proposed_method}

% (MoCap and collision avoidance with neighboring agents) 

Training multiple agents to achieve multiple objectives is a complex and computationally demanding task. In order to have a systematic comparison we first develop our approach for a single agent case and then for multi-agent scenario. Meaning, we train (and then evaluate and compare) two different kinds of agents, and hence, networks. These are i) a single agent with only MoCap objectives, and ii) multi-agents (2 in our case) with both MoCap and collision avoidance objectives.

% 
% \begin{itemize}
%  \item A single agent with only MoCap objectives. 
%  %The actions in this case consist of $a_t$ as described in sub-section~\ref{subsec:actions}
%  \item Multiple agents (2 in our case) with both MoCap and collision avoidance objectives.
%  %trained on a static subject. The actions in this case consist only of $[\mathbf{v}_t]$. 
% \end{itemize}

We hypothesize that, using the first kind of network, an agent will learn to follow the person and orient itself in the direction of the person in order to achieve accurate MoCap from the back-end estimator. On the other hand, using the second network, the agents will learn how to avoid each other and distribute themselves around the person to cover all possible viewpoints. We also hypothesize that the best navigation policies for the robot(s) for the MoCap task should significantly depend on the MoCap's accuracy-related rewards, while other rewards may or may not be required.

% As the translational velocity commands are decoupled from the yaw rate commands for a quadcoptor-type MAV agent, we can combine the outputs of these networks during test time as follows. We use the rotational actions from the first network and the translational actions from the second network to obtain the best MoCap performance in a multi-agent scenario when the subject is in motion. For each kind of network we designed and experimented several variants, each with different MoCap objective-related rewards. Below we describe each network, its variants and the exact observations, actions and rewards used for them.


\subsubsection{Network 1: Single Agent Network}
All variants of single agent network use the following states and observations, where the superscript 1 denotes single agent network. 
\begin{equation}
 s_t^1 = [\xi_t ~~ \xi_t^P]; ~~ o_t^1 = [\mathbf{y}_t^{P} ~~ \mathbf{\dot{y}}_t^{P} ~~ \psi_t^{P}]
\end{equation}
% \begin{equation}
%  o_t^1 = [\mathbf{y}_t^{P} ~~ \mathbf{\dot{y}}_t^{P} ~~ \psi_t^{P}]
% \end{equation}
The actions for all single agent network variants consist of $a_t$ as stated in (\ref{eqn:actions}). They are all trained on a moving subject. These variants differ only in their reward structure as described below. The rewards are computed at every timestep. However, for sake of clarity we drop the subscript $t$ from the reward variables.

\paragraph{Network 1.1 -- Only Centering Reward} 
In this variant we only reward the agent based on the intuitive reasoning of keeping the person as close as possible to the center of the image from the MAV agent's on-board camera. It is calculated as follows. 
\begin{equation}
 r_{\textrm{center}} = 1 - \tanh({c_1d_{\mathrm{px}}}),
 \label{eqn:centrtingreward}
\end{equation}where $d_{\mathrm{px}}$ is the distance between the center of the person's bounding box on the image to the image center, measured in pixels. $c_1=0.01$ is a weighting constant. Note that keeping the person centered in each frame is not the goal of this work. As per the above-stated hypothesis, centering reward may not be required at all. Thus, Network 1.1 will only serve as a comparison benchmark to highlight that a MoCap's accuracy-related reward is explicitly required.


\paragraph{Network 1.2 -- SPIN Reward}
In this variant of the network we reward the agent based on the output accuracy of the MoCap back end. For this, we use SPIN \cite{kolotouros2019spin}, a state-of-the-art method for human pose and shape estimation using monocular images. At every time-step of training, we use SPIN on the image acquired by the agent and compute an estimate of $\mathbf{\hat{x}}_{j,t}^P \forall j ; j=1\cdots14$ corresponding to all 14 joints. In the synthetic training environment we have access to the true values of these joints, denoted by, $\mathbf{\bar{x}}_{j,t}^P \forall j ; j=1\cdots14$. SPIN reward is then given by
\begin{equation}
 r_{\textrm{SPIN}} = 1 - \tanh({c_2d_{\mathrm{J}}}),
\label{eqn:spinreward}
\end{equation}where $d_{\mathrm{J}} = \frac{1}{14}\sum_{j=1}^{14}(||\mathbf{\hat{x}}_{j,t}^P  - \mathbf{\bar{x}}_{j,t}^P||_2)$ and $c_2 = 5$ is a weighting constant.

% where
% \begin{equation}
%  d_{\mathrm{J}} = \frac{1}{14}\sum_{j=1}^{14}(||\mathbf{\hat{x}}_{j,t}^P  - \mathbf{\bar{x}}_{j,t}^P||_2)
%  \label{eqn:mocaperror_single}
% \end{equation} and $c_2 = 5$ is a weighting constant.

%Note that we do not explicitly penalize the errors in the body shape estimation. As SPIN (and MultiViewHMR, introduced later in this section) jointly estimate the body joint positions and the body shape, penalizing only the errors in the joint positions implicitly considers errors in the shape also.   

% \todoAamir{Explain how pose and shape are intertwined for this reward.}

\paragraph{Network 1.3 -- Weighted SPIN Reward}
Network 1.2 rewards the agent equally for the accuracy of each joint. However, the joints further away from the pelvis (also mentioned as the root joint), like hands or foot, have a greater tendency to be in an erratic motion than the ones closer to the root, like hips. To account for this, in the network variant 1.3 we penalized the outward joints more and hence define a Weighted SPIN reward as,
\begin{equation}
 r_{\textrm{WSPIN}} = 1 - \tanh({c_2d_{\mathrm{W}}}), 
 \label{eqn:weightedspinreward}
\end{equation}where
% \begin{equation}
$d_{\mathrm{W}} = \frac{1}{14}\sum_{j=1}^{14}(w_j||\mathbf{\hat{x}}_{j,t}^P  - \mathbf{\bar{x}}_{j,t}^P||_2)$
% \end{equation}
and $w_j$s are positive weights that sum to 1.

\paragraph{Network 1.4 -- Centering and Weighted SPIN Reward}
The last variant of the single agent uses a summed reward given as $r_{\textrm{sum}} = r_{\textrm{center}} + r_{\textrm{WSPIN}}$.
% \begin{equation}
%  r_{\textrm{sum}} = r_{\textrm{center}} + r_{\textrm{WSPIN}}.
% \end{equation}


\smallskip



\subsubsection{Network 2: Multi-Agent Network}
All three variants of the multi agent network, described below, use the state as defined in (\ref{eqn:state_obs}).
The observations for Network variants 2.1 and 2.2 are equal to (\ref{eqn:state_obs}) without  $\mathbf{\dot{y}}_t^{P}$ as these variants are trained on a static subject. In these two variants the action space excludes yaw control. Hence, during their training, we use a separate yaw controller to always orient the agent towards the person. On the other hand, Network 2.3 is trained with the full observation space as stated in (\ref{eqn:state_obs}) on a moving subject, and it uses the full action space is as stated in (\ref{eqn:actions}). Meaning, Network 2.3 also includes yaw-rate control.


The difference in the reward structure is described below.

\paragraph{Network 2.1: Centering, collision avoidance and AlphaPose Triangulation Reward (Trained with Static Subject)} 
In this variant we use a sum of three rewards $r_{\textrm{center}}$, $r_{\textrm{col}}$ and $r_{\textrm{triag}}$. Here, $r_{\textrm{center}}$ is same as defined in (\ref{eqn:centrtingreward}). $r_{\textrm{col}}$ rewards avoiding collisions by penalizing based on the distance from the neighboring robot. It is computed as
\begin{equation}
r_{\textrm{col}} = \begin{cases}
-1 ,& \text{if } \|\mathbf{x}_t-\bar{\mathbf{x}}_t\|_2\geq \mathbf{x}_{thresh}\\
0.2 ,              & \text{otherwise}
\end{cases}
\label{eqn:colreward}
\end{equation}where $\mathbf{x}_{thresh} = 3$m in our implementation.

$r_{\textrm{triag}}$ is a simplified MoCap-specific reward in a 2-agent scenario, which we obtain using a triangulation-based method. AlphaPose \cite{cao2017realtime} is a state-of-the-art human joint detector which provides body joint detections on monocular images. At every time step we use it on the images obtained by the agent and its neighbor to obtain $o_{j,t}\in\mathbb{R}^{14}$ and $\bar{o}_{j,t}\in\mathbb{R}^{14}$, respectively. Using known camera intrinsics and extrinsics (from self-pose estimates) for both agents, a point in the image plane and its corresponding view from another camera, we can estimate the 3-D position of the point using a least squares formulation (equation (14.42) in \cite{prince2012computer}). Therefore, by using $o_j$ and $\bar{o}_{j,t}$, we estimate the 3D positions of all 14 joints of the subject as $\mathbf{\tilde{x}}_{j,t}^P \forall j; j=1\cdots14$ and compare it to ground-truth joint positions $\mathbf{\bar{x}}_{j,t}^P \forall j; j=1\cdots14$. Thus, $r_{\textrm{triag}}$ is given by
\begin{equation}
 r_{\textrm{triag}} = 1 - \tanh({c_3d_{\mathrm{triag}}}),
 \label{eqn:triagreward}
\end{equation}where $d_{\mathrm{triag}} = \frac{1}{14}\sum_{j=1}^{14}(||\mathbf{\tilde{x}}_{j,t}^P  - \mathbf{\bar{x}}_{j,t}^P||_2)$.

% \begin{equation}
%  d_{\mathrm{triag}} = \frac{1}{14}\sum_{j=1}^{14}(||\mathbf{\tilde{x}}_{j,t}^P  - \mathbf{\bar{x}}_{j,t}^P||_2)
% \end{equation}

\begin{figure}[!t]
 \includegraphics[width=0.9\columnwidth]{singleAgentRA_L.pdf}
 \caption{Single Agent Network: Variants of this network are trained with different rewards as described in sub-subsection \ref{subsec:proposed_method}--1.}
 \label{fig:singleagentnet}
\end{figure}




\paragraph{Network 2.2: Centering, collision avoidance and Multiview HMR Reward (Trained with Static Subject)}
In this variant we use a sum of three rewards $r_{\textrm{center}}$, $r_{\textrm{col}}$ and $r_{\textrm{MHMR}}$. The first two are same as (\ref{eqn:centrtingreward}) and (\ref{eqn:colreward}), respectively. $r_{\textrm{MHMR}}$ rewards the agent based on the output accuracy of the MoCap back end using images from multiple agents. For this, we use MultiviewHMR \cite{liang2019shape}. It is a state-of-the-art method for human pose and shape estimation using images from multiple viewpoints. At every timestep of training, we use it on the image acquired by the agent and its neighbor to compute an estimate of $\mathbf{\check{x}}_{j,t}^P \forall j; j=1\cdots14$ corresponding to all 14 joints. The reward is then given by
\begin{equation}
 r_{\textrm{MHMR}} = 1 - \tanh({c_4d_{\mathrm{mhmr}}}),
 \label{eqn:mviewhmr}
\end{equation}where $d_{\mathrm{mhmr}} = \frac{1}{14}\sum_{j=1}^{14}(w_j||\mathbf{\check{x}}_{j,t}^P  - \mathbf{\bar{x}}_{j,t}^P||_2)$ 
% \begin{equation}
%  d_{\mathrm{mhmr}} = \frac{1}{14}\sum_{j=1}^{14}(w_j||\mathbf{\check{x}}_{j,t}^P  - \mathbf{\bar{x}}_{j,t}^P||_2),
% \end{equation} 
and the weights are as described in the previous section.


\paragraph{Network 2.3: Centering, continuous collision avoidance and Multiview HMR Reward (Trained with Moving Subject)}
In this variant we use a sum of three rewards $r_{\textrm{center}}$, $r_{\textrm{concol}}$ and $r_{\textrm{MHMR}}$. Here $r_{\textrm{center}}$ and $r_{\textrm{MHMR}}$ are same as (\ref{eqn:centrtingreward}) and (\ref{eqn:mviewhmr}). The continuous collision avoidance reward is given as follows.
\begin{equation}
r_{\textrm{concol}} = \begin{cases}
-v_{\textrm{pot}} ,& \text{if } \|\mathbf{x}_t-\bar{\mathbf{x}}_t\|_2\leq \mathbf{d}_{lthresh}\\
0.2 ,& \text{if } \mathbf{d}_{lthresh} \leq \|\mathbf{x}_t-\bar{\mathbf{x}}_t\|_2\leq \mathbf{d}_{hthresh}\\
-1 ,              & \text{otherwise}
\end{cases}
\label{eqn:concolreward}
\end{equation}where $\mathbf{d}_{lthresh} = 1.0$m and $\mathbf{d}_{hthresh} = 20$m. $v_{\textrm{pot}}$ is obtained using the potential field functions as described in our previous work \cite{rahul_CASE_2019} (equation 3). Furthermore, the value of $v_{\textrm{pot}}$ is clamped to $1$.


\paragraph{Network 2.4 + Potential Field: Centering and Multiview HMR Reward (Trained with Moving Subject)}
In this variant, we use a sum of two rewards, namely, $r_{\textrm{center}}$ (\ref{eqn:centrtingreward}) and $r_{\textrm{MHMR}}$ (\ref{eqn:mviewhmr}). The key difference in this case w.r.t. Network 2.3 is that here we use a potential field-based collision avoidance method \cite{rahul_CASE_2019} as a part of the environment during the training to keep the robots from colliding with each other at all times. It is not embedded in the reward structure and hence, the robots are not explicitly penalized for it. Testing of this network, during experiments, was also performed with potential field-based collision avoidance as a part of the environment.

% Details of all the network architectures, training process, libraries, instructions on how to run the code, etc., are provided in the attached supplementary material.





% \begin{equation}
% \label{eq:cotforce} 	
% v_{\textrm{pot}} = 	\begin{cases} 	
% 1\ & \text{if} ~ v_{\textrm{pot}}\leq 1 \\ 	
% \frac{\pi}{2} \big(\frac{cot(z)+z -\frac{\pi}{2}}{d_{max}-d_{min}}\big)& \text{if} ~ d_{min} \leq d \leq d_{max} \\ 	0, & \text{if} ~~ d > d_{max} 	
% \end{cases}\;. 	
% \end{equation} 




% The repulsive potential field magnitude w.r.t the $i^{th}$ obstacle, is given as, 
% {\small 
% \begin{equation}
% \label{eq:cotforce} 	
% F^{i}(d) = 	\begin{cases} 	
% F_{max}\ & \text{if} ~ d < d_{min} \\ 	
% \frac{\pi}{2} \big(\frac{cot(z)+z -\frac{\pi}{2}}{d_{max}-d_{min}}\big)& \text{if} ~ d_{min} \leq d \leq d_{max} \\ 	0, & \text{if} ~~ d > d_{max} 	
% \end{cases}\;. 	
% \end{equation}
% } 
% Here, $z = \frac{\pi}{2} \big(\frac{d-d_{min}}{d_{max}-d_{min}}\big)$, argument $d$ is a distance metric between $\x_t^B$ and obstacle $i$. Note that, $	F^{i}(d)$ varies hyperbolically w.r.t $d$. $d_{max}$ and $d_{min}$ are distances defining the region of influence of the potential field and the distance at which the potential field value tends to infinity respectively. In practice, the potential field at $d_{min}$ is clamped to a positive value $F_{max}\geq max(\|\ao_t^{B}\|)$ to ensure obstacle avoidance. \\

% 


\begin{figure}[!t]
 \includegraphics[width=0.9\columnwidth]{multiAgentRA_L.pdf}
 \caption{Multi Agent Network: Variants of this network are trained with different rewards as described in sub-subsection \ref{subsec:proposed_method}--2.}
 \label{fig:multiagentnet}
\end{figure}

% % %########################################### Moved to sup mat
% \begin{algorithm}[h]
% 	\caption{Pseudocode for centralized training} 
% 		\label{alg:1}
% 		\begin{algorithmic}
% 			\begin{footnotesize}
% 				\label{Alg:fc}
% 				\STATE \textbf{Input}: initial policy parameters $\theta_0$, initial value function parameters $\mu_0$
% 				\STATE \textbf{for} m in Total Episodes
% 				\begin{itemize}
% 				 \item Collect trajectories $D_m=\{s_t,o_t,r_t,a_t\}, t = \{1\dots T_{episode}\}$ 
% 				 \item from parallel Gazebo runners, with 2 agents using policy $\pi_{m,\theta}$, estimate advantage $\hat{A}_{\pi_{m,\theta}}(s_t,a_t)$ \cite{schulman2015high} using the value function $V_{m-1,\mu}(s_t)$  and reward-to-go from $D_m$.
% 				 \item Maximize PPO surrogate loss L \cite{schulman2017proximal} w.r.t $\theta$ via SGD with Adam.
% 				 \item Fit value function $V_{m,\mu}(s_t)$ by regressing on current reward-to-go.
% 				\end{itemize}				
% 				\STATE end \textbf{for}
% 			\end{footnotesize}
% 		\end{algorithmic}
% \end{algorithm} 
% 
% 
% \subsection{Algorithmic Details}
% We train the agents using a centralized training and decentralized execution paradigm. Specifically, a centralized fully observable critic and an actor with local observations are learned by collecting experiences of all the robots simultaneously. The robots then execute this common shared policy to collect new data. The fully observable centralized critic and shared actor policy aid in maintaining a stationary environment for our two-agent problem. This enables us to use conventional single-agent reinforcement learning algorithms to train the multi-agent policy network.
% Algorithm \ref{alg:1} summarizes the training methodology used.
% 
% 
% \subsection{Network Architecture}
% Figures~\ref{fig:singleagentnet} and \ref{fig:multiagentnet} show our network training architectures. Both single and multi agent policies are two layer $256\times256$ neural networks with ReLu activations. For estimating the advantage $A_t$, the state-value network $V_\mu(s_t)$ is approximated with a neural network with parameters $\mu$. The value network architecture is a clone of the policy network at each stage. The current value function is fit by regressing over the reward-to-go estimate. We train the networks on Tensorflow with Adam optimizer and a stable baselines software implementation of PPO\footnote{https://stable-baselines.readthedocs.io/en/master/modules/ppo2.html}.
% % %########################################### Moved to sup mat

\section{Experiments and Results}


\subsubsection{Training Setup in Simulation}

% \footnote{http://wiki.ros.org/openai\_ros}
% actor\footnote{http://gazebosim.org/tutorials?tut=actor\&cat=build\_robot}

We train and our networks in simulation. We use Gazebo multi-body dynamics simulator with ROS and OpenAI-Gym to train the MAV agents. For the MAV agent we use AscTec Firefly model with an on-board RGB camera facing down at 45$^\circ$ pitch angle w.r.t.\ the MAV body frame. We run 5 parallel instances of Gazebo and the Alphapose network on multiple computers over a network to render the simulation. The policy network is trained on a dedicated PC which samples a batch of transition and reward tuples from the network of computers to update the networks. We use a simulated human in Gazebo as the MoCap subject and generate random trajectories using a custom plugin. Details of the network architectures, training process, libraries, instructions on how to run the code, etc., are provided in the attached supplementary material.

\subsection{Simulation results}


In this sub-section we evaluate our trained policies in Gazebo simulation environment. We create a test trajectory for the simulated human actor for $120$s on which it walks with varying speeds. The best policy of each network variant, as described in subsection~\ref{subsec:proposed_method}, is run 20 times while the actor walks the trajectory. Thus, results from a total of $2400$s of evaluation run of each network variant is obtained. 

For single agent experiments, in addition to the DRL-based methods, we run 4 other methods: i) `Network 1.4 + AirCap', ii) Orbiting Strategy, iii) Frontal-view Strategy and iv) MPC-based approach \cite{ActiveTallamraju19}. For multi-agent experiments we run 2 additional methods: i) `Network 2.3 + AirCap' and ii) MPC-based approach \cite{ActiveTallamraju19}. All these were also run 20 times for 120s each to allow comparison with our DRL-based policies. 
`Network 1.4 + AirCap' and `Network 2.3 + AirCap' imply running the networks with `true observations' instead of directly using simulator-generated ground-truth observations. To this end, we ran the complete AirCap pipeline \cite{ActiveTallamraju19} during the test by replacing only the MPC-based high-level controller with the DRL policy in it. It executes an NN-based person detector, a Kalman filter-based estimator for person's 3D position estimation (not orientation), cooperative self-localization of the MAVs using simulated GPS measurements with noise as well as communication packet loss. More details regarding this are provided in the supplementary material associated with this article.
`Orbiting Strategy' is essentially a `model-free' approach in which a robot orbits around the person at a fixed distance in order to increase the coverage. In `Frontal-View Strategy' a robot maintains a fixed distance to the person and attempts to always keep the frontal view of the person in the camera image.
Below we discuss the results for single and multi-agent network variants and other aforementioned methods.


\begin{figure}[t]
 \includegraphics[width=\columnwidth]{single_agent_sim_experiments.eps}
 \caption{Simulation results of Single Agent Network variants.}
 \label{fig:sim_experiments_single_agent}
\end{figure}


% Thus, to compute MPE, the SPIN method \cite{kolotouros2019spin} is run on every image acquired and saved by the agents during the evaluation runs and then the equation for $d_{\mathrm{J}}$, as defined for the reward in (\ref{eqn:spinreward}), is used.


% \smallskip

\subsubsection{Single Agent Network Variants}

In order to compare the network variants, we use 2 metrics, i) centering performance error (CPE) and ii) MoCap performance error (MPE). CPE is computed as the pixel distance from the center of the bounding box around the person in the agent's camera image to the image center. MPE, for single agent networks is simply $d_{\mathrm{J}}$, as defined for the reward in (\ref{eqn:spinreward}). 
To compute this, the SPIN method \cite{kolotouros2019spin} is run on the images acquired by the agents during testing.

Note that the metric which quantifies the MoCap accuracy of any method in this paper is MPE (the right side box plots in Fig.~\ref{fig:sim_experiments_single_agent}  and \ref{fig:sim_experiments_multi_agent}). CPE is a metric that we plot only to make the policy performance intuitively explainable and understand `what' the learned RL policies are doing to achieve a good MPE.


 
Figure~\ref{fig:sim_experiments_single_agent} shows the error statistics of the aforementioned metrics. The grey background behind any box plot signifies that the method could not keep the person, even partially, in the MAV FOV, thereby completely losing him/her, for at least some duration of the experiment runs. In these cases, the box plot represents errors computed only for those timesteps when the person was at least partially in the FOV.


% The MPE performance of all approaches except the  Network variant 1.1 is reasonable and similar to each other.


MPE plots in Fig.~5 for single robot experiments show that for all methods the medians of the MPEs are very similar to each other. This is the most significant result, especially because we can demonstrate that in terms of accuracy our DRL-based approach is on par with the state-of-the-art MPC-based approach [2] (or fixed-strategy methods), without the need for hand-crafting observation models and system dynamics (or pre-specified robot trajectories). Furthermore, the MPE for network 1.4 and 1.2 also has significantly less variance of MPE compared to all other methods. Due to these reasons, Network 1.4 and Network 1.2 are the two most successful approaches for the MoCap task.

From Fig.~5 plots, we also see that Network 1.4 keeps the person centered much more than Network 1.2, 1.3 or MPC. This is expected because Network 1.4 is rewarded for centering the person in the image in addition to SPIN-based MoCap rewards. Network 1.2 or 1.3, on the other hand, only has SPIN-based MoCap rewards. Nevertheless, the MPE of Network 1.4 is only slightly better than that of Network 1.3. This signifies that centering the person in the image does not have a great impact on the accuracy of the motion capture estimates.


Network 1.1, which often lost the person in its FOV, outperforms all other methods in its CPE performance for the duration it could `see' the person. This is expected as it is trained with only centering reward. Even though its MPE mean for the person-visible duration is similar to other networks, the variance of its MPE is higher than the other networks. Moreover, the fact that it could keep the person in FOV only $76\%$ of the time as compared to $100\%$ for other networks (1.2--1.4) makes it less desirable even for the MoCap task. 
%The main reason why it loses the person is as follows.



% The next inference we make here is that rewarding the agent for both centering and MoCap performance allows for significantly better centering performance, than when these rewards are treated separately. 
% 
% The most important observation is that to obtain high MoCap performance the  centering rewards play little role.

%In fact, agents with only centering reward tend to lose the person from their FOVs so often that their overall MoCap performance becomes significantly poorer than other network variants. This is due the following reason.

% %########################################### Moved to sup mat
% When the only reward concerns centering (Network variant 1.1), there is only a single image point constraint for the MAV's to keep. In this case, due to the underactuation of the MAV agents and the the fact that the camera is rigidly attached to the MAV frame, the MAVs tend to lose the person completely while making fast and aggressive maneuvers while maintaining the single point constraint.
% 
% On the other hand, when rewarded for only MoCap-related objectives (Networks 1.2 and 1.3), the agents become constrained by many more points on the image plane. Hence, they are more likely to keep the person anywhere on the image plane, irrespective of how far from the image center. Clearly, then combining both these rewards helps to achieve best CPE, while the MPE remains similar to the agents that only got MoCap objective-related rewards.
% %###########################################



% Finally, we also observe that our DRL-based approach performs similar in MoCap accuracy to that of MPC-based approach, while keeping a much low variance in the MoCap errors. MPC-based approach requires hand-crafted observation models. Thus, the fact that our observation model-free DRL-based approach is on par with it in accuracy and better in precision, is a very significant result.


The median MPE of `Network 1.4 + AirCap' is very similar to all other methods. However, it should be noted that there is one drawback in `Network 1.4 + AirCap'. As the `ground truth observations' are not used in this method and the simulated person can rapidly make sudden direction changes, the person is much more susceptible to go out of the FOV of the MAV's camera. Since the network never learned to `search' for the person who is out of the FOV, the method has to `wait' until the person walks back in the FOV. The cooperative estimation method of the AirCap pipeline helps in this regard as the person might still be in another robot's FOV. For a single robot case this is also not possible. Thus, `Network 1.4 + AirCap' loses the person for 35\% of the time. 



The strategy-based methods struggle to keep the person, even partially, in the MAV camera's FOV. While the `Orbiting Strategy' was able to keep the person in the FOV for 73\% of the total time of all experiments combined, the `Frontal-View Strategy' managed to do that only 20\% of the total time. This is because when the person changes his direction or speed of motion, the robot could fly around to reposition itself in the front of the person, thus losing him during the transition. On the other hand, our successful DRL-based approaches, i.e., Network 1.2, 1.3 and 1.4, never lose the person from the camera FOV.
Based on this analysis, we can conclude that the strategy-based methods, while being `model-free', still have a major drawback of losing the person often, if not very carefully hand-crafted. Our DRL-based approaches `explore' the space of these strategies and finds the most suitable one in their policies. 





\begin{figure}[t]
 \includegraphics[width=\columnwidth]{multi_agent_sim_experiments.eps}
 \caption{Simulation results of Multi-Agent Network variants.}
 \label{fig:sim_experiments_multi_agent}
\end{figure}

\subsubsection{Multi-Agent Network Variants}

The MPE in the multi-agent case is also simply $d_{\mathrm{J}}$, as defined for the reward in (\ref{eqn:spinreward}), but instead of using SPIN as in the single agent case, here it is computed by running Multiview HMR \cite{liang2019shape} for pose and shape estimation on every simultaneous pair of images acquired by both the agents during the evaluation runs.
Network 2.1 and 2.2 were trained and tested on a static person. On the other hand, Network 2.3 and Network 2.4 + Potential Field were both trained and tested with a moving person (in the same way as for the single agent experiments). The remaining two methods in the multi-agent case were also tested with moving persons.

%Here, our key observation is that the Network 2.2, trained with MoCap-specific objectives (and centering), has no noticeable difference from the MPC-based approach in terms of MoCap performance. 

% On the other hand, we notice that Network 2.2 does not achieve good performance in centering the person in the MAV's camera FOV in comparison with the MPC-based approach. This again highlights that MoCap performance depends little on whether the person is in the center of the image or not. MPC, however, enforces these constraints directly. Finally, we observe that our multi agent network variant 2.3 shows poorer performance when the subject is moving. This is mainly attributed to the following issues. First, when the subject moves, the MAVs have to follow them. This increases the chances of inter-robot collisions. Hence, the agents might require a lot more training episodes to learn optimal policies.

Figure~\ref{fig:sim_experiments_multi_agent} shows the error statistics of multi-agent simulation experiments. The best performing network in multi-agent case is Network 2.3. It is very similar to the MPC-based method in terms of the MPE median value (See Fig.~6 right side) and has much less MPE variance than MPC. This is a very significant result as MPC required observation models of the subject and our DRL-based approach in Network 2.3 did not. In the MPC approach, the viewpoint configurations for the MAVs emerge out of the joint target perception models. In contrast, in the DRL-based approach the MAVs directly learn the viewpoint configurations from experience. We also notice that the rewards based on a triangulation method assist, to some extent, in achieving acceptable MoCap performance (see results of Network 2.1). However, they remain inferior to the Network 2.3, which used the sophisticated approach taken in Multiview HMR \cite{liang2019shape} for reward computation.


%Furthermore, for the multi-robot case with moving people we added two more methods in the revised manuscript: `Network 2.3 + AirCap' (as explained in \S 3) and `Network 2.4 + Potential field'. The latter is essentially similar to Network 2.3 with the difference that it is not rewarded for collision avoidance with teammate during training or testing. Instead, a potential-field based collision avoidance is run ad-hoc, both during training and testing. 

Furthermore, we find that in terms of MPE, `Network 2.3 + AirCap' is close to both Network 2.3 and MPC. Similar to `Network 1.4 + AirCap', the `Network 2.3 + AirCap' also loses the person from the robots' FOV. However, it is present in at least one robot's FOV for approx.~97\% of the total experiment duration. The increased visibility in the multi-robot case is due to the cooperative estimator module of AirCap pipeline. This assessment signifies the usability of our method in real robots with real observations.

Next, we find that the policy learned by `Network 2.4 + Potential field' was able to achieve MPE median value comparable to Network 2.3 but at the cost of slightly higher MPE variance and loss of person from at least one robot's FOV for several periods (13\% of total duration). This experiment further signifies the key benefit of our DRL-based approach in Network 2.3. It overcomes the need for knowing models, strategies as well as any ad-hoc collision avoidance techniques. In Network 2.3 the learned policy not only achieves good MoCap performance, but it also naturally learns to avoid collisions with the teammates. In the video associated to this paper (also available here -- \url{https://youtu.be/07KwNjc7Sy0}) we show how well Network 2.3 performs.
The networks for the moving person, however, did not ensure very good centering of the person in the image (see the left side of Fig.~6) as compared to the MPC-based approach. Despite this, their MPE performances are only slightly poorer than MPC (MPE median difference is approx.~0.05m only). This further signifies that centering the person on the image has a very low effect on MoCap performance.

%For multi robot experiments we evaluated the trained policy of the best performing network (Network 2.3)\footnote{In the revised manuscript we show that Network 2.3 now works well and results in MPE similar to that of MPC. Please refer \S 4 and \S 5} by running evaluation experiments with this policy and using the complete AirCap pipeline in simulation. These evaluation experiments (henceforth called `Network 2.3 + AirCap') were also run for 20 times, 120 seconds each time like other methods in the previous version of the paper. 

\begin{figure}[t]
\centering
 \includegraphics[width=0.9\columnwidth]{Overlay}
 \caption{A snapshot of the real robot experiment.}
 \label{fig:realrobotexpsfootage}
\end{figure}

Finally, for the multi-agent case, we find that the medians of the MPEs for all multi-agent networks were substantially lowered compared to the MPEs obtained by single-drone experiments (from $\sim$ 0.7m to 0.22m). This highlights the benefit of using multiple drones and hence multiple views to improve MoCap performance. 



\subsection{Real Robot results}




%MPC -- 1243.9/3 = 414
%RL -- 2322.7/3  = 741

% \footnote{https://www.ryzerobotics.com/tello}

In order to validate our approach in a real robot scenario, we used a DJI Ryze Tello drone. It consists of a forward looking camera capturing images at $30$ hz. The drone is controllable using an SDK with ROS interface. Tello has the functionality of vision-based localization, which is highly inaccurate. Hence, we performed experiments within a Vicon hall with markers on top of the drone to estimate its position and velocity. The tracked subject wore a helmet with Vicon markers. Vicon-based position estimate of the person was used to compute the observations for the neural network.

\begin{figure}[!t]
 \includegraphics[width=\columnwidth]{real_world_experiments}
 \caption{Real Robot Experiments: Comparison of single agent network variant 1.1 and MPC-based \cite{ActiveTallamraju19} approach.}
 \label{fig:realrobotexps}
\end{figure}



We performed experiments with $1$ Tello drone and compared our DRL-based approach using Network 1.1 with state-of-the-art MPC-based approach \cite{ActiveTallamraju19}. These were performed for approximately $400$s and $700$s, respectively. Figure~\ref{fig:realrobotexpsfootage} shows an external camera footage of the experiment and the on-board drone view with pose and shape overlay using SPIN. As the ground truth pose and shape of the human subject in real experiment is not available, we only compare the following criteria. We compare i) the length and breadth of the bounding box around the person in the drone images, and ii) proximity of the person to the center of those images, calculated as pixel distance from the image center to the center of the bounding box around the person. The bounding boxes are computed by running Alphapose \cite{cao2017realtime} method on the images recorded by the drone. Figure~\ref{fig:realrobotexps} presents the statistics of these evaluation criteria. We notice that the performance of both approaches is similar in terms of the person's proximity to the image center, with our DRL-based approach performing slightly better. However, we observe that the MPC-based approach is consistently able to keep a larger size (projected height) of the person in the images. This is due the fact that the MPC's objectives enforce it to keep a certain threshold distance to the person. As the DRL-based approach has no such incentive, it varies its distance to the person more, therefore causing a greater variance in the projected height of the person. On the other hand, this enables our DRL-based approach to change its relative orientation with respect to the person such that she/he is is observed from several possible sides. This is evident by the greater variance in the projected width of the person on the images. This property of our DRL-based approach will benefit pose and shape estimation methods, as demonstrated in the simulation experiments.






\section{Conclusions and Future Work}

% \setlength{\belowcaptionskip}{-5pt}
% 
% 
% \setlength{\belowcaptionskip}{-15pt}



In this letter, we presented the first deep reinforcement learning-based approach to human motion capture using aerial robots. 
%We introduce a novel solution to the formation control problem of the MoCap's robotic front-end that learns control policies directly from the MoCap's backend performance. 
Our solution does not depend on hand-crafted system or observation models. Formation control policies are directly learned through experience, which is obtained in synthetic training environments. 
%Through centralized learning and decentralized evaluation paradigm we enable 2 agents to concurrently learn in a parallelized training setup. 
%We proposed several network variants based on the number of agents and the reward structure and evaluate each of them through extensive simulation experiments. 
Through extensive experiments and comparisons we find that DRL-based agents learn extremely good policies, on par with carefully designed model-based (MPC) or model-free, fixed strategy-based methods. These policies even generalize to real robot scenarios. We also find that multiple agents learn even better policies and outperform single agents in performing MoCap.
\
The learning objective (MoCap accuracy) is far simpler to construct than deriving system or observation models \cite{ActiveTallamraju19}. Moreover, strategy based methods, as shown in our experiments, can have various drawbacks, such as losing the person from the field of view. To overcome that, each drawback must be identified and addressed within the fixed strategy. A DRL-based approach overcomes the need for fixing a strategy by `exploring' the space of such strategies. Thus, a major conclusion of our work is that DRL-based approaches are likely the ideal way forward for aerial MoCap systems. Eventually, an end-to-end approach of learning actions directly from images is needed to overcome the need for an additional person-detection method, that has been used so far (e.g., SSD multibox in AirCap \cite{DeepPrice18}). To this end, we are improving our training by using SMPL body models in richer, photorealistic simulated environments.

Our approach would also be applicable in a real robot setting with `real observations' while achieving accuracy similar to an MPC-based approach \cite{ActiveTallamraju19}. Nevertheless, this is valid only for those durations when the person is not lost from the FOV of all cameras. In order for the policy to `search' for the person, network training should be done with the AirCap pipeline's `real observations'. This would involve running several DNN-based detectors and keeping track of delayed measurements. Furthermore, our approach is limited in terms of scaling up to more agents. While addressing this will require more sophisticated network architecture, it should be noted that 2 to 3 aerial robots may be enough to achieve a good MoCap accuracy \cite{MarkerlessNitin19}.

% We have shown that our DRL-based approaches achieve accuracy, similar to MPC or fixed strategy-based methods, without the need for hand-crafted system/observation models. 





\section*{Acknowledgments}
The authors would like to thank Prof.\ Dr.\ Heinrich B\"ulthoff for his constant support and providing us the access to the Vicon tracking hall in MPI for Biological Cybernetics. The authors also thank Igor Martinovi\'c and the anonymous reviewers for extremely helpful suggestions.

















%%%%%%%%%%%%%%%%%%%%%%%%%%%%%%%%%%%%%%%%%%%%%%%%%%%%%%%%%%%%%%%%%%%%%%%%%%%%%%%%%%%%%%%%%%%%%%%%%%%%%%%%%
\bibliographystyle{IEEEtran}  % do not change this line!
\bibliography{bibliography}


\end{document}
