\section{Introduction}

The advent of large language models (LLMs) has significantly advanced the development of autonomous agents capable of performing complex tasks across various domains, including question answering, code generation, and web navigation. These LLM-based agents leverage the extensive knowledge and reasoning capabilities inherent in LLMs to make decisions and plan actions. A prevalent approach in this context is the integration of Monte Carlo Tree Search (MCTS) \cite{coulom2006MCTS} with LLMs, wherein the LLM guides the exploration of potential action sequences to achieve specific goals \cite{hao2023reasoning,zhou2024language,shi2025monteICLR,gan2025master}. This combination has shown promise in enhancing the decision-making processes of autonomous agents.

Despite the recent progress in integrating LLMs with planning algorithms, existing methods \cite{wei2022chaincot,cot-sc2023,yao2023react,shinn2023reflexion,yao2023tree,hao2023reasoning} predominantly adopt a single-model paradigm in which one LLM is queried multiple times with identical or slightly perturbed prompts to simulate diverse action branches during MCTS. The underlying assumption is that the stochasticity or sampling variance of the model is sufficient to generate rollouts that explore a wide range of potential solutions. However, in practice, this approach suffers from a critical limitation: the outputs tend to exhibit high similarity across calls, often reflecting the same dominant reasoning pattern learned by the model \cite{hao2023reasoning,zhou2024language,gan2025master}. As a result, the generated rollouts lack meaningful diversity, leading to narrow and redundant search trajectories.

This deficiency severely constrains the agent’s exploration capability within the solution space. When the search tree is populated with highly similar branches, the planner becomes susceptible to local optima, and its ability to discover novel or unexpected solutions is greatly diminished. In particularly challenging tasks that require compositional reasoning or multi-step tool use, the agent may fail to identify the correct solution path altogether. Even in cases where the solution is eventually found, the process may involve excessive sampling and token consumption, incurring significant computational overhead. These inefficiencies highlight a fundamental mismatch between the need for diverse exploration in planning and the limited variability achievable by repeatedly sampling from a single, monolithic LLM.


To address the above limitations,we propose \textbf{SY}nergistic \textbf{M}ulti-agent \textbf{P}lanning with \textbf{H}eterogene\textbf{O}us La\textbf{N}guage Model Assembl\textbf{Y} (\textbf{SYMPHONY}). The framework integrates multiple language models into a unified planning system that enhances multi-step reasoning through diversity-aware search, adaptive coordination, and reflective adaptation.

A central innovation of SYMPHONY is its heterogeneous agent pool, composed of LLMs with diverse pretraining sources and reasoning styles. Instead of relying on a single agent, SYMPHONY assigns different agents to generate candidate actions at each search node, thereby introducing structural diversity into the search tree. This diversity increases the likelihood of generating complementary reasoning paths, reduces model-specific biases, and improves performance on complex, multi-hop tasks. Empirical results show that expanding model diversity leads to more unique branches per node and consistent gains in task accuracy.

In addition to model heterogeneity, SYMPHONY incorporates several complementary components that further enhance planning performance. A UCB-based scheduling strategy dynamically allocates agents based on historical effectiveness, improving coordination across agents. An entropy-modulated confidence scoring (EMCS) mechanism calibrates value estimates using agent-level uncertainty, yielding more stable evaluations. Finally, a pool-wise memory sharing mechanism enables agents to learn from past failures through natural language reflections, which are shared across the agent pool and incorporated into future prompts. These components together support efficient, adaptive, and robust search behavior.





We evaluate SYMPHONY across three distinct environments that represent key capabilities of LLM-based agents: multi-hop reasoning (HotpotQA), sequential decision making (WebShop), and code generation (MBPP). Experimental results show that SYMPHONY consistently outperforms strong baselines.  In addition to improved performance, SYMPHONY achieves higher planning efficiency, requiring fewer MCTS node expansions to reach correct solutions. Notably, the framework delivers competitive or superior results even when built upon cost-effective models, demonstrating practical value without relying on high-cost large-scale deployments.