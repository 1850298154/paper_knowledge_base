\documentclass{article}


% if you need to pass options to natbib, use, e.g.:
%     \PassOptionsToPackage{numbers, compress}{natbib}
% before loading neurips_2025

 \PassOptionsToPackage{numbers, compress}{natbib}
% ready for submission
\usepackage[final]{neurips_2025}


% to compile a preprint version, e.g., for submission to arXiv, add add the
% [preprint] option:
%     \usepackage[preprint]{neurips_2025}


% to compile a camera-ready version, add the [final] option, e.g.:
%     \usepackage[final]{neurips_2025}


% to avoid loading the natbib package, add option nonatbib:
%    \usepackage[nonatbib]{neurips_2025}


\usepackage[utf8]{inputenc} % allow utf-8 input
\usepackage[T1]{fontenc}    % use 8-bit T1 fonts
% \usepackage{hyperref}       % hyperlinks
\usepackage{url}            % simple URL typesetting
\usepackage{booktabs}       % professional-quality tables
\usepackage{amsfonts}       % blackboard math symbols
\usepackage{nicefrac}       % compact symbols for 1/2, etc.
\usepackage{microtype}      % microtypography
\usepackage{xcolor}         % colors

\usepackage{amsthm}
\newtheorem{theorem}{Theorem}


\usepackage{algorithm}
\usepackage{algpseudocode}
% \usepackage[ruled,vlined]{algorithm2e}
% \usepackage{amsfonts}       % blackboard math symbols
% \usepackage{listings}
% % listings 配置:等宽字体、小字号、自动断行
% \lstset{
%   basicstyle=\ttfamily\small,
%   breaklines=true,           % 自动断行
%   breakatwhitespace=true,    % 在空格处断行
%   breakindent=0pt,           % 续行无额外缩进
%   breakautoindent=false,     % 关闭自动续行缩进
%   columns=fullflexible,
%   keepspaces=true,
%   xleftmargin=0pt,           % 整体左贴环境左边界
%   frame=none
% }
\usepackage{listings} % 加载 listings 包

% 配置 listings 样式(可根据需求调整)
\lstset{
    basicstyle=\ttfamily\small, % 代码字体和字号
    breaklines=true, % 关键:自动换行
    columns=flexible, % 灵活列宽(避免字符挤压)
    tabsize=4, % Tab 缩进为 4 个空格
    showstringspaces=false, % 不显示字符串中的空格
}

\usepackage{amsmath} 
\usepackage{amssymb}
\usepackage{graphicx}       % for including graphics
\usepackage{hyperref}
\usepackage{caption}
\usepackage{makecell}

% \usepackage[UTF8]{ctex}
\newcommand{\comment}[2][red]{\textcolor{#1}{\fbox{\textbf{#2}}}}

\title{
SYMPHONY: Synergistic Multi-agent Planning with Heterogeneous Language Model Assembly
}


% The \author macro works with any number of authors. There are two commands
% used to separate the names and addresses of multiple authors: \And and \AND.
%
% Using \And between authors leaves it to LaTeX to determine where to break the
% lines. Using \AND forces a line break at that point. So, if LaTeX puts 3 of 4
% authors names on the first line, and the last on the second line, try using
% \AND instead of \And before the third author name.


% \author{%
%   David S.~Hippocampus\thanks{Use footnote for providing further information
%     about author (webpage, alternative address)---\emph{not} for acknowledging
%     funding agencies.} \\
%   Department of Computer Science\\
%   Cranberry-Lemon University\\
%   Pittsburgh, PA 15213 \\
%   \texttt{hippo@cs.cranberry-lemon.edu} \\
%   % examples of more authors
%   \And
%   Coauthor \\
%   Affiliation \\
%   Address \\
%   \texttt{email} \\
%   \AND
%   Coauthor \\
%   Affiliation \\
%   Address \\
%   \texttt{email} \\
%   % \And
%   % Coauthor \\
%   % Affiliation \\
%   % Address \\
%   % \texttt{email} \\
%   % \And
%   % Coauthor \\
%   % Affiliation \\
%   % Address \\
%   % \texttt{email} \\
% }



\author{%
  Wei Zhu 
  % \And
  \quad
  Zhiwen Tang\thanks{Corresponding author} 
  % \And
  \quad
  Kun Yue \\
  School of Information Science and Engineering, Yunnan University, Kunming, China \\
  Yunnan Key Laboratory of Intelligent Systems and Computing, Kunming, China \\
  \texttt{zhuwei@stu.ynu.edu.cn, \{zhiwen.tang, kyue\}@ynu.edu.cn} \\
}



\begin{document}


\maketitle


\begin{abstract}
Recent advancements have increasingly focused on leveraging large language models (LLMs) to construct autonomous agents for complex problem-solving tasks. However, existing approaches predominantly employ a single-agent framework to generate search branches and estimate rewards during Monte Carlo Tree Search (MCTS) planning. This single-agent paradigm inherently limits exploration capabilities, often resulting in insufficient diversity among generated branches and suboptimal planning performance.
To overcome these limitations, we propose  \textbf{SY}nergistic \textbf{M}ulti-agent \textbf{P}lanning with \textbf{H}eter\textbf{O}geneous la\textbf{N}gauge model assembl\textbf{Y} (\textbf{SYMPHONY} \footnote{Code is available at \url{https://github.com/ZHUWEI-hub/SYMPHONY}}),  a novel multi-agent planning framework that integrates a pool of heterogeneous language model-based agents. 
By leveraging diverse reasoning patterns across agents, SYMPHONY enhances rollout diversity and facilitates more effective exploration.
Empirical results across multiple benchmark tasks show that SYMPHONY achieves strong performance even when instantiated with open-source LLMs deployable on consumer-grade hardware. When enhanced with cloud-based LLMs accessible via API, SYMPHONY demonstrates further improvements, outperforming existing state-of-the-art baselines and underscoring the effectiveness of heterogeneous multi-agent coordination in planning tasks.

\end{abstract}


\section{Introduction}

One of the most fundamental problems in combinatorial optimization is the traveling salesperson problem (TSP), formalized as early as 1832 (c.f. \cite[Ch 1]{ABCC07}).
In an instance of  TSP we are given a set of $n$ cities $V$ along with their pairwise symmetric distances, $c:V\times V \to\R_{\geq 0}$. The goal is to find a Hamiltonian cycle of minimum cost. In the metric TSP problem, which we study here, the distances satisfy the triangle inequality. Therefore, the problem is equivalent to finding a closed Eulerian connected walk of minimum cost.%\footnote{Given such an Eulerian cycle, we can use the triangle inequality to shortcut vertices visited more than once to get a Hamiltonian cycle.}

It is NP-hard to approximate TSP within a factor of $\frac{123}{122}$ \cite{KLS15}.  An algorithm of Christofides-Serdyukov~\cite{Chr76,Ser78} from four decades ago gives a $\frac32$-approximation for TSP.
Over the years there have been numerous attempts to improve the Christofides-Serdyukov algorithm and exciting progress has been made for various special cases of metric TSP, e.g., \cite{OSS11,MS11,Muc12,SV12,HNR21, KKO20, HN19, GLLM21}.
 Recently, ~\cite{KKO21} gave the first improvement for the general case by demonstrating that the so-called ``max entropy" algorithm of \cite{OSS11} gives a randomized $\frac{3}{2}-\epsilon$ approximation for some $\epsilon > 10^{-36}$.% (see \cite{VS20} for a historical note about TSP)

%After a long line of work %~\cite{Wol80,SW90,BP91,Goe95,CV00,GLS05,BM10,BC11,SWV12, HNR17,HN19, KKO20a} 
	%the best known approximation algorithm for the general case of the problem is $\frac{3}{2}-\epsilon$ for some $\epsilon > 10^{-36}$ due to ~\cite{KKO21}, a result that built upon the work of the third author, Saberi, and Singh ~\cite{OSS11}. 
	The method introduced in \cite{KKO21} exploits the optimum solution to the following linear programming relaxation of metric TSP studied by \cite{DFJ59,HK70,BG93}, also known as the subtour elimination LP:
\begin{equation}\label{eq:tsplp}
\begin{aligned}
	\min \quad& \sum_{u,v} x_{\{u,v\}} c(u,v)& \\
	\text{s.t.,} \quad &  \sum_{u} x_{\{u,v\}} = 2&\forall v\in V,\\
	& \sum_{u\in S, v\notin S} x_{\{u,v\}}\geq 2,&\forall S \subsetneq V, S\not= \emptyset\\
	& x_{\{u,v\}}\geq 0 &\forall u,v\in V.
\end{aligned}	
\end{equation} 
	
	 However, ~\cite{KKO21} did not show that the integrality gap of the subtour elimination polytope is bounded below $\frac{3}{2}$, and therefore did not make progress towards the ``4/3 conjecture" which posits that the integrality gap of LP \eqref{eq:tsplp} is $\frac{4}{3}$. In this work we remedy this discrepancy by proving the following theorem, improving upon the bound of $\frac{3}{2}$ from Wolsey~\cite{Wol80} in 1980:

\begin{theorem}\label{thm:main}
	Let $x$ be a solution to LP \eqref{eq:tsplp} for a TSP instance. For some absolute constant $\epsilon > 10^{-36}$, the \hyperlink{tar:alg}{max entropy algorithm} outputs a TSP tour with expected cost at most $\frac{3}{2}-\epsilon$ times the cost of $x$. Therefore the integrality gap of the subtour elimination LP is at most $\frac{3}{2} - \epsilon$. 
\end{theorem} 

To prove \cref{thm:main}, we amend Section 4 of \cite{KKO21} but keep the remainder of the analysis essentially the same. Unlike \cite{KKO21}, this argument now preserves the integrality gap by avoiding the use of the optimum solution in bounding the cost of the matching. See \cref{sec:overview} for a discussion of our new approach.
%We note that the analysis in this paper is not specialized to the max entropy algorithm (although we rely on many results from \cite{KKO21} to obtain \cref{thm:main} itself); instead, it is valid for any algorithm which samples a spanning tree from the support of a solution to LP \eqref{eq:tsplp} and then adds the minimum cost matching on the odd degree vertices of the tree.  
%Instead, we use the polygon representation of near minimum cuts \cite{Ben95,BG08} to bound  the cost of the matching (see the following section for an overview of our new findings). %An added benefit of avoiding the use of OPT in the analysis is  %We remark this makes the analysis constructive 
%We remark that this allows future analyses to explicitly compute and possibly utilize the relevant laminar family of near minimum cuts (whereas previously one needed to know OPT to find the laminar family used in the analysis in \cite{KKO21}).
%In particular, we show that to get a bound better than $\frac{3}{2}$ for this class of algorithm it is (essentially) sufficient to handle the case in which the near minimum cuts of $x$ are a laminar family.

\subsection{Other Consequences}
\paragraph{Path TSP} In recent exciting work, Traub, Vygen, Zenklusen \cite{TVZ20} showed that an $\alpha$-approximation algorithm for metric TSP can be used as a black box to get a $\alpha(1+\eps)$ approximation algorithm for Path TSP. This together with \cite{KKO21} implies that there is a $3/2-\eps$ approximation algorithm for Path TSP (for $\eps>10^{-36}$). On the other hand, it is a folklore result that the integrality gap of the natural LP relaxation of Path TSP is at least $3/2$.  Therefore, a consequence of the above theorem is that although the best possible approximation factors of the two problem are the same (up to polynomial reductions), the natural LP relaxation of metric TSP has a strictly smaller integrality gap.


\paragraph{2-ECSM} In the 2-edge-connected multi-subgraph problem, or 2-ECSM for short, we are given a weighted graph $G$ and we want to find a minimum cost 2-edge-connected spanning subgraph, where an edge can be chosen multiple times.
The classical Christofides-Serdyukov algorithm gives a 3/2-approximation for 2-ECSM and despite significant attempts \cite{CR98,BFS16,SV14,BCCGISW20} improved algorithms were designed only for special cases of the problem.
Since in \cite{BG93} it is shown that LP \eqref{eq:tsplp} is a valid relaxation for 2-ECSM, we obtain:

\begin{corollary}	
For some absolute constant $\epsilon > 10^{-36}$ the \hyperlink{tar:alg}{max entropy algorithm} is a randomized $\frac{3}{2}-\epsilon$ approximation for the 2-edge-connected multi-subgraph problem.
\end{corollary}
%Beyond these theorems, we believe the analysis in this paper will open new avenues to improve the arguments in ~\cite{KKO21}. The analysis in that work is by nature non-constructive because it uses information about the optimal solution. Here we remove this weakness and could in principle construct the proposed fractional matching in polynomial time. Although of course this has no practical benefit since the algorithm always finds the minimum cost matching, this may allow future works to manipulate the algorithm to better serve the analysis.

%We analyze the max-entropy rounding algorithm introduced in \cite{OSS11} and slightly modified in \cite{KKO20, KKO21}. 

%In other words, we design a feasible vector for the $O$-join polytope to ``satisfy'' all near min cuts ``crossed on both  sides'' 


%Whereas Section 4 of ~\cite{KKO21} only deals with the near minimum cuts of $x$ (where $x$ is a solution to LP \eqref{eq:tsplp}) which lie along the optimal Hamiltonian cycle, we deal with all near minimum cuts of $x$ using the so-called polygon representation of near minimum cuts ~\cite{Ben97,BG08}. %The results give new intuition for the structure of cuts that are within $\frac{6}{5}$ or less of the edge connectivity of the graph.

 %: we show that we can incur a cost of $O(\eta^2) \cdot c(x)$ to ensure that the set of cuts with $x(\delta(S)) \le 2+\eta$ is a laminar family.


\subsection{New techniques and contributions}\label{sub:newtechniques}

This paper can be seen as a case study on how to reason about and deal with {\em near} minimum cuts. One can deduce from the classical cactus representation of a graph $G$ \cite{DKL76} (i) the structure of {\em all} min cuts of $G$ and (ii) the structure of the edges of $G$ in the sense that every edge $\{u,v\}$ maps to a unique {\em path} in the cactus between the images of $u$ and $v$. Furthermore, such a path intersects every cycle of the cactus on at most one cactus edge. The theory has found many application from designing fast algorithms
\cite{Kar00,KP09} to the analysis of approximation algorithms for TSP \cite{KKO20} and connectivity augmentation \cite{BGJ20,CTZ21}.

Two decades later, the theory of min cuts was extended to near min cuts in works of Bencz\'ur and Goemans \cite{Ben95, BG08} where they introduced the polygon representation which represents all cuts of a graph with at most $\frac{6}{5}k$ edges, where $k$ is its edge connectivity. Although these works completely classify the structure of all near min cuts of a given graph $G$, they do not characterize the structure of the \textit{edges} of $G$ with respect to these cuts, which can be important in applications (for example, in many of the recent applications of min cuts,
 one also needs to exploit the structure of the edges in relation to the cactus).
The structure on the edges turns out to be highly relevant in this work as well, and as a byproduct of our analysis we make progress towards classifying the way in which the edges of $G$ relate to the structure of the polygon representation.
 
 % and (to some extent) a classification of the set of edges of $G$ with respect to the polygon representation of Bencz\'ur and Goemans.
 
  %i
 %s to give a better understanding of the structure of edges of $G$ with respect to its near min cuts.

  %One can partition the edges of $G$ into sets $F_1\dots,F_m$ such that the set of edges in every min cut $(S,\overline{S})$ of $G$ is the union of edges in a pair $F_i,F_j$ for $i\ neq j$.
%\Nathan{Shayan can add something} For example...

For motivation, consider a generic family of network design problems in which we want to construct a network such that every pair $u,v$ of vertices has connectivity at least $c_{u,v}$. A natural approach is to write an LP relaxation to find a (minimum cost) vector $x: E \to \R_{\ge 0}$ such that for every cut $S$ separating $u$ and $v$, $x(\delta(S))\geq c_{u,v}$. We can round this LP using independent rounding or a dependent rounding scheme such as sampling from max entropy distributions. Using classical concentration bounds one can show that if $x(\delta(S))\gg c_{u,v}$ then with high probability the rounded solution has at least $c_{u,v}$ edges across this cut. So the main challenge is to ``fix'' near tight cuts, i.e., cuts where $x(\delta(S))\approx c_{u,v}$.  For an explicit instantiation of this scheme see \cite{KKOZ22}. A better understanding of the global structure of the family of near tight cuts has the potential to significantly simplify or even improve the approximation factor of such rounding algorithms. A classical technique to design algorithms for such network design problems is to apply uncrossing to extreme point solutions of the LP. One can view our contribution as an approximate uncrossing technique that deals with all near tight cuts (instead of just tight cuts) as we explain next.
%Next, we explain how our main theorem can be used to give global structure for near tight cuts in the case that $c_{u,v}=2$ for all $u,v$ and we contrast it with the classical uncrossing technique which only deals with tight/min cuts. 


\paragraph{An Approximate Uncrossing Technique.} A fundamental technique in the field of approximation algorithms is the uncrossing technique\footnote{See e.g. \cite{LRS11} for a number of applications of this technique.} of Jain \cite{Jai01}. Given a graph $G=(V,E)$,  a weight vector $x:E\to\R_{\geq 0}$, and a  function $f:V\to\R$, suppose that $x(\delta(S))\geq f(S)$ for all $S\subseteq V$. Let $\cN$ be the family of sets $S$ such that $x(\delta(S)) = f(S)$, i.e., the family of {\em tight} sets with respect to $f$. The uncrossing technique says that if $f$ is (weakly) supermodular then we can refine $\cN$ to a laminar family of sets, $\cH$, such that if all sets of $\cH$ are tight, then all sets of $\cN$ are tight as well. For a concrete example, suppose $f$ is a constant function, say $f(S)=2$ for all $\emptyset\subsetneq S\subsetneq V$. Then, sets of $\cH$ can be constructed using the cactus representation \cite{DKL76} of cuts in $\cN$. The significance of this method is that if $x$ is a basic feasible solution to a LP with constraints $x(\delta(S))\geq f(S)$ for all $S$, one can use this machinery to argue that the support of $x$ has size $O(|V|)$.

Informally, we prove the following, which 
can be seen as  an {\em approximate uncrossing technique}: 
\begin{theorem}[Informal]\label{thm:uncrossing}Suppose we have a vector $x:E\to\R_{\geq 0}$ such that $x(\delta(S))\geq f(S)$ for all $S$; define $\cN$ to be sets $S$ where $x(\delta(S))\leq f(S)(1+\eps)$ for some fixed $\eps>0$. If $f(.)$ is constant, say $f(S)=2$ for all $S$, then there is a set $\cN^*\subseteq \cN$ and a collection of edge sets $F_1,\dots,F_m\subseteq E$ such that the following hold:
\begin{itemize}
	\item $|\cN^*|= O(|V|)$, $m= O(|V|)$.
	\item $x(F_i)\geq 1-\eps/2$ for all $1\leq i\leq m$.
	\item Every edge $e$ is in at most $O(1)$ of the $F_i$'s.
	\item For every set $S\in \cN\smallsetminus \cN^*$ there exists $1\leq i<j\leq m$ such that $F_i\cap F_j=\emptyset$ and $F_i\cup F_j\subseteq \delta(S)$ and for every $S\in \cN^*$, there exists $1\leq i\leq m$ such that $F_i\subseteq \delta(S)$. 
\end{itemize}
\end{theorem}
In words, although we cannot simply refine $\cN$ to a linear number of sets, we can refine the edges in cuts of $\cN$ to a linear number of sets $F_1,\dots, F_m$ such  that we can essentially capture the edges of $\delta(S)$ for any $S\in \cN\smallsetminus \cN^*$ by a pair of disjoint $F_i$'s. We give a slightly weaker condition for cuts in $\cN^*$; namely we only capture half of their edges by $F_i$'s.

\begin{example}For a simple example of the above theorem, suppose $\eps=0$, i.e. $\cN$ is the set of min cuts of a graph $G$. Furthermore, suppose that every proper  cut in $\cN$ is \hyperlink{tar:crossing}{crossed} (recall that $S$ is proper if $1<|S|<|V|-1$) and that $\cN$ has at least one proper cut. 
Then, one can use an uncrossing technique, namely that if $A,B\in \cN$ then $A\cap B\in \cN$, to prove that $G$ must be cycle, namely we can order vertices of $G$, $v_0,\dots,v_{n-1}$ such that $x_{\{v_i,v_{i+1\text{ mod n}}\}}=1$.
In such a case we let $\cN^*=\emptyset$ and $F_i=E(v_i,v_{i+1\text{ mod }n})$.
%partition $V$ into sets $a_0,\dots,a_{m-1}$ such that 
%Let $\C$ be a connected component of crossing cuts of $\cN$, namely, for any pair of sets $A,B\in \C$ there is a path of crossing cuts all from $\C$ that goes from $A$ to $B$.
% and further suppose that $\cN$ can be represented by a cycle $C$ in the sense every min cut of $\cN$ corresponds to a min cut of $C$ and vice versa. Here we assume $a_0,\dots,a_{m-1}$ are the nodes of $C$ where each $a_i$ is identified with a disjoint set of vertices where $V=\uplus_{i=1}^m a_i$. In such a case, we can simply let $\cN^*=\emptyset$ and $F_i=E(a_i,a_{i+1\text{ mod }m})$. 
\label{eg:cycle}\end{example}

\begin{example}\label{eg:laminar}
For a second example, suppose again $\eps=0$ and $\cN$ is the set of mincuts of a graph $G$ where $\cN$ forms a laminar family (no two cuts cross). It turns out that we cannot decompose edges of cuts of $\cN$ into a linear sized collection of sets where every edge appears only a constant number of times. The main reason is that some edges may appear in an unbounded number of cuts. In this case we let $\cN^*=\cN$ and for every $A\in \cN$ (with immediate parent $B\in \cN$ in the laminar family) we add a set $F_A=\delta(A)\smallsetminus \delta(B)$  to our collection.  It is straightforward to show, using the structure of min cuts, that $x(F_A)\geq 1$; furthermore, since the size of a laminar family is linear in $V$, this gives a valid decomposition in the sense of above theorem.
\end{example}
Lastly, if $\eps=0$ and $\cN$ is the set of min cuts of an arbitrary graph, one can represent all min cuts of $\cN$ by a cactus \cite{DKL76} which can be seen as a tree of cycles. In such a case, one can use a construction similar to \cref{eg:cycle} for each cycle where instead of a vertex $v_i$ we have a set $a_i \subseteq V$ and one similar to \cref{eg:laminar} for the tree part of the cactus. For a concrete application of such a decomposition of min cuts see \cite{KKO20}.
%More generally, if $\cN$ corresponds to the set of min cuts of an arbitrary graph, the cuts of $\cN$ can be represented by a {\em cactus graph}. In such a case we add one $F_i$ for every edge of a cycle of the cactus. 


%and further for simplicity assume that there is a single connected component of crossing cuts in $\cN$, namely we can go from any $A$ to $B$ for $A,B\in\cN$ simply following crossing cuts of $\cN$. Then, one can represent cuts in $\cN$ by the set of min cuts of a cycle, namely we can contract vertices of $G$ 

%For a concrete application , suppose we need at least two edges in every set in $\cN^*$, say in a network optimization problem. Then, if we make sure that we have at least one edge in each $F_i$, all typical cuts constraints, $\cN\smallsetminus \cN^*$,  are satisfied, so we  reduce the problem to cuts in $\cN^*$. 


One of the main challenges in dealing with near min cuts relative to min cuts is that if $x(\delta(A)),x(\delta(B))\leq 2+\eps$ then $x(\delta(A\cap B))\leq 2+2\eps$. Therefore, if $\eps=0$, then min cuts are closed under intersection, set difference and union, but this is no longer true when $\eps>0$. So, to employ the classical uncrossing machinery one should be very careful to "uncross" only a constant number of times (independent of $\eps$) to make sure that every cut remains within $2+O(\eps)$. This is the main reason that the polygon representation of near min cuts (see below) is more sophisticated, e.g., we can no longer argue $x(E(a_i, a_{i+1}))\approx 1$, see \cref{fig:nearmincutbadexample}.

Although we don't study it here, we believe it may be worthwhile to find generalizations of \cref{thm:uncrossing} which hold for any (weakly) supermodular function.% That could be helpful in many questions based on the network optimization framework of Jain \cite{Jai01}.

\begin{remark} 
 We do not explicitly prove \cref{thm:uncrossing} in this extended abstract, as it is not used to prove \cref{thm:main}. However it can be deduced from arguments in \cref{sec:twoside} and \cref{app:oneside}. 
%In \cref{sec:overview} we discuss the main ideas of the proof of \cref{thm:uncrossing}. Here, let us explain the main challenge: In principal one might try to simply extend the above decomposition for the case $\eps=0$. The main challenge is that if $x(\delta(A)),x(\delta(B))\leq 2+\eps$ then $x(\delta(A\cap B))\leq 2+2\eps$. Therefore, if $\eps=0$, then min cuts are closed under intersection, set difference and union, but this is no longer true when $\eps>0$. So, to employ the classical uncrossing machinery one should be very careful to "uncross" only a constant number of times (independent of $\eps$) to make sure that every cut remains within $2+O(\eps)$. This is the main reason that the polygon representation of near min cuts (see below) is more sophisticated, e.g., we can no longer argue $x(E(a_i, a_{i+1}))\approx 1$, see \cref{fig:nearmincutbadexample}.
\end{remark}





\paragraph{Extensions to the Polygon Representation} To obtain our uncrossing framework we prove new properties of the polygon representation.
Given a graph $G=(V,E)$, let $k$ be the edge-connectivity of $G$, i.e. the number of edges in a minimum cut of $G$. For $\eps>0$, consider the set of $(1+\eps)$-near minimum cuts of $G$: cuts $(S,\overline{S})$ where $|E(S,\overline{S})| < (1+\eps)k$.
Bencz\'ur \cite{Ben95} and Bencz\'ur, Goemans \cite{BG08} proved that if $\eps \le 1/5$ then the near minimum cuts of $G$ admit a {\em polygon representation}. Namely, every connected component $\cC$ of \hyperlink{tar:crossing}{crossing} $(1+\eps)$ near min cuts can be represented by the diagonals of a convex polygon. In this polygon, the vertices of $G$ are partitioned into sets called \textit{atoms}, and every atom is mapped to a cell of this polygon defined by the diagonals and the boundary of the polygon itself (see \cref{sec:polyrep} for more details). 

The polygon representation can be seen as a generalization of the well-known cactus representation \cite{DKL76} of minimum cuts where a cycle of the cactus is replaced by a convex polygon. Unlike a cycle, some vertices/atoms map to the interior of the polygon, which are called ``inside'' atoms. The inside atoms at first look like a mystery and one can ask many questions about them such as how many can exist and what structures they can exhibit.



 Here, we explain two lemmas we proved which might find further applications beyond TSP in the future. 
%Our results give new intuition and understanding about the structure of polygon representations. These guide our analysis of the integrality gap of the subtour LP.
 %For example, one of our new observations is a 
 First, we give a necessary condition for a cell of a polygon to contain an inside atom:
\begin{lemma}[Informal, see \cref{thm:halfplanes}]
	Consider a polygon $P$ for a connected component $\C$ of a family of $1+\eps$ near min cuts for $\eps \le 1/5$ (where representing diagonals correspond to cuts in $\C$). Any cell of $P$ that has an inside atom must have at least $\Omega(1/\eps)$ many sides. 
\end{lemma}
This can be seen as a generalization of \cite[Lem 22]{BG08} to the case in which the cell is allowed to be adjacent to vertices of the polygon $P$.

Now, we explain our second extension: it follows from the cactus representation of minimum cuts that for a graph $G$ and a min cut $S$ one can partition the set of all min cuts that cross $S$ into two groups ${\cal A}=\{A_1,\dots,A_k\}$ and ${\cal B}=\{B_1,\dots,B_l\}$ for some $k,l\geq 0$ such that $S\cap A_1\subseteq S\cap A_2 \subseteq \dots S\cap A_k$ and, similarly, $S\cap B_1\subseteq \dots\subseteq S\cap B_l$. We prove a generalization of this fact for near min cuts:
\begin{lemma}[Informal, see \cref{lem:crosschain}]
Consider the set of $1+\eps$ near min cuts of a graph $G$ for $\eps\leq 1/10$; for any such near min cut $S$, one can partition the $1+\eps$ near min cuts crossing $S$ into two groups ${\cal A}=\{A_1,\dots,A_k\}$ and ${\cal B}=\{B_1,\dots,B_l\}$ such that $S\cap A_1 \subseteq S\cap A_2\subseteq \dots \subseteq S\cap A_k$ and similarly for cuts in ${\cal B}$.
\end{lemma}

\subsection{Outline of rest of paper} After reviewing preliminaries in \cref{sec:prelims}, we give a high-level overview of our proof technique in \cref{sec:overview}. The main new technical contributions of this paper are in \cref{sec:polyrep} and  \cref{sec:twoside}. The remaining content of the paper essentially follows from ~\cite{KKO21}. %Therefore, the reader may want to skip \cref{sec:proof-of-main}. 







Distributed systems have been maintaining their importance for the last several decades due to the increase in the need for scalable and reliable distributed applications while preserving high performance. 
To analyze distributed systems comprehensively and compare them in terms of features and services, various surveys and evaluations have been published in the past. Surveys on cloud providers, data warehouses, distributed file systems, or metadata services can be counted among them. 

Cloud providers are analyzed and evaluated in terms of elasticity \cite{CMART}, computing power \cite{comperative-benchmarking}, and cost to performance efficiency \cite{fair-benchmarking} in previous efforts. Widely used distributed services are also analyzed in many works, such as a survey on stream processing \cite{stream-benchmarking} or performance and dependability evaluation of MapReduce systems \cite{MapReduce-benchmarking}. Similarly, different aspects of distributed systems are studied in several surveys, like reliability analysis on distributed systems \cite{reliability-survey} and load balancing characteristics of known systems \cite{load-balancing-survey}.

As a big part of distributed systems, data warehouses and file systems are studied for many specifications. Evaluation of distributed data warehouses for the cost-effectiveness of different hardware configurations \cite{ALOJA} and query performance of distinct design choices\cite{benchmarking-data-warehouse} are among the known efforts in these works. Distributed file systems are examined in many past works for general concepts \cite{file-systems-concepts,file-systems-gen1} or specific applications such as distributed access control \cite{access-control-file-systems}. Due to the differences in optimization, design techniques, and the complex interactions between the file systems and other system components like the kernel or operating system, benchmarking distributed file systems is not trivial. To identify the important metrics for the evaluation of distributed file systems, researchers also studied benchmarking file systems \cite{File-system-benchmarking,benchmarking-file-rocket}. 

Analysis of distributed coordination services in terms of general characteristics and importance of coordination \cite{importance-of-coordination} and the comparison of existing algorithms \cite{paxos-made-simple} are among the published works. However, to the best of our knowledge, there is no published work on the evaluation of distributed coordination systems. As mentioned in the Introduction, due to the lack of standard benchmarking tools for distributed coordination services, developers widely use their ad-hoc benchmarks, which are prone to unfair comparisons or limited results for the evaluation of the systems. This study is unique in identifying the metrics and parameters for the evaluation of distributed coordination systems, discussing how each system uses these metrics and parameters for its evaluation, pinpointing the deficiencies of well-known benchmarking suites in evaluating distributed computing systems, and finally discussing the features of an ideal distributed coordination benchmark. 

% \input{preliminary}

% 
\begin{figure*}[t]%% placement specifier
%% Use \includegraphics command to insert graphic files. Place graphics files in 
%% working directory.
\centering%% For centre alignment of image.
\includegraphics[width=0.8\textwidth]{Paper_Figures/Figure1.png}
%% Use \caption command for figure caption and label.
\caption{LLM and digital twin enhanced dynamic robot task allocation}\label{fig:framework}
%% https://en.wikibooks.org/wiki/LaTeX/Importing_Graphics#Importing_external_graphics
\end{figure*}


\section{Methodology}
The methodology developed in this study is illustrated in Fig. \ref{fig:framework}. It introduces a scalable framework for situation-aware decision-making in multi-robot task allocation, enabled by an LLM. The system is capable to operate with or without an initial digital model (e.g., BIM) and has a closed-loop feedback mechanism that synchronizes the physical site with its digital representation.

The Digital Twin serves as a bridge for synchronization between the physical site and the digital model. The digital model integrates: (1) a list of construction tasks and their precedence relationship that can either be presented in the form of a structured list or be automatically extracted from the BIM; (2) site information (i.e., material supplies and locations), which can either be manually created or detected by robots on the physical site and transmitted to the digital model; and, (3) building information that includes object geometries, types, and poses. It includes the completed structures as well as placeholders for components that have not yet been constructed. Each component is associated with attribute information, including its name (reference ID), layer (unbuilt, as-built, materials, inactive), material type, installation pose, and gripping pose. Based on this information, a comprehensive task list that contains the completion status can be extracted. It covers all active tasks (i.e., tasks that are scheduled to be performed in this project stage), including both the ones that are completed and the pending ones waiting to be assigned and executed.

When an update event is triggered, the robots extract the list of tasks to complete and their logic relationships from the digital model. Updates to this list may originate from the physical site, where task completion status and site changes are detected and transmitted via ROS. Conversely, new or modified task instructions are synchronized back to the digital twin. Human supervision plays a critical role in interpreting construction progress and intervening when unforeseen uncertainties occur, such as weather disruptions, equipment failures, or site access limitations. These decisions and environmental conditions are reflected in the digital twin and influence task allocation strategies.

The central processor coordinates the intelligent decision-making layer. Here, an LLM interprets contextual inputs (e.g., natural language updates from human operators), supported by a repository of domain knowledge. Rather than solving the optimization problem directly, the LLM identifies relevant parameters and updates constraints within a pre-defined IP model used for task allocation. The optimized task allocation plan is then sent to the robot controllers, which execute the tasks accordingly. Completion data flows back to the digital twin and task model, ensuring continuous updates and alignment between the virtual and physical environments.




\subsection{Digital twin system}

As a critical component of the proposed framework, the digital twin system integrates the digital model and the system central processor to continuously monitor and manage the project's status.  It also functions as the user interface, connecting human users with  system backends and enabling them to visualize, supervise, and intervene in the project using natural language via keyboard or voice input. The digital twin state at time $t$ is formalized as:
\begin{align}
    S_t = D_0+\mathcal{F}(S_{t-1}, R_t, SI_t, \Delta_{task}, \Delta_{robot})
    \end{align}
Here, $S_t$ represents the current state of the digital twin at time $t$, $D_0$ denote the initial information from the digital model $D$, $\mathcal{F}$ is the state update function, $R_t$ and $SI_t$ denote the robot states (i.e., arm joint states and base locations and orientations) and site information at time $t$,  respectively, and $\Delta_{task}$ and $\Delta_{robot}$  capture updates to task statuses and robot statuses (i.e., high-level task each robot in the team is performing) at time $t$, respectively.  Fig. \ref{fig:dt} illustrates the system architecture and information flow of the technical implementation. The three core capabilities, visualization, supervision, and intervention, are supported by four key modules, including the synchronized visualization module, task status tracker, robot status tracker, and user command receiver. 



\textbf{Visualization} ($D_0, R_t, SI_t \rightarrow S_t$): The synchronized visualization module, adapted from the authors' previous work \cite{wang2024enabling}, is responsible for continuously updating the visual representation of the site and robots in the digital twin based on $SI_t$ and $R_t$. Before the process starts, robot emulators are generated by transmitting the URDF models of the robots and associated mesh files from ROS to the digital twin. In the initial state $S_0$, the basic digital twin scene is generated from $D_0$, which includes both geometric and attribute information. In this step, component geometries received from $D$ are instantiated at the corresponding locations in the digital twin as the environment. Attribute information embedded in $D$ is transmitted to the digital twin to adjust the properties of the corresponding components, such as its name, layer, attachment offset, and visualization (e.g., color).  During the runtime, the robot emulators mirror on-site robots' movements by subscribing to their state data $R_t$, thereby reflecting synchronized robot states to human users. $SI_t$ is similarly integrated through subscription to corresponding ROS topics.

\textbf{Supervision} $(D_0,\Delta_{task}, \Delta_{robot}\rightarrow S_t)$: The supervision function allows the users to quickly get an oversight of construction progress and multi-robot operations to facilitate intervention decision making, which is enabled by the robot status tracker and task status tracker. At $t=0$, the task status tracker extracts from $D$ a comprehensive list of tasks involved. According to the progress, each task is marked as uninitiated, ongoing, or completed. As construction progresses, these labels are updated according to $\Delta_{task}$. Meanwhile, the robot status tracker provides the operation status of each robot in the teams in the form of high-level task name or description. If no task is being performed, the status of the corresponding robot will be marked as "idle".

\textbf{Intervention}: The user command receiver module supports user interventions when users make intervention decisions during the visualization and supervision process or uncertainties occur. In these cases, users can submit high-level instructions in natural language by keyboard typing or directly through voice. The commands are processed by the embedded LLM into actionable modifications. These modifications are then forwarded as valid inputs to the task allocation module, updating $\Delta_{task}$ and triggering state updates via $\mathcal{F}$.

After the completion of tasks or upon reaching a project milestone, the updated task list and building information are transmitted to update the digital model. This information includes components installed during construction and changes in site conditions, such as leftover construction materials on-site. As a result, the digital model maintains an up-to-date record and accurate representation of the project. This closed-loop feedback mechanism ensures that the digital model accurately reflects the project's as-built condition. It also enhances the model's utility in subsequent phases of work, such as facility management, progress auditing, and future renovations.

\begin{figure}[t]

\centering
\includegraphics[width=0.48\textwidth]{Paper_Figures/DT_Framework.png}

\caption{Digital twin system framework}\label{fig:dt}

\end{figure}


\subsection{Multi-robot task allocation: plan creation}
% \bofutodo{ Integer Programming algorithm theory) Done}

This section formally defines the task allocation and scheduling problem, formulates it as an integer program, and describes the algorithms used to solve it.

Suppose there are \(n_\Acal\) capabilities, denoted by \(\Acal = \{1, \ldots, n_\Acal\}\), \(n_\Rcal\) heterogeneous robots, denoted by \(\Rcal = \{1, \ldots, n_\Rcal\}\), and \(n_\Tcal\) tasks, denoted by \(\Tcal = \{1, \ldots, n_\Tcal\}\). Each robot has a subset of capabilities, and each task requires a team of robots with the required capabilities to serve it. When a team of qualified robots \(\Rcal_i \subset \Rcal\) arrive, the task \(i \in \Tcal\) can be completed after time duration \(T^d_{i}\). A robot may only serve one task at a time. Additionally, each task may have a set of dependencies \(\Tcal_i \subset \Tcal\), i.e., other tasks that must be completed beforehand.
The goal is to schedule tasks on a set of available robots in a way that minimizes the makespan, the time by which all tasks have been completed.

We formulate the above-mentioned task allocation and scheduling problem as an integer program. Here, we provide common notations in Table \ref{tab:variable_definition}. The decision variables are \(x_{ir}\) for task assignment and \(t^s_{ir}\), \(t^e_{ir}\), \(t^s_i\), and \(t^e_i\) for scheduling.

\begin{table}[t]
  \caption{Definition of the notation.}
  \label{tab:variable_definition}%
    \begin{tabular}{p{0.06\linewidth}|p{0.82\linewidth}} 
    \toprule
     & Meaning
    \\
    \midrule
    \(x_{ir}\) & = 1 if task \(i \in \Tcal\) is assigned to robot \(r \in \Rcal\).
    \\
    \(t^s_{ir}\) & The start time if robot \(r \in \Rcal\) works on task \(i \in \Tcal\).
    \\
    \(t^e_{ir}\) & The end time if robot \(r \in \Rcal\) works on task \(i \in \Tcal\).
    \\
    \(t^s_i\) & The start time of task \(i \in \Tcal\).
    \\
    \(t^e_i\) & The end time of task \(i \in \Tcal\).
    \\
    \(y_{ijr}\) & Auxiliary variables for robot scheduling.
    \\
    \(y_{ij}\) & Auxiliary variables for task dependencies.
    \\
    \(T^D_{i}\) & The time duration to complete task \(i \in \Tcal\).
    \\
    \(T^s_i\) & The earliest start time for task \(i \in \Tcal\).
    \\
    \(T^e_i\) & The latest end time for task \(i \in \Tcal\).
    \\
    \(T_{\text{large}}\) & A large time constant.
    \\
    \(a_{kr}\) & The amount of capability \(k \in \Acal\) available on robot \(r \in \Rcal\).
    \\
    \(b_{ki}\) & The amount of capability \(k \in \Acal\) required to execute task \(i \in \Tcal\)
    \\
    \bottomrule
    \end{tabular}
\end{table}

\textbf{Objective function:}
The objective function jointly minimizes the makespan and the individual task completion times. The third piece in the objective function penalizes the number of robots assigned to tasks. The makespan is defined as the maximum end time among all tasks. We set \(C_m \gg C_s \approx C_r\) to ensure that the makespan is the primary objective.
\begin{align}
    % \min_{x_{ir}, \ t^s_{ir}, \ t^e_{ir}, \ t^s_i, \ t^e_i} 
    \min_{\substack{x_{ir}, \ t^s_i, \ t^e_i, \\ t^s_{ir}, \ t^e_{ir}}}\ ( C_m \max_{i \in \Tcal} \ t^e_i + C_s \sum_{i \in \Tcal} \ t^e_i + C_r \sum_{i \in \Tcal} \sum_{r \in \Rcal} x_{ir} ) \label{eqn:objective}
\end{align}

\textbf{Variable bound constraints:}
\(x_{ir}\) is a binary task allocation variable, equal to 1 if task \(i \in \Tcal\) is assigned to a team that contains robot \(r \in \Rcal\), and 0 otherwise. The scheduling variables are continuous, with the requirement that the end time is always larger the start time.
\begin{align}
    x_{ir} \in \{0, 1\}, \quad &\forall i \in \Tcal, \forall r \in \Rcal \label{eqn:assignment_var} \\
    0 \leq t^s_i \leq t^d_i, \quad &\forall i \in \Tcal \label{eqn:task_time_var} \\
    0 \leq t^s_{ir} \leq t^d_{ir}, \quad &\forall i \in \Tcal, \forall r \in \Rcal \label{eqn:robot_time_var}
\end{align}

\textbf{Task dependency constraints:}
The task dependency is specified using the time variables. If task \(i\) depends on the completion of task \(j\), then task \(i\) can start only after task \(j\) is completed.
\begin{align}
    t^s_i \geq t^e_j, \quad &\forall j \in \Tcal_i, \forall i \in \Tcal \label{eqn:task_dependency}
\end{align}

\textbf{Task requirement constraints:}
A capability model is introduced to define the task requirement constraints.
Let \(a_{kr} \in \nonnegativerealset\) denote the amount of capability \(k \in \Acal\) available on robot \(r \in \Rcal\), and let \(b_{ki} \in \nonnegativerealset\) denote the amount of capability \(k \in \Acal\) required to execute task \(i \in \Tcal\).
% Each task must be assigned to exactly one robot, as specified in \eqref{eqn:task_assign_to_one_robot}.
The robot team assigned to a task must collectively possess all the capabilities required by that task, as enforced by \eqref{eqn:task_capability_requirement}.
\begin{align}
    % \sum_{r \in R} x_{ir} = 1, \quad & \forall i \in \Tcal \label{eqn:task_assign_to_one_robot} \\
    \sum_{r \in \Rcal} {a_{kr} x_{ir}} \geq b_{ki}, \quad &\forall k \in \Acal, \forall i \in \Tcal \label{eqn:task_capability_requirement}
\end{align}

\textbf{Task schedule constraints:}
Equation \eqref{eqn:task_duration} specifies the relation between the task start and end time.
The constraints \eqref{eqn:task_start_leq_robot_start}-\eqref{eqn:task_end_geq_robot_end} are organized into two groups. The first group, \eqref{eqn:task_start_leq_robot_start}-\eqref{eqn:task_start_geq_robot_start}, ensures that when robot \(r \in \Rcal\) is in the team assigned to task \(i \in \Tcal\), the task start time \(t^s_i\) equals the robot-specific start time \(t^s_{ir}\). Similarly, the second group, \eqref{eqn:task_end_leq_robot_end}-\eqref{eqn:task_end_geq_robot_end}, guarantees that the task end time \(t^e_i\) matches \(t^e_{ir}\) under the same assignment.
% The third group, \eqref{eqn:task_duration_leq}-\eqref{eqn:task_duration_geq}, enforces that the task duration is given by \(t^e_{ir} = t^s_{ir} + T^D_{i}\) when robot \(r\) is in the team for task \(i\).
Note that \(T_{\text{large}}\) is a large time constant.
\begin{align}
    t^e_i = t^s_i + T^D, \quad & \forall i \in \Tcal \label{eqn:task_duration} \\
    t^s_i \leq t^s_{ir} + T_{\text{large}} (1 - x_{ir}), \quad & \forall r \in \Rcal, \forall i \in \Tcal \label{eqn:task_start_leq_robot_start} \\
    t^s_i \geq t^s_{ir} - T_{\text{large}} (1 - x_{ir}), \quad & \forall r \in \Rcal, \forall i \in \Tcal \label{eqn:task_start_geq_robot_start} \\
    t^e_i \leq t^e_{ir} + T_{\text{large}} (1 - x_{ir}), \quad & \forall r \in \Rcal, \forall i \in \Tcal \label{eqn:task_end_leq_robot_end} \\
    t^e_i \geq t^e_{ir} - T_{\text{large}} (1 - x_{ir}), \quad & \forall r \in \Rcal, \forall i \in \Tcal \label{eqn:task_end_geq_robot_end}
    % t^e_{ir} \leq t^s_{ir} + T^D_{i} + T_{\text{large}} (1 - x_{ir}), \quad & \forall r \in \Rcal, \forall i \in \Tcal \label{eqn:task_duration_leq}  \\
    % t^e_{ir} \geq t^s_{ir} + T^D_{i} - T_{\text{large}} (1 - x_{ir}), \quad & \forall r \in \Rcal, \forall i \in \Tcal \label{eqn:task_duration_geq}
\end{align}

\textbf{Robot schedule (no-overlap) constraints:}
Each robot can perform at most one task at a time, so their task schedules must not overlap, as specified in \eqref{eqn:robot_schedule1}-\eqref{eqn:robot_schedule2}. When the auxiliary variable \(y_{ijr} = 1\), task \(i\) is scheduled before task \(j\); otherwise, task \(i\) is scheduled after task \(j\).
\begin{align}
    t^e_{ir} &\leq t^s_{jr} + T_{\text{large}} (1 - y_{ijr}), &\quad \forall i < j \in \Tcal, \forall r \in \Rcal \label{eqn:robot_schedule1} \\
    t^e_{jr} &\leq t^s_{ir} + T_{\text{large}} \ y_{ijr}, &\quad \forall i < j \in \Tcal, \forall r \in \Rcal \label{eqn:robot_schedule2} \\
    y_{ijr} &\in \{0, 1\}, &\quad \forall i < j \in \Tcal, \forall r \in \Rcal 
\end{align}

% \begin{align}
%     t^e_{ir} \leq t^s_{jr} \text{ or }  t^e_{jr} \leq t^s_{ir}, \forall i, j \in \Tcal, \ \text{s.t.} \ i \neq j, \forall r \in \Rcal \label{eqn:robot_schedue}
% \end{align}

% \begin{align}
%     \phi(t^s_{ir}, t^e_{ir}, t^s_{jr}, t^e_{jr}) \neq 0, \quad \forall i, j \in \Tcal, \ \text{s.t.} \ i \neq j, \forall r \in \Rcal \label{eqn:robot_schedue} \\
%     \phi(t^s_{ir}, t^e_{ir}, t^s_{jr}, t^e_{jr}) = \begin{cases}
%          \ \ \ 1, \quad t^s_{ir} \leq t^e_{ir} \leq t^s_{jr} \leq t^e_{jr} \\
%         -1, \quad t^s_{jr} \leq t^e_{jr} \leq t^s_{ir} \leq t^e_{ir} \\
%          \ \ \ 0, \quad \text{Otherwise} \label{eqn:robot_schedue_phi}
%     \end{cases}
% \end{align}

\textbf{Time window constraints (optional):}
The time window constraint is an optional condition that requires a task to be completed within a specified time interval.
\begin{align}
    T^s_i \leq t^s_i \leq t^e_i \leq T^e_i, \quad \forall i \in \Tcal \text{ with a time constraint} \label{eqn:task_time_window}
\end{align}

\textbf{Task conflict constraints (optional):}
The task conflict constraints are optional and ensure that two tasks \(i\) and \(j\) with conflicting resource requirements are not executed concurrently. \(y_{ij}\) is a binary auxiliary variable.
\begin{align}
    t^e_{i} &\leq t^s_{j} + T_{\text{large}} (1 - y_{ij}), &\forall i < j \in \Tcal \ \text{with conflicts} \label{eqn:task_conflict1}\\
    t^e_{j} &\leq t^s_{i} + T_{\text{large}} \ y_{ij}, &\forall i < j \in \Tcal \ \text{with conflicts} \label{eqn:task_conflict2} \\
    y_{ij} &\in \{0, 1\}, &\forall i < j \in \Tcal \ \text{with conflicts}\label{eqn:task_conflict3}
\end{align}


% \begin{align}
%     \phi(t^s_i, t^e_i, t^s_j, t^e_j) \neq 0, \quad \forall i, j \in \Tcal, \ \text{s.t.} \ i \text{ conflicts with } j \label{eqn:task_conflict} \\
%     \phi(t^s_i, t^e_i, t^s_j, t^e_j) = \begin{cases}
%          \ \ \ 1, \quad t^s_i \leq t^e_i \leq t^s_j \leq t^e_j \\
%         -1, \quad t^s_j \leq t^e_j \leq t^s_i \leq t^e_i \\
%          \ \ \ 0, \quad \text{Otherwise}
%     \end{cases}
% \end{align}

The above integer program can be solved using a CP-SAT solver. A CP-SAT solver is an optimization engine that combines constraint programming (CP) techniques with Boolean satisfiability (SAT) solving methods. It is designed to efficiently tackle combinatorial optimization problems - such as scheduling, planning, and assignment tasks - by systematically exploring potential solutions while adhering to a set of constraints. In our work, we employ the CP-SAT solver provided by Google OR-Tools to efficiently solve the integer program.

\renewcommand{\arraystretch}{1.3} % Adjust row height

\begin{table*}[h!]
\footnotesize
\setlength{\tabcolsep}{5pt}
\centering
\caption{Types of constraint and parameter changes}
\begin{tabular}{p{3cm}|p{6cm}|p{0.7cm}|p{6.8cm}}
    \toprule % Top line added
    \textbf{Constraint Type} & \textbf{Definition} & \textbf{Label} & \textbf{Parameter} \\
    \midrule
    Task Dependency & Adjustments in the sequence or prerequisite relationships among tasks & \textbf{1} & \texttt{[task\_id, successors, +/-]} \\

    Task Duration & Variations in the estimated time to complete tasks & \textbf{2} & \texttt{[task\_id, new\_duration]} \\

    Task Starting Time & Changes to tasks' earliest or planned start times & \textbf{3} & \texttt{[task\_id, start\_time\_change]} \\

    Number of Robot & Variations in robot availability & \textbf{4} & \texttt{[new\_robot\_type\_id, robot\_number\_change]} \\

    Task Conflict Constraints & Some tasks cannot be performed at the same time & \textbf{5} & \texttt{[task\_id1, task\_id2]} \\
    \bottomrule % Bottom line added
\end{tabular}
\label{tab:constraint_changes}
\end{table*}

\subsection{LLM-driven adaptive decision-making}

To enable adaptive task reallocation in response to dynamic site conditions, we introduce a formal mapping that captures how the LLM interprets natural language narratives and transforms them into actionable modifications to the task allocation problem. This process serves as the core of the narrative-driven adaptation mechanism, facilitating human-in-the-loop flexibility without requiring direct manipulation of optimization code.

Fig. \ref{fig:modular_system} illustrates the pipeline for enabling adaptive multi-robot task allocation driven by natural language inputs. The process begins with a narrative containing dynamic project updates, such as task sequencing preferences, resource delays, or timing adjustments. An LLM processes this narrative to extract actionable information, identifying relevant task entities (e.g., painting, window installation, wall-drilling) and interpreting relationships such as precedence constraints or temporal shifts. To ensure accurate task reference, a structured database is used to map textual descriptions to corresponding task IDs (e.g., wall-drilling → T6).

The extracted instructions are then categorized into discrete flag types, each representing a specific kind of modification as demonstrated in Table \ref{tab:constraint_changes}. Each flag is paired with a structured parameter representation, which encapsulates the necessary information to modify the optimization code. These flags are designed to align with designated insertion points in a standalone optimization codebase.

Let the input to the LLM be a narrative description
$\mathcal{N}$, such as a sentence or paragraph provided by a site supervisor or worker, which describes changes in site conditions, task statuses, or scheduling preferences. The LLM acts as a function:


\begin{align}
\mathcal{M}: \mathcal{N} \rightarrow\left\{\left(C_k, \theta_k\right)\right\}_{k=1}^K
\end{align}

Where:

$\mathcal{M}$ is the mapping function learned or encoded by the LLM


$C_k$ is the type of constraint (e.g., dependency, time shift)

$\theta_k$ is the parameter set for the update
\\

Each $\left(C_k, \theta_k\right)$ pair defines a flagged constraint to be dynamically injected into the optimization model. These flags are associated with predefined templates in the integer programming codebase, allowing seamless integration with the solver without altering the base model structure. For example:

\begin{itemize}
    \item A narrative like "Our skilled wall-drilling worker will be arriving an hour late" may yield $\left(C= 3, \theta=\{T6, 1\}\right)$.

    
    \item A command like "The owner has requested that painting be completed before window installation" maps to $\left(C= 1, \theta=\{T13 \prec T 9\}\right)$.
\end{itemize}





To ensure consistency, the LLM references a structured task knowledge base $\mathcal{T}=$ $\left\{\left(T_i\right.\right.$, Description$\left.\left._i\right)\right\}$ that maps textual task names to task IDs. This helps avoid ambiguity and ensures each extracted constraint can be accurately linked to the model.





The LLM output is then parsed into a structured JSON format and passed to the optimization backend, which updates the constraint set accordingly:

\begin{align}
\mathcal{C}_{\text {new }}=\mathcal{C}_{\text {original }} \cup\left\{C_k\left(\theta_k\right)\right\}
\end{align}


This interaction between the LLM and the optimizer ensures the system can continuously adapt to evolving conditions while maintaining model transparency and interpretability.

Please note that the optimization algorithm operates independently from the LLM. The LLM's role is limited to interpreting the narrative and modifying only the flagged portions of the code, leaving the optimization logic itself untouched. This separation ensures that the optimization process remains robust and interpretable while gaining the flexibility through natural language instructions. The result is a hybrid, human-in-the-loop system that bridges narrative reasoning and formal optimization in a modular, scalable manner.


\begin{figure}[t]

\centering
\includegraphics[width=0.48\textwidth]{Paper_Figures/Figure3.png}

\caption{Pipeline for narrative-driven adaptive schedule optimization}\label{fig:modular_system}
\end{figure}





\subsection{LLM implementation and prompt design}
Prompt design is critical for LLMs to perform high-quality information extraction, particularly given the intricate nature of construction scheduling tasks \cite{white2023prompt, atreja2024prompt}. To create effective prompts tailored for construction project scheduling, we utilize prompt engineering strategies articulated by White et al. (2023) \cite{white2023prompt}. These strategies emphasize the necessity of integrating context descriptions, structured templates, and explicit output formatting into prompts. In our scenario, context descriptions detail construction tasks, their interdependencies, durations, and associated robotic capabilities, and we further employ JSON as the structured output format to ensure clarity.

In addition, we incorporate two techniques into our prompt design. The first technique is chain-of-thought (CoT) prompting, which guides the LLM through step-by-step reasoning \cite{wei2022chain}. This approach can facilitate accurate task identification and constraint recognition by prompting LLM to engage in deliberate, intermediate reasoning stages before concluding. Specifically, our CoT prompts instruct the LLM to: (1) carefully read the task description, (2) identify the specific task and associated robotic system, (3) determine the constraint type, (4) extract relevant parameters, and (5) format the extracted information. The second technique is few-shot learning. Given that prior research indicates LLMs excel in few-shot learning scenarios \cite{brown2020language, hegselmann2023tabllm}, we provide multiple examples within our prompt. These examples serve as explicit references, allowing LLM to capture task requirements more effectively and thereby enhancing the accuracy of its outputs. By leveraging these prompt engineering techniques, we develop a structured prompt template tailored for prompting LLMs to analyze task descriptions of construction project scheduling (see Appendix \ref{app1}). 

For model selection, we refer to the widely recognized Multi-task Language Understanding (MMLU) benchmark \cite{hendrycks2020measuring}. We utilize two popular LLM families, OpenAI’s GPT series and Anthropic’s Claude series, because of their demonstrated capabilities in complex reasoning and strong performance across diverse NLP tasks \cite{fan2023nphardeval, sonoda2024diagnostic}. Specifically, from OpenAI, we select GPT-4o-mini, GPT-4o, GPT-4.1-mini and GPT-4.1. From Anthropic, we include Claude-Haiku and Claude-Sonnet.

 



\subsection{Multi-robot task allocation: replanning}\label{sec:method-replanning}

Suppose the task schedule and robot assignments have already been determined through the task allocation optimization described in the previous section, and the plan has been partially executed.
At time \(T^R\), updated information becomes available reflecting changes in task conditions, such as modifications to robot capabilities, availability, expected task durations, additional task dependencies, or time window constraints.
These changes may render the current plan infeasible or suboptimal, necessitating the generation of a new plan. At this point, some tasks have been completed, others are in progress, and the rest have not yet started. For the completed and ongoing tasks, their assigned robots and schedules must remain unchanged during the replanning. Furthermore, in many real-world scenarios, modifying the original plan may incur additional operational costs. To account for this, penalties can be introduced into the optimization to discourage unnecessary deviations from the original plan.

Given an original task allocation and scheduling plan, represented by the solution to the variables $^0x_{ir}$, $^0t^{s}_{ir}$, $^0t^e_{ir}$, $^0t^s_i$, and $^0t^e_i$. Let $\Tcal_{-} = \{i \in \Tcal \mid {^0t^s_i} \leq T^R\}$ denote the set of tasks that are ongoing or completed by the replanning time $T^R$, and let $\Tcal_{+} = \{i \in \Tcal \mid {^0t^s_i} > T^R\}$ represent the set of tasks that have not yet started. The replanning optimization is formulated as follows.


\textbf{Replanning optimization:}
The replanning objective jointly minimizes the makespan and individual task completion times while also penalizing deviations from the original plan. We set \(C_m \gg C_s \approx C_x \approx C_t \) to ensure that the makespan remains the primary objective, with the other terms balanced relative to one another. The constraints include the original constraints \eqref{eqn:assignment_var}-\eqref{eqn:task_conflict3}, along with additional constraints \eqref{eqn:schedule_fixed}-\eqref{eqn:assignment_fixed} to ensure plan consistency.
\begin{align}
    \min_{\substack{x_{ir}, \ t^s_i, \ t^e_i, \\ t^s_{ir}, \ t^e_{ir}}} \ ( & C_m \max_{i \in \Tcal} \ t^e_i + C_s \sum_{i \in \Tcal} \ t^e_i + C_r \sum_{i \in \Tcal} \sum_{r \in \Rcal} x_{ir} \nonumber \\
    & + C_x \Delta x + C_t \Delta t) \\
    \text{subject to} & \ \eqref{eqn:assignment_var}-\eqref{eqn:task_conflict3} \text{ and } \eqref{eqn:schedule_fixed}-\eqref{eqn:assignment_fixed} \nonumber
\end{align}

\textbf{Plan change penalty for:}
Changes to the original task assignments and schedules are penalized during replanning to ensure that the new plan accommodates updated task conditions while minimizing deviations from the original plan.
\begin{align}
    \Delta x &= \sum_{\forall i \in \Tcal_{+}} \sum_{\forall r \in \Rcal} |x_{ir} - {^0x_{ir}}| \label{eqn:schedule_change_penalty} \\
    \Delta t &= \sum_{\forall i \in \Tcal_{+}} \left( |t^s_i - {^0t^s_i}| + |t^e_i - {^0t^e_i}| \label{eqn:assignment_change_penalty} \right)
\end{align}

\textbf{Historical plan constraints for completed tasks:} The robot assignments and task schedules for ongoing or completed tasks must remain unchanged to ensure consistency between the updated plan and the actual execution.
\begin{align}
    t^s_i = {^0t^s_i}, \ t^e_i = {^0t^e_i}, \quad \forall i \in \Tcal_{-} \label{eqn:schedule_fixed} \\
    x_{ir} = {^0x_{ir}} \quad \forall r \in \Rcal, \forall i \in \Tcal_{-} \label{eqn:assignment_fixed}
\end{align}













\section{SYMPHONY}
\label{system}
\subsection{Background and Overview}

A Markov Decision Process (MDP) \cite{bellman1966MDP} provides a principled framework for modeling sequential decision-making, defined by the tuple $(S, A, \mathcal{T}, R, \gamma)$, where $S$ is the state space, $A$ is the action space, $\mathcal{T}: S \times A \rightarrow \mathcal{P}(S)$ defines the transition dynamics, $R: S \times A \rightarrow \mathbb{R}$ is the reward function, and $\gamma \in [0,1]$ is the discount factor. At each timestep, the agent observes a state $s \in S$, selects an action $a \in A$, transitions to a new state $s' \sim P(\cdot \mid s, a)$, and receives a reward $R(s, a)$. The objective is to learn a policy that maximizes the expected cumulative discounted return.

LLMs can be naturally integrated into this framework to support high-level reasoning and decision-making. Specifically, an LLM can serve as a \textit{policy} by generating actions conditioned on language-based state representations, as a \textit{value function} by estimating expected returns from textual trajectories, or as a \textit{world model} by predicting future states and rewards through learned knowledge. Unlike traditional reinforcement learning agents that rely on explicit environment modeling and manually designed reward signals, LLM-based agents leverage pretraining on large corpora to internalize commonsense, domain knowledge, and structured reasoning. This allows them to operate effectively in complex, open-ended environments with minimal task-specific engineering.


Monte Carlo Tree Search (MCTS)~\cite{coulom2006MCTS} is a sample-based planning algorithm that incrementally builds a search tree by balancing exploration and exploitation. It has been widely used in sequential decision-making problems and is well-suited for integration with LLM-based agents, as it allows structured reasoning guided by model-generated priors.

Formally, given the MDP setup, MCTS constructs a partial search tree rooted at the initial state $s_0$, iteratively performing four steps: \textit{selection}, which traverses the tree using an upper confidence bound to choose promising actions; \textit{expansion}, which adds new child nodes for unexplored actions; \textit{simulation} (or rollout), which estimates future rewards using a  policy; and \textit{backpropagation}, which updates statistics along the visited path. A detailed description of MCTS can be found in Appendix~\ref{appendix:mcts}.

In this work, we adapt MCTS by incorporating LLMs to guide both the selection and rollout phases, replacing uniform or heuristic strategies with model-informed priors that focus exploration on semantically meaningful regions. Building on this foundation, we introduce \textbf{SYMPHONY}, a synergistic multi-agent planning framework designed to enhance both the efficiency and robustness of LLM-based decision-making. SYMPHONY extends classical MCTS through several key innovations: a heterogeneous ensemble of LLM agents with diverse inductive priors, a UCB-driven adaptive agent scheduling strategy, a pool-wise memory sharing protocol enabling decentralized reflective adaptation, and an entropy-aware utility modulation mechanism for confidence-calibrated evaluation. These components collectively promote diverse trajectory generation, context-aware coordination, coherent information propagation and reliable value estimation. The theoretical analysis and complete pseudocode of SYMPHONY can be found in Appendix \ref{alg:symphony}.



\begin{figure}[h]
		\centering
		\includegraphics[width=\textwidth]{fig/method.png}
		\caption{SYMPHONY System Overview.}
\end{figure}




\subsection{Heterogeneous Agent Pool}

The heterogeneous agent pool in SYMPHONY is designed to enhance rollout diversity by incorporating multiple language models with varied inductive biases and reasoning behaviors. Unlike traditional MCTS approaches that rely on repeated queries to a single language model, SYMPHONY maintains a collection of distinct language models, each serving as an independent agent that contribute complementary perspectives during search. Formally, the agent pool is represented as \(\mathcal{M}^{(k)} = \left \{ M_1^{(k)}, \cdot\cdot\cdot, M_n^{(k)} \right \}  \), where $M_i$ is the $i^{th}$ agent based on a language model after the $k^{th}$ memory update.  

These agents may be instantiated from either open-source models that are deployable on consumer-grade hardware, or large-scale cloud-based models accessible only via remote API.  Different agents exhibit complementary strengths in reasoning depth, factual precision, abstraction ability, and stylistic preferences, which collectively enhance the system’s capacity to explore diverse trajectories in the search space.


SYMPHONY employs a uniform  input-output interface for agents pool. More specifically, the input to the agent pool at the $t^{th}$ step is 
$P_{\phi}(s_t, h_{t-1})$, where  $\phi \in \{\text{expansion}, \text{evaluation}, \text{reflection}\}$  is the function indicator of language models , $P_\phi$ is the corresponding prompt template, and $h_t$ is the interaction history \(h_{t-1}=(s_0,a_0,\cdot\cdot\cdot,s_{t-1},a_{t-1})\). This design choice facilitates modularity. New models can be added or removed without altering the core planning algorithm. It ensures compatibility with future advances in LLMs and facilitates efficient reuse of available computational resources under different deployment settings. Prompts for each stage can be found in case studies (Appendix \ref{appendix:case}). 

% While differing in architecture and training background, all agents conform to a unified interface, abstracting away model-specific details and allowing SYMPHONY to invoke them interchangeably. 
% This design choice facilitates modularity. New models can be added or removed without altering the core planning algorithm. It ensures compatibility with future advances in LLMs and facilitates efficient reuse of available computational resources under different deployment settings. 
%  More specifically, the input to the agent pool at the $t^{th}$ step is, 
% \begin{equation}
%     Prompt = P_{\phi}(s_t, h_{t-1})
% \end{equation}
% where  $\phi \in \{\text{expansion}, \text{evaluation}\}$ is the stage indicator of MCTS, $P_\phi$ is the corresponding prompt template, and $h_t$ is the interaction history   \(h_{t-1}=(s_0,a_0,\cdot\cdot\cdot,s_{t-1},a_{t-1})\).



% By decoupling planning logic from model-specific implementations and enforcing a uniform input-output interface, SYMPHONY transforms diverse LLMs into interoperable planning components. This enables ensemble-style exploration without compromising architectural generality or extensibility, and forms the basis for the adaptive coordination and confidence scoring strategies introduced in subsequent sections.


\subsection{Agent Scheduling}
To operationalize the functional heterogeneity of the agent pool, SYMPHONY implements an adaptive dispatch mechanism grounded in the Upper Confidence Bound (UCB) principle, formulating agent selection at each MCTS rollout step as a structured multi-armed bandit problem. Rather than relying on static sampling heuristics or fixed priority weights, the framework dynamically calibrates agent choice based on  performance statistics, enabling context-sensitive allocation of reasoning capacity.

Formally, for each agent $M_i^{(k)} \in \mathcal{M}^{(k)}$, the scheduler maintains a cumulative utility estimate $\bar{Q}(M^{(k)}_i)$ reflecting empirical rollout effectiveness, Let $S_{M_i^{(k)}}$ denotes the set of nodes generated by agent $M^{(k)}_i$, $S_{M_i^{(k)}} =  \{ s_{t+1}  \sim  \mathcal{T}(s_t, M_i^{(k)} (s_t, h_{t-1}) )  \} $. We record the total invocation count for agent $M^{(k)}_i$ as $N_i^{(k)}$, Similarly, the cumulative average score for agent $M^{(k)}_i$ is defined as $\bar{Q}(M^{(k)}_i) = \ { \sum_{s_t \in S_{M_i^{(k)}}}}R(s_{t}) / |S_{M_i^{(k)}}| $. The selection priority at a search node $s_t$ is governed by the canonical UCB expression:

\begin{equation}\label{eq:ucb}
\mathrm{UCB}(M_i^{(k)}) = \bar{Q}(M_i^{(k)}) + \alpha \cdot \sqrt{\frac{\ln N^{\mathcal{M}^{(k)}}_{total}}{N(M^{(k)}_i) + 1}}
\end{equation}

Here $\alpha$ denotes an exploration–exploitation trade-off hyperparameter, 
$N^{\mathcal{M}^{(k)}}_{total}= \ {\textstyle \sum_{j=1}^n}N(M^{(k)}_j)$ represents the total number of scheduling decisions made thus far, and the denominator smoothing term ensures initialization-phase optimism. This formulation favors agents that exhibit either superior historical returns or low invocation frequency, thereby enabling simultaneous exploitation of high-confidence models and exploration of underutilized reasoning modes.

Crucially, this scheduling mechanism is not an isolated module but is tightly interwoven with the recursive structure of MCTS, encompassing action generation and reflective evaluation. Following UCT-guided node traversal, a  frontier node $s_t$ is expanded by dispatching an $M_i^{(k)} \in \mathcal{M}^{(k)}$
selected via Equation~\ref{eq:ucb}, which is queried using the  expansion prompt:

\begin{equation}
a_t = M_i^{(k)}(P_{\text{expansion}}(s_t, h_{t-1}))
\end{equation}

where $h_{t-1}$ encodes the accumulated interaction trace. The returned actions populate the search frontier with semantically diverse and structurally varied hypotheses.

The agent scheduling mechanism is also used in creating pool-wise reflection memory and node evaluation with EMCS, which will be detailed in the subsequent subsections. 

We further establish, from a theoretical perspective, that sampling agents from the ensemble with non-zero probabilities leads to a strictly lower expected error than deterministically selecting a single agent. The detailed proof is provided in Appendix~\ref{appendix:proofs}.

% These candidate actions subsequently trigger a reflective evaluation cascade, wherein agents perform parallelized, self-consistent analyses of the current search state—assessing logical coherence, goal alignment, and strategic deviation. Their assessments are subsequently fused via the Entropy-Modulated Confidence Scoring (EMCS) mechanism (Section 4.4), producing a calibrated utility landscape that prioritizes consensus-grounded hypotheses while suppressing incoherent or speculative rollouts.

% Furthermore, in the event of terminal failure—where the search fails to reach a successful outcome, the system reengages the scheduling loop for post hoc reflection. The entire trajectory is abstracted into a natural language diagnostic, which is broadcast into the shared memory of the agent pool to inform future behavior via decentralized prompt adaptation and internal alignment tuning.

% Through this tightly integrated scheduling framework, SYMPHONY transcends naive ensemble querying and instead realizes an orchestrated reasoning substrate: one that continuously aligns agent dispatch with contextual task demands and emergent search signals. This not only maximizes epistemic coverage during exploration but also endows the system with the capacity for iterative self-correction, enabling robust operation in high-stakes, dynamically evolving decision environments.



\subsection{Pool-wise Memory Sharing}
To support continual adaptation without parameter updates, SYMPHONY introduces a pool-wise memory sharing mechanism based on decentralized reflection with natural language. Rather than relying on explicit retraining, agents update their behavior by integrating peer-generated reflections into prompt-level memory.

When a trajectory terminates unsuccessfully, $\tau_{\mathrm{fail}} = (s_0, a_0, \dots, s_T)$, a UCB-selected agent $M_i^{(k)}$
generates a structured reflection $\mathcal{R}^{k}_i$ summarizing the failure. This reflection is broadcast to the entire agent pool and treated as a shared memory block. As reflections accumulate from different agents and episodes, they form a diverse collective memory that enhances generalization and coordination.

To manage memory constraints and maintain efficiency, each agent retains a fixed-size buffer updated via a FIFO policy. Reflections are incorporated through prompt-level memory updates:
\begin{equation}
\mathcal{M}^{(k+1)} = \text{Update}(\mathcal{M}^{(k)}, \mathcal{R}^k), \mathcal{R}^k = M_i^{(k)}(P_{\text{reflection}}(s_t, h_{t-1})) 
\end{equation}
This update mechanism enables behavioral adjustment without modifying model parameters, supporting lightweight and scalable adaptation across heterogeneous agents.



% To enable continual adaptation across heterogeneous agents without explicit parameter sharing or retraining, SYMPHONY employs a pool-wise memory sharing mechanism grounded in decentralized natural language reflection. This mechanism supports distributed behavioral adjustment by allowing agents to integrate and respond to peer-generated reflections through prompt-level updates, rather than through weight modification.

% When a planning trajectory terminates unsuccessfully, denoted as $\tau_{\mathrm{fail}} = (s_0, a_0, \dots, s_T)$, the system invokes the agent scheduling algorithm (E.q. \ref{eq:ucb}) to assign a reflective role to one agent from the pool. The selected agent produces a structured natural language reflection $\mathcal{R}^{k}_i$ that captures critical reasoning failures, such as misaligned strategies, overlooked checkpoints, or incoherent decision paths.

% The resulting reflection is treated as a memory block and disseminated to the entire agent pool. As these reflections originate from diverse agents and tasks, they gradually accumulate into a heterogeneous repository of shared experiences, forming a collective memory that encodes varied reasoning patterns. This shared memory enhances the pool's overall adaptability by exposing agents to perspectives beyond their own decision traces.

% To ensure inference efficiency and bounded memory usage, each agent maintains a fixed-size buffer governed by a first-in, first-out (FIFO) policy, retaining only the most recent reflective updates. After every reflection, agents incorporate new shared reflections into their internal memory state via prompt-level integration:
% \begin{equation}
% \mathcal{M}_j^{(k+1)} = \text{Update}(\mathcal{M}_j^{(k)}, \mathcal{R}^k)
% \end{equation}
% Here, $\mathcal{M}^{(k)}_j$ denotes agent $M_j$’s memory at iteration $k$, typically instantiated as an editable prompt template or cached instruction context. The $\text{Update}(\cdot)$ function modifies this memory without altering the model’s parameters, allowing behavioral modulation even in closed-weight or API-based LLMs.

% Through this decentralized reflection process, SYMPHONY enables agents to co-evolve over time via shared linguistic feedback, fostering implicit coordination and continual improvement without centralized training or synchronization.



% To enable continual improvement across heterogeneous agents without explicit parameter sharing or retraining, SYMPHONY introduces a pool-wise memory sharing mechanism grounded in collective natural language reflection. Rather than treating post-hoc reasoning as an isolated agent-specific operation, the framework aggregates reflective signals at the agent-pool level and disseminates them as shared memory updates, thereby enabling distributed behavioral adaptation. This design enables implicit collective reflection to emerge organically: although each agent contributes only locally and episodically, their aggregated reflective outputs gradually construct a shared global memory structure across the pool.

% Concretely, upon termination of an unsuccessful trajectory \(\tau_{fail}=(s_0,a_0,\cdot\cdot\cdot,s_T)\), a UCB-selected agent \(s_T\in\mathcal{T}_{\mathrm{fail}}\) generates a structured natural language reflection \(\mathcal{R}^{k}_i\) encapsulating failure causes, misaligned reasoning paths, or overlooked decision checkpoints. This reflection is treated as a memory block and is immediately broadcast to the full agent pool. Each agent $M_j$ then independently incorporates this shared reflection into its own internal memory state via a prompt-level update:



% \begin{equation}
% \mathcal{M}_j^{(k+1)} = \text{Update}(\mathcal{M}_j^{(k)}, \mathcal{R}^k)
% \end{equation}

% Here, 
% $\mathcal{M}^{(k)}_j$ denotes agent $M_j$’s reflective memory at planning iteration $k$, typically operationalized as a modifiable prompt template or embedded instruction cache. The $\text{Update}(\cdot)$ function performs localized memory editing without altering model parameters, enabling behavior modulation even for closed-weight or remotely hosted models.

% Over successive rollouts, this process results in an emergent distributed memory space: reflections from temporally distinct agents accumulate as shared artifacts, which are selectively absorbed and recontextualized by their peers. Unlike centralized gradient-based learning, this memory sharing framework fosters loose coupling between agents while preserving a continual, low-friction channel for mutual adaptation. Through this mechanism, SYMPHONY instantiates a form of decentralized cognitive coordination, in which reasoning improvements propagate across the ensemble not through weight updates, but through reflective linguistic traces embedded in memory.


\subsection{Entropy-Modulated Node Evaluation}
To improve value estimation during search, SYMPHONY introduces an entropy-modulated node evaluation strategy that adjusts utility scores based on agent confidence. Upon expanding a new node $s_t$, a scheduled agent $M_i^{(k)} \in \mathcal{M}^{(k)}$ performs an internal evaluation, producing a value estimate $Z(s_t) \in [0,1]$ and a confidence score $C(s_t) \in (0,1)$:
\begin{equation}
    Z(s_{t}), C(s_{t}) = M_i(P_{\text{evaluation}}(s_{t}, h_{t-1}))
\end{equation}

To integrate these outputs, SYMPHONY employs Entropy-Modulated Confidence Scoring (EMCS), which penalizes uncertain predictions by down-weighting value estimates using the entropy of a Bernoulli distribution. Here, the confidence score $C(s_t)$ is interpreted as the success probability of a Bernoulli variable: the entropy is maximal at $C(s_t)=0.5$, indicating maximum uncertainty, and approaches zero as $C(s_t)\rightarrow 0$ or $C(s_t)\rightarrow 1$, reflecting high confidence.
\begin{equation}\label{eq:emcs}
     R(s_{t}) = Z(s_t) \cdot (1 - E(s_t))
\end{equation}
where $E(s_t) = -C(s_t)\ln C(s_t) - (1 - C(s_t))\ln(1 - C(s_t))$.

This formulation preserves confident evaluations while suppressing uncertain ones, ensuring that nodes with ambiguous outcomes have reduced influence. Compared to fixed heuristics, EMCS offers uncertainty-aware, real-time modulation with minimal overhead, leading to more stable and reliable planning behavior within the MCTS loop.



% In addition to improving rollout diversity and coordination, SYMPHONY incorporates an entropy-modulated node evaluation strategy to better assess the value of each candidate node. 

% Whenever a new node \(s_t\)  is expanded, SYMPHONY triggers real-time evaluation, wherein a selected agent \(M_i^{(k)} \in \mathcal{M}^{(k)} \) evaluates the state based on internal priors and reasoning heuristics. 
% Each agent produces a value estimate 
% \(V_s \in [0,1]\), representing its assessment of the downstream utility of the node, along with a confidence score 
% \(C_s \in [0,1]\) reflecting its epistemic certainty.
% That is, 
% \begin{equation}
%     Z(s_{t+1}),C(s_{t+1}) = M_i(P_{\text{evaluation}}(s_{t+1}, h_{t}))
% \end{equation}

% To integrate these outputs into the search process, we introduce Entropy-Modulated Confidence Scoring (EMCS), which penalizes uncertain estimates by adjusting value scores based on the information entropy of model confidence. EMCS applies a modulation function based on the entropy of a Bernoulli distribution defined by the confidence. That is 
% \begin{equation}\label{eq:emcs}
%      R(s_{t+1}) = Z(s_t) \cdot (1-E(s_t)) 
% \end{equation}
% where $E(s_t)=-c(s_t)\ln c(s_t)-(1-c(s_t))\ln(1-c(s_t))$. 


% This formulation ensures that when the agent is highly confident ($C\rightarrow0$,$C\rightarrow1$),the entropy approaches zero and the value estimate is preserved. Conversely, when $C=0.5$, the entropy reaches its maximum $\ln2\approx0.693$, and the utility score is significantly suppressed. This entropy-based modulation effectively captures the model's internal uncertainty: the more uncertain the prediction, the less influence its value estimate has on the planning decision. Compared to traditional heuristics, EMCS better avoids over-reliance on uncertain high-value predictions and ensures stable pruning of unreliable branches. Its symmetric, convex form provides smooth, real-time adjustment with negligible computational cost, making it highly suitable for integration into the MCTS loop.

\section{Experiments}\label{sec:exp}

%需要添加每个数据集的baseline为何不同的解释。并且某些baseline的模型不是gpt-4
%选用模型的理由

% We design experiments along three dimensions, reasoning, decision-making, and code generation to validate the effectiveness of our approach. Specifically, we evaluate our method across three tasks: (1) Multi-hop question answering on HotpotQA \cite{yang2018hotpotqa} to assess reasoning ability;(2) Decision-making in the WebShop \cite{yao2022webshop} e-commerce platform to test planning and selection; and(3) Code generation on MBPP \cite{austin2021mbpp} to evaluate whether the model can infer a solution through reasoning and present it in a standardized, executable form.

We evaluate our approach across three representative tasks spanning reasoning, decision-making, and code generation. Specifically, we conduct experiments on: (1) multi-hop question answering using HotpotQA~\cite{yang2018hotpotqa} to assess reasoning capabilities; (2) goal-directed interaction on WebShop~\cite{yao2022webshop} to evaluate decision-making and planning; and (3) code generation on MBPP~\cite{austin2021mbpp} to test the model's ability to reason and produce executable solutions.


\subsection{Experiment Settings}

SYMPHONY supports flexible agent composition and is compatible with a range of language models under different computational constraints. We evaluate two deployment configurations: \textbf{SYMPHONY-S}, designed for consumer-grade hardware, and \textbf{SYMPHONY-L}, which leverages large-scale foundation models via cloud-based APIs.

\textbf{SYMPHONY-S} comprises open-source models that can be executed locally, including  Qwen2.5‑7B‑Instruct‑1M~\cite{yang2024qwen2-5-1M}, Mistral‑7B‑Instruct‑v0.3~\cite{jiang2023mistral7b}, and Llama‑3.1‑8B‑Instruct~\cite{grattafiori2024llama3herdmodels}. This configuration supports efficient inference with minimal deployment cost. In contrast, \textbf{SYMPHONY-L} comprises high-performance models: GPT‑4~\cite{achiam2023gpt4}, Qwen‑Max (2024‑09‑19)~\cite{yang2024qwen2-5}, and DeepSeek‑V3 (2025‑03‑24)~\cite{liu2024deepseek}, which operate through API endpoints within inference-as-a-service infrastructures.

All experiments are carried out under a unified protocol aligned with previous work~\cite{shinn2023reflexion, zhou2024language, gan2025master}. To ensure comparability, we apply consistent prompt formats and fixed hyperparameter settings across both configurations, including decoding temperature, planning depth, rollout budget, and number of demonstrations. To mitigate LLM stochasticity, each experiment is repeated 3 times on the same data set, and the mean accuracy is reported. The detailed hyper-parameter settings are described in Appendix~\ref{appendix:parameter_setting}. % In all the experiments, SYMPHONY-L significantly outperforms SOTA ($p<0.01$). 



% SYMPHONY is agnostic to the choice of underlying agents and supports flexible combinations tailored to different computational settings. We evaluate two deployment configurations: \textbf{SYMPHONY-S}, operates on consumer-grade hardware, and \textbf{SYMPHONY-L}, which relies on API-accessible large-scale models.

% \textbf{SYMPHONY-S} comprises open-source models suitable for local execution, including Qwen2.5‑7B‑Instruct-1M~\cite{yang2024qwen2-5-1M}, Mistral‑7B‑Instruct-v0.3~\cite{jiang2023mistral7b}, and Llama‑3.1‑8B‑Instruct~\cite{grattafiori2024llama3herdmodels}, enabling efficient on-device inference and low-cost experimentation.

% \textbf{SYMPHONY-L} includes proprietary foundation models accessible only via cloud-based APIs, such as GPT‑4~\cite{achiam2023gpt4}, Qwen‑Max (2024‑09‑19)~\cite{yang2024qwen2-5}, and DeepSeek‑V3 (2025‑03‑24)~\cite{liu2024deepseek}, deployed within high-throughput inference infrastructures.


% We follow the experimental protocols established in prior work~\cite{shinn2023reflexion, zhou2024language, gan2025master} across all three datasets. To mitigate variability from LLM stochasticity, each experiment is repeated three times on the same sample set, and mean accuracy is reported.

% To ensure comparability, we adopt consistent prompt formats and fixed hyperparameters across both settings, including decoding temperature, planning depth, and rollout budget.




% We adopt the same experimental setup as previous baselines \cite{} the ReAct, Reflexion, and LATS baselines: for both HotPotQA and WebShop, we randomly sample 100 questions using a fixed random seed to ensure consistency across methods. To reduce variability from LLM nondeterminism, each experiment is repeated three times on the identical sample, and we report the mean accuracy in our results table.  
% We adopt the same experimental setup as previous baselines  \cite{shinn2023reflexion,zhou2024language,gan2025master} for all three datasets. To reduce variability from LLM nondeterminism, each experiment is repeated three times on the identical sample, and we report the mean accuracy in our results table.  


% 没有解释清楚,为什么我们选这几个模型
% 尝试从多样性的角度defend我们的选择
% We categorize our models into two groups: Large-Scale Language Models (LLMs-L) and Small-Scale Language Models (LLMs-S). As part of the Large-Scale Language Models (LLMs-L) group, we include GPT‑4 as a primary baseline and additionally incorporate Qwen‑Max (2025‑01‑25) and DeepSeek V3 (2025‑03‑24), both of which offer comparable capabilities at a lower API cost than GPT‑4. For LLMs-S, smaller LLMs, we benchmark Qwen2.5‑7B‑Instruct‑1M, Mistral‑7B‑Instruct‑v0.3, and Llama‑3.1‑8B‑Instruct, each of which runs efficiently on modest GPU hardware.


% SYMPHONY is agnostic to the specific choice of agents and can accommodate different  agent combinations tailored to different application contexts. We evaluate SYMPHONY under two distinct deployment configurations, differentiated by their compatibility with computational infrastructure: SYMPHONY-S, which supports deployment on consumer-grade hardware, and SYMPHONY-L, which relies on large-scale models accessible exclusively via remote APIs.

% \textbf{SYMPHONY-S} comprises language models that can be executed locally. Specifically, we include Qwen2.5‑7B‑Instruct-1M \cite{yang2024qwen2-5-1M}, Mistral‑7B‑Instruct-v0.3 \cite{jiang2023mistral7b}, and Llama‑3.1‑8B‑Instruct \cite{grattafiori2024llama3herdmodels}, which collectively represent high-performance open-source models suitable for on-device inference and iterative experimentation with minimal deployment cost.

% In contrast, \textbf{SYMPHONY-L} utilizes proprietary foundation models that are not publicly hostable and can only be accessed via cloud-based API endpoints. This configuration includes GPT‑4 \cite{achiam2023gpt4}, Qwen‑Max (2024‑09‑19) \cite{yang2024qwen2-5}, and DeepSeek‑V3 (2025‑03‑24) \cite{liu2024deepseek}, all of which operate within high-throughput inference-as-a-service infrastructures.

% % This bifurcated setup enables systematic evaluation of SYMPHONY under both resource-constrained and compute-rich conditions, demonstrating its applicability across a wide spectrum of real-world deployment scenarios. Importantly, the framework remains agnostic to the specific choice of agents and can accommodate alternative model combinations tailored to different application contexts.

% For all experiments, we ensure consistency by applying identical prompt formats and maintaining fixed hyperparameter settings across both configurations, including decoding temperature, planning depth, and rollout budget.




% We evaluate the proposed method on different sizes of models. More specifically, we construct \textbf{SYMPHONY-S} and \textbf{SYMPHONY-L} that runs on different types of hardware:
% \begin{itemize}
%     \item SYMPHONY-S: Agents that can be hosted by consumer hardware. We include Qwen2.5‑7B‑Instruct‑1M \cite{yang2024qwen2-5-1M} (sliding-window attention), Mistral‑7B‑Instruct‑v0.3 \cite{jiang2023mistral7b} (grouped-query attention), and Llama‑3.1‑8B‑Instruct 
% \cite{grattafiori2024llama3herdmodels} (RMSNorm and RoPE encoding), each representing different design decisions in attention mechanisms, normalization, and positional representation. 
%     \item SYMPHONY-L: Agents that are usually run on large-scale clusters, providing service via API. We  include GPT‑4 \cite{achiam2023gpt4}(Mixture-of-Experts architecture), Qwen‑Max (2024-09-19) \cite{yang2024qwen2-5}(dense Transformer with dynamic gradient clipping), and DeepSeek-V3 \cite{liu2024deepseek}(2025‑03‑24) (sparse attention with memory-efficient design)
% \end{itemize}

% The above selections cover key distinctions in routing, density, and memory optimization, enhancing agent pool diversity, reduces architectural bias, and supports fair evaluation of our method’s effectiveness across heterogeneous model backbones. It should be noted that HARMNOY can be applied to other choices of agent pools as well, including other LLMs depending on the task requirements and developer preferences. 







% For each experimental setting, we use identical prompts and keep all hyperparameters—such as temperature, number of iterations, and the number of expansions, consistent across both proprietary large models and open-source smaller models. 
% This design choice ensures that the observed performance differences are not merely attributable to the underlying capabilities of the models themselves, but rather reflect the true effectiveness of our proposed method.



\begin{table*}[ht]
\centering
\begin{minipage}[t]{0.3\linewidth}
\centering
\caption{HotpotQA. }
\label{tab:hotpotqa}
\setlength{\tabcolsep}{3pt} % 缩小列间距
\resizebox{0.98\linewidth}{!}{ % 适当放大显示比例
\footnotesize
\begin{tabular}{@{}lc@{}}
\toprule
\textbf{Method} & \textbf{Exact Match $\uparrow$} \\
\midrule
\makecell{CoT  \cite{wei2022chaincot}}            & 0.34 \\
\makecell{CoT-SC  \cite{cot-sc2023}}          & 0.38 \\
\makecell{ReAct  \cite{yao2023react}}          & 0.39 \\
\makecell{Reflexion  \cite{shinn2023reflexion}}  & 0.51 \\
\makecell{ToT  \cite{yao2023tree}}            & 0.55 \\
\makecell{RAP  \cite{hao2023reasoning}}         & 0.60 \\
\makecell{LATS  \cite{zhou2024language}}        & 0.71 \\
\makecell{Beam Retrieval  \cite{zhang2023end}}  & 0.73 \\
\makecell{MASTER  \cite{gan2025master}}         & 0.76 \\
\midrule
\textbf{SYMPHONY-S} & \textbf{0.59} \\
\textbf{SYMPHONY-L} & \textbf{0.79} \\
\bottomrule
\end{tabular}
}
\end{minipage}
\hfill
\begin{minipage}[t]{0.34\linewidth}
\centering
\caption{WebShop.}
\label{tab:webshop_results}
\setlength{\tabcolsep}{3pt} % 缩小列间距
\resizebox{0.98\linewidth}{!}{ % 适当放大显示比例
\footnotesize
\begin{tabular}{lcc}
\toprule
\textbf{Method} & \textbf{Score $\uparrow$} & \textbf{SR $\uparrow$} \\
\midrule
\makecell{IL  \cite{yao2022webshop}}          & 0.60 & 0.29 \\
\makecell{IL+RL  \cite{yao2022webshop}}       & 0.62 & 0.29 \\
\makecell{ReAct  \cite{yao2023react}}      & 0.54 & 0.32 \\
\makecell{Reflexion  \cite{shinn2023reflexion}} & 0.64 & 0.35 \\
\makecell{Fine-tuning  \cite{furuta2024multimodal}} & 0.68 & 0.45 \\
\makecell{AgentKit  \cite{wu2024agentkit}}     & 0.70 & --  \\
\makecell{LATS  \cite{zhou2024language}}      & 0.76 & 0.38 \\
\makecell{MASTER  \cite{gan2025master}}          & 0.80 & -- \\
\makecell{Human Expert  \cite{yao2022webshop}}    & 0.82 & 0.60 \\
\midrule
\textbf{SYMPHONY-S}           & \textbf{0.82} & \textbf{0.56} \\
\textbf{SYMPHONY-L}           & \textbf{0.88} & \textbf{0.72} \\
\bottomrule
\end{tabular}
}


\end{minipage}
\hfill
\begin{minipage}[t]{0.34\linewidth}
\centering
\caption{MBPP. }
\label{tab:mbpp_results}
\setlength{\tabcolsep}{3pt} % 缩小列间距
\resizebox{0.98\linewidth}{!}{ % 适当放大显示比例
\footnotesize
\begin{tabular}{lcc}
\toprule
\textbf{Method} & \textbf{\makecell{Pass@1 \\ (Python) $\uparrow$}} & \textbf{\makecell{Pass@1 \\ (Rust) $\uparrow$}}  \\
\midrule
\makecell{GPT-4  \cite{shinn2023reflexion}} & 0.800 & 0.710 \\
\makecell{GPT-4(CoT)  \cite{gan2025master}}  & 0.683 & -- \\
\makecell{GPT-4(ReAct)  \cite{yao2023react}}   & 0.710 & -- \\
\makecell{Reflexion  \cite{shinn2023reflexion}}  & 0.771 & 0.754 \\
\makecell{RAP  \cite{hao2023reasoning}}     & 0.714 & -- \\
\makecell{LATS  \cite{zhou2024language}}        & 0.811 & -- \\
\makecell{MetaGPT  \cite{hong2023metagpt}}     & 0.877 & -- \\
\makecell{AgentVerse  \cite{chen2023agentverse}} & 0.890 & -- \\
\makecell{MASTER  \cite{gan2025master}}          & 0.910 & -- \\
\makecell{AgentCoder  \cite{huang2023agentcoder}} & 0.918 & -- \\
\midrule
\textbf{SYMPHONY-S}           & \textbf{0.927} & \textbf{0.946} \\
\textbf{SYMPHONY-L}           & \textbf{0.965} & \textbf{0.974} \\
\bottomrule
\end{tabular}
}
\end{minipage}
\parbox{\linewidth}{
\footnotesize 
Note: Metrics are normalized to the [0,1] range; A dash (–) marks those not reported in the publication.
% All evaluation metrics are normalized to the [0,1] range. A dash (–) indicates that the corresponding method did not report that metric. SR for AgentKit and MASTER on WebShop are unavailable due to missing implementation details and metric reporting in their original publications.
% Additionally, many baselines on MBPP report results only on Python, without verifying language-agnostic effectiveness.
}
\end{table*}





\subsection{Reasoning:HotpotQA}
\textbf{Setup.} HotpotQA~\cite{yang2018hotpotqa} is a large-scale benchmark for multi-hop question answering, constructed from Wikipedia and containing approximately 113,000 question–answer pairs. In line with prior work~\cite{yao2023react,shinn2023reflexion,zhou2024language,gan2025master}, we employ an oracle feedback setting, where the environment immediately indicates whether a selected answer is correct. This setup is designed to isolate and evaluate the agent’s decision-making capabilities during interaction, rather than its ability to generate final answers.
Evaluation on this dataset is based on the exact match (EM) metric.

We compare SYMPHONY against representative baselines from four categories: (1) \textit{Linear reasoning} methods such as CoT~\cite{wei2022chaincot} and CoT-SC~\cite{cot-sc2023}; (2) \textit{Feedback-driven} approaches including ReAct~\cite{yao2023react} and Reflexion~\cite{shinn2023reflexion}; (3) \textit{Structured reasoning} methods such as ToT~\cite{yao2023tree}, RAP~\cite{hao2023reasoning}, LATS~\cite{zhou2024language}, and Beam Retrieval~\cite{zhang2023end}; and (4) the \textit{multi-agent framework} MASTER~\cite{gan2025master}, which builds multi-agent from the same LLM. Baseline results are taken from~\citet{gan2025master}, where GPT-4 is used uniformly across all methods.

\textbf{Results.} SYMPHONY demonstrates strong performance across all baseline categories. The lightweight \textbf{SYMPHONY-S} outperforms both linear reasoning and feedback-driven baselines, and performs comparably to structured search methods like RAP. The stronger \textbf{SYMPHONY-L} surpasses all structured baselines, including MASTER, achieving state-of-the-art performance on HotpotQA. These improvements reflect SYMPHONY’s ability to combine model heterogeneity with coordinated compositional reasoning. 



% HotPotQA \cite{yang2018hotpotqa} is a large‑scale, Wikipedia‑based multi‑hop QA benchmark containing roughly 113,000 question–answer pairs. % Each question requires reasoning over multiple distinct supporting articles, so an agent must (1) iteratively retrieve candidate passages, (2) sift through distractor documents that introduce noise, and (3) integrate evidence dispersed across non‑contiguous paragraphs to arrive at the correct answer. This noise‑rich setup rigorously tests an agent’s relevance‑scoring and evidence‑aggregation capabilities under multi‑step logical inference. 
% Performance on Hotpot QA is measured with standard exact match. 
% We employ  an oracle feedback mechanism that aligns with established protocols  \cite{yao2023react,shinn2023reflexion,zhou2024language,gan2025master}: the environment immediately verifies answer correctness, ensuring that our evaluation focuses squarely on reasoning proficiency rather than on feedback ambiguity. 
% Performance is measured using standard Exact Match , providing a direct comparison to prior baselines.

% 太长,直接说我们的n和k
% We evaluate each method using a subset of 100 questions along with 3 few-shot examples, applied to both a multi-agent framework composed of proprietary large language models and another composed of open-source small models. While prior work such as LATS (Zhou et al., 2024) reports optimal performance using \(n = 5\) sampled nodes and \(k = 50\) trajectories, our empirical findings suggest that excessive node expansion and repeated simulations yield diminishing returns due to the performance ceiling imposed by model capabilities on this task. As a result, we adopt a more efficient configuration in our experiments, setting \(n = 4\) and \(k = 15\) to balance computational cost with exploration depth.
% 此段重新整理:
% 1、描述实验结果:小模型比XXX好,大模型比XXX好
% 2、分析原因,为什么会有这种结果
% 3、这意味着什么,或者,反常实验结果分析


%Cot、Cot-SC的结果是从LATS中抄过来的,他那标注是GPT-3.5
%ToT、RAP是LATS作者复现的,用的GPT-3.5
%其中React、Reflexion、LATS、BeamRetrieval可以甩锅给MASTER,他说他们拿GPT-4复现了。BeamRetrieval其实是HotPotQA官方榜单的第一名,人家是完整数据集的。

%中文版本:
% 选取的基线方法覆盖了多样化的技术路径与发展阶段。其中,CoT和CoT-SC代表思维链技术的基础版与多路径优化版,验证线性推理的固有缺陷;ReAct和Reflexion作为交互式推理框架,对比动作规划与自我反思机制的效能边界;Beam Retrieval代表传统多文档检索增强方法,凸显动态知识验证的必要性;RAP、LATS、MASTER同样基于MCTS框架,对比MCTS框架下的技术瓶颈。

%选取的基线方法覆盖了多样化的技术路径,,其中基于LLM的规划和推理的有:CoT、CoT-SC(引导推理、动态反馈、树结构)及多智能体协作分类,选取 CoT(引导)、ReAct(反馈)、ToT(树结构)、MASTER(多智能体)等基线,覆盖多元推理机制与策略,通过对比验证本文方法在多样性和鲁棒性上的提升。 

%在SYMPHONY-S设置下,该框架不仅远远超越了Refelxion、ToT等方法,还逼近了RAP。而在SYMPHONY-L设置下,取得了SOTA,尽管指标上仅仅提升0.3,但这是由于Exact Matchd的严格所在。证明SYMPHONY通过协调规划,动态交互和反思性评估进行结构化、组合推理的能力,并且在MCTS框架内实现革新。


% 重新写:基于推理流程演进,HotpotQA 的对比方法分为四类:基础线性推理(CoT、CoT-SC):单路径+自洽性,刻画无多分支时的性能下限;动态反馈修正(ReAct、Reflexion):动作-观察和自我反思,验证结构化搜索在收敛速度与回溯能力上的优势;结构化搜索(ToT、RAP、LATS、Beam Retrieval):树状、奖励和束搜索,衡量异构模型在多路径和全局规划上的提升;多智能体协作(MASTER):同质 agent+MCTS,对比轻量级异构协作与记忆共享的效率与多样性增益。这些基线的实验结果来自{gan2025master}。,他们使用GPT-4作为骨干模型复制了它们。

% We select a representative and diverse set of baselines covering major reasoning paradigms. CoT \cite{wei2022chaincot} and CoT-SC \cite{cot-sc2023} reflect standard and multi-path chain-of-thought prompting, probing the limitations of linear reasoning. ReAct \cite{yao2023react} and Reflexion \cite{yao2023react} exemplify interactive frameworks, contrasting action planning with self-reflection. Beam Retrieval \cite{zhang2023end} represents traditional retrieval-augmented methods, highlighting the need for dynamic knowledge verification. RAP \cite{hao2023reasoning} , LATS \cite{zhou2024language} , and MASTER \cite{gan2025master} , all built on MCTS, facilitate a focused comparison within the same decision framework. The experimental results for these baselines are sourced from \cite{gan2025master}., who reproduced them using GPT-4 as the backbone model.

% We selected representative baselines from the following categories: (1) CoT \cite{wei2022chaincot}, CoT-SC \cite{cot-sc2023}, which uses linear reasoning paths; (2) ReAct \cite{yao2023react}, Reflexion \cite{shinn2023reflexion}, which frames LLM in the RL feedback loop; (3) ToT \cite{yao2023tree}, RAP \cite{hao2023reasoning}, LATS \cite{zhou2024language}, Beam Retrieval\cite{zhang2023end}, which uses tree-based structure or other multi-branch reasoning paths; (4) MASTER \cite{gan2025master}, which constructs multi-agent framework based on the same LLM. Baseline results are reported by \citet{gan2025master}, which uses GPT-4 uniformly as backbone. 

% Following the evolution of reasoning paradigms, we organize the selected HotpotQA baselines into four representative categories. Linear reasoning (CoT \cite{wei2022chaincot}, CoT-SC \cite{cot-sc2023}) captures single-path inference with or without self-consistency, serving as a lower bound in the absence of multi-branch exploration. Dynamic feedback methods (ReAct \cite{yao2023react}, Reflexion \cite{shinn2023reflexion}) incorporate action-observation cycles and self-reflection to enable iterative correction, highlighting the value of adaptive reasoning. Structured search approaches (ToT \cite{yao2023tree}, RAP \cite{hao2023reasoning}, LATS \cite{zhou2024language}, Beam Retrieval\cite{zhang2023end}) leverage tree-based structures, reward signals, or beam strategies to improve global planning and reasoning diversity. Finally, multi-agent collaboration (MASTER \cite{gan2025master}) combines homogeneous agents with MCTS, providing a direct comparison point for evaluating the efficiency and diversity benefits of our lightweight, heterogeneous collaboration framework with memory sharing. Baseline results are taken from \citet{gan2025master}, where GPT-4 was used as the backbone model for reproduction.


% 结果分析
%SYMPHONY在HotpotQA的严格评测中刷新SOTA。SYMPHONY-S设置下远超Reflexion、ToT等方法,逼近RAP;SYMPHONY-L设置下取得SOTA,为这项多跳问答任务树立了新的技术水平

% \textbf{Result.}  SYMPHONY demonstrates strong performance across both deployment configurations. Under the SYMPHONY-S setting, the framework not only surpasses GPT-4 and prominent reasoning baselines such as Reflexion \cite{shinn2023reflexion} and Tree-of-Thought \cite{yao2023tree}, but also closely approaches the performance of RAP \cite{hao2023reasoning}, a leading multi-step reasoning system. With SYMPHONY-L, the framework achieves the highest overall score, establishing a new state-of-the-art on this multi-hop question answering task.


% \textbf{Result.}  SYMPHONY sets a new state of the art on the rigorous HotpotQA benchmark. Under the SYMPHONY-S setting, it significantly outperforms methods such as Reflexion and ToT, and approaches the performance of RAP. Under the SYMPHONY-L setting, it achieves a new SOTA, establishing a new performance bar for multi-hop question answering.


% These results highlight SYMPHONY’s ability to perform structured, compositional reasoning—a core challenge in HotpotQA—through coordinated planning and reflective evaluation, regardless of the underlying model scale. Moreover, SYMPHONY attains this performance with significantly lower resource usage: even with only 4 expansions per step, 10 sampled trajectories, and a minimal 3-shot prompt, it consistently outperforms baselines such as LATS, which rely on heavier sampling strategies.

% \textbf{Result.} SYMPHONY’s results map cleanly onto our four baseline categories. Even the more lightweight SYMPHONY-S outstrips linear reasoning and dynamic feedback, and approaches key structured search methods like RAP. SYMPHONY-L then surpasses all structured-search methods and eclipses the multi-agent MASTER, establishing a new HotpotQA record.

% These gains reflect SYMPHONY’s coordinated compositional reasoning and model heterogeneity: it explores and evaluates multiple paths more effectively than linear, feedback-only, or single-model search, yet uses minimal resources—4 expansions per step, 10 trajectories, and a 3-shot prompt—outperforming heavy-sampling approaches like LATS.



% Experimental results on the small‐scale model configuration show that our method not only outperforms GPT-4 and GPT-4–based reasoning approaches such as Reflexion \cite{shinn2023reflexion} and Tree-of-Thought \cite{yao2023tree}, but also approaches the performance of RAP \cite{hao2023reasoning}. In the large-scale configuration, it achieves the highest overall score. These findings demonstrate that our framework enables compact models—despite having several orders of magnitude fewer parameters—to match or exceed the capabilities of much larger models, and to outperform existing techniques when applied to both small and large models. Moreover, our approach is model-agnostic: when instantiated with large models, it continues to deliver state-of-the-art results. The relatively modest margin over prior SOTA in this setting reflects the benchmark’s exact-match evaluation, which deems any deviation from the reference—even semantically equivalent variants—as incorrect. Indeed, qualitative analysis reveals that many of our “failures” produce semantically correct solutions that differ only in surface form.

% Moreover, our method uses substantially fewer resources compared with other baselines. Despite using substantially lighter settings than LATS—expanding only n = 4 nodes per step, sampling k = 10 trajectories, and employing a 3-example few-shot prompt—our method achieves the highest performance in both the large-model and small-model pools.

% 1、给我们的方法取个名字,一个好的名字满足两点:
% (1) 恰好是首字母缩写
% (2)缩写本身构成新单词,通常这个新单词要朗朗上口,最好有好的寓意
% 2、看能否只占一半的页面

% \begin{table}[ht]
%     \centering
%     \caption{Performance Comparison on HotpotQA. }
%     \label{tab:hotpotqa}  % 修正唯一label
%     \begin{tabular}{@{}lc@{}}
%         \toprule
%         \textbf{Method} & \textbf{Exact Match $\uparrow$} \\
%         \midrule
%         CoT \cite{wei2022chaincot}            & 0.34 \\
%         CoT-SC \cite{cot-sc2023}          & 0.38 \\
%         ReAct \cite{yao2023react}          & 0.39 \\
%         Reflexion \cite{shinn2023reflexion}  & 0.51 \\
%         ToT \cite{yao2023tree}            & 0.55 \\
%         RAP \cite{hao2023reasoning}         & 0.60 \\
%         LATS \cite{zhou2024language}        & 0.71 \\
%         Beam Retrieval \cite{zhang2023end}  & 0.73 \\
%         MASTER \cite{gan2025master}         & 0.76 \\
%         \midrule
%         \textbf{SYMPHONY-S} & \textbf{0.59} \\
%         \textbf{SYMPHONY-L} & \textbf{0.79} \\
%         \bottomrule
%     \end{tabular}
%     % \vspace{0.5em}
%     \parbox{\linewidth}{
%     \footnotesize
    
%     }
% \end{table}
%% 就上面这个实验
%Cot、Cot-SC的结果是从LATS中抄过来的,他那标注是GPT-3.5
%ToT、RAP是LATS作者复现的,用的GPT-3.5
%其中React、Reflexion、LATS、BeamRetrieval可以甩锅给MASTER,他说他们拿GPT-4复现了。BeamRetrieval其实是HotPotQA官方榜单的第一名,人家是完整数据集的。




\subsection{Sequential Decision Making:WebShop}

\textbf{Setup.} WebShop~\cite{yao2022webshop} is a simulated e-commerce platform featuring over 1.18 million products and 12,000 natural language queries. Agents must navigate the website using browser-like operations (e.g., search, click, select) to identify items that satisfy user constraints. Performance is measured by average score, which reflects partial attribute satisfaction, and success rate (SR), which reflects full constraint satisfaction.

We compare SYMPHONY against a comprehensive set of baselines reflecting five categories: (1) \textit{Task-native methods} including imitation learning (IL), IL+RL, and Human Expert~\cite{yao2022webshop}; (2) \textit{Supervised models} such as a fine-tuned LLM~\cite{furuta2024multimodal}; (3) \textit{Feedback-driven reasoning}, including  ReAct, Reflexion; (4) \textit{Structured search} methods including LATS and AgentKit; and (5) the \textit{multi-agent framework}: MASTER~\cite{gan2025master}. All baselines were reproduced under consistent settings by~\citet{gan2025master} using GPT-4, ensuring fair comparison.

\textbf{Results.} SYMPHONY outperforms all baseline categories. Compared to task-native and supervised approaches, it achieves higher task completion while requiring no domain-specific training. Against feedback-driven and structured search methods, it exhibits stronger planning efficiency and generalization. Finally, SYMPHONY-L surpasses the multi-agent MASTER, establishing a new performance benchmark. These results underscore SYMPHONY’s adaptability across different task-specific execution environments. %and its ability to unify general reasoning with environment-specific execution under a lightweight, modular design.


% \begin{table}[ht]
% \centering
% \caption{Performance Comparison on WebShop.   % With results grouped into three categories: prompting, RL-based training, and human performance. Our multi-agent framework achieves a comprehensive victory, delivering the highest score and SR across all categories and even surpassing human expert benchmarks.
% }
% \label{tab:webshop_results} % 修正label命名一致性

% \begin{tabular}{lcc}
% \toprule
% \textbf{Method} & \textbf{Score $\uparrow$} & \textbf{Success Rate (SR) $\uparrow$} \\
% \midrule
% IL \cite{yao2022webshop}          & 0.60 & 0.29 \\
% IL+RL \cite{yao2022webshop}       & 0.62 & 0.29 \\
% ReAct \cite{yao2023react}      & 0.54 & 0.32 \\ % 修正53.8→0.54保持数值范围一致
% Reflexion \cite{shinn2023reflexion}) & 0.64 & 0.35 \\
% Fine-tuning \cite{furuta2024multimodal} & 0.68 & 0.45 \\
% AgentKit \cite{wu2024agentkit}     & 0.70 & -- \\  % 使用en-dash表示缺失值
% LATS \cite{zhou2024language}      & 0.76 & 0.38 \\
% MASTER \cite{gan2025master}          & 0.80 & -- \\
% Human Expert \cite{yao2022webshop}    & 0.82 & 0.60 \\
% \midrule
% \textbf{SYMPHONY-S}           & \textbf{0.82} & \textbf{0.56} \\
% \textbf{SYMPHONY-L}           & \textbf{0.88} & \textbf{0.72} \\
% \bottomrule
% \end{tabular}

% % \vspace{0.4em} % 微调表格与说明间距
% \noindent
% \parbox{0.9\linewidth}{ % 添加parbox包装说明文字
% \footnotesize Note: All scores normalized to [0,1] range. SR for AgentKit and MASTER are unavailable due to missing code and metric reporting in their original publications.
% }


% \end{table}

% 其中IL、IL+RL、Human Expert是Webshop这个实验数据集的本身论文中的实验给的
% 这个Fine-tuning是跟着LATS中表格抄过来的
% AgentKit这个MASTER的gan等人也把他列入表格,不过就这个,他说没复现。不过我翻看AgentKit的原论文,就是GPT-4
% 其中Reflexion、LATS、 MASTER都来自于MASTER的gan等人复现的结果,都是GPT-4 




% 任务介绍不要写不相关的内容,尽可能精简
% For a complex decision-making environment with practical applications, we consider WebShop \cite{yao2022webshop}, a simulated e-commerce platform comprising 1.18 million real-world products and 12,000 user-issued natural language instructions. 
% In this environment, an agent must interact with the website via browser-like operations such as search and click to locate and select a product that aligns with the user’s intent. Performance is measured using average score (partial attribute satisfaction) and success rate (full constraint satisfaction). 
% WebShop presents several core challenges: (1) dynamic web interaction, requiring the agent to perform multi-step navigation, handle page transitions, and respond to real-time feedback; (2) complex semantic understanding, where interpreting implicit user preferences—such as finding “a waterproof smartwatch with high cost-effectiveness”—is critical for task success. These factors make WebShop a rigorous benchmark for evaluating agents’ decision-making, planning, and semantic grounding capabilities in open-ended, real-world contexts.

% 要么删掉,要么强调和别人的evaluation protocol一样
% 如果需要强调结果是你复现的,那么应该写在这里
% 我们的参数设置也是写在这里
% As in previous experiments, we apply substantially smaller parameter settings than LATS \cite{zhou2024language} in both the large-model and small-model pools, using only n = 4 node expansions and k = 10 trajectories, with a 1-example one-shot prompt. Furthermore, since AgentKit \cite{wu2024agentkit} and MASTER \cite{gan2025master} report only average score metrics without providing success rate evaluations, We attempted to re implement MASTER - previously the most advanced method - but as they have not yet released the code, we were unable to reproduce their effect.


% 其中IL、IL+RL、Human Expert是Webshop这个实验数据集的本身论文中的实验给的
% 这个Fine-tuning是跟着LATS中表格抄过来的
% AgentKit这个MASTER的gan等人也把他列入表格,不过就这个,他说没复现。不过我翻看AgentKit的原论文,就是GPT-4
% 其中Reflexion、LATS、 MASTER都来自于MASTER的gan等人复现的结果,都是GPT-4 

%中文版:实验所选基线覆盖了从基础模仿学习到前沿推理框架的完整技术演进路径。其中IL、IL+RL和Human Expert为任务原生方案,体现纯行为克隆与混合强化学习的初始模式,以及通过人类专家界定任务的理论上限;Fine-tuning借助领域自适应训练提升模型针对性,探索中等模型结合领域适配的增益极限;AgentKit以GPT-4作为推理引擎,通过结构化动态图推理优化复杂任务的规划执行。其余基线在前文已有提及,其共有的局限性在 WebShop 场景分析中同样存在,且这些基线均经GAN等人利用GPT-4复现验证,确保对比实验的可靠性。

%尽管仅依赖于轻量级、可本地部署的模型,SYMPHONY-S在这两个指标上都实现了与MASTER相当的性能,并且在Score上已经与人类专家一致,这表明即使在严格的资源约束下,我们的框架也能实现高效的目标条件规划,并且已经达到了任务的理论上限。SYMPHONY-L进一步提高了性能,在该基准上实现了最高的报告成功率,超过了所有先前的方法。

% The selected baselines cover the full progression from imitation learning to advanced reasoning. IL, IL+RL, and Human Expert represent task-native approaches, reflecting behavior cloning, hybrid RL, and expert-defined upper bounds. Fine-tuning explores domain-adaptive training for mid-sized models, while AgentKit employs GPT-4 with dynamic graph reasoning for complex planning. The remaining baselines, discussed earlier, share similar limitations in WebShop and have been reproduced by \cite{gan2025master}. using GPT-4, ensuring result reliability.

% For WebShop, we include additional baselines beyond those used in HotpotQA to capture the task’s unique demands. IL, IL+RL, and Human Expert \cite{yao2022webshop} represent task-native approaches, reflecting behavior cloning, hybrid RL, and expert-defined upper bounds. We also include a fine-tuned LM \cite{furuta2024multimodal} trained directly on task data to reflect end-to-end supervised strategies. AgentKit employs GPT-4 with dynamic graph reasoning for complex planning. The remaining methods follow prior categories, these baselines were reproduced by \citet{gan2025master} using GPT-4, ensuring fair comparison. This selection allows us to assess SYMPHONY's generality and effectiveness across both reasoning-driven and environment-specialized baselines.


% \textbf{Result.} SYMPHONY-S achieves performance on par with MASTER in both metrics, despite relying solely on lightweight, locally deployable models. This indicates that our framework enables efficient goal-conditioned planning even under strict resource constraints. SYMPHONY-L further advances performance, achieving the highest reported success rate on this benchmark—surpassing all prior methods, including those based on reinforcement learning and acting-style prompting, as well as reported human-level baselines.

% \textbf{Result.} Despite using only lightweight, locally deployable models, SYMPHONY-S achieves performance comparable to MASTER, and reaches human-level accuracy in Score, demonstrating that our framework enables efficient goal-conditioned planning under strict resource constraints—approaching the task’s theoretical upper bound. SYMPHONY-L further advances performance, achieving the highest reported success rate and surpassing all prior methods.

% Notably, SYMPHONY achieves these results with significantly lower resource demands. Compared to approaches such as LATS \cite{zhou2024language}, we use only 4 node expansions per step, 10 sampled rollouts, and a minimal one-shot prompt, demonstrating strong efficiency-performance tradeoffs across model scales.

% \textbf{Result.} On WebShop, SYMPHONY consistently outperforms all task-native, supervised, feedback-driven, search-based, and multi-agent baselines. The lightweight SYMPHONY-S achieves a Score of 0.82—matching human expert performance—and a Success Rate of 0.56, exceeding both fine-tuned and MCTS-based methods. Scaling to SYMPHONY-L delivers a new state of the art, surpassing MASTER and LATS. These results confirm SYMPHONY’s broad applicability and its superior efficiency–performance trade-off across diverse reasoning and policy paradigms.


% We compare against acting-based prompting methods and RL-based approaches. Performance is gauged using two metrics: an average score, reflecting the percentage of user-specified attributes met by the selected product, and a success rate, indicating the frequency with which the chosen product fulfills all given conditions. Under our framework, small-scale models (SYMPHONY-S) achieve an average score comparable to that of MASTER, with a nearly identical success rate. This indicates that our method, even when using only small models, performs on par with the previous state-of-the-art. When scaled to large models (SYMPHONY-L), our approach achieves the highest success rate on this task, surpassing not only all prior methods but also reported human-level performance. These results demonstrate that our framework enables highly effective sequential decision-making in realistic settings, with performance that is robust to model size and minimally constrained by computational resources.

% Again, our method uses substantially fewer resources. we apply substantially smaller parameter settings than LATS \cite{zhou2024language} in both the large-model and small-model pools, using only n = 4 node expansions and k = 10 trajectories, with a 1-example one-shot prompt.



\begin{figure}[ht]
		\centering
\includegraphics[width=0.95\textwidth]{fig/diversity_performance_plot.pdf}
		\caption{Branch Diversity vs. Task Performance. Bars and left y-axis shows the branch diversity, while lineplot and right y-axis shows the task performance.
        }
        \label{fig:diversity}
\end{figure}


\subsection{Programming:MBPP}

\textbf{Setup.} The Mostly Basic Programming Problems (MBPP) ~\cite{austin2021mbpp} involves multi-step code generation tasks that require condition decomposition, procedural planning, and implementation. Each task provides a description in natural language and a test suite. Success is defined by passing all tests. We follow~\cite{shinn2023reflexion} to evaluate both Python and Rust versions of the datasets using the MultiPL-E compiler suite~\cite{cassano2022multipl}.

Baselines span three major categories: (1) \textit{Single-agent methods}, including GPT-4 and Reflexion~\cite{shinn2023reflexion}, representing the performance ceiling of basic prompting and reactive reasoning; (2) \textit{Multi-agent frameworks}, such as MetaGPT~\cite{hong2023metagpt}, AgentVerse~\cite{chen2023agentverse}, and AgentCoder~\cite{huang2023agentcoder}, which explore different collaboration strategies; and (3) \textit{Search-based approaches}, including RAP~\cite{hao2023reasoning}, LATS~\cite{zhou2024language}, and MASTER~\cite{gan2025master}, which emphasize structured optimization. All baseline results are drawn from or reproduced by~\citet{gan2025master} under consistent backbone and data settings.

\textbf{Results.} SYMPHONY achieves strong performance across all baseline categories. Compared to single-agent methods, it demonstrates superior reasoning depth and planning efficiency. Against multi-agent frameworks, SYMPHONY provides more effective solution search via the introduction of heterogeneous agent pool.  Compared to search-based approaches, it attains state-of-the-art performance in cross-language settings, including Rust, a programming language usually ignored by previous works. These results confirm SYMPHONY’s robustness, generality, and computational efficiency in code generation.



% The Mostly Basic Programming Problems (MBPP) \cite{austin2021mbpp} benchmark features multi-step code generation tasks involving condition decomposition, planning, and implementation, aligning well with the causal reasoning structure of our method. Each task includes a natural language description and test cases; success is defined as generating a complete program that passes all tests. Performance are evaluated with Pass@1 accuracy.  To evaluate cross-language generalization, we follow \cite{shinn2023reflexion} and translate Python tasks into Rust using the MultiPL-E compiler suite \cite{cassano2022multipl}, conducting experiments in both languages. In terms of parameter settings, we follow the configuration used in LATS, employing \(k = 8\) iterations and sampling \(n = 5\) candidate solutions at each expansion step. All prompts are applied in a zero-shot manner.

% The Mostly Basic Programming Problems (MBPP) \cite{austin2021mbpp} benchmark consists of multi-step code generation tasks that require condition decomposition, logical planning, and implementation, aligning closely with the causal reasoning structure targeted by our method. % This makes MBPP an ideal experimental setting for validating reasoning-based decision-making frameworks. In addition, the diversity of problem types in MBPP allows for evaluating the agent’s generalization ability across different scenarios. For these reasons, we select MBPP as the primary benchmark to evaluate our approach.
% Each MBPP task includes a problem description and a set of test cases for validation. The evaluation metric used on MBPP is \comment{buchong}. The system generates a complete program, and a task is considered successfully solved only if the generated code passes all test cases. % The results, including passed and failed tests as well as compiler outputs, are added to the agent’s context as observations, and evaluate it on the real test suite for the pass@1 accuracy evaluation.
% To validate whether the proposed method can be generalized to different programming languages, we follow previous work \cite{shinn2023reflexion} to translate the original Python code into Rust using MultiPL-E compiler suite \cite{cassano2022multipl}, and experiments on both programming languages.  


% In our experiments, we follow the exact dataset configuration used in prior baselines \comment{cite}, but instead of sampling, we evaluate on the entire MBPP test suite.  



% Furthermore, following Reflexion \cite{shinn2023reflexion}, we use the MultiPL-E compiler suite \cite{cassano2022multipl} to convert a subset of MBPP into Rust. MultiPL-E provides lightweight compilers for translating Python benchmarks into 18 different programming languages. We therefore conduct experiments on both Python and Rust versions of MBPP to demonstrate that our method is implementation-language agnostic and applicable to multiple programming environments.


% 其中GPT-4【实验表格的第一行】、Reflexion的结果【实验表格的第四行】,来自于relfexion论文,都是gpt-4 
% 其中GPT-4(CoT) GPT-4(ReAct)  LATS  MetaGPT AgentVerse  MASTER AgentCoder的结果都来自于master的论文
% 其中RAP 来自 LATS,不过人家是GPT-3.5

%中文版 在 MBPP 基准实验中选取这些 baseline,旨在全面覆盖代码生成核心技术路径。单智能体方法(如 GPT - 4 系列、Reflexion)以 GPT - 4 为基座模型,呈现基础推理范式的性能边界,其数据源自 Reflexion 相关工作。多智能体框架(MetaGPT、AgentVerse、AgentCoder)展示不同协作模式的优劣;搜索优化类(RAP、LATS、MASTER)体现搜索技术对代码生成的改进探索。此外,其余 baseline 均由 Gan 等人基于 GPT - 4 复现,数据统一来源于 Gan 等人的复现工作,确保对比实验在一致的基座模型与数据标准下进行,提升结果的可靠性与可比性。
% The MBPP baselines are selected to cover core technical paths in code generation. Single-agent methods (GPT-4, Reflexion) reflect the performance limits of basic reasoning with GPT-4 as the backbone, with data sourced from \cite{shinn2023reflexion}. . Multi-agent frameworks (MetaGPT\cite{hong2023metagpt}, AgentVerse\cite{chen2023agentverse}, AgentCoder\cite{huang2023agentcoder}) highlight varying collaboration strategies, while search-based methods (RAP \cite{hao2023reasoning}, LATS \cite{zhou2024language}, MASTER \cite{gan2025master}) explore optimization through structured exploration. Remaining baselines are reproduced by \cite{gan2025master}. using GPT-4 under a unified data and model setting, ensuring fair and reliable comparison.


%中文版 实验结果充分凸显了SYMPHONY方法的卓越优势。单智能体与搜索优化类方法中,诸多基线模型未能超越 GPT-4,这表明在代码生成任务里,复杂推理与规划策略的实际效用相对有限。而 SYMPHONY 仅凭借轻量级开源模型及固定的测试时间规划策略,便展现出强大性能,达到了 SOTA 水平。与诸多未涵盖 Rust 语言的基线方法不同,本研究评估覆盖了语言设置,进一步证实了 SYMPHONY 语言无关的有效性。更关键的是,这一成果是在显著降低资源使用率的前提下实现的。
% \textbf{Result.} As shown in the results table, SYMPHONY achieves the highest overall accuracy on both Python and Rust variants. While prior methods—including GPT-4—struggle to solve all tasks reliably, SYMPHONY demonstrates strong performance using only lightweight open-source models and a fixed test-time planning strategy. Unlike many baselines that exclude Rust, our evaluation includes both language settings and confirms SYMPHONY's language-agnostic effectiveness. Importantly, this is achieved with significantly lower resource usage.

% \textbf{Result.} Experimental results highlight the strong advantages of the SYMPHONY method. Many baselines in the single-agent and search-based categories fail to outperform GPT-4, suggesting limited practical gains from complex reasoning or planning in code generation tasks. In contrast, SYMPHONY achieves state-of-the-art performance using only lightweight open-source models and a fixed-time planning strategy. Unlike prior baselines that omit Rust, our evaluation includes it, further demonstrating SYMPHONY's language-agnostic effectiveness. Crucially, these results are achieved with significantly reduced resource consumption.


% GPT-4 does not achieve perfect scores on either the Python or Rust variants of the MBPP benchmark, suggesting that relevant knowledge was not fully encoded during pretraining. In contrast, our method—using only three open-source small models—outperforms all baselines on this dataset via test-time scaling alone, without any fine-tuning or modification of model parameters. Although most prior works omit Rust in their evaluations, our method—following the protocol of Reflexion\cite{shinn2023reflexion}—includes the Rust subset and attains a highest score. This finding confirms that test-time scaling not only fortifies the model’s logical reasoning but also empowers it to solve problems with standardized code in a language-agnostic manner. In terms of parameter settings, we follow the configuration used in LATS, employing \(k = 8\) iterations and sampling \(n = 5\) candidate solutions at each expansion step. All prompts are applied in a zero-shot manner.

% \begin{table}[ht]
% \centering
% \caption{Performance Comparison on MBPP. }
% \label{tab:mbpp_results} % 修正label命名一致性

% \begin{tabular}{lcc}
% \toprule
% \textbf{Method} & \textbf{Python} & \textbf{Rust} \\
% \midrule
% GPT-4 \cite{shinn2023reflexion} & 0.800 & 0.710 \\
% GPT-4(CoT) \cite{gan2025master}  & 0.683 & -- \\
% GPT-4(ReAct) \cite{yao2023react}   & 0.710 & -- \\
% Reflexion \cite{shinn2023reflexion}  & 0.771 & 0.754 \\
% RAP \cite{hao2023reasoning}     & 0.714 & -- \\  
% LATS \cite{zhou2024language}        & 0.811 & -- \\
% MetaGPT  \cite{hong2023metagpt}     & 0.877 & -- \\
% AgentVerse  \cite{chen2023agentverse} & 0.890 & -- \\
% MASTER  \cite{gan2025master}          & 0.910 & -- \\
% AgentCoder \cite{huang2023agentcoder} & 0.918 & -- \\

% % 其中GPT-4【实验表格的第一行】、Reflexion的结果【实验表格的第四行】,来自于relfexion论文,都是gpt-4 
% % 其中GPT-4(CoT) GPT-4(ReAct)  LATS  MetaGPT AgentVerse  MASTER AgentCoder的结果都来自于master的论文
% % 其中RAP 来自 LATS,不过人家是GPT-3.5

% \midrule
% \textbf{SYMPHONY-S}           & \textbf{0.927} & \textbf{0.946} \\
% \textbf{SYMPHONY-L}           & \textbf{0.965} & \textbf{0.974} \\
% \bottomrule
% \end{tabular}

% % \vspace{0.4em} % 微调表格与说明间距
% \parbox{0.9\linewidth}{ % 添加parbox包装说明文字
% \footnotesize All scores are normalized to the [0, 1] range. A dash (–) indicates that the corresponding method does not provide a score on this dataset.}
% \end{table}


\subsection{Diversity Analysis} 
% Diversity in candidate expansions plays a critical role in MCTS, as it directly influences the range and quality of reasoning paths explored. In SYMPHONY, this diversity is modulated by the composition of the agent pool. Different combinations of agents induce different reasoning behaviors, shaping the structure of the search tree. To evaluate this effect, we analyze how varying agent combinations affect both overall task performance and node-level diversity on all three tasks using SYMPHONY-S.

Branch diversity plays a crucial role in effective search. To assess its impact, we evaluate how different agent pool configurations affect both task performance and branch diversity across all three tasks using SYMPHONY-S. The expansion width is fixed at 4, and each node's candidate branches are categorized by output uniqueness: (a) \textbf{4-Unique}: all branches distinct, (b) \textbf{3-Unique}, (c) \textbf{2-Unique}, and (d) \textbf{1-Unique}: all branches identical. Higher frequencies of 3-Unique and 4-Unique indicate more diversified and informative exploration.

As shown in Figure~\ref{fig:diversity}, increasing agent heterogeneity, from single-agent to pairwise and full-trio configurations (e.g., Qwen+Mistral+Llama), leads to a substantial rise in 4-Unique expansions. On MBPP, for example, this proportion exceeds 80\% under the full ensemble, compared to under 20\% in the single-agent setting. This increase in structural diversity strongly correlates with improved accuracy, with SYMPHONY outperforming single-agent baselines by over 30\% on MBPP and showing similar trends on HotpotQA and WebShop. These findings highlight the critical role of model-level diversity in enhancing search coverage and reasoning robustness.




We also experimented with alternative diversity-promoting strategies such as adversarial prompting and temperature scaling, but found their effect to be marginal. Detailed comparisons are included in Appendix~\ref{appendix:diversity}.

% Diversity in candidate expansions plays a critical role. We evaluate how varying agent combinations affect both overall task performance and node-level diversity on all three tasks using SYMPHONY-S. We fix the expansion width to 4 and, for each node, categorize its four generated branches based on the number of unique outputs: (a) \textbf{4-Unique}: all four branches differ; (b) \textbf{3-Unique}: three distinct branches with one duplication; (c) \textbf{2-Unique}: two distinct branches, each repeated; (d) \textbf{1-Unique}: all branches are identical. 
% A higher frequency of 3-Unique and 4-Unique expansions indicates broader and more differentiated exploration.

% As shown in Figure~\ref{fig:diversity}, increasing the heterogeneity of the agent pool, from single-agent to pairwise and full-trio configurations (e.g., Qwen+Mistral+Llama), leads to a clear increase in 4-Unique expansions. On MBPP, for instance, the proportion of 4-Unique expansions surpasses 80\% under the full ensemble, compared to below 20\% in the single-agent setting. This increase in structural diversity is strongly correlated with improved task accuracy, with SYMPHONY outperforming single-agent baselines by over 30\% on MBPP and showing similar gains on HotpotQA and WebShop. These results underscore the importance of model-level diversity in enabling broader search coverage and more robust reasoning behavior.

% We also tested alternatives for diversity enhancement such as adversarial prompting and temperature scaling, but observed limited gains. Detailed results are provided in Appendix A.3.



% Diversity is a critical factor in MCTS, as exploring a wide range of paths increases the likelihood of discovering high-reward solutions. One of our key contributions is the introduction of a heterogeneous model pool, which fundamentally enhances the diversity of child node expansions in MCTS. To systematically evaluate the effectiveness of this design on the HotPotQA benchmark under consistent settings, we examine the following three aspects:  
% 1). Model Pool Ablation: Removing the heterogeneous pool to assess the contribution of multi-model diversity.  
% 2). Varying Pool Size: Investigating how different combinations and numbers of heterogeneous models impact search quality.  
% 3). Alternative Diversity Enhancements: Incorporating methods such as adversarial prompting and temperature variation to further increase model response diversity.

% The first and second ablation settings are primarily designed to evaluate the effectiveness and diversity benefits introduced by our proposed heterogeneous model pool. To this end, we conduct analysis from two complementary perspectives: (1) task-level performance to validate the overall effectiveness of heterogeneous collaboration, and (2) structural analysis of node-level diversity to quantify the heterogeneity across search paths.

% Given that our expansion parameter is set to 4 (see Section 5.7 for details), each layer in the constructed search tree consists of four child nodes. We categorize the node composition of each layer into four types:(i) Fully Diverse: All four nodes are generated by different model types.(ii) Three-Way Diverse: Three distinct model types are present.(iii) Two-Way Diverse: Two distinct types among the four nodes.(iv) Homogeneous: All nodes are generated by the same model type.We then compute the proportion of each category across all layers and tasks. A higher frequency of diverse node layers reflects greater model-level exploration and richer reasoning trajectories. This analysis enables us to quantify how the heterogeneity of the model pool contributes to increased search diversity.

% Figure \ref{fig:diversity} presents the impact of model combination on node diversity and task performance across three benchmarks: HotPotQA, WebShop, and MBPP. We observe a consistent pattern: as the model configuration evolves from homogeneous (e.g., Llama, Qwen, Mistral) to heterogeneous ensembles (e.g., Qwen+Llama, Llama+Mistral, All Three), the proportion of Fully Diverse nodes substantially increases—reaching over 80\% in the final configuration for MBPP. This structural diversity is strongly correlated with improved task accuracy. Specifically, on MBPP, the performance improves from ~60\% with Llama alone to over 90\% with the full model ensemble, marking a > 30 percentage point gain. Similar trends are observed in HotPotQA and WebShop, where full model combinations outperform single-model setups by wide margins (e.g., +25\% in HotPotQA). These results provide strong empirical evidence that incorporating architectural and reasoning diversity via model combinations not only improves robustness but also consistently outperforms powerful individual LLMs.

%要改,还没改。总结图图中的结果
% The key factor in both the first and second ablation studies is the number of heterogeneous models in the pool. When evaluated in isolation, each of the three open-source small models—despite our framework’s auxiliary components—can underperform the Base LM \cite{zhou2024language}. Next, we form all three possible pairwise combinations (pool size = 2). In this configuration, our method matches the performance of GPT-4 augmented with ReAct \cite{yao2023react} and outperforms Reflexion \cite{shinn2023reflexion}. Finally, when using the full pool of three heterogeneous models, our small-model framework not only substantially exceeds GPT-4’s accuracy but also approaches the performance of RAP \cite{hao2023reasoning}, despite RAP’s reliance on a large-scale backbone. As the number of heterogeneous models in the ensemble increases, the performance improves significantly. These results highlight the critical importance of the number of heterogeneous models in the pool, as increased diversity significantly enhances the search effectiveness of MCTS.


%  小模型实验结果表格
% \begin{table}
%     \centering
%     \caption{Ablation study on the number of heterogeneous models in the model pool on HotPotQA. "Llama" denotes Llama-3.1-8B-Instruct, "Qwen" denotes Qwen2.5-7B-Instruct-1M, "Mistral" denotes Mistral-7B-Instruct-v0.3, and "All" represents the ensemble of all three models.}
%     \label{tab:abalation_study}  
%     \begin{tabular}{@{}lc@{}}
%         \toprule
%         \textbf{Model ensemble} & \textbf{EM $\uparrow$} \\
%         \midrule
%         Llama(cite)     & 0.27 \\
%         Qwen(cite)      & 0.30 \\
%         Mistral         & 0.32 \\
%         Qwen+Llama      & 0.47 \\
%         Mistral+Qwen    & 0.50 \\
%         Mistral+Llama   & 0.52 \\
%         All               & 0.59 \\
%         % \midrule
%         % \textbf{Ours:LLMs-S} & \textbf{0.59} \\
%         % \textbf{Ours:LLMs-L} & \textbf{0.79} \\
%         \bottomrule
%     \end{tabular}
%     \vspace{0.5em}
%     \parbox{\linewidth}{
%     \footnotesize
    
%     }
% \end{table}




\subsection{Efficiency  and Cost Analysis}
% To assess the practicality of SYMPHONY, we analyze its efficiency and cost from two complementary perspectives: the size of the search tree required for reasoning, and the computational cost associated with model inference. These two factors jointly determine the real-world feasibility of deploying an LLM-based planning framework.

% We begin by examining the search configuration needed to achieve strong performance. While prior methods such as LATS adopt a large trajectory budget ($K=50$) and a wider expansion width ($n=5$), SYMPHONY achieves comparable or better results with significantly smaller values ($K=10$, $n=4$). This suggests that SYMPHONY requires a much smaller search space to reach high-quality solutions.

% To further quantify this advantage, we evaluate the average number of node expansions needed in MCTS to arrive at a correct answer. As shown in Table~\ref{tab:efficiency_hotpotqa}, SYMPHONY consistently requires fewer expansions than baseline methods, even outperforming LATS when using only one-fifth of its trajectory budget. These results underscore SYMPHONY’s superior sample efficiency and its ability to guide search more effectively.

% Beyond search efficiency, we also consider inference cost, which is a critical bottleneck in LLM-based systems. Unlike previous approaches that rely entirely on high-cost models such as GPT-4, SYMPHONY-L incorporates a heterogeneous agent pool to reduce dependence on any single model. As shown in Figure~\ref{fig:api_usage}, SYMPHONY-L invokes GPT-4 in just 40\% of calls, while delegating the remaining calls to less expensive yet capable models—achieving stronger performance than GPT-4-only pipelines. Token-level cost breakdowns provided in Appendix~\ref{appendix:cost} further confirm the substantial savings.

% Taken together, these results demonstrate that SYMPHONY not only reduces the size of the search tree but also lowers the cost of constructing it—achieving efficient, scalable, and cost-aware planning with LLM-based agents.

\begin{figure}[ht]
\centering
\begin{minipage}{0.39\textwidth}
    \centering
    \captionof{table}{Comparison of the search tree size on HotpotQA.}
    \label{tab:efficiency_hotpotqa}
    \resizebox{\linewidth}{!}{%
    \begin{tabular}{l|ccc}
        \toprule
        \textbf{Method} & \textbf{K} & \textbf{HotpotQA $\uparrow$ } & \textbf{\#Nodes $\downarrow$ }    \\
        \midrule
        ToT      & 10 & 0.34 & 33.97 \\
        RAP & 10 & 0.44 & 31.53 \\ 
        LATS & 10 & 0.44 & 28.42 \\
        \midrule
        ToT           & 50 & 0.49 & 84.05 \\
        RAP           & 50 & 0.54 & 70.60 \\
        LATS          & 50 & 0.61 & 66.65 \\
        
        \midrule
        \textbf{SYMPHONY-S}   & \textbf{10} & \textbf{0.59}  & \textbf{16.39} \\
        \textbf{SYMPHONY-L}   & \textbf{10} & \textbf{0.79}  & \textbf{9.47}\\
        \bottomrule
        \end{tabular}
    }
    
\end{minipage}
\hfill
\begin{minipage}{0.59\textwidth}
    \centering
    \includegraphics[width=\textwidth]{fig/consume400.drawio.png}
    \captionof{figure}{Comparison of model invocation frequency and final performance on HotpotQA. %Performance of MASTER and LATS using GPT-4 is reported by \cite{gan2025master} 
    }\label{fig:api_usage}
    \label{fig:cost_accuracy}
\end{minipage}
\end{figure}


To evaluate SYMPHONY’s practicality, we analyze two key aspects: the size of the search tree and the cost of model inference—both crucial to real-world deployment.

Compared to methods like LATS, which use a large trajectory budget ($K=50$) and wider expansion ($n=5$) on HotpotQA and WebShop, SYMPHONY achieves comparable or better results with much smaller values ($K=10$, $n=4$), indicating a more compact search process.

We further assess efficiency by measuring average node expansions in MCTS on HotpotQA. As shown in Table~\ref{tab:efficiency_hotpotqa}, SYMPHONY consistently requires fewer expansions and even outperforms LATS with a fraction of its search budget, reflecting strong sample efficiency.

In terms of cost, SYMPHONY-L reduces reliance on expensive models by using a heterogeneous agent pool. As shown in Figure~\ref{fig:api_usage}, GPT-4 is used in only 40\% of calls, yet SYMPHONY-L still outperforms GPT-4-only baselines. Token-level cost details are provided in Appendix~\ref{appendix:cost}.

Together, these results show that SYMPHONY achieves efficient and cost-effective planning through smaller search trees and more economical model usage.




% It should be noted that SYMPHONY achieve comparable or even superior performance with much fewer resources. For example, on HotpotQA and WebShop, LATS uses an upper bound of $K=50$ trajectories to search for solutions, while that of SYMPHONY is is set to $K=10$. Moreover, the expansion width $n$ for LATS is set to $5$ for all tasks while that for SYMPHONY is set to $4$. 

% % 任务性能 vs 资源消耗(节点数)
% We look further into this issue by measuring average node expansions in MCTS on HotpotQA under varying trajectory budgets. Baselines include ToT~\cite{yao2023tree}, RAP~\cite{hao2023reasoning}, and LATS~\cite{zhou2024language}.

% As shown in Table~\ref{tab:efficiency_hotpotqa}, SYMPHONY consistently achieves higher accuracy with fewer expansions. Notably, it outperforms LATS with $K=10$ trajectories, even when LATS uses $K=50$, highlighting its superior sample efficiency.




% \comment{Talk about n, k for all three datasets here}
% % HotpotQA
% By evaluating diverse reasoning paths with just 4 expansions per step, 10 sampled trajectories, and a compact 3-shot prompt, SYMPHONY outperforms heavy-sampling methods such as LATS, while maintaining high efficiency.
% % WebShop
% Notably, SYMPHONY achieves these results with significantly lower resource demands. Compared to approaches such as LATS \cite{zhou2024language}, we use only 4 node expansions per step, 10 sampled rollouts, and a minimal one-shot prompt, demonstrating strong efficiency-performance tradeoffs across model scales.
% % MBPP
%  Our parameter configuration mirrors LATS~\cite{zhou2024language}, using $K=8$ iterations and $n=5$ candidate samples per step.
%MBPP的K=8,LATS也等于8,zero-shot
%HotpotQA和webshop的K都等于10,LATS等于50,QA是3-shot,MBPP是1-shot
% shot都和他一样
%我们n都是4,LATS是5
% MCTS里的UCT参数是参照LATS设置为2



% To assess the efficiency of SYMPHONY, we compare the search tree size required to reach a correct answer. Specifically, we measure the average number of node expansions in MCTS across different methods on HotpotQA, under varying trajectory settings. Baselines include ToT \cite{yao2023tree}, RAP \cite{hao2023reasoning}, and LATS \cite{zhou2024language}, all of which provide publicly available implementations.

% As shown in Table~\ref{tab:efficiency_hotpotqa}, SYMPHONY achieves the highest accuracy while requiring the fewest node expansions across all settings. Under identical configurations, it significantly reduces search cost compared to LATS without compromising performance. Notably, even with only $k=10$ trajectories, SYMPHONY surpasses LATS results obtained with $k=50$, demonstrating superior sample efficiency and search effectiveness.




% 大模型API价格 vs 调用情况百分比分析
% As a tree-based framework, our method encodes problem-solving trajectories as nodes in a search tree, where total node count directly reflects computational cost and search efficiency. Because parameter settings—such as expansion breadth and trajectory count—govern how rapidly the tree grows, eliminating redundant exploration is crucial. To this end, we introduce semantic node merging to collapse equivalent states and an early-termination mechanism to prune completed branches, all while leveraging enhanced MCTS-driven diversity to guide the search toward correct solution paths more rapidly.
% We benchmark our efficiency against LATS for two reasons: 1). LATS is a prototypical tree-based, MCTS-driven algorithm that demonstrates strong performance relative to similar frameworks; 2). LATS reports average node counts under the same parameter configurations used by Tree-of-Thought (ToT) \cite{yao2023tree} and RAP \cite{hao2023reasoning}, enabling a direct and fair comparison.

% Given that our primary objective is cost reduction and efficiency improvement, we do not consider the larger parameter configurations used in LATS; instead, we compare only against its minimal setting. Under identical settings, our method requires substantially fewer node expansions than LATS without sacrificing accuracy, thereby underscoring its superior efficiency.Moreover, even with a reduced setting of k = 10, our method outperforms LATS in task performance, surpassing their results obtained with k = 50.





% \begin{table}[ht]
% \centering
% \caption{Comparison of the search tree size on HotPotQA. }
% \label{tab:efficiency_hotpotqa} % 修正label命名一致性

% \begin{tabular}{l|ccc}
% \toprule
% \textbf{Method} & \textbf{k} & \textbf{HotPotQA $\uparrow$ } & \textbf{\#Nodes $\downarrow$ }    \\
% \midrule
% ToT      & 10 & 0.34 & 33.97 \\
% RAP & 10 & 0.44 & 31.53 \\ 
% LATS & 10 & 0.44 & 28.42 \\
% \midrule
% ToT           & 50 & 0.49 & 84.05 \\
% RAP           & 50 & 0.54 & 70.60 \\
% LATS          & 50 & 0.61 & 66.65 \\

% \midrule
% \textbf{SYMPHONY-S}   & \textbf{10} & \textbf{0.59}  & \textbf{16.39} \\
% \textbf{SYMPHONY-L}   & \textbf{10} & \textbf{0.79}  & \textbf{9.47}\\
% \bottomrule
% \end{tabular}

% % \vspace{0.4em} % 微调表格与说明间距
% \parbox{0.9\linewidth}{ % 添加parbox包装说明文字
% \footnotesize }
% \end{table} 

% Previous baseline methods often place stringent demands on model capability, leading them to primarily rely on high-end models such as GPT-4. To ensure a fair and rigorous comparison, we also include GPT-4 as one of our backbone models. However, to better reflect real-world deployment scenarios and encourage architectural diversity, we additionally incorporate models from different vendors and with varying design philosophies, including both proprietary and open-source alternatives.

% By analyzing the API pricing and actual usage distribution for each method, we provide a concrete estimation of the computational cost incurred. Remarkably, our approach not only achieves state-of-the-art performance but does so under significantly lower resource requirements. This demonstrates that our method is not only effective but also highly cost-efficient, making it well-suited for practical applications where scalability and budget are critical factors.

% Prior methods often rely exclusively on GPT-4, which offers strong performance but comes with high computational cost. SYMPHONY adopts a more cost-efficient strategy by incorporating a mix of API-accessible models with varying pricing and capabilities, reducing reliance on the most expensive options while maintaining architectural diversity. To assess this design, we analyze the proportion of different agent usage within SYMPHONY-L and compare overall task performance.

% Figure \ref{fig:api_usage} reports the experiment results. Even with 100\% GPT-4 usage, these baselines do not match our performance: SYMPHONY-L achieves superior accuracy while invoking GPT-4 for only 40\% of all calls. This represents a substantial reduction in computational cost, demonstrating that our approach delivers strong results without reliance on an expensive, high-capacity model.

 

% While prior methods rely solely on GPT-4, SYMPHONY-L employs a heterogeneous mix of API-based models to reduce cost without sacrificing performance.

% As shown in Figure~\ref{fig:api_usage}, SYMPHONY-L outperforms GPT-4-only baselines while using GPT-4 in just 40\% of calls. This highlights the effectiveness of our budget-aware design in achieving strong results with significantly lower computational cost. A detailed comparison of token usage is provided in Appendix~\ref{appendix:cost}.





% In prior work, some baselines mixed GPT-4 and GPT-3.5, but MASTER consistently employed GPT-4 as the backbone to reproduce the methods listed in Table X; accordingly, we directly cite their GPT-4–based results. We then compare (a) the proportion of model invocations in our framework and (b) the resulting HotPotQA performance. Even with 100\% GPT-4 usage, these baselines do not match our performance: our method achieves superior accuracy while invoking GPT-4 for only 40\% of all calls. This represents a substantial reduction in computational cost, demonstrating that our approach delivers strong results without reliance on an expensive, high-capacity model.

% \begin{table}[ht]
% \centering
% \caption{Comparison of API pricing across different models, with both input and output costs measured in USD per million tokens.}
% \label{tab:api_cost} % 修正label命名一致性

% \begin{tabular}{l|ccc}
% \toprule
% \textbf{Model} & \textbf{Input cost (per million tokens/ \textdollar)} & \textbf{Output cost (per million tokens/\textdollar)}    \\
% \midrule
% GPT-4        & 30.00 & 60.00 \\
% Qwen‑Max     & 0.33 & 1.32  \\ 
% DeepSeek-V3  & 0.27 & 1.10  \\
% \bottomrule
% \end{tabular}

% \vspace{0.4em} % 微调表格与说明间距
% \parbox{0.9\linewidth}{ % 添加parbox包装说明文字
% \footnotesize }
% \end{table} 


% \begin{table}[ht]
% \centering
% \caption{Comparison of model invocation frequency and final performance on HotPotQA, highlighting the cost-effectiveness of each method.}
% \label{tab:api_distribution} % 修正label命名一致性

% \begin{tabular}{l|cccc}
% \toprule
% \textbf{Method} & \textbf{GPT-4} & \textbf{Qwen-Max} & \textbf{DeepSeek-V3} & \textbf{HotPotQA $\uparrow$}    \\
% \midrule
% ReAct           & 100\% & -- & -- & 0.42 \\
% Reflexion       & 100\% & -- & -- & 0.51\\ 
% LATS         & 100\% & -- & -- & 0.71 \\
% Beam Retrieval   & 100\%  & -- & -- & 0.73 \\
% MASTER          & 100\% & -- & --   & 0.76 \\
% \midrule
% \textbf{Ours:LLMs-L}   & \textbf{35\%} & \textbf{33\%}  & \textbf{32\%} & \textbf{0.79} \\
% \bottomrule
% \end{tabular}

% \vspace{0.4em} % 微调表格与说明间距
% \parbox{0.9\linewidth}{ % 添加parbox包装说明文字
% \footnotesize }
% \end{table} 

% 上面的价钱和调用次数,画成 y轴为EM分数, x轴为各种方法的矩形图,然后再把调用情况用饼状图表现
% \begin{figure}[ht]
% 		\centering
% 		\includegraphics[width=\textwidth]{fig/consume.pdf}
%         % \vspace{-20pt}
% 		\caption{Comparison of model invocation frequency and final performance on HotPotQA. Performance of MASTER and LATS using GPT-4 is reported by \cite{zhou2024language} } \label{fig:api_usage}
% \end{figure}


\subsection{Ablation Study and Hyperparameter Tuning}
To evaluate the impact of SYMPHONY’s core components, we perform a series of ablation studies by selectively disabling key modules, including UCB-based agent scheduling, pool-wise memory sharing, and EMCS scoring. As presented in Table~\ref{tab:abalation_study_moudle}, removing any of these components leads to consistent performance degradation across tasks. These results underscore the effectiveness of dynamic agent scheduling, collaborative memory sharing, and uncertainty-aware scoring in enhancing overall system performance.

We further conduct hyperparameter tuning for the UCB exploration coefficient $\alpha$ used in agent scheduling, as well as the MCTS parameters $n$ and $K$, which jointly determine the search strategy and computational efficiency. Detailed analyses and results are provided in Appendix~\ref{appendix:alpha} and Appendix~\ref{appendix:parameter}. An extended analysis of architectural robustness under varying agent compositions and noise perturbations is also included in Appendix~\ref{appendix:robustness}, offering deeper insights into SYMPHONY’s stability and adaptability. Case studies are included in Appendix~\ref{appendix:case}.

% To assess the contribution of each core component in the SYMPHONY framework, we removed core reasoning and coordination modules.

% \textbf{Without UCB-Based Agent Scheduling.} We replace the adaptive UCB-based scheduler with a fixed round-robin selection that invokes agents in a predetermined sequence. While this randomization still allows all agents to participate, it removes the structured balancing between exploration and exploitation. This ablation examines whether intelligent scheduling contributes to more efficient reasoning by dynamically adjusting agent allocation beyond uniform participation.

% \textbf{Without Pool-wise memory sharing.} We disable the reflection mechanism that allows agents to analyze and learn from their past failures through verbal self-evaluation. This prevents intra-agent iterative improvement and tests the importance of reasoning-level feedback in guiding future actions.

% \textbf{Without EMCS.}: We remove the  Entropy-Modulated Confidence Scoring (EMCS) component and instead use naive majority voting for path selection. This ablation assesses the value of semantically-informed comparative evaluation in resolving ambiguity and selecting globally coherent solutions.

% Results across benchmarks (Table \ref{tab:abalation_study_moudle}) consistently demonstrate that removing any of these components leads to noticeable performance degradation. Notably, the absence of reflection results in brittle and non-adaptive reasoning trajectories, while a homogeneous agent pool fails to handle complex, multi-faceted queries due to the lack of strategic diversity. Without EMCS, the final answer selection becomes unstable when semantically similar distractors are present. Furthermore, removing the UCB-based agent scheduler eliminates adaptive coordination across agents, leading to inefficient exploration patterns and reduced overall effectiveness in balancing complementary reasoning strategies.


% To assess the contribution of SYMPHONY’s core components, we conduct module-leval ablation studies by disabling UCB-based agent scheduling, pool-wise memory sharing, and EMCS scoring. As shown in Table~\ref{tab:abalation_study_moudle}, performance drops consistently across all settings, confirming the importance of dynamic scheduling, memory sharing, and uncertainty-aware evaluation in SYMPHONY.

% Removing the UCB scheduler reduces coordination efficiency, leading to suboptimal agent allocation. Disabling memory sharing prevents agents from adapting based on prior failures, resulting in brittle reasoning. Replacing EMCS with majority voting degrades answer selection when semantically similar candidates are present.



% We also tune the hyper-parameter $\alpha$ in UCB-based agent scheduling and MCTS coefficient $c$, which are provided in Appendix \ref{appendix:alpha} and Appendix ~\ref{appendix:parameter}, respectively.

\begin{table}
\centering
\caption{Ablation Study. }
\label{tab:abalation_study_moudle}
\resizebox{0.6\linewidth}{!}{
\begin{tabular}{@{}lccc@{}}
    \toprule
Method &HotpotQA(EM)$\uparrow$
&WebShop(SR)$\uparrow$
&MBPP(pass@1)$\uparrow$\\
    \midrule
SYMPHONY-S    
& \textbf{0.59} & \textbf{0.56} & \textbf{0.927} \\
w/o Agent Scheduling  & 0.51 & 0.48 & 0.906\\
w/o Memory Sharing  & 0.45 & 0.46 & 0.871\\
w/o EMCS      & 0.51 & 0.49 & 0.892\\

    % \midrule
    % \textbf{Ours:LLMs-S} & \textbf{0.59} \\
    % \textbf{Ours:LLMs-L} & \textbf{0.79} \\
    \bottomrule
\end{tabular}
}
% \vspace{0.5em}
\end{table}
























% \section{Limitation} \label{sec:limit}
% One current limitation of our work lies in the reliance on manually tuned hyperparameters, which may vary across different tasks to achieve optimal performance and cost-efficiency.  It highlights the need for more robust and automated hyperparameter optimization strategies. We view this as a promising direction for future work and expect that integrating adaptive tuning methods will further enhance the generality and usability of our approach.

% Our framework has several limitations. First, it heavily relies on the LLM's ability to accurately assess the current reasoning state, which contributes to the performance gap observed between SYMPHONY-S and SYMPHONY-L. In addition, users are required to configure certain hyperparameters, such as the maximum number of expansions and trajectories. The optimal values of these parameters may vary across different tasks.

\section{Conclusion and Future Work}
We present SYMPHONY, a multi-agent planning framework that combines MCTS with a diverse pool of language models. By leveraging model heterogeneity and incorporating adaptive scheduling, entropy-modulated confidence scoring, and memory sharing, SYMPHONY improves both search diversity and planning effectiveness. Experiments across multiple benchmarks show consistent gains in accuracy and efficiency. Importantly, SYMPHONY performs well even with models that run on consumer-grade hardware, making it a practical and scalable solution.

Future research will focus on extending SYMPHONY to unstructured or noisy environments, reducing reliance on manually tuned hyperparameters, and integrating fairness and robustness considerations into the planning process. We also plan to explore more efficient memory architectures to support scalable, continual adaptation.

\section*{Acknowledgements}
This work is supported by the Joint Key Project of National Natural Science Foundation of China (U23A20298), Yunnan Fundamental Research Project (202501AT070231), Open Project Program of Yunnan Key Laboratory of Intelligent Systems and Computing (ISC24Y03), and Professional Degree Graduate Practice Innovation Project of Yunnan University (ZC-252514097).



% We introduced SYMPHONY, a synergistic multi-agent planning framework that integrates Monte Carlo Tree Search with a heterogeneous pool of language models. By moving beyond the conventional single-model paradigm, SYMPHONY enhances exploration through diversity-aware rollouts that reduce redundancy and mitigate model-specific biases. The framework further incorporates UCB-based model scheduling, entropy-modulated confidence scoring, and pool-wise memory sharing to enable more effective coordination, uncertainty-aware evaluation, and reflective learning across agents. Experimental results across multiple benchmark datasets demonstrate that SYMPHONY consistently improves both accuracy and planning efficiency. Notably, the framework achieves competitive or superior performance even when using models that can be deployed on consumer-grade GPUs, highlighting its practical value for scalable, accessible autonomous agent design. Future directions include exploring theoretical underpinnings of diversity-driven planning, extending to multimodal agents, and reducing reliance on manually set hyperparameters.



% \section*{References}
\bibliographystyle{plainnat}
\bibliography{reference}





%%%%%%%%%%%%%%%%%%%%%%%%%%%%%%%%%%%%%%%%%%%%%%%%%%%%%%%%%%%%





\newpage
\appendix

%% The Appendices part is started with the command \appendix;
%% appendix sections are then done as normal sections
\appendix
% \onecolumn
\subsection{Prompt Template}
\label{app1}

Figs. \ref{appendix:prompt1}-\ref{appendix:prompt3} (due to the page length, we break it into three parts) show the prompt design for the information extraction in the context of construction project scheduling. 

\begin{figure}[t]
    \vspace{-4.5cm}
    \centering
    \begin{tcolorbox}[colback=gray!10!white, colframe=gray!50!gray, halign=left, boxrule=0.5pt, left=1mm, right=1mm, top=1mm, bottom=1mm]
    \fontsize{8pt}{8pt}\selectfont
    SYSTEM PROMPT: You are a project management assistant specializing in construction scheduling analysis. Your task is to analyze text descriptions of project changes and extract structured information about task relation changes in a construction project.
    \vspace{8pt}
    
    CONTEXT\\
    The project involves the following tasks and their relationships: \\
    \vspace{3pt}
    Task ID \textbar{} Predecessor \textbar{} Duration \textbar{} Description \textbar{} Robot Type
    \begin{itemize}
    \item T1 \textbar{} - \textbar{} 0.25 \textbar{} Move Electrical Conduit \textbar{} R1
    \item T2 \textbar{} - \textbar{} 0.25 \textbar{} Move Window Frame \textbar{} R1
    \item T3 \textbar{} - \textbar{} 0.25 \textbar{} Move Window \textbar{} R1
    \item T4 \textbar{} - \textbar{} 0.25 \textbar{} Move Duct Structural Materials \textbar{} R1
    \item T5 \textbar{} - \textbar{} 0.25 \textbar{} Move Duct \textbar{} R1
    \item T6 \textbar{} - \textbar{} 0.5 \textbar{} Drill Wall \textbar{} R4 or R2
    \item T7 \textbar{} T1, T6 \textbar{} 1 \textbar{} Install Electrical Conduit \textbar{} R5 or R2
    \item T8 \textbar{} T2 \textbar{} 1 \textbar{} Install Window Frame \textbar{} R4 or R2
    \item T9 \textbar{} T3, T8 \textbar{} 0.5 \textbar{} Install Window \textbar{} R3
    \item T10 \textbar{} T4 \textbar{} 2 \textbar{} Duct Structural Framing \textbar{} R4 or R2
    \item T11 \textbar{} T5, T10 \textbar{} 2 \textbar{} Install HVAC Duct \textbar{} R4 or R2
    \item T12 \textbar{} T7 \textbar{} 2 \textbar{} Install Wiring \textbar{} R5 or R2
    \item T13 \textbar{} T12 \textbar{} 1 \textbar{} Wall Painting \textbar{} R6
    \item T14 \textbar{} - \textbar{} 0.5 \textbar{} Construction Site Inspection \textbar{} R7
    \end{itemize}
    \vspace{8pt}
    
    The robot capabilities are listed below: \\
    \vspace{3pt}
    Robot ID \textbar{} Capabilities
    \begin{itemize}
    \item R1: Cargo container
    \item R2: High-payload, Precise parallel gripper, Normal parallel gripper
    \item R3: High-payload, Suction-based gripper
    \item R4: High-payload, Normal parallel gripper
    \item R5: Precise parallel gripper
    \item R6: Sprayer
    \item R7: Camera, IAQ sensors
    \end{itemize}
    \vspace{8pt}
    
    CONSTRAINT TYPES:
    \begin{enumerate}
    \item Task Dependency Adjustments
      \begin{itemize}
        \item Format: [task\_id, successor, +/-]
        \item task\_id: the target task
        \item successor: the successors of the target task
        \item +/-: ``+'' indicates a newly added successor, ``-'' means the dependency has been removed
      \end{itemize}
    \item Task Duration Variations
      \begin{itemize}
        \item Format: [task\_id, new\_duration]
        \item task\_id: the target task
        \item new\_duration: the new duration of the target task in hours
      \end{itemize}
    \item Task Starting Time Changes
      \begin{itemize}
        \item Format: [task\_id, start\_time\_change]
        \item task\_id: the target task
        \item start\_time\_change: the changes in start time of the target task (e.g., +2 means delayed by 2 hours; -2 means ahead by 2 hours)
      \end{itemize}
    \item Number of Robot Variations
      \begin{itemize}
        \item Format: [robot\_type\_id, robot\_number\_change]
        \item robot\_type\_id: the type of robot (e.g., R1, R2, R3, etc.)
        \item robot\_number\_change: the number changes of the robot (e.g., +1 means one more robot; -1 means one less robot)
      \end{itemize}
    \item Task Conflict Constraints
      \begin{itemize}
          \item Format: [task\_id1, task\_id2]
          \item task\_id1: the first task in the conflict
          \item task\_id2: the second task in the conflict
      \end{itemize}
    \end{enumerate}
    \end{tcolorbox}
    \caption{Prompt design for construction project scheduling - Part 1.}
    \label{appendix:prompt1}
\end{figure}


\begin{figure}[t]
    \centering
    \begin{tcolorbox}[colback=gray!10!white, colframe=gray!50!gray, halign=left, boxrule=0.5pt, left=1mm, right=1mm, top=1mm, bottom=1mm]
    \fontsize{8pt}{8pt}\selectfont
    STEP-BY-STEP INSTRUCTIONS:
    \begin{enumerate}
    \item Read through the entire description to understand the context.
    \item For each change mentioned in the description:
       \begin{itemize}
       \item[a.] Identify which task (T1-T14) or robot type (R1-R7) is being affected based on CONTEXT. 
        \begin{itemize}
          \item Be careful to distinguish between similar tasks, for example:
          \begin{itemize}
          \item T2 (Move Window Frame) vs. T3 (Move Window) vs. T8 (Install Window Frame) vs. T9 (Install Window) - These are different tasks.
          \item If text mentions ``window installation'', specifically, it refers to T9 (Install Window), not T3 or T8
          \item If text mentions "window frame installation," it refers to T8 (Install Window Frame), not T2
          \end{itemize}
        \item Be careful to distinguish between similar robots, for example:
          \begin{itemize}
          \item R2, R3, R4, and R5 are different robots. 
          \item Only R2 combines both high-payload and precise parallel gripper capabilities.
          \item If text only mentions ``high-payload and normal parallel gripper'', it refers to R4 not R2.
          \end{itemize}
       \end{itemize}
       \item[b.] Determine which constraint type (1-5) applies to the change based on CONSTRAINT TYPES.
       \item[c.] Extract the specific parameters needed for that constraint type.
       \item[d.] Format the parameters according to the required format for that constraint type.
       \end{itemize}
    \item Compile all identified changes into the JSON output format:
       \begin{itemize}
       \item[a.] Create a JSON object with a ``changes'' array.
       \item[b.] For each change, add an object with ``constraint\_type'' and ``parameters'' fields.
       \item[c.] Ensure numerical values (like durations and time changes) are formatted as numbers, not strings.
       \item[d.] Ensure task IDs, successors, and robot types are formatted as strings.
       \item[e.] For time-related values:
          \begin{itemize}
          \item Simplify all numerical values to their simplest form (e.g., 1.5 not 1.50, 2 not 2.0)
          \item Convert minutes to hours (e.g., 30 minutes = 0.5 hours, 45 minutes = 0.75 hours)
          \item Please be aware that if you identify the constraint as 3, the time change should be associated with ``+'' or ``-''. 
          \end{itemize}
       \item[g.] Please be aware that if you identify the constraint as 4, the robot change should be associated with ``+'' or ``-''.
       \end{itemize}
    \item Double-check your result to ensure all changes mentioned in the description have been captured.
       \begin{itemize}
       \item[a.] Please ensure that your output follows the required format; e.g., for constraint 1, the output should be [task\_id, successor, +/-] (do NOT nest successors in additional brackets) and the task\_id should be the predecessor of the successor. 
       \item[b.] Please ensure that if you identify the constraint as 1, you correctly identify the target task and the successor of the target task and put them in the right order [task\_id, successor, +/-].
       \item[c.] Please ensure that if you identify the constraint as 3, the time change should be associated with ``+'' or ``-''. 
       \item[d.] Please ensure that if you identify the constraint as 4, the robot change should be associated with ``+'' or ``-''. 
       \item[e.] Please ensure that the task description corresponds to the task\_id in the CONTEXT.
       \end{itemize}
    \end{enumerate}
    \end{tcolorbox}
    \caption{Prompt design for construction project scheduling - Part 2.}
    \label{appendix:prompt2}
\end{figure}


\begin{figure}[t]
    \vspace{-2cm}
    \centering
    \begin{tcolorbox}[colback=gray!10!white, colframe=gray!50!gray, halign=left, boxrule=0.5pt, left=1mm, right=1mm, top=1mm, bottom=1mm]
    \fontsize{8pt}{8pt}\selectfont
EXAMPLES:\\
    Example 1:\\
    Input: ``Due to how things are unfolding on-site, it's understood that the drilling machine is not functioning, so the wall will be drilled manually. The task is expected to take two hours, and in light of recent discussions, after coordinating with field staff, it seems that the original worker assigned to install the HVAC duct is no longer available; however, we have secured another worker who can arrive in 150 minutes.''\\
    Output:
    \begin{verbatim}
{"changes": [
{"constraint_type": 2, "parameters": [T6, 2]},
{"constraint_type": 3, "parameters": [T11, +2.5]}
]}
    \end{verbatim}

    Example 2:\\
    Input: ``Recent developments suggest that wall painting takes 1.5 hours instead of 1 hour due to the need for multiple coats, and in light of recent adjustments, a revised understanding across teams indicates that a specialist required for electrical conduit installation calls in sick, preventing work from starting for 2 hours., followed by further refinements as recent developments suggest that wall painting takes 1.5 hours instead of 1 hour due to the need for multiple coats.''\\
    Output:
    \begin{verbatim}
{"changes": [
{"constraint_type": 2, "parameters": [T13, 1.5]},
{"constraint_type": 3, "parameters": [T7, +2]},
{"constraint_type": 2, "parameters": [T13, 1.5]}
]}
    \end{verbatim}

    Example 3:\\
    Input: ``Task dependencies have shifted, and one of the robots capable of handling heavy loads and performing fine, precise tasks is currently out of service due to a mechanical failure. Additionally, in light of recent discussions and the evolving situation on-site, it appears that two robots with high-capacity arms and fine-movement grippers were not charged, and have now run out of power.''\\
    Output:
    \begin{verbatim}
{"changes": [
{"constraint_type": 4, "parameters": [R2, -1]},
{"constraint_type": 4, "parameters": [R2, -2]}
]}
    \end{verbatim}
    
    Now, analyze the following description and extract all task relation changes in the specified JSON format: \{description\}
    
    Please output your response in JSON format and do not output other things. 
    \begin{verbatim}
{"changes": [
{"constraint_type": <number>, 
 "parameters": [<value1>, <value2>, ...]},...
]}
    \end{verbatim}
    \end{tcolorbox}
    \caption{Prompt design for construction project scheduling - Part 3.}
    \label{appendix:prompt3}
\end{figure}

\input{checklist}


\end{document}