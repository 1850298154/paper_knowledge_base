\section{Related projects}
\label{sec:related_projects}

As many algorithms developed in the literature have not been open-sourced, the primary goal of PyVRP is to open-source a state-of-the-art VRP solver that is easy to use and customise.
\replaced{We are aware of the following, related projects:}{There are a number of related projects that we would like to mention:}
\begin{itemize}
    \item HGS-CVRP \citep{vidal2022hybrid} is a simple, \replaced{specialised}{yet state-of-the-art} solver for the CVRP implemented in C++ with \replaced{state-of-the-art}{excellent} performance. 
    While a Python interface, PyHygese~\added{\citep{kwon_pyhygese_2022}}, is available on the Python package index, it does not come with pre-compiled binaries, and thus requires that users have a compiler toolchain already installed.
    Beyond setting the parameters, all types of customisation require changes of the C++ source code.
    \added{HGS-CVRP is open to contributions from the community and has a permissive MIT license.}

    \item LKH-3 \citep{helsgaun2017extension} is a \replaced{heuristic}{solver} that supports a wide variety of VRP \deleted{problem} variants. 
    It solves these by transforming them into a symmetric \replaced{travelling salesman problem}{TSP} problem and applying the Lin-Kernighan-Helsgaun local search heuristic.
    \replaced{While it provides good solutions for many problem variants, the solver is hard to customise as it requires modifying its C source code.}
    {While it has delivered best known solutions for multiple problem variants, it is currently not state-of-the-art for CVRP and VRPTW.}
    \added{Furthermore, it is only available under an academic and non-commercial license, and it is unclear whether community contributions are welcomed.}

    \item VROOM \added{\citep{vroom_v1.13}}, the Vehicle Routing Open-source Optimisation Machine, is an open-source solver that aims to provide good solutions to real-life VRPs.
    \added{In particular, it integrates well with open-source routing software to solve real-life VRPs within limited computation time.}
    It implements many constructive heuristics and a local search algorithm \added{in C++} and can handle different types of \replaced{VRPs}{problems}.
    \replaced{However, it is unable to compete with state-of-the-art algorithms and lacks documentation to customise its underlying solver.}{Although designed to provide good solutions quickly, it is unable to achieve state-of-the-art performance on CVRP and VRPTW even with longer runtimes.}

    \item OR-Tools \citep{ortools} is a general modelling and optimisation toolkit for solving operations research \deleted{(OR)} problems, \added{developed and} maintained by Google. 
    It is written in C++ but can be used from Python, Java or C\#. 
    \added{OR-tools is extensively documented and can be installed directly from the Python packaging index.
    Internally, it uses a constraint programming approach specialised to solve a large variety of routing problems. 
    While this approach allows it to model and solve many problem variants, its performance is far from the state of the art.}
    \deleted{While it is flexible and supports many problems, its performance is far from state-of-the-art for CVRP and VRPTW.}

    \item VRPSolver \citep{pessoa2020generic} is an exact, \added{state-of-the-art} VRP solver which supports different problem variants through a generic model.
    \added{The solver has a C++, Julia and Python interface, the latter of which can be installed directly from the Python package index.}
    VRPSolver can find optimal solutions and prove optimality for VRP solutions of modest size, but it is does not scale to instances with more than a few hundred customers.
    \added{Moreover, the solver's license limits the use of its most powerful components to academic users.}

    \item \added{
        ``A VRP Solver'' is a rich VRP heuristic solver due to \cite{builuk_rosomaxa_2023}, written in Rust and made available under an Apache 2.0 license.
        The project supports many VRP variants, using a custom data format based on JavaScript Object Notation.
        The project is well-tested and a user manual is available, including examples, but appears to be lacking a detailed documentation of the available functional endpoints.
        Furthermore, it is unclear how well the solver performs in general, as results on standard benchmark instances are lacking.
    }
\end{itemize}

While each of these projects has their own merit,\deleted{ many lack extensive documentation, tests or performance, limiting their ease of use.}
PyVRP has a unique combination of scope, performance, flexibility and ease-of-use, making it a useful addition to this set of projects.
