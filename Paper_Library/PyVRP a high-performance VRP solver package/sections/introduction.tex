\section{Introduction}
\label{sec:introduction}

This paper describes PyVRP, a Python package that \replaced{provides}{contains} a high-performance implementation of the hybrid genetic search (HGS) algorithm for vehicle routing problems (VRPs)~\citep{vidal2013hybrid}.
PyVRP currently supports two well-known VRP variants: the capacitated VRP (CVRP) and the VRP with time windows (VRPTW) \citep{toth2014vehicle}. 
The implementation builds on the open-source HGS-CVRP implementation of \citet{vidal2022hybrid}, but has added support for time windows and has been completely redesigned to be easy to use as a highly customisable Python package, while maintaining speed and state-of-the-art performance.

While HGS-CVRP is implemented completely in C++, PyVRP adopts a different philosophy: only performance-critical parts of the algorithm are implemented in C++, whereas all other parts are implemented in Python.
Besides easily using the algorithm, this gives the user additional flexibility to customise the algorithm using Python.
The provided \replaced{HGS}{Hybrid Genetic Search}\ implementation is one example of a large family of VRP algorithms that can be built using the building blocks that PyVRP provides.
As the performance-critical parts remain in C++, the added flexibility and ease-of-use that Python offers barely impact the algorithm performance.

PyVRP comes with pre-compiled binaries for Windows, Mac OS and Linux, and can be easily installed \replaced{from}{through PyPI,} the Python package index using \texttt{pip install pyvrp}.
This allows users to directly solve VRP instances, or implement variants of the HGS algorithm using Python, inspired by the examples in PyVRP's documentation. 
Users can customise various aspects of the algorithm using Python, including population management, crossover strategies, granular neighbourhoods and operator selection in the local search.
Additionally, for advanced use cases such as supporting additional VRP variants, users can build and install PyVRP directly from the source code.
We actively welcome community contributions to develop support for additional VRP variants within PyVRP, \added{and provide some guidelines for this in our online documentation}.

The goal of PyVRP, which is made available under the liberal MIT license, is to provide an easy-to-use, extensible, and well-documented VRP solver that generates state-of-the-art results on a variety of VRP variants. 
This can be used by practitioners to solve practical problems, and by researchers as a starting point or strong baseline when aiming to improve the state of the art. 
The name `PyVRP' has deliberately been chosen to not mention a specific algorithm or VRP variant, \replaced{providing flexibility with respect to the underlying heuristic algorithms and supported VRP variants}{While currently HGS is the state of the art, this may change and nothing prevents us from implementing another algorithm in the future, the implementation of which can reuse components already implemented in PyVRP}. 

Through the Python ecosystem, we enable a wide audience to easily use the software. 
We especially hope that PyVRP will help machine learning (ML) researchers interested in vehicle routing to easily build on the state-of-the-art, and move beyond LKH-3 \citep{helsgaun2017extension} as the most commonly used baseline \citep{accorsi2022guidelines}. 
Using ML for vehicle routing problems is a promising and active research area, but so far it has not been able to advance the state-of-the-art; this may change when ML researchers can build on a flexible and high-quality implementation of a state-of-the-art VRP solver.

\deleted{Compared to the HGS-CVRP code,}PyVRP is a complete Python library for solving multiple VRP variants, accompanied by unit tests, online documentation and examples on its use cases.
The library has been tested in practice: earlier versions of the software ranked 1st in the VRPTW track of the 12th DIMACS implementation challenge \citep{kool2022hybrid} and, after improvements, ranked 1st on the static VRPTW variant of the EURO meets NeurIPS 2022 vehicle routing competition \citep{van_doorn_solving_2022}.
Compared to the versions used to win these challenges, the PyVRP implementation has been simplified \replaced{further, and significantly rewritten to improve its overall design and testability.
In particular,}{and} complex components with limited contribution to the overall performance have been removed to strike a balance between simplicity and performance.
The result is simpler and more robust than the individual challenge solvers, which were quite complex and optimised for the specific challenge problem sets.
In Section~\ref{sec:experiments}, we show that PyVRP still yields excellent performance, using numerical experiments on public benchmarks following the conventions used in the DIMACS implementation challenge.

The rest of this paper is structured as follows.
In Section~\ref{sec:problem_description} we briefly introduce the CVRP and VRPTW, and the benchmarking conventions PyVRP supports.
In Section~\ref{sec:related_projects}, we discuss several other open-source VRP solvers, and highlight the novelty of PyVRP.
Then, in Sections~\ref{sec:technical_implementation} and~\ref{sec:package}, we discuss the technical implementation and PyVRP package, respectively.
In Section~\ref{sec:package} we \replaced{present two examples}{also present a short example} to demonstrate how the package can be used.
Section~\ref{sec:experiments} presents PyVRP's performance on \added{established} benchmark instances.
Finally, Section~\ref{sec:conclusion} concludes the paper.
