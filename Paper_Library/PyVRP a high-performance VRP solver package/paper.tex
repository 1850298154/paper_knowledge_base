%%%%%%%%%%%%%%%%%%%%%%%%%%%%%%%%%%%%%%%%%%%%%%%%%%%%%%%%%%%%%%%%%%%%%%%%%%%%
%% Author template for INFORMS Journal on Computing (ijoc)
%% Mirko Janc, Ph.D., INFORMS, mirko.janc@informs.org
%% ver. 0.95, December 2010
%%%%%%%%%%%%%%%%%%%%%%%%%%%%%%%%%%%%%%%%%%%%%%%%%%%%%%%%%%%%%%%%%%%%%%%%%%%%
%\documentclass[ijoc,blindrev]{informs3}
\documentclass[ijoc, nonblindrev, final]{informs3} % current default for manuscript submission

%%\OneAndAHalfSpacedXI
\OneAndAHalfSpacedXII % current default line spacing
%%\DoubleSpacedXII
%%\DoubleSpacedXI

% If hyperref is used, dvi-to-ps driver of choice must be declared as
%   an additional option to the \documentclass. For example
%\documentclass[dvips,ijoc]{informs3}      % if dvips is used
%\documentclass[dvipsone,ijoc]{informs3}   % if dvipsone is used, etc.

% Private macros here (check that there is no clash with the style)

% Natbib setup for author-year style
\usepackage{natbib}
 \bibpunct[, ]{(}{)}{,}{a}{}{,}%
 \def\bibfont{\small}%
 \def\bibsep{\smallskipamount}%
 \def\bibhang{24pt}%
 \def\newblock{\ }%
 \def\BIBand{and}%

%% Setup of theorem styles. Outcomment only one. 
%% Preferred default is the first option.
\TheoremsNumberedThrough     % Preferred (Theorem 1, Lemma 1, Theorem 2)
%\TheoremsNumberedByChapter  % (Theorem 1.1, Lema 1.1, Theorem 1.2)

%% Setup of the equation numbering system. Outcomment only one.
%% Preferred default is the first option.
\EquationsNumberedThrough    % Default: (1), (2), ...
%\EquationsNumberedBySection % (1.1), (1.2), ...

% In the reviewing and copyediting stage enter the manuscript number.
\MANUSCRIPTNO{JOC-2023-03-SI-0055.R1}

\usepackage{booktabs}
\usepackage{multirow}
\usepackage{longtable}
\usepackage{caption}
\captionsetup[longtable]{labelsep=quad, font={footnotesize,sf,bf}}
% TODO This mimics the IJOC caption style it's nicer if we can re-use the style command.

\usepackage{listings}
\usepackage{pythonhighlight}

\usepackage[xcolor={divpdf,grey}, authormarkup=none]{changes}

%%%%%%%%%%%%%%%%
\begin{document}
%%%%%%%%%%%%%%%%

\renewcommand*{\lstlistlistingname}{Listings}
\renewcommand*{\lstlistingname}{Listing}

% \newcommand{\split}{\texttt{SPLIT}}
\newcommand{\relocate}{\texttt{RELOCATE}}
\newcommand{\swap}{\texttt{SWAP}}
\newcommand{\twoopt}{\texttt{2-OPT}}
\newcommand{\sstar}{\texttt{\textsuperscript{*}}}
\newcommand{\swapstar}{\swap\sstar}
\newcommand{\relocatestar}{\relocate\sstar}
\newcommand{\twooptstar}{\twoopt\sstar}

% Outcomment only when entries are known. Otherwise leave as is and 
%   default values will be used.
%\setcounter{page}{1}
%\VOLUME{00}%
%\NO{0}%
%\MONTH{Xxxxx}% (month or a similar seasonal id)
%\YEAR{0000}% e.g., 2005
%\FIRSTPAGE{000}%
%\LASTPAGE{000}%
%\SHORTYEAR{00}% shortened year (two-digit)
%\ISSUE{0000} %
%\LONGFIRSTPAGE{0001} %
%\DOI{10.1287/xxxx.0000.0000}%

% Author's names for the running heads
% Sample depending on the number of authors;
% \RUNAUTHOR{Jones}
% \RUNAUTHOR{Jones and Wilson}
% \RUNAUTHOR{Jones, Miller, and Wilson}
% \RUNAUTHOR{Jones et al.} % for four or more authors
% Enter authors following the given pattern:
\RUNAUTHOR{Wouda, Lan, and Kool}

% Title or shortened title suitable for running heads. Sample:
\RUNTITLE{PyVRP: a high-performance VRP solver package}
% \RUNTITLE{PyVRP: a high-performance Python solver for the VRPTW}
% Enter the (shortened) title:
%\RUNTITLE{}

% Full title. Sample:
\TITLE{PyVRP: a high-performance VRP solver package}
% \TITLE{PyVRP: a high-performance Python solver for the vehicle routing problem with time windows}
% Enter the full title:
%\TITLE{}

% Block of authors and their affiliations starts here:
% NOTE: Authors with same affiliation, if the order of authors allows, 
%   should be entered in ONE field, separated by a comma. 
%   \EMAIL field can be repeated if more than one author
\ARTICLEAUTHORS{%
\AUTHOR{Niels A. Wouda}
\AFF{Department of Operations, University of Groningen, Groningen, The Netherlands, \EMAIL{n.a.wouda@rug.nl}}
\AUTHOR{Leon Lan}
\AFF{Department of Mathematics, Vrije Universiteit Amsterdam, Amsterdam, The Netherlands, \EMAIL{l.lan@vu.nl}}
\AUTHOR{Wouter Kool}
\AFF{ORTEC, Zoetermeer, The Netherlands, \EMAIL{wouter.kool@ortec.com}}
% Enter all authors
} % end of the block

\ABSTRACT{%
We introduce PyVRP, a Python package that implements hybrid genetic search in a state-of-the-art vehicle routing problem (VRP) solver.
The package is designed for the VRP with time windows (VRPTW), but can be easily extended to support other VRP variants.
PyVRP combines the flexibility of Python with the performance of C++, by implementing (only) performance critical parts of the algorithm in C++, while being fully customisable at the Python level.
PyVRP is a polished implementation of the algorithm that ranked 1st in the 2021 DIMACS VRPTW challenge and, after improvements, ranked 1st on the static variant of the EURO meets NeurIPS 2022 vehicle routing competition.
The code follows good software engineering practices, and is well-documented and unit tested.
PyVRP is freely available under the liberal MIT license.
Through numerical experiments we show that PyVRP achieves state-of-the-art results on the VRPTW and capacitated VRP.
We hope that PyVRP enables researchers \added{and practitioners} to easily and quickly build on a state-of-the-art VRP solver.
}%

% Sample 
%\KEYWORDS{deterministic inventory theory; infinite linear programming duality; 
%  existence of optimal policies; semi-Markov decision process; cyclic schedule}

% Fill in data. If unknown, outcomment the field
\KEYWORDS{Vehicle Routing Problem; Time Windows; Hybrid Genetic Search; Open-Source; C++; Python}
\HISTORY{}

\maketitle
%%%%%%%%%%%%%%%%%%%%%%%%%%%%%%%%%%%%%%%%%%%%%%%%%%%%%%%%%%%%%%%%%%%%%%

% Samples of sectioning (and labeling) in IJOC
% NOTE: (1) \section and \subsection do NOT end with a period
%       (2) \subsubsection and lower need end punctuation
%       (3) capitalization is as shown (title style).
%
%\section{Introduction.}\label{intro} %%1.
%\subsection{Duality and the Classical EOQ Problem.}\label{class-EOQ} %% 1.1.
%\subsection{Outline.}\label{outline1} %% 1.2.
%\subsubsection{Cyclic Schedules for the General Deterministic SMDP.}
%  \label{cyclic-schedules} %% 1.2.1
%\section{Problem Description.}\label{problemdescription} %% 2.

% Text of your paper here



\IEEEPARstart{T}{wo} %
main challenges in the deployment of large-scale swarms are the localization and coordination of vehicles.
Localization methods that rely on external infrastructure 
(e.g., GPS) 
are prone to systematic errors (e.g., multipath effect)
and may not always be available.
Coordination strategies that are centralized can deconflict motion plans to prevent collisions and gridlock, but introduce a single point of failure and are difficult to scale in swarm size due to communication bandwidth limitations.

This paper presents a unified formation flying pipeline for unmanned aerial vehicles (UAVs).
Our pipeline uses \textit{onboard} sensors for localization, which eliminate the need for external positioning systems, and \textit{distributed} techniques for coordination, which enable each vehicle to make decisions independently while communicating their state to a subset of the team.
For \textit{localization}, we use an off-the-shelf commercial visual inertial odometry (VIO) package \cite{VIO}
that fuses inertial measurement unit (IMU) and downward-facing monocular camera measurements to estimate changes in the vehicle pose.
\edit{For \textit{coordination}, we present distributed formation control and task assignment strategies that run onboard the vehicles, do not rely on a common reference frame, and use vehicle-to-vehicle communication.} 
Key features of our formation control strategy include scalability to a large number of vehicles and robustness to disturbances.
The latter is crucial for reaching the desired formations with sensing imperfections.
Our task assignment strategy uses an auction-based algorithm to guarantee conflict-free assignments.
This algorithm can deconflict vehicle gridlocks resulting from distributed collision avoidance (type 3 deadlock~\cite{Wang2017}) and is well-suited for vehicles with limited computational capability and low-bandwidth communication. 


\begin{figure}[t!]
	\begin{center}
		\includegraphics[trim =0mm 10mm 0mm 0mm, clip, width=\columnwidth]{Figs/slanted_plane.png}	
		\caption{
		Six multirotors in a slanted plane formation.
		Vehicles communicate with each other, make distributed decisions onboard, and use VIO for localization.}
		\label{fig:slantedplane}
	\end{center}
\end{figure}


\subsection{Contributions}

This research extends our previous work on UAV formations~\cite{Fathian2019} and presents a unified pipeline consisting of \textit{onboard localization} and \textit{distributed coordination}.
The three main contributions of this work are:
\begin{enumerate}
    \item \edit{scalable formulation of control design suitable for
    onboard sensing without a common reference frame;}
    \item algorithms for deconfliction via \edit{distributed} task assignment of vehicles to desired formation points;    
    \item simulation- and hardware-ready open-source pipeline.
\end{enumerate}
\edit{Our pipeline is tested in hardware with six multirotors (see Fig.~\ref{fig:slantedplane}), and 
to our knowledge is the first demonstration of formation flying that does not rely on external sensing, fiducial markers for localization, a common reference frame, or a centralized base station for coordination.}
The only requirements for the presented pipeline are that the vehicles can communicate, can find the transformation between their VIO start frames, and the environment is sufficiently textured---a standard assumption for VIO systems.
As such, this framework paves the way for future, real-world deployments of aerial vehicle swarms in large numbers and without requiring external localization infrastructure.


\begin{figure} [t!]
\centering
	\begin{subfigure}[b]{0.32\columnwidth}
	   %
	    \includegraphics[width=0.8\textwidth,left]{Figs/Frames2_full.pdf}
	    \caption{\scriptsize full alignment}
	    \label{fig:frame-a}
	\end{subfigure}
	\begin{subfigure}[b]{0.32\columnwidth}
	    \includegraphics[width=0.8\textwidth,center]{Figs/Frames2_orientation.pdf}
	    \caption{\scriptsize orientation alignment}
	    \label{fig:frame-b}
	\end{subfigure}
	\begin{subfigure}[b]{0.32\columnwidth}
	    \includegraphics[width=0.8\textwidth,right]{Figs/Frames2_none.pdf}
	    \caption{\scriptsize no alignment}
        \label{fig:frame-c}
	\end{subfigure}
\caption{\edit{Required alignment of UAV frames in existing swarm strategies: (a) the most restrictive case requiring a common reference frame, i.e., orientation and origin of the frames must be aligned; (b) only the orientation of the frames must be aligned; (c) no alignment restrictions (this work).}}
	\label{fig:Frames}
\end{figure}




\subsection{Related Work}

Existing aerial swarms can be grouped based on the coordination (centralized vs.\ distributed) and localization (external vs.\ onboard) methods used. 
\edit{It is further crucial to distinguish these methods based on the level of alignment required for the vehicle coordinate frames; see Fig.~\ref{fig:Frames}.} 
 
\edit{
Works with \textit{centralized} coordination and \textit{external} localization include~\cite{Preiss2017, Honig2018, Du2019}, which are based on lightweight UAVs with limited onboard computational capability and therefore rely on an external motion capture system and a base station.
Works with \textit{distributed} coordination and \textit{external} localization include \cite{wilson2020robotarium}, \cite{enright2004spheres}, where robots execute distributed controls  based on external localization by motion capture and ultrasonic beacons, respectively.
Works with \textit{centralized} coordination and \textit{onboard} localization include~\cite{Forster2013}, \cite{Loianno2016}, which use a ground station for task assignment among vehicles.
In \cite{Weinstein2018}, formation flying based on VIO is demonstrated, where motion planning and assignment are run on a base station to ensure collision-free trajectories.
The coordination strategies used in aforementioned works require a \textit{common reference frame} (Fig.~\ref{fig:frame-a}).
}


\edit{
Despite the large body of work on formation control~\cite{Oh2015}, and the variety of onboard sensing solutions for localization (e.g., VIO~\cite{Delmerico2018}), few frameworks demonstrated formation flying with \textit{distributed} coordination and \textit{onboard} localization.
A key reason is reliance of many distributed control and assignment algorithms on aligned frames (Fig.~\ref{fig:frame-a}, \ref{fig:frame-b}), which require computation-expensive and/or communication-intensive synchronization/consensus steps for frame alignment.
Equally important, dependence on alignment in existing methods \cite{Wang2017,Turpin2014, van2011reciprocal, morgan2016swarm} diminishes robustness to inherent noise and unobservable errors that cannot be corrected (e.g., disparities between the actual and estimated body frame \textit{orientation} caused by VIO drift).
Leveraging coordination methods that are \textit{robust to misaligned frames} is hence crucial and a focus of this work. 
}






\edit{
Examples of other pipelines with distributed coordination and onboard localization include \cite{Montijano2016,Tron2016}.
Both works demonstrated formation flying on three UAVs, required information from an external motion capture system due to hardware limitations, did not incorporate collision avoidance, and required frame alignment.
}
\edittwo{Note that while~\cite{Montijano2016,Tron2016} can achieve formations with arbitrary headings as illustrated in Fig.~\ref{fig:frame-c}, knowledge of relative orientations is still required; therefore, they belong to the category of Fig.~\ref{fig:frame-b}.}






\if 0

\r{
decentralized coordination setting combined with VIO:
D-CAPT [26]~\cite{}:
ORCA ~\cite{}: 
CBF [2]~\cite{} :
[A]
}

\r{Robusteness in coordination,  with compounded noise/latency, which would eventually break (b).\\


some existing algorithm might as well
work in a similar fully decentralized setting, when combined with VIO
as proposed here. For example, D-CAPT [26], ORCA, CBF [2] might also be
useful for such a task and are computationally even more efficient than
the proposed approach. \\

R2:  onboard sensing for localization ->
 Finally, the related work section only
focuses on this aspect of the pipeline, discussing how many formation papers include
onboard localization but barely sells the advantages of the coordination module (the actual
proposal of the paper) against other competitors such as [26] or [A] or to mention similar
coordination pipelines. \\


Given a solution to this problem, the controller in Section III seems unnecessary, each drone
has a target position and can use a local controller with collision avoidance that drives it to
that position. Note that such controllers exists in the literature (e.g., RVO in any of its
multi-agent variantes), they are distributed in nature and only require local sensing.


}

\fi
\section{Problem description}
\label{sec:problem_description}

PyVRP currently supports two VRP variants: CVRP and VRPTW.
The CVRP aims to construct multiple routes, each starting and ending at the same depot, to serve a set of customers while minimising total distance.
The total demand of customers in a single vehicle is limited by the vehicle capacity. 
The VRPTW generalises the CVRP by adding the constraint that each customer must be visited within a certain time window.
\added{
For both CVRP and VRPTW, PyVRP supports minimising distances, not the number of vehicles used.
A simple procedure to support fleet minimisation can be developed on top of PyVRP, by first finding a feasible solution with some number of vehicles, and then removing one vehicle at a time until no feasible solution is found.
}
PyVRP has been designed to handle instances of these problems with up to several thousand customers.

\subsection{CVRP}
Formally, the capacitated VRP consists of customers $i = 1, ..., n$ with demands $q_i \ge 0$, which must be served from a common depot denoted as $0$.
The goal is to visit all customers using a fixed fleet of vehicles, each of which starts at, and returns to, the depot, while minimising the total distance travelled, where the distance from customer (or depot) $i$ to $j$ is denoted as $d_{ij} \ge 0$.
The total demand in each vehicle should not exceed the vehicle capacity $Q > 0$.

\subsection{VRPTW}
For the VRPTW, each customer additionally has a service time $s_i \ge 0$, an earliest arrival time $e_i \ge 0$ and latest arrival time $l_i \ge 0~(e_i \le l_i)$ in between which service should \emph{start}.
A vehicle can wait at customer $i$ when arriving too early, but cannot arrive after $l_i$.
The time to travel from customer (or depot) $i$ to $j$ is given by $t_{ij} \ge 0$.
\added{While in academic benchmarks the duration $t_{ij}$ is typically set equal to the distance $d_{ij}$, PyVRP supports separate distance and duration matrices, as is commonly encountered in practice.}

\subsection{Conventions}
There are different conventions on the definitions of the constraints and objectives for CVRP and VRPTW, especially relating to rounding of (Euclidean) distances and other data \added{in existing benchmark instances} (see e.g.\ \citet{uchoa2017new}). 
PyVRP supports \emph{integer} distances \added{and durations}, which should be provided \emph{explicitly} by the user.
\added{Additionally, PyVRP can be compiled to use double precision data as well, but that is not enabled by default for performance reasons: we found during initial experimentation that working with double precision data is somewhat slower than using integers.}
\added{For working with benchmark instances,} we rely on the \textsc{VRPLIB} package \added{\citep{lan_vrplib_2023}} to compute distances \added{and durations}, and we provide helper functions to scale and then round or truncate them before converting to integers.
This way, we support various conventions, including the one-decimal precision used in the DIMACS VRPTW challenge, and the CVRPLIB benchmark repository at \url{http://vrp.galgos.inf.puc-rio.br/}.

\section{Related projects}
\label{sec:related_projects}

As many algorithms developed in the literature have not been open-sourced, the primary goal of PyVRP is to open-source a state-of-the-art VRP solver that is easy to use and customise.
\replaced{We are aware of the following, related projects:}{There are a number of related projects that we would like to mention:}
\begin{itemize}
    \item HGS-CVRP \citep{vidal2022hybrid} is a simple, \replaced{specialised}{yet state-of-the-art} solver for the CVRP implemented in C++ with \replaced{state-of-the-art}{excellent} performance. 
    While a Python interface, PyHygese~\added{\citep{kwon_pyhygese_2022}}, is available on the Python package index, it does not come with pre-compiled binaries, and thus requires that users have a compiler toolchain already installed.
    Beyond setting the parameters, all types of customisation require changes of the C++ source code.
    \added{HGS-CVRP is open to contributions from the community and has a permissive MIT license.}

    \item LKH-3 \citep{helsgaun2017extension} is a \replaced{heuristic}{solver} that supports a wide variety of VRP \deleted{problem} variants. 
    It solves these by transforming them into a symmetric \replaced{travelling salesman problem}{TSP} problem and applying the Lin-Kernighan-Helsgaun local search heuristic.
    \replaced{While it provides good solutions for many problem variants, the solver is hard to customise as it requires modifying its C source code.}
    {While it has delivered best known solutions for multiple problem variants, it is currently not state-of-the-art for CVRP and VRPTW.}
    \added{Furthermore, it is only available under an academic and non-commercial license, and it is unclear whether community contributions are welcomed.}

    \item VROOM \added{\citep{vroom_v1.13}}, the Vehicle Routing Open-source Optimisation Machine, is an open-source solver that aims to provide good solutions to real-life VRPs.
    \added{In particular, it integrates well with open-source routing software to solve real-life VRPs within limited computation time.}
    It implements many constructive heuristics and a local search algorithm \added{in C++} and can handle different types of \replaced{VRPs}{problems}.
    \replaced{However, it is unable to compete with state-of-the-art algorithms and lacks documentation to customise its underlying solver.}{Although designed to provide good solutions quickly, it is unable to achieve state-of-the-art performance on CVRP and VRPTW even with longer runtimes.}

    \item OR-Tools \citep{ortools} is a general modelling and optimisation toolkit for solving operations research \deleted{(OR)} problems, \added{developed and} maintained by Google. 
    It is written in C++ but can be used from Python, Java or C\#. 
    \added{OR-tools is extensively documented and can be installed directly from the Python packaging index.
    Internally, it uses a constraint programming approach specialised to solve a large variety of routing problems. 
    While this approach allows it to model and solve many problem variants, its performance is far from the state of the art.}
    \deleted{While it is flexible and supports many problems, its performance is far from state-of-the-art for CVRP and VRPTW.}

    \item VRPSolver \citep{pessoa2020generic} is an exact, \added{state-of-the-art} VRP solver which supports different problem variants through a generic model.
    \added{The solver has a C++, Julia and Python interface, the latter of which can be installed directly from the Python package index.}
    VRPSolver can find optimal solutions and prove optimality for VRP solutions of modest size, but it is does not scale to instances with more than a few hundred customers.
    \added{Moreover, the solver's license limits the use of its most powerful components to academic users.}

    \item \added{
        ``A VRP Solver'' is a rich VRP heuristic solver due to \cite{builuk_rosomaxa_2023}, written in Rust and made available under an Apache 2.0 license.
        The project supports many VRP variants, using a custom data format based on JavaScript Object Notation.
        The project is well-tested and a user manual is available, including examples, but appears to be lacking a detailed documentation of the available functional endpoints.
        Furthermore, it is unclear how well the solver performs in general, as results on standard benchmark instances are lacking.
    }
\end{itemize}

While each of these projects has their own merit,\deleted{ many lack extensive documentation, tests or performance, limiting their ease of use.}
PyVRP has a unique combination of scope, performance, flexibility and ease-of-use, making it a useful addition to this set of projects.

\section{Technical implementation}
\label{sec:technical_implementation}

PyVRP implements a variant of the HGS algorithm of \cite{vidal2013hybrid}.
At its core, our implementation consists of a genetic algorithm, a population, and a local search \replaced{improvement method}{educator}.
We explain these below, but refer to our documentation at \url{https://pyvrp.org/} for a full overview of the class and function descriptions, and other helper classes and methods we do not describe here.

\subsection{Overview of HGS}
HGS is a hybrid algorithm that combines a genetic algorithm with a local search algorithm.
It maintains a population with feasible and infeasible solutions.
Initially, solutions are created by randomly assigning customers to routes (feasibility is not required), which ensures diversity in the search.
Then, in every iteration, two parents are selected from the population, and combined using a crossover operator to create a new \emph{offspring solution}.
We provide an efficient C++ implementation of the selective route exchange (SREX) crossover operator~\citep{nagata2010memetic} by default, but this can easily be \replaced{replaced by}{switched out for} another crossover operator.

In each iteration, the new offspring solution is improved using local search, which considers time windows and capacities as soft constraints by penalising violations.
This way, the local search considers a smoothed version of the problem, which helps the genetic algorithm to converge towards promising regions of the solution space.
The penalty weights are automatically adjusted such that a target percentage of the local search runs results in a feasible solution. 
After the local search, the offspring is inserted into the population.
Once the population exceeds a certain size, a survivor selection mechanism removes solutions which contribute the least to the overall quality and diversity of the population.

\subsection{Genetic algorithm}
The genetic algorithm is implemented in Python and defines the main search loop.
In every iteration of the search loop, the genetic algorithm selects two (feasible or infeasible) parent solutions from the population.
A crossover operators takes the two parent solutions and uses those to generate an offspring solution that inherits features from both parents.

After crossover completes, the offspring solution is improved using local search, and then added to the population.
If this improved offspring solution is feasible and better than our current best solution, it becomes the new best observed solution.
Finally, after the main search loop completes, the genetic algorithm returns a result object that contains the best observed solution and detailed runtime statistics.

\subsection{Local search}
\label{subsec:local_search}

We provide an efficient local search implementation to improve a new offspring solution.
\added{This improvement procedure is typically the most expensive part of the HGS algorithm.
Software profiling suggests that in PyVRP it accounts for 80-90\% of the runtime, which is why the local search is implemented in C++.}
The implementation explores a granular neighbourhood \citep{toth2003granular} in a very efficient manner using user-provided operators.
These operators evaluate moves in different neighbourhoods, and the local search algorithm applies the move as soon as it yields a direct improvement in the objective value of the solution.
The search is repeated until no more improvements can be made. 
We distinguish \emph{node operators} and \emph{route operators}.
Node operators are applied to pairs of customers, and evaluate local moves around these customers.
\added{Node operators may also be applied to a customer and an unassigned vehicle: this evaluates moves placing a customer into an empty route.
To limit the number of vehicles used, these moves are \replaced{evaluated}{checked} only once all moves involving pairs of customer have been exhausted.}
Route operators are applied to pairs of \added{non-empty} routes and evaluate more expensive moves that intensify the search.

Users are free to supply their own node and route operators, but for convenience we already provide a large set of \deleted{very }efficient operators\replaced{, which we describe next}{.
We describe these next}.

\subsubsection{Node operators}
Node operators each evaluate (and possibly apply) a move between two customers $u$ and $v$, with the restriction that $v$ is in the granular neighbourhood $\mathcal{N}(u)$ of $u$.
We provide a default granular neighbourhood of size $k$ for each customer, that takes into account both spatial and temporal aspects of the problem instance.
This default implementation reduces the neighborhood size from $O(n^2)$ to $O(kn)$, but a user can fully customise the neighbourhood structure, or replace it altogether with their own.

\replaced{PyVRP currently implements the following node operators}{The node operators that are implemented in PyVRP are}:

\paragraph{$(N, M)$-exchange} 
This operator considers exchanging a consecutive route segment of $N > 0$ nodes starting at $u$ (inclusive) with a segment of $0 \le M \le N$ nodes starting at $v$ (inclusive).
These segments must not overlap in the same route\added{, and not contain the depot}.
When $M=0$, this operator evaluates \emph{relocate} moves inserting $u$ (and possible subsequent nodes) after $v$.
When $M > 0$, the operator evaluates \emph{swap} moves exchanging the route segments of one or more nodes starting at nodes $u$ and $v$.
This exchange generalises the implementations of \citet{vidal2022hybrid}.
We implement $(N, M)$-exchange using C++'s template mechanism, which after compilation results in efficient, specialised operator implementations for any $N$ and $M$.

\paragraph{MoveTwoClientsReversed}
This operator considers a $(2,0)$-exchange where $u$ and its immediate successor are reversed \added{before inserting them after $v$}. 

\paragraph{2-OPT}
The 2-OPT operator represents the routes of $u$ and $v$ as (directed) line-graphs, where an arc $u \rightarrow x$ indicates $x$ is visited \added{directly} after $u$. 
2-OPT replaces the arcs $u \rightarrow x$ and $v \rightarrow y$ by $u \rightarrow y$ and $v \rightarrow x$, effectively recombining the starts and ends of the two routes if we split them at $u$ and $v$.
When $u$ and $v$ are within the same route, and $u$ precedes $v$, this operator replaces $u \rightarrow x$ and $v \rightarrow y$ by $u \rightarrow v$ and $x \rightarrow y$, thus reversing the route segment from $x$ to $v$.

\subsubsection{Route operators}
Route operators consider moves between route pairs, avoiding the granularity restrictions imposed on the node operators.
This enables the evaluation of much larger neighbourhoods, while additional caching opportunities ensure these evaluations remain fast.
PyVRP provides two route operators by default:

\paragraph{RELOCATE*}
The RELOCATE* operator finds and applies the best $(1, 0)$-exchange move between two routes.
\added{RELOCATE* uses the $(N, M)$-exchange node operator (with $N = 1$ and $M = 0$) to evaluate each move between the two routes.}

\paragraph{SWAP*}
The SWAP* operator due to \cite{vidal2022hybrid} considers the best swap move between two routes, but does not require that the swapped customers are inserted in each others place.
Instead, each is inserted into the best location in the other route.
We enhance the implementation of \cite{vidal2022hybrid} with time window support, further caching, and earlier stopping when evaluating `known-bad' moves. 

\subsection{Population management}

The population is implemented in Python, using feasible and infeasible sub-populations that are implemented in C++ for performance.
New solutions can be added to the population, and parent solutions can be requested from it for crossover.
These parents are selected by a \replaced{$k$-way}{binary} tournament on the relative fitness of each parent~\added{\citep{Team_SB}}.
\added{By default, $k=2$, which results in a binary tournament.}

The population is initialised with a minimal set of random solutions.
New solutions obtained by the genetic algorithm are added to it as they are generated.
Once a sub-population reaches its maximal size, survivor selection is performed that reduces the sub-population to its minimal size.
This survivor selection is done by first removing duplicate solutions, and then by removing those solutions that have worst fitness based on the biased fitness criterion of \cite{vidal2022hybrid}.
This fitness criterion balances solution quality based on the solution's objective value and diversity w.r.t. to other solutions in the sub-population, evaluated using a diversity measure supplied to the population.
We implement a directed variant of the broken pairs distance, but a user can also supply their own diversity measure.

\section{The PyVRP package}
\label{sec:package}

The PyVRP package is developed in a GitHub repository located at \url{https://github.com/PyVRP/PyVRP}. 
The repository contains the C++ and Python source code, including unit and integration tests, as well as documentation and examples introducing new users to PyVRP.
Additionally, the repository uses automated workflows that build PyVRP for different platforms (currently Linux, Windows, and Mac OS), such that a user can install PyVRP directly from the Python package index using \texttt{pip install pyvrp} without having to compile the C++ extensions themselves.

\subsection{Package structure}
The top-level \texttt{pyvrp} namespace contains some of the components of Section~\ref{sec:technical_implementation}, and important additional classes.
These include \added{the \texttt{Model} modelling interface,} the \texttt{GeneticAlgorithm} and \texttt{Population}, along with a \texttt{read} function that can be used to read benchmark instances in various formats (through the VRPLIB Python package).
Crossover operators that can be used together with the \texttt{GeneticAlgorithm} are provided in \texttt{pyvrp.crossover}.
Further, the \texttt{pyvrp.diversity} namespace constains diversity measures that can be used with the \texttt{Population}.
The \texttt{pyvrp.\replaced{search}{educate}} namespace contains the \texttt{LocalSearch} class, the \deleted{node and route} operators, and the \texttt{compute\_neighbours} function that computes a granular neighbourhood.
Stopping criteria for the genetic algorithm are provided by \texttt{pyvrp.stop}.
These include stopping criteria based on a maximum number of iterations or runtime, but also variants that stop after a number of iterations without improvement.
Finally, \texttt{pyvrp.plotting} provides utilities for plotting and analysing solutions.

\subsection{Example use}
\added{
We present two examples, for different audiences.
The first example in Listing~\ref{lst:model_api} shows the modelling interface of PyVRP, and how that can be used to define and solve a CVRP instance.
This interface is particularly convenient for practitioners interested in solving VRPs using PyVRP.
The second example in Listing~\ref{lst:pyvrp} shows the different components in PyVRP, and how they can be used to solve a VRPTW instance.
This example is helpful for understanding how PyVRP's implementation of HGS works, and can be used as a basis to customise the solution algorithm.
}

\added{
We will first present the modelling interface in Listing~\ref{lst:model_api}.
}

\lstinputlisting[style=mypython, language=Python, caption=Using PyVRP's modelling interface to solve a CVRP instance., label=lst:model_api]{examples/api.py}

\added{
The modelling interface is available as \texttt{Model}, and can be used to define all relevant instance attributes: the depot, clients, vehicle types, and the edges connecting all locations.
After defining an instance, it can be solved by calling the \texttt{solve} method on the model.
Once solving finishes, a result object \texttt{res} is returned.
This object contains the best-found solution (\texttt{res.best}) and statistics about the solver run.
The result object can be printed to display the solution and some relevant statistics.
Additionally, the results can be plotted, which we will show how to do in Listing~\ref{lst:pyvrp}.
}

In Listing~\ref{lst:pyvrp} we solve the 1000 customer \texttt{RC2\_10\_5} instance of the Homberger and Gehring VRPTW set of benchmarks.
\added{Rather than using the modelling interface's high-level \texttt{solve} method, here we set everything up explicitly.}
The code assumes that the \texttt{RC2\_10\_5} instance is available locally.

\lstinputlisting[style=mypython, language=Python, caption=PyVRP example usage., label=lst:pyvrp]{examples/example.py}

\added{
Listing~\ref{lst:pyvrp} first reads a benchmark instance in standard format and constructs a random number generator with fixed seed.
It then defines the local search method.
We use the default granular neighbourhood computed by \texttt{compute\_neighbours}, but this can easily be customised by providing an alternative neighbourhood definition.
Then, we add all node and route operators described in Section~\ref{subsec:local_search} to the local search object.
This is not required: any subset of these operators is also allowed, and might even improve the solver performance in specific cases.
Finally, the penalty manager and population are initialised.
These track, respectively, the weights of constraint violation penalties, and the feasible and infeasible solution subpopulations.
An initial population should also be provided to the genetic algorithm: here we generate 25 random solutions.
A user may wish to apply alternative population generation methods here.
Finally, the genetic algorithm is initialised and run until a stopping criterion is met: in this case, the stopping criterion is 60 seconds of runtime.
We plot the solver trajectory and best observed solution in Figure~\ref{fig:RC2_10_5}.
}

\begin{figure}
    \centering
    \includegraphics[width=\textwidth]{figures/RC2_10_5.pdf}
    \caption{
        Detailed statistics collected from a single run of Listing \ref{lst:pyvrp}. 
        The top-left figure shows the average diversity of the feasible and infeasible sub-populations.
        It is clear from this figure that periodic survivor selection improves diversity.
        The middle-left figure shows the best and average objectives of both sub-populations, which improve over time as the search progresses.
        The bottom-left figure shows iteration runtimes (in seconds), \added{including a trendline}.
        Finally, the figure on the right plots the best observed solution.
    }
    \label{fig:RC2_10_5}
\end{figure}

\subsection{\added{Extending PyVRP}}

\added{
Before writing new code for PyVRP, a few things must be decided about the new constraint.
Hard constraints might require changes to PyVRP data structures.
Soft constraints typically require modifications to the cost evaluation functions.
Additionally, the new constraint likely requires additional data attributes that must be added to PyVRP's data instance object, and solution representation.
Once that new data is available, the search method can be updated to compute the correct cost deltas of each available move.
Some of the cost delta evaluation may need to be cached to ensure an efficient implementation---this is particularly the case for time-related costs, which PyVRP already supports.
Entirely new problem aspects might need to develop such caching as part of the extension.
}

\added{
Since we have developed several extensions to PyVRP already, there are some examples available of previous work.
We have summarised guidelines for extending PyVRP in our online documentation, available at \protect\url{https://pyvrp.org/dev/new_vrp_variants.html}.
}

%!TEX root = main.tex

\section{Experimental Evaluation}
\seclabel{experiments}

We first evaluated our algorithms
in an offline setting~(\secref{offline-expr}), where we record execution traces and evaluate different approaches on the \emph{same} input.
This eliminates biases due to non-deterministic thread scheduling.
Next, we consider an online setting~(\secref{online-expr}),
where we instrument programs and perform the analyses during runtime.
We conducted all our experiments on a standard laptop with \SI{1.8}{GHz} Intel Core i7 processor and \SI{16}{GB} RAM.

% We evaluated our algorithms in two experimental settings. 
% The first setting is offline experiments~(\secref{offline-expr}),
% in which we record execution traces and evaluate different approaches.
% This has the benefit that different approaches can be compared on the \emph{same} input,
% thereby eliminating biases due to non-deterministic thread scheduling.
% The second setting is online experiments~(\secref{online-expr}),
% in which we instrument programs and perform the analyses during runtime.
% We conducted all our experiments on a standard laptop with 1.8GHz Intel Core i7 processor and 16GB RAM.

%We compare our predictive algorithm with the \dlfuzzer tool. 
%This setting enables us to assess the applicability of our technique in a runtime monitoring system.
%The state-of-the-art deadlock predictors \dirk and \seqc work offline by design.
%Hence, they are not applicable for a comparison in this setting.
%As discussed in~\secref{otf}, offline methods are not directly applicable in runtime monitoring systems.

%Talk about implementation - name of tool + prog lang + traces are logged + filtering phase + cycle detection phase + online vs offline + trace conversion for seqc
%
%Setup - benchmarks + log traces using so-and-so-tools + 1-trace-per-benchmark  + cluster details + how many runs per trace + Timeout (if any)
%
%benchmark details: number of shared locks
%\subsection{Evaluation}
%
%Some suggested experiments:
%
%\begin{itemize}
%	\item Comparison with other tool(s) - \dirk and \seqc
%	\item Comparison of online vs online algorithms
%	\item Scalability with number of events
%	\item Scalability with number of threads in the trace
%	\item Scalability with number of threads in the deadlock pattern
%	\item Time spent in each kinds of events (this will be useful to guide future research)
%\end{itemize}
%
%\Andreas{For matching reports, we slightly adapt the algorithm to report all sets of program locations that contain a deadlock pattern (hence we might have more than one reports per abstract deadlock pattern, corresponding to different program locations)}


% Here we report on an implementation and experimental evaluation of our algorithms.

% \subsection{Experimental Setup}



\subsection{Offline Experiments}
\seclabel{offline-expr}


\Paragraph{Experimental setup}
The goal of the first set of experiments is to evaluate 
$\SyncPDOffline$, and compare it
against prior algorithms for dynamic deadlock prediction.
In order for our evaluation to be precise we evaluate all algorithms on the \emph{same} execution trace.
We implemented $\SyncPDOffline$ in Java inside the \toolname analysis tool~\cite{rapid}, 
following closely the pseudocode in \algoref{offline}.
\toolname takes as input execution traces, as defined in \secref{prelim}.
These also include fork, join, and lock-request events.
We compare $\SyncPDOffline$ with two state-of-the-art, 
theoretically-sound albeit computationally more expensive, deadlock predictors,
\seqc~\cite{Cai2021} and \dirk~\cite{Kalhauge2018}, both of which also work on execution traces.




On the theoretical side, the complexity of \seqc is $\Otilde(\NumEvents^4)$, 
as opposed to the $\Otilde(\NumEvents)$ complexity of $\SyncPDOffline$. 
Moreover, \seqc only predicts deadlocks of size $2$, and though it could be extended to handle deadlocks of any size, this would degrade performance further.
\seqc may miss sync-preserving deadlocks even of size $2$, 
but can  detect deadlocks that are not sync-preserving.
Thus \seqc and $\SyncPDOffline$ are theoretically incomparable in their detection capability.
We refer to\begin{pldi}~\cite{arxiv}\end{pldi}\begin{arxiv}~\appref{incomp}\end{arxiv} for examples.
We noticed that \seqc fails on traces with non-well-nested locks --- we encountered one such case in our dataset.
\dirk's algorithm is theoretically complete, i.e., it can find all predictable deadlocks in a trace.
In addition, it can find deadlocks beyond the predictable ones, by reasoning about event values.
However, \dirk relies on heavyweight SMT-solving and
employs windowing techniques to scale to large traces. 
Due to windowing, it can miss deadlocks between events that are outside the given window. 
%\hunkar{
As with previous works~\cite{Cai2021, Kalhauge2018}, we set a window size of $10$K for \dirk.
%}

Our dataset consists of several benchmarks 
from standard benchmark suites --- IBM Contest suite~\cite{Farchi03}, Java Grande suite~\cite{Smith01},
DaCapo~\cite{Blackburn06}, and
SIR~\cite{doESE05} ---
and recent literature~\cite{Kalhauge2018, Cai2021, jula2008deadlock, Joshi2009}.
Each benchmark was instrumented with RV-Predict~\cite{rvpredict} or Wiretap~\cite{Kalhauge2018} and
executed in order to log a single execution trace.

%!TEX root = ../main.tex

\begin{table}[h!]
\caption{
%\hcomment{Should we remove some not very important rows from this table?}
Trace characteristics, abstract lock graph statistics and performance comparison.
%Column 1 denotes the name of the benchmark.
Columns 2-6 show the number of events, threads, variables, locks
and total number of lock acquire and request events.
Columns 7-9 show the number of cycles, abstract and concrete deadlock patterns in the abstract lock graph.
Columns 10 - 15 show the number of deadlocks reported and the times (in seconds) taken. 
by \dirk, \seqc, and \SyncPDOffline.
%Statistics on the lock graph $\lkevgraph{\tr}$ of each trace $\tr$.
% We denote by $\mathcal{N}$, $\mathcal{T}$, $\mathcal{M}$, $\mathcal{L}$ and $\NumAcquires + \mathcal{R}$ the total number of events, number of threads, number of memory locations, number of locks and number of acquire and request events in the benchmarks, respectively. 
%All times are in seconds. 
Time out (T.O) was set to $3$h.
F stands for technical failure.
\label{tab:expr-results}
}
\vspace{-0.17cm}
\setlength\tabcolsep{3pt}
\renewcommand{\arraystretch}{0.91}
\centering
\scalebox{0.86}{
\begin{tabular}{|r|c|c|c|c|c||c|c|c||c|c||c|c||c|c|}
\hline
1 & 2 & 3 & 4 & 5 & 6 & 7 & 8 & 9 & 10 & 11 & 12 & 13 & 14 & 15 \\
\hline
\multirow{2}{*}{\textbf{Benchmark}}& 
\multirow{2}{*}{$\mathcal{N}$} & \multirow{2}{*}{$\mathcal{T}$} & \multirow{2}{*}{$\mathcal{V}$} & \multirow{2}{*}{$\mathcal{L}$} & \multirow{2}{*}{$\NumAcquires/\mathcal{R}$} 
& \multicolumn{3}{c||}{ \textsf{A. Lock Graph }}
& \multicolumn{2}{c||}{ {\dirk} } 
& \multicolumn{2}{c||}{ {\seqc}} 
& \multicolumn{2}{c|}{ {\textsf{$\SyncPDOffline$}}} \\
\cline{7-15}
& & & & & 
& \textsf{|$\textsf{Cyc}$|}
& \textsf{\textsf{A. P.}}
& \textsf{\textsf{C. P.}}
& \textbf{Dlk} 
& {\textbf{Time}} 
&{\textbf{Dlk}} 
& {\textbf{Time}}
&{\textbf{Dlk}} 
&{\textbf{Time}} \\
\hline
Deadlock & 39 & 3 & 4 & 3 & 8 & 1 & 1 & 1 & 1 & 0.02 & 0 & 0.09 & 0 & 0.16\\
NotADeadlock & 60 & 3 & 4 & 5 & 16 & 1 & 1 & 1 & 0 & 0.02 & 0 & 0.09 & 0 & 0.16\\
Picklock & 66 & 3 & 6 & 6 & 20 & 2 & 2 & 2 & 1 & 0.02 & 1 & 0.10 & 1 & 0.18\\
Bensalem & 68 & 4 & 5 & 5 & 22 & 2 & 2 & 2 & 1 & 0.02 & 1 & 0.12 & 1 & 0.16\\
Transfer & 72 & 3 & 11 & 4 & 12 & 1 & 1 & 1 & 1 & 0.02 & 0 & 0.09 & 0 & 0.15\\
Test-Dimmunix & 73 & 3 & 9 & 7 & 26 & 2 & 2 & 2 & 2 & 0.02 & 2 & 0.10 & 2 & 0.17\\
StringBuffer & 74 & 3 & 14 & 4 & 16 & 1 & 3 & 6 & 2 & 0.02 & 2 & 0.12 & 2 & 0.19\\
Test-Calfuzzer & 168 & 5 & 16 & 6 & 48 & 2 & 1 & 1 & 1 & 0.02 & 1 & 0.12 & 1 & 0.17\\
DiningPhil & 277 & 6 & 21 & 6 & 100 & 1 & 1 & 3K & 1 & 1.60 & 0 & 0.09 & 1 & 0.17\\
HashTable & 318 & 3 & 5 & 3 & 174 & 1 & 2 & 43 & 2 & 0.19 & 2 & 0.12 & 2 & 0.19\\
Account & 706 & 6 & 47 & 7 & 134 & 3 & 1 & 12 & 0 & 0.19 & 0 & 0.09 & 0 & 0.18\\
Log4j2 & 1K & 4 & 334 & 11 & 43 & 1 & 1 & 1 & 1 & 0.65 & 1 & 0.11 & 1 & 0.20\\
Dbcp1 & 2K & 3 & 768 & 5 & 56 & 2 & 2 & 3 & - & F & 2 & 0.11 & 2 & 0.19\\
Dbcp2 & 2K & 3 & 592 & 10 & 76 & 1 & 2 & 4 & - & F & 0 & 0.10 & 0 & 0.18\\
Derby2 & 3K & 3 & 1K & 4 & 16 & 1 & 1 & 1 & 1 & 0.23 & 1 & 0.10 & 1 & 0.17\\
RayTracer & 31K & 5 & 5K & 15 & 976 & 0 & 0 & 0 & - & F & 0 & 0.15 & 0 & 0.19\\
jigsaw & 143K & 21 & 8K & 2K & 67K & 172 & 12 & 70 & - & F & 2 & 0.36 & 1 & 1.55\\
elevator & 246K & 5 & 727 & 52 & 48K & 0 & 0 & 0 & 0 & 1.65 & 0 & 0.33 & 0 & 0.27\\
hedc & 410K & 7 & 109K & 8 & 32 & 0 & 0 & 0 & 0 & 2.09 & 0 & 0.50 & 0 & 0.24\\
JDBCMySQL-1 & 442K & 3 & 73K & 11 & 13K & 2 & 4 & 6 & 2 & 28.45 & 2 & 0.24 & 2 & 0.48\\
JDBCMySQL-2 & 442K & 3 & 73K & 11 & 13K & 4 & 4 & 9 & 1 & 3.37 & 1 & 0.22 & 1 & 0.33\\
JDBCMySQL-3 & 443K & 3 & 73K & 13 & 13K & 5 & 8 & 16 & 1 & 31.23 & 1 & 0.25 & 1 & 0.45\\
JDBCMySQL-4 & 443K & 3 & 73K & 14 & 13K & 5 & 10 & 18 & 2 & 5.51 & 2 & 0.28 & 2 & 0.49\\
cache4j & 775K & 2 & 46K & 20 & 35K & 0 & 0 & 0 & 0 & 5.86 & 0 & 0.46 & 0 & 0.39\\
ArrayList & 3M & 801 & 121K & 802 & 176K & 9 & 3 & 672 & 3 & 8.7K & 3 & 21.98 & 3 & 1.68\\
IdentityHashMap & 3M & 801 & 496K & 802 & 162K & 1 & 3 & 4 & 1 & 443.93 & 1 & 8.51 & 1 & 1.45\\
Stack & 3M & 801 & 118K & 2K & 405K & 9 & 3 & 481 & 1 & T.O & 3 & 25.34 & 3 & 2.94\\
Sor & 3M & 301 & 2K & 3 & 719K & 0 & 0 & 0 & 0 & 15.89 & 0 & 44.12 & 0 & 0.61\\
LinkedList & 3M & 801 & 290K & 802 & 176K & 9 & 3 & 10K & 3 & 4.7K & 3 & 48.02 & 3 & 2.06\\
HashMap & 3M & 801 & 555K & 802 & 169K & 1 & 3 & 10K & 3 & 4.4K & 2 & 504.36 & 2 & 1.65\\
WeakHashMap & 3M & 801 & 540K & 802 & 169K & 1 & 3 & 10K & - & T.O & 2 & 499.68 & 2 & 1.70\\
Swing & 4M & 8 & 31K & 739 & 2M & 0 & 0 & 0 & - & F & 0 & 0.72 & 0 & 0.88\\
Vector & 4M & 3 & 15 & 4 & 800K & 1 & 1 & 1B & - & T.O & 1 & 1.52 & 1 & 1.90\\
LinkedHashMap & 4M & 801 & 617K & 802 & 169K & 1 & 3 & 10K & 2 & 40.74 & 2 & 492.87 & 2 & 1.69\\
montecarlo & 8M & 3 & 850K & 3 & 26 & 0 & 0 & 0 & 0 & 2.6K & 0 & 1.81 & 0 & 0.79\\
TreeMap & 9M & 801 & 493K & 802 & 169K & 1 & 3 & 10K & 2 & 105.45 & 2 & 480.11 & 2 & 1.92\\
hsqldb & 20M & 46 & 945K & 403 & 419K & 0 & 0 & 0 & - & F & - & - & 0 & 2.38\\
sunflow & 21M & 16 & 2M & 12 & 1K & 0 & 0 & 0 & - & F & 0 & 8.35 & 0 & 1.62\\
jspider & 22M & 11 & 5M & 15 & 10K & 0 & 0 & 0 & - & F & 0 & 8.49 & 0 & 1.95\\
tradesoap & 42M & 236 & 3M & 6K & 245K & 2 & 1 & 4 & - & F & 0 & 108.16 & 0 & 7.06\\
tradebeans & 42M & 236 & 3M & 6K & 245K & 2 & 1 & 4 & - & F & 0 & 116.23 & 0 & 7.26\\
eclipse & 64M & 15 & 10M & 5K & 377K & 9 & 5 & 280 & - & F & 0 & 26.67 & 0 & 9.90\\
TestPerf & 80M & 50 & 599 & 9 & 197K & 0 & 0 & 0 & 0 & 795.04 & 0 & 47.56 & 0 & 4.30\\
Groovy2 & 120M & 13 & 13M & 10K & 69K & 0 & 0 & 0 & 0 & 1.7K & 0 & 38.06 & 0 & 8.92\\
Tsp & 200M & 6 & 24K & 3 & 882 & 0 & 0 & 0 & 0 & 7.6K & 0 & 72.62 & 0 & 12.70\\
lusearch & 203M & 7 & 3M & 98 & 273K & 0 & 0 & 0 & 0 & 1.3K & 0 & 75.88 & 0 & 14.44\\
biojava & 221M & 6 & 121K & 79 & 16K & 0 & 0 & 0 & - & F & 0 & 63.79 & 0 & 12.65\\
graphchi & 241M & 20 & 25M & 61 & 1K & 0 & 0 & 0 & - & F & 0 & 102.05 & 0 & 25.25\\
\hline\hline\textbf{Totals} & \textbf{1B} & \textbf{7K} & \textbf{70M} & \textbf{37K} & \textbf{8M} & \textbf{256} & \textbf{93} & \textbf{1B} & \textbf{35} & \textbf{>18h} & \textbf{40} & \textbf{2801} & \textbf{40} & \textbf{135}\\\hline
\end{tabular}
}
\end{table}

\Paragraph{Evaluation}
\cref{tab:expr-results} presents our results.
A bug identifies a unique tuple of source
code locations corresponding to events participating in the deadlock.
%Columns 2-6 present the characteristics of our benchmark traces.
Trace lengths vary vastly from $39$ to about $241$M, while the number of threads ranges from $3$ to about $800$,
which are representative features of real-world settings.
\texttt{Hsqldb} contains critical sections that are not well nested, 
and \seqc was not able to handle this benchmark;
our algorithm does not have such a restriction.
%We were unable to run \dirk on certain benchmarks due to \dirk's technical issues, which are marked with F.


\vspace{-0.1cm}
\SubParagraph{\underline{Abstract vs Concrete Patterns}}
Columns 7-9 present statistics on the 
abstract lock graph $\lkevgraph{\tr}$ of each trace $\tr$.
Many traces have a large number of concrete deadlock patterns 
but much fewer abstract deadlock patterns;
a single abstract deadlock pattern can 
comprise up to an order of $10^4$ more concrete patterns (Column $8$ v/s Column $9$).
Unlike all prior sound techniques, 
our algorithms analyze 
abstract deadlock patterns, instead of concrete ones. 
We thus expect our algorithms to be much more scalable in practice.


\SubParagraph{\underline{Deadlock-detection capability}}
In total, both \seqc and $\SyncPDOffline$ reported 40 deadlocks.
\seqc misses a deadlock of size $5$ in \texttt{DiningPhil},
which is detected by $\SyncPDOffline$,
and $\SyncPDOffline$ misses a deadlock in \texttt{jigsaw} which is detected by \seqc.
As $\SyncPDOffline$ is complete for sync-preserving deadlocks, we conclude that there are no more such deadlocks in our dataset.
Overall, $\SyncPDOffline$ and \seqc miss only three deadlocks reported by \dirk. 
On closer inspection, we found that these deadlocks are not witnessed by correct reorderings, and require reasoning about event values.
On the other hand, \dirk struggles to analyze even moderately-sized benchmarks and times out in $3$ of them. %(timeout is 3h).
This results in \dirk failing to report 5 deadlocks after $9$ hours, all of which are reported by $\SyncPDOffline$ in under a minute.
Similar conclusions were recently made in~\cite{Cai2021}.  
Overall, our results strongly indicate that the notion of sync-preservation characterizes most deadlocks that other tools are able to predict.


\SubParagraph{\underline{Unsoundness of \dirk}}
In our evaluation, we discovered that the soundness guarantee 
underlying \dirk~\cite{Kalhauge2018} is broken, resulting in it reporting false positives.
% Although \dirk is portrayed as sound, we have found two independent sources of unsoundness 
% resulting in it reporting false positives.
First, its constraint formulation~\cite{Kalhauge2018} 
does not rule out deadlock patterns when acquire events in the pattern hold common locks, 
in which case mutual exclusion disallows such a pattern to be a real predictable deadlock.
Second, \dirk also models conditional statements, allowing it to reason about witnesses beyond correct reorderings.
While this relaxation allows \dirk to predict additional deadlocks in \texttt{Transfer}, \texttt{Deadlock} and \texttt{HashMap}, 
its formalization is not precise and its implementation is erroneous.
We elaborate these aspects further in\begin{pldi}~\cite{arxiv}. \end{pldi}\begin{arxiv}~\appref{unsound-dirk}.\end{arxiv}


%We noticed that \dirk reports false positives in certain cases, which contradicts its theoretical soundness guarantee.
%We provide two such cases in~\appref{unsound-dirk}.
%The first case is a modified version of \texttt{Transfer}.
%Here, \dirk falsely reports a deadlock because it inadequately models conditional statements, which are used to allow more trace reorderings that can expose a deadlock.
%This relaxation enables \dirk to predict deadlocks in the benchmarks \texttt{Transfer} and \texttt{Deadlock}, which are missed by $\SyncPDOffline$ and \seqc. 
%However, this relaxation is not formalized and its implementation is erroneous.
%In the second case, \dirk falsely reports a deadlock due to missing that a common lock protects an otherwise deadlock pattern.

\SubParagraph{\underline{Running time}}
Our experimental results indicate that \dirk, backed by SMT solving, 
is the least efficient technique in terms of running time ---
it takes considerably longer or times out on large benchmark instances.
$\SyncPDOffline$ analyzed the entire set of traces $\sim\!\!\!21\times$ faster than \seqc.
On the most demanding benchmarks, such as 
HashMap and TreeMap, $\SyncPDOffline$ is more than $200\times$ faster than \seqc.
Although \seqc employs a polynomial-time algorithm for deadlock prediction, 
and thus significantly faster than the SMT-based \dirk,
the large polynomial complexity in its running time hinders scalability on 
execution traces coming from benchmarks that are more representative of realistic workloads.
In contrast, the linear time guarantees of $\SyncPDOffline$ are realized in practice, 
allowing it to scale on even the most challenging inputs.
More importantly, the improved performance comes while preserving essentially the same precision.


%\begin{figure}[!h]
\def\scatterscale{0.192}
\def\scatterwidth{0.23\textwidth}

\centering
\begin{subfigure}[b]{\scatterwidth}
\includegraphics[scale=\scatterscale]{figures/LinkedListDeadlockTest.pdf}
\label{subfig:sca-one-lock-clique}
\caption{LinkedList}
\end{subfigure}
\begin{subfigure}[b]{\scatterwidth}
\includegraphics[scale=\scatterscale]{figures/StackDeadlockTest.pdf}
\caption{Stack}
\label{subfig:sca-skewed}
\end{subfigure}


\caption{
Scalability experiments.
}
\label{fig:scalability}
\end{figure}


%\Andreas{We might revisit/remove the following}

%\textcolor{red}{check the numbers again.}
\SubParagraph{\underline{False negatives}}
Our benchmark set contains $93$ abstract deadlock patterns, $40$ of which are confirmed sync-preserving deadlocks.
We inspected the remaining $53$ abstract patterns to see if any of them are predictable deadlocks
missed by our sync-preserving criterion, independently of the compared tools.
$48$ of these $53$ patterns are in fact not predictable deadlocks ---
for every such pattern $D$, 
the set $S_D$ of events in the downward-closure of $\prev{}(D)$ with respect to $\tho{}$ and $\rf{}$,
already contains an event from $D$, disallowing any correct reordering
(sync-preserving or not) in which $D$ can be enabled.
Of the remaining, $4$ deadlock patterns obey the following scheme:
there are two acquire events $\acq_1, \acq_2$ participating in the deadlock pattern, 
each $\acq_i$ is preceded by a critical section on a lock that appears in 
$\lheld{}(\acq_{3-i})$, again disallowing a correct reordering that witnesses the pattern.
Thus, \emph{only one} predictable deadlock is not sync-preserving in our whole dataset.
This analysis supports that the notion of sync-preservation is not overly conservative in practice.
%\hunkar{I think here we should clarify that this analysis is not overly conservative as far as predictable deadlocks go.}

%\hunkar{
The above analysis concerns false negatives wrt. predictable deadlocks.
Some deadlocks are beyond the common notion of predictability we have adopted here, as they can only be exposed by reasoning about event values and control-flow dependencies, a problem that is $\NP$-hard even for 3 threads~\cite{Gibbons1997}.
We noticed $3$ such deadlocks in our dataset, found by \dirk,
though, as mentioned above, \dirk's reasoning for capturing such deadlocks is unsound in practice.
%}

%\hunkar{Maybe stress again that this reasoning about event values is non-trivial as it relies on heavyweight SMT solving. Also, maybe have the section "Unsoundness of Dirk" after this one and say that next we show that it is also tricky to implement in practice.}
%sync-preserving deadlocks are likely to be permissive and not lead to a high false negative rate.
%We conduct an analyses based on our benchmark set and characterize potential false negatives of our technique.
%Recall the conditions imposed on correct reorderings (see \secref{prelim}).
%One of the main conditions imposed by the definition of a correct reordering is that
%every read event reads from the same write as in the original trace. 
%We investigated the effect of this condition on the potential false negatives.
%Our benchmark set contains $93$ abstract deadlock patterns and our technique is able to find $42$ deadlocks.
%The remaining $51$ deadlock patterns constitute potential false negatives.
%If we only impose the above condition, then $46$ of the deadlock patterns cannot be realized to a real deadlock.
%The additional conditions that are specific to sync-preserving deadlocks (e.g., the order of critical sections on the same lock cannot be reversed) prevents the remaining $5$ deadlock patterns from being realizable.
%We remark that the definition of correct reorderings adopted in this work originate from the standard model that is widely used in this domain~\cite{Smaragdakis12,serbanuta2013,Koushik05}.
%Hence, this analyses further supports that given this standard program model, the additional restrictions introduced with the
%sync-preserving deadlocks are likely to be permissive and not lead to a high false negative rate.


\subsection{Online Experiments}
\seclabel{online-expr}

\Paragraph{Experimental setup}
%Dynamic analyses can incur high runtime overheads.
The objective of our second set of experiments is to evaluate 
the performance 
of our proposed algorithms in an \emph{online} setting.
For this, we implemented our $\SyncPDOnline$ algorithm inside the
framework of \dlfuzzer~\cite{Joshi2009} following closely the pseudocode in \algoref{online}. 
This framework instruments a concurrent program so that it can
perform analysis on-the-fly while executing it.
If a deadlock occurs during execution, it is reported and the execution halts.
However, if a deadlock is predicted in an alternate interleaving, 
then this deadlock is reported and the execution continues to search further deadlocks.
We used the same dataset as in \secref{offline-expr}, 
after discarding some benchmarks that could not be instrumented by \dlfuzzer.

To the best of our knowledge, all prior deadlock prediction techniques work offline.
For this reason, we only compared our online tool with the randomized 
scheduling technique of~\cite{Joshi2009} already
implemented inside the same \dlfuzzer framework.	
At a high level, this random scheduling technique works as follows.
Initially, it
(i)~executes the input program with a random scheduler, 
(ii)~constructs a \emph{lock dependency relation}, and 
(iii)~runs a cycle detection algorithm to discover deadlock patterns. 
For each deadlock pattern thus found, it spawns new executions that attempt 
to realize it as an actual deadlock.
To increase the likelihood of hitting the deadlock,
\dlfuzzer biases the random scheduler by pausing threads at specific locations.
%(e.g., before acquiring a certain lock).

The second, confirmation phase of~\cite{Joshi2009}
acts as a best-effort proxy for sound deadlock prediction.
On the other hand, $\SyncPDOnline$ is already sound and predictive, and thus does not require
additional confirmation runs, making it more efficient.
Towards effective prediction, we also implemented a simple bias to the scheduler.
If a thread $t$ attempts to write on a shared variable $x$ while holding a lock, then 
our procedure randomly decides to pause this operation for a short duration.
This effectively explores race conditions in different orders.
Overall, implementing $\SyncPDOnline$ inside \dlfuzzer provided the added advantage of supplementing a powerful prediction technique with a biased randomized scheduler.
%As an added advantage of implementing our algorithms in the
% framework, we are able to achieve
%the complementary objective of evaluating how well prediction supplements
%the effectiveness of an otherwise naive
%controlled concurrency testing
%technique like randomized scheduling.
To our knowledge, our work is the first to effectively 
combine these two orthogonal techniques.
We also remark that such a bias is of no benefit to \dlfuzzer itself
since it does not employ any predictive reasoning.

% a randomized testing procedure, even though the latter is rather simple.

For this experiment, we run \dlfuzzer on each benchmark, and for each deadlock pattern found in the initial execution, 
we let it spawn $3$ new executions trying to realize the deadlock, 
as per standard (\href{https://github.com/ksen007/calfuzzer}{https://github.com/ksen007/calfuzzer}).
We repeated this process $50$ times and recorded the total time taken.
Then, we allocated the same time for $\SyncPDOnline$ to repeatedly execute the same program and perform deadlock prediction.
We measured all deadlocks found by each technique.
%We also noticed that \dlfuzzer fails to work properly if the given input program deadlocks in the initial run.
%This results in \dlfuzzer itself to go into deadlock without producing any deadlock reports.  
%Hence, we made a modest modification on \dlfuzzer allowing it to work properly and report deadlocks in such cases.
%We used this modified version of \dlfuzzer in our evaluation.





%!TEX root = ../main.tex

\begin{table*}
\caption{
Performance comparison of $\SyncPDOnline$ ($\syncpdshort$) and $\dlfuzzer$ ($\dlfshort$).
%%Column 1 denotes the name of the benchmark.
Columns 2-3 show the total number of bug reports. 
Columns 4-6 show the total number of unique bugs found by each tool, and their union. 
Columns 7-12 show the hit rate on each bug.
Columns 13-16 show the runtime overhead of the tools.
%We refer to $\dlfuzzer$ as $\dlfshort$, to $\SyncPDOnline$ as $\syncpdshort$, and the instrumentation phases using the suffix $\mathsf{\tt -I}$.
%Performance comparison of $\SyncPDOnline$ ($\syncpdshort$) and $\dlfuzzer$ ($\dlfshort$).
%Column 1 denotes the name of the benchmark.
%Columns 2-3 show the total number of bug reports. 
%Columns 4-6 show the total number of unique bugs found by each tool, and their union. 
%Columns 7-12 show the hit rate on each bug.
%Columns 13-17 show the runtime overhead of the tools.
%We use the suffix $\mathsf{\tt -I}$ for the instrumentation phases.
\label{tab:expr-dlf-results}
}
\vspace{-0.15cm}
\setlength\tabcolsep{3pt}
\renewcommand{\arraystretch}{0.9}
\small
\centering
\scalebox{1}{
\begin{tabular}{|r|c|c|c|c|c|c|c|c|c|c|c|c|c|c|c|c|c|c|c|c|c|c|c|c|c|c|c|}
\hline
1 & 2 & 3 & 4 & 5 & 6 & 7 & 8 & 9 & 10 & 11 & 12 & 13 & 14 & 15 & 16 \\
\hline
\multirow{2}{*}{\textbf{Benchmark}}& 
\multicolumn{2}{c|}{ {\textsf{Bug Hits}}} & 
\multicolumn{3}{c|}{ {\textsf{Unique Bugs}}} 
& \multicolumn{2}{c|}{ \textsf{Bug 1}} &
\multicolumn{2}{c|}{ \textsf{Bug 2}} &
\multicolumn{2}{c|}{ \textsf{Bug 3}} & \multicolumn{4}{c|}{ {\textsf{Runtime Overhead}}}  \\
\cline{7-16}
\cline{2-6}
& \textsf{$\syncpdshort$} & \textsf{$\dlfshort$} & \textsf{$\syncpdshort$} & \textsf{$\dlfshort$}  & All 
& \textsf{\textsf{$\syncpdshort$}} 
& \textbf{$\dlfshort$} 
& {\textbf{$\syncpdshort$}} 
&{\textbf{$\dlfshort$}} 
& {\textbf{$\syncpdshort$}}
&{\textbf{$\dlfshort$}} & {\textbf{$\syncpdshortinstr$}}  & {\textbf{$\syncpdshort$}}  & 
{\textbf{$\dlfshortinstr$}}  &
 {\textbf{$\dlfshort$}} \\
\hline
%\multirow{2}{*}{\textbf{Benchmark}}& \multirow{2}{*}{$\mathcal{N}$} & \multirow{2}{*}{$\mathcal{T}$} & \multirow{2}{*}{$\mathcal{M}$}
%& \multirow{2}{*}{$\mathcal{L}$} &
%\multirow{2}{*}{$\NumAcquires+\mathcal{R}$} &\multicolumn{2}{c||}{ {\seqc}} & \multicolumn{2}{c||}{ {\textsf{$\SyncPDOffline$}}} & \multicolumn{2}{c|}{ {\textsf{$\SyncPDOnline$}}} \\
%\cline{7-12}
%& & & & & &  \textbf{$\#$Deadlocks} & { \textbf{Time (s)}} &{ \textbf{$\#$Deadlocks}} & { \textbf{Time (s)}} &{ \textbf{$\#$Deadlocks}} &{ \textbf{Time (s)}} \\
\hline
Deadlock & 50 & 50 & 1 & 1 & 1 & 50 & 50 & - & - & - & - & 2$\times$ & 3$\times$ & 2$\times$ & 4$\times$\\
Picklock & 227 & 97 & 2 & 1 & 2 & 226 & 97 & 1 & 0 & - & - & 2$\times$ & 2$\times$ & 2$\times$ & 5$\times$\\
Bensalem & 355 & 32 & 2 & 1 & 2 & 8 & 0 & 347 & 32 & - & - & 2$\times$ & 2$\times$ & 2$\times$ & 6$\times$\\
Transfer & 54 & 50 & 1 & 1 & 1 & 54 & 50 & - & - & - & - & 2$\times$ & 2$\times$ & 1$\times$ & 4$\times$\\
Test-Dimmunix & 702 & 0 & 2 & 0 & 2 & 351 & 0 & 351 & 0 & - & - & 2$\times$ & 2$\times$ & 2$\times$ & 4$\times$\\
StringBuffer & 153 & 131 & 2 & 2 & 2 & 128 & 118 & 25 & 13 & - & - & - & - & - & -\\
Test-Calfuzzer & 177 & 44 & 1 & 1 & 1 & 177 & 44 & - & - & - & - & 2$\times$ & 2$\times$ & 2$\times$ & 4$\times$\\
DiningPhil & 162 & 100 & 1 & 1 & 1 & 162 & 100 & - & - & - & - & - & - & - & -\\
HashTable & 169 & 120 & 2 & 2 & 2 & 82 & 21 & 87 & 99 & - & - & - & - & - & -\\
Account & 19 & 188 & 1 & 1 & 1 & 19 & 188 & - & - & - & - & 2$\times$ & 8$\times$ & 2$\times$ & 16$\times$\\
Log4j2 & 290 & 100 & 2 & 1 & 2 & 145 & 100 & 145 & 0 & - & - & - & - & - & -\\
Dbcp1 & 265 & 138 & 2 & 2 & 2 & 264 & 61 & 1 & 77 & - & - & - & - & - & -\\
Dbcp2 & 129 & 126 & 2 & 2 & 2 & 86 & 99 & 43 & 27 & - & - & - & - & - & -\\
RayTracer & 0 & 0 & 0 & 0 & 0 & - & - & - & - & - & - & 122$\times$ & 124$\times$ & 109$\times$ & 111$\times$\\
Tsp & 0 & 0 & 0 & 0 & 0 & - & - & - & - & - & - & 47$\times$ & 60$\times$ & 37$\times$ & 40$\times$\\
jigsaw & 1189 & 1 & 1 & 1 & 2 & 1189 & 0 & 0 & 1 & - & - & - & - & - & -\\
elevator & 0 & 0 & 0 & 0 & 0 & - & - & - & - & - & - & 2$\times$ & 2$\times$ & 2$\times$ & 2$\times$\\
JDBCMySQL-1 & 349 & 117 & 2 & 3 & 3 & 1 & 21 & 0 & 4 & 348 & 92 & 3$\times$ & 4$\times$ & 2$\times$ & 13$\times$\\
JDBCMySQL-2 & 559 & 73 & 1 & 1 & 1 & 559 & 73 & - & - & - & - & 2$\times$ & 4$\times$ & 2$\times$ & 18$\times$\\
JDBCMySQL-3 & 560 & 224 & 1 & 1 & 1 & 560 & 224 & - & - & - & - & 2$\times$ & 5$\times$ & 2$\times$ & 24$\times$\\
JDBCMySQL-4 & 1717 & 101 & 3 & 1 & 3 & 95 & 0 & 834 & 0 & 788 & 101 & 3$\times$ & 5$\times$ & 2$\times$ & 31$\times$\\
hedc & 0 & 0 & 0 & 0 & 0 & - & - & - & - & - & - & 2$\times$ & 2$\times$ & 1$\times$ & 2$\times$\\
cache4j & 0 & 0 & 0 & 0 & 0 & - & - & - & - & - & - & 2$\times$ & 2$\times$ & 2$\times$ & 2$\times$\\
lusearch & 0 & 0 & 0 & 0 & 0 & - & - & - & - & - & - & 16$\times$ & 17$\times$ & 13$\times$ & 16$\times$\\
ArrayList & 47 & 45 & 3 & 3 & 3 & 20 & 22 & 3 & 5 & 24 & 18 & 50$\times$ & 69$\times$ & 18$\times$ & 79$\times$\\
Stack & 44 & 27 & 3 & 3 & 3 & 18 & 13 & 8 & 4 & 18 & 10 & 69$\times$ & 91$\times$ & 64$\times$ & 86$\times$\\
IdentityHashMap & 68 & 62 & 2 & 2 & 2 & 13 & 47 & 55 & 15 & - & - & 4$\times$ & 8$\times$ & 3$\times$ & 10$\times$\\
LinkedList & 48 & 26 & 3 & 2 & 3 & 21 & 17 & 7 & 0 & 20 & 9 & 16$\times$ & 28$\times$ & 14$\times$ & 32$\times$\\
Swing & 0 & 0 & 0 & 0 & 0 & - & - & - & - & - & - & 5$\times$ & 6$\times$ & 4$\times$ & 6$\times$\\
Sor & 0 & 0 & 0 & 0 & 0 & - & - & - & - & - & - & 2$\times$ & 7$\times$ & 2$\times$ & 2$\times$\\
HashMap & 46 & 44 & 2 & 2 & 2 & 18 & 11 & 28 & 33 & - & - & 7$\times$ & 11$\times$ & 4$\times$ & 8$\times$\\
Vector & 126 & 50 & 1 & 1 & 1 & 126 & 50 & - & - & - & - & 2$\times$ & 2$\times$ & 2$\times$ & 3$\times$\\
LinkedHashMap & 57 & 43 & 2 & 2 & 2 & 22 & 10 & 35 & 33 & - & - & 10$\times$ & 10$\times$ & 4$\times$ & 8$\times$\\
WeakHashMap & 29 & 40 & 2 & 2 & 2 & 6 & 11 & 23 & 29 & - & - & 7$\times$ & 12$\times$ & 4$\times$ & 8$\times$\\
montecarlo & 0 & 0 & 0 & 0 & 0 & - & - & - & - & - & - & 16$\times$ & 100$\times$ & 13$\times$ & 126$\times$\\
TreeMap & 42 & 47 & 2 & 2 & 2 & 16 & 15 & 26 & 32 & - & - & 9$\times$ & 12$\times$ & 5$\times$ & 9$\times$\\
eclipse & 0 & 0 & 0 & 0 & 0 & - & - & - & - & - & - & 2$\times$ & 2$\times$ & 2$\times$ & 2$\times$\\
TestPerf & 0 & 0 & 0 & 0 & 0 & - & - & - & - & - & - & 2$\times$ & 2$\times$ & 2$\times$ & 2$\times$\\
\hline\hline\textbf{Total} & \textbf{7633} & \textbf{2076} & \textbf{49} & \textbf{42} & \textbf{51} & - & - & - & - & - & - & - & - & - & - \\ 
\hline
\end{tabular}
}
\end{table*}


\Paragraph{Evaluation}
\cref{tab:expr-dlf-results} presents our experimental results.
%A bug identifies a unique tuple of source code locations corresponding to events
%participating in the deadlock.
Columns $2$-$3$ of the table display the total number of bug hits,
which is the total number of times a bug was predicted by $\SyncPDOnline$ in the entire duration,
or was confirmed in any trial of \dlfuzzer.
Columns $4$-$6$ display the unique bugs (i.e., unique tuples of source code locations leading to a deadlock) 
found by the techniques.
The employed techniques are able to find a maximum of $3$ unique bugs for each benchmark
in our benchmark set. 
Respectively, columns $7$-$12$ display the detailed information on the number 
of times a particular bug was found by each technique.
Runtime overheads are displayed in the columns $13$-$16$, with $\mathsf{\tt -I}$ denoting the instrumentation phase only.


\SubParagraph{\underline{Deadlock-detection capability}}
\dlfuzzer had $2076$ bug reports in total, and it found $42$ unique bugs.
In contrast, $\SyncPDOnline$ flagged $7633$ bug reports, corresponding to $49$ unique bugs.
In more detail, \dlfuzzer missed $9$ bugs reported by \SyncPDOnline whereas 
$\SyncPDOnline$ missed $2$ bugs reported by \dlfuzzer.
Also, \SyncPDOnline significantly outperformed \dlfuzzer in total number of bugs hits.
Our experiments again support that the notion of sync-preservation  captures most deadlocks that occur in practice, to the extent that other state-of-the-art techniques can capture.
%\hunkar{
A further observation is that in the offline experiments, \SyncPDOffline  was not able to find deadlocks in \texttt{Transfer} and \texttt{Deadlock}. 
However, the random scheduling procedure allowed \SyncPDOnline 
to navigate to executions from which deadlocks can be predicted.
This demonstrates the potential of combining predictive dynamic
techniques with controlled concurrency testing.
%; a direction we find promising to be pursued further by the community.


\SubParagraph{\underline{Runtime overhead}}
We have also measured the runtime overhead of both \SyncPDOnline and \dlfuzzer,
both as incurred by instrumentation, as well as by the deadlock analysis.
The latter is the time taken by \algoref{online} for the case of $\SyncPDOnline$,
and the overhead introduced due to the new executions in the second confirmation phase for the case of \dlfuzzer.
Our results show that the instrumentation overhead of \SyncPDOnline is, in fact, comparable to that of \dlfuzzer, though somewhat larger. 
This is expected, as \SyncPDOnline needs to also instrument memory access events, while \dlfuzzer only instruments lock events, but at the same time surprising because the number of memory access events
is typically much larger than the number of lock events.
On the other hand, the analysis overhead is often larger for \dlfuzzer, 
even though it reports fewer bugs.
It was not possible to measure the runtime overhead in certain benchmarks as 
either they were always deadlocking or the computation was running indefinitely.



\section{Conclusions}
In this paper, we set out to address the problem of multi-tasking robots in multi-robot tasks. 
%A fundamental limitation of existing multi-robot systems was addressed by the removal of a restrictive assumption that was often made--robots are single-tasking.
%Our method allowed coalitions to overlap thus enabling multi-tasking robots. 
We observed that the key underlying challenge was to reason about the physical constraints that could be synergistically satisfied.
%which directly affected the feasibility of multi-tasking.
To address the challenge, we developed our method based on the information invariant theory and modeled constraints as information instances. 
%This allowed us to reason about the relationships between constraints by reasoning about those between information requirements. 
Thereby, a formal and general framework to achieve multi-tasking robots was developed. 
We showed that our algorithm was sound and complete under our problem settings. 
%Our method was integrated with a simple greedy heuristic for task allocation.
Simulation  results  were  provided  to  show  the  effectiveness  of  our approach under resource-constrained situations and in handling challenging situations. % in a multi-UAV simulator. 

% The idea of multi-tasking is attractive in many ways. 
% Humans are living in multi-tasking environments--at any point of time, 
% we are optimizing for more than one task. 
% Multi-task often leads to more efficient task performance since it allows us to exploit task synergies. 
% The work presented in this paper takes us one step forward in realizing multi-tasking robots. 
% In particular, we started looking at the feasibility of multi-tasking. 
% There are many potential directions to pursue along this direction. First, several limitations are present in the current approach. 
% For example, although our method guarantees that there exists a physical configuration that satisfies all the constraints, it does not explicitly take the environmental influence into account. For example, a narrow corridor may prevent a robot formation from passing through, even though all the constraints for the formation do not introduce any conflicts. In this sense, our work should better be characterized as establishing a necessary condition for multi-tasking. Also, our method is mainly focused on the ``{\it planning}'' phase and hence does not address how the robots reach the desired configuration and maintain the constraints. These issues are assumed to be handled by the execution layer.

% More generally, the question of how to execute the tasks with overlapping coalitions is not addressed in this work. 
% As we already discussed, executing individual tasks with non-overlapping coalitions is straightforward but task synergies impose additional requirements on the task execution: how should the robots that are assigned multiple tasks execute them? Should they consider them in a prioritized strategy~\cite{van2005prioritized}? Or should they combine the different tasks in a way that is similar to motor schemas~\cite{arkin2}. 
% Communication requirements for maintaining the constraints must also be taken into account. How should the robots optimize their communication to improve the task performance? 

% The stringency of the physical constraints is another interesting question. It may be desirable to relax the constraints in certain situations (e.g., due to environmental influences). In such cases, it may be important to consider the problem where the constraints are least violated~\cite{kim2012revision}, or specify task constraints in different ways to increase the diversity of the configurations~\cite{srivastava2007domain} so as to make it robust to different environments. 


% Acknowledgments here
\ACKNOWLEDGMENT{%
We are grateful to Penghui Guo for transferring the `PyVRP' name on the Python package index to us. 
Leon Lan would like to thank TKI Dinalog, Topsector Logistics and the Dutch Ministry of Economic Affairs and Climate Policy for funding this project.
}% Leave this (end of acknowledgment)


\section{On Presentations of Pure Braid Groups}\label{appendix:pn}
\citet{bhattacharya2018path} gave a presentation of a homotopy group for path planning with $n$ agents on a plane, which is a pure braid group $P_n$, with the following generators:
\begin{equation}
\left\{u_{i,j/\gamma_{i+1},\ldots,\gamma_{j-1}}\middle|1\leq i<j\leq n,\,\gamma_{i+1},\ldots,\gamma_{j-1}\in\{+,-\}\right\}.
\end{equation}

The relations are as follows.
\begin{itemize}
    \item For $i<j<k$, $\alpha_{i+1},\ldots,\alpha_{j-1}\in\{+,-\}$, and $\beta_{j+1},\ldots,\beta_{k-1}\in\{+,-\}$,
    \begin{equation}
    \begin{split}
    &u_{i,j/\alpha_{i+1},\ldots,\alpha_{j-1}}
    \cdot
        u_{i,k/\alpha_{i+1},\ldots,\alpha_{j-1},-,\beta_{j+1},\ldots,\beta_{k-1}}
        \cdot u_{j,k/\beta_{j+1},\ldots,\beta_{k-1}}\\
        &\cdot u^{-1}_{i,j/\alpha_{i+1},\ldots,\alpha_{j-1}}
    \cdot
        u^{-1}_{i,k/\alpha_{i+1},\ldots,\alpha_{j-1},+,\beta_{j+1},\ldots,\beta_{k-1}} 
        \cdot u^{-1}_{j,k/\beta_{j+1},\ldots,\beta_{k-1}},
    \end{split}
    \end{equation}
    \begin{equation}
    \begin{split}
    &u_{i,j/\alpha_{i+1},\ldots,\alpha_{j-1}}
    \cdot
        u_{i,k/\alpha_{i+1},\ldots,\alpha_{j-1},-,\beta_{j+1},\ldots,\beta_{k-1}}\\
        &\cdot u^{-1}_{i,j/\alpha_{i+1},\ldots,\alpha_{j-1}}\cdot
        u^{-1}_{i,k/\alpha_{i+1},\ldots,\alpha_{j-1},+,\beta_{j+1},\ldots,\beta_{k-1}},
    \end{split}
    \end{equation}
    and
    \begin{equation}\label{eq:p-rel}
    \begin{split}
    &u_{i,k/\alpha_{i+1},\ldots,\alpha_{j-1},-,\beta_{j+1},\ldots,\beta_{k-1}}
        \cdot u_{j,k/\beta_{j+1},\ldots,\beta_{k-1}}\\
    &\cdot u^{-1}_{i,k/\alpha_{i+1},\ldots,\alpha_{j-1},+,\beta_{j+1},\ldots,\beta_{k-1}} 
    \cdot u^{-1}_{j,k/\beta_{j+1},\ldots,\beta_{k-1}}.
    \end{split}
    \end{equation}
\item Let $i,j,i',j'$ be distinct indices with $i<j,\,i'<j',\,i<i'$. Let $\gamma_{i+1},\ldots,\gamma_{j-1}\in\{+,-\}$ and $\gamma'_{i'+1},\ldots,\gamma'_{j'-1}\in\{+,-\}$ be signs.
When $i<i'<j<j'$, we assume that $\gamma_{i'}\neq \gamma'_j$ and $(\gamma_k,\gamma'_k)\neq(-\gamma_{i'},\gamma_{i'})$ for all $i'<k<j$.
When $i<i'<j'<j$, we assume that $\gamma_{i'}= \gamma_{j'}$ and $(\gamma_k,\gamma'_k)\neq(-\gamma_{i'},\gamma_{i'})$ for all $i'<k<j'$. For such tuples,
    \begin{equation}\label{eq:commute}
    \begin{split}
    &u_{i,j/\gamma_{i+1},\ldots,\gamma_{j-1}}
    \cdot u_{i',j'/\gamma'_{i'+1},\ldots,\gamma'_{j'-1}}
    \cdot u^{-1}_{i,j/\gamma_{i+1},\ldots,\gamma_{j-1}}
    \cdot u^{-1}_{i',j'/\gamma'_{i'+1},\ldots,\gamma'_{j'-1}}.
    \end{split}
    \end{equation}
\end{itemize}
The description in the original paper is incomplete as it omits the condition for the relation (\ref{eq:commute}).

We consider the word
\begin{equation}
w=u_{1,4/--}u_{2,4/-}u_{3,4}u_{1,4/++}^{-1}u_{2,4/+}^{-1}u_{3,4}^{-1},
\end{equation}
which is irreducible when using Dehn's algorithm.
On the other hand,
\begin{equation}
\begin{split}
w &=u_{2,4/-}u_{1,4/+-}u_{3,4}u_{1,4/++}^{-1}u_{2,4/+}^{-1}u_{3,4}^{-1}\\
&=u_{2,4/-}u_{3,4}u_{2,4/+}^{-1}u_{3,4}^{-1}\\
&=u_{3,4}u_{3,4}^{-1}\\
&=e,
\end{split}
\end{equation}
where the first, second, and third equalities are deduced from (\ref{eq:p-rel}) with $(i,j,k)=(1,2,4)$, $(1,3,4)$, and $(2,3,4)$, respectively. Thus, Dehn's algorithm is incomplete for this presentation when $n\geq 4$.

The pure braid group $P_n$ has a standard presentation~\citep{rolfsen2010tutorial} using generators $\left\{a_{i,j}\middle|1\leq i<j\leq n\right\}$ with notation (\ref{eq:def_a})
and the following relations.
\begin{itemize}
\item For $1\leq i<j<k\leq n$,
\begin{equation}
    a_{i,j}a_{i,k}a_{j,k}a_{i,j}^{-1}a_{j,k}^{-1}a_{i,k}^{-1},
\end{equation}
and
\begin{equation}
    a_{i,k}a_{j,k}a_{i,j}a_{i,k}^{-1}a_{i,j}^{-1}a_{j,k}^{-1}.
\end{equation}
\item For $1\leq i<j<k<l\leq n$,
\begin{equation}
    a_{i,j}a_{k,l}a_{i,j}^{-1}a_{k,l}^{-1},
\end{equation}
\begin{equation}
    a_{i,l}a_{j,k}a_{i,l}^{-1}a_{j,k}^{-1},
\end{equation}
and
\begin{equation}
    a_{i,k}a_{j,k}a_{j,l}a_{j,k}^{-1}a_{i,k}^{-1}a_{j,k}a_{j,l}^{-1}a_{j,k}^{-1}.
\end{equation}
\end{itemize}
Let
\begin{equation}
u_{i,j/\sigma_{i+1},\ldots,\sigma_{j-1}}=a_{i,i+1}^{d_{i+1}}\cdots a_{i,j-1}^{d_{j-1}} a_{i,j}a_{i,j-1}^{-d_{j-1}}\cdots a_{i,i+1}^{-d_{i+1}},
\end{equation}
where $d_k=1$ if $\sigma_k=+$, and $d_k=0$ if $\sigma_k=-$.
Then, the relations for $\{u_{i,j/\sigma_{i+1},\ldots,\sigma_{j-1}}\}$ can be deduced from the relations for $\{a_{i,j}\}$ and vice versa. Therefore, we can translate words from Bhattacharya and Ghrist's presentation to words for the standard presentation. However, this translation increases the lengths of words by a factor of $O(n)$.
On the other hand, lengths of words constructed by Bhattacharya and Ghrist's method are equal to those of words constructed by the method in \S~\ref{subsec:word construction}.
To the best of our knowledge, there is no algorithm for the word problem of the pure braid group that is efficient enough to compensate for these drawbacks. This is why we use elements of the braid group to label homotopy classes, instead of those of the pure braid group.

\section{Proof of Proposition~\ref{prop:homotopy_inj}}\label{appendix:proof}
The map from $\mathcal{C}_{r+n}(\mathbb{R}^2)$ to $\mathcal{C}_r(\mathbb{R}^2)$, which sends $(p_1,\ldots,p_{r+n})$ to $(p_1,\ldots,p_r)$, is a fiber bundle~\citep{fadell1962configuration}, and the fiber at $(o_1,\ldots, o_r)$ is the image of the embedding (\ref{eq:embedding}). Thus, the following exact sequence of homotopy groups is induced:
\begin{equation}        
\pi_2\left(\mathcal{C}_r(\mathbb{R}^2)\right)\to
\pi_1\left(\mathcal{C}_n(\mathbb{R}^2\setminus\{o_1,\ldots,o_r\})\right)\to \pi_1\left(\mathcal{C}_{r+n}(\mathbb{R}^2)\right),
\end{equation}
where $\pi_2\left(\mathcal{C}_r(\mathbb{R}^2)\right)$ is the second homotopy group of $\mathcal{C}_r(\mathbb{R}^2)$~\citep{hatcher2002algebraic}.
Moreover, $\pi_2\left(\mathcal{C}_r(\mathbb{R}^2)\right)$ is trivial~\citep{knudsen2018configuration}.



% References here (outcomment the appropriate case) 

% CASE 1: BiBTeX used to constantly update the references 
%   (while the paper is being written).
\bibliographystyle{informs2014} % outcomment this and next line in Case 1
\bibliography{references.bib} % if more than one, comma separated

% CASE 2: BiBTeX used to generate mypaper.bbl (to be further fine tuned)
%\input{mypaper.bbl} % outcomment this line in Case 2

\end{document}
