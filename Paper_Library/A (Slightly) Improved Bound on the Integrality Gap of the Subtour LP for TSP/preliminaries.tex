%\subsection{Proof Overview} 

\section{Preliminaries}
\label{sec:prelims}
In the interest of getting quickly to the overview in \cref{sec:overview}, on a first pass reading of this paper, the reader may wish to skip over the (short) proofs later in this section.

\subsection{Algorithm}
%First, we recall the classical Christofides-Serdyukov algorithm: Given an instance of TSP, choose a minimum spanning tree and then add the minimum cost matching on the odd degree vertices of the tree. The algorithm we study is very similar, except we choose a random spanning tree based on the standard linear programming relaxation of TSP. 

Let $x^0$ be an optimum solution of LP \eqref{eq:tsplp}. 
%Without loss of generality, we assume that there is 
%It can be shown that $x^*$ always has an edge with fraction 1 \cite{BP90}).
Without loss of generality we assume $x^0$ has an edge $e_0=\{u_0,v_0\}$ with $x^0_{e_0}=1, c(e_0)=0$.
(To justify this, consider the following process: given $x^0$, pick an arbitrary node, $u$, split it into two nodes $u_0,v_0$ and set $x_{\{u_0,v_0\}}=1, c(e_0)=0$ and assign half of every edge incident to $u$ to $u_0$ and the other half to $v_0$.) %This allows us to assume without loss of generality that $x^0$ has an edge $e_0=(u_0,v_0)$ such that  $x_{e_0}=1, c(e_0)=0$. 

Let $E_0=E\cup\{e_0\}$ be the support of $x^0$ and let $x$ be $x^0$ restricted to $E$ and $G=(V,E)$. Note $x^0$ restricted to $E$ is in the spanning tree polytope  \eqref{eq:spanningtreelp} of $G$.

For a vector $\lambda:E\to\R_{\geq 0}$, a $\lambda$-uniform distribution $\mu_\lambda$ over spanning trees of $G=(V,E)$ is a distribution where for every spanning tree $T\subseteq E$, $\PP{\mu_\lambda}{T}=\frac{\prod_{e\in T} \lambda_e}{\sum_{T'} \prod_{e\in T'} \lambda_e}$.
The second step of the algorithm is to find a vector $\lambda$ such that for every edge $e\in E$, $\PP{T \sim \mu_\lambda}{e\in T}=x_e(1\pm\eps)$, for some $\eps<2^{-n}$. Such a vector $\lambda$ can be found using the multiplicative weight update algorithm \cite{AGMOS10} (see \cref{thm:maxentropycomp}) or by applying interior point methods \cite{SV12} or the ellipsoid method \cite{AGMOS10}. (We note that the multiplicative weight update method can only guarantee $\eps<1/\text{poly}(n)$ in polynomial time.)

Finally, similar to Christofides' algorithm, we sample a tree $T\sim\mu_{\lambda}$ and then add the minimum cost matching on the odd degree vertices of $T$.

\hypertarget{tar:alg}{\begin{algorithm}[h]
\begin{algorithmic}
	\State Find an optimum solution $x^0$ of  \cref{eq:tsplp}, and let $e_0=\{u_0,v_0\}$ be an edge with $x^0_{e_0}=1,c(e_0)=0$.
	\State Let $E_0=E\cup \{e_0\}$ be the support of $x^0$ and $x$ be $x^0$ restricted to $E$ and $G=(V,E)$.
	\State Find a vector $\lambda:E\to\R_{\geq 0}$ such that for any $e\in E$, $\PP{T \sim \mu_\lambda}{e \in T}=x_e(1\pm 2^{-n})$.
	\State Sample a tree $T\sim\mu_\lambda$.
	\State Let $M$ be the minimum cost matching on odd degree vertices of $T$.
	\State Output $T \cup M$.
\end{algorithmic}
\caption{Max Entropy Algorithm for TSP}\label{alg:tsp}
\end{algorithm}}
The above algorithm from \cite{KKO21} is a slight modification of the algorithm proposed in  \cite{OSS11}. While the proof of \cref{thm:main} heavily utilizes properties of max entropy trees, we note that \cref{thm:cutsbothsideswithinside} (the main contribution of this paper) only uses the fact that the spanning tree distribution respects the marginals of $x$.
\begin{theorem}[\cite{AGMOS10}]
\label{thm:maxentropycomp}
Let $z$ be a point in the spanning tree polytope (see \eqref{eq:spanningtreelp}) of a graph $ G=(V, E)$.
For any $\eps>0$, a vector $\lambda:E\to\R_{\geq 0}$ can be found such that the corresponding $\lambda$-uniform spanning tree distribution, $\mu_\lambda$, satisfies
%we define the exponential family distribution
%$$\tilde{p}(T):=\frac{1}{P}\exp(\sum_{e\in T} \tilde{\gamma}_e)$$ for
%all $T\in {\cal T}$ where $$P:=\sum_{T\in {\cal T}}\exp(\sum_{e\in T}
%\tilde{\gamma}_e)$$ then, for every edge $e\in E$,
$$%\tilde{z}_e :=
\sum_{T\in {\cal T}: T \ni e} \PP{\mu_\lambda}{T}  \leq (1+\varepsilon)z_e,\hspace{3ex}\forall e\in E,$$
i.e., the marginals are approximately preserved.  In the above ${\cal T}$ is the set of all spanning trees of $(V,E)$. The running
time is polynomial in $n=|V|$, $- \log \min_{e\in E} z_e$ and $\log(1/\eps)$.
\end{theorem}

%\subsection{Held-Karp relaxation and class of algorithms}
%
%Let $x^0$ be an optimum solution of the following TSP   linear program relaxation \cite{DFJ59,HK70}:
%\begin{equation}\label{eq:tsplp}
%\begin{aligned}
%	\min \quad& \sum_{u,v} x_{(u,v)} c(u,v)& \\
%	\text{s.t.,} \quad &  \sum_{u} x_{(u,v)} = 2&\forall v\in V,\\
%	& \sum_{u\in S, v\notin S} x_{(u,v)}\geq 2,&\forall S\subsetneq V,\\
%	& x_{(u,v)}\geq 0 &\forall u,v\in V.
%\end{aligned}	
%\end{equation}
%
%Given  $x^0$, we pick an arbitrary node, $u$, split it into two nodes $u_0,v_0$ and set $x_{(u_0,v_0)}=1, c(u_0,v_0)=0$ and we assign half of every edge incident to $u$ to $u_0$ and the other half to $v_0$.  This allows us to assume without loss of generality that $x^0$ has an edge $e_0=(u_0,v_0)$  such that  $x_{e_0}=1, c(e_0)=0$. 
%%if such an edge does not exist, we (also 
%
%Let $E_0=E\cup\{e_0\}$ be the support of $x^0$ and let $x$ be $x^0$ restricted to $E$ and $G=(V,E)$.
%% be the support of $x$ excluding the edge $e_0$, and let $E_0=E\cup \{e_0\}$.
%$x^0$ restricted to $E$ is in the spanning tree polytope. The algorithm samples a tree from $x^0$ and then adds the edge $e_0$. 

\subsection{Notation}
We write $[n]:=\{1,\dots,n\}$ to denote the set of integers from $1$ to $n$.
For a set of edges $A\subseteq E$ and (a tree) $T\subseteq E$, we write\footnote{We put this notation in a box because it is so important and ubiquitous in this paper.} 
$$\boxed{\hypertarget{tar:AT}{A_T = |A \cap T|}.}$$
For a set $S\subseteq V$, we write 
$$E(S)=\{\{u,v\}\in E: u,v\in S\}$$ to denote the set of edges in $S$ and we write 
$$\delta(S)=\{\{u,v\}\in E: |\{u,v\}\cap S|=1\}$$ 
to denote the  set of edges that leave $S$. 
For two {\em disjoint} sets of vertices $A,B\subseteq V$, we write
$$ E(A,B)=\{\{u,v\}\in E: u\in A, v\in B\}.$$
For a set $A\subseteq E$ and a function $x:E\to\R$ we write
$$ x(A):=\sum_{e\in A} x_e.$$
\hypertarget{tar:crossing}{For two sets $A,B\subseteq V$, we say $A$ {\em crosses} $B$ if all of the following sets are non-empty:
$$ A\cap B, A\smallsetminus B, B\smallsetminus A, \overline{A\cup B}.$$}
\hypertarget{tar:G=(V,E,x)}{We write $G=(V,E,x)$ to denote an (undirected) graph $G$ together with special vertices $u_0,v_0$ and a weight function $x:E\to\R_{\geq 0}$. Similarly, let $G_0 = (V,E_0,x^0)$ and let $G_{/ e_0} = G_0/\{e_0\}$, i.e. $G_{/ e_0}$ is the graph $G_0$ with the edge $e_0$ contracted.}% }% such that 
%$$x(\delta(S))\geq 2, \quad\quad \forall S\subsetneq V: u_0,v_0\notin S.$$}




\subsection{Polyhedral background}
For any graph $G=(V,E)$,
Edmonds \cite{Edm70} gave the following description for the convex hull of spanning trees of a graph $G=(V,E)$, known as the {\em spanning tree polytope}.
\begin{equation}
\begin{aligned}
& z(E) = |V|-1 & \\
& z(E(S)) \leq |S|-1 &  \forall S\subseteq V\\
& z_e \geq 0 & \hspace{6ex} \forall e\in E.
\end{aligned}
\label{eq:spanningtreelp}
\end{equation}
Edmonds \cite{Edm70} proved that the extreme point solutions of this polytope are the characteristic vectors of the spanning trees of $G$. 



%We formally define tight sets in \cref{subsec:algorithm} but for now assume $S$ is tight if $x(\delta(S))=2$. In the half-integral case this corresponds to $|\delta(S)|=4$.
\begin{fact} \label{fact:sptreepolytope}
Let $x^0$ be a feasible solution of \eqref{eq:tsplp} such that $x^0_{e_0}=1$ with support $E_0=E\cup \{e_0\}$. %Let $E$ be the support of $x$ excluding $e_0$. Then, 
Let $x$ be $x^0$ restricted to $E$; then $x$ is in the spanning tree polytope of $G=(V,E)$. 
\end{fact}
\begin{proof}
%Let $x$ be the restriction of $x$ to $E$.
For any set $S\subseteq V$ such that $u_0,v_0\notin S$, $x(E(S))=\frac{2|S|-x^0(\delta(S))}{2}\leq |S|-1$.
If $u_0\in S, v_0\notin S$, then
$x(E(S)) = \frac{2|S|-1 - (x^0(\delta(S)) -1 )}{2}\leq |S|-1$.
Finally, if $u_0,v_0\in S$, then 
$x(E(S)) = \frac{2|S|-2 - x^0(\delta(S))}{2} \leq |S|-2$.
The claim follows because $x(E)=x^0(E_0)-1=n-1$.
 \end{proof}


Since $c(e_0)=0$, the following fact is immediate.
\begin{fact} \label{fact:expcostT}Let $G=(V,E,x)$ where  $x$ is in the spanning tree polytope. If $\mu$ is any distribution of spanning trees with marginals $x$ then $\EE{T\sim\mu}{c(T \cup e_{0})}=c(x)$.
 \end{fact}
 
 To bound the cost of the min-cost matching on the set $O$ of odd degree vertices of the tree $T$, we use the following characterization of the $O$-join polyhedron\footnote{The standard name for this is the $T$-join polyhedron. Because we reserve $T$ to represent our tree, we call this the $O$-join polyhedron, where $O$ represents the set of odd vertices in the tree.} due to Edmonds and Johnson \cite{EJ73}.%%%copied from tsp-journal
\begin{proposition}
\label{prop:tjoin}
For any graph $G=(V,E)$, cost function $c: E \to \R_+$, and a set $O\subseteq V$ with an even number of vertices,  the minimum weight of an $O$-join equals the optimum value of the following integral linear program.
\begin{equation}
\begin{aligned}
\min \hspace{4ex} & \cost(y) \\
\st \hspace{3ex} & y(\delta(S)) \geq 1 & \forall S \subseteq V, |S\cap  O| \text{ odd}\\
& y_e \geq 0 & \forall e\in E
\end{aligned}
\label{eq:tjoinlp}
\end{equation}
\end{proposition}

\begin{definition}[Satisfied cuts]\label{def:satisfiedcuts}
\hypertarget{tar:satisfy}{For a set $S\subseteq V$ such that $u_0,v_0\notin S$ and a spanning tree $T\subseteq E$ we say a vector $y:E\to\R_{\geq 0}$ 	satisfies $S$ if one of the following holds:
\begin{itemize}
\item $\delta(S)_T$ is even, or
\item $y(\delta(S))\geq 1$.	
\end{itemize}}
\end{definition}
To analyze this class of algorithms, the main challenge is to construct a (random) vector $y$ that satisfies all cuts (with probability 1) and for which $\E{c(y)}\leq (1/2-\eps)c(x)$.  
 
%To bound the cost of the min-cost matching on the set $O$ of odd degree vertices of the tree $T$, we use the following characterization of the $O$-join polytope\footnote{The standard name for this is the $T$-join polytope. Because we reserve $T$ to represent our tree, we call this the $O$-join polytope, where $O$ represents the set of odd vertices in the tree.} due to Edmonds and Johnson \cite{EJ73}.
%%%%copied from tsp-journal
%\begin{proposition}
%\label{prop:tjoin}
%For any graph $G=(V,E)$, cost function $c: E \to \R_+$, and a set $O\subseteq V$ with an even number of vertices,  the minimum weight of an $O$-join equals the optimum value of the following integral linear program.
%\begin{equation}
%\begin{aligned}
%\min \hspace{4ex} & \cost(y) \\
%\st \hspace{3ex} & y(\delta(S)) \geq 1 & \forall S \subseteq V, |S\cap  O| \text{ odd}\\
%& y_e \geq 0 & \forall e\in E
%\end{aligned}
%\label{eq:tjoinlp}
%\end{equation}
%\end{proposition}

%\begin{definition}[Satisfied cuts]\label{def:satisfiedcuts}
%\hypertarget{tar:satisfy}{For a set $S\subseteq V$ such that $u_0,v_0\notin S$ and a spanning tree $T\subseteq E$ we say a vector $y:E\to\R_{\geq 0}$ 	satisfies $S$ if one of the following holds:
%\begin{itemize}
%\item $\delta(S)_T$ is even, or
%\item $y(\delta(S))\geq 1$.	
%\end{itemize}}
%\end{definition}

\subsection{Near Min Cuts}
\begin{definition}\hypertarget{tar:nearmincut}{For $G=(V,E,x)$, we say a cut $S\subseteq V$ is an {\em $\eta$-near min cut} if $x(\delta(S))< 2+\eta$.\footnote{Note this differs slightly from the notation in \cite{Ben95, BG08} and \cref{sub:newtechniques} in which an $\eta$ near min cut is said to be within a $1+\eta$ factor of the edge connectivity of the graph.}}
\end{definition}

%For a vertex, $v$, we say a cut $(\{v\}, \overline{\{v\}})$ is a {\em singleton} cut. 

The following lemma is a standard fact about crossing near min cuts:
\begin{lemma}\label{lem:cutdecrement}
For $G=(V,E,x)$, let $A,B\subsetneq V$ be two crossing $\eps_A, \eps_B$ near min cuts respectively. Then,
$A\cap B, A\cup B, A\smallsetminus B, B\smallsetminus A$ are $\eps_A+\eps_B$ near min cuts.
\end{lemma}
\begin{proof}
We prove the lemma only for $A\cap B$; the rest of the cases can be proved similarly.
%Since the cut function $|\delta(.)|$ is a submodular function we have
By submodularity,
$$ x(\delta(A\cap B)) + x(\delta(A\cup B)) \leq x(\delta(A)) + x(\delta(B)) \leq 4+\eps_A+\eps_B.$$
Since $x(\delta(A\cup B))\geq 2$, we have $x(\delta(A\cap B))\leq 2+\epsilon_A+\eps_B$, as desired.
\end{proof}

\begin{lemma}\label{lem:sub-NMC-shared}
If $A,B\subsetneq V$ are disjoint and $C=A \cup B$ is an $\epsilon$ near min cut then $x(E(A,B)) \ge 1 - \frac{\epsilon}{2}$.
\end{lemma}
\begin{proof}
$$2+\epsilon \ge x(\delta(C)) = x(\delta(A)) + x(\delta(B)) - 2 \cdot x(E(A,B)) \ge 4 - 2 \cdot x(E(A,B))$$ 
And the claim follows.
\end{proof}

The following lemma is proved in \cite{Ben97}:
\begin{lemma}[{\cite[Lem 5.3.5]{Ben97}}]
\label{lem:nmcuts_largeedges}
For $G=(V,E,x)$, let $A,B\subsetneq V$ be two crossing $\epsilon$-near minimum cuts. %such that %$x(\delta(A)),x(\delta(B)) \le 2+\epsilon$. Then 
Then, $$x(E(A\cap B, A\setminus B)),x(E(A\cap B, B\setminus A)), x(E(\overline{A\cup B}, A\setminus B)), x(E( \overline{A\cup B}, B\setminus A)) \geq (1-\epsilon/2).$$
%Let $(A,\overline{A})$ and $(B,\overline{B})$ be two crossing $(1+\epsilon)$ near minimum cuts of $G$. Then $|E(A\cap B, A\setminus B)| \geq (1-\epsilon)\frac{\con}{2}$.
\end{lemma}

\begin{lemma}
\label{lem:shared-edges}
For $G=(V,E,x)$, let $A,B\subsetneq V$ be two $\eps$ near min cuts  such that $A \subsetneq B$. Then 
$$x(\delta(A) \cap \delta(B)) = x(E(A,\overline{B}))\le 1 + \eps, \text{ and }$$
$$x(E(A,B \smallsetminus A))\geq 1-\eps/2. $$
%Let $S,T$ be two sets such that $S \subsetneq T$ and $|\delta(S)|,|\delta(T))| \le \con(1+\epsilon)$. Then 
%$$|\delta(S) \cap \delta(T)| = |E(S,\overline{T})|\le \bigg(\frac{1}{2}+\epsilon\bigg)\con.$$
\end{lemma}
\begin{proof}
Notice
\begin{align*}&2+\epsilon \ge x(\delta(A)) = x(E(A,B \smallsetminus A)) + x(E(A,\overline{B}))\\
&2+\epsilon \ge x(\delta(B)) = x(E(B \smallsetminus A,\overline{B})) + x(E(A,\overline{B}))
\end{align*}
Summing these up, we get
$$2x(E(A,\overline{B})) + x(E(A,B \smallsetminus A)) + x(E(B \smallsetminus A, \overline{B})) = 2x(E(A,\overline{B}))+x(\delta(B\smallsetminus A)) \le 4+2\eps.$$
Since $B \smallsetminus A$ is non-empty,
	$x(\delta(B\smallsetminus A)) \ge 2$,
which implies the first inequality.
To see the second one, let $C=B\smallsetminus A$ and note
$$ 4\leq x(\delta(A))+x(\delta(C)) = 2 x(E(A,C)) + x(\delta(B))\leq 2 x(E(A,C))+ 2+\eps$$
which implies $x(E(A,C))\geq 1-\eps/2$.
\end{proof}



\subsection{Random spanning trees}
The following simple lemmas appear in e.g. \cite{KKO21}:

\begin{lemma}\label{lem:treeconditioning}
Let $G=(V,E,x)$, and let $\mu$ be any distribution over spanning trees with marginals $x$. 
For any  $\eps$-near min cut $S\subseteq V$ 
(such that none of the endpoints of $e_0=(u_0,v_0)$ are in $S$), we have
%Write $x$ on $E\smallsetminus e_0$ as any distribution $\mu$ of spanning trees; then 
$$\PP{T\sim\mu}{S \text{ is a subtree of $T$}} = \PP{T \sim \mu}{|T \cap E(S)| = |S|-1} \ge 1- \eps/2.$$ 
\end{lemma}
\begin{proof}
First, observe that
$$\E{E(S)_T}= x(E(S)) \geq  \frac{2|S|-x(\delta(S))}{2} \geq |S|-1 -\eps/2,$$ 
where we used that since $u_0,v_0\notin S$, for any $v\in S$ we have $\E{\delta(v)_T)} = x(\delta(v))=2$.

Let $p_S=\PP{T \sim \mu}{S\text{ is a subtree of $T$}}$.
Then, we must have
$$ |S|-1 - (1-p_S) = p_S(|S|-1) + (1-p_S)(|S|-2)\ge \E{E(S)_T} \geq |S|-1 - \eps/2.$$
Therefore, $p_S\geq 1-\eps/2$.
\end{proof}
\begin{corollary}\label{lem:treeoneedge}
Let  $A,B\subseteq V$ be disjoint sets such that  $A,B,A\cup B$ are $\eps_A,\eps_B,\eps_{A\cup B}$-near minimum cuts w.r.t., $x$ respectively, where none of them  contain endpoints of $e_0$.  Then for any distribution $\mu$ of spanning trees on $E$ with marginals $x$,
$$\PP{T\sim \mu}{E(A,B)_T=1}\geq 1-(\eps_A+\eps_B+\eps_{A\cup B})/2.$$	
\end{corollary}
\begin{proof}
By the union bound, with probability at least $1-(\eps_A+\eps_B+\eps_{A\cup B})/2$, $A,B,$ and $A\cup B$ are trees. 
But this implies that we must have exactly one edge between $A,B$.
\end{proof}

The following simple fact also holds by the union bound.
\begin{fact}\label{fact:0edgerandomspanningtree}
Let $G=(V,E,x)$ and let $\mu$ be a distribution over spanning trees with marginals $x$. For any set $A\subseteq E$	, we have
$$ \PP{T\sim\mu}{T\cap A=\varnothing} \geq 1-x(A).$$
\end{fact}


