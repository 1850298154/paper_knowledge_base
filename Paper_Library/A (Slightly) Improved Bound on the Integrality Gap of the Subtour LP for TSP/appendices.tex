
\appendix
\section{Cuts crossed on one side}\label{app:oneside}



\subsection{Cuts crossed on one side}\label{sec:oneside}
%${\cal N}$ all near mincuts, ${\cal N}_1}$ crossed on onside and ${\cal N}_2$
In \cref{thm:cutsbothsideswithinside}, we found a vector which satisfies all cuts crossed on both sides. Consequently, we can study the structure of cuts which remain after deleting all cuts crossed on both sides, i.e. connected components of cuts crossed on one side. The arguments in this section closely follow Section 4.3 of \cite{KKO21}. Even though in that work these families were handled using OPT edges, the extension to charging to LP edges is very natural in this setting and requires little modification. 

\begin{lemma}\label{obs:one-side-interval}
%Let $\eta \le 2/5$ and suppose $\cC$ is a connected component of cuts in $\cN_\eta$.  Suppose $\cC_1$ is a connected component of $\cC \cap \cN_{\eta,1}$ with $|\cC_1| \ge 2$  and polygon $P_1$. (Recall $\cC \cap \cN_{\eta,1}$ are the cuts which are crossed on one side in the polygon $P$ corresponding to $\cC$.) Then, the following are true:
%\begin{enumerate}
%\item $P_1$ has no inside atoms.
%\item Every cut in $\cC_1$ is crossed on one side in $P_1$. 	
%\end{enumerate}
Suppose $\mathcal{C}$ is a connected component of cuts in $\cN_{\eta,1}$ with $|\cC| \ge 2$. If $\eta \le \frac{2}{5}$, the corresponding polygon $P$ of $\mathcal{C}$ has no inside atoms. %Furthermore, each cut $C \in \mathcal{C}$ is crossed on exactly one side in $P$.
\end{lemma}
\begin{proof}
By \cref{def:inside}, to show that the polygon of $\cC$ has no inside atoms it is sufficient to show that there are no $k$-cycles for any integer $k$. 
Since $\eta\leq 2/5$, by \cref{lem:no-kcycle} there are no $3$ or $4$-cycles.
By way of contradiction suppose there was a $k$-cycle $C_1,\dots,C_k \in \cC$ with $k\geq 5$. Then, perhaps after renaming, we can assume that $C_1,C_2,C_3$ do not contain the root, $C_1$ and $C_3$ each cross $C_2$, and $C_1 \cap C_3 = \emptyset$. Then by \cref{lem:characterize-N2} (below), $C_2 \in \cN_{\eta,2}$, which is a contradiction since we assumed $\cC \subseteq \cN_1$. Therefore, $P$ has no $k$-cycles for $k\geq 5$. So, $P$ has no inside atoms. 
%Next, fix a cut $C \in \mathcal{C}$. We will prove that $C$ is crossed on exactly one side in $P$. Note that $C$ must be crossed on at least one side (since $\cC$ is connected). So, for contradiction, suppose $C$ is crossed by $C_1$ on the left and $C_2$ on the right. Since $P$ has no inside atoms, $(C_1 \smallsetminus C) \cap (C_2 \smallsetminus C) = \emptyset$. By \cref{lem:characterize-N2} the lemma follows.
\end{proof} 

Also note that by \cref{lem:characterize-N2}, if $\cC$ is a connected component of cuts in $\cN_{\eta,1}$, then every cut $C \in \cC$ is crossed on one side in the polygon of $\cC$. (In other words, deleting the cuts in $\cN_{\eta,2}$ does not allow a cut previously crossed on one side to be crossed on both sides in its new polygon.)

%\begin{lemma}\label{lem:characterize-N2}
%Let $C \in \cN_\eta$. Suppose that there are two cuts $A,B \in \cN$ which cross $C$ such that $(A \smallsetminus C) \cap (B \smallsetminus C) = \emptyset$. Then, $C \in \cN_{\eta,2}$. 
%\end{lemma}
%\begin{proof}
%	Since $A$ and $B$ both cross $C$, it must be that $C,A,B$ are in the same connected component of cuts $\cC \subseteq \cN$ (i.e. including all cuts in $\cN_{\eta,2}$). Let $P'$ be the corresponding polygon.
%	
%	%By \cref{obs:Ocross}, $O(A)$ and $O(C)$ cross and $O(B)$ and $O(C)$ cross. \anna{Why do you need the last sentence?} 
%	Suppose by way of contradiction that $A,B$ both crossed $C$ on the left (if they both cross on the right the argument is similar). Then $A,B$ must both contain the outside atom immediately to the left of the leftmost atom of $C$, which contradicts $(A \smallsetminus C) \cap (B \smallsetminus C) = \emptyset$.
%\end{proof}
%\begin{definition}	
%	 Let $\cN_{\eta,\leq 1}$ be the family of cuts in $\cN_\eta$ which are crossed on one side. Note that $\cN\neq \cN_1\cup\cN_{\eta,2}$ as it has cuts which are not crossed.
%\end{definition}

%By \cref{obs:one-side-interval}, 
%Consider the family ${\cal N}_1$ of all $2+\eta$ near min cuts which are not crossed on both sides in their polygon representation. In this section we will consider a connected component of cuts $\cC$ of ${\cal N}_1$ with $|\cC| \ge 2$ with corresponding polygon $P$.
%Before stating the main result of this section, we first prove that, perhaps surprisingly, $P$ has no inside atoms. So, of polygon representations of cuts crossed on one side is much simpler.

\subsection{Notation and results from \cite{KKO21}}

As before suppose $\cC$ is a connected component of cuts crossed on one side with corresponding polygon $P$. Now assume $P$ has outside atoms $a_0,\dots,a_{m-1}$, and WLOG assume $a_0$ is the root (recall there are no inside atoms). %, and arcs labelled $1,\dots,m$ where arc $i$ connects $a_{i-1}$ to $a_i$.
%We remark that this section handles such families in an almost identical way to \cite{KKO21}. 

\begin{definition}[Leftmost and Rightmost cuts]
	  We call any cut $C\in {\cal C}$ with leftmost atom $a_1$  a {\em leftmost} cut of $P$, and any cut $C\in {\cal C}$ with rightmost atom $a_{m-1}$  a {\em rightmost} cut of $P$. 
		We also call $a_1$ the leftmost atom of $P$ (resp.  $a_{m-1}$ the rightmost atom).
	%if $z(\delta(a_{1}))\leq 2+\eta$ (resp. $z(\delta(a_{m-1}))\leq 2+\eta$).
\end{definition}
%Observe that by \cref{cor:leftmostrightmostancestor}, any cut that is not a leftmost or a rightmost cut has a strict ancestor. 

In \cite{KKO21}, it was shown that polygons of cuts crossed on one side have a simple structure. In particular, they look like a near-integral cycle: 

\begin{theorem}[Structure of Polygons of $\cN_{\eta,1}$ (Theorem 4.9 from \cite{KKO21})]\label{thm:poly-structure}
For  $\eps_{\eta}\geq 7\eta$ and any polygon of cuts crossed on one side with atoms $a_0...a_{m-1}$ (where $a_0$ is the root) the following holds:
	\begin{itemize}
	\item For all adjacent atoms $a_i,a_{i+1}$ (also including $a_0,a_{m-1}$), we have $x(E(a_i,a_{i+1})) \ge 1-\eps_\eta$. 
	\item All atoms $a_i$ (including the root) have $x(\delta(a_i)) \le 2+\eps_\eta$. 
	\item $x(E(a_0, \{a_2,\dots,a_{m-2}\}))\leq \eps_{\eta}$.
 %In addition $x(\{a_0...a_{m-1}\}) \le 2+\eps_\eta$. 
	\end{itemize}
\end{theorem}



\begin{definition}[$A,B,C$-Polygon Partition]\label{def:abcpolygonpartitioning}
The $A,B,C$-polygon partition of a polygon $P$ is a partition of edges of $\delta(a_0)$ into sets $A=E(a_1,a_0)$ and $B=E(a_{m-1},a_0)$, $C=\delta(a_0)\smallsetminus A\smallsetminus B$. 
\end{definition}

\begin{definition}[Happy Polygons]
%Let $P$ be a polygon with polygon partition $A,B,C$.
For a spanning tree $T$, we say that a polygon  $P$ of cuts crossed on one side is {\em happy} if 
$$A_T \text{ and } B_T \text{ odd}, C_T=0.$$

We say that $P$ is {\em left-happy} (respectively {\em right-happy}) if 
$$A_T \text{ odd}, C_T=0,$$
(respectively $B_T \text{ odd}, C_T=0$).
\end{definition}



%Recall for each edge $i$ of the polygon, $x(E(a_{i-1},a_i)) \ge 1-\epsilon_\eta$. 

%Recall for a cut $L\in\cL$, $L_r$ is the near minimum cut crossing $L$ on the right that minimizes the intersection.
\begin{definition}[Happy Cut]We say a leftmost cut $L\in \cC$  is {\em happy} if
$$ E(L, \overline{L\cup a_0})_T=1. % (E(L,\overline{a_0\cup L\cup L_r})\cup E(L\smallsetminus L_r, L_r))\cap T=\emptyset,
$$
%where $L_r$ is defined in \cref{def:SlSL}
%Also,  we say the leftmost atom $a_1$ is {\em happy} if
%$$ |T\cap E(a_1,a_2)|=1, E(a_1,a_3\cup \dots\cup a_{m-1})\cap T=\emptyset.$$
Similarly, the leftmost atom $a_1$ is {\em happy} if $E(a_1,\overline{a_0\cup a_1})_T=1$. Define  rightmost cuts in $u$ or the rightmost atom in $u$ to be happy, similarly.
\end{definition}
Note that, by definition, if leftmost cut $L$ is happy and $P$ is left happy then $L$ is even, i.e., $\delta(L)_T=2$. Similarly, $a_1$ is even if it is happy and $P$ is left-happy.

%We will slightly amend the proof of Theorem 4.24 from \cite{KKO21} to show the following:
\begin{definition}[Relevant Cuts]
%Given a polygon $P$ corresponding to a connected component ${\cal C}$ of cuts crossed on one side with atoms $a_0,\dots,a_{m-1}$, 
Define the family of relevant cuts of a polygon $P$ representing a connected component $\cC \subseteq \cN_{\eta,1}$ as follows:
$$\cC_{+}={\cal C}\cup \{a_i:  1\leq i\leq m-1 \land x(\delta(a_i))\leq 2+\eta\}.$$
\end{definition}	


\begin{lemma}[{\cite[Lemma 4.28]{KKO21}}]\label{lem:4cutmapping}%Let $E_1=E(a_1,a_2),\dots,E_{m-2}=E(a_{m-2},a_{m-1})$.
There is a {\em mapping} of cuts in $\cC_+$ to the collections of edges $E(a_1,a_2),\dots,E(a_{m-2},a_{m-1})$ such that each set $E(a_i,a_{i+1})$ has at most 4 cuts mapped to it, every cut $C \in \cC_+$ containing atoms $a_i$ through $a_j$ is mapped to either $E(a_{i-1},a_i)$ or $E(a_j,a_{j+1})$ (or both), and every atom of the polygon in ${\cal C}'$ gets mapped to two (not necessarily distinct) groups of edges $E(a_i,a_{i+1})$, $E(a_j,a_{j+1})$. 
\end{lemma}

Note in the following three statements, we gain a factor of two compared to \cite{KKO21} as we look at $\eta$-near min cuts instead of $2\eta$-near min cuts.

\begin{lemma}[{\cite[Lemma 4.26]{KKO21}}]\label{lem:Astrictpeven}
For every cut $A\in {\cal C}$ that is not a leftmost or a rightmost cut, $\P{\delta(A)_T=2} \geq 1-11\eta$.
\end{lemma}

\begin{lemma}[{\cite[Lemma 4.27]{KKO21}}]\label{lem:atoms-even-whp}
	For any atom $a_i\neq a_0$ that is not the leftmost or the rightmost atom we have
	$$\P{\delta(a_i)_T = 2} \ge 1-21\eta.$$
\end{lemma}

\begin{lemma}[{\cite[Lemma 4.30]{KKO21}}]\label{lem:even-cuts-cond-on-happy}
%If $u$ is left-happy, then 
For every leftmost or rightmost cut $A$ in $P$ that is an $\eta$-near min cut, $\P{A\text{ happy}}\geq 1-5\eta$, and for the leftmost atom $a_1$ (resp. rightmost atom $a_{m-1}$), if it is an $\eta$-near min cut then  $\P{a_1\text{ happy}}\geq 1-12\eta$ (resp. $\P{a_{m-1}\text{ happy}}\geq 1-12\eta$).
\end{lemma}




\subsection{Main theorem for cuts crossed on one side}

The following is our extension of Theorem 4.24 in \cite{KKO21}. This is the key theorem used to deal with components cuts crossed on one side and the atoms in $\cN_\eta$ which compose them.

\begin{restatable}[Happy Polygons (Similar to Theorem 4.24 in \cite{KKO21})]{theorem}{thmcrossedoneside}\label{thm:crossed-one-side}
	Let $G=(V,E,x)$ for an LP solution $x$.
	Let $\mu$ be an arbitrary distribution of spanning trees with marginals $x$. For any $\alpha>0$, $\eta\leq 1/10$, and $\eps_\eta=7\eta$, there is a random vector $s^*:E\to\R_{\geq 0}$ (as a function of $T\sim\mu$) such that 
	% Then there is a procedure that increases OPT edges in the $O$-Join such that the following holds:
	\begin{itemize}
		\item For a connected component ${\cal C}$ of cuts crossed on one side with corresponding polygon $P$ and atoms $a_0,a_1...a_{m-1}$ and cycle partition $A,B,C$ the following holds:
		\begin{itemize}
		\item For any cut $S\in \cC_+$ which is not a leftmost/rightmost cut/atom if  $\delta(S)_T$ is odd then we have $s^*(\delta(S))\geq \alpha(1-\epsilon_\eta)$,
		\item If $P$ is left happy, then for any $S\in \cC_+$ that is a leftmost cut or the leftmost atom, if $\delta(S)_T$ is odd, then we have 
		$s^*(\delta(S))\geq \alpha(1-\epsilon_\eta)$. 
		\item Similarly, if $P$ is right happy then for any cut $S\in \cC_+$ that is  a rightmost cut or the rightmost atom, if $\delta(S)_T$ is odd, then $s^*(\delta(S))\geq \alpha(1-\epsilon_\eta)$.		
		\end{itemize}
		\item $\E{s^*_{e}}\leq 44\alpha\eta x_e$ for all $e \in E$.
	\end{itemize}
%	Moreover, the increase of any OPT edge due to this procedure is at most $210\eta(\eta/4)$ in expectation. 
\end{restatable} 





%\begin{theorem}[{\cite[Theorem 4.9]{KKO21}}]\label{thm:poly-structure}
%For any connected component of cuts in $\cN_{\eta,\leq 1}$ with (outside) atoms $a_0...a_{m-1}$ (where $a_0$ is the root) the following\footnote{Note that in \cite{KKO21} we look at $2+2\eta$ near minimum cuts of $x$ but here we look at $2+\eta$ near minimum cuts. Therefore we have divided $\eps_\eta$ by two compared to the original.} holds for $\eps_\eta = 7\eta$.
%	\begin{itemize}
%	\item For all adjacent atoms $a_i,a_{i+1}$ (also including $a_0,a_{m-1}$), we have $x(E(a_i,a_{i+1})) \ge 1-\eps_\eta$. 
%	\item All atoms $a_i$ (including the root) have $x(\delta(a_i)) \le 2+\eps_\eta$. 
%	\item $x(E(a_0, \{a_2,\dots,a_{m-2}\}))\leq \eps_{\eta}$.
% %In addition $x(\{a_0...a_{m-1}\}) \le 2+\eps_\eta$. 
%	\end{itemize}
%\end{theorem}
%The interpretation of this theorem is that the structure of a polygon of cuts crossed on one side converges to the structure of an actual integral cycle as $\eta\to 0$.
%
%Note that by \cref{thm:poly-structure}, $x(A),x(B)\geq 1-\eps_\eta$ and $x(C)\leq \eps_\eta$ where $\eps_\eta= 7\eta$ is defined in \cref{thm:poly-structure}.

Before proving the theorem, we study a special case.
\begin{lemma}[\cref{thm:crossed-one-side} Holds for Triangles]\label{lem:trianglereduction}
	Let $S=X\cup Y$ where $X,Y,S$ are $\eps_{\eta}$-near min cuts which do not cross. Then, letting $X$ be $a_1$ and $Y$ be $a_2$ (and $a_0=\overline{X\cup Y}$) \cref{thm:crossed-one-side} holds. %there is a random vector $s^*:E^*\to\R_{\geq 0}$ such that 
\end{lemma}
\begin{proof}
In this case this system has cycle partition $A=E(a_1,a_0), B=E(a_2,a_0), C=\emptyset$.
For the edges $E(a_1,a_2)$ we define an increase event $\cI$ when at least one of $T\cap E(X)$, $T\cap E(Y)$, $T\cap E(S)$ is not a tree.
Whenever this happens we define $s^*_{e}=\alpha x_e$ for all $e \in E(a_1,a_2)$.
If $S$ is left-happy we need to show when $\delta(X)_T$ is odd, then $s^*(\delta(X))\geq \alpha(1-\epsilon_\eta)$. This is because when $S$ is left-happy we have $A_T=1$ (and $C_T=0$), so either the increase event $\cI$ does not happen and we get $\delta(X)_T=2$ or it happens in which case $s^*(\delta(X)) = \alpha \cdot x(E(a_1,a_2)) \geq \alpha(1-\epsilon_\eta)$ by \cref{lem:sub-NMC-shared}.
Finally, observe that by \cref{lem:treeoneedge}, 
$\P{\cI}\le 3 \eps_\eta/2$, so
$\E{s^*_{e}}=1.5\eps_\eta \alpha x_e < 44\alpha\eta x_e$ for all $e \in E(a_1,a_2)$. 
\end{proof}




%The above statements allow us to prove \cref{thm:crossed-one-side}. 
\begin{proof}[Proof of \cref{thm:crossed-one-side}]
Fix a connected component $\cC$ of ${\cN_1}$ with corresponding polygon $P$. Fix $1<i<m$. By \cref{thm:poly-structure} $x(E(a_{i-1},a_i)) \ge 1-\epsilon_\eta$. 

For the at most four cuts mapped to $E(a_{i-1},a_i)$ in \cref{lem:4cutmapping}, we define the following three events:
\begin{enumerate}[i)]
\item A leftmost cut mapped to $E(a_{i-1},a_i)$ is not happy.
\item A rightmost cut mapped to $E(a_{i-1},a_i)$ is not happy.
\item A cut which is not leftmost or rightmost mapped to $E(a_{i-1},a_i)$ is odd.
\end{enumerate}
Observe that the cuts in (i) and (ii) are assigned to $E(a_{i-1},a_i)$ in \cref{lem:4cutmapping}. We say an atom $a$ is singly-mapped to $E(a_{i-1},a_i)$ if in the matching $a$ is only mapped to $E(a_{i-1},a_i)$ once, otherwise we say it is doubly-mapped to $E(a_{i-1},a_i)$.

We say an event $\cI(E(a_{i-1},a_i))$ occurs if either (i), (ii), or (iii) occurs. If $\cI(E(a_{i-1},a_i))$ occurs then for all $e \in E(a_{i-1},a_i)$, we set:
\begin{align*}
	s^*_{e} = \begin{cases}\alpha x_e & \text{If (i),(ii), or (iii) occurred for at least one non-atom cut in $\cC'$, or for an atom}\\ &\text{which is doubly-mapped to $E(a_{i-1},a_i)$} \\
		\alpha x_e /2 & \text{Otherwise.}
	 \end{cases}
\end{align*}
If $\cI(E(a_{i-1},a_i))$ does not occur we set $s^*_{e} = 0$ for all $e \in E(a_{i-1},a_i)$. 

%Whenever the increase event of $E(a_{i-1},a_i)$ occurs, we define $s^*_e = \alpha x_e$ for all $e \in E(a_{i-1},a_i)$. Otherwise, we define $s^*_e = 0$. 

First, observe that for any non-atom cut $S \in \cC_+$ (i.e. any relevant cut) that is not a leftmost or a rightmost cut/atom, if $\delta(S)_T$ is odd, then if $E(a_{i-1},a_i)$ is the set of edges that $S$ is mapped to, it satisfies $s^*(\delta(S)) \ge \alpha \cdot x(E(a_{i-1},a_i)) \ge \alpha(1-\epsilon_\eta)$. So, these cuts satisfy the conditions of the theorem.

The same inequality holds for non-leftmost/rightmost atom cuts $a\in {\cal C}'$ which are doubly-mapped to $E(a_{i-1},a_i)$. For non-leftmost/rightmost atom cuts $a\in {\cal C}'$ which are singly-mapped to $E(a_{i-1},a_i)$, $a$ is mapped (possibly even twice) to another edge $E(a_{j-1},a_j)$ (note $j=i-1$ or $i+1$), and in this case $s^*(\delta(S)) \ge \alpha/2 \cdot 2(1-\epsilon_\eta) = \alpha(1-\epsilon_\eta)$, and again the above inequality holds.

Now, suppose $S \in \cC$ is a leftmost cut of $P$ and $\delta(S)_T$ is odd, and the rightmost atom of $S$ is $a_{i-1}$ (i.e. it is mapped to $E(a_{i-1},a_i)$). If $P$ is not left-happy then there is nothing to prove. If $P$ is left-happy, we may assume $S$ is not happy. Then $\cI(E(a_{i-1},a_i))$ happens, so as in the above inequality $s^*(\delta(S)) \ge \alpha(1-\epsilon_\eta)$. We obtain the same condition for rightmost cuts and leftmost/rightmost atoms that are assigned to $P$ (note leftmost/rightmost atoms are always doubly-mapped: $a_1$ to $E(a_{1},a_2)$ and $a_{m-1}$ to $E(a_{m-2},a_{m-1})$).

%So it only remains to upper bound $\E{s_e^*}$ for any $e \in E(a_{i-1},a_i)$. We first bound the cost from criteria (1) of the increase event. An edge $e$ is mapped to at most four cuts due to \cref{lem:4cutmapping}. By \cref{lem:Astrictpeven} and \cref{lem:atoms-even-whp}, all of these cuts are even with probability at least $1-64\eta$. Note that here we are using that only atoms $a_{i-1},a_{i}$ may possibly be mapped to $E(a_{i-1},a_i)$.

%Now, we bound the contribution from (2) and (3). At most one leftmost cut and at most one rightmost cut are mapped to $e$, and $s^*_e$ is set to $\alpha x_e$ if either is not happy. By \cref{lem:even-cuts-cond-on-happy}, in the worst case, at most one of these two cuts is the leftmost or the rightmost atom in which case $e$ increases with probability at most $17\eta$. Summing the contribution from (1) - (3), we have $\E{s_e^*} \le 81\alpha\eta x_e$.

It remains to upper bound $\E{s^*_e}$ for any edge $e\in E(a_{i-1},a_i)$. By \cref{lem:4cutmapping}, at most four cuts are mapped to $E(a_{i-1},a_i)$.

First suppose exactly one atom is doubly-mapped to $E(a_{i-1},a_i)$. Then there are at most three cuts mapped to $E(a_{i-1},a_i)$, including that atom. The probability of an event of type (i) or (ii) occurring for the leftmost or rightmost atom is at most $1-12\eta$ by \cref{lem:even-cuts-cond-on-happy}. Atoms which are not leftmost or rightmost are even with probability at least $1-21\eta$ by \cref{lem:atoms-even-whp}. Therefore, in the worst case, the doubly-mapped atom is not leftmost or rightmost. For the remaining two cuts, leftmost and rightmost cuts are happy with probability at least $1-5\eta$ by \cref{lem:even-cuts-cond-on-happy}, and (non-atom) non leftmost/rightmost cuts are even with probability at least $1-11\eta$ by \cref{lem:Astrictpeven}. Therefore in the worst case the remaining two (non-atom) cuts mapped to $E(a_{i-1},a_i)$ are not leftmost/rightmost. Therefore, if an atom is doubly-mapped to $E(a_{i-1},a_i)$, for any $e \in E(a_{i-1},a_i)$ we have
$$\E{s^*(e)}\leq 21\eta \alpha x_e + 2\cdot 11\eta \alpha x_e < 44\eta \alpha x_e$$
Note if two atoms are doubly-mapped to $E(a_{i-1},a_i)$, there are no other mapped cuts and in the worst case the atoms are not leftmost/rightmost, so for any $e \in E(a_{i-1},a_i)$,
$$\E{s^*(e)}\leq 2\cdot 21\eta \alpha x_e < 44\eta \alpha x_e$$

Otherwise, any atoms mapped to $E(a_{i-1},a_i)$ are singly-mapped. In this case, if only an atom cut is odd/unhappy, we set $s^*(e) = x_e\alpha/2$. The probability of an event of type (i) or (ii) occurring for the leftmost or rightmost atom is at most $1-12\eta$ by \cref{lem:even-cuts-cond-on-happy}, so we can bound the contribution of this event to $\E{s^*(e)}$ by $12\eta \alpha x_e/2$. Atoms which are not leftmost or rightmost are even with probability at least $1-21\eta$ by \cref{lem:atoms-even-whp}, and so we can bound their contribution by $21\eta \alpha x_e/2$. Therefore, in the worst case four non-leftmost/rightmost \textit{non}-atom cuts are mapped to $E(a_{i-1},a_i)$, in which case, for any $e \in E(a_{i-1},a_i)$,
%To see this, note that by \cref{lem:even-cuts-cond-on-happy} events of type (i) and (ii) above occur with probability at most $1-10\eta$ when the cut is not an atom and $1-24\eta$ when the cut is an atom (but we divide the increase in half for atoms) so these cuts incur a cost of at most $12\eta^2/3.9$. Atoms which are not leftmost or rightmost are even with probability at least $1-42\eta$ by \cref{lem:atoms-even-whp}, so the expected cost for them (again dividing by two) is at most $21\eta^2/3.9$. Therefore the highest cost comes from cuts in $\cC$ that are not leftmost or rightmost, for which we pay at most $22\eta^2/3.9$ by \cref{lem:Astrictpeven}, and so in the worst case $e^*_i$ is matched to four of these cuts. 
%Therefore, by the union bound:
$$\E{s^*(e)}\leq 4\cdot 11\eta \alpha x_e = 44\eta \alpha x_e$$ as desired.
\end{proof}
%First, suppose that $i\neq 2,m-1$. 
%By \cref{lem:4cutmapping} at most 4 cuts are mapped to $E(a_{i-1},a_i)$. Since atoms are mapped to twice, we may divide the increase by two (if an atom is assigned to a set of edges twice, we may treat is as two separate events). Therefore in the worst case, there are four cuts which are not leftmost or rightmost mapped to $E(a_{i-1},a_i)$.  Therefore, by the union bound,
% $\E{s^*_e}\leq \alpha x_e (4\cdot 11\eta)=44\alpha\eta x_e$.
%
%This implies for all edges $e$ connecting non-root atoms of $P$, $\E{s^*_e}\leq 44\alpha\eta x_e$. To finish the proof of the theorem observe that any edge $e$ (of $G$) connects non-root atoms of at most one polygon in the polygon representation of cuts in $\cN_{\eta,\leq 1}$.
%Summary: C' is all the NCM plus all the atoms. The matching lemma says we can match all of these cuts to OPT edges (now just polygon edges) such that each edge gets at most 4 cuts. So the above proof first deals with non leftmost/rightmost cuts and non atoms. The second deals with leftmost/rightmost cuts. Remember there are many leftmost/rightmost cuts but any edge has at most 2 for sure (one on each side)
\section{Proof of \cref{thm:maintechnical}}\label{app:proofbeforetechnical}

%\subsection{The hierarchy and payment}\label{sub:hierarchy}
In this section, we use the previous section and \cref{thm:cutsbothsideswithinside} to prove  \cref{thm:maintechnical}. %As noted %in \cref{sec:proof-of-main}, 
%A similar strategy can be used to prove \cref{thm:hierarchyKKO} by simply ``leaving out" the slack vector constructed in \cref{thm:cutsbothsideswithinside}

%The following definition can be found in \cite{KKO21}. We set $\eps_\eta=7\eta$ in the following. 
%\begin{definition}[Hierarchy,\cite{KKO21}]\label{def:hierarchy}
%\hypertarget{tar:hierarchy}{For an LP solution $x^0$ with support $E_0=E\cup \{e_0\}$ where $x$ is $x^0$ restricted to $E$, a hierarchy ${\cal H}\subseteq \cN_{\eps_\eta}$ is a {\em laminar} family with maximal set $V\smallsetminus \{u_0,v_0\}$, where $\eps_\eta=7\eta$ and every cut $S\in \cH$  is either an outer polygon, a triangle (i.e. a cut with exactly two children in $\cH$), or neither of the preceding two, in which case we call it a degree cut. %\footnote{Note in \cite{KKO21}, there are two kinds of near-cycle cuts: those with exactly two children, which are called triangles, and those with more than two children called ``polygons." Here we intentionally remove the use of ``polygon" in this context to avoid confusion with the main topic of this paper. For simplicity of exposition we call both types near-cycle cuts.}.  
%For any (non-root) cut $S\in \cH$, define the parent of $S$, $\p(S)$, to be the smallest cut $S'\in\cH$ such that $S\subsetneq S'$.}
%
%%\hypertarget{tar:AS}{Furthermore, assume that for a cut $S\in \cH$, 
%%let $\cA(S):=\{a\in \cH: \p(a)=S\}$. If $S$ is a ???, then we can order cuts in $\cA(S)$, $a_1,\dots,a_{m-1}$ such that 
%%\begin{itemize}
%%\item $A=E(\overline{S},a_1), B=E(a_{m-1},\overline{S})$ satisfy $x(A),x(B)\geq 1-\eps_\eta$.
%%\item For any $1\leq i<m-1$, $x(E(a_i,a_{i+1}))\geq 1-\eps_\eta$.
%%\item $C=\cup_{i=2}^{m-2} E(a_i,\overline{S})$ satisfies $x(C)\leq \eps_\eta$.
%%\end{itemize}}
%%\anna{Notation $u_i$ vs $a_i$}
%
%%We call the sets $A,B,C$ the cycle partition of edges in $\delta(S)$. We say $S$ is left-happy when $A_T$ is odd and  $C_T=0$ and right happy when $B_T$ is odd and $C_T=0$ and happy when $A_T,B_T$ are odd and $C_T=0$.
%%We mark some of the degree cuts in the hierarchy as {\em important cuts}.
%
%We abuse notation and for an  edge $e=\{u,v\}$ that is not a neighbor of $u_0,v_0$, we write $\p(e)$ to denote the smallest\footnote{In the sense of the number of vertices that it contains} cut $S\in \cH$ such that $u,v\in S$. We say edge $e$ is a \textbf{bottom edge} if $\p(e)$ is an outer polygon cut or a triangle and we say it is a \textbf{top edge} if $\p(e)$ otherwise.
%
%\end{definition}

\begin{definition}[Hierarchy, \cite{KKO21}]\label{def:hierarchy}
\hypertarget{tar:hierarchy}{For an LP solution $x^0$ with support $E_0=E\cup \{e_0\}$ where $x$ is $x^0$ restricted to $E$, a hierarchy ${\cal H}\subseteq \cN_{\eps_\eta}$ is a {\em laminar} family  with root $V\smallsetminus \{u_0,v_0\}$, where every cut $S\in \cH$ is called either a ``near-cycle" cut or a degree cut. In the special case that $S$ has exactly two children we call it a triangle cut. Furthermore, every cut $S$ is the union of its children. 
For any (non-root) cut $S\in \cH$, define the parent of $S$, $\p(S)$, to be the smallest cut $S'\in\cH$ such that $S\subsetneq S'$.}

\hypertarget{tar:AS}{For a cut $S\in \cH$, 
let $\cA(S):=\{a\in \cH: \p(a)=S\}$. If $S$ is called a ``near-cycle" cut, then we can order cuts in $\cA(S)$, $a_1,\dots,a_{m-1}$ such that 
\begin{itemize}
\item $A=E(\overline{S},a_1), B=E(a_{m-1},\overline{S})$ satisfy $x(A),x(B)\geq 1-\eps_\eta$.
\item For any $1\leq i<m-1$, $x(E(a_i,a_{i+1}))\geq 1-\eps_\eta$.
\item $C=\cup_{i=2}^{m-2} E(a_i,\overline{S})$ satisfies $x(C)\leq \eps_\eta$.
\end{itemize}}
%\anna{Notation $u_i$ vs $a_i$}

We call the sets $A,B,C$ the ``near-cycle" partition of edges in $\delta(S)$. We say $S$ is left-happy when $A_T$ is odd and  $C_T=0$ and right happy when $B_T$ is odd and $C_T=0$ and happy when $A_T,B_T$ are odd and $C_T=0$.
%We mark some of the degree cuts in the hierarchy as {\em important cuts}.

We abuse notation and for an  edge $e=(u,v)$ that is not a neighbor of $u_0,v_0$, we write $\p(e)$ to denote the smallest\footnote{in the sense of the number of vertices that it contains} cut $S'\in \cH$ such that $u,v\in S'$. We say edge $e$ is a \textbf{bottom edge} if $\p(e)$ is a polygon cut and we say it is a \textbf{top edge} if $\p(e)$ is a degree cut.


%\hypertarget{tar:AS}{For a cut $S\in \cH$, 
%let $\cA(S):=\{u\in \cH: \p(u)=S\}$. If $S$ is a polygon cut, then by \cref{thm:poly-structure} and \cref{lem:trianglereduction} the polygon of $S$ has no inside atoms and where the outside atoms are $u_1,\dots,u_{m-1}$, the polygon satisfies: 
%\begin{itemize}
%\item $A=E(\overline{S},u_1), B=E(u_{m-1},\overline{S})$ satisfy $x(A),x(B)\geq 1-\eps_\eta$.
%\item For any $1\leq i<m-1$, $x(E(u_i,u_{i+1})\geq 1-\eps_\eta$.
%\item $C=\cup_{i=2}^{m-2} E(u_i,\overline{S})$ satisfies $x(C)\leq \eps_\eta$.
%\end{itemize}}
%%\anna{Notation $u_i$ vs $a_i$}
%
%We call the sets $A,B,C$ the polygon partition of edges in $\delta(S)$. We say $S$ is left-happy when $A_T$ is odd and  $C_T=0$ and right happy when $B_T$ is odd and $C_T=0$ and happy when $A_T,B_T$ are odd and $C_T=0$.
%%We mark some of the degree cuts in the hierarchy as {\em important cuts}.
%
%We abuse notation, and for an (LP) edge $e=(u,v)$ that is not a neighbor of $u_0,v_0$, let $\p(e)$ denote the smallest\footnote{In the sense of the number of vertices that it contains.} cut $S'\in \cH$ such that $u,v\in S'$. We say edge $e$ is a \textbf{bottom edge} if $\p(e)$ is a polygon cut and we say it is a \textbf{top edge} if $\p(e)$ is a degree cut.
\end{definition}

The terminology of the above differs slightly from \cite{KKO21}, where we replace ``polygon" cut with ``near-cycle" cut and ``polygon" partition with ``near-cycle" partition.

By \cref{thm:poly-structure}, an example of a near-cycle cut is the union of non-root atoms  of a connected component of cuts crossed on one side (i.e. its outer polygon cut). Another example is the non-root atoms of a connected component of minimum cuts (i.e. a cycle of a cactus of length at least four).

In the following, we will define a hierarchy $\cH$ satisfying the above definition such that every cut $S \in \cN_{\eta,\le 1}$ is either in $\cH$ or there is a near-cycle cut $P \in \cH$ representing a connected component $\cC$ such that $S \in \cC$.

We will use the following ``main payment theorem" from ~\cite{KKO21}.

\begin{restatable}[Main Payment Theorem (4.33 in \cite{KKO21})]{theorem}{paymentmain}
\label{thm:payment-main}
For an LP solution $x^0$ where $x$ is $x^0$ restricted to $E$ and a hierarchy $\cH$ for some $\eps_\eta\leq 10^{-10}$ and any $\decrease > 0$,
%and $z^*=(x+OPT)/2$, 
the maximum entropy distribution $\mu$ with marginals $x$ satisfies the following:
\begin{enumerate}[i)]
\item There is a set of {\em good} edges $E_g\subseteq E\smallsetminus \delta(\{u_0,v_0\})$ such that any bottom edge $e$ is in $E_g$ and  for any (non-root) $S\in \cH$ such that $\p(S)$ is not a near-cycle cut, we have $x(E_g\cap \delta(S))\geq 3/4$. 
\item There is a random vector $s:E_g \to \R$  (as a function of $T\sim\mu$) such that for all $e$, $s_e\geq -x_e \decrease$ (with probability 1), and \label{payment:non-near-min-cuts}
\item If a near-cycle cut $S$ with cycle partition $A,B,C$ is not left happy, then for any set $F\subseteq E$ with $\p(e)=S$ for all $e\in F$ and $x(F)\geq 1-\eps_\eta/2$, we have
$$ s(A)+s(F)+s^-(C)\geq 0,$$
where $s^-(C)=\sum_{e\in C} \min\{s_e,0\}$.
A similar inequality holds if $S$ is not right happy.
%\textbf{(Leftmost and rightmost cuts are satisfied)} For any odd near minimum cut $S$ that is a leftmost cut of a polygon $u$, if $u$ is not left-happy, then $s(\delta(S))\geq 0$. The analogous claim holds for rightmost cuts.
%\label{payment:satisfy-poly-cuts}
\item %\textbf{(Non-polygon cuts are satisfied)} 
For every cut $S\in \cH$ such that $\p(S)$ is not an near-cycle cut, if $\delta(S)_T$ is odd, then $s(\delta(S))\geq 0$. \label{payment:satisfy-non-poly-cuts}
\item %\textbf{(Good edges have slack)} 
For a good edge $e\in E_g$, $\E{s_e} \le  - \eps_P \decrease x_e$ (where $\eps_P \ge 3.12 \cdot 10^{-16}$) . %, for some $ $ (see \cref{eq:constants} for the values of these constants).\label{payment:good-edges-slack}
\end{enumerate}
\end{restatable}


In \cref{app:proofbeforetechnical}, we show how the main payment theorem along with \cref{thm:crossed-one-side} and \cref{thm:cutsbothsideswithinside} implies the following:

\begin{restatable}{theorem}{beforetechnical}\label{lem:beforetechnicalthm}
Let $x^0$ be a feasible solution of LP \eqref{eq:tsplp} with support $E_0=E\cup\{e_0\}$ with $x$ the restriction of $x^0$  to $E$.
Let $\mu$ be the max entropy distribution with marginals $x$.
For $\eta\leq 10^{-12}$, $\decrease > 0$, there is a set $E_g\subset E\smallsetminus \delta(\{u_0,v_0\})$  of {\em good} edges
%\footnote{The edge sets $E_g$ and $E^*$ need not be disjoint.} with $x_e > 0$. 
and two functions $s: E_0\rightarrow \R$ and $s^*: E \rightarrow \R _{\ge 0}$ (as functions of $T\sim\mu$) such that
	\begin{itemize}
\item[(i)] 	For each edge $e \in E_g$, $s_e \ge -x_e \decrease$ and for any $e\in E\smallsetminus E_g$, $s_e=0$.
\item[(ii)] For each $\eta$-near min cut $S$, including those for which $\{u_0,v_0\}\in\delta(S)$,  if $\delta(S)_T$ is odd, then $  s(\delta(S)) + s^*(\delta(S)) \ge  0.$
\item[(iii)] We have $\E{s_e} \le -\epsilon_P \decrease x_e$  for all edges $e \in E_g$  and $\E{s^*_{e}} \le 125 \eta \decrease x_e$ for all edges $e \in E$, where $\eps_P$ is defined in \cref{thm:payment-main}.
\item[(iv)] For every cut $S$ crossed on at most one side such that $S\neq \overline{\{u_0,v_0\}}$, $x(\delta(S)\cap E_g) \ge 3/4.$
\end{itemize}
\end{restatable}

Now we will use it to prove the appendix theorem, which we already showed implies \cref{thm:main}:
\maintechnical*
\begin{proof}[Proof of \cref{thm:maintechnical}]
Let $E_g$ be the good edges defined in \cref{lem:beforetechnicalthm} and let $E_b:=E\smallsetminus E_g$ be the set of bad edges; in particular, note all edges in $\delta(\{u_0,v_0\})$ are bad edges. We define a new vector $\tilde{s}:E\cup \{e_0\}\to\R$ as follows: 
\begin{equation}\label{eq:tildesdef}\tilde{s}(e)\gets \begin{cases}\infty & \text{if } e=e_0\\
-x_e(4\decrease/5)(1-2\eta) & \text{if } e\in E_b,\\
x_e(4\decrease/3) & \text{otherwise.}
\end{cases}
\end{equation}
Let $\tilde{s}^*$ be the vector $s^*$ from \cref{thm:cutsbothsideswithinside} called with $\alpha = 2\decrease$.
We claim that for any  $\eta$-near minimum cut $S$ such that $\delta(S)_T$ is odd, we have 
$$ \tilde{s}(\delta(S))+\tilde{s}^*(\delta(S))\geq 0.$$
To check this note by (iv) of \cref{lem:beforetechnicalthm} for every set $S\in \cN_{\eta,\leq 1}$ such that $S\neq V\smallsetminus \{u_0,v_0\}$, we have $x(E_g \cap \delta(S)) \ge \frac{3}{4}$, so we have
\begin{equation}\label{eq:tildeScutsok}\tilde{s}(\delta(S))+\tilde{s}^*(\delta(S))\geq \tilde{s}(\delta(S)) = \frac{4\decrease}{3} x(E_g\cap\delta(S)) - \frac{4\decrease}{5}(1-2\eta)x(E_b\cap\delta(S))\geq 0.	
\end{equation}

For $S=V\smallsetminus \{u_0,v_0\}$, we have $\delta(S)_T=\delta(u_0)_T + \delta(v_0)_T=2$ with probability 1, so condition ii) is satisfied for these cuts as well.
Finally, consider cuts $S\in\cN_{\eta,2}$. By \cref{thm:cutsbothsideswithinside}, if $\delta(S)_T$ is odd, then $\tilde{s}^*(\delta(S)) \ge \alpha(1-\eta) = 2\decrease(1-\eta)$. Therefore, in such a case we have:
\begin{equation}\label{eq:tildeScutstwo}\tilde{s}(\delta(S)) + \tilde{s}^*(\delta(S)) \geq  2\decrease(1-\eta) - \frac{4\decrease}{5}(1-2\eta) x(\delta(S))  \geq 0	
\end{equation}
where we use that $x(\delta(S)) \le 2+\eta$. 
%w $\eta<10^{-12}$.)

Now, we are ready to define $s,s^*$. Let $\hat{s},\hat{s}^*$ be the $s,s^*$ of \cref{lem:beforetechnicalthm} respectively.
Define $s=\gamma \tilde{s} + (1-\gamma) \hat{s}$ and similarly define $s^*=\gamma\tilde{s}^*+(1-\gamma)\hat{s}^*$ for some $\gamma$ that we choose later. 
We prove all three conclusions of \cref{thm:maintechnical} for $s,s^*$. (i) follows by (i) of \cref{lem:beforetechnicalthm} and  \cref{eq:tildesdef}. (ii) follows by (ii) of \cref{lem:beforetechnicalthm} and \cref{eq:tildeScutsok,eq:tildeScutstwo} above. %\anna{This is a bit brutal. Maybe in one of the theorems switch from (i),(ii) etc to (a) (b) etc?}
It remains to verify (iii). For edge $e \in E$, $\E{s^*_{e}}\leq 125\eta \decrease x_e$ by (iii) of \cref{lem:beforetechnicalthm} and the construction of $s^*$. On the other hand, by (iii) of \cref{lem:beforetechnicalthm} and \cref{eq:tildesdef},
\begin{align*} 
\E{s_e} \begin{cases}\leq  x_e  (\gamma\frac{4}{3}\decrease - (1-\gamma)\eps_P\decrease ) & \forall e\in E_g,\\	
= -x_e\gamma\cdot (\frac{4}{5}\decrease)(1-2\eta)& \forall e\in E_b.
\end{cases}
\end{align*}
Setting $\gamma=\frac{15}{32}\eps_P$ we get $\E{s_e}\leq -\frac{1}{3}\eps_P\decrease x_e$ for $e\in E_g$ and $\E{s_e}\leq -\frac{1}{3}x_e\decrease\epsilon_P$ for $e\in E_b$ as desired.
\end{proof}






%\begin{proof}
%First observe that whenever $|{\cal C}|=1$ the unique cut in ${\cal C}$ is an $\eta$ near min cut which is not crossed. For a polygon cut $S$ in the hierarchy (i.e. $S$ is the union of all non-root atoms of a polygon), by \cref{thm:poly-structure}, the set $S$ is a $\eps_\eta$ near min cut, and for any atom $a_i$ of a polygon, $a_i\in \cN_{\eta}$.
%
%Now, it remains to show that for a polygon cut $S$ we have a valid ordering $u_1,\dots,u_k$ of cuts in $\cA(S)$. If $S$ is a non-triangle polygon cut, the $u_1,\dots,u_k$ are exactly atoms of the polygon of $S$ and $x(A),x(B)\geq 1-\eps_\eta$ and $x(C)\leq \eps_\eta$ and $x(E(u_i,u_{i+1}))\geq 1-\eps_\eta$ follow by \cref{thm:poly-structure}.
%For a triangle cut $S=X\cup Y$, because $S,X,Y$  are $\eps_\eta$-near min cuts (by the previous paragraph), we get $x(A),x(B)\geq 1-\eps_\eta$ as desired, by \cref{lem:shared-edges}. Finally, since $x(\delta(X)),x(\delta(Y))\geq 2$ we have $x(E(X,Y))\geq 1-\eps_\eta$.
%%Finally, for any cut $S'\in \cH$ that is a descendant of $S$, we must have that $S'$ is a subset of an atom of $S$. Say $S$ has atoms $a_0,\dots,a_{m-1}$; if $S'\subseteq a_1$, then $\delta(S')\cap \delta(S)\subseteq E(a_0,a_1)=A$. Otherwise if $S'\subseteq a_{m-1}$, then $\delta(S')\cap\delta(S)\subseteq E(a_{m-1},a_0)=B$. Otherwise, $\delta(S')\cap\delta(S)\subseteq C$ as desired.
%\end{proof}
\beforetechnical*
\begin{proof}
We start by explaining how to construct $\cH$.
Run the following procedure on $\cN_{\eta,\leq 1}$ (of $x$):
For every connected component ${\cal C}$ of $\cN_{\eta,\leq 1}$, if $|{\cal C}|=1$ then add the unique cut in ${\cal C}$ to the hierarchy. Otherwise, ${\cal C}$ corresponds to a polygon $P$ of cuts crossed on one side with atoms $a_0,\dots,a_{m-1}$ (for some $m>3$).
By \cref{obs:one-side-interval} all these atoms are outside atoms. 
 Add $a_1,\dots,a_{m-1}$ to $\cH$\footnote{Notice that an atom may already correspond to a connected component, in such a case we do not need to add it in this step.} and $\cup_{i=1}^{m-1} a_i$ to $\cH$.
Note that since $x(\delta(\{u_0,v_0\}))=2$, the root of the hierarchy is always $V\smallsetminus \{u_0,v_0\}$.
%Once we process all connected components %,  then add $V$ and call it a degree cut and stop.

Now, we name every cut in the hierarchy.
For a cut $S$, if there is a connected component of at least two cuts with union equal to $S$, then call $S$ a near-cycle cut with $A,B,C$ partition as defined in \cref{def:hierarchy}.
If $S$ is a cut with exactly two children $X,Y$ in the hierarchy (i.e. a triangle), then let $A=E(X,\overline{X}\smallsetminus Y)$, $B=E(Y,\overline{Y}\smallsetminus X)$ and $C=\emptyset$.
Otherwise, call $S$ a degree cut.

\begin{fact}[{\cite[Fact 4.34]{KKO21}}] The above procedure produces a valid hierarchy.	
\end{fact}

The following observation simply follows from the fact that the new cuts that we introduce in the above hierarchy, i.e., atoms and union of non-root atoms of a polygon, are not crossed and are never part of a non-singleton connected component.
\begin{fact}\label{fact:nonsingletoncompcorr}
The set of non-singleton connected components $\cC_1,\cC_2,\dots$  that the above procedure produces are in one-to-one correspondence to the set of non-singleton connected components of $\cN_{\eta,\leq 1}$. 
\end{fact}

Let $E_g$ and $s$ be defined as in \cref{thm:payment-main} for the hierarchy defined above, and let {$s_{e_0}=\infty$}.
Also, let $s^*$ be the sum of the $s^*:E\to\R_{\geq 0}$ vectors from \cref{thm:cutsbothsideswithinside} and \cref{thm:crossed-one-side} called with $\alpha = \frac{2+\eta}{1-\eps_\eta}\decrease$.
(i) follows from (ii) of \cref{thm:payment-main}. Then,
$\E{s^*_{e^*}}\leq (18+44)\eta (\frac{2+\eta}{1-\eps_\eta}\decrease) \le  125\eta \beta$ follows from \cref{thm:cutsbothsideswithinside} and \cref{thm:crossed-one-side} and using that $\eta \le 10^{-12}$ and $\eps_\eta = 7\eta$. 
 Also, $\E{s_e}\leq -\eps_P \decrease x_e$ for edges $e\in E_g$ follows from (v) of \cref{thm:payment-main}.
 
 Now, we verify (iv): For any (non-root) cut $S\in \cH$ such that $\p(S)$ is not a near-cycle cut $x(\delta(S)\cap E_g)\geq 3/4$ by (i) of \cref{thm:payment-main}. The only remaining case is $\eta$-near minimum cuts  which are either atoms or near minimum cuts in a polygon. Fix such a set $S$ in a polygon $P$. 
 Let $S'$ be the union of the non-root atoms of $P$. Then by \cref{lem:shared-edges}, $x(\delta(S)\cap \delta(S'))\leq 1+\eps_\eta$. All edges in  $\delta(S)\smallsetminus \delta(S')$ are bottom edges, so by (i) of \cref{thm:payment-main} are in $E_g$. Therefore, $x(\delta(S)\cap E_g))\geq 1-\eps_\eta\geq 3/4$.
 
 It remains to verify (ii): We consider 5 groups of cuts:
 
{\bf Type 1}: Cuts $S$ such that $e_0\in \delta(S)$. Then, since $s_{e_0}=\infty$, $s(\delta(S))+s^*(\delta(S))\geq 0$.

 {\bf Type 2:} Cuts $S\in\cN_{\eta,2}$. By \cref{thm:cutsbothsideswithinside} and the fact that $\alpha=\frac{2+\eta}{1-\eps_\eta}\decrease$, if $\delta(S)_T$ is odd then 
 $$s^*(\delta(S)) \ge \frac{2+\eta}{1-\eps_\eta}\decrease (1-\eta) \ge (2+\eta)\decrease \ge -s(\delta(S))$$
where we use that $s_e\geq -\decrease x_e$ for all edges $e$ and $x(\delta(S)) \le 2+\eta$.
 
 {\bf Type 3}: Cuts $S\in {\cH}\cap \cN_{\eta}$ where $\p(S)$ is not a near-cycle cut. By (iv) of \cref{thm:payment-main} and that $s^*\geq 0$ the inequality follows.
  
 
  {\bf Type 4:} Cuts $S$  such that either $S\in\cN_{\eta,\leq 1}\smallsetminus\cH$ or  $S\in \cH\cap\cN_\eta$ and $\p(S)$ is a (non-triangle) near-cycle cut.
In this case either $S$ is an atom or an $\eta$-near minimum cut of a non-singleton connected component $\cC$ of $\cN_{\eta,\leq 1}$ with corresponding polygon $P$ of cuts crossed on one side and the cycle partition $A,B,C$.  If $S$ is not a leftmost cut/atom or a rightmost cut/atom, then  by \cref{thm:crossed-one-side}, whenever $\delta(S)_T$ is odd, we have (similar to Type 2):
\begin{equation}s^*(\delta(S)) \geq \frac{2+\eta}{1-\eps_\eta}\decrease (1-\eps_\eta) = (2+\eta)\beta \ge -s(\delta(S))\end{equation} 

Otherwise, 
suppose $S$ is  a leftmost cut. If $P$ is left-happy then by \cref{thm:crossed-one-side}, similar to above, $s^*(\delta(S))+s(\delta(S))\geq 0$ if $\delta(S)_T$ is odd. Otherwise, let $S'$ be the union of the non-root atoms of $P$ and $F=\delta(S)\smallsetminus \delta(S')$. By \cref{lem:shared-edges}, we have $x(F)\geq 1-\eps_\eta/2$. Therefore, by (iii) of \cref{thm:payment-main} we have
$$ s(\delta(S))+s^*(\delta(S))\geq  s(A)+s(F)+s^-(C) \geq 0$$
as desired. 
Note that since $S$ is a leftmost cut, we always have $A\subseteq \delta(S)$. But $C$ may have an unpredictable intersection with $\delta(S)$; in particular, in the worst case only edges of $C$ with negative slack belong to $\delta(S)$. This is why we need to use $s^-(C)$ instead of $s(C)$. 
A similar argument holds when $S$ is the leftmost atom or a rightmost cut/atom.

	{\bf Type 5:} Cuts $S\in \cH\cap\cN_\eta$ where $\p(S)$ is a triangle $P$. This is similar to the previous case except we use \cref{lem:trianglereduction} to argue that the inequality is satisfied when $P$ is left/right happy.	
\end{proof}





%\section{An attempt}
%
%Here we use \cref{thm:crossed-one-side} and \cref{thm:payment-main} to prove the appendix theorem.
%
%\begin{theorem}[The Appendix Theorem]
%\label{thm:hierachyKKO}
%Given a family $\cN_{\eta,\leq 1}$ of near-min cuts containing {\bf  no} cuts crossed on both sides, for any $\decrease>0$, there is a vector $s: E \rightarrow R$
% depending on $T$ such that	\begin{enumerate}
%\item [(i)] $\forall e \in E$, $s_e  \ge -\decrease  x_e$ %and $s_e^* \ge 0$ (both with probability 1)
%\item [(ii)] $\E{s_e} < -\epsilon \decrease x_e$ for some absolute constant $\epsilon > 0$, independent of $\eta$, %and $\E{s^*_e} = O(\eta^2 x_e)$ 
%where the expectations are over the choice of $T$.
%\item [(iii)] If $S \in \cN_{\eta,\leq 1}$ is a cut such that $\delta(S)_T$ is odd, then $s_T(\delta(S)) = \sum_{e \in \delta(S)} s_e \ge 0$.
%\end{enumerate}
%\end{theorem}
%\begin{proof}
%We start by explaining how to construct a hierarchy $\cH$ from $\cN_{\eta,\leq 1}$ so that we can call \cref{thm:payment-main}:
%For every connected component ${\cal C}$ of $\cN_{\eta,\leq 1}$, if $|{\cal C}|=1$ then add the unique cut in ${\cal C}$ to the hierarchy. Otherwise, ${\cal C}$ corresponds to a polygon $P$ of cuts crossed on one side with atoms $a_0,\dots,a_{m-1}$ (for some $m>3$).
%By \cref{obs:one-side-interval} all these atoms are outside atoms. 
% Add $a_1,\dots,a_{m-1}$ to $\cH$\footnote{Notice that an atom may already correspond to a connected component, in such a case we do not need to add it in this step.} and $\cup_{i=1}^{m-1} a_i$ to $\cH$.
%Note that since $x(\delta(\{u_0,v_0\}))=2$, the root of the hierarchy is always $V\smallsetminus \{u_0,v_0\}$.
%%Once we process all connected components %,  then add $V$ and call it a degree cut and stop.
%
%Now, we name every cut in the hierarchy.
%For a cut $S$, if there is a connected component of at least two cuts with union equal to $S$, then call $S$ a polygon cut with $A,B,C$ partition as defined in \cref{def:abcpolygonpartitioning}.
%If $S$ is a cut with exactly two children $X,Y$ in the hierarchy (i.e. a triangle), then let $A=E(X,\overline{X}\smallsetminus Y)$, $B=E(Y,\overline{Y}\smallsetminus X)$ and $C=\emptyset$.
%Otherwise, call $S$ a degree cut.
%
%\begin{fact}[{\cite[Fact 4.34]{KKO21}}] The above procedure produces a valid hierarchy.	
%\end{fact}
%
%The following observation simply follows from the fact that the new cuts that we introduce in the above hierarchy, i.e., atoms and unions of non-root atoms of a polygon, are not crossed and are never part of a non-singleton connected component.
%\begin{fact}\label{fact:nonsingletoncompcorr}
%The set of non-singleton connected components $\cC_1,\cC_2,\dots$  that the above procedure produces are in one-to-one correspondence to the set of non-singleton connected components of $\cN_{\eta,\leq 1}$. 
%\end{fact}
%
%Let $E_g$ and $s$ be defined as in \cref{thm:payment-main} for the hierarchy defined above using the parameter $\decrease$ given as input to this theorem. Let $E_b:=E\smallsetminus E_g$ be the set of bad edges; in particular, note all edges in $\delta(\{u_0,v_0\})$ are bad edges.
%
%Let $s^*$ be the $s^*:E\to\R_{\geq 0}$ vector from \cref{thm:crossed-one-side} called with $\alpha = \frac{2+\eta}{1-\eps_\eta}\decrease$.
%Then,
%%$\E{s^*_{e^*}}\leq 44\eta (\frac{2+\eta}{1-\eps_\eta}\decrease) <  89\eta \decrease$  and using that $\eta \le 10^{-12}$ and $\eps_\eta = 7\eta$. 
%From (v) of \cref{thm:payment-main} we have $\E{s_e}\leq -\eps_P \decrease x_e$ for edges $e\in E_g$. 
%
%We define a new vector $\tilde{s}:E\cup \{e_0\}\to\R$ as follows: 
%\begin{equation}\label{eq:tildesdef}\tilde{s}(e)\gets \begin{cases}\infty & \text{if } e=e_0\\
%-x_e(4\decrease/5)(1-2\eta) & \text{if } e\in E_b,\\
%x_e(4\decrease/3) & \text{otherwise.}
%\end{cases}
% 
% \newpage 
% 
% Now, we verify (iv): For any (non-root) cut $S\in \cH$ such that $\p(S)$ is not a polygon cut $x(\delta(S)\cap E_g)\geq 3/4$ by (i) of \cref{thm:payment-main}. The only remaining case is $\eta$-near minimum cuts  which are either atoms or near minimum cuts in a polygon. Fix such a set $S$ in a polygon $P$. 
% Let $S'$ be the union of the non-root atoms of $P$. Then by \cref{lem:shared-edges}, $x(\delta(S)\cap \delta(S'))\leq 1+\eps_\eta$. All edges in  $\delta(S)\smallsetminus \delta(S')$ are bottom edges, so by (i) of \cref{thm:payment-main} are in $E_g$. Therefore, $x(\delta(S)\cap E_g))\geq 1-\eps_\eta\geq 3/4$.
% 
% It remains to verify (ii): We consider three groups of cuts:
%  
% {\bf Type 1}: Cuts $S\in {\cH}\cap \cN_{\eta,\le 1}$ where $\p(S)$ is not a polygon cut. By (iv) of \cref{thm:payment-main} and that $s^*\geq 0$ the inequality follows.
%  
% 
%  {\bf Type 2:} Cuts $S$  such that either $S\in\cN_{\eta,\leq 1}\smallsetminus\cH$ or  $S\in \cH\cap\cN_\eta$ and $\p(S)$ is a (non-triangle) polygon cut.
%In this case either $S$ is an atom or an $\eta$-near minimum cut of a non-singleton connected component $\cC$ of $\cN_{\eta,\leq 1}$ with corresponding polygon $P$ of cuts crossed on one side and the cycle partition $A,B,C$.  If $S$ is not a leftmost cut/atom or a rightmost cut/atom, then  by \cref{thm:crossed-one-side}, whenever $\delta(S)_T$ is odd, we have (similar to Type 2):
%\begin{equation}s^*(\delta(S)) \geq \frac{2+\eta}{1-\eps_\eta}\decrease (1-\eps_\eta) = (2+\eta)\beta \ge -s(\delta(S))\end{equation} 
%
%Otherwise, 
%suppose $S$ is  a leftmost cut. If $P$ is left-happy then by \cref{thm:crossed-one-side}, similar to above, $s^*(\delta(S))+s(\delta(S))\geq 0$ if $\delta(S)_T$ is odd. Otherwise, let $S'$ be the union of the non-root atoms of $P$ and $F=\delta(S)\smallsetminus \delta(S')$. By \cref{lem:shared-edges}, we have $x(F)\geq 1-\eps_\eta/2$. Therefore, by (iii) of \cref{thm:payment-main} we have
%$$ s(\delta(S))+s^*(\delta(S))\geq  s(A)+s(F)+s^-(C) \geq 0$$
%as desired. 
%Note that since $S$ is a leftmost cut, we always have $A\subseteq \delta(S)$. But $C$ may have an unpredictable intersection with $\delta(S)$; in particular, in the worst case only edges of $C$ with negative slack belong to $\delta(S)$. This is why we need to use $s^-(C)$ instead of $s(C)$. 
%A similar argument holds when $S$ is the leftmost atom or a rightmost cut/atom.
%
%	{\bf Type 3:} Cuts $S\in \cH\cap\cN_\eta$ where $\p(S)$ is a triangle $P$. This is similar to the previous case except we use \cref{lem:trianglereduction} to argue that the inequality is satisfied when $P$ is left/right happy.	
%	
%	
%
%Let $E_g$ be the good edges defined in \cref{lem:beforetechnicalthm} and let $E_b:=E\smallsetminus E_g$ be the set of bad edges; in particular, note all edges in $\delta(\{u_0,v_0\})$ are bad edges. We define a new vector $\tilde{s}:E\cup \{e_0\}\to\R$ as follows: 
%\begin{equation}\label{eq:tildesdef}\tilde{s}(e)\gets \begin{cases}\infty & \text{if } e=e_0\\
%-x_e(4\decrease/5)(1-2\eta) & \text{if } e\in E_b,\\
%x_e(4\decrease/3) & \text{otherwise.}
%\end{cases}
%\end{equation}
%Let $\tilde{s}^*$ be the vector $s^*$ from \cref{thm:cutsbothsideswithinside} called with $\alpha = 2\decrease$.
%We claim that for any  $\eta$-near minimum cut $S$ such that $\delta(S)_T$ is odd, we have 
%$$ \tilde{s}(\delta(S))+\tilde{s}^*(\delta(S))\geq 0.$$
%To check this note by (iv) of \cref{lem:beforetechnicalthm} for every set $S\in \cN_{\eta,\leq 1}$ such that $S\neq V\smallsetminus \{u_0,v_0\}$, we have $x(E_g \cap \delta(S)) \ge \frac{3}{4}$, so we have
%\begin{equation}\label{eq:tildeScutsok}\tilde{s}(\delta(S))+\tilde{s}^*(\delta(S))\geq \tilde{s}(\delta(S)) = \frac{4\decrease}{3} x(E_g\cap\delta(S)) - \frac{4\decrease}{5}(1-2\eta)x(E_b\cap\delta(S))\geq 0.	
%\end{equation}
%
%For $S=V\smallsetminus \{u_0,v_0\}$, we have $\delta(S)_T=\delta(u_0)_T + \delta(v_0)_T=2$ with probability 1, so condition ii) is satisfied for these cuts as well.
%Finally, consider cuts $S\in\cN_{\eta,2}$. By \cref{thm:cutsbothsideswithinside}, if $\delta(S)_T$ is odd, then $\tilde{s}^*(\delta(S)) \ge \alpha(1-\eta) = 2\decrease(1-\eta)$. Therefore, in such a case we have:
%\begin{equation}\label{eq:tildeScutstwo}\tilde{s}(\delta(S)) + \tilde{s}^*(\delta(S)) \geq  2\decrease(1-\eta) - \frac{4\decrease}{5}(1-2\eta) x(\delta(S))  \geq 0	
%\end{equation}
%where we use that $x(\delta(S)) \le 2+\eta$. 
%%w $\eta<10^{-12}$.)
%
%Now, we are ready to define $s,s^*$. Let $\hat{s},\hat{s}^*$ be the $s,s^*$ of \cref{lem:beforetechnicalthm} respectively.
%Define $s=\gamma \tilde{s} + (1-\gamma) \hat{s}$ and similarly define $s^*=\gamma\tilde{s}^*+(1-\gamma)\hat{s}^*$ for some $\gamma$ that we choose later. 
%We prove all three conclusions of \cref{thm:maintechnical} for $s,s^*$. (i) follows by (i) of \cref{lem:beforetechnicalthm} and  \cref{eq:tildesdef}. (ii) follows by (ii) of \cref{lem:beforetechnicalthm} and \cref{eq:tildeScutsok,eq:tildeScutstwo} above. %\anna{This is a bit brutal. Maybe in one of the theorems switch from (i),(ii) etc to (a) (b) etc?}
%It remains to verify (iii). For edge $e \in E$, $\E{s^*_{e}}\leq 125\eta \decrease x_e$ by (iii) of \cref{lem:beforetechnicalthm} and the construction of $s^*$. On the other hand, by (iii) of \cref{lem:beforetechnicalthm} and \cref{eq:tildesdef},
%\begin{align*} 
%\E{s_e} \begin{cases}\leq  x_e  (\gamma\frac{4}{3}\decrease - (1-\gamma)\eps_P\decrease ) & \forall e\in E_g,\\	
%= -x_e\gamma\cdot (\frac{4}{5}\decrease)(1-2\eta)& \forall e\in E_b.
%\end{cases}
%\end{align*}
%Setting $\gamma=\frac{15}{32}\eps_P$ we get $\E{s_e}\leq -\frac{1}{3}\eps_P\decrease x_e$ for $e\in E_g$ and $\E{s_e}\leq -\frac{1}{3}x_e\decrease\epsilon_P$ for $e\in E_b$ as desired.
%\end{proof}
%
