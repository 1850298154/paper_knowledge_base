\documentclass[conference]{IEEEtran}
\IEEEoverridecommandlockouts
% The preceding line is only needed to identify funding in the first footnote. If that is unneeded, please comment it out.
\usepackage{cite}
\usepackage{amsmath,amssymb,amsfonts}
\usepackage{algorithm}
\usepackage{algorithmic}
\usepackage{graphicx}
\usepackage{textcomp}
\usepackage{xcolor}


\usepackage{tcolorbox}
\usepackage{xcolor}
\usepackage{caption} 
\usepackage{multirow}
\usepackage{booktabs}
\usepackage{graphicx}
\usepackage{lscape}
\usepackage{subcaption}
\usepackage{url}

\def\BibTeX{{\rm B\kern-.05em{\sc i\kern-.025em b}\kern-.08em
    T\kern-.1667em\lower.7ex\hbox{E}\kern-.125emX}}
\begin{document}

\title{Neural Algorithmic Reasoners informed Large Language Model for Multi-Agent Path Finding

\thanks{*Corresponding author: Wenjun Wu (Email: wwj09315@buaa.edu.cn). This work was supported by the National Science and Technology Major Project (No. 2022ZD0116401)}
}



% \author{\IEEEauthorblockN{Michael Shell\IEEEauthorrefmark{1},
% Homer Simpson\IEEEauthorrefmark{2},
% James Kirk\IEEEauthorrefmark{3}, 
% Montgomery Scott\IEEEauthorrefmark{3}, and
% Eldon Tyrell\IEEEauthorrefmark{4},~\IEEEmembership{Fellow,~IEEE}}
% \IEEEauthorblockA{\IEEEauthorrefmark{1}School of Electrical and Computer Engineering,
% Georgia Institute of Technology, Atlanta, GA 30332 USA}
% \IEEEauthorblockA{\IEEEauthorrefmark{2}Twentieth Century Fox, Springfield, USA}
% \IEEEauthorblockA{\IEEEauthorrefmark{3}Starfleet Academy, San Francisco, CA 96678 USA}
% \IEEEauthorblockA{\IEEEauthorrefmark{4}Tyrell Inc., 123 Replicant Street, Los Angeles, CA 90210 USA}% <-this % stops an unwanted space


\author{
    \IEEEauthorblockN{Pu Feng\textsuperscript{1}, Size Wang\textsuperscript{2}, Yuhong Cao\textsuperscript{4}, Junkang Liang\textsuperscript{2}, Rongye Shi\textsuperscript{1,2,3}, *Wenjun Wu\textsuperscript{1,2,3}}
    \IEEEauthorblockA{\textsuperscript{1}State Key Laboratory of Complex \& Critical Software Environment, Beihang University, Beijing, China}
    \IEEEauthorblockA{\textsuperscript{2}School of Artificial Intelligence, Beihang University, Beijing, China}
    \IEEEauthorblockA{\textsuperscript{3}Hangzhou International Innovation Institute, Hangzhou, China}
    \IEEEauthorblockA{\textsuperscript{4}Department of Mechanical Engineering, National University of Singapore, Singapore}
}

% \author{\IEEEauthorblockN{Pu Feng}
% \IEEEauthorblockA{\textit{State Key Laboratory of Complex  \& }\\
% \textit{Critical Software Environment} \\
% \textit{Beihang University}\\
% Beijing, China \\
% fengpu@buaa.edu.cn}
% \and
% \IEEEauthorblockN{Size Wang}
% \IEEEauthorblockA{\textit{School of Artificial Intelligence} \\
% \textit{Beihang University}\\
% Beijing, China \\
% szwang0312@gmail.com}
% \and
% \IEEEauthorblockN{Yuhong Cao}
% \IEEEauthorblockA{\textit{Department of Mechanical Engineering} \\
% \textit{National University of Singapore}\\
% Singapore\\
% caoyuhong@u.nus.edu}
% \and
% \IEEEauthorblockN{Junkang Liang}
% \IEEEauthorblockA{\textit{School of Artificial Intelligence} \\
% \textit{Beihang University}\\
% Beijing, China  \\
% liangjunkang1999@gmail.com}
% \and
% \IEEEauthorblockN{Rongye Shi}
% \IEEEauthorblockA{\textit{School of Artificial Intelligence} \\
% \textit{Hangzhou International Innovation Institute} \\
% \textit{Beihang University}\\
% Beijing; Hangzhou, China \\
% shirongye@buaa.edu.cn}
% \and
% \IEEEauthorblockN{Wenjun Wu*}
% \IEEEauthorblockA{\textit{School of Artificial Intelligence} \\
% \textit{Hangzhou International Innovation Institute} \\
% \textit{Beihang University}\\
% Beijing; Hangzhou, China \\
% wwj09315@buaa.edu.cn}
% }

\maketitle

\begin{abstract}
The development and application of large language models (LLM) have demonstrated that foundational models can be utilized to solve a wide array of tasks. However, their performance in multi-agent path finding (MAPF) tasks has been less than satisfactory, with only a few studies exploring this area. MAPF is a complex problem requiring both planning and multi-agent coordination. To improve the performance of LLM in MAPF tasks, we propose a novel framework, LLM-NAR, which leverages neural algorithmic reasoners (NAR) to inform LLM for MAPF. LLM-NAR consists of three key components: an LLM for MAPF, a pre-trained graph neural network-based NAR, and a cross-attention mechanism. This is the first work to propose using a neural algorithmic reasoner to integrate GNNs with the map information for MAPF, thereby guiding LLM to achieve superior performance. LLM-NAR can be easily adapted to various LLM models. Both simulation and real-world experiments demonstrate that our method significantly outperforms existing LLM-based approaches in solving MAPF problems.
\end{abstract}

\begin{IEEEkeywords}
Large Language Model, Multi-Agent Path Finding, Neural Algorithmic Reasoner
\end{IEEEkeywords}

\section{Introduction}
In recent years, large language models (LLM) have rapidly advanced and proven highly effective in addressing a variety of complex tasks. Their success is largely attributed to extensive pre-training that embeds a wide range of knowledge, making them highly valuable across numerous applications. Meanwhile, integrating LLM with other computational techniques to solve specific problems has attracted considerable interest~\cite{zhu2024knowagent}.


In multi-agent systems, where coordination and cooperation are crucial, LLM has been increasingly used to enhance inter-agent collaboration, improving its ability to tackle complex tasks. While research has focused on agent interactions, communication, and world simulation~\cite{guo2024large}, studies on multi-agent planning remain limited.


This paper focuses on the multi-agent path finding (MAPF) problem, where multiple agents must navigate from start to goal locations without collisions. MAPF is vital in real-world applications like warehouse management and swarm control~\cite{surynek2022problem}. Traditional approaches include classical methods like Conflict-Based Search (CBS)~\cite{sharon2015conflict}, as well as learning-based strategies using reinforcement learning~\cite{sartoretti2019primal,li2023mixture, feng2023mact,yu2024leveraging} and graph neural networks~\cite{li2020graph}. Leveraging LLM for MAPF offers a way to mitigate the slow training and high computational demands of current methods. However, existing work~\cite{chen2024solving} on integrating LLM into MAPF is limited and does not fully explore their potential due to challenges in understanding spatial constraints and collaborative strategy formulation—areas where LLMs still fall short.





\begin{figure}[t]
  \centering
  \includegraphics[scale=0.35]{figure/intro.pdf}
  \caption{Limitations of existing LLM for MAPF. Informed with GNN-based NAR, our method performs better.}
  \label{intro}
  \vspace{-0.25in}
\end{figure}

To improve the performance of LLM in MAPF tasks, we propose a novel framework, LLM-NAR, which leverages neural algorithmic reasoners (NAR) to enhance LLM's ability to process spatial map information for MAPF. The framework integrates three key components: an LLM for MAPF, a GNN-based NAR, and a cross-attention mechanism.

The first component, LLM for MAPF, employs a tailored prompt interaction strategy specifically designed for MAPF tasks. Scenario-specific information is fed into LLM to generate directives for each agent at each timestep. We periodically update the LLM’s understanding of the map's state to maintain the accuracy and relevance of its outputs. A key feature of our approach is the reset mechanism, which helps keep the LLM "aware" and prevents information loss. The second component, GNN-based NAR, builds a graphical representation that captures the map's intricacies and the spatial relationships between agents. This graphical model distills critical spatial and relational insights, which are essential for effective path planning. The third component, the cross-attention mechanism, fuses the token outputs from the LLM with the graph representation produced by the GNN-based NAR. This fusion enhances contextual understanding by aligning linguistic instructions with spatial data. Finally, we train the cross-attention network by minimizing the loss between the action output from the final layer and the expert strategy derived from CBS~\cite{sharon2015conflict}.


Our approach leverages a GNN-based NAR to enhance the LLM's understanding of map information for the MAPF task, requiring only a few thousand training steps to train the cross-attention mechanism. In contrast to the hundreds of thousands or even millions of steps typically required by other learning-based methods, our approach is significantly more efficient, demanding far fewer training steps. LLM-NAR can also be easily adapted to various large language models. The key contributions of this paper are as follows:

% Our approach leverages GNN-based NAR to enhance the LLM's understanding of map information for the MAPF task, requiring only a few steps of training. 强化学习等方法往往需要约500K左右step训练步数,而我们方法仅需要around 5K steps来训练cross attention。 LLM-NAR can be easily adapted to various large language models. The key contributions of this paper are as follows:
\begin{enumerate} 
    \item We propose a novel LLM prompt mechanism tailored specifically for MAPF tasks. 
    \item We develop a GNN-based NAR to effectively represent map information. 
    \item We introduce the LLM-NAR framework, which integrates LLM and GNN-based NAR outputs via a cross-attention mechanism. 
    \item We release an open-source dataset\footnote{https://github.com/fpgod/LLM-NAR} to facilitate LLM-NAR interactions in MAPF tasks, including tokens for LLM interaction and NAR training data.
    \item We demonstrate the effectiveness of our approach through both simulation and real-world experiments. 
\end{enumerate}






% 在LLM-NAR主要包含三个部分,分别为LLM for mapf、GNN for mapf and cross attention。首先在llm部分,我们设计了适合于mapf任务的prompt交互模式,我们给大模型场景信息,生成每个智能体的指令动作。其中我们在每个step都对碰撞情况进行检测,排除碰撞行为,直到大模型输出可行动作。特别一点是我们在每个step都对llm的地图状态进行reset observation操作,确保了在llm每步获取信息的稳健性。第二部分GNN for mapf,我们利用GNN来构建mapf中的地图和邻居agent信息的图表示,to 提取mapf任务中的地图和neighbor的关键信息。最后我们将llm提供的token和gnn提供的node、edge表示通过cross attention进行聚合。为了能对cross attention进行训练,我们采集了CBS完成mapf的动作数据集,作为训练的label。我们方法充分利用了neural algorithm reasoner来帮助llm理解mapf任务中的地图信息,在仿真实验和真实实验中也验证了我们方法的有效性,以下是本文的贡献:1.提出了一套适合mapf的llm prompt机制
% 2.构建neural algorithmic reasoners以gnn,对于地图信息进行表示
% 3.提出llmnar框架,利用crossattention对llm和nar信息进行融合。
% 4.在仿真和真实实验中验证了我们方法的有效性

\section{Related Work}
\subsection{Multi-agent Path Finding}
Multi-agent path finding~\cite{stern2019multi} tasks are classic group collaboration problems, and a significant amount of research has been conducted in this area. Current MAPF methods are mainly divided into two categories. The first category comprises traditional planning algorithms, which compute optimal or sub-optimal routes by designing specific rules; this category is represented by Conflict-Based Search (CBS)~\cite{sharon2015conflict} and its extended versions~\cite{andreychuk2021improving}. The second category consists of learning-based methods, primarily including reinforcement learning~\cite{sartoretti2019primal,feng2024safe,feng2024hierarchical}, graph neural networks~\cite{li2020graph} and transformers~\cite{wang2023scrimp}. Currently, there is relatively little research on applying large language models (LLM) to MAPF. Notably, some studies~\cite{chen2024solving} have identified existing challenges of LLM in solving MAPF problems, highlighting this as a very valuable research direction.
% 多智能体路径规划任务是一类经典的群体协作任务,已经有大量的工作开展研究。mapf的方法目前主要分为两类,一类是传统的规划算法,通过设计规则来计算最优或者次优路线,这一类研究以cbs及其扩展版本为代表。另一类是基于学习的方法,主要包括基于强化学习的方法和基于gnn、transoformer的方法。目前基于llm的mapf研究还较少,其中提出了目前llm在解决mapf问题中存在的问题,这是一个非常有价值的研究方向。
\subsection{Large Language Model for Multi-agent Cooperation}
Large Language Models (LLM) have begun to influence the field of multi-agent systems by enhancing communication, coordination, and decision-making among agents~\cite{ni2024tree,ni2025shieldlearner}. Leveraging their advanced natural language understanding and generation capabilities, LLM enables agents to interpret complex instructions and interact more naturally with humans and other agents~\cite{zhang2024llm}. LLM has also been used to construct interactive environments for multi-agent tasks and explore social collaboration patterns through dialogue and communication between agents~\cite{pan2024agentcoord}. These methods represent initial explorations of LLM in the multi-agent domain. Our paper goes a step further by combining LLM with graph neural networks to enhance cooperative capabilities in Multi-agent Path Finding.

% Large Language Models (LLM) have begun to influence the field of multi-agent systems by enhancing communication, coordination, and decision-making among agents. Leveraging their advanced natural language understanding and generation capabilities, LLM enable agents to interpret complex instructions and interact more naturally with humans and other agents.llm也被用于构建多智能体任务交互环境,通过智能体之间的对话交流探索社会协作模式。也有研究通过分层方法,在上层使用llm进行智能体之间的任务分配,在下层使用基础算法进行动作执行。这些方法是llm在多智能体方向上的一些探索,我们的论文更进一步将llm与gnn结合,提升mapf中的协作能力。

\subsection{Neural algorithmic reasoning}
Neural Algorithmic Reasoning (NAR) is, in general terms, the art of building neural networks capable of capturing algorithmic computation \cite{velivckovic2021neural}, bridging the gap between precise but inflexible classical algorithms and adaptable yet less interpretable neural networks. By embedding algorithmic reasoning within neural architectures, NAR enables neural networks to perform tasks requiring logical deduction, planning, and problem-solving, enhancing their generalization and reasoning capabilities \cite{xu2018powerful}. Graph Neural Networks (GNNs) are a natural fit for NAR due to their effectiveness in representing and processing graph-structured data \cite{hechtlinger2017generalization}. Recent research~\cite{bounsi2024transformers} has explored the use of GNN-based NARs to enhance transformers in solving reasoning tasks. GNNs can also approximate classical algorithms operating on graphs, such as shortest path computation and optimization problems \cite{cappart2023combinatorial}, making them well-suited for complex tasks like MAPF, where agent relationships and environments are naturally represented as graphs \cite{ma2022graph}.


\section{Problem Formulation}
MAPF tasks involve a set of agents navigating from their initial locations to designated destinations without collisions, while minimizing travel time. Consider a set of $N$ agents, represented as $v = \{v_1, v_2, \ldots, v_N\}$, operating on an undirected graph $G = (V, E)$, where $V$ is the set of vertices and $E$ represents the edges connecting these vertices. Each agent starts at an initial vertex $s_i \in V$ and must reach a unique destination vertex $d_i \in V$. The sets of start and destination vertices are denoted by $S$ and $D$, respectively. At each discrete time step $t$, agents can either move to an adjacent vertex or remain stationary, as represented by the action set $u^t$. Collisions occur if two agents attempt to occupy the same vertex simultaneously or if an agent moves into a vertex occupied by an obstacle. The joint action set $U^t$ formed by the actions of all agents at time step $t$, is valid only if it satisfies the following conditions: (1) no two agents occupy the same vertex at any given time step ($u^t_i \neq u^t_j$ for any pair of agents $i$ and $j$), and (2) no pair of agents swap vertices within a single time step $\left(u^{t+1}_i = u^t_j \Longleftrightarrow u^{t+1}_j \neq u^t_i\right)$. The goal is to determine a sequence of valid joint action sets that allows all agents to move from their initial positions in $S$ to their destinations in $D$ in the minimum time steps.

% 多智能体路径规划任务是一群智能体从起始位置出发找到一条没有碰撞的路径到达目标点同时最小化travel time的问题。在mapf问题中,我们定义v={v_1,v_2,v_N}表示任务中的N个智能体,and an undirected graph $G=(V, E, )$ where $E$ is the set of edges connecting the set of vertices $V$.每个智能体有自己的初始位置$\left(s_i \in V\right)$ and a unique destination/goal vertex $\left(d_i \in V\right)$.我们将the set of start position 和destination position定义为S和D.在每个时刻t,智能体可以执行动作a^t,包括移动到其他位置和保持不动。当两个智能体同时进入到同一个cell或者智能体进入障碍物所在cell的时候是发生碰撞。
% The set of actions that all agents perform at a time step $t_j$ is referred to as a joint set $J_{t_j}$. A joint set of actions is considered valid if no two agents occupy the same vertex at any time step: $a_{i, t_j} \neq a_{k, t_j}$, and if no two agents swap vertices in a single time step $\left(a_{i, t_{j+1}}=a_{k, t_j} \Longleftrightarrow\right.$ $a_{k, t_{j+1}} \neq a_{i, t_j}$ ), where $a_{i, t_j}$ denotes the position of agent $a_i$ at time $t_j$ Our objective is to determine a series of valid joint sets that guide agents from their source vertices $S$ to their goal vertices $D$ in the least amount of time steps.




\section{Method}

\begin{figure*}[h]
  \centering
  \includegraphics[scale=0.5]{figure/framework2.pdf}
  \caption{Framework of LLM-NAR, which consists of LLM for MAPF, GNN-based NAR and cross-attention.}
  \label{framework}
  \vspace{-0.2in}
\end{figure*}

The LLM-NAR framework is designed to address MAPF challenges by integrating three key components: LLM for MAPF, GNN-based NAR, and a cross-attention mechanism. First, we construct a novel prompt framework for LLM to handle MAPF tasks. Then, the GNN-based NAR plays a crucial role in understanding the spatial structure of the map, capturing essential relationships between agents and their environment. We leverage the capabilities of the GNN-based NAR through a cross-attention mechanism, allowing the LLM to enhance its performance. Fig.~\ref{framework} illustrates the architecture of the LLM-NAR framework. 

\begin{algorithm}[h]
    \caption{Neural Algorithmic Reasoners informed Large Language Model for Multi-Agent Path Finding}
    \label{alg1}
    \begin{algorithmic}[1] %[1] enables line numbers
        \STATE Initialize MAPF task map set $\mathcal{M}$, paprameter of GNN-NAR network.
        
        \FOR{each MAPF task in $\mathcal{M}$}
            \STATE Use CBS to obtain optimal actions $\{(U^t)^*\}$ for the task.
            \STATE Train GNN-NAR network based on the obtained $\{(U^t)^*\}$ and optimize it according to Eq.~\ref{gnn loss}.
        \ENDFOR
        
        \STATE Initialize cross-attention network, max training steps, and episode length T.

        
        \FOR{each episode}
            \FOR{t=1 to T}
                \STATE Based on the current observation $O^t$, get GNN representation $X_L$.
                \STATE Based on the current observation $O^t$, get LLM output $\Theta^t$.
                \STATE Combine $X_L$ and $\Theta^t$ into the cross-attention mechanism to get the LLM-NAR action $U^t_{\text{LLM-NAR}}$.
                \STATE Update cross-attention network according to Eq.~\ref{llm-nar loss}.
            \ENDFOR
        \ENDFOR
    \end{algorithmic}
\end{algorithm}

\subsection{Overview of LLM-NAR}
% 在我们整个算法的框架中,我们融合了三部分内容,分别是GNN-NAR、LLM以及将这两者能力组合起来的cross attention部分。在算法1中,我们提供了我们方法的伪代码。我们首先在每个地图任务中利用cbs获得最优路线数据,并利用这个数据对gnn-nar进行了预训练,从而获得有一定准确性的gnn-nar。我们认为此时的gnn-nar具备了对mapf地图信息的表示能力,即获得了X_L。同时基于每个步骤的状态信息,我们构建了一套新颖的适用于llm求解mapf任务的prompt形式,并获得了llm的token 输出。将\theta_t和X_L同时输入到cross attention中,我们最终获得了LLM-NAR的动作输出,我们将$U^t_{\text{LLM-NAR}}$同cbs的行为动作u_t计算loss,更新我们的cross attention网络。整个过程中我们通过融合gnn对于地图的表示能力、llm对于任务的规划能力来改进单纯依靠llm或者gnn无法很好解决mapf任务的问题。

In our algorithmic framework, we integrate three core components: GNN-NAR, LLM, and a cross-attention mechanism that synergizes the strengths of these two. The pseudocode for our method is presented in Algorithm~\ref{alg1}. 

CBS, as an optimal algorithm, has the drawback of high computational cost but offers the advantage of providing optimal solutions, making it suitable for use as training data labels. We first employ CBS~\cite{sharon2015conflict} to generate optimal path data for each MAPF task and use this data to pretrain the GNN-NAR network, resulting in a reasonably accurate GNN-NAR model. At this stage, we assume that GNN-NAR has acquired the capability to represent MAPF map information, denoted as $X_L$. Concurrently, based on the state information at each time step, we design a novel prompt format tailored specifically for MAPF tasks, which allows the LLM to generate token outputs.

By feeding both $X_L$ and the LLM-generated token outputs $\Theta^t$ into the cross-attention mechanism, we derive the final action output of LLM-NAR, $U^t_{\text{LLM-NAR}}$. The loss is then computed between $U^t_{\text{LLM-NAR}}$ and the CBS-generated optimal action $u^t$, and this loss is used to update the cross-attention network.

Throughout this process, we leverage the GNN's ability to effectively encode map structures and the LLM's planning capabilities to address the MAPF task more comprehensively. This integration mitigates the limitations of relying solely on either GNN or LLM for solving MAPF challenges. A detailed explanation of each component is provided below.




% llm-nar由LLM for MAPF、GNN based NAR和cross attention三个部分组成,通过gnn informed llm来解决mapf问题。图1是llm-nar的框架图,下面我们将介绍每一个部分in detail。
% 这里画一个framework的图
\subsection{LLM for MAPF}

In this module, we focus on utilizing LLM to address the MAPF problem. Inspired by~\cite{chen2024solving}, we construct our LLM framework for MAPF with several improvements, which features a specially designed scene description prompt and a reset mechanism for LLM, as shown in Fig.~\ref{llmworkflow}.


\begin{figure}[h]
  \centering
  \includegraphics[scale=0.35]{figure/llmformapfg.pdf}
  \caption{Workflow of LLM for MAPF}
  \label{llmworkflow}
  \vspace{-0.1in}
\end{figure}

We provide LLM with a system prompt of role information, allowing it to function as the solver for MAPF problem. Based on the specific map and task conditions, we progressively provide LLM with scene descriptions step by step. This information includes the agents' current positions, target locations, a text-based map description, and the positions of both agents and obstacles, as shown in Fig.~\ref{fig:llm map prompt}. The step-by-step design breaks the entire planning task into manageable single-step tasks. 

We extract the solutions from LLM output tokens $\Theta^t$ at timestep $t$. Any answers that violate the rules or are deemed invalid are corrected to the "stay" action to ensure the smooth operation of the workflow. Then we feed the predicted actions into the environment, enabling the state transition and proceeding to the next step. Additionally, a reset mechanism is employed to modify the prompt process based on an evaluation of the LLM's performance.


% LLM will be reset if certain conditions are met.


\begin{figure}[h]
    \centering
    \begin{tcolorbox}[sharp corners, boxrule=0.5mm, colback=white]

        \textbf{Agent 1} is at (6, 7), wants to go to (5, 5).\\
        \textbf{Agent 2} is at (3, 0), wants to go to (3, 5).\\
        \textcolor{blue}{The map is as follows, where '@' denotes a cell with an obstacle that an agent cannot pass, and '.' denotes an empty cell that an agent can pass.}\\
        \textcolor{blue}{The lower-left cell is (0,0) and the lower-right cell is (0,7):}\\
        \textcolor{blue}{...}\\
        \textcolor{blue}{......@.}\\
        \textcolor{blue}{..@.....}\\
        \textcolor{blue}{...}\\
        \textcolor{red}{The coordinates of the obstacles: (2,5) (1,7) (6,3)... }\\
        \textcolor{red}{The coordinates of the agents: (6,7) (3,0).} 
        
        % 添加图标题
    \end{tcolorbox}
    \caption{Example of the user prompt for describing the scenario. The black text describes the mission information. The blue text describes the map information. The red text is a mathematical representation of the scene.}
    \label{fig:llm map prompt}
    \vspace{-0.1in}
\end{figure}

Compared to previous approaches that employed conflict checkers to generate feasible actions prompt at each step~\cite{chen2024solving}, we simplify the workflow by integrating the conflict checker into the environment transition process and abandoned the feasible action prompt. Instead, we directly provide step-by-step scenario information as input before the LLM generates actions. We find that in later stages of problem-solving, LLM often encountered issues such as loss or confusion of scene information (e.g., forgetting target locations or providing invalid actions). Prompting with feasible actions proved insufficient to mitigate these problems. To emphasize the agents' objectives, we utilize detailed scene descriptions instead of merely feasible actions. Our approach ensures that LLM receives information prompts at each step. This assists LLM in understanding its current stage in the process and aligning their actions with the intended goals. 

Additionally, we introduce a reset mechanism for the LLM. The LLM is reset when poor performance is detected, and the task continues from the current position, treating it as the new starting point. We consider LLM needs a restart either after completing a set number of rounds (denoted as $m$, $m$=5 in our experiments) or when it generates only invalid solutions for three consecutive rounds. In such cases, we can promptly stop LLM from going down the wrong path and "clear its head" by rebooting it. The task is considered complete when the agents reach their goals or the entire planning process (including resets) reaches three times the map length, which allows sufficient room for agents to traverse the longest path and make any necessary detours.


% 在本模块中,我们专注于利用大型语言模型(LLM)直接解决多智能体路径寻找(MAPF)问题。我们遵循并在\cite{chen2024solvingmultiagentpathfinding}的框架基础上进行了多项改进。通过整合大模型的操作方法并与其他模块协调,我们设计了图\ref{}中所示的工作流程。

% 实验开始时,我们向LLM提供一个角色提示,让其作为MAPF问题的解决器。然后,根据具体的地图和任务条件,我们逐步向LLM提供场景描述。这些信息包括代理的当前位置、目标位置、基于文本的地图描述以及代理和障碍物的位置,如图\ref{fig map prompt}所示。

% 我们从LLM的输出中提取解决方案。任何违反规则或被认为无效的行动都将被纠正为静止行动,以确保工作流的顺利进行。如果必要,这些行动会通过其他模块进行处理。然后我们将纠正后的行动输入到环境(env)中,促使(马尔可夫)游戏中的状态转换,并进入下一步。经过几轮规划后,我们会重置LLM。

% 与以前使用冲突检查器的方法相比,我们简化了工作流程。在早期的工作中,LLM在问题解决的后期阶段常常会遇到场景信息丢失或混淆的问题(例如,忘记目标位置或提供无效/不可行的行动)。我们的方法确保LLM在每一步都接收到信息提示。这种逐步设计在早期研究中已被证明是一种高效的策略,并且与我们的其他模块无缝协作。通过实施重置机制,我们进一步减少了违规行动的发生。





\subsection{GNN-based NAR}

In this section, we detail the deployment of a pre-trained GNN, configured as a Neural Algorithmic Reasoner (NAR), as depicted in Fig.~\ref{nar}, to assist the LLM in interpreting spatial and relational dynamics within the MAPF scenario. The GNN plays a critical role in enhancing the understanding of the map's structure and the proximity between agents, improving decision-making by extracting and integrating complex spatial features and state information.


Each agent, identified by an index \(i\), gathers local environmental data at each timestep \(t\), represented as \(o^t_i\). These observations are assembled into a matrix for all agents: $O^t = [o^t_1, o^t_2, \ldots, o^t_N]$, where \(N\) denotes the total number of agents. This observation matrix \(O^t\) is processed by a Convolutional Neural Network (CNN), which translates the raw data into a series of feature vectors \(X^t\):
\begin{equation}
X^t = \text{CNN}(O^t) = [x^t_1, x^t_2, \ldots, x^t_N].
\end{equation}

Please note that our ultimate goal is to enable the LLM to derive superior strategies from more effectively processed map information. Within the framework of our GNN-based NAR, we have defined a connectivity structure among agents, represented by the adjacency matrix \( C^t \), to delineate the relationships between agents. Intuitively, if an agent \( i \)'s strategy might potentially collide with neighbor agent \( j \), then the state information of \( j \) is invaluable for training the strategy of \( i \). To this end, we use a graph convolution to aggregate observational data using \( C^t \), expressed through the operation \( \mathcal{A}(X^t; C^t) \):
\begin{equation}
\mathcal{A}(X^t; C^t) = [C^t X^t]_{i} = \sum_{j=1}^N [C^t]_{ij} [X^t]_{j} = \sum_{j: v_j \in \mathcal{N}_i} c^t_{i} x^t_{j}.
\end{equation}



% 在本文中,我们最终目的是希望llm能够根据更加有效的地图信息输入来获得better的策略,所以在gnn based nar模块,我们定义了一个智能体之间的联通结构,并用邻接矩阵C^t来表示智能体之间的关联。直观来说,如果智能体i的策略因为与j存在潜在碰撞,那么j的状态信息对于i策略训练非常有价值。为此,我们利用C^t将观测信息聚合,记作\mathcal{A}(X_t; C_t):
% \[
% \mathcal{A}(X_t; C_t) = [C_t X_t]_{if} = \sum_{j=1}^N [C_t]_{ij} [X_t]_{jf} = \sum_{j: v_j \in \mathcal{N}_i} c_t^{ij} x_t^{jf},
% \]

The GNN architecture consists of \(L\) layers, with each layer updating its state by integrating outputs from the previous layer through the defined graph convolution:
\begin{equation}
X_{\ell} = \sigma[\mathcal{A}_{\ell}(X_{\ell-1}; C)] \quad \text{for} \quad \ell = 1, \ldots, L,
\end{equation}
where \(\sigma\) represents a non-linear activation function that enhances the model’s capacity to discern complex patterns. After processing through the GNN layers, the resulting features are fed into a Multi-Layer Perceptron (MLP) to generate the predicted actions $\hat{U}^t = \text{MLP}(X_L^t)$.


% The GNN optimization process aims to minimize the discrepancies between the predicted actions \(\hat{U}_t\) and the optimal actions \(U_t^*\), typically derived from an expert policy Conflict-Based Search (CBS): 
% \[
% \min_{\mathrm{CNN}, \{\mathrm{A}_{\ell}\}, \mathrm{MLP}} \sum_{(\{U_t\}, \{\mathbf{O}_t^i\}) \in \mathcal{T}} \sum_t \mathcal{L}(U_t^*,\hat{U}_t ).
% \]
% Here, \(\mathcal{L}\) is the loss function, and \(\mathcal{T}\) represents the training set comprising sequences of observations and actions. 我们通过优化gnn的动作靠近专家策略,获得更加准确的node信息表示,为后续cross attention提供nar的knowledge。

The optimization process for the GNN is aimed at minimizing the discrepancies between the predicted actions $\hat{U}^t$ and the optimal actions $(U^t)^*$, derived from CBS~\cite{sharon2015conflict} as the expert policy. The objective is to minimize the loss, which can be formally expressed as:
\begin{equation}
\min_{\mathrm{CNN}, \{\mathrm{A}_{\ell}\}, \mathrm{MLP}} \sum_{(\{U^t\}, \{\mathbf{O}^t_i\}) \in \mathcal{T}} \sum_t \mathcal{L}((U^t)^*, \hat{U}^t).
\label{gnn loss}
\end{equation}
Here, \(\mathcal{L}\) denotes the loss function, which quantifies the fidelity of the GNN's predictions in replicating the expert policy's actions. The set \(\mathcal{T}\) represents the training dataset, consisting of sequences of observations and corresponding actions.

By optimizing the GNN to closely align with the expert policy, we enhance the accuracy of the graph representations, which in turn supports more effective information processing. Through this process, we obtain a pre-trained GNN-based NAR that provides knowledge-rich spatial information for the subsequent cross-attention process in the LLM-NAR framework, facilitating superior decision-making.
\begin{figure}[h]
  \centering
  \includegraphics[scale=0.28]{figure/NAR.pdf}
  \caption{Pretrained GNN-based NAR}
  \label{nar}
  \vspace{-0.2in}
\end{figure}


% 在这部分,我们构建一个pretrained Gnn来作为nar帮助llm来理解mapf任务的地图和邻居信息。我们通过最小化gnn输出动作与最优动作集的偏差来优化gnn网络。在本部分,我们将mapf的graph表示扩展为$G = (V, E, C)$.其中C表示重要邻居之间的connection。特别指出,gnn based nar的最终目的是找到有相互影响的邻居节点,并考虑这些邻居节点的状态信息,为LLm的决策进一步蒸馏提取有效状态信息。下面我们介绍如何C如何发挥作用。
% 假设每个智能体都有自己的局部观测o^t_i,let O^t是观测矩阵包含每个智能体i=1,...N:O^t= [O^t_1,O^t_2...O^t_N].首先我们讲观测数据O^t通过一个cnn,获得编码后的特征向量X:X^t= [x^t_1,x^t_2...x^t_N].


% 我们引入一个邻接矩阵C^t,用于表示node之间的关联情况。我们将C^t相互关联的邻居智能体的观测线性组合获得聚合观测:$$\mathcal{A}\left(\mathrm{X}_t ; \mathrm{S}_t\right) =\left[\mathrm{S}_t \mathrm{X}_t\right]_{i f}=\sum_{j=1}^N\left[\mathrm{~S}_t\right]_{i j}\left[\mathrm{X}_t\right]_{j f}=\sum_{j: v_j \in \mathcal{N}_i} s_t^{i j} x_t^{j f}$$,where \mathcal{A}\left(\mathrm{X}_t ; \mathrm{S}_t\right)我们用graph convolution进行实现。
% L layer的gnn根据如下更新:

% \begin{equation}
% \mathbf{X}_{\ell}=\sigma\left[\mathcal{A}_{\ell}\left(\mathrm{X}_{\ell-1} ; \mathrm{S}\right)\right] \quad \text { for } \quad \ell=1, \ldots, L
% \end{equation}

% gnn通过如下优化问题进行训练:
% \begin{equation}
% \min _{\mathrm{CNN},\left\{\mathrm{A}_{\ell k}\right\}, \mathrm{MLP}} \sum_{\left(\left\{\mathrm{U}_t\right\},\left\{\mathbf{Z}_t^i\right\}\right) \in \mathcal{T}} \sum_t \mathcal{L}\left(\mathbf{U}_t^*, \mathcal{F}\left(\left\{\mathbf{Z}_t^i\right\}, \mathcal{G}_t\right)\right) .
% \end{equation}
% 其中我们利用cbs产生的动作座位u*.






\subsection{Cross-Attention}
% 
Inspired by the Flamingo framework \cite{bounsi2024transformers} in multi-modal tasks, our approach utilizes a multi-layer cross-attention mechanism to fuse diverse modalities of information. This section describes the implementation of our cross-attention mechanism, which is pivotal in synthesizing the final policy by integrating outputs from both LLM and the GNN-based NAR. Within our method, both Self-Attention and Gated Cross-Attention blocks are utilized to refine the integration of linguistic and spatial data. Specifically, we obtain the LLM output token $\Theta^t$, alongside the graph representation $X_{L}^t$ derived from GNN-based NAR. The token $\Theta^t$ is first processed through a Self-Attention block and further enhanced via a residual connection. Subsequently, these tokens are subjected to cross-attention with the graph representations $X_{L}^{t}$ through the Gated Cross-Attention block. 

The final output of the LLM-NAR interaction at the subsequent timestep is formulated as:
\begin{equation}
\mathbf{T}^{t}=\operatorname{FFN}\left(\operatorname{softmax}\left(\frac{\left(\mathbf{\Theta}^{t} \mathbf{Q}^t\right)^{\top} \mathbf{X}_L^{t} \mathbf{K}^t}{\sqrt{d_k}}\right) \mathbf{X}_L^{t} \mathbf{V}^t\right)
\label{cross}
\end{equation}

In this equation, \(\mathbf{Q}^t, \mathbf{K}^t \in \mathbb{R}^{k \times d_k}, \mathbf{V}^t \in \mathbb{R}^{k \times k}\) represent the query, key, and value transformations used in the cross-attention mechanism, respectively. FNN is a feed-forward network. Notably, the queries are derived from the LLM tokens, while the keys and values are obtained from the graph representations. To maintain the LLM’s intrinsic knowledge during the initial phase of training, we initially close the gate of the cross-attention mechanism, gradually introducing more complex interactions as training progresses.

% 然后Self-Attention block and one Gated Cross-Attention 被使用。首先,The output tokens from the LLM $\Theta^{t}$ are  passed through the Self-Attention block and processed via a residual connection. Subsequently, these tokens cross attend to the node of gnn  through the Gated Cross-Attention block.这个过程重复三次,如图所示。and the final outcome of the LLM-NAR is:as in: 
% \begin{equation}
% \mathbf{T}^{(t+1)}=\operatorname{FFN}\left(\operatorname{softmax}\left(\frac{\left(\mathbf{\Theta}^{(t)} \mathbf{Q}_t^{\times}\right)^{\top} \mathbf{X}_L^{(t)} \mathbf{K}_t^{\times}}{\sqrt{d_k}}\right) \mathbf{X}_L^{(t)} \mathbf{V}_t^{\times}\right)
% \end{equation}

% where $\mathbf{Q}_t^{\times}, \mathbf{K}_t^{\times} \in \mathbb{R}^{k \times d_k}, \mathbf{V}_t^{\times} \in \mathbb{R}^{k \times k}$ are the key, query and value transformations of the cross-attention, respectively. 值得注意的是:, the queries are derived from the LLM tokens, while the keys and values are obtained from the graph representations 。To preserve the LLM' intrinsic knowledge during the early training phase, we initially close the gate of the cross-attention mechanism.

% The module consists of one Self-Attention block and one Gated Cross-Attention block\cite{alayrac2022flamingovisuallanguagemodel}, repeated three times in sequence, as shown in Figure \ref{}. The output tokens from the LLM are first passed through the Self-Attention block and processed via a residual connection. Subsequently, these tokens cross attend to the graph representations through the Gated Cross-Attention block. In this configuration, the queries are derived from the LLM tokens, while the keys and values are obtained from the graph representations, as in: 

% \begin{equation}
% \mathbf{T}^{(t+1)}=\operatorname{FFN}\left(\operatorname{softmax}\left(\frac{\left(\mathbf{\Theta}^{(t)} \mathbf{Q}_t^{\times}\right)^{\top} \mathbf{X}_L^{(t)} \mathbf{K}_t^{\times}}{\sqrt{d_k}}\right) \mathbf{X}_L^{(t)} \mathbf{V}_t^{\times}\right)
% \end{equation}

% where $\mathbf{Q}_t^{\times}, \mathbf{K}_t^{\times} \in \mathbb{R}^{k \times d_k}, \mathbf{V}_t^{\times} \in \mathbb{R}^{k \times k}$ are the key, query and value transformations of the cross-attention, respectively. To preserve the LLM' intrinsic knowledge during the early training phase, we initially close the gate of the cross-attention mechanism. 


\begin{table*}[htbp]
\centering
\renewcommand{\arraystretch}{1.2} % 增加行距
\caption{Success Rate of different method}
\label{success rate}

\resizebox{\textwidth}{!}{
\begin{tabular}{c||ccccc||ccccc||ccccc}
\toprule
% \multirow{3}{*}{Model}  & \multicolumn{15}{c}{Success Rate}  \\ \cmidrule{2-16}
  \multirow{2}{*}{Model}      & \multicolumn{5}{c||}{8x8 empty map with 2,4,8,10,16 agents}    & \multicolumn{5}{c||}{20x20 empty map with 2,4,8,10,16 agents}     & \multicolumn{5}{c}{28x28 empty map with 2,4,8,10,16 agents}       \\ 
    & 2 & 4 & 8 & 10 & {16} & 2 & 4 & 8 & 10 & {16} & 2 & 4 & 8 & 10 & 16 \\ \midrule 
Qwen2 & 55.00\% & 50.00\% & 30.00\% & 18.75\% & {9.38\%} & 50.00\% & 25.00\% & 10.00\%& 9.38\%  & {4.69\%} & 37.50\% & 18.75\% & 7.50\% & 4.69\% & 3.13\%  \\
Gemma2 & 100.00\% & 93.75\% & 68.75\% & 55.00\% & {43.75\%} & 100.00\% & 93.75\% & 53.13\%& {46.88\%} & 27.50\%  & 100.00\% & 87.50\% & 40.63\% & 40.00\% & 28.13\% \\
LLaMA3 & 100.00\% & 81.25\% & 71.25\% & 45.00\% & {31.13\%} & 93.75\% & 56.25\% & 50.00\% & 30.00\% & {21.88\%} & 100.00\% & 75.00\% & 60.00\% & 56.25\% & 26.56\% \\
GPT-3.5-turbo& 100.00\% & 90.63\% & 75.00\% & {59.38\%}& 50.00\%  & 75.00\%  & 71.88\% & 65.00\% & 62.50\%& {56.94\%}& 100.00\% & 75.00\% & 62.50\% & 56.25\% & 46.88\% \\ \midrule
LLM-NAR & \textbf{100.00\%} & \textbf{100.00\%} & \textbf{93.75\%} & \textbf{75.00\%} & {\textbf{67.19\%}} & \textbf{100.00\%} & \textbf{93.75\%} & \textbf{93.75\%} & \textbf{80.00\%} & {\textbf{65.63\%}} & \textbf{100.00\%} & \textbf{100.00\%} & \textbf{75.00\%} & \textbf{70.00\%} & \textbf{60.94\%} \\ \midrule \midrule 
       & \multicolumn{5}{c||}{8x8 map (10\% obstacles) with 2,4,8,10,16 agents}    & \multicolumn{5}{c||}{16x16 map (10\% obstacles) with 2,4,8,10,16 agents}     & \multicolumn{5}{c}{20x20 map (10\% obstacles) with 2,4,8,10,16 agents}       \\ 
    & 2 & 4 & 8 & 10 & {16} & 2 & 4 & 8 & 10 & {16} & 2 & 4 & 8 & 10 & 16 \\ \midrule 
Qwen2  &  37.50\% & {23.44\%} & 15.00\% & 12.50\% & 6.25\% &   25.00\% & 12.50\% & 6.25\% & 6.25\%  & {3.13\%}& 25.00\%& 12.50\%  & 9.38\% & 4.69\%& 2.50\%  \\
Gemma2 & 75.60\% & 62.50\%& {54.69\%}&  37.50\%  & 20.00\%  & 75.00\% & {42.19\%}& 40.63\%& 22.50\%&  12.50\%    & 75.00\% & 43.75\% & 31.25\%& 20.00\%& 18.75\%   \\
LLaMA3 & 81.25\%&  62.50\%  & 52.50\%& 46.88\%  & {39.06\%} & 37.50\%& 25.00\%&  17.50\% &  12.50\%   & {7.81\%}& 75.00\% & 50.00\% & 17.50\%& 12.50\%  & 7.81\% \\
GPT-3.5-turbo  & 81.25\%& 75.00\%  & 50.00\% & 42.50\% & {36.06\%}  & 62.50\% & 46.88\% & 50.00\% & {50.00\%}& 37.50\%& 75.00\%  & 71.88\% & 62.50\%& 60.00\% & 48.44\% \\ \midrule
LLM-NAR &  \textbf{100.00\%}& \textbf{100.00\%} & \textbf{78.13\%} & \textbf{72.50\%} & {\textbf{65.75\%}} &  \textbf{87.50\%} & \textbf{75.00\%} & \textbf{69.38\%} & \textbf{64.50\%} & {\textbf{60.25\%}}& \textbf{87.50\%} & \textbf{81.25\%} & \textbf{75.00\%} & \textbf{65.00\%} & \textbf{59.13\%} \\ \bottomrule


\end{tabular}}
\end{table*}

% \begin{table*}[htbp]
% \centering
% \renewcommand{\arraystretch}{1.2} % 增加行距
% \caption{Success Rate of different method}
% \label{}

% \resizebox{\textwidth}{!}{
% \begin{tabular}{c||ccccc||ccccc||ccccc}
% \toprule
% % \multirow{3}{*}{Model}  & \multicolumn{15}{c}{Success Rate}  \\ \cmidrule{2-16}
%   \multirow{2}{*}{Model}      & \multicolumn{5}{c||}{8x8 empty map with 2,4,8,10,16 agents}    & \multicolumn{5}{c||}{20x20 empty map with 2,4,8,10,16 agents}     & \multicolumn{5}{c}{28x28 empty map with 2,4,8,10,16 agents}       \\ 
%     & 2 & 4 & 8 & 10 & {16} & 2 & 4 & 8 & 10 & {16} & 2 & 4 & 8 & 10 & 16 \\ \midrule 
% Qwen2 & 50.00\% & 50.00\% & 18.75\% & 30.00\% & {9.38\%} & 50.00\% & 25.00\% & 9.38\% & 10.00\% & {4.69\%} & 37.50\% & 18.75\% & 3.13\% & 7.50\% & 4.69\% \\
% Gemma2 & 100.00\% & 93.75\% & 68.75\% & 55.00\% & {43.75\%} & 100.00\% & 93.75\% & 53.13\% & 27.50\% & {46.88\%} & 100.00\% & 87.50\% & 40.63\% & 40.00\% & 28.13\% \\
% Llama3 & 100.00\% & 81.25\% & 81.25\% & 35.00\% & {28.13\%} & 50.00\% & 93.75\% & 56.25\% & 30.00\% & {21.88\%} & 75.00\% & 100.00\% & 56.25\% & 60.00\% & 26.56\% \\
% GPT3.5-turbo& 100.00\% & 75.00\% & 90.63\% & 50.00\% & {59.38\%} & 75.00\% & 62.50\% & 71.88\% & 65.00\% & {60.94\%}& 100.00\% & 56.25\% & 75.00\% & 62.50\% & 46.88\% \\ \midrule
% LLM-NAR & \textbf{100.00\%} & \textbf{100.00\%} & \textbf{93.75\%} & \textbf{70.00\%} & {\textbf{67.19\%}} & \textbf{100.00\%} & \textbf{93.75\%} & \textbf{93.75\%} & \textbf{80.00\%} & {\textbf{65.63\%}} & \textbf{100.00\%} & \textbf{100.00\%} & \textbf{75.00\%} & \textbf{70.00\%} & \textbf{60.94\%} \\ \midrule \midrule 
%        & \multicolumn{5}{c||}{8x8 map (10\% obstacles) with 2,4,8,10,16 agents}    & \multicolumn{5}{c||}{16x16 map (10\% obstacles) with 2,4,8,10,16 agents}     & \multicolumn{5}{c}{20x20 map (10\% obstacles) with 2,4,8,10,16 agents}       \\ 
%     & 2 & 4 & 8 & 10 & {16} & 2 & 4 & 8 & 10 & {16} & 2 & 4 & 8 & 10 & 16 \\ \midrule 
% Qwen2  &  37.50\% & 6.25\% & 12.50\% & 15.00\% & {23.44\%} &  25.00\% & 6.25\% & 6.25\% & 12.50\% & {3.13\%}& 12.50\% & 25.00\% & 9.38\% & 2.50\% & 4.69\% \\
% Gemma2 &  37.50\% & 75.00\% & 62.50\% & 20.00\% & {54.69\%} &  12.50\% & 75.00\% & 40.63\% & 22.50\% & {42.19\%} & 75.00\% & 43.75\% & 18.75\% & 20.00\% & 31.25\% \\
% Llama3 &  62.50\% & 81.25\% & 46.88\% & 52.50\% & {39.06\%} &  12.50\% & 37.50\% & 25.00\% & 17.50\% & {7.81\%}& 75.00\% & 50.00\% & 12.50\% & 17.50\% & 7.81\% \\
% GPT3.5-turbo  & 75.00\% & 81.25\% & 50.00\% & 42.50\% & {39.06\%} & 37.50\% & 62.50\% & 46.88\% & 50.00\% & {50.00\%}& 75.00\% & 62.50\% & 71.88\% & 60.00\% & 48.44\% \\ \midrule
% LLM-NAR &  \textbf{100.00\%}& \textbf{100.00\%} & \textbf{78.13\%} & \textbf{72.50\%} & {\textbf{68.75\%}} &  \textbf{87.50\%} & \textbf{75.00\%} & \textbf{59.38\%} & \textbf{57.50\%} & {\textbf{56.25\%}}& \textbf{87.50\%} & \textbf{81.25\%} & \textbf{75.00\%} & \textbf{65.00\%} & \textbf{53.13\%} \\ \bottomrule


% \end{tabular}}
% \end{table*}

\begin{table*}[htbp]
\centering
\renewcommand{\arraystretch}{1.2} % 增加行距
\caption{Average Step of different method}
\label{average step}

\resizebox{\textwidth}{!}{
\begin{tabular}{c||ccccc||ccccc||ccccc}
\toprule
% \multirow{3}{*}{Model}  & \multicolumn{15}{c}{Success Rate}  \\ \cmidrule{2-16}
  \multirow{2}{*}{Model}    & \multicolumn{5}{c||}{8x8 empty map with 2,4,8,10,16 agents}    & \multicolumn{5}{c||}{20x20 empty map with 2,4,8,10,16 agents}     & \multicolumn{5}{c}{28x28 empty map with 2,4,8,10,16 agents}       \\ 
    & 2 & 4 & 8 & 10 & {16} & 2 & 4 & 8 & 10 & {16} & 2 & 4 & 8 & 10 & 16 \\ \midrule 
Qwen2  & \hspace{1.5mm}0.70\hspace{1.5mm} & \hspace{1.5mm}0.81\hspace{1.5mm} & \hspace{1.5mm}0.85\hspace{1.5mm} & \hspace{1.5mm}0.88\hspace{1.5mm} & \hspace{1.5mm}{0.97}\hspace{1.5mm} & \hspace{1.5mm}0.86\hspace{1.5mm} & \hspace{1.5mm}0.90\hspace{1.5mm} & \hspace{1.5mm}0.92\hspace{1.5mm} & \hspace{1.5mm}0.98\hspace{1.5mm} &\hspace{1.5mm}1.00\hspace{1.5mm} & \hspace{1.5mm}0.79\hspace{1.5mm} & \hspace{1.5mm}1.00\hspace{1.5mm} & \hspace{1.5mm}1.00\hspace{1.5mm} & \hspace{1.5mm}1.00\hspace{1.5mm} & \hspace{1.5mm}1.00\hspace{1.5mm} \\
Gemma2 & 0.25 & 0.46 & 0.65 & 0.75 & {0.90} & 0.51 & 0.53 & 0.79 & 0.88 & {0.93} & 0.52 & 0.67 & 0.80 & 0.83 & 0.94 \\
LLaMA3 & 0.31 & 0.44 & 0.52 & 0.72 & {0.89} & 0.54 & 0.63 & 0.75 & 0.90 & {0.95} & 0.49 & 0.55 & 0.64 & 0.83 & 0.88 \\
GPT-3.5-turbo& 0.36 & 0.49 & 0.59 & 0.65 & {0.83} & 0.51 & 0.55 & 0.64 & 0.69 & {0.79}& 0.35 & 0.55 & 0.61 & 0.63 & 0.70 \\ \midrule
LLM-NAR & \textbf{0.22} & \textbf{0.29} & \textbf{0.48} & \textbf{0.47} & {\textbf{0.63}} & \textbf{0.31} & \textbf{0.39} & \textbf{0.47} & \textbf{0.54} & {\textbf{0.64}} & \textbf{0.23} & \textbf{0.45} & \textbf{0.48} & \textbf{0.50} & \textbf{0.54} \\ \midrule \midrule 
       & \multicolumn{5}{c||}{8x8 map (10\% obstacles) with 2,4,8,10,16 agents}    &  \multicolumn{5}{c||}{16x16 map (10\% obstacles) with 2,4,8,10,16 agents}     & \multicolumn{5}{c}{20x20 map (10\% obstacles) with 2,4,8,10,16 agents}       \\ 
    & 2 & 4 & 8 & 10 & 16 & 2 & 4 & 8 & 10 & {16} & 2 & 4 & 8 & 10 & 16 \\ \midrule 
Qwen2  &  0.88 & 1.00 & 1.00 & 1.00 & {1.00} &  0.90 & 0.93 & 0.95 & 1.00 & {1.00} & 0.95 & 1.00 & 1.00 & 1.00 & 1.00 \\
Gemma2 &  0.50 & 0.56 & 0.70 & 0.76 & {0.91} &  0.61 & 0.70 & 0.81 & 0.89 &{0.93} & 0.58 & 0.71 & 0.84 & 0.92 & 0.98 \\
LLaMA3 &  0.63 & 0.66 & 0.75 & 0.98 & {1.00} &  0.75 & 0.79 & 0.84 & 0.94 & {1.00} & 0.51 & 0.74 & 0.98 & 1.00 & 1.00 \\
GPT-3.5-turbo &  0.56 & 0.59 & 0.80 & 0.87 & {0.88} &  0.56 & 0.57 & 0.74 & 0.78 & {0.81} & 0.45 & 0.68 & 0.71 & 0.78 & 0.84 \\ \midrule
LLM-NAR &  \textbf{0.31} & \textbf{0.34} & \textbf{0.51} & \textbf{0.60} & {\textbf{0.62}} &  \textbf{0.32} & \textbf{0.52} & \textbf{0.60} & \textbf{0.65} & {\textbf{0.72}} & \textbf{0.33} & \textbf{0.55} & \textbf{0.61} & \textbf{0.63} & \textbf{0.71} \\ 


\bottomrule


\end{tabular}}
\end{table*}



This process repeats \(N_l\) times, which is set to three in this paper. As we enter the next round of cross-attention, the previous \(T^t\) is used as the input \(\Theta^t\) for the next layer, and a new \(T^t\) is generated according to Eq.~\ref{cross}. Subsequently, the predicted actions are extracted from the final layer output \(T^t_{N_l}\). During the training of the LLM-NAR, we freeze the parameters of both the LLM and the pre-trained GNN-based NAR, focusing only on updating the cross-attention component. The optimization process is similar to that used during the pre-training of the GNN. We train the model by minimizing the discrepancy between the actions output of the LLM-NAR, denoted \(U_{LLM-NAR}\), and the optimal actions \((U^t)^*\) derived from CBS~\cite{sharon2015conflict}:
\begin{equation}
\min \mathcal{L}(U_{LLM-NAR}^t, (U^t)^*),
\label{llm-nar loss}
\end{equation}
% 通过LLM-NAR,我们将llm和gnn based nar的两种能力进行融合,利用gnn来提升了llm对于地图信息的理解。我们这个方法适用于不同规模mapf任务,因为llm是适用于不同规模mapf任务的,同时gnn也具有一定的规模泛化能力。同时在实际应用中,我们的方法仅需要训练cross attention部分,llm可以通过本地部署或者调用大模型的api,而gnn也可以通过预训练获得。
By integrating an LLM with a GNN-based NAR in the LLM-NAR framework, we enhance the LLM's ability to comprehend map information. This approach is well-suited for MAPF tasks at various scales due to the scalability of both the LLM and GNN. In practice, only the cross-attention component requires training. LLMs, such as GPT or LLaMA, can function locally or via an API, while the GNN, as detailed in \cite{li2020graph}, can be utilized through pre-training, streamlining deployment in real-world applications. Notably, training the cross-attention mechanism requires only a small number of samples and steps, significantly fewer than methods like reinforcement learning. 





% This process repeats $N_l$ times, which is set three in this paper. Then the predicted actions is read from the final layer output $T_{N_l}$.在训练LLM-NAR时,我们冻结LLM和pretrained GNN based NAR的参数,更新cross attention部分。而优化过程与pretrain gnn时类似,我们通过最小化LLM-NAR输出的aciton U_{LLM-NAR}与传统方法中的最优策略CBS的动作U^t*之间的差异来进行训练:


\section{Experiment Results}
This section demonstrates our LLM-NAR's superiority via experiments in both simulation and real-world MAPF tasks.

\subsection{Experiment Setup}
% 为了验证我们方法的有效性,我们在多个不同size的地图和不同障碍物密度的任务中进行了实验。我们的LLM-NAR方法适用于不同的LLM算法,在本文实验中,我们采用的基础LLM模型是GPT3.5-turbo。我们选择了现有常用的多个LLM模型作为baseline。to be honest,由于费用的限制,我们暂时没有对gpt4进行完整测试,但是我们认为我们方法通过改造gpt3.5获得提升,那么这类方法也适用于未来更加前进的模型,我们的核心实验目的是为了测试我们方法的实际价值。




To validate the effectiveness of our approach, we conducted experiments on maps with varying sizes and obstacle densities. Given the specific conditions of each task and potential performance variations of the LLM, we ran each algorithm 10 times on every map and averaged the results to obtain our experimental data. Notably, we use 100 execution cases to train the cross-attention mechanism and complete the process in just 5K training steps.

We employed two evaluation metrics to assess performance: \textit{success rate} and \textit{average step}. The \textit{success rate} (\(R\)) is
the percentage of agents that reach their goals in all episodes, expressed as $R = n_{\text{success}}/{n}$, where \(n_{\text{success}}\) represents the number of agents who succeed, and \(n\) is the total number of agents. The \textit{average step} (\(\delta\)) is calculated as the mean executed path length for all agents, normalized by the maximum number of steps allowed in the map. It is computed by dividing the total number of steps taken by all agents by the product of the number of agents and the maximum steps: $\delta = \frac{\text{Total Steps}}{\text{Number of Agents} \times \text{Max Steps}}$.


\begin{figure}[ht]
    \centering
    \begin{subfigure}{0.48\linewidth}
        \centering
        \includegraphics[width=\textwidth]{figure/gpt_tra.pdf}
        \caption{Trajectories of the robots with GPT.}
        \label{gpt_tra}
    \end{subfigure}
    \hfill
    \begin{subfigure}{0.48\linewidth}
        \centering
        \includegraphics[width=\textwidth]{figure/LLMnar_tra.pdf}
        \caption{Trajectories of the robots with our method.}
        \label{ours_tra}
    \end{subfigure}
    \caption{Comparison of the agent trajectories in the $20 \times 20$ map with a 10\% obstacle ratio and 10 agents.}
    \label{sim_tra}
    \vspace{-0.1in}
\end{figure}





Our LLM-NAR method is compatible with various LLM models. In this study, we primarily used GPT-3.5-turbo as the foundational model, while also selecting several commonly used LLMs as baselines for comparative analysis. Due to budgetary constraints, we have not performed comprehensive testing on GPT-4. However, we believe that if our method demonstrates improvements with GPT-3.5, it should similarly enhance more advanced models in the future. The primary objective of our experiments is to showcase the practical value of our approach, highlighting its applicability and scalability to newer, more advanced models.


\begin{figure*}[ht]
    \centering
    \begin{minipage}{0.32\textwidth}
        \centering
        \includegraphics[width=\textwidth]{figure/real1.pdf}
        \subcaption{Two robots}
        \label{tra-gpt}
    \end{minipage}
    \hfill
    \begin{minipage}{0.32\textwidth}
        \centering
        \includegraphics[width=\textwidth]{figure/real2.pdf}
        \subcaption{Three robots}
        \label{tra-llmnar}
    \end{minipage}
    \hfill
    \begin{minipage}{0.32\textwidth}
        \centering
        \includegraphics[width=\textwidth]{figure/real3.pdf}
        \subcaption{Four robots}
        \label{real3}
    \end{minipage}
    \caption{Real-world experiment with two, three, and four robots.}
    \label{real}
    \vspace{-0.3in}
\end{figure*}

\subsection{Simulation Results}
% 我们在仿真中对不同大小地图、不同数量智能体和有无障碍物的任务进行了测试.我们以success rate指标来表示算法成功率,以average step来体现算法的路径效率。其中success rate如表1所示,average step 如表2所示。
% 从表1可以看到,在每个地图中,随着智能体数量的增加,各种方法的成功率均有所下降。这表明随着智能体数量的增加,任务难度增加。在智能体数量较少时,baseline modelLLAMA3和GPT有不错的表现,但是随着智能体数量增加,其成功率快速下降。而LLM-NAR的成功率在不同智能体数量下均优于其他baseline,且在智能体数量较多时也有不错的表现。而在更复杂的有障碍物的任务中,LLM-NAR也比其他的baseline有更好的表现。
We conducted simulations to test tasks on maps of different sizes, with varying numbers of agents, and in environments both with and without obstacles. We used the \textit{success rate} metric to represent the algorithm's ability to successfully complete tasks, and the \textit{average step} to reflect the algorithm's path efficiency. The \textit{success rate} is shown in Table~\ref{success rate}, and the \textit{average step} is presented in Table~\ref{average step}.

As shown in Table~\ref{success rate}, on each map, the success rates of all methods decrease as the number of agents increases. This suggests that task difficulty increases with the number of agents. When the number of agents is small, the baseline models LLaMA3 and GPT perform well, but their success rates decline rapidly as the number of agents increases. In contrast, LLM-NAR consistently achieves higher success rates than the other baselines across different numbers of agents, and maintains good performance even when the number of agents is large. In the 20$\times$20 map without obstacles, when the number of agents is 10, GPT achieves a success rate of 62.50\%, while LLM-NAR reaches 80.00\%, significantly higher than other baselines. When the number of agents increases to 16, the success rates of the baseline models drop below 60\%, whereas LLM-NAR maintains a success rate of 65.63\%.


Furthermore, in more complex tasks involving obstacles, LLM-NAR also outperforms the other baselines. Due to the impact of obstacles, Qwen2's performance is relatively poor. LLaMA3 performs relatively well when the number of agents is small, while GPT performs better than other baselines when the number of agents is large. However, LLM-NAR consistently outperforms all baseline models across different map sizes and numbers of agents, indicating that in environments with obstacles, LLM-NAR enhances task performance.

Table~\ref{average step} presents the results for the average step metric. Overall, across tasks with different map sizes and numbers of agents, LLM-NAR requires the fewest average steps to reach the target points, resulting in shorter path lengths. In the 20$\times$20 map without obstacles, when the number of agents is 10, LLM-NAR reaches the goal using only $0.54 \times \text{Max Steps}$, while other methods require at least $0.69 \times \text{Max Steps}$. When the number of agents increases to 16, LLM-NAR's average step of $0.64$ still outperforms GPT's $0.79$. In maps with obstacles, the number of steps required by each method generally increases compared to the empty map. However, when comparing different methods, our LLM-NAR still shows significant advantages, outperforming all baseline models across all map sizes and numbers of agents.

% Furthermore, in more complex tasks involving obstacles, LLM-NAR also outperforms the other baselines.由于障碍物的影响,Qwen2的表现较差,LLama3在智能体数量少的情况下表现相对较好,gpt在数量较多时表现优于其他baseline,而llm-nar在不同大小地图和智能体数量均优于所有baseline model,表明在有障碍物情况,llm-nar通过gnn based nar的inform提升了任务能力。

% 表2为average step指标上的结果。整体上,在不同大小地图和智能体数量任务中,LLM-NAR到达目标点所需要的路径step数量更少,路径长度更短。在没有障碍物的20*20地图中,当智能体数量为10时候,LLM-NAR仅需要0.54*max steps就可以到达,而其他方法至少需要0.69*max steps。当智能体数量为16时,LLM-NAR 的0.64同样优于GPT的0.79。在有障碍物地图中,整体上每种方法的step数量均比empty map时增加一些,而对比不同方法之间,我们的llm-nar还是非常有优势,在所有大小地图和智能体数量情况下均优于所有baseline model。
% 图1 为gpt和LLM-nar在20 x 20 map(10%obstacles)with 10 agent地图中的任务轨迹图。为了方便对比,我们在同样的障碍物分布中进行了测试。从轨迹可以看到,agent with我们的方法llm-nar均到达了目标点,而gpt agents中一个未能正确到达目标点。同时our proposed method的路线比gpt agents的轨迹更短,路径效率更高。
Fig.~\ref{sim_tra} illustrates the task trajectories of GPT-3.5-turbo and LLM-NAR on a $20 \times 20$ map with a 10\% obstacle ratio and 10 agents. Both methods were tested under the same obstacle setting. From the trajectory visualization, it is evident that all agents using our proposed LLM-NAR successfully reached their respective goal positions on this map, whereas one agent in the GPT-based approach failed to reach its designated target. Additionally, the paths generated by LLM-NAR are shorter and more efficient compared to those of GPT, demonstrating superior path efficiency.





% \subsection{Practical Analysis}
% 在上述的实验结果中可以看到,我们提出的LLM-NAR有效提升了LLM在mapf任务中的表现,包括更多数量智能体、更大规模地图任务中的成功率和路线的效率。从这个层面来说,我们首次提出的llm-nar框架利用gnn信息的方式为llm求解mapf问题提供新的解决思路。除了和llm进行比较,我们也开展了和传统规划方法,例如cbs,以及强化学习方法例如primal、dhc、scrimp的实验对比。
\subsection{Practical Analysis}

From the experimental results presented above, it can be observed that our proposed LLM-NAR significantly improves the performance of LLM in MAPF tasks. This improvement is evident in terms of higher success rates and more efficient paths, especially in tasks involving a larger number of agents and larger-scale maps. From this perspective, our proposed LLM-NAR framework provides a novel approach to solving MAPF problems by leveraging GNN-based information.

% In addition to comparisons with LLM, we also conducted experiments comparing our method with traditional planning approaches, such as CBS, and reinforcement learning methods, including Primal, DHC, and SCRIMP.首先,对于learning based 的方法,一个比较重要的指标是训练效率,也就是完成训练所需要的step数,表1为不同rl method策略收敛所需的训练步数,可以看到,我们llm-nar远小于其他rl方法。而在执行时间也就是计算代价方面,我们方法在执行时的时间低于cbs等planning方法,且随着智能体数量的增加,runningtime方面的优势更大。
In addition to comparisons with LLM, we also conducted experiments comparing our method with traditional planning approaches, such as CBS, and reinforcement learning methods, including PRIMAL~\cite{sartoretti2019primal}, DHC~\cite{ma2021distributed}, and SCRIMP~\cite{wang2023scrimp}. 

One crucial metric for learning-based methods is training efficiency, which is typically measured by the number of steps required for convergence. Table~\ref{tab:training_steps} presents the training steps needed for different RL-based methods. As shown in the table, our LLM-NAR framework requires significantly fewer training steps compared to other RL-based approaches.

\begin{table}[h]
    \centering
    \caption{Required Training Steps}
    \label{tab:training_steps}
    \begin{tabular}{c|cccc}
        \hline
        \textbf{Method} & \textbf{LLM-NAR} & \textbf{PRIMAL} & \textbf{DHC} & \textbf{SCRIMP} \\ 
        \hline
        \textbf{Training Steps} & $5 \times 10^3$ & $3 \times 10^5$ & $3 \times 10^5$ & $3 \times 10^5$ \\
        \hline
    \end{tabular}
\end{table}

As shown in Table~\ref{tab:running_time_20x20}, which presents execution time as an indicator of computational cost, our method achieves significantly lower execution time compared to planning-based approaches such as CBS. Notably, as the number of agents increases, the runtime efficiency gap between LLM-NAR and CBS widens, further highlighting the scalability advantage of our approach.

\begin{table}[h]
    \centering
    \caption{Running Time in 20×20 Map (seconds)}
    \label{tab:running_time_20x20}
    \begin{tabular}{c|cc}
        \hline
        \textbf{Agent Number} & \textbf{LLM-NAR} & \textbf{CBS} \\ 
        \hline
        4  & $1.7 \pm 0.5$  & $3.1 \pm 1.3$  \\ 
        8  & $1.9 \pm 0.5$  & $9.3 \pm 3.6$  \\ 
        16 & $2.0 \pm 0.6$  & $32.3 \pm 10.4$ \\ 
        \hline
    \end{tabular}
    \vspace{-0.2in}
\end{table}






\subsection{Real-world Results}
In addition to the simulation study, we further tested the proposed method in real-world experiments using LIMO mobile robots. In the experiments, we obtained the robots' position information at the base station through the Nokov motion capture system. The LIMO robots utilize Mecanum wheel motion, allowing them to perform the up, down, left, and right movements required in MAPF tasks.


% \begin{figure}[h]
%     \centering
%     \includegraphics[width=0.8\linewidth]{real.png}
%     \caption{Real-world experiment}
%     \label{real}
% \end{figure}


In the real-world tests, due to limitations in available space and the number of robots, we evaluated task execution with two to four robots in a $5 \times 4$ size map using LLM-NAR, GPT, and LLaMA3. In the task involving two agents, all three methods successfully completed the task, but the paths generated by LLM-NAR were shorter compared to those produced by GPT and LLaMA3. In the task with four agents, GPT had one robot that failed to reach its target point during execution, and LLaMA3 had two robots fail. In contrast, all robots controlled by LLM-NAR reached their target points, and the paths were shorter. Fig.~\ref{real} shows the trajectories of our LLM-NAR method in the two, three, and four robot tasks. 


% For more detailed experimental results, please refer to our video materials\footnote{\url{https://github.com/fpgod/LLM-NAR/blob/main/ijcnn_video.mp4}}.


% In addition to the simulation study, we further tested the proposed method in a real-world experiment on the mobile robots LIMO.实验中,我们在基站上通过nokov动作捕捉系统获得机器人位置信息,并通过ROS2进行通讯与指令发送。Limo机器人采用麦克伦运动模式,因此可以执行mapf任务中的上下左右动作。在real-world 测试中,我们测试了在2-6个机器人数量下的任务执行情况。在2个智能体任务时,LLM-NAR和GPT顺利完成了任务,但是LLM—NAR的路径相比于GPT更短。在6个智能体任务中,GPT在任务执行中有一个机器人未能到达目标点,而LLM-NAR控制的机器人均到达目标点,且路径更短。


\section{Conclusion}
In this paper, we addressed the issue of the poor performance of Large Language Models (LLM) in Multi-agent Path Finding (MAPF) by proposing the Neural Algorithmic Reasoners informed Large Language Model for Multi-agent Path Finding (LLM-NAR). We introduced an improved LLM prompting method for MAPF and constructed a Graph Neural Network (GNN)-based Neural Algorithmic Reasoner specifically for MAPF. By integrating the information from both components through a cross-attention mechanism, we developed the LLM-NAR policy. Simulations and real-world experiments demonstrated that our method outperforms other LLM models across various map settings and numbers of agents. In future work, we will further explore utilizing methods such as temporal graph neural networks to optimize the performance of LLM in MAPF tasks.



\bibliographystyle{IEEEtran}
\bibliography{ref}


\end{document}
