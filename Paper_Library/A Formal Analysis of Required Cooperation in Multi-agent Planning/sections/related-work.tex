\section{Related Work}
\label{sec:related}


One of the earlier works on MAP is the PGP framework by \cite{Durfee91,Decker92}. Recently, the MAP problem has started to receive an increasing amount of attention. Most of these recent research works consider agents separately for planning, and have been concentrated on how to explore the structure of agent interactions to reduce the search space, as well as solving the problem in a distributed fashion. \cite{Nissim2010} provide a search method by compiling MAP into a constraint satisfaction problem (CSP), and then using a distributed CSP framework to solve it. The MAP formulation is based on an extension of the STRIPS language called MA-STRIPS \cite{brafman2008}. In MA-STRIPS, actions are categorized into public and private actions. Public actions can influence other agents while private actions cannot. In this way, it is shown by \cite{brafman2008} that the search complexity of MAP is exponential in the tree-width of the agent interaction graph. Due to the poor performance of DisCSP based approaches, \cite{Nissim2012} apply the $A^*$ search algorithm in a distributed manner, which represents one of the state-of-art MAP solvers. \cite{torreno2012} propose a POP-based distributed planning framework for MAP, which uses a cooperative refinement planning technique that can handle planning with any level of coupling between the agents. Each agent at any step proposes a refinement step to improve the current group plan. Their approach does not assume complete information. A similar paradigm is taken by \cite{kvarnstrom11}. An iterative best-response planning and plan improvement technique using standard SAP algorithms is provided by \cite{jonsson2011}, which considers the previous singe agent plans as constraints to be satisfied while the following agents perform planning.


Given a problem, all of these MAP approaches solve it using the given set of agents, without first asking whether multiple agents are really required, let alone what is the minimum number of agents required. Answers to these questions not only separate MAP from SAP in a fundamental way, but also have real world applications when the team compositions can dynamically change. In this paper, we analyze these questions using the SAS$^+$ formalism \cite{backstrom96} with {\em causal graph} \cite{Knoblock94,helmert06}, which is often discussed in the context of factored planning \cite{bacchus93,amir03,brafman06,brafman2013}. The causal graph captures the interaction between different variables; intuitively, it can also capture the interactions between agents since agents affect each other through these variables. In fact, \cite{brafman2013} mention the causal graph's relation to the agent interaction graph when each variable is associated with a single agent.




























