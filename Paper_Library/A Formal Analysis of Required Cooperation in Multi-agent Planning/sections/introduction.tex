\section{Introduction}
\label{sec:intro}


A multi-agent planning (MAP) problem differs from a single agent planning (SAP) problem in that more than one agent is used in planning. While a (non-temporal) MAP problem can be compiled into a SAP problem by considering agents as resources, the search space grows exponentially with the number of such resources. Given that a SAP problem with a single such resource is in general PSPACE-complete \cite{Bylander1991}, running a single planner to solve MAP is inefficient. Hence, previous research has generally agreed that agents should be considered as separate entities for planning, and thus has been mainly concentrated on how to explore the interactions between the agents (i.e., loosely-coupled vs. tightly-coupled) to reduce the search space, and how to perform the search more efficiently in a distributed fashion. 


However, there has been little discussion on whether multiple agents are required for a planning problem in the first place. If a single agent is sufficient, solving the problem with multiple agents becomes an efficiency matter, e.g., shortening the makespan of the plan. Problems of this nature can be solved in two separate steps: planning with a single agent and optimizing with multiple agents. In such a way, the difficulty of finding a solution may potentially be reduced. 


In this paper, we aim to answer the following questions: $1)$ Given a problem with a set of agents, what are the conditions that make cooperation between multiple agents {\em required} to solve the problem; $2)$ How to determine the minimum number of agents required for the problem. We show that providing the exact answers is intractable. Instead, we attempt to provide approximate answers. To facilitate our analysis, we first divide MAP problems into two classes -- MAP problems with heterogeneous agents, and MAP problems with homogeneous agents. Consequently, the MAP problems that {\em require cooperation} (referred to as RC problems) are also divided into two classes -- type-$1$ RC (RC with heterogeneous agents) and type-$2$ RC (RC with homogeneous agents) problems. Figure \ref{fig:map-div} shows these divisions.


For the two classes of RC problems, we aim to identify all the conditions that can cause RC. Figure \ref{fig:rc-div} presents these conditions and their relationships to the two classes of RC problems. We establish that at least one of these conditions must be present in order to have RC. Furthermore, we show that most of the problems in common planning domains belong to type-$1$ RC, which is identified by three conditions in the problem formulation that define the heterogeneity of agents; most of the problems in type-$1$ RC can be solved by a {\em super agent}. For type-$2$ RC, we show that RC is only caused when the state space is not {\em traversable} or when there are {\em causal loops} in the causal graph. We provide upper bounds for the answer of the second question for type-$2$ RC problems, based on different relaxations of the conditions that cause RC, which are associated with, for example, how certain causal loops can be broken in the causal graph.


\begin{figure}
\centering
{
    \includegraphics{pdf/map.png}
}
\caption{Division of MAP problems into MAP with heterogeneous and homogeneous agents. Consequently, RC problems are also divided into two classes: type-$1$ RC involves RC problems with heterogeneous agents and type-$2$ RC involves RC problems with homogeneous agents.}
\label{fig:map-div}
\end{figure}


\begin{figure}
\centering
{
    \includegraphics{pdf/rc.png}
}
\caption{Causes of required cooperation in RC problems.}
\label{fig:rc-div}
\end{figure}


The answers to these questions not only enrich our fundamental understanding of MAP, but also have many applications. For example, in a human robot teaming scenario, a human may be remotely working with multiple robots. When a robot is assigned a task that it cannot achieve, it is useful to determine whether the failure is due to the fact that the task is simply unachievable or the task requires more than one robot. In the latter case, it is useful then to determine how many extra robots must be sent to help. The answers can also be applied to multi-robot systems, and are useful in general to any multi-agent systems in which the team compositions can dynamically change (e.g., when the team must be divided to solve different problems). 


The rest of the paper is organized as follows. After a review of the related literature, we start the discussion of required cooperation for MAP, in which we answer the above questions in an orderly fashion. We conclude afterward. 




