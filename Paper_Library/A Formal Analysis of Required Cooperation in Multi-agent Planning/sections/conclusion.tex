\section{Conclusion}
\label{sec:con}


In this paper, we introduce the notion of required cooperation (RC), which answers two questions: $1)$ whether more than one agent is required for a solvable MAP problem, and $2)$ what is the minimum number of agents required for the problem. We show that the exact answers to these questions are difficult to provide. To facilitate our analysis, we first divide RC problems into two class -- type-$1$ RC involves heterogeneous agents and type-$2$ RC involves homogeneous agents. For the first question, we show that most of the problems in the common planning domains belong to type-$1$ RC; the set of type-$1$ RC problems in which RC is only caused by DVC can be solved with a super agent. For type-$2$ RC problems, we show that RC is caused when the state space is not traversable or when there are causal loops in the causal graph; we provide upper bounds for the answer of the second question, based on different relaxations of the conditions that cause RC in type-$2$ RC problems. These relaxations are associated with, for example, how certain causal loops can be broken in the causal graph.

