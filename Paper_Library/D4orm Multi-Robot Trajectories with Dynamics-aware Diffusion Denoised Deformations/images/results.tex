{
\newcommand{\entry}[3]{
  \node[] at (\xsize * #1, #2) {
    \includegraphics[width=0.3\linewidth]{images/#3}
  };
}

\begin{figure}
\vspace{1pt}
\centering
\begin{tikzpicture}
  \scriptsize
  \tikzmath{
    \xsize = 2.75;
    \ya = 0;
    \yb = -5.0;
    \yc = -2.5;
  }
  %
  \node[] at (\xsize * 0, 1.25) {\textbf{Differential Drive}};
  \node[] at (\xsize * 1, 1.25) {\textbf{2D Holonomic}};
  \node[] at (\xsize * 2, 1.25) {\textbf{3D Holonomic}};
  %
  % scalability
  \entry{0}{\ya}{time_cmp_multi2d}
  \entry{1}{\ya}{time_cmp_multi2dholo}
  \entry{2}{\ya}{time_cmp_multi3dholo}
  \node[rotate=90] at (-\xsize * 0.5 - 0.2, \ya) {$\leftarrow$ runtime [s]};
  \node[] at (\xsize * 1, \ya-1.2) {number of robots (\textcolor{red}{$\times$}: exclude failed runs, $\bullet$: all runs succeeded)};
  %
  % convergence
  \entry{0}{\yb}{rew_cmp_multi2d_16_dot}
  \entry{1}{\yb}{rew_cmp_multi2dholo_16_dot}
  \entry{2}{\yb}{rew_cmp_multi3dholo_16_dot}
  \node[rotate=90] at (-\xsize * 0.5 - 0.2, \yb) {reward $\rightarrow$};
  \node[] at (\xsize * 1, \yb-1.2) {number of total steps ($\times 10 ^ 2$)};
  %
  % realtime response
  \entry{0}{\yc}{sr_cmp_multi2d_16}
  \entry{1}{\yc}{sr_cmp_multi2dholo_16}
  \entry{2}{\yc}{sr_cmp_multi3dholo_16}
  \node[rotate=90] at (-\xsize * 0.5 - 0.2, \yc) {success rate $\rightarrow$};
  \node[] at (\xsize * 1, \yc-1.2) {planning time [s]};
  %
\end{tikzpicture}
\caption{
Empirical results against baseline methods.
The upper figures show the time required by each method to find initial feasible solutions over different numbers of agents.
The middle ones show the planner's ability to find a solution in the given time, focusing on the case of 16 robots.
The bottom ones show how quickly each method finds plausible solutions, along with the total number of steps ($\mathrm{iters} \times N$ for \textsc{D4orm}), using 16 robots each.
Instance-wise, when to find the initial solution is marked with $\bullet$.
}
\label{fig:results}
\end{figure}
}
