{
\newcommand{\entry}[4]{
\node[anchor=west] at (#1 * \xstep, 0) {\includegraphics[width=0.19\linewidth]{images_supp/#3}};
\node[anchor=west,inner sep=0,draw] at (#1 * \xstep + 0.51, 0.56) {\includegraphics[width=0.07\linewidth,height=0.07\linewidth,trim={30 30 5 5},clip]{images_supp/#2}};
\node[anchor=center] at (#1 * \xstep + 2.0, 1.4) {\small{#4}};
}

{
\newcommand{\entrysimple}[4]{
\node[anchor=west] at (#1 * \xstep, 0) {\includegraphics[width=0.19\linewidth]{images_supp/#3}};
}
%
\begin{figure*}[tp]
  \tikzmath{
    \xstep = 3.45;
  }
\centering
\begin{tikzpicture}
  \entry{0}{16a_1obs_big}{time_cmp_multi2dholo_obs1_large}{1 large obs.}
  \entry{1}{16a_1obs}{time_cmp_multi2dholo_obs1}{1 small obs.}
  \entry{2}{16a_2obs}{time_cmp_multi2dholo_obs2}{2 obs.}
  \entry{3}{16a_3obs}{time_cmp_multi2dholo_obs3}{3 obs.}
  \entry{4}{16a_4obs}{time_cmp_multi2dholo_obs4}{4 obs.}

  \node[anchor=center,rotate=90] at (-0.2, 0.2) {\small runtime [s]};
  % \node[anchor=center] at (9.0, -1.4) {\small \#agents};
\end{tikzpicture}

\begin{tikzpicture}
  \entrysimple{0}{16a_1obs_big}{sr_cmp_multi2dholo_obs1_large}{1 large obs.}
  \entrysimple{1}{16a_1obs}{sr_cmp_multi2dholo_obs1}{1 small obs.}
  \entrysimple{2}{16a_2obs}{sr_cmp_multi2dholo_obs2}{2 obs.}
  \entrysimple{3}{16a_3obs}{sr_cmp_multi2dholo_obs3}{3 obs.}
  \entrysimple{4}{16a_4obs}{sr_cmp_multi2dholo_obs4}{4 obs.}

  \node[anchor=center,rotate=90] at (-0.2, 0.2) {\small success rate};
  \node[anchor=center] at (9.0, -1.4) {\small number of robots};
\end{tikzpicture}

\caption{
Antipodal navigation scenarios for 16 robots with varying numbers of obstacles. Top row: The computation time for finding initial solutions is provided, averaged across 30 runs, excluding any failed runs. Bottom row: The success rate over 30 runs, with a maximum of 50 iterations.
}
\label{fig:obstacles}
\end{figure*}
}

\appendix
We now present some additional studies with \textsc{D4orm} to showcase our ability to handle general target assignment scenarios (not only antipodal points on a circle/sphere), scaling up to more number of robots, and handling more complex workspaces that contain obstacles.
\newline{}

\medskip\noindent\textbf{Generic Target Assignment.}
To validate that the success of our proposed method does not depend on exploiting biases or patterns in the problem, we test it on $30$ random configurations with $\{8\ldots 16\}$ 2D holonomic robots.
The environments have random initial and target positions, all generated within a square of size \( D \times D \), with \( D \) as the diameter of the circle with antipodal points used in previous experiments.
\cref{fig:time_cmp_random_16} (left) shows a solution instance for 16 robots.
We also observe in \cref{fig:time_cmp_random_16} (right) that the method works reliably for all environments and is generally faster; the average time required to generate the first feasible solution for 16 robots is about \SI{1}{\second}, which is less than the circle with antipodal points scenario ($\approx\SI{2}{s}$).
This is because some of the trajectories do not interfere with each other due to geometric relationships in the random scenario, and are therefore often much easier to deconflict.

\begin{figure}[htbp]
    %\centering
    \begin{tikzpicture}
        \node[] at (0,0)
        {   
            \includegraphics[width=0.37\columnwidth]
                {images_supp/16a_0obs_random.pdf}
            \hspace{0.5cm}
            \includegraphics[width=0.49\columnwidth]
                {images_supp/time_cmp_multi2dholorandom.pdf}
        };
        \node[] at (2.2, -1.8)
        {
            {\small \#robots}
        };
        \node[rotate=90] at (-0.4,0)
        {
            {\small runtime [s]}
        };
        \draw[fill=gray!50,draw=none] (-3.8,-1.3)
            rectangle ++(1.8,0.35)
            node[pos=.5]
            {{\small\texttt{\#robots=16}}};
    \end{tikzpicture}
    \caption{
    A solution instance for a scenario with random targets, and time required by \textsc{D4orm} to find initial feasible solutions.
    % \textbf{(left)} A solution instance for a scenario with random targets.
    % \textbf{(right)} Time required by \textsc{D4orm} to find initial feasible solutions for different number of agents.
    % }
    }
    \label{fig:time_cmp_random_16}
\end{figure}

\medskip\noindent\textbf{Presence of Obstacles.}
We also tested our method in environments with obstacles, which can be easily addressed by a slight modification of the reward function of \cref{eqn:generic-reward}.
In particular, for each state in the trajectories, we add a binary penalty term for whether the robot collides with any obstacle.
\cref{fig:obstacles} shows that \textsc{D4orm} is capable of handling obstacles.
Meanwhile, as the workspace becomes more complex due to the addition of obstacles, the time required to obtain the first feasible solution increases, and \textsc{D4orm} also encounters failures where a few robots are sacrificed for collision in order to achieve a high team reward.
Dealing with obstacle-rich environments in a more systematic way (as opposed to a na\"{i}ve binary penalty) is an interesting future direction.
