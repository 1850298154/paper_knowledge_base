\documentclass[a4paper,11pt]{article}
\usepackage{amsmath,amssymb,graphicx,amsthm}

\usepackage[hidelinks]{hyperref}
\usepackage{algorithm}
\usepackage{algpseudocode}
\usepackage{subcaption}
\usepackage{tikz}
\usetikzlibrary{calc}
\usepackage{tkz-berge}
\usetikzlibrary{arrows.meta}
\usepackage{placeins}
\usepackage{array}
\usetikzlibrary{arrows.meta,decorations.markings,decorations}
\usepackage{mathtools}
\usepackage{multirow}
\usepackage{enumitem}
\usepackage{makecell}
\usepackage{multicol}

\usepackage{pgfplots}
\pgfplotsset{width=8.5cm}
\pgfplotsset{compat=1.8}

\usepackage[
top    = 3cm,
bottom = 3.6cm,
left   = 2.5cm,
right  = 2.5cm]{geometry}

\renewcommand{\baselinestretch}{1.0}

\usepackage[margin=1cm]{caption}

\newtheorem{THM}{THEOREM}[section]

\newtheorem{theorem}[THM]{Theorem}
\newtheorem{lemma}[THM]{Lemma}
\newtheorem{claim}[THM]{Claim}
\newtheorem{corollary}[THM]{Corollary}
\newtheorem{proposition}[THM]{Proposition}
\newtheorem{conjecture}[THM]{Conjecture}
\newtheorem{example}[THM]{Example}

\theoremstyle{definition}
\newtheorem{remark}{Remark}[section]
\newtheorem{problem}{Problem}   % number problems independently

% algo
\newcommand{\Input}{\item[\textbf{Input:}]}
\newcommand{\Output}{\item[\textbf{Output:}]}

\pgfdeclarelayer{background}
\pgfdeclarelayer{foreground}
\pgfsetlayers{background,main,foreground}

\newcommand{\convexpath}[2]{
  [
  create hullnodes/.code={
    \global\edef\namelist{#1}
    \foreach [count=\counter] \nodename in \namelist {
      \global\edef\numberofnodes{\counter}
      \node at (\nodename) [draw=none,name=hullnode\counter] {};
    }
    \node at (hullnode\numberofnodes) [name=hullnode0,draw=none] {};
    \pgfmathtruncatemacro\lastnumber{\numberofnodes+1}
    \node at (hullnode1) [name=hullnode\lastnumber,draw=none] {};
  },
  create hullnodes
  ]
  ($(hullnode1)!#2!-90:(hullnode0)$)
  \foreach [
  evaluate=\currentnode as \previousnode using \currentnode-1,
  evaluate=\currentnode as \nextnode using \currentnode+1
  ] \currentnode in {1,...,\numberofnodes} {
    let
    \p1 = ($(hullnode\currentnode)!#2!-90:(hullnode\previousnode)$),
    \p2 = ($(hullnode\currentnode)!#2!90:(hullnode\nextnode)$),
    \p3 = ($(\p1) - (hullnode\currentnode)$),
    \n1 = {atan2(\y3,\x3)},
    \p4 = ($(\p2) - (hullnode\currentnode)$),
    \n2 = {atan2(\y4,\x4)},
    \n{delta} = {-Mod(\n1-\n2,360)}
    in
    {-- (\p1) arc[start angle=\n1, delta angle=\n{delta}, radius=#2] -- (\p2)}
  }
  -- cycle
}


\usetikzlibrary{decorations.pathreplacing,calc}
\tikzset{draw half paths/.style 2 args={%
    decoration={show path construction,
      lineto code={
        \draw [#1] (\tikzinputsegmentfirst) --
        ($(\tikzinputsegmentfirst)!0.2!(\tikzinputsegmentlast)$);
        \draw [#2] ($(\tikzinputsegmentfirst)!0.2!(\tikzinputsegmentlast)$)
        -- (\tikzinputsegmentlast);
      }
    }, decorate
  }}

\tikzset{
  my box/.style = {
    , line cap = round
    , line join = round
  }
}
\newcommand{\highlight}[3]{
  \path [my box, line width = #1, draw = #2, transparency group, opacity=1] #3;
}

\usepackage{graphicx,accents}
\usepackage{esvect}
\makeatletter
\DeclareRobustCommand{\cev}[1]{%
  {\mathpalette\do@cev{#1}}%
}


\newcommand{\FBOX}{\hspace*{\fill}$\rule{0.19cm}{0.19cm}$}
\renewcommand\qedsymbol{\FBOX}

\newcommand{\SFBOX}{\hspace*{\fill}$\rule{0.16cm}{0.16cm}$}
\newenvironment{subproof}[1][\proofname]{%
  \renewcommand{\qedsymbol}{\SFBOX}%
  \begin{proof}[#1]%
  }{%
  \end{proof}%
}

% todonotes
\setlength{\marginparwidth}{2.15cm}
\usepackage[textsize=tiny]{todonotes}
\newcommand{\peter}[1]{\todo[backgroundcolor=green!50]{P: #1}}
\newcommand{\nora}[1]{\todo[backgroundcolor=red!50]{N: #1}}

\def\R{\mathbb R}
\def\C{\mathbb C}
\def\Q{\mathbb Q}
\def\Z{\mathbb Z}
\def\N{\mathbb N}
\def\E{\mathbb E}
\def\1{\mathbb 1}
\DeclareMathOperator*{\argmax}{arg\,max}
\DeclareMathOperator*{\argmin}{arg\,min}
\DeclareMathOperator{\poly}{poly}

\def\niceArrow{-{Stealth[length=2.25mm]}}
\def\NiceArrow{-{Stealth[length=2.5mm]}}

% \cev balra nyil karakter felé
\usepackage{graphicx,accents}
\usepackage{esvect}
\makeatletter
\DeclareRobustCommand{\cev}[1]{%
  {\mathpalette\do@cev{#1}}%
}
\newcommand{\do@cev}[2]{%
  \vbox{\offinterlineskip
    \sbox\z@{$\m@th#1 x$}%
    \ialign{##\cr
      \hidewidth\reflectbox{$\m@th#1\vec{}\mkern0mu$}\hidewidth\cr
      \noalign{\kern-\ht\z@}
      $\m@th#1#2$\cr
    }
  }
}
\makeatother



\newcolumntype{x}[1]{>{\centering\let\newline\\\arraybackslash\hspace{0pt}}m{#1}}


% wavy edge style
\usetikzlibrary{snakes}
\tikzset{wavy/.style={decorate,decoration={snake,amplitude=.4mm,segment length=2mm,post length=0mm,pre length=0mm},line width=.5}}

\tikzset{draw third paths/.style 2 args={%
    decoration={show path construction,
      lineto code={
        \draw [#1] (\tikzinputsegmentfirst) -- ($(\tikzinputsegmentfirst)!0.05!(\tikzinputsegmentlast)$);
        \draw [#2] ($(\tikzinputsegmentfirst)!0.05!(\tikzinputsegmentlast)$) -- ($(\tikzinputsegmentfirst)!0.95!(\tikzinputsegmentlast)$);
      }
    }, decorate
  }}
% pl.: \path [draw third paths={draw=none}{dotted}] (s)--(t);

% parallel edge style, e.g. \draw (u) to[paralleledge=3] (uv);
\tikzset{paralleledge/.style={to path={
      \pgfextra{%
        \pgfmathsetmacro{\startf}{-(#1-1)/2}
        \pgfmathsetmacro{\endf}{-\startf}
        \pgfmathsetmacro{\stepf}{\startf+1}}
      \ifnum 1=#1 -- (\tikztotarget)  \else
        let \p{mid}=($(\tikztostart)!0.5!(\tikztotarget)$)
        in
        \foreach \i in {\startf,\stepf,...,\endf}
        {%
          (\tikztostart) .. controls ($ (\p{mid})!\i*6pt!90:(\tikztotarget) $) .. (\tikztotarget)
        }
      \fi
      \tikztonodes
    }
  }
}

% e.g. \path[circleAroundEdges={.15}{.33}{.5}] (u1)--(u2) node [left=.29cm, midway] {$\ell$};
\tikzset{circleAroundEdges/.style n args={3}{
    decorate, decoration = {markings, mark=at position #3 with {\draw[double=black,white,double distance=.6pt] (0,#2) arc [x radius = #1, y radius = #2, start angle = 90, end angle = -125];} , mark=at position #3 with{\draw[double=black,white,double distance=.6pt] (0,-#2) arc [x radius = #1, y radius = #2, start angle = 270, end angle = 360+125];}}
  }
}

\usepackage{titling}
\usepackage[hang,flushmargin]{footmisc}
\settowidth{\thanksmarkwidth}{*}
\setlength{\thanksmargin}{-\thanksmarkwidth}

\usepackage[yyyymmdd]{datetime}

\newcommand\nnfootnote[1]{%
  \begin{NoHyper}
    \renewcommand\thefootnote{}\footnote{#1}%
    \addtocounter{footnote}{-1}%
  \end{NoHyper}
}

\begin{document}

\title{Separable convex optimization over indegree polytopes
}

\author{N\'ora A.\ Borsik \thanks{Department of Operations Research, E\"otv\"os Lor\'and University, P\'azm\'any P.\ s.\ 1/c, Budapest, Hungary. E-mail: {\tt nborsik@gmail.com}}
  \and P\'eter Madarasi \thanks{HUN-REN Alfr\'ed R\'enyi Institute of Mathematics, and Department of Operations Research, E\"otv\"os Lor\'and University, P\'azm\'any P.\ s.\ 1/c, Budapest, Hungary. E-mail: {\tt madarasip@staff.elte.hu} (corresponding author)}
}

\date{\vspace*{-10pt}}

\maketitle

\begin{abstract}
  We study egalitarian (acyclic) orientations of undirected graphs under indegree-based objectives, such as minimizing the $\varphi$\nobreakdash-sum of indegrees for a strictly convex function $\varphi$, decreasing minimization (dec-min), and increasing maximization (inc-max).
  In the non-acyclic setting of Frank and Murota (2022), a single orientation simultaneously optimizes these three objectives, however, restricting to acyclic orientations confines us to the corners of the indegree polytope, where these fairness objectives do diverge.

  We establish strong hardness results across a broad range of settings: minimizing the $\varphi$\nobreakdash-sum of indegrees is NP-hard for \emph{every} discrete strictly convex function $\varphi$; dec-min and inc-max are NP-hard for every indegree bound $k \geq 2$, as well as without a bound; and the complementary inc-min and dec-max problems are NP-hard even on $3$\nobreakdash-regular graphs.

  On the algorithmic side, we give a polynomial-time algorithm for minimizing the maximum weighted indegree via a weighted smallest-last ordering.
  We also provide an exact exponential-time algorithm for minimizing general separable discrete convex objectives over indegrees, and a polynomial-time algorithm for the non-acyclic case.
  Finally, for maximizing the sum of the products of indegrees and outdegrees, we prove NP-hardness on graphs of maximum degree~$4$, give an algorithm for maximum degree $3$, and provide a $3$\nobreakdash-approximation algorithm.
  Our results delineate the algorithmic frontier of convex integral optimization over indegree (base-)polytopes, and highlight both theoretical consequences and practical implications, notably for scheduling and deadlock-free routing.
  \\

  \noindent{\bf Keywords:} graph orientation, egalitarian orientations, vertex ordering, decreasing minimization, base polyhedron, NP-hardness, approximation algorithms
\end{abstract}


\section{Introduction}\label{sec:introduction}

This paper considers the minimization of a separable discrete convex objective over the indegree vectors of the (acyclic) orientations of an undirected graph, in particular, we investigate the following problem.

\begin{problem}[$\min\sum_{v\in V} \varphi_v(\varrho(v))$ orientation]\label{prob:sumPhi_VRhoV}
  Given a loop-free multigraph $G = (V, E)$ and, for each $v \in V$, a discrete convex function $\varphi_v: \Z_+ \to \R$ (i.e., a function satisfying $\varphi_v(z+1) + \varphi_v(z-1) \geq 2\varphi_v(z)$ for every positive integer $z$), find an orientation that minimizes $\sum_{v \in V} \varphi_v(\varrho(v))$, where $\varrho(v)$ denotes the indegree of vertex $v$.
\end{problem}

This problem includes several notable special cases, such as maximizing $\prod_{v \in V} \varrho(v)$, maximizing $\sum_{v\in V}\varrho(v)\delta(v)$, and minimizing $\sum_{v \in V} \left([\alpha(v) - \varrho(v)]^+ + [\varrho(v) - \beta(v)]^+\right)$ for $\alpha,\beta : V \to \R$.

Of distinguished interest is the so-called \emph{symmetric} case of Problem~\ref{prob:sumPhi_VRhoV}, when our goal is to minimize $\sum_{v \in V}\varphi(\varrho(v))$ for a discrete convex function $\varphi : \Z_+ \to \R$.
As we will see in Section~\ref{sec:decMinIncMax}, this includes as special cases the \emph{decreasingly-minimal (dec-min)} and \emph{increasingly-maximal (inc-max)} objectives, aiming to lexicographically minimize the indegree vector sorted in non-increasing order and lexicographically maximize the indegree vector sorted in non-decreasing order, respectively.
These objectives enforce that the indegree vector becomes ``smooth'', ``equitable'', or ``egalitarian'', and have been studied by Frank and Murota~\cite{frank2022decreasing2, frank2022decreasing1} under the name \emph{egalitarian orientation problems}.
They proved that a single orientation simultaneously optimizes these three types of objectives.
However, restricting to acyclic orientations confines us to the corners of the indegree polytope, where these fairness objectives do diverge, see Sections~\ref{sec:convexHullOfIndegrees}~and~\ref{sec:decMinIncMax}.

\medskip
A significant part of the present paper is devoted to the investigation of such egalitarian orientation problems under the additional requirement that the orientation must be \emph{acyclic}:
\begin{problem}[$\min\sum_{v\in V} \varphi(\varrho(v))$ acyclic orientation]\label{prob:sumPhiRhoV}
  Given a loop-free multigraph $G = (V, E)$ and a discrete strictly convex function $\varphi: \Z_+ \to \R$ (i.e., a function satisfying $\varphi(z+1) + \varphi(z-1) > 2\varphi(z)$ for every positive integer $z$), find an acyclic orientation that minimizes $\sum_{v \in V} \varphi(\varrho(v))$.
\end{problem}

\begin{problem}[decreasingly minimal (dec-min) acyclic orientation]\label{prob:decMin}
  Given a loop-free multigraph $G = (V, E)$, find an acyclic orientation in which the indegree vector sorted in non-increasing order is lexicographically minimal.
\end{problem}

\begin{problem}[increasingly maximal (inc-max) acyclic orientation]\label{prob:incMax}
  Given a loop-free multigraph $G = (V, E)$, find an acyclic orientation in which the indegree vector sorted in non-decreasing order is lexicographically maximal.
\end{problem}

In contrast to the preceding problems, which consider only the indegree distribution, we also investigate the acyclic version of another natural special case of Problem~\ref{prob:sumPhi_VRhoV} that incorporates both the indegree and outdegree of each vertex.
\begin{problem}[$\max\sum_{v\in V}\varrho(v)\delta(v)$ acyclic orientation]\label{prob:rhoTimesDelta}
  Given a loop-free multigraph $G = (V, E)$, find an acyclic orientation that maximizes the sum $\sum_{v \in V} \varrho(v)\delta(v)$, where $\varrho(v)$ and $\delta(v)$ denote the indegree and outdegree of vertex $v$, respectively.
\end{problem}

An important observation is that each of the acyclic orientation problems introduced above can be equivalently reformulated as a \emph{vertex ordering problem}.
Rather than directly searching for an optimal acyclic orientation, one may instead seek a topological ordering of an optimal orientation.
In this reformulation, indegrees $\varrho$ correspond to left-degrees $\cev{d}$, and outdegrees $\delta$ to right-degrees $\vec{d}$.
An ordering of the vertices is optimal for the vertex ordering problem if and only if the corresponding acyclic orientation --- obtained by directing each edge from the earlier to the later vertex in the order --- is optimal for the original orientation problem.
Throughout this paper, we do not explicitly distinguish between acyclic orientation problems and their vertex ordering counterparts, though we primarily adopt the vertex ordering perspective.

\medskip
We note that Problem~\ref{prob:sumPhi_VRhoV} has a self-generalizing nature, namely, one can pose lower and upper bounds $f, g : V \to \Z_+$ on the indegrees of the vertices.
To this end, we introduce a modified discrete convex function $\varphi_v^{fg}$ for each $v \in V$, which turns $\varphi_v$ into a fast-decreasing linear function before $f(v)$, and a fast growing linear function after $g(v)$.
Formally, we define the \emph{$(f, g)$-lifting} of $\varphi_v$ as

\begin{equation}\label{eq:phivfg}
  \varphi_v^{fg}(z)
  = \begin{cases}
    \varphi_v(z)                    &\text{if } f(v) \leq z \leq g(v),\\
    \varphi_v(f(v))+(f(v) - z) M    &\text{if } z < f(v),\\
    \varphi_v(g(v))+(z - g(v)) M    &\text{if } g(v) < z,
  \end{cases}
\end{equation}
where $M$ is a sufficiently large integer.
This modified function penalizes those orientations that violate the indegree bounds $f$ and $g$, restricting the search to $(f,g)$-indegree-bounded orientations if such an orientation exists.
Furthermore, our framework trivially extends to optimizing over the orientations of mixed graphs by removing each directed arc and horizontally shifting the $\varphi_v$ function of its target by one.


\paragraph{Motivation}
We provide a purely mathematical motivation for Problems~\ref{prob:sumPhiRhoV}-\ref{prob:incMax}.
Consider the indegree vectors of all orientations of a loop-free undirected multigraph $G = (V, E)$, and take their convex hull in $\R^V$.
The resulting polytope is called the \emph{indegree polytope} and is denoted by $P_{\varrho}(G)$.
It is known that $P_{\varrho}(G)$ is a base polyhedron and its integer points are precisely the indegree vectors of orientations of $G$~\cite{frank2022decreasing2,frank2022decreasing1}.
Therefore, the unconstrained (possibly cyclic) versions of Problems~\ref{prob:sumPhiRhoV}-\ref{prob:incMax} aim to optimize the corresponding objective function ($\min \sum_{v\in V} \varphi(\varrho(v))$, dec-min, or inc-max) over the integer points of $P_{\varrho}(G)$.
These problems are known to be solvable in strongly polynomial time~\cite{borradaile2017egalitarian,frank2022decreasing2,frank2022decreasing1}, despite the fact that the objective functions are non-linear --- so standard linear programming theorems do not apply directly.
Optimal solutions to these objectives often lie in the interior of the polytope.
In fact, it is not difficult to prove that no corner of $P_{\varrho}(G)$ is optimal unless $G$ is a forest.

It is therefore natural to ask for an optimal \emph{corner} of $P_{\varrho}(G)$.
Surprisingly, the corners of $P_{\varrho}(G)$ --- that is, its vertices or extreme points --- are exactly the indegree vectors of \emph{acyclic} orientations of $G$; see Theorem~\ref{thm:VerticesOfBasePolyAreAcyclicIndigrees}.
Thus, restricting the search to corners naturally yields Problems~\ref{prob:sumPhiRhoV}-\ref{prob:incMax}.


\paragraph{Applications}
First, we propose an application of Problem~\ref{prob:sumPhi_VRhoV}, which is a generalization of the scheduling problem investigated in~\cite{burcea2016scheduling}.
Consider a set $J$ of unit-sized jobs and a set $T$ of timeslots.
Each job $j \in J$ must be assigned to exactly one timeslot from a predefined feasible set $I_j \subseteq T$ associated to the job $j$.
The load $\ell(t)$ of a timeslot $t \in T$ is the total number of jobs assigned to it.
The goal is to minimize the total cost $\sum_{t \in T} h(\ell(t))$ for a discrete convex function $h$, which measures the cost of each timeslot $t$ based on the load $\ell(t)$.
This problem is known to be solvable in polynomial time~\cite{burcea2016scheduling}.

We show how this problem fits into the framework of Problem~\ref{prob:sumPhi_VRhoV}.
In fact, we consider a more general scheduling problem in which each timeslot $t \in T$ has its own discrete convex cost function $h_t$, and our goal is to minimize $\sum_{t \in T} h_t(\ell(t))$.
Consider a bipartite graph $G = (J, T; E)$, where $J$ is the set of jobs, $T$ is the set of timeslots, and $E$ contains an edge between the job $j$ and the timeslot $t$ if and only if $t \in I_j$.
Let $\varphi_t = h_t$, and define $\varphi_j$ as the $(f,g)$-lifting of the all-zero function with $f(j) = g(j) = |I_j| - 1$, enforcing that exactly one arc leaves each job $j$, which enters the timeslot in which the job $j$ is scheduled.
Clearly, the indegree of each timeslot $t$ is exactly its load $\ell(t)$.
Thus, there is a natural cost-preserving correspondence between feasible schedules and feasible orientations for Problem~\ref{prob:sumPhi_VRhoV}.

\smallskip
Furthermore, a practical application of egalitarian acyclic orientation problems arises in the context of network routing.
In this setting, an undirected graph models the network, where vertices represent switches and edges correspond to communication links that provide two channels in the two directions between the endpoints, each with specified capacity.
In \emph{wormhole routing}, each packet is divided into small units called \emph{flits}.
The first and last of these are the so-called \emph{header flit} and the \emph{tail flit}, respectively.
The header flit carries routing information, including source and destination identifiers.
All flits of a packet traverse the network in sequence.
As the header flit travels, it reserves channel capacity for the packet; the reservation is released when the tail flit passes through the given channel.
During routing, each switch attempts to forward any flit waiting at an incoming channel to the next channel along its path --- provided that the next channel has available capacity.

A major challenge is to prevent \emph{deadlock}, a condition in which flits are indefinitely blocked by cyclic dependencies among channels, thereby stalling the system.
Figure~\ref{fig:deadlock} illustrates such a scenario.
\begin{figure}[t]
  \centering
  \begin{tikzpicture}[scale=0.5]
    \tikzset{VertexStyle/.append style = {minimum size = 29pt,inner sep=0pt, shape=rectangle}}
    \SetVertexNoLabel
    \Vertex[x=0, y=0, L=A]{A}
    \Vertex[x=4, y=0, L=B]{B}
    \Vertex[x=4, y=-4, L=C]{C}
    \Vertex[x=0, y=-4, L=D]{D}

    \draw [blue] (0.75,2) node (p1) {$P_1$};
    \draw [\NiceArrow, line width=1.5pt, blue] (0.15,3)--(0.15,1);
    \draw [\NiceArrow, line width=1.5pt, black!35] (-0.15,1)--(-0.15,3);
    \draw [\niceArrow, blue] (0.15,1)--(0.15,-1);
    \draw [\NiceArrow, line width=1.5pt, blue] (0.15,-1)--(0.15,-3);
    \draw [\NiceArrow, line width=1.5pt, black!35] (-0.15,-3)--(-0.15,-1);
    \draw [\niceArrow, blue, dashed] (0.15,-3)--(1,-3.85);

    \draw [cyan] (-2,-3.25) node (p2) {$P_2$};
    \draw [\NiceArrow, line width=1.5pt, cyan] (-3,-3.85)--(-1,-3.85);
    \draw [\NiceArrow, line width=1.5pt, black!35] (-1,-4.15)--(-3,-4.15);
    \draw [\niceArrow, cyan] (-1,-3.85)--(1,-3.85);
    \draw [\NiceArrow, line width=1.5pt, cyan] (1,-3.85)--(3,-3.85);
    \draw [\NiceArrow, line width=1.5pt, black!35] (3,-4.15)--(1,-4.15);
    \draw [\niceArrow, cyan, dashed] (3,-3.85)--(3.85,-3);

    \draw [red] (3.25,-6) node (p3) {$P_3$};
    \draw [\NiceArrow, line width=1.5pt, red] (3.85,-7)--(3.85,-5);
    \draw [\NiceArrow, line width=1.5pt, black!35] (4.15,-5)--(4.15,-7);
    \draw [\niceArrow, red] (3.85,-5)--(3.85,-3);
    \draw [\NiceArrow, line width=1.5pt, red] (3.85,-3)--(3.85,-1);
    \draw [\NiceArrow, line width=1.5pt, black!35] (4.15,-1)--(4.15,-3);
    \draw [\niceArrow, red, dashed] (3.85,-1)--(3,-0.15);

    \draw [green] (6,-0.75) node (p4) {$P_4$};
    \draw [\NiceArrow, line width=1.5pt, green] (7,-0.15)--(5,-0.15);
    \draw [\NiceArrow, line width=1.5pt, black!35] (5,0.15)--(7,0.15);
    \draw [\niceArrow, green] (5,-0.15)--(3,-0.15);
    \draw [\NiceArrow, line width=1.5pt, green] (3,-0.15)--(1,-0.15);
    \draw [\NiceArrow, line width=1.5pt, black!35] (1,0.15)--(3,0.15);
    \draw [\niceArrow, green, dashed] (1,-0.15)--(0.15, -1);

    \tikzset{VertexStyle/.append style = {minimum size = 20pt,inner sep=0pt, shape=rectangle, fill=blue}}
  \end{tikzpicture}
  \caption{Illustration of a deadlock.
    The squares and the thick back-and-forth arcs represent the switches and the channels of the network, respectively.
    Each of the four colors illustrates the route of a packet.
    The gray channels remain unused.
    Each packet waits to be forwarded along the dashed arrow of its color to a channel blocked by another packet.}\label{fig:deadlock}
\end{figure}
A well-established method for avoiding deadlocks is \emph{up-down routing}~\cite{alfaro2000performance, sancho2000new, schroeder1991autonet, wittorff2009implementation}, which begins by selecting an acyclic orientation of the network graph.
Forwarding is then restricted: subpaths in which two consecutive edges are both directed into the same vertex are disallowed.

This restriction ensures that the resulting \emph{channel dependency graph} is acyclic, which in turn guarantees deadlock-free routing~\cite{dally1987deadlock}.
In this dependency graph, vertices represent channels, and there is a directed arc from channel $c_i$ to $c_j$ if a packet can move from $c_i$ to $c_j$.
A cycle in the channel dependency graph corresponds to a closed walk $W$ in the original network where all consecutive edges are traversable, that is, no two are oriented into the same vertex.
This, in turn, implies that $W$ is oriented cyclically in the chosen orientation, contradicting the assumption of acyclicity.
Therefore, no such $W$ exists, and the routing is deadlock-free.

The number of forbidden subpaths at a vertex depends solely on its indegree.
Thus, choosing an appropriate acyclic orientation is critical for minimizing the number of routing constraints.
For example, a \emph{dec-min} acyclic orientation minimizes the maximum number of forbidden subpaths at any vertex, while a $\min \sum_{v\in V} \binom{\varrho(v)}{2}$ orientation minimizes the total number of such subpaths.
Later, we will see that the optimal solutions to this latter problem coincide with those that minimize $\sum_{v\in V} \varrho(v)^2$.
Note that in real-world implementations, additional constraints --- such as rooted connectivity --- are often imposed to ensure that there exist feasible routes between certain pairs of switches~\cite{alfaro2000performance}.


\paragraph{Related work}
The unconstrained (possibly cyclic) counterparts of Problems~\ref{prob:sumPhiRhoV}-\ref{prob:incMax}, known as egalitarian orientation problems, along with graph orientations in general, have been extensively studied in the literature~\cite{borradaile2017egalitarian,frank1980orientation,frank2022decreasing2,frank2022decreasing1}.
A fundamental fairness objective is to minimize the maximum indegree in an (acyclic) orientation.
Graphs that admit an acyclic orientation with maximum indegree at most $k$ are called \emph{$k$\nobreakdash-degenerate}, a notion first introduced by Lick and White~\cite{lick1970k}.
The smallest such $k$ --- the \emph{degeneracy} of the graph --- is denoted by $\cev{d}_{\min}$.
A simple greedy algorithm can construct in linear time an acyclic orientation in which the indegree of every vertex is at most the degeneracy number~\cite{matula1983smallest}.

The problem of minimizing the maximum indegree has also been studied for orientations without the acyclicity constraint.
In this case, a simple augmenting-path approach yields a polynomial-time algorithm~\cite{asahiro2007graph,de1995regular,venkateswaran2004minimizing}.
However, when a weight function $w : E \to \R_+$ is introduced and the objective is to minimize the maximum \emph{weighted} indegree, the problem becomes NP-hard --- even when each weight is either $1$ or $k$ for some fixed $k \geq 2$~\cite{asahiro2011approximation, asahiro2007graph}.

A stricter notion of fairness is captured by the dec-min orientation and dec-min acyclic orientation problems introduced by Borradaile et al.~\cite{borradaile2017egalitarian}.
These problems seek orientations where the indegree vector sorted in non-increasing order is lexicographically minimal.
Notably, any dec-min acyclic orientation guarantees that the maximum indegree is at most the degeneracy number $\cev{d}_{\min}$.
The dec-min acyclic orientation problem is NP-hard, whereas the unconstrained (possibly cyclic) variant remains solvable in polynomial time via a dipath-reversing method~\cite{borradaile2017egalitarian}.

Frank and Murota~\cite{frank2022decreasing2,frank2022decreasing1} further investigated the dec-min orientation problem without the acyclicity constraint.
They proved that the optimal solutions to the dec-min and inc-max orientation problems coincide.
Moreover, these solutions also minimize the sum $\sum_{v \in V} \varphi(\varrho(v))$ for any discrete strictly convex function $\varphi: \Z_+ \to \R$.
Their framework leverages the theory of M-convex sets, providing a strongly polynomial-time algorithm for computing a dec-min element in such a set, particularly one arising from an integral base polyhedron --- extending well beyond graph orientations.


Special cases of Problem~\ref{prob:sumPhi_VRhoV} were already studied in the literature.
If $\varphi_v$ is the $(f,g)$-lifting of the all-zero function for each $v \in V$, then we obtain a straightforward generalization of the indegree-bounded orientation problem~\cite{hakimi1965degrees}.
The indegree-bounded acyclic orientation problem is known to be NP-hard~\cite{kiraly2018acyclic}, thus, the acyclic version of Problem~\ref{prob:sumPhi_VRhoV} is NP-hard even if $\varphi_v$ is the $(f,g)$-lifting of the all-zero function for each $v\in V$.
If $\varphi_u \equiv \varphi_v$ for every $u, v \in V$, then we obtain the unconstrained (possibly cyclic) version of Problem~\ref{prob:sumPhiRhoV}.

Another line of work on equitable acyclic orientations was proposed by Biedl et al.~\cite{biedl2005balanced}, who introduced the concept of balanced acyclic orientations, where the goal is to minimize the \emph{total imbalance}, defined as the sum of absolute differences between indegree and outdegree over all vertices --- corresponding to Problem~\ref{prob:sumPhi_VRhoV} for $\varphi_v(z)=|2z-d(v)|$.
They showed that computing a balanced acyclic orientation is NP-hard even for graphs with maximum degree at most $6$.
Later work refined this hardness result, showing NP-completeness for graphs with maximum degree $4$ and even for $5$\nobreakdash-regular graphs~\cite{kara2005complexity}.
In contrast, the problem is solvable in polynomial time for graphs with maximum degree at most $3$~\cite{biedl2005balanced}.
Note that the unconstrained (possibly cyclic) orientations minimizing the total imbalance are the Eulerian or almost-Eulerian orientations, and such an orientation can be computed in polynomial time.


\paragraph{Our contribution}
In Section~\ref{sec:convexHullOfIndegrees}, we prove that the corners of the indegree polytope in $\R^V$ are exactly the indegree vectors of acyclic orientations.
This provides a direct equivalence between acyclic orientation problems and the search for optimal corners of this polytope.
Moreover, we define a polyhedron by a polynomial number of inequalities whose projection yields the indegree polytope, which enables a unified treatment of egalitarian and certain non-linear orientation problems.
In particular, we give a simple polynomial-time algorithm for Problem~\ref{prob:sumPhi_VRhoV}.
As special cases, this solves Problems~\ref{prob:sumPhiRhoV}-\ref{prob:rhoTimesDelta} without the acyclicity requirement, including other natural problems, such as $\max \prod_{v \in V} \varrho(v)$ and $\min \sum_{v \in V} \left([\alpha(v) - \varrho(v)]^+ + [\varrho(v) - \beta(v)]^+\right)$ for $\alpha,\beta : V \to \R$.

In Section~\ref{sec:minSumH}, we examine Problem~\ref{prob:sumPhiRhoV}.
Our main result is that this problem is NP-hard for loop-free multigraphs for \emph{every} discrete strictly convex function $\varphi$, although the optimal acyclic orientations themselves do depend on the choice of $\varphi$, see Section~\ref{sec:decMinIncMax}.
This immediately implies that Problems~\ref{prob:sumPhiRhoV}-\ref{prob:incMax} and $\max \prod_{v \in V} \varrho(v)$ are all NP-hard.
In fact, this general hardness result holds for every discrete (not necessarily strictly) convex function $\varphi$ that has a breakpoint at a specified value.
This settles the hardness for minimizing $ \sum_{v \in V} \left([\alpha(v) - \varrho(v)]^+ + [\varrho(v) - \beta(v)]^+\right)$ over acyclic orientations.
Furthermore, we also obtain as a corollary that minimizing $\sum_{v\in V} [\varrho(v) - g(v)]^+$ over acyclic orientations is NP-hard, while one can decide in polynomial time whether the optimal objective value is zero, i.e., whether an acyclic orientation exists with $\varrho(v)\leq g(v)$ for each $v\in V$~\cite{kiraly2018acyclic}.
Complementing our main hardness result, even Problem~\ref{prob:sumPhi_VRhoV} turns out to be solvable in polynomial time under the acyclicity constraint provided that the functions $\varphi_v$ are linear.


In Section~\ref{sec:minSquareSum}, we explore the special case of minimizing the square-sum of indegrees over acyclic orientations.
Here, we prove NP-hardness even for simple graphs and analyze the approximation ratio of a natural greedy algorithm.

Section~\ref{sec:decMinIncMax} addresses the dec-min and inc-max acyclic orientation problems --- both of which turn out to be special cases of Problem~\ref{prob:sumPhiRhoV} --- with and without an upper bound $k$ on the indegrees.
We show that without indegree bound, the optimal solutions to the two problems may differ, but they coincide when the bound is $k=2$.
It was shown in~\cite{borradaile2017egalitarian} that the $k$-indegree-bounded dec-min acyclic orientation problem is NP-hard for every odd $k \geq 5$ and also in the unbounded case, which we strengthen by showing NP-hardness for all $k \geq 2$.
We further establish that finding an inc-max acyclic orientation is NP-hard for all $k \geq 2$ and also in the unbounded case.
We introduce the complementary inc-min and dec-max problems and prove that they are NP-hard for both acyclic and unconstrained orientations.
Furthermore, we investigate the problem of minimizing the maximum (weighted) indegree in acyclic orientations --- as the first step of finding a dec-min solution.
Minimizing the maximum indegree in the unweighted case can be done in polynomial time~\cite{matula1983smallest}, which we generalize to the weighted case.
In contrast, the corresponding problem without the acyclicity constraint is polynomial-time solvable in the unweighted setting~\cite{asahiro2007graph,de1995regular,venkateswaran2004minimizing} but becomes NP-hard once weights are introduced~\cite{asahiro2011approximation,asahiro2007graph}.

Furthermore, we present an exact exponential-time dynamic programming algorithm for minimizing $\sum_{v\in V}\varphi_v(\varrho(v))$ over acyclic orientations, where $\varphi_v: \Z_+ \to \R$ may be arbitrary.
The algorithm runs in $O(2^{|V|}\poly(|V|,|E|))$ time, provided that these functions can be evaluated in polynomial~time.

Finally, Section~\ref{sec:maxSumLeftRight} addresses the objective $\max\sum_{v\in V}\varrho(v)\delta(v)$ over acyclic orientations.
We leverage results from~\cite{biedl2005balanced, kara2005complexity} to show NP-hardness for simple graphs with maximum degree at most $4$, and provide a polynomial-time algorithm for graphs with maximum degree at most $3$.
In addition, we prove that orienting by a random topological ordering achieves an expected approximation ratio of~$3$ on loop-free multigraphs, and we give a deterministic algorithm that attains the same ratio.


\paragraph{Notation}
Throughout this paper, we work with an undirected graph $G = (V, E)$.
Parallel edges are allowed, while loops are excluded unless explicitly stated otherwise.
For a vertex $v \in V$, its degree is denoted by $d_G(v)$, or simply $d(v)$ when the graph is clear from context.
If loops are present, then each loop contributes one to the degree of its incident vertex.
The \emph{maximum degree} of $G$ is denoted by $\Delta(G)$.
For a subset $V' \subseteq V$, we write $G[V']$ for the subgraph induced by $V'$, and $d(v, V')$ for the degree of $v$ within this subgraph.
We denote a directed graph~(\emph{digraph} for short) by $D = (V, A)$.
The \emph{indegree} and \emph{outdegree} of a vertex $v$ in $D$ are denoted by $\varrho_D(v)$ (or simply $\varrho(v)$) and $\delta_D(v)$ (or simply $\delta(v)$), respectively.
An \emph{ordering} of the vertices is written as $\sigma = (\sigma_1, \dots, \sigma_n)$, where $\sigma_i \in V$ is the vertex in position $i$.
The set of all vertex orderings is denoted by $\mathcal{S}_V$.
For a vertex $v = \sigma_i$ in an ordering $\sigma$, the \emph{left-degree} is $\cev{d}_{\sigma}(v) = d(v, \{\sigma_1, \dots, \sigma_i\})$, and the \emph{right-degree} is $\vec{d}_{\sigma}(v) = d(v, \{\sigma_{i+1}, \dots, \sigma_n\})$.
We often omit~$\sigma$ when the ordering is clear and write $\cev{d}(v)$ and $\vec{d}(v)$.
When a weight function $w : E \to \R_+$ is given, we extend these notations with a superscript $w$ for the weighted degrees, for example, $d^w(v) = \sum_{e \in E : v \in e} w(e)$ denotes the \emph{weighted degree} of $v$.
An ordering $\sigma$ is called \emph{$k$\nobreakdash-bounded} if $\cev{d}(v) \leq k$ for all $v \in V$.
The \emph{degeneracy} of $G$, denoted by $\cev{d}_{\min}(G)$ or simply $\cev{d}_{\min}$, is the smallest $k$ for which a $k$\nobreakdash-bounded ordering exists.
A discrete function $\varphi : \Z_+\to \R$ is \emph{convex} if $\varphi(z+1) + \varphi(z-1) \geq 2 \varphi(z)$ holds for every positive integer $z$.
We say that $z$ is a breakpoint of such a function $\varphi$ if the previous condition holds with strict inequality for $z$.
A discrete \emph{strictly convex} function is a discrete convex function that has a breakpoint at every positive integer $z$.
We use the notation $[x]^+ = \max\{x, 0\}$ for $x \in \R$, and we let $\chi_A$ denote the \emph{indicator function} of $A$, which equals $1$ if the statement $A$ holds and $0$ otherwise.


\section{The indegree polytope in $\R^V$}\label{sec:convexHullOfIndegrees}
In this section, we examine the convex hull of the indegree vectors in $\R^V$ corresponding to all orientations of a given graph~$G$.
We denote this polytope by $P_{\varrho}(G)$.
It is straightforward to show that the integer points of $P_{\varrho}(G)$ are precisely the indegree vectors of orientations of~$G$.
Consequently, minimizing $\sum_{v \in V} \varphi(\varrho(v))$ for a discrete strictly convex function~$\varphi$, finding a dec-min orientation, and finding an inc-max orientation is equivalent to optimizing the corresponding objective over the integer points of~$P_{\varrho}(G)$.

Frank and Murota~\cite{frank2022decreasing2,frank2022decreasing1} described $P_{\varrho}(G)$ as a base polyhedron.
Later in this section, we show that $P_{\varrho}(G)$ can also be obtained as the projection of a circulation polytope.
Furthermore, we present a simple algorithm for minimizing $\sum_{v\in V}\varphi_v(\varrho(v))$ over the integer points of $P_{\varrho}(G)$.

First, we turn to the study of the vertices of $P_{\varrho}(G)$.
A key insight, suggested by Andr\'as Frank, characterizes the corners of $P_{\varrho}(G)$, as formalized in the following theorem.
\begin{theorem}\label{thm:VerticesOfBasePolyAreAcyclicIndigrees}
  For every graph $G = (V, E)$, the corners of $P_{\varrho}(G)$ are precisely the indegree vectors of the acyclic orientations of $G$.
\end{theorem}
\begin{proof}
  We first show that the indegree vector of any non-acyclic orientation can be written as a convex combination of other indegree vectors.
  Let $D$ be a non-acyclic orientation of $G$ with indegree vector $x = (\varrho(v_1), \dots, \varrho(v_n))$, where $v_1, \dots, v_n$ are the vertices of $G$.
  Choose a directed cycle $C$ in $D$.
  Without loss of generality, assume that $C$ consists of the arcs $v_1 v_2, \dots, v_{k-1} v_k, v_k v_1$.
  Construct $k$ orientations $D^1, \dots, D^k$, where $D^j$ is obtained from $D$ by reversing the $j^{\text{th}}$ arc of $C$.
  The indegree vector of $D^j$ is $x^j=(\varrho(v_1),\dots,  \varrho(v_{j-1}), \varrho(v_j)+1, \varrho(v_{j+1})-1,\varrho(v_{j+2}),\dots, \varrho(v_n))$ for $j\in\{1,\dots,k-1\}$, and the indegree vector of $D^k$ is $x^k=(\varrho(v_1)-1,\varrho(v_2),\dots,  \varrho(v_{k-1}), \varrho(v_k)+1, \varrho(v_{k+1}),\dots, \varrho(v_n))$.
  Then $x = \frac{1}{k} \sum_{j=1}^k x^j$, and thus $x$ is not a vertex of $P_{\varrho}(G)$.

  Second, we show that the indegree vectors of acyclic orientations are vertices of $P_{\varrho}(G)$.
  We proceed by induction on $|V|$.
  For the base case $|V|=1$, the claim is trivial.
  Suppose the claim holds for graphs with $(|V| - 1)$ vertices.
  Let $D$ be an acyclic orientation of $G$, and denote its indegree vector by $x$.
  Suppose there exist orientations $D^1, \dots, D^k$ with indegree vectors $x^1, \dots, x^k$, and coefficients $\lambda_1, \dots, \lambda_k > 0$ with $\sum_{j=1}^k \lambda_j = 1$ such that $x = \sum_{j=1}^{k} \lambda_j x^j$.
  We show that $x^j=x$ for $j\in \{1,\dots,k\}$.
  Since $D$ is acyclic, some vertex has indegree $0$; without loss of generality, let this vertex be $v_n$.
  Let $\widetilde G = G \setminus \{v_n\}$, and let $\widetilde D, \widetilde D^1, \dots, \widetilde D^k$ denote the orientations $D,D^1, \dots, D^k$ restricted to the vertex set of $\widetilde G$, and let $\tilde x, \tilde x^1, \dots, \tilde x^k$ denote their indegree vectors, respectively.
  Then $\tilde x = \sum_{j=1}^k \lambda_j \tilde x^j$.
  By the inductive hypothesis, $\tilde x^j = \tilde x$ for every $j$.
  Since $v_n$ has indegree $0$ in $D$, the indegree of $v_n$ must be $0$ in $D^j$ for every $j$, implying $x^j = x$ for every $j$.
  Thus, $x$ is a corner of $P_{\varrho}(G)$.
\end{proof}

This characterization further motivates the study of egalitarian \emph{acyclic} orientations, as finding an optimal acyclic orientation for Problems~\ref{prob:sumPhi_VRhoV}-\ref{prob:rhoTimesDelta} is equivalent to finding a corner of $P_{\varrho}(G)$ optimizing the corresponding objective.
For any linear objective function, it follows that there always exists a minimizer among the indegree vectors of acyclic orientations.
Later, in Theorem~\ref{thm:minLinearFunctionAcyclic}, we provide a simple greedy algorithm for finding an acyclic orientation whose indegree vector minimizes a linear objective.

Next, we describe a polyhedron whose projection yields the polytope $P_{\varrho}(G)$.
Consider the following construction: let $G' = (V, E; E')$, where $E'$ contains an edge between $e \in E$ and $v \in V$ if $v$ is an endpoint of $e$ in $G$.
Let $A_{G'}$ denote the incidence matrix of this bipartite graph $G'$.
Extend the rows corresponding to $V$ with a $|V| \times |V|$ negated identity matrix, and the rows corresponding to $E$ with a $|V| \times |E|$ zero matrix.
Let $A$ denote the resulting matrix, which is clearly totally unimodular.
Let $y \in \R_+^{E'}$ be the variables corresponding to columns of $A_{G'}$, and $x \in \R_+^V$ the variables corresponding to the additional columns.
Define $b = (0^V, 1^E)$.
We prove the following.

\begin{theorem}\label{thm:TUMatrixForIndegreeVectors}
  Consider the polytope $P = \{ (y,x) \in \R_+^{E'} \times \R_+^V : A(y,x) = b \}$.
  Then the projection
  $P|_x = \{ x \in \R_+^V : (y,x) \in P \text{ for some } y \in \R_+^{E'} \}$
  is the polytope $P_{\varrho}(G)$ of indegree vectors of orientations of $G = (V, E)$.
\end{theorem}
\begin{proof}
  We first show that $P_{\varrho}(G)\subseteq P|_x$.
  Fix an orientation of $G$ and define a vector $(y,x)\in \R_+^{E'}\times\R_+^V$ as follows:
  for each edge $e = uv \in E$, set $y_{eu} = 1$ if $e$ is oriented towards $u$, and otherwise $0$; and for each vertex $v \in V$, set $x_v = \varrho(v)$.
  By definition, $(y,x)$ satisfies all defining constraints of $P$; hence $(y,x) \in P$, and its projection $x$ lies in $P|_x$.
  Because this holds for every orientation of $G$, $P|_x$ contains all indegree vectors of orientations of $G$, proving $P_{\varrho}(G) \subseteq P|_x$.

  Since $P$ and $P|_x$ are bounded polyhedra, they are the convex hulls of their corners.
  We show that every corner $(y,x)$ of $P$ corresponds to an orientation of $G$ with indegree vector $x$, and that every corner $x$ of $P|_x$ arises from such a solution $(y,x)$.

  As $A$ is totally unimodular and $b$ is integral, the corners of $P$ are integral.
  Consider such an integral vector $(y,x) \in P$.
  For each edge $e = uv \in E$, the constraint $y_{eu} + y_{ev} = 1$ implies that exactly one of $y_{eu}, y_{ev}$ is $1$ and the other $0$.
  Orient $e$ towards $u$ if $y_{eu} = 1$, and towards $v$ otherwise.
  Then the indegree of $v$ is $x_v$, as required.

  Now let $x$ be a corner of $P|_x$.
  There exists a $y \in \R_+^{E'}$ so that $(y,x)$ is in $P$.
  Then $(y,x)$ can be written as a convex combination of distinct corners $(y_1, x_1), \dots, (y_k, x_k) \in P$.
  Projecting, this would give $x = \sum_{j=1}^k \lambda_j x_j$, contradicting the extremality of $x$ unless $x_j = x$ for all $j$.
  In that case, $(y_j, x)$ is a desired corner of $P$.
  Therefore, $P|_x$ is the indegree polytope of orientations of $G$, as claimed.
\end{proof}

We show how this simple polyhedral description can be leveraged to solve the unconstrained (possibly cyclic) versions of our problems.
Meyer~\cite{meyer1977class} presented a method for solving non-linear integer programs of the form $\min \sum_{i=1}^n \varphi_i(x_i) \text{ s.t.\ } Ax = b,\ x \in \Z_+^n$, where each $\varphi_i : \Z \to \R$ is a discrete convex function, $A \in \Z^{m \times n}$ is a totally unimodular matrix, and $b \in \Z^m$.
By applying this to our polyhedral description from Theorem~\ref{thm:TUMatrixForIndegreeVectors}, Meyer’s method solves the following orientation problem: given a discrete convex function $\varphi_v: \Z_+ \to \R$ for each vertex $v \in V$, find an unconstrained (possibly cyclic) orientation that minimizes the sum $\sum_{v \in V} \varphi_v(\varrho(v))$.
In contrast, we later prove that the \emph{acyclic} version of our original problem --- minimizing $\sum_{v \in V} \varphi(\varrho(v))$ over acyclic orientations --- is NP-hard; see Corollary~\ref{cor:minSumHLooplessNPC}.
This implies that, for the class of non-linear integer programs considered in~\cite{meyer1977class}, restricting the solution space to corner solutions makes the problem intractable in general.

\medskip
In the rest of this section, we present an alternative, strongly polynomial-time algorithm for solving the problem of finding an orientation of a graph $G$ that minimizes $\sum_{v \in V} \varphi_v(\varrho(v))$, where each $\varphi_v$ is a given discrete (not necessarily strictly) convex function.
Our approach gives a reduction to the minimum-cost integral flow problem, for which a strongly polynomial-time algorithm is known~\cite{tardos1985strongly}.
Following an idea similar to Meyer’s, we linearize the objective by expressing each discrete convex function as a sum of its increments and leveraging that they are non-decreasing.
What distinguishes our approach is the direct exploitation of the connection between orientations and flows, yielding a formulation that is not only more efficient, but also conceptually elegant.
\begin{theorem}\label{thm:nonAcyclicWithFlow}
  Let $G = (V, E)$ be a graph, and let $\varphi_v: \Z_+ \to \R$ be a discrete (not necessarily strictly) convex function for each $v \in V$.
  An orientation that minimizes $\sum_{v \in V} \varphi_v(\varrho(v))$ can be computed in strongly polynomial time, provided that $\varphi_v$ can be evaluated in $O(\poly(|V|, |E|))$ time for each $v \in V$.
\end{theorem}
\begin{proof}
  We reduce the problem to a minimum-cost integral flow problem.
  Let $G = (V, E)$ denote the graph for which we want to find an optimal orientation, and let $V = \{v_1, \dots, v_n\}$ and $E = \{e_1, \dots, e_m\}$.
  If $\varphi_v(0) \neq 0$, then define $\varphi_v'(z) = \varphi_v(z) - \varphi_v(0)$ for $z \in \Z_+$.
  Since
  \[
    \sum_{v \in V} \varphi_v(\varrho(v)) = \sum_{v \in V} \varphi_v'(\varrho(v)) + \sum_{v \in V} \varphi_v(0),
  \]
  the set of optimal orientations remains unchanged.
  Thus, we may assume without loss of generality that $\varphi_v(0) = 0$ for all $v \in V$.

  Construct a network $D = (V', A)$ as follows.
  The vertex set $V'$ consists of a source $s$, a sink $t$, a vertex $v'_i$ for each $v_i \in V$, and a vertex $e'_j$ for each $e_j \in E$.
  For each $v_i \in V$, add $d_G(v_i)$ parallel arcs from $s$ to $v'_i$.
  Add an arc from $v'_i$ to $e'_j$ if $e_j$ is incident to $v_i$ in $G$.
  Finally, add an arc from each $e'_j$ to $t$.
  Figure~\ref{fig:integralFlow} illustrates this construction.
  \begin{figure}[t]
    \centering
    \begin{tikzpicture}[xscale=1.1,yscale=1.45]
      \tikzset{VertexStyle/.append style = {minimum size = 20pt,inner sep=0pt}}
      \SetVertexMath
      \Vertex[x=-3, y=0, L=s]{s}
      \Vertex[x=0, y=1.5, L=v'_1]{v1}
      \Vertex[x=0, y=0, L=v'_i]{vi}
      \Vertex[x=0, y=-1.5, L=v'_n]{vn}

      \Vertex[x=3, y=1.5, L=e'_1]{e1}
      \Vertex[x=3, y=0, L=e'_j]{ej}
      \Vertex[x=3, y=-1.5, L=e'_m]{em}
      \Vertex[x=6, y=0, L=t]{t}

      \node(d1) at (0,0.85) {$\vdots$};
      \node(d1) at (0,-0.7) {$\vdots$};
      \node(d1) at (3,0.85) {$\vdots$};
      \node(d1) at (3,-0.7) {$\vdots$};

      \draw[\niceArrow] (s) to [out=33,in=203,looseness=1] (v1);
      \draw[\niceArrow] (s) to [out=23,in=213,looseness=1] (v1);

      \draw[\niceArrow] (s) to [out=5,in=175,looseness=1] (vi);
      \draw[\niceArrow] (s) to [out=-5,in=-175,looseness=1] (vi);
      \path[circleAroundEdges={.08}{.20}{.5}] (s)--(v1) node [above=3mm, pos=.45] {$d(v_1)$};
      \path[circleAroundEdges={.08}{.20}{.5}] (s)--(vi) node [above=1.5mm, pos=.55] {$d(v_i)$};
      \path[circleAroundEdges={.08}{.20}{.5}] (s)--(vn) node [below=3mm, pos=.45] {$d(v_n)$};

      \draw[\niceArrow] (s) to [out=-33,in=-203,looseness=1] (vn);
      \draw[\niceArrow] (s) to [out=-23,in=-213,looseness=1] (vn);

      \draw[\niceArrow] (vi)--(ej);
      \node(if1) at (1.5,-0.3) {if $v_i$ and $e_j$};
      \node(if2) at (1.5,-0.6) {are incident};

      \draw[\niceArrow] (e1)--(t);
      \draw[\niceArrow] (ej)--(t);
      \draw[\niceArrow] (em)--(t);
      \draw[\niceArrow] (vi)--(ej);
    \end{tikzpicture}
    \caption{The digraph $D = (V', A)$ constructed in the reduction to the minimum cost flow problem.
      The vertices $v'_1,\dots,v'_n$ correspond to the vertices of $G$ and the vertices $e'_1,\dots,e'_m$ correspond to the edges of $G$.}\label{fig:integralFlow}
  \end{figure}
  Let the capacity of each arc be $1$.
  For each $i \in \{1, \dots, n\}$ and $\ell \in \{1, \dots, d(v_i)\}$, set the cost of the arc $a_\ell^i$ to $c(a_\ell^i) = \varphi_{v_i}(\ell) - \varphi_{v_i}(\ell - 1)$.
  Set the cost $c(a)$ of all other arcs to $0$.

  We now prove that the minimum cost of an integral $s$\nobreakdash-$t$ flow of amount $|E|$ is exactly $\sum_{v_i \in V} \varphi_{v_i}(\varrho(v_i))$ for an optimal orientation of $G$.
  First, observe that the total cost of sending $z$ units of flow along the first $z$ parallel arcs from $s$ to $v'_i$, for some $v_i \in V$, is
  \[
    \sum_{\ell=1}^z c(a_\ell^i) = \sum_{\ell=1}^z \big(\varphi_{v_i}(\ell) - \varphi_{v_i}(\ell - 1)\big) = \varphi_{v_i}(z) - \varphi_{v_i}(0) = \varphi_{v_i}(z).
  \]

  On the one hand, consider an optimal orientation, and construct an $s$\nobreakdash-$t$ flow $f : A \to \Z_+$ of amount $|E|$ as follows.
  For each $i \in \{1, \dots, n\}$, set $f(a_\ell^i) = 1$ if $\ell \leq \varrho(v_i)$.
  For each $j \in \{1, \dots, m\}$, set $f(v'_i e'_j) = 1$ if $e_j$ is oriented toward $v_i$ in $G$.
  Set $f(e'_j t) = 1$ for all $j$.
  Set $f(a) = 0$ for all other arcs.
  The flow $f$ is clearly feasible, with cost equal to $\sum_{v_i \in V} \varphi_{v_i}(\varrho(v_i))$, as it uses the first $\varrho(v_i)$ parallel $s v'_i$ arcs for each $v_i$.

  On the other hand, consider a minimum-cost integral $s$\nobreakdash-$t$ flow $f : A \to \Z_+$ of value $|E|$.
  Since $f(e'_j t) = 1$ for all $j$, each $e'_j$ has exactly one incoming arc $v'_i e'_j$ with $f(v'_i e'_j) = 1$.
  We orient $e_j$ toward $v_i$ if $f(v'_i e'_j) = 1$.
  The indegree $\varrho(v_i)$ is thus the number of parallel $sv'_i$ arcs used by $f$.
  Since $\varphi_{v_i}$ is convex, the costs satisfy
  \[
    c(a_\ell^i) = \varphi_{v_i}(\ell) - \varphi_{v_i}(\ell - 1) \leq \varphi_{v_i}(\ell') - \varphi_{v_i}(\ell' - 1) = c(a_{\ell'}^i)
  \]
  whenever $\ell < \ell'$.
  Therefore, $f$ can be assumed to use the first $\varrho(v_i)$ parallel arcs from $s$ to $v'_i$ for each $i$.
  Thus, the cost of the flow is $\sum_{v_i \in V} \varphi_{v_i}(\varrho(v_i))$, matching the objective value of the corresponding orientation.

  This shows that an orientation minimizing $\sum_{v_i \in V} \varphi_{v_i}(\varrho(v_i))$ can be found in polynomial time by solving the minimum-cost integral $s$\nobreakdash-$t$ flow problem in the network above.
  Since the latter is solvable in strongly polynomial-time~\cite{tardos1985strongly}, the theorem follows.
\end{proof}

Theorem~\ref{thm:nonAcyclicWithFlow} clearly solves Problems~\ref{prob:sumPhi_VRhoV}-\ref{prob:rhoTimesDelta} without the acyclicity requirement, as well as $\max \prod_{v \in V} \varrho(v)$ and $\min \sum_{v \in V} \left([\alpha(v) - \varrho(v)]^+ + [\varrho(v) - \beta(v)]^+\right)$ for every $\alpha,\beta : V \to \R$.
As already noted in the introduction, the self-generalizing nature of Problem~\ref{prob:sumPhi_VRhoV} immediately extends Theorem~\ref{thm:nonAcyclicWithFlow} to mixed graphs with lower and upper bounds for the indegrees of the vertices.

\section{Minimizing \texorpdfstring{$\sum_{v\in V} \varphi(\varrho(v))$ over acyclic orientations}{Sum phi(rho(v))}}\label{sec:minSumH}
This section studies the problem of minimizing the objective $\sum_{v\in V} \varphi(\varrho(v))$ for some discrete (strictly) convex function $\varphi: \Z_+ \to \R$ over the vertices of the indegree polytope.
By Theorem~\ref{thm:VerticesOfBasePolyAreAcyclicIndigrees}, this problem is equivalent to finding an \emph{acyclic} orientation of a graph $G = (V, E)$ that minimizes the same objective.

A closely related problem --- where acyclicity is not required --- was previously studied by Frank and Murota~\cite{frank2022decreasing2, frank2022decreasing1}.
They showed that in the unconstrained setting, the optimal orientations do not depend on the particular choice of the strictly convex function $\varphi$; and also that the same orientations are simultaneously optimal for both the decreasing-minimum (dec-min) and increasing-maximum (inc-max) orientation problems.
Furthermore, such orientations can be found in strongly polynomial time~\cite{borradaile2017egalitarian,frank2022decreasing2, frank2022decreasing1}, see also Theorem~\ref{thm:nonAcyclicWithFlow}.

Once acyclicity is imposed, however, this universality disappears: the optimal orientations generally do depend on the specific function $\varphi$, see Section~\ref{sec:decMinIncMax}.
Our first main result establishes hardness in this setting.
We show that for \emph{every} discrete strictly convex function $\varphi$, the problem becomes NP-hard when parallel edges are allowed --- in fact, this general hardness result extends to any discrete convex function that has a breakpoint at a specific value depending on the graph.
We further analyze three special cases: $\varphi(z) = z^2$, which corresponds to finding an acyclic orientation that minimizes the square-sum of indegrees; and the dec-min and inc-max problems.
Then, we propose an exact dynamic programming algorithm for solving a generalization of our original problem.

\begin{remark}
  In Section~\ref{sec:decMinIncMax}, we will see that the dec-min and inc-max acyclic orientation problems can be expressed as special cases of $\min \sum_{v\in V} \varphi(\varrho(v))$, corresponding to the choices $\varphi(z) = |V|^z$ and $\varphi(z) = |V|^{-z}$, respectively.
  We will show an example where the sets of optimal acyclic orientations for the dec-min and inc-max problems are disjoint --- in sharp contrast with the case of unconstrained (possibly cyclic) orientations.
  Beyond the corresponding exponential functions, we also observed divergence for polynomial objectives.
  For instance, the optimal acyclic orientations for $\varphi(z) = z^2$ and for $\varphi(z) = z^3$ do not coincide in general, and we conjecture that for every integer exponent $c \geq 2$, the problems for $\varphi(z) = z^c$ admit distinct sets of optimal solutions.
  Surprisingly, the cubic objective does not simply refine the quadratic one: there exist graphs where an optimal solution for $\varphi(z) = z^3$ fails to be optimal for $\varphi(z) = z^2$.
  This underlines the richness of the family of the problems defined by $\min \sum_{v\in V} \varphi(\varrho(v))$ under acyclicity, and highlights a clear distinction from the universality in the unconstrained (possibly cyclic) setting.~$\bullet$
\end{remark}
In the next section, we prove that minimizing $\sum_{v\in V} \varphi(\varrho(v))$ over acyclic orientations is NP-hard for \emph{every} discrete strictly convex function $\varphi$.
From now on, we mostly adopt the vertex ordering perspective, that is, we seek a topological ordering of the desired orientation, as discussed in Section~\ref{sec:introduction}.


\subsection{Hardness results}\label{sec:minSumHComplexity}\label{sec:minSumHNPH}
The goal of this section is to investigate the computational complexity of finding a vertex order of a graph $G = (V, E)$ that minimizes $\sum_{v \in V} \varphi(\cev{d}(v))$ for various discrete convex functions $\varphi : \Z_+ \to \R$.
Our main result is the following theorem.

\begin{theorem}\label{thm:minSumHLooplessNPCGeneral}
  Let $G = (V, E)$ be a graph and let $q=\left\lceil \frac{|E|}{|V|}\right\rceil$.
  For every discrete convex function $\varphi$ with a breakpoint at $q$, the problem of finding a vertex order of $G$ that minimizes $\sum_{v\in V}\varphi(\cev d(v))$ is NP-hard.
\end{theorem}
\begin{proof}
  We proceed in two steps. First, we establish the analogous hardness result for graphs that may contain loops, then we eliminate the loops.

  \medskip
  \noindent\textbf{Step 1:} hardness for graphs with loops.
  When loops are present, we slightly adapt the definitions of $\cev{d}(v)$ and $d(v)$: each loop incident to a vertex $v$ contributes exactly one both to its left-degree $\cev{d}(v)$ and to its degree $d(v)$.
  Note that, although a loop constitutes a cycle of length one --- thus no acyclic orientation exists --- the vertex-ordering formulation of the problem remains meaningful.

  The proof is by reduction from the set cover problem, which is NP-complete~\cite{karp1972reducibility}.
  Consider an instance of set cover with ground set $S = \{s_1, \dots, s_n\}$ and a collection of subsets $\mathcal{H} = \{e_1, \dots, e_m\} \subseteq 2^S$.
  The task is to decide whether there exist $k$ subsets $e_{j_1}, \dots, e_{j_k} \in \mathcal{H}$ such that $\bigcup_{\ell=1}^{k} e_{j_{\ell}} = S$.
  Without loss of generality, we assume that each element $s_i \in S$ appears in at least two subsets, and that $n > k$ and $m > k$.

  We now construct a graph $G' = (V', E')$ corresponding to this instance.
  For each subset $e_j \in \mathcal{H}$, introduce a vertex $e_j$.
  For each element $s_i \in S$, introduce a vertex $s_i$.
  For every incidence $s_i \in e_j$, introduce two auxiliary vertices $a_{s_i}^{e_j}$ and $b_{s_i}^{e_j}$.
  Connect $a_{s_i}^{e_j}$, $b_{s_i}^{e_j}$, and $e_j$ into a triangle, and add an edge between $b_{s_i}^{e_j}$ and $s_i$.
  Introduce a special vertex $s$, and connect it to all vertices $e_j$ and all vertices $s_i$.
  Finally, for technical reasons, we add loops so that $d'(e_j)=q+1$, $d'(s_i)=q+1$, $d'(a_{s_i}^{e_j})=q+1$, $d'(b_{s_i}^{e_j})=q+2$, and $d'(s)=q+n+k$ hold, where $d'$ is the degree function of $G'$ and $q = 2 \sum_{e_j\in \mathcal{H}} |e_j|+1$.
  Figure~\ref{fig:minSumHLoopsNPC} illustrates the construction (with loops omitted for clarity).

  In Claim~\ref{cl:multisetSecondProperty}, we verify that the particular choice of $q$ above satisfies the requirement $q = \left\lceil \frac{|E'|}{|V'|} \right\rceil$ from the statement of the theorem.
  Since $q$ is polynomial in the size of the set cover instance, so is the size of the constructed graph $G'$.

  \begin{figure}[t]
    \centering
    \begin{tikzpicture}[yscale=1]
      \tikzset{VertexStyle/.append style = {minimum size = 20pt,inner sep=0pt}}
      \SetVertexMath
      \Vertex[x=0, y=3,L=e_1]{e1}
      \draw (1.5,3) node () {$\dots$};
      \Vertex[x=3, y=3, L=e_j]{ej}
      \draw (4.5,3) node () {$\dots$};
      \Vertex[x=6, y=3, L=e_m]{em}

      \Vertex[x=2, y=2.25, L=a_{s_i}^{e_j}]{aij}
      \Vertex[x=3, y=1.5, L=b_{s_i}^{e_j}]{bij}
      \draw (4.25,1.5) node () {if $s_i\in e_j$};

      \Vertex[x=0, y=0,L=s_1]{s1}
      \draw (1.5,0) node () {$\dots$};
      \Vertex[x=3, y=0, L=s_i]{si}
      \draw (4.5,0) node () {$\dots$};
      \Vertex[x=6, y=0, L=s_n]{sn}

      \Vertex[x=-3,y=1.5, L=s]{s}

      \draw (ej)--(aij);
      \draw (ej)--(bij);
      \draw (aij)--(bij);
      \draw (bij)--(si);
      \draw (s) to [out=50,in=180,looseness=0.75] (e1);

      \draw (s) to [out=70,in=140,looseness=0.75] (ej);

      \draw (s) to [out=90,in=130,looseness=0.75] (em);

      \draw (s) to [out=-50,in=-180,looseness=0.75] (s1);

      \draw (s) to [out=-70,in=-140,looseness=0.75] (si);

      \draw (s) to [out=-90,in=-130,looseness=0.75] (sn);
    \end{tikzpicture}
    \caption{Illustration of the graph $G'$ constructed in the first step of the proof of Theorem~\ref{thm:minSumHLooplessNPCGeneral}.}\label{fig:minSumHLoopsNPC}
  \end{figure}

  Before that, we state a key lemma on degree distributions, which requires introducing some notation.
  Define $\mathcal{L}_{n',m',q',k'}$ as the set of multisets $L$ of non-negative integers such that
  \begin{enumerate}[label=\textbf{(\arabic*)}]
  \item $|L| = n'$,\label{item:L1}
  \item $\sum_{\ell \in L} \ell = m'$, where $m' = n'(q'-1) + r$ for some residual $r \in \{2k',\dots,n'-1\}$, and\label{item:L2}
  \item $\gamma_L \geq k'$, where $\gamma_L = \sum_{\ell \in L} [\ell - q']^+$.\label{item:L3}
  \end{enumerate}

  Let $L^* \in \mathcal{L}_{n',m',q',k'}$ be the multiset consisting of $k'$ copies of $(q' + 1)$, $(r - 2k')$ copies of $q'$, and $(n'+k'-r)$ copies of $(q'-1)$, where $r$ is the residual as in~\ref{item:L2}.
  Note that later we show that there exists a set cover of size $k$ if and only if there exists a vertex ordering of $G'$ with $\sum_{v\in V'}\varphi(\cev{d'}(v)) \;\leq\; \sum_{\ell\in L^*}\varphi(\ell)$, where $\cev{d'}(v)$ denotes the left-degree of $v$ in $G'$.

  The following claim establishes that $L^*$ minimizes $\sum_{\ell\in L}\varphi(\ell)$ over all $L \in \mathcal{L}_{n',m',q',k'}$ for any discrete convex function $\varphi$ with a breakpoint at $q'$.
  Moreover, it identifies a structural property that every minimizer must satisfy.
  \begin{claim}\label{cl:uniqueMinimizer}
    Let $\varphi: \Z_+ \to \R$ be a discrete convex function with a breakpoint at $q'$.
    Then for any $L'\in \mathcal{L}_{n',m',q',k'}$ minimizing $\sum_{\ell\in L}\varphi(\ell)$, we have $\gamma_{L'}=k'$.
    Moreover, $L^*$ is a minimizer of $\sum_{\ell\in L}\varphi(\ell)$ over $\mathcal{L}_{n',m',q',k'}$.
  \end{claim}
  \begin{subproof}
    We begin with the first part of the claim.
    Let $L \in \mathcal{L}_{n',m',q',k'}$ be a multiset with $\gamma_{L}>k'$.
    Then $L$ must contain an element which is at least $(q' + 1)$.
    Therefore, it also contains an element which is at most $(q' - 1)$, because of~\ref{item:L2}.
    Modify $L$ by decreasing an element of value at least $(q' + 1)$ by one, and increasing an element of value at most $(q' - 1)$ by one.
    This reduces $\gamma_L$ by exactly one, so condition~\ref{item:L3} still holds and the resulting multiset remains in $\mathcal{L}_{n',m',q',k'}$.
    Moreover, $\sum_{\ell\in L} \varphi(\ell)$ strictly decreases, since $\varphi$ has a breakpoint at $q'$.
    Thus $L$ was not a minimizer.

    \medskip
    Next, we prove that $L^*$ is optimal by showing that any optimal $L' \in \mathcal{L}_{n',m',q',k'} \setminus \{L^*\}$ can be transformed into $L^*$ without increasing the objective value.
    By the first part, we assume $\gamma_{L'} = k'$ throughout the transformation.

    First, we eliminate elements smaller than $(q' - 1)$.
    We show that if $L'$ contains such an element, then $q' \in L'$.
    Suppose for contradiction that $q' \notin L'$.
    Then
    \begin{align*}
      m' - \gamma_{L'}
      &=\sum_{\ell\in L'}(\ell-[\ell-q']^+)=\sum_{\ell\in L'}\min\{\ell,q'\}<\sum_{\ell\in L'}\left(q'-1+\chi_{\ell\geq q'}\right)\\
      &=n'(q'-1)+\#\{\ell\in L':\ell\geq q'\}=n'(q'-1)+\#\{\ell\in L':\ell\geq q'+1\}\\
      &\leq n'(q'-1)+\gamma_{L'},
    \end{align*}
    where the strict inequality holds because $L'$ contains an element smaller than $(q' - 1)$, the last equation uses the indirect assumption that $q' \notin L'$, and all other steps follow by definition and rearrangements.
    Thus $r = m'-n'(q'-1) < 2\gamma_{L'}$ holds.
    Since $r \geq 2k'$ by~\ref{item:L2}, we get that $\gamma_{L'}>k'$, contradicting the optimality of $L'$.
    Hence $q'\in L'$.
    Now decrease an element equal to $q'$ by one, and increase an element smaller than $(q' - 1)$ by one.
    This does not increase $\sum_{\ell \in L'}\varphi(\ell)$ by convexity of $\varphi$, keeps $\gamma_{L'}$ unchanged, and strictly decreases the number of elements of value smaller than $(q' - 1)$.
    Iterating this operation, we obtain a minimizer containing no element smaller than $(q' - 1)$.

    Second, we eliminate elements larger than $(q'+1)$.
    If each element of the minimizer $L'$ is at least $(q' - 1)$ and there exists an element in $L'$ that is larger than $(q'+1)$, then $(q'-1) \in L'$ by~\ref{item:L2}.
    It follows that
    \begin{align*}
      m' &= n'(q'-1)+\gamma_{L'}+\#\{\ell\in L': \ell=q'\}+\#\{\ell\in L': \ell\geq q'+1\}\\
         &< n'(q'-1)+2\gamma_{L'}+\#\{\ell\in L': \ell=q'\}\\
         &=n'(q'-1)+2k'+\#\{\ell\in L': \ell=q'\},
    \end{align*}
    where the inequality is strict because there exists an element in $L'$ that is larger than $(q'+1)$, which implies that $\#\{\ell\in L': \ell\geq q'+1\}<\gamma_{L'}$.
    Hence $r = m' - n'(q'-1) < 2k' + \#\{\ell\in L' : \ell=q'\}$, and since $r \geq 2k'$, it follows that $q' \in L'$.
    Now decrease an element larger than $(q'+1)$ by one, and increase an element equal to $q'$ by one.
    This again keeps $\gamma_{L'}$ unchanged, does not increase the objective value, does not introduce any element smaller than $(q'-1)$, and decreases the number of elements that are larger than $(q' + 1)$.
    Iterating this operation, we obtain a minimizer containing no element below $(q'-1)$ and no element above $(q'+1)$.

    \medskip
    Therefore, by applying the modifications above, we obtain an optimal $L'\in \mathcal{L}_{n',m',q',k'}$ that contains only elements $(q'-1)$, $q'$, and $(q'+1)$.
    This forces $L'$ to contain exactly $k'$ copies of $(q'+1)$.
    Then conditions~\ref{item:L1} and~\ref{item:L2} uniquely determine the remaining multiplicities in $L'$, namely it contains $(r - 2k')$ copies of $q'$ and $(n'+k'-r)$ copies of $(q'-1)$.
    Hence $L'=L^*$, proving that $L^*$ is optimal.
  \end{subproof}

  We remark that if $\varphi$ is strictly convex, then $L^*$ is in fact the unique minimizer of $\sum_{\ell\in L}\varphi(\ell)$ over $\mathcal{L}_{n',m',q',k'}$, which can be proved in the same way as Claim~\ref{cl:uniqueMinimizer}.

  We now continue the proof of Theorem~\ref{thm:minSumHLooplessNPCGeneral}.
  Consider the multisets in $\mathcal{L}_{n',m',q',k'}$ with parameters $n'=|V'|$, $m'=|E'|$, $q'=q$, and $k'=k$, as specified in the construction illustrated by Figure~\ref{fig:minSumHLoopsNPC}.
  The next step is to verify that these parameters satisfy condition~\ref{item:L2} in the definition of $\mathcal{L}_{n',m',q',k'}$, and thus $q=\left\lceil\frac{|E'|}{|V'|}\right\rceil$, as stated in the theorem.

  \begin{claim}\label{cl:multisetSecondProperty}
    For the graph $G' = (V', E')$ defined at the beginning of the proof, shown in Figure~\ref{fig:minSumHLoopsNPC}, we have $|E'| = |V'|(q-1) + r$ for some $r \in \{2k,\dots,|V'|-1\}$, where $q = 2 \sum_{e_j\in \mathcal{H}} |e_j|+1$ as defined above and $k$ is the number of sought covering sets.
  \end{claim}
  \begin{subproof}
    First, observe that
    \begin{equation}\label{eq:loopsV}
      |V'| = 2 \sum_{e_j\in \mathcal{H}} |e_j| + n + m + 1,
    \end{equation}
    and that $E'$ contains $4 \sum_{e_j\in \mathcal{H}} |e_j| + n + m$ non-loop edges.
    Next, we compute $|E'|$ by summing degrees.
    After adding loops, we have, $\sum_{v\in V'} d'(v)=(2q+3) \sum_{e_j\in \mathcal{H}} |e_j|+(q+1)(n+m)+q+n+k$, where each loop is counted once and each non-loop edge is counted twice in the sum.
    Therefore,
    \begin{equation}\label{eq:loopsE}
      |E'|=\sum_{v\in V'}d'(v)- \#\{\text{non-loop edges in } E'\}=(2q-1) \sum_{e_j\in \mathcal{H}} |e_j| + q(n+m+1) + n + k.
    \end{equation}
    The residual $r$ can be expressed using~\eqref{eq:loopsV} and~\eqref{eq:loopsE} as
    \begin{equation}\label{eq:loopsResidual}
      r = |E'| - |V'|(q-1)  = \sum_{e_j\in \mathcal{H}} |e_j| + n + m + 1 + n + k.
    \end{equation}
    Finally, we verify that $r \in \{2k, \dots, |V'|-1\}$.
    Since $m > k$, it follows immediately from~\eqref{eq:loopsResidual} that $r \geq 2k$.
    Moreover, $n + k < 2n \leq \sum_{e_j\in \mathcal{H}} |e_j|$, where the first inequality uses $n> k$ and the second uses that each element $s_i\in S$ appears in at least two subsets in $\mathcal{H}$.
    Substituting this into~\eqref{eq:loopsResidual} implies $r < |V'|$, completing the proof.
  \end{subproof}

  Recall from Claim~\ref{cl:uniqueMinimizer} that for any discrete convex function $\varphi$ with a breakpoint at $q'$, the multiset $L^*$ minimizes $\sum_{\ell\in L}\varphi(\ell)$ over $\mathcal{L}_{n',m',q',k'}$.
  Let $\operatorname{opt}_{L^*}$ denote this minimum value.
  \begin{claim}\label{cl:setCover}
    There exist subsets $e_{j_1},\dots,e_{j_k}\in\mathcal{H}$ covering $S$ if and only if there exists a vertex order of $G'$ such that $\sum_{v\in V'} \varphi(\cev{d'}(v))\leq\operatorname{opt}_{L^*}$.
  \end{claim}
  \begin{subproof}
    First, suppose $e_{j_1},\dots,e_{j_k}\in\mathcal{H}$ form a set cover of $S$.
    We construct a suitable order of the vertices as follows.
    For each $i=1,\dots,k$, place the vertex $e_{j_i}$ to the last available position, remove it from $G'$, and while there exists a vertex of degree at most $q$, place a vertex of minimum degree to the last available position and remove it from $G'$.
    After processing these $k$ vertices, place the vertex $s$ next, and then order the remaining vertices in the same manner: for each $e_j$ with $j\notin\{j_1,\dots,j_k\}$, put $e_j$ to the last free position, remove it from $G'$, and again repeatedly place vertices of minimum degree until none remain.
    In the resulting order, every vertex has left-degree $(q-1)$, $q$, or $(q+1)$, and exactly the vertices $e_{j_1},\dots,e_{j_k}$ have degree $(q+1)$, by construction of $G'$.
    Thus the multiset of left-degrees $\{\cev{d'}(v):v\in V'\}$ coincides with $L^*$, yielding $\sum_{v\in V'} \varphi(\cev{d'}(v))=\operatorname{opt}_{L^*}$, which proves the first direction of the statement.

    Second, suppose there exists a vertex order with $\sum_{v\in V'} \varphi(\cev{d'}(v)) \leq \operatorname{opt}_{L^*}$.
    Let $L=\{\cev{d'}(v) : v \in V'\}$ denote the corresponding multiset of left-degrees.
    Then $\gamma_L \leq k$, since otherwise Claim~\ref{cl:uniqueMinimizer} would be violated.

    Let $\varepsilon_e$ be the number of vertices $e_j$ placed after $s$, and $\varepsilon_s$ the number of vertices $s_i$ placed after $s$.
    With this notation,
    \[
      k
      \geq \gamma_L
      \geq [\cev{d'}(s)-q]^++\sum_{\substack{v\in V':\\s\text{ precedes } v}} [\cev{d'}(v)-q]^+
      \geq (n+k-\varepsilon_e-\varepsilon_s)+\varepsilon_e
      = n+k-\varepsilon_s.
    \]
    Here, the term $[\cev{d'}(s) - q]^+$ equals $(n+k-\varepsilon_e-\varepsilon_s)$, and each $e_j$ appearing after $s$ contributes at least one vertex $v$ (namely, one of $e_j,a_{s_i}^{e_j},b_{s_i}^{e_j}$) with $\cev{d'}(v) \geq q+1$.
    It follows that $\varepsilon_s \geq n$, so all vertices $s_1,\dots,s_n$ must lie to the right of $s$.
    Using the (at most) $k$ vertices with left-degree at least $(q+1)$, we now construct a set cover as follows.
    If $\cev{d'}(e_j)=q+1$ or $\cev{d'}(v)\geq q+1$ for some $v\in\{a_{s_i}^{e_j},b_{s_i}^{e_j}\}$, then include $e_j$ in the cover.
    If $\cev{d'}(s_i)=q+1$, then include any set $e_j\in\mathcal{H}$ containing $s_i$.
    What is left is to show that these subsets cover the whole ground set $S$.
    If $\cev{d'}(s_i)=q+1$, then $s_i$ is clearly covered.
    Otherwise, if $\cev{d'}(s_i) < q+1$, then the rightmost vertex $v$ from $\{e_j, a_{s_{\ell}}^{e_j},b_{s_\ell}^{e_j}: s_i,s_{\ell}\in e_j,s_\ell\in S, e_j\in \mathcal{H}\}$ has $\cev{d'}(v) > q+1$ by the construction of $G'$, hence the subset $e_j\in \mathcal{H}$ corresponding to $v$ covers $s_i$.
    Thus we obtain a valid set cover of size at most $k$, completing the proof.
  \end{subproof}

  Claim~\ref{cl:setCover} completes the first step of the proof of Theorem~\ref{thm:minSumHLooplessNPCGeneral}.
  Now we eliminate loops.

  \medskip
  \noindent\textbf{Step 2:} hardness for graphs without loops.
  We have established NP-hardness of minimizing $\sum_{v\in V}\varphi(\cev{d}(v))$ for graphs that may contain loops.
  Here we construct a loop-free graph whose optimal ordering preserves the structure of an optimal solution for the graph $G' = (V', E')$ from that proof (see Figure~\ref{fig:minSumHLoopsNPC}, except for the loops).
  The construction consists of two steps.

  \medskip
  First, we construct an auxiliary graph $G'' = (V'', E'')$ as follows.
  Take a copy of $G'$, add a new vertex $u$, and for each vertex $v\in V'$, replace each loop on $v$ with an edge between $u$ and $v$.
  Finally, add $(q+n+k)$ loops to $u$.

  We claim that in some optimal order of $G''$, $u$ appears first.
  We prove that from an arbitrary optimal order of $G''$, we can construct another where $u$ appears first.
  Consider an optimal order where some neighbor of $u$ precedes it, and let $v$ be the closest such neighbor.
  Observe that $\cev{d''}(v)\leq q+n+k<q+1+n+k\leq\cev{d''}(u)$, where $d''$ is the degree function of $G''$, which means moving $u$ directly before $v$ does not increase the objective value.
  Repeating this argument shows that $u$ can be placed before all of its neighbors, and hence moved to the first position without loss of optimality.

  Now consider such an optimal order $(u, \sigma)$ of $G''$, where $\sigma$ is an order of the vertices in $V'$.
  For each $v' \in V'$, let $v'' \in V''$ be its copy in $G''$.
  Then $\cev{d''}_{\!\!\!(u, \sigma)}(v'') = \cev{d'}_{\!\!\sigma}(v')$, which implies that the order $(u, \sigma)$ is optimal for $G''$ if and only if $\sigma$ is optimal for $G'$.

  \medskip
  Second, we construct a loop-free graph $G = (V, E)$ as follows.
  Take two disjoint copies $G_1 = (V_1, E_1)$ and $G_2 = (V_2, E_2)$ of $G''$, with $v_1\in V_1$ and $v_2\in V_2$ denoting the copies of $v\in V''$.
  Remove the $(q+n+k)$ loops from $u_1$ and $u_2$, and instead add $(q+n+k)$ parallel edges between $u_1$ and $u_2$, making $G$ loop-free.
  Consider an optimal order for $G$.
  Without loss of generality, assume $u_1$ appears before $u_2$ (otherwise swap the role of $G_1$ and $G_2$).
  Then the restriction of this order to $V_2$ must be an optimal order for $G''$, since otherwise we could improve the order of $G$.
  By the first part of the proof, we can assume that $u_2$ appears first among the vertices of $V_2$.

  Thus, from an optimal order of the loop-free graph $G$, we obtain an optimal order for $G'$ by restricting to $G_2$ and omitting $u_2$.
  This shows that minimizing $\sum_{v\in V}\varphi(\cev{d}(v))$ remains NP-hard even for loop-free graphs.

  \medskip
  Finally, we verify that $q=\left\lceil \frac{|E|}{|V|} \right\rceil$ as claimed.
  \begin{claim}
    For the graph $G = (V, E)$, we have $|E|=|V|(q-1)+r$ for some $r \in \{1, \dots, |V|-1\}$.
  \end{claim}
  \begin{subproof}
    By the construction of $G = (V, E)$, it follows that $|V|=2|V'|+2=4 \sum_{e_j\in \mathcal{H}} |e_j| + 2(n+m+1) + 2$ and
    $|E|=2|E'|+q+n+k=(4q-2) \sum_{e_j\in \mathcal{H}} |e_j| + 2q(n+m+1) + 3(n+k) + q$, where $|E'|$ and $|V'|$ are as expressed in~\eqref{eq:loopsV} and~\eqref{eq:loopsE}, respectively.
    Therefore,
    \[
      r
      = |E| - |V|(q-1)
      = 2 \sum_{e_j\in \mathcal{H}} |e_j| + 2(n+m+1) + 3(n+k) - q + 2.
    \]
    Since $q=2 \sum_{e_j\in \mathcal{H}} |e_j|+1$, as defined at the beginning of the proof of Theorem~\ref{thm:minSumHLooplessNPCGeneral}, it immediately follows that $r \geq 1$.
    Moreover, using the assumptions $n > k$ and that each $s_i\in S$ appears in at least two sets in $\mathcal{H}$, we obtain $3(n+k) - (q-1) < 6n - (q-1) \leq 3 \sum_{e_j\in \mathcal{H}} |e_j| - (q-1) = \sum_{e_j\in \mathcal{H}} |e_j|$.
    Substituting this into the expression for $r$ gives $r < |V|$.
    Hence $r \in \{1, \dots, |V|-1\}$, as required.
  \end{subproof}

  This completes the proof of Theorem~\ref{thm:minSumHLooplessNPCGeneral}.
\end{proof}

Theorem~\ref{thm:minSumHLooplessNPCGeneral} yields the following two natural corollaries concerning important special cases.

\begin{corollary}\label{cor:minSumHLooplessNPC}
  For every discrete strictly convex function $\varphi : \Z_+ \to \R$, finding a vertex order that minimizes $\sum_{v \in V} \varphi(\cev{d}(v))$ is NP-hard.
  \FBOX
\end{corollary}
\noindent
This immediately implies that Problems~\ref{prob:sumPhiRhoV}-\ref{prob:incMax} and $\max \prod_{v \in V} \cev{d}(v)$ are NP-hard.
We also obtain that, within the class of non-linear integer programs considered in~\cite{meyer1977class}, restricting attention to optimal corner solutions --- rather than arbitrary integer points of the polyhedron --- becomes NP-hard, as already discussed in Section~\ref{sec:convexHullOfIndegrees}.

Now we state another corollary of Theorem~\ref{thm:minSumHLooplessNPCGeneral}.
\begin{corollary}\label{cor:oneBreakPointSumHLooplessNPC}
  For every piecewise-linear discrete convex function $\varphi : \Z_+ \to \R$ with a single breakpoint at $q = \left\lceil \frac{|E|}{|V|} \right\rceil$, finding a vertex order that minimizes $\sum_{v \in V} \varphi(\cev{d}(v))$ is NP-hard.
  \FBOX
\end{corollary}
\noindent
Thus, minimizing $\sum_{v \in V} \left([\alpha(v) - \cev{d}(v)]^+ + [\cev{d}(v) - \beta(v)]^+\right)$ is NP-hard for $\alpha \equiv \beta \equiv \left\lceil\frac{|E|}{|V|}\right\rceil$.
Furthermore, minimizing $\sum_{v\in V} [\cev{d}(v) - g(v)]^+$ is NP-hard even for $g\equiv \left\lceil\frac{|E|}{|V|}\right\rceil$, while one can decide in polynomial time whether the optimal objective value is zero, i.e., whether an acyclic orientation exists with $\varrho(v)\leq g(v)$ for each $v\in V$~\cite{kiraly2018acyclic}.

\medskip
Complementing Corollary~\ref{cor:oneBreakPointSumHLooplessNPC}, the problem becomes polynomial-time solvable if $\varphi$ is linear.
We prove this in a more general setting in which, instead of a single function $\varphi$, each vertex $v \in V$ has its own linear function $\varphi_v$.
Although this result can also be derived from Theorems~\ref{thm:VerticesOfBasePolyAreAcyclicIndigrees} and~\ref{thm:TUMatrixForIndegreeVectors}, we present here a simpler argument, that also implies an even more efficient algorithm.
\begin{theorem}\label{thm:minLinearFunctionAcyclic}
  Let $G = (V, E)$ be a graph and let $\varphi_v : \Z_+ \to \R$ be a discrete linear function for each $v \in V$.
  Then ordering the vertices in non-increasing order of the slopes of $\varphi_v$ minimizes $\sum_{v \in V} \varphi_v(\cev{d}(v))$.
\end{theorem}
\begin{proof}
  Let $a_v, b_v \in \R$ be such that $\varphi_v(z) = a_vz + b_v$ for every $v \in V, z \in \Z_+$.
  For an arbitrary order $\sigma$ of the vertices, define $\pi_\sigma(v) = b_v $ for each $v \in V$, and let
  \[
    \pi_\sigma(uv) =
    \begin{cases}
      a_v & \text{if $u$ precedes $v$ in $\sigma$},\\
      a_u & \text{otherwise}
    \end{cases}
  \]
  for each edge $uv \in E$.
  Then we can express the objective as
  \[
    \sum_{v \in V} \varphi_v(\cev{d}(v))
    = \sum_{v \in V} a_v \cev{d}(v) + b_v
    = \sum_{e \in E} \pi_\sigma(e) + \sum_{v \in V} \pi_\sigma(v).
  \]
  In any vertex order, for each edge $uv$, we clearly have $\pi_\sigma(uv) \geq \min\{a_u, a_v\}$, with equality achieved if the vertex with the larger slope precedes the other.
  Thus, by ordering the vertices in non-increasing order of $a_v$, every edge contributes exactly $\min\{a_u,a_v\}$, yielding an optimal ordering.
\end{proof}

As an immediate corollary, if all vertices share the same linear function $\varphi$, then every vertex order attains the same objective value, and hence all orders are optimal.
This also implies that minimizing $\sum_{v\in V} \varrho(v)^2$ and $\sum_{v\in V} \binom{\varrho(v)}{2}$ are exactly the same problems, as we stated in relation to the routing application in Section~\ref{sec:introduction}.


\subsection{A special case: minimizing \texorpdfstring{$\sum_{v \in V} \varrho(v)^2$ over acyclic orientations}{Sum rho(v)^2}}\label{sec:minSquareSum}
In this section, we study the problem of finding an acyclic orientation that minimizes the square-sum of indegrees, that is, minimizing $\sum_{v \in V} \cev{d}(v)^2$ over all vertex orders.
This is a natural special case of minimizing $\sum_{v \in V} \varphi(\cev{d}(v))$, obtained by setting $\varphi(z) = z^2$.
The NP-hardness of this problem follows immediately from Corollary~\ref{cor:minSumHLooplessNPC}.

Next, we strengthen this result by showing that the problem remains NP-hard for simple graphs.
We then propose a greedy algorithm and analyze its approximation guarantee.

\subsubsection{Hardness for simple graphs}
We now prove that minimizing the square-sum of indegrees remains NP-hard even for simple graphs.

\begin{theorem}\label{thm:squaresumNPForSimple}
  It is NP-hard to find a vertex order that minimizes the sum $\sum_{v \in V} \cev{d}(v)^2$ even for simple graphs.
\end{theorem}
\begin{proof}
  Let $G' = (V', E')$ be the graph constructed in the first step of the proof of Theorem~\ref{thm:minSumHLooplessNPCGeneral}, which contains loops.
  Without loss of generality, assume that each vertex has at least two loops; this can be ensured by increasing the parameter $q$, which simply adds an additional loop to every vertex.

  We construct a simple graph $G = (V, E)$.
  Start with a copy of $G'$, and for each vertex $v \in V'$, proceed as follows.
  Let $x$ denote the number of loops incident to $v$.
  Remove these $x$ loops and add a gadget $H_v$ defined as follows:
  Take a clique $K_M$ on $M = 3d(v)$ vertices, denoted by $k_1,\dots,k_M$.
  Introduce $x$ new vertices $v_1,\dots,v_x$.
  For each $i \in \{1, \dots, x\}$, connect $v_i$ to $v$ and to every vertex of $K_M$.
  Equivalently, $H_v$ consists of $x$ vertices adjacent both to $v$ and to all of $K_M$, forming a complete bipartite graph between $\{v_1,\dots,v_x\}$ and $K_M$.
  An illustration of the construction is given in Figure~\ref{fig:sqrtSumSimpleNPH}.

  \begin{figure}[t]
    \centering
    \begin{tikzpicture}[yscale=1]
      \tikzset{VertexStyle/.append style = {minimum size = 20pt,inner sep=0pt}}
      \SetVertexMath
      \Vertex[x=0, y=2, L=k_1]{k1}
      \Vertex[x=0, y=1, L=k_2]{k2}
      \Vertex[x=0, y=-2, L=k_M]{kM}
      \Vertex[x=3, y=1.25, L=v_1]{v1}
      \Vertex[x=3, y=0.25, L=v_2]{v2}
      \Vertex[x=3, y=-1.25, L=v_x]{vx}
      \Vertex[x=6, y=0, L=v]{v}

      \begin{pgfonlayer}{background}
        \highlight{10mm}{black!15}{(k1.center) -- (kM.center)}
      \end{pgfonlayer}
      \node(k) at (-1.5,0) {$K_M$};
      \node(s1) at (0,0) {$\cdot$};
      \node(s2) at (0,-0.5) {$\cdot$};
      \node(s3) at (0,-1) {$\cdot$};
      \node(t) at (3,-0.4) {$\vdots$};


      \draw[wavy] (v) to (7.5,0);
      \draw[wavy] (v) to (7.5,0.5);
      \draw[wavy] (v) to (7.5,-0.5);
      \path[circleAroundEdges={.15}{.37}{0.33}] (v)--(7.5,0) node [above=.45cm, pos=2/3] {$d(v) - x$};

      \draw (v)--(v1);
      \draw (v)--(v2);
      \draw (v)--(vx);
      \path[circleAroundEdges={.15}{.4}{0.125}] (v)--(3,0) node [above=.45cm, pos=1/8] {\phantom{(}$x$\phantom{)}};
      \draw (v1)--(k1);
      \draw (v1)--(k2);
      \draw (v1)--(kM);
      \draw (v2)--(k1);
      \draw (v2)--(k2);
      \draw (v2)--(kM);
      \draw (vx)--(k1);
      \draw (vx)--(k2);
      \draw (vx)--(kM);


      \draw (k1) to [out=-120,in=120,looseness=1] (k2);
      \draw[dotted] (k1) to [out=-130,in=130,looseness=1] (-0.1,-0.5);
      \draw (k1) to [out=-150,in=140,looseness=1] (kM);
      \draw[dotted] (k2) to [out=-120,in=120,looseness=1] (-0.1,-0.5);
      \draw (k2) to [out=-140,in=130,looseness=1] (kM);
      \draw[dotted] (kM) to [out=120,in=-120,looseness=1] (-0.1,-0.5);


      \path [draw third paths={draw=none}{dotted}] (3, -0.5)--(v);

      \path [draw third paths={draw=none}{dotted}] (3, -0.5)--(k1);

      \path [draw third paths={draw=none}{dotted}] (3, -0.5)--(k2);


      \path [draw third paths={draw=none}{dotted}] (0, -0.5)--(v1);
      \path [draw third paths={draw=none}{dotted}] (0, -0.5)--(v2);
      \path [draw third paths={draw=none}{dotted}] (0, -0.5)--(vx);

    \end{tikzpicture}
    \caption{The gadget $H_v$ for replacing $x$ loops incident to the vertex $v$ of total degree $d(v)$.
      The wavy edges on the right illustrate the non-gadget edges.}\label{fig:sqrtSumSimpleNPH}
  \end{figure}

  \begin{claim}\label{cl:minSquareSumGadget}
    There exists an order of $G$ that minimizes $\sum_{v \in V} \cev{d}(v)^2$ such that, for each $v\in V'$, the vertices of its gadget $H_v$ appear immediately before $v$ in the order $k_1, \dots, k_M, v_1, \dots, v_x, v$.
  \end{claim}
  \begin{proof}
    Let $\sigma$ be an optimal order that minimizes the number of inversions relative to the order of the gadgets stated in the claim, that is, the number of pairs of vertices $u_1, u_2 \in H_v$ such that $u_1$ precedes $u_2$ in $\sigma$, but $u_2$ precedes $u_1$ in the order $k_1, \dots, k_M, v_1, \dots, v_x, v$ for some $v \in V'$.
    We show that in fact no inversions occur.

    Suppose for contradiction that in some gadget $H_v$, two vertices are reversed in $\sigma$ relative to their intended order $k_1, \dots, k_M, v_1, \dots, v_x, v$.
    Since the left-degree of each vertex in $H_v \setminus \{v\}$ depends only on the relative order within $H_v$, we may assume that the vertices of $H_v$ appear contiguously in $\sigma$.
    Since the vertices $k_1, \dots, k_M$ are symmetric, we can assume they appear in order of increasing index; otherwise, we could swap them without changing the optimality and reduce inversions.
    The same holds for $v_1, \dots, v_x$.
    We focus first on the order within $H_v \setminus \{v_1\}$, and we place $v_1$ correctly at the end of the argument.

    Suppose some $k_j$ appears after $v_2$ in $\sigma$, and let $j$ be the smallest such index.
    Let $u$ denote the vertex immediately before $k_j$ in $\sigma$, which must be either $v$ or $v_i$ for some $i \in \{2, \dots, x\}$.
    If $u = v$, then, since $v$ and $k_j$ are not adjacent, swapping them leaves all left-degrees unchanged, hence preserves optimality but reduces inversions, contradicting the choice of $\sigma$.
    If $u=v_i$ for some $i \in \{2, \dots, x\}$, then $\cev{d}(k_j)\geq j+1 > j \geq \cev{d}(v_i)$, so swapping again preserves optimality but reduces inversions, contradicting the choice of $\sigma$.
    In either case, we contradict the choice of $\sigma$.
    Thus, the vertices of $H_v \setminus \{v\}$ appear in $\sigma$ in the form $k_1, \dots, k_{j-1}, v_1, k_j, \dots, k_M, v_2, \dots, v_x$ for some $j \in \{1, \dots, M+1\}$.

    Next, we prove that $v$ must be the last vertex of the gadget $H_v$ in $\sigma$.
    If $v$ lies between $v_1$ and $v_x$, then we argue that swapping it with its successor $w$ cannot increase the sum.
    If $w \in K_M$, then the left-degrees stay the same, so the order remains optimal and has fewer inversions, contradicting the choice of $\sigma$.
    Otherwise, $w = v_i$ for some $i \in \{2, \dots, x\}$, and then $\cev{d}(v)$ increases by one, $\cev{d}(w)$ decreases by one.
    Since $\cev{d}(v) \leq d(v) - 1$ and $\cev{d}(w) = M + 1$ with $d(v) \leq M$, the sum strictly decreases, contradicting the optimality of $\sigma$.

    Otherwise if $v$ and exactly $\ell$ vertices of $K_M$ precede $v_1$, then moving $v$ after $v_x$ increases $\cev{d}(v)$ from $a$ to $(a + x)$, decreases $\cev{d}(v_1)$ from $(\ell + 1)$ to $\ell$, and reduces each $\cev{d}(v_i)$ from $(M + 1)$ to $M$ for $i = \{2, \dots, x\}$.
    The net decrease $\Delta$ in the sum is
    \begin{align*}
      \Delta
      &= a^2 + (\ell+1)^2 + (x - 1)(M+1)^2 - \bigl((a + x)^2 + \ell^2 + (x - 1)M^2\bigr)\\
      &= -2ax - x^2 + 2\ell + 1 + (x - 1)(2M + 1)\\
      &> -3d(v)x + 2Mx - 2M,
    \end{align*}
    where the inequality follows from $a \leq d(v)$, $x \leq d(v)$, and $\ell\ge 0$.
    Since $M = 3d(v)$ and $x \geq 2$, the value of $\Delta$ is non-negative, so the order remains optimal and has fewer inversions, contradicting the choice of $\sigma$.

    Therefore, $v$ must be the last vertex of $H_v$ in $\sigma$.
    Moreover, the vertices of the gadget $H_v$ appear in $\sigma$ in the order $k_1, \dots, k_{j-1}, v_1, k_j, \dots, k_M, v_2, \dots, v_x, v$ for some $j\in\{1,\dots, M+1\}$.
    If $j \neq M + 1$, then swapping $v_1$ and $k_j$ would not increase the sum, as $\cev{d}(v_1) = j - 1 < j = \cev{d}(k_j)$, but would reduce inversions, another contradiction.
    Hence, the vertices of $H_v$ appear in $\sigma$ in the order $k_1, \dots, k_M, v_1, \dots, v_x, v$, as claimed.
  \end{proof}

  We continue the proof of Theorem~\ref{thm:squaresumNPForSimple}.
  Let $\sigma_{H_v}$ denote the order $k_1, \dots, k_M, v_1, \dots, v_x, v$, and let $\operatorname{opt}_{H_v} = \sum_{u \in H_v \setminus \{v\}} \cev{d}_{\sigma_{H_v}}(u)^2$ denote the minimum contribution of the gadget $H_v$.
  Let $\operatorname{opt}_G$ and $\operatorname{opt}_{G'}$ denote the optimal values for $G$ and $G'$, respectively.
  To complete the proof, it suffices to show that $\operatorname{opt}_G = \operatorname{opt}_{G'} + \sum_{v \in V'} \operatorname{opt}_{H_v}$ and that if $\sigma$ is an optimal order for $G$, then its restriction to $V'$ is an optimal order for $G'$.

  First, consider an optimal order for $G'$.
  By inserting the vertices of each gadget $H_v$ immediately before $v$ in the order $\sigma_{H_v}$, we obtain an order for $G$ of objective value $\operatorname{opt}_{G'} + \sum_{v \in V'} \operatorname{opt}_{H_v}$.
  Thus, $\operatorname{opt}_G \leq \operatorname{opt}_{G'} + \sum_{v \in V'} \operatorname{opt}_{H_v}$.

  Next, let $\sigma$ be an optimal order for $G$, and let $\sigma'$ denote its restriction to $V'$.
  By Claim~\ref{cl:minSquareSumGadget}, we may assume that for each $v \in V'$, the vertices of $H_v$ appear immediately before $v$ in the order $\sigma_{H_v}$, otherwise we could rearrange them, preserving the optimality and the relative order of the vertices in $V'$.
  Thus, for all $u \in V$,
  \[
    \cev{d}_{\sigma}(u) =
    \begin{cases}
      \cev{d}_{\sigma'}(u) & \text{if } u \in V', \\
      \cev{d}_{\sigma_{H_v}}(u) & \text{if } u \in H_v \setminus \{v\} \text{ for some } v \in V'.
    \end{cases}
  \]
  It follows that $\operatorname{opt}_G = \sum_{v \in V'} \cev{d}_{\sigma'}(v)^2 + \sum_{v \in V'} \operatorname{opt}_{H_v} \geq \operatorname{opt}_{G'} + \sum_{v \in V'} \operatorname{opt}_{H_v}$.

  These together show that $\operatorname{opt}_G = \operatorname{opt}_{G'} + \sum_{v \in V'} \operatorname{opt}_{H_v}$ and that $\sigma'$ is an optimal order for $G'$.
  This completes the proof, since finding an optimal order for $G'$ is NP-hard, as shown in the first step of the proof of Theorem~\ref{thm:minSumHLooplessNPCGeneral}.
\end{proof}

\subsubsection{Greedy approximation algorithm}

We analyze the approximation performance of the following greedy algorithm:
at each iteration, the algorithm selects a vertex of minimum degree, assigns it to the last available position in the order, and removes it from the graph.
Observe that when multiple vertices share the minimum degree, the output may depend on which vertex is chosen.
We note that this algorithm is essentially the same as the algorithm for computing the degeneracy $\cev{d}_{\min}$~\cite{matula1983smallest}.

In what follows, we establish two different upper bounds on the approximation ratio of this algorithm for minimizing the square-sum of left-degrees.
The first bound provides a stronger guarantee for sparse graphs, while the second is tighter for dense graphs.

\begin{theorem}\label{thm:degApprox}
  Let $G = (V, E)$ be a graph.
  The greedy algorithm finds a $\cev{d}_{\min}(G)$\nobreakdash-approximate order for minimizing $\sum_{v \in V} \cev{d}(v)^2$.
\end{theorem}
\begin{proof}
  Consider the order produced by the greedy algorithm.
  For each edge $e$, let $\pi(e) = \cev{d}(v)$, where $v$ is the endpoint of $e$ that appears later in the order.
  Then, $\sum_{e \in E} \pi(e) = \sum_{v \in V} \cev{d}(v)^2$.

  Since the algorithm is known to compute the degeneracy number by finding a $\cev{d}_{\min}(G)$ bounded order, we have $\cev{d}(v) \leq \cev{d}_{\min}(G)$ for every $v$.
  Therefore, $\pi(e) \leq \cev{d}_{\min}(G)$ for all $e \in E$, and the objective value satisfies $\operatorname{obj} \leq |E| \cdot \cev{d}_{\min}(G)$.
  On the other hand, each edge must contribute at least $1$ to the sum of squared left-degrees in any order, so $\operatorname{opt} \geq |E|$.

  Thus, the approximation ratio is bounded by $\frac{\operatorname{obj}}{\operatorname{opt}} \leq \frac{|E| \cdot \cev{d}_{\min}(G)}{|E|} = \cev{d}_{\min}(G)$.
\end{proof}

We now derive another upper bound on the approximation ratio of the greedy algorithm.

\begin{theorem}
  Let $G = (V, E)$ be a graph.
  The greedy algorithm produces a $4\eta_n$\nobreakdash-approximate order for minimizing $\sum_{v \in V} \cev{d}(v)^2$, where $\eta_n = \sum_{i=1}^n \frac{1}{i}$ is the $n^{\text{th}}$ harmonic number.
\end{theorem}
\begin{proof}
  Let $v_1, \dots, v_n$ denote the order produced by the greedy algorithm, and let $\operatorname{obj}$ denote its objective value.
  For $i = 1, \dots, n$, define $V_i = \{ v_1, \dots, v_i \}$, and let $G[V_i] = (V_i, E_i)$ be the subgraph of $G$ induced by $V_i$.
  Let $m_i = |E_i|$ and $\operatorname{opt}(V_i)$ denote the minimum square-sum of left-degrees in $G[V_i]$.

  We first give a lower bound:
  \begin{equation}\label{eq:lowerBoundForSquareSum}
    \operatorname{opt}(V) \geq \operatorname{opt}(V_i) \geq \sum_{v \in V_i} \left( \frac{m_i}{i} \right)^2 = \frac{m_i^2}{i},
  \end{equation}
  where the first inequality follows because any optimal order for $G$ induces an order on $V_i$, and the second by Jensen's inequality and the convexity of the square function.

  For the greedy order, we have
  \[
    \operatorname{obj} = \sum_{i=1}^n \cev{d}(v_i)^2 \leq \sum_{i=1}^n \left( \frac{2 m_i}{i} \right)^2 = 4 \sum_{i=1}^n \frac{1}{i} \frac{m_i^2}{i},
  \]
  where the inequality follows because $v_i$ is a vertex with minimum degree in $G[V_i]$, in which the average degree is $\frac{2m_i}{i}$.

  Applying~\eqref{eq:lowerBoundForSquareSum}, this gives
  \[
    \operatorname{obj} \leq 4 \sum_{i=1}^n \frac{1}{i} \operatorname{opt}(V) = 4 \eta_n \operatorname{opt}(V).
  \]
  \vspace{-8mm}

\end{proof}

\begin{figure}[t]
  \centering
  \begin{tikzpicture}[yscale=1.4,xscale=0.9]
    \tikzset{VertexStyle/.append style = {minimum size = 20pt,inner sep=0pt}}
    \SetVertexMath
    \Vertex[x=0,y=0, L=v_1^1]{v1}
    \Vertex[x=1,y=1, L=v_1^2]{v2}
    \Vertex[x=2,y=0, L=v_1^3]{v3}
    \Vertex[x=4,y=0, L=u_2]{v4}
    \Vertex[x=6,y=0, L=v_2^1]{v5}
    \Vertex[x=7,y=1, L=v_2^2]{v6}
    \Vertex[x=8,y=0, L=v_2^3]{v7}

    \draw (v1)--(v2);
    \draw (v2)--(v3);
    \draw (v3)--(v1);
    \draw (v5)--(v6);
    \draw (v6)--(v7);
    \draw (v7)--(v5);
    \draw (v3)--(v4);
    \draw (v4)--(v5);

    \draw [thick, decorate,decoration={brace, amplitude=4pt}, xshift=0cm,yshift=-0.3cm]
    (8.4,0) -- (2.4,0) node [black,midway,xshift=0cm,yshift=-0.3cm]
    {\footnotesize $\times (k - 1)$};

  \end{tikzpicture}
  \caption{Illustration of the graph sequence $G_k$ for which the approximation ratio of the greedy algorithm approaches $\frac{9}{7}$.
    Here, $k$ denotes the number of triangles.
    Each $v_i^j$ represents the $i^\text{th}$ vertex of the $j^{\text{th}}$ triangle ($j \in \{1, 2, 3\}$), and $u_i$ connects $v_{i-1}^3$ to $v_i^1$ for $i \in \{2, \dots, k\}$.}\label{fig:squareSum9/7Approx}
\end{figure}
The graph family shown in Figure~\ref{fig:squareSum9/7Approx} provides an example where the worst-case approximation ratio of the greedy algorithm approaches $\frac{9}{7}$, which is the best lower bound we have found so far.
To analyze the worst case, we compare the performance of the greedy algorithm against the optimal order for the graph shown in Figure~\ref{fig:squareSum9/7Approx}, consisting of $k$ triangles and paths of length two between them.
Consider the execution of the greedy algorithm that orders the vertices as follows: it begins with the vertices $v_i^1, v_i^2, v_i^3$ for $i = 1, \dots, k$, followed by the vertices $u_i$ for $i = 2, \dots, k$.
Note that for every $\ell \in \{1, \dots, n\}$, the $\ell^\text{th}$ vertex in this order has minimum degree in the subgraph induced by the first $\ell$ vertices.
Thus, this order is a possible output of the greedy algorithm, and the objective value is $k(0^2 + 1^2 + 2^2) + (k-1) (2^2) = 9k - 4$.
On the other hand, consider the following optimal order: list the vertices $v_1^1$, $v_1^2$, $v_1^3$ first, and then for each $i = 2, \dots, k$, list $u_i$, $v_i^1$, $v_i^2$, $v_i^3$ in this order.
The corresponding objective value is $0^2 + 1^2 + 2^2 + (k-1)(1^2 + 1^2 + 1^2 + 2^2) = 7k - 2$.
Therefore, the approximation ratio is $\frac{9k - 4}{7k - 2}$, which tends to $\frac{9}{7}$ as $k$ goes to infinity.

We tested the approximation ratio of the worst solution found by the greedy algorithm for every simple graph on at most $12$ vertices, and found that the ratio for the graph $G_k$ shown in Figure~\ref{fig:squareSum9/7Approx} serves as an upper bound for every simple graph on at most $(4k - 1)$ vertices.
It remains an open question whether the worst-case approximation ratio of the greedy algorithm for simple graphs is indeed $\frac{9}{7}$.


\subsection{A special case: lexicographically optimal acyclic orientations}\label{sec:decMinIncMax}
This section investigates the computational complexities of Problems~\ref{prob:decMin} and~\ref{prob:incMax}, along with two complementary problems, namely the \emph{increasingly minimal (inc-min)} and \emph{decreasingly maximal (dec-max)} acyclic orientation problems.
As a relaxation of the dec-min acyclic orientation problem, we also consider the problem of minimizing the maximum (weighted) indegree.

We proceed with a simple observation about acyclic orientations that connects the dec-min and inc-max objectives to minimizing $\sum_{v\in V} \varphi(\varrho(v))$ for a discrete strictly convex function~$\varphi$.
\begin{claim}\label{cl:decMinIncMaxAreSpecial}
  The dec-min and inc-max acyclic orientation problems are special cases of minimizing $\sum_{v\in V} \varphi(\varrho(v))$ over acyclic orientations, corresponding to the choices $\varphi(z) =|V|^z$ and $\varphi(z) = |V|^{-z}$, respectively.
\end{claim}
\begin{proof}
  We present the proof for the dec-min case only, as the other one follows analogously.
  Consider two distinct orientations $D$ and $D'$ of a given graph $G = (V, E)$ with non-increasingly ordered indegree sequences $\ell_1 \geq \dots \geq \ell_n$ and $\ell'_1 \geq \dots \geq \ell'_n$, respectively.
  We first show that if $D$ is better than $D'$ for the dec-min objective, then $\sum_{i=1}^n n^{\ell_i} < \sum_{i=1}^n n^{\ell'_i}$, where $n = |V|$.
  Let $k$ denote the smallest index such that $\ell_i = \ell'_i$ for all $i < k$, and $\ell_k < \ell'_k$.
  Observe that the following upper bound holds for the objective value of $D$:
  \begin{equation}\label{eq:objDecMin}
    \sum_{i=1}^n n^{\ell_i} \leq \sum_{i=1}^{k-1} n^{\ell_i}+(n-k+1)n^{\ell_k} \leq \sum_{i=1}^{k-1} n^{\ell_i}+n^{\ell_k+1}.
  \end{equation}
  The first inequality follows by increasing the last $(n-k+1)$ indegrees to $\ell_k$, and the second inequality by simple rearrangement.
  For the objective value of $D'$, we obtain a complementary lower bound:
  \begin{equation}\label{eq:objDecRandomOrder}
    \sum_{i=1}^n n^{\ell'_i}>\sum_{i=1}^k n^{\ell'_i} = \sum_{i=1}^{k-1} n^{\ell_i}+n^{\ell'_k}\geq \sum_{i=1}^{k-1} n^{\ell_i}+n^{\ell_k+1}.
  \end{equation}
  The strict inequality holds because two distinct indegree sequences always differ in at least two positions, thus $k < n$; and the other two steps follow by the definition of $k$ and by simple rearrangements.
  Combining~\eqref{eq:objDecMin} and~\eqref{eq:objDecRandomOrder}, we obtain that $D$ is also strictly better than $D'$ for the objective $\min\sum_{v\in V} \varphi(\varrho(v))$ for $\varphi(z) = n^z$.
  This immediately implies that comparing two orientations with respect to the dec-min objective and to minimizing $\sum_{v\in V} \varphi(\varrho(v))$ for $\varphi(z) =|V|^z$ yields the same result, which completes the proof.
\end{proof}
Note that this proof also shows that the claim does not only hold over acyclic orientations but also over an arbitrary subset of orientations.


\medskip
Recall that, when orientations need not be acyclic, the dec-min and inc-max orientation problems are equivalent.
This raises the natural question of whether this equivalence continues to hold for acyclic orientations.
We answer this in the negative by exhibiting a counterexample.
\begin{figure}[t]
  \centering
  \begin{tikzpicture}[yscale=0.75]
    \tikzset{VertexStyle/.append style = {minimum size = 20pt,inner sep=0pt}}
    \SetVertexMath
    \Vertex[x=0, y=0, L=v_9]{v8}
    \SetVertexNoLabel
    \grEmptyCycle[form=1,x=0,y=0,RA=2, rotation=-67.5, prefix=v]{8}
    \AssignVertexLabel{v}{$v_1$, $v_2$, $v_3$, $v_4$, $v_5$, $v_6$, $v_7$, $v_8$}

    \draw (v0)--(v1);
    \draw (v2)--(v3);
    \draw (v3)--(v4);
    \draw (v4)--(v5);
    \draw (v5)--(v6);
    \draw (v6)--(v7);
    \draw (v7)--(v0);
    \draw (v0)--(v2);
    \draw (v0)--(v3);
    \draw (v1)--(v3);
    \draw (v4)--(v6);
    \draw (v5)--(v7);

    \draw (v8)--(v1);
    \draw (v8)--(v2);
    \draw (v8)--(v4);
    \draw (v8)--(v5);
    \draw (v8)--(v6);
    \draw (v8)--(v7);
  \end{tikzpicture}
  \caption{The smallest simple graph for which the acyclic dec-min and inc-max problems are not the same.}\label{fig:decMinIncMaxDifference}
\end{figure}
\begin{claim}\label{cl:decMinIncMaxDifference}
  For the simple graph shown in Figure~\ref{fig:decMinIncMaxDifference}, the sets of optimal acyclic orientations for the dec-min and inc-max problems are disjoint.
\end{claim}
\begin{proof}
  We adopt the vertex ordering perspective of the dec-min and inc-max problems, and proceed in two steps.
  First, we show that the non-increasing left-degree sequence of dec-min orders is $3,3,3,3,2,2,1,1,0$.
  Second, we prove that there exists an inc-max order whose non-decreasing left-degree sequence is $0,1,2,2,2,2,2,3,4$.
  Since the latter is better for inc-max than the (reverse of the) former, the set of optimal acyclic orientations for the two problems must be~disjoint.

  For the first part, observe that there exist $3$\nobreakdash-bounded orders for this graph; consequently, any dec-min order must be $3$\nobreakdash-bounded.
  Initially, only $v_2$ and $v_3$ have degree at most $3$; after deleting one of them, the only additional vertex that turns to be of degree at most $3$ is either $v_1$ or $v_4$.
  Therefore, the last two vertices in any $3$\nobreakdash-bounded order must belong among $v_1,v_2,v_3,v_4$, and both of these last two vertices have left-degree $3$.
  Next, consider the subgraph induced by the vertices $v_5, \dots, v_9$.
  Observe that each vertex has at least three neighbors within this induced subgraph; only $v_5$ and $v_8$ have exactly three, and deleting one of them does not create a vertex with degree smaller than $3$.
  Therefore, in any $3$\nobreakdash-bounded order, the last two vertices among $v_5,\dots,v_9$ must have left-degree at least $3$.
  Combining these observations with the fact that the first vertex has left-degree $0$, we conclude that the non-increasing left-degree sequence of any dec-min order is lexicographically at least $3,3,3,3,2,2,1,1,0$.
  The order $v_9, v_8, v_7, v_6, v_5, v_4, v_3, v_1, v_2$ attains this, so this left-degree sequence is the optimal dec-min sequence.

  For the second part, one can verify directly that the order $v_1, v_4, v_2, v_3, v_9, v_5, v_6, v_7, v_8$ produces the non-decreasing left-degree sequence $0,1,2,2,2,2,2,3,4$.
  This sequence is lexicographically larger than the non-decreasing sequence of dec-min orders, namely $0,1,1,2,2,3,3,3,3$.
  Therefore, the sets of optimal dec-min and inc-max orientations are disjoint.
\end{proof}

Note that the same example demonstrates that the optimal solutions for $\varphi(z) = z^2$ differ from those for the dec-min and inc-max problems over acyclic orientations.
It is also worth noting that the maximum indegree in every dec-min acyclic orientation is at most the degeneracy number $\cev{d}_{\min}(G)$, but the example in Figure~\ref{fig:decMinIncMaxDifference} demonstrates that this does not necessarily apply for inc-max acyclic orientations.

\medskip
Theorem~\ref{thm:minSumHLooplessNPCGeneral} and Claim~\ref{cl:decMinIncMaxAreSpecial} immediately imply that both the dec-min and inc-max acyclic orientation problems are NP-hard for multigraphs.
In what follows, we give an in-depth complexity analysis of the dec-min and inc-max acyclic orientation problems under indegree bounds, and restricted to simple graphs.


\subsubsection{Complexity}\label{sec:decMinIncMax:complexity}
We call an ordering \emph{$k$\nobreakdash-bounded} if the left-degree of each vertex is at most $k$.
Next, we consider the $k$\nobreakdash-bounded dec-min and inc-max ordering problems.
For $k = 1$, it is easy to see that a graph $G = (V, E)$ has a $1$\nobreakdash-bounded order if and only if it is a forest, moreover, in any $1$\nobreakdash-bounded order there are exactly $|E|$ vertices with left-degree $1$ and $(|V| - |E|)$ vertices with left-degree~$0$.
Therefore, any $1$\nobreakdash-bounded order is both a dec-min and an inc-max solution, meaning that we can solve the problems in polynomial time.
Next, we prove that for $k = 2$, the dec-min and inc-max $k$\nobreakdash-bounded orderings also coincide.
We then analyze the computational complexity of finding a dec-min or inc-max $k$\nobreakdash-bounded order, and show that both problems are NP-hard for any $k \geq 2$ as well as in the unrestricted case (without bound on left-degrees).
Notably, the NP-hardness of finding a dec-min $k$\nobreakdash-bounded order was previously established for all odd $k \geq 5$ and also for the unbounded case~\cite{borradaile2017egalitarian}.
We strengthen this result by proving NP-hardness for all $k \geq 2$.

\begin{lemma}\label{lem:2BoundedDecMinIncMax}
  For every graph $G = (V, E)$, the dec-min and inc-max $2$\nobreakdash-bounded orders coincide.
\end{lemma}
\begin{proof}
  In any $2$\nobreakdash-bounded order, the left-degree of each vertex is $0$, $1$, or $2$, and the sum of the left-degrees is $|E|$.
  A dec-min ordering minimizes the number of vertices with left-degree $2$, while an inc-max ordering minimizes the number of vertices with left-degree $0$.

  Let $\sigma$ be a $2$\nobreakdash-bounded order, and let $n_i^{\sigma}$ denote the number of vertices with left-degree $i$ for $i = 0, 1, 2$.
  Since every vertex has left-degree at most $2$, by definition, $n_1^{\sigma} = |V| - n_0^{\sigma} - n_2^{\sigma}$ and the sum of the left-degrees is $|E| = n_1^{\sigma} + 2 n_2^{\sigma} = |V| - n_0^{\sigma} + n_2^{\sigma}$.
  Now, let $\sigma$ and $\sigma'$ be two $2$\nobreakdash-bounded orders.
  Since both have the same edge count, $|V| - n_0^{\sigma} + n_2^{\sigma} = |V| - n_0^{\sigma'} + n_2^{\sigma'}$, which simplifies to $n_2^{\sigma} - n_2^{\sigma'} = n_0^{\sigma} - n_0^{\sigma'}$.

  Thus, minimizing the number of vertices with left-degree $2$ is equivalent to minimizing the number of those with left-degree $0$.
  Therefore, the dec-min and inc-max $2$\nobreakdash-bounded orders coincide.
\end{proof}

Next, we prove the NP-hardness of the inc-max $k$\nobreakdash-bounded ordering problem for any $k \geq 2$.
\begin{theorem}\label{thm:sortedLexMaxNPC}
  For every integer $k \geq 2$, minimizing the number of vertices with left-degree $0$ in a $k$\nobreakdash-bounded order is NP-hard even for simple graphs.
\end{theorem}
\begin{proof}
  We prove the claim for graphs with parallel edges and then deduce the case of simple graphs.
  The proof is by reduction from the vertex cover problem in $3$\nobreakdash-regular simple graphs, which is known to be NP-hard~\cite{alimonti2000some}.
  In the vertex cover problem, the goal is to find a minimum-size subset $C \subseteq V$ such that every edge of $G$ has at least one endpoint in $C$.
  Given a $3$\nobreakdash-regular instance $G = (V, E)$ of the vertex cover problem, construct a graph $G' = (V', E')$ as follows.
  Let $H$ denote the gadget shown in Figure~\ref{fig:sortedLexMaxNPCGadget}, where the number of parallel edges depends on $k$.
  For each vertex $v \in V$, introduce a disjoint copy $H_v$ of $H$; for each edge $uv \in E$, add an edge to $G'$ between one of $t_1$, $t_2$, or $t_3$ in $H_u$ and one of $t_1$, $t_2$, or $t_3$ in $H_v$, ensuring that these vertices have exactly one incident non-gadget edge (illustrated with a wavy line) in every gadget.
  The size of the constructed graph is clearly polynomial in the size of the input to the vertex cover problem.

  \begin{figure}[t]
    \centering
    \begin{tikzpicture}[scale=0.8]
      \centering
      \tikzset{VertexStyle/.append style = {minimum size = 20pt,inner sep=0pt}}
      \SetVertexMath
      \Vertex[x=-6, y=3.5, L=t_1]{t1}
      \Vertex[x=0, y=3.5, L=t_2]{t2}
      \Vertex[x=6, y=3.5, L=t_3]{t3}
      \Vertex[x=0, y=1, L=t_4]{t4}
      \Vertex[x=-4, y=0, L=t_5]{t5}
      \Vertex[x=4, y=0, L=t_8]{t8}
      \Vertex[x=-5, y=-2, L=t_7]{t7}
      \Vertex[x=-3, y=-2, L=t_6]{t6}
      \Vertex[x=3, y=-2, L=t_9]{t9}
      \Vertex[x=5, y=-2, L=t_{10}]{t10}
      \Vertex[x=-6, y=-3.5, L=t_{11}]{t11}
      \Vertex[x=-2, y=-3.5, L=t_{12}]{t12}
      \Vertex[x=2, y=-3.5, L=t_{13}]{t13}
      \Vertex[x=6, y=-3.5, L=t_{14}]{t14}

      \Vertex[x=-6, y=5, empty]{u}
      \Vertex[x=0, y=5, empty]{v}
      \Vertex[x=6, y=5, empty]{w}

      \draw[wavy] (t1) to (u);
      \draw[wavy] (t2) to (v);
      \draw[wavy] (t3) to (w);

      \draw (t1) to[paralleledge=3] (t4);
      \path[circleAroundEdges={.09}{.19}{0.5}] (t4)--(t1) node [above=.25cm, midway] {$k\!-\!1$};
      \draw (t2) to[paralleledge=3] (t4);
      \path[circleAroundEdges={.09}{.20}{0.5}] (t2)--(t4) node [left=.2cm, midway] {$k\!-\!1$};
      \draw (t3) to[paralleledge=3] (t4);
      \path[circleAroundEdges={.09}{.19}{0.5}] (t4)--(t3) node [above=.25cm, midway] {$k\!-\!1$};

      \draw (t4) to[paralleledge=3] (t5);
      \path[circleAroundEdges={.09}{.19}{0.5}] (t4)--(t5) node [below=.15cm, midway] {$\left\lfloor\frac{k}{2}\right\rfloor$};
      \draw (t4) to[paralleledge=3] (t8);
      \path[circleAroundEdges={.09}{.19}{0.5}] (t4)--(t8) node [below=.15cm, midway] {$\left\lfloor\frac{k}{2}\right\rfloor$};

      \draw (t5) to[paralleledge=3] (t7);
      \path[circleAroundEdges={.09}{.20}{0.5}] (t5)--(t7) node [left=.15cm, midway] {$\left\lceil\frac{k}{2}\right\rceil$};
      \draw (t5) to[paralleledge=3] (t6);
      \path[circleAroundEdges={.09}{.20}{0.5}] (t5)--(t6) node [right=.15cm, midway] {$\left\lfloor\frac{k}{2}\right\rfloor$};
      \draw (t7) to[paralleledge=3] (t6);
      \path[circleAroundEdges={.09}{.20}{0.5}] (t6)--(t7) node [below=.2cm, midway] {$k$};
      \draw (t7) -- (t11);

      \draw (t8) to[paralleledge=3] (t9);
      \path[circleAroundEdges={.09}{.20}{0.5}] (t8)--(t9) node [left=.15cm, midway] {$\left\lfloor\frac{k}{2}\right\rfloor$};
      \draw (t8) to[paralleledge=3] (t10);
      \path[circleAroundEdges={.09}{.20}{0.5}] (t8)--(t10) node [right=.15cm, midway] {$\left\lceil\frac{k}{2}\right\rceil$};
      \draw (t9) to[paralleledge=3] (t10);
      \path[circleAroundEdges={.09}{.20}{0.5}] (t9)--(t10) node [below=.2cm, midway] {$k$};
      \draw (t10) -- (t14);

      \draw (t11) -- (t12);
      \draw (t12) -- (t13);
      \draw (t13) -- (t14);

      \draw (t1) to[paralleledge=1] (t11);
      \draw (t2) to[paralleledge=1] (t13);
      \draw (t3) to[paralleledge=1] (t14);

    \end{tikzpicture}
    \caption{The gadgets corresponding to $v\in V$ for the proof of Theorem~\ref{thm:sortedLexMaxNPC}.}\label{fig:sortedLexMaxNPCGadget}
  \end{figure}


  We show that the minimum number of vertices with left-degree $0$ in a $k$\nobreakdash-bounded order of $G'$ equals the size of a minimum vertex cover in $G$ plus $|V|$.

  First, suppose $X \subseteq V$ is a minimum vertex cover in $G$.
  We construct a $k$\nobreakdash-bounded order for $G'$ with exactly $(|X| + |V|)$ vertices having left-degree $0$.
  For each $v \in X$, order $H_v$ as $t_{12}, t_{11}, t_{13}, t_{14}, t_1, t_2, \dots, t_{10}$, then, for each $u \in V \setminus X$, order $H_u$ as $t_1, t_2, \dots, t_{14}$.
  In this order, the left-degree of every vertex is at most $k$, and exactly two vertices in each $H_v$ for $v \in X$ (namely $t_7$ and $t_{10}$) and one vertex in each $H_u$ for $u \in V\setminus X$ (namely $t_{14}$) have left-degree $0$.

  Conversely, suppose we are given a $k$\nobreakdash-bounded order of $G'$.
  Define $X$ as the set of those vertices $v \in V$ for which $H_v$ contains at least two vertices with left-degree $0$ in the order.
  We prove that $X$ is a vertex cover by showing that if $u'$ precedes $v'$ in the $k$\nobreakdash-bounded order for some edge $uv \in E$, then $v$ must be in $X$, where $u'v'$ is the unique non-gadget (wavy) edge between $H_u$ and $H_v$.
  Here, $u' \in \{t_1, t_2, t_3\} \subset H_u$ and $v' \in \{t_1, t_2, t_3\} \subset H_v$.
  Consider the ordering of vertices in $H_v$.
  In any $k$\nobreakdash-bounded order, the last vertex of the $3$\nobreakdash-cycle $t_5 t_6 t_7$ must be $t_5$, and the last vertex of $t_8 t_9 t_{10}$ must be $t_8$.
  Both must precede $t_4$, which in turn must precede $t_1$, $t_2$, and $t_3$ --- one of which is $v'$.
  If $v'$ is $t_1$, then it must precede $t_{11}$, otherwise, all of its neighbors would precede $v'$, exceeding left-degree $k$.
  Similarly, if $v'$ is $t_2$, then it must precede $t_{13}$; if $v'$ is $t_3$, then it must precede $t_{14}$.
  This means that $t_1$, $t_2$, $t_3$, $t_4$ and at least one of $t_{11}$, $t_{12}$, $t_{13}$, and $t_{14}$ all succeed the $3$\nobreakdash-cycles $t_5t_6t_7$ and $t_8t_9t_{10}$.
  Thus, the first $9$ vertices of the gadget $H_v$ induces a subgraph of $H_v\setminus \{t_1, t_2, t_3, t_4\}$ that excludes at least one of $t_{11}$, $t_{12}$, $t_{13}$ and $t_{14}$, therefore, it is not connected.
  Clearly, the first vertex of each connected component has left-degree $0$, hence $H_v$ contains at least two vertices with left-degree $0$, thus $v\in X$.
  Finally, note that in any $k$\nobreakdash-bounded order, the first vertex of each gadget $H_u$ must have left-degree $0$, because only $t_1$, $t_2$, and $t_3$ have a non-gadget incident edge, and none of these can be first (otherwise, $t_4$ or the last vertex of the $3$\nobreakdash-cycle $t_5t_6t_7$ would exceed left-degree $k$).
  This proves that $X$ is a vertex cover, and its size satisfies the desired bound, which completes the proof of the theorem for graphs with parallel~edges.

  \medskip
  Now we modify the construction so that the resulting graph becomes simple.
  For any $p \in \left\{ \left\lfloor \frac{k}{2} \right\rfloor, \left\lceil \frac{k}{2} \right\rceil, k - 1, k \right\}$ and any pair of distinct vertices $t_i, t_j$ of $H_v$, if the gadget contains $p$ parallel edges between $t_i$ and $t_j$, then we replace these edges as follows.
  We introduce a complete bipartite graph $K_{k+1, k+1} = (L, R; E'')$.
  We connect $t_i$ to $p$ distinct vertices of $L$ and $t_j$ to $p$ distinct vertices of $R$.
  Then, we remove a matching of size $p$ among these $2p$ vertices from the bipartite graph.
  This replacement ensures that the new subgraph is connected, the degrees of $t_i$ and $t_j$ remain unchanged, and every new vertex has degree $(k+1)$.
  Thus, in any $k$\nobreakdash-bounded order, the last vertex among the new vertices and $t_i, t_j$ must be either $t_i$ or $t_j$.
  Furthermore, we can place the new vertices replacing the parallel edges between $t_i$ and $t_j$ in any order of the original vertices of the gadget without increasing the number of vertices with left-degree zero.
  By symmetry, assume $t_i$ precedes $t_j$ in the order.
  Insert the new vertices after $t_i$ and before $t_j$ in the following order.
  First, list the vertices of $L$ adjacent to $t_i$, then the vertices of $R$ not adjacent to $t_j$, followed by the remaining vertices of $L$, and finally the remaining vertices of $R$.
  This shows that from any order of the graph, we can derive a corresponding order of the modified simple graph without increasing the number of vertices with left-degree zero.

  For the reverse direction, consider an order for the modified graph and remove the new vertices.
  We claim that this removal does not increase the number of vertices with left-degree zero.
  Indeed, for any pair $t_i, t_j$, if the degree of $t_i$ drops to zero upon removal, then $t_i$ preceded $t_j$ and at least one of the new vertices replacing the parallel edges between $t_i$ and $t_j$ preceded $t_i$ in the order.
  The first such new vertex must have left-degree zero.
  Therefore, for each original vertex $t_i$ whose left-degree becomes zero, we account for at least one removed new vertex with left-degree zero.

  This establishes that any optimal order for the modified simple graph corresponds to an optimal order for the original graph, completing the proof.
\end{proof}

Since an inc-max $k$\nobreakdash-bounded order minimizes the number of vertices with left-degree $0$, the previous theorem directly yields the following corollary.
\begin{corollary}\label{cor:boundedIncMaxNPC}
  For every integer $k \geq 2$, finding an inc-max $k$\nobreakdash-bounded order is NP-hard even for simple graphs.
  \FBOX
\end{corollary}

\begin{figure}[t]
  \centering
  \begin{tikzpicture}[scale=1]
    \centering
    \tikzset{VertexStyle/.append style = {minimum size = 20pt,inner sep=0pt}}
    \SetVertexMath
    \Vertex[x=-1.5, y=2, L=t_1]{t1}
    \Vertex[x=-1.5, y=0, L=t_2]{t2}
    \Vertex[x=-1.5, y=-2, L=t_3]{t3}
    \Vertex[x=1.5, y=-2, L=t_4]{t4}
    \Vertex[x=1.5, y=0, L=t_5]{t5}
    \Vertex[x=1.5, y=2, L=t_6]{t6}

    \Vertex[x=-1.5, y=3.5, empty]{t7}
    \Vertex[x=.5, y=3.5, empty]{t8}
    \Vertex[x=2.5, y=3.5, empty]{t9}

    \draw[wavy] (t1) to (t7);
    \draw[wavy] (t6) to (t8);
    \draw[wavy] (t6) to (t9);

    \draw (t3) to[paralleledge=1] (t4);

    \draw (t1) to[paralleledge=3] (t2);
    \path[circleAroundEdges={.1136}{.25}{.5}] (t1)--(t2) node [left=.25cm, midway] {$\left\lfloor\frac{k-1}{2}\right\rfloor$};
    \draw (t2) to[paralleledge=3] (t3);
    \path[circleAroundEdges={.1136}{.25}{.5}] (t2)--(t3) node [left=.25cm, midway] {$\left\lfloor\frac{k-1}{2}\right\rfloor$};
    \draw (t5) to[paralleledge=3] (t4);
    \path[circleAroundEdges={.1136}{.25}{.5}] (t5)--(t4) node [right=.25cm, midway] {$\left\lfloor\frac{k-1}{2}\right\rfloor$};
    \draw (t6) to[paralleledge=3] (t5);
    \path[circleAroundEdges={.1136}{.25}{.5}] (t6)--(t5) node [right=.25cm, midway] {$\left\lfloor\frac{k-1}{2}\right\rfloor$};

    \path[circleAroundEdges={.09}{.2}{1/4}] (t1)--(t5) node [above=.3cm, pos=1/3] {$\left\lfloor\frac{k}{2}\right\rfloor$};
    \draw (t1) to[paralleledge=3] (t5);
    \path[circleAroundEdges={.09}{.2}{1/4}] (t6)--(t2) node [above=.3cm, pos=1/3] {$\left\lfloor\frac{k}{2}\right\rfloor$};
    \draw (t6) to[paralleledge=3] (t2);
    \path[circleAroundEdges={.09}{.2}{1/4}] (t2)--(t4) node [above=.3cm, pos=1/3] {$\left\lfloor\frac{k}{2}\right\rfloor$};
    \draw (t2) to[paralleledge=3] (t4);
    \path[circleAroundEdges={.09}{.2}{1/4}] (t5)--(t3) node [above=.3cm, pos=1/3] {$\left\lfloor\frac{k}{2}\right\rfloor$};
    \draw (t5) to[paralleledge=3] (t3);
  \end{tikzpicture}
  \caption{The gadgets corresponding to $v\in V$ for the proof of Theorem~\ref{thm:sortedLexMinNPC}.}\label{fig:sortedLexMinNPCGadget}
\end{figure}

We now turn to the complexity of the dec-min $k$\nobreakdash-bounded ordering problem.
This problem was previously shown to be NP-hard for all odd $k \geq 5$ in~\cite{borradaile2017egalitarian}, which we strengthen by proving NP-hardness for all integers $k \geq 2$.
\begin{theorem}\label{thm:sortedLexMinNPC}
  For every integer $k \geq 2$, minimizing the number of vertices with left-degree exactly $k$ in a $k$\nobreakdash-bounded order is NP-hard even for simple graphs.
\end{theorem}
\begin{proof}
  The case $k = 2$ follows from Lemma~\ref{lem:2BoundedDecMinIncMax} and Corollary~\ref{cor:boundedIncMaxNPC}, thus we assume that $k \geq 3$.
  We first prove the theorem for graphs with parallel edges, and then show how to make the graph simple.
  We give a reduction from the vertex cover problem on $3$\nobreakdash-regular simple graphs~\cite{alimonti2000some}, similar to the proof of Theorem~\ref{thm:sortedLexMaxNPC}.

  Given a $3$\nobreakdash-regular instance $G = (V, E)$ of vertex cover, we construct a graph $G' = (V', E')$ as follows.
  For each $v \in V$, introduce a disjoint copy $H_v$ of the gadget shown in Figure~\ref{fig:sortedLexMinNPCGadget}.
  For each edge $uv \in E$, add a non-gadget (wavy) edge between $t_1 \in H_u$ or $t_6 \in H_u$ and $t_1 \in H_v$ or $t_6 \in H_v$, ensuring that within each gadget exactly one and two such edges are incident to $t_1$ and $t_6$, respectively.
  These non-gadget edges correspond to the edges of $G$.
  The size of the constructed graph is clearly polynomial in the size of the input to the vertex cover problem.

  We show that the size of a minimum vertex cover $X \subseteq V$ in $G$ equals the minimum number of vertices with left-degree $k$ in a $k$\nobreakdash-bounded order of $G'$.

  First, suppose $X$ is a minimum vertex cover.
  We construct a $k$\nobreakdash-bounded order of $G'$ where at most $|X|$ vertices have left-degree $k$.
  For each $v \in X$, remove $t_3$ from $H_v$ in $G'$, and recursively remove vertices of degree smaller than $k$.
  When a vertex is removed, place it in the next rightmost position of the order.
  We argue that this process removes all vertices, and thus yields a valid order of $V'$.
  Indeed, if $v \in X$, then removing $t_3$ eliminates $H_v$.
  If $u \in V\setminus X$, then for each edge $uv \in E$, some $v \in X$ ensures that $H_v$ is removed, removing all non-gadget edges incident to $H_u$ and triggering the removal of $H_u$.
  Thus all vertices are removed, forming a $k$\nobreakdash-bounded order.
  The only vertices with left-degree $k$ arise from removing $t_3$ in some $H_v$ with $v \in X$, so at most $|X|$ such vertices appear.

  Conversely, suppose we have a $k$\nobreakdash-bounded order of $G'$.
  We construct a vertex cover whose size matches the number of vertices with left-degree $k$.
  Define $X$ as the set of vertices $v \in V$ for which $H_v$ contains a vertex of left-degree $k$ in the given order.
  We show that $X$ is a vertex cover.
  Suppose for contradiction that some edge $uv \in E$ is not covered by $X$.
  Consider the non-gadget edge $u'v'$ joining $H_u$ and $H_v$.
  Since $u \in V\setminus X$, no vertex in $H_u$ has left-degree $k$.
  By the construction of the gadget, the last two vertices among $t_1, \dots, t_6$ in $H_u$ must be $t_1$ and $t_6$, or else a vertex of $H_u$ would have left-degree at least $k$.
  For $t_1$ and $t_6$ to avoid left-degree $k$, all their non-gadget neighbors (including $v'$) must appear later in the order, so $u'$ precedes $v'$.
  By symmetry, applying the same reasoning to $v \in V\setminus X$ forces $v'$ to precede $u'$, which is a contradiction.
  Therefore, $X$ is a vertex cover, and its size equals the number of vertices with left-degree $k$.
  This completes the proof for graphs with parallel edges.

  \medskip
  Now we modify the construction to obtain a simple graph.
  For any $p \in \left\{ \left\lfloor \frac{k-1}{2} \right\rfloor, \left\lfloor \frac{k}{2} \right\rfloor \right\}$ and any pair of distinct vertices $t_i, t_j \in H_v$, if the gadget in Figure~\ref{fig:sortedLexMinNPCGadget} contains $p$ parallel $t_i t_j$ edges, then we replace these edges with the following graph.
  Take a clique $K_{k+1}$, choose $2p$ distinct vertices, connect $p$ of them to $t_i$, the remaining $p$ to $t_j$, and remove a matching of size $p$ among these $2p$ vertices within the clique.

  The resulting replacement is connected, preserves the degrees of the original vertices of the gadgets, and ensures that every new vertex has degree $k$.

  We show that an optimal order for this modified graph contains the same number of vertices with left-degree $k$ as an optimal order for the original graph containing parallel edges.
  Given an optimal order for the modified graph, removing the new vertices does not increase the number of vertices with left-degree $k$.
  Conversely, starting from an optimal order for the original graph, we can insert the new vertices between the corresponding $t_i$ and $t_j$ as follows (assuming $t_i$ precedes $t_j$; otherwise, swap their roles).
  First insert the vertices adjacent to $t_i$, then those not adjacent to $t_j$, and finally those adjacent to $t_j$.

  In this ordering, no new vertex attains left-degree $k$, and the left-degrees of the original vertices remain unchanged.
  Thus, the problem remains NP-hard even for simple graphs, completing the proof.
\end{proof}

Since dec-min $k$\nobreakdash-bounded orders minimize the number of vertices with left-degree exactly $k$, Theorem~\ref{thm:sortedLexMinNPC} directly implies that finding a dec-min $k$\nobreakdash-bounded order is NP-hard for all $k \geq 2$.
\begin{corollary}\label{cor:decMinNPC}
  For every integer $k \geq 2$, finding a dec-min $k$\nobreakdash-bounded order is NP-hard even for simple graphs.
  \FBOX
\end{corollary}

Next, we show that finding an inc-max order without any bound on the left-degrees is NP-hard.
Unlike Theorem~\ref{thm:sortedLexMinNPC}, which proves NP-hardness for dec-min orders, Theorem~\ref{thm:sortedLexMaxNPC} does not directly determine the complexity of the inc-max problem in the unbounded case.
The reason is that a graph may admit a $k$\nobreakdash-bounded order, yet the optimal inc-max order may not be $k$\nobreakdash-bounded.
An example illustrating this distinction is provided in Figure~\ref{fig:decMinIncMaxDifference}.
Now we prove the complexity of inc-max ordering.

\begin{theorem}\label{thm:incMaxNPC}
  Finding an inc-max order is NP-hard even for simple graphs.
\end{theorem}
\begin{proof}
  We prove NP-hardness by reduction from the MAX\nobreakdash-$3$\nobreakdash-SAT\nobreakdash-$4$ problem.
  Here the input is a Boolean formula in conjunctive normal form where each variable appears in exactly four clauses and each clause contains exactly three distinct literals.
  The task is to find an assignment that maximizes the number of satisfied clauses.
  This problem is APX-hard with an approximation threshold of $\frac{1900}{1899}$, that is, no polynomial-time algorithm can achieve a better approximation ratio unless $\mathrm{P} = \mathrm{NP}$~\cite{berman2003approximation}.

  An inc-max order of a simple graph maximizes the largest integer $\ell$ such that there exists exactly one vertex with left-degree $i$ for each $i \in \{0, \dots, \ell - 1\}$.
  We show that if such an order could be computed in polynomial time, then it would lead to an approximation ratio better than the threshold for MAX\nobreakdash-$3$\nobreakdash-SAT\nobreakdash-$4$, thereby implying NP-hardness.

  Given an instance of MAX\nobreakdash-$3$\nobreakdash-SAT\nobreakdash-$4$ with $m$ clauses, we construct a graph $G = (V, E)$ as follows.
  For each clause $j \in \{1, \dots, m\}$, and for each literal position $r \in \{1, 2, 3\}$ within the clause, introduce a vertex $c_r^j$ corresponding to the $r^{\text{th}}$ literal of clause $j$.
  We add an edge between $c_{r_1}^{j_1}$ and $c_{r_2}^{j_2}$ whenever $j_1, j_2$ are distinct and the literals $r_1,r_2$ are not negations of each other.
  The size of the constructed graph is clearly polynomial in the size of the input to the MAX\nobreakdash-$3$\nobreakdash-SAT\nobreakdash-$4$ problem.

  Let $q$ denote the maximum number of satisfiable clauses in the formula, and let $\ell$ denote the largest integer such that there exists an order of $V$ where exactly one vertex has left-degree $i$ for each $i \in \{0, \dots, \ell - 1\}$.
  We prove that $q - 4 \leq \ell \leq q$.

  For the upper bound, suppose that we are given an order of the vertices such that there is a unique vertex $v_i$ of left-degree $i$ for each $i\in\{0, \dots, \ell - 1\}$.
  All remaining vertices in $V \setminus \{v_0, \dots, v_{\ell-1}\}$ must have left-degree at least $\ell$, and since $G$ is simple, these vertices must each have at least $\ell$ preceding neighbors.
  Thus, $v_0, \dots, v_{\ell-1}$ are the first $\ell$ vertices in the order and they form a clique, implying that they correspond to literals from distinct clauses that are not negations of each other.
  Assigning these literals to be true satisfies at least $\ell$ clauses, yielding~$\ell \leq q$.

  For the lower bound $q-4 \leq \ell$, consider an optimal truth assignment satisfying exactly $q$ clauses.
  Choose one true literal per satisfied clause, yielding vertices $v_0, \dots, v_{q-1}$.
  These vertices form a clique, as they come from different clauses, and are not negations of one another.
  An arbitrary order beginning with the vertices $v_0,\dots,v_{q-1}$ yields left-degrees $0,\dots, (q-1)$, respectively.
  Any remaining vertex $v \in V \setminus \{v_0, \dots, v_{q-1}\}$ has at least $(q - 4)$ neighbors among $v_0, \dots, v_{q-1}$, since at most one literal from the same clause and three negations can be non-neighbors and hence excluded.
  Consequently, an order exists where exactly one vertex has each left-degree up to $(q - 5)$, ensuring $q - 4 \leq \ell$.

  Consequently, determining $\ell$ gives a $\frac{q}{q - 4}$\nobreakdash-approximation of $q$.
  If $q \leq 7600$, then we can solve MAX\nobreakdash-$3$\nobreakdash-SAT\nobreakdash-$4$ exactly in polynomial time.
  Otherwise, $\frac{q}{q - 4} < \frac{1900}{1899}$, which is below the approximability threshold, thus, finding an inc-max order is NP-hard.
\end{proof}

We have established that both the dec-min and inc-max acyclic orientation problems are NP-hard.
We now introduce two natural counterparts: the \emph{increasingly minimal (inc-min)} and \emph{decreasingly maximal (dec-max)} orientation problems.
The inc-min (acyclic) orientation problem asks for an (acyclic) orientation in which the indegree vector sorted in non-decreasing order is lexicographically minimal.
Similarly, in the dec-max (acyclic) orientation problem, the indegree vector sorted in non-increasing order is to be lexicographically maximized.
We prove that both the inc-min and dec-max problems are NP-hard, regardless of whether the orientation is required to be acyclic.

\begin{theorem}\label{thm:incMinNPC}
  The inc-min orientation and inc-min acyclic orientation problems are NP-hard even for $3$\nobreakdash-regular simple graphs.
\end{theorem}
\begin{proof}
  We prove the NP-hardness of both problems simultaneously by reduction from the maximum independent set problem, which is known to be NP-hard for $3$\nobreakdash-regular simple graphs~\cite{alimonti2000some}.
  In the maximum independent set problem, the goal is to find a largest possible subset $I \subseteq V$ such that no two vertices in $I$ are adjacent.

  In an optimal inc-min (acyclic) orientation, the number of vertices with indegree zero is maximized.
  Let $n_0$ and $n_0^{\text{DAG}}$ denote the maximum number of zero-indegree vertices achievable in any orientation and in any acyclic orientation, respectively.
  Let $G = (V, E)$ be a given $3$\nobreakdash-regular simple graph, and let $k$ denote the size of a maximum independent set in $G$.
  We prove that $n_0 = n_0^{\text{DAG}} = k$.

  Clearly, $n_0 \geq n_0^{\text{DAG}}$.
  To see that $n_0^{\text{DAG}} \geq k$, take any independent set of size $k$ and place its vertices first in an ordering.
  Orient all edges from left to right, and thus obtain an acyclic orientation in which all members of the independent set have indegree zero.
  Next, observe that $n_0 \leq k$ since any set of zero-indegree vertices must be independent.

  Combining $n_0 \geq n_0^{\text{DAG}} \geq k$ and $n_0 \leq k$ gives $n_0 = n_0^{\text{DAG}} = k$.
  Therefore, solving the inc-min (acyclic) orientation problem would allow us to compute a maximum independent set.
\end{proof}

In the case of $3$\nobreakdash-regular simple graphs, an orientation is an inc-min acyclic orientation if and only if its reversal is a dec-max acyclic orientation.
Reversing an orientation preserves acyclicity and transforms the indegree $\varrho(v)$ into $(3 - \varrho(v))$, thereby leading directly to the following corollary of Theorem~\ref{thm:incMinNPC}.
\begin{corollary}\label{cor:decMaxNPC}
  The dec-max orientation and dec-max acyclic orientation problems are NP-hard even for $3$\nobreakdash-regular simple graphs.
  \FBOX
\end{corollary}


Table~\ref{tab:lexComplexities} summarizes the complexities of the considered orientation and acyclic orientation problems.

\setlength{\tabcolsep}{6pt}
\renewcommand{\arraystretch}{1.5}
\newcolumntype{L}{>{\centering\arraybackslash}m{90pt}}
\newcolumntype{M}{>{\centering\arraybackslash}m{60pt}}
\begin{table}[h]
  \centering
  \begin{tabular}{|L|M|M|M|M|}
    \hline
    & dec-min & inc-max & inc-min & dec-max \\
    \hline
    orientation
    & in P~\cite{borradaile2017egalitarian}
              & in P~\cite{frank2022decreasing2,frank2022decreasing1}
                        & \multirow{3}{*}{\makecell[c]{NP-hard,\\ Thm~\ref{thm:incMinNPC}}}
    & \multirow{3}{*}{\makecell[c]{NP-hard,\\ Cor~\ref{cor:decMaxNPC}}} \\
    \cline{1-3}
    \multirow{2}{*}{acyclic orientation}
    & \multirow{2}{*}{\makecell[c]{NP-hard~\cite{borradaile2017egalitarian},\\ Cor~\ref{cor:decMinNPC}}}
    & \multirow{2}{*}{\makecell[c]{NP-hard,\\ Thm~\ref{thm:incMaxNPC}}}
    &  &  \\
    &  &  &  &  \\
    \hline
  \end{tabular}
  \caption{Complexities of lexicographical orientation problems without indegree bounds.}
  \label{tab:lexComplexities}
\end{table}

\subsubsection{A relaxation: minimizing the maximum (weighted) indegree}\label{sec:decMinIncMax:relax}
The first step toward finding a dec-min orientation is minimizing the maximum indegree.
This section focuses on this problem, which has been studied extensively in the context of unconstrained (possibly cyclic) orientations~\cite{asahiro2007graph, de1995regular, venkateswaran2004minimizing}.

A widely used approach begins with an arbitrary orientation and iteratively reverses directed paths to reduce the indegree of a vertex with highest indegree.
In a natural generalization, we consider edge-weighted graphs: given a graph $G = (V, E)$ with a non-negative weight function $w : E \to \R_+$, our goal is to find an orientation of the edges that minimizes the maximum weighted indegree across all vertices.
Unlike the unweighted case, this problem is weakly NP-hard even for simple bipartite planar graphs~\cite{asahiro2007graph}.
Furthermore, this result was strengthened in~\cite{asahiro2011approximation}, which shows NP-hardness even when the weight function is restricted to values in $\{1, k\}$ for some fixed $k \geq 2$.

We now turn to the problem of minimizing the maximum (weighted) indegree under the additional constraint that the orientation is acyclic.
As discussed in Section~\ref{sec:introduction}, this is equivalent to finding a vertex ordering that minimizes the maximum (weighted) left-degree.

For unweighted graphs, this problem is closely related to the concept of \emph{graph degeneracy}, introduced in~\cite{lick1970k}.
A graph $G = (V, E)$ is called $k$\nobreakdash-degenerate if every induced subgraph $G[V']$ contains a vertex in $V'$ of degree at most $k$.
The degeneracy number $\cev{d}_{\min}(G)$ is the smallest such~$k$.
It is well known that $G$ is $k$\nobreakdash-degenerate if and only if there exists an ordering of its vertices in which each vertex has at most $k$ left-neighbors --- often referred to as a \emph{$k$\nobreakdash-bounded order}.
This directly implies that $G$ admits an acyclic orientation with maximum indegree at most $k$ if and only if it is $k$\nobreakdash-degenerate.
A linear-time algorithm for computing $\cev{d}_{\min}(G)$ was given in~\cite{matula1983smallest}.
It constructs a $\cev{d}_{\min}(G)$\nobreakdash-bounded order by repeatedly removing a vertex of minimum degree from the current subgraph and placing it at the last available position.
Orienting the edges from earlier to later vertices yields an acyclic orientation that minimizes the maximum indegree.

We now extend this approach to the weighted setting.
Given a graph $G = (V, E)$ with a non-negative weight function $w : E \to \R_+$, our goal is to find an acyclic orientation that minimizes the maximum weighted indegree.
Equivalently, we seek a vertex ordering that minimizes the maximum weighted left-degree.
Algorithm~\ref{alg:weigthedMinMaxIndegreeAcyclic} solves this problem in strongly polynomial time.

\begin{algorithm}
  \caption{\hspace{0.5cm}\textsc{Weighted smallest-last ordering}}\label{alg:weigthedMinMaxIndegreeAcyclic}
  \begin{algorithmic}[1]
    \Input A graph $G = (V, E)$ with a weight function $w : E \to \R_+$.
    \Output A vertex order of $G$ that minimizes $\max \{\cev{d}^w(v) : v \in V\}$.
    \State $V' \coloneqq V$; $n \coloneqq |V|$
    \State Let $\sigma_1, \dots, \sigma_n$ denote the desired vertex order.
    \For {$i = n, \dots, 1$}
    \State Choose $\sigma_i \in \argmin \{d^w_{G[V']}(v) : v \in V'\}$.\label{alg:line:select}
    \State $V' \coloneqq V' \setminus \{\sigma_i\}$
    \EndFor
    \State \Return $\sigma_1, \dots, \sigma_n$
  \end{algorithmic}
\end{algorithm}
\FloatBarrier

The algorithm constructs the vertex order from right to left.
At each step, it selects a vertex with the smallest weighted degree in the current subgraph and places it at the last available position.

We now show the correctness of this algorithm.

\begin{theorem}\label{thm:weigthedMinMaxIndegreeAcyclic}
  Given a graph $G = (V,E)$ with a non-negative weight function $w: E \to \R_+$, Algorithm~\ref{alg:weigthedMinMaxIndegreeAcyclic} produces a vertex ordering that minimizes the maximum weighted left-degree.
\end{theorem}
\begin{proof}
  Let $\sigma = (\sigma_1, \dots, \sigma_n)$ be the order returned by the algorithm, and let $\sigma' = (\sigma'_1, \dots, \sigma'_n)$ be an arbitrary vertex order.
  Take an index $i$ such that $\sigma_i$ is a vertex with maximum weighted left-degree in $\sigma$.
  Let $\sigma'_j$ be the last vertex in $\sigma'$ that appears among ${\sigma_1, \dots, \sigma_i}$.
  Then
  \[
    \cev{d}_{\sigma}^w(\sigma_i) = d^w(\sigma_i, \{\sigma_1, \dots, \sigma_i\}) \leq d^w(\sigma'_j, \{\sigma_1, \dots, \sigma_i\}) \leq d^w(\sigma'_j, \{\sigma'_1, \dots, \sigma'_j\}) = \cev{d}_{\sigma'}^w(\sigma'_j).
  \]
  The first and last equations follow by definition of the weighted left-degree.
  The first inequality holds because $\sigma'_j$ belongs to $\{\sigma_1, \dots, \sigma_i\}$, and $\sigma_i$ was selected as the vertex of minimum weighted degree in this set.
  The second inequality follows from the non-negativity of $w$ and the inclusion $\{\sigma_1, \dots, \sigma_i\} \subseteq \{\sigma'_1, \dots, \sigma'_j\}$.

  This clearly shows that the maximum weighted left-degree in $\sigma$ is no larger than in $\sigma'$, which completes the proof.
\end{proof}

Hence, we can compute an acyclic orientation that minimizes the maximum weighted indegree in strongly polynomial time.
This stands in contrast to the unconstrained (possibly cyclic) case, where the same problem is NP-hard.

The complexity results discussed in this section are summarized in Table~\ref{tab:minMaxIndegreeComplexities}.

\setlength{\tabcolsep}{6pt}
\renewcommand{\arraystretch}{1.5}
\begin{table}[h]
  \centering
  \begin{tabular}{|c|c|c|}
    \hline
    & min max $\varrho$ & min max $\varrho^w$  \\
    \hline
    orientation & in P~\cite{asahiro2007graph, de1995regular, venkateswaran2004minimizing} & NP-hard~\cite{asahiro2011approximation, asahiro2007graph} \\
    \hline
    acyclic orientation & in P~\cite{matula1983smallest} & in P, Thm~\ref{thm:weigthedMinMaxIndegreeAcyclic}\\
    \hline
  \end{tabular}
  \caption{Complexities of finding an orientation that minimizes the maximum (weighted) indegree}\label{tab:minMaxIndegreeComplexities}
\end{table}

It may seem natural to consider the analogous first step of the inc-max problem, that is, finding an orientation which maximizes the minimum (weighted) indegree.
However, this is not meaningful in the acyclic case, since every acyclic orientation necessarily includes at least one source vertex with indegree zero.


\subsection{An exponential dynamic programming algorithm}\label{sec:DinProg}
In Section~\ref{sec:minSumHComplexity}, we proved that it is NP-hard to find a vertex order that minimizes $\sum_{v \in V} \varphi(\cev{d}(v))$ for any discrete strictly convex function $\varphi : \Z_+ \to \R$.
In this section, we provide an exact method for solving the following more general problem.

Let $G = (V, E)$ be a graph, and let $\varphi_v : \Z_+ \to \R$ be a discrete (not necessarily convex) function given for each $v \in V$, which can be evaluated in $O(\poly(|V|, |E|))$ time.
Our goal is to find an order that minimizes $\sum_{v \in V} \varphi_v(\cev{d}(v))$.
The natural brute-force approach is to try all $|V|!$ permutations of the vertices and select one that minimizes the objective value.
Now we present a dynamic programming algorithm that finds an order that minimizes $\sum_{v \in V} \varphi_v(\cev{d}(v))$ in $O(2^{|V|} \poly(|V|, |E|))$ steps.

Let $f(\emptyset) = 0$.
For each non-empty $V' \subseteq V$, in increasing order by $|V'|$, compute and memoize
\begin{equation}\label{eq:defF}
  f(V') = \min_{v \in V'} \left\{ f(V' \setminus \{v\}) + \varphi_v(d(v, V')) \right\},
\end{equation}
and choose
\begin{equation}\label{eq:defG}
  g(V') \in \argmin_{v \in V'} \left\{ f(V' \setminus \{v\}) + \varphi_v(d(v, V')) \right\}.
\end{equation}

Afterwards, construct the optimal order by repeating the following step until no vertices remain: place $g(V')$ to the last free position, where $V'$ is the current vertex set, and remove $g(V')$ from the graph.

We prove correctness by showing that for each $V' \subseteq V$, $f(V')$ gives the optimal value and $g(V')$ can be chosen as the last vertex in an optimal order for $G[V']$.
\begin{theorem}
  For each non-empty subset $V' \subseteq V$, the minimum value of $\sum_{v \in V'} \varphi_v(\cev{d}(v))$ for the graph $G[V']$ is $f(V')$ as defined in~\eqref{eq:defF}, and there exists an optimal order in which the last vertex is $g(V')$ as defined in~\eqref{eq:defG}.
\end{theorem}
\begin{proof}
  We proceed by induction on $|V'|$.
  Suppose the claim holds for all subsets of size smaller than $|V'|$.
  Consider an optimal order $\sigma = v_1, \dots, v_n$ for $G[V']$.
  Then $v_1, \dots, v_{n-1}$ is an optimal order for $G[V' \setminus \{v_n\}]$, so its objective value is $f(V' \setminus \{v_n\})$ by induction.
  Since $v_n$ is last, $\varphi_{v_n}(\cev{d}(v_n)) = \varphi_{v_n}(d(v_n, V'))$.
  Thus the objective value of $\sigma$ is $\left(f(V' \setminus \{v_n\}) + \varphi_{v_n}(d(v_n, V'))\right)$.
  This shows that $f(V')$ is at most the optimal value.
  Conversely, we construct an order for $G$ with objective value $f(V')$.
  Let $v_n \in g(V')$.
  By induction, $f(V' \setminus \{v_n\})$ is the optimal value for $G[V' \setminus \{v_n\}]$.
  Appending $v_n$ to an optimal order for $G[V' \setminus \{v_n\}]$ achieves objective value $f(V')$, completing the proof.
\end{proof}

The running time of this dynamic programming algorithm is $O(2^{|V|} \poly(|V|, |E|))$, since we compute $f(V')$ and $g(V')$ for each of the $2^{|V|}$ subsets, and each computation takes polynomial time in $|V|$ and $|E|$.

\section{Maximizing \texorpdfstring{$\sum_{v\in V} \varrho(v)\delta(v)$ over acyclic orientations}{Sum rho(v)delta(v)}}\label{sec:maxSumLeftRight}

This section is devoted to Problem~\ref{prob:rhoTimesDelta}, another notion of ``equitable'' acyclic orientations, which we define in terms of both the in- and outdegrees, as opposed to Problems~\ref{prob:sumPhiRhoV}-\ref{prob:incMax}, which focus solely on the indegrees.
Our goal is to find an acyclic orientation of a graph $G = (V, E)$ that maximizes $\sum_{v \in V} \varrho(v)\delta(v)$.
The optimal orientations are equitable in the sense that the product $\varrho(v)\delta(v)$ for a vertex $v$ is maximized when its total degree is distributed as evenly as possible between indegree and outdegree.

Without the acyclicity condition, this problem fits into the framework of Problem~\ref{prob:sumPhi_VRhoV}, hence it can be solved in polynomial time by Theorem~\ref{thm:nonAcyclicWithFlow}.
In fact, the optimal orientations are either Eulerian (that is, $\varrho(v) = \delta(v)$ for every $v \in V$) or almost-Eulerian (that is, $|\varrho(v) - \delta(v)| \leq 1$ for every $v \in V$), and such orientations are known to always exists and be computable in polynomial time.

However, the acyclicity condition makes the problem much harder.
Finding an acyclic orientation of $G$ that maximizes $\sum_{v \in V} \varrho(v)\delta(v)$ is equivalent to finding a vertex order that maximizes $\sum_{v \in V} \cev{d}(v)\vec{d}(v)$.
In~\cite{biedl2005balanced}, the authors studied similar ordering problems where the goal was to minimize $\sum_{v \in V} |\cev{d}(v) - \vec{d}(v)|$.
They called an order \emph{perfectly balanced} if $|\cev{d}(v) - \vec{d}(v)| \leq 1$ holds for every vertex $v$, and proved that deciding whether a given graph has a perfectly balanced order is NP-complete even for bipartite simple graphs with maximum degree $6$.
In~\cite{kara2005complexity}, it was further shown that the problem is NP-complete for planar simple graphs with maximum degree $4$ and for $5$\nobreakdash-regular graphs.

For a single vertex $v$, the value of $\cev{d}(v)\vec{d}(v)$ is maximized if and only if $|\cev{d}(v) - \vec{d}(v)| \leq 1$.
Thus, if $G$ has a perfectly balanced order, then any such order maximizes $\sum_{v \in V} \cev{d}(v)\vec{d}(v)$.
By the NP-hardness of finding a perfectly balanced order, we obtain the following result.

\begin{theorem}
  It is NP-hard to find a vertex order that maximizes $\sum_{v \in V} \cev{d}(v)\vec{d}(v)$ even for bipartite simple graphs with maximum degree $6$, planar simple graphs with maximum degree $4$, and for $5$\nobreakdash-regular graphs.
  \FBOX
\end{theorem}

In the next section, we prove that the problem becomes tractable when the maximum degree is at most $3$.

\subsection{Polynomial-time algorithm for graphs with maximum degree at most $3$}

The problem of finding a vertex ordering that minimizes $\sum_{v \in V} |\cev{d}(v) - \vec{d}(v)|$ can be solved in polynomial time for graphs whose maximum degree is at most~$3$.
This is based on the algorithm from~\cite{biedl2005balanced}, which computes an ordering that minimizes the number of vertices $v$ for which either $\cev{d}(v) = 0$ or $\vec{d}(v) = 0$.
We show that a modification of this algorithm solves our maximization problem, as long as the graph has maximum degree at most~$3$.

Before describing the algorithm, we introduce some necessary definitions.
A vertex order of a graph $G = (V, E)$ is called an \emph{$s$\nobreakdash-$t$ order} if vertices $s$ and $t$ appear first and last, respectively, and for every vertex $v \in V \setminus \{s, t\}$ with degree at least $2$, we have both $\cev{d}(v) \geq 1$ and $\vec{d}(v) \geq 1$.
It is known that every biconnected graph admits an $s$\nobreakdash-$t$ order for any pair of distinct vertices $s, t \in V$, and such an order can be computed in polynomial time~\cite{even1976computing}.
Clearly, for biconnected $3$\nobreakdash-regular graphs, any $s$\nobreakdash-$t$ order maximizes the expression $\sum_{v \in V} \cev{d}(v) \vec{d}(v)$.
We extend this result to all connected graphs with maximum degree at most~$3$.
Let $B_1, \dots, B_r$ denote the biconnected components of $G$.
These components form a tree structure $T$, where each vertex represents either a component or a cut vertex of the graph, and there is an edge between a component and a cut vertex if the component contains the cut vertex.
A component is called an \emph{end-component} if it contains exactly one cut vertex --- that is, end-components correspond to the leaves of $T$.

The following algorithm is a modified version of the one in~\cite{biedl2005balanced}, adapted to maximize the sum $\sum_{v\in V}\cev{d}(v)\vec{d}(v)$ for graphs with maximum degree at most~$3$.

\begin{algorithm}[H]
  \caption{\hspace{0.5cm}\textsc{Combine $s$\nobreakdash-$t$ orderings}}\label{alg:combineST}
  \begin{algorithmic}[1]
    \Input A connected graph $G = (V, E)$ with maximum degree at most $3$ and $|V| \geq 2$.
    \Output A vertex order of $G$ that maximizes $\sum_{v \in V} \cev{d}(v) \vec{d}(v)$.
    \State Compute the tree $T$ of biconnected components of $G$.
    \State Choose an arbitrary end-component $B_1$ of $T$.
    \State Perform a depth-first traversal of $T$ starting from $B_1$, and label the components in traversal order as $B_1, \dots, B_r$.\label{alg:line:depthFirst}
    \State Let $t_1 \in B_1$ be the unique cut vertex of $B_1$.
    \State Choose $s_1 \in B_1 \setminus \{t_1\}$ such that $d_G(s_1)$ is minimized.\label{alg:line:chooseS_1}
    \State Compute an $s_1$\nobreakdash-$t_1$ order of $B_1$, denoted by $\sigma^1$.
    \For{$i = 2,\dots, r$}
      \State Let $s_i$ be a cut vertex shared with a previously processed block $B_j$ for some $j < i$.
      \If{$B_i$ is an end-component}
      \State Choose $t_i \in B_i \setminus \{s_i\}$ such that $d_G(t_i)$ is minimized.\label{alg:line:chooseT_i}
      \Else
        \State Let $t_i$ be a cut vertex shared with a later component $B_j$ for some $j > i$.
      \EndIf
      \State Compute an $s_i$\nobreakdash-$t_i$ order $v_1^i, \dots, v_{n_i}^i$ of $B_i$.
      \State Append $v_2^i, \dots, v_{n_i}^i$ to the end of the current order $\sigma^{i-1}$ to obtain $\sigma^i$.
    \EndFor
    \State \Return $\sigma^r$
  \end{algorithmic}
\end{algorithm}

Algorithm~\ref{alg:combineST} computes $s$\nobreakdash-$t$ orders for the biconnected components of $G$ and combines them into a single vertex ordering.

\begin{theorem}
  For a connected graph $G = (V, E)$ with maximum degree at most~$3$, Algorithm~\ref{alg:combineST} computes, in polynomial time, a vertex order that maximizes the sum $\sum_{v \in V} \cev{d}(v)\vec{d}(v)$.
\end{theorem}
\begin{proof}
  The algorithm clearly runs in polynomial time.

  To prove correctness, we define the \emph{imbalance} of a vertex $v$ in an ordering $\sigma$ as
  \[
    \mathcal{I}_\sigma(v) = \left\lfloor \frac{d(v)}{2} \right\rfloor \left\lceil \frac{d(v)}{2} \right\rceil - \cev{d}_\sigma(v) \vec{d}_\sigma(v).
  \]
  This quantity is always non-negative, as the product $\cev{d}_\sigma(v)\vec{d}_\sigma(v)$ is maximized when the indegree and outdegree are as balanced as possible.
  The term $\left\lfloor \frac{d(v)}{2} \right\rfloor \left\lceil \frac{d(v)}{2} \right\rceil$ is an upper bound on this product.
  Therefore, maximizing $\sum_{v \in V} \cev{d}(v)\vec{d}(v)$ is equivalent to minimizing $\sum_{v \in V} \mathcal{I}_\sigma(v)$.
  We prove that the ordering $\sigma^r$ returned by the algorithm achieves this minimum.

  For any vertex $v$ and order $\sigma$, the imbalance is given by
  \[
    \mathcal{I}_\sigma(v) =
    \begin{cases}
      2 & \text{if } d(v) = 3 \text{ and } \cev{d}_\sigma(v) = 0 \text{ or } \vec{d}_\sigma(v) = 0, \\
      1 & \text{if } d(v) = 2 \text{ and } \cev{d}_\sigma(v) = 0 \text{ or } \vec{d}_\sigma(v) = 0, \\
      0 & \text{otherwise.}
    \end{cases}
  \]
  That is, $\mathcal{I}_\sigma(v) = d(v) - 1$ when all edges of $v$ are directed either forward or backward, and $0$ otherwise.

  Let $T$ denote the tree of biconnected components of $G$, and let $B_1, \dots, B_r$ denote the components in the depth-first traversal order used in Line~\ref{alg:line:depthFirst} of the algorithm.
  Let $B_{i_1}, \dots, B_{i_\ell}$ be the end-components of $T$, where $B_{i_1} = B_1$.
  For each $j \in \{1, \dots, \ell\}$, let $q_{i_j}$ denote the unique cut vertex of $B_{i_j}$.
  By construction, using the notation of Algorithm~\ref{alg:combineST}, we have $q_{i_1} = t_{i_1}$ and $q_{i_j} = s_{i_j}$ for $j \geq 2$.

  In the final ordering $\sigma^r$ produced by the algorithm, every vertex $v \in V \setminus \{s_{i_1}, t_{i_2}, \dots, t_{i_\ell}\}$ satisfies $\cev{d}_{\sigma^r}(v) \geq 1$ and $\vec{d}_{\sigma^r}(v) \geq 1$, and thus $\mathcal{I}_{\sigma^r}(v) = 0$.
  It follows that
  \begin{equation}\label{eq:imbalanceAlg}
    \sum_{v \in V} \mathcal{I}_{\sigma^r}(v) = \mathcal{I}_{\sigma^r}(s_{i_1}) + \sum_{j = 2}^\ell \mathcal{I}_{\sigma^r}(t_{i_j}) \leq (d(s_{i_1}) - 1) + \sum_{j = 2}^\ell (d(t_{i_j}) - 1).
  \end{equation}

  Now consider an arbitrary vertex ordering $\sigma$.
  In each end-component $B_{i_j}$, the only vertex connected to the rest of the graph is $q_{i_j}$.
  Therefore, the first or last vertex in $B_{i_j}$ must lie in $B_{i_j} \setminus \{q_{i_j}\}$, and hence there exists some $v_{i_j} \in B_{i_j} \setminus \{q_{i_j}\}$ such that $\cev{d}_\sigma(v_{i_j}) = 0$ or $\vec{d}_\sigma(v_{i_j}) = 0$.

  Thus,
  \[
    \mathcal{I}_\sigma(v_{i_j}) = d(v_{i_j}) - 1,
  \]
  and so we obtain the lower bound
  \begin{equation}\label{eq:ImbalanceRandom}
    \sum_{v \in V} \mathcal{I}_\sigma(v) \geq \sum_{j = 1}^\ell \mathcal{I}_\sigma(v_{i_j}) = \sum_{j = 1}^\ell (d(v_{i_j}) - 1).
  \end{equation}

  From the choice of $s_{i_1}$ and $t_{i_j}$ in Lines~\ref{alg:line:chooseS_1} and~\ref{alg:line:chooseT_i}, we have $d(s_{i_1}) \leq d(v_{i_1})$ and $d(t_{i_j}) \leq d(v_{i_j})$ for each $j \in \{ 1, \dots, \ell \}$.
  Combining these inequalities with~\eqref{eq:imbalanceAlg} and~\eqref{eq:ImbalanceRandom}, we obtain
  \[
    \sum_{v \in V} \mathcal{I}_{\sigma^r}(v) \leq \sum_{v \in V} \mathcal{I}_\sigma(v),
  \]
  which proves that $\sigma^r$ minimizes the total imbalance and thus maximizes $\sum_{v \in V} \cev{d}(v)\vec{d}(v)$.
\end{proof}

So far, we have shown that the problem of maximizing $\sum_{v \in V} \cev{d}(v)\vec{d}(v)$ can be solved in polynomial time for graphs with maximum degree at most $3$.
However, the problem becomes NP-hard when restricted to simple graphs with maximum degree at most $4$.

We now turn to approximation algorithms for the general case of maximizing $\sum_{v \in V} \cev{d}(v)\vec{d}(v)$.
It is worth noting that Biedl et al.~\cite{biedl2005balanced} presented a $\frac{13}{8}$\nobreakdash-approximation algorithm for the related problem of minimizing $\sum_{v \in V} \max\{\cev{d}(v), \vec{d}(v)\}$.
Their algorithm incrementally constructs an ordering by inserting vertices one at a time, always placing each vertex at the position that minimizes the imbalance among its already inserted neighbors --- typically, by placing it in the middle of its neighbors.
While this approach appears promising for our objective as well, it turns out that it performs poorly in the worst case.
In fact, one can construct a family of graphs for which the approximation ratio of this strategy for maximizing $\sum_{v \in V} \cev{d}(v)\vec{d}(v)$ tends to infinity.

In what follows, we analyze the expected approximation ratio achieved by a uniformly random vertex order.
We then present a deterministic algorithm that guarantees the same approximation ratio.

\subsection{Approximation ratio of orienting by a random permutation}

Next, we analyze the expected approximation ratio achieved by a uniformly random ordering of the vertices for the problem of maximizing $\sum_{v \in V} \cev{d}(v)\vec{d}(v)$.

\begin{theorem}\label{thm:random3Approx}
  A uniformly random permutation of the vertices yields a $3$\nobreakdash-approximate solution in expectation for maximizing $\sum_{v \in V} \cev{d}(v)\vec{d}(v)$.
\end{theorem}
\begin{proof}
  Let $n$ denote the number of vertices, and let $\mathcal{S}_V$ denote the set of all permutations of $V$.
  For each $v \in V$, define $\E(v)$ as the expected value of $\cev{d}_\sigma(v)\vec{d}_\sigma(v)$ when $\sigma$ is drawn uniformly at random from $\mathcal{S}_V$, that is,
  \[
    \E(v)=\frac{1}{n!}\sum_{\sigma\in \mathcal{S}_V}\cev{d}_{\sigma}(v)\vec{d}_{\sigma}(v).
  \]
  Then the expected objective value of a random vertex order is
  \[
    \frac{\sum_{\sigma\in \mathcal{S}_V}\sum_{v\in V}\cev{d}_{\sigma}(v)\vec{d}_{\sigma}(v)}{n!}=\sum_{v\in V}\frac{\sum_{\sigma\in \mathcal{S}_V}\cev{d}_{\sigma}(v)\vec{d}_{\sigma}(v)}{n!}=\sum_{v\in V}\E(v).
  \]

  Thus, to prove the theorem, it suffices to show that $\E(v) \geq \frac{1}{3} \cev{d}_\sigma(v)\vec{d}_\sigma(v)$ for each vertex $v \in V$ for every optimal vertex order $\sigma$.

  Observe that for any vertex order $\sigma$ and any vertex $v \in V$, the value $\cev{d}_\sigma(v)\vec{d}_\sigma(v)$ equals the number of length-two paths $uvw$ such that $u$ precedes $v$ and $w$ follows $v$ in $\sigma$.
  Therefore, for each vertex $v\in V$, the product $\cev{d}(v)\vec{d}(v)$ in an optimal vertex order is at most the maximum number of such paths, while $\E(v)$ is their expected number in a random vertex order.
  Since there are six possible orderings of the vertices $u$, $v$, and $w$, and $v$ appears between $u$ and $w$ in exactly two of them, the probability that $v$ lies between $u$ and $w$ in a random vertex order is $\frac{1}{3}$.
  Hence, $\E(v)$ is the third of the maximum number of length-two paths with inner vertex $v$.
  Summing over all vertices yields that a random vertex order achieves, in expectation, an approximation ratio~$3$.
\end{proof}


\subsection{Derandomized approximation algorithm}
Now we present a deterministic polynomial-time $3$\nobreakdash-approximation algorithm for maximizing the objective function $\sum_{v\in V} \cev{d}(v)\vec{d}(v)$.

Let $G = (V, E)$ be a graph, and let $n = |V|$.
For $0 \leq i \leq n$ and a sequence of distinct vertices $v_1, \dots, v_i \in V$, let $\E_{v_1,\dots,v_i}$ denote the expected value of the objective function over all permutations of the vertex set $V$ in which the first $i$ positions are fixed to $v_1, \dots, v_i$ in this order.
Formally, we define
\[
  \E_{v_1,\dots,v_i} = \frac{1}{(n-i)!}\sum_{\substack{\sigma\in \mathcal{S}_V, \\\sigma_j=v_j\ \forall j\leq i}}\sum_{v\in V} \cev{d}_{\sigma}(v)\vec{d}_{\sigma}(v)
\]
for every sequence of distinct vertices $v_1, \dots, v_i \in V$.

We now describe the deterministic algorithm, which is a derandomized version of the randomized algorithm from the previous section, using the method of conditional expectations.
The algorithm fixes the vertices in the order one by one from left to right.

At the $i^{\text{th}}$ step, vertices $v_1, \dots, v_{i-1}$ have already been fixed in positions $1$ through $(i - 1)$.
For each remaining vertex $v \in V \setminus \{v_1, \dots, v_{i-1}\}$, we compute $\E_{v_1,\dots,v_{i-1},v}$, and choose a vertex $v$ that maximizes this expected value to be placed in position $i$.

To establish that the algorithm runs in polynomial time, it suffices to show that the expected value $\E_{v_1,\dots,v_i}$ can be computed in polynomial time for any fixed sequence of distinct vertices $v_1,\dots,v_i \in V$.
We proceed to prove this next.
\begin{claim}
For every sequence of distinct vertices $v_1,\dots,v_i \in V$, the expected value $\E_{v_1,\dots,v_i}$ can be computed in polynomial time.
\end{claim}
\begin{proof}
We call a permutation \emph{relevant} if it begins with the vertices $v_1,\dots,v_i$ in this order.

For each vertex $v \in V$, let $\E_{v_1,\dots,v_i}(v)$ denote the expected value of the term $\cev{d}(v)\vec{d}(v)$ under a uniformly random relevant permutation of $V$.
By linearity of expectation and simple rearrangement, we have
\[
  \E_{v_1,\dots,v_i}
  = \frac{1}{(n-i)!}\sum_{\substack{\sigma\in \mathcal{S}_V,\\ \sigma_j=v_j\ \forall j\leq i}}\sum_{v\in V} \cev{d}_{\sigma}(v)\vec{d}_{\sigma}(v)
  = \sum_{v\in V}\frac{1}{(n-i)!}\sum_{\substack{\sigma\in \mathcal{S}_V,\\ \sigma_j=v_j\ \forall j\leq i}}\!\!\!\!\!\cev{d}_{\sigma}(v)\vec{d}_{\sigma}(v)
  = \sum_{v\in V}\E_{v_1,\dots,v_i}(v).
\]

Thus, it suffices to show that each term $\E_{v_1,\dots,v_i}(v)$ can be computed in polynomial time.
Let $V_j = \{v_1, \dots, v_j\}$ for $j\in\{1,\dots,i\}$.

First, suppose that $v = v_j$ for some $j \leq i$.
In this case, $\E_{v_1,\dots,v_i}(v_j) = d(v_j, V_{j-1}) d(v_j, V \setminus V_j)$, because the left and right-degrees of the vertex $v_j$ remain unchanged across all relevant permutations.

Second, suppose that $v \in V \setminus V_i$.
We present a dynamic programming method for computing $\E_{v_1,\dots,v_i}(v)$, exploiting the fact that only the vertices in $V \setminus V_i$ may appear to the right of $v$ in any relevant permutation.
Let us introduce the following notations.
Denote the neighbors of $v$ within $V \setminus V_i$ by $w_1, \dots, w_p$.
Define $f(j, k, \ell)$ to be the number of permutations of the set $\{v, w_1, \dots, w_j\}$ in which $v$ has exactly $k$ succeeding vertices, and the right-degree of $v$ is exactly $\ell$.
Here, $j, k \in \{0, \dots, p\}$ and $\ell \in \{0, \dots, d(v, V \setminus V_i)\}$.
We set $f(j, k, \ell) = 0$ for all values outside these ranges.
As the base case, observe that
\[
  f(0, k, \ell) =
  \begin{cases}
    1 & \text{if } k = \ell = 0, \\
    0 & \text{otherwise}.
  \end{cases}
\]

\begin{claim}
The function $f(j,k,\ell)$ satisfies the following recurrence
\[
  f(j,k,\ell) = k \cdot f(j-1, k-1, \ell - d(v, \{w_j\})) + (j - k) \cdot f(j-1, k, \ell).
\]
\end{claim}
\begin{proof}
  Let $\sigma$ be a permutation counted by $f(j,k,\ell)$.
  There exists a unique permutation $\sigma'$ such that $\sigma$ is obtained from $\sigma'$ by inserting $w_j$ into an appropriate position --- we say that $\sigma$ \emph{corresponds} to $\sigma'$.
  Depending on the relative position of $w_j$ with respect to our vertex $v$, this $\sigma'$ is counted by either $f(j-1, k-1, \ell - d(v,\{w_j\}))$ or $f(j-1, k, \ell)$.

  Now, consider a permutation $\sigma'$ counted by $f(j-1, k-1, \ell - d(v,\{w_j\}))$.
  Since there are exactly $k$ possible positions to insert $w_j$ after $v$, there are precisely $k$ permutations $\sigma$ corresponding to this $\sigma'$.
  Similarly, if $\sigma'$ is counted by $f(j-1, k, \ell)$, then there are exactly $(j - k)$ positions to insert $w_j$ before $v$, resulting in $(j - k)$ corresponding permutations $\sigma$.
  This establishes the validity of the recurrence.
\end{proof}

Observe that if we compute the values $f(j,k,\ell)$ in increasing order by $j$ and memoize the results, then all values on the right-hand side of the recurrence are readily available at each step.
Consequently, we can compute $f(j,k,\ell)$ for all $j,k \in \{0,\dots,p\}$ and $\ell \in \{0,\dots,d(v, V \setminus V_i)\}$ in $O(|V|^2|E|)$ time.

Note that each value $f(j,k,\ell) \leq |V|!$, so storing them requires at most $O(|V| \log |V|)$ bits.
Together, these observations imply that the dynamic programming algorithm runs in polynomial time.

We now show how to compute $\E_{v_1,\dots,v_i}(v)$ using the precomputed values $f(p,k,\ell)$, where $p$ is the number of neighbors of $v$ in $V \setminus V_i$.
By definition, every relevant permutation begins with $v_1,\dots,v_i$ in the first $i$ positions, followed by a permutation of the remaining vertices in $V \setminus V_i$.
Thus, there are $|\mathcal{S}_{V \setminus V_i}|$ relevant permutations in total.
In exactly $\sum_{k=0}^p f(p,k,\ell)$ of these permutations, vertex $v$ has right-degree $\ell$.
Therefore, the value $\E_{v_1,\dots,v_i}(v)$ is the weighted average of the expressions $(d(v) - \ell)\ell$, where each weight is $\sum_{k=0}^p f(p,k,\ell)$, that is,
\[
  \E_{v_1,\dots,v_i}(v)=\frac{1}{|\mathcal{S}_{V\setminus V_i}|}\sum_{\ell=0}^{d(v, V\setminus V_i)}\sum_{k=0}^p f(p,k,\ell) (d(v)-\ell) \ell
\]
Since all terms involved can be computed in polynomial time, it follows that the derandomized algorithm also runs in polynomial time.
\end{proof}


\begin{remark}
In the case of simple graphs, the vertex $v$ is succeeded by exactly $\ell$ of its neighbors in precisely
\[
  \frac{|\mathcal{S}_{V \setminus V_i}|}{d(v, V \setminus V_i) + 1}
\]
of the relevant permutations for each $\ell \in \{0, \dots, d(v, V \setminus V_i)\}$.
Therefore, we have
\begin{align*}
  \E_{v_1,\dots,v_i}(v) &= \frac{1}{d(v,V\setminus V_i)+1}\sum_{\ell=0}^{d(v,V\setminus V_i)}(d(v)-\ell)\ell\\
  % =\frac{d(v)\sum_{\ell=0}^{d(v,V\setminus V_i)}\ell}{d(v,V\setminus V_i)+1}-\frac{\sum_{\ell=0}^{d(v,V\setminus V_i)}\ell^2}{d(v,V\setminus V_i)+1}\\
  &=\frac{d(v)}{d(v,V\setminus V_i)+1} \sum_{\ell=0}^{d(v,V\setminus V_i)}\ell - \frac{1}{d(v,V\setminus V_i)+1} \sum_{\ell=0}^{d(v,V\setminus V_i)}\ell^2\\
  &=\frac{d(v)d(v,V\setminus V_i)}{2}-\frac{d(v,V\setminus V_i)(2d(v,V\setminus V_i)+1)}{6}\\
  &=\frac{d(v,V\setminus V_i)(3d(v)-2d(v,V\setminus V_i)-1)}{6}.
\end{align*}
Hence, for simple graphs, the value $\E_{v_1,\dots,v_i}(v)$ can be computed directly using this closed-form expression, avoiding the need for dynamic programming.
$\bullet$
\end{remark}

In the rest of this section, we prove that our algorithm produces a $3$\nobreakdash-approximate order.

\begin{theorem}
The derandomized algorithm yields a $3$\nobreakdash-approximation for $\max \sum_{v \in V} \cev{d}(v)\vec{d}(v)$.
\end{theorem}
\begin{proof}
Let $\operatorname{opt}$ denote the value of an optimal solution.
Consider the $i^{\text{th}}$ step of the algorithm, where the first $i$ vertices $v_1, \dots, v_i$ have already been fixed.
Let $V_i = \{v_1, \dots, v_i\}$.
We prove by induction on $i$ that $\E_{v_1, \dots, v_i} \geq \frac{1}{3} \operatorname{opt}$ for all $i \in \{0, 1, \dots, n\}$.
That is, fixing the first $i$ vertices and randomly permuting the remaining ones gives, in expectation, a $3$\nobreakdash-approximation of the~optimum.

The base case $i = 0$ holds by Theorem~\ref{thm:random3Approx}, which states that a uniformly random permutation yields a $3$\nobreakdash-approximation in expectation.

Assume the claim holds for some $i < n$, i.e., $\E_{v_1, \dots, v_i} \geq \frac{1}{3} \operatorname{opt}$.
We show it also holds for $(i+1)$.
From the definition of $\E_{v_1,\dots,v_i}$ and simple rearrangements, we have
\begin{align*}
\E_{v_1,\dots,v_i}
&= \frac{1}{(n - i)!} \sum_{\substack{\sigma \in \mathcal{S}_V \\ \sigma_j = v_j\ \forall j \leq i}} \sum_{v \in V} \cev{d}_\sigma(v)\vec{d}_\sigma(v) \\
&= \frac{1}{(n - i)!} \sum_{u \in V \setminus V_i} \sum_{\substack{\sigma \in \mathcal{S}_V \\ \sigma_j = v_j\ \forall j \leq i,\\ \sigma_{i+1} = u}} \sum_{v \in V} \cev{d}_\sigma(v)\vec{d}_\sigma(v) \\
&= \frac{1}{(n - i)!} \sum_{u \in V \setminus V_i} (n - i - 1)! \cdot \E_{v_1,\dots,v_i,u} \\
&= \frac{1}{n - i} \sum_{u \in V \setminus V_i} \E_{v_1,\dots,v_i,u}.
\end{align*}
Since the algorithm selects $v_{i+1} \in V \setminus V_i$ to maximize $\E_{v_1,\dots,v_i,u}$, it follows that $\E_{v_1,\dots,v_i,v_{i+1}} \geq \E_{v_1,\dots,v_i}$.
By the inductive hypothesis, we then have $\E_{v_1,\dots,v_{i+1}} \geq \E_{v_1,\dots,v_i} \geq \frac{1}{3} \operatorname{opt}$.
This completes the induction and proves the theorem.
\end{proof}

\section{Open questions}
One of our main results, established in Section~\ref{sec:minSumH}, states that minimizing $\sum_{v \in V} \varphi(\cev{d}(v))$ is NP-hard for every discrete strictly convex function $\varphi: \Z_+ \to \R$ when parallel edges are allowed.
The complexity of this problem in the case of \emph{simple} graphs, however, remains open.

In Section~\ref{sec:decMinIncMax}, we presented a detailed analysis of the computational complexity of various lexicographically optimal vertex ordering problems.
We proved that both the dec-min and inc-max $k$\nobreakdash-bounded ordering problems are NP-hard for every $k \geq 2$.

This result also implies the NP-hardness of the dec-min problem for $k$\nobreakdash-degenerate graphs with $k \geq 2$.
However, the complexity of the inc-max problem for $k$\nobreakdash-degenerate graphs remains unresolved.

For the special case $\varphi(z) = z^2$, we extended the NP-hardness proof to simple graphs.
Moreover, we analyzed a natural greedy algorithm for this setting.
It is still an open question whether this greedy algorithm achieves a constant-factor approximation.
Among the graph instances we examined, the worst-case approximation ratio tends to $\frac{9}{7}$.

\section*{Acknowledgment}
We are grateful to Andr\'as Frank for insightful discussions and for drawing our attention to relevant literature.

\medskip
This research has been implemented with the support provided by the Ministry of Innovation and Technology of Hungary from the National Research, Development and Innovation Fund, financed under the ELTE TKP 2021-NKTA-62 funding scheme, and by the Ministry of Innovation and Technology NRDI Office within the framework of the Artificial Intelligence National Laboratory Program, by the Lend\"ulet Programme of the Hungarian Academy of Sciences --- grant number LP2021-1/2021.
The first author was supported by the Ministry of Innovation and Technology of Hungary from the National Research, Development and Innovation Fund --- grant number ADVANCED 150556.
The second author was supported by the EK\" OP-24 University Excellence Scholarship Program of the Ministry for Culture and Innovation from the source of the National Research, Development and Innovation Fund.

\bibliographystyle{plain}
\bibliography{bibliography}
\end{document}
