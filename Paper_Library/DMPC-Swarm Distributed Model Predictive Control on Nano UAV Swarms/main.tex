%Version 3 October 2023
% See section 11 of the User Manual for version history
%
%%%%%%%%%%%%%%%%%%%%%%%%%%%%%%%%%%%%%%%%%%%%%%%%%%%%%%%%%%%%%%%%%%%%%%
%%                                                                 %%
%% Please do not use \input{...} to include other tex files.       %%
%% Submit your LaTeX manuscript as one .tex document.              %%
%%                                                                 %%
%% All additional figures and files should be attached             %%
%% separately and not embedded in the \TeX\ document itself.       %%
%%                                                                 %%
%%%%%%%%%%%%%%%%%%%%%%%%%%%%%%%%%%%%%%%%%%%%%%%%%%%%%%%%%%%%%%%%%%%%%

%%\documentclass[referee,sn-basic]{sn-jnl}% referee option is meant for double line spacing

%%=======================================================%%
%% to print line numbers in the margin use lineno option %%
%%=======================================================%%

%%\documentclass[lineno,sn-basic]{sn-jnl}% Basic Springer Nature Reference Style/Chemistry Reference Style

%%======================================================%%
%% to compile with pdflatex/xelatex use pdflatex option %%
%%======================================================%%

%%\documentclass[pdflatex,sn-basic]{sn-jnl}% Basic Springer Nature Reference Style/Chemistry Reference Style


%%Note: the following reference styles support Namedate and Numbered referencing. By default the style follows the most common style. To switch between the options you can add or remove �Numbered� in the optional parenthesis. 
%%The option is available for: sn-basic.bst, sn-vancouver.bst, sn-chicago.bst%  
 
%%\documentclass[sn-nature]{sn-jnl}% Style for submissions to Nature Portfolio journals
%%\documentclass[sn-basic]{sn-jnl}% Basic Springer Nature Reference Style/Chemistry Reference Style
%%\documentclass[sn-mathphys-num]{sn-jnl}% Math and Physical Sciences Numbered Reference Style 
\documentclass[sn-mathphys-ay, iicol]{sn-jnl}% Math and Physical Sciences Author Year Reference Style
%%\documentclass[sn-aps]{sn-jnl}% American Physical Society (APS) Reference Style
%%\documentclass[sn-vancouver,Numbered]{sn-jnl}% Vancouver Reference Style
%%\documentclass[sn-apa]{sn-jnl}% APA Reference Style 
%%\documentclass[sn-chicago]{sn-jnl}% Chicago-based Humanities Reference Style

%%%% Standard Packages
%%<additional latex packages if required can be included here>

\usepackage{graphicx}%
\usepackage{multirow}%
\usepackage{amsmath,amssymb,amsfonts}%
\usepackage{amsthm}%
\usepackage{mathrsfs}%
\usepackage[title]{appendix}%
\usepackage{xcolor}%
\usepackage{textcomp}%
\usepackage{manyfoot}%
\usepackage{booktabs}%
\usepackage{algorithm}%
\usepackage{algorithmicx}%
\usepackage{algpseudocode}%
\usepackage{listings}%
%%%%

%%%%%=============================================================================%%%%
%%%%  Remarks: This template is provided to aid authors with the preparation
%%%%  of original research articles intended for submission to journals published 
%%%%  by Springer Nature. The guidance has been prepared in partnership with 
%%%%  production teams to conform to Springer Nature technical requirements. 
%%%%  Editorial and presentation requirements differ among journal portfolios and 
%%%%  research disciplines. You may find sections in this template are irrelevant 
%%%%  to your work and are empowered to omit any such section if allowed by the 
%%%%  journal you intend to submit to. The submission guidelines and policies 
%%%%  of the journal take precedence. A detailed User Manual is available in the 
%%%%  template package for technical guidance.
%%%%%=============================================================================%%%%

%% as per the requirement new theorem styles can be included as shown below
\theoremstyle{thmstyleone}%
\newtheorem{theorem}{Theorem}%  meant for continuous numbers
%%\newtheorem{theorem}{Theorem}[section]% meant for sectionwise numbers
%% optional argument [theorem] produces theorem numbering sequence instead of independent numbers for Proposition
\newtheorem{proposition}[theorem]{Proposition}% 
%%\newtheorem{proposition}{Proposition}% to get separate numbers for theorem and proposition etc.

\theoremstyle{thmstyletwo}%
\newtheorem{example}{Example}%
\newtheorem{remark}{Remark}%

\theoremstyle{thmstylethree}%
\newtheorem{definition}{Definition}%

\raggedbottom
%%\unnumbered% uncomment this for unnumbered level heads

%%%%%%% my imports

\usepackage{times}
\usepackage{booktabs}
\usepackage{tablefootnote}

\usepackage{float}

\usepackage[binary-units=true]{siunitx}

\usepackage{amsmath}
\usepackage{amssymb}
\usepackage{amsthm}

\usepackage{graphicx}

\usepackage{amssymb}% http://ctan.org/pkg/amssymb
\usepackage{pifont}% http://ctan.org/pkg/pifont

\usepackage{makecell}

\usepackage{wrapfig}

\usepackage{algorithm}
\usepackage{algpseudocode}

% from https://people.inf.ethz.ch/markusp/teaching/guides/guide-tables.pdf
\usepackage{booktabs}
\newcommand{\ra}[1]{\renewcommand{\arraystretch}{#1}}

\usepackage{makecell}

\usepackage{pgfplots}
\usepackage{tikz}

\usepackage{float}

\usepackage{enumitem}


\usepgfplotslibrary{external}
\usetikzlibrary{shapes}
\usetikzlibrary{positioning}
\usetikzlibrary{arrows,decorations.markings}
\usetikzlibrary{calc,spy,shapes}
\usetikzlibrary{shapes}
\usepgfplotslibrary{fillbetween}

\usetikzlibrary{pgfplots.groupplots}
\usetikzlibrary{patterns}

\usepackage{caption}

\usepackage{subcaption}

\renewcommand\theadalign{bc}
\renewcommand\theadfont{\bfseries}
\renewcommand\theadgape{\Gape[4pt]}
\renewcommand\cellgape{\Gape[4pt]}

\usepackage{xcolor}
\definecolor{applegreen}{rgb}{0.55, 0.9, 0.0}
\definecolor{amaranth}{rgb}{0.9, 0.17, 0.31}

\newtheorem*{theorem-non}{Theorem}
\newtheorem{lemma}{Lemma}
\newtheorem{corollary}{Corollary}
\newtheorem{assumption}{Assumption}

\newcommand{\fillopacity}{0.4}
\newcommand{\colordarkgreen}{green!50!black}
\newcommand{\colordarkblue}{blue!50!black}
\newcommand{\colordarkred}{blue!50!black}

\newcommand{\lwboxplot}{0.7pt}

\newcommand{\emphkey}[1]{\textbf{#1}}

\newcommand{\myswarm}{\textsc{DMPC-Swarm}}

\newcommand{\cmarkgreen}{\textcolor{green!70!black}{\textbf{\large\cmark}}}
\newcommand{\xmarkred}{\textcolor{red!70!black}{\textbf{\large\xmark}}}

\newcommand{\opindent}{\phantom{=}}

\newcommand{\reviewerone}[1]{#1}
\newcommand{\reviewertwo}[1]{#1}
\newcommand{\reviewerthree}[1]{#1}
\newcommand{\reviewerall}[1]{#1}

\usepackage{ifthen}
	\newboolean{authnotes}
	\setboolean{authnotes}{true}
	% \setboolean{authnotes}{false}

\ifthenelse{\boolean{authnotes}}
% if the boolean authnotes is true define the macros
{
\newcommand{\thought}[1]{{\color[rgb]{0.2,0.39,0.66}(#1)}}
\newcommand{\todo}[1]{{\color[rgb]{1.0,0.0,0.0}(#1)}}
\newcommand{\fm}[1]{\footnote{{\bf\color{blue} Fabian: #1}}}
\newcommand{\mz}[1]{\footnote{{\bf\color{orange!50!black} Marco: #1}}}
\newcommand{\st}[1]{\footnote{{\bf\color{green!50!black} Sebastian: #1}}}
\newcommand{\ag}[1]{\footnote{{\bf\color{red!50!black} Alexander: #1}}}
}
% otherwise, just define the macros to avoid compilation errors, but ignore
% the content of the notes
{
\newcommand{\thought}[1]{}
\newcommand{\todo}[1]{}
\newcommand{\fm}[1]{}
\newcommand{\mz}[1]{}
\newcommand{\st}[1]{}
\newcommand{\ag}[1]{}
}



%%%%%%%%%%%Makros%%%%%%%%%%%%%%%%%
\newcommand{\capt}[1]{\mdseries{\emph{#1}}}
\newcommand{\cone}[0]{\textbf{C1)}}
\newcommand{\ctwo}[0]{\textbf{C2)}}
\newcommand{\cthree}[0]{\textbf{C3)}}
\newcommand{\cfour}[0]{\textbf{C4)}}

\newcommand{\safetyguarnum}[0]{\textbf{R1}}
\newcommand{\distcontrnum}[0]{\textbf{R2}}
\newcommand{\reseffnum}[0]{\textbf{R3}}

% \newcommand{\compdistrcomp}[0]{\hyperref[comp:distrcomp]{\textbf{C1}}}
% \newcommand{\compcom}[0]{\hyperref[comp:com]{\textbf{C2}}}
% \newcommand{\compmlr}[0]{\hyperref[comp:mlr]{\textbf{C3}}}



\newcommand{\distcontr}[0]{\emphkey{distributed control}}
\newcommand{\safetyguar}[0]{\emphkey{safety guarantees}}
\newcommand{\reseff}[0]{\emphkey{resource efficiency}}

\newcommand{\distcontrc}[0]{\emphkey{Distributed control}}
\newcommand{\safetyguarc}[0]{\emphkey{Safety guarantees}}
\newcommand{\reseffc}[0]{\emphkey{Resource efficiency}}

\newcommand{\distcontrcc}[0]{\emphkey{Distributed Control}}
\newcommand{\safetyguarcc}[0]{\emphkey{Safety Guarantees}}
\newcommand{\reseffcc}[0]{\emphkey{Resource Efficiency}}

\newcommand{\ul}[1]{_\mathrm{#1}}
\newcommand{\uli}[1]{_{#1}}

\newcommand{\fakepar}[1]{\vspace{2mm}\noindent\textbf{#1}}




\newcommand{\cmark}{\textcolor{applegreen!80!black}{\ding{51}}}%
\newcommand{\xmark}{\textcolor{amaranth!80!black}{\ding{55}}}%


\pgfplotsset{
	box plot/.style={
		/pgfplots/.cd,
		black,
		only marks,
		mark=-,
		mark size=\pgfkeysvalueof{/pgfplots/box plot width},
		/pgfplots/error bars/y dir=plus,
		/pgfplots/error bars/y explicit,
		/pgfplots/table/x index=\pgfkeysvalueof{/pgfplots/box plot x index},
	},
	box plot box/.style={
		/pgfplots/error bars/draw error bar/.code 2 args={%
			\draw  ##1 -- ++(\pgfkeysvalueof{/pgfplots/box plot width},0pt) |- ##2 -- ++(-\pgfkeysvalueof{/pgfplots/box plot width},0pt) |- ##1 -- cycle;
		},
		/pgfplots/table/.cd,
		y index=\pgfkeysvalueof{/pgfplots/box plot box top index},
		y error expr={
			\thisrowno{\pgfkeysvalueof{/pgfplots/box plot box bottom index}}
			- \thisrowno{\pgfkeysvalueof{/pgfplots/box plot box top index}}
		},
		/pgfplots/box plot
	},
	box plot top whisker/.style={
		/pgfplots/error bars/draw error bar/.code 2 args={%
			\pgfkeysgetvalue{/pgfplots/error bars/error mark}%
			{\pgfplotserrorbarsmark}%
			\pgfkeysgetvalue{/pgfplots/error bars/error mark options}%
			{\pgfplotserrorbarsmarkopts}%
			\path ##1 -- ##2;
		},
		/pgfplots/table/.cd,
		y index=\pgfkeysvalueof{/pgfplots/box plot whisker top index},
		y error expr={
			\thisrowno{\pgfkeysvalueof{/pgfplots/box plot box top index}}
			- \thisrowno{\pgfkeysvalueof{/pgfplots/box plot whisker top index}}
		},
		/pgfplots/box plot
	},
	box plot bottom whisker/.style={
		/pgfplots/error bars/draw error bar/.code 2 args={%
			\pgfkeysgetvalue{/pgfplots/error bars/error mark}%
			{\pgfplotserrorbarsmark}%
			\pgfkeysgetvalue{/pgfplots/error bars/error mark options}%
			{\pgfplotserrorbarsmarkopts}%
			\path ##1 -- ##2;
		},
		/pgfplots/table/.cd,
		y index=\pgfkeysvalueof{/pgfplots/box plot whisker bottom index},
		y error expr={
			\thisrowno{\pgfkeysvalueof{/pgfplots/box plot box bottom index}}
			- \thisrowno{\pgfkeysvalueof{/pgfplots/box plot whisker bottom index}}
		},
		/pgfplots/box plot
	},
	box plot median/.style={
		/pgfplots/box plot,
		/pgfplots/table/y index=\pgfkeysvalueof{/pgfplots/box plot median index}
	},
	box plot width/.initial=1em,
	box plot x index/.initial=0,
	box plot median index/.initial=1,
	box plot box top index/.initial=2,
	box plot box bottom index/.initial=3,
	box plot whisker top index/.initial=4,
	box plot whisker bottom index/.initial=5,
}

\newcommand{\boxplot}[2][]{
	\addplot [box plot median,#1] table {#2};
	\addplot [forget plot, box plot box,#1] table {#2};
	\addplot [forget plot, box plot top whisker,#1] table {#2};
	\addplot [forget plot, box plot bottom whisker,#1] table {#2};
}

%%%%%%%%%%%%%%%%% END OF PREAMBLE %%%%%%%%%%%%%%%%

\usepackage{fancyhdr}
\pagestyle{fancy}
\fancyhead{} % clear all header fields
\fancyhead[C]{\small\textit{Accepted for publication at Springer Autonomous Robots}}


\begin{document} 
% {\onecolumn \begin{center} Accepted for publication at Springer Autonomous Robots\end{center}
% }
% \twocolumn
% 		\newpage

\title[DMPC-Swarm: Distributed Model Predictive Control on Nano UAV swarms]{DMPC-Swarm: Distributed Model Predictive Control on Nano UAV Swarms}

\author*[1]{\fnm{Alexander} \sur{Gräfe}}\email{alexander.graefe@dsme.rwth-aachen.de}

\author[1]{\fnm{Joram} \sur{Eickhoff}}\email{joram.eickhoff@dsme.rwth-aachen.de}

\author[2]{\fnm{Marco} \sur{Zimmerling}}\email{marco.zimmerling@tu-darmstadt.de}

\author[1]{\fnm{Sebastian} \sur{Trimpe}}\email{trimpe@dsme.rwth-aachen.de}

\affil[1]{\orgdiv{Institute for Data Science in Mechanical Engineering}, \orgname{RWTH Aachen University}, \orgaddress{\street{Theaterstraße 35-39}, \city{Aachen}, \postcode{52062}, \country{Germany}}}

\affil[2]{\orgdiv{Networked Embedded Systems Lab}, \orgname{TU Darmstadt}, \orgaddress{\street{Mornewegstraße 30}, \city{Darmstadt}, \postcode{64293}, \country{Germany}}}




\begin{abstract}

Constraint Programming (CP) and Machine Learning (ML) face challenges in text generation due to CP's struggle with implementing ``meaning'' and ML's difficulty with structural constraints. This paper proposes a solution by combining both approaches and embedding a Large Language Model (LLM) in CP. The LLM handles word generation and meaning, while CP manages structural constraints. This approach builds on 
%
GenCP, an improved version of On-the-fly Constraint Programming Search (OTFS) using LLM-generated domains.
Compared to Beam Search (BS), a standard NLP method, this combined approach
%
(GenCP with LLM)
is faster and produces better results, ensuring all constraints are satisfied. This fusion of CP and ML presents new possibilities for enhancing text generation under constraints.


\end{abstract}
\keywords{Robot Swarms, Safety, Collision Avoidance, Distributed Control}

\maketitle

% }
% \twocolumn
% 
\begin{abstract}

Constraint Programming (CP) and Machine Learning (ML) face challenges in text generation due to CP's struggle with implementing ``meaning'' and ML's difficulty with structural constraints. This paper proposes a solution by combining both approaches and embedding a Large Language Model (LLM) in CP. The LLM handles word generation and meaning, while CP manages structural constraints. This approach builds on 
%
GenCP, an improved version of On-the-fly Constraint Programming Search (OTFS) using LLM-generated domains.
Compared to Beam Search (BS), a standard NLP method, this combined approach
%
(GenCP with LLM)
is faster and produces better results, ensuring all constraints are satisfied. This fusion of CP and ML presents new possibilities for enhancing text generation under constraints.


\end{abstract}



% In setting up this template for *Science* papers, we've used both
% the \section* command and the \paragraph* command for topical
% divisions.  Which you use will of course depend on the type of paper
% you're writing.  Review Articles tend to have displayed headings, for
% which \section* is more appropriate; Research Articles, when they have
% formal topical divisions at all, tend to signal them with bold text
% that runs into the paragraph, for which \paragraph* is the right
% choice.  Either way, use the asterisk (*) modifier, as shown, to
% suppress numbering.
\clearpage
%\section*{Introduction}
%\begin{enumerate}
%	\item Robot swarm challenges
%	\begin{enumerate}
%		\item \textbf{C1)} Information exchange (widespread, mobile (UAVs themselves + infrastructure), reliable)
%		\item \textbf{C2)} Distributed decision making (information structure)
%		\item \textbf{C3)} Adaptivity (UAVs join/leave the network)
%		\item \textbf{C4)} Safety $\rightarrow$ Theoretical guarantees. In particular challenging due to system complexity. Caused by integrating C1)-C3) in a real robot swarm. \cite{zhou2022swarm} state guarantees are impossible.	\end{enumerate}
%	\item Related work $\rightarrow$ none that does all.
%	\item \textbf{Proposal/method}
%	\begin{enumerate} \item Communication/algorithm codesign.
%	\begin{enumerate}
%		\item design philosophy
%		\item leads to cyber physical architecture (Mixer, heterogeneous computing architecture, event-triggering)
%	\end{enumerate}
%	\item Together they solve C1)-C3).
%	\item Codesign is basis for C4). However guarantees setting specific.
%	\item We will show how C4) can be achieved on relevant design example. DMPC for trajectory planning.
%	\item First time: DMPC for trajectory planning implemented on a real swarm with real communication.
%	\end{enumerate}
%\end{enumerate}
%\section*{Results}
%\begin{enumerate}
%	\item \textbf{Hardware-results} (videos + plots)
%	\begin{enumerate}
%		\item \textbf{C1)}: 2 separated swarms, on in the lab, one above in our office, move with CUs between swarms.
%		\item \textbf{C2)}: Formation flight using DMPC, a lot of drones in testbed.
%		\item \textbf{C3)}: Adding/removing drones 
%	\end{enumerate}
%	\item Model/simulation match reality $\rightarrow$ Basis for \textbf{C4)}
%	\item \textbf{C4):} Theorem (DMPC) (Light version). Because model matches real world guarantees transfer.
%	\item Show message loss cannot be be ignored in DMPC with simulation study.
%	\item everything open source.
%\end{enumerate}
%
%\section*{Discussion}
%\begin{enumerate}
%	\item Guarantees through codesign. Theory in accordance with real world. Guarantees in theory transfer to real world.
%	\item Simulation shows MLR is important. Message loss cannot be ignored.
%	\item What effect has the communication system on performance and what can be done on communication side to enhance the performance?
%	\item What effect has the control side on the performance and what can be done on control side?
%\end{enumerate}
%\clearpage
%-------------------------------- content ---------------------------------
% \section*{Summary}
% This paper presents \textsc{SafeSwarm}, the first flying robot swarm architecture and hardware implementation that combines collision avoidance guarantees, distributed control and resource-efficient communication and computation.
% This paper presents the first combination of a trajectory planning algorithm with guarantees under message loss based on distributed model predictive control and evaluation on a quadcopter swarm with a real ad-hoc communication network.



\IEEEPARstart{T}{wo} %
main challenges in the deployment of large-scale swarms are the localization and coordination of vehicles.
Localization methods that rely on external infrastructure 
(e.g., GPS) 
are prone to systematic errors (e.g., multipath effect)
and may not always be available.
Coordination strategies that are centralized can deconflict motion plans to prevent collisions and gridlock, but introduce a single point of failure and are difficult to scale in swarm size due to communication bandwidth limitations.

This paper presents a unified formation flying pipeline for unmanned aerial vehicles (UAVs).
Our pipeline uses \textit{onboard} sensors for localization, which eliminate the need for external positioning systems, and \textit{distributed} techniques for coordination, which enable each vehicle to make decisions independently while communicating their state to a subset of the team.
For \textit{localization}, we use an off-the-shelf commercial visual inertial odometry (VIO) package \cite{VIO}
that fuses inertial measurement unit (IMU) and downward-facing monocular camera measurements to estimate changes in the vehicle pose.
\edit{For \textit{coordination}, we present distributed formation control and task assignment strategies that run onboard the vehicles, do not rely on a common reference frame, and use vehicle-to-vehicle communication.} 
Key features of our formation control strategy include scalability to a large number of vehicles and robustness to disturbances.
The latter is crucial for reaching the desired formations with sensing imperfections.
Our task assignment strategy uses an auction-based algorithm to guarantee conflict-free assignments.
This algorithm can deconflict vehicle gridlocks resulting from distributed collision avoidance (type 3 deadlock~\cite{Wang2017}) and is well-suited for vehicles with limited computational capability and low-bandwidth communication. 


\begin{figure}[t!]
	\begin{center}
		\includegraphics[trim =0mm 10mm 0mm 0mm, clip, width=\columnwidth]{Figs/slanted_plane.png}	
		\caption{
		Six multirotors in a slanted plane formation.
		Vehicles communicate with each other, make distributed decisions onboard, and use VIO for localization.}
		\label{fig:slantedplane}
	\end{center}
\end{figure}


\subsection{Contributions}

This research extends our previous work on UAV formations~\cite{Fathian2019} and presents a unified pipeline consisting of \textit{onboard localization} and \textit{distributed coordination}.
The three main contributions of this work are:
\begin{enumerate}
    \item \edit{scalable formulation of control design suitable for
    onboard sensing without a common reference frame;}
    \item algorithms for deconfliction via \edit{distributed} task assignment of vehicles to desired formation points;    
    \item simulation- and hardware-ready open-source pipeline.
\end{enumerate}
\edit{Our pipeline is tested in hardware with six multirotors (see Fig.~\ref{fig:slantedplane}), and 
to our knowledge is the first demonstration of formation flying that does not rely on external sensing, fiducial markers for localization, a common reference frame, or a centralized base station for coordination.}
The only requirements for the presented pipeline are that the vehicles can communicate, can find the transformation between their VIO start frames, and the environment is sufficiently textured---a standard assumption for VIO systems.
As such, this framework paves the way for future, real-world deployments of aerial vehicle swarms in large numbers and without requiring external localization infrastructure.


\begin{figure} [t!]
\centering
	\begin{subfigure}[b]{0.32\columnwidth}
	   %
	    \includegraphics[width=0.8\textwidth,left]{Figs/Frames2_full.pdf}
	    \caption{\scriptsize full alignment}
	    \label{fig:frame-a}
	\end{subfigure}
	\begin{subfigure}[b]{0.32\columnwidth}
	    \includegraphics[width=0.8\textwidth,center]{Figs/Frames2_orientation.pdf}
	    \caption{\scriptsize orientation alignment}
	    \label{fig:frame-b}
	\end{subfigure}
	\begin{subfigure}[b]{0.32\columnwidth}
	    \includegraphics[width=0.8\textwidth,right]{Figs/Frames2_none.pdf}
	    \caption{\scriptsize no alignment}
        \label{fig:frame-c}
	\end{subfigure}
\caption{\edit{Required alignment of UAV frames in existing swarm strategies: (a) the most restrictive case requiring a common reference frame, i.e., orientation and origin of the frames must be aligned; (b) only the orientation of the frames must be aligned; (c) no alignment restrictions (this work).}}
	\label{fig:Frames}
\end{figure}




\subsection{Related Work}

Existing aerial swarms can be grouped based on the coordination (centralized vs.\ distributed) and localization (external vs.\ onboard) methods used. 
\edit{It is further crucial to distinguish these methods based on the level of alignment required for the vehicle coordinate frames; see Fig.~\ref{fig:Frames}.} 
 
\edit{
Works with \textit{centralized} coordination and \textit{external} localization include~\cite{Preiss2017, Honig2018, Du2019}, which are based on lightweight UAVs with limited onboard computational capability and therefore rely on an external motion capture system and a base station.
Works with \textit{distributed} coordination and \textit{external} localization include \cite{wilson2020robotarium}, \cite{enright2004spheres}, where robots execute distributed controls  based on external localization by motion capture and ultrasonic beacons, respectively.
Works with \textit{centralized} coordination and \textit{onboard} localization include~\cite{Forster2013}, \cite{Loianno2016}, which use a ground station for task assignment among vehicles.
In \cite{Weinstein2018}, formation flying based on VIO is demonstrated, where motion planning and assignment are run on a base station to ensure collision-free trajectories.
The coordination strategies used in aforementioned works require a \textit{common reference frame} (Fig.~\ref{fig:frame-a}).
}


\edit{
Despite the large body of work on formation control~\cite{Oh2015}, and the variety of onboard sensing solutions for localization (e.g., VIO~\cite{Delmerico2018}), few frameworks demonstrated formation flying with \textit{distributed} coordination and \textit{onboard} localization.
A key reason is reliance of many distributed control and assignment algorithms on aligned frames (Fig.~\ref{fig:frame-a}, \ref{fig:frame-b}), which require computation-expensive and/or communication-intensive synchronization/consensus steps for frame alignment.
Equally important, dependence on alignment in existing methods \cite{Wang2017,Turpin2014, van2011reciprocal, morgan2016swarm} diminishes robustness to inherent noise and unobservable errors that cannot be corrected (e.g., disparities between the actual and estimated body frame \textit{orientation} caused by VIO drift).
Leveraging coordination methods that are \textit{robust to misaligned frames} is hence crucial and a focus of this work. 
}






\edit{
Examples of other pipelines with distributed coordination and onboard localization include \cite{Montijano2016,Tron2016}.
Both works demonstrated formation flying on three UAVs, required information from an external motion capture system due to hardware limitations, did not incorporate collision avoidance, and required frame alignment.
}
\edittwo{Note that while~\cite{Montijano2016,Tron2016} can achieve formations with arbitrary headings as illustrated in Fig.~\ref{fig:frame-c}, knowledge of relative orientations is still required; therefore, they belong to the category of Fig.~\ref{fig:frame-b}.}






\if 0

\r{
decentralized coordination setting combined with VIO:
D-CAPT [26]~\cite{}:
ORCA ~\cite{}: 
CBF [2]~\cite{} :
[A]
}

\r{Robusteness in coordination,  with compounded noise/latency, which would eventually break (b).\\


some existing algorithm might as well
work in a similar fully decentralized setting, when combined with VIO
as proposed here. For example, D-CAPT [26], ORCA, CBF [2] might also be
useful for such a task and are computationally even more efficient than
the proposed approach. \\

R2:  onboard sensing for localization ->
 Finally, the related work section only
focuses on this aspect of the pipeline, discussing how many formation papers include
onboard localization but barely sells the advantages of the coordination module (the actual
proposal of the paper) against other competitors such as [26] or [A] or to mention similar
coordination pipelines. \\


Given a solution to this problem, the controller in Section III seems unnecessary, each drone
has a target position and can use a local controller with collision avoidance that drives it to
that position. Note that such controllers exists in the literature (e.g., RVO in any of its
multi-agent variantes), they are distributed in nature and only require local sensing.


}

\fi
Distributed systems have been maintaining their importance for the last several decades due to the increase in the need for scalable and reliable distributed applications while preserving high performance. 
To analyze distributed systems comprehensively and compare them in terms of features and services, various surveys and evaluations have been published in the past. Surveys on cloud providers, data warehouses, distributed file systems, or metadata services can be counted among them. 

Cloud providers are analyzed and evaluated in terms of elasticity \cite{CMART}, computing power \cite{comperative-benchmarking}, and cost to performance efficiency \cite{fair-benchmarking} in previous efforts. Widely used distributed services are also analyzed in many works, such as a survey on stream processing \cite{stream-benchmarking} or performance and dependability evaluation of MapReduce systems \cite{MapReduce-benchmarking}. Similarly, different aspects of distributed systems are studied in several surveys, like reliability analysis on distributed systems \cite{reliability-survey} and load balancing characteristics of known systems \cite{load-balancing-survey}.

As a big part of distributed systems, data warehouses and file systems are studied for many specifications. Evaluation of distributed data warehouses for the cost-effectiveness of different hardware configurations \cite{ALOJA} and query performance of distinct design choices\cite{benchmarking-data-warehouse} are among the known efforts in these works. Distributed file systems are examined in many past works for general concepts \cite{file-systems-concepts,file-systems-gen1} or specific applications such as distributed access control \cite{access-control-file-systems}. Due to the differences in optimization, design techniques, and the complex interactions between the file systems and other system components like the kernel or operating system, benchmarking distributed file systems is not trivial. To identify the important metrics for the evaluation of distributed file systems, researchers also studied benchmarking file systems \cite{File-system-benchmarking,benchmarking-file-rocket}. 

Analysis of distributed coordination services in terms of general characteristics and importance of coordination \cite{importance-of-coordination} and the comparison of existing algorithms \cite{paxos-made-simple} are among the published works. However, to the best of our knowledge, there is no published work on the evaluation of distributed coordination systems. As mentioned in the Introduction, due to the lack of standard benchmarking tools for distributed coordination services, developers widely use their ad-hoc benchmarks, which are prone to unfair comparisons or limited results for the evaluation of the systems. This study is unique in identifying the metrics and parameters for the evaluation of distributed coordination systems, discussing how each system uses these metrics and parameters for its evaluation, pinpointing the deficiencies of well-known benchmarking suites in evaluating distributed computing systems, and finally discussing the features of an ideal distributed coordination benchmark. 
%!TEX root = ../scifile.tex

\section{\myswarm{} --- Architecture}
\noindent
This section presents \myswarm{}, providing an overview of how it enables communication (\textbf{C1}), distributed computation (\textbf{C2}), and safe control (\textbf{C3}). 
%We also briefly describe our hardware implementation, with experimental results presented in Section~\ref{sec:experiments}.
The subsequent Section~\ref{sec:mlrdmpc} details the methodologies of \myswarm{}.


\begin{figure*}[h]
    \centering
    \includegraphics[width=0.9\textwidth]{Images/MLRDMPCOverview.pdf}

	\caption{\myswarm{} overview.
    \capt{\textbf{Left:} The physical swarm consisting of of UAVs and CUs connected via a wireless mesh network. \textbf{Right:} Swarm operations are structured in synchronized rounds, alternating between computation and many-to-all communication phases, facilitated by Mixer.}}

    % \caption{\myswarm{} overview. 
    % \capt{\textbf{Left:} The UAVs and CUs are connected over a wireless mesh network. \textbf{Right:} The whole swarm operation is structured in synchronized rounds, consisting of alternating calculation phases and many-to-all communication phases achieved by a protocol based on synchronous transmissions.
    % }}
    \label{fig:approach}
\end{figure*}

% \subsection{Overview} 
\label{sec:method:overview}

\subsection{Communication}
As described in Section~\ref{sec:relatedwork}, DMPC requires synchronized round-based many-to-all communication.
% Reliable many-to-all communication with bounded latency is crucial for \myswarm{}'s distributed operation. 
However, the inherent unreliability of wireless networks, combined with the high dynamics of moving UAVs, makes this challenging.

%For this, we leverage a wireless mesh protocol based on synchronous transmissions called Mixer~\citep{Mixer}.
%Mixer exploits the capture effect~\citep{Leentvaar1976} to accept overlapping transmissions typically avoided by traditional routing protocols and combines this with random linear network coding~\citep{Ho2006}.
%Through this, it achieves order-optimal scaling with the number of devices and messages. 
%Synchronous transmission protocols like Mixer have proven effective in control applications~\citep{FeedbackControlGoesWireless, baumann2019fast, PredictiveTriggering, Trobinger2021}.

Over the past decade, wireless mesh protocols based on synchronous transmissions have proven superior to traditional point-to-point routing approaches~\citep{zimmerling20st}.
The key insight is that packet collisions from transmission that overlap in time, space and frequency, can be successfully decoded due to the capture effect~\citep{Leentvaar1976} and non-destructive interference~\citep{herrmann22rssispy}.
This has two crucial implications: (\emph{i}) Unlike traditional routing protocols, synchronous transmission methods do not need to track topology changes to avoid packet collisions, making them highly resilient to such changes. (\emph{ii}) Synchronous transmissions enable distributed network nodes to achieve efficient time synchronization with sub-microsecond accuracy~\citep{glossy,chaos}, providing the foundation for real-time communication with formally proven end-to-end deadline guarantees~\citep{zimmerling17realtime}. 
As a result, protocols based on synchronous transmissions have enabled a range of powerful wireless control applications~\citep{FeedbackControlGoesWireless, baumann2019fast, PredictiveTriggering, Trobinger2021}.

%By transmitting and receiving messages over synchronized time slots, Mixer efficiently floods information throughout the network, establishing the many-to-all communication essential for DMPC. 
%This flooding mechanism eliminates the need to track continuous network topology changes caused by moving UAVs making it an efficient solution for UAV swarm communication.
%Furthermore, Mixer's high reliability minimizes message loss, allowing control to treat losses as exceptional events.

In \myswarm{}, we build on this prior work by leveraging a synchronous transmissions based protocol called Mixer~\citep{Mixer}. 
Mixer provides many-to-all communication across dynamic wireless mesh networks, essential for DMPC.
It does this also efficiently: Mixer achieves order-optimal scaling with the number of messages by integrating synchronous transmissions with random linear network coding~\citep{Ho2006}.
The network coding approach also provides additional reliability in real networks with fast-moving nodes like UAVs. %, boosting Mixer's communication reliability above 99.999\,\% 
This high robustness and reliability allows us to treat message losses as exceptional events when designing the control system in \myswarm{}.

In \myswarm{}, Mixer synchronizes the devices into discrete-time rounds indexed by~$k$ and with a fixed duration~$T$ (see Figure~\ref{fig:approach}).
Each round comprises a computation and a communication phase, which last~$T_\mathrm{calc}$ and~$T_\mathrm{com}$, respectively. 
Devices begin computation simultaneously at~$t = kT$ and begin to communicate at $t = kT + T_\mathrm{calc}$. During communication, devices concurrently send messages in dedicated, synchronized time slots.
Mixer efficiently broadcasts these messages, all devices can receive the complete set after $T_\mathrm{com}$. This mechanism provides the timely, reliable data exchange essential for MLR-DMPC.

\subsection{Distributed Computation}

Nano-UAV swarms face a fundamental conflict between limited onboard compute resources and the high computational demands of DMPC. 
To address this, we leverage the swarm's $M$ CUs for heavy computations (cf.\ Section~\ref{sec:introduction:problemsetting}). %following prior work \citep{Graefe2022} 
% Because $M>1$, control computations will be distributed, thus DMPC-Swam enjoys the benefits of distributed implementation such as scalability.
% Due to weight constraints limiting UAVs' onboard computational power, we leverage the swarm's CUs (cf.\ Section~\ref{sec:introduction:problemsetting}).
% % This network includes digital devices from swarm users or ground-based robots with greater computational capabilities, called CUs. 
% Building on previous DMPC work~\citep{Graefe2022}, we offload demanding computations to these CUs.

Although each CU can compute a trajectory for one UAV at once via DMPC, maintaining a one-to-one ratio of CUs to UAVs ($M = N$) is inefficient---it is costly, overloads the compute network, and demands high communication bandwidth. 
Therefore, \myswarm{} employs significantly fewer CUs ($M < N$). 
In each round~$k$, the CUs compute new trajectories for $M$ of the $N$ UAVs, while the rest follow their previously planned paths. 
A distributed, priority-based event trigger determines which UAVs need new trajectories and schedules computations on the CUs (see Section~\ref{sec:mlrdmpc:algorithm:eventtrigger}).



\subsection{Overview of MLR-DMPC}
\label{sec:method:mldrmpc}

\myswarm{} leverages Mixer's round-based structure of alternating computation and communication phases for MLR-DMPC (see Figure~\ref{fig:approach}). 
A key innovation of MLR-DMPC over existing DMPCs is its ability to handle message loss. 
We provide a brief overview here; details are in Section~\ref{sec:mlrdmpc}.

To ensure effective DMPC, the CUs must be informed of all UAV actions, which Mixer's many-to-all communication facilitates.
The CUs monitor received UAV activities using information trackers.

At the start of each computation phase, each CU checks whether its information trackers are up-to-date. 
If they are, the CU solves a QP for its assigned UAV and broadcasts the solution during the next communication phase. 
If crucial information is missing---indicating a critical message loss that prevents safe trajectory calculation---the CU does not proceed. 
Instead, the CU identifies the missing information and requests it in the subsequent communication phase.
Although such instances are rare due to Mixer's reliability, they must be addressed. 

Consequently, UAVs receive only recursively feasible trajectories that ensure their safety—even if they do not receive an update due to message loss or not being selected by the trigger. 
The UAVs track these trajectories using high-frequency, low-level controllers common in DMPC~\citep{Park2022, Park2023, Chen2023, Chen2022, Graefe2022}. 
% For this purpose, \myswarm{} employs a standard controller~\citep{mellinger2011minimum}.

% At the start of each computation phase, each CU checks if it has information to compute a safe trajectory. 
% If crucial information is missing, a \textbf{critical message loss} has occurred. 
% Although such cases are rare due to Mixer's reliability, addressing them is essential for guarantees.
% If no critical message loss has occurred, the CU solves a QP for its assigned UAV and broadcasts the solution during the next communication phase. 
% If a critical message loss has occurred, the CU identifies missing information, and requests it during the next communication phase. 
% % The next MLR-DMPC round begins after communication.

% Meanwhile, each UAV runs a high-frequency, low-level controller to track the trajectories provided by the CUs as commonly used in DMPC~\citep{Park2022, Park2023, Chen2023, Chen2022, Graefe2022}. 
% \myswarm{} employs a standard controller~\citep{mellinger2011minimum} for this purpose.


\section{Distributed Model Predictive Control with Message Loss Recovery}
\label{sec:mlrdmpc}

This section details MLR-DMPC. 
After describing the UAV model forming the base for MLR-DMPC, we provide an in-depth explanation of MLR-DMPC and prove collision avoidance.

\subsection{UAV Model}

All $N$ UAVs have nonlinear dynamics
\begin{align}
    \label{eq:uavdynamics}
    \forall i \in \mathcal{A}:\quad&\dot{x}_i(t) = f_i(x_i(t), u_i(t)) + v_i(t),\nonumber\\ &x_i(0) = x_{i,0},
\end{align}
where $x_i \in \mathcal{X} \subseteq \mathbb{R}^n$ is the state vector, $u_i \in \mathcal{U} \subseteq \mathbb{R}^m$ is the control input, $f_i: \mathcal{X}\times\mathcal{U}\to\mathcal{X}$, and $v_i \in \mathcal{V} \subseteq \mathbb{R}^n$ represents disturbances.
The position of UAV $i$ at time $t$ is given by $p_i(t) = g_{\mathrm{p}, i}(x_i(t))$, where $g_{\mathrm{p}, i}: \mathcal{X} \rightarrow \mathcal{P} \subseteq \mathbb{R}^3$ maps the state to the position.
Typically, $x_i$ includes the position, making $g_{\mathrm{p}, i}$ straightforward (e.g., $p_i(t) = [I, 0] x_i(t)$).
For detailed quadcopter dynamics, see~\citep{mahony2012multirotor, antal2023modelling, panerati2021learning}.
Each UAV must navigate from its initial position $p_{i, \mathrm{init}} = g_{\mathrm{p}, i}(x_{i,0})$ to its target $p_{i, \mathrm{target}}$ while avoiding collisions (Equation~\ref{eq:truecollisionavoidance}).
% Collision avoidance is defined by ensuring the vector difference between any two UAV positions lies outside an ellipsoid:
% \begin{align} 
%     \label{eq:truecollisionavoidance} 
%     \forall t\geq 0,&;~\forall i,j\in\mathcal{A},~i\neq j:\\ 
%     &\left\| \Theta^{-1}[p_j(t)-p_i(t)] \right\|_2 \geq d_\mathrm{min},\nonumber 
% \end{align}
% where $\Theta$ is a scaling matrix for downwash effects, and $d_\mathrm{min}$ is the minimum allowable distance~\citep{Luis2019, Luis2020, Chen2022, Chen2023, Graefe2022}.

% Due to the complexity of $f_i$, we simplify the dynamics for real-time optimization, as common in (D)MPC~\citep{augugliaro2012generation, Luis2019, Luis2020, Chen2022, Chen2023, Graefe2022}.
\reviewerall{
Instead of using the complex UAV dynamics $f_i$, we define a simpler nominal system for each UAV $i \in \mathcal{A}$, as common in (D)MPC~\citep{augugliaro2012generation, Luis2019, Luis2020, Chen2022, Chen2023, Graefe2022}
\begin{equation}
    \label{eq:nominalsystem}
    \dot{\hat{x}}_i = \hat{f}_i(\hat{x}_i, \hat{u}_i),\quad \hat{x}_i(0) = \hat{x}_{i,0},
\end{equation}
where $\hat{x}_i \in \hat{\mathcal{X}}$ is the nominal state with $\hat{x}_{i,0}$ its initial value, $\hat{u}_i\in\hat{\mathcal{U}}$ and $\hat{f}_i:\hat{\mathcal{X}}\times\hat{\mathcal{U}}\to\hat{\mathcal{X}}$.
We define the function $\hat{g}_{\mathrm{p}, i}:\hat{\mathcal{X}}\to\mathcal{P}$ that maps the nominal state to the nominal position $\hat{p}_i(t) = \hat{g}_{\mathrm{p}, i}(\hat{x}_i(t))$.

We further design a controller $c_i: \mathcal{X} \times \hat{\mathcal{X}} \times \hat{\mathcal{U}} \rightarrow \mathcal{U}$ that controls the UAV
\begin{equation}
    \dot{x}_i = f_i\big(x_i,c_i(x_i, \hat{x}_i, \hat{u}_i)\big) + v_i,
\end{equation} 
steering it towards the nominal state.
In the following, we will not explicitly specify $\hat{f}_i$ and $c_i$, instead, we will make the following assumption similar to incremental asymptotic stability~\citep{kohler2018novel}.

\begin{assumption}
\label{as:tracking}
% For each UAV $i \in \mathcal{A}$, define the nominal system
% \begin{equation}
%     \label{eq:nominalsystem}
%     \dot{\hat{x}}_i = \hat{f}_i(\hat{x}_i, \hat{u}_i),\quad \hat{x}_i(0) = \hat{x}_{i,0},
% \end{equation}
% where $\hat{x}_i \in \hat{\mathcal{X}}$ is the nominal state with $\hat{x}_{i,0}$ its initial value, $\hat{u}_i\in\hat{\mathcal{U}}$ and $\hat{f}_i:\hat{\mathcal{X}}\times\hat{\mathcal{U}}\to\hat{\mathcal{X}}$.
% We define the function $\hat{g}_{\mathrm{p}, i}:\hat{\mathcal{X}}\to\mathcal{P}$ that maps the nominal state to the nominal position $\hat{p}_i(t) = \hat{g}_{\mathrm{p}, i}(\hat{x}_i(t))$.

We assume that for arbitrary inputs $\hat{u}_i \in \hat{\mathcal{U}} \subseteq \mathbb{R}^{\hat{m}}$, there exists a constant $\Delta d_\mathrm{min}\in\mathbb{R}_{\geq 0}$ such that if $||\hat{p}_i(0)-p_i(0)||=||\hat{g}_{\mathrm{p}, i}(\hat{x}_i(0))-g_{\mathrm{p}, i}(x_i(0))||\leq \Delta d_\mathrm{min}$,
then for all $t\in\mathbb{R}_{\geq 0}$:
\begin{equation}
    \left\| p_i(t) - \hat{p}_i(t) \right\|_2 \leq \Delta d_\mathrm{min}.
\end{equation}
\end{assumption}
}

% Under this assumption, the CU can generate reference trajectories using $\hat{f}_i$ and the UAVs can apply the controller $c_i$ to ensure they follow them closely.

We can satisfy this assumption by selecting $\hat{f}_i$ as a fourth-order integrator in combination with an appropriate controller leveraging the differential flatness of quadcopters~\citep{mellinger2011minimum}. 
However, empirical studies show that second- or third-order integrators suffice for generating smooth, trackable trajectories while reducing QP complexity~\citep{augugliaro2012generation, Luis2019, Luis2020, Chen2022, Chen2023, Graefe2022}. 
Therefore, \myswarm{} uses a third-order integrator for $\hat{f}_i$, with input $\hat{u}_i$ representing the jerk~\citep{Graefe2022}, balancing simplicity and trajectory smoothness. 
These trajectories are followed by a standard controller~\citep{mellinger2011minimum} acting as $c_i$.
\reviewerall{As our theoretical analysis does not depend on a specific formulation of $\hat{f}_i$, we will continue with an arbitrary $\hat{f}_i$ without giving a specific formula.}

During each round~$k$, a UAV has a predicted reference trajectory $\hat{x}_i(\tau|k)$ and input $\hat{u}_i(\tau|k)$ of the nominal system~\eqref{eq:nominalsystem} computed by a CU, where $\tau$ denotes the time within the prediction horizon. 
With these predictions, the UAV generates the references
\begin{align}
    \hat{x}_i(t) &:= \hat{x}_i(\tau|k), \\
    \hat{u}_i(t) &:= \hat{u}_i(\tau|k),
\end{align}
and follows them using its low-level controller $c_i$ at time $t = \tau + kT$:
\begin{equation}
    u_i(t) = c_i\big[x_i(t),\hat{x}_i(\tau|k),\hat{u}_i(\tau|k)\big].
\end{equation}
When a UAV receives a new trajectory $\hat{u}\uli{i,w}$ from a CU $w$ during the communication phase at round $k$, it updates its reference trajectory at the beginning of round $k+1$. Otherwise, it reuses its old trajectory, either because no CU computed a new trajectory for it or because the UAV missed it due to message loss:
\begin{align} 
	\label{eq:nottriggered} 
	\hat{u}\uli{i}&(\tau|k+1) \\ 
	&= \begin{cases} \hat{u}\uli{i,w}(\tau+T|k) & \parbox[c]{.4\linewidth}{if received new trajectory from a CU $w$,}\\ 
		\hat{u}\uli{i}(\tau+T|k) & \text{otherwise.}\nonumber \end{cases} 
\end{align}


\subsection{MLR-DMPC Algorithm}
\label{sec:mlrdmpc:algorithm}

\definecolor{communicationcolor}{HTML}{fdd48f}
\definecolor{information-trackercolor}{HTML}{d2c0cd}
\definecolor{dmpccolor}{HTML}{bfbfbf}
\definecolor{mlrcolor}{HTML}{8ebae5}

\setlength{\fboxrule}{0pt}
\setlength{\fboxsep}{3pt}

\newcommand{\myindent}{\hspace*{2em}}

\newcommand{\algcolorboxpar}[2]{\vspace*{-\fboxsep}\hspace*{-\fboxsep}\colorbox{#1}{\parbox{\dimexpr\linewidth-2\fboxsep}{#2}}}

\newcommand{\algcolorbox}[2]{%
  {\hskip-\ALG@thistlm\vspace*{-\fboxsep}\hspace*{-\fboxsep}\colorbox{#1}{\parbox{\dimexpr\linewidth-2\fboxsep}{{\hskip\ALG@thistlm\relax {\tiny\strut} #2}}}}%
}

\newcommand{\myif}[1]{\textbf{if} #1 \textbf{then}}
\newcommand{\myelif}[1]{\textbf{elif} #1 \textbf{then}}
\newcommand{\myendif}{\textbf{endif}}
\newcommand{\mywhile}[1]{\textbf{while} #1 \textbf{do}}
\newcommand{\myendwhile}[0]{\textbf{endwhile}}
\newcommand{\varendash}[1][5pt]{%
  \makebox[#1]{\leaders\hbox{--}\hfill\kern0pt}%
}


\begin{algorithm*}[h]
	\caption{MLR-DMPC running on CU $w$.}\label{alg:mlrdmpc}
	\fontsize{9pt}{9pt}\selectfont
	\begin{algorithmic}[1]
		\State\algcolorbox{white}{$k\gets 0$  \Comment{Current round}}
		\State\algcolorbox{white}{$\forall i\in \mathcal{A}$~$\mathcal{D}\uli{iw}(k) \gets \{\}$~~~$\mathcal{D}_w(k)\gets\{\mathcal{D}\uli{iw}(k)|\forall i\in \mathcal{A}\}$ \Comment{Init information-trackers as empty.}}
		\State\algcolorbox{white}{$\forall i\in \mathcal{A}$ setDeprecated($\mathcal{D}\uli{iw}(k)$)}
		
		\State\algcolorbox{white}{state $\gets$ REQUEST\_TRAJECTORY \Comment{Because information-trackers are empty.}}
		
		\State\algcolorbox{white}{UAV\_messages, cu\_messages $\gets \emptyset$}
		
		\State\algcolorbox{white}{\mywhile{true}}
			%%%%%%%%%%%%%%%%%%%% information-tracker %%%%%%%%%%%%%%%%%%%%%
			\State \algcolorbox{information-trackercolor}{\myindent\varendash[0.3\linewidth]\textit{Trajectory-Tracker Update} \varendash[0.3\linewidth]}
			\State  \algcolorbox{information-trackercolor}{\myindent$\mathcal{D}_w(k)\gets$updateInformationTrackers(UAV\_messages, cu\_messages, $\mathcal{D}_w(k-1)$) \label{line:updateinformation-tracker} \Comment{Algorithm~\ref{alg:processmessages}}}  \label{line:computationphase:begin}
			\State \algcolorbox{information-trackercolor}{\myindent\myif{allInformationTrackersUpToDate()} state $\gets$ RUN\_DMPC}
			\State \algcolorbox{information-trackercolor}{\myindent\myelif{state $==$ RUN\_DMPC} state $\gets$ WAIT}\label{lst:line:activatemlr}
			\State \algcolorbox{information-trackercolor}{\myindent\myendif{}}

			%%%%%%%%%%%%%%%%%%%% DMPC %%%%%%%%%%%%%%%%%%%%%
			\State \algcolorbox{dmpccolor}{\myindent\varendash[0.3\linewidth]\textit{DMPC} \varendash[0.475\linewidth]}
			\State \algcolorbox{dmpccolor}{\myindent\myif{state $==$ RUN\_DMPC}}
				\State \algcolorbox{dmpccolor}{\myindent\myindent prios$\gets$ prioConsensus(cu\_messages) \label{line:prio}\Comment{Event-trigger}}
				\State \algcolorbox{dmpccolor}{\myindent\myindent selected\_UAV $\gets$ selectUAV(prios) \label{line:et}}
				\State \algcolorbox{dmpccolor}{\myindent\myindent tx\_message $\gets$ EmptyMessage}
				\State \algcolorbox{dmpccolor}{\myindent\myindent \myif{$|\mathcal{D}_{\mathrm{selected\_UAV},w}(k)|==1$}} \label{lst:line:onlyoneentrance}
					\State \algcolorbox{dmpccolor}{\myindent\myindent\myindent $\hat{x}_{i,w}(\tau|k),\hat{u}_{i,w}(\tau|k)\gets$ solveOptimization(selected\_UAV,$\mathcal{D}_w(k)$) \label{line:optimization}}
					\State \algcolorbox{dmpccolor}{\myindent\myindent\myindent tx\_message $\gets$ TrajectoryMessage($\hat{x}_{i,w}(\tau|k),\hat{u}_{i,w}(\tau|k)$)} 
				\State \algcolorbox{dmpccolor}{\myindent\myindent\myendif}

			%%%%%%%%%%%%%%%%%%%% MLR %%%%%%%%%%%%%%%%%%%%%
			\State \algcolorbox{mlrcolor}{\myindent\varendash[0.3\linewidth]\textit{MLR} \varendash[0.49\linewidth]}
			\State \algcolorbox{mlrcolor}{\myindent\myelif{state $==$ WAIT}   \label{line:mlrbegin}					\Comment{\parbox[t]{.5\linewidth}{Immediately after entering MLR, we do not exactly know which trajectories are unknown.}}} %Hence, we wait one timestep, to compare the information-trackers with the metadata.}}	
				\State \algcolorbox{mlrcolor}{\myindent\myindent tx\_message $\gets$ EmptyMessage}
				\State \algcolorbox{mlrcolor}{\myindent\myindent state $\gets$ REQUEST\_TRAJECTORY}
			\State\algcolorbox{mlrcolor}{\myindent\myelif{state $==$ REQUEST\_TRAJECTORY} \label{line:reqtraj}}
				\State \algcolorbox{mlrcolor}{\myindent\myindent requested\_UAV $\gets$ selectUAVWithDeprecatedInformationTracker($\mathcal{D}_w(k)$)}
				\State \algcolorbox{mlrcolor}{\myindent\myindent tx\_message $\gets$ TrajectoryRequestMessage(requested\_UAV)}
				\State \algcolorbox{mlrcolor}{\myindent\myindent state $\gets$ WAIT\_FOR\_UPDATE}
			\State\algcolorbox{mlrcolor}{\myindent\myelif{state $==$ WAIT\_FOR\_UPDATE}}
				\State \algcolorbox{mlrcolor}{\myindent\myindent tx\_message $\gets \emptyset$
                    \Comment{\parbox[t]{.5\linewidth}{Do not send anything, because requested UAV uses communication resources.}} \label{line:sendnothing}}
				\State \algcolorbox{mlrcolor}{\myindent\myindent state $\gets$ REQUEST\_TRAJECTORY}
			\State \algcolorbox{mlrcolor}{\myindent\myendif{}		 \label{line:mlrend}}	\label{line:computationphase:end}

			%%%%%%%%%%%%%%%%%%%% Communication %%%%%%%%%%%%%%%%%%%%%
			\State \algcolorbox{communicationcolor}{\myindent\varendash[0.3\linewidth]\textit{Communication Phase} \varendash[0.3\linewidth]} \label{line:communicationphase:begin}
			\State \algcolorbox{communicationcolor}{\myindent tx\_message $\gets$ \{tx\_message, calcPrio($D_w(k)$)\} \textbf{if} tx\_message $!=$ $\emptyset$ \label{line:calcprio}}
			\State \algcolorbox{communicationcolor}{\myindent UAV\_messages, cu\_messages = wirelessBusRound(tx\_message) \label{line:mixerround}}
			\State  \algcolorbox{communicationcolor}{\myindent$k\gets k+1$} \label{line:communicationphase:end}
		\State \algcolorbox{white}{\myendwhile}
	\end{algorithmic}
\end{algorithm*}




\begin{algorithm*}[t]
	\caption{Algorithm that updates the trajectory information-trackers of the MLR-DMPC.}\label{alg:processmessages}
	\fontsize{9pt}{9pt}\selectfont
	\begin{algorithmic}[1]
		\Procedure{updateInformationTrackers}{UAV\_messages, cu\_messages, $\mathcal{D}_w(k-1)$}
		\State $\forall \mathcal{D}\uli{iw}(k-1)\in \mathcal{D}_w(k-1):$ ~ $\mathcal{\tilde{D}}\uli{iw}(k) \gets \mathcal{D}\uli{iw}(k-1)$ \label{lst:line:metadatacompbegin}
		\For{($i$, message) $\in$ enumerate(UAV\_messages)} \Comment{\parbox[t]{.3\linewidth}{Which trajectories received in the last round}}
		\For{trajectory $\in\mathcal{\tilde{D}}\uli{iw}(k)$}
		\If{trajectory.metadata $==$ message.metadata}
		\State $\mathcal{\tilde{D}}\uli{iw}(k) \gets \{\text{trajectory}\}$ \label{lst:line:update}
		\State setInformationTrackerUpToDate($i$)
		\State \textbf{break}
		\EndIf
		\EndFor
		\EndFor \label{lst:line:metadatacompend}
		\If{$\exists i\in\mathcal{A}$ s.t. $|\mathcal{\tilde{D}}\uli{iw}(k)| > 1$}
		\State setAllInformationTrackersDeprecated() \label{lst:line:deprecatedone}
		\EndIf
		\State \label{lst:line:updateinformationtrackerbegin}
		\State $\mathcal{D}\uli{iw}(k) \gets \mathcal{\tilde{D}}\uli{iw}(k)$
		\If{\textbf{not} length(cu\_messages) == $M$}
		\State setAllInformationTrackersDeprecated() \label{lst:line:deprecatedtwo}
		\EndIf
		\For{(message, $i$) $\in$ enumerate(cu\_messages)}
		\If{type(message) \textbf{is} TrajectoryMessage}
		\State $\mathcal{D}\uli{iw}(k) \gets \mathcal{D}\uli{iw}(k) \cup$\{message\} \label{lst:line:setnewtraj}
		\EndIf
		\EndFor \label{lst:line:updateinformationtrackerend}
		\State \Return $\{\mathcal{D}\uli{iw}(k)|\forall i\in\mathcal{A}\}$
		\EndProcedure
	\end{algorithmic}
\end{algorithm*}

% \myswarm{} computes trajectories using a DMPC~\citep{Luis2019,Luis2020,Park2022,Park2023,Chen2022,Chen2023,van2017distributed,Graefe2022}. 
% The fundamental idea is to generate reference trajectories by solving optimization problems at each round on multiple devices (e.g., on the UAVs~\citep{Luis2019,Luis2020,Park2022,Park2023,Chen2022,Chen2023,van2017distributed}, or on  CUs~\citep{Graefe2022}) and sharing the results among them.
% However, existing approaches are not robust against message loss.
% By contrast, \myswarm{} uses MLR-DMPC, a novel approach combining an MLR-module with constraints in the DMPC to handle message loss.

Using MLR-DMPC, the CUs concurrently calculate the trajectories $\hat{u}\uli{i,w}$ for the UAVs. 
Building upon the brief overview presented in Section~\ref{sec:method:mldrmpc}, we now present its details step by step going through Algorithm~\ref{alg:mlrdmpc}.

Each CU alternates between computation (Lines~\ref{line:computationphase:begin}--\ref{line:computationphase:end}) and communication phases (Lines~\ref{line:communicationphase:begin}--\ref{line:communicationphase:end}), with a state machine scheduling operations during computation.

A key component of the computation phase is the information tracker $\mathcal{D}\uli{iw}(k)$, which CU $w$ uses to store trajectory candidates for UAV $i \in \mathcal{A}$ at time $k$, considering one trajectory as currently followed by the UAV.
\reviewerthree{We mark trajectories saved in these trackers with a tilde symbol, e.g., $\tilde{x}$. 
For simplicity and brevity, we abuse the ``element of $\mathcal{D}$'' notation for states, inputs, and positions ambiguously, i.e., we note $\tilde{u}\in \mathcal{D}\uli{iw}(k), \tilde{x}\in \mathcal{D}\uli{iw}(k), \tilde{p} \in \mathcal{D}\uli{iw}(k)$, although $\tilde{u}$, $ \tilde{x}$ and $\tilde{p}$ are of different types and dimensions.}
Due to the delay $T$ between rounds, each trajectory in $\mathcal{D}\uli{iw}(k)$ begins at $k - 1$, e.g., $\tilde{u}_i(\cdot\,|\,k - 1) \in \mathcal{D}\uli{iw}(k)$.

At the start of each computation phase, the CU updates its information trackers (Line~\ref{line:updateinformation-tracker}) using messages from the previous communication phase (Line~\ref{line:mixerround}). 
If it detects critical message loss, it marks the affected trackers as deprecated, indicating they may not contain the UAV's actual trajectory (details in Section~\ref{sec:mlrdmpc:algorithm:informationtracker}).

If all information trackers are up-to-date, the CU executes a DMPC step: selecting a UAV $i$ using the event trigger (Lines~\ref{line:prio} and~\ref{line:et}; Section~\ref{sec:mlrdmpc:algorithm:eventtrigger}) and solving an optimization problem (Line~\ref{line:optimization}) to compute the new reference trajectory $\hat{x}\uli{i,w}(\tau\,|\,k)$ and $\hat{u}\uli{i,w}(\tau\,|\,k)$.
The optimization problem uses the information trackers to formulate anti-collision constraints (details in Section~\ref{sec:mlrdmpc:algorithm:opt}).

If any information tracker is deprecated, the CU initiates MLR (Lines~\ref{line:mlrbegin}--\ref{line:mlrend}) by requesting a missing trajectory from a UAV (Line~\ref{line:reqtraj}). 
In the next round, the UAV transmits its current reference trajectory, using the communication resources typically reserved for the CU because its own are occupied with state information and target position. Accordingly, the CU refrains from transmitting during this round to free communication resources for the UAV (Line~\ref{line:sendnothing}).

At the end of each step, during the communication phase, the CU broadcasts a message containing the MLR-DMPC results and the computed UAV priorities for the next round's event trigger to all other CUs and UAVs (Line~\ref{line:mixerround}), unless it has donated its communication resources to a UAV.


\subsubsection{Information-Tracker Updates}
\label{sec:mlrdmpc:algorithm:informationtracker}

Algorithm~\ref{alg:processmessages} details how each CU updates its information trackers. 
It first extracts the UAVs' current reference trajectories from the received messages. 
Each UAV's message includes unique metadata, specifically, the calculation time and the responsible CU, about its current reference trajectory. 
The CU compares this metadata with that in its information trackers; if they match, it retains the trajectory as the one the UAV is following (Line~\ref{lst:line:update}).

If no new trajectory was planned for a UAV in the previous round, the stored trajectory is the current one. 
However, if a CU did plan a new trajectory, the UAV may be following either the new trajectory (if it received it) or the old one (if it missed the message). 
The CU therefore retains both trajectories in the information tracker (Line~\ref{lst:line:setnewtraj}).

With this information, the CU checks for critical message loss. 
If it received fewer CU messages than the total number of CUs, it might lack the actual trajectory of an unknown UAV, so it marks all information trackers as deprecated (Line~\ref{lst:line:deprecatedtwo}). 
If the CU did not receive a required message from a UAV with necessary metadata to determine which trajectory the UAV followed, it marks that UAV's information tracker as deprecated (Line~\ref{lst:line:deprecatedone}).


\subsubsection{Event-Trigger}
\label{sec:mlrdmpc:algorithm:eventtrigger}

The event-trigger aims to assign $M$ of the $N$ UAVs to the CUs in a distributed manner, avoiding a single point of failure, using priorities similar to~\cite{PredictiveTriggering, Graefe2022}.

At each round $k-1$, each CU $w$ calculates a priority $J\uli{i,w}(k-1)$ for every UAV $i$ (see Algorithm~\ref{alg:mlrdmpc}, Line~\ref{line:calcprio}).
Appendix~\ref{app:et} outlines three triggers with their priority calculation: round-robin (RR) selecting UAVs periodically; distance-based (DB) selecting UAVs farthest from their targets; and a hybrid trigger (HT) combining RR and DB, selecting UAVs based on both their distance to targets and time since last assigned to a CU.

Message loss (e.g., missed target positions) can cause discrepancies among CUs, leading them to compute different priorities for the same UAVs. To address this, CUs employ a consensus algorithm. During communication, they share their computed priorities.
CUs unify them by taking the maximum across all CUs:
\begin{equation}
\label{eq:priosmax}
J\uli{i}(k) = \max_{w} J\uli{i,w}(k-1).
\end{equation}
They sort the UAVs based on these unified priorities and store the top $M$ UAVs in the set $\mathcal{A}_{\mathrm{ET}}(k)$. 
Since all active CUs receive the same priority information, they compute the same set $\mathcal{A}_{\mathrm{ET}}(k)$.

Each CU $w$ selects the UAV at position $((k + w) \bmod M)$ in $\mathcal{A}_{\mathrm{ET}}(k)$\footnote{The modulo operation ensures each priority rank is selected, even when a CU misses multiple messages.}. 
The set of UAVs for which new trajectories are computed is denoted by $\tilde{\mathcal{A}}_{\mathrm{ET}}(k) \subseteq \mathcal{A}_{\mathrm{ET}}(k)$, since some CUs may be in MLR mode and not compute a new trajectory.


\subsubsection{Trajectory Calculation}
\label{sec:mlrdmpc:algorithm:opt}

\reviewerthree{The trajectory computed by CU $w$ for UAV $i$ has a piecewise-constant input with sampling time $T\ul{s}$ and horizon $h\ul{s}$.
The constant input from time $T\ul{s}\kappa$ till $T\ul{s}(\kappa+1)$ for $\kappa\in\{0,...,h\ul{s}-1\}$ is denoted as $u\uli{i,w, \kappa|k}$.
Formally, the trajectory of the system is
\begin{equation}
    \hat{u}\uli{i,w}(\tau+T|k) = \sum_{\kappa=0}^{h\ul{s}-1}\Gamma\ul{T\ul{s}}(\tau-\kappa T\ul{s})u\uli{i,w, \kappa|k},
\end{equation}
where $\Gamma\ul{T\ul{s}}(t) = 1$ if $0 \leq t \leq T\ul{s}$ and $\Gamma\ul{T\ul{s}}(t) = 0$ otherwise~\citep{Graefe2022}. 
Hence, $\Gamma\ul{T\ul{s}}(\tau-\kappa T\ul{s})u\uli{i,w, \kappa|k}=u\uli{i,w, \kappa|k}$, between $T\ul{s}\kappa$ till $T\ul{s}(\kappa+1)$, and it is zero otherwise.
We require $\tfrac{T}{T\ul{s}} = \tfrac{h\ul{s}}{H}$, where $H \in \mathbb{N}$ is the prediction horizon, so the prediction extends to time $kT + HT + T$. 
An offset of $T$ accounts for the latency of one round.
}

\reviewerthree{
The CU aims to determine this input trajectory by minimizing the quadratic distance between the state trajectory and its target, scaled by positive definite matrices $Q\uli{i}$ and $R\uli{i}$. 
This trajectory is constrained by the nominal UAV model $\hat{f}_i$ (Assumption~\ref{as:tracking}). 
The initial state $\tilde{x}\uli{i}(2T|k-1)$ is the sole entry in $\mathcal{D}_{i,w}(k)$ (cf.\ Algorithm~\ref{alg:mlrdmpc}, Line~\ref{lst:line:onlyoneentrance}). 
Both the input and state are confined to $\hat{\mathcal{U}}$ and $\hat{\mathcal{X}}$, respectively, thereby restricting the movement area of the UAV.
Through an equality constraint, the CU ensures that the UAV halts at the end of the computed trajectory, thereby guaranteeing recursive feasibility of the optimization problem~\citep{Graefe2022,Park2022,Chen2022,Chen2023}. Lastly, to ensure collision-free trajectories, the CU imposes time-varying BVC constraints described below.

In summary, the CU solves:

% The CU obtains this trajectory by solving:
\begin{subequations}
	\label{eq:opt}
	\begin{align}
		\label{eq:objectfunc}
		\min_{\hat{u}\uli{i,w, \cdot|k} }\sum_{\kappa=0}^{h\ul{o}}\big[&||\hat{x}\uli{i,w}(\kappa T\ul{o}+T|k)-\hat{x}\uli{i, \mathrm{target}}||_{Q\uli{i}}^2\nonumber\\
		& + ||\hat{u}\uli{i,w}(\kappa T\ul{o}+T|k)||_{R\uli{i}}^2\big]
	\end{align}
	\begin{align}
		\label{eq:dynconst}
		\text{s.t.~} & \dot{\hat{x}}\uli{i,w}(t|k) = \hat{f}\uli{i}(\hat{x}\uli{i,w}(t|k), \hat{u}\uli{i,w}(t|k))\\
		&\hat{x}\uli{i,w}(T|k)=\tilde{x}\uli{i}(2T|k-1)\nonumber
	\end{align}
	\begin{equation}
		\label{eq:inputconst}
		% u\uli{i, \mathrm{min}} \leq \hat{u}\uli{i,k, \ell|k} \leq u\uli{i, \mathrm{max}}, \forall\ell\in\left\{0,...,h\ul{s}-1\right\}
            \hat{u}\uli{i,w, \ell|k} \in \hat{\mathcal{U}} ~\forall\ell\in\left\{0,...,h\ul{s}-1\right\}
	\end{equation}
	\begin{equation}
		\label{eq:stateconst}
		\hat{x}\uli{i,w}(\kappa T\ul{b} + T|k)\in\hat{\mathcal{X}}~
		\forall\kappa\in \{0,...,h\ul{b}\}
	\end{equation}
	\begin{equation}
		\label{eq:feascond}
		0 = \hat{f}\uli{i}(\hat{x}\uli{i,w}(HT+T|k), 0)
	\end{equation}

	\begin{align}
		\forall  j&\in\mathcal{A}\backslash\{i\}:\label{eq:anticollisionconst}\\
		&A\uli{i, j, \mathrm{c}}\begin{bmatrix}
			\hat{p}\uli{i,w}(T|k)\\
			\hat{p}\uli{i,w}(T+T\ul{c}|k)\\
			\vdots\\
			\hat{p}\uli{i,w}(T+h\ul{c}T\ul{c}|k)
		\end{bmatrix} \leq b\uli{i, j, \mathrm{c}}(\mathcal{D}_{j,w}(k))\nonumber
	\end{align}
\end{subequations}
\noindent
where $\hat{x}\uli{i, \mathrm{target}}$ satisfies $p\uli{i, \mathrm{target}} = \hat{g}\uli{\mathrm{p}, i}(\hat{x}\uli{i, \mathrm{target}})$. 
}
% The objective function and constraints have different sampling times $T\ul{o}$, $T\ul{b}$, and $T\ul{c}$ with $\tfrac{T}{T\ul{o}} = \tfrac{h\ul{o}}{H}$, $\tfrac{T}{T\ul{b}} = \tfrac{h\ul{b}}{H}$, and $\tfrac{T}{T\ul{c}} = \tfrac{h\ul{c}}{H} \in \mathbb{N}$ to allow for an independent adjustment of their resolution~\citep{Graefe2022}.

% The CU aims to determine a trajectory that minimizes the quadratic distance between the trajectory and its target, scaled by $Q\uli{i}$ and $R\uli{i}$. The trajectory is constrained by the nominal UAV model $\hat{f}_i$ (Assumption~\ref{as:tracking}). 
% The initial state is the sole entry in $\mathcal{D}_{i,w}(k)$, $\tilde{x}\uli{i}(2T|k-1)$ (cf.\ Algorithm~\ref{alg:mlrdmpc}, Line~\ref{lst:line:onlyoneentrance}). 
% The input and state are limited to $\hat{\mathcal{U}}$ and $\hat{\mathcal{X}}$, respectively, restricting the UAV's movement area. Equation~(\ref{eq:feascond}) ensures the UAV stops at the end of the computed trajectory, guaranteeing recursive feasibility of the optimization problem~\citep{Graefe2022,Park2022,Chen2022,Chen2023}.

Constraint (\ref{eq:anticollisionconst}) prevents collisions between the UAVs.
The matrices $A\uli{i, j, \mathrm{c}}$ and $b\uli{i, j, \mathrm{c}}$ are constructed using time-variant BVC~\citep{Graefe2022,van2017distributed,Chen2022,Chen2023}. 
However, we relax the constraints in case a UAV is not recomputed.
The collision avoidance constraints are computed between the UAV $i$ and every other UAV $j\in\mathcal{A}\backslash\{i\}$.
The CU uses the reference positions $\hat{p}\uli{i}(\cdot|k-1)$ that UAV $i$ is currently following (known because $\mathcal{D}_{i,w}(k)$ contains only one element, cf. Alg.~\ref{alg:mlrdmpc} Line~\ref{lst:line:onlyoneentrance}) and all trajectories $\tilde{p}\uli{j}(\cdot|k-1)\in\mathcal{D}_j(k)$ in the information-tracker for the other UAV $j$.
Thus, the CU constrains the position of UAV $i$ based on all trajectories which it guesses UAV $j$ might fly.
First, we define the difference vector 
\begin{align}
	&n\uli{ij}(hT\ul{c}+2T|k-1)\\
	&= \Theta^{-1}[\tilde{p}\uli{j}(hT\ul{c}+2T|k-1)-\hat{p}\uli{i}(hT\ul{c}+2T|k-1)],\nonumber
\end{align}
which is the normal vector of a plane spanned between the UAVs.
UAV $i$ is constrained to stay on its side of the plane
\begin{align}
	\label{eq:tvb}
	&n\uli{\mathrm{0}, ij}(hT\ul{c}+2T|k-1)^T\Theta^{-1}\\
	&\opindent\times[\tilde{p}\uli{j}(hT\ul{c}+2T|k-1)-\hat{p}\uli{i,w}(hT\ul{c}+T|k)]\nonumber \\
	&\geq \begin{cases}
		\parbox[c]{.55\linewidth}{$\frac{1}{2}(\hat{d}\ul{min}$\\ \hphantom{A}+ $||n\uli{ij}(hT\ul{c}+2T|k-1)||\ul{2})$} & \parbox[c]{.28\linewidth}{if $j\in\mathcal{A}\ul{ET}(k)$}\\
		\hat{d}\ul{min} & \text{else}
	\end{cases} \nonumber
\end{align}
with $n\uli{\mathrm{0}, ij}=n\uli{ij} / ||n\uli{ij}||\ul{2}$ and $\hat{d}\ul{min}$ as the minimum distance between UAVs' reference positions with some safety gap to $d\ul{min}$ (cf. Section~\ref{sec:mdethod:collisionavoidance}).
If UAV $j\in\mathcal{A}\ul{ET}(k)$, we span the plane midway between it and UAV $i$. Else, relaxing the constraints, we move this plane close to $j$, because $j$ remains on its old trajectory.

In this work, we assume a UAV flying space devoid of static obstacles, such as buildings. However, incorporating such obstacles into MLR-DMPC is straightforward using established methods from other DMPCs~\citep{Park2022,Park2023,Chen2023}.


\subsection{Theoretical Analysis}
Our theoretical analysis of \myswarm{} is based on the system model derived under Assumption~\ref{as:tracking}. We first analyze the information trackers $\mathcal{D}$, then demonstrate that the reference trajectories $\hat{p}$ are collision-free, and finally prove the same for the actual positions $p$.

\subsubsection{Content of the Information Trackers}

We present two properties of the information trackers~$\mathcal{D}\uli{iw}(k)$, which follow directly from the construction of Algorithm~\ref{alg:processmessages} and are proven in Appendix~\ref{app:proofs}.
\begin{lemma}
    \label{col:information-tracker}
    If the information-tracker  $\mathcal{D}\ul{iw}(k)$ is not deprecated, then it contains
    the true trajectory $\hat{p}\uli{i}$, the UAV $i$ is following. 
\end{lemma}

\begin{lemma}
    \label{col:information-trackerequal}
    If after executing Algorithm~\ref{alg:processmessages} the two information-trackers $\mathcal{D}\ul{iw}(k)$ and $\mathcal{D}\ul{iv}(k)$ are not deprecated for different $w, v$, then $\mathcal{D}\ul{iw}(k) = \mathcal{D}\ul{iv}(k)$.
\end{lemma}

The CUs hence perform the DMPC step with the same information, which is crucial for guaranteeing collision avoidance.
We thus write $\mathcal{D}\ul{iw}(k) = \mathcal{D}\ul{iv}(k) = \mathcal{D}\ul{i}(k)$ when a CU is not deprecated.
In the following, we treat $\mathcal{D}\ul{i}(k)$ as the global knowledge shared by all CUs executing the DMPC. 
\reviewerone{This simplifies the notation for subsequent derivations.}

When all CUs are in the MLR state, we consider $\mathcal{D}\ul{i}(k)$ as the information tracker of a virtual CU that does not compute new trajectories but listens to all messages without message loss. 
Since no new trajectories are recomputed in this case, the information tracker remains unchanged.

\begin{figure*}[h]
    \centering
    \includegraphics[width=0.9\textwidth]{Images/HardwareImplementationV2.pdf}

	\caption{Hardware Implementation of \myswarm{}.
    \capt{Sixteen Crazyflie~2.1 quadcopters operate in a space of $3.4\SI{}{\meter}\times3.4\SI{}{\meter}\times2.6\SI{}{\meter}$. Three laptops serving as CUs compute trajectories using MLR-DMPC. The quadcopters and CUs communicate over a wireless mesh network with at least two hops using the Mixer protocol.
	}}
    \label{fig:hardwareimplementation}
\end{figure*}

\subsubsection{Collision Avoidance of MLR-DMPC}
\label{sec:mdethod:collisionavoidance}
The reference trajectories under constraints (\ref{eq:anticollisionconst}) are collision free as Appendix~\ref{app:proofs} proves.
\begin{corollary}
\label{cor:anitcol}
	If $i,j\in\mathcal{\tilde{A}}\ul{ET}(k)$ and constraint~(\ref{eq:anticollisionconst}) is fulfilled for both $i$ and $j$ on CUs $w$ and $v$, then for all $h\in\{1, 2,\cdots, h_\mathrm{c}\}$ %for all $\tilde{p}\uli{j}\in\mathcal{D}\ul{j}(k)$
	\begin{equation}
		||\Theta^{-1}[\hat{p}\uli{j,v}(hT\ul{c}+T|k)-\hat{p}\uli{i,w}(hT\ul{c}+T|k)]||\ul{2} \geq \hat{d}\ul{min}.
	\end{equation}
\end{corollary}

\begin{lemma}
	\label{lem:anticolone}
	If $i\in\mathcal{\tilde{A}}\ul{ET}(k)$, $j\notin\mathcal{\tilde{A}}\ul{ET}(k)$ and constraint~(\ref{eq:anticollisionconst}) is fulfilled, then for all $\tilde{p}\uli{j}\in\mathcal{D}\ul{j}(k+1)$ and for all $h\in\{1, 2,\cdots, h_\mathrm{c}\}$
	\begin{equation}
		||\Theta^{-1}[\tilde{p}\uli{j}(hT\ul{c}+T|k)-\hat{p}\uli{i,w}(hT\ul{c}+T|k)]||\ul{2} \geq \hat{d}\ul{min}.
	\end{equation}
\end{lemma}

\begin{lemma}
	\label{lem:anticoltwo}
	If for all $i\notin\mathcal{\tilde{A}}\ul{ET}(k)$, $j\notin\mathcal{\tilde{A}}\ul{ET}(k)$, for all $\tilde{p}\uli{j}\in\mathcal{D}\ul{j}(k)$ and $\tilde{p}\uli{i}\in\mathcal{D}\ul{j}(k)$ and for all $h\in\{1, 2,\cdots, h_\mathrm{c}\}$
	\begin{align}
		||\Theta^{-1}[\tilde{p}\uli{j}(hT\ul{c}+T|k-1)&-\tilde{p}\uli{i}(hT\ul{c}+T|k-1)]||\ul{2}\nonumber \\
		&\geq \hat{d}\ul{min},
	\end{align}
	then for all $\tilde{p}\uli{j}\in\mathcal{D}\ul{j}(k+1)$ and $\tilde{p}\uli{i}\in\mathcal{D}\ul{i}(k+1)$ and for all $h\in\{1, 2,\cdots, h_\mathrm{c}\}$
	\begin{equation}
		||\Theta^{-1}[\tilde{p}\uli{j}(hT\ul{c}|k)-\tilde{p}\uli{i}(hT\ul{c}|k)]||\ul{2} \geq \hat{d}\ul{min}.
	\end{equation}
\end{lemma}

We now can prove collision avoidance of all reference trajectories at discrete timepoints.

\begin{theorem}
	\label{th:feas}
	If for all pairwise different $i, j\in\mathcal{A}$, $\forall \tilde{p}\uli{i}\in\mathcal{D}\uli{i}(k)$, $\forall \tilde{p}\uli{j}\in\mathcal{D}\uli{j}(k)$ and $\forall h\in\{0, 1,\cdots, h_\mathrm{c}-1\}$
	\begin{align}
		\label{eq:recfeascond}
		||\Theta^{-1}[\tilde{p}\uli{j}(hT\ul{c}+2T|k-1)&-\tilde{p}\uli{i}(hT\ul{c}+2T|k-1)]||\ul{2}\nonumber\\ 
		&\geq \hat{d}\ul{min},
	\end{align}
	and all $\tilde{x}\uli{i}\in\mathcal{D}\uli{i}(k)$ satisfy constraints~(\ref{eq:dynconst})--(\ref{eq:feascond}), then the optimization problems (\ref{eq:opt}) are feasible for arbitrary message loss.
	Additionally, for all pairwise different $i, j\in\mathcal{A}(k-1)$, $\forall \tilde{p}\uli{i}\in\mathcal{D}\uli{i}(k+1)$, $\forall \tilde{p}\uli{j}\in\mathcal{D}\uli{j}(k+1)$ and for all $h\in\{1, 2,\cdots, h_\mathrm{c}\}$
	\begin{equation}
		\label{eq:colfree}
		||\Theta^{-1}[\tilde{p}\uli{j}(hT\ul{c}+T|k)-\tilde{p}\uli{i}(hT\ul{c}+T|k)]||\ul{2} \geq \hat{d}\ul{min}
	\end{equation}
    and all $\tilde{p}\uli{i}$ also fulfill constraints~(\ref{eq:dynconst})--(\ref{eq:stateconst}).
\end{theorem}
\begin{proof}
	We split the UAVs into two sets. 
	The first set contains all UAVs that were not recomputed, and the second the UAVs that were recomputed. 
	For the first set, we know that the shifted trajectories $\tilde{x}\uli{i}(t+T|k)=\tilde{x}\uli{i}(t+2T|k-1)$ fulfill the constraints~(\ref{eq:dynconst})--(\ref{eq:stateconst}) for all $t<(H-1)T$, as $T$ is a multiple of the sampling times $T\ul{s}$ and $T\ul{b}$~\citep{Graefe2022}. 
	For $t\geq(H-1)T$, it is $\tilde{x}\uli{i}(t+T|k)=\tilde{x}\uli{i}((H-1)T+T|k)=\tilde{x}\uli{i}((H-1)T+2T|k-1)$ and thus also~(\ref{eq:dynconst})--(\ref{eq:stateconst}) and~(\ref{eq:feascond}) are fulfilled. 
	
	For the UAVs that are recomputed, with the same argumentation, we can show that the shifted trajectory $\bar{x}\uli{i}(t+T|k)=\tilde{x}\uli{i}(t+2T|k-1)$ fulfills constraints~(\ref{eq:dynconst})--(\ref{eq:stateconst}).
    Also it holds that for all $j\in\mathcal{A}\backslash\{i\}$ and all $\tilde{p}_j\in\mathcal{D}_j(k)$

    \begin{align}
		n&\uli{\mathrm{0}, ij}(hT\ul{c}+2T|k-1)^T\Theta^{-1}\nonumber\\&\opindent\times[\tilde{p}\uli{j}(hT\ul{c}+2T|k-1)-\bar{p}\uli{i}(hT\ul{c}+T|k)]\nonumber\\
		&=n\uli{\mathrm{0}, ij}(hT\ul{c}+2T|k-1)^T\Theta^{-1}\nonumber\\&\opindent\times[\tilde{p}\uli{j}(hT\ul{c}+2T|k-1)-\tilde{p}\uli{i}(hT\ul{c}+2T|k-1)]\nonumber\\
		&=||\Theta^{-1}[\tilde{p}\uli{j}(hT\ul{c}+2T|k-1)\nonumber\\
		&\opindent-\bar{p}\uli{i}(hT\ul{c}+2T|k-1)]||\ul{2}\nonumber\\
		&\geq\frac{1}{2}(\hat{d}\ul{min}  + ||n\uli{ij}(hT\ul{c}+2T|k-1)||\ul{2}) \nonumber\\
		&\geq \hat{d}\ul{min}.
    \end{align}
    Hence, $\bar{x}$ also fulfills constraint~(\ref{eq:anticollisionconst}) and is a candidate solution optimization problem~(\ref{eq:opt}).
    The optimization problem is thus feasible.
    
    Algorithm~\ref{alg:processmessages} then incorporates the solutions the CUs computed into the information-trackers of every CU after a communication step. 
    First, it deletes some trajectories in the information-trackers (Lines~\ref{lst:line:metadatacompbegin}--\ref{lst:line:metadatacompend}). 
    Because the remaining trajectories were already in the former information-tracker for CUs not in the MLR-state (no deprecated information-tracker), they fulfill the constraints~(\ref{eq:dynconst})--(\ref{eq:feascond}) as argued above. 
    In Lines~\ref{lst:line:updateinformationtrackerbegin}--\ref{lst:line:updateinformationtrackerend}, the algorithm adds new trajectories to the information-tracker, resulting in the information-trackers $\mathcal{D}_i(k+1), \forall~i\in\mathcal{A}$ stored on the CUs with non-deprecated information-trackers.
    These also fulfill constraints~(\ref{eq:dynconst})--(\ref{eq:anticollisionconst}) as they are solutions to the corresponding optimization problem.

    Corollary~\ref{cor:anitcol}, Lemma~\ref{lem:anticolone} and~\ref{lem:anticoltwo} then lead to equation (\ref{eq:colfree}).
\end{proof}

From this, we can derive that the UAVs also fly on collision-free trajectories.

\begin{theorem}
	\label{th:collisionfreeness}
	If the reference trajectories of all UAVs fulfill~(\ref{eq:dynconst})--(\ref{eq:feascond}) at $k=0$ and Equation (\ref{eq:recfeascond}), then for all $k$, arbitrary message loss, all pairwise different $i, j\in\mathcal{A}$ and and for all $h\in\{0, 1,\cdots, h_\mathrm{c}-1\}$
	\begin{equation}
		||\Theta^{-1}[\hat{p}\uli{j}(hT\ul{c}+kT)-\hat{p}\uli{i}(hT\ul{c}+kT)]||_2 \geq \hat{d}_\mathrm{min}.
	\end{equation}
\end{theorem}
\begin{proof}
First, we apply induction to Theorem~\ref{th:feas}, which is possible as the non-deprecated information-trackers are all equal to $\mathcal{D}_{i}$ due to Lemma~\ref{col:information-trackerequal}, and $T$ is a multiple of $T_\mathrm{c}$~\citep{Graefe2022}.
Thus, for pairwise different $i, j\in\mathcal{A}$, the information-tracker $\mathcal{D}_{i}$ contains trajectories, which are collision-free to all other trajectories in the information-tracker $\mathcal{D}_{j}$. 
Because of Lemma~\ref{col:information-tracker}, each UAV follows one trajectory in the information-tracker, and its reference positions are collision-free.
\end{proof}

Based on Assumption~\ref{as:tracking}, we now derive guarantees for the actual UAV positions at continuous times $t$.

\begin{theorem}
\label{th:colfreerealworld}
    Let
        \begin{align}
        \label{eq:contopt}
            \Delta d_\mathrm{min,cont}&=\hat{d}_\mathrm{min}\\&-\min_{\substack{\hat{x}_{0, i},\hat{x}_{0, j}\in\hat{\mathcal{X}}\\ \tau\in[0,T_\mathrm{c}]\\\hat{u}_i(t), \hat{u}_j(t)\in\hat{\mathcal{U}}}}||\Theta^{-1}(\hat{p}_i(\tau)-\hat{p}_j(\tau))||_2\nonumber\\
        \text{s.t.~~} \dot{x}_i&=\hat{f}_i(\hat{x}_i(t), \hat{u}_i(t)), \hat{x}_i(0)=\hat{x}_{0,i}\nonumber\\
        \dot{x}_j&=\hat{f}_i(\hat{x}_j(t), \hat{u}_j(t)), \hat{x}_j(0)=\hat{x}_{0,j}\nonumber\\
        \hat{p}_i(t) &= g_i(\hat{x}_i(t))\nonumber\\
        \hat{p}_j(t) &= g_i(\hat{x}_j(t))\nonumber\\
        ||\Theta^{-1}(\hat{p}_i(0)&-\hat{p}_j(0))||_2 \geq \hat {d}_\mathrm{min}\nonumber\\
        ||\Theta^{-1}(\hat{p}_i(T_\mathrm{c})&-\hat{p}_j(T_\mathrm{c}))||_2 \geq \hat {d}_\mathrm{min}.\nonumber\\
        \end{align}
    If
    \begin{equation}
        \hat{d}_\mathrm{min} \geq d_\mathrm{min} + \Delta d_\mathrm{min} + \hat{d}_\mathrm{min,cont},
    \end{equation}
    then Equation (\ref{eq:truecollisionavoidance}) holds, i.e., the UAVs are collision free.
\end{theorem}

\begin{proof}
    Follows directly from Assumption (\ref{eq:truecollisionavoidance}), construction of $\Delta d_\mathrm{min,cont}$ and triangle inequality.
\end{proof}

For collision avoidance guarantees, $\Delta d_\mathrm{min}$ and $\Delta d_\mathrm{min,cont}$ must be known. 
Solving optimization problem~(\ref{eq:contopt}) directly yields $\Delta d_\mathrm{min,cont}$, while $\Delta d_\mathrm{min}$ must be determined through experiments or engineering intuition.

\begin{remark}
Under strong external disturbances like gusts, Assumption~\ref{as:tracking} may not hold. 
If a UAV's state deviates significantly from the reference, we reinitialize its current state and assign a high priority $J\uli{i}$ to ensure replanning~\citep{Luis2020}. 
However, under these conditions, the guarantees of MLR-DMPC do not hold, as such disturbances may lead the UAV onto a collision course.
\end{remark}

\reviewerone{
\begin{remark}
Theorem~\ref{th:colfreerealworld} requires collision-free initial trajectories. 
In practice, UAVs typically start their flight by hovering while maintaining a safe distance from one another. 
Therefore, the initial trajectories can often be initialized as this constant hovering state.
\end{remark}
}
% section: results

% dev set decision table
% manually created from 25-official-html.tgz
% with custom editing

\begin{table}
\centering
\begin{tabular}{l|rr}
\toprule
 & \multicolumn{2}{c}{\textbf{ELAS F1}} \\
\textbf{Treebank} & \textbf{sem-frag}
 & \textbf{heuristic}\\
%\hline
\midrule
% e7 elmo udpf task ar padt	allennlp 090 dm lbert luxfb ar padt 20200424 080759
% e7 elmo udpf task ar padt	copy2e arcase mark rel
ar\_padt & \textbf{70.99} & 59.74 \\
% e5 elmo udpf task bg btb	allennlp 090 dm pbert u bg btb 20200312 003243	
% e7 elmo udpf task bg btb	copy2e encase mark rel	
bg\_btb & \textbf{88.09} & 86.19 \\
%\hline
\midrule
% 	e3 elmo udpf task cs cac	allennlp 090 dm pbert luxfb cs cac 20200424 002226
%	e5 elmo udpf task cs cac	copy2e arcase mark rel	
cs\_cac & \textbf{86.51} & 74.41 \\
% e3 elmo udpf task cs fictree	allennlp 090 dm pbert luxfb cs cac 20200424 002226
% e5 elmo udpf task cs fictree	copy2e arcase mark rel	77.37
cs\_fictree & \textbf{83.23} & 77.37  \\
% e3 elmo udpf task cs fictree	allennlp 090 dm pbert u cs cac 20200419 171603
% e3 elmo udpf task cs cac	copy2e arcase mark
cs\_pdt & \textbf{79.58} & 	71.19 \\
%\hline
\midrule
% e7 elmo udpf task en ewt + ud25 en gum + ud25 en lines + ud25 en partut	allennlp 090 dm lbert u en ewt 20200312 051351
% e7 elmo udpf task en ewt + ud25 en gum + ud25 en lines + ud25 en partut	copy2e encase mark cc rel
en\_ewt & \textbf{84.71} & 	82.86 \\
% e3 elmo udpf task et edt + task et ewt	allennlp 090 dm mbert u et 20200419 234001
% e3 elmo udpf task et edt + task et ewt	copy2e arcase mark
et\_edt & 62.74 & \textbf{69.35} \\
% e5 elmo udpf task fi tdt fasttext udpf task fi tdt	allennlp 090 dm lbert u fi tdt 20200420 050020
% e7 elmo udpf task fi tdt	copy2e arcase mark rel
fi\_tdt & \textbf{83.64} & 71.84 \\ 
% e5 elmo udpf task fr sequoia + ud25 fr gsd + ud25 fr partut + ud25 fr spoken	allennlp 090 dm mbert u fr sequoia 20200312 072651
% e3 elmo udpf task fr sequoia + ud25 fr gsd + ud25 fr partut + ud25 fr spoken	copy2e	
fr\_sequoia & \textbf{88.65} &  87.53 \\
% e3 elmo udpf task it isdt + ud25 it partut + ud25 it postwita + ud25 it twittiro + ud25 it vit	allennlp 090 dm lbert u it isdt 20200419 172143	
% e7 elmo udpf task it isdt	copy2e encase mark cc rel	
it\_isdt & \textbf{90.13} & 88.28 \\
% e3 plain udpf task lt alksnis	allennlp 090 dm mbert u lt alksnis 20200420 014618	
% e3 plain udpf task lt alksnis	copy2e arcase mark	
lt\_alksnis & \textbf{73.63} & 57.84 \\
% e7 elmo udpf task lv lvtb	allennlp 090 dm mbert luxfb lv lvtb 20200423 191418	
% e5 elmo udpf task lv lvtb fasttext udpf task lv lvtb	copy2e encase rel	
lv\_lvtb & \textbf{81.82} & 71.29 \\
%\hline
\midrule
% 	e7 elmo udpf task nl alpino + task nl lassysmall	allennlp 090 dm lbert u nl alpino 20200312 025649	
% e7 elmo udpf task nl alpino + task nl lassysmall	copy2e encase mark cc rel	
nl\_alpino & \textbf{89.93} & 89.00 \\
% e7 elmo udpf task nl alpino + task nl lassysmall	allennlp 090 dm lbert u nl alpino 20200312 025649	
% e7 elmo udpf task nl alpino + task nl lassysmall	copy2e encase mark cc rel	
nl\_lassysmall & 79.00 & \textbf{81.24} \\
%\hline
\midrule
% e5 elmo udpf task pl lfg + task pl pdb	allennlp 090 dm mbert luxfb pl lfg 20200423 222537	
% e5 elmo udpf task pl lfg + task pl pdb	copy2e encase mark rel	
pl\_lfg & \textbf{94.12} & 93.84 \\
% e3 fasttext udpf task pl lfg + task pl pdb	allennlp dev dm lbert luxf pl 20200416 194726	
% e5 elmo udpf task pl lfg + task pl pdb	copy2e arcase mark rel	
pl\_pdb & \textbf{82.25} & 78.27 \\
%\hline
\midrule
% e7 elmo udpf task ru syntagrus	allennlp 090 dm lbert lufb ru syntagrus 20200423 210055	
% e7 elmo udpf task ru syntagrus	copy2e arcase mark	
ru\_syntagrus & \textbf{88.48} & 80.03 \\
% e3 elmo udpf task sk snk plain udpf task sk snk	allennlp 090 dm mbert u sk snk 20200420 020636	
% e7 elmo udpf task sk snk	copy2e arcase mark	75.98
sk\_snk  & \textbf{81.30} & 75.98 \\
% e3 elmo udpf task sv talbanken	allennlp 090 dm lbert u sv talbanken 20200419 195336	
% e7 elmo udpf task sv talbanken	copy2e encase mark cc rel	
sv\_talbanken & \textbf{84.54} & 81.32 \\
% e3 plain udpf task ta ttb	allennlp 090 dm mbert u ta ttb 20200419 232103	
%	e3 plain udpf task ta ttb	copy2e arcase	
ta\_ttb & \textbf{55.68} & 43.94 \\
% e3 elmo udpf task uk iu	allennlp 090 dm mbert u uk iu 20200420 004219	
% e7 elmo udpf task uk iu	copy2e arcase mark	
uk\_iu & \textbf{82.41} & 76.88 \\
%\hline
\bottomrule
\end{tabular}
\caption{Development set ELAS F1 score %f-score
        for the best semantic parser evaluated without connecting
            fragmented graphs (sem-frag)
        and
        for the best combination of heuristic rules
            (heuristic)
}
\label{devresults:decision_custom}
\end{table}

% eof

Table~\ref{devresults:decision_custom} compares the semantic parser against the heuristic approach on the ELAS F1 metric.
The evaluation script was run without connecting fragmented graphs and format validation.
For all but two treebanks, the semantic parser performs better than the
best
heuristic approach.
For some languages, the difference in performance is large.
For \texttt{et\_ewt}, which does not have a development set,
we suspect that we overfitted our semantic parser on the
\texttt{et\_ewt} training data
by allowing it to train for 75 epochs.

% test set results table
% manually created from eval pages linked on
% https://quest.ms.mff.cuni.cz/sharedtask/cgi-bin/overview.pl

% main body generated by copy and pasting the qualitative tables,
% then using `cut -f1,16` to get the right columns, pasting them
% together with `paste` and tabs converted to `&` and \\ added to
% lines in `vim`

\begin{table}
\centering
\begin{tabular}{l|rrr}
\toprule
 & \multicolumn{3}{c}{\textbf{ELAS F1}} \\
\textbf{Treebank} & \textbf{subm}
 & \textbf{frag fix} & \textbf{re-run}\\
\midrule
Arabic-PADT         &  57.19  &  70.08  &  \bf 70.40  \\
Bulgarian-BTB       &  77.29  &  89.58  &  \bf 89.60  \\
Czech-FicTree       &  70.04  &  80.77  &  \bf 81.63  \\
Czech-CAC           &  71.72  &  86.00  &  \bf 86.38  \\
Czech-PDT           &  65.94  &  79.03  &  \bf 79.84  \\
Czech-PUD           &  64.34  &  77.37  &  \bf 78.08  \\
Dutch-Alpino        &  71.44  &  87.61  &  \bf 87.77  \\
Dutch-L.Small       &  64.03  &  77.39  &  \bf 77.24  \\
English-EWT         &  70.61  &  \bf 83.56  & \bf 83.56  \\
English-PUD         &  70.25  &  86.88  & \bf 87.03  \\
Estonian-EDT        &  62.29  &  68.20  &  \bf 68.37  \\
Estonian-EWT        &  55.70  &  \bf 61.19  &  60.67  \\
Finnish-TDT         &  73.02  &  \bf 84.36  &  84.33  \\
Finnish-PUD         &  71.58  & \bf 84.62  & \bf 84.62  \\
French-Sequoia      &  77.44  &  87.58  & \bf 88.60  \\
French-FQB          &  74.30  &  82.68  & \bf 83.26  \\
Italian-ISDT        &  71.98  &  \bf 90.24  &  90.23  \\
Latvian-LVTB        &  72.41  &  81.81  &  \bf 82.40  \\
Lithuanian-AL.      &  58.36  &  68.76  &  \bf 68.84  \\
Polish-LFG          &  61.23  &  \bf 70.89  &  70.71  \\
Polish-PDB          &  67.68  &  80.93  &  \bf 82.43  \\
Polish-PUD          &  65.64  &  79.77  & \bf 80.79  \\
Russian-SynT.       &  75.27  &  89.21  & \bf 89.47  \\
Slovak-SNK          &  68.43  &  81.63  &  \bf 81.97  \\
Swedish-Talb.       &  71.86  &  86.78  & \bf 86.72  \\
Swedish-PUD         &  64.70  &  79.35  & \bf 79.37  \\
Tamil-TTB           &  48.47  &  \bf 57.28  &  57.10  \\
Ukrainian-IU        &  66.43  &  79.81  & \bf 82.92  \\
\midrule
Average             &  67.49  &  79.76  & \bf 80.15  \\
\bottomrule
\end{tabular}
\caption{Test set results:
    subm = submitted,
    frag fix = using our own fragment connector and quick-fix.pl without connect-to-root,
    re-run = a re-run with bug fixes, no new models but new model selection
}
\label{testresults_custom}
\end{table}

% eof

Table~\ref{testresults_custom} shows test set ELAS obtained on the shared task
submission site for
\textit{(a)} our submission fully relying on the organiser's
             \texttt{quick-fix} tool to fix issues in the output of
             our system,
\textit{(b)} the same predictions post-processed by our own
             fragment connector that aims to minimise the
             number of root edges added, and
\textit{(c)} a re-run of our pipeline using the same models
             for system components as before but with all
             bugs fixed during development applied to all
             predictions and new decisions which models
             to apply to the test sets.
While the \texttt{quick-fix} tool enabled us to make a valid submission
in time, its
approach of adding edges from the root node to
all unreachable tokens
has a strong negative impact on 
precision, \eg 62.26 ELAS precision on the Czech CAC development set
\vs 87.37 without post-processing.
Our own post-competition fix avoids this
and would have brought us to the top half of the competition.

% eof

\section{Discussion}
\label{sec:discussion}

We have constructed and analyzed mathematical models of dynamic
task allocation in a multi-robot system. The models are general
and can be easily extended to other systems in which robots use a
history of local observations of the environment as a basis for
making decisions about future actions. These models are based on
theory of stochastic processes. In order to study a robot's
behavior, we do not need to know its exact trajectory or the
trajectories of other robots; instead, we derive a probabilistic
model that governs how a robot's behavior changes in time. In some
simple cases these models can be solved analytically. However,
stochastic models are usually too complex for exact analytic
treatment. Thus, in the scenario described in
\secref{sec:pucksonly} in which only observations of tasks are
made, though the individual model is tractable, the stochastic
model of the collective behavior is not. Instead, we use averaging
and approximation techniques to quantitatively study the dynamics
of the collective behavior. Such models, therefore, do not
describe the robots' behavior in a single experiment, but rather
the behavior that has been averaged over many experimental or
simulations runs. Fortunately, results of experiments and
simulations are usually presented as an average over many runs;
therefore, mathematical models of average collective behavior can
be used to describe experimental results. In fact, the stochastic
model produces excellent agreement with experimental results under
all experimental conditions and without using any adjustable
parameters.


Phenomenological models are more straightforward to construct and
analyze than exact stochastic models --- in fact, they can be
easily constructed from details of the individual robot
controller~\cite{Lerman04sab}. The ease of use comes at a price,
namely, the number of simplifying assumptions that were made in
order to produce a mathematically tractable model. First, we
assume that the robots are functioning in a dilute limit, where
they are sufficiently separated that their actions are largely
independent of one another. Second, we assume that the transition
rates can be represented by aggregate quantities that are
spatially uniform and independent of the details of the individual
robot's actions or history. We also assume the system is
homogeneous, with modeled robots characterized by a set of
parameters, each of them representing the mean value of some real
robot feature: mean speed, mean duration for performing a certain
maneuver, and so on. Real robot systems are heterogeneous: even if
the robots are executing the same controller, there will always be
variations due to inherent differences in hardware. We do not
consider parameter distributions in our models as would be
necessary to describe such heterogeneous systems. Finally,
phenomenological models more reliably describe systems where
fluctuations (deviations from the mean behavior) can be neglected,
as happens in large systems or when many experimental runs are
aggregated. However, even if phenomenological models don't agree
with experiments exactly, as we saw in \secref{sec:results2}, they
can still reliably predict most behaviors of interest even in
not-so-large systems. They are, therefore, a useful tool for
modeling and analyzing multi-robot systems.




\section*{Acknowledgments}
This work was supported by the German Research Foundation (DFG) within the priority program 1914 (grant TR 1433/2) and within the Emmy Noether project NextIoT (ZI 1635/2-1), and by the LOEWE initiative (Hesse, Germany) within the emergenCITY center (LOEWE/1/12/519/03/05.001(0016)/72). 

The authors gratefully acknowledge the computing time provided to them at the NHR Center NHR4CES at RWTH Aachen University (p0022034). This is funded by the Federal Ministry of Education and Research, and the state governments participating on the basis of the resolutions of the GWK for national high performance computing at universities (www.nhr-verein.de/unsere-partner).

We thank Sebastian Giedyk for his help with the quadcopters and the testbed, and Shengsi Xu for his help with the software development for the experiments, and Fabian Mager, Henrik Hose, Alexander von Rohr and Pierre-François Massiani for helpful discussions.

\bibliography{references}

\begin{appendices}


\section{Event Triggers}
\label{app:et}

% We detail the event-trigger mechanism.
% To minimize communication overhead, we quantize priorities to \SI{8}{\bit} unsigned integers instead of \SI{32}{\bit} floats.

% Due to the one-round delay, when a UAV's trajectory is recalculated, its priority still reflects its state before recalculation. 
% To address this, if a CU has just recalculated UAV~$i$, it sets its priority to zero. 
% We also adjust the maximum operation in Equation~\ref{eq:priosmax} to return zero if any element is zero; otherwise, it returns the maximum. 
% The CU calculates priorities using the following equations.

To minimize communication overhead in the event-trigger mechanism, we quantize priorities to 8-bit unsigned integers instead of 32-bit floats. 
Due to the one-round delay, a recalculated UAV's priority may not reflect its updated state. 
To address this, if a CU has just recalculated UAV~$i$, it sets its priority to zero. 
We adjust the maximum operation in Equation~(\ref{eq:priosmax}) to return zero if any element is zero; otherwise, it returns the maximum. The CU calculates priorities using the following equations.


\subsection{Round-Robin Event Trigger (RR)}
The RR calculates priorities as:
\begin{equation}
    J_{iw}(k) = k - k_{i, \text{calc}}(k),
\end{equation}
where $k_{i, \text{calc}}(k)$ is the last round in which UAV $i$'s trajectory was calculated. If the CU's database contains multiple trajectories, it uses the first one.

\subsection{Distance-Based Event Trigger (DT)}
The DT calculates priorities as:
\begin{equation}
    J_{iw}(k) = \left\| p_{i, \mathrm{target}} - p_i\big(T\,|\,(k - 1)T\big) \right\|.
\end{equation}

\subsection{Hybrid Event Trigger (HT)}
The HT combines the previous methods:
\begin{align}
    J_{iw}(k) &= \left\| p_{i, \mathrm{target}} - p_i\big(T\,|\,(k - 1)T\big) \right\|\nonumber\\ 
    &\opindent\times\left[ k - k_{i, \text{calc}}(k) \right].
\end{align}


\subsection{Deadlock-Aware Triggering}

Since recalculating a deadlocked UAV's trajectory will not change it, we introduce a deadlock-aware triggering mechanism. 
The CU uses \cite[Theorem~1, Equation~(13)]{Chen2022} to detect deadlocks. 
If a UAV is in deadlock, the CU sets its priority to one.
\section{Deadlock Avoidance}
\label{app:deadlock}

The main idea behind these constraints is to incorporate rotation into the swarm constraints~\citep{Chen2023}. 
To achieve this, we modify the left side of constraint~(\ref{eq:tvb}):

\begin{align}
    \label{eq:tvbsoftconstraints}
    &n\uli{\mathrm{0}, ij}(hT\ul{c}+2T|k-1)^T\Theta^{-1} \\
    &\opindent\times[\tilde{p}\uli{j}(hT\ul{c}+2T|k-1)-\hat{p}\uli{i,w}(hT\ul{c}+T|k)] \nonumber \\
    &\geq \begin{cases}
        \parbox[c]{.55\linewidth}{$\frac{1}{2}(\hat{d}\ul{min}$\\ \hphantom{A}+ $||n\uli{ij}(hT\ul{c}+2T|k-1)||\ul{2})$\\ \hphantom{A}+ $\epsilon$} & \text{if~} j\in\mathcal{A}\ul{ET}(k)\\
        \hat{d}\ul{min} + \epsilon & \text{else}
    \end{cases},\nonumber
\end{align}
where $\epsilon\in \mathbb{R}_{\geq 0}$ is an additional optimization variable.
\reviewerone{$\epsilon \geq 0$ hereby ensures that the constraints in the limit of $\epsilon=0$ are the same as the constraints (\ref{eq:tvb}), ensuring that the guarantees still hold~\cite{Chen2023}}

Following~\cite{Chen2023}, we increase the weight of $\epsilon$ in the objective function when UAV $j$ is to the right of UAV $i$, pushing each UAV to its right and causing approaching UAVs to rotate around each other. 
Unlike~\cite{Chen2023}, we keep the soft constraint hyperparameters constant, as they perform well without adaptation.


\subsection{High-Level Path Planner}


We combine the soft constraints with a high-level planner similar to~\cite{Park2022}. 
In normal operation, the desired target state $x\uli{i, \mathrm{target}}$ in optimization problem~(\ref{eq:opt}) equals the UAV's actual target. 
When a deadlock occurs---detected when all velocities fall below a threshold—a CU activates the high-level planner, and $x\uli{i, \mathrm{target}}$ is set to an intermediate target calculated by the planner.

Each CU runs its portion of the high-level planner for its assigned subset of UAVs in parallel with the communication round, avoiding additional computation time. 
The intermediate target positions are communicated in the subsequent communication round.

The high-level planner classifies UAVs into those continuing towards their targets and those that should make room for others. 
For each assigned UAV~$i$, the CU determines if it should make room for another UAV~$j$ based on the following conditions:
\begin{itemize}
    \item UAV~$i$ is within a certain distance of UAV~$j$,
    \item UAV~$j$ is not closer to its target than UAV~$i$,
    \item UAV~$i$ is moving towards UAV~$j$, or UAV~$i$ lies between UAV~$j$ and its target.
\end{itemize}
The CU notes all UAVs for which UAV~$i$ should make room and adjusts UAV~$i$'s path for the closest such UAV. 
Following~\cite{Park2022}, UAV~$i$ sets its new target to its current position plus a vector pointing away from UAV~$j$, adding random noise to resolve symmetric deadlocks.

UAVs follow these intermediate targets until the deadlock resolves or the conditions necessitating making room no longer apply. 
The high-level planner then stops, and all UAVs resume flying to their actual target positions.

% \section{Communication Phase Latency}
% \label{app:tcom}

% 	Before deriving how we calculate the communication phase length, we first have to explain our communication system more into detail.
% 	Because of the network coding in Mixer, all messages sent via Mixer must have the same size.
% 	In \myswarm{}, however, devices send messages with different message sizes.
% 	To support this, we extend Mixer with a message layer.
% 	In the message layer, $K$ message fields with different message sizes $S_i$ are defined beforehand
% 	\begin{equation}
% 	\label{eq:messagelayer}
% 	\mathcal{S} = \{S_i | i\in\{0, \cdots, K-1\}\},
% 	\end{equation}
% 	where $K$ is the overall number of sent messages.
% 	These message fields are then split onto smaller messages that have a constant message size $S_\mathrm{m}$.
%     Mixer then transmits these smaller messages.
	
% 	Their size $S_\mathrm{m}$ can be chosen freely.
% 	Because it might not be possible to split a message field onto an integer number of messages, some spaces in the messages are empty and do not contain information.
% 	If we use a small $S_\mathrm{m}$, the size of empty spaces in the messages is small, but the communication overhead is bigger.
% 	For big $S_\mathrm{m}$, the communication overhead is smaller, but the empty spaces are bigger.
% 	We therefore aim to find an $S_\mathrm{m}$ that leads to minimal communication phase latency.
	
% 	The number of Mixer messages $M_\mathrm{m}$ is (together with the initiator message sent by one device to start the communication phase and share the current round time $k$ among all devices)
% 	\begin{equation}
% 	M_\mathrm{m} = 1 +  \sum_{S_i\in\mathcal{S}}\lceil\frac{S_i}{S_\mathrm{m}}\rceil.
% 	\end{equation}

%     We choose the number of slots per communication phase using the heuristic~\citep{Mixer}
%     \begin{equation}
%         M_\mathrm{s} = \max(3M_\mathrm{m}, 170).
%     \end{equation}

%     The length of one slot $T_\mathrm{s}$ depends not only on $S_\mathrm{m}$ and $M_\mathrm{m}$, but also on the hardware and the physical layer used.
%     The exact calculation of $T_\mathrm{s}$ for our implementation can be found in the code provided upon publication of this paper.
	
%     The optimal message size and length of one communication phase is then
% 	\begin{equation}
% 	S_\mathrm{p}^* = \arg \min_{S_\mathrm{m}\in\mathbb{N}_+}(M_\mathrm{s}T_\mathrm{s})~~~~ T_\mathrm{com} = \min_{S_\mathrm{m}\in\mathbb{N}_+}(M_\mathrm{s}T_\mathrm{s}).
% 	\label{eq:roundlength}
% 	\end{equation}
\section{Additional Proofs}
\label{app:proofs}


\begin{proof}[Proof of Lemma 1]
    If UAV $i$ was not recalculated in the previous round, then $|\mathcal{D}_{iw}| = 1$. If the CU received the trajectory metadata, it directly knows the trajectory UAV $i$ is following. If the metadata was not received due to message loss, since the database is not deprecated (it would have been marked otherwise), the CU still knows the trajectory from prior rounds.

    If UAV $i$ was recalculated in the previous round, then at Line~\ref{lst:line:deprecatedtwo} in Algorithm~\ref{alg:processmessages}, the database contains the trajectory UAV $i$ was following two rounds ago. After processing, it includes this trajectory and the new one calculated and transmitted in the last round. Depending on whether UAV $i$ received the new trajectory, it is following one of these two trajectories. Thus, the CU knows the possible trajectories UAV $i$ may be following.
\end{proof}

\begin{proof}[Proof of Lemma 2]
    For UAVs not recalculated in the last round, the intermediate databases $\tilde{\mathcal{D}}_{iv}(k)$ and $\tilde{\mathcal{D}}_{iw}(k)$ contain their actual trajectories (Lines~\ref{lst:line:metadatacompbegin}--\ref{lst:line:metadatacompend}, Lemma~\ref{col:information-tracker}). For UAVs recalculated in the last round, the databases contain their trajectories from the last or second-to-last round, depending on recalculations due to the event trigger. After updating $\tilde{\mathcal{D}}$, the newly calculated trajectories are added (Lines~\ref{lst:line:updateinformationtrackerbegin}--\ref{lst:line:updateinformationtrackerend}). The databases are not deprecated only if all new trajectories are received, ensuring $\mathcal{D}_{iw}(k) = \mathcal{D}_{iv}(k)$.
\end{proof}

% \begin{proof}[Proof of Lemma 1]
%     If UAV $i$ was not recalculated in the round before, we know that $|\mathcal{D}\ul{iw}| = 1$.
%     If the CU has received  the metadata of the trajectory, 
%     we directly know that this is the trajectory the UAV is following. 
%     If the CU has not received the metadata (because of message loss), this follows from the fact, that in at least one of the rounds before the metadata must have matched, otherwise the database would be deprecated. 
    
%     If UAV $i$ was recalculated in the round before, we know that at Line~\ref{lst:line:deprecatedtwo} in Algorithm~\ref{alg:processmessages}, the content of the database is the trajectory the UAV was planning to follow two rounds ago. 
%     At the end of the algorithm, this database contains this trajectory and the trajectory that was calculated and transmitted in the last round. 
%     One of these trajectories is the true trajectory, depending if UAV $i$ received the new trajectory or not.
% \end{proof}

% \begin{proof}[Proof of Lemma 2]
%     We know that the intermediate databases $\mathcal{\tilde{D}}\uli{iv}(k)$ and $\mathcal{\tilde{D}}\uli{iw}(k)$ contain the trajectories the UAVs are really following, if these UAVs were not recalculated in the last round ((Lines~\ref{lst:line:metadatacompbegin}--\ref{lst:line:metadatacompend}), Lemma~\ref{col:information-tracker}). 
%     For the UAVs that were recalculated in the last round, we know that the database contains the trajectories that the UAVs were following in in the last round (because of the event-trigger they were not recalculated in the second last round). 
%     For the ones that were recalculated, they contain the trajectories the UAVs were following in the second last round. 
%     After calculating $\tilde{D}$, the algorithm adds the newly calculated trajectories to the database (Lines~\ref{lst:line:updateinformationtrackerbegin}--\ref{lst:line:updateinformationtrackerend}). 
%     As the resulting databases are not deprecated only if all newly calculated trajectories are received $\mathcal{D}\ul{iw}(k) = \mathcal{D}\ul{iv}(k)$. 
% \end{proof}

\begin{proof}[Proof of Corrolary 1]
    As both UAVs are selected for computation, we know $|\mathcal{D}\uli{j}| = 1$.
    The corollary then follows from Lemma 1 in \citep{Graefe2022}.
\end{proof}
    
\begin{proof}[Proof of Lemma 3] 
    We know that because UAV $j$ was not recalculated $\tilde{p}\uli{j}(\cdot|k)=p\uli{j}(\cdot+T|k-1)\in\mathcal{D}\ul{j}(k-1)$ (Equation (\ref{eq:nottriggered})). 
    It is
    \begin{align}
        ||&\Theta^{-1}[\tilde{p}\uli{j}(hT\ul{c}+T|k)-\hat{p}\uli{i,w}(hT\ul{c}+T|k)]||\ul{2}\nonumber\\ 
        &=||n\uli{\mathrm{0}, ij}(hT\ul{c}+2T|k-1)||\ul{2}||\Theta^{-1}\nonumber\\
        &\opindent\times[\tilde{p}\uli{j}(hT\ul{c}+T|k)-\hat{p}\uli{i,w}(hT\ul{c}+T|k)]||\ul{2}\nonumber \\
        &\geq n\uli{\mathrm{0}, ij}(hT\ul{c}+2T|k-1)^T\Theta^{-1}\nonumber\\
        &\opindent\times[p\uli{j}(hT\ul{c}+T|k)- \hat{p}\uli{i,w}(hT\ul{c}+T|k)]\nonumber\\
        &= n\uli{\mathrm{0}, ij}(hT\ul{c}+2T|k-1)^T\Theta^{-1}\nonumber\\
        &\opindent\times[p\uli{j}(hT\ul{c}+2T|k-1)- \hat{p}\uli{i,w}(hT\ul{c}+T|k)]\nonumber\\
        &\geq \hat{d}\ul{min}
    \end{align}
\end{proof}
    

\begin{proof}[Proof of Lemma 4] 
Since no new entries are added to the databases, there exists a trajectory $\tilde{p}\uli{i}(\cdot+T|k-1)\in\mathcal{D}_{i}(k)$ with $\tilde{p}\uli{i}(\cdot|k-1)=\tilde{p}\uli{i}(\cdot+T|k)$ (same for $j$), and thus collision freeness follows directly.
\end{proof}

\end{appendices}
%Materials and Methods\\
%Supplementary Text\\
%Figs. S1 to S3\\
%Tables S1 to S4\\
%References \textit{(4-10)}


\end{document}
