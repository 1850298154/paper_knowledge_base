\section{Related Work}
\label{sec:relatedwork}

Before presenting our solution, we analyze related DMPC methods. %, as our main goal is to realize DMPC on swarms. 
For comprehensive reviews of general UAV swarm control methods, we refer to~\cite{gugan2023path, wang2024review, yu2023overview}.


% Before presenting our solution, we analyze the related work, dividing our discussion into two parts. 
% First, we review general methods for controlling robot swarms. Second, we examine related Distributed Model Predictive Control (DMPC) methods.

% \subsection{Trajectory Planning for Robot Swarms}
% Over the last decade, numerous approaches for controlling UAV swarms have been developed; comprehensive reviews can be found in~\cite{huang2019collision,gugan2023path}. 
% Some of these approaches~\citep{augugliaro2012generation,schoellig2014so,michael2010grasp} achieve impressive dynamic swarm maneuvers but rely on centralized control. 
% Consequently, they suffer from inherent disadvantages such as a single point of failure and limited scalability. 
% % Therefore, we focus exclusively on distributed control methods, emphasizing those that have demonstrated real-world feasibility through hardware experiments.

% Many studies~\citep{turpin2012trajectory,kushleyev2013towards,zhou2018agile,Park2022,Park2023,Chen2023,Chen2022} present distributed controllers but include hardware implementations where a central computer manages the entire control system by emulating distributed computation and communication. 
% Although these approaches achieve impressive swarm maneuvers, they do not fully capture real-world conditions such as limited onboard computing capabilities, communication delays, and message loss.

% Other approaches achieve true distributed computation~\citep{schwager2011eyes,franchi2012shared,li2020autonomous,preiss2017crazyswarm,mcguire2019minimalnavigationsolution,guo2023collision,zhou2022swarm,quan2022formation,lupashin2014flyingmachinearena}, necessitating communication between devices. 
% Some~\citep{preiss2017crazyswarm,mcguire2019minimalnavigationsolution,guo2023collision,lupashin2014flyingmachinearena} rely on a central base station, such as a WLAN router. 
% The Flying Machine Arena~\citep{lupashin2014flyingmachinearena} further features a distributed computation network of off-board computers that communicate via Ethernet or an industrial wireless channel. 
% \cite{li2020autonomous} use a single-hop network for communication. 
% Others~\citep{schwager2011eyes,franchi2012shared,zhou2022swarm,quan2022formation} implement \emph{mesh networks}, achieving distributed communication. 
% Although all these approaches demonstrate their feasibility on hardware, none of them offer formal guarantees for collision avoidance.

% \cite{riviere2020glas} achieve such guarantees, including a distributed hardware implementation. 
% They use imitation learning to train a central planner via a neural network, which the UAVs execute in a distributed manner. 
% Safety is ensured through a suitable potential function approach. 
% The hardware implementation includes up to 12 quadcopters, with each receiving position information about every other quadcopter from a central computer; thus, it does not consider mesh communication with latency and message loss.

% In contrast to these studies, our proposed method \myswarm{} employs distributed computation with mesh communication and offers provable guarantees even under conditions of communication latency and message loss.
% We achieve this by leveraging DMPC for drone swarms, which we evaluate next.

% \subsection{DMPC}
% We provided a brief overview of DMPC methods in Section~\ref{sec:introduction:background}. 
% In this section, we will elaborate on them in greater detail.

\label{sec:introduction:background}
% \begin{figure}
%     \centering
%     \includegraphics[width=0.99\linewidth]{Images/DMPCV2.pdf}
%     \caption{Overview of DMPC.}
%     \label{fig:dmpc}
% \end{figure}
% All these approaches have the same structure (cf. Figure~\ref{fig:dmpc}):
% \begin{enumerate}
%     \item Each UAV receives the trajectories planned in the last timestep from all other UAVs.
%     \\item Each UAV solves a QP to determine its trajectory. The constraints of the QP enforce collision avoidance with the trajectories received from other UAVs, while the objective function minimizes the distance to the target position.
%     \item Each UAV executes the initial part of its planned trajectory using local trajectory following controllers.
%     \item In parallel, each UAV broadcasts its planned trajectory to all other UAVs and resumes with step~1.
% \end{enumerate} 
%The initial DMPC approaches by \cite{Luis2019,Luis2020} did not provide any theoretical guarantees. 

 \reviewerthree{
    DMPC encompasses a broad category of methods.
    \cite{peng2024distributed} provide an overview of various DMPC types specifically applied to swarm control.
    Among these variants, the approaches presented in \cite{Luis2019,Luis2020,Park2022,Park2023,Chen2023,Chen2022,Graefe2022} form the backbone of our work.
    For brevity, we refer to these approaches simply as DMPC throughout the paper.

    It is important to emphasize that alternative DMPC methods also exist. For instance, some approaches solve centralized optimization problems using distributed techniques~\citep{stomberg2021distributed,stomberg2023dmpc,camponogara2002distributed}. In contrast to the DMPC variant employed here, these methods typically confine information exchange to the local neighborhood of agents.


The considered DMPC variant originated from \cite{augugliaro2012generation}, which introduced a centralized Sequential Convex Programming approach for trajectory planning by linearizing nonlinear collision constraints and solving Quadratic Programs (QPs).}
\cite{Luis2019} extended this method to distributed computation by assigning one QP per UAV; after solving their QPs and simulating a step forward, the swarm executes the final trajectories together. 
\cite{Luis2020} adapted the method for real-time control, synchronizing UAVs into periodic rounds.
During such a round, each UAV first runs a QP.
Afterwards, the UAVs exchange their solution in many-to-all communication rounds.
In parallel, each UAV executes its initial trajectory segment. 

Subsequent works extended DMPC to provide theoretical collision avoidance guarantees.
\cite{Park2022} utilized Bernstein polynomials for trajectory representation, leveraging their convex hull properties and final state constraints to ensure collision avoidance, recursive feasibility, and static obstacle avoidance. 
\cite{Park2023} extended this approach to dynamic obstacles, mitigating deadlocks through heuristics that temporarily adjust target positions.
Similarly, \cite{Chen2022} showed that soft constraints can prevent deadlocks with theoretical guarantees, which \cite{Chen2023} extended to include static obstacles. Both use time-varying separating planes, the dynamic extension of Buffered Voronoi Cells (BVC)~\citep{Zhou2017}, and final state constraints for collision avoidance and recursive feasibility.
Concurrently, \cite{Graefe2022} studied the feasibility of DMPC for nano UAV swarms by proposing to offload computations to ground agents and using event-triggered scheduling, however, without a distributed hardware implementation.
Further, they employ time-varying BVCs with final state constraints for collision avoidance guarantees.

In addition to theoretical advancements, the presented works' hardware experiments have demonstrated DMPC's practical suitability.
However, these rely on a \emph{single} central computer that simulates distributed computation and communication, streaming position commands to the UAVs.
This setup fails to capture real-world conditions like communication delays from limited communication bandwidth, message loss, and constrained computational power.

In contrast, we present a distributed hardware implementation.
Building upon \cite{Graefe2022}, we employ ground-based distributed event-triggered computation. 
To ensure collision avoidance even in the presence of message loss, a common occurrence in wireless communication, we propose MLR-DMPC, an extension of DMPC. 
Furthermore, to effectively mitigate deadlocks, we incorporate soft constraints~\citep{Chen2022, Chen2023} and high-level planning heuristics~\citep{Park2022, Park2023}.

Finally, we distinguish these methods from others also termed DMPC~\citep{peng2024distributed}.
In \cite{stomberg2021distributed, stomberg2023dmpc}, for example, centralized optimization problems are solved using distributed techniques.
While effective for controlling ground robots via distributed off-board computations, these methods require multiple communication steps per optimization run, making it too slow when running under limited communication bandwidth. 


% Over the last decade, numerous UAV swarm control approaches have been developed to address some challenges described above.
% Some of these approaches~\citep{augugliaro2012generation,schoellig2014so,michael2010grasp} achieve impressive dynamic swarm maneuvers, but rely on central control.
% Consequently, they experience the inherent disadvantages associated with central control, such as a single point of failure and limited scalability.
% This review therefore focuses exclusively on distributed control methods,
% emphasizing those that have demonstrated real-world feasibility through hardware experiments.

% The combination of challenges outlined in the introduction makes real-world distributed and safe robot swarms difficult to achieve. 
% Consequently, existing approaches meet only a subset of the three requirements \safetyguarnum{}--\reseffnum{}
% and none achieve all the requirements simultaneously.
% Tab.~\ref{tab:relatedworkaspects} categorizes existing works on distributed swarm control, dividing each requirement into two parts and indicating which have been effectively addressed.
% Following this table's structure, we now review the listed approaches.

% \begin{table}[t]
% \centering
%     \ra{1.2}
% 	\scriptsize 
% 	\fontsize{9.7pt}{9.7pt}\selectfont
%     \begin{tabular}{@{}ccccccc@{}}
%     \toprule
% 	&\multicolumn{2}{c}{\safetyguarnum{}: \safetyguarcc{}}&\multicolumn{2}{c}{\distcontrnum{}: \distcontrcc{}}&\multicolumn{2}{c}{\reseffnum{}: \reseffcc{}}\\
% 	\cmidrule(lr){2-3}\cmidrule(lr){4-5}\cmidrule(lr){6-7}
%     &\makecell{collision avoidance\\ without\\ message loss}&\makecell{collision avoidance\\ under\\ message loss}&\makecell{distributed\\ computation}&\makecell{mesh\\network}&\makecell{lightweight\\UAVs}&\makecell{low-bandwidth\\communication}\\
%     \midrule
%      \citep{schwager2011eyes}&\xmarkred&\xmarkred&\cmarkgreen&\cmarkgreen&\xmarkred&\cmarkgreen\\
%     \citep{franchi2012shared}&\xmarkred&\xmarkred&\cmarkgreen&\cmarkgreen&\xmarkred&\xmarkred\\
%     \citep{turpin2012trajectory}&\xmarkred&\xmarkred&\xmarkred&\xmarkred&\xmarkred&\xmarkred\\
%     \citep{kushleyev2013towards}&\xmarkred&\xmarkred&\xmarkred&\xmarkred&\cmarkgreen&\xmarkred\\

%     \citep{lupashin2014flyingmachinearena}&\xmarkred&\xmarkred&\cmarkgreen&\xmarkred&\xmarkred&\xmarkred\\
%     \citep{zhou2018agile}&\xmarkred&\xmarkred&\xmarkred&\xmarkred&\xmarkred&\xmarkred\\
%     \citep{li2020autonomous}&\xmarkred&\xmarkred&\cmarkgreen&\xmarkred&\cmarkgreen&\cmarkgreen\\
%     \citep{preiss2017crazyswarm,mcguire2019minimalnavigationsolution}&\xmarkred&\xmarkred&\cmarkgreen&\xmarkred&\cmarkgreen&\cmarkgreen\\
% 	\citep{guo2023collision}&\xmarkred&\xmarkred&\cmarkgreen&\xmarkred&\xmarkred&\xmarkred\\  
%     \citep{zhou2022swarm,quan2022formation}&\xmarkred&\xmarkred&\cmarkgreen&\xmarkred/\cmarkgreen&\xmarkred&\xmarkred\\
%     %\citep{stomberg2023dmpc}&\xmarkred&\xmarkred&\cmarkgreen&\xmarkred&\xmarkred\\
%     \citep{Park2022,Park2023,Chen2023,Chen2022}&\cmarkgreen&\xmarkred&\xmarkred&\xmarkred&\xmarkred&\xmarkred\\
% 	% \citep{Graefe2022}&\cmarkgreen&\xmarkred&\xmarkred&\xmarkred&\cmarkgreen&\cmarkgreen\\ 
%     This work&\cmarkgreen&\cmarkgreen&\cmarkgreen&\cmarkgreen&\cmarkgreen&\cmarkgreen\\
%     \bottomrule
%     \end{tabular}
%     \caption{Comparison of features of the proposed work with related studies on distributed control for UAV swarms. 
%     \capt{The table evaluates existing work based on the three requirements \safetyguarnum{}---\reseffnum{}.
%     Each requirement is split into two parts.
%     Given that every work in the table employs a distributed control method, we evaluate \distcontr{} based on whether the hardware implementation includes \emph{distributed computation} and \emph{mesh networks}.
%     Our work is the first to achieve \safetyguar{}, \distcontr{} and \reseff{} on a real UAV swarm.}}
%     \label{tab:relatedworkaspects}
% \end{table}

% \fakepar{\safetyguarnum{}: \safetyguarcc{}.} We differentiate between guarantees for \emph{collision avoidance without message loss}, frequently an unrealistic assumption for wireless communication, and guarantees for \emph{collision avoidance under message loss}.
% Distributed model predictive control (DMPC) based approaches~\citep{Park2022,Park2023,Chen2023,Chen2022} are the only ones offering collision avoidance guarantees.
% However, these guarantees can be compromised under conditions of message loss, as experimentally demonstrated in this article.
% All other approaches achieve impressive performance but lack formal guarantees. 
% As a consequence, we cannot exclude the possibility that these approaches may fail to avoid collisions in rare scenarios.


% \fakepar{\distcontrnum{}: \distcontrcc{}.}
% Every approach listed in Tab.~\ref{tab:relatedworkaspects} features a \distcontr{} method.
% Tab.~\ref{tab:relatedworkaspects} assesses how \distcontr{} is realized in the hardware implementation by determining whether it incorporates \emph{distributed computation} and whether all devices communicate through a \emph{mesh network}.

% Many works focusing on methodology~\citep{turpin2012trajectory,kushleyev2013towards,zhou2018agile,Park2022,Park2023,Chen2023,Chen2022} do not feature both of these elements as their implementations typically run the controller on a central computer that emulates distributed computation and communication. 
% Although they achieve impressive swarm maneuvers, this does not fully capture real world conditions, such as limited onboard computing, communication delay and message loss.

% Works realizing \emph{distributed computation}~\citep{schwager2011eyes,franchi2012shared,li2020autonomous,preiss2017crazyswarm,mcguire2019minimalnavigationsolution,guo2023collision,zhou2022swarm,quan2022formation,lupashin2014flyingmachinearena} must implement communication between devices.
% Some~\citep{preiss2017crazyswarm,mcguire2019minimalnavigationsolution,guo2023collision,lupashin2014flyingmachinearena} rely on a central base station, like a WLAN router.
% The Flying Machine Arena~\citep{lupashin2014flyingmachinearena} further features distributed off-board computers that communicate via Ethernet or an industrial wireless channel.
% Li et al.~\citep{li2020autonomous} use a single-hop network for communication.
% Others~\citep{schwager2011eyes,franchi2012shared,zhou2022swarm,quan2022formation} implement \emph{mesh networks}, achieving fully distributed systems, however, without achieving \safetyguarnum{} and \reseffnum{} simultaneously.

% \fakepar{\reseffnum{}: \reseffcc{}.} In terms of \reseffnum{}, we concentrate on \emph{lightweight UAVs} ($<\SI{100}{\gram}$) and \emph{low-bandwidth communication} ($<\SI{10}{\mega\bit\per\second}$).
% Few works support lightweight UAVs~\citep{li2020autonomous,preiss2017crazyswarm,mcguire2019minimalnavigationsolution}.
% Notably, although some~\citep{kushleyev2013towards,Park2022,Park2023,Chen2023,Chen2022} are evaluated on lightweight UAVs, they do so by emulating the control on a central computer.
% If these methods were to be run distributed on the UAVs as intended, they would need to be considerably heavier. 

% \emph{Lightweight UAVs} typically use \emph{low-bandwidth communication}~\citep{li2020autonomous,preiss2017crazyswarm,mcguire2019minimalnavigationsolution}.
% Schwager et al. \citep{schwager2011eyes} also use \emph{low-bandwidth communication} (\SI{10}{\kilo\bit\per\second}), albeit with heavier quadcopters.
% Other studies rely on higher bandwidth radios like Wi-Fi~\citep{lupashin2014flyingmachinearena,zhou2018agile,guo2023collision,zhou2022swarm,quan2022formation}, or emulate wireless communication~\citep{turpin2012trajectory,kushleyev2013towards,Park2022,Park2023,Chen2023,Chen2022}.

% In summary, no existing approaches simultaneously meet all requirements: \safetyguar{}, \distcontr{} and \reseff{}.
% To advance UAV swarm robotics, this work develops, implements, and experimentally evaluates a novel swarm architecture, \myswarm{}, that fulfills all three (Fig.~\ref{fig:swarmoverview}).