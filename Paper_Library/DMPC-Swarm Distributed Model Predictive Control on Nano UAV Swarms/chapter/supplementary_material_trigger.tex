\section{Event Triggers}
\label{app:et}

% We detail the event-trigger mechanism.
% To minimize communication overhead, we quantize priorities to \SI{8}{\bit} unsigned integers instead of \SI{32}{\bit} floats.

% Due to the one-round delay, when a UAV's trajectory is recalculated, its priority still reflects its state before recalculation. 
% To address this, if a CU has just recalculated UAV~$i$, it sets its priority to zero. 
% We also adjust the maximum operation in Equation~\ref{eq:priosmax} to return zero if any element is zero; otherwise, it returns the maximum. 
% The CU calculates priorities using the following equations.

To minimize communication overhead in the event-trigger mechanism, we quantize priorities to 8-bit unsigned integers instead of 32-bit floats. 
Due to the one-round delay, a recalculated UAV's priority may not reflect its updated state. 
To address this, if a CU has just recalculated UAV~$i$, it sets its priority to zero. 
We adjust the maximum operation in Equation~(\ref{eq:priosmax}) to return zero if any element is zero; otherwise, it returns the maximum. The CU calculates priorities using the following equations.


\subsection{Round-Robin Event Trigger (RR)}
The RR calculates priorities as:
\begin{equation}
    J_{iw}(k) = k - k_{i, \text{calc}}(k),
\end{equation}
where $k_{i, \text{calc}}(k)$ is the last round in which UAV $i$'s trajectory was calculated. If the CU's database contains multiple trajectories, it uses the first one.

\subsection{Distance-Based Event Trigger (DT)}
The DT calculates priorities as:
\begin{equation}
    J_{iw}(k) = \left\| p_{i, \mathrm{target}} - p_i\big(T\,|\,(k - 1)T\big) \right\|.
\end{equation}

\subsection{Hybrid Event Trigger (HT)}
The HT combines the previous methods:
\begin{align}
    J_{iw}(k) &= \left\| p_{i, \mathrm{target}} - p_i\big(T\,|\,(k - 1)T\big) \right\|\nonumber\\ 
    &\opindent\times\left[ k - k_{i, \text{calc}}(k) \right].
\end{align}


\subsection{Deadlock-Aware Triggering}

Since recalculating a deadlocked UAV's trajectory will not change it, we introduce a deadlock-aware triggering mechanism. 
The CU uses \cite[Theorem~1, Equation~(13)]{Chen2022} to detect deadlocks. 
If a UAV is in deadlock, the CU sets its priority to one.