%!TEX root = ../scifile.tex
\newcommand{\distributed}[1]{\textcolor{magenta!50!black}{#1}}
\newcommand{\groundbased}[1]{\textcolor{green!50!black}{#1}}
\section{Introduction}

Robot swarms have the potential to transform fields like smart farming, civil protection, and planetary exploration~\citep{trianni2016saga, blender2016xaverproject, cantizani2022bluetoothsearchandrescue, couceiro2013collectivesearchandrescue, staudinger2021robotswarmcommunication, zhang2020autonomousswarmnavigation, kang2019marsbee}. 
Although currently confined to laboratories, they are expected to become commonplace by 2030~\citep{dorigo2020reflections}.
This holds especially for swarms of unmanned aerial vehicles (UAVs) relevant for tasks such as smart farming~\citep{Walter2017, trianni2016saga, r2018research}, aerial surveillance~\citep{saska2016swarm, chung2018survey}, and wildlife observation~\citep{shah2020multidrone}. 
% In this work, we focus on multirotor UAV swarms.

A fundamental principle in robot swarms is \emph{distributed control}, where agents collaboratively make decisions through distributed computation and communication, unlike \emph{centralized control} by a single agent. 
Distributed control leverages the collective's computational power, enhances scalability, and improves resilience by avoiding single points of failure~\citep{mondada2004swarm, Luis2019, ge2017distributed}.

Building upon these advantages, recent work proposed a distributed optimization approach based on \emph{Distributed Model Predictive Control} (DMPC), which is one of the most promising distributed swarm control methods in recent years~\citep{Luis2019,Luis2020,Park2022,Park2023,Chen2023,Chen2022,Graefe2022}.
These approaches enable dynamic swarm maneuvers by continuously solving distributed optimization problems across the agents. 
Moreover, by incorporating specific constraints into its optimization problem, they can formally guarantee collision avoidance~\citep{Park2022, Park2023, Chen2023, Chen2022, Graefe2022}, which is crucial for ensuring safety in swarm operations.

% Building upon these advantages, \emph{Distributed Model Predictive Control} (DMPC) has emerged as one of the most promising distributed swarm control methods in recent years~\citep{Luis2019,Luis2020,Park2022,Park2023,Chen2023,Chen2022,Graefe2022}. 
% DMPC enables dynamic swarm maneuvers by continuously solving distributed optimization problems across the agents. 
% Moreover, by incorporating specific constraints into its optimization problem, DMPC can formally guarantee collision avoidance~\citep{Park2022, Park2023, Chen2023, Chen2022, Graefe2022}, which is crucial for ensuring safety in swarm operations.

However, despite promising theoretical analyses and simulations demonstrating DMPC's potential, existing hardware implementations execute DMPC on a central computer, while simulating selected effects of distributed computation and communication~\citep{Luis2020,Park2022,Park2023,Chen2023,Chen2022,Graefe2022}. 
Although these implementations confirm DMPC's conceptual suitability for swarm control, they do not realize the practical benefits of distributed architectures. 
Moreover, centralized implementations oversimplify real-world conditions, particularly communication delays and message loss, rendering them inadequate for evaluating DMPC-controlled swarms in scenarios where the advantages of distributed systems are essential.

To bridge this gap, we develop and implement \myswarm{}, the first \emph{distributed realization} of DMPC for swarm control over wireless networks. 
We identify three key challenges to a real-world realization of DMPC, which we address in this work:

\begin{enumerate}
    \item[\textbf{C1}] \textbf{Communication:} 
    DMPC requires communication between agents over a wireless network. 
    In \myswarm{}, we aim to achieve this communication over self-organizing wireless mesh networks, where agents transmit data for one another based on device-to-device communication. 
    This preserves the distributed nature of the swarm and enhances scalability, reliability, and efficiency compared to a network in which agents are limited to direct communicate with a dedicated base station~\citep{laneman04}.
    However, wireless mesh networks introduce complexities, especially as robots move and the network topology continuously evolves~\citep{khan2019hybrid, oubbati2019routing, namdev2021optimized}. 
    Consequently, achieving the communication required by DMPC approaches becomes even more challenging.

    \item[\textbf{C2}] \textbf{Computation:} 
    DMPC necessitates significant computational power because it involves repeatedly solving optimization problems. However, UAVs, especially nano-quadcopters used in applications such as agriculture~\citep{gago2020nano}, indoor source seeking~\citep{duisterhof2021tiny, karaguzel2023shadows} and indoor visual inspection~\citep{tavasoli2023real}, have limited onboard computing capabilities due to constraints in size, weight, and cost.
    Despite these limitations, most existing DMPC implementations fail to address the computational restrictions inherent in such UAVs. Furthermore, swarm systems often comprise heterogeneous computing architectures with varying computational capabilities, for instance, quadcopters of different sizes, mobile robots, and operator devices. Current DMPC approaches generally do not exploit this diversity in computational resources.
    
    \item[\textbf{C3}] \textbf{Unrealistic assumptions of DMPC:} 
    Moreover, existing DMPC methods often overlook practical challenges such as computational delays due to limited processing power, communication delays and message loss resulting from unreliable wireless communication. 
    These issues pose significant hurdles, as they may cause the collision avoidance guarantees provided by DMPC methods to fail in practice, thereby limiting their real-world applicability. 
    Solving these challenges on the algorithmic side is crucial for the successful deployment of DMPC in practical swarm applications.

\end{enumerate}

We address these challenges through a novel combination of a stateless communication protocol, a distributed and event-triggered compute architecture leveraging multiple heterogeneous compute nodes, as well as algorithmic extensions to DMPC, achieving the first distributed realization of DMPC.

\reviewerthree{We note that different concepts and notions of distributed systems exist. In our work, we define ``distributed'' as collaborative problem-solving among multiple agents, sharing workloads and operating without a single coordinator~\citep{hu2018centralize,cheraghi2022pastpresentfuture,lupashin2014flyingmachinearena}. 

One prevalent alternative view on distributed systems emphasizes interaction and information exchange with agents in a neighborhood and peer-to-peer fashion~\citep{antonelli2013interconnected, ge2017distributed}. 
Although such approaches offer scalability benefits, they can also involve downsides in other settings.
For example, defining and tracking which agents qualify as neighbors is often problematic, particularly in scenarios involving rapidly moving swarms, which complicates the theoretical analysis of algorithms.
This is especially relevant when the interaction involves wireless communication, where the behavior and existence of communication links to neighbors can be difficult to predict due to the significant influence of environmental factors. 

Following our view on ``distributed'' as collaborative problem-solving and distribution of workloads, we consider any system that facilitates these functions as a potential solution, rather than restricting interaction to a peer-to-peer fashion.
\reviewerall{Accordingly, we design and integrate all three system components, algorithms, computing infrastructure, and communication infrastructure, into an architecture featuring distributed entities at each component.}
For example the distributed communication solution that we propose is based on physical neighbor-to-neighbor communication, while enabling a many-to-all information exchange and global interaction as part of the solution.

%Given these challenges, we favor an approach that permits direct communication and collaboration among agents beyond immediate neighbors, based on the broader notion of distributed systems described above. In this context, our work represents the first distributed implementation of DMPC.
}

% TODO add discussion distributed. The notion of distributed systems has multiple aspects, in particular communication, computing and algorithms. Here, we consider distributed as collaborative problem solving... as in literature. To solve this, 
% We jointly consider all three aspects as our solution space. different approaches, definitions and solutions exist in literature, for example, in communication, algorithms, there are peer-to-peer approaches, which use distributed communication, which hcan have advantages in case of scalability, but also downsides... .

% Our approach is the first realization of distributed..., in the sense of our notion of distributed.

Before detailing our contributions, we make the swarm control problem precise.


\subsection{Problem Setting}
\label{sec:introduction:problemsetting}

% \begin{figure}
%     \centering
%     \includegraphics[width=0.99\linewidth]{Images/Setting.pdf}

%     \caption{Problem Setting.
%     \capt{Tracking of target positions $p_{i, \mathrm{target}}$ is considered as the swarm control problem. A\myswarm{} realizes for the first time DMPC in hardware, leveraging a mesh network, multiple compute units (CUs), like ground-based robots or devices from users, and a DMPC robust against message loss.}}
% \label{fig:setting}
% \end{figure}


We consider a swarm $\mathcal{A}=\{1,\cdots,N\}$ of $N$ UAVs.  % as illustrated in Figure~\ref{fig:setting}. 
At any time $t$, each UAV $i \in \mathcal{A}$ has a position $p_{i}(t) \in \mathbb{R}^3$. 
A UAV can measure its own position but not the positions of the other UAVs.
Each must navigate from its initial position $p_{i, \mathrm{init}} = p_{i}(0)$ to a target position $p_{i, \mathrm{target}}$ without colliding with other UAVs.

The target positions $p_{i, \mathrm{target}}$ may change over time, e.g., when a UAV receives a new task at a different location. 
This dynamic environment necessitates frequent real-time trajectory replanning.
Such distributed position-following problem is the basis of many swarming tasks and is the standard problem setting of DMPC~\citep{Luis2019,Luis2020,Park2023,Park2022,Chen2023,Chen2022}.

Each UAV can run a low-level trajectory-following controller but lacks the resources to solve the optimization problems in DMPC. 
However, the network---in addition to the $N$ UAVs---also includes $M$ agents ($M > 1$) with higher computational power, called \emph{compute units} (CUs). 
We assume $M < N$ to account for limited capacities. 
Such CUs are common in practice, e.g., in smart farming or manufacturing, UAVs may connect to digital devices (e.g., smartphones or laptops) or rely on edge clouds and computational resources from other agents like ground robots or larger UAVs~\citep{baumann2020wireless, sankaranarayanan2023paced, liu2021boost}.

All UAVs and CUs communicate via a wireless mesh network, each device equipped with a wireless transceiver. 
Given the limited range of these transceivers, devices can directly communicate only with their immediate neighbors. 
As the agents move, the network topology changes dynamically, continuously altering the quality and availability of direct communication links, which is known to be challenging for efficient and reliable communication~\citep{khan2019hybrid, oubbati2019routing,namdev2021optimized,chriki2019fanet,lv2023survey}.

\textbf{Problem Statement.} Our goal is to develop a swarm architecture for the safe control of real UAV swarms using DMPC, realized on distributed physical hardware and a real wireless mesh network. 
The DMPC computations must be distributed across multiple CUs, considering their limited number, and should ensure that the UAVs' positions converge to the target positions over time
\begin{equation}
 \forall i\in \mathcal{A}: p_{i}(t)\to p_{i,\mathrm{target}},
\end{equation}
while provably avoiding collisions between UAVs
\begin{align} 
    \label{eq:truecollisionavoidance} 
    \forall t\in\mathcal{R}_0^+,&~\forall i,j\in\mathcal{A},~i\neq j:\\ 
    &\left\| \Theta^{-1}[p_j(t)-p_i(t)] \right\|_2 \geq d_\mathrm{min},\nonumber 
\end{align}
where $\Theta$ is a scaling matrix for downwash effects, and $d_\mathrm{min}$ is the minimum allowable distance~\citep{Luis2019, Luis2020, Chen2022, Chen2023,Park2022, Park2023, Graefe2022}.
The DMPC framework should function under real-world conditions, such as communication message loss, delays and limited computational resources.


\subsection{Contributions}

Our solution, \myswarm{}, effectively solves this problem by addressing challenges \textbf{C1}--\textbf{C3}.
Overall, we make the following contributions:

\begin{enumerate}
    \item We present \myswarm{}, a novel swarm architecture combining low-power wireless communication, distributed computing and a new DMPC algorithm. 
    The communication protocol is resilient to rapid device movement and provides the communication required by DMPC.
    The architecture building on top of it efficiently balances the DMPC computational load across UAVs and CUs. 

    \item \reviewerthree{We introduce message-loss-recovery DMPC (MLR-DMPC) as the algorithmic part of \myswarm{}.}. This DMPC method handles message loss and communication delays, providing collision avoidance guarantees under realistic conditions.

    \item We implement \myswarm{}, achieving the first distributed hardware realization of DMPC-based swarm control on nano-quadcopters. 
    Our experimental setup features up to 16 nano-quadcopters communicating over a wireless mesh network using Bluetooth Low Energy (BLE). 
    The entire system is based on the popular Crazyflie platform, with its software and hardware files accessible at \url{https://github.com/Data-Science-in-Mechanical-Engineering/DMPC-Swarm}.
    A video demonstrating our results is available at \url{http://tiny.cc/DMPCSwarm}.
\end{enumerate}