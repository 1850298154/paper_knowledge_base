% \section{Communication Phase Latency}
% \label{app:tcom}

% 	Before deriving how we calculate the communication phase length, we first have to explain our communication system more into detail.
% 	Because of the network coding in Mixer, all messages sent via Mixer must have the same size.
% 	In \myswarm{}, however, devices send messages with different message sizes.
% 	To support this, we extend Mixer with a message layer.
% 	In the message layer, $K$ message fields with different message sizes $S_i$ are defined beforehand
% 	\begin{equation}
% 	\label{eq:messagelayer}
% 	\mathcal{S} = \{S_i | i\in\{0, \cdots, K-1\}\},
% 	\end{equation}
% 	where $K$ is the overall number of sent messages.
% 	These message fields are then split onto smaller messages that have a constant message size $S_\mathrm{m}$.
%     Mixer then transmits these smaller messages.
	
% 	Their size $S_\mathrm{m}$ can be chosen freely.
% 	Because it might not be possible to split a message field onto an integer number of messages, some spaces in the messages are empty and do not contain information.
% 	If we use a small $S_\mathrm{m}$, the size of empty spaces in the messages is small, but the communication overhead is bigger.
% 	For big $S_\mathrm{m}$, the communication overhead is smaller, but the empty spaces are bigger.
% 	We therefore aim to find an $S_\mathrm{m}$ that leads to minimal communication phase latency.
	
% 	The number of Mixer messages $M_\mathrm{m}$ is (together with the initiator message sent by one device to start the communication phase and share the current round time $k$ among all devices)
% 	\begin{equation}
% 	M_\mathrm{m} = 1 +  \sum_{S_i\in\mathcal{S}}\lceil\frac{S_i}{S_\mathrm{m}}\rceil.
% 	\end{equation}

%     We choose the number of slots per communication phase using the heuristic~\citep{Mixer}
%     \begin{equation}
%         M_\mathrm{s} = \max(3M_\mathrm{m}, 170).
%     \end{equation}

%     The length of one slot $T_\mathrm{s}$ depends not only on $S_\mathrm{m}$ and $M_\mathrm{m}$, but also on the hardware and the physical layer used.
%     The exact calculation of $T_\mathrm{s}$ for our implementation can be found in the code provided upon publication of this paper.
	
%     The optimal message size and length of one communication phase is then
% 	\begin{equation}
% 	S_\mathrm{p}^* = \arg \min_{S_\mathrm{m}\in\mathbb{N}_+}(M_\mathrm{s}T_\mathrm{s})~~~~ T_\mathrm{com} = \min_{S_\mathrm{m}\in\mathbb{N}_+}(M_\mathrm{s}T_\mathrm{s}).
% 	\label{eq:roundlength}
% 	\end{equation}