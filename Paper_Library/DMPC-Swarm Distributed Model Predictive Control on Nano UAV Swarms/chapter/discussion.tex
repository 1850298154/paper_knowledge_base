%!TEX root = ../scifile.tex

% \section{Discussion}


% The previous section demonstrated that \myswarm{} succesfully realized all three components \compdistrcomp{}--\compmlr{}. 
% We now discuss the results and associated limitations with respect to these three components.

% \subsection{Distributed Computation}
% \myswarm{} utilizes ground-based distributed computations on CUs, as proposed in~\citep{Graefe2022}.
% We are the first to implement this approach in hardware and demonstrate its real-world feasibility.
% By running MLR-DPMPC, an event-triggered \distcontr{} method, on the CUs, our system enables flexible scaling, i.e., the number of CUs can be adjusted according to the number of UAVs.
% Our approach also supports scenarios where bandwidth and hardware only allow for a few CUs, ensuring efficient resource usage.

% \subsection{Efficient and Tailored Communication}
% \myswarm{} uses a mesh wireless network and Mixer, a communication protocol using synchronous transmission.
% The protocol is robust against rapid device movement, which occurs in UAV swarms, has high reliability and features low latency.
% In this work, we have demonstrated that Mixer's synchronized many-to-all communication structure is ideal for realizing DMPC.

% Apart from this, this work is the first to demonstrate that protocols using synchronous transmissions are effective for UAV swarm communication.
% This holds true beyond the specific type of control discussed in this work (MLR-DMPC). 
% Future research should investigate the benefits of this communication architecture for other types of swarm control.


% \subsection{Communication Aware DMPC}
% Ensuring collision avoidance is a fundamental requirement for any UAV swarm and a core aspect of swarm safety.
% In theory and practice, our experiments demonstrate that \myswarm{} provides such guarantees.
% Its control method, MLR-DMPC, specifically accounts for message loss of wireless communication.
% Notably, existing DMPC approaches~\citep{Park2022,Park2023,Chen2022,Chen2023,Graefe2022} do not focus on guarantees under message loss.
% Our experimental results highlight that neglecting this aspect can lead to collisions between robots when message loss occurs, even with a low message loss probability.

% \myswarm{}'s guarantees rely on mild assumptions on the low-level controller, which are fulfilled by standard PID control.
% However, depending on the UAV, this might require careful controller tuning.

% While we can prove collision avoidance, formally guaranteeing that all quadcopters reach their targets remains an open challenge, a known issue in the context of DMPC~\citep{Park2022,Chen2022,Chen2023}.
% To increase the chances of all targets being reached, our method combines direction-dependent soft constraints~\citep{Chen2023} with a high-level planner similar to~\citep{Park2022} (S2).
% Empirically, our results demonstrate that this approach is practical, with over $\SI{99}{\percent}$ of UAVs successfully reaching their target positions despite flying in a confined space.

% \subsection{Limitations}
% Although our distributed control approach is efficient for small and medium-sized swarms, it does not scale well to huge swarms of hundreds to thousands of agents.
% First, the communication rounds would take too long due to the many-to-all structure.
% Second, solving the optimization problems would also take a long time as MLR-DMPC includes all UAVs in anti-collision constraints.
% In such large-scale systems, a many-to-all communication approach is not necessary.
% For example, \cite{Chen2022} showed that for DMPC, UAVs only need to consider other UAVs in their neighborhood.
% Although directly integrating this idea into MLR-DMPC is feasible, there is currently no efficient means of achieving local many-to-many communication via synchronous transmissions.

% In this work, we assume a UAV flying space devoid of static obstacles, such as buildings. However, incorporating such obstacles into MLR-DMPC is straightforward using established approaches from other DMPC methods \citep{Park2022,Park2023,Chen2023}.


% \subsection{\reseffcc{}}
% \myswarm~achieves multiple aspects of \reseff{}.
% By offloading expensive computations, like online optimization, to the ground, our approach supports lightweight UAVs, such as Crazyflie nano-quadcopters.
% The combination of distributed computation with event-triggering also enhances bandwidth and hardware efficiency.
% We have shown experimentally that the right choice of an event-trigger can significantly reduce the number of CUs required, conserving hardware and bandwidth resources while maintaining or even improving performance.

% At present, our event-triggering system relies on heuristics, a method that works well but also has its limitations.
% Machine learning methods, e.g., reinforcement learning~\citep{baumann2018deep,kesper2023toward,funk2021learning}, could be used in the future to obtain optimal performing triggers. 


\section{Conclusion}

We introduced \myswarm{}, the first hardware implementation of DMPC-based swarm control with distributed computation and wireless mesh communication. 
\myswarm{} employs the communication protocol Mixer, showing that its synchronized many-to-all communication structure, robust to rapid device movement, is ideal for realizing DMPC. 
\myswarm{} uses ground-based, event-triggered distributed computations on CUs, enabling flexible scaling with the number of UAVs and efficient resource usage, even with limited bandwidth and hardware resources. 
Finally, we developed a novel DMPC algorithm that provably prevents collisions even under message loss. 
Our hardware experiments demonstrate that this combination enables DMPC on nano UAV swarms for the first time.

Our findings on the suitability of synchronous transmission protocols for UAV swarm communication extend beyond DMPC. 
Future research could explore the benefits of this communication architecture for other UAV swarm control methods.

\reviewerthree{Although our distributed control approach is efficient for small and medium-sized swarms, it does not readily scale to huge swarms of hundreds to thousands of agents.
First, the communication rounds would take too long due to the many-to-all structure.
Second, solving the optimization problems would also take a long time as MLR-DMPC includes all UAVs in anti-collision constraints.
In such large-scale systems, a many-to-all communication approach is not necessary.
For example, \cite{Chen2022} showed that for DMPC, UAVs only need to consider other UAVs in their neighborhood.
Although directly integrating this idea into MLR-DMPC is feasible, there is currently no efficient means of achieving local many-to-many communication via synchronous transmissions, which is an interesting challenge for future work.}

Another interesting future challenge is to eliminate Assumption~\ref{as:tracking}. 
For strong external disturbances, like sudden gusts, this Assumption~\ref{as:tracking} is most likely not valid or too conservative (i.e., $\Delta d_\mathrm{min}$ would be very large).
One possible way to eliminate this assumption would be to include recent event-triggered robust MPC~\citep{grafe2025event}.

% \section{Conclusion and Outlook}

% In this work, we introduced \myswarm{}, the first hardware implementation of DMPC-based swarm control featuring distributed computation and communication, encompassing all essential components for successful DMPC deployment.
% Frist, \myswarm{} employs wireless mesh communication using synchronous transmission protocols, specifically Mixer. We demonstrated that its synchronized many-to-all communication structure, robust to rapid device movement, is ideal for implementing DMPC.
% Second, \myswarm{} implements ground-based, event-triggered distributed computations on control units, demonstrating the practical feasibility of the theoretical work in~\cite{Graefe2022}. This approach allows flexible scaling with the number of UAVs, and the event trigger ensures efficient resource usage, even with limited bandwidth and hardware resources.
% Third, we developed a novel DMPC method that provably prevents collisions even under message loss.
% In hardware experiments, we showed that this combination enables DMPC on nano UAV swarms for the first time.

% Our findings on the suitability of synchronous transmission protocols for UAV swarm communication are not limited to MLR-DMPC. 
% Future research should explore the benefits of this communication architecture for other UAV swarm control methods.


% In future research, we will explore the benefits of this communication architecture for other swarm control methods.

% Although we can prove collision avoidance, formally guaranteeing that all quadcopters reach their targets remains an open challenge in DMPC~\citep{Park2022, Park2023, Chen2022, Chen2023}. To enhance the likelihood of reaching all targets, we combine direction-dependent soft constraints~\citep{Chen2023} with a high-level planner similar to~\cite{Park2022} (Appendix~\ref{app:deadlock}). Empirically, over $\SI{99}{\percent}$ of UAVs successfully reached their target positions, demonstrating the practicality of our approach even in confined spaces.

% \fakepar{Communication.}
% \myswarm{} employs a mesh wireless network using synchronous transmission protocols, specifically Mixer. This protocol is robust to the rapid device movements typical of UAV swarms, offers high reliability, and features low latency. 
% We demonstrated that its synchronized many-to-all communication structure is ideal for realizing DMPC.

% Additionally, our work is the first to show that synchronous transmission protocols are effective for UAV swarm communication beyond MLR-DMPC. Future research should explore the benefits of this communication architecture for other swarm control methods.

% \fakepar{Distributed Computation.}
% \myswarm{} implements ground-based, event-triggered distributed computations on CUs, demonstrating the practical feasibility of the theoretical work in~\cite{Graefe2022}. 
% This approach allows flexible scaling with the number of UAVs, and the event-trigger ensures efficient resource usage, even with limited bandwidth and hardware.

% \fakepar{MLR-DMPC.}
% Existing DMPC approaches~\citep{Park2022, Park2023, Chen2022, Chen2023, Graefe2022} lack guarantees under message loss. 
% Our experimental results highlight that neglecting this aspect can lead to collisions between robots when message loss occurs. To address this issue, we introduced MLR-DMPC, which provably prevents collisions under message loss.

% Although we can prove collision avoidance, formally guaranteeing that all quadcopters reach their targets remains an open challenge in DMPC~\citep{Park2022, Park2023, Chen2022, Chen2023}. To enhance the likelihood of reaching all targets, we combine direction-dependent soft constraints~\citep{Chen2023} with a high-level planner similar to~\cite{Park2022} (Appendix~\ref{app:deadlock}). Empirically, over $\SI{99}{\percent}$ of UAVs successfully reached their target positions, demonstrating the practicality of our approach even in confined spaces.

% In this work, we assume a UAV flying space devoid of static obstacles, such as buildings. However, incorporating such obstacles into MLR-DMPC is straightforward using established approaches from other DMPC methods \citep{Park2023,Chen2023,Park2022}.


% \subsection{Limitations}
% Although our distributed control approach is efficient for small and medium-sized swarms, it does not scale well to huge swarms of hundreds to thousands of agents.
% First, the communication rounds would take too long due to the many-to-all structure.
% Second, solving the optimization problems would also take a long time as MLR-DMPC includes all UAVs in anti-collision constraints.
% In such large-scale systems, a many-to-all communication approach is not necessary.
% For example, \cite{Chen2022} showed that for DMPC, UAVs only need to consider other UAVs in their neighborhood.
% Although directly integrating this idea into MLR-DMPC is feasible, there is currently no efficient means of achieving local many-to-many communication via synchronous transmissions.

% In this work, we assume a UAV flying space devoid of static obstacles, such as buildings. However, incorporating such obstacles into MLR-DMPC is straightforward using established approaches from other DMPC methods \citep{Park2022,Park2023,Chen2023}.











% \myswarm{} ensured collision avoidance by accounting for message loss in wireless communication. Unlike existing approaches, it provided guarantees under message loss conditions. While formally guaranteeing all quadcopters reach their targets remains an open challenge, our method showed practical success with over 99% target achievement in confined spaces.

% The approach is efficient for small to medium-sized swarms but does not scale well to very large swarms due to time-consuming communication rounds and optimization problem-solving. Future research should explore integrating local many-to-many communication methods into MLR-DMPC for better scalability. Additionally, while static obstacles were not considered in this work, incorporating them using established DMPC methods is straightforward.

% Overall, \myswarm{} demonstrates significant advancements in UAV swarm control while highlighting areas for future improvement to handle larger scales and more complex environments effectively.
