\begin{figure*}[t]
	\centering
	\fontsize{9.3pt}{9.3pt}\selectfont
	\newcommand{\radiusplot}{0.1}
	\newcommand{\plotlinewidth}{1.0pt}
	\newcommand{\myfigwidth}{0.4\textwidth}
	\newcommand{\myfigheight}{0.3\textwidth}
	\newcommand{\marksize}{1.0pt}
	\newcommand{\marksizedrone}{3.0pt}
	\newcommand{\boxshift}{0.05cm}
	\newcommand{\boxsize}{0.05cm}
	\newcommand{\constraintlinewidth}{1pt}
	\newcommand{\lw}{1.2pt}
	\newcommand{\marksizequadcopter}{2.0pt}
	\newcommand{\colormlrdmpc}{green!50!black}

	\newcommand{\colorone}{red}
	\newcommand{\colortwo}{blue}
	\newcommand{\colorthree}{green}

	\newcommand{\newtasklinewidth}{1.2pt}
	\newcommand{\labelheight}{4.6}

    \begin{subfigure}{0.95\textwidth}
        \includegraphics[width=0.99\linewidth]{Images/BigSwarm.pdf}
        \caption{The swarm's maneuvers in the experiments.}
        \label{fig:hardware:distrcomp:maneouver}
    \end{subfigure}

	\begin{subfigure}{0.95\textwidth}
		\centering
		\begin{tikzpicture}[spy using outlines={circle, magnification=3, size=2cm, connect spies, every spy on node/.append style={thick}}]
	%\coordinate[right=1cm of coord] (coord2);
			\begin{axis}[
                    axis on top,
                    fill between/on layer={main},
                    xmax=100.1, xmin=-0.1,
                    ymax=5, ymin=-0.1,
                    xlabel={Time (\SI{}{\second})},
                    xlabel style={yshift=3mm},
                    xtick pos=bottom, 
                    ylabel near ticks,
                    ylabel style={align=center, xshift=-3mm, yshift=-2mm}, ylabel={Distance\\ to target (\SI{}{\meter})},
                    legend style={at={(axis cs: 100.1,3)},anchor=south west ,draw=black,fill=white,align=left, fill opacity=0.8, nodes={scale=0.6, transform shape}}, 
                    height=\myfigheight, width=0.9\textwidth, box plot width=0.15cm,
				% grid=both, ymajorgrids=true, yminorgrids=true, xmajorgrids=true, grid style={line width=0.5pt, dashed, draw=gray!50},
				]
					
                    \addplot[mark=none, color=\colorone, line width=\lw, name path=lowertemp, forget plot] table [x=t, y=dmin, col sep=comma] {plot_data/HardwareExperimentFigures_1CUs.csv};
                    
                    \addplot[mark=none, color=\colorone, line width=\lw, name path=uppertemp, forget plot] table [x=t, y=dmax, col sep=comma] {plot_data/HardwareExperimentFigures_1CUs.csv};
                    
                    \addplot[color=\colorone, opacity=\fillopacity, legend image post style={opacity=1.0}] fill between[of=uppertemp and lowertemp];
                    \addlegendentry{1 CU};

				% \addplot[mark=none, color=red, line width=\lw] table [x=t, y=mean, col sep=comma] {plot_data/HardwareExperimentFigures_1CU.csv};

				%%%%%%%%%%%% 2 CUS %%%%%%%%%%

                \addplot[mark=none, color=\colortwo, line width=\lw, name path=lowertempp, forget plot] table [x=t, y=dmin, col sep=comma] {plot_data/HardwareExperimentFigures_2CUs.csv};
                
                \addplot[mark=none, color=\colortwo, line width=\lw, name path=uppertempp, forget plot] table [x=t, y=dmax, col sep=comma] {plot_data/HardwareExperimentFigures_2CUs.csv};

				
				% such that color is not purple
                \addplot[color=white, legend image post style={opacity=1.0}, forget plot] fill between[of=uppertempp and lowertempp];
                
                \addplot[color=\colortwo, opacity=\fillopacity, legend image post style={opacity=1.0}] fill between[of=uppertempp and lowertempp];
                \addlegendentry{2 CUs};

				% \addplot[mark=none, color=\colortwo, line width=\lw] table [x=t, y=mean, col sep=comma] {plot_data/HardwareExperimentFigures_2CU.csv};

				%%%%%%%%%%%% 3 CUS %%%%%%%%%%

                \addplot[mark=none, color=\colorthree, line width=\lw, name path=lowertemppp, forget plot] table [x=t, y=dmin, col sep=comma] {plot_data/HardwareExperimentFigures_3CU.csv};
                
                \addplot[mark=none, color=\colorthree, line width=\lw, name path=uppertemppp, forget plot] table [x=t, y=dmax, col sep=comma] {plot_data/HardwareExperimentFigures_3CUs.csv};

                \addplot[color=\colorthree, opacity=\fillopacity, legend image post style={opacity=1.0}] fill between[of=uppertemppp and lowertemppp];
                \addlegendentry{3 CUs};

				% \addplot[mark=none, color=\colorthree, line width=\lw] table [x=t, y=mean, col sep=comma] {plot_data/HardwareExperimentFigures_3CU.csv};
				
				%%%%%%%%%% descriptions
                \draw[solid, line width=\newtasklinewidth] (axis cs:22,-2) -- (axis cs:22,8);
                \draw[solid, line width=\newtasklinewidth] (axis cs:44,-2) -- (axis cs:44,8);
                \draw[solid, line width=\newtasklinewidth] (axis cs:66,-2) -- (axis cs:66,8);


                \coordinate (planecoordone) at (axis cs: -7, \labelheight);
                \coordinate (pyramidcoord) at (axis cs: 11, \labelheight);
                \coordinate (cubecoord) at (axis cs: 33, \labelheight);
                \coordinate (spherecoord) at (axis cs: 55, \labelheight);
                \coordinate (planecoord) at (axis cs: 77, \labelheight);

			\end{axis}

                \node[anchor=center] (planeone) at (planecoordone) {\small \textbf{Plane}\strut};
                \node[anchor=center] (pyramid) at (pyramidcoord) {\small \textbf{Pyramid}\strut};
                
                \node[anchor=center] (cube) at (cubecoord) {\small \textbf{Cube}\strut};
                
                \node[anchor=center] (sphere) at (spherecoord) {\small \textbf{Sphere}\strut};
                
                \node[anchor=center] (plane) at (planecoord) {\small \textbf{Plane}\strut};
                
                \draw[dashed, line width=\newtasklinewidth, ->] (planeone) -- (pyramid);
                \draw[dashed, line width=\newtasklinewidth, ->] (pyramid) -- (cube);
                \draw[dashed, line width=\newtasklinewidth, ->] (cube) -- (sphere);
                \draw[dashed, line width=\newtasklinewidth, ->] (sphere) -- (plane);

            \end{tikzpicture}
            \caption{Performance of the swarm with respect to the number of CUs.}
            \label{fig:hardware:distrcomp:numcu}
        \end{subfigure}

        \begin{subfigure}{0.95\textwidth}
            \centering
            \begin{tikzpicture}[spy using outlines={circle, magnification=3, size=2cm, connect spies, every spy on node/.append style={thick}}]
            \begin{axis}[
                    axis on top,
                    fill between/on layer={main},
                    xmax=100.1, xmin=-0.1,
                    ymax=5.0, ymin=-0.1,
                    xlabel={Time (\SI{}{\second})}, 
                    xlabel style={yshift=3mm},
                    xtick pos=bottom,
                    ylabel near ticks,
                    ylabel style={align=center, xshift=-3mm, yshift=0mm}, ylabel={Distance\\ to target (\SI{}{\meter})},
                    legend style={at={(axis cs: 100,3)},anchor=south west ,draw=black,fill=white,align=left, fill opacity=0.8, nodes={scale=0.6, transform shape}}, 
                    height=\myfigheight, width=0.9\textwidth, box plot width=0.15cm,
				% grid=both, ymajorgrids=true, yminorgrids=true, xmajorgrids=true, grid style={line width=0.5pt, dashed, draw=gray!50},
				]
					
                    \addplot[mark=none, color=\colorone, line width=\lw, name path=lowertemp, forget plot] table [x=t, y=dmin, col sep=comma] {plot_data/HardwareExperimentFigures_RR.csv};
                    
                    \addplot[mark=none, color=\colorone, line width=\lw, name path=uppertemp, forget plot] table [x=t, y=dmax, col sep=comma] {plot_data/HardwareExperimentFigures_RR.csv};
                    
                    \addplot[color=\colorone, opacity=\fillopacity, legend image post style={opacity=1.0}] fill between[of=uppertemp and lowertemp];
                    \addlegendentry{RR};
                    
				% \addplot[mark=none, color=red, line width=\lw, forget plot] table [x=t, y=mean, col sep=comma] {plot_data/HardwareExperimentFigures_RR.csv};

				%%%%%%%%%%%% 3 CUS %%%%%%%%%%

				\addplot[mark=none, color=\colortwo, line width=\lw, name path=lowertempp, forget plot] table [x=t, y=dmin, col sep=comma] {plot_data/HardwareExperimentFigures_DT.csv};

				\addplot[mark=none, color=\colortwo, line width=\lw, name path=uppertempp, forget plot] table [x=t, y=dmax, col sep=comma] {plot_data/HardwareExperimentFigures_DT.csv};

				% such that color is not purple
				\addplot[color=white, legend image post style={opacity=1.0}, forget plot] fill between[of=uppertempp and lowertempp];

				\addplot[color=\colortwo, opacity=\fillopacity, legend image post style={opacity=1.0}] fill between[of=uppertempp and lowertempp];
				\addlegendentry{DT};

				% \addplot[mark=none, color=blue, line width=\lw, forget plot] table [x=t, y=mean, col sep=comma] {plot_data/HardwareExperimentFigures_DT.csv};

				%%%%%%%%%%%% 3 CUS %%%%%%%%%%

				\addplot[mark=none, color=\colorthree, line width=\lw, name path=lowertempp, forget plot] table [x=t, y=dmin, col sep=comma] {plot_data/HardwareExperimentFigures_2CUs.csv};

				\addplot[mark=none, color=\colorthree, line width=\lw, name path=uppertempp, forget plot] table [x=t, y=dmax, col sep=comma] {plot_data/HardwareExperimentFigures_2CUs.csv};

				\addplot[color=\colorthree, opacity=\fillopacity, legend image post style={opacity=1.0}] fill between[of=uppertempp and lowertempp];
				\addlegendentry{HT};

				% \addplot[mark=none, color=\colordarkgreen, line width=\lw, forget plot] table [x=t, y=mean, col sep=comma] {plot_data/HardwareExperimentFigures_2CU.csv};

				%%%%%%%%%% descriptions
                \draw[solid, line width=\newtasklinewidth] (axis cs:22,-2) -- (axis cs:22,8);
                \draw[solid, line width=\newtasklinewidth] (axis cs:44,-2) -- (axis cs:44,8);
                \draw[solid, line width=\newtasklinewidth] (axis cs:66,-2) -- (axis cs:66,8);


                \coordinate (planecoordone) at (axis cs: -7, \labelheight);
                \coordinate (pyramidcoord) at (axis cs: 11, \labelheight);
                \coordinate (cubecoord) at (axis cs: 33, \labelheight);
                \coordinate (spherecoord) at (axis cs: 55, \labelheight);
                \coordinate (planecoord) at (axis cs: 77, \labelheight);

			\end{axis}

			\node[anchor=center] (planeone) at (planecoordone) {\small \textbf{Plane}\strut};
			\node[anchor=center] (pyramid) at (pyramidcoord) {\small \textbf{Pyramid}\strut};

			\node[anchor=center] (cube) at (cubecoord) {\small \textbf{Cube}\strut};

			\node[anchor=center] (sphere) at (spherecoord) {\small \textbf{Sphere}\strut};

			\node[anchor=center] (plane) at (planecoord) {\small \textbf{Plane}\strut};

			\draw[dashed, line width=\newtasklinewidth, ->] (planeone) -- (pyramid);
			\draw[dashed, line width=\newtasklinewidth, ->] (pyramid) -- (cube);
			\draw[dashed, line width=\newtasklinewidth, ->] (cube) -- (sphere);
			\draw[dashed, line width=\newtasklinewidth, ->] (sphere) -- (plane);

		\end{tikzpicture}
        \caption{Performance of the swarm with respect to the event-trigger.}
        \label{fig:hardware:distrcomp:et}
	\end{subfigure}

    \caption{Hardware experiment results on distributed computation.
    \capt{
    \textbf{(a)} Visualization of flown formations.
    \textbf{(b)} Swarm performance improves with more CUs, illustrating the trade-off between resource efficiency and performance.
    \textbf{(c)} Comparison of event triggers: RR periodically selects all UAVs; DT selects based on proximity to targets; HT combines both. RR performs worst overall, while DT and HT each excel in different scenarios. A video of the experiments is available at \url{http://tiny.cc/DMPCSwarmComputation}.
    }}

	% \caption{Results of our hardware experiments on distributed computation. 
    %     \capt{
    %     \textbf{(a)} Visualization of formations flown.
    %     \textbf{(b)} Increasing the number of CUs (computational units) enhances swarm performance, illustrating the trade-off between resource efficiency and performance.
    %     \textbf{(c)} Comparison of different event triggers: RR (round-robin) periodically selects all UAVs; DT (distance trigger) selects based on distance to targets; HT (hybrid trigger) combines both methods. While RR performs worst in most scenarios, DT and HT each perform best in different situations. A video of the experiments is available at \url{http://tiny.cc/DMPCSwarmComputation}.
    %     }}
	\label{fig:hardware:distrcomp}
\end{figure*}