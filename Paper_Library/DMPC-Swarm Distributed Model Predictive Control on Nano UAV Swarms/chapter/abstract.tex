%!TEX root = ../scifile.tex

% Place your abstract within the special {sciabstract} environment.

\abstract{
	Swarms of unmanned aerial vehicles (UAVs) are increasingly becoming vital to our society, undertaking tasks such as search and rescue, surveillance and delivery. 
	A special variant of Distributed Model Predictive Control (DMPC) has emerged as a promising approach for the safe management of these swarms by combining the scalability of distributed computation with dynamic swarm motion control.
	In this DMPC method, multiple agents solve local optimization problems with coupled anti-collision constraints, periodically exchanging their solutions.
	Despite its potential, existing methodologies using this DMPC variant have yet to be deployed on distributed hardware that fully utilize true distributed computation and wireless communication.
	This is primarily due to the lack of a communication system tailored to meet the unique requirements of mobile swarms and an architecture that supports distributed computation while adhering to the payload constraints of UAVs.
	We present \myswarm{}, a new swarm control methodology that integrates an efficient, stateless low-power wireless communication protocol with a novel DMPC algorithm that provably avoids UAV collisions even under message loss. 
	By utilizing event-triggered and distributed off-board computing, \myswarm{} supports nano UAVs, allowing them to benefit from additional computational resources while retaining scalability and fault tolerance. 
	In a detailed theoretical analysis, we prove that \myswarm{} guarantees collision avoidance under realistic conditions, including communication delays and message loss.
	Finally, we present \myswarm{}'s implementation on a swarm of up to 16 nano-quadcopters, demonstrating the first realization of these DMPC variants with computation distributed on multiple physical devices interconnected by a real wireless mesh networks.
	A video showcasing \myswarm{} is available at \url{http://tiny.cc/DMPCSwarm}.

	% Swarms of unmanned aerial vehicles (UAVs) are increasingly becoming vital to our society, undertaking tasks such as search and rescue, surveillance and delivery. 
	% Distributed Model Predictive Control (DMPC) has emerged as a promising approach for the safe management of these swarms by combining the scalability of distributed computation with dynamic swarm motion control.
	% TODO: A varian of DMPC has emerged as a promising apporach for the safe mangement of these quadcopter swarms. In this DMPC, the optim are coupled....
	% Despite its potential, existing DMPC methodologies have yet to be deployed on distributed hardware that fully utilize true distributed computation and wireless communication. 
	% This is primarily due to the lack of a communication system tailored to meet the unique requirements of mobile swarms and an architecture that supports distributed computation while adhering to the payload constraints of UAVs.
	% We present \myswarm{}, a new DMPC methodology that integrates an efficient, stateless low-power wireless communication protocol with a novel DMPC algorithm that provably avoids UAV collisions even under message loss. 
	% By utilizing event-triggered and distributed off-board computing, \myswarm{} supports nano UAVs, allowing them to benefit from additional computational resources while retaining scalability and fault tolerance. 
	% In a detailed theoretical analysis, we prove that \myswarm{} guarantees collision avoidance under realistic conditions, including communication delays and message loss.
	% Finally, we present \myswarm{}'s implementation on a swarm of up to 16 nano-quadcopters, demonstrating the first realization of DMPC with computation distributed on multiple physical devices interconnected by a real real wireless mesh networks.
	% A video showcasing \myswarm{} is available at \url{http://tiny.cc/DMPCSwarm}.


	% Swarms of unmanned aerial vehicles (UAVs) are expected to become an integral part of daily life, performing tasks such as search and rescue, surveillance, and delivery. 
	% To safely control these swarms, Distributed Model Predictive Control (DMPC) has emerged as a promising approach, combining the scalability of distributed control methods with the advantages of dynamic optimization-based control.
	% However, existing DMPC approaches have not been implemented on distributed hardware including true distributed computation and wireless communication, primarily due to challenges in communication and computation.
	% In this paper, we present \myswarm{}, a new DMPC approach that combines a new DMPC algorithm with efficient, stateless low-power wireless communication. 
	% This combination allows us to achieve the first distributed implementation of DMPC for swarm control over wireless mesh networks. 
	% \myswarm{} integrates recent efficient low-power wireless communication protocols with a novel DMPC algorithm that provably avoids collisions of UAVs even under communication message loss. 
	% We combine our approach with event-triggered, distributed off-board computing to support nano-quadcopter swarms. 
	% This allows lightweight nano-quadcopters to benefit from additional computational resources such as from multiple ground nodes, while retaining the scalability and fault tolerance of distributed systems.
	% In a detailed theoretical analysis, we prove that \myswarm{} can guarantee collision avoidance under realistic conditions, including communication delays and message loss. 
	% We realize \myswarm{} on a swarm consisting of up to 16 nano-quadcopters. 
	% Our experimental results are the first to realize DMPC in a distributed fashion on physical devices and real wireless mesh networks our experimental results validate the theoretical guarantee 
	% A video of our experiments can be found in \url{http://tiny.cc/SafeSwarmMovie1}.
	%This work bridges the gap between theoretical DMPC methods and practical swarm control applications, providing a robust solution for safe UAV swarm operation in real-world environments.
	% Swarms of unmanned aerial vehicles (UAVs) will soon become a feature of daily life, performing tasks like search and rescue, surveillance and delivery. Ensuring these swarms are both reliable and safe is crucial, especially during deployment in sensitive areas like cities or disaster sites. We thus present SafeSwarm, a novel UAV swarm architecture that combines safety guarantees, distributed control and resource efficiency in one hardware implementation.	
	% SafeSwarm guarantees that UAVs avoid colliding with each other while flying to their destinations, despite possible communication imperfections such as delay and message loss, validated both via a theoretical analysis and hardware experiments on up to 16 quadcopters.
	% To achieve this guaranteed collision avoidance, SafeSwarm combines Distributed Model Predictive Control (DMPC) with distributed event-triggered ground-based computing and Mixer, a recently developed mesh network communication protocol based on synchronous transmissions. 
	% Our architecture builds on Mixer's ability to abstract the underlying network topology into a bus-like structure, even amid rapid device movement.
	% Leveraging the ability of the protocol, our novel DMPC enables the UAVs to reliably avoid collisions, explicitly accounting for delay and message loss of wireless communication.
	% Further, SafeSwarm achieves resource efficiency by offloading intensive computations to ground-based systems, thereby reducing UAV weight, and by employing event-triggering to minimize ground-based computing power and communication bandwidth without compromising performance.
	% SafeSwarm addresses critical challenges in UAV swarm robotics, providing a resource-efficient, scalable, reliable, and safe solution.
}