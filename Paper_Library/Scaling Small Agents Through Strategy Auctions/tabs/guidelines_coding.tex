\textcolor{darkgray}{\small{\textit{Read these instructions very carefully. Only after you have understood them well, navigate to the tasks in the next tab.\\
\\
\textbfrm{Goal}\\
\\
The goal of this annotation exercise is to label how much time it will take a human (not an LLM) to solve a given coding question.\\
\\
You will be provided with coding questions and you will need to solve each by writing code.\\
\\
You will need to use a stopwatch to measure your task completion time.\\
\\
Task completion time must be reported in the format <HH hours MM minutes SS seconds>, for example, 25 seconds would be written as <00 hours 00 minutes 25 seconds> . DO NOT REPORT MILLISECONDS, EVEN IF YOUR STOPWATCH SHOWS THEM.\\
\\
BE FAST: We are trying to measure a human’s *BEST* completion time, so please complete the task (correctly) as quickly as you can. You are allowed to use web search to look up syntax, however please do not overuse web search unnecessarily, as it tends to increase the completion time.\\ 
\\
If the question requires writing code, you MUST use a Python shell which allows running code at the click of a button. For example, use Google Colab or https://pythonhow.com/python-shell . For code-writing questions, you will be provided with one single test to check your code. We will run your code on more tests later to validate its correctness.\\ 
\\
It is assumed that the Python shell and the search engine are already open in a window. To avoid wasting time unnecessarily, please arrange the windows on your screen so that you can see both the question text, the coding editor and the search engine side by side at the same time.\\
\\
Solve the task by following these steps:\\
\\
- Step 1: Read the question first, slowly and carefully.\\
- Step 1a: If the question requires writing a Python function, copy the function header and, at the bottom, the given test into your Python shell *BEFORE* you start the stop watch. The required function name and arguments will be clear from the test.\\
\hspace*{20pt}For example you may have:} \texttt{def my\_function(my\_arg):}\\
\\
\hspace*{145pt}\texttt{assert my\_function(3)==True}\\
\textit{\hspace*{20pt}So that when the stopwatch starts you will only need to write the function body.\\
- Step 2: Start the stopwatch.\\
- Step 3a: If the question is multiple choice, stop the stopwatch as soon as the correct answer has been identified (no need to type it anywhere) and record the completion time.\\ 
- Step 3b: If the answer requires writing code, stop the stopwatch as soon as you have completed and run the code, and record the completion time.\\
- Step 4: Provide the answer and the task completion time (as per the stopwatch).\\ 
\\
Note: You are allowed to use Google, but not allowed to use AI Assistants.\\
\\
\textbfrm{Examples:}\\
\\
Question:\\
Which of the following lines of code is the correct way to raise a to the power of b in python? Give only the number corresponding to the answer, and nothing else.}\\
\\
1: \texttt{a\textasciicircum b} \\
2: \texttt{a**b}\\
\\
\textit{Completion Time: 00 hours 00 minutes 02 seconds\\
Your Answer: 2\\
\\
—\\
\\
Question:\\
Write a python function to find the first even number in a given list of numbers.\\
\\
Your function should satisfy the following test:}\\
\texttt{assert first\_even ([1, 3, 5, 7, 4, 1, 6, 8]) == 4}\\
\\
\textit{Completion Time: 00 hours 00 minutes 39 seconds\\
Your Answer:}\\
\texttt{def first\_even(nums):\\
\hspace*{20pt}first\_even = next((el for el in nums if el\%2==0), -1)\\
\hspace*{20pt}return first\_even}}}