\vspace{10pt}
\begin{figure}[H]
  \centering
  \begin{tcolorbox}[promptbox]
    \textbf{\large{Judge Prompt}}\\[0.4em]
    \ttfamily
    \\
    \\
    Provide an integer reward score between 0 and 5 (inclusive) for the quality of the provided plan steps, using strict evaluation standards. Ensure the reward reflects how effectively the plan contributes to progressing toward the correct solution.\\
\\
\\
Problem Statement:\\
***begin problem statement***\\
\{task\}\\
***end problem statement***\\
\\
Plan:\\
\{plan\}\\
\\
\\
Be harsh in your evaluation. Only plans that you are extremely confident will succeed should be assigned the maximum score.\\
\\
\\
Score: $[$Strictly provide an integer reward score between 0 and 5$]$\\
  \end{tcolorbox}
  \label{fig:judge-prompt}
\end{figure}

%%%%%%%%%%%%%%%%%%%%%%%%%%%%%%%%%%%%%%%%%%%%%%%%%%%%%%%%%%%%%%%%%%%%%%

\vspace*{64pt}
\begin{figure}[H]
  \centering
  \begin{tcolorbox}[promptbox]
    \textbf{\large{Strategy Prompt (Deep Search)}}\\[0.4em]
    \ttfamily
    \\
    \\
    You are a world expert at making efficient plans to solve any task using a set of carefully crafted tools.\\
\\
Now for the given task, develop a step-by-step high-level plan taking into account the following inputs and list of facts.\\
This plan should involve individual tasks based on the available tools, that if executed correctly will yield the correct answer.\\
Do not skip steps, do not add any superfluous steps. Only write the high-level plan, DO NOT DETAIL INDIVIDUAL TOOL CALLS.\\
After writing the final step of the plan, write the '<end\_plan>' tag and stop there.\\
Always search for the exact task at the beginning. If you are given an external file, always inspect it first to explore its content.\\
Do a very concise plan that only focus on the given task.\\
Do not attempt to answer the question without calling tools, even if you know the answer. You must always use at least one tool to find the answer.\\
\\
\\
\\
Here is your task:\\
\\
Task:\\

\textasciigrave\textasciigrave\textasciigrave\\
\{task\}\\
\textasciigrave\textasciigrave\textasciigrave\\
\\
Your plan can leverage any of these tools:\\
\{tool\_descriptions\}\\
\\
List of facts that you know:\\
\textasciigrave\textasciigrave\textasciigrave\\
\{answer\_facts\}\\
\textasciigrave\textasciigrave\textasciigrave\\
\\
Now begin! Write your plan below.\\
  \end{tcolorbox}
  \label{fig:strategy-prompt-search}
\end{figure}

%%%%%%%%%%%%%%%%%%%%%%%%%%%%%%%%%%%%%%%%%%%%%%%%%%%%%%%%%%%%%%%%%%%%%%

\vspace*{100pt}
\begin{figure}[H]
  \centering
  \begin{tcolorbox}[promptbox]
    \textbf{\large{Strategy Prompt (Coding)}}\\[0.4em]
    \ttfamily
    \\
    \\
    You are a world expert at making efficient plans to solve any task using a set of carefully crafted tools.\\
\\
Now for the given task, develop a step-by-step high-level plan taking into account the following inputs.\\
This plan should involve individual tasks based on the available tools, that if executed correctly will yield the correct answer.\\
Do not skip steps, do not add any superfluous steps. Only write the high-level plan, DO NOT DETAIL INDIVIDUAL TOOL CALLS.\\
After writing the final step of the plan, write the '<end\_plan>' tag and stop there.\\
Do a very concise plan that only focus on the given task.\\
Do not attempt to answer the question without calling tools, even if you know the answer. You must always use at least one tool to find the answer.\\
\\
\\
\\
Here is your task:\\
\\
Task:\\

\textasciigrave\textasciigrave\textasciigrave\\
\{task\}\\
\textasciigrave\textasciigrave\textasciigrave\\
\\
Your plan can leverage any of these tools:\\
\{tool\_descriptions\}\\
\\
Now begin! Write your plan below.\\
  \end{tcolorbox}
  \label{fig:strategy-prompt-code}
\end{figure}

%%%%%%%%%%%%%%%%%%%%%%%%%%%%%%%%%%%%%%%%%%%%%%%%%%%%%%%%%%%%%%%%%%%%%%

\vspace*{10pt}
\begin{figure}[H]
  \centering
  \begin{tcolorbox}[promptbox]
    \textbf{\large{Strategy Refinement Prompt (Deep Search)}}\\[0.4em]
    \ttfamily
    \\
    \\
    You are a world expert at making efficient plans to solve any task using a set of carefully crafted tools.\\
\\
Now for the given task, develop a step-by-step high-level plan taking into account the following inputs and list of facts.\\
This plan should involve individual tasks based on the available tools, that if executed correctly will yield the correct answer.\\
Do not skip steps, do not add any superfluous steps. Only write the high-level plan, DO NOT DETAIL INDIVIDUAL TOOL CALLS.\\
After writing the final step of the plan, write the '<end\_plan>' tag and stop there.\\
Always search for the exact task at the beginning. If you are given an external file, always inspect it first to explore its content.\\
Do a very concise plan that only focus on the given task.\\
Do not attempt to answer the question without calling tools, even if you know the answer. You must always use at least one tool to find the answer.\\
\\
\\
\\
Here is your task:\\
\\
Task:\\

\textasciigrave\textasciigrave\textasciigrave\\
\{task\}\\
\textasciigrave\textasciigrave\textasciigrave\\
\\
Your plan can leverage any of these tools:\\
\{tool\_descriptions\}\\
\\
List of facts that you know:\\
\textasciigrave\textasciigrave\textasciigrave\\
\{answer\_facts\}\\
\textasciigrave\textasciigrave\textasciigrave\\
\\
Below you will find some example tasks followed by two corresponding plans - one plan that lost in a previous plan competition and one that won. Use these examples to understand what makes a plan lose or win.\\
\\
\{retrieved\_tasks\_and\_plans\}\\
\\
Now apply what you have learned and given the task and a corresponding losing plan, write a winning plan.\\
\\
\{previous\_losing\_plan\}\\
Winning plan:\\
  \end{tcolorbox}
  \label{fig:refine-prompt-search}
\end{figure}

%%%%%%%%%%%%%%%%%%%%%%%%%%%%%%%%%%%%%%%%%%%%%%%%%%%%%%%%%%%%%%%%%%%%%%

\vspace*{50pt}
\begin{figure}[H]
  \centering
  \begin{tcolorbox}[promptbox]
    \textbf{\large{Strategy Refinement Prompt (Coding)}}\\[0.4em]
    \ttfamily
    \\
    \\
    You are a world expert at making efficient plans to solve any task using a set of carefully crafted tools.\\
\\
Now for the given task, develop a step-by-step high-level plan taking into account the following inputs.\\
This plan should involve individual tasks based on the available tools, that if executed correctly will yield the correct answer.\\
Do not skip steps, do not add any superfluous steps. Only write the high-level plan, DO NOT DETAIL INDIVIDUAL TOOL CALLS.\\
After writing the final step of the plan, write the '<end\_plan>' tag and stop there.\\
Do a very concise plan that only focus on the given task.\\
Do not attempt to answer the question without calling tools, even if you know the answer. You must always use at least one tool to find the answer.\\
\\
\\
\\
Here is your task:\\
\\
Task:\\

\textasciigrave\textasciigrave\textasciigrave\\
\{task\}\\
\textasciigrave\textasciigrave\textasciigrave\\
\\
Your plan can leverage any of these tools:\\
\{tool\_descriptions\}\\
\\
Below you will find some example tasks followed by two corresponding plans - one plan that lost in a previous plan competition and one that won. Use these examples to understand what makes a plan lose or win.\\
\\
\{retrieved\_tasks\_and\_plans\}\\
\\
Now apply what you have learned and given the task and a corresponding losing plan, write a winning plan.\\
\\
\{previous\_losing\_plan\}\\\\
Winning plan:\\
  \end{tcolorbox}
  \label{fig:refine-prompt-code}
\end{figure}

%%%%%%%%%%%%%%%%%%%%%%%%%%%%%%%%%%%%%%%%%%%%%%%%%%%%%%%%%%%%%%%%%%%%%%