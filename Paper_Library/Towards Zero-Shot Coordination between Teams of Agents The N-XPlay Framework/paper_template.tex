\documentclass[conference]{IEEEtran}
\usepackage{times}

% numbers option provides compact numerical references in the text. 
\usepackage[numbers]{natbib}
\usepackage{multicol}
\usepackage[bookmarks=true]{hyperref}

\usepackage{subcaption}
\usepackage{float}
\usepackage{amsmath}
\usepackage{amssymb}
\usepackage{xcolor}
\usepackage{graphicx}
% \usepackage[inline]{enumitem}
\usepackage[capitalize]{cleveref}

\pdfinfo{
   /Author (Homer Simpson)
   /Title  (Robots: Our new overlords)
   /CreationDate (D:20101201120000)
   /Subject (Robots)
   /Keywords (Robots;Overlords)
}

% \newcommand{\ava}[1]{{\color{purple}Ava:#1} }

\begin{document}

% paper title
\title{
Towards Zero-Shot Coordination between Teams of Agents: The N-XPlay Framework
% Template paper for the \\Robotics: Science and Systems Conference
 }

% You will get a Paper-ID when submitting a pdf file to the conference system
\author{
Ava Abderezaei*, Chi-Hui Lin*, Joseph Miceli, Naren Sivagnanadasan, \\
Stéphane Aroca-Ouellette, Jake Brawer, Alessandro Roncone
}


\maketitle

\begin{abstract}
Zero-shot coordination (ZSC)—the ability to collaborate with unfamiliar partners—is essential to making autonomous agents effective teammates. Existing ZSC methods evaluate coordination capabilities between two agents who have not previously interacted. However, these scenarios do not reflect the complexity of real-world multi-agent systems, where coordination often involves a hierarchy of sub-groups and interactions between teams of agents, known as Multi-Team Systems (MTS). To address this gap, we first introduce N-player Overcooked, an N-agent extension of the popular two-agent ZSC benchmark, enabling evaluation of ZSC in N-agent scenarios. We then propose N-XPlay for ZSC in N-agent, multi-team settings. Comparison against Self-Play across two-, three- and five-player Overcooked scenarios, where agents are split between an ``ego-team'' and a group of unseen collaborators shows that agents trained with N-XPlay are better able to simultaneously balance ``intra-team'' and ``inter-team'' coordination than agents trained with SP.
\end{abstract}

\IEEEpeerreviewmaketitle

\section{Introduction}
Zero-Shot Coordination (ZSC) is an open-problem in multi-agent systems, which challenges agents to efficiently and robustly collaborate with previously unseen teammates \cite{peterstone}. Existing work on ZSC mostly centers on systems of two agents. One agent---often referred to as the ego-agent---will be trained and evaluated on its ability to coordinate with unseen partners, such as a human collaborator, in two-agent teams  \cite{gamma,pecan,fcp}. However, many real-world scenarios involve coordination of many agents across a hierarchy of sub-groups \cite{lin2024}. Agents may have advanced capabilities within particular local teams, yet they must be able to work with members of other groups to accomplish organization-level goals. Humans manage these different relationships---namely, intra-team and inter-team---in multi-team systems (MTS) \cite{teamsofteams3}, and are able to decompose tasks within a sub-group and across the hierarchy of sub-groups. This MTS formulation of the ZSC problem---where agents must balance coordination within an `ego-team' while still effectively collaborating with a group of previously unseen teammates---is underexplored. The ability to accommodate multiple forms of coordination or generally multiple strategies simultaneously has implications beyond MTS, and will be a necessary feature of agents deployed in large scale, heterogeneous organizations such as Human-AI teams. 

To lay the groundwork for studying ZSC in more complex organizational structures, we extend the the popular ZSC environment Overcooked \cite{carroll2020utilitylearninghumanshumanai} from a two-player game to an N-player game. Overcooked is a timed collaborative game where agents must coordinate to prepare and deliver meals. In expanding Overcooked to N-players, complexities not seen in dyadic systems begin to appear such as the structure and composition of teams. These complexities exacerbate the ZSC problem, creating a gap between current techniques and their applicability to settings with more than two agents. To address this gap, we propose \textsl{N-XPlay}, a framework to train agents for use in multi-team systems. This approach is motivated by the ``Teams of Teams'' (\cite{teamsofteams1}) concept, where disparate groups must operate effectively both within and across teams. N-XPlay (\cref{fig:n-2}) naturally extends the existing body of work in ZSC from two-agent to N-agent settings to learn a policy that can both collaborate with replicas of itself and unseen collaborators simultaneously. In training, a subset of $N - X$ agents sharing a policy (similar to Self-Play; SP) are paired with $X$ agents independently sampled from a diverse population of pre-trained policies, similar to existing Population-Based (PB) methods \cite{pecan, gamma, fcp}. This combination of SP and PB enables learning both intra-team coordination within the ``ego-team'' and inter-team coordination with an unseen group of agents.

\begin{figure}
    \centering
    \includegraphics[width=0.75\linewidth]{figs/n-xp_nolayout.png}
    \caption{\textbf{N-XPlay}: The ego-team (blue) consists of $N-X$ agents using identical policies, aiming to maximize intra-team performance while coordinating with $X$ collaborators. During training, collaborators are sampled from a population-based method such as \cite{fcp,pecan, trajedi} to enable inter-team collaboration.}
    % \caption{N-XPlay: The ego-team (blue) is composed of $N-X$ agents that use identical policies. These agents are challenged to find a policy which both maximizes their own intra-team performance while still effectively coordinating with a group of $X$ collaborators. During training, the collaborator group is formed by sampling $X$ agents independently from a population based method such as \cite{fcp,pecan}.}
    \label{fig:n-2}
    \vspace{-15pt}
\end{figure}

% As team size increases, it substantially complicates the ZSC problem.
% Extending ZSC to larger teams substantially increases the complexity of the problem. Notably, in two-agent settings, there is only one possible team composition from the perspective of the ego-agent: the ego-agent itself and one unseen teammate. In contrast, in teams of three or more agents, a wider variety of team compositions emerge; the ego-agent may be familiar with a subset of teammates while remaining unfamiliar with others. These complex team dynamics exacerbate the ZSC problem and create a gap between current ZSC techniques and their applicability to settings with more than two agents. 


We analyze N-XPlay on our N-player Overcooked environment in two-, three-, and five-player settings. We compare the performance of ego-teams comprising of agents trained using N-XPlay against ego-teams composed of agents trained using N-agent SP when paired with varying numbers of unseen collaborators. In summary, our contributions are twofold: (1) we extend the Overcooked environment popularly used for two-agent ZSC tasks, to the N-agent setting \footnote{Open-source code available at: \href{https://github.com/HIRO-group/multiHRI}{https://github.com/HIRO-group/multiHRI}.} and, (2) we propose N-XPlay, a method that extends existing two-agent ZSC approaches to N-agent multi-team settings.
\begin{figure}
    \centering
    \includegraphics[width=\linewidth]{figs/layout_specific_eval_2.png}
    \caption{\textbf{N-1Play vs SP on two-agent layouts:} With no unseen teammate, SP outperforms N-1Play. With 1 unseen, SP degrades, and N-1Play outperforms it on two layouts.}% Coordination Ring and Cramped Room.}
    \label{fig:2-results}
    \vspace{-10pt}
\end{figure}
\section{Method and Experiments}
\textbf{\textsl{N-XPlay Framework.}}
N-XPlay extends two-agent ZSC methods to N-agent settings, addressing challenges presented in complex organizational structures. It models these organization compositions using a ``Teams of Teams'' (\cite{teamsofteams1}) approach, creating agents that coordinate effectively with each other while capable of collaborating with unknown agents. To do so, it first creates a population of diverse reinforcement learning agents using existing population-based methods such as FCP \cite{fcp} and MEP \cite{zhao2021maximum}. In training the ego-team policy for an $N$-agent setting, for each episode, a subset of agents sized $N - X$ share the same policy similar to SP, while paired with $X$ agents independently sampled from the population. As an example in \cref{fig:n-2}, in a environment of $N = 5$ agents where $X = 2$, during training 3 teammates will share and learn the same policy and 2 will be sampled from the pre-trained population.


\textbf{\textsl{Experiments.}}
With our experiments, we wished to explore the impact of the N-XPlay training regime on team performance across various team compositions and sizes. To do this, we build on top of \cite{stephane, stephane2}'s implementation and generate an unseen collaborator population similar to FCP \cite{fcp} where we train four SP agents and store three checkpoints during their training and categorize them into---high, medium, and low performance---based on their average rewards. N-XPlay policies are then trained using the method described above. We then evaluate collective performance in Overcooked where some proportion of the agents are N-XPlay or agents trained with N-player SP and the remaining agents are drawn from a previously unseen population.
% and grouped into high, medium, and low performers. Each model is evaluated with all three groups, and reported results are averaged across them.
For a thorough analysis, we evaluate N-XPlay across various organization sizes and team compositions:
(i) Two-player: N-1Play, with one agent trained alongside a sampled teammate. 
(ii) Three-player: N-1Play (two trained, one sampled) and N-2Play (one trained, two sampled).
(iii) Five-player: N-1, N-3, and N-4Play, training four, two, and one agent(s), with the rest sampled from the population. \cref{fig:2-results}, \cref{fig:3-results}, and \cref{fig:5-results} compare the performance of SP and versions of N-XPlay across these layouts when paired with varying numbers of unseen teammates.
\section{Discussion and Conclusion}
\label{sec:conclusion}
To examine the effect of organization composition, \cref{fig:ratio-unseen-over-teamsize} plots the performance of SP and the best-performing variant of N-XPlay (referred to as N-XPlay in the rest of this section), averaged across layouts, against the unseen-to-total agent ratio. As this ratio increases i.e., less of the organization is represented by the ego-team, N-XPlay begins to outperform SP. For instance, N-XPlay outperforms SP with one unseen teammate in two-agent teams ($1/2$), but not in five-agent teams ($1/5$). In the latter, N-XPlay only overtakes SP when at least three teammates are unseen ($3/5$). We also see some layouts, like No Counter Space (\cref{fig:5-results}), require little coordination and allow agents to work independently, allowing SP to excel regardless of unseen teammates.
%We also see that each N-XPlay variant tends to perform best when evaluated with the same number of unseen teammates it was trained with. For instance, in \cref{fig:3-results}, N-2Play outperforms N-1Play and SP when paired with two unseen teammates.
\begin{figure}
    \centering
    \includegraphics[width=0.75\linewidth]{figs/layout_specific_eval_3.png}
    \caption{\textbf{N-XPlay vs SP on three-agent layouts}: As the number of unseen teammates increases, performances shift: N-1Play is best with one unseen teammate, N-2Play is best with two.}
    \label{fig:3-results}
    % \vspace{-10pt}
\end{figure}
Our results reveal two key conclusions to be expanded on in future work: (1) As the ratio of unseen over collective size increases (i.e. the ego-team represents a smaller portion of the collective), the performance of both SP and N-XPlay declines with N-XPlay starting to outperform SP in layouts requiring coordination among the collective. (2) Focusing solely on maximizing intra-team performance, as in the case of SP, increases the collective’s dependence on the ego-team's ability to independently perform the org-level task. When the ego-team cannot accomplish the overall task alone, the absence of inter-team coordination leads to substantial performance loss. Therefore, the ability to coordinate both within and across teams is critical in complex MTS settings.
%(1) The empowerment of the ego-team directly impacts the collective performance in MTS. As empowerment decreases (correlated with the proportion of ego-agents), the collective performance of both SP and N-XPlay decline. The finding that low-coordination layouts see more sustained performance as ego-team proportion decreases supports this conclusion. 
% (2) Solely maximizing intra-team performance in MTS leads collective reward to be more strictly dependent on the empowerment of the ego-team. The ability to simultaneously perform intra-team and inter-team coordination is critical to minimizing lost potential collective performance, in cases where the ego-team cannot independently accomplish the goal.
% Certain layouts allow agents to work independently, without needing to coordinate or avoid collisions with teammates. In these cases, SP agents perform well because they do not need to adapt to their partners. An example is the No Counter Space layout in \cref{fig:5-layouts}, where we observe that SP consistently outperforms N-XPlay regardless of the number of unseen teammates.   
% Removed due to space:
% To provide quantitative results, we focus on the crossover points in \cref{fig:ratio-unseen-over-teamsize}---where N-XPlay begins to outperform SP---as these indicate when team compositions become too complex for SP to handle effectively, allowing N-XPlay to demonstrate its advantage. In two-agent teams, N-XPlay performs \(27.5\%\) worse than SP when paired with a copy of itself (i.e., 0 unseen teammates), but \(5.4\%\) better when paired with 1 unseen teammate. In three-agent teams, N-XPlay goes from \(6.3\%\) worse with 0 unseen teammates to \(25.5\%\) better with 1 unseen teammate. In five-agent teams, N-XPlay improves from \(10.4\%\) worse with 2 unseen teammates to \(3.2\%\) better with 3 unseen teammates. Interestingly, the relative performance shift is greatest in the three-agent setting ($\sim$500\%), compared to 119\% and 131\% in the two- and five-agent settings, respectively. We are interested in exploring these differences further in future work to better understand where N-XPlay shines and how various factors impact its effectiveness.
\begin{figure}
    \centering
    \includegraphics[width=\linewidth]{figs/layout_specific_eval_5.png}
    \caption{\textbf{N-XPlay vs SP on five-agent layouts:} Layout configuration significantly impacts the evaluation. N-XPlay lags behind SP in low-coordination layouts like No Counter Space but outperforms SP in high-coordination settings.}
    \label{fig:5-results}
    \vspace{-15pt}
\end{figure}

\begin{figure}
    \vspace{-15pt}
    \centering
\includegraphics[width=0.8\linewidth]{figs/unseen_vs_performance_comparison.png}
    \caption{
    \textbf{Best performing variant of N-XPlay vs SP:} As the ratio of unseen teammates (X/N) increases, SP and N-XPlay performance decline with N-XPlay ultimately surpassing SP.}
    \label{fig:ratio-unseen-over-teamsize}
    \vspace{-15pt}
\end{figure}
To conclude, here we introduced N-player Overcooked---an extension of the widely used two-player Overcooked environment---to enable the study of zero-shot coordination (ZSC) in settings with more than two agents. Building on this, we proposed N-XPlay, a framework that generalizes ZSC to N-agent settings by modeling the complex team compositions that arise in larger groups. Our results demonstrate N-XPlay is better suited than SP at finding policies that can simultaneously perform both intra- and inter-team collaboration.
% In this paper, we introduced N-XPlay, a framework that extends zero-shot coordination from two-agent to N-agent settings by modeling the complex team compositions that arise in larger teams.
% Inspired by the "teams of teams" concept, N-XPlay creates teams of agents that can coordinate effectively both within their own group (intra-team) and with unfamiliar agents from other groups (inter-team). 
% To that end, we extend the popular Overcooked environment to support N-agent ZSC problems, and used it to perform extensive evaluation on two-, three-, five- Overcooked scenarios. Our comparison with SP demonstrated that N-XPlay starts to outperform Self-Play (SP) as the number of unseen teammates increases. Specifically, N-XPlay outperforms Self-Play (SP) by 5.4\%, 25.5\%, and 3.2\% when paired with one unseen teammate in two-agent and three-agent settings, and with three unseen teammates in five-agent settings, respectively. For future work, we are interested in further investigating how factors such as layout configuration and the ratio of unseen teammates to team size influence the relative performance of SP and N-XPlay, and why these effects vary across different team sizes. Additionally, while we currently adopt a population generation method similar to Fictitious Co-Play (FCP) \cite{fcp}, we plan to explore alternative population generation strategies to better understand their impact on coordination performance in N-agent settings.


\bibliographystyle{plainnat}
\bibliography{references}

\end{document}


