%!TEX root=./main.tex

\section{Introduction}
\seclabel{intro}


%\begin{itemize}
%	\item Concurrency and its importance + bugs
%	\item Deadlocks - define briefly, lead to resource wastage, violation of fairness. In more practical settings, they lead to poor performance of end user apps (like mobile apps hanging etc). x\% of bugs were reported to be deadlock bugs [Shan Lu's paper] \cite{Lu08}. Maybe introduced when resolving other kinds of bugs like races/atomicity violations
%	\item In this paper: dynamic deadlock detector - what is dynamic deadlock detection
%	\item In previous works - either goodlock style approaches (unsound), or sound (predictive) approaches based on SMT reasoning or  graph based exhaustive backtracking
%	\item In this work we make two contributions - we first outline sources of hardness in dynamic deadlock prediction and then propose an efficient yet precise dynamic deadlock prediction algorithm, and show that it works in practice.
%	\item Contributions: (a) OV and W[1] hardness of checking deadlock patterns, (b) W[1] hardness of 2 thread deadlock prediction \ucomment{We should add this}, (c) notion of syncpreserving deadlock, (d) algorithms - both offline and online + complexity, (e) evaluation
%\end{itemize}

The verification of concurrent programs is a major challenge due to the non-deterministic behavior 
intrinsic to them.
%Unanticipated scheduling patterns lead to concurrency bugs, which are easy to introduce during development but very hard to reproduce during in-house testing, often referred to as \emph{heisenbugs}~\cite{Musuvathi2008}.
Certain scheduling patterns may be unanticipated by the programmers,
which may then lead to introducing concurrency bugs.
Such bugs are easy to introduce during development but can be very hard to
reproduce during in-house testing, and have been notoriously called \emph{heisenbugs}~\cite{Musuvathi2008}.
Among the most notorious concurrency bugs are deadlocks, occurring when the system blocks its execution because each thread is waiting for another thread to finish a task in a circular fashion.
Deadlocks account for a large fraction of concurrency bugs in the wild across various programming languages~\cite{Lu08,Tu2019}
while they are often introduced accidentally when fixing other concurrency bugs~\cite{Yin2011}.

Deadlock-detection techniques can be broadly classified into static and dynamic techniques.
As usual, static techniques analyze source code and have the potential to prove the absence of %deadlocks~\cite{Boyapati2002,Williams2005a,Engler2003,Naik2009,Ng2016,Liu2021}.
deadlocks~\cite{Naik2009,Ng2016,Liu2021}.
However, as static analyses face simultaneously two dimensions of non-determinism, namely in inputs and scheduling, they  lead to poor performance in terms of scalability and false positives,
and are less suitable when the task at hand is to help software developers proactively find bugs.
Dynamic analyses, on the other hand, have the more modest goal of discovering deadlocks by analyzing program executions, allowing for better scalability and few (or no) false positives.
Although dynamic analyses cannot prove the absence of bugs, 
they offer \emph{statistical} and \emph{coverage} guarantees.
These advantages have rendered dynamic techniques a standard practice in 
principled testing for various bugs,
such as data races, atomicity violations, deadlocks, and 
others~\cite{Flanagan09,threadsanitizer,Bensalem2005,Flanagan2008,Mathur2020,Biswas14,Savage97,Pozniansky03}.
A recent trend in this direction advocates for 
\emph{predictive analysis}~\cite{Smaragdakis12,Huang14,Kini2017,Flanagan2008,Huang2018,Kalhauge2018,Genc19},
where the goal is to enhance coverage by additionally reasoning about alternative
%reorderings of the observed execution trace that \emph{could} have taken place manifesting the bug.
reorderings of the observed execution trace that \emph{could} have taken place and also manifest the bug.
% In order to increase coverage, such techniques are often \emph{predictive}, meaning that they try to infer the presence of a bug  even when observing bug-free executions; this is achieved by reasoning about alternative scheduling patterns (i.e., reorderings of the observed trace) 


%\Andreas{Do we need to refer to certain tools by name in this paragraph, e.g. deadlock fuzzer?}
Due to the difficulty of the problem, many dynamic deadlock analyses focus on detecting \emph{deadlock patterns}, broadly defined as cyclic lock-acquisition patterns in the observed execution trace.
One of the earliest works in this direction is the Goodlock algorithm~\cite{Havelund2000}.
As deadlock patterns are necessary but insufficient conditions for the presence of deadlocks, subsequent work has focused on refining this notion in order to reduce false-positives~\cite{Bensalem2005,Agarwal2005}.
Further techniques reduce the size of the lock graph to improve scalability~\cite{Cai2012,Cai2020}.
To further address the unsoundness (false positives) problem, various works propose controlled-scheduling techniques that attempt to realize deadlock warnings via program re-execution~\cite{Bensalem2006,Joshi2009,Samak2014,Samak2014b,Sorrentino2015}
and exhaustive exploration of all reorderings~\cite{Joshi2010,Koushik05}.

Fully sound deadlock prediction has traditionally relied
on explicitly~\cite{Joshi2010,Koushik05} or symbolically (SMT-based)~\cite{Eslamimehr2014,Kalhauge2018} producing all sound witness reorderings.
The heavyweight nature of such techniques
limits their applicability to executions of realistic size, 
which is often in the order of millions of events.
The first steps for sound, polynomial-time deadlock prediction were made recently with 
\seqc~\cite{Cai2021}, an extension of M2~\cite{Pavlogiannis2020} that targets data races.
%\seqc was shown to have the same predictive power as Dirk, but work significantly faster, and even managing to handle input sizes that were far out of reach for Dirk.

This line of work highlights the need for a most-efficient sound deadlock predictor, approaching the golden standard of \emph{linear time}.
Moreover, dynamic analyses are often employed as runtime monitors, and must thus operate \emph{online}, reporting bugs as soon as they occur. Unfortunately, most existing online algorithms only report \emph{deadlock patterns}, 
thus suffering false positives.
%\hcomment{I don't understand this discussion. Here, if we are talking about predictive techniques then there is no such online algorithm. If we are talking about deadlock detectors in general then deadlockfuzzer is online and reports real deadlocks.}
The lack of such a deadlock predictor is even more pronounced when contrasted to 
the problem of dynamic race prediction, which has 
seen a recent surge of sound, online, \emph{linear-time} predictors~(e.g.,~\cite{Kini2017,Roemer20}), and highlights the bigger challenges that deadlocks entail.
We address these challenges in this work, 
by presenting the first high-precision, sound dynamic deadlock-prediction algorithm 
that works online and in linear time.

% The first ingredient towards our algorithm is the 
% observation that the core technical challenge behind
% predicting deadlocks is similar to the one behind predicting data races (also observed in prior works such as~\cite{Kalhauge2018,Cai2021}) --- in both these contexts,
% one attempts to identify whether some well-defined subset of events is \emph{concurrent}, 
% i.e., can be simultaneously executed by the underlying program. 
% Thus, algorithmic advances in predicting data races efficiently
% have the potential of speeding up the task of predicting deadlocks.
% Towards this, we define the class of \emph{synchronization-preserving deadlocks},
% inspired recent advances in data race prediction.
% %inspired from the recent 
% %notion of \emph{synchronization-preservation}~\cite{Mathur2021} 
% %in the context of data races.
% We illustrate synchronization-preserving deadlocks using an example in~\cref{subsec:spd_intro}.
% % and provide the formal definition in~\secref{syncp}.

The task of checking if a potential deadlock is a real predictable
deadlock, in general, involves searching for the reordering of the original execution 
that witnesses the deadlock.
The first ingredient towards our technique is the notion of 
\emph{synchronization-preserving reorderings}~\cite{Mathur2021} that help systematize this search space.
% executions that order two critical sections on the same lock
% in the same order as they appear in the original observed execution.
\emph{Synchronization-preserving deadlocks} are then those 
predictable deadlocks that can be witnessed in some synchronization-preserving reordering.
We illustrate synchronization-preserving deadlocks using an example in~\cref{subsec:spd_intro}.

This notion of synchronization-preservation, by itself, is not sufficient
when it comes to deadlock detection as the prerequisite step towards
predicting deadlocks also involves identifying  \emph{potential deadlock patterns}.
Unlike data races, where \emph{potential races}
can be identified in polynomial-time, the identification of deadlock patterns
is in general, intractable; we prove this in~\secref{lower-bounds}.
As a result, an approach that works by explicitly enumerating
cycles in a \emph{lock graph} and then checking if any of these cycles is realizable 
to a deadlock is likely to be not scalable.
To tackle this, we propose the novel notion of \emph{abstract deadlock patterns}
which, informally, represent clusters of deadlock patterns of the same signature. 
%\hunkar{
Intuitively, a set of deadlock patterns have the same signature 
if the threads and locks that participate in the patterns are the same.
%}
%and check if there is an abstract deadlock pattern that is realizable as a sync-preserving deadlock.
Our next \emph{key observation} is that a single abstract deadlock pattern
can be checked for sync-preserving deadlocks in \emph{linear total time} in the length of the execution, 
\emph{regardless} of how many concrete deadlock patterns it represents.
Our first deadlock prediction algorithm \SyncPDOffline builds upon this --- 
it enumerates all abstract deadlock patterns
in a first phase and then checks their realizability in a second phase,
while running in linear time per abstract deadlock pattern.
Since the number of abstract deadlock patterns
is typically \emph{far smaller} than the number of (concrete)
deadlock patterns (see \cref{tab:expr-results} in \secref{experiments}), this approach achieves high scalability.
Our second algorithm \SyncPDOnline works in a single streaming pass --- it computes abstract deadlock patterns
that involve only two threads and checks their realizability \emph{on-the-fly} simultaneously in overall linear time in the length of the execution.
% \SyncPDOffline, on the other hand, explicitly enumerates all abstract deadlock patterns
% in a first phase and then checks their realizability in a second phase,
% while running in linear time per abstract deadlock pattern.

%!TEX root=./main.tex

\subsection{Synchronization-Preserving Deadlocks}
\label{subsec:spd_intro}

% Here we illustrate our new notion of \emph{sync(hronization)-preserving deadlocks}, which forms a subclass of predictable deadlocks.
% Later we provide formal definition, and develop an online, efficient, sound and complete algorithm for predicting sync-preserving deadlocks.

%%!TEX root=../main.tex

\begin{figure}[t]
\centering
\scalebox{0.9}{
\execution{4}{
\figev{1}{$\acq(\LockColorOne{\lk_1})$}
\figev{1}{$\rel(\LockColorOne{\lk_1})$}
\figev{2}{\underline{$\acq(\LockColorTwo{\lk_2})$}}
\figev{2}{$\mathbf{\Bacq(\LockColorThree{\lk_3})}$}
\figev{2}{$\wt(z)$}
\figev{2}{$\rel(\LockColorThree{\lk_3})$}
\figev{2}{$\rel(\LockColorTwo{\lk_2})$}
\figev{4}{$\acq(\LockColorOne{\lk_1})$}
\figev{4}{\underline{$\wt(y)$}}
\figev{4}{$\rd(z)$}
}
\hspace{1cm}
\execution{4}{
	\figevoffset{10}{4}{$\rel(\LockColorOne{\lk_1})$}
	\figevoffset{10}{1}{$\acq(\LockColorThree{\lk_3})$}
	\figevoffset{10}{1}{$\wt(x)$}
	\figevoffset{10}{1}{$\rd(y)$}
	\figevoffset{10}{1}{\underline{$\rel(\LockColorThree{\lk_3})$}}
	\figevoffset{10}{3}{$\acq(\LockColorThree{\lk_3})$}
	\figevoffset{10}{3}{\underline{$\rd(x)$}}
	\figevoffset{10}{3}{$\Bacq(\LockColorTwo{\lk_2})$}
	\figevoffset{10}{3}{$\rel(\LockColorTwo{\lk_2})$}
	\figevoffset{10}{3}{$\rel(\LockColorThree{\lk_3})$}
}

%\begin{tikzpicture}[]
%	\tikzset{rectangle/.append style={draw=black}}
%	\scalebox{0.84}{
%		\def\xstep{2.2}
%		\def\ystep{0.5}
%		
%		\node[rectangle, align=center] (t1) at (-1, -7){
%			$t_2, \LockColorThree{\lk_3}, \{ \LockColorTwo{\lk_2} \}$
%		};
%		\node[rectangle, align=center] (t31) at (-1, -8.7){
%			$t_3, \LockColorTwo{\lk_2}, \{ \LockColorThree{\lk_3} \}$
%		};
%	
%		\draw[->, thick, -Latex, bend right=10] ([xshift=-3mm]t1.south) to ([xshift=-3mm]t31.north);% (t31.north);
%		\draw[->, thick, -Latex, bend right=10] (t31) to (t1);
%		
%		\node[opacity=0] at (0*\xstep, 1){};
%	}
%\end{tikzpicture}
}

% \caption{
% A  trace $\tr$ with a sync-preserving deadlock, stalling $t_2$ on $e_4$ and $t_3$ on $e_{18}$.
% Underlined events indicate the slice of $\tr$ that forms a sync-preserving lower set of $\tr$,
% and serves as a valid witness for the deadlock.
% \figlabel{motivating}
% }
\caption{
A  trace $\tr$ with a sync-preserving deadlock, stalling $t_2$ on $e_4$ and $t_3$ on $e_{18}$.
\hunkar{
Modify the example to have two deadlock patterns where one of the deadlock patterns is be realizable and the other is not, in order to demonstrate that deadlock patterns are an insufficient condition.
}
}
\figlabel{motivating}
\end{figure}
%!TEX root = ../main.tex



\begin{figure}[t]
\begin{subfigure}[t]{0.22\textwidth}
\scalebox{0.855}{
\execution{2}{
\figev{1}{$\acq(\LockColorOne{\lk_1})$}
\figev{1}{$\Bacq(\LockColorTwo{\lk_2})$}
\figev{1}{\underline{$\wt(x)$}}
\figev{1}{$\rel(\LockColorTwo{\lk_2})$}
\figev{1}{$\rel(\LockColorOne{\lk_1})$}
\figev{2}{$\acq(\LockColorTwo{\lk_2})$}
\figev{2}{\underline{$\rd(x)$}}
\figev{2}{$\Bacq(\LockColorOne{\lk_1})$}
\figev{2}{$\rel(\LockColorOne{\lk_1})$}
\figev{2}{$\rel(\LockColorTwo{\lk_2})$}
}
}
\caption{
A trace $\tr_1$ with no\\ predictable deadlock.
} 
\label{fig:motivating-no-dl}
\end{subfigure}
~~\hspace{-0.1cm}
\begin{subfigure}[t]{0.75\textwidth}
\scalebox{0.855}{
\execution{4}{
\figev{1}{$\acq(\LockColorOne{\lk_1})$}
\figev{1}{$\rel(\LockColorOne{\lk_1})$}
\figev{2}{\underline{$\acq(\LockColorTwo{\lk_2})$}}
\figev{2}{$\mathbf{\Bacq(\LockColorThree{\lk_3})}$}
\figev{2}{$\wt(z)$}
\figev{2}{$\rel(\LockColorThree{\lk_3})$}
\figev{2}{$\rel(\LockColorTwo{\lk_2})$}
\figev{4}{$\acq(\LockColorOne{\lk_1})$}
\figev{4}{\underline{$\wt(y)$}}
\figev{4}{$\rd(z)$}
}
\execution{4}{
	\figevoffset{10}{4}{$\rel(\LockColorOne{\lk_1})$}
	\figevoffset{10}{1}{$\acq(\LockColorThree{\lk_3})$}
	\figevoffset{10}{1}{$\wt(x)$}
	\figevoffset{10}{1}{$\rd(y)$}
	\figevoffset{10}{1}{\underline{$\rel(\LockColorThree{\lk_3})$}}
	\figevoffset{10}{3}{$\acq(\LockColorThree{\lk_3})$}
	\figevoffset{10}{3}{\underline{$\rd(x)$}}
	\figevoffset{10}{3}{$\Bacq(\LockColorTwo{\lk_2})$}
	\figevoffset{10}{3}{$\rel(\LockColorTwo{\lk_2})$}
	\figevoffset{10}{3}{$\rel(\LockColorThree{\lk_3})$}
}
}
\caption{
A  trace $\tr_2$ with a sync-preserving deadlock, stalling $t_2$ on $e_4$ and $t_3$ on $e_{18}$.
}
\label{fig:motivating-dl}
\end{subfigure}
\caption{
Traces with no predictable deadlock (\subref{fig:motivating-no-dl}), and with a sync-preserving deadlock (\subref{fig:motivating-dl}).
} 
\label{fig:motivating}
\end{figure}

% \ucomment{Get back.}

%Consider the trace $\tr$ in \figref{motivating} consisting of 
%$20$ events and four threads.
Consider the trace $\tr_1$ in \cref{fig:motivating-no-dl} consisting of 
$10$ events and two threads.
We use $e_i$ to denote the $i$-th event of $\tr_1$.
%The events $e_4$ and $e_{18}$ form a \emph{deadlock pattern}:~
The events $e_2$ and $e_{8}$ form a \emph{deadlock pattern}:~
%each one acquires a lock already held when the other is executed  
%while no common lock protects both.
%they respectively acquire the locks $\LockColorThree{\lk_3}$ and $\LockColorTwo{\lk_2}$ while holding the locks $\LockColorTwo{\lk_2}$ and $\LockColorThree{\lk_3}$,
they respectively acquire the locks $\LockColorTwo{\lk_2}$ and $\LockColorOne{\lk_1}$ while holding the locks $\LockColorOne{\lk_1}$ and $\LockColorTwo{\lk_2}$,
and no common lock protects these operations.

A deadlock pattern is a necessary but insufficient condition for an actual deadlock:~a sound algorithm must examine whether it can be realized to a deadlock via a witness.
A witness is a reordering $\rho$ of (a slice of) $\tr_1$ 
%that is also a valid trace, and such that $e_4$ and $e_{18}$ are locally enabled in their respective threads at the end of $\rho$.
that is also a valid trace, and such that $e_2$ and $e_{8}$ are locally enabled in their respective threads at the end of $\rho$.
%\hunkar{Observe that arriving at such a state where all the events that participate in the deadlock pattern are locally enabled in their respective threads corresponds to detecting an actual deadlock as none of those enabled events can be executed.}
In general, the problem of checking if a deadlock pattern
can be realized is intractable
(\thmref{w1-hardness-pattern}).
% Prior works encode the problem as a SAT or SMT constraint
% and invoke a solver~\cite{Kalhauge2018,Eslamimehr2014},
% but cannot scale to large traces without
% \emph{windowing} where one limits the analysis to fragments of smaller size.
% In doing so, they tend to miss simple deadlocks too.
%In this work we focus on checking whether $\pattern{e_4,e_{18}}$ forms a 
In this work we focus on checking whether a given deadlock pattern forms a
\emph{sync-preserving deadlock}, which is a subclass of the class of
all predictable deadlocks.
%Intuitively, the process goes as follows.

A deadlock pattern is said to be sync-preserving deadlock
if it can be witnessed in a \emph{sync-preserving reordering}.
A reordering $\rho^{\sf SP}$ of a trace $\tr$ is said to be sync-preserving if 
it preserves the control
flow taken by the original observed trace $\tr$, and further
it preserves the mutual order of any two critical sections (on the same lock)
that appear in the reordering $\rho^{\sf SP}$.
Consider, for example, the sequence $\rho_1 = e_1..e_3\,e_6..e_7$ where $e_i..e_j$ denote the contiguous sequence of events
that starts from $e_i$ and ends at $e_j$. 
We call $\rho_1$ a \emph{correct reordering} of $\tr_1$, being a slice of $\tr_1$ closed under the thread order and preserving the writer of each read in $\tr_1$;
the precise definition is presented in \secref{prelim}.
%\hunkar{
In this case, however, $\rho_1$ does not witness the deadlock as the event $e_2$ is not \emph{enabled} in $\rho_1$.
In fact, due to the dependency between the events $e_3$ and $e_7$, there are no correct reorderings of $\tr_1$ which make both $e_2$ and $e_8$ enabled.
This makes the deadlock pattern $\pattern{e_2, e_{8}}$ non-predictable.
%}
Consider now $\tr_2$ in \cref{fig:motivating-dl}, and the sequence $\rho_2 = e_3..e_7\,e_8..e_{11}\,e_1 e_2$.
Observe that $\rho_2$ is also a correct reordering.
However, $\rho_2$ is not sync-preserving as the order of 
the two critical sections
on lock $\lk_1$ in $\rho_2$ is different from their original order in $\tr_2$.
On the other hand, $\rho_3 = e_1 e_2 e_3 e_8 e_9 \,e_{12}..e_{15}\,e_{16} e_{17}$
is a correct reordering that is also sync-preserving --- all pairs of critical 
sections on the same lock appear in the same order in $\rho_3$ as they did in $\tr_2$.
Further, $\rho_3$ also witnesses the deadlock as the events $e_4$ and $e_{18}$
are both \emph{enabled} in $\rho_3$.
This makes the deadlock pattern $\pattern{e_4, e_{18}}$ a sync-preserving deadlock.
%\hunkar{
%We note that in the witness $\rho_2$ the events $e_4$ and $e_{18}$ are enabled but not executed.
%Hence, having executed $e_{12}$ in $\rho_2$, which appears after $e_{4}$ in the original order, does not %lead to switching the order of critical sections on lock $\LockColorThree{\lk_3}$.
%}
\begin{comment}
We will attempt to construct a certain slice of $\tr$, which we call a \emph{sync-preserving closure set} $\mathcal{I}$ for the focal events $e_4$ and $e_{18}$.
Since these events must be  enabled in their respective threads, their immediate local predecessors are taken in $\mathcal{I}$, i.e., $e_3, e_{17}\in \mathcal{I}$.
As $\mathcal{I}$ is a slice of $\tr$, all the indirect local predecessors of all events in $\mathcal{I}$ are also in $\mathcal{I}$, i.e., all events of $t_2$ and $t_3$ up to $e_3$ and $e_{17}$, respectively.
Note that $e_{17}$ is a read event, reading its value from the write event $e_{13}$.
For this reason, we take $e_{13}$, as well as all its local predecessors in $\mathcal{I}$.
At this point, the critical section of $t_1$ on $\lk_3$ opened by $e_{12}$ is open in $\mathcal{I}$,
as the matching release event $e_{15}\not \in \mathcal{I}$.
At the same time, $e_{16}$ opens a critical section also on $\lk_3$, while also $e_{16}\in \mathcal{I}$.
The sync-preservation property requires that, since $e_{16}$ appears \emph{after} $e_{13}$ in $\tr$, we have to include the matching release event $e_{15}$ of $e_{13}$ (as well as its local predecessors) in $\mathcal{I}$.
Note that this brings the read event $e_{14}$ in $\mathcal{I}$.
Similar reasoning as before will bring $e_{8}\in \mathcal{I}$, which is an acquire event on lock $\lk_1$.
At this point we must again decide on whether we will close this critical section, i.e., bring $e_{11}$ in $\mathcal{I}$.
Note that there is another acquire event on lock $\lk_1$ already in $\mathcal{I}$, namely $e_1$.
However $e_1$ appears before $e_{8}$ in $\tr$, sync-preservation allows the critical section on $e_8$ to remain open.
At this point we have completed the computation of $\mathcal{I}$, and as we have $e_4, e_{18}\not \in \mathcal{I}$,
we conclude that these two events indeed constitute a predictable deadlock, witnessed by $\rho=\tr\Project \mathcal{I}$,
i.e., by the same trace $\tr$ when sliced on $\mathcal{I}$.
Note that if we included $e_{10}$ in $\mathcal{I}$ then, due to $e_9$, we would also have to include $e_5$ in $\mathcal{I}$, which would result in $e_4\in \mathcal{I}$, in which case $\tr\Project \mathcal{I}$ would not be a valid witness for the deadlock.
\end{comment}


In this work we show that sync-preserving deadlocks enjoy two remarkable properties.
First, all sync-preserving deadlocks 
of a given \emph{abstract} deadlock pattern can be checked in linear time.
%Second, sync-preservation appears to capture almost all deadlocks that exist in the practice.
Second, our extensive experimental evaluation on standard benchmarks indicates that 
sync-preservation captures 
%almost all the deadlocks that are known to exist in practice.
a vast majority of deadlocks in practice.
In combination, these two benefits suggest that sync-preservation is the 
right notion of deadlocks to be targeted by dynamic deadlock predictors.

\subsection{Contributions of this work}

Our main contribution is \textit{Learning-to-Fly} (L2F)\footnote{Videos of the simulations and demonstrations in this paper can be viewed at \url{https://tinyurl.com/vvvuukh}}, a scheme 
for real-time, on-the-fly collision avoidance between two UAS whose main features are:

\begin{enumerate}
\item \textit{Systematic composition of machine learning and control theory:} We combine learning-based decision-making, and linear programming-based control to solve the problem in a decentralized manner. Unlike many other ad-hoc Machine Learning-based solutions, we provide a sound theoretical justification for our approach in Theorem \ref{th:MILP_CAMPC_relation}. We also provide a sufficient condition for the scheme to work successfully (Theorem \ref{th:CAMPC_success}).

\item  \textit{A notion of priority among the UAS} can be encoded naturally in L2F, where the UAS with higher priority does not have to deviate from its originally planned trajectory until absolutely necessary.

\item \textit{Computationally lightweight enough for real-time implementation:} Experimental results show that L2F, with a computation time in milliseconds can be used in a real-time implementation at a high-rate ($10$ Hz). 

\item \textit{High performance:} In the best case, L2F successfully results in 2-UAS collision avoidance $100\%$ of the test cases, gracefully degrading to $90\%$ for the worst case. Comparisons with other methods also show the superior performance of L2F. 

\item \textit{Enabling fast, independent planning for UAS with temporal logic objectives}, as individual UAS, or fleets of UAS run by the same operator, can plan for themselves without considering other UAS in the airspace while calling upon L2F for on-the-fly collision avoidance. For a 4-UAS case study, we demonstrate a speed up of $3.5\times$ over the centralized planning method of \cite{pant2018fly}.

\item \textit{Proof-of-concept demonstration} on Crazyflie quad-rotor robots to show feasibility on real UAS.

\end{enumerate}

%consisting of:
%\textcolor{green}{Too much repetition, shorten this.
%}
%\begin{enumerate}
%\item \textit{Learning-based conflict resolution:} Given the planned trajectories of two UAS that have a conflict in the near future, the scheme chooses a sequence of maneuvers such that the two UAS avoid colliding with each other without violating their higher level mission objective. We develop a supervised learning-based method for conflict resolution, \textit{\crLSTM} (Section~\ref{sec:learning_supervised}).
%
%\item \textit{CA-MPC: A distributed, cooperative convex Model Predictive 
%Control (MPC) algorithm for collision avoidance:} that takes in the desired 
%maneuvers from the conflict resolution scheme as constraints for each UAS to 
%satisfy such that the resulting trajectories are collision-free and also 
%satisfy the mission objective (Section \ref{sec:CA_MPC}).
%\end{enumerate}
%
%This collision avoidance scheme allows us to build a UAS planning and traffic management framework where we have:
%
%\begin{enumerate}
%\item  \textit{Fast, independent planning for UAS with temporal logic objectives}, as individual UAS, or fleets of UAS run by the same operator, can plan for themselves without having to consider the other UAS in the airspace. 
%\item  \textit{A notion of priority among the UAS}, where the UAS with higher priority does not have to deviate from its originally planned trajectory until absolutely necessary.
%\item  \textit{Real-time collision avoidance} as the CA framework relies on executing one learning based step and two convex optimizations. Experimental evaluations of our implementation show their low computation times.
%\end{enumerate}

%Through extensive simulations, we evaluate the performance of this collision avoidance scheme and show that this framework for planning and collision avoidance allows for faster and more scalable planning for UAS fleets with Signal Temporal Logic specifications than the centralized approach of \cite{pant2018fly}. Finally, through a proof-of-concept implementation on real quadrotor drones, we show this framework can be used in real-world settings. While our method of relying on a learning-based component does not result in collision-free trajectories for the UAS 1\% of the time on average, it outperforms other baseline approaches, and can be used for real-time implementations. \textcolor{green}{Shrink this and make consistent w/ final results}.