%!TEX root = main.tex

\section{Details of Algorithms}

%\subsection{Verifying Abstract Deadlock Patterns}
%\applabel{verify-abstract-patterns}

%Given an abstract deadlock pattern of length $k \geq 2$,
%there can be $O(\NumEvents^k)$ instantiations of it,
%giving a naive enumerate-and-check algorithm running 
%in time $O(\NumThreads\cdot\NumEvents^{k+1})$.
%Instead we exploit monotonicity properties of 
%the sync-preserving closure (\propref{spclosure-monotone})
%and instantiations of an abstract pattern (\corref{pattern-monotone}) 
%that allow for an \emph{incremental}
%algorithm that iteratively checks successive instantiations
%of a given abstract deadlock pattern, while still spending total $O(\NumEvents\cdot \NumThreads)$ time.
%% , i.e., the running time is linear in $\NumEvents$, which is the dominating parameter.
%The first observation allows us to re-use a prior
%computation when checking later deadlock patterns.
%
%% We first observe that the sync-preserving closure is monotonic
%% with respect to the thread-order, and thus, when computing closure
%% of later events, one can avoid re-computation.
%\begin{restatable}{proposition}{spclosure-monotone}
%\proplabel{spclosure-monotone}
%For a trace $\tr$ and sets $S, S' \subseteq \events{\tr}$.
%If for every event $e \in S$, there is an event $e' \in S$
%such that $e \tho{\tr} e'$, then
%$\SPClosure{\tr}(S) \subseteq \SPClosure{\tr}(S')$.
%\end{restatable}
%
%%\begin{proof}
%%Follows from the definition of $\SPClosure{\tr}(S)$.
%%\end{proof}
%
%Next, we extend \propref{spclosure-monotone} to avoid redundant
%computations when a sync-preserving deadlock is not found and later deadlock patterns must be checked.
%Given two deadlock patterns
%$D_1 = e_0, \ldots, e_{k-1}$ and $D_2 = f_0, \ldots, f_{k-1}$
%of the same length $k$, 
%we say $D_1 \prec D_2$ if
%they are instantiations of a common abstract pattern $\abst{D}$
%(i.e., $D_1, D_2 \in \abst{D}$) and 
%for every $i<k$, $e_i \tho{\tr} f_i$.
%\begin{restatable}{corollary}{pattern-monotone}
%\corlabel{pattern-monotone}
%Let $\tr$ be a trace and let $D_1 = e_0, \ldots, e_{k-1}$ and 
%$D_2 = f_0, \ldots f_{k-1}$ be deadlock
%patterns of size $k$ in $\tr$ such that $D_1 \prec D_2$.
%Let $S_1 = \set{e_0, \ldots, e_{k-1}}$ and $S_2 = \set{f_0, \ldots, f_{k-1}}$.
%If $\SPClosure{\tr}(\prev{\tr}(S_1))\cap S_2 \neq \emptyset$,
%then $\SPClosure{\tr}(\prev{\tr}(S_2))\cap S_2 \neq \emptyset$.
%\end{restatable}
%
%The algorithm for checking if an abstract deadlock 
%pattern contains a sync-preserving instantiation
%is presented in~\algoref{abstract-pattern}.
%At a high level, it iterates over the sequences $F_0, \ldots, F_{k-1}$ of 
%acquires (one for each abstract acquire) in trace order.
%For this, the algorithm maintains
% indices $i_0, \ldots, i_{k-1}$ that point to entries in $F_0, \ldots, F_{k-1}$.
%At each step, it determines whether the current
%deadlock pattern $D = e_0, \ldots, e_{k-1}$
%by computing the sync-preserving closure
%of the predecessors of the deadlock pattern, and checks
%if it is disjoint from the deadlock.
%If so, it terminates.
%Otherwise, it looks for the next eligible deadlock,
%which it determines based on \corref{pattern-monotone}.
%In particular, it advances the pointer $i_j$
%all the way until an entry which is outside of the
%closure computed so far.
%Also observe that the timestamp $T$ of the closure
%computed in an iteration is being used in later iterations;
%this is a consequence of \propref{spclosure-monotone}.
%Further, we ensure that the list of acquires $\AcqLst_{t, \lk}$
%used in the function \fixpoint (\algoref{compute-closure})
%be reused across iterations, and not need be
%re-assigned to the original list of all acquire events;
%correctness of this optimization follows from~\propref{spclosure-monotone}.
%Let us now calculate the running time for this algorithm.
%Each of the $\AcqLst_{t, \lk}$ in \fixpoint is traversed
%at most once. 
%Next, each element of the sequences $F_0, \ldots, F_{k-1}$
%is also traversed atmost once.
%For each of these acquires, the algorithm spends $O(\NumThreads)$
%time for vector clock updates.
%The total time required is thus $O(\NumEvents \cdot \NumThreads)$.
%This concludes the proof of \lemref{abstract-pattern-linear-time}.

\subsection{The Online $\SyncPDOffline$ Algorithm}
\applabel{app_online}
%\seclabel{app_online}
\Andreas{We can consider bringing this in the main paper. We should have space now. Some reviewers requested this, and it does make the paper self contained.}
In this section we give a detailed pseudocode description of the online $\SyncPDOffline$ algorithm (\algoref{online}).
%!TEX root = main.tex

\small
\begin{algorithm*}[H]
% \BlankLine
\begin{multicols}{2}
\myfun{\init{}}{
	\ForEach{$t \in \threads{}$ $\cdot$}{
		$\Cc_t$ := $\bot$
	}
	\ForEach{$x \in \vars{}$ $\cdot$}{
		$\LW_x$ := $\bot$
	}
	\ForEach{$\lk \in \locks{}$ $\cdot$}{
		$\gId_\lk$ := $0$
	}
	\ForEach{$t_1 {\neq}t_2 \in \threads{}, \lk_1 {\neq} \lk_2 \in \locks{}$}{
		$\view{\Ii}{}{\tuple{t_1, \lk_1, t_2, \lk_2}}$ := $\bot$ \;
		\ForEach{$t \in \threads{}, \lk \in \locks{}$}{
			$\view{\AcqLst}{t, \lk}{\tuple{t_1, \lk_1, t_2, \lk_2}}$ := $\epsilon$
		}
	}
	\ForEach{$u \in \threads{}$}{
		\ForEach{$t\in \threads{}, \lk_1, \lk_2 \in \locks{}$}{
			$\view{\DPLst}{t, \lk_1, \lk_2}{\tuple{u}}$ := $\epsilon$
		}
	}
}
\BlankLine
% {\tiny \tcp{$Lst$ is a list of triplets $(g, C, C')$ in increasing order of $C$}}
% {\tiny \tcp{Returns the last element $(g_{\max}, C_{\max}, C'_{\max})$ s.t. $C_{\max} \cle U$}}
% {\tiny \tcp{Removes all other elements $(g,C, C')$ from $Lst$ with $C \cle U$}}
\myfun{\maxLB{$U$, $Lst$}}{
	$(g_{\max}, C_{\max}, C'_{\max})$ := $(0, \bot, \bot)$ \; 
	\While{$\NOT\, \isEmpty{Lst}$}{
		$(g, C, C')$ := $\first{Lst}$ \;
		\If{$C \cle U$}{
			$(g_{\max}, C_{\max}, C'_{\max})$ := $(g, C, C')$
		}
		\Else{
			\Break
		}
		$\removeFirst{Lst}$
	}
	\Return $(g_{\max}, C_{\max}, C'_{\max})$
}

\BlankLine

\myfun{\fixpoint{$I$, $\tuple{t_1, \lk_1, t_2, \lk_2}$}}{
	\Repeat{$I$ does not change}{
		\ForEach{$\lk \in \locks{}$}{
			\ForEach{$t \in \threads{}$}{
				($g_{\lk, t}, C_{\lk, t}, C'_{\lk, t}$) := 
				\maxLB{$I$, $\view{\AcqLst}{t, \lk}{\tuple{t_1, \lk_1, t_2, \lk_2}}$} \;
			}
			$t_{\max}$ := argmax$_{t \in \threads{}}\set{g_{\lk, t}}$ \;
			$I$ := $I \mx \bigsqcup_{t \neq t_{\max} \in \threads{}} C'_{\lk, t}$	
		}
	}
	\Return $I$
}

\myfun{\checkDeadlock{$Lst$, $I$, $\tuple{t_1, \lk_1, t_2, \lk_2}$}}{
	\While{$\NOT\, \isEmpty{Lst}$}{
		($C_{\prev{}}, C$) := $\first{Lst}$ \;
		$I$ := \fixpoint{$I \mx C_{\prev{}}$, $\tuple{t_1, \lk_1, t_2, \lk_2}$} \;
		\If{$C \not\cle I$}{ \linelabel{check-containment}
			\declare `Deadlock' \;
			\Break
		}
		$\removeFirst{Lst}$
	}
	\Return $I$
}

\myhandler{\rdhandler{$t$, $x$}}{
	$\Cc_t$ := $\Cc_t \mx \LW_x$ \linelabel{clkrd}
}

\myhandler{\wthandler{$t$, $x$}}{
	$\LW_x$ := $\Cc_t$ \linelabel{clkwt};\\
	$\Cc_t$ := $\Cc_t[t \mapsto \Cc_t(t)+1]$
}

\myhandler{\acqhandler{$t$, $\lk$}}{
	$C_{\prev{}}$ := $\Cc_t$ \;
	$\Cc_t$ := $\Cc_t[t \mapsto \Cc_t(t)+1]$\\ $\gId_\lk$ := $\gId_\lk + 1$ \linelabel{clkacq}\;
	\ForEach{$t_1, t_2 \in \threads{}, \lk_1, \lk_2 \in \locks{}$}{
		$\addLst{\view{\AcqLst}{t, \lk}{\tuple{t_1, \lk_1, t_2, \lk_2}}}{$(\gId_\lk, \Cc_t, \bot)$}$
	}
	\ForEach{$u\in \threads{}$, $\lk' \in \lheld{}$}{
		$\addLst{\view{\DPLst}{t, \lk, \lk'}{\tuple{u}}}{($C_{\prev{}}$, $\Cc_t$)}$
	}
	\ForEach{$u \neq t \in \threads{}$, $\lk' \in \lheld{}$}{
		$I$ := $\view{\Ii}{}{\tuple{u, \lk' t, \lk}} \mx C_{\prev{}}$ \;
		$\view{\Ii}{}{\tuple{u, \lk' t, \lk}}$ := \checkDeadlock{$\view{\DPLst}{u, \lk', \lk\,}{\tuple{t}}$, $I$, $\tuple{u, \lk' t, \lk}$}
	}

}

\myhandler{\relhandler{$t$, $\lk$}}{
	$\Cc_t$ := $\Cc_t[t \mapsto \Cc_t(t)+1]$ \linelabel{clkrel} \;
	\ForEach{$t_1, t_2 \in \threads{}, \lk_1, \lk_2 \in  \locks{}$}{
		$\updateRel{\lst{\view{\AcqLst}{t, \lk}{\tuple{t_1, \lk_1, t_2, \lk_2}}}}{$\Cc_t$}$
	}
}


\end{multicols}
% \vspace*{0.25cm}
\normalsize
\caption{$\SyncPDOnline$.}
\algolabel{online}
\end{algorithm*}
\normalsize