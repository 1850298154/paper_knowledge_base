%!TEX root = ../main.tex
%\begin{figure}
\begin{subfigure}[b]{0.475\textwidth}
	\newcommand{\xdisposition}{0}
	\newcommand{\ydisposition}{0}
	\newcommand{\xtstep}{0.75}
	\newcommand{\ytstep}{1}
	\newcommand{\ybias}{-0.3 }
	\newcommand{\xstep}{2.5}
	\newcommand{\ystep}{-0.475}
	\newcommand{\xtscale}{0.8}
	\def \numevents{9.5}
	\newcommand{\eventA}[4]{
		\node[event, draw=black, fill=white] (A#1) at (#1*\xstep, #2*\ystep) {\footnotesize $#2(x_{#3})$};
		%\node[] at (0*\xstep-\xtstep, {#1*\ystep}) {\small $#2(x_{#3})$};
	}
	\scalebox{0.9}{
		\begin{tikzpicture}[thick,
			pre/.style={<-,shorten >= 2pt, shorten <=2pt, very thick},
			post/.style={->,shorten >= 3pt, shorten <=3pt,   thick},
			seqtrace/.style={line width=2},
			und/.style={very thick, draw=gray},
			event/.style={rectangle, minimum height=0.8mm, minimum width=15mm,  line width=1pt, inner sep=0.5,},
			virt/.style={circle,draw=black!50,fill=black!20, opacity=0}]
			\footnotesize
			
			
%			\node[] at (0.5*\xstep, -1.5*\ystep){
%				\small
%				$\cs(\lk_i,\lk_j)=\acq(\lk_i) \cdot\acq(\lk_j) \cdot\rel(\lk_j) \cdot \rel(\lk_i)$
%			};
			

				
			\begin{scope}[shift={(0,-3*\ystep)}]
				
				\draw[dashed] (-0.9*\xstep,6.5*\ystep) rectangle (-0.5*\xstep,8.5*\ystep);
				\node (A1) at (-0.7*\xstep,7*\ystep) {\normalsize [1, 1]};
				\node (A2) at (-0.7*\xstep,8*\ystep) {\normalsize [1, 0]};
				\node (A) at (-0.7*\xstep,9*\ystep) {\large  $A$};
				
				\draw[dashed] (1.5*\xstep,6.5*\ystep) rectangle (1.9*\xstep,8.5*\ystep);
				\node (B1) at (1.7*\xstep,7*\ystep) {\normalsize [1, 0]};
				\node (B2) at (1.7*\xstep,8*\ystep) {\normalsize [0, 1]};
				\node (B) at (1.7*\xstep,9*\ystep) {\large  $B$};
				
				
			\end{scope}
			

			
			\node[] (S11) at (0*\xstep,0.15) {\normalsize $t_A$};
			\node[] (S12) at (0*\xstep,\numevents * \ystep) {};
			\node[] (S21) at (1*\xstep,0.15) {\normalsize $t_B$};
			\node[] (S22) at (1*\xstep,\numevents * \ystep) {};
			
			\draw[seqtrace] (S11) to (S12);
			\draw[seqtrace] (S21) to (S22);
			
			
			\node[event, draw=black, fill=white] (11) at (0*\xstep, 1*\ystep + 0*\ybias) {$\acq(\LockColorTwo{\lk_2})$};
			\node[event, draw=black, fill=white] (12) at (0*\xstep, 2*\ystep + 0*\ybias) {$\acq(\LockColorOne{\lk_1})$};
			\node[event, draw=black, fill=white, dotted] (13) at (0*\xstep, 3*\ystep + 0*\ybias) {$\cs(m_0,m_1)$};
			%\node[event, draw=black, fill=white, dotted] (14) at (0*\xstep, 4*\ystep + 0*\ybias) {$\acq(\lk')$};
			%\node[event, draw=black, fill=white, dotted] (15) at (0*\xstep, 5*\ystep + 0*\ybias) {$\rel(\lk')$};
			%\node[event, draw=black, fill=white, dotted] (16) at (0*\xstep, 6*\ystep + 0*\ybias) {$\rel(\lk)$};
			\node[event, draw=black, fill=white] (17) at (0*\xstep, 4*\ystep + 0*\ybias) {$\rel(\LockColorOne{\lk_1})$};
			\node[event, draw=black, fill=white] (18) at (0*\xstep, 5*\ystep + 0*\ybias) {$\rel(\LockColorTwo{\lk_2})$};
			
			\node[event, draw=black, fill=white] (19) at (0*\xstep, 6*\ystep + 1*\ybias) {$\acq(\LockColorOne{\lk_1})$};
			\node[event, draw=black, fill=white, dotted] (110) at (0*\xstep, 7*\ystep + 1*\ybias) {$\cs(m_0,m_1)$};
			%\node[event, draw=black, fill=white, dotted] (111) at (0*\xstep, 11*\ystep + 1*\ybias) {$\acq(\lk')$};
			%\node[event, draw=black, fill=white, dotted] (112) at (0*\xstep, 12*\ystep + 1*\ybias) {$\rel(\lk')$};
			%\node[event, draw=black, fill=white, dotted] (113) at (0*\xstep, 13*\ystep + 1*\ybias) {$\rel(\lk)$};
			\node[event, draw=black, fill=white] (114) at (0*\xstep, 8*\ystep + 1*\ybias) {$\rel(\LockColorOne{\lk_1})$};
			
			%%%%%%%%%%%%%%%%
			
			\node[event, draw=black, fill=white] (21) at (1*\xstep, 2*\ystep + 0*\ybias) {$\acq(\LockColorOne{\lk_1})$};
			\node[event, draw=black, fill=white, dotted] (22) at (1*\xstep, 3*\ystep + 0*\ybias) {$\cs(m_1,m_0)$};
			%\node[event, draw=black, fill=white, dotted] (23) at (1*\xstep, 4*\ystep + 0*\ybias) {$\acq(\lk)$};
			%\node[event, draw=black, fill=white, dotted] (24) at (1*\xstep, 5*\ystep + 0*\ybias) {$\rel(\lk)$};
			%\node[event, draw=black, fill=white, dotted] (25) at (1*\xstep, 6*\ystep + 0*\ybias) {$\rel(\lk')$};
			\node[event, draw=black, fill=white] (26) at (1*\xstep, 4*\ystep + 0*\ybias) {$\rel(\LockColorOne{\lk_1})$};
			
			\node[event, draw=black, fill=white] (27) at (1*\xstep, 5*\ystep + 1*\ybias) {$\acq_1(\LockColorTwo{\lk_2})$};
			\node[event, draw=black, fill=white, dotted] (28) at (1*\xstep, 6*\ystep + 1*\ybias) {$\cs(m_1,m)$};
			%\node[event, draw=black, fill=white, dotted] (29) at (1*\xstep, 11*\ystep + 1*\ybias) {$\acq(\lk)$};
			%\node[event, draw=black, fill=white, dotted] (210) at (1*\xstep, 12*\ystep + 1*\ybias) {$\rel(\lk)$};
			%\node[event, draw=black, fill=white, dotted] (211) at (1*\xstep, 13*\ystep + 1*\ybias) {$\rel(\lk')$};
			\node[event, draw=black, fill=white] (212) at (1*\xstep, 7*\ystep + 1*\ybias) {$\rel_1(\LockColorTwo{\lk_2})$};
			
			\begin{scope}[]
				\node[below left=of 212] (212b) {};
			\end{scope}
	
		\end{tikzpicture}
	}
	\caption{
		Reduction for OV-hardness proof from an instance of size $n = 2$ and $d = 2$.
		% We use the shortcut $\cs(\lk_i,\lk_j)$ to denote two nested critical sections on $\lk_i$ and $\lk_j$, as indicated in the figure.
	}
	\figlabel{ov-hardness}
\end{subfigure}
%\end{figure}