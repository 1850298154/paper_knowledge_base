%!TEX root=../main.tex

\setlength{\textfloatsep}{10pt}
\begin{figure}[t]
%\begin{wrapfigure}{r}{0.35\textwidth}
\begin{subfigure}[t]{0.2\textwidth}
\scalebox{0.84}{
\begin{tikzpicture}[yscale=0.9]
	\tikzset{rectangle/.append style={draw=black}}
		\def\xstep{2.2}
		\def\ystep{0.5}
		\node[rectangle, align=center] (t1) at (-1, -7){
			$t_1, \LockColorTwo{\lk_2}, \{ \LockColorOne{\lk_1}\}, \sequence{e_{2}}$
		};
		\node[rectangle, align=center] (t31) at (-1, -8.7){
			$t_2, \LockColorOne{\lk_1}, \{ \LockColorTwo{\lk_2} \}, \sequence{e_{8}}$
		};
		\draw[->, thick, -Latex, bend right=10] ([xshift=-3mm]t1.south) to ([xshift=-3mm]t31.north);% (t31.north);
		\draw[->, thick, -Latex, bend right=10] (t31) to (t1);
\end{tikzpicture}
}
\end{subfigure}
~~~~
\begin{subfigure}[t]{0.2\textwidth}
\scalebox{0.84}{
\begin{tikzpicture}[yscale=0.9]
	\tikzset{rectangle/.append style={draw=black}}
		\def\xstep{2.2}
		\def\ystep{0.5}
		\node[rectangle, align=center] (t1) at (-1, -7){
			$t_2, \LockColorThree{\lk_3}, \{ \LockColorTwo{\lk_2}\}, \sequence{e_{4}}$
		};
		\node[rectangle, align=center] (t31) at (-1, -8.7){
			$t_3, \LockColorTwo{\lk_2}, \{ \LockColorThree{\lk_3} \}, \sequence{e_{18}}$
		};
		\draw[->, thick, -Latex, bend right=10] ([xshift=-3mm]t1.south) to ([xshift=-3mm]t31.north);% (t31.north);
		\draw[->, thick, -Latex, bend right=10] (t31) to (t1);
\end{tikzpicture}
}
\end{subfigure}
~~~~
% \hspace*{\fill}
\begin{subfigure}[t]{0.4\textwidth}
\scalebox{0.84}{
\begin{tikzpicture}[yscale=0.9]
	\tikzset{rectangle/.append style={draw=black}}
		\node[rectangle, align=center] (t1) at (-1, -7){
					$t_1, \LockColorTwo{\lk_2}, \{ \LockColorOne{\lk_1} \}$,
					$\sequence{e_2, e_4, e_{29}}$
				};
				\node[rectangle,align=center] (t2) at (2.5, -7){
					$t_2, \LockColorOne{\lk_1}, \{ \LockColorFour{\lk_4} \}$,
					$\sequence{e_{23}}$
				};
				\node[rectangle, align=center] (t31) at (-1, -8.7){
					$t_3, \LockColorOne{\lk_1}, \{ \LockColorTwo{\lk_2} \}$,
					$\sequence{e_{16}, e_{19}}$
				};
				\node[rectangle,align=center] (t32) at (2.5, -8.7){
					$t_3, \LockColorThree{\lk_3}, \{ \LockColorTwo{\lk_2} \}$,
					$\sequence{e_{13}}$
				};
				\draw[->, thick, -Latex] (t2) -- (t1);
				\draw[->, thick, -Latex] ([xshift=6mm]t1.south) to (t32.west);
				\draw[->, thick, -Latex, bend right=10] ([xshift=-3mm]t1.south) to ([xshift=-3mm]t31.north);
				\draw[->, thick, -Latex, bend right=10] (t31) to (t1);
\end{tikzpicture}
}
\end{subfigure}
% 
% \caption{
% A  trace $\tr$ with a sync-preserving deadlock, stalling $t_2$ on $e_4$ and $t_3$ on $e_{18}$.
% Underlined events indicate the slice of $\tr$ that forms a sync-preserving lower set of $\tr$,
% and serves as a valid witness for the deadlock.
% \figlabel{motivating}
% }
\caption{
Abstract lock graphs of the traces from~\figref{motivating-no-dl} (left), \figref{motivating-dl} (middle) and \figref{syncp_example} (right).
% \Andreas{@Umang: You can use this space for more figures, examples}
\figlabel{motivating-graph}
}
 \end{figure}
%\end{wrapfigure}