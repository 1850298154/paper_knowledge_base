%!TEX root=../main.tex

\begin{figure}[t]
\centering
\scalebox{0.9}{
\execution{3}{
\figev{1}{$\acq(\LockColorOne{\lk_1})$}
\figev{1}{$\acq(\LockColorTwo{\lk_2})$}
\figev{1}{$\wt(x)$}
\figev{1}{$\rel(\LockColorTwo{\lk_2})$}
\figev{1}{$\rel(\LockColorOne{\lk_1})$}
\figev{2}{$\acq(\LockColorTwo{\lk_2})$}
\figev{2}{$\rd(x)$}

\figev{2}{$\acq(\LockColorOne{\lk_1})$}
\figev{2}{$\rel(\LockColorOne{\lk_1})$}
\figev{2}{$\rel(\LockColorTwo{\lk_2})$}

\figev{3}{$\acq(\LockColorTwo{\lk_2})$}
\figev{3}{$\acq(\LockColorOne{\lk_1})$}
%\figevoffset{13}{2}{$\rd(y)$}
\figev{3}{$\rel(\LockColorOne{\lk_1})$}
\figev{3}{$\rel(\LockColorTwo{\lk_2})$}



%\figev{3}{$\acq(\LockColorOne{\lk_1})$}
%\figev{3}{$\wt(y)$}

}
}
% \hspace{1cm}
% \execution{3}{
% %\figevoffset{11}{3}{$\rel(\LockColorOne{\lk_1})$}

% \figevoffset{11}{2}{$\acq(\LockColorThree{\lk_3})$}
% \figevoffset{11}{2}{$\acq(\LockColorFour{\lk_4})$}
% %\figevoffset{13}{2}{$\rd(y)$}
% \figevoffset{11}{2}{$\rel(\LockColorFour{\lk_4})$}
% \figevoffset{11}{2}{$\rel(\LockColorThree{\lk_3})$}

% %\figevoffset{11}{1}{$\acq(\LockColorOne{\lk_1})$}
% %\figevoffset{13}{1}{$\wt(y)$}
% %\figevoffset{13}{1}{$\rel(\LockColorOne{\lk_1})$}

% \figevoffset{11}{3}{$\acq(\LockColorFour{\lk_4})$}
% \figevoffset{11}{3}{$\acq(\LockColorThree{\lk_3})$}
% %\figevoffset{13}{3}{$\rd(y)$}
% \figevoffset{11}{3}{$\rel(\LockColorThree{\lk_3})$}
% \figevoffset{11}{3}{$\rel(\LockColorFour{\lk_4})$}

% }
% %\begin{tikzpicture}[]
% %	\tikzset{rectangle/.append style={draw=black}}
% %	\scalebox{0.84}{
% %		\def\xstep{2.2}
% %		\def\ystep{0.5}
% %		
% %		\node[rectangle, align=center] (t1) at (-1, -7){
% %			$t_2, \LockColorThree{\lk_3}, \{ \LockColorTwo{\lk_2} \}$
% %		};
% %		\node[rectangle, align=center] (t31) at (-1, -8.7){
% %			$t_3, \LockColorTwo{\lk_2}, \{ \LockColorThree{\lk_3} \}$
% %		};
% %	
% %		\draw[->, thick, -Latex, bend right=10] ([xshift=-3mm]t1.south) to ([xshift=-3mm]t31.north);% (t31.north);
% %		\draw[->, thick, -Latex, bend right=10] (t31) to (t1);
% %		
% %		\node[opacity=0] at (0*\xstep, 1){};
% %	}
% %\end{tikzpicture}
% }

% \caption{
% A  trace $\tr$ with a sync-preserving deadlock, stalling $t_2$ on $e_4$ and $t_3$ on $e_{18}$.
% Underlined events indicate the slice of $\tr$ that forms a sync-preserving lower set of $\tr$,
% and serves as a valid witness for the deadlock.
% \figlabel{motivating}
% }
\caption{
A  trace $\tr$ with a sync-preserving deadlock, stalling $t_1$ on $e_2$ and $t_3$ on $e_{12}$.
}
\figlabel{motivating}
\end{figure}