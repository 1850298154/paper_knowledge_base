%!TEX root=../main.tex

\begin{figure}[t]
\centering
\scalebox{0.9}{
\execution{4}{
\figev{1}{$\acq(\LockColorOne{\lk_1})$}
\figev{1}{$\rel(\LockColorOne{\lk_1})$}
\figev{2}{\underline{$\acq(\LockColorTwo{\lk_2})$}}
\figev{2}{$\mathbf{\Bacq(\LockColorThree{\lk_3})}$}
\figev{2}{$\wt(z)$}
\figev{2}{$\rel(\LockColorThree{\lk_3})$}
\figev{2}{$\rel(\LockColorTwo{\lk_2})$}
\figev{4}{$\acq(\LockColorOne{\lk_1})$}
\figev{4}{\underline{$\wt(y)$}}
\figev{4}{$\rd(z)$}
}
\hspace{1cm}
\execution{4}{
	\figevoffset{10}{4}{$\rel(\LockColorOne{\lk_1})$}
	\figevoffset{10}{1}{$\acq(\LockColorThree{\lk_3})$}
	\figevoffset{10}{1}{$\wt(x)$}
	\figevoffset{10}{1}{$\rd(y)$}
	\figevoffset{10}{1}{\underline{$\rel(\LockColorThree{\lk_3})$}}
	\figevoffset{10}{3}{$\acq(\LockColorThree{\lk_3})$}
	\figevoffset{10}{3}{\underline{$\rd(x)$}}
	\figevoffset{10}{3}{$\Bacq(\LockColorTwo{\lk_2})$}
	\figevoffset{10}{3}{$\rel(\LockColorTwo{\lk_2})$}
	\figevoffset{10}{3}{$\rel(\LockColorThree{\lk_3})$}
}

%\begin{tikzpicture}[]
%	\tikzset{rectangle/.append style={draw=black}}
%	\scalebox{0.84}{
%		\def\xstep{2.2}
%		\def\ystep{0.5}
%		
%		\node[rectangle, align=center] (t1) at (-1, -7){
%			$t_2, \LockColorThree{\lk_3}, \{ \LockColorTwo{\lk_2} \}$
%		};
%		\node[rectangle, align=center] (t31) at (-1, -8.7){
%			$t_3, \LockColorTwo{\lk_2}, \{ \LockColorThree{\lk_3} \}$
%		};
%	
%		\draw[->, thick, -Latex, bend right=10] ([xshift=-3mm]t1.south) to ([xshift=-3mm]t31.north);% (t31.north);
%		\draw[->, thick, -Latex, bend right=10] (t31) to (t1);
%		
%		\node[opacity=0] at (0*\xstep, 1){};
%	}
%\end{tikzpicture}
}

% \caption{
% A  trace $\tr$ with a sync-preserving deadlock, stalling $t_2$ on $e_4$ and $t_3$ on $e_{18}$.
% Underlined events indicate the slice of $\tr$ that forms a sync-preserving lower set of $\tr$,
% and serves as a valid witness for the deadlock.
% \figlabel{motivating}
% }
\caption{
A  trace $\tr$ with a sync-preserving deadlock, stalling $t_2$ on $e_4$ and $t_3$ on $e_{18}$.
\hunkar{
Modify the example to have two deadlock patterns where one of the deadlock patterns is be realizable and the other is not, in order to demonstrate that deadlock patterns are an insufficient condition.
}
}
\figlabel{motivating}
\end{figure}