%!TEX root=../main.tex

\begin{figure}[t]
\centering
\scalebox{1}{
\execution{4}{
\figev{4}{$\acq(\LockColorOne{\lk_1})$}
\figev{4}{$\rel(\LockColorOne{\lk_1})$}
\figev{3}{$\acq(\LockColorTwo{\lk_2})$}
\figev{3}{$\mathbf{\Bacq(\LockColorThree{\lk_3})}$}
\figev{3}{$\rel(\LockColorThree{\lk_3})$}
\figev{3}{$\rel(\LockColorTwo{\lk_2})$}
\figev{3}{$\wt(y)$}
\figev{1}{$\acq(\LockColorOne{\lk_1})$}
\figev{1}{$\wt(x)$}
\figev{1}{$\rd(y)$}
\figev{1}{$\rel(\LockColorOne{\lk_1})$}
\figev{2}{${\acq(\LockColorThree{\lk_3})}$}
\figev{2}{$\rd(x)$}
\figev{2}{$\mathbf{\Bacq(\LockColorTwo{\lk_2})}$}
\figev{2}{$\rel(\LockColorTwo{\lk_2})$}
\figev{2}{$\rel(\LockColorThree{\lk_3})$}
}
}
% \caption{
% A  trace $\tr$ with a sync-preserving deadlock, stalling $t_2$ on $e_4$ and $t_3$ on $e_{18}$.
% Underlined events indicate the slice of $\tr$ that forms a sync-preserving lower set of $\tr$,
% and serves as a valid witness for the deadlock.
% \figlabel{motivating}
% }
\caption{
A trace $\tr$ with a sync-preserving deadlock, stalling $t_2$ on $e_{14}$ and $t_3$ on $e_{4}$.
\figlabel{incomp1}
}
\end{figure}