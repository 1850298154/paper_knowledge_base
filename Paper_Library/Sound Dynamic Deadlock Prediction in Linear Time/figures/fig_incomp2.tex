%!TEX root=../main.tex

\begin{figure}[t]
\centering
\scalebox{1}{
\execution{2}{
\figev{1}{$\acq(\LockColorOne{\lk_1})$}
\figev{1}{$\mathbf{\Bacq(\LockColorTwo{\lk_2})}$}
\figev{1}{$\rel(\LockColorTwo{\lk_2})$}
\figev{1}{$\rel(\LockColorOne{\lk_1})$}
\figev{2}{${\acq(\LockColorTwo{\lk_2})}$}
\figev{2}{$\mathbf{\Bacq(\LockColorOne{\lk_1})}$}
\figev{2}{$\rel(\LockColorOne{\lk_1})$}
\figev{2}{$\mathbf{\Bacq(\LockColorOne{\lk_1})}$}
\figev{2}{$\rel(\LockColorOne{\lk_1})$}
\figev{2}{$\rel(\LockColorTwo{\lk_2})$}
}
}
% \caption{
% A  trace $\tr$ with a sync-preserving deadlock, stalling $t_2$ on $e_4$ and $t_3$ on $e_{18}$.
% Underlined events indicate the slice of $\tr$ that forms a sync-preserving lower set of $\tr$,
% and serves as a valid witness for the deadlock.
% \figlabel{motivating}
% }
\caption{
A  trace $\tr$ with a sync-preserving deadlock, stalling $t_1$ on $e_{2}$ and $t_2$ on $e_{6}$ and a predictable but not sync-preserving deadlock stalling $t_1$ on $e_{2}$ and $t_2$ on $e_{8}$.
\figlabel{incomp2}
}
\end{figure}