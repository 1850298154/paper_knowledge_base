%!TEX root = main.tex

\section{Proofs of \secref{syncp}}
\seclabel{sec:app_proofs_characterize_patterns}

We first prove \lemref{spclosure-necessary-sufficient}.

\spclosureNecessarySufficient*

% \begin{restatable}{lemma}{spclosureNecessarySufficient}
% \lemlabel{spclosure-necessary-sufficient}
% Let $\tr$ be a trace and let $S \subseteq \events{\tr}$.
% For any sync-preserving correct reordering $\rho$ of $\tr$, 
% 	if $S \subseteq \events{\rho}$, then $\SPClosure{\tr}(S) \subseteq \events{\rho}$.
% 	Further, there is a sync-preserving correct reordering $\rho$ of 
% 	$\tr$ such that $\events{\rho} = \SPClosure{\tr}(S)$.
% \end{restatable}

\begin{proof}
Let $\rho$ be a sync-preserving correct reordering of $\tr$ (and thus $\events{\rho} \subseteq \events{\tr}$)
such that $S \subseteq  \events{\rho}$.
Since, $\rho$ is a correct-reordering of $\tr$, for every event $e \in \events{\rho}$,
we must have $f \in \events{\rho}$ for every $f \tho{\tr} e$.
Likewise, for every read event $e \in \events{\rho}$, we must also have $\rf{\tr}(e) \in \events{\rho}$,
again because $\rho$ is a correct reordering of $\tr$.
Finally, let $e \in \events{\rho}$ be an acquire event (with $\OpOf{e} = \acq(\lk)$), and let
$f \neq e$ be another acquire event (with $\OpOf{e} = \acq(\lk)$)
with $f \trord{\tr} e$.
Since $\rho$ is a sync-preserving correct reordering of $\rho$, we must also have
$\match{\tr}(f) \in \events{\rho}$.
Thus, $\rho$ must contain $\SPClosure{\tr}(S)$.

Let us now argue that we can construct a sync-preserving correct reordering whose events are precisely $\SPClosure{\tr}(S)$.
Consider the sequence of events $\rho$ obtained by projecting $\tr$
to $\SPClosure{\tr}(S)$.
We will argue that (a) $\rho$ is a well-formed trace, 
(b) $\rho$ is a correct reordering of $\tr$, and 
(c) $\rho$ preserves the order of critical section on the same lock as in $\tr$.
For (a), we need to argue that critical sections on the same lock do not overlap in $\rho$.
This follows because if they did, then we have two acquire events $e_1 \neq e_2 \in \SPClosure{\tr}(S)$
such that $\OpOf{e_1} = \OpOf{e_2} = \acq(\lk)$ (for some $\lk$),
$e_1 \trord{\tr} e_2$ (and thus $e_1$ also appears before $e_2$ in $\rho$ by construction),
but $\match{\tr}(e_1)$ either does not appear in $\rho$ or appears later than $e_2$.
In the first case, we arrive at a contradiction to the definition of $\SPClosure{\tr}(S)$,
and in the second case, we arrive at a contradiction that $\rho$ is obtained without changing the relative order of the residual events.
Arguing (b) is straightforward: $\SPClosure{\tr}(S)$ is closed under both $\tho{\tr}$ as well as $\rf{\tr}$,
and further $\rho$ preserves the order of events as in $\tr$.
Likewise,  (c) is easy to argue because $\events{\rho} = \SPClosure{\tr}(S)$ are closed
under sync-preservation and the order of events in $\rho$ does not change with respect to $\tr$. 
\end{proof}

Let us now turn our attention to \lemref{spclosure-deadlock}.

\spclosureDeadlock*

% \begin{restatable}{lemma}{spclosureDeadlock}
% \lemlabel{spclosure-deadlock}
% Let $\tr$ be a trace and let $D = \pattern{e_0, \ldots, e_{k-1}}$
% be a deadlock pattern of size $k$ in $\tr$.
% $D$ is a sync-preserving deadlock of $\tr$ iff
% $\SPClosure{\tr}(\prev{\tr}(S)) \cap S = \emptyset$, where
% $S = \set{e_0, \ldots, e_{k-1}}$.
% \end{restatable}

\begin{proof}
($\Rightarrow$). Assume $D$ is a sync-preserving deadlock of $\tr$.
Let $\rho$ be the sync-preserving correct reordering of $\tr$ that witnesses $D$.
Thus, each of $e_0, \ldots, e_{k-1}$ are enabled in $\rho$.
This means that $\prev{\tr}(S) \subseteq \events{\rho}$, and also $S \cap \events{\rho} = \emptyset$.
By \lemref{spclosure-necessary-sufficient}, we have
$\SPClosure{\tr}(\prev{\tr}(S)) \subseteq \events{\rho}$, giving us
$\SPClosure{\tr}(\prev{\tr}(S)) \cap S = \emptyset$.

($\Leftarrow$). Assume $\SPClosure{\tr}(\prev{\tr}(S)) \cap S = \emptyset$.
From \lemref{spclosure-necessary-sufficient}, there exists a sync-preserving correct reordering
$\rho$ with $\prev{\tr}(S) \subseteq \SPClosure{\tr}(\prev{\tr}(S)) = \events{\rho}$.
Since $S \cap \events{\rho} = \emptyset$, the entire deadlock pattern $D$ is also enabled in $\rho$.
Thus, $D$ is a sync-preserving deadlock of $\tr$.
\end{proof}

The proof of the following proposition follows directly from the definition of sync-preserving closure.

\spclosureMonotone*

\begin{proof}
Follows from the definition of $\SPClosure{\tr}(S)$.
\end{proof}