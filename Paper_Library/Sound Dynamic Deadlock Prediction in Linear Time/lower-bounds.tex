%!TEX root = main.tex

\section{The Complexity of Dynamic Deadlock Prediction}
\seclabel{lower-bounds}

Detecting deadlock patterns and predictable deadlocks is clearly a problem in $\NP$,
as any witness for either problem can be verified in polynomial time.
However, little has been known about the hardness of the problem in terms of rigorous lower bounds.
Here we settle these questions, by proving strong intractability results.
Due to space constraints, we state and explain the main results here, and refer to\begin{pldi}~our technical report \cite{arxiv}\end{pldi}\begin{arxiv}~\cref{sec:sec:app_proofs_lower_bounds}\end{arxiv} for the full proofs.

\myparagraph{Parametrized hardness for detecting deadlock patterns}{
We show that the basic problem of checking the existence of a deadlock \emph{pattern} is itself
hard parameterized by the size $k$
of the pattern.
% \ucomment{Can be rewritten as: We first consider the problem of checking the existence of deadlock \emph{patterns}. Intuitively, a solution to the problem of deadlock prediction must resolve this basic problem first. Our first result shows that this problem is hard, parametrized by the size of the pattern we are looking for. In particular, the following result implies}

\begin{restatable}{theorem}{patternwonehardness}
\thmlabel{pattern-w1-hardness-pattern}
Checking if a trace $\tr$ contains a deadlock pattern of size $k$
is $\W{1}$-hard in the parameter $k$.
Moreover, the problem remains $\NP$-hard even when the lock-nesting depth of $\tr$ is constant.
\end{restatable}
}
%!TEX root = main.tex

\begin{proof}
	We show that there is a polynomial-time fixed parameter tractable reduction from
	INDEPENDENT-SET(c) to the problem of checking the existence of deadlock-patterns
	of size $c$.
	Our reduction takes as input an undirected graph $G$ and outputs a trace $\tr$
	such that $G$ has an independent set of size $c$ iff $\tr$
	has a deadlock pattern of size $c$.
	%Here present the construction and refer to \appref{proofs} for the proof of correctness.
	
	\begin{figure}[]
		\input{figures/construction-w1-hardness}
		\hfill ~~ %\unskip\ \vrule\ \hfill
		%!TEX root = ../main.tex
%\begin{figure}
\begin{subfigure}[b]{0.475\textwidth}
	\newcommand{\xdisposition}{0}
	\newcommand{\ydisposition}{0}
	\newcommand{\xtstep}{0.75}
	\newcommand{\ytstep}{1}
	\newcommand{\ybias}{-0.3 }
	\newcommand{\xstep}{2.5}
	\newcommand{\ystep}{-0.475}
	\newcommand{\xtscale}{0.8}
	\def \numevents{9.5}
	\newcommand{\eventA}[4]{
		\node[event, draw=black, fill=white] (A#1) at (#1*\xstep, #2*\ystep) {\footnotesize $#2(x_{#3})$};
		%\node[] at (0*\xstep-\xtstep, {#1*\ystep}) {\small $#2(x_{#3})$};
	}
	\scalebox{0.9}{
		\begin{tikzpicture}[thick,
			pre/.style={<-,shorten >= 2pt, shorten <=2pt, very thick},
			post/.style={->,shorten >= 3pt, shorten <=3pt,   thick},
			seqtrace/.style={line width=2},
			und/.style={very thick, draw=gray},
			event/.style={rectangle, minimum height=0.8mm, minimum width=15mm,  line width=1pt, inner sep=0.5,},
			virt/.style={circle,draw=black!50,fill=black!20, opacity=0}]
			\footnotesize
			
			
%			\node[] at (0.5*\xstep, -1.5*\ystep){
%				\small
%				$\cs(\lk_i,\lk_j)=\acq(\lk_i) \cdot\acq(\lk_j) \cdot\rel(\lk_j) \cdot \rel(\lk_i)$
%			};
			

				
			\begin{scope}[shift={(0,-3*\ystep)}]
				
				\draw[dashed] (-0.9*\xstep,6.5*\ystep) rectangle (-0.5*\xstep,8.5*\ystep);
				\node (A1) at (-0.7*\xstep,7*\ystep) {\normalsize [1, 1]};
				\node (A2) at (-0.7*\xstep,8*\ystep) {\normalsize [1, 0]};
				\node (A) at (-0.7*\xstep,9*\ystep) {\large  $A$};
				
				\draw[dashed] (1.5*\xstep,6.5*\ystep) rectangle (1.9*\xstep,8.5*\ystep);
				\node (B1) at (1.7*\xstep,7*\ystep) {\normalsize [1, 0]};
				\node (B2) at (1.7*\xstep,8*\ystep) {\normalsize [0, 1]};
				\node (B) at (1.7*\xstep,9*\ystep) {\large  $B$};
				
				
			\end{scope}
			

			
			\node[] (S11) at (0*\xstep,0.15) {\normalsize $t_A$};
			\node[] (S12) at (0*\xstep,\numevents * \ystep) {};
			\node[] (S21) at (1*\xstep,0.15) {\normalsize $t_B$};
			\node[] (S22) at (1*\xstep,\numevents * \ystep) {};
			
			\draw[seqtrace] (S11) to (S12);
			\draw[seqtrace] (S21) to (S22);
			
			
			\node[event, draw=black, fill=white] (11) at (0*\xstep, 1*\ystep + 0*\ybias) {$\acq(\LockColorTwo{\lk_2})$};
			\node[event, draw=black, fill=white] (12) at (0*\xstep, 2*\ystep + 0*\ybias) {$\acq(\LockColorOne{\lk_1})$};
			\node[event, draw=black, fill=white, dotted] (13) at (0*\xstep, 3*\ystep + 0*\ybias) {$\cs(m_0,m_1)$};
			%\node[event, draw=black, fill=white, dotted] (14) at (0*\xstep, 4*\ystep + 0*\ybias) {$\acq(\lk')$};
			%\node[event, draw=black, fill=white, dotted] (15) at (0*\xstep, 5*\ystep + 0*\ybias) {$\rel(\lk')$};
			%\node[event, draw=black, fill=white, dotted] (16) at (0*\xstep, 6*\ystep + 0*\ybias) {$\rel(\lk)$};
			\node[event, draw=black, fill=white] (17) at (0*\xstep, 4*\ystep + 0*\ybias) {$\rel(\LockColorOne{\lk_1})$};
			\node[event, draw=black, fill=white] (18) at (0*\xstep, 5*\ystep + 0*\ybias) {$\rel(\LockColorTwo{\lk_2})$};
			
			\node[event, draw=black, fill=white] (19) at (0*\xstep, 6*\ystep + 1*\ybias) {$\acq(\LockColorOne{\lk_1})$};
			\node[event, draw=black, fill=white, dotted] (110) at (0*\xstep, 7*\ystep + 1*\ybias) {$\cs(m_0,m_1)$};
			%\node[event, draw=black, fill=white, dotted] (111) at (0*\xstep, 11*\ystep + 1*\ybias) {$\acq(\lk')$};
			%\node[event, draw=black, fill=white, dotted] (112) at (0*\xstep, 12*\ystep + 1*\ybias) {$\rel(\lk')$};
			%\node[event, draw=black, fill=white, dotted] (113) at (0*\xstep, 13*\ystep + 1*\ybias) {$\rel(\lk)$};
			\node[event, draw=black, fill=white] (114) at (0*\xstep, 8*\ystep + 1*\ybias) {$\rel(\LockColorOne{\lk_1})$};
			
			%%%%%%%%%%%%%%%%
			
			\node[event, draw=black, fill=white] (21) at (1*\xstep, 2*\ystep + 0*\ybias) {$\acq(\LockColorOne{\lk_1})$};
			\node[event, draw=black, fill=white, dotted] (22) at (1*\xstep, 3*\ystep + 0*\ybias) {$\cs(m_1,m_0)$};
			%\node[event, draw=black, fill=white, dotted] (23) at (1*\xstep, 4*\ystep + 0*\ybias) {$\acq(\lk)$};
			%\node[event, draw=black, fill=white, dotted] (24) at (1*\xstep, 5*\ystep + 0*\ybias) {$\rel(\lk)$};
			%\node[event, draw=black, fill=white, dotted] (25) at (1*\xstep, 6*\ystep + 0*\ybias) {$\rel(\lk')$};
			\node[event, draw=black, fill=white] (26) at (1*\xstep, 4*\ystep + 0*\ybias) {$\rel(\LockColorOne{\lk_1})$};
			
			\node[event, draw=black, fill=white] (27) at (1*\xstep, 5*\ystep + 1*\ybias) {$\acq_1(\LockColorTwo{\lk_2})$};
			\node[event, draw=black, fill=white, dotted] (28) at (1*\xstep, 6*\ystep + 1*\ybias) {$\cs(m_1,m)$};
			%\node[event, draw=black, fill=white, dotted] (29) at (1*\xstep, 11*\ystep + 1*\ybias) {$\acq(\lk)$};
			%\node[event, draw=black, fill=white, dotted] (210) at (1*\xstep, 12*\ystep + 1*\ybias) {$\rel(\lk)$};
			%\node[event, draw=black, fill=white, dotted] (211) at (1*\xstep, 13*\ystep + 1*\ybias) {$\rel(\lk')$};
			\node[event, draw=black, fill=white] (212) at (1*\xstep, 7*\ystep + 1*\ybias) {$\rel_1(\LockColorTwo{\lk_2})$};
			
			\begin{scope}[]
				\node[below left=of 212] (212b) {};
			\end{scope}
	
		\end{tikzpicture}
	}
	\caption{
		Reduction for OV-hardness proof from an instance of size $n = 2$ and $d = 2$.
		% We use the shortcut $\cs(\lk_i,\lk_j)$ to denote two nested critical sections on $\lk_i$ and $\lk_j$, as indicated in the figure.
	}
	\figlabel{ov-hardness}
\end{subfigure}
%\end{figure}
		\caption{
		Construction of $\W{1}$-hardness (\subref{fig:w1-hardness}) and OV-hardness (\subref{fig:ov-hardness}) results. 
		We use the shortcut $\cs(\lk,\lk')$ to denote two nested critical sections on $\lk$ and $\lk'$. That is,
	$\cs(\lk,\lk')=\acq(\lk) \cdot\acq(\lk') \cdot\rel(\lk') \cdot \rel(\lk)$.
		} 
		%\label{fig:main}
	\end{figure}


	\myparagraph{Construction}{
		%Let $G = (V, E)$ be the input graph and let $c$ be the parameter.
		Let $V = \set{v_1, v_2, \ldots, v_n}$.
		We assume a total ordering $<_E$ on the set of edges $E$.
		The trace $\tr$ we construct is a concatenation of $c$ sub-traces: 
		$
		\tr = \tr^{(1)} \cdot \tr^{(2)} \cdots \tr^{(c)}
		$
		and uses $c$ threads $\set{t_1, t_2, \ldots t_c}$ and
		$|E| + c$ locks $\set{\lk_{\set{u, v}}}_{\set{u, v} \in E} \uplus \set{\lk_0, \lk_1 \ldots, \lk_{c-1}}$.
		The $i^\text{th}$ sub-trace $\tr^{(i)}$ is a sequence of events performed by thread $t_i$, and
		is obtained by concatenation of $n = |V|$ sub-traces:
		$
		\tr^{(i)} = \tr^{(i)}_1 \cdot \tr^{(i)}_2 \cdots \tr^{(i)}_n
		$.
		Each sub-trace $\tr^{(i)}_j$ with $(i \leq c, j \leq n)$ 
		comprises of nested critical sections over locks of the 
		form $\lk_{\set{v_j, u}}$, where $u$ is a neighbor of $v_j$.
		Inside the nested block we have critical 
		sections on locks $\lk_{i \% c}$ and $\lk_{(i+1) \% c}$.
		Formally, let $\set{v_j, v_{k_1}}, \ldots, \set{v_j, v_{k_d}}$
		be the neighboring edges of $v_j$ (ordered according to $<_E$).
		Then, $\tr^{(i)}_j$ is the unique string generated by the
		grammar having $d+1$ non-terminals $S_0, S_1, \ldots, S_d$, start symbol $S_d$
		and the following production rules:
		\begin{itemize}
			\item $S_0 \to \ev{t_i, \acq(\lk_{i \% c})} \cdot \ev{t_i, \acq(\lk_{(i+1) \% c})} \cdot \ev{t_i, \rel(\lk_{(i+1) \% c})} \cdot \ev{t_i, \rel(\lk_{i \% c})}$.
			\item for each $1 \leq r \leq d$, $S_r \to \ev{t_i, \acq(\lk_{\set{v_j, v_{k_r}}})} \cdot S_{r-1} \cdot \ev{t_i, \rel(\lk_{\set{v_j, v_{k_r}}})}$.
		\end{itemize}
		\figref{w1-hardness} illustrates this construction for a graph with $3$ nodes
		and parameter $c = 3$.
		Finally, observe that the lock-nesting depth in $\tr$ is bounded by 2 + the degree of $G$.
	}
\end{proof}








\thmref{pattern-w1-hardness-pattern} implies that the problem is not only $\NP$-hard, 
but also unlikely to be \emph{fixed parameter tractable} in the size $k$ of the deadlock pattern.
In fact, under the well-believed Exponential Time Hypothesis (ETH), the parametrized
problem INDEPENDENT-SET(c) cannot be solved in time $f(c) \cdot n^{o(c)}$~\cite{chen2006strong}.
The above reduction preserves the parameter $k=c$, thus under ETH,
detecting deadlock patterns of size $k$ is unlikely to be solvable 
in time complexity $f(k) \cdot \NumEvents^{g(k)}$, where 
$g(k)$ is $o(k)$ (such as $g(k) = \sqrt{k}$ or even $g(k) = k/\log(k)$).
% It is known that INDEPENDENT-SET(c) cannot
% be solved in f (c) · n
% o(c)
% time under ETH [11]. As our reduction
% to rf-poset realizability and dynamic data-race prediction uses
% k = O(c) threads, each of these problems does not have a
% f (k) · n
% o(k)
% -time algorithm under ETH.
% In turn this means that it is unlikely to obtain an algorithm for detecting deadlock patterns that runs in time $O(\NumEvents^{c}\cdot f(k))$, where $c$ is a constant and $f$ is \emph{any} function independent of $n$.
% This is important as $k$ is typically constant; 
% $\W{1}$-hardness indicates that we can only expect algorithms with running time $O(\NumEvents^{g(k)})$, i.e., the degree of the polynomial depends on the size of the deadlock patterns which can be as high as the number of threads.
The problem of checking the existence of deadlock patterns 
is, intuitively, a precursor to the deadlock prediction problem. 
Thus, an approach for 
deadlock prediction that first identifies the existence of arbitrary deadlock patterns and 
then verifying their feasibility is unlikely to be tractable. 
In practice, the synchronization patterns corresponding to the hard instances are uncommon in executions, and our proposed algorithms (\secref{syncp} and \secref{otf}) can effectively expose predictable deadlocks (\secref{experiments}).


\myparagraph{Fine-grained hardness for deadlock pattern detection}{
\begin{comment}
	We now establish a fine-grained hardness for detecting
	deadlock patterns --- for each $k \geq 2$, we cannot check for the existence of 
	patterns of size $k$ in time $O(\NumEvents^{k-\epsilon})$ for any $\epsilon > 0$,
	under the popular Orthogonal Vectors hypothesis (OV).
	%This result is based on a reduction from the popular Orthogonal Vectors (OV) problem.
	For a fixed $k \geq 2$, the $k$-OV problem takes
	$k$ sets of $d$-dimensional vectors
	$A_1, A_2, \ldots, A_k \subseteq \set{0, 1}^d$, each of cardinality $|A_i|= n$
	($1 \leq i \leq k$) as input, 
	and asks if
	there are vectors $a_1 \in A_1, \ldots, a_k \in A_k$
	such that the extended dot product $a_1 \cdot a_2 \cdots a_k = \sum_{p =1}^d (a_1[p]\cdot a_2[p] \cdots a_k[p])= 0$.
	For a $k\geq 2$, the $k$-OV hypothesis states that for any 
	$\epsilon >0$, there is no $O(n^{k-\epsilon}\cdot \poly{d})$ algorithm for $k$-OV.
	The Strong Exponential Time Hypothesis (SETH) implies $k$-OV ~\cite{williams2005}.
	Our next theorem is based on showing that, for every $k \geq 2$, 
	detecting deadlock patterns of size $k$ is at least as hard as solving $k$-OV.
 Note the difference between \thmref{pattern-w1-hardness-pattern} and \thmref{pattern-ov-hardness}:~the former allows algorithms with running time of the form $\NumEvents^{k/2}$ (even under ETH), but the latter excludes them, requiring that $k$ fully appears in the exponent.
 The two results are based on different hypotheses and also incomparable since 
 it could turn out that $k$-OV is false but ETH is true.
 We thus establish both results, in order to develop a deeper understanding of the intricacies of the problem.
 \end{comment}
% %!TEX root = main.tex

\begin{proof}
	We show a fine-grained reduction from the Orthogonal Vectors Problem to
	the problem of checking for deadlock patterns of size $k$.
	For this, we start with two sets
	$A_1, A_2, \ldots, A_k \subseteq \set{0, 1}^d$
	of $d$-dimensional vectors with $|A_i| = n$ for every $1 \leq i \leq k$.
	We write the $j^{th}$ vector in $A_i$ as $A_{i, j}$.
	
	%%!TEX root = ../main.tex
%\begin{figure}
\begin{subfigure}[b]{0.475\textwidth}
	\newcommand{\xdisposition}{0}
	\newcommand{\ydisposition}{0}
	\newcommand{\xtstep}{0.75}
	\newcommand{\ytstep}{1}
	\newcommand{\ybias}{-0.3 }
	\newcommand{\xstep}{2.5}
	\newcommand{\ystep}{-0.475}
	\newcommand{\xtscale}{0.8}
	\def \numevents{9.5}
	\newcommand{\eventA}[4]{
		\node[event, draw=black, fill=white] (A#1) at (#1*\xstep, #2*\ystep) {\footnotesize $#2(x_{#3})$};
		%\node[] at (0*\xstep-\xtstep, {#1*\ystep}) {\small $#2(x_{#3})$};
	}
	\scalebox{0.9}{
		\begin{tikzpicture}[thick,
			pre/.style={<-,shorten >= 2pt, shorten <=2pt, very thick},
			post/.style={->,shorten >= 3pt, shorten <=3pt,   thick},
			seqtrace/.style={line width=2},
			und/.style={very thick, draw=gray},
			event/.style={rectangle, minimum height=0.8mm, minimum width=15mm,  line width=1pt, inner sep=0.5,},
			virt/.style={circle,draw=black!50,fill=black!20, opacity=0}]
			\footnotesize
			
			
%			\node[] at (0.5*\xstep, -1.5*\ystep){
%				\small
%				$\cs(\lk_i,\lk_j)=\acq(\lk_i) \cdot\acq(\lk_j) \cdot\rel(\lk_j) \cdot \rel(\lk_i)$
%			};
			

				
			\begin{scope}[shift={(0,-3*\ystep)}]
				
				\draw[dashed] (-0.9*\xstep,6.5*\ystep) rectangle (-0.5*\xstep,8.5*\ystep);
				\node (A1) at (-0.7*\xstep,7*\ystep) {\normalsize [1, 1]};
				\node (A2) at (-0.7*\xstep,8*\ystep) {\normalsize [1, 0]};
				\node (A) at (-0.7*\xstep,9*\ystep) {\large  $A$};
				
				\draw[dashed] (1.5*\xstep,6.5*\ystep) rectangle (1.9*\xstep,8.5*\ystep);
				\node (B1) at (1.7*\xstep,7*\ystep) {\normalsize [1, 0]};
				\node (B2) at (1.7*\xstep,8*\ystep) {\normalsize [0, 1]};
				\node (B) at (1.7*\xstep,9*\ystep) {\large  $B$};
				
				
			\end{scope}
			

			
			\node[] (S11) at (0*\xstep,0.15) {\normalsize $t_A$};
			\node[] (S12) at (0*\xstep,\numevents * \ystep) {};
			\node[] (S21) at (1*\xstep,0.15) {\normalsize $t_B$};
			\node[] (S22) at (1*\xstep,\numevents * \ystep) {};
			
			\draw[seqtrace] (S11) to (S12);
			\draw[seqtrace] (S21) to (S22);
			
			
			\node[event, draw=black, fill=white] (11) at (0*\xstep, 1*\ystep + 0*\ybias) {$\acq(\LockColorTwo{\lk_2})$};
			\node[event, draw=black, fill=white] (12) at (0*\xstep, 2*\ystep + 0*\ybias) {$\acq(\LockColorOne{\lk_1})$};
			\node[event, draw=black, fill=white, dotted] (13) at (0*\xstep, 3*\ystep + 0*\ybias) {$\cs(m_0,m_1)$};
			%\node[event, draw=black, fill=white, dotted] (14) at (0*\xstep, 4*\ystep + 0*\ybias) {$\acq(\lk')$};
			%\node[event, draw=black, fill=white, dotted] (15) at (0*\xstep, 5*\ystep + 0*\ybias) {$\rel(\lk')$};
			%\node[event, draw=black, fill=white, dotted] (16) at (0*\xstep, 6*\ystep + 0*\ybias) {$\rel(\lk)$};
			\node[event, draw=black, fill=white] (17) at (0*\xstep, 4*\ystep + 0*\ybias) {$\rel(\LockColorOne{\lk_1})$};
			\node[event, draw=black, fill=white] (18) at (0*\xstep, 5*\ystep + 0*\ybias) {$\rel(\LockColorTwo{\lk_2})$};
			
			\node[event, draw=black, fill=white] (19) at (0*\xstep, 6*\ystep + 1*\ybias) {$\acq(\LockColorOne{\lk_1})$};
			\node[event, draw=black, fill=white, dotted] (110) at (0*\xstep, 7*\ystep + 1*\ybias) {$\cs(m_0,m_1)$};
			%\node[event, draw=black, fill=white, dotted] (111) at (0*\xstep, 11*\ystep + 1*\ybias) {$\acq(\lk')$};
			%\node[event, draw=black, fill=white, dotted] (112) at (0*\xstep, 12*\ystep + 1*\ybias) {$\rel(\lk')$};
			%\node[event, draw=black, fill=white, dotted] (113) at (0*\xstep, 13*\ystep + 1*\ybias) {$\rel(\lk)$};
			\node[event, draw=black, fill=white] (114) at (0*\xstep, 8*\ystep + 1*\ybias) {$\rel(\LockColorOne{\lk_1})$};
			
			%%%%%%%%%%%%%%%%
			
			\node[event, draw=black, fill=white] (21) at (1*\xstep, 2*\ystep + 0*\ybias) {$\acq(\LockColorOne{\lk_1})$};
			\node[event, draw=black, fill=white, dotted] (22) at (1*\xstep, 3*\ystep + 0*\ybias) {$\cs(m_1,m_0)$};
			%\node[event, draw=black, fill=white, dotted] (23) at (1*\xstep, 4*\ystep + 0*\ybias) {$\acq(\lk)$};
			%\node[event, draw=black, fill=white, dotted] (24) at (1*\xstep, 5*\ystep + 0*\ybias) {$\rel(\lk)$};
			%\node[event, draw=black, fill=white, dotted] (25) at (1*\xstep, 6*\ystep + 0*\ybias) {$\rel(\lk')$};
			\node[event, draw=black, fill=white] (26) at (1*\xstep, 4*\ystep + 0*\ybias) {$\rel(\LockColorOne{\lk_1})$};
			
			\node[event, draw=black, fill=white] (27) at (1*\xstep, 5*\ystep + 1*\ybias) {$\acq_1(\LockColorTwo{\lk_2})$};
			\node[event, draw=black, fill=white, dotted] (28) at (1*\xstep, 6*\ystep + 1*\ybias) {$\cs(m_1,m)$};
			%\node[event, draw=black, fill=white, dotted] (29) at (1*\xstep, 11*\ystep + 1*\ybias) {$\acq(\lk)$};
			%\node[event, draw=black, fill=white, dotted] (210) at (1*\xstep, 12*\ystep + 1*\ybias) {$\rel(\lk)$};
			%\node[event, draw=black, fill=white, dotted] (211) at (1*\xstep, 13*\ystep + 1*\ybias) {$\rel(\lk')$};
			\node[event, draw=black, fill=white] (212) at (1*\xstep, 7*\ystep + 1*\ybias) {$\rel_1(\LockColorTwo{\lk_2})$};
			
			\begin{scope}[]
				\node[below left=of 212] (212b) {};
			\end{scope}
	
		\end{tikzpicture}
	}
	\caption{
		Reduction for OV-hardness proof from an instance of size $n = 2$ and $d = 2$.
		% We use the shortcut $\cs(\lk_i,\lk_j)$ to denote two nested critical sections on $\lk_i$ and $\lk_j$, as indicated in the figure.
	}
	\figlabel{ov-hardness}
\end{subfigure}
%\end{figure}
	
	\myparagraph{Construction}{
		We will construct a trace $\tr$ such that $\tr$ has a deadlock
		pattern of length $k$ iff $(A_1, A_2, \ldots, A_k)$ is a positive $k$-OV instance.
		The trace $\tr$ is of the form $\tr = \tr^{A_1} \cdot \tr^{A_2} \cdots \tr^{A_k}$
		and uses $k$ threads $\set{t_{A_1}, \ldots, t_{A_k}}$ and $d+k$ distinct 
		locks $\lk_1, \ldots, \lk_d, m_1, m_2, \ldots, m_k$. 
		Intuitively, the sub-trace $\tr^{A_i}$ encodes the given set of vectors $A_i$.
		The sub-traces $\tr^{A_i} = \tr^{A_i}_1 \cdot \tr^{A_i}_2 \cdots \tr^{A_i}_n$
		% and $\tr^B = \tr^B_1 \cdot \tr^B_2 \cdots \tr^B_n$ 
		are defined as follows.
		For each $j \in \set{1, 2, \ldots, n}$
		the sub-trace $\tr^{A_i}_j$ is the unique string generated by the
		grammar having $d+1$ non-terminals $S_0, S_1, \ldots, S_d$, start symbol $S_d$
		and the following production rules:
		\begin{itemize}
			\item $S_0 \to \ev{t_Z, \acq(m_i)} \cdot \ev{t_Z, \acq(m_{i \% k + 1})} \cdot \ev{t_Z, \rel(m_{i \% k + 1})} \cdot \ev{t_Z, \rel(m_i)}$.
			% where $(m, m') = (\lk, \lk')$ if $Z = A$, and $(m, m') = (\lk', \lk)$.
			\item for each $1 \leq p \leq d$, $S_p \to S_{p-1}$ if $A_{i, j}[p] = 0$.
			Otherwise (if $A_{i, j}[p] =1$), $S_p \to \ev{t_{A_i}, \acq(\lk_p)} \cdot S_{p-1} \cdot \ev{t_{A_i}, \rel(\lk_p)}$.
		\end{itemize}
		In words, all events of $\tr^{A_i}$ are performed by thread $t_{A_i}$.
		 % and those in $\tr^B$ are performed by $t_B$.
		Next, the $j^{th}$ sub-trace of $\tr^{A_i}$, denoted $\tr^{A_i}_j$ corresponds to the vector $A_{i, j}$
		as follows --- $\tr^{A_i}_j$ is a nested block of critical sections,
		with the innermost critical section being on lock $m_{i \% k + 1}$,
		which is immediately enclosed in a  critical section on lock $m_i$.
		Further, in the sub-trace $\tr^{A_i}_j$, 
		the lock $\lk_p$ occurs iff $A_{i, j}[p] = 1$.
		% The sub-traces $\tr^B_i$ is similarly constructed, except that the order
		% of the two innermost critical sections is inverted.
		\figref{ov-hardness} illustrates the construction for an OV-instance with $k=2$ $n=2$ and $d=2$.
	}
\end{proof}

% \myparagraph{The complexity of size-$2$ patterns}{
We next consider the problem of
detecting deadlock patterns of size $2$, as these form the most common case in practice~\cite{Lu08}.
Observe that \thmref{pattern-w1-hardness-pattern} has no implications on this case, as here $k$ is fixed.
The problem admits a folklore $O(\NumEvents^2)$ time algorithm, by iterating over all pairs of lock-acquisition events of the input trace, and checking whether any such pair forms a deadlock pattern.
Perhaps surprisingly, here we show that, despite its simplicity, this algorithm is optimal, i.e., we cannot hope to improve over this quadratic bound.
This result is based on a reduction from the popular Orthogonal Vectors (OV) problem.
Given two sets of $d$-dimensional vectors
$A, B \subseteq \set{0, 1}^d$
of cardinality $|A| = |B| = n$, the OV problem asks if there are $a \in A, b \in B$
such that $a \cdot b = \sum_i a[i]\cdot b[i]= 0$.
The OV hypothesis states that for any 
$\epsilon >0$, there is no $O(n^{2-\epsilon}\cdot \poly{d})$ algorithm for solving OV.
This is also a consequence of the famous Strong Exponential Time Hypothesis (SETH)~\cite{williams2005}.
We next show that detecting deadlock patterns of size $2$ is at least as hard as solving OV.
% }

\begin{restatable}{theorem}{patternovhardness}
\thmlabel{pattern-ov-hardness}
Given a trace $\tr$ of size $\NumEvents$ and $\NumLocks$ locks, 
for any $\epsilon>0$,
there is no algorithm that determines in $O(\NumEvents^{2-\epsilon}\cdot \poly{\NumLocks})$ time
whether $\tr$ has a deadlock pattern of size $2$, under the OV hypothesis.
\end{restatable}

% %!TEX root = main.tex

\begin{proof}
	We show a fine-grained reduction from the Orthogonal Vectors Problem to
	the problem of checking for deadlock patterns of size $k$.
	For this, we start with two sets
	$A_1, A_2, \ldots, A_k \subseteq \set{0, 1}^d$
	of $d$-dimensional vectors with $|A_i| = n$ for every $1 \leq i \leq k$.
	We write the $j^{th}$ vector in $A_i$ as $A_{i, j}$.
	
	%%!TEX root = ../main.tex
%\begin{figure}
\begin{subfigure}[b]{0.475\textwidth}
	\newcommand{\xdisposition}{0}
	\newcommand{\ydisposition}{0}
	\newcommand{\xtstep}{0.75}
	\newcommand{\ytstep}{1}
	\newcommand{\ybias}{-0.3 }
	\newcommand{\xstep}{2.5}
	\newcommand{\ystep}{-0.475}
	\newcommand{\xtscale}{0.8}
	\def \numevents{9.5}
	\newcommand{\eventA}[4]{
		\node[event, draw=black, fill=white] (A#1) at (#1*\xstep, #2*\ystep) {\footnotesize $#2(x_{#3})$};
		%\node[] at (0*\xstep-\xtstep, {#1*\ystep}) {\small $#2(x_{#3})$};
	}
	\scalebox{0.9}{
		\begin{tikzpicture}[thick,
			pre/.style={<-,shorten >= 2pt, shorten <=2pt, very thick},
			post/.style={->,shorten >= 3pt, shorten <=3pt,   thick},
			seqtrace/.style={line width=2},
			und/.style={very thick, draw=gray},
			event/.style={rectangle, minimum height=0.8mm, minimum width=15mm,  line width=1pt, inner sep=0.5,},
			virt/.style={circle,draw=black!50,fill=black!20, opacity=0}]
			\footnotesize
			
			
%			\node[] at (0.5*\xstep, -1.5*\ystep){
%				\small
%				$\cs(\lk_i,\lk_j)=\acq(\lk_i) \cdot\acq(\lk_j) \cdot\rel(\lk_j) \cdot \rel(\lk_i)$
%			};
			

				
			\begin{scope}[shift={(0,-3*\ystep)}]
				
				\draw[dashed] (-0.9*\xstep,6.5*\ystep) rectangle (-0.5*\xstep,8.5*\ystep);
				\node (A1) at (-0.7*\xstep,7*\ystep) {\normalsize [1, 1]};
				\node (A2) at (-0.7*\xstep,8*\ystep) {\normalsize [1, 0]};
				\node (A) at (-0.7*\xstep,9*\ystep) {\large  $A$};
				
				\draw[dashed] (1.5*\xstep,6.5*\ystep) rectangle (1.9*\xstep,8.5*\ystep);
				\node (B1) at (1.7*\xstep,7*\ystep) {\normalsize [1, 0]};
				\node (B2) at (1.7*\xstep,8*\ystep) {\normalsize [0, 1]};
				\node (B) at (1.7*\xstep,9*\ystep) {\large  $B$};
				
				
			\end{scope}
			

			
			\node[] (S11) at (0*\xstep,0.15) {\normalsize $t_A$};
			\node[] (S12) at (0*\xstep,\numevents * \ystep) {};
			\node[] (S21) at (1*\xstep,0.15) {\normalsize $t_B$};
			\node[] (S22) at (1*\xstep,\numevents * \ystep) {};
			
			\draw[seqtrace] (S11) to (S12);
			\draw[seqtrace] (S21) to (S22);
			
			
			\node[event, draw=black, fill=white] (11) at (0*\xstep, 1*\ystep + 0*\ybias) {$\acq(\LockColorTwo{\lk_2})$};
			\node[event, draw=black, fill=white] (12) at (0*\xstep, 2*\ystep + 0*\ybias) {$\acq(\LockColorOne{\lk_1})$};
			\node[event, draw=black, fill=white, dotted] (13) at (0*\xstep, 3*\ystep + 0*\ybias) {$\cs(m_0,m_1)$};
			%\node[event, draw=black, fill=white, dotted] (14) at (0*\xstep, 4*\ystep + 0*\ybias) {$\acq(\lk')$};
			%\node[event, draw=black, fill=white, dotted] (15) at (0*\xstep, 5*\ystep + 0*\ybias) {$\rel(\lk')$};
			%\node[event, draw=black, fill=white, dotted] (16) at (0*\xstep, 6*\ystep + 0*\ybias) {$\rel(\lk)$};
			\node[event, draw=black, fill=white] (17) at (0*\xstep, 4*\ystep + 0*\ybias) {$\rel(\LockColorOne{\lk_1})$};
			\node[event, draw=black, fill=white] (18) at (0*\xstep, 5*\ystep + 0*\ybias) {$\rel(\LockColorTwo{\lk_2})$};
			
			\node[event, draw=black, fill=white] (19) at (0*\xstep, 6*\ystep + 1*\ybias) {$\acq(\LockColorOne{\lk_1})$};
			\node[event, draw=black, fill=white, dotted] (110) at (0*\xstep, 7*\ystep + 1*\ybias) {$\cs(m_0,m_1)$};
			%\node[event, draw=black, fill=white, dotted] (111) at (0*\xstep, 11*\ystep + 1*\ybias) {$\acq(\lk')$};
			%\node[event, draw=black, fill=white, dotted] (112) at (0*\xstep, 12*\ystep + 1*\ybias) {$\rel(\lk')$};
			%\node[event, draw=black, fill=white, dotted] (113) at (0*\xstep, 13*\ystep + 1*\ybias) {$\rel(\lk)$};
			\node[event, draw=black, fill=white] (114) at (0*\xstep, 8*\ystep + 1*\ybias) {$\rel(\LockColorOne{\lk_1})$};
			
			%%%%%%%%%%%%%%%%
			
			\node[event, draw=black, fill=white] (21) at (1*\xstep, 2*\ystep + 0*\ybias) {$\acq(\LockColorOne{\lk_1})$};
			\node[event, draw=black, fill=white, dotted] (22) at (1*\xstep, 3*\ystep + 0*\ybias) {$\cs(m_1,m_0)$};
			%\node[event, draw=black, fill=white, dotted] (23) at (1*\xstep, 4*\ystep + 0*\ybias) {$\acq(\lk)$};
			%\node[event, draw=black, fill=white, dotted] (24) at (1*\xstep, 5*\ystep + 0*\ybias) {$\rel(\lk)$};
			%\node[event, draw=black, fill=white, dotted] (25) at (1*\xstep, 6*\ystep + 0*\ybias) {$\rel(\lk')$};
			\node[event, draw=black, fill=white] (26) at (1*\xstep, 4*\ystep + 0*\ybias) {$\rel(\LockColorOne{\lk_1})$};
			
			\node[event, draw=black, fill=white] (27) at (1*\xstep, 5*\ystep + 1*\ybias) {$\acq_1(\LockColorTwo{\lk_2})$};
			\node[event, draw=black, fill=white, dotted] (28) at (1*\xstep, 6*\ystep + 1*\ybias) {$\cs(m_1,m)$};
			%\node[event, draw=black, fill=white, dotted] (29) at (1*\xstep, 11*\ystep + 1*\ybias) {$\acq(\lk)$};
			%\node[event, draw=black, fill=white, dotted] (210) at (1*\xstep, 12*\ystep + 1*\ybias) {$\rel(\lk)$};
			%\node[event, draw=black, fill=white, dotted] (211) at (1*\xstep, 13*\ystep + 1*\ybias) {$\rel(\lk')$};
			\node[event, draw=black, fill=white] (212) at (1*\xstep, 7*\ystep + 1*\ybias) {$\rel_1(\LockColorTwo{\lk_2})$};
			
			\begin{scope}[]
				\node[below left=of 212] (212b) {};
			\end{scope}
	
		\end{tikzpicture}
	}
	\caption{
		Reduction for OV-hardness proof from an instance of size $n = 2$ and $d = 2$.
		% We use the shortcut $\cs(\lk_i,\lk_j)$ to denote two nested critical sections on $\lk_i$ and $\lk_j$, as indicated in the figure.
	}
	\figlabel{ov-hardness}
\end{subfigure}
%\end{figure}
	
	\myparagraph{Construction}{
		We will construct a trace $\tr$ such that $\tr$ has a deadlock
		pattern of length $k$ iff $(A_1, A_2, \ldots, A_k)$ is a positive $k$-OV instance.
		The trace $\tr$ is of the form $\tr = \tr^{A_1} \cdot \tr^{A_2} \cdots \tr^{A_k}$
		and uses $k$ threads $\set{t_{A_1}, \ldots, t_{A_k}}$ and $d+k$ distinct 
		locks $\lk_1, \ldots, \lk_d, m_1, m_2, \ldots, m_k$. 
		Intuitively, the sub-trace $\tr^{A_i}$ encodes the given set of vectors $A_i$.
		The sub-traces $\tr^{A_i} = \tr^{A_i}_1 \cdot \tr^{A_i}_2 \cdots \tr^{A_i}_n$
		% and $\tr^B = \tr^B_1 \cdot \tr^B_2 \cdots \tr^B_n$ 
		are defined as follows.
		For each $j \in \set{1, 2, \ldots, n}$
		the sub-trace $\tr^{A_i}_j$ is the unique string generated by the
		grammar having $d+1$ non-terminals $S_0, S_1, \ldots, S_d$, start symbol $S_d$
		and the following production rules:
		\begin{itemize}
			\item $S_0 \to \ev{t_Z, \acq(m_i)} \cdot \ev{t_Z, \acq(m_{i \% k + 1})} \cdot \ev{t_Z, \rel(m_{i \% k + 1})} \cdot \ev{t_Z, \rel(m_i)}$.
			% where $(m, m') = (\lk, \lk')$ if $Z = A$, and $(m, m') = (\lk', \lk)$.
			\item for each $1 \leq p \leq d$, $S_p \to S_{p-1}$ if $A_{i, j}[p] = 0$.
			Otherwise (if $A_{i, j}[p] =1$), $S_p \to \ev{t_{A_i}, \acq(\lk_p)} \cdot S_{p-1} \cdot \ev{t_{A_i}, \rel(\lk_p)}$.
		\end{itemize}
		In words, all events of $\tr^{A_i}$ are performed by thread $t_{A_i}$.
		 % and those in $\tr^B$ are performed by $t_B$.
		Next, the $j^{th}$ sub-trace of $\tr^{A_i}$, denoted $\tr^{A_i}_j$ corresponds to the vector $A_{i, j}$
		as follows --- $\tr^{A_i}_j$ is a nested block of critical sections,
		with the innermost critical section being on lock $m_{i \% k + 1}$,
		which is immediately enclosed in a  critical section on lock $m_i$.
		Further, in the sub-trace $\tr^{A_i}_j$, 
		the lock $\lk_p$ occurs iff $A_{i, j}[p] = 1$.
		% The sub-traces $\tr^B_i$ is similarly constructed, except that the order
		% of the two innermost critical sections is inverted.
		\figref{ov-hardness} illustrates the construction for an OV-instance with $k=2$ $n=2$ and $d=2$.
	}
\end{proof}

%!TEX root = main.tex


\begin{proof}
	We show a fine-grained reduction from the Orthogonal Vectors Problem to
	the problem of checking for deadlock patterns of size $2$.
	For this, we start with two sets
	$A, B \subseteq \set{0, 1}^d$
	of $d$-dimensional vectors with $|A| = |B| = n$.
	We write the $i^{th}$ vector in $A$ as $A_i$ and that in $B$ as $B_i$.
	
	%%!TEX root = ../main.tex
%\begin{figure}
\begin{subfigure}[b]{0.475\textwidth}
	\newcommand{\xdisposition}{0}
	\newcommand{\ydisposition}{0}
	\newcommand{\xtstep}{0.75}
	\newcommand{\ytstep}{1}
	\newcommand{\ybias}{-0.3 }
	\newcommand{\xstep}{2.5}
	\newcommand{\ystep}{-0.475}
	\newcommand{\xtscale}{0.8}
	\def \numevents{9.5}
	\newcommand{\eventA}[4]{
		\node[event, draw=black, fill=white] (A#1) at (#1*\xstep, #2*\ystep) {\footnotesize $#2(x_{#3})$};
		%\node[] at (0*\xstep-\xtstep, {#1*\ystep}) {\small $#2(x_{#3})$};
	}
	\scalebox{0.9}{
		\begin{tikzpicture}[thick,
			pre/.style={<-,shorten >= 2pt, shorten <=2pt, very thick},
			post/.style={->,shorten >= 3pt, shorten <=3pt,   thick},
			seqtrace/.style={line width=2},
			und/.style={very thick, draw=gray},
			event/.style={rectangle, minimum height=0.8mm, minimum width=15mm,  line width=1pt, inner sep=0.5,},
			virt/.style={circle,draw=black!50,fill=black!20, opacity=0}]
			\footnotesize
			
			
%			\node[] at (0.5*\xstep, -1.5*\ystep){
%				\small
%				$\cs(\lk_i,\lk_j)=\acq(\lk_i) \cdot\acq(\lk_j) \cdot\rel(\lk_j) \cdot \rel(\lk_i)$
%			};
			

				
			\begin{scope}[shift={(0,-3*\ystep)}]
				
				\draw[dashed] (-0.9*\xstep,6.5*\ystep) rectangle (-0.5*\xstep,8.5*\ystep);
				\node (A1) at (-0.7*\xstep,7*\ystep) {\normalsize [1, 1]};
				\node (A2) at (-0.7*\xstep,8*\ystep) {\normalsize [1, 0]};
				\node (A) at (-0.7*\xstep,9*\ystep) {\large  $A$};
				
				\draw[dashed] (1.5*\xstep,6.5*\ystep) rectangle (1.9*\xstep,8.5*\ystep);
				\node (B1) at (1.7*\xstep,7*\ystep) {\normalsize [1, 0]};
				\node (B2) at (1.7*\xstep,8*\ystep) {\normalsize [0, 1]};
				\node (B) at (1.7*\xstep,9*\ystep) {\large  $B$};
				
				
			\end{scope}
			

			
			\node[] (S11) at (0*\xstep,0.15) {\normalsize $t_A$};
			\node[] (S12) at (0*\xstep,\numevents * \ystep) {};
			\node[] (S21) at (1*\xstep,0.15) {\normalsize $t_B$};
			\node[] (S22) at (1*\xstep,\numevents * \ystep) {};
			
			\draw[seqtrace] (S11) to (S12);
			\draw[seqtrace] (S21) to (S22);
			
			
			\node[event, draw=black, fill=white] (11) at (0*\xstep, 1*\ystep + 0*\ybias) {$\acq(\LockColorTwo{\lk_2})$};
			\node[event, draw=black, fill=white] (12) at (0*\xstep, 2*\ystep + 0*\ybias) {$\acq(\LockColorOne{\lk_1})$};
			\node[event, draw=black, fill=white, dotted] (13) at (0*\xstep, 3*\ystep + 0*\ybias) {$\cs(m_0,m_1)$};
			%\node[event, draw=black, fill=white, dotted] (14) at (0*\xstep, 4*\ystep + 0*\ybias) {$\acq(\lk')$};
			%\node[event, draw=black, fill=white, dotted] (15) at (0*\xstep, 5*\ystep + 0*\ybias) {$\rel(\lk')$};
			%\node[event, draw=black, fill=white, dotted] (16) at (0*\xstep, 6*\ystep + 0*\ybias) {$\rel(\lk)$};
			\node[event, draw=black, fill=white] (17) at (0*\xstep, 4*\ystep + 0*\ybias) {$\rel(\LockColorOne{\lk_1})$};
			\node[event, draw=black, fill=white] (18) at (0*\xstep, 5*\ystep + 0*\ybias) {$\rel(\LockColorTwo{\lk_2})$};
			
			\node[event, draw=black, fill=white] (19) at (0*\xstep, 6*\ystep + 1*\ybias) {$\acq(\LockColorOne{\lk_1})$};
			\node[event, draw=black, fill=white, dotted] (110) at (0*\xstep, 7*\ystep + 1*\ybias) {$\cs(m_0,m_1)$};
			%\node[event, draw=black, fill=white, dotted] (111) at (0*\xstep, 11*\ystep + 1*\ybias) {$\acq(\lk')$};
			%\node[event, draw=black, fill=white, dotted] (112) at (0*\xstep, 12*\ystep + 1*\ybias) {$\rel(\lk')$};
			%\node[event, draw=black, fill=white, dotted] (113) at (0*\xstep, 13*\ystep + 1*\ybias) {$\rel(\lk)$};
			\node[event, draw=black, fill=white] (114) at (0*\xstep, 8*\ystep + 1*\ybias) {$\rel(\LockColorOne{\lk_1})$};
			
			%%%%%%%%%%%%%%%%
			
			\node[event, draw=black, fill=white] (21) at (1*\xstep, 2*\ystep + 0*\ybias) {$\acq(\LockColorOne{\lk_1})$};
			\node[event, draw=black, fill=white, dotted] (22) at (1*\xstep, 3*\ystep + 0*\ybias) {$\cs(m_1,m_0)$};
			%\node[event, draw=black, fill=white, dotted] (23) at (1*\xstep, 4*\ystep + 0*\ybias) {$\acq(\lk)$};
			%\node[event, draw=black, fill=white, dotted] (24) at (1*\xstep, 5*\ystep + 0*\ybias) {$\rel(\lk)$};
			%\node[event, draw=black, fill=white, dotted] (25) at (1*\xstep, 6*\ystep + 0*\ybias) {$\rel(\lk')$};
			\node[event, draw=black, fill=white] (26) at (1*\xstep, 4*\ystep + 0*\ybias) {$\rel(\LockColorOne{\lk_1})$};
			
			\node[event, draw=black, fill=white] (27) at (1*\xstep, 5*\ystep + 1*\ybias) {$\acq_1(\LockColorTwo{\lk_2})$};
			\node[event, draw=black, fill=white, dotted] (28) at (1*\xstep, 6*\ystep + 1*\ybias) {$\cs(m_1,m)$};
			%\node[event, draw=black, fill=white, dotted] (29) at (1*\xstep, 11*\ystep + 1*\ybias) {$\acq(\lk)$};
			%\node[event, draw=black, fill=white, dotted] (210) at (1*\xstep, 12*\ystep + 1*\ybias) {$\rel(\lk)$};
			%\node[event, draw=black, fill=white, dotted] (211) at (1*\xstep, 13*\ystep + 1*\ybias) {$\rel(\lk')$};
			\node[event, draw=black, fill=white] (212) at (1*\xstep, 7*\ystep + 1*\ybias) {$\rel_1(\LockColorTwo{\lk_2})$};
			
			\begin{scope}[]
				\node[below left=of 212] (212b) {};
			\end{scope}
	
		\end{tikzpicture}
	}
	\caption{
		Reduction for OV-hardness proof from an instance of size $n = 2$ and $d = 2$.
		% We use the shortcut $\cs(\lk_i,\lk_j)$ to denote two nested critical sections on $\lk_i$ and $\lk_j$, as indicated in the figure.
	}
	\figlabel{ov-hardness}
\end{subfigure}
%\end{figure}
	
	\myparagraph{Construction}{
		We will construct a trace $\tr$ such that $\tr$ has a deadlock
		pattern of length $2$ iff $(A, B)$ is a positive OV instance.
		The trace $\tr$ is of the form $\tr = \tr^A \cdot \tr^B$
		and uses $2$ threads $\set{t_A, t_B}$ and $d+2$ distinct locks $\lk_1, \ldots, \lk_d, m_0, m_1$. 
		Intuitively, $\tr^A$ and $\tr^B$  encode the given sets of vectors $A$ and $B$.
		The sub-traces $\tr^A = \tr^A_1 \cdot \tr^A_2 \cdots \tr^A_n$
		and $\tr^B = \tr^B_1 \cdot \tr^B_2 \cdots \tr^B_n$ are defined as follows.
		For each $i \in \set{1, 2, \ldots, n}$ and $Z \in \set{A, B}$, the
		sub-trace $\tr^Z_i$ is the unique string generated by the
		grammar having $d+1$ non-terminals $S_0, S_1, \ldots, S_d$, start symbol $S_d$
		and the following production rules:
		\begin{itemize}
			\item $S_0 \to \ev{t_Z, \acq(m)} \cdot \ev{t_Z, \acq(m')} \cdot \ev{t_Z, \rel(m')} \cdot \ev{t_Z, \rel(m)}$,
			where $(m, m') = (m_0, m_1)$ if $Z = A$, and $(m, m') = (m_1, m_0)$.
			\item for each $1 \leq j \leq d$, $S_j \to S_{j-1}$ if $Z_i[j] = 0$.
			Otherwise (if $Z_i[j] =1$), $S_j \to \ev{t_Z, \acq(\lk_j)} \cdot S_{j-1} \cdot \ev{t_Z, \rel(\lk_j)}$.
		\end{itemize}
		In words, all events of $\tr^A$ are performed by thread $t_A$ and those
		in $\tr^B$ are performed by $t_B$.
		Next, the $i^{th}$ sub-trace of $\tr^A$, denoted $\tr^A_i$ corresponds to the vector $A_i$
		as follows --- $\tr^A_i$ is a nested block of critical sections,
		with the innermost critical section being on lock $\lk'$,
		which is immediately enclosed in a  critical section on lock $\lk$.
		Further, in the sub-trace $\tr^A_i$, the lock $\lk_j$
		occurs iff $A_i[j] = 1$.
		The sub-traces $\tr^B_i$ is similarly constructed, except that the order
		of the two innermost critical sections is inverted.
		\figref{ov-hardness} illustrates the construction for an OV-instance with $n=2$ and $d=2$.}
\end{proof}



\myparagraph{The complexity of deadlock prediction}{
Finally, we settle the complexity of the prediction problem for deadlocks,
and show that, even for deadlock patterns of size $2$, the problem is $\W{1}$-hard
parameterized by the number of threads.
In contrast, recall that the $\W{1}$-hardness of 
\thmref{pattern-w1-hardness-pattern} concerns deadlock patterns of arbitrary size.
Our result is based on a similar hardness that was established recently for predicting data races~\cite{Mathur2020b}.


\begin{restatable}{theorem}{wonehardness}
\thmlabel{w1-hardness-pattern}
The problem of checking if a trace $\tr$ has a predictable deadlock of size $2$
is $\W{1}$-hard in the number of threads $\NumThreads$ appearing in $\tr$, and thus
is also $\NP$-hard.
\end{restatable}
}