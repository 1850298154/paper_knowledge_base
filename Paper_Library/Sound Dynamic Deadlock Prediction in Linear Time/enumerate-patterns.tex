%!TEX root = main.tex

\subsection{The Algorithm \SyncPDOffline}
\seclabel{enumerate-patterns}

\begin{comment}
As with prior work, we model the problem of finding deadlock patterns as a cycle-detection problem
over a graph. % $\lkevgraph{\tr}$ whose nodes are elements of the trace.
%However, instead of a naive lock graph (with locks as nodes),
%the nodes in our graph store additional metadata that helps us
%to rule out false positives, and eventually enumerate all deadlock
%patterns, for which we need information about concrete dynamic events in the trace.
However, instead of considering locks as vertices (as in prior works~\cite{Bensalem2006,Bensalem2005}),
we use an \emph{abstract lock graph} $\lkevgraph{\tr}$.
% of the graph, our 
% representation, called \emph{abstract lock graph} $\lkevgraph{\tr}$ 
% effectively exploits \lemref{abstract-pattern-linear-time}.
% The key novelty compared to such graphs in the literature
% comes from combining it with the notion of sync-preserving deadlocks.
Every node of $\lkevgraph{\tr}$ is an abstract acquire of $\tr$, and thus
every deadlock pattern in $\tr$ appears in some abstract deadlock pattern defined by a cycle in $\lkevgraph{\tr}$.
Hence, we can detect \emph{all} sync-preserving deadlocks in $\tr$ by enumerating each of the cycles of $\lkevgraph{\tr}$, and if it constitutes an abstract deadlock pattern, use \lemref{abstract-pattern-linear-time} to check if it contains a sync-preserving deadlock in $\Otilde(\NumEvents)$ time.
% Hence, in linear time, our algorithm checks a whole abstract deadlock pattern, that typically summarizes polynomially many concrete deadlock patterns. 
\end{comment}

We now present the final ingredients of \SyncPDOffline.
We construct the \emph{abstract lock graph},
enumerate cycles in it, check whether
any cycle is an abstract deadlock pattern,
and if so, whether it contains sync-preserving deadlocks.

%!TEX root=../main.tex

\setlength{\textfloatsep}{10pt}
\begin{figure}[t]
%\begin{wrapfigure}{r}{0.35\textwidth}
\begin{subfigure}[t]{0.2\textwidth}
\scalebox{0.84}{
\begin{tikzpicture}[yscale=0.9]
	\tikzset{rectangle/.append style={draw=black}}
		\def\xstep{2.2}
		\def\ystep{0.5}
		\node[rectangle, align=center] (t1) at (-1, -7){
			$t_1, \LockColorTwo{\lk_2}, \{ \LockColorOne{\lk_1}\}, \sequence{e_{2}}$
		};
		\node[rectangle, align=center] (t31) at (-1, -8.7){
			$t_2, \LockColorOne{\lk_1}, \{ \LockColorTwo{\lk_2} \}, \sequence{e_{8}}$
		};
		\draw[->, thick, -Latex, bend right=10] ([xshift=-3mm]t1.south) to ([xshift=-3mm]t31.north);% (t31.north);
		\draw[->, thick, -Latex, bend right=10] (t31) to (t1);
\end{tikzpicture}
}
\end{subfigure}
~~~~
\begin{subfigure}[t]{0.2\textwidth}
\scalebox{0.84}{
\begin{tikzpicture}[yscale=0.9]
	\tikzset{rectangle/.append style={draw=black}}
		\def\xstep{2.2}
		\def\ystep{0.5}
		\node[rectangle, align=center] (t1) at (-1, -7){
			$t_2, \LockColorThree{\lk_3}, \{ \LockColorTwo{\lk_2}\}, \sequence{e_{4}}$
		};
		\node[rectangle, align=center] (t31) at (-1, -8.7){
			$t_3, \LockColorTwo{\lk_2}, \{ \LockColorThree{\lk_3} \}, \sequence{e_{18}}$
		};
		\draw[->, thick, -Latex, bend right=10] ([xshift=-3mm]t1.south) to ([xshift=-3mm]t31.north);% (t31.north);
		\draw[->, thick, -Latex, bend right=10] (t31) to (t1);
\end{tikzpicture}
}
\end{subfigure}
~~~~
% \hspace*{\fill}
\begin{subfigure}[t]{0.4\textwidth}
\scalebox{0.84}{
\begin{tikzpicture}[yscale=0.9]
	\tikzset{rectangle/.append style={draw=black}}
		\node[rectangle, align=center] (t1) at (-1, -7){
					$t_1, \LockColorTwo{\lk_2}, \{ \LockColorOne{\lk_1} \}$,
					$\sequence{e_2, e_4, e_{29}}$
				};
				\node[rectangle,align=center] (t2) at (2.5, -7){
					$t_2, \LockColorOne{\lk_1}, \{ \LockColorFour{\lk_4} \}$,
					$\sequence{e_{23}}$
				};
				\node[rectangle, align=center] (t31) at (-1, -8.7){
					$t_3, \LockColorOne{\lk_1}, \{ \LockColorTwo{\lk_2} \}$,
					$\sequence{e_{16}, e_{19}}$
				};
				\node[rectangle,align=center] (t32) at (2.5, -8.7){
					$t_3, \LockColorThree{\lk_3}, \{ \LockColorTwo{\lk_2} \}$,
					$\sequence{e_{13}}$
				};
				\draw[->, thick, -Latex] (t2) -- (t1);
				\draw[->, thick, -Latex] ([xshift=6mm]t1.south) to (t32.west);
				\draw[->, thick, -Latex, bend right=10] ([xshift=-3mm]t1.south) to ([xshift=-3mm]t31.north);
				\draw[->, thick, -Latex, bend right=10] (t31) to (t1);
\end{tikzpicture}
}
\end{subfigure}
% 
% \caption{
% A  trace $\tr$ with a sync-preserving deadlock, stalling $t_2$ on $e_4$ and $t_3$ on $e_{18}$.
% Underlined events indicate the slice of $\tr$ that forms a sync-preserving lower set of $\tr$,
% and serves as a valid witness for the deadlock.
% \figlabel{motivating}
% }
\caption{
Abstract lock graphs of the traces from~\figref{motivating-no-dl} (left), \figref{motivating-dl} (middle) and \figref{syncp_example} (right).
% \Andreas{@Umang: You can use this space for more figures, examples}
\figlabel{motivating-graph}
}
 \end{figure}
%\end{wrapfigure}
\myparagraph{Abstract lock graph}{
The abstract lock graph of $\tr$ is a directed graph
$
\lkevgraph{\tr} = (V_\tr, E_\tr)
$,
where
\begin{itemize}
	\item $V_\tr=\{\tuple{t_1, \lk_1, L_1, F_1},\dots, \tuple{t_k, \lk_k, L_k, F_k}\}$ is the set of abstract acquires of $\tr$, and
	\item for every $\eta_1 {=} \tuple{t_1, \lk_1, L_1, F_1}, 
	\eta_2 {=} \tuple{t_2, \lk_2, L_2, F_2} \in V_\tr$, we have  $(\eta_1, \eta_2) \in E_\tr$ iff 
	$t_1 \neq t_2$, $\lk_1 \in L_2$, and $L_1 \cap L_2 = \emptyset$.
\end{itemize}
A node $\tuple{t_1, \lk_1, L_1, F_1}$ signifies that there is an event $\acq_1(\lk_1)$ performed by thread $t_1$ while holding the locks in $L_1$.
The last component $F_1$ is a list which contains all such events $\acq_1$ in order of appearance in $\tr$.
An edge $(\eta_1, \eta_2)$ signifies that the lock $\lk_1$ acquired by each of the
events $\acq_1 \in F_1$ was held by $t_2$ when it executed each of $\acq_2\in F_2$ while not holding a common lock.
%See \figref{lock_graph} for the lock graph to the trace $\tr_2$ of \figref{syncp_example}.
The abstract lock graph can be constructed incrementally
as new events appear in $\tr$.
For $\NumEvents$ events, $\NumLocks$ locks and nesting depth $\NestingDepth$,
the graph has
$|V_\tr| = O\big(\NumThreads \cdot \NumLocks^{\NestingDepth}\big)$ vertices,
$|E_\tr| = O(|V_\tr|\cdot \NumLocks^{\NestingDepth - 1})$ edges and can be
constructed in $O(\NumEvents \cdot \NestingDepth)$ time.
See \figref{motivating-graph} for examples.
In the left graph, the cycle marks an abstract deadlock pattern and its single concrete deadlock pattern
$\abst{D}=\{ e_2 \} \times \{ e_{8} \}$, and similarly for the middle graph where $\abst{D}=\{ e_4 \} \times \{ e_{18} \}$.
%In the graph on the right, we have the abstract lock graph of the trace $\tr_3$ of  \figref{syncp_example}.
In the right graph, there is a unique cycle which marks an abstract deadlock pattern of $6$ concrete deadlock patterns
$\abst{D}=\{ e_2, e_4, e_{29} \} \times \{ e_{16}, e_{19} \}$.


% \begin{example}
% \figref{lock_graph} illustrates the lock graph of the trace $\tr_2$ in \figref{syncp_example}.
% We have a unique cycle, of length $2$, that marks a corresponding abstract deadlock pattern of $6$ concrete deadlock patterns
% $\abst{D}=\{ e_2, e_4, e_{29} \} \times \{ e_{16}, e_{19} \}$.
% \end{example}
}
%!TEX root = main.tex
\setlength{\textfloatsep}{10pt}
\begin{algorithm}[h]
\small
\DontPrintSemicolon
\SetInd{0.4em}{0.4em}
\KwIn{
A trace $\tr$.
}
\KwOut{All abstract deadlock patterns of $\tr$ that contain a sync-preserving deadlock.}
\BlankLine
Construct the abstract lock graph $\lkevgraph{\tr}$\\
\ForEach{cycle $C=\tuple{\eta_0,\dots, \eta_{k-1}}$ in $G$}{
Let $\eta_i=\tuple{t_i, \lk_i, L_i, F_i}$\\
\uIf(\tcp*[h]{$C$ is an abstract deadlock pattern}){$\forall i\neq j$ we have $t_i\neq t_j$ and $\lk_i\neq \lk_j$ and $L_i\cap L_j=\emptyset$ }{
%
\lIf{$\checkAbsDeadP(C)$}{
%Report the deadlock pattern in $C$ that constitutes a sync-preserving deadlock
Report that $C$ contains a sync-preserving deadlock
}
}
}
\caption{
Algorithm $\SyncPDOffline$.
\algolabel{offline}
}
\end{algorithm}
\normalsize

% \vspace{-0.2in}
\myparagraph{Algorithm $\SyncPDOffline$}{
	It is straightforward to verify that every abstract deadlock pattern of $\tr$ appears as a (simple) cycle in $\lkevgraph{\tr}$.
	However, the opposite is not true.
	A cycle $C = \eta_0, \eta_1, \ldots, \eta_{k-1}$ of $\lkevgraph{\tr}$,
	where $\eta_i = \tuple{t_i, \lk_i, L_i, F_i}$ defines an abstract deadlock pattern
	if additionally every thread $t_i$ is distinct, all every lock $\lk_i$ is distinct, and all sets $L_i$ are pairwise disjoint.
	This gives us a simple recipe for enumerating all abstract deadlock patterns, by using
	Johnson's algorithm~\cite{Johnson1975} to enumerate
	every simple cycle $C$ in $\lkevgraph{\tr}$, and check whether $C$ is an abstract deadlock pattern.
	We thus arrived at our offline algorithm $\SyncPDOffline$ (\algoref{offline}).
	The running time depends linearly on the length of $\tr$
	and the number of cycles in $\lkevgraph{\tr}$.% (\thmref{syncp_offline}).
}




\begin{restatable}{theorem}{syncpoffline}
\thmlabel{syncp_offline}
Consider a trace $\tr$ of $\NumEvents$ events, $\NumThreads$ threads and $\textsf{Cyc}_\tr$ cycles in $\lkevgraph{\tr}$.
The algorithm $\SyncPDOffline$ reports all sync-preserving deadlocks of $\tr$ in time $O(\NumEvents \cdot \NumThreads\cdot \textsf{Cyc}_\tr)$.
% When the number of threads and lock-nesting depth is small, $\SyncPDOffline$ runs in $\Otilde(\NumEvents \cdot \poly{\NumLocks})$ time.
\end{restatable}

Although, in principle, we can have exponentially many cycles in $\lkevgraph{\tr}$,
because the nodes of $\lkevgraph{\tr}$ are \emph{abstract} acquire events (as opposed to \emph{concrete}), we expect that the number of cycles (and thus abstract deadlock patterns) in $\lkevgraph{\tr}$ remains small, even though the number of \emph{concrete} deadlock patterns can grow exponentially.
Since $\SyncPDOffline$ spends linear time per abstract deadlock pattern, we  have an efficient procedure overall 
for constant $\NumThreads$ and $\NumLocks$.
We evaluate $\textsf{Cyc}_\tr$ experimentally in \secref{experiments}, and confirm that it is very small compared to the number of concrete deadlock patterns in $\tr$.
%\hunkar{
Nevertheless, $\textsf{Cyc}_\tr$ can become exponential when $\NumThreads$ and $\NumLocks$ are large, making~\cref{algo:offline} run in exponential time.
Note that this barrier is unavoidable in general, as proven in \thmref{pattern-w1-hardness-pattern}.
%}

%!TEX root = main.tex

\begin{example}
    \exlabel{ex4}
    We illustrate how the lock graph is integrated inside $\SyncPDOffline$.	
    Consider the trace $\tr_3$ in \figref{syncp_example}.
    It contains $6$ concrete deadlock patterns $D_1 \ldots D_6$.
    A naive algorithm would enumerate each pattern explicitly until it finds a deadlock. 
    However, the tight interplay between the abstract lock graph and sync-preservation enables a more efficient procedure.
    $\SyncPDOffline$ starts by computing the sync-preserving closure of $D_1$,
    $\SPClosure{\tr_3}(\prev{\tr_3}(\{ e_{2},e_{16}\}))=\set{e_1, \ldots, e_6, \, e_8, \ldots, e_{15}}$.
    As $e_2 \in \SPClosure{\tr_3}(\prev{\tr_3}(\{ e_{2},e_{16}\}))$, we conclude that $D_1$ is not a sync-preserving deadlock.
    The algorithm further deduces that the deadlock patterns $D_2$, $D_3$ and $D_4$ are also not sync-preserving deadlocks, as follows.
    $D_2=\pattern{e_2, e_{19}}$ shares a common event $e_2$ with $D_1$ but contains the event $e_{19}$ instead of $e_{16}$, while $e_5 \in \SPClosure{\tr_3}(\prev{\tr_3}(\{ e_{2},e_{16}\}))$.
    Since $e_{16} \tho{\tr_3} e_{19}$, and the sync-preserving closure grows monotonically (\propref{spclosure-monotone}), the sync-preserving closure of $e_2$ and $e_{19}$ will also contain $e_5$ (and thus $e_2$).
    Therefore, $D_2$ cannot be a sync-preserving deadlock.
    This reasoning is formalized in \corref{pattern-monotone}, and also applies to $D_3$ and $D_4$.
    Next, the algorithm proceeds with $D_5$.
    The above reasoning does not hold for $D_5$ as $\SPClosure{\tr_3}(\prev{\tr_3}(\{ e_{2},e_{16}\})) \cap S_5 = \emptyset$ 
    where $S_5=\set{e_{29}, e_{16}}$.
    The algorithm then computes the sync-preserving closure of $D_5$, reports a deadlock (\exref{syncp-ex}) and stops analyzing this abstract deadlock pattern.
    %Hence, $D_6$ is also not considered.
    %As the remaining pattern $D_6$ originates from the same abstract deadlock pattern, it need not be considered.
    In the end, we have only explicitly enumerated the deadlock patterns $D_1$ and $D_5$.
\end{example}

\begin{remark}
Although the concept of lock graphs exists in the literature~\cite{Havelund2000,Bensalem2005,Cai2020,cai14magiclock},
our notion of \emph{abstract} lock graphs is novel and tailored to sync-preserving deadlocks.
The closest concept to abstract lock graphs is that of equivalent cycles~\cite{cai14magiclock}. 
However, equivalent cycles unify all the concrete patterns of a given abstract pattern and lead to unsound deadlock detection, which was indeed their use. 
\end{remark}