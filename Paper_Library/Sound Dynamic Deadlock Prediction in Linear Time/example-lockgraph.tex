%!TEX root = main.tex

\begin{example}
    \exlabel{ex4}
    We illustrate how the lock graph is integrated inside $\SyncPDOffline$.	
    Consider the trace $\tr_3$ in \figref{syncp_example}.
    It contains $6$ concrete deadlock patterns $D_1 \ldots D_6$.
    A naive algorithm would enumerate each pattern explicitly until it finds a deadlock. 
    However, the tight interplay between the abstract lock graph and sync-preservation enables a more efficient procedure.
    $\SyncPDOffline$ starts by computing the sync-preserving closure of $D_1$,
    $\SPClosure{\tr_3}(\prev{\tr_3}(\{ e_{2},e_{16}\}))=\set{e_1, \ldots, e_6, \, e_8, \ldots, e_{15}}$.
    As $e_2 \in \SPClosure{\tr_3}(\prev{\tr_3}(\{ e_{2},e_{16}\}))$, we conclude that $D_1$ is not a sync-preserving deadlock.
    The algorithm further deduces that the deadlock patterns $D_2$, $D_3$ and $D_4$ are also not sync-preserving deadlocks, as follows.
    $D_2=\pattern{e_2, e_{19}}$ shares a common event $e_2$ with $D_1$ but contains the event $e_{19}$ instead of $e_{16}$, while $e_5 \in \SPClosure{\tr_3}(\prev{\tr_3}(\{ e_{2},e_{16}\}))$.
    Since $e_{16} \tho{\tr_3} e_{19}$, and the sync-preserving closure grows monotonically (\propref{spclosure-monotone}), the sync-preserving closure of $e_2$ and $e_{19}$ will also contain $e_5$ (and thus $e_2$).
    Therefore, $D_2$ cannot be a sync-preserving deadlock.
    This reasoning is formalized in \corref{pattern-monotone}, and also applies to $D_3$ and $D_4$.
    Next, the algorithm proceeds with $D_5$.
    The above reasoning does not hold for $D_5$ as $\SPClosure{\tr_3}(\prev{\tr_3}(\{ e_{2},e_{16}\})) \cap S_5 = \emptyset$ 
    where $S_5=\set{e_{29}, e_{16}}$.
    The algorithm then computes the sync-preserving closure of $D_5$, reports a deadlock (\exref{syncp-ex}) and stops analyzing this abstract deadlock pattern.
    %Hence, $D_6$ is also not considered.
    %As the remaining pattern $D_6$ originates from the same abstract deadlock pattern, it need not be considered.
    In the end, we have only explicitly enumerated the deadlock patterns $D_1$ and $D_5$.
\end{example}