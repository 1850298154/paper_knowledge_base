%!TEX root = main.tex

\subsection{Proof of \thmref{pattern-ov-hardness}}
\seclabel{ov-hardness}




\patternovhardness*
\begin{proof}
	
	We show a fine-grained reduction from the Orthogonal Vectors Problem to
	the problem of checking for deadlock patterns of size $k$.
	For this, we start with two sets
	$A_1, A_2, \ldots, A_k \subseteq \set{0, 1}^d$
	of $d$-dimensional vectors with $|A_i| = n$ for every $1 \leq i \leq k$.
	We write the $j^{th}$ vector in $A_i$ as $A_{i, j}$.
	
	%%!TEX root = ../main.tex
%\begin{figure}
\begin{subfigure}[b]{0.475\textwidth}
	\newcommand{\xdisposition}{0}
	\newcommand{\ydisposition}{0}
	\newcommand{\xtstep}{0.75}
	\newcommand{\ytstep}{1}
	\newcommand{\ybias}{-0.3 }
	\newcommand{\xstep}{2.5}
	\newcommand{\ystep}{-0.475}
	\newcommand{\xtscale}{0.8}
	\def \numevents{9.5}
	\newcommand{\eventA}[4]{
		\node[event, draw=black, fill=white] (A#1) at (#1*\xstep, #2*\ystep) {\footnotesize $#2(x_{#3})$};
		%\node[] at (0*\xstep-\xtstep, {#1*\ystep}) {\small $#2(x_{#3})$};
	}
	\scalebox{0.9}{
		\begin{tikzpicture}[thick,
			pre/.style={<-,shorten >= 2pt, shorten <=2pt, very thick},
			post/.style={->,shorten >= 3pt, shorten <=3pt,   thick},
			seqtrace/.style={line width=2},
			und/.style={very thick, draw=gray},
			event/.style={rectangle, minimum height=0.8mm, minimum width=15mm,  line width=1pt, inner sep=0.5,},
			virt/.style={circle,draw=black!50,fill=black!20, opacity=0}]
			\footnotesize
			
			
%			\node[] at (0.5*\xstep, -1.5*\ystep){
%				\small
%				$\cs(\lk_i,\lk_j)=\acq(\lk_i) \cdot\acq(\lk_j) \cdot\rel(\lk_j) \cdot \rel(\lk_i)$
%			};
			

				
			\begin{scope}[shift={(0,-3*\ystep)}]
				
				\draw[dashed] (-0.9*\xstep,6.5*\ystep) rectangle (-0.5*\xstep,8.5*\ystep);
				\node (A1) at (-0.7*\xstep,7*\ystep) {\normalsize [1, 1]};
				\node (A2) at (-0.7*\xstep,8*\ystep) {\normalsize [1, 0]};
				\node (A) at (-0.7*\xstep,9*\ystep) {\large  $A$};
				
				\draw[dashed] (1.5*\xstep,6.5*\ystep) rectangle (1.9*\xstep,8.5*\ystep);
				\node (B1) at (1.7*\xstep,7*\ystep) {\normalsize [1, 0]};
				\node (B2) at (1.7*\xstep,8*\ystep) {\normalsize [0, 1]};
				\node (B) at (1.7*\xstep,9*\ystep) {\large  $B$};
				
				
			\end{scope}
			

			
			\node[] (S11) at (0*\xstep,0.15) {\normalsize $t_A$};
			\node[] (S12) at (0*\xstep,\numevents * \ystep) {};
			\node[] (S21) at (1*\xstep,0.15) {\normalsize $t_B$};
			\node[] (S22) at (1*\xstep,\numevents * \ystep) {};
			
			\draw[seqtrace] (S11) to (S12);
			\draw[seqtrace] (S21) to (S22);
			
			
			\node[event, draw=black, fill=white] (11) at (0*\xstep, 1*\ystep + 0*\ybias) {$\acq(\LockColorTwo{\lk_2})$};
			\node[event, draw=black, fill=white] (12) at (0*\xstep, 2*\ystep + 0*\ybias) {$\acq(\LockColorOne{\lk_1})$};
			\node[event, draw=black, fill=white, dotted] (13) at (0*\xstep, 3*\ystep + 0*\ybias) {$\cs(m_0,m_1)$};
			%\node[event, draw=black, fill=white, dotted] (14) at (0*\xstep, 4*\ystep + 0*\ybias) {$\acq(\lk')$};
			%\node[event, draw=black, fill=white, dotted] (15) at (0*\xstep, 5*\ystep + 0*\ybias) {$\rel(\lk')$};
			%\node[event, draw=black, fill=white, dotted] (16) at (0*\xstep, 6*\ystep + 0*\ybias) {$\rel(\lk)$};
			\node[event, draw=black, fill=white] (17) at (0*\xstep, 4*\ystep + 0*\ybias) {$\rel(\LockColorOne{\lk_1})$};
			\node[event, draw=black, fill=white] (18) at (0*\xstep, 5*\ystep + 0*\ybias) {$\rel(\LockColorTwo{\lk_2})$};
			
			\node[event, draw=black, fill=white] (19) at (0*\xstep, 6*\ystep + 1*\ybias) {$\acq(\LockColorOne{\lk_1})$};
			\node[event, draw=black, fill=white, dotted] (110) at (0*\xstep, 7*\ystep + 1*\ybias) {$\cs(m_0,m_1)$};
			%\node[event, draw=black, fill=white, dotted] (111) at (0*\xstep, 11*\ystep + 1*\ybias) {$\acq(\lk')$};
			%\node[event, draw=black, fill=white, dotted] (112) at (0*\xstep, 12*\ystep + 1*\ybias) {$\rel(\lk')$};
			%\node[event, draw=black, fill=white, dotted] (113) at (0*\xstep, 13*\ystep + 1*\ybias) {$\rel(\lk)$};
			\node[event, draw=black, fill=white] (114) at (0*\xstep, 8*\ystep + 1*\ybias) {$\rel(\LockColorOne{\lk_1})$};
			
			%%%%%%%%%%%%%%%%
			
			\node[event, draw=black, fill=white] (21) at (1*\xstep, 2*\ystep + 0*\ybias) {$\acq(\LockColorOne{\lk_1})$};
			\node[event, draw=black, fill=white, dotted] (22) at (1*\xstep, 3*\ystep + 0*\ybias) {$\cs(m_1,m_0)$};
			%\node[event, draw=black, fill=white, dotted] (23) at (1*\xstep, 4*\ystep + 0*\ybias) {$\acq(\lk)$};
			%\node[event, draw=black, fill=white, dotted] (24) at (1*\xstep, 5*\ystep + 0*\ybias) {$\rel(\lk)$};
			%\node[event, draw=black, fill=white, dotted] (25) at (1*\xstep, 6*\ystep + 0*\ybias) {$\rel(\lk')$};
			\node[event, draw=black, fill=white] (26) at (1*\xstep, 4*\ystep + 0*\ybias) {$\rel(\LockColorOne{\lk_1})$};
			
			\node[event, draw=black, fill=white] (27) at (1*\xstep, 5*\ystep + 1*\ybias) {$\acq_1(\LockColorTwo{\lk_2})$};
			\node[event, draw=black, fill=white, dotted] (28) at (1*\xstep, 6*\ystep + 1*\ybias) {$\cs(m_1,m)$};
			%\node[event, draw=black, fill=white, dotted] (29) at (1*\xstep, 11*\ystep + 1*\ybias) {$\acq(\lk)$};
			%\node[event, draw=black, fill=white, dotted] (210) at (1*\xstep, 12*\ystep + 1*\ybias) {$\rel(\lk)$};
			%\node[event, draw=black, fill=white, dotted] (211) at (1*\xstep, 13*\ystep + 1*\ybias) {$\rel(\lk')$};
			\node[event, draw=black, fill=white] (212) at (1*\xstep, 7*\ystep + 1*\ybias) {$\rel_1(\LockColorTwo{\lk_2})$};
			
			\begin{scope}[]
				\node[below left=of 212] (212b) {};
			\end{scope}
	
		\end{tikzpicture}
	}
	\caption{
		Reduction for OV-hardness proof from an instance of size $n = 2$ and $d = 2$.
		% We use the shortcut $\cs(\lk_i,\lk_j)$ to denote two nested critical sections on $\lk_i$ and $\lk_j$, as indicated in the figure.
	}
	\figlabel{ov-hardness}
\end{subfigure}
%\end{figure}
	
	\myparagraph{Construction}{
		We will construct a trace $\tr$ such that $\tr$ has a deadlock
		pattern of length $k$ iff $(A_1, A_2, \ldots, A_k)$ is a positive $k$-OV instance.
		The trace $\tr$ is of the form $\tr = \tr^{A_1} \cdot \tr^{A_2} \cdots \tr^{A_k}$
		and uses $k$ threads $\set{t_{A_1}, \ldots, t_{A_k}}$ and $d+k$ distinct 
		locks $\lk_1, \ldots, \lk_d, m_1, m_2, \ldots, m_k$. 
		Intuitively, the sub-trace $\tr^{A_i}$ encodes the given set of vectors $A_i$.
		The sub-traces $\tr^{A_i} = \tr^{A_i}_1 \cdot \tr^{A_i}_2 \cdots \tr^{A_i}_n$
		% and $\tr^B = \tr^B_1 \cdot \tr^B_2 \cdots \tr^B_n$ 
		are defined as follows.
		For each $j \in \set{1, 2, \ldots, n}$
		the sub-trace $\tr^{A_i}_j$ is the unique string generated by the
		grammar having $d+1$ non-terminals $S_0, S_1, \ldots, S_d$, start symbol $S_d$
		and the following production rules:
		\begin{itemize}
			\item $S_0 \to \ev{t_Z, \acq(m_i)} \cdot \ev{t_Z, \acq(m_{i \% k + 1})} \cdot \ev{t_Z, \rel(m_{i \% k + 1})} \cdot \ev{t_Z, \rel(m_i)}$.
			% where $(m, m') = (\lk, \lk')$ if $Z = A$, and $(m, m') = (\lk', \lk)$.
			\item for each $1 \leq p \leq d$, $S_p \to S_{p-1}$ if $A_{i, j}[p] = 0$.
			Otherwise (if $A_{i, j}[p] =1$), $S_p \to \ev{t_{A_i}, \acq(\lk_p)} \cdot S_{p-1} \cdot \ev{t_{A_i}, \rel(\lk_p)}$.
		\end{itemize}
		In words, all events of $\tr^{A_i}$ are performed by thread $t_{A_i}$.
		 % and those in $\tr^B$ are performed by $t_B$.
		Next, the $j^{th}$ sub-trace of $\tr^{A_i}$, denoted $\tr^{A_i}_j$ corresponds to the vector $A_{i, j}$
		as follows --- $\tr^{A_i}_j$ is a nested block of critical sections,
		with the innermost critical section being on lock $m_{i \% k + 1}$,
		which is immediately enclosed in a  critical section on lock $m_i$.
		Further, in the sub-trace $\tr^{A_i}_j$, 
		the lock $\lk_p$ occurs iff $A_{i, j}[p] = 1$.
		% The sub-traces $\tr^B_i$ is similarly constructed, except that the order
		% of the two innermost critical sections is inverted.
		\figref{ov-hardness} illustrates the construction for an OV-instance with $k=2$ $n=2$ and $d=2$.
	}

\myparagraph{Correctness}{
We now argue for the correctness of the construction.
First, if $(A_1, A_2, \ldots, A_k)$ is a positive $OV$ instance, then
there are $k$ indices $\alpha_1, \alpha_2, \ldots, \alpha_k \in \set{1 ,\ldots, n}$
such that $\sum_{p=1}^d \prod_{i=1}^k A_{i, \alpha_i}[p] = 0$.
 % and $B_\beta$ are orthogonal.
Thus, for every $1 \leq p \leq d$, there is an $i$ such that $A_{i, \alpha_i}[p] = 0$.
% or $B_\beta[j] = 0$.
Based on the construction, this implies that for every lock $\lk_p$
(with $1 \leq p \leq d$),
$\lk_p$ occurs in at most one of the sub-traces $\tr^{A_1}_{\alpha_1}, \ldots, \tr^{A_k}_{\alpha_k}$.
 % and $\tr^B_\beta$.
This gives a deadlock pattern $D = \pattern{e_1, e_2, \ldots, e_k}$,
where $e_i$ is the unique event in the sub-trace $\tr^{A_i}_{\alpha_i}$ with $\OpOf{e} = \acq(m_{i\%k +1})$.
% and $e'$ is the unique event in $\tr^B_\beta$ with $\OpOf{e'} = \acq(\lk)$.
This is because $m_i \in \lheld{\tr}(e_i)$,
and each lock $\lk_p$ exists in at most one of $\set{\lheld{\tr}(e_i)}_{i\leq n}$
% or $\lheld{\tr}(e')$ 
(and thus $\lheld{\tr}(e_i) \cap \lheld{\tr}(e_{i'}) = \emptyset$ for every $i \neq i' \leq n$).

Now, consider the case when there is a deadlock pattern $D = \pattern{e_1, e_2, \ldots, e_k}$ in $\tr$.
W.l.o.g, we can assume that
$\ThreadOf{e_i} = t_{A_i}$.
This means there are $\alpha_1, \alpha_2, \ldots, \alpha_k$
 % and $\beta$ such that
 such that
$e_i$ occurs in $\tr^{A_i}_{\alpha_i}$.
 % and $e'$ occurs in $\tr^B_\beta$.
Next, observe that the events $\set{e_i}_{i \leq k}$
 % and $e'$ 
 can only be acquire events on locks $\set{m_1, m_2, \ldots, m_k}$
and not on the locks $\set{\lk_1, \ldots, \lk_d}$.
This is because the order of acquisition amongst $\lk_1, \ldots, \lk_d$
is always fixed and further, each of these are never acquired inside of critical sections
of $m_1, \ldots, m_k$.
 % or $\lk'$.
By the choice of threads, $\OpOf{e_i} = \acq(m_{i\%k + 1})$ for every $i \leq k$.
% and $\OpOf{e'} = \acq(\lk)$.
Since $\lheld{\tr}(e_i) \cap \lheld{\tr}(e_{i'}) = \emptyset$ for every $i \neq i' \leq k$, 
we also have that
for every $p \leq d$, $\lk_p \not\in \lheld{\tr}(e) \cap \lheld{\tr}(e')$.
This means that at, for each $p \leq d$, there is atmost one $i \leq k$
for which $A_{i, \alpha_i}[p] = 1$.
This means that the vectors $A_{1, \alpha_1}, A_{2, \alpha_2} \ldots, A_{k, \alpha_k}$
witness that the given OV instance is a positive instance.
}


	\myparagraph{Time complexity}{
		For any fixed $k\geq 2$, the total size of the trace is $\NumEvents = O(n\cdot d)$
		and it has $\NumLocks = d + k $ locks.
		The construction can be performed in time $O(n\cdot d)$.
		Thus, if for some $\epsilon > 0$, 
		there is a $O(\poly{\NumLocks} \cdot \NumEvents^{k-\epsilon}) = O(\poly{d} \cdot n^{k-\epsilon})$ 
		algorithm for the problem of detecting deadlock patterns of length $k$,
		we would get a $O(\poly{d} \cdot n^{k-\epsilon} + n\cdot d) = O(\poly{d} \cdot n^{k-\epsilon})$
		time algorithm for $k$-OV, falsifying the $k$-OV hypothesis.
	}
\end{proof}
