%!TEX root = main.tex

% Drawing executions. Arguments: # of threads, and execution.
\newcommand{\execution}[2]{
\scalebox{0.85}{
  \begin{tikzpicture}%
    \foreach \x in {1,...,#1}
    \node[right] at (1.5*\x+0.2,0.25) {$t_{\x}$};
    \draw (1.2,0) -- (#1*1.5+1.3,0);%
    \pgfmathsetmacro{\y}{1};%
    #2%
    \draw (1.2,0) -- (1.2,-0.4*\y);%
    \draw (#1*1.5+1.3,0) -- (#1*1.5+1.3,-0.4*\y);%
    \foreach \x in {2,...,#1}
    % \draw[dashed] (1.5*\x-0.3,0) -- (1.5*\x-0.3,-0.4*\y);%
    \draw (1.2,-0.4*\y) -- (#1*1.5+1.3,-0.4*\y);%
  \end{tikzpicture}
}
}

\newcommand{\executioncounter}[2]{
	\scalebox{0.85}{
		\begin{tikzpicture}%
			\foreach \x in {3,...,#1}
			\node[right] at (1.5*\x+0.2,0.25) {$t_{\x}$};
			\draw (1.2,0) -- (#1*1.5+1.3,0);%
			\pgfmathsetmacro{\y}{1};%
			#2%
			\draw (1.2,0) -- (1.2,-0.4*\y);%
			\draw (#1*1.5+1.3,0) -- (#1*1.5+1.3,-0.4*\y);%
			\foreach \x in {2,...,#1}
			% \draw[dashed] (1.5*\x-0.3,0) -- (1.5*\x-0.3,-0.4*\y);%
			\draw (1.2,-0.4*\y) -- (#1*1.5+1.3,-0.4*\y);%
		\end{tikzpicture}
	}
}

% \figev arguments: thread id, event, variable/lock
\newcommand{\figev}[2]{
\pgfmathsetmacro{\y}{\y+1};
\pgfmathsetmacro{\y}{\y-1};
\node [left] at (1.25,-0.4*\y)  {\pgfmathprintnumber{\y} };%
\node at (#1*1.5 + 0.45,-0.4*\y) {#2};%
\pgfmathsetmacro{\y}{\y+1};
}


\newcommand{\executionlarge}[2]{
\scalebox{0.85}{
  \begin{tikzpicture}%
    \foreach \x in {1,...,#1}
    \node[right] at (1.9*\x+0.2,0.25) {$t_{\x}$};
    \draw (1.2,0) -- (#1*1.9+1.9,0);%
    \pgfmathsetmacro{\y}{1};%
    #2%
    \draw (1.2,0) -- (1.2,-0.4*\y);%
    \draw (#1*1.9+1.9,0) -- (#1*1.9+1.9,-0.4*\y);%
    \foreach \x in {2,...,#1}
    % \draw[dashed] (1.5*\x-0.3,0) -- (1.5*\x-0.3,-0.4*\y);%
    \draw (1.2,-0.4*\y) -- (#1*1.9+1.9,-0.4*\y);%
  \end{tikzpicture}
}
}



% \figev arguments: thread id, event, variable/lock
\newcommand{\figevlarge}[2]{
\pgfmathsetmacro{\y}{\y+1};
\pgfmathsetmacro{\y}{\y-1};
\node [left] at (1.25,-0.4*\y)  {\pgfmathprintnumber{\y}};%
\node at (#1*1.9 + 0.45,-0.4*\y) {$ #2 $};%
\pgfmathsetmacro{\y}{\y+1};
}


% \figev arguments: thread id, event, variable/lock, 3rd arg = event id offset

% \newcommand{\mathresult}[1]{\pgfmathparse{#1} \pgfmathresult}
% \newcommand{\addmathresult}[1]{\pgfmathparse{2 *\mathresult{#1}} \pgfmathresult}

\newcommand{\figevoffset}[3]{
\pgfmathparse{#1}
\pgfmathsetmacro{\offset}{\pgfmathresult};
\pgfmathparse{\y+\offset}
\pgfmathsetmacro{\newindex}{\pgfmathresult};
\node [left] at (1.25,-0.4*\y)  {\pgfmathprintnumber{\newindex}};%
\node at (#2*1.5 + 0.45,-0.4*\y) {$ $#3$ $};%
\pgfmathsetmacro{\y}{\y+1};
}

%\orderedeg{t1}{ind1}{off1}{t2}{ind2}{off2} denotes an edge from an event e1
% to an event e2 such that ei appears at indi in the trace and belongs to thread ti.
% off1 and off2 are offsets from the center.
\newcommand{\orderedge}[6]{
  \draw (#1*1.5 + 0.45 + #3, -0.4*#2) edge[-{Latex[length=2mm, width=2mm]},  ->,>=stealth] (#4*1.5 + 0.45 + #6, -0.4*#5);
}

% The 7th arg is the label and the 8th argument is the position
\newcommand{\orderedgewithlabel}[8]{
  \draw (#1*1.5 + 0.45 + #3, -0.4*#2) edge[-{Latex[length=2mm, width=2mm]},  ->,>=stealth] node[anchor=center, sloped, #8] { #7} (#4*1.5 + 0.45 + #6, -0.4*#5);
}

\newcommand{\executionnonumber}[2]{\begin{tikzpicture}%
                             \foreach \x in {1,...,#1}
                               \node[right] at (2*\x+0.5,0.25) {$t_{\x}$};
                             \draw (1.9,0) -- (#1*2+1.6,0);%
                             \pgfmathsetmacro{\z}{1};%
                             #2%
                             \draw (1.9,0) -- (1.9,-0.5*\z);%
                             \draw (#1*2+1.6,0) -- (#1*2+1.6,-0.5*\z);%
                             \foreach \x in {2,...,#1}
                               \draw (2*\x-0.3,0) -- (2*\x-0.3,-0.5*\z);%
                             \draw (1.9,-0.5*\z) -- (#1*2+1.6,-0.5*\z);%
                           \end{tikzpicture}}                    

\newcommand{\figevnonumber}[2]{\node [left] at (1.75,-0.4*\z) {};%
\node [right] at (#1*2 + 0.3,-0.5*\z) {$ #2 $};%
\pgfmathsetmacro{\z}{\z+1};}

% Drawing programs. Arguments: # of threads, and program
\newcommand{\figprogram}[2]{\begin{tikzpicture}%
                              \foreach \x in {1,...,#1}
                                \node at (2.3*\x-1.15,0.25) {$t_{\x}$};
                              \draw (0,0) -- (#1*2.3+0.6,0);% 
                              \pgfmathsetmacro{\y}{1};%
                              #2%
                              \draw (0,0) -- (0,-0.4*\y);% 
                              \draw (#1*2.3+0.6,0) -- (#1*2.3+0.6,-0.4*\y);% 
                              \draw (0,-0.4*\y) -- (#1*2.3+0.6,-0.4*\y);% 
                            \end{tikzpicture}}
% figstmt arguments: thread id, offset, statement
\newcommand{\figstmt}[3]{\node [left] at (-0.15,-0.4*\y) {\pgfmathprintnumber{\y}};%
                      \node[right] at (#1*2.3 + #2*0.3 - 2.15,-0.4*\y) {\texttt{#3}};
                      \pgfmathsetmacro{\y}{\y+1};}

\newcommand{\specialcell}[2][c]{%
  \begin{tabular}[#1]{@{}c@{}}#2\end{tabular}}

\newtheorem{observation}{Observation}

\newcommand{\nil}{\texttt{nil}}

\newcommand{\drawarrow}{\raisebox{2pt}{\scalebox{0.7}{\tikz{ \draw[-{Latex[length=2mm, width=2mm]},  ->,>=stealth](0,0) -- (7mm,0);} }}}

% \separatorlight{num_threads}{ind}{off1}{off2} --- draw a dashed line
% after event at index ind. off1 and off2 are left and right offsets.
\newcommand{\separatorlight}[4]{
  \draw[loosely dashed] (1.5 + 0.45 + #3, -0.4*#2 -0.2) -- (#1*1.5 + 0.45 + #4, -0.4*#2 - 0.2);
}

\newcommand{\separatordark}[4]{
  \draw[thick, dash dot] (1.5 + 0.45 + #3, -0.4*#2 -0.2) -- (#1*1.5 + 0.45 + #4, -0.4*#2 - 0.2);
  \draw[thick, dash dot] (1.5 + 0.45 + #3, -0.4*#2 -0.25) -- (#1*1.5 + 0.45 + #4, -0.4*#2 - 0.25);
}

% \demarcate{num_threads}{ind1}{ind2}{off}{text} -- draw a double headed arrow
% from event at ind1 to event at ind2, with an offset of off
% to the right of the execution, label is the text
\newcommand{\demarcate}[5]{
  \draw (#1*1.5 + 0.45 + #4, -0.4*#2) edge[-{Latex[length=2mm, width=2mm]},  <->,>=stealth]  (#1*1.5 + 0.45 + #4, -0.4*#3);
  \node[fill=white] (dummy) at (#1*1.5 + 0.45 + #4, -0.2*#2 -0.2*#3) {#5};
}