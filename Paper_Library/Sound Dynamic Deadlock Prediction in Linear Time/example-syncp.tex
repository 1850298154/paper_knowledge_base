%!TEX root = main.tex

\begin{example}
\exlabel{syncp-ex}
%We illustrate sync-preserving closure on our example traces.
% We will use the notation 
% $\prev{\tr}^*(S) =
% \setpred{e \in \events{\tr}}{\exists f \in S, e \tho{\tr} f}$ 
% for a trace $\tr$ and a subset $S \subseteq \events{\tr}$.
% For ease of notation, given a trace $\tr$ and set of events $S\subseteq \events{\tr}$, we denote by $\LowerSet_{\tr}(S)=\{e\colon \exists e'. e\tho{\tr}e' \}$ these events and their local predecessors in $\tr$.\\
%
Consider the trace $\tr_2$ in \figref{motivating-dl}, 
and the deadlock pattern $D=\pattern{e_{4}, e_{18}}$.
We have 
$\SPClosure{\tr_2}(\prev{\tr_2}(\{ e_{4}, e_{18} \})) = 
\set{e_1, e_2, e_3, e_8, e_9, e_{12}, \ldots, e_{17}}$.
Since we have that $e_{4}, e_{18} \not \in$ $\SPClosure{\tr_2}( \prev{\tr_2}(\set{e_{4},  e_{18}}))$, $D$ is a sync-preserving deadlock. 
%
Now consider the trace $\tr_3$ in \figref{syncp_example},
and the deadlock patterns
$D_1=\allowbreak\pattern{e_2, e_{16}}$, 
$D_5=\allowbreak\pattern{e_{29}, e_{16}}$, and
$D_6=\allowbreak\pattern{e_{29}, e_{19}}$.
We have
$\SPClosure{\tr_3}(\prev{\tr_3}(\{ e_{2},e_{16}\}))\allowbreak=
\set{e_1, \ldots, e_6, e_8, \ldots, e_{15}}$, 
$\SPClosure{\tr_3}(\prev{\tr_3}(\allowbreak\{ e_{29},e_{16}\}))=
\set{
e_1, \ldots, e_{15}, e_{28}}$, and $\SPClosure{\tr_3}(\prev{\tr_3}(\{ e_{29},e_{19}\}))=
\set{
e_1, \ldots, e_{18}, e_{28}}$.
Since $e_2 \in \allowbreak \SPClosure{\tr_3}(\allowbreak\prev{\tr_3}(\{ e_{2},e_{16}\}))$,
$D_1$ is not a sync-preserving deadlock.
However, $e_{29}, e_{16} \not \in \SPClosure{\tr_3}(\prev{\tr_3}(\{ e_{29},e_{16}\}))$, and $e_{29}, e_{19} \not \in \SPClosure{\tr_3}(\prev{\tr_3}(\{ e_{29},e_{19}\}))$, thus $D_5$ and $D_6$ are sync-preserving deadlocks (as we also concluded in \exref{sp-deadlocks}).
\end{example}