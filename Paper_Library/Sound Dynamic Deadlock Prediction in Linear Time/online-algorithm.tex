%!TEX root = main.tex

\section{On-the-fly Deadlock Prediction}
\seclabel{otf}

Although $\SyncPDOffline$ is efficient, both theoretically (\thmref{syncp_offline})
and in practice (Table~\ref{tab:expr-results}),
it runs in two passes, akin to other predictive deadlock-detection methods~\cite{Kalhauge2018,Cai2021}.
%In other words, it is \emph{offline} and requires access to the input trace $\tr$ upfront.
In a runtime monitoring setting, it is desirable to operate in an \emph{online} fashion.
Recall that $\checkAbsDeadP(\cdot)$ indeed operates online (\secref{verify-abstract-patterns}),
while the offline nature of $\SyncPDOffline$
is tied to the offline construction of the abstract lock graph $\lkevgraph{\tr}$.
% , and performs cycle detection in order to find abstract deadlock patterns.
%We note that $\lkevgraph{\tr}$ may also be maintained online, and be periodically checked for cycles, as, e.g., in~\cite{Havelund2000,Agarwal2005,Cai2020}.
%However, this operation becomes costly when there are many cycles.
%as there can be many cycles of short length, while long cycles take long to discover.
To achieve the golden standard of online, linear-time, 
sound deadlock prediction, we focus on deadlocks of size $2$.
This focus is barely restrictive as most deadlocks in the wild have size $2$~\cite{Lu08}.
Further, deadlocks of size $2$ enjoy the following computational benefits:
\begin{enumerate*}[label=(\alph*)]
\item cycles of length $2$ can be detected instantaneously without performing graph traversals, and
\item \emph{every} cycle of length $2$ is an abstract deadlock pattern.
\end{enumerate*}


\myparagraph{Algorithm $\SyncPDOnline$}{
% We are now ready to outline our algorithm $\SyncPDOnline$ for the online detection of all sync-preserving deadlocks of size 2. 
%For completeness we refer to \appref{app_online} for the pseudocode.
The algorithm $\SyncPDOnline$ maintains all abstract acquires of the form $\eta=\tuple{t, \lk_1, \{\lk_2\}, F}$,
i.e., we only focus on one lock $\lk_2$ that is protecting each such acquire.
When a new acquire event $e=\ev{t,\acq(\lk_1)}$ is encountered, 
the algorithm iterates over all the locks $\lk_2\in\lheld{\tr}(e)$ that are held in $e$,
and append the event $e$ to the sequence $F$ of the corresponding abstract acquire $\eta=\tuple{t, \lk_1, \{ \lk_2 \}, F}$; $F$ is maintained as FIFO queue.
Recall that we use timestamps on the acquire events in $F$ to determine membership in our closure computation.
Our online algorithm computes these timestamps on-the-fly and stores them in these queues together with the events.
%together
%, instead of storing the actual event.
% \Andreas{@Umang: I am missing some logic about vector clocks here, please update accordingly}
Then the algorithm calls $\checkAbsDeadP(C)$ on the abstract deadlock pattern $C$ formed between $\eta$ and $\eta'=\tuple{t'\neq t, \lk_2, \{ \lk_1 \}, F'}$, in order to check for sync-preserving deadlocks between the deadlock patterns in $F\times F'$.
If $\checkAbsDeadP(C)$ reports no deadlock, the contents of $F'$ are emptied
%as the monotonicity of \corref{pattern-monotone} 
as we are guaranteed that $F'$ will not cause a sync-preserving deadlock with any further acquire of thread $t$ on lock $\lk_1$.
% 
For a trace with $\NumEvents$ events, $\NumThreads$ threads and $\NumLocks$ locks.
The algorithm calls $\checkAbsDeadP$ for each of the $O(\NumThreads^2\cdot \NumLocks^2)$ abstract
deadlock patterns of size $2$, each call taking $O(\NumEvents\cdot \NumThreads)$ time.

%!TEX root = main.tex

\small
\begin{algorithm*}[t!]
% \BlankLine
 \vspace*{-0.35cm}
\begin{multicols}{2}
\iffalse
\myfun{\maxLB{$U$, $Lst$}}{
	$(g_{\max}, C_{\max}, C'_{\max})$ := $(0, \bot, \bot)$ \; 
	\While{$\NOT\, \isEmpty{Lst}$}{
		$(g, C, C')$ := $\first{Lst}$ \;
		\If{$C \cle U$}{
			$(g_{\max}, C_{\max}, C'_{\max})$ := $(g, C, C')$
		}
		\lElse{
			\Break
		}
		$\removeFirst{Lst}$
	}
	\Return $(g_{\max}, C_{\max}, C'_{\max})$
}

\BlankLine

\myfun{\fixpoint{$I$, $\tuple{t_1, \lk_1, t_2, \lk_2}$}}{
	\Repeat{$I$ does not change}{
		\ForEach{$\lk \in \locks{}$}{
			\ForEach{$t \in \threads{}$}{
				($g_{\lk, t}, C_{\lk, t}, C'_{\lk, t}$) := 
				\maxLB{$I$, $\view{\AcqLst}{t, \lk}{\tuple{t_1, \lk_1, t_2, \lk_2}}$} \;
			}
			$t_{\max}$ := argmax$_{t \in \threads{}}\set{g_{\lk, t}}$ \;
			$I$ := $I \mx \bigsqcup_{t \neq t_{\max} \in \threads{}} C'_{\lk, t}$	
		}
	}
	\Return $I$
}
\fi

\myfun{\checkDeadlock{$Lst$, $I$, $\tuple{t_1, \lk_1, t_2, \lk_2}$}}{
	\While{$\NOT\, \isEmpty{Lst}$}{
		($C_{\prev{}}, C$) := $\first{Lst}$ \;
		$I$ := \fixpoint{$I \mx C_{\prev{}}$, $\tuple{t_1, \lk_1, t_2, \lk_2}$} \;
		\lIf{$C \not\cle I$}{ \linelabel{check-containment}
			\declare `Deadlock' and \Break
		}
		$\removeFirst{Lst}$
	}
	\Return $I$
}

\myhandler{\wthandler{$t$, $x$}}{
	$\LW_x$ := $\Cc_t$ \linelabel{clkwt}\\
	$\Cc_t$ := $\Cc_t[t \mapsto \Cc_t(t)+1]$ \linelabel{clkctw}
}

\myhandler{\relhandler{$t$, $\lk$}}{
	$\Cc_t$ := $\Cc_t[t \mapsto \Cc_t(t)+1]$ \linelabel{clkrel} \;
	\ForEach{$t_1, t_2 \in \threads{}, \lk_1, \lk_2 \in  \locks{}$}{
		$\updateRel{\lst{\view{\AcqLst}{t, \lk}{\tuple{t_1, \lk_1, t_2, \lk_2}}}}{$\Cc_t$}$
	}
}

\myhandler{\acqhandler{$t$, $\lk$}}{
	$C_{\prev{}}$ := $\Cc_t$ \;
	$\Cc_t$ := $\Cc_t[t \mapsto \Cc_t(t)+1]$\linelabel{clkacq}\\ $\gId_\lk$ := $\gId_\lk + 1$ \linelabel{gidlk}\;
	\ForEach{$t_1, t_2 \in \threads{}, \lk_1, \lk_2 \in \locks{}$}{
		$\addLst{\view{\AcqLst}{t, \lk}{\tuple{t_1, \lk_1, t_2, \lk_2}}}{$(\gId_\lk, \Cc_t, \bot)$}$
	}
	\ForEach{$u\in \threads{}$, $\lk' \in \lheld{}$}{
		$\addLst{\view{\DPLst}{t, \lk, \lk'}{\tuple{u}}}{($C_{\prev{}}$, $\Cc_t$)}$ \linelabel{acqhist}
	}
	\ForEach{$u \neq t \in \threads{}$, $\lk' \in \lheld{}$}{
		$I$ := $\view{\Ii}{}{\tuple{u, \lk' t, \lk}} \mx C_{\prev{}}$ \;
		$\view{\Ii}{}{\tuple{u, \lk' t, \lk}}$ := \checkDeadlock{$\view{\DPLst}{u, \lk', \lk\,}{\tuple{t}}$, $I$, $\tuple{u, \lk' t, \lk}$} \linelabel{checkdlk}
	}

}

\myhandler{\rdhandler{$t$, $x$}}{
	$\Cc_t$ := $\Cc_t \mx \LW_x$ \linelabel{clkrd}
}

\BlankLine

\end{multicols}
 \vspace*{-0.25cm}
\normalsize
\caption{$\SyncPDOnline$.}
\algolabel{online}
\end{algorithm*}
\normalsize

$\SyncPDOnline$ is shown in detail in \algoref{online}.
The pseudocode contains handlers for processing the different events of $\tr$ in a streaming fashion, as well as a helper function for checking deadlocks.
The main data structures of the algorithm are
(i)~vector clocks $\Cc_t$, $\LW_x$, and $\view{\Ii}{}{\tuple{t_1, \lk_1, t_2, \lk_2}}$,
(ii)~scalar $\gId_\lk$, and
(iii)~FIFO queues of vector clocks $\view{\AcqLst}{t, \lk}{\tuple{t_1, \lk_1, t_2, \lk_2}}$ and $\view{\DPLst}{t, \lk_1, \lk_2}{\tuple{u}}$,
where $t, t_1, t_2, u$ range over threads, $x$ ranges over variables, and $\lk, \lk_1, \lk_2$ range over locks.
$\Cc_t$ stores the timestamp $\TS{\tr}^e$ where $e$ is the last event in thread $t$.
%(Lines \ref{line:clkctw},  \ref{line:clkrel}, \ref{line:clkacq}, \ref{line:clkrd}).
$\LW_x$ keeps track of the $\TS{\tr}^e$ where $e$ is the last event such that $\OpOf{e} = \wt(x)$. %(\cref{line:clkwt}).
$\view{\Ii}{}{\tuple{t_1, \lk_1, t_2, \lk_2}}$ stores the computed sync-preserving closures for every tuple $\tuple{t_1, \lk_1, t_2, \lk_2}$.
%(\cref{line:checkdlk}).
The scalar variable $\gId_\lk$ keeps track of the index of the last acquire event on lock $\lk$. %(\cref{line:gidlk}).
Similar to \cref{algo:compute-closure}, the FIFO queue $\view{\AcqLst}{t, \lk}{\tuple{t_1, \lk_1, t_2, \lk_2}}$ is maintained to keep track of the critical section history of thread $t$ and lock $\lk$.
Lastly, for an acquire event $e$, $\view{\DPLst}{t, \lk_1, \lk_2}{\tuple{u}}$ maintains a queue of tuples of the form $\tuple{C_{\prev{}}, \Cc_t}$ where $C_{\prev{}}$ and $\Cc_t$ are the timestamps of $\prev{\sigma}(e)$ and $e$, respectively
%(\cref{line:acqhist}).
These tuples are utilized when checking for deadlocks (\cref{line:checkdlk}).
}

\begin{restatable}{theorem}{syncponline}
\thmlabel{syncp_online}
Consider a trace $\tr$ of $\NumEvents$ events, $\NumThreads$ threads and $\NumLocks$ locks.
The online $\SyncPDOnline$ algorithm reports all sync-preserving deadlocks of size $2$ of $\tr$ in $O(\NumEvents\cdot  \NumThreads^3\cdot \NumLocks^2)$ time.
%$O(\NumEvents\cdot \NumThreads+ \NumThreads^3\cdot \NumLocks^2\cdot \NumAcquires)$ time.
\end{restatable}

