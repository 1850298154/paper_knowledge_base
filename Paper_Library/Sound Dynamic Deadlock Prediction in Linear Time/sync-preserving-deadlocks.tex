%!TEX root = main.tex

\section{Synchronization-Preserving Deadlocks and their Prediction}
\seclabel{syncp}
Having established the intractability of general deadlock prediction in \secref{lower-bounds}, we now define the subclass of predictable deadlocks called synchronization-preserving (\emph{sync-preserving}, for short) in \secref{syncp-def}.
The key benefit of sync-preserving deadlocks is that, unlike arbitrary deadlocks, they can be detected efficiently; we develop our algorithm $\SyncPDOffline$ for this task in Sections 4.2-4.5.
Our experiments later indicate that most predictable deadlocks are actually sync-preserving, hence the benefit of fast detection comes at the cost of little-to-no precision loss in practice.
%We start with a brief overview.

\myparagraph{Overview of the algorithm}{
	There are several insights behind our algorithm.
	First, given a deadlock pattern, one can verify 
	if it is a sync-preserving deadlock in linear time (\secref{verify-patterns});
	this is based on our sound and complete characterization of sync-preserving deadlocks (\secref{characterize-patterns}).
%	This is in sharp contrast with prior works
%such as~\cite{Cai2021,Kalhauge2018} that resort to
%SMT-solving or expensive (i.e., of large polynomial) graph-based analyses. 
	Next, instead of verifying single deadlock patterns one-by-one, 
	we consider \emph{abstract deadlock patterns},
	which are essentially collections of deadlock patterns
	that share the same signature;
	the formal definition is given in \secref{verify-abstract-patterns}.
	We show that our basic algorithm can
	be extended to \emph{incrementally}
	verify \emph{all} the concretizations of an abstract deadlock pattern in linear time (\secref{verify-abstract-patterns}), in a single pass (\lemref{abstract-pattern-linear-time}).
	%Finally, we extend this algorithm to all deadlock patterns in the trace,
	%by first enumerating all the abstract deadlock patterns by constructing an \emph{abstract lock graph},
	%and then verifying each of them using the algorithm thus far (\secref{enumerate-patterns}).
	Finally, we feed this algorithm all the abstract deadlock patterns of the input trace,
    by constructing an \emph{abstract lock graph} and enumerating cycles in it (\secref{enumerate-patterns}).
	
	% We next discuss these in detail \ucomment{and conclude with a summary of our algorithm in section 4.6}.
}

\subsection{Synchronization-Preserving Deadlocks}
\seclabel{syncp-def}

Our notion of sync-preserving deadlocks builds on the recently introduced concept of sync-preserving correct reorderings~\cite{Mathur2021}.

\begin{definition}[Sync-preserving Correct Reordering]
\deflabel{syncp-correct-reordering}
A correct reordering $\rho$ of a trace $\tr$ is
\emph{sync-preserving} if for every lock $\lk \in \locks{\rho}$
and every two acquire events $e_1 \neq e_2 \in \events{\rho}$
with $\OpOf{e_1} = \OpOf{e_2} = \acq(\lk)$, the order of $e_1$ and
$e_2$ is the same in $\tr$ and $\rho$, i.e., 
$e_1 \trord{\rho} e_2$ iff $e_1 \trord{\tr} e_2$.
\end{definition} 
A sync-preserving correct reordering
preserves the order of those critical sections (on the same lock) that actually appear in 
the reordering, but allows 
intermediate critical sections to be dropped completely.
% 
% Since a correct reordering $\pi$ of $\tr$ may only contain
% a prefix of events in some thread $t$, some critical section of $\tr$
% in thread $t$ may not be present in $\pi$.
% In particular, sync-preserving correct reorderings allow intermediate 
% critical sections (of the same lock) to be dropped, 
% as long as the order on the remaining ones is preserved.
This style of reasoning is more permissive than
the space of reorderings %(and their prefixes) 
induced by the Happens-Before partial order~\cite{Lamport78},
that implicitly enforces that all intermediate critical sections on a lock
be present.
Sync-preserving deadlocks can now be defined naturally.
\begin{definition}[Sync-preserving Deadlocks]
\deflabel{syncp-deadlock}
Let $\tr$ be a trace and $D = \pattern{e_0, e_1, \ldots, e_{k-1}}$
be a deadlock pattern.
We say that $D$ is a sync-preserving deadlock of $\tr$
if there is a sync-preserving correct reordering $\rho$ of $\tr$
such that each of $e_0, \ldots, e_{k-1}$ is $\tr$-enabled in $\rho$.
\end{definition}

%!TEX root=../main.tex

\begin{figure}
	\centering
	\begin{subfigure}[t]{0.17\textwidth}
		\scalebox{0.9}{
			\execution{3}{
				\figev{1}{$\acq(\LockColorOne{\lk_1})$}
				\figev{1}{$\acq(\LockColorTwo{\lk_2})$}
				\figev{1}{$\rel(\LockColorTwo{\lk_2})$}
				\figev{1}{$\acq(\LockColorTwo{\lk_2})$}
				\figev{1}{$\wt(y)$}
				\figev{1}{$\rel(\LockColorTwo{\lk_2})$}
				\figev{1}{$\rel(\LockColorOne{\lk_1})$}
				\figev{2}{$\acq(\LockColorThree{\lk_3})$}
				\figev{2}{$\wt(x)$}
				\figev{2}{$\rd(y)$}
				\figev{2}{$\rel(\LockColorThree{\lk_3})$}
				\figev{3}{$\acq(\LockColorTwo{\lk_2})$}
				\figev{3}{$\acq(\LockColorThree{\lk_3})$}
				\figev{3}{$\rd(x)$}
				\figev{3}{$\rel(\LockColorThree{\lk_3})$}
				\figev{3}{$\acq(\LockColorOne{\lk_1})$}
			}
		}
	\end{subfigure}
	\hspace*{\fill}
	\begin{subfigure}[t]{0.17\textwidth}
		\scalebox{0.9}{
			\execution{3}{
				\figevoffset{16}{3}{$\wt(v)$}
				\figevoffset{16}{3}{$\rel(\LockColorOne{\lk_1})$}
				\figevoffset{16}{3}{$\acq(\LockColorOne{\lk_1})$}
				\figevoffset{16}{3}{$\rel(\LockColorOne{\lk_1})$}
				\figevoffset{16}{3}{$\rel(\LockColorTwo{\lk_2})$}
				\figevoffset{16}{2}{$\acq(\LockColorFour{\lk_4})$}
				\figevoffset{16}{2}{$\acq(\LockColorOne{\lk_1})$}
				\figevoffset{16}{2}{$\wt(z)$}
				\figevoffset{16}{2}{$\rd(v)$}
				\figevoffset{16}{2}{$\rel(\LockColorOne{\lk_1})$}
				\figevoffset{16}{2}{$\rel(\LockColorFour{\lk_4})$}
				\figevoffset{16}{1}{$\acq(\LockColorOne{\lk_1})$}
				\figevoffset{16}{1}{$\acq(\LockColorTwo{\lk_2})$}
				\figevoffset{16}{1}{$\rd(z)$}
				\figevoffset{16}{1}{$\rel(\LockColorTwo{\lk_2})$}
				\figevoffset{16}{1}{$\rel(\LockColorOne{\lk_1})$}
			}
		}
	\end{subfigure}
	\hspace*{\fill}
	\begin{subfigure}[t]{0.35\textwidth}
		\begin{tikzpicture}[baseline=-180pt]
			\tikzset{rectangle/.append style={draw=black}}
			\scalebox{0.84}{
				\def\xstep{2.2}
				\def\ystep{0.5}
				\node[align=left] (tt1) at (0*\xstep, -3.5){
					$
					\eta_1=
					\tuple{t_1, \LockColorTwo{\lk_2}, \{ \LockColorOne{\lk_1} \},
						\sequence{e_2, e_4, e_{29}}}
					$\\
					$
					\eta_2=
					\tuple{
						t_2, \LockColorOne{\lk_1}, \{ \LockColorFour{\lk_4} \},
						\sequence{e_{23}}}
					$\\
					$
					\eta_3=
					\tuple{
						t_3, \LockColorOne{\lk_1}, \{ \LockColorTwo{\lk_2} \},
						\sequence{e_{16}, e_{19}}}
					$\\
					$
					\eta_4=
					\tuple{
						t_3, \LockColorThree{\lk_3}, \{ \LockColorTwo{\lk_2} \},
						\sequence{e_{13}}}
					$
				};
				
				\node[align=left] (tt2) at (0*\xstep, -4.8){
					$\abst{D}=\pattern{\eta_1, \eta_3}$
				};
				
				\node[align=left] (tt3) at (-0.5*\xstep, -6){
					$D_1=\pattern{e_2, e_{16}}$\\
					$D_2=\pattern{e_2, e_{19}}$\\
					$D_3=\pattern{e_4, e_{16}}$\\
				};
				\node[align=left] (tt4) at (0.5*\xstep, -6){
					$D_4=\pattern{e_4, e_{19}}$\\
					$D_5=\pattern{e_{29}, e_{16}}$\\
					$D_6=\pattern{e_{29}, e_{19}}$\\
				};
			}
		\end{tikzpicture}
	\end{subfigure}
	% \vspace{-40pt}
	\caption{
		A trace $\tr_3$, its abstract acquires $\eta_i$, unique abstract deadlock pattern $\abst{D}$, concrete patterns $D_i\in \abst{D}$. 
	}
	% \vspace{-10pt}
	\figlabel{syncp_example}
\end{figure}


\begin{example}
\exlabel{sp-deadlocks}
Consider the trace $\tr_2$ in \figref{motivating-dl}.
The deadlock pattern $D = \pattern{e_4, e_{18}}$ is a sync-preserving deadlock, witnessed by the sync-preserving correct reordering
$\rho_3 = e_1e_2e_3 e_8 e_9 \,e_{12}..e_{15}\,e_{16} e_{17}$.
Now consider the trace $\tr_3$ from \figref{syncp_example}
and the deadlock pattern $D_5 = \pattern{e_{29}, e_{16}}$.
This is a predictable deadlock, witnessed by the correct reordering
$\rho_5 = e_1..e_7e_8..e_{11}e_{12}..e_{15}\,e_{28}$.
Observe that $\rho_5$ is a sync-preserving reordering, which makes $D_5$ a sync-preserving deadlock.
%\hunkar{
A key aspect in $\rho_5$ is that the events $e_{22}..e_{27}$ are dropped, as otherwise $e_{16}$ cannot be $\sigma_3$-enabled.
%}
%Another reordering that witnesses the deadlock $D_5$ is
%$\rho_5 = e_8\,e_1..e_3\,e_9\,e_4..e_7\,e_{12}e_{10}\,e_{11}e_{28}\,e_{13}..e_{15}$.
%This reordering $\rho_5$ inverts the order between some events,
%but is sync-preserving because
%it preserves the order between all pairs of critical sections
%on the same lock. % is the same as in $\tr_2$.
% Hence, $\rho_5$ is also a sync-preserving correct reordering.
A similar reasoning applies for the deadlock pattern $D_6$, and it is also a sync-preserving deadlock.
The other deadlock patterns ($D_1, D_2, D_3, D_4$)
are not predictable deadlocks.
Intuitively, the reason for this is that realizing these deadlock patterns require executing the read event $e_{14}$, which then enforces to execute the events $e_8..e_{11}$ and $e_1..e_6$.
This prevents the deadlocks from becoming realizable as the events 
$e_2$ or $e_4$ that appear in these deadlock patterns are no longer $\sigma_3$-enabled.
This point is detailed in \exref{syncp-ex}.
\end{example}






%!TEX root = main.tex

\subsection{Characterizing Sync-Preserving Deadlocks}
\seclabel{characterize-patterns}

There are two fundamental tasks in 
searching for a correct reordering that witnesses a deadlock --- 
(i)~determining the set of events in the correct reordering, and 
(ii)~identifying a total order on such events --- both of which are intractable~\cite{Mathur2020b}.
On the contrary, for sync-preserving deadlocks, we show that
(a)~the search for a correct reordering can be reduced
to the problem of checking if some well-defined set of events (\defref{spclosure})
does not contain the events appearing in the deadlock pattern (\lemref{spclosure-deadlock}),
and that (b)~this set can be constructed efficiently.
%These results directly extend the results of prior work
%on race detection~\cite{Mathur2021} to the more general
%case of checking concurrency of \emph{arbitrarily many} events.
%Let us define this set next.
\begin{definition}[Sync-Preserving Closure]
\deflabel{spclosure}
Let $\tr$ be a trace and $S \subseteq \events{\tr}$.
The sync-preserving closure of $S$, denoted
$\SPClosure{\tr}(S)$ is the smallest set $S'$ such that
\begin{enumerate*}[label=(\alph*)]
	\item \itmlabel{def-item-spclosure-a} $S \subseteq S'$,
	\item \itmlabel{def-item-spclosure-b} for every $e, e' \in \events{\tr}$ such that 
	$e \stricttho{\tr} e'$ or $e = \rf{\tr}(e')$, if $e' \in S'$, then $e \in S'$, and
	\item\itmlabel{acq-order-closure} for every lock $\lk$ and every two distinct events $e, e'\in S'$ with $\OpOf{e} = \OpOf{e'} = \acq(\lk)$,
	if $e \trord{\tr} e'$ then $\match{\tr}(e) \in S'$.
\end{enumerate*}
\end{definition}


\defref{spclosure} resembles 
the notion of correct reorderings (\defref{syncp-correct-reordering}). 
Indeed, \lemref{spclosure-necessary-sufficient} justifies using this set --- it is
both a necessary and a sufficient set for sync-preserving correct reorderings.

\begin{restatable}{lemma}{spclosureNecessarySufficient}
\lemlabel{spclosure-necessary-sufficient}
Let $\tr$ be a trace and let $S \subseteq \events{\tr}$.
For any sync-preserving correct reordering $\rho$ of $\tr$, 
	if $S \subseteq \events{\rho}$, then $\SPClosure{\tr}(S) \subseteq \events{\rho}$.
	Further, there is a sync-preserving correct reordering $\rho$ of 
	$\tr$ such that $\events{\rho} = \SPClosure{\tr}(S)$.
% \begin{enumerate}[label=(\arabic*)]
% 	\item For any sync-preserving correct reordering $\rho$ of $\tr$, 
% 	if $S \subseteq \events{\rho}$, then $\SPClosure{\tr}(S) \subseteq \events{\rho}$.
% 	\item There is a sync-preserving correct reordering $\rho$ of 
% 	$\tr$ such that $\events{\rho} = \SPClosure{\tr}(S)$.
% \end{enumerate}
\end{restatable}

%\hunkar{
For an intuition, consider again \cref{fig:syncp_example} and the sync-preserving correct reordering $\rho_5 = e_1..e_7e_8..e_{11}e_{12}..e_{15}\,e_{28}$ computed in \cref{ex:sp-deadlocks}.
According to \cref{lem:spclosure-necessary-sufficient}, $\SPClosure{\tr_3}(S) \subseteq \events{\rho_5}$ holds for all $S$ such that $S \subseteq \events{\rho_5}$.
For example, if we take $S=\set{e_1, e_{15}}$ then observe that $S \subseteq \events{\rho_5} $ and  $\SPClosure{\tr_3}(S)=\set{e_1, \ldots, e_6, \, e_8, \ldots e_{15}} \subseteq \events{\rho_5}$ holds.
%}

%\hunkar{
%Can we provide any intuition on the second part of the lemma?
%}
Based on~\lemref{spclosure-necessary-sufficient}, we 
present a sound and complete characterization of sync-preserving deadlocks
(\lemref{spclosure-deadlock}).
For a set $S \subseteq \events{\tr}$, 
we let $\prev{\tr}(S)$ denote the set of immediate thread
predecessors of events in $S$. 
That is,
$\prev{\tr}(S) = \setpred{e \in \events{\tr}}{\exists f \in S, e \stricttrord{\tr} f \text{ and } \forall e' \stricttrord{\tr} f, e' \tho{\tr} e}$.
% \begin{restatable}{lemma}{spclosure-iff-syncp-correct-reordering}
% \lemlabel{spclosure-iff-syncp-correct-reordering}
% Let $\tr$ be a trace and let $S \subseteq \events{\tr}$.
% There is a sync-preserving correct reordering $\rho$ of $\tr$
% in which each event $e \in S$ is $\tr$-enabled iff 
% $\SPClosure{\tr}(\prev{\tr}(S)) \cap S = \emptyset$.
% \end{restatable}

\begin{restatable}{lemma}{spclosureDeadlock}
\lemlabel{spclosure-deadlock}
Let $\tr$ be a trace and let $D = \pattern{e_0, \ldots, e_{k-1}}$
be a deadlock pattern of size $k$ in $\tr$.
$D$ is a sync-preserving deadlock of $\tr$ iff
$\SPClosure{\tr}(\prev{\tr}(S)) \cap S = \emptyset$, where
$S = \set{e_0, \ldots, e_{k-1}}$.
\end{restatable}

%!TEX root = main.tex

\begin{example}
\exlabel{syncp-ex}
%We illustrate sync-preserving closure on our example traces.
% We will use the notation 
% $\prev{\tr}^*(S) =
% \setpred{e \in \events{\tr}}{\exists f \in S, e \tho{\tr} f}$ 
% for a trace $\tr$ and a subset $S \subseteq \events{\tr}$.
% For ease of notation, given a trace $\tr$ and set of events $S\subseteq \events{\tr}$, we denote by $\LowerSet_{\tr}(S)=\{e\colon \exists e'. e\tho{\tr}e' \}$ these events and their local predecessors in $\tr$.\\
%
Consider the trace $\tr_2$ in \figref{motivating-dl}, 
and the deadlock pattern $D=\pattern{e_{4}, e_{18}}$.
We have 
$\SPClosure{\tr_2}(\prev{\tr_2}(\{ e_{4}, e_{18} \})) = 
\set{e_1, e_2, e_3, e_8, e_9, e_{12}, \ldots, e_{17}}$.
Since we have that $e_{4}, e_{18} \not \in$ $\SPClosure{\tr_2}( \prev{\tr_2}(\set{e_{4},  e_{18}}))$, $D$ is a sync-preserving deadlock. 
%
Now consider the trace $\tr_3$ in \figref{syncp_example},
and the deadlock patterns
$D_1=\allowbreak\pattern{e_2, e_{16}}$, 
$D_5=\allowbreak\pattern{e_{29}, e_{16}}$, and
$D_6=\allowbreak\pattern{e_{29}, e_{19}}$.
We have
$\SPClosure{\tr_3}(\prev{\tr_3}(\{ e_{2},e_{16}\}))\allowbreak=
\set{e_1, \ldots, e_6, e_8, \ldots, e_{15}}$, 
$\SPClosure{\tr_3}(\prev{\tr_3}(\allowbreak\{ e_{29},e_{16}\}))=
\set{
e_1, \ldots, e_{15}, e_{28}}$, and $\SPClosure{\tr_3}(\prev{\tr_3}(\{ e_{29},e_{19}\}))=
\set{
e_1, \ldots, e_{18}, e_{28}}$.
Since $e_2 \in \allowbreak \SPClosure{\tr_3}(\allowbreak\prev{\tr_3}(\{ e_{2},e_{16}\}))$,
$D_1$ is not a sync-preserving deadlock.
However, $e_{29}, e_{16} \not \in \SPClosure{\tr_3}(\prev{\tr_3}(\{ e_{29},e_{16}\}))$, and $e_{29}, e_{19} \not \in \SPClosure{\tr_3}(\prev{\tr_3}(\{ e_{29},e_{19}\}))$, thus $D_5$ and $D_6$ are sync-preserving deadlocks (as we also concluded in \exref{sp-deadlocks}).
\end{example}

%!TEX root = main.tex

\subsection{Verifying Deadlock Patterns}
\seclabel{verify-patterns}

Given a deadlock pattern, we check if it constitutes a sync-preserving deadlock
by constructing the sync-preserving closure (\lemref{spclosure-deadlock}) in linear time.
Based on \defref{spclosure}, this can be done in 
an iterative manner. 
%start with the set of strict predecessors
%of events in the deadlock pattern, and in each iteration, add
%$\tho{}$ and $\rf{}$ predecessors of the current set of events,
%and additionally identify the release events that must be added to the set.
We 
(i)~start with the set of $\tho{}$ predecessors of the events in the deadlock pattern, and
(ii)~iteratively  add
$\tho{}$ and $\rf{}$ predecessors of the current set of events.
Additionally, we identify and add the release events that must be included in the set.
% 
% \defref{spclosure} suggests that the closure can be computed in
% an iterative manner as follows --- starting with the set of strict predecessors
% of the events in the deadlock pattern,  until a fixpoint is reached,
% in each iteration, we can add events that are $\tho{}$
% or $\rf{}$ predecessors of the current set of events,
% and additionally, identify the release events that must be added to the set.
%In order to ensure that the \emph{entire} fixpoint computation is
%performed in linear time, we use \emph{timestamps}.
%\hunkar{
We utilize \emph{timestamps} to ensure that the \emph{entire} fixpoint computation is
performed in linear time.
%}
% to ensure that indeed each iteration is performed in time 
% proportional to only the number of events actually added in that iteration.

\newcommand{\TS}[1]{\mathsf{TS}_{\tr}}

\myparagraph{Thread-read-from timestamps}{
	Given a set $\threads{}$ of threads, a timestamp 
	is simply a mapping $T : \threads{} \to \nats$.
	Given timestamps $T_1, T_2$, we 
	use the notations 
	$T_1 \cle T_2$ and $T_1 \mx T_2$ for pointwise comparison
	and pointwise maximum, respectively.
	% $T_1 \cle T_2 \equiv \forall p \in P, T_1(p) \leq T_2(p)$
	% and
	% $T_1 \mx T_2 = \lambda p \in P, \max\set{T_1(p), T_2(p)}$.
	For a set $U$ of timestamps, we write $\bigsqcup U$ to denote the
	pointwise maximum over all elements of $U$.
	% Further, for a set $U$ of $\threads{}$-indexed timestamps, we write $\bigsqcup U$ to denote
	% $\lambda p \in P, \max\set{T(p)}_{T \in U}$.
	Let $\trf{\tr}$ be the reflexive transitive closure
	of the relation $(\tho{\tr} \cup \setpred{(\rf{\tr}(e), e)}{\exists x \in \vars{\tr}, \OpOf{e} = \rd(x)})$;
	observe that
	$\trf{\tr}$ is a partial order.
	We define the 
	% threads-read-from
	 timestamp $\TS{\tr}^e$ of an event $e$ in $\tr$
	to be a $\threads{\tr}$-indexed timestamp as follows: $\TS{\tr}^e(t) = |\setpred{f}{f \trf{\tr} e}|$.
	% such that for every $t \in \threads{\tr}$,
	% \[
	% 	\TS{\tr}^e(t) = |\setpred{f}{f \trf{\tr} e}|
	% \]
	This ensures that
	for two events $e, e' \in \events{\tr}$,
	$e \trf{\tr} e'$ iff $\TS{\tr}^e \cle \TS{\tr}^{e'}$.
	For a set $S \subseteq \events{\tr}$, we overload the notation
	and say the timestamp of $S$ is $\TS{\tr}^S = \bigsqcup \set{\TS{\tr}^e}_{e \in S}$.
	% Observe that, for a set $S$ that is $\trf{\tr}$-downward closed, we have $\TS{\tr}^e \cle \TS{\tr}^S$ is true iff $e \in S$.
	% In other words, timestamps allow for a succinct representation of 
	%$\trf{\tr}$-downward closed sets.
	Given a trace $\tr$ with $\NumEvents$ events and
	$\NumThreads$ threads we can compute these timestamps for all the events in $O(\NumEvents \cdot \NumThreads)$ time, using a simple vector clock algorithm~\cite{Mattern89,Fidge91}.
}


\myparagraph{Computing sync-preserving closures}{
	Recall the basic template of the fixpoint computation.
	In each iteration, we identify the set of release events
	that must be included in the set, together with their $\trf{\tr}$-closure.
	In order to identify such events efficiently, for every
	thread $t$ and lock $\lk$, we maintain
	a FIFO queue $\AcqLst_{t, \lk}$ (\emph{critical section history} of $t$ and $\lk$) 
	to store 
	the sequence of events that acquire $\lk$ in thread $t$.
	In each iteration, we traverse each list
	to determine the last acquire event that belongs to the current set.
	For a given lock, we need to add the matching release events of
	all thus identified events to the closure, 
	except possibly the matching release event of the latest acquire event (see~\defref{spclosure}).
	This computation is performed using timestamps, as shown in \algoref{compute-closure}.
	Starting with a set $S$, the algorithm
	runs in time $O(|S|\cdot\NumThreads + \NumThreads\cdot\NumAcquires)$,
	where $\NumThreads$ and $\NumAcquires$ are respectively
	the number of threads and acquire events in $\tr$.
}

\begin{minipage}{0.45\textwidth}
	%!TEX root = main.tex

% Input: trace tr, ad a deadlock pattern D = <e0, e1, .., ek>,

% SPClosure(tr, S):
% 	closure = emptyset
% 	while closure changes:

\small
%\setlength{\textfloatsep}{0pt}
\begin{algorithm}[H]
\Input{Trace $\tr$, Timestamp $T_0$}
\BlankLine
% \myfun{\fixpoint{$\sigma$, $T_0$, $\set{\AcqLst_{t,\lk}}_{\lk\in \locks{\tr}, t \in \threads{\tr}}$}}{
	\Let $\set{\AcqLst_{t,\lk}}_{\lk\in \locks{\tr}, t \in \threads{\tr}}$ be the lock-acquisition histories in $\tr$ \;
	$T \gets T_0$ \;
	\Repeat{$T$ does not change}{
		\For{$\lk \in \locks{}$}{
			\ForEach{$t \in \threads{}$}{
				\Let $e_t$ be the last event in $\AcqLst_{t,\lk}$ with $\TS{\tr}^{e_t} \cle T$ \;
				Remove all earlier events in $\AcqLst_{t, \lk}$ 
			}
			\Let $e_{t*}$ be the last event in $\set{e_t}_{t\in \threads{\tr}}$ according to $\trord{\tr}$ \;
			$T$ := $T \mx \bigsqcup \setpred{\TS{\tr}^{\match{\tr}(e_t)}}{e_t \neq e_{t*}}$	\;
		}
	}
	\Return $T$
% }
\caption{CompSPClosure:\\ Computing sync-preserving closure.}
\algolabel{compute-closure}
\end{algorithm}
\normalsize

\end{minipage}
\hfill ~ %\unskip\ \vrule\ \hfill
\begin{minipage}{0.48\textwidth}
	%!TEX root = main.tex

\small
\begin{algorithm}[H]
%\Input{Trace $\tr$, Abstract deadlock pattern $\abst{D}$ of length $k$}
\Input{Trace $\tr$, $\abst{D}$ of length $k$}
\BlankLine
	\Let $F_0, \ldots, F_{k-1}$ be the sequences of acquires in $\abst{D}$ \;
	\Let $n_0, \ldots, n_{k-1}$ be the lengths of $F_0, \ldots, F_{k-1}$ \;
	\lForEach{$j \in \set{0, \ldots, k-1}$}{
		$i_j \gets 1$
	}
	$T \gets \lambda t, 0$ %\tcp{Timestamp of sync-preserving closure}
	\While{$\bigwedge\limits_{j=0}^{k-1} i_j < n_j$}{
		\Let $e_0 = F_0[i_0], \ldots, e_{k-1} = F_{k-1}[i_{k-1}]$ \;
		$S \gets \prev{\tr}{\set{e_0, \ldots, e_{k-1}}}$\;
		$T \gets$ \fixpoint{$\tr, T \mx \TS{\tr}^S$}\\   \linelabel{call-comp-closure}
		%\tcp{\algoref{compute-closure}}
		\If{$\forall j < k, \TS{\tr}^{e_{j}} \cle T$}{
			\report pattern $D = e_0, \ldots, e_{k-1}$ and \exit
		}
		\ForEach{$j \in \set{0, \ldots, k-1}$}{
			$i_j = \min\setpred{l \leq n_j}{\TS{\tr}^{F_j[l]} \not\cle T}$ %\tcp{Update index in $F_j$}
		}
	}
\caption{CheckAbsDdlck:\\ Checking an abstract deadlock pattern.}
\algolabel{abstract-pattern}
\end{algorithm}
\normalsize

\end{minipage}



\myparagraph{Checking a deadlock pattern}{
	After computing the timestamp $T$ of
	the closure (output of \algoref{compute-closure}, starting with the set of events in the given deadlock pattern),
	determining whether a given deadlock pattern
	$D = e_0, \ldots, e_{k-1}$ is a sync-preserving deadlock 
	can be performed in time $O(k\cdot\NumThreads)$ --- simply check if  $\forall i, \TS{\tr}(e_i) \not\cle T$.
	This gives an algorithm for checking if a deadlock pattern
	of length $k$ is sync-preserving that runs in time 
	$O(\NumThreads\cdot \NumEvents + k\cdot\NumThreads + \NumThreads\cdot \NumAcquires) = O(\NumEvents\cdot \NumThreads)$.
}
%!TEX root = main.tex

\subsection{Verifying Abstract Deadlock Patterns}
\seclabel{verify-abstract-patterns}

\myparagraph{Abstract acquires and abstract deadlock patterns}{
Given a thread $t$, a lock $\lk$ and a set of locks 
$L\subseteq \locks{\tr}\neq \emptyset$ with $\lk\not \in L$,
we define the \emph{abstract acquire} $\eta =\tuple{t, \lk, L, F}$, 
where $F=\sequence{e_1,\dots, e_n}$ is the sequence of all events
$e_i\in \events{\tr}$ (in trace-order) such that for each $i$, we have 
(i)~$\ThreadOf{e_i}=t$,
(ii)~$\OpOf{e_i} = \acq(\lk)$, and
(iii)~$\lheld{\tr}(e_i) = L$.
In words, the abstract acquire $\eta$ contains the sequence of all acquire events 
%of the same \emph{signature}.
%\hunkar{
of a specific thread, that access a specific lock and hold the same set of locks when executed, ordered as per thread order.
%}
An \emph{abstract deadlock pattern} of size $k$ in a trace $\tr$ is a sequence
$
\abst{D} = \eta_0,\dots,\eta_{k-1}
$
of abstract acquires $\eta_i=\tuple{t_i, \lk_i, L_i, F_i}$ such that
$t_0, \ldots, t_{k-1}$ are distinct threads, 
$\lk_0,  \ldots, \lk_{k-1}$ are distinct locks,
and $L_0, L_1, \ldots, L_{k-1} \subseteq \locks{\tr}$ are
such that $\lk_i \not\in L_i$, $\lk_i \in L_{(i+1)\% k}$ for every $i$,
and $L_i \cap L_j= \emptyset$ for every $i\neq j$.
Thus, an abstract deadlock pattern $\abst{D}$ succinctly encodes all 
concrete deadlock patterns $F_0\times F_1\times \dots \times F_{k-1}$,
called \emph{instantiations} of $\abst{D}$.
We also write $D\in \abst{D}$ to denote that 
$D\in F_0\times F_1\times \dots \times F_{k-1}$.
We say that $\abst{D}$ contains a sync-preserving deadlock if there 
exists some instantiation $D\in \abst{D}$ that is a sync-preserving deadlock.
See \figref{syncp_example} for an example.
}
Our next result is stated below, followed by its proof idea.
\begin{restatable}{lemma}{abstractlineartime}
\lemlabel{abstract-pattern-linear-time}
Consider a trace $\tr$ with $\NumEvents$ events and $\NumThreads$ threads, 
and an abstract deadlock pattern $\abst{D}$ of $\tr$.
We can determine if $\abst{D}$ contains a sync-preserving deadlock in $O(\NumThreads\cdot \NumEvents)$ time.
\end{restatable}

%The algorithm for checking if a given abstract pattern
%$\abst{D}$ contains a sync-preserving deadlock is presented in 
%\algoref{abstract-pattern}.
%The algorithm relies on exploiting monotonicity properties
%enjoyed by sync-preserving deadlocks that allow us to
%avoid a naive enumeration of all $O(\NumEvents^k)$ instantiations
%of $\abst{D}$.
%A detailed description of the algorithm is presented in
%\appref{verify-abstract-patterns}.

%\hcomment{moved the text from the appendix -- start}


An abstract deadlock pattern of length $k \geq 2$ can have $\NumEvents^k$ instantiations,
giving a naive enumerate-and-check algorithm running 
in time $O(\NumThreads\cdot\NumEvents^{k+1})$, which is prohibitively large.
Instead, we exploit 
(i)~the monotonicity properties of 
the sync-preserving closure (\propref{spclosure-monotone}) and
(ii)~instantiations of an abstract pattern (\corref{pattern-monotone}) 
that allow for an \emph{incremental}
algorithm that iteratively checks successive instantiations
of a given abstract deadlock pattern, while spending total $O(\NumEvents\cdot \NumThreads)$ time.
% , i.e., the running time is linear in $\NumEvents$, which is the dominating parameter.
The first observation allows us to re-use a prior
computation when checking later deadlock patterns.

% We first observe that the sync-preserving closure is monotonic
% with respect to the thread-order, and thus, when computing closure
% of later events, one can avoid re-computation.
\begin{restatable}{proposition}{spclosureMonotone}
	\proplabel{spclosure-monotone}
	For a trace $\tr$ and sets $S, S' \subseteq \events{\tr}$.
	If for every event $e \in S$, there is an event $e' \in S'$
	such that $e \tho{\tr} e'$, then
	$\SPClosure{\tr}(S) \subseteq \SPClosure{\tr}(S')$.
\end{restatable}

%\begin{proof}
%Follows from the definition of $\SPClosure{\tr}(S)$.
%\end{proof}

%\hunkar{
Consider $\tr_3$ in~\cref{fig:syncp_example} and let $S=\prev{\tr_3}{(\set{e_{29}, e_{16}})}$, and $S'=\prev{\tr_3}{(\set{e_{29}, e_{19}})}$.
The sets $S, S'$ satisfy the conditions of~\cref{prop:spclosure-monotone}, hence $\SPClosure{\tr_3}(S) \subseteq \SPClosure{\tr_3}(S')$, as computed in~\cref{ex:syncp-ex}.
%}
Next, we extend \propref{spclosure-monotone} to avoid redundant
computations when a sync-preserving deadlock is not found and later deadlock patterns must be checked.
Given two deadlock patterns
$D_1 = e_0, \ldots, e_{k-1}$ and $D_2 = f_0, \ldots, f_{k-1}$
of the same length $k$, 
we say $D_1 \prec D_2$ if
they are instantiations of a common abstract pattern $\abst{D}$
(i.e., $D_1, D_2 \in \abst{D}$) and 
for every $i<k$, $e_i \tho{\tr} f_i$.
\begin{restatable}{corollary}{pattern-monotone}
	\corlabel{pattern-monotone}
	Let $\tr$ be a trace and let $D_1 = e_0, \ldots, e_{k-1}$ and 
	$D_2 = f_0, \ldots f_{k-1}$ be deadlock
	patterns of size $k$ in $\tr$ such that $D_1 \prec D_2$.
	Let $S_1 = \set{e_0, \ldots, e_{k-1}}$ and $S_2 = \set{f_0, \ldots, f_{k-1}}$.
	If $\SPClosure{\tr}(\prev{\tr}(S_1))\cap S_2 \neq \emptyset$,
	then $\SPClosure{\tr}(\prev{\tr}(S_2))\cap S_2 \neq \emptyset$.
\end{restatable}


We now describe how \cref{prop:spclosure-monotone} and \cref{cor:pattern-monotone} are used in our algorithms, 
and illustrate them later in  \cref{ex:ex4}. % in \cref{sec:enumerate-patterns} provides an example.
% \hunkar{
% , we illustrate the application of  \propref{spclosure-monotone} and \corref{pattern-monotone}.
% }
% \hunkar{Provide an intuition here too?}
\algoref{abstract-pattern} checks if an abstract deadlock 
pattern contains a sync-preserving deadlock.
The algorithm iterates over the sequences $F_0, \ldots, F_{k-1}$ of 
acquires (one for each abstract acquire) in trace order.
For this, it maintains
indices $i_0, \ldots, i_{k-1}$ that point to entries in $F_0, \ldots, F_{k-1}$.
%At each step, it determines whether the current
%deadlock pattern $D = e_0, \ldots, e_{k-1}$
%by computing the sync-preserving closure
%of the predecessors of the deadlock pattern, and checks
%if it is disjoint from the deadlock.
%If so, it terminates.
%\hunkar{
At each step, it determines whether the current deadlock pattern $D = e_0, \ldots, e_{k-1}$
constitutes a sync-preserving deadlock by computing the sync-preserving closure
of the thread-local predecessors of the events of the deadlock pattern.
%The algorithm reports a deadlock if the sync-preserving closure is disjoint from the timestamps of the events $e_0, \ldots, e_{k-1}$.
The algorithm reports a deadlock if the sync-preserving closure does not contain any of $e_0, \ldots, e_{k-1}$.
%}
Otherwise, it looks for the next eligible deadlock pattern,
which it determines based on \corref{pattern-monotone}.
In particular, it advances the pointer $i_j$
all the way until an entry which is outside of the
closure computed so far.
Observe that the timestamp $T$ of the closure
computed in an iteration is being used in later iterations;
this is a consequence of \propref{spclosure-monotone}.
%\hunkar{
Furthermore, in the call to the \algoref{compute-closure} at \lineref{call-comp-closure}, we ensure that the list of acquires $\AcqLst_{t, \lk}$,
used in the function \fixpoint
is reused across iterations, and not
re-assigned to the original list of all acquire events.
%}
The correctness of this optimization follows from~\propref{spclosure-monotone}.
Let us now calculate the running time of \algoref{abstract-pattern}.
Each of the $\AcqLst_{t, \lk}$ in \fixpoint is traversed
at most once. 
Next, each element of the sequences $F_0, \ldots, F_{k-1}$
is also traversed at most once.
For each of these acquires, the algorithm spends $O(\NumThreads)$
time for vector clock updates.
The total time required is thus $O(\NumEvents \cdot \NumThreads)$.
This concludes the proof of \lemref{abstract-pattern-linear-time}.

%\hcomment{moved the text from the appendix -- end}
%!TEX root = main.tex

\subsection{The Algorithm \SyncPDOffline}
\seclabel{enumerate-patterns}

\begin{comment}
As with prior work, we model the problem of finding deadlock patterns as a cycle-detection problem
over a graph. % $\lkevgraph{\tr}$ whose nodes are elements of the trace.
%However, instead of a naive lock graph (with locks as nodes),
%the nodes in our graph store additional metadata that helps us
%to rule out false positives, and eventually enumerate all deadlock
%patterns, for which we need information about concrete dynamic events in the trace.
However, instead of considering locks as vertices (as in prior works~\cite{Bensalem2006,Bensalem2005}),
we use an \emph{abstract lock graph} $\lkevgraph{\tr}$.
% of the graph, our 
% representation, called \emph{abstract lock graph} $\lkevgraph{\tr}$ 
% effectively exploits \lemref{abstract-pattern-linear-time}.
% The key novelty compared to such graphs in the literature
% comes from combining it with the notion of sync-preserving deadlocks.
Every node of $\lkevgraph{\tr}$ is an abstract acquire of $\tr$, and thus
every deadlock pattern in $\tr$ appears in some abstract deadlock pattern defined by a cycle in $\lkevgraph{\tr}$.
Hence, we can detect \emph{all} sync-preserving deadlocks in $\tr$ by enumerating each of the cycles of $\lkevgraph{\tr}$, and if it constitutes an abstract deadlock pattern, use \lemref{abstract-pattern-linear-time} to check if it contains a sync-preserving deadlock in $\Otilde(\NumEvents)$ time.
% Hence, in linear time, our algorithm checks a whole abstract deadlock pattern, that typically summarizes polynomially many concrete deadlock patterns. 
\end{comment}

We now present the final ingredients of \SyncPDOffline.
We construct the \emph{abstract lock graph},
enumerate cycles in it, check whether
any cycle is an abstract deadlock pattern,
and if so, whether it contains sync-preserving deadlocks.

%!TEX root=../main.tex

\setlength{\textfloatsep}{10pt}
\begin{figure}[t]
%\begin{wrapfigure}{r}{0.35\textwidth}
\begin{subfigure}[t]{0.2\textwidth}
\scalebox{0.84}{
\begin{tikzpicture}[yscale=0.9]
	\tikzset{rectangle/.append style={draw=black}}
		\def\xstep{2.2}
		\def\ystep{0.5}
		\node[rectangle, align=center] (t1) at (-1, -7){
			$t_1, \LockColorTwo{\lk_2}, \{ \LockColorOne{\lk_1}\}, \sequence{e_{2}}$
		};
		\node[rectangle, align=center] (t31) at (-1, -8.7){
			$t_2, \LockColorOne{\lk_1}, \{ \LockColorTwo{\lk_2} \}, \sequence{e_{8}}$
		};
		\draw[->, thick, -Latex, bend right=10] ([xshift=-3mm]t1.south) to ([xshift=-3mm]t31.north);% (t31.north);
		\draw[->, thick, -Latex, bend right=10] (t31) to (t1);
\end{tikzpicture}
}
\end{subfigure}
~~~~
\begin{subfigure}[t]{0.2\textwidth}
\scalebox{0.84}{
\begin{tikzpicture}[yscale=0.9]
	\tikzset{rectangle/.append style={draw=black}}
		\def\xstep{2.2}
		\def\ystep{0.5}
		\node[rectangle, align=center] (t1) at (-1, -7){
			$t_2, \LockColorThree{\lk_3}, \{ \LockColorTwo{\lk_2}\}, \sequence{e_{4}}$
		};
		\node[rectangle, align=center] (t31) at (-1, -8.7){
			$t_3, \LockColorTwo{\lk_2}, \{ \LockColorThree{\lk_3} \}, \sequence{e_{18}}$
		};
		\draw[->, thick, -Latex, bend right=10] ([xshift=-3mm]t1.south) to ([xshift=-3mm]t31.north);% (t31.north);
		\draw[->, thick, -Latex, bend right=10] (t31) to (t1);
\end{tikzpicture}
}
\end{subfigure}
~~~~
% \hspace*{\fill}
\begin{subfigure}[t]{0.4\textwidth}
\scalebox{0.84}{
\begin{tikzpicture}[yscale=0.9]
	\tikzset{rectangle/.append style={draw=black}}
		\node[rectangle, align=center] (t1) at (-1, -7){
					$t_1, \LockColorTwo{\lk_2}, \{ \LockColorOne{\lk_1} \}$,
					$\sequence{e_2, e_4, e_{29}}$
				};
				\node[rectangle,align=center] (t2) at (2.5, -7){
					$t_2, \LockColorOne{\lk_1}, \{ \LockColorFour{\lk_4} \}$,
					$\sequence{e_{23}}$
				};
				\node[rectangle, align=center] (t31) at (-1, -8.7){
					$t_3, \LockColorOne{\lk_1}, \{ \LockColorTwo{\lk_2} \}$,
					$\sequence{e_{16}, e_{19}}$
				};
				\node[rectangle,align=center] (t32) at (2.5, -8.7){
					$t_3, \LockColorThree{\lk_3}, \{ \LockColorTwo{\lk_2} \}$,
					$\sequence{e_{13}}$
				};
				\draw[->, thick, -Latex] (t2) -- (t1);
				\draw[->, thick, -Latex] ([xshift=6mm]t1.south) to (t32.west);
				\draw[->, thick, -Latex, bend right=10] ([xshift=-3mm]t1.south) to ([xshift=-3mm]t31.north);
				\draw[->, thick, -Latex, bend right=10] (t31) to (t1);
\end{tikzpicture}
}
\end{subfigure}
% 
% \caption{
% A  trace $\tr$ with a sync-preserving deadlock, stalling $t_2$ on $e_4$ and $t_3$ on $e_{18}$.
% Underlined events indicate the slice of $\tr$ that forms a sync-preserving lower set of $\tr$,
% and serves as a valid witness for the deadlock.
% \figlabel{motivating}
% }
\caption{
Abstract lock graphs of the traces from~\figref{motivating-no-dl} (left), \figref{motivating-dl} (middle) and \figref{syncp_example} (right).
% \Andreas{@Umang: You can use this space for more figures, examples}
\figlabel{motivating-graph}
}
 \end{figure}
%\end{wrapfigure}
\myparagraph{Abstract lock graph}{
The abstract lock graph of $\tr$ is a directed graph
$
\lkevgraph{\tr} = (V_\tr, E_\tr)
$,
where
\begin{itemize}
	\item $V_\tr=\{\tuple{t_1, \lk_1, L_1, F_1},\dots, \tuple{t_k, \lk_k, L_k, F_k}\}$ is the set of abstract acquires of $\tr$, and
	\item for every $\eta_1 {=} \tuple{t_1, \lk_1, L_1, F_1}, 
	\eta_2 {=} \tuple{t_2, \lk_2, L_2, F_2} \in V_\tr$, we have  $(\eta_1, \eta_2) \in E_\tr$ iff 
	$t_1 \neq t_2$, $\lk_1 \in L_2$, and $L_1 \cap L_2 = \emptyset$.
\end{itemize}
A node $\tuple{t_1, \lk_1, L_1, F_1}$ signifies that there is an event $\acq_1(\lk_1)$ performed by thread $t_1$ while holding the locks in $L_1$.
The last component $F_1$ is a list which contains all such events $\acq_1$ in order of appearance in $\tr$.
An edge $(\eta_1, \eta_2)$ signifies that the lock $\lk_1$ acquired by each of the
events $\acq_1 \in F_1$ was held by $t_2$ when it executed each of $\acq_2\in F_2$ while not holding a common lock.
%See \figref{lock_graph} for the lock graph to the trace $\tr_2$ of \figref{syncp_example}.
The abstract lock graph can be constructed incrementally
as new events appear in $\tr$.
For $\NumEvents$ events, $\NumLocks$ locks and nesting depth $\NestingDepth$,
the graph has
$|V_\tr| = O\big(\NumThreads \cdot \NumLocks^{\NestingDepth}\big)$ vertices,
$|E_\tr| = O(|V_\tr|\cdot \NumLocks^{\NestingDepth - 1})$ edges and can be
constructed in $O(\NumEvents \cdot \NestingDepth)$ time.
See \figref{motivating-graph} for examples.
In the left graph, the cycle marks an abstract deadlock pattern and its single concrete deadlock pattern
$\abst{D}=\{ e_2 \} \times \{ e_{8} \}$, and similarly for the middle graph where $\abst{D}=\{ e_4 \} \times \{ e_{18} \}$.
%In the graph on the right, we have the abstract lock graph of the trace $\tr_3$ of  \figref{syncp_example}.
In the right graph, there is a unique cycle which marks an abstract deadlock pattern of $6$ concrete deadlock patterns
$\abst{D}=\{ e_2, e_4, e_{29} \} \times \{ e_{16}, e_{19} \}$.


% \begin{example}
% \figref{lock_graph} illustrates the lock graph of the trace $\tr_2$ in \figref{syncp_example}.
% We have a unique cycle, of length $2$, that marks a corresponding abstract deadlock pattern of $6$ concrete deadlock patterns
% $\abst{D}=\{ e_2, e_4, e_{29} \} \times \{ e_{16}, e_{19} \}$.
% \end{example}
}
%!TEX root = main.tex
\setlength{\textfloatsep}{10pt}
\begin{algorithm}[h]
\small
\DontPrintSemicolon
\SetInd{0.4em}{0.4em}
\KwIn{
A trace $\tr$.
}
\KwOut{All abstract deadlock patterns of $\tr$ that contain a sync-preserving deadlock.}
\BlankLine
Construct the abstract lock graph $\lkevgraph{\tr}$\\
\ForEach{cycle $C=\tuple{\eta_0,\dots, \eta_{k-1}}$ in $G$}{
Let $\eta_i=\tuple{t_i, \lk_i, L_i, F_i}$\\
\uIf(\tcp*[h]{$C$ is an abstract deadlock pattern}){$\forall i\neq j$ we have $t_i\neq t_j$ and $\lk_i\neq \lk_j$ and $L_i\cap L_j=\emptyset$ }{
%
\lIf{$\checkAbsDeadP(C)$}{
%Report the deadlock pattern in $C$ that constitutes a sync-preserving deadlock
Report that $C$ contains a sync-preserving deadlock
}
}
}
\caption{
Algorithm $\SyncPDOffline$.
\algolabel{offline}
}
\end{algorithm}
\normalsize

% \vspace{-0.2in}
\myparagraph{Algorithm $\SyncPDOffline$}{
	It is straightforward to verify that every abstract deadlock pattern of $\tr$ appears as a (simple) cycle in $\lkevgraph{\tr}$.
	However, the opposite is not true.
	A cycle $C = \eta_0, \eta_1, \ldots, \eta_{k-1}$ of $\lkevgraph{\tr}$,
	where $\eta_i = \tuple{t_i, \lk_i, L_i, F_i}$ defines an abstract deadlock pattern
	if additionally every thread $t_i$ is distinct, all every lock $\lk_i$ is distinct, and all sets $L_i$ are pairwise disjoint.
	This gives us a simple recipe for enumerating all abstract deadlock patterns, by using
	Johnson's algorithm~\cite{Johnson1975} to enumerate
	every simple cycle $C$ in $\lkevgraph{\tr}$, and check whether $C$ is an abstract deadlock pattern.
	We thus arrived at our offline algorithm $\SyncPDOffline$ (\algoref{offline}).
	The running time depends linearly on the length of $\tr$
	and the number of cycles in $\lkevgraph{\tr}$.% (\thmref{syncp_offline}).
}




\begin{restatable}{theorem}{syncpoffline}
\thmlabel{syncp_offline}
Consider a trace $\tr$ of $\NumEvents$ events, $\NumThreads$ threads and $\textsf{Cyc}_\tr$ cycles in $\lkevgraph{\tr}$.
The algorithm $\SyncPDOffline$ reports all sync-preserving deadlocks of $\tr$ in time $O(\NumEvents \cdot \NumThreads\cdot \textsf{Cyc}_\tr)$.
% When the number of threads and lock-nesting depth is small, $\SyncPDOffline$ runs in $\Otilde(\NumEvents \cdot \poly{\NumLocks})$ time.
\end{restatable}

Although, in principle, we can have exponentially many cycles in $\lkevgraph{\tr}$,
because the nodes of $\lkevgraph{\tr}$ are \emph{abstract} acquire events (as opposed to \emph{concrete}), we expect that the number of cycles (and thus abstract deadlock patterns) in $\lkevgraph{\tr}$ remains small, even though the number of \emph{concrete} deadlock patterns can grow exponentially.
Since $\SyncPDOffline$ spends linear time per abstract deadlock pattern, we  have an efficient procedure overall 
for constant $\NumThreads$ and $\NumLocks$.
We evaluate $\textsf{Cyc}_\tr$ experimentally in \secref{experiments}, and confirm that it is very small compared to the number of concrete deadlock patterns in $\tr$.
%\hunkar{
Nevertheless, $\textsf{Cyc}_\tr$ can become exponential when $\NumThreads$ and $\NumLocks$ are large, making~\cref{algo:offline} run in exponential time.
Note that this barrier is unavoidable in general, as proven in \thmref{pattern-w1-hardness-pattern}.
%}

%!TEX root = main.tex

\begin{example}
    \exlabel{ex4}
    We illustrate how the lock graph is integrated inside $\SyncPDOffline$.	
    Consider the trace $\tr_3$ in \figref{syncp_example}.
    It contains $6$ concrete deadlock patterns $D_1 \ldots D_6$.
    A naive algorithm would enumerate each pattern explicitly until it finds a deadlock. 
    However, the tight interplay between the abstract lock graph and sync-preservation enables a more efficient procedure.
    $\SyncPDOffline$ starts by computing the sync-preserving closure of $D_1$,
    $\SPClosure{\tr_3}(\prev{\tr_3}(\{ e_{2},e_{16}\}))=\set{e_1, \ldots, e_6, \, e_8, \ldots, e_{15}}$.
    As $e_2 \in \SPClosure{\tr_3}(\prev{\tr_3}(\{ e_{2},e_{16}\}))$, we conclude that $D_1$ is not a sync-preserving deadlock.
    The algorithm further deduces that the deadlock patterns $D_2$, $D_3$ and $D_4$ are also not sync-preserving deadlocks, as follows.
    $D_2=\pattern{e_2, e_{19}}$ shares a common event $e_2$ with $D_1$ but contains the event $e_{19}$ instead of $e_{16}$, while $e_5 \in \SPClosure{\tr_3}(\prev{\tr_3}(\{ e_{2},e_{16}\}))$.
    Since $e_{16} \tho{\tr_3} e_{19}$, and the sync-preserving closure grows monotonically (\propref{spclosure-monotone}), the sync-preserving closure of $e_2$ and $e_{19}$ will also contain $e_5$ (and thus $e_2$).
    Therefore, $D_2$ cannot be a sync-preserving deadlock.
    This reasoning is formalized in \corref{pattern-monotone}, and also applies to $D_3$ and $D_4$.
    Next, the algorithm proceeds with $D_5$.
    The above reasoning does not hold for $D_5$ as $\SPClosure{\tr_3}(\prev{\tr_3}(\{ e_{2},e_{16}\})) \cap S_5 = \emptyset$ 
    where $S_5=\set{e_{29}, e_{16}}$.
    The algorithm then computes the sync-preserving closure of $D_5$, reports a deadlock (\exref{syncp-ex}) and stops analyzing this abstract deadlock pattern.
    %Hence, $D_6$ is also not considered.
    %As the remaining pattern $D_6$ originates from the same abstract deadlock pattern, it need not be considered.
    In the end, we have only explicitly enumerated the deadlock patterns $D_1$ and $D_5$.
\end{example}

\begin{remark}
Although the concept of lock graphs exists in the literature~\cite{Havelund2000,Bensalem2005,Cai2020,cai14magiclock},
our notion of \emph{abstract} lock graphs is novel and tailored to sync-preserving deadlocks.
The closest concept to abstract lock graphs is that of equivalent cycles~\cite{cai14magiclock}. 
However, equivalent cycles unify all the concrete patterns of a given abstract pattern and lead to unsound deadlock detection, which was indeed their use. 
\end{remark}
%%!TEX root = main.tex
\subsection{Overall Algorithm}
\seclabel{overall-offline}




