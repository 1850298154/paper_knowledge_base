%!TEX root = main.tex

\subsection{Verifying Deadlock Patterns}
\seclabel{verify-patterns}

Given a deadlock pattern, we check if it constitutes a sync-preserving deadlock
by constructing the sync-preserving closure (\lemref{spclosure-deadlock}) in linear time.
Based on \defref{spclosure}, this can be done in 
an iterative manner. 
%start with the set of strict predecessors
%of events in the deadlock pattern, and in each iteration, add
%$\tho{}$ and $\rf{}$ predecessors of the current set of events,
%and additionally identify the release events that must be added to the set.
We 
(i)~start with the set of $\tho{}$ predecessors of the events in the deadlock pattern, and
(ii)~iteratively  add
$\tho{}$ and $\rf{}$ predecessors of the current set of events.
Additionally, we identify and add the release events that must be included in the set.
% 
% \defref{spclosure} suggests that the closure can be computed in
% an iterative manner as follows --- starting with the set of strict predecessors
% of the events in the deadlock pattern,  until a fixpoint is reached,
% in each iteration, we can add events that are $\tho{}$
% or $\rf{}$ predecessors of the current set of events,
% and additionally, identify the release events that must be added to the set.
%In order to ensure that the \emph{entire} fixpoint computation is
%performed in linear time, we use \emph{timestamps}.
%\hunkar{
We utilize \emph{timestamps} to ensure that the \emph{entire} fixpoint computation is
performed in linear time.
%}
% to ensure that indeed each iteration is performed in time 
% proportional to only the number of events actually added in that iteration.

\newcommand{\TS}[1]{\mathsf{TS}_{\tr}}

\myparagraph{Thread-read-from timestamps}{
	Given a set $\threads{}$ of threads, a timestamp 
	is simply a mapping $T : \threads{} \to \nats$.
	Given timestamps $T_1, T_2$, we 
	use the notations 
	$T_1 \cle T_2$ and $T_1 \mx T_2$ for pointwise comparison
	and pointwise maximum, respectively.
	% $T_1 \cle T_2 \equiv \forall p \in P, T_1(p) \leq T_2(p)$
	% and
	% $T_1 \mx T_2 = \lambda p \in P, \max\set{T_1(p), T_2(p)}$.
	For a set $U$ of timestamps, we write $\bigsqcup U$ to denote the
	pointwise maximum over all elements of $U$.
	% Further, for a set $U$ of $\threads{}$-indexed timestamps, we write $\bigsqcup U$ to denote
	% $\lambda p \in P, \max\set{T(p)}_{T \in U}$.
	Let $\trf{\tr}$ be the reflexive transitive closure
	of the relation $(\tho{\tr} \cup \setpred{(\rf{\tr}(e), e)}{\exists x \in \vars{\tr}, \OpOf{e} = \rd(x)})$;
	observe that
	$\trf{\tr}$ is a partial order.
	We define the 
	% threads-read-from
	 timestamp $\TS{\tr}^e$ of an event $e$ in $\tr$
	to be a $\threads{\tr}$-indexed timestamp as follows: $\TS{\tr}^e(t) = |\setpred{f}{f \trf{\tr} e}|$.
	% such that for every $t \in \threads{\tr}$,
	% \[
	% 	\TS{\tr}^e(t) = |\setpred{f}{f \trf{\tr} e}|
	% \]
	This ensures that
	for two events $e, e' \in \events{\tr}$,
	$e \trf{\tr} e'$ iff $\TS{\tr}^e \cle \TS{\tr}^{e'}$.
	For a set $S \subseteq \events{\tr}$, we overload the notation
	and say the timestamp of $S$ is $\TS{\tr}^S = \bigsqcup \set{\TS{\tr}^e}_{e \in S}$.
	% Observe that, for a set $S$ that is $\trf{\tr}$-downward closed, we have $\TS{\tr}^e \cle \TS{\tr}^S$ is true iff $e \in S$.
	% In other words, timestamps allow for a succinct representation of 
	%$\trf{\tr}$-downward closed sets.
	Given a trace $\tr$ with $\NumEvents$ events and
	$\NumThreads$ threads we can compute these timestamps for all the events in $O(\NumEvents \cdot \NumThreads)$ time, using a simple vector clock algorithm~\cite{Mattern89,Fidge91}.
}


\myparagraph{Computing sync-preserving closures}{
	Recall the basic template of the fixpoint computation.
	In each iteration, we identify the set of release events
	that must be included in the set, together with their $\trf{\tr}$-closure.
	In order to identify such events efficiently, for every
	thread $t$ and lock $\lk$, we maintain
	a FIFO queue $\AcqLst_{t, \lk}$ (\emph{critical section history} of $t$ and $\lk$) 
	to store 
	the sequence of events that acquire $\lk$ in thread $t$.
	In each iteration, we traverse each list
	to determine the last acquire event that belongs to the current set.
	For a given lock, we need to add the matching release events of
	all thus identified events to the closure, 
	except possibly the matching release event of the latest acquire event (see~\defref{spclosure}).
	This computation is performed using timestamps, as shown in \algoref{compute-closure}.
	Starting with a set $S$, the algorithm
	runs in time $O(|S|\cdot\NumThreads + \NumThreads\cdot\NumAcquires)$,
	where $\NumThreads$ and $\NumAcquires$ are respectively
	the number of threads and acquire events in $\tr$.
}

\begin{minipage}{0.45\textwidth}
	%!TEX root = main.tex

% Input: trace tr, ad a deadlock pattern D = <e0, e1, .., ek>,

% SPClosure(tr, S):
% 	closure = emptyset
% 	while closure changes:

\small
%\setlength{\textfloatsep}{0pt}
\begin{algorithm}[H]
\Input{Trace $\tr$, Timestamp $T_0$}
\BlankLine
% \myfun{\fixpoint{$\sigma$, $T_0$, $\set{\AcqLst_{t,\lk}}_{\lk\in \locks{\tr}, t \in \threads{\tr}}$}}{
	\Let $\set{\AcqLst_{t,\lk}}_{\lk\in \locks{\tr}, t \in \threads{\tr}}$ be the lock-acquisition histories in $\tr$ \;
	$T \gets T_0$ \;
	\Repeat{$T$ does not change}{
		\For{$\lk \in \locks{}$}{
			\ForEach{$t \in \threads{}$}{
				\Let $e_t$ be the last event in $\AcqLst_{t,\lk}$ with $\TS{\tr}^{e_t} \cle T$ \;
				Remove all earlier events in $\AcqLst_{t, \lk}$ 
			}
			\Let $e_{t*}$ be the last event in $\set{e_t}_{t\in \threads{\tr}}$ according to $\trord{\tr}$ \;
			$T$ := $T \mx \bigsqcup \setpred{\TS{\tr}^{\match{\tr}(e_t)}}{e_t \neq e_{t*}}$	\;
		}
	}
	\Return $T$
% }
\caption{CompSPClosure:\\ Computing sync-preserving closure.}
\algolabel{compute-closure}
\end{algorithm}
\normalsize

\end{minipage}
\hfill ~ %\unskip\ \vrule\ \hfill
\begin{minipage}{0.48\textwidth}
	%!TEX root = main.tex

\small
\begin{algorithm}[H]
%\Input{Trace $\tr$, Abstract deadlock pattern $\abst{D}$ of length $k$}
\Input{Trace $\tr$, $\abst{D}$ of length $k$}
\BlankLine
	\Let $F_0, \ldots, F_{k-1}$ be the sequences of acquires in $\abst{D}$ \;
	\Let $n_0, \ldots, n_{k-1}$ be the lengths of $F_0, \ldots, F_{k-1}$ \;
	\lForEach{$j \in \set{0, \ldots, k-1}$}{
		$i_j \gets 1$
	}
	$T \gets \lambda t, 0$ %\tcp{Timestamp of sync-preserving closure}
	\While{$\bigwedge\limits_{j=0}^{k-1} i_j < n_j$}{
		\Let $e_0 = F_0[i_0], \ldots, e_{k-1} = F_{k-1}[i_{k-1}]$ \;
		$S \gets \prev{\tr}{\set{e_0, \ldots, e_{k-1}}}$\;
		$T \gets$ \fixpoint{$\tr, T \mx \TS{\tr}^S$}\\   \linelabel{call-comp-closure}
		%\tcp{\algoref{compute-closure}}
		\If{$\forall j < k, \TS{\tr}^{e_{j}} \cle T$}{
			\report pattern $D = e_0, \ldots, e_{k-1}$ and \exit
		}
		\ForEach{$j \in \set{0, \ldots, k-1}$}{
			$i_j = \min\setpred{l \leq n_j}{\TS{\tr}^{F_j[l]} \not\cle T}$ %\tcp{Update index in $F_j$}
		}
	}
\caption{CheckAbsDdlck:\\ Checking an abstract deadlock pattern.}
\algolabel{abstract-pattern}
\end{algorithm}
\normalsize

\end{minipage}



\myparagraph{Checking a deadlock pattern}{
	After computing the timestamp $T$ of
	the closure (output of \algoref{compute-closure}, starting with the set of events in the given deadlock pattern),
	determining whether a given deadlock pattern
	$D = e_0, \ldots, e_{k-1}$ is a sync-preserving deadlock 
	can be performed in time $O(k\cdot\NumThreads)$ --- simply check if  $\forall i, \TS{\tr}(e_i) \not\cle T$.
	This gives an algorithm for checking if a deadlock pattern
	of length $k$ is sync-preserving that runs in time 
	$O(\NumThreads\cdot \NumEvents + k\cdot\NumThreads + \NumThreads\cdot \NumAcquires) = O(\NumEvents\cdot \NumThreads)$.
}