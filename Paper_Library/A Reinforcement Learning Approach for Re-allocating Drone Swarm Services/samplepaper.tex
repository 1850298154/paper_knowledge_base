% This is samplepaper.tex, a sample chapter demonstrating the
% LLNCS macro package for Springer Computer Science proceedings;
% Version 2.20 of 2017/10/04
%
\documentclass[runningheads]{llncs}
%
\usepackage{graphicx}
\usepackage{color,soul}
\usepackage{algorithmic}
\usepackage{algorithm}
\usepackage[table,xcdraw,usenames,dvipsnames]{xcolor}
\usepackage{subcaption}
\usepackage{amssymb,mathtools}
\usepackage{enumitem}
\setlength{\intextsep}{0pt}
\setlength{\textfloatsep}{0pt}
\setlength{\floatsep}{0pt}

%% Save the class definition of \subparagraph
\let\llncssubparagraph\subparagraph
%% Provide a definition to \subparagraph to keep titlesec happy
\let\subparagraph\paragraph
%% Load titlesec
\usepackage[compact]{titlesec}
%% Revert \subparagraph to the llncs definition
\let\subparagraph\llncssubparagraph

\usepackage[compact]{titlesec}  
%\titlespacing{\section}{0pt}{0pt}{0pt}


\renewcommand{\algorithmiccomment}[1]{#1}



% \setlength{\textfloatsep}{1\baselineskip plus 0.2\baselineskip minus 0.5\baselineskip}


% Used for displaying a sample figure. If possible, figure files should
% be included in EPS format.
%
% If you use the hyperref package, please uncomment the following line
% to display URLs in blue roman font according to Springer's eBook style:
% \renewcommand\UrlFont{\color{blue}\rmfamily}

\begin{document}
%
\title{A Reinforcement Learning Approach for Re-allocating Drone Swarm Services}
%

\author{Balsam Alkouz\inst{1}\orcidID{0000-0001-7938-4438} \and
Athman Bouguettaya\inst{1}\orcidID{0000-0003-1254-8092}}


\institute{University of Sydney, NSW, Australia
\email{\{balsam.alkouz,athman.bouguettaya\}@sydney.edu.au}}

\titlerunning{An RL Approach for Re-allocating Drone Swarm Services}
%
\maketitle              % typeset the header of the contribution
%
\begin{abstract}
We propose a novel framework for the re-allocation of drone swarms for delivery services known as Swarm-based Drone-as-a-Service (SDaaS). The re-allocation framework ensures maximum profit to drone swarm providers while meeting the time requirement of service consumers. The constraints in the delivery environment (e.g., limited recharging pads) are taken into consideration. We utilize reinforcement learning (RL) to select the best allocation and scheduling of drone swarms given a set of requests from multiple consumers. We conduct a set of experiments to evaluate and compare the efficiency of the proposed approach considering the provider's profit and run-time efficiency.


\keywords{Drones swarm \and Service composition \and Swarm re-allocation \and Homogeneous swarms \and Provider-centric \and Congestion-aware}
\end{abstract}
%
%
%
\section{Introduction}
%%sdaas why needed and added capabilities
%dependency during covid on delivery
\emph{Swarm-based Drone-as-a-Service (SDaaS)} is a concept that describes services offered by swarms of drones \cite{alkouz2020swarm}. The SDaaS notion is an augmentation on the Drone-as-a-Service (DaaS) concept that describes services offered by single drones \cite{hamdi2021drone}. It offers added capabilities to cover services a single drone is not capable of achieving. Examples of these services include search and rescue \cite{cardona2019robot}, sky shows and entertainment \cite{waibel2017drone}, and delivery of goods \cite{alkouz2020formation}. Our focus is on the use of drone swarms in delivery. An increasing dependency on drone delivery is perceived especially during pandemics, as they are contact-less and fast. Therefore, robust and effective deliveries of multiple/heavier packages are needed. Such deliveries are only possible using a swarm of drones as flight regulations only allow the use of small drones (payload$<$2.5 kg) to deliver in the city\footnote{https://www.faa.gov/uas/advanced\_operations/package\_delivery\_drone}. In addition, swarms of drones in delivery are capable of covering longer trips by distributing the payload over several drones decreasing the rate of battery consumption \cite{alkouz2020swarm}.
% \cite{euchi2020drones}
%%formal definition and services functional and nonfunctional
%skyway
Swarm-based drone deliveries operating in a city are assumed to be flying within line of sight segments in a skyway network \cite{lee2021package}. The skyway network nodes are assumed to be building rooftops equipped with recharging pads that a swarm may land on to extend its flight range \cite{shahzaad2020game}. We formally define an SDaaS as a swarm carrying packages and travelling in a skyway segment frome node A to node B. The composition of optimal segments between a source node and a destination node would result in an optimal composite SDaaS service. An SDaaS service maps to the key components of service computing, i.e. functional and non-functional attributes \cite{shahzaad2021robust}. The function of an SDaaS is the successful delivery of packages by a swarm between two nodes. The non-functional attributes or the Quality of Services (QoS) include the delivery time, cost, etc.

%% steps involved in sdaas and need for allocation
Three main steps are involved in a successful SDaaS delivery. First, an optimal swarm members allocation approach is essential to serve multiple consumers requests in a day. Second, an optimal path composition method is required to optimize the QoS. Third, a failure-recovery solution is necessary in case of uncertainties. In this paper, we focus on the first step, with respect to the composition, to optimally allocate swarms to consumers requests from a provider point of view. The last step, i.e. failure recovery, is the future extension of this work. An optimal allocation is key in assuring that a provider owned drones are optimally \textit{utilized} and \textit{re-utilized} within a day. Therefore, fulfilling as many consumers requests as possible and increasing a provider profit.\looseness=-1

%%challenges in allocation and what we propose
%reuse of drones
%optimal scheduling
There are several challenges in the swarms allocation problem. First, a provider owns a \textit{limited set of drones} that needs to be utilized maximally. Second, the delivery time of consumers requests may overlap as they need to be delivered within \textit{strict time windows}. Hence, requests that maximize the providers profit need to be allocated. Third, the requests need to be served in a way that optimizes the \textit{re-utilization} of drones. Hence, within a time window, a swarm may be reused if its round trip time to the first request is smaller than the time window. This problem is challenging since the allocation of any swarm is highly dependent on the availability of other drones because they are \textit{re-allocatable}. In addition, each swarm is bounded by a \textit{Round Trip Time} from the source to the destination and back to the destination. This means that the allocation of any request highly \textit{affects the allocation of other requests in the same time-window and other windows} as the provider owns a limited set of drones.
%Previous approaches in multi-robot task allocations only assumed machine specific tasks \cite{carlier1989algorithm}, time unconstrained tasks  \cite{gigliotta2018equal}, and a single machine use within a time window \cite{khamis2015multi}.
We propose to allocate \textit{any} available drones to \textit{multiple time-constrained} requests, and \textit{re-utilize} the drones multiple times within a time window to \textit{maximize a providers profit}. We summarize our main contributions as following: \looseness=-1
\begin{itemize}[nosep]
     %\item Define a new set of SDaaS services re-allocation constraints.
    \item A modified A* congestion-aware algorithm to compose SDaaS services.
    \item An RL SDaaS allocation algorithm to maximize providers profit.
\end{itemize}


% \subsection{Motivating Scenario and Problem Statement}


% \begin{figure}
% \includegraphics[width=0.9\textwidth]{Figures/Motivating Scenario.png}
% \caption{Swarm-based services Composition and Allocation in a Skyway network} 
% %\vspace{-2cm}
% \label{motivating}
% \end{figure}

% Let's assume that, within a day, several medical facilities request multiple pieces of medical equipment from a warehouse. The requests are received at least a night before the delivery with a specified preferred time window for the arrival of packages. Since each facility is requesting multiple items then a swarm of drones needs to deliver these items at the same time together. The warehouse, on the other hand, receives the requests in batch and needs to allocate swarms of drones to serve these requests. However, the warehouse owns a limited fleet of homogeneous drones that would form swarms to serve the requests (See Fig. \ref{motivating}). The trip a swarm takes to a certain facility is determined by the composition of the optimal SDaaS services in a skyway network. We assume that a skyway network is made up of nodes representing building rooftops equipped with recharging pads. The number of pads is different at every node. The segments connecting the nodes are line of sight paths following the flight aviation authorities regulations. Since multiple swarms would be operating at the same time, congestion may occur at intermediate recharging nodes. Hence, the services composition should cater for these constraints. %We assume the environment is deterministic, i.e. the state of the nodes is known at a given time.

% The goal of this paper is to optimise the swarms-requests \textit{allocation} from a provider perspective. Most profitable requests need to be allocated while ensuring the maximum utilization of the provider owned drones. For this purpose, the swarms need to be \textit{allocated} and \textit{re-allocated} multiple times to serve as many requests as possible. We propose to \textit{use the drones multiple times within a time window} if possible. The proposed framework should consider overlapping requests and the optimal placement of requests in the timeline to serve as many requests as possible while adhering to the delivery time window constraint. As shown in Fig.\ref{motivating}, the requests arrival time (AT) to the destination need to be within the specified time window. However, the full Round Trip Time (RTT) needs to be considered to serve as many overlapping requests as possible. This is essential since the provider only owns a \textit{limited} set of drones that may serve a limited number of requests at a time.\looseness=-1

\section{Related Work}
%%swarms in different applications 
A robotic swarm is a set of robots that collectively solve a problem to achieve a common goal. %They are inspired by swarms in natural systems, such as swarms of bees, birds, and fish. 
% Some of the main advantages of swarms include adaptability, robustness, and scalability \cite{dorigo2013swarmanoid}. Recent literature spotlights applications of swarms that are being tested or envisioned for the future. Such applications include drone shows \cite{waibel2017drone}, reconnaissance missions \cite{mishra2020drone}, and the delivery of packages \cite{alkouz2020swarm}.\looseness=-1
%Swarms are capable of solving complex tasks through their collective behaviour \cite{shrit2017new}.
%% swarms and delivery
%%swarms need for allocation 
In delivery, majority of literature refer to swarms of drones as multiple single independent drones managed to deliver multiple independent deliveries \cite{kuru2019analysis}. However, we refer to a swarm as a set of drones carrying multiple packages for a single delivery operation. In this regard, a sequential and parallel delivery services composition using a swarm of drones was proposed \cite{alkouz2020swarm}. 
%A sequential composition uses a static swarm that is formed at the source and does not split midway. A parallel composition uses a dynamic swarm that is capable of splitting and merging throughout the trip. 
%Moreover, the study of flight formation in swarm drone deliveries was explored to study the effect on the energy consumption \cite{alkouz2020formation}. 
%In comparison, vehicle routing problems (VRP) with drones mainly deal with finding the optimal set of routes for a fleet of vehicles to deliver to a set of consumers \cite{wang2019vehicle}. Although the problem proposed in this paper is similar to VRP in terms of serving multiple consumers by a fleet of drones, it is fundamentally different in that it is not a routing problem but a \textit{resource allocation and scheduling problem}, and a swarm is assumed to deliver to a \textit{single destination at a time}. 
While drone swarms in delivery represent a major advancement, developing \textit{swarm allocation} methods is essential to unlock their full potential and obtain teamwork benefit \cite{gigliotta2018equal}.\looseness=-1

%In this regard, Unmanned Life demonstrated its emergency response delivery operation using a swarm of drones\footnote{https://unmanned.life/}.

%% multirobot task allocation
%%types and lackings
%% MRTA and MRTS
Multi-Robot Task Allocation (MRTA) addresses the assignment of set of tasks to a set of robots \cite{elfakharany2020towards}. The robots need to be optimally allocated to tasks to optimize the overall team performance \cite{khamis2015multi}. Multi Robot Task Scheduling (MRTS) deals with the scheduling of the tasks to minimize the overall cost, make it be: time, money, or energy. Most multi-robot systems deal with MRTA and MRTS as two different steps. However, the decoupling of these steps leads to partial observability and lack of full insights \cite{elfakharany2020towards}. In addition, to the best of our knowledge, most work done in MRTA does not deal with the \textit{multiple re-allocations} of the robots in a \textit{time-constrained} environment. Hence, we propose to couple the MRTA and MTRS problems and deal with multiple re-allocations of drone swarms in a time-constrained environment using a \textit{service-oriented approach}.\looseness=-1



%%deep reinforcement learning in services composition
%% services and swarms
The service paradigm is a key enabler of drone deliveries in a skyway network. It ensures congruent and effective provisioning of drone-based deliveries \cite{shahzaad2021resilient}. 
%Single drone delivery services is presented as Drone-as-a-Service (DaaS) \cite{hamdi2021drone}. However, to cover more complex tasks, Swarm-based Drone-as-a-Service (SDaaS) is introduced \cite{alkouz2020swarm}. 
Previous works discuss the optimal composition of services, i.e. composing the best path from the source to the destination \cite{alkouz2020swarm}. 
%This work extends and augments the SDaaS framework and introduces the re-allocations of drones to serve SDaaS services. The approach proposed uses reinforcement learning to optimally allocate and schedule the requests to serve as many requests as possible and increase the providers profit. 
In a different application, a reinforcement learning approach to compose moving WiFi hotspot services was proposed \cite{gharineiat2021deep}. Majority of the existing work uses deep reinforcement learning for services composition and not allocation \cite{wang2010adaptive}. Hence, this work is the first that deals with the \textit{re-allocation} of SDaaS services to optimize the QoS. %from a \textit{provider perspective}. 
This work takes into consideration the optimal \textit{SDaaS composition} and challenges due to the simultaneous use of the skyway network by multiple swarms.\looseness=-1
% However, this approach does not discuss the allocation of resources in a time-constrained environment.

%DaaS and SDaaS

\section{Swarm-based Drone-as-a-Service Model}
In this section, we present a swarm-based drone delivery service model. We abstract a swarm carrying packages and travelling in a skyway segment between two nodes as a service (Fig. \ref{fig:framework}).\\ 
%Then, we formulate the problem and the objective function. Later, we discuss in details the constraints that surround the re-allocation of swarms in a time-constrained environment.
\textbf{Definition 1: Swarm-based Drone-as-a-Service (SDaaS).} An SDaaS is defined as a set of drones, carrying packages and travelling in a skyway segment. It is represented as a tuple of $<SDaaS\_id, S, F>$, where
\begin{itemize}[nosep]
    \item $SDaaS\_id$ is a unique service identifier
    \item $S$ is the swarm travelling in SDaaS. $S$ consists of $D$ which is the set of drones forming $S$, a tuple of $D$ is presented as $<d_1,d_2,..,d_m>$. $S$ also contains the properties including the current battery levels of every $d$ in $D$ $<b_1,b_2, ..,b_m>$, the payloads every $d$ in $D$ is carrying $<p_1,p_2,..,p_m>$, and the current node $n$ the swarm S is at.
    \item F describes the delivery function of a swarm on a skyway segment between two nodes, A and B. F consists of the segment distance $dist$, travel time $tt$,  charging time $ct$, and waiting time $wt$ when recharging pads are not enough to serve $D$ simultaneously in node B.
\end{itemize}
% \textbf{Definition 2: SDaaS Service Provider.} A provider is presented as a tuple of $<D, \alpha>$, where
% \begin{itemize}[nosep]
%     \item $D$ is the finite set of $n$ drones owned by the provider. $D$ is a tuple of $<d_1,d_2,..,d_n>$ and every drone $d_i$ consists of a tuple of $<b,p,s>$ where $b$ is the maximum battery capacity of the drone, $b$ is the maximum payload capacity of the drone, and $s$ is the maximum speed of the drone.
%     \item $\alpha$ is the providers locations, i.e. source node.
% \end{itemize}
\textbf{Definition 2: SDaaS Request.} A request is a tuple of $< R\_id,\beta, P, T>$, where
\begin{itemize}[nosep]
    \item $R\_id$ is the request unique identifier.
    \item $\beta$ is the request destination node.
    \item $P$ are the weights of the packages requested, where $P$ is $<p_1,p_2,..,p_m>$.
    \item $T$ is the time window of the expected delivery, it is represented as a tuple of the window start and end times $<st,et>$.
    %where $st$ is the start time of the requested delivery window and $et$ is the end time of the requested delivery window.
\end{itemize}
%% formal definition
%% formal problem definition

% \subsection{Problem Formulation}

% Given a set of consumers requests $S_{R}=\{R_1, R_2, .. ,R_n\}$ and a query $Q=<P_D,t>$ where $P_D$ are the drones owned by a provider and $t$ is the time window containing all the requests received. The problem is formulated as allocating the drones to the best subset of requests $R_i \in S_{R}$ that can maximize the profit $\max(profit)$ for the provider. The allocated requests should be delivered at the specified and strict request time windows $<R_{st},R_{et}>$ and enough drones $P_D$ should be available at a time to serve the concurrent requests $R_i\{P\}<=P_D$. This problem is considered an NP-Complete problem as it has an exact solution (i.e. Brute force) and any given solution can be validated in a polynomial time. The validation is done by summing the profit of all the correctly allocated requests, i.e., requests the comply with all the problem constraints. Although a brute force approach provides an exact and optimal solution, this approach is not computationally feasible with larger number of requests as the computation time and resources needed grow exponentially. %All the possible combinations of requests and possible delivery times are generated and the valid combination (in terms of overlapping requests and available drones) with the best profit would be selected. 

% \subsection{SDaaS allocation constraints}
% \begin{figure}
% \centering
% \includegraphics[width=0.6\textwidth]{Figures/Constraints.png}
% \caption{Re-allocation and Scheduling Constraints in SDaaS}
% %\vspace{-3cm}
% \label{fig:constraints}
% \end{figure}

% The following constraints need to be addressed to optimally allocate the provider owned drones to the most profitable requests. The requests need to be scheduled optimally to ensure the re-utilization and re-allocation of the drones.

% %%time overlapped requests
% %% congestions
% %% SDaaSmenvironment constraints
% \begin{itemize}[nosep]
%     \item \textit{Reuse of drones:}
%     %% limited owned drones
%     %% multiple requests (reuses of smae drone) within one window
%     As the provider owns a limited number of $N$ drones, these drones need to be re-allocated multiple times to serve as many requests as possible. Hence, over different time windows a drone may be re-used. In addition, we allow the re-allocation of drones within one time window as shown in Fig.\ref{fig:constraints}. $R1$ and $R5$ re-use the same set of drones within one time window between 9:00-10:00 AM. $R1$ and $R7$ re-use the drones over two time windows.\looseness=-1
    
%     \item \textit{Strict AT and flexible RTT:}
%     %% strict arrival time
%     %% request start time not at start of window
%     %A delivery request is strictly bounded by a time window of packages arrival (AT). A consumer would not be satisfied by a late delivery. In addition, a consumer may not be available at the drop-off location if the packages arrive before the specified time window. Hence, 
%     The arrival time of packages need to strictly be within a consumer specified time window. However, the full Round Trip Time of the swarm from the source to the destination and back to the source is not bounded  by a time window. Therefore, the start time of the round trip may begin before the AT time window as shown in $R6$ in Fig.\ref{fig:constraints}.
    
%     \item \textit{Requests overlapping:}
%     %% we need highest profit
%     There might be instances where multiple consumers would request packages to be delivered within the same time window. This will cause some requests to overlap ($R1$, $R2$ and $R3$). 
%     %Since the provider owns a finite set of $N$ drones some of these overlapped requests may not be served. 
%     Hence, the drones need to be allocated to the requests with more profit return.% since we solve the allocation problem from a provider perspective. 
%     Therefore, between $R2$ and $R3$, $R2$ would be allocated as it returns more profit. In addition, the allocation of some requests may be dependant on the allocation of other requests. If $R4$ and $R6$ are allocated then $R5$ may not be allocated. Hence, it only makes sense to allocate $R5$ instead of $R4$ to be able to optimize the profit.\looseness=-1
    
%     \item \textit{Requests optimal scheduling:}
%     %% where to put the request
%     %The start time of a request's RTT is flexible but the AT needs to be within the consumer's specified time window. Therefore,
%     There may be instances where a request needs to be pushed earlier or later to be able to cater for more requests. For example, if $R7$ was scheduled in position 1 then the request wouldn't have been allocated as there are no enough drones to serve it. However, if it is scheduled to position 2, the request would be allocated and the profit would increase.\looseness=-1
    
%     \item \textit{SDaaS environment and composition constraints:}
%     In addition to the aforementioned constraints, the authors of \cite{alkouz2020swarm} identify constraints that surround the optimal composition of swarm-based drone delivery services. The constraints include the limited number of recharging pads at a node that may cause sequential charging and added waiting times. They also include the different battery consumption rates within a swarm due to different payloads. Moreover, in the case of multiple re-allocations, a congestion-aware composition needs to address the issues of using the same skyway network medium for delivery. The number of sequential charges need to be reduced due to congestion of multiple swarms. %We assume the environment is deterministic, i.e. we know the state of the nodes at a time $t$ from other providers. Hence, the congestion-aware algorithm should address the congestion caused by the multiple swarms travelling owned by the same provider.
% \end{itemize}

\section{SDaaS Members Re-allocation Framework}
\begin{figure}
\centering
\includegraphics[width=\linewidth]{Figures/Framework.png}
\caption{SDaaS Members Re-allocation Framework}
%\vspace{-2cm}
\label{fig:framework}
\end{figure}

The SDaaS members allocation and scheduling framework composes of two main modules. In the first module, the composition of SDaaS services for every received request is performed. The output of the first module is the maximum time taken for the packages to arrive at the destination (AT), the maximum round trip time back to the source (RTT), and the profit if the request is served. In the second module, the AT, RTT, and profit are used to allocate and re-allocate the provider owned drones to the most profitable requests and schedule them in a way that serves as many requests as possible.

\subsection{SDaaS Pre-Allocation}
\label{composition}

% \begin{algorithm}[!ht]
%  \caption{Congestion-aware Services Composition Algorithm}
%  \label{congestionlAlg}
%  \scriptsize
%  \begin{algorithmic}[1]
%  \renewcommand{\algorithmicrequire}{\textbf{Input:}}
%  \renewcommand{\algorithmicensure}{\textbf{Output:}}
%  \REQUIRE $S$, $R$
%  \ENSURE  $AT$, $RTT$, Profit
%  \STATE  $AT$=0, $RTT$ = 0
%     \WHILE{$S$ not at dest $\rightarrow$ src}
%         \STATE distance to dest/src= \textbf{Dijkstra}(current, dest/src)
%         \STATE \textbf{compute} energy consumption for every $d$ in $S$ based on $R$ payload and distance to dest/src
%         \IF{all $d$ in $S$ can reach dest/src directly}
%         \STATE $S$ travels to dest/src
%         \STATE $RTT$ += tt \label{alg:RTT}
%         \ELSE
%         \STATE \textbf{find} all nearest neighbor nodes
%             \IF {$P_D-S_D < m$}
%             \STATE available\_pads=total\_pads - ($P_D-S_D$)\label{alg:rest}
%             \ELSE
%             \STATE available\_pads=total\_pads - $m$
%             \ENDIF
%         \STATE \textbf{select and travel} to best neighboring node(neighbor\_nodes, $S_D$)
%         \STATE $RTT$ += $tt$ + $ct$ + $wt$
%         \ENDIF
%         \IF{$S$ at dest}
%         \STATE {$AT$ = $RTT$}
%         \ENDIF
%     \ENDWHILE
%     \STATE Profit = \textbf{size($S_D$)} * RTT * Constant
%  \RETURN $AT$, $RTT$, Profit
%  \end{algorithmic}
% \end{algorithm}

The pre-allocation module mainly consists of the SDaaS optimal composition of all requests to their respective destinations. The optimal path that reduces the delivery time is composed. The intermediate nodes contain different numbers of recharging pads. The composition should consider the optimal selection of nodes that would reduce the charging times. In addition, contention may occur at a node if two swarms serving different requests take the same path at a time causing \textit{congestion} \cite{alkouz2021provider}. We assume that the weight of the packages do not exceed a drones payload capacity. We also assume that a drone may carry a single package at a time. The swarm is assumed to serve one request in a single trip. Therefore, the size of a swarm, serving a request, is equivalent to the number of packages in the request. A request is assumed to have a maximum capacity of $m$ packages. The goal of this module is to compute the maximum time a swarm would take to serve a request (AT) and come back to the source (RTT). The RTT is the maximum possible time of a trip with the existence of other swarms in the network at the same time. Hence, the composition is considered congestion-aware. The composed path is an optimal path in terms of delivery time that a swarm may take while considering the probability of having other swarms utilizing the charging pads, i.e. congestion. Hence, we propose a modified congestion-aware A* approach for SDaaS composition. The AT of the packages at the destination and the RTT is key in scheduling the requests to serve as many requests as possible. We assume that the environment is deterministic, i.e. we know the availability of recharging pads considering other providers using the network.

%The waiting time at a node would typically increase due to the sequential charging of the drones if the number of recharging pads is less than the swarm size.
%The round trip time (RTT) contains the time back at the destination to charge the swarm fully.
%The RTT along with the number of drones used determine the profit of the provider if the request gets allocated to a swarm in subsection \ref{allocation}.
% \hl{The computation of the $AT$ and $RTT$ is essential for the optimal allocation and scheduling of drones to requests in subsection} \ref{allocation}.
%% algorithm explanation



The composition is initiated with a set of swarm drones ($S_D$), fully charged at the source. The swarm is assumed to be static \cite{akram2017security}, i.e. it traverses the network without splitting midway. 
%For a full round trip, the swarm needs to traverse the network to the destination and back to the source. %The arrival time of a swarm is the time taken to reach the destination ($AT$). The round trip time is the $AT$ plus the time back to the source ($RTT$). 
%The time back to the source is typically shorter as the drone releases its payload at the destination and consumes less energy. 
While the drone is not at the destination and back at the source, the algorithm computes the likelihood for the swarm to reach the dest/src nodes using Dijkstra's shortest path without stopping at intermediate nodes. The likelihood of reaching is computed based on the payload of all the drones and the energy consumption rate over the distance travelled. If the swarm is capable of reaching the dest/src directly, it traverses the network and the $RTT$ gets updated with the travel time $tt$. Otherwise, if the swarm is not capable of reaching the dest/src node directly, it selects the optimal neighbouring node. An optimal neighbor is a neighbouring node with the least travel time $tt$ and node time $nt$. The $nt$ is dependant on the number of available recharging pads at a node. The $nt$ composes of the charging times $ct$ and the waiting times $wt$ due to sequential charging in case the number of pads is less than the size of the swarm. We assume that a node may be used by a maximum of two swarms at a time. At every node, we consider the potential of congestion to compute the maximum possible $AT$ and $RTT$. We assume that each drone is occupied by all the other drones owned by the provider $P_D$ if they are less than the maximum swarm size $m$. Otherwise, we assume a station is used by another swarm of size $m$. We compute the node time considering the number of available recharging pads under congestion. When the best neighbour is selected, the swarm traverses to the node and charges fully. The swarm attempts again to reach the dest/src directly. The process continues until the swarm is at the dest/src. The $RTT$ is updated to include the charging time back at the source. The profit is computed using the number of drones utilized to serve a request $S_D$ and the $RTT$ of the trip.\looseness=-1 %The profit is computed per mile per drone taking the charging constraints into consideration. The output of this module will serve as the input for the allocation and scheduling module as shown in Fig.\ref{fig:framework}. \looseness=-1

%Hence, the energy consumption reduces significantly increasing the flight range. This results in less number of stops at intermediate nodes to recharge.


\subsection{SDaaS Allocation and Scheduling}
\label{allocation}
The composed services from subsection \ref{composition} are used to allocate drones to the most profitable requests for the provider. There might be instances where aggregated less profitable requests result in a better total profit than few high profit requests. Hence, the allocation and scheduling algorithm needs to maximize the total profit per day. These allocated requests need to be scheduled in the timeline efficiently to serve as many possible requests. The allocation and scheduling should take in consideration the limited number of provider owned drones. At a time $t$ a provider may serve a maximum of $N$ packages at a time. Therefore, we propose a reinforcement learning allocation and scheduling algorithm.\looseness=-1

% \subsubsection{Profit-based greedy.}
% In this method, the requests are first sorted in an descending order of their profit. Most profitable requests are allocated first. Each hour is assumed to be divided into equal time windows, e.g., 5 minutes length each. The most profitable request is allocated and scheduled to arrive ($AT$) at the first window of the request delivery hour. For example, if the packages are specified by the consumer to arrive between 10:00AM and 11:00AM then the request $Scheduled\_AT$ is specified to 10:00AM. The start time and the end time of the delivery is computed using the $AT$ and the $RTT$  durations of the request. If the $AT$ is 30 mins and the full $RTT$ is 50 mins, then the start time is 9:30AM and the end time is 10:20 AM. The second most profitable request gets allocated in a similar manner after checking the constraints. For example, if the second request also needs to reach the consumer between 10:00AM and 11:00AM, and the number of available drones are not sufficient at 10:00AM because request 1 uses them all and it overlaps with it, then the $Scheduled\_AT$ of request 2 is shifted 5 minutes interval at a time until sufficient drones are available. If all the 12 time windows have no sufficient available drones, the request does not get allocated.
% %The process is repeated until all requests are passed through. 
% Every time a request is allocated, it's checked against all the previously allocated overlapping requests. The total profit of all allocated requests and the number of utilized drones are computed at the end. %Algorithm \ref{profit-based} depicts the profit-based greedy algorithm.\looseness=-1

% % \begin{algorithm}[!hb]
% %  \caption{Profit-based Greedy Allocation Algorithm}
% %  \label{profit-based}
% %  \scriptsize
% %  \begin{algorithmic}[1]
% %  \renewcommand{\algorithmicrequire}{\textbf{Input:}}
% %  \renewcommand{\algorithmicensure}{\textbf{Output:}}
% %  \REQUIRE $R$, $P_D$
% %  \ENSURE  allocated\_R, total\_profit, drones\_utilized 
% %  \STATE total\_profit = 0 , drones\_utilized = 0 , allocated\_R = $[\:]$ 
% %  \STATE R = \textbf{sort\_by} (R, profit)
% %  \FOR{\textbf{each} r \textbf{in} R}
% %     \FOR [\hfill \%loop through 5 minutes windows in a request delivery hour]{i in range (12,5)}
% %         \STATE overlapped\_R = \textbf{overlap}(r, allocated\_R)
% %         \IF{D(overlapped\_R)+D(r)$<=P_D$}
% %         \STATE allocated\_R.\textbf{append}(r)
% %         \STATE \textbf{break}
% %         \ENDIF
% %     \ENDFOR
% %  \ENDFOR
% %  \STATE total\_profit=$\sum profit(allocated\_R)$, drones\_utilized =$\sum packages(allocated\_R)$
% %  \RETURN (allocated\_R, total\_profit, drones\_utilized)
% %  \end{algorithmic}
% %  \end{algorithm}

% \subsubsection{RTT-based greedy.}
% Similar to the profit-based greedy algorithm, the RTT-based greedy sorts the received service requests. However, instead of sorting by profit, this algorithm sorts the requests by the $RTT$ duration in an ascending order. The intuitive reason behind this algorithm is to see the effect of serving shorter requests on the overall number of served requests and total profit. Serving multiple shorter requests may allow the drones to be re-allocated again and again to serve more requests. After sorting, the rest of the algorithm checks the validity of the allocation of new requests similar to the profit-based greedy algorithm.

% \subsubsection{Time-based greedy}
% In this greedy algorithm, the service requests received are sorted by two conditions. First, the requests are sorted ascendingly by the request start time, i.e. the start of the specified time window by the consumer. Second, within one time window, the requests are sorted by their profits descendingly. The rest of the algorithm checks the validity of the new requests allocation similar to Algorithm \ref{profit-based}  and computes the  total  profit, drones utilized, and the set of the served and allocated requests.

\subsubsection{Reinforcement Learning based allocation.}
The proposed framework aims to allocate the provider owned drones $P_D$ to the consumers requests $R$ in the best possible manner, i.e. maximize profit. This imposes the maximum utilization of drones and scheduling the request in the best possible timings to be able to re-allocate the drones over and over again. We leverage \textit{Reinforcement Learning (RL)} to find, allocate, and schedule the requests. In RL, an \textit{agent} learns about an \textit{environment's} behaviour through \textit{explorations}. RL is capable of discovering the best set of requests to be allocated and schedule them at the most optimal time to facilitate the re-use of drones. The main reason for our choice of RL is its \textit{ability to discover the “cumulative” optimal set of requests} to be allocated. The RL does that by assigning rewards for every action the agent invokes. In our work, the actions are the service requests and time slots that a swarm can get allocated to. The agent’s role is to pick the next service request and allocation time that would maximize the overall reward. Therefore, the agent should not only consider the current requests to make the selection but also future requests and available drones. The \textit{environment} that the agent interacts with in this solution is designed to be problem specific. The environment checks for requests validity, overlapping allocations, drones’ availability, and time inter-dependencies and permits only valid actions to be taken by the agent.

We define the agent's \textit{actions} as a tuple of request ID and time slot $<R_{id}, AT_w>$. The time slot represents the arrival time within the consumer specified delivery time window $<R_{st}, R_{et}>$. The agent at every step takes an action, i.e. adds a specific request to the environment at a certain time window. The environment checks the validity of allocating the request by looking at the overlapped allocated requests and the availability of the provider owned drones. %Every time the agent successfully allocates a request, it receives a \textit{reward} of the request profit value. If the agent is not successful, because the move is invalid, it gets \textit{penalized} e.g., -20. 
The \textit{state} is updated at every step with the total accumulated profit of the allocated requests. 
%Since the state is represented by the accumulative profit only, the state-action space only increases in one dimension (actions). Hence, there is no need for a neural network or deep reinforcement learning to overcome an exponential increase in state-action pairs. Therefore, 
We implement a \textit{Q-learning} algorithm that seeks to find the best action to take given the current state \cite{watkins1992q}. 
%The agent interacts with the environment through \textit{exploration} and \textit{exploitation}. With exploitation, the agent uses a q-table to view all the possible actions for a given state. The agent selects the maximum value of those actions. In exploration, the agent does not depend on the q-table for the action, but rather takes a random action. This allows the agent to discover new states. The exploration/exploitation rates are controlled using an epsilon $\epsilon$ that decays over time. Therefore, the agent explores more at the beginning and exploits more later in the learning process. The following equation shows how a q-table value gets updated over time:

% \begin{equation}
% Q(s, a) \gets (1 - \alpha) \cdot \overbrace{Q(s, a)}^{old\: value} + \alpha \cdot \overbrace{(r + \gamma \cdot \max_{a'} Q(s', a')}^{learned\: value})   
% \end{equation}

% where $Q(s, a)$ is the new value for the state-action pair. $\alpha$ is the learning rate, we specify $\alpha$ to 0.001 for the agent to adopt the new values slowly. $r$ is the expected reward of taking action $a$ in state $s$. $\gamma$ is the discount factor that quantifies the importance we give for the future reward, we specify $\gamma$ to 0.99. Hence, the agent will consider future rewards with greater weight.

% We compare the three aforementioned allocation methods on a set of 10 requests and 6 provider owned drones. Fig. \ref{fig:allocation} shows the allocated requests and scheduled times. Note that at no time $t$ there exists allocated requests with over 6 drones in total. The total profit of the allocated requests with the profit-based greedy is 266.9, RTT greedy is 240.9, and RL allocation is 295.3. For a small set of requests the profit-based is performing better than the RTT greedy algorithm as requests with high profits are allocated first.  The RTT greedy is performing the worst here because typically a short request means a low profitable one. However, as will be seen in the experiments section, the RTT greedy outperforms the profit-based with higher number of requests. This is because %the accumulation of many short requests lead to higher profits than long but small in number profitable requests. I
% it allows a more stacked scheduling that facilitates the reuse of drones over and over again. The reinforcement learning allocation outperforms all the greedy methods as it smartly learns the optimal set of requests to be served and at what time they're allocated to maximize its reward. It was able to fulfill 8 requests compared to the 7 requests of the other two methods.\looseness=-1

% \begin{figure}
% \centering
% \includegraphics[width=\textwidth]{Figures/schedules.jpg}
% \caption{Requests Re-allocation and Scheduling with Proposed Methods} 
% %\vspace{-1cm}
% \label{fig:allocation}
% \end{figure}

% Profit greedy 266.9
% RTT greedy 240.9
% Time greedy 259.9

% However, this behaviour will not be maintained with higher number of received requests as will be described in \ref{experiments}. Between the greedy algorithms the Time-based greedy result in most profitable allocations. This is because the requests are stacked by the start time leaving less wasted spaces. In addition, the requests within a time are allocated by profit resulting in higher total profits.





\section{Experiments}
\label{experiments}


In this section, we evaluate the performance in terms of total profit gained and the execution time of the proposed algorithm. %The proposed problem is fundamentally different than existing resource allocation and scheduling problems \cite{carlier1989algorithm} \cite{salkin1975knapsack} in that it is machine (drone) unspecific, re-allocatable, and time constrained. The proposed problem is machine unspecific unlike job shop \cite{carlier1989algorithm} where each task is served by a specific machine. In the proposed problem, any drone can serve any request. In addition, the proposed problem allocates time constrained requests unlike the knapsack problem \cite{salkin1975knapsack} that is not constrained by time but is similar in that it aims to maximize the weight (profit). 
A brute force baseline is time and memory extensive and is not feasible as described earlier. Therefore, we compare the proposed RL allocation method to the First Come First Served (FCFS) algorithm \cite{tanenbaum2015modern}.  In the FCFS approach, the first request received gets allocated first. If a request can’t be allocated due to the limited number of drones being occupied at a time window, the request does not get allocated and the next arriving request gets checked and allocated. 
%In the genetic algorithm, we define a chromosome as a set of requests allocated at certain time windows. The GA tries to increase its score by a series of crossovers and mutations selecting the best genes (requests and times) at every step. We specify the population size to be 200 and the mutation rate as 0.01. The fitness of each individual is evaluated at every step to rank the best individuals as proposed by \cite{sapru2010time}. The fitness score is computed using the reciprocal of violated constraints and the accumulative profit of the allocated requests. The algorithm terminates at fitness score 1 or when it reaches the maximum number of generations specified, i.e. $5*10^5$.  

An urban road network dataset from the city of london is used to mimic the arrangement of a skyway network \cite{karduni2016protocol}. The dataset consists of nodes representing intersections and segments connecting those nodes. For the experiments, we extracted a sub-network consisting of 129 connected nodes. Each node is allocated with different number of recharging pads randomly. A source node is then selected and $r$ service requests are generated with different destination nodes. For each request, we synthesize maximum 5 packages payload and a maximum weight of 1.4kg. The drone model is assumed to be the DJI phantom 3. All the power consumption computation is based on this model, the distance travelled, and payload carried. We used the congestion-aware SDaaS composition algorithm to compute the $AT$, $RTT$, and profit for each request given the recharging pads constraints. These requests are assigned to different time windows randomly. Each time window is assumed to be one hour. Hence, the $AT$ of the package should lie within this hour. The experiments were run on 7th Gen Intel® Core™ i7- 7700HQ Processor (2.8 GHz), 16 GB RAM, 64-bit Windows OS PC. \looseness=-1
% Please add the following required packages to your document preamble:
% \usepackage[table,xcdraw]{xcolor}
% If you use beamer only pass "xcolor=table" option, i.e. \documentclass[xcolor=table]{beamer}
% \begin{table}[t]
% \centering
% \scriptsize
% \caption{Experiment Variables}
% \label{tab:variables}
% \begin{tabular}{|l|l|}
% \hline
% Variable                                                                                     & Value                                                                                                                                        \\ \hline
% PC information                                                                               & {\color[HTML]{333333} \begin{tabular}[c]{@{}l@{}}7th Gen Intel® Core™ i7- 7700HQ Processor ( 2.8 GHz) \\16 GB RAM, 64-bit OS\end{tabular}} \\
% \begin{tabular}[c]{@{}l@{}}No. of nodes in the largest \\ connected sub-network\end{tabular} & 195                                                                                                                                          \\
% Max no. of packages in a request                                                             & 5                                                                                                                                            \\
% Max weight of a package                                                                      & 1.4 kg                                                                                                                                       \\
% Drone model                                                                                  & DJI Phantom 3                                                                                                                                \\
% Time for drone to fully charge                                                               & 30 mins                                                                                                                                      \\
% Battery capacity                                                                             & 4480 mAh                                                                                                                                     \\
% Drone speed                                                                                  & 15.6 m/s                                                                                                                                     \\ \hline
% \end{tabular}
% \end{table}

%Second best is the Time-based greedy that is capable of stacking the request by time and profit leading to a highly condensed profitable schedule.

\begin{figure}[!ht]
\begin{subfigure}{.5\textwidth}
  \centering
  % include first image
  \includegraphics[width=.9\linewidth]{Figures/profit-requests-short.png}  
  \caption{Profit with varying number of requests}
  \label{fig:sub-first}
\end{subfigure}
\begin{subfigure}{.5\textwidth}
  \centering
  % include second image
  \includegraphics[width=.9\linewidth]{Figures/profit-drones-short.png}  
  \caption{Profit with varying number of drones}
  \label{fig:sub-second}
\end{subfigure}


% \begin{subfigure}{.5\textwidth}
%   \centering
%   % include third image
%   \includegraphics[width=.9\linewidth]{Figures/fulfilled-requests.png}  
%   \caption{Fulfilled requests with varying requests}
%   \label{fig:sub-third}
% \end{subfigure}
% \begin{subfigure}{.5\textwidth}
%   \centering
%   % include fourth image
%   \includegraphics[width=.9\linewidth]{Figures/fulfilled-drones.png}  
%   \caption{Fulfilled requests with varying drones}
%   \label{fig:sub-fourth}
% \end{subfigure}


% \begin{subfigure}{.5\textwidth}
%   \centering
%   % include third image
%   \includegraphics[width=.9\linewidth]{Figures/utilization-requests.png}  
%   \caption{Drones Utilization with varying requests}
%   \label{fig:sub-fifth}
% \end{subfigure}
% \begin{subfigure}{.5\textwidth}
%   \centering
%   % include fourth image
%   \includegraphics[width=.9\linewidth]{Figures/utilization-drones.png}  
%   \caption{Drones Utilization with varying drones}
%   \label{fig:sub-sixth}
% \end{subfigure}

\begin{subfigure}{.5\textwidth}
  \centering
  % include third image
  \includegraphics[width=.9\linewidth]{Figures/execution-short.png}  
  \caption{Execution times}
  \label{fig:sub-seventh}
\end{subfigure}
\begin{subfigure}{.5\textwidth}
  \centering
  % include fourth image
  \includegraphics[width=.8\linewidth]{Figures/Reward.png}  
  \caption{RL rewards convergence}
  \label{fig:sub-eighth}
\end{subfigure}
\caption{Proposed Method Effectiveness}
\label{fig:fig}
\vspace{-0.0cm}
\end{figure}

In the first experiment, we study the effect of varying the number of received requests a day on the profit. We assume the provider owns a fixed set of 30 drones. As shown in Fig.\ref{fig:sub-first}, the RL allocation outperforms the FCFS. This is because of its ability to learn the optimal allocation and scheduling of the requests to maximize the profit. 
%The greedy by profit and by RTT are preforming the worse. This is because allocating by profit means that there are instances where multiple less profitable requests are not allocated that lead to a total of more profit. The RTT greedy algorithm interestingly preforms better than the profit-based on larger number of requests. This is because a bigger number of shorter requests are allocated leading to a higher total profit. 
The FCFS is performing worse than RL because allocating services in an FCFS manner does not consider any order in terms of most profitable request and round trip times. Therefore, the non-optimal set of requests gets allocated at non-optimal time windows. %However, the FCFS slightly preformed better than the greedy algorithms at larger number of requests and this is due to the random order of allocated requests that may combine the benefits of the two greedy algorithms, i.e. more chances to allocate more profitable and shorter time requests. 
%The GA performed worse than the proposed RL approach because RL uses a mathematically grounded framework of Markov decision processes, whereas GA is largely based on heuristics. The value function update in RL is a gradient-based update, whereas GA doesn’t use such gradients. As the number of requests increase, the size of an individual increases, this makes it very difficult to reach an optimal or near optimal solution based on heuristics only. 

The same behaviour is noted with varying the number of provider owned drones for a set of 50 requests as shown in Fig.\ref{fig:sub-second}. The RL allocation converges to the maximum possible profit earlier by serving all the 50 requests. This performance of the RL allocation method comes with the cost of execution. Fig.\ref{fig:sub-seventh} shows the execution times varying the number of requests received a day. The left  y-axis  represents  the  execution times of the FCFS. The right y-axis represents the execution time for the RL based algorithm. Since the number of state-action pairs in RL only increase in one dimension and converges at almost the 20000 episode (Fig.\ref{fig:sub-eighth}), the execution time does not increase significantly. We assume the requests are received in batch a day earlier, hence, the learning could occur overnight.


% In the second experiment, we measure the number of fulfilled requests a day. A fulfilled  request  is  a  request  that  is successfully  allocated  to a swarm of drones to be served. As shown in Fig.\ref{fig:sub-third}, the RTT greedy algorithm serves a larger number of requests per day when more requests are received. This is because the probability of having many short requests is more. Hence, it is able to allocate and serve more shorter requests. The RL allocation algorithm is steadily performing the best. Since more requests are served, more consumers are satisfied. This is reflected by the profit the provider gains (Fig.\ref{fig:sub-first}). Such graph, i.e. Fig.\ref{fig:sub-third}, would help the provider determine the number of requests to receive a day if they own a finite fleet of drones and are incapable of enlarging it. Fig.\ref{fig:sub-fourth} measures  the  percentages  of fulfilled  requests with  a  fixed  number  of  requests  (50)  and  varied  number  of provider’s owned drones. The graph shows how the RL allocation method outperforms the other two methods. The graph shows how the RL algorithm converges earlier (with 30 drones only) as the number of owned drones increases. This graph would allow the provider to decide how many drones should they own if they receive a number of $r$ requests a day.

% In the last experiment, we measure the effectiveness of the algorithms on the drones utilization. %A  provider,  as described earlier,  owns  a  finite  set  of  drones.  
% An  optimal method is  a  method  that  re-uses  the finite  drones  as much as possible to serve multiple requests. Fig.\ref{fig:sub-fifth} shows the percentage of utilized drones, i.e. how many times were the finite set of 30 drones re-allocated. As shown in the figure, as the number of requests received increases, the percentage  of  drone’s  utilization increases as the drones are capable of serving more requests. The RL allocation method outperforms the other methods as it is capable of serving more requests (with larger number of packages) in a day. At 150 requests, the RL algorithm uses the finite set of 30 drones 8 times a day approximately. The RTT greedy algorithm uses the drones many times as it serves a large number of requests (Fig.\ref{fig:sub-third}). However, it is fulfilling low profitable requests as shown in Fig.\ref{fig:sub-first}. The GA algorithm, interestingly, utilizes the drones the least though it serves more requests than the profit-based greedy algorithm as shown in Fig.\ref{fig:sub-third}. This means that the GA is allocating requests that have less packages and need less drones which is also reflected in the profit as these requests are generally less profitable. Fig.\ref{fig:sub-sixth}  shows  the  percentages  of  drones  utilized  as  the  number of  owned  drones  increases  with  a fixed  number  of  received requests a day (50). As the number of owned drones increases the  drones  utilization  decreases  as  they  will  be  re-allocated less. The  RL allocation method,  as  shown,  outperforms  the other  methods  in  terms of  drones  utilization. 

% At 45 requests, both Time-based and Profit-based greedy algorithms utilize the same number of drones. This is because the Profit-based greedy algorithm allocate the more profitable requests first which are typically requests with more served packages.



\section{Conclusion}
We proposed a provider-centric re-allocation of drone swarm services known as, Swarm-based Drone-as-a-Service (SDaaS). A congestion-aware SDaaS composition algorithm is proposed to compute the maximum delivery and round trip times a swarm may take to serve a request taking the constraints at intermediate nodes (limited recharging pads and congestion) in consideration. A reinforcement learning allocation  method was proposed with the goal of increasing the provider's profit. The efficiency of the proposed approach was evaluated in terms of profit maximization and execution time. Experimental results show the outperformance of the RL allocation approach to the baseline FCFS approach. In the future work, the problem could be expanded to cover multi-objectives, e.g. profit and time. In addition, we will consider heterogeneous swarms allocation to serve multiple requests and extend the work to deal with SDaaS failures.


\section*{Acknowledgment}
This research was partly made possible by DP160103595 and LE180100158 grants from the Australian Research Council. The statements made herein are solely the responsibility of the authors.
% \subsubsection{Sample Heading (Third Level)} Only two levels of
% headings should be numbered. Lower level headings remain unnumbered;
% they are formatted as run-in headings.

% \paragraph{Sample Heading (Fourth Level)}
% The contribution should contain no more than four levels of
% headings. Table~\ref{tab1} gives a summary of all heading levels.

% \begin{table}
% \caption{Table captions should be placed above the
% tables.}\label{tab1}
% \begin{tabular}{|l|l|l|}
% \hline
% Heading level &  Example & Font size and style\\
% \hline
% Title (centered) &  {\Large\bfseries Lecture Notes} & 14 point, bold\\
% 1st-level heading &  {\large\bfseries 1 Introduction} & 12 point, bold\\
% 2nd-level heading & {\bfseries 2.1 Printing Area} & 10 point, bold\\
% 3rd-level heading & {\bfseries Run-in Heading in Bold.} Text follows & 10 point, bold\\
% 4th-level heading & {\itshape Lowest Level Heading.} Text follows & 10 point, italic\\
% \hline
% \end{tabular}
% \end{table}


% \noindent Displayed equations are centered and set on a separate
% line.
% \begin{equation}
% x + y = z
% \end{equation}
% Please try to avoid rasterized images for line-art diagrams and
% schemas. Whenever possible, use vector graphics instead (see
% Fig.~\ref{fig1}).

% \begin{figure}
% \includegraphics[width=\textwidth]{fig1.eps}
% \caption{A figure caption is always placed below the illustration.
% Please note that short captions are centered, while long ones are
% justified by the macro package automatically.} \label{fig1}
% \end{figure}

% \begin{theorem}
% This is a sample theorem. The run-in heading is set in bold, while
% the following text appears in italics. Definitions, lemmas,
% propositions, and corollaries are styled the same way.
% \end{theorem}
%
% the environments 'definition', 'lemma', 'proposition', 'corollary',
% 'remark', and 'example' are defined in the LLNCS documentclass as well.
%
% \begin{proof}
% Proofs, examples, and remarks have the initial word in italics,
% while the following text appears in normal font.
% \end{proof}
% For citations of references, we prefer the use of square brackets
% and consecutive numbers. Citations using labels or the author/year
% convention are also acceptable. The following bibliography provides
% a sample reference list with entries for journal
% articles~\cite{ref_article1}, an LNCS chapter~\cite{ref_lncs1}, a
% book~\cite{ref_book1}, proceedings without editors~\cite{ref_proc1},
% and a homepage~\cite{ref_url1}. Multiple citations are grouped
% \cite{ref_article1,ref_lncs1,ref_book1},
% \cite{ref_article1,ref_book1,ref_proc1,ref_url1}.
%
% ---- Bibliography ----
%
% BibTeX users should specify bibliography style 'splncs04'.
% References will then be sorted and formatted in the correct style.
%

\bibliographystyle{splncs04}
\bibliography{samplepaper}
%
% \begin{thebibliography}{8}
% \bibitem{ref_article1}
% Author, F.: Article title. Journal \textbf{2}(5), 99--110 (2016)

% \bibitem{ref_lncs1}
% Author, F., Author, S.: Title of a proceedings paper. In: Editor,
% F., Editor, S. (eds.) CONFERENCE 2016, LNCS, vol. 9999, pp. 1--13.
% Springer, Heidelberg (2016). \doi{10.10007/1234567890}

% \bibitem{ref_book1}
% Author, F., Author, S., Author, T.: Book title. 2nd edn. Publisher,
% Location (1999)

% \bibitem{ref_proc1}
% Author, A.-B.: Contribution title. In: 9th International Proceedings
% on Proceedings, pp. 1--2. Publisher, Location (2010)

% \bibitem{ref_url1}
% LNCS Homepage, \url{http://www.springer.com/lncs}. Last accessed 4
% Oct 2017
% \end{thebibliography}
\end{document}
