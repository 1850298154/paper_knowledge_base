\section{Conclusions}
In this paper, we set out to address the problem of multi-tasking robots in multi-robot tasks. 
%A fundamental limitation of existing multi-robot systems was addressed by the removal of a restrictive assumption that was often made--robots are single-tasking.
%Our method allowed coalitions to overlap thus enabling multi-tasking robots. 
We observed that the key underlying challenge was to reason about the physical constraints that could be synergistically satisfied.
%which directly affected the feasibility of multi-tasking.
To address the challenge, we developed our method based on the information invariant theory and modeled constraints as information instances. 
%This allowed us to reason about the relationships between constraints by reasoning about those between information requirements. 
Thereby, a formal and general framework to achieve multi-tasking robots was developed. 
We showed that our algorithm was sound and complete under our problem settings. 
%Our method was integrated with a simple greedy heuristic for task allocation.
Simulation  results  were  provided  to  show  the  effectiveness  of  our approach under resource-constrained situations and in handling challenging situations. % in a multi-UAV simulator. 

% The idea of multi-tasking is attractive in many ways. 
% Humans are living in multi-tasking environments--at any point of time, 
% we are optimizing for more than one task. 
% Multi-task often leads to more efficient task performance since it allows us to exploit task synergies. 
% The work presented in this paper takes us one step forward in realizing multi-tasking robots. 
% In particular, we started looking at the feasibility of multi-tasking. 
% There are many potential directions to pursue along this direction. First, several limitations are present in the current approach. 
% For example, although our method guarantees that there exists a physical configuration that satisfies all the constraints, it does not explicitly take the environmental influence into account. For example, a narrow corridor may prevent a robot formation from passing through, even though all the constraints for the formation do not introduce any conflicts. In this sense, our work should better be characterized as establishing a necessary condition for multi-tasking. Also, our method is mainly focused on the ``{\it planning}'' phase and hence does not address how the robots reach the desired configuration and maintain the constraints. These issues are assumed to be handled by the execution layer.

% More generally, the question of how to execute the tasks with overlapping coalitions is not addressed in this work. 
% As we already discussed, executing individual tasks with non-overlapping coalitions is straightforward but task synergies impose additional requirements on the task execution: how should the robots that are assigned multiple tasks execute them? Should they consider them in a prioritized strategy~\cite{van2005prioritized}? Or should they combine the different tasks in a way that is similar to motor schemas~\cite{arkin2}. 
% Communication requirements for maintaining the constraints must also be taken into account. How should the robots optimize their communication to improve the task performance? 

% The stringency of the physical constraints is another interesting question. It may be desirable to relax the constraints in certain situations (e.g., due to environmental influences). In such cases, it may be important to consider the problem where the constraints are least violated~\cite{kim2012revision}, or specify task constraints in different ways to increase the diversity of the configurations~\cite{srivastava2007domain} so as to make it robust to different environments. 