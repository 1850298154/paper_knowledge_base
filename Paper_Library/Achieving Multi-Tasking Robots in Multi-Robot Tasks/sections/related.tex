\section{Related Work}

Much effort in multi-robot systems has been dedicated to the task allocation problem~\cite{gerkey}. 
When the robots are single-tasking and tasks are single-robot tasks, 
the problem is also known as the assignment problem, 
which has efficient solutions~\cite{kuhn1955hungarian, Liu1}.
Addressing multi-robot tasks, however, significantly increases the problem complexity~\cite{gerkey},
for which there has been a noteworthy amount of focus on approximate solutions~\cite{shehory, vig, sandholm, adams2011coalition, zhang2013considering},
and realistic concerns while implementing task allocation with distributed robot systems~\cite{arkin2, botelho, dias, fua, gerkey2, kalra, parker2, vig, yu2, yu3, zlot}.
As far as we know, there exists little discussion on generalizing the problem
to multi-tasking robots with instantaneous assignments~\cite{gerkey}.\footnote{Note that our work differs from the task scheduling problem (i.e., time extended assignments) where a robot can be assigned to different tasks at different times. Our problem setting requires the tasks to be assigned and executed at the same time.} 
Although the focus here is not on task allocation, 
it is an interesting direction to study how to integrate our method with task allocation algorithms. 

The ability to achieve multi-tasking robots may appear similar to the existing notions of task synergy~\cite{liemhetcharat2014weighted, parker1999cooperative}
 and overlapping coalition structure~\cite{shehory1996formation, dang2006overlapping}.
However, these prior methods concern mainly with the optimization problem of utility maximization with the influence between the assignments given a priori, i.e., how the assignment of a coalition to a task may contribute to the overall utility given the other assignments.
Such influence is often assumed to be captured by a pre-specified heuristic function
that is difficult to compute in real-world settings.
For one, modeling task synergy involves complex reasoning
about the physical constraints as we argued. 
Our study goes beyond prior work by explicitly modeling the influence of overlapping coalitions on task feasibility due to the consideration of physical constraints, 
and thus addresses the fundamental question of whether task synergies are present.  
A similar observation was made in ~\cite{vig} about the impact of physical constraints on task allocation and the issue was addressed by manually specifying a feasibility function.
A general framework that can automatically reason over the  constraint space is missing. 


The information invariant theory is introduced to capture the equivalence of sensori-computational systems~\cite{donald1995information}. It has since then been used to develop systems that have demonstrated an impressive level of flexibility~\cite{yu3, tang2005asymtre}. Information invariant is well connected to the notion of information space~\cite{lavalle2006planning},
where both must reason about the relationships between different information requirements. 
The difference being that the latter is often focused on the minimalistic aspect~\cite{tovar2004gap}. 
To the best of our knowledge, however, we are the first to apply the information invariant theory to reason about synergies between multi-robot tasks. 


