%%%%%%%%%%%%%%%%%%%%%%%%%%%%%%%%%%%%%%%%%%%%%%%%%%%%%%%%%%%%%%%%%%%%%%%%%%%%%%%%
\section{Introduction}
\label{sec:introduction} 


To address a multi-robot task, one simplifying assumption made in the literature
is that the robots are single-tasking. 
This assumption, however, is impractical in situations where the robots must coordinate closely to share their capabilities,
such as when a robot has a capability that is required in multiple tasks.
To handle such resource-constrained situations,
a simple solution is to achieve the tasks sequentially. 
Unfortunately, such a solution, besides having a negative impact on task efficiency, is feasible only when no concurrent execution is required among the tasks for which capabilities must be shared.
As a result, it significantly limits the capabilities of multi-robot systems. 
Instead, in this paper, we consider multi-tasking robots in multi-robot tasks
to enable a robot to operate on multiple tasks at the same time.



While multi-tasking robots are desirable when there are resource contentions,
the fundamental question regarding its feasibility must be carefully considered. In particular, it is affected heavily by the compatibility of the physical constraints to be satisfied for achieving the tasks.\footnote{While there may be other factors that affect the feasibility, such as limitation on the communication bandwidth, our focus is on physical constraints as the influences of other factors are usually less direct or critical.}
For example, for a robot to share its localization capability, 
it must stay within the proximity of the robot that requires its assistance;
for a UAV to share its camera sensor
with a ground vehicle that is assigned to a monitoring task, it must maintain its camera head direction towards the target. 
Hence, the main challenge to enable multi-tasking robots lies in identifying synergies between the underlying physical constraints for the tasks.  



The proposed approach is built on the information invariant theory~\cite{donald1995information} that specifies interactions between information requirements. 
In our work, the physical constraints to be satisfied are mapped to information instances,
which are categorized by information types.
Fig. \ref{fig:demo} illustrates a scenario with two tasks:
one of them is a centroid task that requires the three robots to maintain their centroid over the base station,
which is specified as a constraint on the centroid information over the three robots;
the other one is a monitoring task that requires one of the robots to maintain a target within its observation range,
which is specified as a constraint on the relative position information from the monitoring robot to the target. 
The interactions between the constraints can then be described as information interactions, which allows 
us to identify task synergies using the information invariant framework. 
More specifically, given a set of 
physical constraints for the tasks, 
our approach checks if they are compatible 
according to a set of rules that 
govern information interactions. 
A task synergy is identified, for example, when the constraints for different tasks can be satisfied simultaneously even when there are shared resources (i.e., robots) among them.
Fig. \ref{fig:demo} illustrates a synergy between the two tasks. 
 
 
 \begin{figure}[t!]
    \centering
    \begin{subfigure}
        \centering
        \includegraphics[width=0.41\columnwidth]{figures/demo1.png}
    \end{subfigure}%
    ~ 
    \begin{subfigure}
        \centering
        \includegraphics[width=0.3\columnwidth]{figures/demo2.png}
    \end{subfigure}
    \caption{Scenario that illustrates a synergy between two tasks, where one robot is shared: a centroid task and a monitoring task. The left and right figures show the change of robot configurations as the target changes its position.}
    \vskip-15pt
    \label{fig:demo}
\end{figure}



To the best of our knowledge, our work represents the first general framework for achieving multi-tasking robots in multi-robot tasks~\cite{gerkey}.
It removes a restrictive assumption made in multi-robot systems
by enabling overlapping coalitions.
A formal framework is presented based on the information invariant theory. 
Simulation results show that our approach not only achieves better efficiency but also extends the capabilities of such systems in a multi-UAV simulator,
especially in situations where resources are limited. 


