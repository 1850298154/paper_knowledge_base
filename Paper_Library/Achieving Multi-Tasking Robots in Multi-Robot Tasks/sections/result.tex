\section{Results}



\subsection{Tasks Considered}

We introduced three types of UAV tasks that were considered in our experimental results.

\begin{itemize}
    \item \textbf{Monitoring task}: a target must be monitored within a close proximity by an air vehicle. The constraint for the monitoring task is the relative position ($F_R$) between the vehicle and target, and  the global position ($F_G$) of the target (since we do not have control over it). 
    \item \textbf{Centroid task}: a group of vehicles must maintain their centroid with respect to a specific location or target. The constraint is the centroid information  defined over either $2$ or $3$ robots (denoted by $F_{C_2}$ or $F_{C_3}$). The centroid information can be derived (inferred) from the global position information of the vehicles involved in the centroid task. 
    \item \textbf{Communication maintenance task}: a vehicle must maintain its position in between two other vehicles to maintain communication links. The constraint introduced here is the communication maintenance information (denoted by $F_M$)  that takes $3$ vehicles, which can be derived from the relative positions between vehicles $1$ and $3$, and $2$ and $3$, assuming that $3$ is the vehicle that is between the other two.
    %any of the two agents involved (that introduces three inference rules). 
\end{itemize}

\subsection{Synthetic evaluation} %\YZ{What should go here?}



In this experiment, we tested with the first two types of tasks only. 
The goal is to see how beneficial our approach is under resource constrained situations. 
We ran several experiments to determine the efficacy of our synergistic approach compared to a baseline. In the baseline, vehicles were assumed to be single-tasking, and hence did not accept new tasks once they were assigned to a task. 
%For example, if vehicles 1 and 2 are assigned to a centroid task, then neither vehicle may accept a monitoring task, even though it would be possible for one of them to accept such a task and still have a compatible set of constraints. 
In our experiments, we randomly generated sets of tasks for specific sets of agents, and attempted to assign tasks alternately between centroid and monitoring tasks to vehicles,
until every generated task has been attempted. 
%The only exception to this is that we always attempt to assign a monitoring task, followed by a centroid task, followed by a monitoring task, etc. until every generated task has been attempted. 

In Fig. \ref{fig:1}, we set out to evaluate how our approach performs as the number of vehicles increases from $3$ to $7$.
In the figures, we varied the ratio between the numbers of the centroid and monitoring tasks, which was expected to have a noticeable effect on the performance. 
The top figure shows a configuration where the numbers of both tasks are half of the number of vehicles. 
This evaluates situations where the robotic resources are more abundant. 
In such cases, the performance between the two methods did not differ much even though our synergy based method still outperformed the baseline.
The middle figures shows a configuration with many more monitoring (simple) tasks,
in which we observed more task assignments for both methods. 
This illustrates situations where resources are more constrained. 
Our synergy method not only performed significantly better, with almost 1.5 times tasks assigned, but also more consistent: the random task generation had much less effect on our method with almost zero standard deviations. 
In the bottom figure, we increased the numbers of both tasks for task-saturated situations. 
We can see that the influence on our synergy method was still less apparent than that on the baseline. 
In every configuration examined, our approach resulted in more tasks assigned than the baseline approach.
%and in the vast majority of cases examined, it assigns more tasks than the baseline. 


 In Fig. \ref{fig:2} (top), 
 we studied the differences of the assignment process between the two approaches in more detail. 
 We generated a new task at each iteration of the assignment process and checked how many tasks were allocated in each iteration accumulatively. We can see that at very low numbers of tasks relative to the vehicles, the synergy and baseline task assignments assigned the same number of tasks, which was to be expected since we had many more vehicles available. 
 However,  as more tasks were added to the assignment, the synergy method pulled ahead of the baseline. 
 Also, our method consistently assigned more tasks in each iteration. 
 In Fig. \ref{fig:2} (bottom), we studied the influence of the ratio between the numbers of the monitoring tasks and centroid tasks in detail. 
 We randomly generated $25$ tasks with the ratio between the tasks gradually decreasing, so that initially we had $25$ monitoring and $0$ centroid task, and $0$ monitoring and $25$ centroid task at the end. 
 We can see that our method was affected a lot less by the baseline. The performance gap kept increasing as this ratio decreased.




% In the plots of Fig. 3, it is shown that with any number of centroid tasks considered, synergy assignment beats the baseline, even with relatively few multi-vehicle tasks assigned.



\begin{figure}[t!]
    \centering
    \includegraphics[width=0.6\columnwidth]{sections/unnamed5.png}
    \includegraphics[width=0.6\columnwidth]{sections/unnamed4.png}
    \caption{%\YZ{just a note. first is 0.5 times on both, second is 0.5 times centroid and 3 times monitoring, and third is 3 times both}
    (top) Plot showing the number of tasks assigned by each approach as the number of tasks generated increases. (bottom) Plot showing the number of tasks assigned by each approach with $25$ tasks as the ratio between the monitoring and centroid tasks gradually decreases.}
    \vskip-10pt
    \label{fig:2}
\end{figure}


% \begin{figure}
%     \centering
%     \includegraphics[width=0.6\columnwidth]{sections/unnamed5.png}
%     \caption{
%     %\YZ{Another note. This is 7 vehicles, centroid and monitoring tasks equal to iteration number}
%     Plot showing the number of tasks assigned by each approach as the number of tasks generated increases.}
%     \label{fig:2}
% \end{figure}



% \begin{figure}
%     \centering
%     \includegraphics[width=0.6\columnwidth]{sections/unnamed4.png}
%     \caption{%\YZ{Another note. This is 7 vehicles, 25 tasks, centroid starts at 0 and goes to 25, monitoring starts at 25 and goes to 0}
%     Plot showing the number of tasks assigned by each approach with $25$ tasks as the ratio between the monitoring and centroid tasks gradually decreases.}
%     \label{fig:3}
% \end{figure}


\begin{figure}
    \centering
    \begin{subfigure}%[]
        \centering
        \includegraphics[width=0.48\columnwidth]{Screenshot_1.png}
    \end{subfigure}%
    ~ 
    \begin{subfigure}%[]
        \centering
        \includegraphics[width=0.48\columnwidth]{Screenshot_2.png}
        %\caption{Lorem ipsum, lorem ipsum,Lorem ipsum, lorem ipsum,Lorem ipsum}
    \end{subfigure}
    ~ 
    \begin{subfigure}%[]
        \centering
        \includegraphics[width=0.475\columnwidth]{Screenshot_3.png}
    \end{subfigure}
    ~
        \begin{subfigure}%[]
        \centering
        \includegraphics[width=0.471\columnwidth]{Screenshot_4.png}
        %\caption{Lorem ipsum}
    \end{subfigure}%
    ~ 
    \begin{subfigure}%[]
        \centering
        \includegraphics[width=0.475\columnwidth]{Screenshot_5.png}
        %\caption{Lorem ipsum, lorem ipsum,Lorem ipsum, lorem ipsum,Lorem ipsum}
    \end{subfigure}
    ~ 
    \begin{subfigure}%[]
        \centering
        \includegraphics[width=0.47\columnwidth]{Screenshot_6.png}
    \end{subfigure}
     \vskip-5pt
    \caption{Screenshots of the simulated task, going from left to right, and top to bottom, as the simulation progresses. Each vehicle is represented by a chevron with a unique color and an ID label. Each controlled vehicle has its goal destination shown by a dot of its color. Vehicles 1, 5, 6, and 8 are not controlled. The first shows the initial setting, and the last figure shows the scenario in a different scale from others. 
    %and (b) shows the same after initial target destinations are calculated. Target destinations are indicated by a dot of the vehicle's color. Notice that vehicles 1, 5, 6, and 8 do not have a target destination; this is because they are not controlled by the system and their plans are unknown.
    }
    \vskip-15pt
    \label{fig:4}
\end{figure}








\subsection{Simulation Scenario}


\subsubsection{Simulation Environment and Settings}
The OpenAMASE simulation environment was developed by the Air Force Research Lab\cite{afrlAMASE} as a testing ground for their aerial vehicle control software, UxAS\cite{afrlUxAS}. Together, these two pieces of software form the basis of the simulation environment. AMASE gives access to a GUI and simulates the vehicles over time, and UxAS handles the passing of all relevant messages to and from all modules of the software. Any module can subscribe to any type of message, and will then receive any message of that type sent by any other module. Our software uses the AirVehicleState, containing a ``heartbeat'' of information about each vehicle for each simulation tick, and the AirVehicleConfiguration, containing capability information about each vehicle, to decide where vehicles should move to.


\subsubsection{Simulation Result}
In this simulation, we tested our system on a realistic scenario involving a multi-vehicle convoy task with intruder detection. 
AMASE and UxAS can only handle up to twelve vehicles, and we have limited our simulation to $8$ controlled vehicles,  $2$ intruder vehicles, and $1$ static vehicle that simulates a control station. 
Fig. \ref{fig:4} shows snapshots from running the task. 
%However, there is no "hard cap" on the number of vehicles our system can control.
A convoy is formed (centered around vehicle $1$) and protected by three vehicles ($2, 3, 4$), which are assigned to a centroid task with vehicle $1$ as the centroid.
In the mean time, vehicle $7$ must maintain the communication between the convoy and a ground control station (vehicle $8$). 
As the convoy is moving towards its target, 
two intruders are detected and two of the vehicles ($2$ and $3$) that
are already assigned the centroid task take advantage of
the synergy between centroid and monitoring tasks by executing two tasks at the same time.
%They are dynamically assigned to the monitoring tasks as the new tasks are introduced. 
% The tasks are as follows: Vehicles 2, 3, and 4 are assigned to a centroid, c(2, 3, 4). Vehicle 2 is assigned to monitor vehicle 5, m(2, 5). Vehicle 3 is assigned to monitor vehicle 6, m(3, 6). Vehicle 7 is assigned to act as a communication relay between vehicles 1 and 8, comm(7, 1, 8). We are given control of vehicles 2, 3, 4, and 7. Vehicle 1 is treated as a friendly convoy moving through contested space, vehicles 5 and 6 are treated as potentially hostile agents, and vehicle 8 is treated as a ground station.


