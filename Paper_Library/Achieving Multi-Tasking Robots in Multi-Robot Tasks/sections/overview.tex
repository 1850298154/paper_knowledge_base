%%%%%%%%%%%%%%%%%%%%%%%%%%%%%%%%%%%%%%%%%%%%%%%%%%%%%%%%%%%%%%%%%%%%%%%%%%%%%%%%
\section{Approach}
\label{sec:overview}

A physical constraint in our work is modeled as an information instance or a combination of information instances,
which are categorized by information types. 
Next, we first formally introduce these two notions. 

\begin{definition}[Information Instance]
    An information instance, denoted as $F(\textbf{E})$, captures the {\textbf{\textit{semantics}}} of information where $F$ is the information type and $\textbf{E}$ is an ordered set of referents.
    \label{def:info-inst}
    \qed
\end{definition}

%Essentially, a configuration constraint specifies a physical constraint that must be satisfied by the relevant entities,
%in order to retrieve the corresponding information instance. 
\begin{table*}
    \renewcommand{\arraystretch}{1.3} 
    \caption{Examples of inference rules used in this work} 
    \centering
    \begin{tabular}{| l | c |}
        \hline
        \multicolumn{1}{| c |} {Rule} & Description \\
        \hline
        $F_{G}(X) + F_{R}(Y, X) \Rightarrow F_{G}(Y)$ & global position of $X$ + relative position of $Y$ to $X$ $\Rightarrow$ global position of $X$\\
        \hline
        $F_{R}(Y, X) \Rightarrow F_{R}(X, Y)$ & relative position of $Y$ to $X$ $\Rightarrow$ relative position of $X$ to $Y$\\
        \hline
        $F_{R}(X, Z) + F_{R}(Y, Z) \Rightarrow F_{R}(X, Y)$  & relative position of $X$ to $Z$ + relative position of $Y$ to $Z$ $\Rightarrow$ relative position of $X$ to $Y$\\
         \hline
   \end{tabular}
    \label{tab:rules}
\end{table*}

Information instances are used to label the actual information. 
In this work, we use capital $F$ to denote information instance and type,
and $f$ to denote the {\textbf{\textit{value}}} of the actual information. 
For example, $F_R(r_1, r_2)$ is used  to refer to ``{\it the relative position between $r_1$ and $r_2$}'',
where the suffix $R$ denotes relative position;
$f_R(r_1, r_2)$ corresponds to a specific value of this information. 
For brevity, we often use $F$ without the referents to denote an information instance. 
Next, we more formally define physical constraint as follows:

\begin{definition}[Physical Constraint]
    A physical constraint is
     a constraint on the value of an information instance $F$. 
    \label{def:info-config}
    \qed
\end{definition}

Notice that the exact value for a constraint may depend on the environment settings and robot configurations dynamically, and hence is not always specified a priori.  
For example, in a tracking task, a constraint specified with respect to the target may be influenced by the environment settings (e.g., whether occlusions are likely to occur). 
This value is assumed to be determined by the execution module (which is not the focus of this work). 
%and always exist for any constraint under any environment settings. 
To infer about information invariant, we define
information inference:


\begin{definition}[Inference Rules]
    Given a set of information instances, $S$, and an information instance $F$, 
    an inference rule defines a relationship such that any value set for $S$, i.e., $\{f_1: F_1 \in S\}$, uniquely determines the value of $F$ (i.e., $f$), or written as $S \Rightarrow F$.
    \qed
    \label{def:infer-iis} 
\end{definition}

For example, 
$\{F_R(r_1, r_2), F_G(r_2)\}$ (the relative position between $r_1$ and $r_2$ and the global position of $r_2$) can be used to infer $F_G(r_1)$ (the global position of $r_1$).
See Tab. \ref{tab:rules} for a few more examples. 


