\section{Introduction}
\label{sec:1}

The parallelism offered by multi-robot systems is a natural fit for missions in which large numbers of tasks must be achieved as quickly as possible \cite{brambilla2013swarm}. In realistic settings, each task requires robots with specific \emph{skills}, such as specific sensors, actuators, or computational resources. However, as the diversity of the tasks increases and the set of required skills grows, it becomes infeasible to envision swarms of identical robots that can interchangeably perform any task. Rather, specialization and redundancy become desirable due to better scale economy and expected long-term resilience \cite{ramachandran2019resilience,prorok2021beyond}.

The goal of this paper is to contribute to realizing this vision. In our setting, a heterogeneous swarm of multi-skilled robots must perform a set of tasks as quickly as possible. We assume that, due to the diversity of the tasks, the robot must form \emph{coalitions}, i.e., teams of robots that, combined, offer the required skills for each task to be completed \cite{barton2008coalitionSocial,capezzuto2020coalitions,capezzuto2021schedulingCoalition}. In addition to heterogeneity, our problem setting has two crucial features: \emph{(i)} the diversity of the tasks also requires the coalitions to be \emph{dynamically} formed and disbanded on a per-task basis; and \emph{(ii)} all the required robots in a coalition must be present at the same time for the task to progress. These two features imply that, along with the combinatorial problem of forming coalitions, the robots must also \emph{schedule} the optimal task agenda in a coherent manner.

The key difference between our work and existing works is that we consider \emph{simultaneously} multi-skill coalition formation and multi-robot task scheduling for a complete coverage problem.
With reference to the Korsah \textit{et al.} taxonomy \cite{korsah2013taxonomy}, this problem is an instance of cross-schedule dependencies (XD), single-task robots (ST), multi-robot tasks (MR), and time-extended assignment (TA). In addition, to make our problem setting more realistic, we enrich our formulation with \emph{stochastic travel times} across tasks \cite{nam2016optimalAssignment}.

We study two centralized algorithms to solve this problem. The first algorithm is optimal, but it scales poorly with the number of tasks, robots, and skills. In contrast, the second method scales to hundreds of tasks, robots, and skills. Even though the latter method offers no optimality guarantees, we empirically show that its performance is within a factor of 2 with respect to the optimum.

% \begin{figure}[h]
% \centering
% \includegraphics[scale=0.5]{Images/banner_image.png}
% \caption{An example graph}
% \label{fig:x cubed graph}
% \end{figure}

\begin{figure}[t!]
    \centering
    \includegraphics[width=0.45\textwidth]{Images/banner_image.png}
    % \includegraphics{Images/greedy_solve_time.png}
    % \includegraphics[width=0.4\textwidth]{figures/graphs/fixing/0.5.1.png}
    % \includegraphics[width=0.5\textwidth]{figures/updated_figures/modified/counter_on_max_detractor_log.jpg}
    \caption{A heterogeneous swarm of multi-skilled robots forming coalitions. \textmd{The tasks in this diagram are excavation of blue resources (bottom left), and green material sampling (center). The excavation task requires a resource sensor, an excavating arm, and a bucket to carry resources. Material sampling requires a scanner to find and locate the sample, and an arm on a legged robot.}}
    \label{fig:banner_image}
\end{figure}

The rest of this paper is structured as follows. In Sec. 2 we survey related work on coalition formation and multi robot scheduling. In Sec. 3 we discuss the problem formulation for both of our approaches. In Sec. 4 we analyze the results of both the methods and compare them with each other. We conclude the paper in Sec. 5.

% * some intro on MRTA
% * why heterogeneous MRTA is important

% * Brief about coalition and skills
% * example of law enforcement and compare the two


% * Brief our paper. Whats input and whats output. 
% * Explain the dynamic coalition formation 
% * What separates us? dynamic coalition+stochasticity+scheduling+(possibly degrading skills)

% <Make sure there's no passive voice>
% <Can't get ABT structure>

% In the recent few years, heterogeneous multi robot systems have proven their potential to be able to tackle the complex problems effectively. 
% When a large number of diverse tasks need to be executed, it is usually cheaper to have a heterogeneous set of robots such that fewer robots have the expensive sensors. 
% As humanity is creeping towards being a multi-planetary civilization, we must start focusing on techniques to optimize the utilization of the limited resources that we may have on another planet. The demands will be diverse, and the available resources will be sparse. In addition to that, the collaboration between multiple robots is also going to be crucial. We already have two completely different robots, Ingenuity and Perseverance, working together on Mars. It is not far-fetched to imagine a swarm of heterogeneous robots working together towards the same goal.

% For example, let us consider sending fleet robots with a bunch of sensors to Mars. We already know that it is expensive to put things on another planet. Hence, it would be better to divide and share the sensors among all the robots instead of equipping each robot with every sensor possible. That way, we can send all the necessary sensors while cutting down on the total number of sensors sent.

% Also, the tasks expected from the robots may be diverse. 
% We may have a digging site that needs both an `ultrasonic sensor' and a `visual sensor.' At the same time an inspection site may require a pair of a `visual sensor' and a `laser sensor' for mapping the region. The robots will need to collaborate to ensure all the tasks fulfilling their requirement. However, the robots available are heterogeneous, and the tasks may require more than one robot for task completion. This condition makes it difficult to assign tasks to the robots. The complications are even more when we consider the task start condition. The task can only start when all the necessary robots are present simultaneously. Therefore, in this work, we propose two methodologies that will be able to produce an optimal and a sub-optimal feasible solutions for such a problem. 
% Furthermore, considering the unknown terrain of travel, we will be assuming stochastic travel times across tasks.

% In this problem, we can see it as a combination of two problems that have been in the literature for a long time. We would like to categorize this problem into the following categories: 
% \begin{enumerate}
%     % \item Multi-robot task allocation (MRTA): Explicit or intentional cooperation in which tasks are explicitly assigned to a robot or sub-team of robots [korsah]
%     \item \textbf{Coalition formation:} Creating sub-teams of the available robots and assigning them the tasks.
%     \item \textbf{Multi traveling salesman problem (MTSP):} An extension of the famous traveling salesman problem, which has multiple salesmen instead of a single one. 
%     % \item \textbf{Stochastic time of travel:} Considering the variance in time taken for travel between two destination.
% \end{enumerate}

% A single robot may or may not be enough to fulfill the requirements of a task. Hence, we can frame this as a `coalition formation' problem in which we have to form sub-teams of the available robots. The formed team then needs to execute a task as a single entity. But in our case, the formed coalition cannot be permanent. Because if a robot has a `rare sensor,' all the teams may need to share that robot. Hence, we will need to consider the coalitions to be dynamic. The `dynamic' part of the coalition comes from the fact that the coalitions form just for the tasks and disband at the end of the tasks.

% % Similarly, to start the task execution, all the assigned robots must be present at the task simultaneously. This implies that the early arriving robot may need to wait until other robots arrive. Hence, we will need to schedule the robots' arrival and departure from the tasks such that minimum time is lost waiting for the other robots. Also, our main objective is to complete all the tasks as fast as possible. Hence, the sequence of the tasks attended by the robots should be optimal to not waste much time in travel. We can interpret this part of the problem as an MTSP. However, unlike a standard MTSP, in our case, a task can and should be attended by multiple robots. Hence, we must modify the MTSP such that multiple robots can attend a single task if needed, but at the same time there should not be an overload of the sensors present at the task.

% Our main objective is to complete all the tasks as fast as possible. And the robots' interwoven schedules mean that each robot's trajectory must be optimum. But the task execution can only start when all the required robots are present simultaneously. This implies that the early arriving robot may need to wait until other robots arrive. Hence, in our method we will consider not only each robot's trajectory but also the wait time needed at each task. In addition to that, we will also be considering the travel times between the two tasks to be stochastic. 

% % Finally, solution should be resilient enough to consider that the robots might get delayed or they might take longer than the estimation to reach the destination. 

% All the problems discussed here are (NP?) hard and non-convex. Hence, they are difficult to solve even on their own, and combining them makes this problem even harder. Therefore, in this paper, we will discuss two methods to solve the combined problem of modified MTSP with the dynamic coalition. One of the proposed methods uses an off-the-shelf optimizer Gurobi and produces an optimal solution. But as the problem is non-convex, the method takes too long to produce a solution and is not scalable. The second discussed method is a greedy method that is very quick but produces a sub-optimal solution. 