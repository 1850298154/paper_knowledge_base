%%%%%%%%%%%%%%%%%%%%%%%%%%%%%%%%%%%%%%%%%%%%%%%%%%%%%%%%%%%%%%%%%%%%%%%%%%%%%%%%
%2345678901234567890123456789012345678901234567890123456789012345678901234567890
%        1         2         3         4         5         6         7         8

\documentclass[letterpaper, 10 pt, conference]{ieeeconf}  % Comment this line out if you need a4paper

%\documentclass[a4paper, 10pt, conference]{ieeeconf}      % Use this line for a4 paper

\IEEEoverridecommandlockouts                              % This command is only needed if 
                                                          % you want to use the \thanks command

\overrideIEEEmargins                                      % Needed to meet printer requirements.

%In case you encounter the following error:
%Error 1010 The PDF file may be corrupt (unable to open PDF file) OR
%Error 1000 An error occurred while parsing a contents stream. Unable to analyze the PDF file.
%This is a known problem with pdfLaTeX conversion filter. The file cannot be opened with acrobat reader
%Please use one of the alternatives below to circumvent this error by uncommenting one or the other
%\pdfobjcompresslevel=0
%\pdfminorversion=4

% See the \addtolength command later in the file to balance the column lengths
% on the last page of the document

% The following packages can be found on http:\\www.ctan.org
% \usepackage{graphics} % for pdf, bitmapped graphics files
%\usepackage{epsfig} % for postscript graphics files
%\usepackage{mathptmx} % assumes new font selection scheme installed
%\usepackage{times} % assumes new font selection scheme installed
\usepackage{amsmath} % assumes amsmath package installed
%\usepackage{amssymb}  % assumes amsmath package installed

\usepackage{balance} % for balancing columns on the final page
\usepackage{algorithm}
\usepackage{algpseudocode}
\usepackage{tikz}
\usepackage{booktabs}
\usepackage{graphicx}
\usepackage{caption}
\usepackage{subcaption}
\usepackage{lineno}
\usepackage[noadjust]{cite}
% \usepackage{subfigure}
\def\checkmark{\tikz\fill[scale=0.4](0,.35) -- (.25,0) -- (1,.7) -- (.25,.15) -- cycle;} 
\newcommand{\ashay}[1]{\textbf{\textcolor{blue}{Ashay: #1}}}

\title{\LARGE \bf
Heterogeneous Coalition Formation and Scheduling \\
with Multi-Skilled Robots}

\author{Ashay Aswale and Carlo Pinciroli% <-this % stops a space
% \thanks{*This work was not supported by any organization}% <-this % stops a space
\thanks{All the authors are with the Dept. of Robotics Engineering, Worcester Polytechnic Institute, Worcester, MA, USA (email: {\tt\small \{asaswale, cpinciroli\}@wpi.edu}).}}


\begin{document}



\maketitle
\thispagestyle{empty}
\pagestyle{empty}


%%%%%%%%%%%%%%%%%%%%%%%%%%%%%%%%%%%%%%%%%%%%%%%%%%%%%%%%%%%%%%%%%%%%%%%%%%%%%%%%
\begin{abstract}

We present an approach to task scheduling in heterogeneous multi-robot systems. In our setting, the tasks to complete require diverse skills. We assume that each robot is multi-skilled, i.e., each robot offers a subset of the possible skills. This makes the formation of heterogeneous teams (\emph{coalitions}) a requirement for task completion. We present two centralized algorithms to schedule robots across tasks and to form suitable coalitions, assuming stochastic travel times across tasks. The coalitions are dynamic, in that the robots form and disband coalitions as the schedule is executed. The first algorithm we propose guarantees optimality, but its run-time is acceptable only for small problem instances. The second algorithm we propose can tackle large problems with short run-times, and is based on a heuristic approach that typically reaches 1x-2x of the optimal solution cost. %To the best of our knowledge, this work is the first to simultaneously consider coalition formation of multi-skilled robots and task scheduling.

\end{abstract}

% Re-written by Carlo


\IEEEPARstart{T}{wo} %
main challenges in the deployment of large-scale swarms are the localization and coordination of vehicles.
Localization methods that rely on external infrastructure 
(e.g., GPS) 
are prone to systematic errors (e.g., multipath effect)
and may not always be available.
Coordination strategies that are centralized can deconflict motion plans to prevent collisions and gridlock, but introduce a single point of failure and are difficult to scale in swarm size due to communication bandwidth limitations.

This paper presents a unified formation flying pipeline for unmanned aerial vehicles (UAVs).
Our pipeline uses \textit{onboard} sensors for localization, which eliminate the need for external positioning systems, and \textit{distributed} techniques for coordination, which enable each vehicle to make decisions independently while communicating their state to a subset of the team.
For \textit{localization}, we use an off-the-shelf commercial visual inertial odometry (VIO) package \cite{VIO}
that fuses inertial measurement unit (IMU) and downward-facing monocular camera measurements to estimate changes in the vehicle pose.
\edit{For \textit{coordination}, we present distributed formation control and task assignment strategies that run onboard the vehicles, do not rely on a common reference frame, and use vehicle-to-vehicle communication.} 
Key features of our formation control strategy include scalability to a large number of vehicles and robustness to disturbances.
The latter is crucial for reaching the desired formations with sensing imperfections.
Our task assignment strategy uses an auction-based algorithm to guarantee conflict-free assignments.
This algorithm can deconflict vehicle gridlocks resulting from distributed collision avoidance (type 3 deadlock~\cite{Wang2017}) and is well-suited for vehicles with limited computational capability and low-bandwidth communication. 


\begin{figure}[t!]
	\begin{center}
		\includegraphics[trim =0mm 10mm 0mm 0mm, clip, width=\columnwidth]{Figs/slanted_plane.png}	
		\caption{
		Six multirotors in a slanted plane formation.
		Vehicles communicate with each other, make distributed decisions onboard, and use VIO for localization.}
		\label{fig:slantedplane}
	\end{center}
\end{figure}


\subsection{Contributions}

This research extends our previous work on UAV formations~\cite{Fathian2019} and presents a unified pipeline consisting of \textit{onboard localization} and \textit{distributed coordination}.
The three main contributions of this work are:
\begin{enumerate}
    \item \edit{scalable formulation of control design suitable for
    onboard sensing without a common reference frame;}
    \item algorithms for deconfliction via \edit{distributed} task assignment of vehicles to desired formation points;    
    \item simulation- and hardware-ready open-source pipeline.
\end{enumerate}
\edit{Our pipeline is tested in hardware with six multirotors (see Fig.~\ref{fig:slantedplane}), and 
to our knowledge is the first demonstration of formation flying that does not rely on external sensing, fiducial markers for localization, a common reference frame, or a centralized base station for coordination.}
The only requirements for the presented pipeline are that the vehicles can communicate, can find the transformation between their VIO start frames, and the environment is sufficiently textured---a standard assumption for VIO systems.
As such, this framework paves the way for future, real-world deployments of aerial vehicle swarms in large numbers and without requiring external localization infrastructure.


\begin{figure} [t!]
\centering
	\begin{subfigure}[b]{0.32\columnwidth}
	   %
	    \includegraphics[width=0.8\textwidth,left]{Figs/Frames2_full.pdf}
	    \caption{\scriptsize full alignment}
	    \label{fig:frame-a}
	\end{subfigure}
	\begin{subfigure}[b]{0.32\columnwidth}
	    \includegraphics[width=0.8\textwidth,center]{Figs/Frames2_orientation.pdf}
	    \caption{\scriptsize orientation alignment}
	    \label{fig:frame-b}
	\end{subfigure}
	\begin{subfigure}[b]{0.32\columnwidth}
	    \includegraphics[width=0.8\textwidth,right]{Figs/Frames2_none.pdf}
	    \caption{\scriptsize no alignment}
        \label{fig:frame-c}
	\end{subfigure}
\caption{\edit{Required alignment of UAV frames in existing swarm strategies: (a) the most restrictive case requiring a common reference frame, i.e., orientation and origin of the frames must be aligned; (b) only the orientation of the frames must be aligned; (c) no alignment restrictions (this work).}}
	\label{fig:Frames}
\end{figure}




\subsection{Related Work}

Existing aerial swarms can be grouped based on the coordination (centralized vs.\ distributed) and localization (external vs.\ onboard) methods used. 
\edit{It is further crucial to distinguish these methods based on the level of alignment required for the vehicle coordinate frames; see Fig.~\ref{fig:Frames}.} 
 
\edit{
Works with \textit{centralized} coordination and \textit{external} localization include~\cite{Preiss2017, Honig2018, Du2019}, which are based on lightweight UAVs with limited onboard computational capability and therefore rely on an external motion capture system and a base station.
Works with \textit{distributed} coordination and \textit{external} localization include \cite{wilson2020robotarium}, \cite{enright2004spheres}, where robots execute distributed controls  based on external localization by motion capture and ultrasonic beacons, respectively.
Works with \textit{centralized} coordination and \textit{onboard} localization include~\cite{Forster2013}, \cite{Loianno2016}, which use a ground station for task assignment among vehicles.
In \cite{Weinstein2018}, formation flying based on VIO is demonstrated, where motion planning and assignment are run on a base station to ensure collision-free trajectories.
The coordination strategies used in aforementioned works require a \textit{common reference frame} (Fig.~\ref{fig:frame-a}).
}


\edit{
Despite the large body of work on formation control~\cite{Oh2015}, and the variety of onboard sensing solutions for localization (e.g., VIO~\cite{Delmerico2018}), few frameworks demonstrated formation flying with \textit{distributed} coordination and \textit{onboard} localization.
A key reason is reliance of many distributed control and assignment algorithms on aligned frames (Fig.~\ref{fig:frame-a}, \ref{fig:frame-b}), which require computation-expensive and/or communication-intensive synchronization/consensus steps for frame alignment.
Equally important, dependence on alignment in existing methods \cite{Wang2017,Turpin2014, van2011reciprocal, morgan2016swarm} diminishes robustness to inherent noise and unobservable errors that cannot be corrected (e.g., disparities between the actual and estimated body frame \textit{orientation} caused by VIO drift).
Leveraging coordination methods that are \textit{robust to misaligned frames} is hence crucial and a focus of this work. 
}






\edit{
Examples of other pipelines with distributed coordination and onboard localization include \cite{Montijano2016,Tron2016}.
Both works demonstrated formation flying on three UAVs, required information from an external motion capture system due to hardware limitations, did not incorporate collision avoidance, and required frame alignment.
}
\edittwo{Note that while~\cite{Montijano2016,Tron2016} can achieve formations with arbitrary headings as illustrated in Fig.~\ref{fig:frame-c}, knowledge of relative orientations is still required; therefore, they belong to the category of Fig.~\ref{fig:frame-b}.}






\if 0

\r{
decentralized coordination setting combined with VIO:
D-CAPT [26]~\cite{}:
ORCA ~\cite{}: 
CBF [2]~\cite{} :
[A]
}

\r{Robusteness in coordination,  with compounded noise/latency, which would eventually break (b).\\


some existing algorithm might as well
work in a similar fully decentralized setting, when combined with VIO
as proposed here. For example, D-CAPT [26], ORCA, CBF [2] might also be
useful for such a task and are computationally even more efficient than
the proposed approach. \\

R2:  onboard sensing for localization ->
 Finally, the related work section only
focuses on this aspect of the pipeline, discussing how many formation papers include
onboard localization but barely sells the advantages of the coordination module (the actual
proposal of the paper) against other competitors such as [26] or [A] or to mention similar
coordination pipelines. \\


Given a solution to this problem, the controller in Section III seems unnecessary, each drone
has a target position and can use a local controller with collision avoidance that drives it to
that position. Note that such controllers exists in the literature (e.g., RVO in any of its
multi-agent variantes), they are distributed in nature and only require local sensing.


}

\fi
\section{Related work}\label{background} 
Task allocation is a well-studied problem, posing ongoing challenges in various computing environments \cite{Stavrinides2019, Genez2020, Jayanetti2022, Kanbar2022, Kritikakou2022, Peixoto2022, Mo2023}.
However, previous related research efforts do not consider the edge/hub/cloud architecture, nor all of the parameters investigated in this work. This is demonstrated in \cref{table:comparison}, which summarizes our qualitative comparison with relevant state-of-the-art approaches. 
The comparison is made with respect to the objectives and parameters considered in this work, the applicability of each approach to applications comprising multiple tasks with precedence relationships among them (i.e., applications with a task flow graph structure), as well as the optimality of the solution provided by each method.
An overview of the related literature, as well as a comparison with our preliminary research, are provided in the remainder of this section.


\begin{table*}[!ht]
\centering
\caption{Qualitative comparison of this work with relevant research efforts.}
\label{table:comparison}
\footnotesize
\resizebox{0.85\textwidth}{!}{
    \begin{tabular}{@{\extracolsep{4pt}}lcccccccccc@{}} 
        \toprule
        \multirow{3}{*}{Reference} & \multicolumn{2}{c}{Objectives} & \multicolumn{6}{c}{Considered Parameters} & \multirow{2}{*}{Task Flow} & \multirow{2}{*}{Optimal}\\
         \cline{2-3}   \cline{4-9} 
        & Latency & Energy & Comp. & Comp. & Comm. &  Comm. &  \multirow{2}{*}{Memory} & \multirow{2}{*}{Storage} & \multirow{2}{*}{Graph} & \multirow{2}{*}{Solution}\\
        & Min. & Min. & Latency & Energy & Latency & Energy & & & & \\
        
        \hline
        %-- Latency Minimization References
        %                               Latency         Energy       Comp.          Comp.        Comm.        Comm.        Memory        Storage       TFG           Optimal Solution
        \cite{Alfakih2021}              & \checkmark    & -           & \checkmark  & -           & -          & -          & \checkmark  & \checkmark & -          & -           \\
        \cite{Guevara2022}              & \checkmark    & -           & \checkmark  & -           & \checkmark & -          & \checkmark  & \checkmark & \checkmark & -           \\
        \cite{Weikert2022}              & \checkmark    & -           & \checkmark  & -           & \checkmark & \checkmark & \checkmark  & -          & \checkmark & -           \\
        \cite{Lai2022}                  & \checkmark    & -           & \checkmark  & -           & \checkmark & -          & \checkmark  & \checkmark & -          & -           \\
        \cite{Barijough2019}            & \checkmark    & -           & \checkmark   & -          & \checkmark & \checkmark & -           & -          & \checkmark & \checkmark  \\
        \cite{Tang2022}                 & \checkmark    & -           & \checkmark   & -          & \checkmark & -          & -           & \checkmark & -          & \checkmark  \\
        \cite{Kuang2021}                & \checkmark    & -           & \checkmark   & \checkmark & \checkmark & \checkmark & -           & -          & -          & -           \\
        
        %-- Energy Minimization References
        %                               Latency         Energy       Comp.          Comp.        Comm.        Comm.        Memory      Storage         TFG           Optimal Solution                  
        \cite{Avgeris2022}              & -             & \checkmark  & \checkmark  & \checkmark  &\checkmark & -           & -           & -          & -          & \checkmark \\
        \cite{Khalil2018}               & -             & \checkmark  & -           & \checkmark  & -         & \checkmark  & -           & -          & -          & -          \\
        \cite{Kritikakou2023}           & -             & \checkmark  & \checkmark  & \checkmark  & -         & -           & -           & -          & \checkmark & -          \\
        \cite{Hu2020}                   & -             & \checkmark & \checkmark   & \checkmark & \checkmark & \checkmark & -          & -            & -          & -          \\
        \cite{Azizi2022}                & -             & \checkmark & \checkmark   & \checkmark & \checkmark & -          & -          & -            & -          & -          \\
        \cite{Li2022}                   & -             & \checkmark & \checkmark   & \checkmark & \checkmark & \checkmark & -          & -            & -          & -          \\

        %-- Latency and Energy Minimization References
        %                               Latency         Energy       Comp.          Comp.        Comm.        Comm.        Memory      Storage         TFG          Optimal Solution          
        \cite{Zhang2021}                & \checkmark    & \checkmark & \checkmark   & \checkmark & \checkmark & \checkmark & -          & -            & -          & -          \\
        \cite{Dinh2017}                 & \checkmark    & \checkmark & \checkmark   & \checkmark & \checkmark & \checkmark & -          & -            & -          & -          \\        
        \cite{Tong2023}                 & \checkmark    & \checkmark & \checkmark   & \checkmark & \checkmark & \checkmark & -          & -            & -          & -          \\

        
        This work & \checkmark & \checkmark & \checkmark & \checkmark & \checkmark & \checkmark & \checkmark & \checkmark & \checkmark  & \checkmark \\
        \bottomrule
    \end{tabular}
}
%\vspace{-3mm}
\end{table*}



\subsection{Latency minimization}
A number of works on task allocation in edge computing and multi-tier environments have a primary focus on latency minimization.
For instance, Alfakih et al. \cite{Alfakih2021} explore the minimization of the computational latency of task execution in an edge computing system, based on an accelerated particle swarm optimization algorithm combined with a dynamic programming approach.
Guevara et al. \cite{Guevara2022} present a reinforcement learning-based resource allocation technique for minimizing the total execution time of tasks in a fog-cloud environment.
On the other hand, Weikert et al. \cite{Weikert2022} propose an algorithm for task allocation in an IoT platform, aiming to optimize the overall latency.
Furthermore, Lai et al. \cite{Lai2022} propose an online Lyapunov optimization-based method to tackle the problem of allocating user tasks in an edge computing environment, utilizing a stochastic approach. 
Barijough et al. \cite{Barijough2019} introduce a technique for allocating real-time streaming applications under latency and quality constraints.  
Tang et al. \cite{Tang2022} propose a framework for managing the physical resources of the edge and cloud layers, so that the response time is minimized and the system throughput is improved.
Moreover, Kuang et al. \cite{Kuang2021} present an iterative algorithm based on Lagrangian dual decomposition in order to minimize latency in an edge computing system. 


\subsection{Energy consumption minimization}
Several studies are focused on task allocation strategies aiming to reduce the total energy consumption.
Specifically, Avgeris et al. \cite{Avgeris2022} propose a resource allocation technique based on mixed integer linear programming in order to minimize the energy consumption of edge servers.
Within this context, Khalil et al. \cite{Khalil2018} present a framework for energy-efficient task allocation in an IoT environment, utilizing evolutionary-based meta-heuristics. 
Cui et al. \cite{Kritikakou2023} propose a heuristic algorithm for minimizing the total energy consumption of a platform comprising homogeneous processors, utilizing dynamic voltage and frequency scaling (DVFS).  
On the other hand, Hu et al. \cite{Hu2020} introduce a game-theoretic approach for task allocation in an edge computing environment to minimize the system energy consumption within an acceptable delay range.
Similarly, Azizi et al. \cite{Azizi2022} propose two priority-aware semi-greedy algorithms for allocating  IoT tasks in a heterogeneous fog platform, so that the total energy consumption is optimized, while meeting the deadline of each task.
Furthermore, Li et al. \cite{Li2022} examine a two-stage iterative algorithm, in which the resource allocation problem is decomposed into two sub-problems to obtain a suboptimal solution. 




\subsection{Latency and energy consumption minimization}
On the other hand, certain related works consider both optimization objectives, the minimization of latency and energy consumption.
For instance, Zhang et al. \cite{Zhang2021} present a game theory-based scheme for task allocation in a UAV-assisted edge computing environment. The goal of the proposed approach is to minimize the weighted latency and energy consumption of the system, considering resource allocation constraints.
Dinh et al. \cite{Dinh2017} propose a semi-definite relaxation-based optimization framework for allocating tasks in an edge architecture. The particular framework aims to minimize the total latency of the tasks, as well as the total energy consumption of the system.
On the other hand, Tong et al. \cite{Tong2023} present a latency and energy-aware Stackelberg game-based task allocation strategy, considering an edge device with limited computational resources.


\subsection{Our approach vs. state-of-the-art}
Overall, none of the aforementioned research efforts considers the specific edge/hub/cloud architecture examined in this work. 
Furthermore, some approaches do not take into account the energy required for the execution of the tasks \cite{Alfakih2021, Guevara2022, Weikert2022, Lai2022, Barijough2019, Tang2022} or the energy consumed for inter-task communication \cite{Alfakih2021, Guevara2022, Lai2022, Tang2022, Avgeris2022, Kritikakou2023, Azizi2022}. 
The majority of the related studies consider devices with unlimited resources, such as memory \cite{Barijough2019, Tang2022, Kuang2021, Avgeris2022, Khalil2018, Kritikakou2023, Hu2020, Azizi2022, Li2022} and storage \cite{Weikert2022, Barijough2019, Kuang2021, Avgeris2022, Khalil2018, Kritikakou2023, Hu2020, Azizi2022, Li2022}, an assumption that is not realistic, especially in the case of resource-limited devices at the edge of the network.     
Moreover, several approaches are only applicable to single-task applications \cite{Alfakih2021, Lai2022, Tang2022, Kuang2021, Avgeris2022, Khalil2018, Hu2020, Azizi2022, Li2022} or cannot provide an optimal solution to each of the objectives considered in this work \cite{Alfakih2021, Guevara2022, Weikert2022, Lai2022, Kuang2021, Khalil2018, Kritikakou2023, Hu2020, Azizi2022, Li2022}.

Related studies that are closer to ours \cite{Zhang2021, Dinh2017, Tong2023}, even though they consider both the latency and energy aspects of the problem, do not take into account the memory and storage limitations of the devices. Furthermore, they cannot be applied to applications with precedence relationships among their tasks, and can only provide suboptimal solutions.
Hence, our proposed approach aims to fill these gaps, by incorporating all of the important parameters that characterize an edge/hub/cloud environment, providing an optimal allocation for a task flow graph. 


\begin{figure*}[t]
    \centering
    \includegraphics[width=.85\textwidth]{coins_journal_tfg_to_etfg_v6.1.1.pdf}
    \caption{Overview of proposed optimization framework. The task flow graph transformation, including the encapsulated energy model, is described in \cref{extended,subsec:energyModel}. The formulation of the optimization problem is presented in \cref{subsec:optimization}.}
    \label{flow}
    %\vspace{-3mm}
\end{figure*}


\subsection{Comparison with our preliminary research}
The foundational concepts of this work were first presented in a preliminary form in \cite{Kouloumpris2019}.
Below, we outline the main differences and contributions of the current study with respect to our preliminary research:
\begin{enumerate}
    \item We streamlined and enhanced the mathematical representation of all aspects of the proposed approach, from the description of the task flow graph transformation to the modeling of the optimization problem.
    
    \item We extended our optimization framework to consider a new objective for the minimization of overall energy consumption (in addition to the latency objective), based on an improved energy model.
     
    \item We developed suitable synthetic benchmarks to further validate and evaluate the efficiency and scalability of our framework, by extending our transformation method to randomly generated task flow graphs.
    
    \item We conducted extensive experimentation with alternative configurations of different devices, for both the real-world use-case scenario and the synthetic benchmarks.  
\end{enumerate}

\subsection{Sensing and Communication Topology}
\label{subsec:graphs}
An (undirected) \textit{hypergraph} refers to a pair $(\mathcal V, \mathcal E)$ constituting a set of vertices $\mathcal V$ and a set of hyperedges $\mathcal E$, where each element of $\mathcal E$ (called a hyperedge) is a subset of $\mathcal V$ \cite[Sec. 1.10]{diestel2017}. Consider a multi-agent system represented as a hypergraph, $\mathcal G=(\mathcal V, \mathcal E)$, such that the vertices correspond to the agents (of which there are $|\mathcal V|$ in total), and each hyperedge represents the availability of a measurement that depends on the states of the corresponding agents. As hypergraphs generalize graphs by allowing more than two (or even zero or one) vertices to be connected, they enable us to represent measurements that are functions of three or more agents' states, such as subtended angle and time-difference-of-arrival (TDoA) measurements \cite{weak_rigidity,tdoa_2017}. Let $\mathcal V$ and $\mathcal E$ be endowed with arbitrary orderings, so that we may write $\mathcal V = \lbrace 1, 2, \dots, |\mathcal V|\rbrace$ and $\mathcal E= \lbrace \mathcal E^{(1)}, \mathcal E^{(2)}, \dots, \mathcal E^{(|\mathcal E|} \rbrace$.
Agents $i$ and $j$ are said to be \textit{neighbors} of each other if, for some $l\in\lbrace 1, \dots, |\mathcal E|\rbrace$, both $i$ and $j$ are in $\mathcal E^{(l)}$. The set of neighbors of agent $i$ is denoted by $\mathcal N_i \subseteq \mathcal V$.
% ; we also say that agent $i$ is able to \textit{sense} agent $j$, and vice versa. 
Each agent in $\mathcal G$ is assumed to be a Cyber-Physical System (CPS) that has a physical state, an embedded computer, and the capabilities to communicate and (potentially) obtain measurements through a variety of sensors. We make the following assumption about the communication capabilities of the agents.

\vspace{2pt}
\begin{assumption}[Communication Topology]
Each agent is able to communicate with its neighbors, and the agents are able to synchronize\footnote{The meaning of `synchronize' as it is used in Assumption \ref{ass:1} can be understood unambiguously by studying the algorithm which we develop in Section \ref{sec:distributed}.} their communications.
\label{ass:1}
\end{assumption}
\vspace{2pt}

% \begin{remark} Furthermore, the inter-agent communications can be synchronized using the distributed algorithm presented in \cite{sync2010}.\end{remark}\vspace{2pt}

Thus, we have implicitly made the assumption that if agent $i$ is able to sense agent $j$, then agents $i$ and $j$ are able to establish a bidirectional communication channel between them as well. 
% We can represent the resulting communication topology using the \textit{adjacency graph} of $\mathcal G$, which is defined as the graph $\mathcal G_{Ad}=(\mathcal V, \mathcal E_{Ad})$ where $\mathcal E_{Ad}\subseteq \mathcal V \times \mathcal V$, such that $(i, j)$ and $(j,i)$ are in $\mathcal E_{Ad}$ if and only if agents $i$ and $j$ are neighbors in $\mathcal G$.
% While the hypergraph $\mathcal G$ simplifies the forthcoming notation, the graph $\mathcal G_{Ad}$ is easier to visualize.

\subsection{Agent States and Estimates}
The collective state of the multi-agent system is represented by a block vector $\mathbf p$, which has the form 
\begin{equation}
\mathbf p = \begin{bmatrix} \mathbf p[1]^\top & \mathbf p [2]^\top & \dots & \mathbf p [|\mathcal V|]^\top \end{bmatrix}^\top \in \mathbb R^n
\end{equation}
where $\mathbf p[i] \in \mathbb R^{n_i}$ is the $i^{th}$ agent's state, $n_1, n_2, \dots, n_{|\mathcal V|}$ are positive integers representing the dimensions of each of the agents' states, and $n \coloneqq \sum_{i\in \mathcal V} n_i$. We call the pair $(\mathcal G, \mathbf p)$ (or equivalently,  the vector $\mathbf p$) a \textit{configuration}, and $\mathbb R^n$ is called the configuration space.
% Similar to \cite{whiteley1985generating} and \cite{anderson2010formal}, we say that a configuration is \textit{generic} if its state vector $\mathbf p$ does not lie in a measure-zero subset of $\mathbb R^n$ where a certain property fails to hold. The corresponding property is said to be a generic property of the configuration space. An example of a nongeneric configuration is one where the states of three of the agents are precisely collinear; such cases are excluded when we study generic configurations and their properties.


Each agent uses a suite of onboard sensors to estimate its own state and self-reports the estimated state to its neighbors. The reported state of agent $i$ is denoted by $\hat {\mathbf p}[i]$, which may or may not coincide with $\mathbf p[i]$. The discrepancy between $\mathbf p$ and $\hat {\mathbf p}$ is encapsulated in the error vector, defined as $\mathbf x \coloneqq \mathbf p - \hat {\mathbf p}$.  We say that there is a \textit{fault} or an \textit{error} at agent $i$ to mean that $\mathbf p[i] \neq \hat{\mathbf p}[i]$, or equivalently, $\mathbf x[i] \neq \mathbf 0$. This can occur if agent $i$ has wrongly estimated its state due to sensor bias, miscalibration, modeling errors, or adversarial sensor spoofing attacks.
% \begin{remark}
% Our paper does not consider the problem of categorizing the source of the error into either of the preceding cases; rather, we develop an algorithm for reconstructing $\mathbf x$ in a distributed manner, so that the error at agent $i$, $\mathbf x[i]$, can be determined irrespective of its source.
% \end{remark}
Let $\mathcal D$ be the set of agents that have errors. We have,
\begin{equation}
|\mathcal D| = \sum_{i\in \mathcal V} \mathbb I\left(\mathbf p[i] \neq \hat{\mathbf p}[i]\right) = \|\mathbf x\|_{2,0},
\end{equation}
which is in turn
equal to the number of non-zero blocks of $\mathbf x$, referred to as its \textit{block-sparsity}.
We say that the errors are sparse, and that $\mathbf x$ is block-sparse, if 
$\|\mathbf x\|_{2,0} \ll|\mathcal V|$.

\vspace{2pt}
\begin{remark}
In the literature on rigidity theory, it is assumed that the identities of the agents in $\mathcal D^\complement$, which are called the \textit{anchors}, are known to all the agents in the network \cite{zhao2019bearing}.
However, we have dropped the assumption that any of the elements of $\mathcal D$ or $\mathcal D^\complement$ are known \textit{a priori}, allowing us to diagnose the entire network for errors, rather than ruling out a subset of the agents (i.e., the anchors) from having errors. 
\label{rem:anchors}
\end{remark}
\vspace{2pt}

In the single-agent fault detection, identification, and reconstruction (FDIR) problem, the objective is to identify the subset of sensors that are faulty, as well as recover the nominal state estimation performance. We can extend this idea to the multi-agent setting by treating each agent of the network as a \textit{sensor} that is observing the collective state vector, $\mathbf p$. With this interpretation, the onboard state estimates of the agents can also be thought of as measurements of $\mathbf p$:
\begin{align}
    \hat {\mathbf p}[i] = \bigl[\ 
        \mathbf 0 \ \ \mathbf 0 \ \ &\dots \ \ 
        \underset{
            \substack{\vspace{1pt}\\
                        \textstyle \uparrow \vspace{1pt}\\
                        \scalebox{0.8}{$i^{th}\ \textrm{block}$}
                    }
                }{\mathbf I} 
        \ \ \dots \ \ \mathbf 0\ \bigr]\ \mathbf p + \mathbf x[i], 
    \label{eq:onboard}
\end{align}
where $i\in \mathcal V$. Similarly, $\mathbf x[i]$ represents an unknown additive signal whose presence or absence must be determined at each sensor, as part of the multi-agent FDIR problem. However, the measurements in (\ref{eq:onboard}) are decoupled, as only agent $i$ is able to measure $\mathbf p[i]$ and $\mathbf x[i]$. This makes it impossible for the multi-agent network to collaboratively identify faults unless an additional set of measurements is available.


\subsection{Inter-Agent Measurements}

A distinctive feature of many real-world multi-agent systems is the availability of relative measurements between them, which are often nonlinear functions of the agents' states. We allow each hyperedge in $\mathcal E$ to correspond to a different type of nonlinear measurement.
Let the inter-agent measurement model corresponding to $\mathcal E^{(l)}$, which is the $l^{th}$ edge in $\mathcal E$, be denoted by $
\mathbf \Phi^{(l)}: U^{(l)} \rightarrow \mathbb R^{m_l}$, where $m_l$ is the dimension of the measurement and $U^{(l)}$ is an open subset of $\mathbb R^n$. Some examples of inter-agent measurement models that arise in practical applications are given in Table \ref{tab:iamms}, in each of which the state vector $\mathbf p[i]$ represents the position of agent $i$ in $2$ or $3$-dimensional space.

\begin{table}[h!]
\centering
\caption{Examples of inter-agent measurement models, where $\lbrace\mathbf p[i]\rbrace_{i\in \mathcal V}$ represent positions of the agents in $\mathbb R^2$ or $\mathbb R^3$}
\bgroup
\def\arraystretch{2.0}
\setlength{\tabcolsep}{1.0em}
% \vspace{10pt}
{\rowcolors{2}
{gray!3}{gray!12}
% {green!80!yellow!50}{green!70!yellow!40}
\begin{tabular}{|p{2.5cm}|p{5.0cm}|}
\hline
Measurement Type & Expression for $\mathbf \Phi^{(l)}$ \\
\hline
Displacement \cite{oh2015survey} & $\ \mathbf p[i] - \mathbf p[j]$ \vspace{4pt}  \\
Distance \cite{oh2015survey,topology_const2015observability} & $\  \big\|\mathbf p[i] - \mathbf p[j]\big\|$ \vspace{4pt}  \\
Bearing \cite{zhao2019bearing} & $(\mathbf p[i] - \mathbf p[j])\Big/\big\|\mathbf p[i] - \mathbf p[j]\big\|$ \vspace{4pt}\\
Time-Difference-of-Arrival (TDoA) \cite{tdoa_2017} & $\big\|\mathbf p[i] - \mathbf p[j]\big\| 
- \big\|\mathbf p[i] - \mathbf p[k]\big\|$ \\
Subtended Angle \cite{weak_rigidity} &  $\bigstrut[t]
\arccos
\biggl(
\cfrac{\mathbf p[i] - \mathbf p[j]}{\|\mathbf p[i] - \mathbf p[j]\|}\overset{^\top}{^{\ }}
\cfrac{\mathbf p[i] - \mathbf p[k]}{\|\mathbf p[i] - \mathbf p[k]\|}
\biggr)$ \vspace{3pt} \\
\hline
\end{tabular}}
\egroup
\label{tab:iamms}
\end{table}

\begin{remark}
Other inter-agent measurement models, such as Frequency-Difference-of-Arrival (FDoA) \cite{tdoa_2017}, may involve the relative velocities between agents, in which case $\mathbf p[i]$ may be a block vector constituting the $i^{th}$ agent's position and velocity vectors. Similarly, the agents' orientations can be appended to their state vectors in order to model relative pose measurements (which can be obtained using cameras) \cite{aragues2011relative_pose}.
\end{remark}
\vspace{2pt}

For the measurement models given in Table \ref{tab:iamms}, the domain $U^{(l)}$ of the $l^{th}$ measurement model $\mathbf \Phi^{(l)}$ can be chosen to exclude the set of points in $\mathbb R^n$ where two agents' positions are coincident, as $\mathbf \Phi ^{(l)}$ and/or its derivatives may not be well-defined at these points. With the appropriate choices of the domains, each of the measurement models discussed thus far satisfy the following assumption.

\vspace{2pt}
\begin{assumption}[Smoothness]
For all $l=1, 2, \dots, |\mathcal E|$, $\mathbf \Phi^{(l)}$ is a smooth (i.e., infinitely differentiable) map.
\end{assumption}
\vspace{2pt}

The inter-agent measurements are collectively represented by the block vector-valued function,
\begin{align}
    \mathbf \Phi:\ &U \rightarrow \ \mathbb R^m\\
    & \mathbf p \mapsto \begin{bmatrix}
        \mathbf \Phi^{(1)}(\mathbf p)\\
        \mathbf \Phi^{(2)}(\mathbf p)\\
        \vdots\\
        \mathbf \Phi^{(|\mathcal E|)}(\mathbf p)
    \end{bmatrix}
\end{align}
where $U=U^{(1)}\times \dots \times U^{(|\mathcal E|)} \subseteq \mathbb R^n$ is the domain of $\mathbf \Phi$. Let $\mathbf y = \mathbf \Phi(\mathbf p)$ be the block vector of inter-agent measurements, which is partitioned such that $\mathbf y[l]= \mathbf \Phi^{(l)}(\mathbf p)$  $\forall l=1, 2, \dots, |\mathcal E|$. We assume that agent $i$ has access to the set of inter-agent measurements obtained in its neighborhood in $\mathcal G$, which is the set $\lbrace \mathbf y[l] \hsp \vert \hsp i \in \mathcal E^{(l)}\rbrace$. Letting, $\mathbf J_{\mathbf \Phi}(\mathbf p)$ denote the Jacobian of $\mathbf \Phi$ evaluated at the point $\mathbf p$, which is partitioned such that $\mathbf J_{\mathbf \Phi}(\mathbf p)[l,i]\in \mathbb R^{m_l \times n_i}$, the following assumption formalizes the functional dependence between the measurement models and the agents.

\vspace{2pt}
\begin{assumption}[Inter-Agent Sensing Topology]
We have, $\mathbf J_{\mathbf \Phi}(\mathbf p)[l, i] \neq \mathbf 0\ \Leftrightarrow\ i \in \mathcal E^{(l)}$. This means that the $l^{th}$ block in $\mathbf \Phi(\mathbf p)$ only depends on the states of the agents in $\mathcal E^{(l)}$.
\label{ass:jacobian}
\end{assumption}
\vspace{2pt}

With the above definitions in place, the objective of multi-agent FDIR can be stated as follows: the agents must collectively reconstruct an error vector $\hat{\mathbf x}\in \mathbb R^n$ that solves the equation $\mathbf y = \mathbf \Phi(\hat{\mathbf p} + \hat {\mathbf x})$, i.e., it explains the observed inter-agent measurements. Moreover, we require the error reconstruction to be done in a distributed manner, while ensuring that the computational and communication costs of the algorithm do not scale with $|\mathcal V|$. Suppose the error vector is reconstructed correctly, such that $\hat{\mathbf x} = \mathbf x$, then the indices of the non-zero blocks of $\hat{\mathbf x}$ can be identified with the faulty agents (i.e., the agents in $\mathcal D$). Finally, the true configuration of the multi-agent system can be determined as $(\mathcal G, \hat{\mathbf p} + \hat {\mathbf x})$, thereby solving the FDIR problem. However, as we show in the next section, the set of error vectors that explain the observed measurements, $\lbrace\hat{\mathbf x} \in \mathbb R^n \hsp \vert \hsp\mathbf y = \mathbf \Phi(\hat{\mathbf p} + \hat {\mathbf x})\rbrace$, may have infinitely many elements.
Therefore, additional assumptions and/or regularization techniques are needed to uniquely reconstruct ${\mathbf x}$.

% \vspace{2pt}
% \begin{remark}[Localization using Anchors]
% In order to cast our problem formulation into the special case where the identities of the anchors in $\mathcal D^\complement$ is known \textit{a priori}, we can include the anchors' measurements in $\mathbf \Phi$. For instance, letting the $l^{th}$ hyperedge in $\mathcal E$ correspond to an anchor's measurement of its own state, we have $\mathbf \Phi^{(l)}=\mathbf p[i]$, where $\mathcal E^{(l)} = \lbrace i\rbrace$, and $i \in \mathcal D^\complement$.
% \end{remark}
% \vspace{2pt}

% to be re-written
\subsection{Greedy formulation}

The previously discussed method produces an optimal result, but experimental evaluation reveals that it takes a long time to reach a solution. This motivates the need for another method that can solve the same problem quickly, although at the cost of optimality. We propose a simple, but effective greedy solver that produces a quick but sub-optimal result. 

%%%%%%%%%%%%%%%%%%%%%%%%%%%%%%%%%%%%%%%%%%%%
%%%% ALGORITHM FOR COALITION/SCHEDULING %%%%
%%%%%%%%%%%%%%%%%%%%%%%%%%%%%%%%%%%%%%%%%%%%

\begin{algorithm}
\begin{algorithmic}[1]
\small
% \State $P$ holds all the robots' schedules
% \State $O \gets R$

\While{Any task is unsatisfied}

\State $(i_1, k_1),... \gets \text{Robot-task pairs with max contribution}$
% \State $i_1, k^c \gets \underset{i,k \in I_1, K_1}{\arg\min} (y^{\max}_{S_{i,-1}} + t^{e}_{S_{i,-1}} + t_{iS_{i,-1}k} + t^s_{S_{i,-1}k})$
% \State $i^c, k^c \gets \underset{(i,k) \in (i_1, k_1),...}{\arg\min} (y^*_{iP_{i,-1}k})$
% \State $i^c, k^c \gets \underset{(i,k) \in (i_1, k_1),...}{\arg\min} (\footnotesize\text{estimated time of arrival of robot }i\text{ at task } k)$
\State $(i^c, k^c) \gets \text{The earliest robot-task pair from} (i_1, k_1), ...$

\State  Assign task $k^c$ to the robot $i^c$ using Algorithm \ref{alg:assign_task}

\While{Unaddressed skill at task $k^c$}
% \State $I_2 \gets$ $\underset{i}{\arg\max} (q_{i} \cdot r_{k^c}^T)$
% \State $i_2 \gets \underset{i \in I_2}{\arg\min} (y^{\max}_{S_{i,-1}} + t^{e}_{S_{i,-1}} + t_{iS_{i,-1}k^c} + t^s_{S_{i,-1}k^c})$
\State $i^d_1, ... \gets $Robots with max contributions from remaining skills at $k^c$
\State $i^d \gets $ The earliest robot from $i^d_1, ...$

\State Assign task $k^c$ to the robot $i^d$ using Algorithm \ref{alg:assign_task}
\EndWhile

\State $y^{\max}_{k^c} \gets \underset{i}{\max} \ y_{ik^c}$

\EndWhile
\end{algorithmic}
\caption{The proposed greedy algorithm}\label{alg:greedy_algo}
\end{algorithm}


%%%%%%%%%%%%%%%%%%%%%%%%%%%%%%%%%%%%%%
%%%% ALGORITHM FOR TASK ASSIGNING %%%%
%%%%%%%%%%%%%%%%%%%%%%%%%%%%%%%%%%%%%%
\begin{algorithm}
\begin{algorithmic}[1]
\small

\State $j \gets$ The current task of robot $i$
\State $x_{ijk}=1$
\State $y_{i,k} \gets y^{\max}_j + t^{e}_{j} + t_{ijk} + t^s_{jk}$
\State Update the list of unaddressed skills at task $k$

\caption{Assign task $k$ to robot $i$}\label{alg:assign_task}
\end{algorithmic}
\end{algorithm}


\subsubsection{Methodology}
% \paragraph{Methodology}

In this work, we assume that task execution can only start when all the required skills are fulfilled simultaneously. Thus, a coalition might cause its robots to wait idly until the last robot in the coalition arrives. It is thus desirable to have as small coalitions as possible with robots that cover as many skills as possible. On the other hand, only seeking a solution with small coalitions might require few, powerful robots to spend significant time travelling across the environment to attend the assigned tasks. In such a scenario, the generated paths for the robots are not optimal due to the absence of any mechanism to shorten the robots' travel path. Motivated by these observations, we propose a greedy algorithm that promotes forming small coalitions while also minimizing the distance traveled by the robots.

Our algorithm first finds the robots that can contribute the most to a task and arrive the soonest. We define a robot's `contribution' as the number of previously unoffered skills it can bring to a task. We identify all the robot-task pairs that maximize the robots' contributions to the tasks (Alg. \ref{alg:greedy_algo}, line 2). If multiple robots contribute equally, we choose the one that can reach the task location first (Alg. \ref{alg:greedy_algo}, line 3). This estimated time of arrival is calculated with the same logic as in Sec. \ref{subsec:arrival_times}. 

We use Alg. \ref{alg:assign_task} to add the task to the robot's schedule (line 2), update its arrival time (line 3), and update the task's requirements (line 4) to account for the skills provided by the attending robot. 

We now choose a robot coalition to fulfill the skills required for task $k^c$. If the task still requires additional skills to start (Alg. \ref{alg:greedy_algo}, line 5) we select the robots that can offer the highest number of the remaining skills (Alg. \ref{alg:greedy_algo}, line 6). If multiple robots are tied, we choose the one that can reach the task location first  (Alg. \ref{alg:greedy_algo}, line 7). We then use Algorithm \ref{alg:assign_task} to add the task to the robot's schedule  (Alg. \ref{alg:greedy_algo}, line 8). We repeat this process until all of the task requirements have been fulfilled (Alg. \ref{alg:greedy_algo}, line 5). We then update the task start time for the chosen task $k^c$ according to the attending coalition (Alg. \ref{alg:greedy_algo}, line 10).

% To assign more robots to a task if needed, we first consider the robot that can offer the highest number of remaining skills. If multiple robots are tied, we select the one that can reach the task location first. Algorithm \ref{alg:assign_task} is used to add the task to the robot's schedule. We repeat this process until all task requirements are met.

% In this method, we will consider a pool of robots $P$ to distribute the tasks evenly across all the robots. In each cycle, a random robot in the pool chooses a task and removes itself from the pool. When the pool runs out of robots, we reset it with all the robots, and the cycle starts again. Algirithm \ref{alg:greedy_algo} describes the methodology of this approach.

% A variable $S$ maintains the schedule of all the robots. The $m^{\text{th}}$ task in robot $i$'s schedule is given by $S_{im}$. We will use the notation $S_{i,-1}$ to refer to the last task assigned to the robot $i$, as per the latest schedule.

% At the start, we choose a random robot from the available pool. The robot chooses a task that it can contribute to the most. The contribution is the number of skills it can offer to the task. In case of a tie, the robot chooses a task it is estimated to reach earliest. The estimated time of arrival is given by the same logic as in the section \ref{subsec:arrival_times}, with the variables carrying the same meaning. The stochastic buffer time consideration is also the same as in section \ref{subsec:stoch_time} and is denoted by $t^s$. 

% Once the robot chooses a task $k^c$, it removes itself from the pool $P$ if it exists in the pool. Then the robot adds the task to its schedule and updates the time of arrival at that task (algorithm \ref{alg:assign_task})

% If the chosen task has more requirements than the robot can offer, the robot searches for a companion robot for the coalition.
% The robot chooses a companion that can offer the most number of remaining skills. Here we refer to the `remaining skills' as the skills that the task requires but the current coalition does not offer them. If multiple robots can offer the same number of remaining skills, then the robot that is estimated to arrive at the task earliest is chosen as a companion. To get an estimate of each robot's arrival times, we first refer to their last allocated task in the schedule $S$. Then, using the same logic as in section \ref{subsec:arrival_times}, we estimate how long it takes for the robot to arrive at the current task $k^c$.

% The earliest arriving robot is chosen for a coalition. We again use the algorithm \ref{alg:assign_task} to update this robot's schedule and the time of attending the task. This cycle of adding robots to the coalition continues until the coalition satisfies all the task requirements. And once they are satisfied, the task is removed from the task pool $O$










% %%%%%%%%%%%%%%%%%%%%%%%%%%%%%%%%%%%%%%%%%%%%
% %%%% ALGORITHM FOR COALITION/SCHEDULING %%%%
% %%%%%%%%%%%%%%%%%%%%%%%%%%%%%%%%%%%%%%%%%%%%

% \begin{algorithm}
% \begin{algorithmic}
% \State $P \gets$ all robots
% \State $O \gets$ all tasks
% \State $S$ holds all the robots' schedules
% % \State $y$ holds each robot's arrival times at tasks

% % \For{$i$ in all robots} $y_i$ = 0\EndFor
% % \State $\forall i \quad y_i = 0$

% \While{$O$ has tasks remaining}
% \If{$P$ is empty}
% \State $P \gets$ all robots
% \EndIf
% \State $i \gets$ random robot out of $P$

% \State $K \gets$ all tasks where $\underset{k}{\arg\max} (q_{i}.q_{k}^T$)
% \State $j \gets S_{i,-1}$
% % \State $y^{\max}_{j} \gets \underset{i_2}{\max} \ y_{i_2j}$
% \State $k^c \gets \underset{k \in K}{\arg\min} (y^{\max}_{j} + t^{e}_{j} + t_{ijk} + t^s_{jk})$
% \State Assign task $k^c$ to the robot $i$ using Algorithm \ref{alg:assign_task}

% % \State $y_{ik^c} \gets y^{\max}_{j} + t^{e}_{j} + t_{ijk^c} + t^s_{jk^c}$
% % \State $r \gets $ vector of all requirements of the task $j$

% \While{any element of $(r_{j}-q_{i})>0$}
% \State $r_{j} \gets (r_{j}-q_{i})$
% \State Set negative values in $r_{j}$ as 0
% \State $I \gets$ all robots where $\underset{i}{\arg\min} \Big(\sum^l_{s=0}\big((r_{j}-q_{i})>0\big)\Big)$

% \State $i_2 \gets \underset{i \in I}{\arg\min} (y^{\max}_{S_{i,-1}} + t^{e}_{S_{i,-1}} + t_{iS_{i,-1}k^c} + t^s_{S_{i,-1}k^c})$

% % \State $j \gets S_{i_2,-1}$
% % \State $y^{\max}_{j} \gets \underset{i \in C}{\max} \ y_{ij}$
% \State Assign task $k^c$ to the robot $i_2$ using Algorithm \ref{alg:assign_task}
% % \State $y_{i_2k^c} \gets y^{\max}_{S_{i_2,-1}} + t^{e}_{S_{i_2,-1}} + t_{i_2S_{i_2,-1}k^c} + t^s_{S_{i_2,-1}k^c}$
% \EndWhile


% \State remove $k^c$ from $O$
% \State $y^{\max}_{k^c} \gets \underset{i}{\max} \ y_{ik^c}$


% \EndWhile
% \end{algorithmic}
% \caption{A greedy algorithm}\label{alg:greedy_algo}
% \end{algorithm}


\subsubsection{Correctness of the algorithm}
We assert that our algorithm yields a feasible solution in which all tasks are allocated to suitable robots. To establish this claim, we demonstrate that the algorithm assigns each task to a set of appropriate robots. Suppose there is an unassigned task $k$ with unfulfilled requirements, which means the sum of its requirements is greater than 0. The solver must continue until this task is assigned a group of robots that can fulfil all of its requirements. Hence, eventually a robot will choose this task, even if it can only provide a single skill. Once the solver has found a robot for task $k$, it will search for other robots to fulfil any remaining requirements. A feasible solution requires at least one robot to contribute at least one skill to the remaining requirements. As long as such a robot exists, it will be assigned to task $k$. Moreover, the solver will not choose a robot that cannot contribute to the task as long as there is a robot that can contribute at least one skill. Therefore, there can be no redundant robot assigned to any task. We conclude that our algorithm always terminates with a feasible solution. Therefore, we can conclude that the proposed greedy algorithm for task allocation is correct and produces a feasible solution.
% section: results

% dev set decision table
% manually created from 25-official-html.tgz
% with custom editing

\begin{table}
\centering
\begin{tabular}{l|rr}
\toprule
 & \multicolumn{2}{c}{\textbf{ELAS F1}} \\
\textbf{Treebank} & \textbf{sem-frag}
 & \textbf{heuristic}\\
%\hline
\midrule
% e7 elmo udpf task ar padt	allennlp 090 dm lbert luxfb ar padt 20200424 080759
% e7 elmo udpf task ar padt	copy2e arcase mark rel
ar\_padt & \textbf{70.99} & 59.74 \\
% e5 elmo udpf task bg btb	allennlp 090 dm pbert u bg btb 20200312 003243	
% e7 elmo udpf task bg btb	copy2e encase mark rel	
bg\_btb & \textbf{88.09} & 86.19 \\
%\hline
\midrule
% 	e3 elmo udpf task cs cac	allennlp 090 dm pbert luxfb cs cac 20200424 002226
%	e5 elmo udpf task cs cac	copy2e arcase mark rel	
cs\_cac & \textbf{86.51} & 74.41 \\
% e3 elmo udpf task cs fictree	allennlp 090 dm pbert luxfb cs cac 20200424 002226
% e5 elmo udpf task cs fictree	copy2e arcase mark rel	77.37
cs\_fictree & \textbf{83.23} & 77.37  \\
% e3 elmo udpf task cs fictree	allennlp 090 dm pbert u cs cac 20200419 171603
% e3 elmo udpf task cs cac	copy2e arcase mark
cs\_pdt & \textbf{79.58} & 	71.19 \\
%\hline
\midrule
% e7 elmo udpf task en ewt + ud25 en gum + ud25 en lines + ud25 en partut	allennlp 090 dm lbert u en ewt 20200312 051351
% e7 elmo udpf task en ewt + ud25 en gum + ud25 en lines + ud25 en partut	copy2e encase mark cc rel
en\_ewt & \textbf{84.71} & 	82.86 \\
% e3 elmo udpf task et edt + task et ewt	allennlp 090 dm mbert u et 20200419 234001
% e3 elmo udpf task et edt + task et ewt	copy2e arcase mark
et\_edt & 62.74 & \textbf{69.35} \\
% e5 elmo udpf task fi tdt fasttext udpf task fi tdt	allennlp 090 dm lbert u fi tdt 20200420 050020
% e7 elmo udpf task fi tdt	copy2e arcase mark rel
fi\_tdt & \textbf{83.64} & 71.84 \\ 
% e5 elmo udpf task fr sequoia + ud25 fr gsd + ud25 fr partut + ud25 fr spoken	allennlp 090 dm mbert u fr sequoia 20200312 072651
% e3 elmo udpf task fr sequoia + ud25 fr gsd + ud25 fr partut + ud25 fr spoken	copy2e	
fr\_sequoia & \textbf{88.65} &  87.53 \\
% e3 elmo udpf task it isdt + ud25 it partut + ud25 it postwita + ud25 it twittiro + ud25 it vit	allennlp 090 dm lbert u it isdt 20200419 172143	
% e7 elmo udpf task it isdt	copy2e encase mark cc rel	
it\_isdt & \textbf{90.13} & 88.28 \\
% e3 plain udpf task lt alksnis	allennlp 090 dm mbert u lt alksnis 20200420 014618	
% e3 plain udpf task lt alksnis	copy2e arcase mark	
lt\_alksnis & \textbf{73.63} & 57.84 \\
% e7 elmo udpf task lv lvtb	allennlp 090 dm mbert luxfb lv lvtb 20200423 191418	
% e5 elmo udpf task lv lvtb fasttext udpf task lv lvtb	copy2e encase rel	
lv\_lvtb & \textbf{81.82} & 71.29 \\
%\hline
\midrule
% 	e7 elmo udpf task nl alpino + task nl lassysmall	allennlp 090 dm lbert u nl alpino 20200312 025649	
% e7 elmo udpf task nl alpino + task nl lassysmall	copy2e encase mark cc rel	
nl\_alpino & \textbf{89.93} & 89.00 \\
% e7 elmo udpf task nl alpino + task nl lassysmall	allennlp 090 dm lbert u nl alpino 20200312 025649	
% e7 elmo udpf task nl alpino + task nl lassysmall	copy2e encase mark cc rel	
nl\_lassysmall & 79.00 & \textbf{81.24} \\
%\hline
\midrule
% e5 elmo udpf task pl lfg + task pl pdb	allennlp 090 dm mbert luxfb pl lfg 20200423 222537	
% e5 elmo udpf task pl lfg + task pl pdb	copy2e encase mark rel	
pl\_lfg & \textbf{94.12} & 93.84 \\
% e3 fasttext udpf task pl lfg + task pl pdb	allennlp dev dm lbert luxf pl 20200416 194726	
% e5 elmo udpf task pl lfg + task pl pdb	copy2e arcase mark rel	
pl\_pdb & \textbf{82.25} & 78.27 \\
%\hline
\midrule
% e7 elmo udpf task ru syntagrus	allennlp 090 dm lbert lufb ru syntagrus 20200423 210055	
% e7 elmo udpf task ru syntagrus	copy2e arcase mark	
ru\_syntagrus & \textbf{88.48} & 80.03 \\
% e3 elmo udpf task sk snk plain udpf task sk snk	allennlp 090 dm mbert u sk snk 20200420 020636	
% e7 elmo udpf task sk snk	copy2e arcase mark	75.98
sk\_snk  & \textbf{81.30} & 75.98 \\
% e3 elmo udpf task sv talbanken	allennlp 090 dm lbert u sv talbanken 20200419 195336	
% e7 elmo udpf task sv talbanken	copy2e encase mark cc rel	
sv\_talbanken & \textbf{84.54} & 81.32 \\
% e3 plain udpf task ta ttb	allennlp 090 dm mbert u ta ttb 20200419 232103	
%	e3 plain udpf task ta ttb	copy2e arcase	
ta\_ttb & \textbf{55.68} & 43.94 \\
% e3 elmo udpf task uk iu	allennlp 090 dm mbert u uk iu 20200420 004219	
% e7 elmo udpf task uk iu	copy2e arcase mark	
uk\_iu & \textbf{82.41} & 76.88 \\
%\hline
\bottomrule
\end{tabular}
\caption{Development set ELAS F1 score %f-score
        for the best semantic parser evaluated without connecting
            fragmented graphs (sem-frag)
        and
        for the best combination of heuristic rules
            (heuristic)
}
\label{devresults:decision_custom}
\end{table}

% eof

Table~\ref{devresults:decision_custom} compares the semantic parser against the heuristic approach on the ELAS F1 metric.
The evaluation script was run without connecting fragmented graphs and format validation.
For all but two treebanks, the semantic parser performs better than the
best
heuristic approach.
For some languages, the difference in performance is large.
For \texttt{et\_ewt}, which does not have a development set,
we suspect that we overfitted our semantic parser on the
\texttt{et\_ewt} training data
by allowing it to train for 75 epochs.

% test set results table
% manually created from eval pages linked on
% https://quest.ms.mff.cuni.cz/sharedtask/cgi-bin/overview.pl

% main body generated by copy and pasting the qualitative tables,
% then using `cut -f1,16` to get the right columns, pasting them
% together with `paste` and tabs converted to `&` and \\ added to
% lines in `vim`

\begin{table}
\centering
\begin{tabular}{l|rrr}
\toprule
 & \multicolumn{3}{c}{\textbf{ELAS F1}} \\
\textbf{Treebank} & \textbf{subm}
 & \textbf{frag fix} & \textbf{re-run}\\
\midrule
Arabic-PADT         &  57.19  &  70.08  &  \bf 70.40  \\
Bulgarian-BTB       &  77.29  &  89.58  &  \bf 89.60  \\
Czech-FicTree       &  70.04  &  80.77  &  \bf 81.63  \\
Czech-CAC           &  71.72  &  86.00  &  \bf 86.38  \\
Czech-PDT           &  65.94  &  79.03  &  \bf 79.84  \\
Czech-PUD           &  64.34  &  77.37  &  \bf 78.08  \\
Dutch-Alpino        &  71.44  &  87.61  &  \bf 87.77  \\
Dutch-L.Small       &  64.03  &  77.39  &  \bf 77.24  \\
English-EWT         &  70.61  &  \bf 83.56  & \bf 83.56  \\
English-PUD         &  70.25  &  86.88  & \bf 87.03  \\
Estonian-EDT        &  62.29  &  68.20  &  \bf 68.37  \\
Estonian-EWT        &  55.70  &  \bf 61.19  &  60.67  \\
Finnish-TDT         &  73.02  &  \bf 84.36  &  84.33  \\
Finnish-PUD         &  71.58  & \bf 84.62  & \bf 84.62  \\
French-Sequoia      &  77.44  &  87.58  & \bf 88.60  \\
French-FQB          &  74.30  &  82.68  & \bf 83.26  \\
Italian-ISDT        &  71.98  &  \bf 90.24  &  90.23  \\
Latvian-LVTB        &  72.41  &  81.81  &  \bf 82.40  \\
Lithuanian-AL.      &  58.36  &  68.76  &  \bf 68.84  \\
Polish-LFG          &  61.23  &  \bf 70.89  &  70.71  \\
Polish-PDB          &  67.68  &  80.93  &  \bf 82.43  \\
Polish-PUD          &  65.64  &  79.77  & \bf 80.79  \\
Russian-SynT.       &  75.27  &  89.21  & \bf 89.47  \\
Slovak-SNK          &  68.43  &  81.63  &  \bf 81.97  \\
Swedish-Talb.       &  71.86  &  86.78  & \bf 86.72  \\
Swedish-PUD         &  64.70  &  79.35  & \bf 79.37  \\
Tamil-TTB           &  48.47  &  \bf 57.28  &  57.10  \\
Ukrainian-IU        &  66.43  &  79.81  & \bf 82.92  \\
\midrule
Average             &  67.49  &  79.76  & \bf 80.15  \\
\bottomrule
\end{tabular}
\caption{Test set results:
    subm = submitted,
    frag fix = using our own fragment connector and quick-fix.pl without connect-to-root,
    re-run = a re-run with bug fixes, no new models but new model selection
}
\label{testresults_custom}
\end{table}

% eof

Table~\ref{testresults_custom} shows test set ELAS obtained on the shared task
submission site for
\textit{(a)} our submission fully relying on the organiser's
             \texttt{quick-fix} tool to fix issues in the output of
             our system,
\textit{(b)} the same predictions post-processed by our own
             fragment connector that aims to minimise the
             number of root edges added, and
\textit{(c)} a re-run of our pipeline using the same models
             for system components as before but with all
             bugs fixed during development applied to all
             predictions and new decisions which models
             to apply to the test sets.
While the \texttt{quick-fix} tool enabled us to make a valid submission
in time, its
approach of adding edges from the root node to
all unreachable tokens
has a strong negative impact on 
precision, \eg 62.26 ELAS precision on the Czech CAC development set
\vs 87.37 without post-processing.
Our own post-competition fix avoids this
and would have brought us to the top half of the competition.

% eof
\section{Conclusions}
In this paper, we set out to address the problem of multi-tasking robots in multi-robot tasks. 
%A fundamental limitation of existing multi-robot systems was addressed by the removal of a restrictive assumption that was often made--robots are single-tasking.
%Our method allowed coalitions to overlap thus enabling multi-tasking robots. 
We observed that the key underlying challenge was to reason about the physical constraints that could be synergistically satisfied.
%which directly affected the feasibility of multi-tasking.
To address the challenge, we developed our method based on the information invariant theory and modeled constraints as information instances. 
%This allowed us to reason about the relationships between constraints by reasoning about those between information requirements. 
Thereby, a formal and general framework to achieve multi-tasking robots was developed. 
We showed that our algorithm was sound and complete under our problem settings. 
%Our method was integrated with a simple greedy heuristic for task allocation.
Simulation  results  were  provided  to  show  the  effectiveness  of  our approach under resource-constrained situations and in handling challenging situations. % in a multi-UAV simulator. 

% The idea of multi-tasking is attractive in many ways. 
% Humans are living in multi-tasking environments--at any point of time, 
% we are optimizing for more than one task. 
% Multi-task often leads to more efficient task performance since it allows us to exploit task synergies. 
% The work presented in this paper takes us one step forward in realizing multi-tasking robots. 
% In particular, we started looking at the feasibility of multi-tasking. 
% There are many potential directions to pursue along this direction. First, several limitations are present in the current approach. 
% For example, although our method guarantees that there exists a physical configuration that satisfies all the constraints, it does not explicitly take the environmental influence into account. For example, a narrow corridor may prevent a robot formation from passing through, even though all the constraints for the formation do not introduce any conflicts. In this sense, our work should better be characterized as establishing a necessary condition for multi-tasking. Also, our method is mainly focused on the ``{\it planning}'' phase and hence does not address how the robots reach the desired configuration and maintain the constraints. These issues are assumed to be handled by the execution layer.

% More generally, the question of how to execute the tasks with overlapping coalitions is not addressed in this work. 
% As we already discussed, executing individual tasks with non-overlapping coalitions is straightforward but task synergies impose additional requirements on the task execution: how should the robots that are assigned multiple tasks execute them? Should they consider them in a prioritized strategy~\cite{van2005prioritized}? Or should they combine the different tasks in a way that is similar to motor schemas~\cite{arkin2}. 
% Communication requirements for maintaining the constraints must also be taken into account. How should the robots optimize their communication to improve the task performance? 

% The stringency of the physical constraints is another interesting question. It may be desirable to relax the constraints in certain situations (e.g., due to environmental influences). In such cases, it may be important to consider the problem where the constraints are least violated~\cite{kim2012revision}, or specify task constraints in different ways to increase the diversity of the configurations~\cite{srivastava2007domain} so as to make it robust to different environments. 


\bibliographystyle{IEEEtran}
\bibliography{references}
\balance

% Some dummy text to

\end{document}
