\section{Related work}
\label{sec:2}

% To classify our work, we would first like to define it with the terminologies from taxonomy by Korsah et al. \cite{korsah2013taxonomy}. our work can be classified as 'Cross schedule dependencies (XD), Single task robots (ST), Multi-robot tasks (MR), Time-extended assignment (TA)': XD[MT-MR-TA]. 

% There exists a vast majority of literature in coalition formation papers. But a relatively small proportion of the work focuses on the heterogeneous swarm or scheduling the coalitions.

The problems of task scheduling and coalition formation have received wide attention in the literature. %As shown in Table \ref{tab:backg}, 
We identified several axes to categorize relevant work in Table \ref{tab:backg}.

% Several works study coalition formation without a scheduling component. Rahwan et al. \cite{rahwan2007coalitionAlgos} and Guo et al.
% \cite{guo2020repairs} focus on \textit{homogeneous} coalition formation. Rahwan \textit{et al.} \cite{rahwan2007coalitionAlgos} considers the \textit{static} case in which coalitions, once formed, are kept constant throughout the experiment. Guo \textit{et al.}
% \cite{guo2020repairs} compare the cases of static and \textit{dynamic} coalition formation, in which the robots are allowed to change coalition from task to task.  

Several works study coalition formation without scheduling. Rahwan \textit{et al.} \cite{rahwan2007coalitionAlgos} and Guo \textit{et al.} \cite{guo2020repairs} focus on \textit{homogeneous} coalition formation. Rahwan  \textit{et al.} \cite{rahwan2007coalitionAlgos} considers \textit{static} coalitions in which coalitions, once formed, are kept constant throughout the experiment. Guo \textit{et al.} \cite{guo2020repairs} compare \textit{static} and \textit{dynamic} coalition formation, in which the robots are allowed to change coalition from task to task.  

Recently, several researchers considered coalition formation of heterogeneously skilled robots  \cite{hu2022coverage,huang2020inner,lin2021deployment}. However, they focus on coverage and connectivity problems, neglecting scheduling aspects. Similarly, Barton \textit{et al.} \cite{barton2008coalitionSocial} study coalition formation in a network of \textit{heterogeneously skilled} agents. %% Ashay's comment: These works were in the table but were missing from the text. Hence added. We can delete them if too out of place
% In contrast, recent works study how to schedule heterogeneous robots across tasks without requiring coalition formation \cite{parker2016exploiting,visser2018integrating}.
% On the other hand, Parker \textit{et. al.} \cite{parker2016exploiting} and Visser \textit{et. al.} \cite{visser2018integrating} investigate search and rescue scenarios using a heterogeneous set of robots. Their research is centered on scheduling these robots to perform various tasks without the need for coalition formation.
At the opposite end of the spectrum, Parker \textit{et al.} \cite{parker2016exploiting} and Visser \textit{et al.} \cite{visser2018integrating}  study search-and-rescue scenarios with a heterogeneous set of robots and focus on scheduling them without coalition formation.

Significant effort has been devoted to combining coalition formation with task scheduling \cite{guerrero2017deadlines,ramchurn2010coalitionSpacialTemporal,koes2006cocoa,capezzuto2020coalitions,capezzuto2021schedulingCoalition}.
However, all of these works assume the robots to be homogeneously skilled.

% A popular competition in this context is \textit{RoboCup Rescue} \cite{kitano1999robocup} on disaster relief. The problem involves a heterogeneous set of agents (e.g., ambulance, fire trucks, police) working in a dynamic environment. Tasks constantly appear and disappear. Even though this problem considers robots with heterogeneous capabilities and their schedules, the coalitions are not necessarily required. However, the schedule of one agent does affect another.


% Plenty of variation of studies exists on coalition formation. Rahwan et al. \cite{rahwan2007coalitionAlgos} worked on coalition formation, but in their work, a robot can only be in one coalition throughout the experiment. i.e. the coalitions formed are static in their work. Whereas Guo et al.
% \cite{guo2020repairs} focused on comparing the performances of the existence of a single coalition and multiple coalitions throughout the task execution. 

% \cite{korsah2013taxonomy} and \cite{nunes2017taxonomy}
% \cite{parker2016exploiting} focuses on comparing the performances of team vs non-team and smart team allocation vs greedy allocation in RoboCup

% ==

% simple MRTA
Some of the closest research to this paper studies coalition formation with heterogeneous robots that also produces an optimal plan with cross-scheduling dependencies \cite{Leahy2022scalable,lippi2021HRI,korsah2012xbots,mansfield2021mrscheduling}. 
However, these works do not consider \textit{multi-skilled} robots; rather, each robot is a specialist of a specific skill.

Prorok \textit{et al.} \cite{prorok2017impact,prorok2016fastRedistribution} and Kosak \textit{et al.} \cite{kosak2018multipotent} address multi-skilled robots. In the work of Prorok \textit{et al.}, \cite{prorok2017impact,prorok2016fastRedistribution}, the robots offer a subset of the possible skills, whereas Kosak \textit{et al.} \cite{kosak2018multipotent} allow ``multipotent'' robots to modify themselves and adapt to the task at hand. However, all these works focus on coalition formation without a scheduling component. 

% Tkach \textit{et. al.} \cite{tkach2021towards} solve a similar problem of `Law enforcement problem' (\textit{LEP}) that allocate police officers to dynamic tasks whose locations, arrival times, and importance levels are unknown a priori. Though \textit{LEP} do consider the cross-scheduling dependencies for \textit{multi-skilled} agents, their problem statement is fundamentally different from ours. The \textit{LEP} tasks are time sensitive (have deadlines), they have differing importance, and are unknown a priori. Whereas, we do not consider the tasks to be time sensitive and all our tasks have equal importance. Furthermore, in \textit{LEP} the agents can abandon the tasks and can ignore the tasks if the agent does not get a higher utility our of the tasks. Whereas, in our problem setting the agents cannot abandon a task once allocated, and all the tasks must be finished by the agents. 

% Amador \textit{et. al.} \cite{amador2014dynamic, tkach2021towards} tackled a similar problem called the `Law enforcement problem' (\textit{LEP}) which involves allocating police officers to dynamic tasks with unknown locations, arrival times, and importance levels. While \textit{LEP} considers cross-scheduling dependencies for multi-skilled agents, their problem statement is fundamentally different from ours. Their approach considers time-sensitive tasks with varying importance levels and agents who can abandon or ignore tasks if they don't offer a higher utility. In contrast, our problem assumes that all tasks are of equal importance and are not time-sensitive. Furthermore, our agents cannot abandon any task once allocated and all the tasks must be completed by the agents.
Amador \textit{et. al.} \cite{amador2014dynamic, tkach2021towards} addressed a comparable issue, the `Law enforcement problem' (\textit{LEP}), which assigns police officers to tasks with unknown locations, arrival times, and importance levels. Although \textit{LEP} considers cross-scheduling dependencies for multi-skilled agents, their problem statement differs significantly from ours. Their approach concerns time-sensitive tasks with various importance levels and agents who can abandon or ignore tasks if they do not offer a higher utility. In contrast, our problem assumes all tasks are equally important and not time-sensitive. Moreover, our agents cannot abandon any tasks, and all tasks must be completed.

 We are also interested in the presence of stochastic aspects in the problem statement. The works considered so far lack such a component, but recent research has started to include it. Nam \textit{et al.}  \cite{nam2016optimalAssignment}, for example, include stochastic travel times in multi-agent task scheduling; however, their work does not require coalitions formation. In the literature on coalitions, stochastic aspects concern resilience and reconfiguration \cite{mayya2021resilient,ramachandran2019resilience,ramachandran2021resilient}, without a scheduling component. 


% How others have tried to solve it. ref papers. no points now.
% Explain work on:

% * Scheduling

% * Coalition formation

% * law enforcement problem

% * Deadlines/tema orienteering problems


% What we do is all this, plus more . 

\begin{table}[t]
\caption{Comparison of the related work to the present work. Legend: Coalt: Coalition, Heter: Heterogeneous set of robots, Sched: Scheduling, Stoch: Stochasticity of any nature, M-Skill: Multi-skilled robots.}
\begin{tabular}{cccccc}
 \toprule
 \textbf{Work} & \textbf{Coalt} & \textbf{Heter} & \textbf{Sched} & \textbf{Stoch} & \textbf{M-Skill} \\ 
\midrule
\cite{rahwan2007coalitionAlgos,guo2020repairs} & \checkmark & & & & \\
\cite{barton2008coalitionSocial,hu2022coverage,huang2020inner,lin2021deployment} & \checkmark & \checkmark &&&\\
\cite{guerrero2017deadlines,ramchurn2010coalitionSpacialTemporal,koes2006cocoa,capezzuto2020coalitions,capezzuto2021schedulingCoalition} & \checkmark & & \checkmark & & \\
\cite{parker2016exploiting,visser2018integrating} & & \checkmark & \checkmark & & \\
\cite{Leahy2022scalable,lippi2021HRI,korsah2012xbots,mansfield2021mrscheduling} & \checkmark & \checkmark & \checkmark & & \\
\cite{mayya2021resilient,ramachandran2019resilience,ramachandran2021resilient} & \checkmark & \checkmark & & \checkmark & \\
\cite{nam2016optimalAssignment} & & & \checkmark & \checkmark & \\
\cite{prorok2017impact,prorok2016fastRedistribution,kosak2018multipotent} & \checkmark & \checkmark & & & \checkmark\\
\cite{amador2014dynamic, tkach2021towards} & \checkmark & \checkmark & \checkmark & & \checkmark\\
\textit{this work} & \checkmark & \checkmark & \checkmark & \checkmark & \checkmark\\
%  \hline
% %  Not sure if this should be considered heterogeneous. The robots are homo
% \cite{kosak2018multipotent} & \checkmark & \checkmark & & & \\
 \bottomrule
\end{tabular}
\label{tab:backg}
\end{table}
% $\checkmark^{*}$ refers to static coalition or team orienteering

% To the best of our knowledge, no work simultaneously considers heterogeneous coalition and scheduling among multi-skilled robots.