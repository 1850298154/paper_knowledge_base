%%%%%%%%%%%%%%%%%%%%%%%%%%%%%%%%%%%%%%%%%%%%%%%%%%%%%%%%%%%%%%%%%%%%%%%%%%%%%%%%
%2345678901234567890123456789012345678901234567890123456789012345678901234567890
%        1         2         3         4         5         6         7         8

\documentclass[letterpaper, 10 pt, conference]{ieeeconf}  % Comment this line out if you need a4paper



%\documentclass[a4paper, 10pt, conference]{ieeeconf}      % Use this line for a4 paper

\IEEEoverridecommandlockouts                              % This command is only needed if 
% you want to use the \thanks command

\overrideIEEEmargins                                      % Needed to meet printer requirements.

%In case you encounter the following error:
%Error 1010 The PDF file may be corrupt (unable to open PDF file) OR
%Error 1000 An error occurred while parsing a contents stream. Unable to analyze the PDF file.
%This is a known problem with pdfLaTeX conversion filter. The file cannot be opened with acrobat reader
%Please use one of the alternatives below to circumvent this error by uncommenting one or the other
%\pdfobjcompresslevel=0
%\pdfminorversion=4

% See the \addtolength command later in the file to balance the column lengths
% on the last page of the document

% The following packages can be found on http:\\www.ctan.org
%\usepackage{graphics} % for pdf, bitmapped graphics files
%\usepackage{epsfig} % for postscript graphics files
%\usepackage{mathptmx} % assumes new font selection scheme installed
%\usepackage{times} % assumes new font selection scheme installed
%\usepackage{amsmath} % assumes amsmath package installed
%\usepackage{amssymb}  % assumes amsmath package installed


\usepackage{color} % Usually already available

\title{\LARGE \bf
	A Neuro-inspired Theory of Joint Human-Swarm Interaction
}




\author{Jonas D. Hasbach$^{1,2}$ and Maren Bennewitz$^{2}$% <-this % stops a space
%	\thanks{*This work was not supported by any organization}% <-this % stops a space
	\thanks{\textcolor{red}{\textbf{This short paper was accepted and presented at the Workshop on Human-Swarm Interaction at IEEE ICRA 2020.}}}
	\thanks{$^{1}$Jonas D. Hasbach is with the Department of Human-Machine-Systems, Fraunhofer FKIE,
		Wachtberg, Germany.
		{\tt\small jonas.hasbach@fkie.fraunhofer.de}}%
	\thanks{$^{2}$Jonas D. Hasbach and Maren Bennewitz are with the Humanoid Robots Lab, Computer Science,
		University of Bonn, Germany.
		{\tt\small maren@cs.uni-bonn.de}}%
}


\begin{document}
	


	
	\maketitle
	\thispagestyle{empty}
	\pagestyle{empty}
	

	
	
	\begin{abstract}
		Human-swarm interaction (HSI) is an active research challenge in the realms of swarm robotics and human-factors engineering. Here we apply a cognitive systems engineering perspective and introduce a neuro-inspired joint-systems theory of HSI. The mindset defines predictions for adaptive, robust and scalable HSI dynamics and therefore has the potential to inform human-swarm loop design.
	\end{abstract}
	

\section{Motivation}
%SR HSI


For the real world application of swarm robotics, human operators are required to be part of the system loop. Reasons are (1) the swarm's inability to achieve mission goals independently \cite{Conant1970a}, (2) human out of loop phenomena \cite{Bainbridge1983} as well as (3) legal and ethical concerns \cite{Verbruggen2019}. 
%the allocation of tasks to humans because of ethical concerns \cite{Verbruggen2019}. 

The objective of human-swarm interaction (HSI) \cite{Kolling2016} is to combine the distributed nature of robot swarms with the centralized control and feedback demands of humans into one loop \cite{Barca2013}. For example, swarm robots may interact locally with the operator in a fire fighting scenario \cite{Penders2011}.

% a human operator can control a subset of robots, which then influence the behaviour of the whole robotic swarm via local interaction rules \cite{Goodrich2012, Kolling2016}. \alert{pSHI}


%pHSI and system theory
HSI requires holistic theories about human-swarm loops that can be used to inform design \cite{Woods2006a}. Therefore, here we formulate a holistic neuro-inspired theory of how to best combine human- and swarm properties into one joint human-swarm loop. Testable predictions are deducted which will allow for the adjustment of the theory by empirical probing. 


\section{Joint Human-Swarm Loop} %dynamics

%CSE
Rather than focusing on the interaction between operator and machine, cognitive systems engineering applies a cybernetic perspective which focuses on how the operator and machine can jointly accomplish system goals \cite{Hollnagel2005}.

%system goals (see Hussein et al. \cite{Hussein2018} and Harriott et al. \cite{Harriott2014} for performance measurements in HSI)
The system goals, and therefore the desired human-swarm-loop implementation, depend on the mission scenario. However, there are three system properties of HSI that seem desirable independent of mission goals; adaptation, robustness and scalability. We selected these variables guided by system theorist Beer's rule: \textit{'The purpose of a system is what it does'} \cite[p. 99]{Beer1985}. While adaptation is important for human-machine loops in general \cite{Woods2006a,Woods2015}, robustness and scalability have been defined as desired swarm properties \cite{Kolling2016,Barca2013}. %(although distributed perspectives also surface in human factors \alert{team SA})



%adaptivity, robustness, scalability and resilience %\textbf{Robustness} is sometimes seen as equivalent to resilience \cite{Woods2015}. We prefer the definition of 

\textbf{Adaptivity} is a system property that holds critical system variables in an acceptable range over different situations \cite{Ashby1960}, i.e. adaptive systems cope with surprises \cite{Woods2006a, Woods2015}. Compared to swarms, human operators are capable of much greater flexible dynamics with which they can adapt to uncertainties during missions. In contrast, \textbf{robustness} is the ability of a system to cope with demands that are expected \cite{Woods2015} (e.g. continue the mission after some swarm robots are lost). A system is \textbf{scalable} if it is capable of adjusting the number of network nodes during deployment, such as two separated bird flocks forming interactions and becoming one unity. 

%In sum, the overall objective of HSI may be formulated as identifying the combination of design solutions that allow for the emergence of sustained adaptive, robust and scalable goal-oriented performance patterns over a range of situations.

Now the \textbf{challenge of HSI} (CoHSI) can be defined as \textit{joining the centralized nature of the operator with the distributed nature of the swarm \cite{Barca2013} into one goal-directed system while promoting the emergence of adaptive, robust and scalable behaviour}.


\section{Neuro-inspiration}

As humans are biological systems and robotic swarms are often bio-inspired, it seems appealing that HSI could benefit from bio-inspiration as well. Similarities between neural- and swarm principles have been discussed under the label 'swarm cognition' \cite{Trianni2011a}. In the following, we approach the CoHSI from a neuro-inspired angle.

%sah %The bias signals introduced into the CPGs are transmitted via local interactions between the neurons, without the neurons possessing global information.
%In addition, reflexes in the biological nervous system lead to a fast reaction to environmental stimuli without cognitive participation.

\subsection{Swarm-Amplified Humans} In the biological nervous system, stereotypical locomotion is generated by low-level ('front end') \textbf{central pattern generators} (CPG) \cite{EveMarder2001, Ijspeert2008}. While higher-order ('back end') cognition signals may modulate the activity of the lower-order CPGs, CPG function is also influenced by sensory feedback. Thus, at least some neural circuits of the sensory-motor loop seem to be capable of producing stereotypical behaviour while sensitive to higher-order and environmental modulatory signals. Intriguingly, this state of affairs may be mapped onto the CoHSI, because a cognition or operator signal modulates a distributed and semi-autonomous system (CPG or swarm). We therefore investigate transferring neurocomputational principles to the design of HSI.

%role of operator and swarm
In this neuro-inspired theory that we call the \textbf{swarm-amplified human}, the swarm is seen as an extension of the human nervous system. From an operator perspective, the swarm becomes an extension of the human body \cite{Hollnagel2005,Miller2018}; an artificial body part. This shifts the focus from the design of the human-swarm channel to the interaction between human and environment which is \textit{interfaced by the swarm} \cite{LeGoc2016}. From a swarm perspective, swarm robots form behavioural subgroups ('body parts') determined by their respective local environments and human modulation while the latter is conveyed by a distributed neural pathway overlay (i.e. a flexible hop network). The research question in the context of swarm-amplified humans is: \textit{'How can the operator interact with the environment through the swarm?'}.

 

%From a swarm engineering perspective, a neural network overlay (hop network) may implement human-environment interaction by a distributed mechanism.

%adaptivity, robustness and scalable
This perspective joins human and swarm into an adaptive, robust and scalable human-swarm loop. It highlights system adaptivity by giving the operator high-level control over his swarm 'body', as the human component is the more capable adaptive subsystem. Similar to CPGs, the swarm is capable of reacting to the environment in a closed-loop (i.e. sort out some mission goals autonomously). The swarm therefore is biased by the human component but not fully centralized to it. Robustness and scalability is preserved by the decentralized nature of the neuronal overlay.

%example
For example, human states (e.g. cognitive states such as vigilance) may be injected into the swarm network where the human bias signal interacts with local environmental classifications of robots (e.g. heat classification). In a fire fighting scenario, a combination of $ humanClassifier_{vigilance}=high $ and $ robotSensor_{heat}=high $  could bias parts of the swarm network to flip their observable behaviour to 'danger - get out!' \cite{Penders2011,Payton2005}. This may be similar to switching from walking to running in a reactive flight response that may be an interplay of cognitive and sensory-motor computation. Note that the swarm requires no deliberate operator control and serves as a sensory organ; it is a semi-autonomous part of the human-environment loop.

%the operator does not have to deliberately control the swarm while the later serves as a sensory organ; the swarm is seen as a functional part of the human body.

\subsection{Deduced Hypotheses}

We shortly list four deduced hypotheses about how to bring together neural- and swarm principles in the interdependent human-swarm loop components. Elaboration will be given in later work due to space restrictions.

\subsubsection{Human Output - Comprehensible Passive Interaction}
It is predicted that comprehensible passive interaction is a beneficial design selection, as body coordination seems controllable and often automatic to the conscious mind. See e.g. the sense of agency \cite{Braun2018} and human state classifications \cite{Penders2011,Karavas2017}.

%How to get the control signal out from the operator? From a human perspective, a lot of body coordination seems automatic to the conscious mind. For example, when walking, one often does not think about walking. Instead, one might think about going to a particular place and the nervous system generates the locomotion without concious effort (not least by the curtsy of the CPMs). When treating the swarm as an artificial body part, the swarm should be able to adapt to the state of the human without requiring him to spend mental capacity to control the swarm, in order to induce the feeling of unconscious automatism. Note that explicit (i.e. deliberate, concious) control is still possible by changing into the human state which is known to result into intended swarm or environmental states. \textit{It is predicted that passive interaction is an effective design selection.} \alert{+guardians}

\subsubsection{Swarm Input - Flexible and Distributed Signal Propagation}
It is predicted that flexible and distributed signal propagation by a neural overlay (hop network) while minimizing transmitted information for robustness and scalability is a beneficial design selection. See e.g. Hebb's rule \cite{Hebb1949} that increases neural connections by a positive feedback mechanism conceptually similar to collective path selection in ants \cite{Deneubourg1990}. %for dynamic routing

% neural convergence \alert{REF}.
%\subsection{Swarm Input}
%How to inject the control signal into the swarm? To not decrease swarm robustness and scalability, the neural overlay must preserve distributed computation and signal transmission. Importantly, no global information about the swarm network architecture is available. The challenge therefore becomes one of signal distribution in an unknown and dynamic network, while absolutely minimal bandwidth is used (i.e. no GPS or agent identifiers). Consensus about bias signals then may be inferred by statistical principles, i.e. by increasing the the stability of the human bias signals so that ultimately the swarm network flips into a new attractor mode \alert{ref}. Work on this component may investigate how neurocomputational principles, like Hebbs Rule (neurons that fire together wire together) may be used in robust signal distribution. For now, it is predicted that human bias signal distribution by the swarm hop network itself while minimizing transmitted information for robustness and scalability is an effective design selection.

\subsubsection{Swarm Output - Inferring Global Information by Local Interactions}
It is predicted that inferred global information about the swarm and environment by local human-robot- or hub-robot interactions for preserving distributive coding is a beneficial design selection. See e.g. the virtual pheromone \cite{Payton2005} where robots summarize non-local information up the signal path conceptually similar to neural convergence \cite{Hubel1962}.  %that shares similarities to neural convergence.

\subsubsection{Human Input - Correlating Multi-modal Stimulation}
It is predicted that spatially and temporally correlating multi-sensory operator stimulation is a beneficial design selection, as body parts feature correlated sensory inputs. See e.g. work on tool- \cite{Miller2018} and body part integration \cite{Braun2018} as well as a multi-modal interface \cite{Penders2011}.



\section{Conclusion}

Future empirical work must explore the benefits of transferring neurocomputational principles to HSI. In any case, HSI should be seen as joining together human- and swarm capabilities in the context of system purpose. 



% -> pHSI MAYBE INTEGRATE PHSI HERE
%However, there are relevant scenarios which require a human operator to be deployed with the swarm on the field. For example, an army soldier deployed on the battlefield or a rescue worker on a search and rescue mission. In the future, one even may look forward to a hunter who uses swarm robots to track game or an astronaut who is surrounded by a swarm that protect him from projectiles. Such Proximal Human-Swarm Interaction (pHSI), in which the human has a direct line-of-sight relationship to at least one swarm robot, requires specialized interaction approaches because the operator is situated in a complex and noisy environment where he moves freely.

%\bibliographystyle{ieeetran} 
\bibliographystyle{IEEEtran} 
\bibliography{IEEEabrv,lib}


\end{document}