\section{Conclusions}\label{conc}
Motivated by the task allocation challenges in the edge/hub/cloud continuum for applications requiring an edge device, a hub device, and a cloud server, we proposed a comprehensive BILP formulation, encapsulated in an application-driven design-time framework, to optimally and efficiently address the task allocation problem in the particular architecture.
Our approach transforms the task flow graph of an application into an extended form to facilitate the formulation of the examined problem. 
It supports two distinct optimization objectives, the minimization of either latency or energy, while incorporating parameters and constraints often ignored in related studies, such as the computational and communication latency and energy required for the processing of the tasks, as well as the memory, storage, and energy limitations of the devices.
We validated our framework using a real-world application under varied device configurations and communication channel characteristics. We further evaluated its scalability to applications of different structures and sizes utilizing appropriate synthetic benchmarks we developed for this purpose.
Through extensive experimentation, we demonstrated that the proposed method not only yields optimal and scalable results, but it also enables efficient design space exploration with respect to different devices and their connectivity.

\vspace{-10pt}
\section*{Data availability}
\vspace{-5pt}
The datasets of our synthetically generated task flow graphs are publicly available (open-access) at https://doi.org/10.5281/zenodo.10654551.

\vspace{-10pt}
\section*{Acknowledgments}
\vspace{-5pt}
This work has been supported by the European Union’s Horizon 2020 research and innovation programme under grant agreement No. 739551 (KIOS CoE) and from the Government of the Republic of Cyprus through the Cyprus Deputy Ministry of Research, Innovation and Digital Policy.

\vspace{-10pt}
\section*{Appendix A. Supplementary data}
\label{sec:appendix}
\vspace{-5pt}
Supplementary material related to this article can be found online at https://doi.org/10.1016/j.future.2024.02.005.
