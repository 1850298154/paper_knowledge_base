\section{Preliminaries}
\label{preliminaries}
\vspace{-5pt}
In this section we briefly present the Hadoop architecture and some of the basic of the CSP language and the tool PAT, which will be used in the rest of the paper. This is in order to better understand the different steps of our proposed methodology. 
\vspace{-5pt}

\subsection{Hadoop Architecture}
Hadoop~\cite{SOUALHIA2017} is an open source implementation of the MapReduce programming model~\cite{WOHA2014}. MapReduce is designed to perform parallel processing of large datasets using a large number of computers. A MapReduce job is comprised of two functions: a map and reduce, and the input data~\cite{SOUALHIA2017}.
Hadoop has become the \textit{de facto} standard for processing large data
in today's cloud environment. It is a master-slave-based framework, the master node consists of a JobTracker and NameNode. The worker/slave node consists of a TaskTracker and DataNode. The Hadoop Distributed File System (HDFS) is the storage unit responsible for controlling access to data in Hadoop.
Hadoop is equipped with a centerpiece; the scheduler which distributes tasks across worker nodes according to their availability. The default scheduler in Hadoop is based on the First In First Out (FIFO) principle.
Facebook and Yahoo! have developed two new schedulers for Hadoop: Fair scheduler and Capacity scheduler, respectively~\cite{SOUALHIA2017}.


\subsection{CSP and PAT}
CSP is a formal language used to model and analyze the behavior of processes in concurrent systems. It has been practically applied in modeling several real time systems and protocols~\cite{CSP}.   
In the sequel, we present a subset of the CSP language, which will be used in this work, where \textit{P} and \textit{Q} are processes, \textit{a} is an event, \textit{c} is a channel, and \textit{e} and \textit{x} are values:\\
\texttt{\normalsize{
P , Q ::= Stop $\mid$ Skip $\mid$ a $\rightarrow$ P $\mid$ P ; Q $\mid$ P $\mid\mid$ Q $\mid$ c!e $\rightarrow$ P $\mid$ c?x $\rightarrow$ P
}} 
\begin{itemize}
\item \texttt{Stop}: indicates that a process is in the state of deadlock.
\item \texttt{Skip}: indicates a successfully terminated process.
\item \texttt{a $\rightarrow$ P}: means that an object first engages in the event \textit{a} and then behaves exactly as described by \textit{P}.
\item \texttt{P ; Q}: denotes that \textit{P} and \textit{Q} are sequentially executed.
\item \texttt{P $\mid\mid$ Q}: denotes that \textit{P} and \textit{Q} are processed in parallel. The two processes are synchronized with the same communication events.
\item \texttt{c!e $\rightarrow$ P}: indicates that a value \textit{e} was sent through a channel \textit{c} and then a process \textit{P}.
\item \texttt{c?x $\rightarrow$ P}: indicates a value was received through a channel \textit{c} and stored in a variable \textit{x} and then a process \textit{P}.
\end{itemize}
\noindent PAT~\cite{PAT} is a CSP-based tool used to simulate and verify concurrent, real-time systems, etc.~\cite{Sun2009}. It implements different model checking techniques for system analysis and properties verification in distributed systems (\eg{} deadlock-freeness, reachability, etc.). Different advanced optimizations techniques, such as partial order reduction, symmetry reduction, etc., are available in PAT to reduce the number of explored states and CPU time. 



 

