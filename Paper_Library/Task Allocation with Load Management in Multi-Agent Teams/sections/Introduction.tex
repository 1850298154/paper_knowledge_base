Multi-agent systems have significant potential in generating management-level strategies for real-world large-scale operations such as humanitarian relief, search and rescue, transportation, warehouse management, and various other scenarios that involve teaming of heterogeneous agents or human-autonomy collaboration. In a team of collaborative agents, instead of being managed by a centralized system, decentralized agents have equal authorities in task planning and execution and ensuring operation effectiveness, with better team resilience in unexpected situations and scalability. With the help of emerging computing systems, recent advances in multi-agent task allocation \cite{MATAreview,MATA2,MATA3} have deliberated on distributed systems and deep reinforcement learning methods \cite{DEC-POMDP,HOOSHANGI2017,DQN,omidshafiei2017deep,HT-DEC-POMDP} for complex scenarios in addition to centralized approaches \cite{Nair2013,emam2020adaptive,QMDPNet}. Most coordination and collaboration strategies in efficient task allocation \cite{MATARL} define the objective as accomplishing tasks \cite{HT-DEC-POMDP} or reaching task locations \cite{omidshafiei2017deep, QMDPNet} in fewer steps. Although these approaches could sufficiently accomplish the problem objective, there is a lack of consideration of risk management since unanticipated situations might require additional strategic amendments in real-world operations. Agents in a team should be intelligent enough to prepare for the unknown.

To mitigate such risks and potentially allow agents to be available for unexpected events, some agents should idle. Being constantly active would overload the agents, would be especially harmful to human agents, and would cost extra resources. In the literature, load is handled as management or load allocation problems \cite{ResourceManage,ResourceAllocation,ResAlloMA} in the field of multi-agent computing. In human-robot collaboration, load usually refers to the workload measured by human heart-rate variability \cite{MAWorkload}, serving as a feedback to reduce stress and improve overall team performance \cite{AdaptiveWLA}. In the context of multi-agent task allocation in dynamic environments, the idling time is a performance measure to indicate task load balance \cite{AuctionTAforMR}, a precondition/constraint to perform other tasks \cite{Noureddine2017MultiagentDR}, or a time indicator for inoccupation \cite{idleinoccu}.
%Reviewer1Major1

In addition, optimization-based methods for load reduction have been studied as centralized strategies in the field of production scheduling \cite{prodSche}, energy management \cite{energyManage}, and cloud computing \cite{cloudComp}.
Limited research has explored the effect of idling as an option and to investigate the effects of load reduction on performance for decentralized multi-agent teams. Therefore, this work considers idling behaviors as an incentive to reduce task load and save resources represented by agent capabilities and task-reassignment.

In this paper, we leverage our previous work \cite{HT-DEC-POMDP} on task allocation for heterogeneous multi-agent teams to model task load management based on agent preference in decision-making as a secondary objective along with maximizing team performance.
%Reviewer1Major6
Multi-agent teams with a reduced load respond more reliably in emergencies that require extra resources, or in situations where one of the team members is compromised \cite{resilientTA}. An optimal idling strategy for agents may have negligible impact on team effectiveness, but makes a team more resilient in unexpected situations. We identify idling strategies by quantifying the importance of each agent in a team.
The proposed work also accounts for individual agent preferences in task assignment. The analysis on the impact of load reduction and the importance of collaboration highlights the benefits of the proposed decision-making framework.

The contributions of this work are:
\begin{itemize}
    \item Efficient learning of load reduction in multi-agent teams with heterogeneous capabilities without compromising team performance,
    \item Embedding agent preferences in decision-making by customizing individual reward functions,
    \item Inferring team resilience by characterizing the importance of each agent on team performance.
\end{itemize}
