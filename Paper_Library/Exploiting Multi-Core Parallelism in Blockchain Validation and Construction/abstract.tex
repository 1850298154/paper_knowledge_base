\begin{abstract}
Blockchain validators can reduce block processing time by exploiting multi-core CPUs, but deterministic execution must preserve a given total order while respecting transaction conflicts and per-block runtime limits. This paper systematically examines how validators can exploit multi-core parallelism during both block construction and execution without violating blockchain semantics.  
We formalize two validator-side optimization problems: (i) executing an already ordered block on \(p\) cores to minimize makespan while ensuring equivalence to sequential execution; and (ii) selecting and scheduling a subset of mempool transactions under a runtime limit \(B\) to maximize validator reward. For both, we develop exact Mixed-Integer Linear Programming (MILP) formulations that capture conflict, order, and capacity constraints, and propose fast deterministic heuristics that scale to realistic workloads. 

Using Ethereum mainnet traces and including a Solana-inspired declared-access baseline (Sol) for ordered-block scheduling and a simple reward-greedy baseline (RG) for block construction, we empirically quantify the trade-offs between optimality and runtime. MILPs quickly become intractable as heterogeneity or core count increases, whereas our heuristics run in milliseconds and achieve near-optimal quality. For ordered-block execution, heuristic makespans are typically within a few percent of the MILP solutions (and can even surpass the MILP incumbent when the solver times out), yielding up to $1.5$ speedup with $p=2$ and $2.3$ speedup with $p=8$ over sequential execution, despite tight ordering constraints. For block construction, the heuristic achieves $99$--$100\%$ of the MILP optimum reward on homogeneous workloads, and $74$--$100\%$ of an LP-relaxation upper bound on heterogeneous workloads, where exact optimization often times out. The resulting block-construction throughput scales close to linearly with $p$, reaching up to $7.9$ speedup with $p=8$ in our experiments. These results demonstrate that lightweight, conflict-aware scheduling and selection can unlock substantial parallelism in blockchain validation, bridging the gap between sequential execution and the true potential of multi-core hardware.
\end{abstract}
