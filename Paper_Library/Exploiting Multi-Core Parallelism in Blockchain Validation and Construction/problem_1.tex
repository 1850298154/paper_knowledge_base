\section{Ordered-Block Scheduling (OBS)} \label{sec:problem1}

In this section we propose algorithms to solve the Ordered-Block Scheduling (OBS) problem. All solutions start by creating a dependency graph. Given a block containing an ordered sequence of transactions $T=(\tau_1, \ldots, \tau_n)$, we construct a \textbf{directed dependency graph} $G=(T,E)$, where each node represents a transaction and each directed edge $(i,j) \in E$ indicates that transaction $\tau_i$ must precede $\tau_j$. An edge exists if (i) the two transactions conflict (as defined in \Cref{subsubsec:conflicts}), and (ii) $i<j$, ensuring consistency with the block’s total order. This graph captures all read/write dependencies that restrict concurrent execution. 
%Due to space constraints, we defer the MILP formulation for OBS and its discussion to~\Cref{sec:OBS_MILP}.
 
 %Create a directed conflict graph $G=(T,E)$, so that the nodes are the transactions in $T$, and there is a link from $\tau_i$ to $\tau_j$ (i.e., $(i,j) \in E$) if (i) $\tau_i$ and $\tau_j$ have a conflict (explained in Section \ref{subsubsec:conflicts}) and (ii) $i<j$.

\subsection{OBS MILP Formulations}
%\section{Ordered Block Scheduling (OBS) - MILPs} 
\label{sec:OBS_MILP}

\begin{figure}[t!]
\centering
\begin{tcolorbox}[enhanced, arc=1mm, boxrule =0.8pt, colback=white, colframe=black, width=\textwidth, left =1pt, right=6pt,  ams align,fontupper=\footnotesize,]
%\begin{align}
\min \quad & \sum_{r \in [R]} y_{r} 
    && && \text{(Minimize the number of rounds used)} \label{obj:assign} \\[0.5ex]
\text{s.t.}\quad 
& \sum_{r \in [R]} x_{ir} = 1 
    && \forall i \in [n] 
    && \text{(Schedule each transaction once)} \label{con:onetrans} \\
%& \sum_{i \in [n]} x_{ir} \leq p 
%&& \forall r \in [R] 
&\sum_{i\in[n]} x_{ir} \leq p \cdot y_r 
    && \forall r \in [R]
    && \text{(At most $p$ transactions per $r$ used)} \label{con:p-cores} \\
% & x_{ir} + x_{jr'} \leq 1 
%     && \forall (i,j) \in E,\; \forall r' \leq r 
%     && \text{(Precedence order)} \label{con:order} \\
& \sum_{t=1}^r x_{jt} - \sum_{t=1}^{r-1} x_{it} \leq 0
    && \forall (i,j) \in E, \forall r \in [R]
    && \text{($j$ cannot be scheduled before $i$)} \label{con:order} \\
% & y_r \geq x_{ir} 
%     && \forall i \in [n],\; \forall r \in [R] 
%     && \text{(Round $r$ usage)} \label{con:used-rounds} \\
% & y_r \in [0,1] 
%     && \forall r \in [R] 
%     && \text{(Linear assignment)} \label{con:linear-assign} \\
& y_r \geq y_{r+1} &&  \forall r \in [1 .. R-1 ]
    && \text{(Consecutive round usage)} \label{con:no-gaps}\\
& x_{ir},y_r \in \{0,1\} 
    && \forall i \in [n],\; \forall r \in [R] 
    && \text{(Binary assignment)} \label{con:binary-assign} %\end{align}
\end{tcolorbox}
\captionsetup{labelformat=simple,labelsep=colon,name={MILP}, justification=centering}
\caption{Ordered-Block Scheduling - Homogeneous Transactions}
\label{milp-pbm1-simple}
\end{figure}

\begin{figure}[t!]
\centering
\begin{tcolorbox}[enhanced, arc=1mm, boxrule =0.8pt, colback=white, colframe=black, width=\textwidth, left =1pt, right=6pt,  ams align,fontupper=\footnotesize,]
\min \quad & M 
    && && \text{(Minimize makespan)} \label{obj:min-mspan-c1} \\[0.5ex]
\text{s.t.}\quad
& \sum_{c \in [p]} x_{ic} = 1 
    && \forall i \in [n] 
    && \text{(Each trans.~in 1 core)} \label{con:onetrans-c1} \\
& s_i \geq 0 
    && \forall i \in [n] 
    && \text{(Non-negative start)} \label{con:noneg-c1} \\
& s_j \geq e_i 
    && \forall (i,j) \in E 
    && \text{(Precedence constr.)} \label{con:order-c1} \\
& e_i = s_i + t_i 
    && \forall i \in [n] 
    && \text{(End time Definition)} \label{con:runtime-c1} \\
& M \geq e_i 
    && \forall i \in [n] 
    && \text{(Makespan is max end)} 
    \label{con:mspan-c1} \\
&w_{ijc} \leq x_{ic} 
    &&  \forall (i,j) \in \mathcal{P}_{\text{non}} , \forall c \in [p]\\
&w_{ijc} \leq x_{jc} 
    &&  \forall (i,j) \in \mathcal{P}_{\text{non}} , \forall c \in [p]\\
&w_{ijc} \geq x_{ic} + x_{jc} -1 
    &&  \forall (i,j) \in \mathcal{P}_{\text{non}}, \forall c \in [p]\\
&z_{ij} =\sum_{c \in [p]} w_{ijc} 
    &&  \forall (i,j) \in \mathcal{P}_{\text{non}}
    && \text{(Same core flag)}\\
&e_i \le s_j + B \bigl( (1 - y_{ij}) + (1-z_{ij}) \bigr), 
    && \forall (i,j) \in \mathcal{P}_{\text{non}}\label{con:order1}\\
&e_j \le s_i + B \bigl( y_{ij} + (1-z_{ij}) \bigr), 
    && \forall (i,j) \in \mathcal{P}_{\text{non}}\label{con:order2}  \\
& M,\, s_i,\, e_i \in \mathbb{N} 
    && \forall i \in [n] 
    && \text{(Continuous variables)}  \label{con:linear-assign-c1} \\
& x_{ic} \in \{0,1\} 
    && \forall i \in [n],\; \forall c \in [p] 
    && \text{(Binary assignment)} \label{con:binary-assign-c1} \\
&w_{ijc} \in \{0,1\} 
    && \forall (i,j) \in \mathcal{P}_{\text{non}}, \forall c \in [p]\\
&y_{ij},z_{ij} \in \{0,1\} 
    && \forall (i,j) \in \mathcal{P}_{\text{non}}
\end{tcolorbox}
\captionsetup{labelformat=simple,labelsep=colon,name={MILP}}
\caption{Ordered-Block Scheduling - Homogeneous Transactions. Let $\mathcal{P}_{\text{non}}:=\{ (i,j) : i<j \land (i,j) \notin E \land (j,i) \notin E \}$ be the set of pairs of transactions without conflicts.}
\label{milp-pbm1-simple&complex}
\end{figure}

\subsubsection{Homogeneous Transactions}

We first consider the homogeneous case in which all the transactions have identical execution time. Since each transaction takes the same amount of time $\executiontime{\tau}$, we discretize time into rounds $R$ of uniform length. MILP \ref{milp-pbm1-simple} defines the corresponding MILP formulation. Here, $R$ denotes the maximum number of rounds, which must be an upper bound of the rounds required and has to be provided. 

%\smallskip
\textbf{Correctness.}
In the solution of MILP \ref{milp-pbm1-simple}, $x_{ir}=1$ iff $\tau_i$ is executed in round $r$; and $y_r$ indicates whether the round $r$ is used by any transaction. The objective is to minimize the number of rounds used, minimizing total execution time (makespan).

Constraint~\eqref{con:p-cores} enforces the per-round parallelism limit. 
Since the binary variable $x_{ir}$ indicates that transaction~$\tau_i$ executes in round $r$,
the sum $\sum_i x_{ir}$ counts the number of transactions assigned to that round. 
At most $p$ cores are available, and hence we require $\sum_i x_{ir} \le p \cdot y_r$, where $y_r = 1$ only if round~$r$ is actually used. 
Therefore, no round can host more than $p$ concurrent transactions, and unused rounds ($y_r = 0$) cannot schedule any work.
Observe that in this case, the solution does not need to assign transactions to specific cores.

Constraint~\eqref{con:order} enforces precedence. 
For every directed edge $(i,j) \in E$, transaction~$\tau_i$ must complete before~$\tau_j$ begins. 
The term $\sum_{t=1}^r x_{jt}$ equals~1 if $\tau_j$ is scheduled in a round $\le r$, while $\sum_{t=1}^{r-1} x_{it}$ equals~1 if $\tau_i$ is scheduled strictly before round~$r$. 
The inequality
\[
\sum_{t=1}^r x_{jt} - \sum_{t=1}^{r-1} x_{it} \le 0, \quad \forall r,
\]
therefore excludes any assignment in which $\tau_j$ is placed before $\tau_i$. 
Together, constraints~\eqref{con:p-cores} and~\eqref{con:order} ensure that (i) no more than $p$ transactions execute in parallel in any round and (ii) all precedence edges are respected in the resulting schedule.

\subsubsection{Heterogeneous Transactions}

We now extend the model to the heterogeneous case, in which a block contains both simple and complex transactions with different execution times. MILP~\ref{milp-pbm1-simple&complex} encodes this mixed setting. The total execution time (makespan) found is denoted by $M$, and $B$ is a parameter that must be set to be larger than any possible $M$. The objective is to minimize the makespan $M$ subject to the assignment, timing, and ordering constraints.

Each transaction $\tau_i$ is assigned to exactly one core via a binary variable $x_{ic}$ and scheduled with start and end times $s_i$ and $e_i$, respectively.  Precedence edges $(i,j) \in E$ enforce dependency order via \Cref{con:order-c1}, ensuring that conflicting transactions follow the sequence order. For non-conflicting pairs, overlap is permitted unless the two transactions are assigned to the same core. Same-core assignment is captured by the auxiliary variable $w_{ijc}$ (the logical AND of $x_{ic}$ and $x_{jc}$), and $z_{ij}$ indicates whether the pair shares any core. If $z_{ij}=1$, the binary selector $y_{ij}$ determines their order: $y_{ij}=1$ means $\tau_i$ precedes $\tau_j$; otherwise, $y_{ij}=0$ means the reverse. 
%
In \Cref{con:order1} and \Cref{con:order2}, observe the following cases:
\begin{itemize}
    \item $z_{ij}=0$: the constraints have no effect due to the runtime limit term $B$.
    \item $z_{ij}=1, y_{ij}=1$:  \Cref{con:order2} has no effect, only \Cref{con:order1} applies, yielding $e_i \le s_j$.
    \item $z_{ij}=1, y_{ij}=0$: \Cref{con:order1} has no effect, only \Cref{con:order2} applies, yielding $e_j \le s_i$.
\end{itemize}
%
These conditional constraints ensure that same-core transactions are serialized.
%, while all others may execute concurrently.

%\smallskip
\textbf{Correctness.}
Constraint~\eqref{con:onetrans-c1} ensures that every transaction is assigned to exactly one core. 
Precedence edges $(i,j) \in E$ are enforced by~\eqref{con:order-c1}, which requires that any dependent transaction $\tau_j$ starts only after $\tau_i$ has completed. 
Together, these constraints preserve the total order implied by the dependency graph.
%
For non-conflicting pairs, the coupling of $w_{ijc}$, $z_{ij}$, and $y_{ij}$ ensures that same-core pairs are serialized by 
\eqref{con:order1}--\eqref{con:order2}, while pairs on different cores remain unconstrained. 
Together, these properties ensure that MILP \ref{milp-pbm1-simple&complex} produces a feasible schedule that respects all dependencies and minimizes the overall makespan~$M$.

%\smallskip
\subsubsection{Complexity} 

Both MILP formulations grow quickly with the number of transactions $n$ and cores $p$. 
The homogeneous case (MILP \ref{milp-pbm1-simple}) involves $\mathcal{O}(nR)$ binary variables and $\mathcal{O}(nR + |E|R)$ constraints, while the heterogeneous case (MILP~\ref{milp-pbm1-simple&complex}) expands to $\mathcal{O}(n^2p)$ binary variables, $\mathcal{O}(n)$ integer variables, and $\mathcal{O}(n^2p + |E|)$ constraints. 
Each formulation generalizes well-known NP-hard multiprocessor scheduling problems, implying that exact solutions become computationally prohibitive as $n$ and $p$ increase. 
%In practice, solver time rises steeply with block size and heterogeneity, motivating the heuristics introduced next, which trade a small loss in optimality for orders-of-magnitude speed.

    \begin{algorithm}[H]
      \caption{$\builddag{}$. Recall the conflict definition from \Cref{subsubsec:conflicts}}%that $\mathrm{conflict}(i,j) \equiv \Big((\writeset{\tau_i}\cap \writeset{\tau_j})\cup(\writeset{\tau_i}\cap \readset{\tau_j})\cup(\readset{\tau_i}\cap \writeset{\tau_j})\Big)\neq\varnothing.$}%
      \label{heu:build-DAG}%
      \footnotesize
      \begin{algorithmic}[1]
        \Function{$\builddag{T}$}{}
          \State $V \gets \{\tau:\tau \in T\}$; $E \gets \emptyset$
          \For{\textbf{each} $i \in [1..n-1]$}
             \For{\textbf{each} $j \in [i+1..n]$}
                \If{$\mathrm{conflict}(\tau_i,\tau_j)$} $E \gets E \cup \{(\tau_i,\tau_j)\}$
                \EndIf
            \EndFor
          \EndFor
          \State \textbf{return} $G=(V,E)$
      \EndFunction
      \end{algorithmic}
      \end{algorithm}
      % \Function{$\builddag{T}$}{}
      %     \State $V \leftarrow \{\tau:\tau \in T\}$; $E \leftarrow \emptyset$
      %     \State $last\_write \leftarrow \perp$; $last\_read \leftarrow \emptyset$
          
      %     \For{$\tau \in T$ in block order}
    
      %       \For{$k \in R(\tau)$}
      %           \If{$last\_write[k] \neq \perp$}
      %               \State $E \leftarrow E \cup \{(last\_write[k],$\id{\tau}$)\}$
      %           \EndIf
      %           \State $last\_read[k] \leftarrow last\_read[k] \cup \{ $\id{\tau}$\}$
      %       \EndFor
    
      %       \For{$k \in W(\tau)$}
      %           \If{$last\_write[k] \neq \perp$}
      %               \State $E \leftarrow E \cup \{(last\_write[k],$\id{\tau}$)\}$
      %           \EndIf
      %           \For{$r \in last\_read[k]$}
      %               \State $E \leftarrow E \cup \{(r,$\id{\tau}$)\}$
      %           \EndFor
      %           $last\_write[k] \leftarrow  $\id{\tau}$; last\_read[k] \leftarrow \emptyset $
      %       \EndFor
            
      %     \EndFor

      %     \State \textbf{return} $G=(V,E)$
      % \EndFunction


    \begin{algorithm}[H]
      \caption{$\preprocess{}$}%
      \label{heu:preprocessing_problem1}%
      \footnotesize
      \begin{multicols}{2}
      \begin{algorithmic}[1]
      \Function{$\preprocess{G}$}{}
        \For{\textbf{each} $\tau \in V$}
            \State $\successor{[\tau]} \gets \{\tau' : (\tau, \tau') \in E\}$
            \State $\predecessor{[\tau]} \gets \{\tau' : (\tau', \tau) \in E\}$
            \State $\outdegree{[\tau]} \gets | \successor{[\tau]} |$
            \State $\indegree{[\tau]} \gets | \predecessor{[\tau]} |$
        \EndFor
        % \For{$(u,v) \in E$}
        %     \State $succ(u) \gets succ(u) \cup \{v\}$
        %     \State $ pred(v) \gets pred(v) \cup \{u\}$
        % \EndFor
        % \State for each $i$: $\indegree{i} \gets |pred[i]|$
        \State $Q \gets \{\tau: \outdegree{[\tau]}=0\}$ \Comment{Set of sinks}
        \While{$Q \neq \emptyset$}
            \State $\tau \gets$ any element in $Q$
            \State $Q \gets Q \setminus \{\tau\}$
            \State $\height{[\tau]} \gets \executiontime{\tau}+\max_{\tau' \in \successor{[\tau]}}\{\height{[\tau']}\}$
            \State $\volume{[\tau]} \gets \executiontime{\tau}+\sum_{\tau' \in \successor{[\tau]}}\{\volume{[\tau']}\}$
            \For{\textbf{each} $\tau' \in \predecessor{[\tau]}$}
                \State $\outdegree{[\tau']} \gets \outdegree{[\tau']} - 1$
                \If{$\outdegree{[\tau']} = 0$} $Q \gets Q \cup \{\tau'\}$
                \EndIf
            \EndFor
        \EndWhile
        \For{\textbf{each} $\tau \in V$}
            \State $\priority{[\tau]} \gets (-\height{[\tau]},-\volume{[\tau]},-|\successor{[\tau]}|,\id{\tau})$
        \EndFor
        \State \textbf{return}$(\priority,\indegree,\successor)$
      \EndFunction
      \end{algorithmic}
      \end{multicols}
      \end{algorithm}
      % \State $Q \gets \text{Queue of nodes with in-degree} = 0$
        % \State $topo \gets []$
        % \While{$Q \neq \emptyset$}
        %     \State $x \gets dequeue(Q)$
        %     \State append $x$ to $topo$
        %     \For{$y \in succ(x)$}
        %         \State $\indegree{y} \gets \indegree{y} - 1$
        %         \If{$\indegree{y} = 0$}
        %             \State $enqueue$ $y$ to $Q$
        %         \EndIf
        %     \EndFor   
        % \EndWhile
        % \State assert $|topo| = |V|$
        % \For{$i \in V$}
        %     \State $h(i) \gets \executiontime{i}; v(i) \gets \executiontime{i}$
        % \EndFor
        % \For{$x \in reverse(topo)$}
        %     \If{$succ(x) \neq \emptyset$}
        %         \State $h(x) \gets \executiontime{x}+max\{h(y):y \in succ(x)\}$
        %         \State $v(x) \gets \executiontime{x}+\sum\{v(y):y \in succ(x)\}$
        %     \EndIf
        % \EndFor
        % \State for each $i$: $\indegree{i} \gets |pred[i]|$
        % \State for each $i$: $\outdegree{i} \gets |succ[i]|$
        % \State $K(i)=(-h(i),-v(i),-\outdegree{i},\id{i})$
        % \State \textbf{return}$(K,pred,succ)$
      % \EndFunction

      % \end{algorithmic}
      % \end{algorithm}

\begin{algorithm}[H]
\caption{$\findschedulePone{}$}%
\label{heu:scheduling_problem1}%
\footnotesize
\begin{multicols}{2}
\begin{algorithmic}[1]
\Function{$\findschedulePone{T, p, \priority, \indegree, \successor}$}{}
    \State $now \gets 0$
    \State $free \gets [1..p]$ \Comment{Free cores}
    \State $running \gets \emptyset$ \Comment{Trans.~that are running}
    \State $ready \gets \{\tau \in T: \indegree{[\tau]}=0\}$ \Comment{Ready tx}
    \Repeat
        \While{$(ready \neq \emptyset) \land (free \neq \emptyset)$}
            \State $\tau \gets \arg \min \{ \priority{[\tau']}: \tau' \in ready \}$
            \State $ready \gets ready \setminus \{\tau\}$
            \State $running \gets running \cup \{\tau\}$
            \State $\start{[\tau]} \gets now$
            \State $\finish{[\tau]} \gets now + \executiontime{\tau}$
            \State $\core{[\tau]} \gets$ any element in $free$
            \State $free \gets free \setminus \{\core{[\tau]}\}$
        \EndWhile
        \State $now \gets \min\{\finish{[\tau]} : \tau \in running\}$
        \State $S \gets \{\tau \in running \;:\; \finish{[\tau]} = now\}$
        \State $running \gets running \setminus S$
        \For{\textbf{each} $\tau \in S$}
        % \State $\tau \gets \arg \min_{\tau' \in running} \{\finish{[\tau']}\}$
        % \State $running \gets running \setminus \{\tau\}$
        % \State $now \gets \finish{[\tau]}$
        \For{\textbf{each} $\tau' \in \successor{[\tau]}$}
            \State $\indegree{[\tau']} \gets \indegree{[\tau']} - 1$
            \If{$\indegree{[\tau']} = 0$} 
                \State $ready \gets ready \cup \{\tau'\}$
            \EndIf
        \EndFor
        \State $free \gets free \cup \{\core{[\tau]}\}$
        \EndFor
    \Until{$(running = \emptyset) \land (ready = \emptyset)$}
    \State \textbf{return} $(\start, \core)$
\EndFunction
\end{algorithmic}
\end{multicols}
\end{algorithm}
    % \While{$(ready \neq \emptyset) \land (free \neq \emptyset)$}
    %         \State $\tau \gets \arg \min_{\tau' \in ready} \{\priority{[\tau']}\}$
    %         \State $ready \gets ready \setminus \{\tau\}$
    %         \State $running \gets running \cup \{\tau\}$
    %         \State $\start{[\tau]} \gets now$
    %         \State $\finish{[\tau]} \gets now + \executiontime{\tau}$
    %         \State $\core{[\tau]} \gets$ any element in $free$
    %         \State $free \gets free \setminus \{\core{[\tau]}\}$
    % \EndWhile
    % \State \textbf{return} $(\start, \core)$
    % \EndFunction

    %   \end{algorithmic}
    %   \end{algorithm}
      
    % \Function{SCHEDULE\_TRANSACTIONS}{T, succ, h v}
    %     \State $ready \leftarrow$ min-heap keyed by $K(i)$
    %     \For{$i \in T$}
    %         \If{$\indegree{i}=0$} \State insert $i$ into $ready$ \EndIf
    %     \EndFor
    %     \State $r \leftarrow 0$
    %     \While{$ready \neq \emptyset$}
    %         \State $r \leftarrow r+1$; \ $S \leftarrow \emptyset$
    %         \For{$c=1$ \textbf{to} $p$}
    %             \If{$ready=\emptyset$} \textbf{break} \EndIf
    %             \State $i \leftarrow$ extract-min($ready$); \ $S \leftarrow S \cup \{i\}$; \ $round(i) \leftarrow r$
    %         \EndFor
    %         \For{$i \in S$}
    %             \For{$j \in succ(i)$}
    %                 \State $\indegree(j) \leftarrow \indegree(j)-1$
    %                 \If{$\indegree(j)=0$} \State insert $j$ into $ready$ \EndIf
    %             \EndFor
    %         \EndFor
    %     \EndWhile
    %     \State \textbf{return} $round$
    %   \EndFunction

    % \Function{$\findschedule{T, p, \priority, \indegree, \successor}$}{}
    %     \State $now \gets 0$
    %     \State for each $i \in T$: $start[i]\leftarrow\bot$, $finish[i]\leftarrow\bot$, $core[i]\leftarrow\bot$
    %     \State for each $i \in T$: $\indegree{i} \gets |pred[i]|$
    %     \State $ready \gets$ min-heap keyed by $K(i)$ in lexicographic order
    %     \State $running \gets$ min-heap keyed by finish time
    %     \State $free \gets \{1,2,\dots,p\}$

    %     \For{\textbf{each} $i \in T$}
    %       \If{$\indegree{i}=0$} \State push $i$ into $ready$ \EndIf
    %     \EndFor

    %     \While{$ ready \neq \emptyset$ \textbf{and} $free \ne \emptyset$ }
    %         \State $i \gets \text{extract-min}(ready)$
    %         \State pick and remove any $c \in free$
    %         \State $core[i]\gets c$; \ $start[i]\gets now$; \ $finish[i]\gets now + \executiontime{i}$
    %         \State push $(finish[i], i, c)$ into $running$
    %     \EndWhile

    %     \While{$running \neq \emptyset$ \textbf{or} $ready \neq \emptyset$}
    %         \State $(t\_next, \_,\_) \gets$ extract-min($running$) \Comment{Advance to the next completion event}
    %         \State $now \gets t\_next$

            
    %         \State $S \gets \emptyset$ \Comment{Collect and finalize all jobs finishing at t\_next}
    %         \While{$running \neq \emptyset$ \textbf{and}  $\text{top}(running).\text{finish\_time} = now$}
    %             \State $(_,j,c) \gets$ extract-min($running$)
    %             \State add $(j,c)$ to $S$
    %         \EndWhile

    %         \For{ \textbf{each} $(j,c) \in S$}
    %             \State insert $c$ into $free$
    %         \EndFor
            
    %         \For{ \textbf{each} $(j,c) \in S$}
    %             \For{\textbf{each} $k \in succ(j)$}
    %                 \State $\indegree{k} \gets \indegree{k} -1$
    %                 \If{$\indegree{k}=0$}
    %                     \State push $k$ into $ready$
    %                 \EndIf
    %             \EndFor
    %         \EndFor

    %         \While{$ ready \neq \emptyset$ \textbf{and} $free \neq \emptyset$} 
    %             \State $i \gets$ extract-min($ready$)
    %             \State pick and remove any $c \in free$
    %             \State $core[i]\gets c$; \ $start[i]\gets now$; \ $finish[i]\gets now + \executiontime{i}$
    %             \State push $(\text{finish}[i], i,c)$ into $running$
    %         \EndWhile
        
    %     \EndWhile
    %     \State \textbf{return} $(start, core)$
    % \EndFunction

    %   \end{algorithmic}
    %   \end{algorithm}

  \begin{algorithm}[H]
  \footnotesize
    \caption{OBS Heuristics}
    \label{heu:problem1}
    \begin{algorithmic}[1]
        \State \textbf{Input:} 
        Block $T=(\tau_1,\dots,\tau_n)$.  
        %For each $\tau \in T$, read set $\readset{\tau}$, write set $\writeset{\tau}$, execution time $\executiontime{\tau}$, identifier $\id{\tau}$. 
        Number of cores  $p$.
        \State \textbf{Output:} A schedule with the start time $\start{[\tau]}$ and core $\core{[\tau]}$ assigned to each transaction $\tau \in T$.
        
      
      % \Statex
      % \State $pred$: array of sets $\leftarrow [\emptyset, ..., \emptyset]$ %\comment{Vector mapping each transaction to a set of conflicting predecessors}
      % \State $succ$: array of sets $\leftarrow [\emptyset, ..., \emptyset]$
      % \State $h$: array of integers
      % \State $v$: array of integers

      \Statex
      \State $G \gets \builddag{T}$
      \State $(\priority,\indegree, \successor) \gets \preprocess{G}$
      \State $(\start, \core) \gets \findschedulePone{T, p, \priority, \indegree, \successor}$
    \end{algorithmic}
  \end{algorithm}

\subsection{Heuristics}

Our heuristics (\Cref{heu:build-DAG,heu:preprocessing_problem1,heu:scheduling_problem1,heu:problem1}, 
%deferred to~\cref{sec:OBS_heuristicalgorithms}) 
use a conflict-aware greedy scheduler, guided by transaction priorities. It proceeds in three stages:

%\begin{itemize}
    %\item \textbf{DAG Construction:} For the given set of transactions $T$, we build a conflict DAG following the definition of conflicts presented in \cref{subsubsec:conflicts}.
    %\item 
    \noindent
    \textbf{(1) DAG construction.} For the given block $T=(\tau_1,\dots,\tau_n)$, we build a dependency DAG $G=(V,E)$ (\Cref{heu:build-DAG}). Each node is a transaction $\tau \in T$, and we add an edge $(\tau_i,\tau_j)$ whenever $\tau_i$ and $\tau_j$ conflict (as defined in \Cref{subsubsec:conflicts}) and $i<j$. This DAG encodes all precedence constraints that must be respected at execution time.
    %\item \textbf{Priority Scores:} First, the node-level dependency profile is extracted from the DAG. A single backward pass computes per-node height (critical-path execution length) and volume (sum of subtree execution times). The priority key is $\priority{[\tau]} \gets (-\height{[\tau]},-\volume{[\tau]},-|\successor{[\tau]}|,\id{\tau})$, favouring long critical paths, then heavier subtrees, then higher fan-out. The transaction id breaks ties deterministically.
    
    %\item 
       \noindent
       \textbf{(2) Priority scoring.} We then extract structural information from $G$ (\Cref{heu:preprocessing_problem1}). For each transaction $\tau$, we compute:
    \begin{itemize}
        \item $\height{[\tau]}$: the critical-path length rooted at $\tau$ (i.e., $\tau$'s own execution time plus the maximum downstream height),
        \item $\volume{[\tau]}$: the total execution volume of $\tau$ and its descendants,
        \item $|\successor{[\tau]}|$: the fan-out of $\tau$.
    \end{itemize}
    A single backward pass over the DAG computes these values. We assign each transaction a deterministic priority key
    $
        \priority{[\tau]} := \big(-\height{[\tau]},\; -\volume{[\tau]},\; -|\successor{[\tau]}|,\; \id{\tau}\big),
    $
    which favors long critical paths first, then heavier subtrees, then higher fan-out, and $\id{\tau}$ breaks ties.
    %\item \textbf{Transaction Scheduling:} We maintain a set of ready transactions and a set of running jobs. Whenever one or more cores become free, we dispatch the best ready job. When jobs finish, we decrease their successors’ in-degrees and enqueue those successors that reach zero in-degrees to ready. This works for both simple-only and simple-complex transaction workloads.
    
    %\item    
    \noindent
    \textbf{(3) Transaction scheduling.} Finally, we simulate the block execution on $p$ cores (\Cref{heu:scheduling_problem1}). The scheduler maintains (i) a set of \emph{ready} transactions (initially those with in-degree zero in $G$), (ii) a set of \emph{running} transactions with assigned cores and finish times, and (iii) the current time $now$. Whenever any core becomes free, we schedule the highest-priority ready transaction to that core. When a transaction finishes, we decrement the in-degree of its successors and add any successor with zero in-degree to the set of ready transactions. This event-driven loop continues until all transactions are scheduled. \Cref{heu:problem1} combines these steps into a complete pipeline.
%\end{itemize}

%\smallskip
\textbf{Complexity.}
The DAG construction in \Cref{heu:build-DAG} compares each ordered pair $(\tau_i,\tau_j)$ with $i<j$, and is therefore $\mathcal{O}(n^2)$ in the worst case.\footnote{In practice, this can often be pruned with sparse access lists, but we analyze the dense upper bound.} The preprocessing pass in \Cref{heu:preprocessing_problem1} performs a single reverse topological traversal plus constant work per edge, i.e., $\mathcal{O}(n + |E|)$. The scheduler in \Cref{heu:scheduling_problem1} runs as an event-driven simulation with at most $n$ schedule events and $n$ completion events; using a priority queue for the ready set, this is $\mathcal{O}\big(n \log n + |E|\big)$. Overall, the heuristic end-to-end runtime is dominated by DAG construction and is $\mathcal{O}(n^2)$ in the worst case.
Regarding the approximation ratio, since the heuristic implements a list scheduling, we know that it gives a
schedule with makespan at most $2-1/p$ times the optimal.

