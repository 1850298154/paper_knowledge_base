\section{Conclusions} \label{sec:conclusion}

In this paper, we presented a systematic exploration of validator-side parallelism in blockchain transaction execution. We formulated two complementary optimization problems: (i) executing an already ordered block on multiple cores while minimizing makespan, and (ii) selecting and scheduling a subset of mempool transactions under a gas budget to maximize validator reward. For both, we developed exact MILP formulations to serve as optimal baselines, and efficient deterministic heuristics that scale to realistic workloads. We also include a Sealevel-inspired declared-access baseline (Sol) to represent a practical, conflict-avoiding in-order execution strategy.

Our evaluation on Ethereum mainnet traces demonstrate that MILP runtimes grow steeply with problem size, heterogeneity, and core count, confirming the computational hardness of the problems. Our heuristics remain sub-second across all settings, produce solutions that are close to those of the MILP, and reduce makespan and increase reward by a sensible amount with respect to sequential scheduling. Across our experiments, GH consistently matches or improves upon Sol in solution quality, with particularly clear advantages as conflict intensity increases, while maintaining comparable (sub-second) scheduling overhead.

From a systems perspective, our findings show that even simple, conflict-aware scheduling strategies can make effective use of available cores, narrowing the gap between theoretical and practical parallelism in blockchain execution. In particular, the mempool-based formulation provides a path toward constructing blocks that are inherently “parallel-friendly,” coupling selection and scheduling to improve both performance and economic efficiency.

\textbf{Limitations and Future Work.} Our evaluation uses the \texttt{gasUsed} field from Ethereum traces as a proxy for execution time. This provides a consistent and realistic measure for the ordered-block problem, but it is only an approximation of the actual runtime. In practice, execution time and gas usage may diverge due to hardware, client, or EVM implementation differences. Moreover, in the mempool formulation, reordering transactions could change their effective gas consumption in the case of conflicts. A second limitation is that our current heuristics largely resolve conflicts through local, greedy decisions based on static conflict information. A promising direction is to design more holistic conflict-aware policies that reason globally about contention (e.g., hotspots, conflict clusters, or anticipated serialization) when selecting and scheduling transactions.

Exploring how to model such non-determinism in execution time while maintaining the safety guarantees provided by conservative gas limits is left for future work. Similarly, an open systems question is how best to maintain and update dependency graphs efficiently as new transactions arrive and blocks are mined. This also suggests online heuristics that update priorities as conflicts evolve, rather than relying on a fixed, one-shot ordering. Designing low-overhead data structures or incremental graph-maintenance schemes for this purpose will be essential for bringing these scheduling techniques into production blockchain environments.
