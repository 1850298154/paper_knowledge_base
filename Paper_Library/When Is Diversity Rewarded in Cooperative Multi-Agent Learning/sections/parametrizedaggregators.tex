\headline{Parameterizable Families of Aggregators} \label{sec:parametrizedaggregators}
A core topic of this work is \emph{reward design}: how can we craft team objectives that either advantage or disadvantage behavioral diversity?  To do this, it is helpful to first identify an appropriate search space. Our theoretical analysis enables us to narrow down this search space, and focus on aggregators whose \textit{curvature} can be parametrized.  Many \emph{family of aggregator functions} 
\(\{\,f_t(\cdot)\}_{t \in \mathbb{R}}\) can be parametrized by a scalar 
\(t\) which controls whether the aggregation is \emph{Schur‐convex} or 
\emph{Schur‐concave}, and how strongly it penalizes (or favors) inequalities among the components. For example, the \textit{softmax aggregator} $\sum_{i=1}^N 
\frac{\exp\bigl(t \cdot r_{i j}\bigr)}{\sum_{\ell=1}^N \exp\bigl(t \cdot r_{\ell j}\bigr)}$ is parametrized by its temperature, $t$, transitioning from being strictly Schur-concave when $t < 0$ to strictly Schur-convex when $t > 0$. We can define a space of reward functions by selecting both the task scores and outer aggregator to be softmax functions: let \(
T_j(\mathbf{A})
=
\sum_{i=1}^N 
\frac{\exp\bigl(t \cdot r_{i j}\bigr)}{\sum_{\ell=1}^N \exp\bigl(t \cdot r_{\ell j}\bigr)}
\; r_{i j},
\) and let \(
U\bigl(T_1(\mathbf{a}_1) , \ldots T_M(\mathbf{a}_m) )
=
\sum_{j=1}^M 
\frac{\exp\bigl(\tau \cdot T_j(\mathbf{A})\bigr)}
     {\sum_{\ell=1}^M \exp\bigl(\tau \cdot T_\ell(\mathbf{A})\bigr)}
\; T_j(\mathbf{A})
\), where $t, \tau \in \mathbb{R}$ parametrize the inner and outer aggregators, respectively. $\Delta R$ is then dependent on $t$ and $\tau$. \autoref{fig:deltaR-vs-softmax} plots $\Delta R$ when $N = M =  2$. As a case study, we derive lower bounds on $\Delta R$ when $N = M$ in \autoref{thm:gap_NeqM_softmax_hetgap}. 

% We \textit{conjecture} the lower bound of \autoref{thm:gap_NeqM_softmax_hetgap} is actually the exact value of $\Delta R(t,\tau;N)$, but we were unable to prove this.  \autoref{fig:deltaR-vs-softmax} (righthand side) plots the heterogeneity gains when $N = M =  2$. 

\begin{theorem}[Softmax \heterogeneitygap for $N=M$]
\label{thm:gap_NeqM_softmax_hetgap}
Assume $N = M \geq 2$, and let \(\sigma(t,N):=\frac{e^{t}}{e^{t}+N-1}
\). The \heterogeneitygap for softmax aggregators \textbf{(i)} equals $\Delta R(t,\tau;N)=0$ when $t \leq 0$; \textbf{(ii)} is bounded below by $\sigma(t,N)-\dfrac1N$ when $t>0, \tau\le 0$; and  \textbf{(iii)} is bounded below by $\max\!\bigl\{\sigma(t,N)-\sigma(\tau,N),0\bigr\}$ when $t>0, \tau\ge 0.$
% \[
% \boxed{%
% \Delta R(t,\tau;N)\geq
% \begin{cases}
% \sigma(t,N)-\dfrac1N, & t>0,\;\;\tau\le 0,\\[10pt]
% \max\!\bigl\{\sigma(t,N)-\sigma(\tau,N),\,0\bigr\}, & t>0,\;\;\tau\ge 0.
% \end{cases}}
% \] otherwise, where 
% \(\sigma(t,N)\;:=\;\frac{e^{t}}{e^{t}+N-1}.
% \)
\end{theorem}

\autoref{tab:param-agg-extended} contains more examples of aggregation operators parameterized by $t$. These families provide a search space for potential reward functions, allowing us to sweep smoothly from $\Delta R = 0$ to $\Delta R > 0$ reward regimes.  As $t \to \pm \infty$, most such aggregators converge to either $\min$ or $\max$, and often reduce to the arithmetic mean for certain parameter choices, motivating us to ask what the \heterogeneitygap is when the outer and inner aggregator belong to the set $\{\min, \text{mean}, \max\}$. These aggregators are of special interest, since ``$\min$'' can be seen as a ``maximally'' Schur-concave function, ``$\max$'' can be seen as a ``maximally'' Schur-convex function, and ``mean'' is both Schur-convex and Schur-concave. Hence, it is worth asking what the \heterogeneitygap is when the outer and inner aggregator belong to the set $\{\min, \text{mean}, \max\}$. We derive these gains in two cases: continuous allocations where $r_{ij} \in [0,1]$, and discrete effort allocations where $r_{ij} \in \{0,1\}$. The results are summarized in \autoref{fig:deltaR-vs-softmax}, lefthand side (formal derivation available in \autoref{appendix:formal_analysis}). 


%–––––  Preamble needs –––––
% \usepackage{graphicx}    % \includegraphics
% \usepackage{adjustbox}   % \begin{adjustbox}{…}
% \usepackage{caption}     % \captionof   (standard in article/report)

\begin{figure}[t]            % → figure* in two-column styles
  \centering
  %================ LEFT BLOCK (ΔR table) ================
  \begin{minipage}{0.5\textwidth}
    \centering
    \footnotesize
    \captionof*{table}{\small Discrete and continuous heterogeneity gains}
    %
    \begin{adjustbox}{max width=\linewidth}
      \tiny                        % table font only
      \setlength{\tabcolsep}{3pt}  % tight columns
      \renewcommand{\arraystretch}{1.15}
      \begin{tabular}{l|ccc}
        & $T=\min$ & $T=\text{mean}$ & $T=\max$\\\hline
        \multicolumn{4}{c}{\it Outer $U=\min$}\\\hline
        $\Delta R_{\mathrm{F}}$ & 0 & 0 & $(M-1)/M$\\
        $\Delta R_{\mathrm{D}}$ & 0 & $\lfloor N/M\rfloor/N$ &
          $\mathbf 1_{\{N\ge M\}}$\\\hline
        \multicolumn{4}{c}{\it Outer $U=\text{mean}$}\\\hline
        $\Delta R_{\mathrm{F}}$ & 0 & 0 & $(M-1)/M$\\
        $\Delta R_{\mathrm{D}}$ & 0 & 0 & $(\min\{M,N\}-1)/M$\\\hline
        \multicolumn{4}{c}{\it Outer $U=\max$}\\\hline
        $\Delta R_{\mathrm{F}}$ & 0 & 0 & 0\\
        $\Delta R_{\mathrm{D}}$ & 0 & 0 & 0\\\hline
      \end{tabular}
    \end{adjustbox}
  \end{minipage}%   ←– prevents newline
  \hfill%            ←– prevents newline after glue
  %================ RIGHT BLOCK (graphic) ================
  \begin{minipage}{0.5\textwidth}
    \centering
    \includegraphics[width=\linewidth]{images/hetgap_N2M2_softmax.pdf}
    % \captionof*{figure}{\small $N{=}2,\;M{=}2$ soft-max heterogeneity gains}
  \end{minipage}

  \caption{\textbf{Left:} Discrete ($\Delta R_{\mathrm D}$) and continuous-allocation
           ($\Delta R_{\mathrm F}$) heterogeneity gains for all
           $U,T\!\in\!\{\min,\text{mean},\max\}$. The indicator
           $\mathbf 1_{\{N\ge M\}}$ equals 1 if \(N\ge M\) and 0 otherwise. 
           \textbf{Right:} We plot the parametrized heterogeneity gains $\Delta R(t,\tau;N)$ when $U$ and $T$ are soft-max aggregators.  }
  \label{fig:deltaR-vs-softmax}
\end{figure}


% ----


% We derive these \heterogeneitygaps in Table \ref{tab:softmaxextremes}.  The ``discrete'' case where each agent must allocate all effort to a single task (i.e., for each $i$, $r_{ij} = 1$ for some $j$ and $0$ for all $j' \neq j$) is sometimes also of interest. \Heterogeneitygaps for this case are summarized in Table \ref{tab:discrete-finite-tau}.

% ----

