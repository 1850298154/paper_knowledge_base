
This work introduces tools for both \emph{diagnosing} and \emph{designing} reward functions that incentivize heterogeneity in cooperative MARL. In task allocation settings, our theory shows that the advantage of behavioral diversity is a predictable consequence of reward \emph{curvature}: if the inner aggregator is Schur-convex, amplifying inequality, and the outer aggregator is Schur-concave, amplifying uniformity, heterogeneous policies are strictly superior; reversing the curvature removes the benefit. Complementing this analysis, and covering settings where our theory doesn't apply, the proposed HetGPS algorithm automatically steers underspecified environments to either side of the diversity boundary, letting us encourage or suppress heterogeneity and providing a sandbox for studying its advantages. Together, these results help turn the choice of heterogeneity from an ad-hoc heuristic into a controllable design dimension, and help reconcile past mixed results on parameter sharing. 

A key remaining open question concerns how the environment’s \emph{transition dynamics} interact with reward curvature to shape heterogeneity gains.  We expand on this open question, other directions for future work, and our scope/limitations, in \autoref{appendix:limitations}.

\section*{Acknowledgements and Disclosure of Funding}
This work is supported by European Research Council (ERC) Project 949940 (gAIa) and ARL DCIST CRA W911NF-17-2-0181. We gratefully acknowledge their support.



% In this work we introduced practical instruments for \emph{diagnosing} and \emph{designing} heterogeneity in cooperative MARL. Our analysis shows that incentives for behavioral diversity are a predictable consequence of reward \emph{curvature}: inner aggregators that amplify inequality, combined with outer aggregators that preserve it, make heterogeneous policies strictly superior, whereas the opposite pairing removes any advantage. Complementing these results, our \textbf{Heterogeneity Gain Parameter Search} (HetGPS) algorithm automatically tunes underspecified tasks to lie on either side of this diversity boundary, enabling one to incentivize or suppress specialization as required, and offering a sandbox for studying heterogeneity in cases where theory breaks down. We view our results as a step toward turning the assessment of the benefits of heterogeneity from a heuristic into a well-understood, controllable design dimension, reconciling past mixed findings on parameter sharing. In future work, we hope that this kind of research opens the door to studying richer forms of diversity—such as evolving roles or emergent communication—through the same analytic lens. An extended discussion of limitations and open questions appears in \autoref{appendix:limitations}.

% In this work, we developed instruments for both diagnosing and designing heterogeneity in cooperative MARL. Our results show that the incentive for behavioral diversity in cooperative MARL is a predictable consequence of the curvature designed into the reward: when inner aggregators reward inequality and outer aggregators preserve it, heterogeneous teams enjoy a performance advantage, whereas the opposite curvature neutralizes that benefit entirely. The HetGPS algorithm further enables one to discover environments where heterogeneity is beneficial (or, alternatively, where homogeneity is), enabling a deeper exploration and study when our theory fails. Together, the curvature test and HetGPS turn heterogeneity from a heuristic choice into a controllable design dimension, clarify past mixed results on parameter sharing, and lay the groundwork for studying richer forms of diversity—such as evolving roles or emergent communication—under the same analytic lens. An extended discussion of limitations and open questions is provided in \autoref{appendix:limitations}.

% By turning this qualitative insight into a simple convexity test and an optimisation routine (HetGPS) that can sculpt environments toward or away from diversity, we offer both a diagnostic and a design tool: theorists gain a clear criterion for analysing existing benchmarks, while practitioners can purpose-build tasks that either demand or discourage heterogeneity, depending on the application. Beyond clarifying conflicting empirical reports on when separate networks help, our framework suggests a principled way to trade off sample-efficiency against specialisation, to co-optimise rewards and team architecture, and to benchmark MARL algorithms under systematically varying incentives for diversity. We hope this work will encourage a shift from ad-hoc heuristics toward reward-aware heterogeneity design, and will spark follow-up studies on richer forms of diversity—e.g., temporally evolving roles, communication protocols, or mixed-motives teams—using the same curvature lens. An extended discussion of current limitations and open questions is provided in \autoref{appendix:limitations}.


% An extended discussion on the limitations of our work is available in \autoref{appendix:limitations}. 