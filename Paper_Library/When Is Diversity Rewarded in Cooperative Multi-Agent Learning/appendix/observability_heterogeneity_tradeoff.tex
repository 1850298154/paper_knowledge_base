\section{Observability-Heterogeneity Trade-Off}
\label{app:observehettradeoff}

% \begin{figure}[t] 
%     \centering  % Centers the figure
%     \includegraphics[width=\textwidth]{images/ctf_embodied_lidar.pdf} 
%     \caption{\Heterogeneitygap for {\capturetheflag} over training iterations as a function of the LIDAR sensor radius.
%     The plot shows that the gap decreases as the agent observability increases.
%     We report mean and standard deviation after 30 million training frames over 4 different random seeds.}  % Add your figure caption
%     \label{fig:ctf_gap_lidar} 
% \end{figure}

\begin{figure}[t]
  \centering
   \centering
     \includegraphics[width=0.6\textwidth]{images/ctf_embodied_lidar.pdf} 
    \caption{Gain w.r.t. observability when $U=\min,T=\max$.
    }  % Add your figure caption
    \label{fig:ctf_gap_lidar} 
  \caption{\Heterogeneitygap for {\capturetheflag} throughout training when the agents' observation range is gradually increased from $0$ to $0.35$ over 4 random seeds (4 random seeds suffice as this phenomenon is established in the literature \citep{bettini2023hetgppo}, and we only wish to show its emergence in the context of our work.)}
\end{figure}

In this Appendix, we crystallize the relationship between environment observability and empirical \heterogeneitygaps. It is well known that neurally homogeneous agents (i.e., sharing the same parameters) can achieve behavioral heterogeneity by conditioning their actions on diverse input contexts (behavioral typing). 
%It is well known that \textit{neurally} homogeneous agents (i.e., sharing the same neural network) can emulate diverse behavior by conditioning on the input context (behavioral typing)~\citep{leibo2019malthusian}.
This can be achieved by naively appending the agent index to its observation~\citep{gupta2017cooperative} or by providing relevant observations that allows the agents to infer their role~\citep{bettini2023hetgppo}.
Behavioral typing is impossible in matrix games, as these games are observationless.
However, it is possible in more complex games, such as our {\capturetheflag} scenario.
We augment agents in the positive gain scenario ($U=\min,T=\max$) with a range sensor, providing proximity readings for other agents within a radius.
In \autoref{fig:ctf_gap_lidar}, we show that the \heterogeneitygap decreases as the agent visibility increases (higher sensing radius).
This is because, with a higher range, homogeneous agents can sense each other and coordinate to pursue different goals.
This result highlights the tight interdependence between the \heterogeneitygap and agents' observations. 
% We note that, even if homogeneous agents may be able to narrow the gap by leveraging high observability, this makes them less resilient and more brittle to noise and adversarial attacks due to their increased reliance on the input context~\citep{bettini2023hetgppo}. 