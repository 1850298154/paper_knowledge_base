
\documentclass{article} % For LaTeX2e
\usepackage{iclr2026_conference,times}
\iclrfinalcopy
\pagestyle{plain}  

% Optional math commands from https://github.com/goodfeli/dlbook_notation.
\input{math_commands.tex}


\usepackage[utf8]{inputenc} % allow utf-8 input
\usepackage[T1]{fontenc}    % use 8-bit T1 fonts
\usepackage[colorlinks=true,
    linkcolor=black,   % cross-references (\ref, \eqref)
    citecolor=black,   % citations
    urlcolor=blue]     % ONLY \href and \url
{hyperref}
\usepackage{url}            % simple URL typesetting
\usepackage{booktabs}       % professional-quality tables
\usepackage{amsfonts}       % blackboard math symbols
\usepackage{nicefrac}       % compact symbols for 1/2, etc.
\usepackage{microtype}      % microtypography
\usepackage{xcolor}         % colors
\usepackage{hyperref}

\usepackage{amsmath}    % For advanced math typesetting
\usepackage{amssymb}    % For additional math symbols
\usepackage{amsthm}
\usepackage{bm}         % For bold math symbols
\usepackage{graphicx}   % Required for inserting images
\usepackage{natbib}
\usepackage{enumitem}
\usepackage{hyperref}
\usepackage{subcaption}
\usepackage{tikz}
% \usepackage[dvipsnames]{xcolor}
\usepackage{booktabs}
\usepackage{listings}
\usepackage{multirow}
\usepackage{graphicx,subcaption,adjustbox}
\usepackage[OMLmathbf]{isomath}
% Attempt to make hyperref and algorithmic work together better:
\newcommand{\theHalgorithm}{\arabic{algorithm}}
\RequirePackage{algorithm}
\RequirePackage[noend]{algorithmic}

\usepackage{etoolbox}   
\AtBeginDocument{%
  \captionsetup[figure]{skip=4pt}     
}
\setlength{\textfloatsep}{8pt plus 2pt minus 2pt}  
\usepackage{titlesec}
\titlespacing*{\section}{0pt}{0.75\baselineskip}{0.35\baselineskip}
\titlespacing*{\subsection}{0pt}{0.45\baselineskip}{0.15\baselineskip}


\theoremstyle{plain} % or definition or remark, depending on your preference
\newtheorem{definition}{Definition}[section]
\newtheorem{observation}{Observation}[section]
\newtheorem{proposition}{Proposition}[section]
\newtheorem{lemma}{Lemma}[section]
\newtheorem{theorem}{Theorem}[section]
\newtheorem{corollary}{Corollary}[section]
\newtheorem{example}{Example}[section]

\newcommand{\algorithmautorefname}{Alg.}%
\newcommand{\definitionautorefname}{Def.}
\newcommand{\propertyautorefname}{Property}
\newcommand{\propositionautorefname}{Prop.}
\newcommand{\corollaryautorefname}{Corollary}
\renewcommand{\figureautorefname}{Fig.}%
\renewcommand{\tableautorefname}{Tab.}%
\renewcommand{\equationautorefname}{Eq.}%
\renewcommand{\sectionautorefname}{Sec.}%
\renewcommand{\subsectionautorefname}{Sec.}%
\renewcommand{\subsubsectionautorefname}{Sec.}%
\renewcommand{\appendixautorefname}{App.}%
\renewcommand{\theoremautorefname}{Thm.}

\usepackage{xspace}
\newcommand{\capturetheflag}{Multi-goal-capture\xspace}
\newcommand{\heterogeneitygap}{heterogeneity gain\xspace}
\newcommand{\Heterogeneitygap}{Heterogeneity gain\xspace}
\newcommand{\heterogeneitygaps}{heterogeneity gains\xspace}
\newcommand{\Heterogeneitygaps}{Heterogeneity gains\xspace}
\newcommand{\HeterogeneityGap}{Heterogeneity Gain\xspace}

\newcommand{\websiteurl}{https://sites.google.com/view/hetgps}

\newcommand{\websiteurlctf}{https://sites.google.com/view/hetgps\#h.3f4tphhd9pn8
}

\newcommand{\websiteurltag}{https://sites.google.com/view/hetgps\#h.tzxrmpjqhomw
}

\newcommand{\websiteurlfootball}{https://sites.google.com/view/hetgps\#h.2y559rx4xjjx}

\makeatletter
\def\blfootnote{\gdef\@thefnmark{}\@footnotetext}
\makeatother


\newcommand{\fixit}[1]{\textcolor{red}{#1}}
% % Inner aggregation operator: open circle with a small "A" at bottom-right.
% \newcommand{\inneragg}{%
%   \mathop{%
%     \begin{tikzpicture}[baseline=-0.5ex, scale=1]
%       % Draw an open circle similar to \bigoplus
%       \draw[line width=0.8pt] (0,0) circle (0.6em);
%       % Draw the plus sign inside the circle
%       \draw[line width=0.8pt] (-0.4em,0) -- (0.4em,0);
%       \draw[line width=0.8pt] (0,-0.4em) -- (0,0.4em);
%     \end{tikzpicture}%
%   }%
% }

\DeclareMathOperator*{\inneragg}{\bigoplus}
\DeclareMathOperator*{\outeragg}{\bigoplus}
% \newcommand{\R}{\mathbb{R}}


\newcommand{\headline}[1]{\noindent\textbf{#1.}}

\title{When Is Diversity Rewarded in Cooperative Multi-Agent Learning?}
% \author{Anonymous Author(s)}
\author{Michael Amir\thanks{Equal contribution, listed alphabetically.} \quad Matteo Bettini$^{*}$ \quad Amanda Prorok \\
 Department of Computer Science and Technology\\
  University of Cambridge\\
  \texttt{\{ma2151,mb2389,asp45\}@cl.cam.ac.uk} \\
}

\blfootnote{\texttt{Supplementary website: \websiteurl }}

\begin{document}

\maketitle

\begin{figure}[!h] 
    \centering  % Centers the figure
    \includegraphics[width=0.8\textwidth]{images/nips_hero_gain.pdf}  % Adjust width as necessary, replace 'example-image' with your image file path
    \caption{We study and categorize what reward structures lead to the need for behavioral heterogeneity in multi-agent multi-task environments.}  % Add your figure caption
    \label{fig:hero}  % Add a label for referencing the figure in text
\end{figure}


\begin{abstract}
The success of teams in robotics, nature, and society often depends on the division of labor among diverse specialists; however, a principled explanation for \emph{when} such diversity surpasses a homogeneous team is still missing. Focusing on multi-agent task allocation problems, we study this question from the perspective of reward design: what kinds of objectives are best suited for heterogeneous teams? We first consider an instantaneous, non-spatial setting where the global reward is built by two generalized aggregation operators: an \emph{inner} operator that maps the \(N\) agents’ effort allocations on individual tasks to a task score, and an \emph{outer} operator that merges the \(M\) task scores into the global team reward. We prove that the curvature of these operators determines whether heterogeneity can increase reward, and that for broad reward families this collapses to a simple convexity test. Next, we ask what incentivizes heterogeneity to \textit{emerge} when embodied, time-extended agents must \emph{learn} an effort allocation policy. To study heterogeneity in such settings, we use multi-agent reinforcement learning (MARL) as our computational paradigm, and introduce \emph{Heterogeneity Gain Parameter Search (HetGPS)}, a gradient-based algorithm that optimizes the parameter space of underspecified MARL environments to find scenarios where heterogeneity is advantageous.  Across different environments, we show that HetGPS rediscovers the reward regimes predicted by our theory to maximize the advantage of heterogeneity, both validating HetGPS and connecting our theoretical insights to reward design in MARL. Together, these results help us understand when behavioral diversity delivers a measurable benefit.
\end{abstract}

% \newpage


% \begin{abstract}
% While behavioral diversity is often considered advantageous in multi-agent reinforcement learning (MARL), particularly for cooperative settings requiring role specialization, its practical implementation can introduce significant computational overhead that hinders learning. 
% Therefore, understanding which types of tasks genuinely necessitate and benefit from behavioral heterogeneity is crucial for efficient MARL design. 
% We investigate this question from the lens of \textit{co-design}, investigating which aspects of a task or environment can be tuned to either create or eliminate an advantage for heterogeneous agents. 
% We focus on multi-task environments where the reward structure depends on agents' effort allocation. We model the global reward as the sum of two generalized aggregators: 
% an \textit{inner} aggregator that maps agent efforts within each task to a task score, and an \textit{outer} aggregator that combines the resulting task scores. We then prove that the \textit{curvature}--specifically the \textit{Schur-concavity} or \textit{Schur-convexity}--of these aggregation functions determines the performance advantage of heterogeneous agents over homogeneous ones. For broad classes of reward structures, these conditions simplify to simple convexity/concavity tests, potentially offering researchers insights into when to leverage agent heterogeneity. For scenarios not covered by our theoretical analysis, we introduce \emph{Heterogeneity Gain Parameter Search (HetGPS)}, an algorithm that optimizes MARL environment parameters in underspecified tasks to identify settings where behavioral heterogeneity is advantageous.  Experiments in embodied \capturetheflag and matrix games validate our theory and demonstrate HetGPS as a valuable tool for studying the relation between agent diversity and task performance.
% \end{abstract} 

% \newpage

\section{Introduction}
\label{sec:intro}
\section{Introduction}

One of the most fundamental problems in combinatorial optimization is the traveling salesperson problem (TSP), formalized as early as 1832 (c.f. \cite[Ch 1]{ABCC07}).
In an instance of  TSP we are given a set of $n$ cities $V$ along with their pairwise symmetric distances, $c:V\times V \to\R_{\geq 0}$. The goal is to find a Hamiltonian cycle of minimum cost. In the metric TSP problem, which we study here, the distances satisfy the triangle inequality. Therefore, the problem is equivalent to finding a closed Eulerian connected walk of minimum cost.%\footnote{Given such an Eulerian cycle, we can use the triangle inequality to shortcut vertices visited more than once to get a Hamiltonian cycle.}

It is NP-hard to approximate TSP within a factor of $\frac{123}{122}$ \cite{KLS15}.  An algorithm of Christofides-Serdyukov~\cite{Chr76,Ser78} from four decades ago gives a $\frac32$-approximation for TSP.
Over the years there have been numerous attempts to improve the Christofides-Serdyukov algorithm and exciting progress has been made for various special cases of metric TSP, e.g., \cite{OSS11,MS11,Muc12,SV12,HNR21, KKO20, HN19, GLLM21}.
 Recently, ~\cite{KKO21} gave the first improvement for the general case by demonstrating that the so-called ``max entropy" algorithm of \cite{OSS11} gives a randomized $\frac{3}{2}-\epsilon$ approximation for some $\epsilon > 10^{-36}$.% (see \cite{VS20} for a historical note about TSP)

%After a long line of work %~\cite{Wol80,SW90,BP91,Goe95,CV00,GLS05,BM10,BC11,SWV12, HNR17,HN19, KKO20a} 
	%the best known approximation algorithm for the general case of the problem is $\frac{3}{2}-\epsilon$ for some $\epsilon > 10^{-36}$ due to ~\cite{KKO21}, a result that built upon the work of the third author, Saberi, and Singh ~\cite{OSS11}. 
	The method introduced in \cite{KKO21} exploits the optimum solution to the following linear programming relaxation of metric TSP studied by \cite{DFJ59,HK70,BG93}, also known as the subtour elimination LP:
\begin{equation}\label{eq:tsplp}
\begin{aligned}
	\min \quad& \sum_{u,v} x_{\{u,v\}} c(u,v)& \\
	\text{s.t.,} \quad &  \sum_{u} x_{\{u,v\}} = 2&\forall v\in V,\\
	& \sum_{u\in S, v\notin S} x_{\{u,v\}}\geq 2,&\forall S \subsetneq V, S\not= \emptyset\\
	& x_{\{u,v\}}\geq 0 &\forall u,v\in V.
\end{aligned}	
\end{equation} 
	
	 However, ~\cite{KKO21} did not show that the integrality gap of the subtour elimination polytope is bounded below $\frac{3}{2}$, and therefore did not make progress towards the ``4/3 conjecture" which posits that the integrality gap of LP \eqref{eq:tsplp} is $\frac{4}{3}$. In this work we remedy this discrepancy by proving the following theorem, improving upon the bound of $\frac{3}{2}$ from Wolsey~\cite{Wol80} in 1980:

\begin{theorem}\label{thm:main}
	Let $x$ be a solution to LP \eqref{eq:tsplp} for a TSP instance. For some absolute constant $\epsilon > 10^{-36}$, the \hyperlink{tar:alg}{max entropy algorithm} outputs a TSP tour with expected cost at most $\frac{3}{2}-\epsilon$ times the cost of $x$. Therefore the integrality gap of the subtour elimination LP is at most $\frac{3}{2} - \epsilon$. 
\end{theorem} 

To prove \cref{thm:main}, we amend Section 4 of \cite{KKO21} but keep the remainder of the analysis essentially the same. Unlike \cite{KKO21}, this argument now preserves the integrality gap by avoiding the use of the optimum solution in bounding the cost of the matching. See \cref{sec:overview} for a discussion of our new approach.
%We note that the analysis in this paper is not specialized to the max entropy algorithm (although we rely on many results from \cite{KKO21} to obtain \cref{thm:main} itself); instead, it is valid for any algorithm which samples a spanning tree from the support of a solution to LP \eqref{eq:tsplp} and then adds the minimum cost matching on the odd degree vertices of the tree.  
%Instead, we use the polygon representation of near minimum cuts \cite{Ben95,BG08} to bound  the cost of the matching (see the following section for an overview of our new findings). %An added benefit of avoiding the use of OPT in the analysis is  %We remark this makes the analysis constructive 
%We remark that this allows future analyses to explicitly compute and possibly utilize the relevant laminar family of near minimum cuts (whereas previously one needed to know OPT to find the laminar family used in the analysis in \cite{KKO21}).
%In particular, we show that to get a bound better than $\frac{3}{2}$ for this class of algorithm it is (essentially) sufficient to handle the case in which the near minimum cuts of $x$ are a laminar family.

\subsection{Other Consequences}
\paragraph{Path TSP} In recent exciting work, Traub, Vygen, Zenklusen \cite{TVZ20} showed that an $\alpha$-approximation algorithm for metric TSP can be used as a black box to get a $\alpha(1+\eps)$ approximation algorithm for Path TSP. This together with \cite{KKO21} implies that there is a $3/2-\eps$ approximation algorithm for Path TSP (for $\eps>10^{-36}$). On the other hand, it is a folklore result that the integrality gap of the natural LP relaxation of Path TSP is at least $3/2$.  Therefore, a consequence of the above theorem is that although the best possible approximation factors of the two problem are the same (up to polynomial reductions), the natural LP relaxation of metric TSP has a strictly smaller integrality gap.


\paragraph{2-ECSM} In the 2-edge-connected multi-subgraph problem, or 2-ECSM for short, we are given a weighted graph $G$ and we want to find a minimum cost 2-edge-connected spanning subgraph, where an edge can be chosen multiple times.
The classical Christofides-Serdyukov algorithm gives a 3/2-approximation for 2-ECSM and despite significant attempts \cite{CR98,BFS16,SV14,BCCGISW20} improved algorithms were designed only for special cases of the problem.
Since in \cite{BG93} it is shown that LP \eqref{eq:tsplp} is a valid relaxation for 2-ECSM, we obtain:

\begin{corollary}	
For some absolute constant $\epsilon > 10^{-36}$ the \hyperlink{tar:alg}{max entropy algorithm} is a randomized $\frac{3}{2}-\epsilon$ approximation for the 2-edge-connected multi-subgraph problem.
\end{corollary}
%Beyond these theorems, we believe the analysis in this paper will open new avenues to improve the arguments in ~\cite{KKO21}. The analysis in that work is by nature non-constructive because it uses information about the optimal solution. Here we remove this weakness and could in principle construct the proposed fractional matching in polynomial time. Although of course this has no practical benefit since the algorithm always finds the minimum cost matching, this may allow future works to manipulate the algorithm to better serve the analysis.

%We analyze the max-entropy rounding algorithm introduced in \cite{OSS11} and slightly modified in \cite{KKO20, KKO21}. 

%In other words, we design a feasible vector for the $O$-join polytope to ``satisfy'' all near min cuts ``crossed on both  sides'' 


%Whereas Section 4 of ~\cite{KKO21} only deals with the near minimum cuts of $x$ (where $x$ is a solution to LP \eqref{eq:tsplp}) which lie along the optimal Hamiltonian cycle, we deal with all near minimum cuts of $x$ using the so-called polygon representation of near minimum cuts ~\cite{Ben97,BG08}. %The results give new intuition for the structure of cuts that are within $\frac{6}{5}$ or less of the edge connectivity of the graph.

 %: we show that we can incur a cost of $O(\eta^2) \cdot c(x)$ to ensure that the set of cuts with $x(\delta(S)) \le 2+\eta$ is a laminar family.


\subsection{New techniques and contributions}\label{sub:newtechniques}

This paper can be seen as a case study on how to reason about and deal with {\em near} minimum cuts. One can deduce from the classical cactus representation of a graph $G$ \cite{DKL76} (i) the structure of {\em all} min cuts of $G$ and (ii) the structure of the edges of $G$ in the sense that every edge $\{u,v\}$ maps to a unique {\em path} in the cactus between the images of $u$ and $v$. Furthermore, such a path intersects every cycle of the cactus on at most one cactus edge. The theory has found many application from designing fast algorithms
\cite{Kar00,KP09} to the analysis of approximation algorithms for TSP \cite{KKO20} and connectivity augmentation \cite{BGJ20,CTZ21}.

Two decades later, the theory of min cuts was extended to near min cuts in works of Bencz\'ur and Goemans \cite{Ben95, BG08} where they introduced the polygon representation which represents all cuts of a graph with at most $\frac{6}{5}k$ edges, where $k$ is its edge connectivity. Although these works completely classify the structure of all near min cuts of a given graph $G$, they do not characterize the structure of the \textit{edges} of $G$ with respect to these cuts, which can be important in applications (for example, in many of the recent applications of min cuts,
 one also needs to exploit the structure of the edges in relation to the cactus).
The structure on the edges turns out to be highly relevant in this work as well, and as a byproduct of our analysis we make progress towards classifying the way in which the edges of $G$ relate to the structure of the polygon representation.
 
 % and (to some extent) a classification of the set of edges of $G$ with respect to the polygon representation of Bencz\'ur and Goemans.
 
  %i
 %s to give a better understanding of the structure of edges of $G$ with respect to its near min cuts.

  %One can partition the edges of $G$ into sets $F_1\dots,F_m$ such that the set of edges in every min cut $(S,\overline{S})$ of $G$ is the union of edges in a pair $F_i,F_j$ for $i\ neq j$.
%\Nathan{Shayan can add something} For example...

For motivation, consider a generic family of network design problems in which we want to construct a network such that every pair $u,v$ of vertices has connectivity at least $c_{u,v}$. A natural approach is to write an LP relaxation to find a (minimum cost) vector $x: E \to \R_{\ge 0}$ such that for every cut $S$ separating $u$ and $v$, $x(\delta(S))\geq c_{u,v}$. We can round this LP using independent rounding or a dependent rounding scheme such as sampling from max entropy distributions. Using classical concentration bounds one can show that if $x(\delta(S))\gg c_{u,v}$ then with high probability the rounded solution has at least $c_{u,v}$ edges across this cut. So the main challenge is to ``fix'' near tight cuts, i.e., cuts where $x(\delta(S))\approx c_{u,v}$.  For an explicit instantiation of this scheme see \cite{KKOZ22}. A better understanding of the global structure of the family of near tight cuts has the potential to significantly simplify or even improve the approximation factor of such rounding algorithms. A classical technique to design algorithms for such network design problems is to apply uncrossing to extreme point solutions of the LP. One can view our contribution as an approximate uncrossing technique that deals with all near tight cuts (instead of just tight cuts) as we explain next.
%Next, we explain how our main theorem can be used to give global structure for near tight cuts in the case that $c_{u,v}=2$ for all $u,v$ and we contrast it with the classical uncrossing technique which only deals with tight/min cuts. 


\paragraph{An Approximate Uncrossing Technique.} A fundamental technique in the field of approximation algorithms is the uncrossing technique\footnote{See e.g. \cite{LRS11} for a number of applications of this technique.} of Jain \cite{Jai01}. Given a graph $G=(V,E)$,  a weight vector $x:E\to\R_{\geq 0}$, and a  function $f:V\to\R$, suppose that $x(\delta(S))\geq f(S)$ for all $S\subseteq V$. Let $\cN$ be the family of sets $S$ such that $x(\delta(S)) = f(S)$, i.e., the family of {\em tight} sets with respect to $f$. The uncrossing technique says that if $f$ is (weakly) supermodular then we can refine $\cN$ to a laminar family of sets, $\cH$, such that if all sets of $\cH$ are tight, then all sets of $\cN$ are tight as well. For a concrete example, suppose $f$ is a constant function, say $f(S)=2$ for all $\emptyset\subsetneq S\subsetneq V$. Then, sets of $\cH$ can be constructed using the cactus representation \cite{DKL76} of cuts in $\cN$. The significance of this method is that if $x$ is a basic feasible solution to a LP with constraints $x(\delta(S))\geq f(S)$ for all $S$, one can use this machinery to argue that the support of $x$ has size $O(|V|)$.

Informally, we prove the following, which 
can be seen as  an {\em approximate uncrossing technique}: 
\begin{theorem}[Informal]\label{thm:uncrossing}Suppose we have a vector $x:E\to\R_{\geq 0}$ such that $x(\delta(S))\geq f(S)$ for all $S$; define $\cN$ to be sets $S$ where $x(\delta(S))\leq f(S)(1+\eps)$ for some fixed $\eps>0$. If $f(.)$ is constant, say $f(S)=2$ for all $S$, then there is a set $\cN^*\subseteq \cN$ and a collection of edge sets $F_1,\dots,F_m\subseteq E$ such that the following hold:
\begin{itemize}
	\item $|\cN^*|= O(|V|)$, $m= O(|V|)$.
	\item $x(F_i)\geq 1-\eps/2$ for all $1\leq i\leq m$.
	\item Every edge $e$ is in at most $O(1)$ of the $F_i$'s.
	\item For every set $S\in \cN\smallsetminus \cN^*$ there exists $1\leq i<j\leq m$ such that $F_i\cap F_j=\emptyset$ and $F_i\cup F_j\subseteq \delta(S)$ and for every $S\in \cN^*$, there exists $1\leq i\leq m$ such that $F_i\subseteq \delta(S)$. 
\end{itemize}
\end{theorem}
In words, although we cannot simply refine $\cN$ to a linear number of sets, we can refine the edges in cuts of $\cN$ to a linear number of sets $F_1,\dots, F_m$ such  that we can essentially capture the edges of $\delta(S)$ for any $S\in \cN\smallsetminus \cN^*$ by a pair of disjoint $F_i$'s. We give a slightly weaker condition for cuts in $\cN^*$; namely we only capture half of their edges by $F_i$'s.

\begin{example}For a simple example of the above theorem, suppose $\eps=0$, i.e. $\cN$ is the set of min cuts of a graph $G$. Furthermore, suppose that every proper  cut in $\cN$ is \hyperlink{tar:crossing}{crossed} (recall that $S$ is proper if $1<|S|<|V|-1$) and that $\cN$ has at least one proper cut. 
Then, one can use an uncrossing technique, namely that if $A,B\in \cN$ then $A\cap B\in \cN$, to prove that $G$ must be cycle, namely we can order vertices of $G$, $v_0,\dots,v_{n-1}$ such that $x_{\{v_i,v_{i+1\text{ mod n}}\}}=1$.
In such a case we let $\cN^*=\emptyset$ and $F_i=E(v_i,v_{i+1\text{ mod }n})$.
%partition $V$ into sets $a_0,\dots,a_{m-1}$ such that 
%Let $\C$ be a connected component of crossing cuts of $\cN$, namely, for any pair of sets $A,B\in \C$ there is a path of crossing cuts all from $\C$ that goes from $A$ to $B$.
% and further suppose that $\cN$ can be represented by a cycle $C$ in the sense every min cut of $\cN$ corresponds to a min cut of $C$ and vice versa. Here we assume $a_0,\dots,a_{m-1}$ are the nodes of $C$ where each $a_i$ is identified with a disjoint set of vertices where $V=\uplus_{i=1}^m a_i$. In such a case, we can simply let $\cN^*=\emptyset$ and $F_i=E(a_i,a_{i+1\text{ mod }m})$. 
\label{eg:cycle}\end{example}

\begin{example}\label{eg:laminar}
For a second example, suppose again $\eps=0$ and $\cN$ is the set of mincuts of a graph $G$ where $\cN$ forms a laminar family (no two cuts cross). It turns out that we cannot decompose edges of cuts of $\cN$ into a linear sized collection of sets where every edge appears only a constant number of times. The main reason is that some edges may appear in an unbounded number of cuts. In this case we let $\cN^*=\cN$ and for every $A\in \cN$ (with immediate parent $B\in \cN$ in the laminar family) we add a set $F_A=\delta(A)\smallsetminus \delta(B)$  to our collection.  It is straightforward to show, using the structure of min cuts, that $x(F_A)\geq 1$; furthermore, since the size of a laminar family is linear in $V$, this gives a valid decomposition in the sense of above theorem.
\end{example}
Lastly, if $\eps=0$ and $\cN$ is the set of min cuts of an arbitrary graph, one can represent all min cuts of $\cN$ by a cactus \cite{DKL76} which can be seen as a tree of cycles. In such a case, one can use a construction similar to \cref{eg:cycle} for each cycle where instead of a vertex $v_i$ we have a set $a_i \subseteq V$ and one similar to \cref{eg:laminar} for the tree part of the cactus. For a concrete application of such a decomposition of min cuts see \cite{KKO20}.
%More generally, if $\cN$ corresponds to the set of min cuts of an arbitrary graph, the cuts of $\cN$ can be represented by a {\em cactus graph}. In such a case we add one $F_i$ for every edge of a cycle of the cactus. 


%and further for simplicity assume that there is a single connected component of crossing cuts in $\cN$, namely we can go from any $A$ to $B$ for $A,B\in\cN$ simply following crossing cuts of $\cN$. Then, one can represent cuts in $\cN$ by the set of min cuts of a cycle, namely we can contract vertices of $G$ 

%For a concrete application , suppose we need at least two edges in every set in $\cN^*$, say in a network optimization problem. Then, if we make sure that we have at least one edge in each $F_i$, all typical cuts constraints, $\cN\smallsetminus \cN^*$,  are satisfied, so we  reduce the problem to cuts in $\cN^*$. 


One of the main challenges in dealing with near min cuts relative to min cuts is that if $x(\delta(A)),x(\delta(B))\leq 2+\eps$ then $x(\delta(A\cap B))\leq 2+2\eps$. Therefore, if $\eps=0$, then min cuts are closed under intersection, set difference and union, but this is no longer true when $\eps>0$. So, to employ the classical uncrossing machinery one should be very careful to "uncross" only a constant number of times (independent of $\eps$) to make sure that every cut remains within $2+O(\eps)$. This is the main reason that the polygon representation of near min cuts (see below) is more sophisticated, e.g., we can no longer argue $x(E(a_i, a_{i+1}))\approx 1$, see \cref{fig:nearmincutbadexample}.

Although we don't study it here, we believe it may be worthwhile to find generalizations of \cref{thm:uncrossing} which hold for any (weakly) supermodular function.% That could be helpful in many questions based on the network optimization framework of Jain \cite{Jai01}.

\begin{remark} 
 We do not explicitly prove \cref{thm:uncrossing} in this extended abstract, as it is not used to prove \cref{thm:main}. However it can be deduced from arguments in \cref{sec:twoside} and \cref{app:oneside}. 
%In \cref{sec:overview} we discuss the main ideas of the proof of \cref{thm:uncrossing}. Here, let us explain the main challenge: In principal one might try to simply extend the above decomposition for the case $\eps=0$. The main challenge is that if $x(\delta(A)),x(\delta(B))\leq 2+\eps$ then $x(\delta(A\cap B))\leq 2+2\eps$. Therefore, if $\eps=0$, then min cuts are closed under intersection, set difference and union, but this is no longer true when $\eps>0$. So, to employ the classical uncrossing machinery one should be very careful to "uncross" only a constant number of times (independent of $\eps$) to make sure that every cut remains within $2+O(\eps)$. This is the main reason that the polygon representation of near min cuts (see below) is more sophisticated, e.g., we can no longer argue $x(E(a_i, a_{i+1}))\approx 1$, see \cref{fig:nearmincutbadexample}.
\end{remark}





\paragraph{Extensions to the Polygon Representation} To obtain our uncrossing framework we prove new properties of the polygon representation.
Given a graph $G=(V,E)$, let $k$ be the edge-connectivity of $G$, i.e. the number of edges in a minimum cut of $G$. For $\eps>0$, consider the set of $(1+\eps)$-near minimum cuts of $G$: cuts $(S,\overline{S})$ where $|E(S,\overline{S})| < (1+\eps)k$.
Bencz\'ur \cite{Ben95} and Bencz\'ur, Goemans \cite{BG08} proved that if $\eps \le 1/5$ then the near minimum cuts of $G$ admit a {\em polygon representation}. Namely, every connected component $\cC$ of \hyperlink{tar:crossing}{crossing} $(1+\eps)$ near min cuts can be represented by the diagonals of a convex polygon. In this polygon, the vertices of $G$ are partitioned into sets called \textit{atoms}, and every atom is mapped to a cell of this polygon defined by the diagonals and the boundary of the polygon itself (see \cref{sec:polyrep} for more details). 

The polygon representation can be seen as a generalization of the well-known cactus representation \cite{DKL76} of minimum cuts where a cycle of the cactus is replaced by a convex polygon. Unlike a cycle, some vertices/atoms map to the interior of the polygon, which are called ``inside'' atoms. The inside atoms at first look like a mystery and one can ask many questions about them such as how many can exist and what structures they can exhibit.



 Here, we explain two lemmas we proved which might find further applications beyond TSP in the future. 
%Our results give new intuition and understanding about the structure of polygon representations. These guide our analysis of the integrality gap of the subtour LP.
 %For example, one of our new observations is a 
 First, we give a necessary condition for a cell of a polygon to contain an inside atom:
\begin{lemma}[Informal, see \cref{thm:halfplanes}]
	Consider a polygon $P$ for a connected component $\C$ of a family of $1+\eps$ near min cuts for $\eps \le 1/5$ (where representing diagonals correspond to cuts in $\C$). Any cell of $P$ that has an inside atom must have at least $\Omega(1/\eps)$ many sides. 
\end{lemma}
This can be seen as a generalization of \cite[Lem 22]{BG08} to the case in which the cell is allowed to be adjacent to vertices of the polygon $P$.

Now, we explain our second extension: it follows from the cactus representation of minimum cuts that for a graph $G$ and a min cut $S$ one can partition the set of all min cuts that cross $S$ into two groups ${\cal A}=\{A_1,\dots,A_k\}$ and ${\cal B}=\{B_1,\dots,B_l\}$ for some $k,l\geq 0$ such that $S\cap A_1\subseteq S\cap A_2 \subseteq \dots S\cap A_k$ and, similarly, $S\cap B_1\subseteq \dots\subseteq S\cap B_l$. We prove a generalization of this fact for near min cuts:
\begin{lemma}[Informal, see \cref{lem:crosschain}]
Consider the set of $1+\eps$ near min cuts of a graph $G$ for $\eps\leq 1/10$; for any such near min cut $S$, one can partition the $1+\eps$ near min cuts crossing $S$ into two groups ${\cal A}=\{A_1,\dots,A_k\}$ and ${\cal B}=\{B_1,\dots,B_l\}$ such that $S\cap A_1 \subseteq S\cap A_2\subseteq \dots \subseteq S\cap A_k$ and similarly for cuts in ${\cal B}$.
\end{lemma}

\subsection{Outline of rest of paper} After reviewing preliminaries in \cref{sec:prelims}, we give a high-level overview of our proof technique in \cref{sec:overview}. The main new technical contributions of this paper are in \cref{sec:polyrep} and  \cref{sec:twoside}. The remaining content of the paper essentially follows from ~\cite{KKO21}. %Therefore, the reader may want to skip \cref{sec:proof-of-main}. 




\subsection{Related Works}
\section{Other Related Work}
\label{sec:related}
There has been a renewed interest in integrated task and motion planning algorithms. Most research in this direction has been focused on deterministic environments~\citep{cambon09_asymov,plaku10_sampling,dornhege12_semantic,kaelbling11_hierarchical,garrett15_ffrob,dantam16_incremental}. \cite{kaelbling13_hpnPOMDP} consider  a partially observable formulation of the problem. Their approach utilizes regression modules on belief fluents to develop a regression-based solution algorithm. \cite{sucan12_tmp_mdp} use an explicit multigraph to represent the plan or policy for which motion planning refinements are desired.  \cite{hadfield15_modular} address problems where the high-level formulation is deterministic and the low-level is determinized using most likely observations. In contrast, our approach employs abstraction to bridge MDP solvers and motion planners to solve problems where the high-level model is stochastic. In addition, the transitions in our MDP formulation depend on properties of the refined motion planning trajectories (e.g., battery usage). 

Principles of abstraction in MDPs have been well studied~\citep{hostetler14_state,bai16_markovian,li06_abstractMDP,singh95_abstractRL}. However, these directions of work assume that the full, unabstracted MDP can be efficiently expressed as a discrete MDP. \cite{marecki06_cmdp} consider continuous time MDPs with finite sets of states and actions. In contrast, our focus is on MDPs with high-dimensional, uncountable state and action spaces. Recent work on deep reinforcement learning  (e.g., \citep{hausknecht16_iclr,mnih15_drl}) presents  approaches for using deep neural networks in conjunction  with reinforcement learning to solve MDPs with continuous state spaces. We believe that these approaches can be used in a complementary fashion with our proposed approach. They could be used to learn maneuvers spanning shorter-time horizons, while our approach could be used to efficiently abstract their representations and to use them as actions or macros in longer-horizon tasks. 

Efforts towards improved representation languages are orthogonal to our contributions~\citep{fox02_pddl+}. The fundamental computational complexity results indicating growth in complexity with increasing sizes of state spaces, branching factors, and time horizons remain true regardless of the solution approach taken. It is unlikely that a uniformly precise model, a simulator at the level of precision of individual atoms, or even circuit diagrams of every component used by the agent will help it solve the kind of complex tasks on which humans would appreciate assistance. On the other hand, not using any model at all would result in dangerous agents that would not be able to safely evaluate the possible outcomes of their actions. Our results show that these divides can be bridged using hierarchical modeling and solution approaches that simplify the representational requirements and offer computational advantages that could make autonomous robots feasible in the real world. 



\section{Problem Setting}
\label{sec:problemsetting}
Consider a set of \(N\) agents and \(M\) tasks. Each agent \( i \in \{1,\ldots,N\} \)
allocates \textit{effort} among the tasks according to the budget constraints:
\(
r_{i1}, r_{i2}, \ldots, r_{iM} \geq 0 
 \text{ with }  
\sum_{j=1}^{M} r_{ij} \leq 1
\), where $r_{ij}$ is \textit{defined} as the effort agent $i$ puts into task $j$. Here, ``effort'' $r_{ij}$ is a scalar input to the reward function representing the agent's contribution to the task, such as resource allocation (\autoref{appendix:examples_of_marl_environments}) or realized goal proximity (\autoref{sec:experiments}).
%Here, ``effort'' is an abstract quantity that can represent various things such as resource allocation or goal proximity. 
We can consider both continuous allocations ($r_{ij}$ can be any real number) and discrete allocations ($r_{ij}$ restricted to some finite set of options), with most results in this work focusing on the continuous case. We collect all
agents' allocations into an \(N \times M\) matrix:
\(
\mathbf{A} 
= 
[r_{ij}]
\)\footnote{All results in this work can be extended to the case where \(
r_{i1}, r_{i2}, \ldots, r_{iM} \geq B_{min} 
 \text{ and }  
\sum_{j=1}^{M} r_{ij} \leq B_{max}
\) for some arbitrary $B_{min}, B_{max} \in \mathbb{R}$.}.

For each task \(j\) let the \(j\)-th column of the effort matrix be \( \mathbf a_j=[r_{1j},\dots,r_{Nj}]^\top\).  A \emph{task-level aggregator} \(T_j:\R^{N}\!\to\!\R\) maps these efforts to a \textit{task score}, and an \emph{outer aggregator} \(U:\R^{M}\!\to\!\R\) combines the \(M\) scores into the team reward, \(R(\mathbf A)=U\bigl(T_1(\mathbf a_1),\dots,T_M(\mathbf a_M)\bigr)\).  Both \(T_j\) and \(U\) are \emph{generalised aggregators}: symmetric and coordinate-wise non-decreasing, mirroring the familiar properties of \(\sum\). When every task shares the same inner aggregator we simply drop the subscript and write \(T\). To highlight the analogy with a double sum we also write \(R(\mathbf A)=\outeragg_{j=1}^{M}\inneragg_{i=1}^{N} r_{ij}\), where (in abuse of notation) the outer symbol \(\outeragg\) denotes \(U\) and the inner symbol \(\inneragg\) denotes \(T_j\).  



\headline{Homogeneous vs. Heterogeneous Strategies} A \emph{homogeneous strategy} is one where all agents have the same allocation (i.e., devote the same amount of effort to a given task $j$):
\(
r_{ij} = c_j 
 \forall\, i, j
\). In this case, the allocation matrix \(\mathbf{A}\) consists of identical rows. We define
\(
R_{\mathrm{hom}}
=
\max_{(c_1,\ldots,c_M) \in \Delta_{\!\le}^{M-1}} 
\;
R\Bigl(\mathbf{A}\Bigr)
\)
where \( \Delta_{\!\le}^{M-1} = \{(c_1,\ldots,c_M)\mid c_j\ge 0,\; \sum_j c_j \leq 1\}\) is the closed unit simplex. A \emph{heterogeneous strategy} allows each agent \( i \) to choose any 
\((r_{i1}, \ldots, r_{iM}) \in \Delta_{\!\le}^{M-1}\) independently. Then
\(
R_{\mathrm{het}}
=
\max_{\mathbf{A} \in (\Delta_{\!\le}^{M-1})^N}
\;
R\Bigl(\mathbf{A}\Bigr).
\)
%
We define the \textit{\heterogeneitygap} as:
\(
\Delta R = R_{\mathrm{het}} - R_{\mathrm{hom}}.
\)
This quantity measures how much greater the overall reward can be when agents are
allowed to specialize differently across tasks, compared to when they must behave identically.  Characterizing when \(\Delta R > 0\) is our main focus in this work. 

\headline{MARL extension}
In MARL, the effort value $r_{ij}$ represents the contribution of agent $i$ to task $j$ \textit{as computed by the environment based on agent $i$'s actions}. The aggregate reward $R(A)$ can represent:
%In MARL, the value $r_{ij}$ represents the `local rewards' agents receive based on task performance, and the aggregate reward \(R(\mathbf{A})\) can represent:
(i) the payoff of a one-shot effort-allocation game, (ii) the return or sparse terminal reward of an episode, or (iii) the stepwise reward, giving the discounted return \(\sum_{t= 0}^T\gamma^{t}\,R\bigl(\mathbf{A}_{t}\bigr)\) for a sequence \(\bigl(\mathbf{A}_{t}\bigr)_{t=1,\ldots T}\) of allocations\footnote{To extend this further, our theoretical results hold even if the reward function varies over time, $R_t(A_t)$.}. \(\Delta R>0\) implies that the best heterogeneous policies outperform the best homogeneous ones.
In practice, this is \textit{evidence of} an advantage~to~heterogeneity and not~a formal~guarantee, as learning agents may not always converge to optimal policies.

% \emph{if} the heterogeneous agents can always attain the optimal allocation; if not (as in some \autoref{sec:experiments} experiments), this is \textit{evidence of} an advantage~to~heterogeneity--not~a formal~guarantee.

\headline{Examples} \autoref{appendix:parametrized_aggregator_table} contains examples of generalized aggregators. Our framework is flexible, and can be applied to many settings, including ones not ordinarily thought of as ``task allocation'': in \autoref{sec:experiments}, we apply it to one-shot allocation games, multi-agent navigation, tag, and football. Furthermore, in \autoref{appendix:examples_of_marl_environments}, we analyze the \heterogeneitygap of two well-known environments from the literature: Team Colonel Blotto games \citep{noel2022reinforcementcolonelblotto} and level-based foraging \citep{papoudakis2021benchmarking_multilevelforaging}.



\section{Analysis}


Focusing on continuous allocations, we ask what properties of aggregators guarantee $\Delta R > 0$. We draw  on the concept of Schur-convexity. Schur-convex functions can be understood as generalizing symmetric, convex aggregators: every convex and symmetric function is Schur-convex, but a Schur-convex function is not necessarily convex \citep{roberts1974convex, peajcariaac1992convex}. Proofs for all results are available in \autoref{appendix:formal_analysis}.

Since both the outer aggregator $U$ and the task-level aggregators $T_j$ are non-decreasing, an optimal effort allocation will always have each agents' efforts summing to $1$. Hence, from here on, we \textbf{assume} without loss of generality that $\sum_{j=1}^{M} r_{ij} = 1$. We call such allocations \textbf{admissible}. 


\begin{definition}[Majorization]
\label{def:majorization}
Let \(x=(x_1,\dots,x_N)\) and \(y=(y_1,\dots,y_N)\) be two vectors in \(\mathbb{R}^N\) such that $\sum_{i=1}^N x_{(i)} = \sum_{i=1}^N y_{(i)}$. Let \(x_{(1)} \ge x_{(2)} \ge \cdots \ge x_{(N)}\) and \(y_{(1)} \ge y_{(2)} \ge \cdots \ge y_{(N)}\) be the components of \(x\) and \(y\) sorted in descending order. We say that \(x\) \emph{majorizes} \(y\) (written \(x \succ y\)) if
\(
\sum_{i=1}^k x_{(i)} \ge \sum_{i=1}^k y_{(i)}\) for \( k = 1,2,\dots, N-1,N.
\)
\end{definition}

% Occasionally, the relation in Definition \ref{def:majorization} is called ``weak majorization'', with majorization further requiring $\sum_{i=1}^N x_{(i)} = \sum_{i=1}^N y_{(i)}$. In this work, we do not assume equality of sums.

\begin{definition}[Schur-Convex Function]
A symmetric function \(f:\mathbb{R}^N \to \mathbb{R}\) is \emph{Schur-convex} if for any two vectors \(x,y \in \mathbb{R}^N\) with \(x \succ y\), we have
\(
f(x) \ge f(y).
\)
If the inequality is strict whenever \(x\) and \(y\) are not permutations of each other, then \(f\) is said to be \emph{strictly Schur-convex}. Similarly, \(f\) is \emph{Schur-concave} if \(f(x) \le f(y)\) whenever \(x \succ y\). 
\end{definition}

Intuitively, $x \succ y$ means one can obtain $y$ from $x$ by repeatedly moving mass from larger to smaller coordinates, thereby making the vector more uniform. Schur-convexity is then a statement on a function's  \textit{curvature}: $f$ is \emph{Schur-convex} if it increases with inequality, or is \emph{Schur-concave} if it increases with uniformity. We show here a connection between Schur-convexity (concavity) and $\Delta R$.

Call an allocation matrix $\mathbf A$ \emph{trivial} if there exists a task $j^\star$ such that every agent allocates its entire budget to that task, i.e.\ $r_{ij^\star}=B_{\max}$ and $r_{ij}=0\ \forall i,\; \forall j\neq j^\star$; otherwise $A$ is \emph{non-trivial}. Then:

\begin{theorem}[Positive \HeterogeneityGap via Schur-convex Inner Aggregators]
\label{thm:heterogeneity-gap-schurconvex}
% Let $N,M \ge 2$, and consider a reward function
% \[
%   R(\mathbf{A})
%   \;=\;
%   U\bigl( T_1(\mathbf{a}_1),\dots,T_M(\mathbf{a}_M) \bigr),
%   \quad
%   \text{where each } \mathbf{a}_j = \bigl(r_{1j},\dots,r_{Nj}\bigr)^\top.
% \]
Let $N,M \ge 2$, and assume that (i) each \emph{task‐level aggregator} $T_j$ is strictly Schur-convex and (ii) the \emph{outer aggregator} $U$ is coordinate-wise strictly increasing.  Then either all admissible optimal homogeneous allocations are trivial, or $\Delta R > 0$. 
% \begin{enumerate}[label=(\roman*)]
% \item 
% Each \emph{task‐level aggregator} $T_j$ is strictly Schur-convex. 
% \item 
% The \emph{outer aggregator} $U$ is strictly increasing in each coordinate.
% \end{enumerate}

% Furthermore, assume that in the best homogeneous solution, at least two tasks receive nonzero allocations. Then  
% \(
%   \Delta R
%   \;=\;
%   R_{\mathrm{het}} - R_{\mathrm{hom}}
%   \;>\;
%   0.
% \)
\end{theorem}

If the task-level aggregator is instead Schur-concave, we can show there is no \heterogeneitygap:

\begin{theorem}[No \HeterogeneityGap via Schur-concave Inner Aggregators]
\label{thm:heterogeneity-gap-schurconcave}
Let $N,M \ge 2$. If each task‐level aggregator $T_j$ is \emph{Schur-concave} then 
\(
\Delta R 
   \;=\;
   0.
\)
\end{theorem}


% We see then that Schur-convexity is a property of the inner aggregator that induces a positive \heterogeneitygap ($\Delta R > 0$), and Schur-concavity implies no gain. What about the outer aggregator $U$? Here, the notion of Schur-convexity or Schur-concavity is tricky to apply directly, since $U$ receives as its argument vectors of the form  $\bigl(T_1(\mathbf{a}_1) , \ldots T_M(\mathbf{a}_m) )$ for different effort allocations $\textbf{a}$, and $\sum_1^M T_i(\mathbf{a}_i)$ can vary, which means task scores are incomparable in terms of majorization. However, if we select our inner aggregators such that $\sum_1^M T_i(\mathbf{a}_i)$ always equals some constant $C$ (e.g., by normalizing the task-level rewards), we can investigate the relationship between the outer aggregator and heterogeneity from the lens of Schur-convexity. In particular, the relationship is the reverse of what it was for inner aggregators: when the outer aggregator is Schur-convex, the \heterogeneitygap is $0$. Let us prove this. 

We see that Schur-convexity of the inner aggregator produces \( \Delta R > 0 \), whereas Schur-concavity implies \( \Delta R = 0 \). Analyzing the outer aggregator \( U \) is trickier, because it acts on task-score vectors \(\bigl(T_{1}(\mathbf a_{1}), \dots, T_{M}(\mathbf a_{M})\bigr)\) whose sum \( \sum_{i=1}^{M} T_{i}(\mathbf a_{i}) \) may vary, so majorization is not directly applicable. However, we can extend our analysis to $U$ if our inner aggregators are \textit{normalized} to keep the sum constant: \(\sum_{i=1}^{M} T_{i}(\mathbf a_{i}) = C\) for any admissible allocation. Assuming this, we can  invoke majorization again, and the relationship between convexity and $\Delta R$ reverses: if the outer aggregator \( U \) is Schur-convex, the \heterogeneitygap vanishes. Let us prove this.

\begin{theorem}[No \HeterogeneityGap for Schur-Convex $U$ with Constant-Sum Task Scores]
\label{thm:no-gap-schurconvex-outer-detailed}
Let $N,M \ge 2$.  Suppose that for any admissible  allocation $\mathbf{A}$, (i) every task score is non-negative, and obeys $T_i(0, \ldots, 0) = 0$, and (ii)  the sum of task score is always
\(
  \sum_{j=1}^M T_j\bigl(\mathbf{a}_j\bigr) \;=\; C
\). If  $U$ is \emph{strictly Schur-convex} function, then 
\(
  \Delta R = 0.
\)
\end{theorem}

\textbf{Sum-Form Aggregators.} In \autoref{appendix:sumform_aggregators}, we show that the above results reduce to a simple convexity test for \textit{sum-form aggregators}: a broad class of aggregators that describes most reward structures we consider in this work. This makes testing whether $\Delta R > 0$ a simple computation in many cases.

\headline{Parameterizable Families of Aggregators} \label{sec:parametrizedaggregators}
A core topic of this work is \emph{reward design}: how can we craft team objectives that either advantage or disadvantage behavioral diversity?  To do this, it is helpful to first identify an appropriate search space. Our theoretical analysis enables us to narrow down this search space, and focus on aggregators whose \textit{curvature} can be parametrized.  Many \emph{family of aggregator functions} 
\(\{\,f_t(\cdot)\}_{t \in \mathbb{R}}\) can be parametrized by a scalar 
\(t\) which controls whether the aggregation is \emph{Schur‐convex} or 
\emph{Schur‐concave}, and how strongly it penalizes (or favors) inequalities among the components. For example, the \textit{softmax aggregator} $\sum_{i=1}^N 
\frac{\exp\bigl(t \cdot r_{i j}\bigr)}{\sum_{\ell=1}^N \exp\bigl(t \cdot r_{\ell j}\bigr)}$ is parametrized by its temperature, $t$, transitioning from being strictly Schur-concave when $t < 0$ to strictly Schur-convex when $t > 0$. We can define a space of reward functions by selecting both the task scores and outer aggregator to be softmax functions: let \(
T_j(\mathbf{A})
=
\sum_{i=1}^N 
\frac{\exp\bigl(t \cdot r_{i j}\bigr)}{\sum_{\ell=1}^N \exp\bigl(t \cdot r_{\ell j}\bigr)}
\; r_{i j},
\) and let \(
U\bigl(T_1(\mathbf{a}_1) , \ldots T_M(\mathbf{a}_m) )
=
\sum_{j=1}^M 
\frac{\exp\bigl(\tau \cdot T_j(\mathbf{A})\bigr)}
     {\sum_{\ell=1}^M \exp\bigl(\tau \cdot T_\ell(\mathbf{A})\bigr)}
\; T_j(\mathbf{A})
\), where $t, \tau \in \mathbb{R}$ parametrize the inner and outer aggregators, respectively. $\Delta R$ is then dependent on $t$ and $\tau$. \autoref{fig:deltaR-vs-softmax} plots $\Delta R$ when $N = M =  2$. As a case study, we derive lower bounds on $\Delta R$ when $N = M$ in \autoref{thm:gap_NeqM_softmax_hetgap}. 

% We \textit{conjecture} the lower bound of \autoref{thm:gap_NeqM_softmax_hetgap} is actually the exact value of $\Delta R(t,\tau;N)$, but we were unable to prove this.  \autoref{fig:deltaR-vs-softmax} (righthand side) plots the heterogeneity gains when $N = M =  2$. 

\begin{theorem}[Softmax \heterogeneitygap for $N=M$]
\label{thm:gap_NeqM_softmax_hetgap}
Assume $N = M \geq 2$, and let \(\sigma(t,N):=\frac{e^{t}}{e^{t}+N-1}
\). The \heterogeneitygap for softmax aggregators \textbf{(i)} equals $\Delta R(t,\tau;N)=0$ when $t \leq 0$; \textbf{(ii)} is bounded below by $\sigma(t,N)-\dfrac1N$ when $t>0, \tau\le 0$; and  \textbf{(iii)} is bounded below by $\max\!\bigl\{\sigma(t,N)-\sigma(\tau,N),0\bigr\}$ when $t>0, \tau\ge 0.$
% \[
% \boxed{%
% \Delta R(t,\tau;N)\geq
% \begin{cases}
% \sigma(t,N)-\dfrac1N, & t>0,\;\;\tau\le 0,\\[10pt]
% \max\!\bigl\{\sigma(t,N)-\sigma(\tau,N),\,0\bigr\}, & t>0,\;\;\tau\ge 0.
% \end{cases}}
% \] otherwise, where 
% \(\sigma(t,N)\;:=\;\frac{e^{t}}{e^{t}+N-1}.
% \)
\end{theorem}

\autoref{tab:param-agg-extended} contains more examples of aggregation operators parameterized by $t$. These families provide a search space for potential reward functions, allowing us to sweep smoothly from $\Delta R = 0$ to $\Delta R > 0$ reward regimes.  As $t \to \pm \infty$, most such aggregators converge to either $\min$ or $\max$, and often reduce to the arithmetic mean for certain parameter choices, motivating us to ask what the \heterogeneitygap is when the outer and inner aggregator belong to the set $\{\min, \text{mean}, \max\}$. These aggregators are of special interest, since ``$\min$'' can be seen as a ``maximally'' Schur-concave function, ``$\max$'' can be seen as a ``maximally'' Schur-convex function, and ``mean'' is both Schur-convex and Schur-concave. Hence, it is worth asking what the \heterogeneitygap is when the outer and inner aggregator belong to the set $\{\min, \text{mean}, \max\}$. We derive these gains in two cases: continuous allocations where $r_{ij} \in [0,1]$, and discrete effort allocations where $r_{ij} \in \{0,1\}$. The results are summarized in \autoref{fig:deltaR-vs-softmax}, lefthand side (formal derivation available in \autoref{appendix:formal_analysis}). 


%–––––  Preamble needs –––––
% \usepackage{graphicx}    % \includegraphics
% \usepackage{adjustbox}   % \begin{adjustbox}{…}
% \usepackage{caption}     % \captionof   (standard in article/report)

\begin{figure}[t]            % → figure* in two-column styles
  \centering
  %================ LEFT BLOCK (ΔR table) ================
  \begin{minipage}{0.5\textwidth}
    \centering
    \footnotesize
    \captionof*{table}{\small Discrete and continuous heterogeneity gains}
    %
    \begin{adjustbox}{max width=\linewidth}
      \tiny                        % table font only
      \setlength{\tabcolsep}{3pt}  % tight columns
      \renewcommand{\arraystretch}{1.15}
      \begin{tabular}{l|ccc}
        & $T=\min$ & $T=\text{mean}$ & $T=\max$\\\hline
        \multicolumn{4}{c}{\it Outer $U=\min$}\\\hline
        $\Delta R_{\mathrm{F}}$ & 0 & 0 & $(M-1)/M$\\
        $\Delta R_{\mathrm{D}}$ & 0 & $\lfloor N/M\rfloor/N$ &
          $\mathbf 1_{\{N\ge M\}}$\\\hline
        \multicolumn{4}{c}{\it Outer $U=\text{mean}$}\\\hline
        $\Delta R_{\mathrm{F}}$ & 0 & 0 & $(M-1)/M$\\
        $\Delta R_{\mathrm{D}}$ & 0 & 0 & $(\min\{M,N\}-1)/M$\\\hline
        \multicolumn{4}{c}{\it Outer $U=\max$}\\\hline
        $\Delta R_{\mathrm{F}}$ & 0 & 0 & 0\\
        $\Delta R_{\mathrm{D}}$ & 0 & 0 & 0\\\hline
      \end{tabular}
    \end{adjustbox}
  \end{minipage}%   ←– prevents newline
  \hfill%            ←– prevents newline after glue
  %================ RIGHT BLOCK (graphic) ================
  \begin{minipage}{0.5\textwidth}
    \centering
    \includegraphics[width=\linewidth]{images/hetgap_N2M2_softmax.pdf}
    % \captionof*{figure}{\small $N{=}2,\;M{=}2$ soft-max heterogeneity gains}
  \end{minipage}

  \caption{\textbf{Left:} Discrete ($\Delta R_{\mathrm D}$) and continuous-allocation
           ($\Delta R_{\mathrm F}$) heterogeneity gains for all
           $U,T\!\in\!\{\min,\text{mean},\max\}$. The indicator
           $\mathbf 1_{\{N\ge M\}}$ equals 1 if \(N\ge M\) and 0 otherwise. 
           \textbf{Right:} We plot the parametrized heterogeneity gains $\Delta R(t,\tau;N)$ when $U$ and $T$ are soft-max aggregators.  }
  \label{fig:deltaR-vs-softmax}
\end{figure}


% ----


% We derive these \heterogeneitygaps in Table \ref{tab:softmaxextremes}.  The ``discrete'' case where each agent must allocate all effort to a single task (i.e., for each $i$, $r_{ij} = 1$ for some $j$ and $0$ for all $j' \neq j$) is sometimes also of interest. \Heterogeneitygaps for this case are summarized in Table \ref{tab:discrete-finite-tau}.

% ----





% \section{Discounted Rewards Over a Time Horizon}
% \label{sec:rl_transition}

% The framework introduced in the Problem Formulation section describes a static optimization problem where agents simultaneously choose their effort allocation \( \mathbf{r}_i = (r_{i1}, \ldots, r_{iM}) \) across tasks, forming the matrix \( \mathbf{A} \). The overall reward \( R(\mathbf{A}) \) is then calculated based on this joint allocation. However, many real-world multi-agent systems operate dynamically over time. To connect our above analysis to such settings, we can consider an action space such at time \( t \), agent $i$ puts $r_{ij,t}$ effort into task $j$. We can then consider, at every time step $t$, the corresponding instantaneous allocation matrix \( \mathbf{A}_t \). The reward given to the agents at time $t$ is then $R_t = R(\mathbf{A}_t) = \outeragg_{j=1}^M \inneragg_{i=1}^N r_{ij,t}$.

% In this temporal setting, agents' effort allocations are dependent on their policy \( \pi_i \), and their goal is to maximize the expected discounted sum of future rewards:
% \[
% J(\{\pi_i\}_{i=1}^N) = \mathbb{E}_{\{\pi_i\}, P} \left[ \sum_{t=0}^\infty \gamma^t R_t \right],
% \]
% where \( \gamma \in [0, 1) \) is the discount factor and \( P \) denotes the state transition dynamics (if any). This reward structure is directly applicable to reinforcement learning tasks, and in our experiments, we shall show that it allows one to predict the \heterogeneitygap in such settings.



\section{Heterogeneity Gain Parameter Search (HetGPS)}
\label{sec:HetGPS}
In complex scenarios where theory might be less applicable, we study  heterogeneity through algorithmic search. We consider the setting of a Parametrized Dec-POMDP (PDec-POMDP, defined in \autoref{app:pdecpomdp}). A PDec-POMDP represents a Dec-POMDP~\citep{oliehoek2016concise}, where the observations, transitions, or reward  are conditioned on parameters $\theta$. Hence, 
the return obtained by the agents, $G^\theta(\mathbf{\pi})$, can be differentiated with respect to $\theta$: $\nabla_\theta G^\theta(\mathbf{\pi})=\frac{\partial}{\partial_\theta} G^\theta(\mathbf{\pi})$.
In particular, computing this gradient in a differentiable simulator allows us to back-propagate through time and optimize $\theta$ via gradient ascent\footnote{Although the same approach can train policies \citep{xu2022accelerated,song2024learning}, HetGPS instead optimizes environment parameters and policies separately, using standard zeroth-order policy-gradient methods, to avoid being trapped in local minima. The overhead of implementing HetGPS in this way is modest: it increased training time by roughly 25\% in our \autoref{sec:experiments} experiments compared to training on a fixed environment.}.


% We introduce \textbf{Heterogeneity Gain Parameter Search (HetGPS)}, an algorithm that automatically discovers multi-agent environments that maximize the \heterogeneitygap.

% Note that such a method can be used to learn the policies as well \citep{xu2022accelerated,song2024learning}. In this work, however, we learn policies and environment parameters separately to avoid them collectively overfitting to local minima. Policies are learnt through traditional zeroth-order policy gradient.

% We introduce \textbf{Heterogeneity Gain Parameter Search (HetGPS)}, an algorithm that automatically discovers multi-agent environments that maximize the \heterogeneitygap.

% HetGPS is able to search over a space of underspecified, parametrizable environments by using gradient descent on the agents' returns. 


\begin{algorithm}[ht]
   \caption{Heterogeneity Gain Parameter Search (HetGPS)}
   \label{alg:HetGPS}
\begin{algorithmic}[1]
   \INPUT Environment parameters $\theta$, environment learning rate $\alpha$, heterogeneous agent policy $\mathbf{\pi}_\mathrm{het}$, homogeneous agent policy $\mathbf{\pi}_\mathrm{hom}$
   \FOR{$i$ in iterations}
        \STATE Batch$_\mathrm{het}^\theta$= Rollout($\theta$,$\mathbf{\pi}_\mathrm{het}$) 
        \COMMENT{{\color{blue}rollout het policies in environment $\theta$}}
        \STATE Batch$_\mathrm{hom}^\theta$= Rollout($\theta$,$\mathbf{\pi}_\mathrm{hom}$) \COMMENT{{\color{blue}rollout hom policies in environment $\theta$}}
        \STATE HetGain$^\theta$ = ComputeGain(Batch$_\mathrm{het}^\theta$,Batch$_\mathrm{hom}^\theta$)
        \IF{\lstinline[language=Python]{train_env}($i$)}
            \STATE $\theta \leftarrow \theta + \alpha\nabla_\theta  \mathrm{HetGain}^\theta$ \COMMENT{{\color{blue} train environment via backpropagation}}
        \ENDIF
        \IF{train\_agents($i$)}
            \STATE  $\mathbf{\pi}_\mathrm{het} \leftarrow$ MarlTrain($\mathbf{\pi}_\mathrm{het}$,Batch$_\mathrm{het}^\theta$) \COMMENT{{\color{blue}train het policies via MARL}}
            \STATE  $\mathbf{\pi}_\mathrm{hom} \leftarrow$ MarlTrain($\mathbf{\pi}_\mathrm{hom}$,Batch$_\mathrm{hom}^\theta$) \COMMENT{{\color{blue}train hom policies via MARL}}
        \ENDIF
   \ENDFOR
   \OUTPUT final environment configuration $\theta$, policies $\mathbf{\pi}_\mathrm{het},\mathbf{\pi}_\mathrm{hom}$
\end{algorithmic}
\end{algorithm}

\headline{Heterogeneity Gain Parameter Search (HetGPS)}
We now consider the problem of learning the environment parameters $\theta$ to maximize the \textit{empirical} \heterogeneitygap.
The empirical \heterogeneitygap is defined as the difference in performance between heterogeneous and homogeneous teams in a given PDec-POMDP parametrization. We compare \textit{neurally heterogeneous} agents (independent parameters) with \textit{neurally homogeneous} agents (shared parameters).
%Heterogeneous agents each learn different parameters, whereas homogeneous agents learn the same policy. 
We denote their policies as $\mathbf{\pi}_\mathrm{het}$ and $\mathbf{\pi}_\mathrm{hom}$.
Then, we can simply write the gain as: $\mathrm{HetGain}^\theta(\mathbf{\pi}_\mathrm{het},\mathbf{\pi}_\mathrm{hom}) =  G^\theta(\mathbf{\pi}_\mathrm{het}) -G^\theta(\mathbf{\pi}_\mathrm{hom})$, 
% \begin{equation}
%     \mathrm{HetGain}^\theta(\mathbf{\pi}_\mathrm{het},\mathbf{\pi}_\mathrm{hom}) =  G^\theta(\mathbf{\pi}_\mathrm{het}) -G^\theta(\mathbf{\pi}_\mathrm{hom}) ,
% \end{equation}
representing the return of heterogeneous agents minus that of homogeneous agents on environment parametrization $\theta$.
HetGPS, shown in \autoref{alg:HetGPS}, learns $\theta$ by performing gradient ascent to maximize the gain:
$\theta \leftarrow \theta + \alpha\nabla_\theta  \mathrm{HetGain}^\theta(\mathbf{\pi}_\mathrm{het},\mathbf{\pi}_\mathrm{hom})$.
The environment and the agents are trained in an iterative, bilevel optimization process. We discuss this process, and \textit{alternatives when the simulator is non-differentiable}, in \autoref{app:stability_of_bilevel_optimization}.
At every training iteration, HetGPS collects roll-out batches in the current environment $\theta$ for both heterogeneous and homogeneous teams, computing the \heterogeneitygap on the collected data.
Then, it updates $\theta$ to maximize the \heterogeneitygap.
Finally, to train the agents, it uses MARL, with any on-policy algorithm (e.g., MAPPO~\citep{yu2022surprising}).
The functions \verb|train_env| and \verb|train_agents| determine when to train each of the components in HetGPS. We consider two possible training regimes: (1) \textit{alternated}: where HetGPS performs cycles of $x$ agent training iterations followed by $y$ environment training iterations and (2) \textit{concurrent}: where agents train at every iteration and the environment is updated every $x$ iterations. Note that by performing descent instead of ascent, HetGPS can also be used to \textit{minimize} the heterogeneity gain.


\section{Experiments}
\label{sec:experiments}
%!TEX root = main.tex

\section{Experimental Evaluation}
\seclabel{experiments}

We first evaluated our algorithms
in an offline setting~(\secref{offline-expr}), where we record execution traces and evaluate different approaches on the \emph{same} input.
This eliminates biases due to non-deterministic thread scheduling.
Next, we consider an online setting~(\secref{online-expr}),
where we instrument programs and perform the analyses during runtime.
We conducted all our experiments on a standard laptop with \SI{1.8}{GHz} Intel Core i7 processor and \SI{16}{GB} RAM.

% We evaluated our algorithms in two experimental settings. 
% The first setting is offline experiments~(\secref{offline-expr}),
% in which we record execution traces and evaluate different approaches.
% This has the benefit that different approaches can be compared on the \emph{same} input,
% thereby eliminating biases due to non-deterministic thread scheduling.
% The second setting is online experiments~(\secref{online-expr}),
% in which we instrument programs and perform the analyses during runtime.
% We conducted all our experiments on a standard laptop with 1.8GHz Intel Core i7 processor and 16GB RAM.

%We compare our predictive algorithm with the \dlfuzzer tool. 
%This setting enables us to assess the applicability of our technique in a runtime monitoring system.
%The state-of-the-art deadlock predictors \dirk and \seqc work offline by design.
%Hence, they are not applicable for a comparison in this setting.
%As discussed in~\secref{otf}, offline methods are not directly applicable in runtime monitoring systems.

%Talk about implementation - name of tool + prog lang + traces are logged + filtering phase + cycle detection phase + online vs offline + trace conversion for seqc
%
%Setup - benchmarks + log traces using so-and-so-tools + 1-trace-per-benchmark  + cluster details + how many runs per trace + Timeout (if any)
%
%benchmark details: number of shared locks
%\subsection{Evaluation}
%
%Some suggested experiments:
%
%\begin{itemize}
%	\item Comparison with other tool(s) - \dirk and \seqc
%	\item Comparison of online vs online algorithms
%	\item Scalability with number of events
%	\item Scalability with number of threads in the trace
%	\item Scalability with number of threads in the deadlock pattern
%	\item Time spent in each kinds of events (this will be useful to guide future research)
%\end{itemize}
%
%\Andreas{For matching reports, we slightly adapt the algorithm to report all sets of program locations that contain a deadlock pattern (hence we might have more than one reports per abstract deadlock pattern, corresponding to different program locations)}


% Here we report on an implementation and experimental evaluation of our algorithms.

% \subsection{Experimental Setup}



\subsection{Offline Experiments}
\seclabel{offline-expr}


\Paragraph{Experimental setup}
The goal of the first set of experiments is to evaluate 
$\SyncPDOffline$, and compare it
against prior algorithms for dynamic deadlock prediction.
In order for our evaluation to be precise we evaluate all algorithms on the \emph{same} execution trace.
We implemented $\SyncPDOffline$ in Java inside the \toolname analysis tool~\cite{rapid}, 
following closely the pseudocode in \algoref{offline}.
\toolname takes as input execution traces, as defined in \secref{prelim}.
These also include fork, join, and lock-request events.
We compare $\SyncPDOffline$ with two state-of-the-art, 
theoretically-sound albeit computationally more expensive, deadlock predictors,
\seqc~\cite{Cai2021} and \dirk~\cite{Kalhauge2018}, both of which also work on execution traces.




On the theoretical side, the complexity of \seqc is $\Otilde(\NumEvents^4)$, 
as opposed to the $\Otilde(\NumEvents)$ complexity of $\SyncPDOffline$. 
Moreover, \seqc only predicts deadlocks of size $2$, and though it could be extended to handle deadlocks of any size, this would degrade performance further.
\seqc may miss sync-preserving deadlocks even of size $2$, 
but can  detect deadlocks that are not sync-preserving.
Thus \seqc and $\SyncPDOffline$ are theoretically incomparable in their detection capability.
We refer to\begin{pldi}~\cite{arxiv}\end{pldi}\begin{arxiv}~\appref{incomp}\end{arxiv} for examples.
We noticed that \seqc fails on traces with non-well-nested locks --- we encountered one such case in our dataset.
\dirk's algorithm is theoretically complete, i.e., it can find all predictable deadlocks in a trace.
In addition, it can find deadlocks beyond the predictable ones, by reasoning about event values.
However, \dirk relies on heavyweight SMT-solving and
employs windowing techniques to scale to large traces. 
Due to windowing, it can miss deadlocks between events that are outside the given window. 
%\hunkar{
As with previous works~\cite{Cai2021, Kalhauge2018}, we set a window size of $10$K for \dirk.
%}

Our dataset consists of several benchmarks 
from standard benchmark suites --- IBM Contest suite~\cite{Farchi03}, Java Grande suite~\cite{Smith01},
DaCapo~\cite{Blackburn06}, and
SIR~\cite{doESE05} ---
and recent literature~\cite{Kalhauge2018, Cai2021, jula2008deadlock, Joshi2009}.
Each benchmark was instrumented with RV-Predict~\cite{rvpredict} or Wiretap~\cite{Kalhauge2018} and
executed in order to log a single execution trace.

%!TEX root = ../main.tex

\begin{table}[h!]
\caption{
%\hcomment{Should we remove some not very important rows from this table?}
Trace characteristics, abstract lock graph statistics and performance comparison.
%Column 1 denotes the name of the benchmark.
Columns 2-6 show the number of events, threads, variables, locks
and total number of lock acquire and request events.
Columns 7-9 show the number of cycles, abstract and concrete deadlock patterns in the abstract lock graph.
Columns 10 - 15 show the number of deadlocks reported and the times (in seconds) taken. 
by \dirk, \seqc, and \SyncPDOffline.
%Statistics on the lock graph $\lkevgraph{\tr}$ of each trace $\tr$.
% We denote by $\mathcal{N}$, $\mathcal{T}$, $\mathcal{M}$, $\mathcal{L}$ and $\NumAcquires + \mathcal{R}$ the total number of events, number of threads, number of memory locations, number of locks and number of acquire and request events in the benchmarks, respectively. 
%All times are in seconds. 
Time out (T.O) was set to $3$h.
F stands for technical failure.
\label{tab:expr-results}
}
\vspace{-0.17cm}
\setlength\tabcolsep{3pt}
\renewcommand{\arraystretch}{0.91}
\centering
\scalebox{0.86}{
\begin{tabular}{|r|c|c|c|c|c||c|c|c||c|c||c|c||c|c|}
\hline
1 & 2 & 3 & 4 & 5 & 6 & 7 & 8 & 9 & 10 & 11 & 12 & 13 & 14 & 15 \\
\hline
\multirow{2}{*}{\textbf{Benchmark}}& 
\multirow{2}{*}{$\mathcal{N}$} & \multirow{2}{*}{$\mathcal{T}$} & \multirow{2}{*}{$\mathcal{V}$} & \multirow{2}{*}{$\mathcal{L}$} & \multirow{2}{*}{$\NumAcquires/\mathcal{R}$} 
& \multicolumn{3}{c||}{ \textsf{A. Lock Graph }}
& \multicolumn{2}{c||}{ {\dirk} } 
& \multicolumn{2}{c||}{ {\seqc}} 
& \multicolumn{2}{c|}{ {\textsf{$\SyncPDOffline$}}} \\
\cline{7-15}
& & & & & 
& \textsf{|$\textsf{Cyc}$|}
& \textsf{\textsf{A. P.}}
& \textsf{\textsf{C. P.}}
& \textbf{Dlk} 
& {\textbf{Time}} 
&{\textbf{Dlk}} 
& {\textbf{Time}}
&{\textbf{Dlk}} 
&{\textbf{Time}} \\
\hline
Deadlock & 39 & 3 & 4 & 3 & 8 & 1 & 1 & 1 & 1 & 0.02 & 0 & 0.09 & 0 & 0.16\\
NotADeadlock & 60 & 3 & 4 & 5 & 16 & 1 & 1 & 1 & 0 & 0.02 & 0 & 0.09 & 0 & 0.16\\
Picklock & 66 & 3 & 6 & 6 & 20 & 2 & 2 & 2 & 1 & 0.02 & 1 & 0.10 & 1 & 0.18\\
Bensalem & 68 & 4 & 5 & 5 & 22 & 2 & 2 & 2 & 1 & 0.02 & 1 & 0.12 & 1 & 0.16\\
Transfer & 72 & 3 & 11 & 4 & 12 & 1 & 1 & 1 & 1 & 0.02 & 0 & 0.09 & 0 & 0.15\\
Test-Dimmunix & 73 & 3 & 9 & 7 & 26 & 2 & 2 & 2 & 2 & 0.02 & 2 & 0.10 & 2 & 0.17\\
StringBuffer & 74 & 3 & 14 & 4 & 16 & 1 & 3 & 6 & 2 & 0.02 & 2 & 0.12 & 2 & 0.19\\
Test-Calfuzzer & 168 & 5 & 16 & 6 & 48 & 2 & 1 & 1 & 1 & 0.02 & 1 & 0.12 & 1 & 0.17\\
DiningPhil & 277 & 6 & 21 & 6 & 100 & 1 & 1 & 3K & 1 & 1.60 & 0 & 0.09 & 1 & 0.17\\
HashTable & 318 & 3 & 5 & 3 & 174 & 1 & 2 & 43 & 2 & 0.19 & 2 & 0.12 & 2 & 0.19\\
Account & 706 & 6 & 47 & 7 & 134 & 3 & 1 & 12 & 0 & 0.19 & 0 & 0.09 & 0 & 0.18\\
Log4j2 & 1K & 4 & 334 & 11 & 43 & 1 & 1 & 1 & 1 & 0.65 & 1 & 0.11 & 1 & 0.20\\
Dbcp1 & 2K & 3 & 768 & 5 & 56 & 2 & 2 & 3 & - & F & 2 & 0.11 & 2 & 0.19\\
Dbcp2 & 2K & 3 & 592 & 10 & 76 & 1 & 2 & 4 & - & F & 0 & 0.10 & 0 & 0.18\\
Derby2 & 3K & 3 & 1K & 4 & 16 & 1 & 1 & 1 & 1 & 0.23 & 1 & 0.10 & 1 & 0.17\\
RayTracer & 31K & 5 & 5K & 15 & 976 & 0 & 0 & 0 & - & F & 0 & 0.15 & 0 & 0.19\\
jigsaw & 143K & 21 & 8K & 2K & 67K & 172 & 12 & 70 & - & F & 2 & 0.36 & 1 & 1.55\\
elevator & 246K & 5 & 727 & 52 & 48K & 0 & 0 & 0 & 0 & 1.65 & 0 & 0.33 & 0 & 0.27\\
hedc & 410K & 7 & 109K & 8 & 32 & 0 & 0 & 0 & 0 & 2.09 & 0 & 0.50 & 0 & 0.24\\
JDBCMySQL-1 & 442K & 3 & 73K & 11 & 13K & 2 & 4 & 6 & 2 & 28.45 & 2 & 0.24 & 2 & 0.48\\
JDBCMySQL-2 & 442K & 3 & 73K & 11 & 13K & 4 & 4 & 9 & 1 & 3.37 & 1 & 0.22 & 1 & 0.33\\
JDBCMySQL-3 & 443K & 3 & 73K & 13 & 13K & 5 & 8 & 16 & 1 & 31.23 & 1 & 0.25 & 1 & 0.45\\
JDBCMySQL-4 & 443K & 3 & 73K & 14 & 13K & 5 & 10 & 18 & 2 & 5.51 & 2 & 0.28 & 2 & 0.49\\
cache4j & 775K & 2 & 46K & 20 & 35K & 0 & 0 & 0 & 0 & 5.86 & 0 & 0.46 & 0 & 0.39\\
ArrayList & 3M & 801 & 121K & 802 & 176K & 9 & 3 & 672 & 3 & 8.7K & 3 & 21.98 & 3 & 1.68\\
IdentityHashMap & 3M & 801 & 496K & 802 & 162K & 1 & 3 & 4 & 1 & 443.93 & 1 & 8.51 & 1 & 1.45\\
Stack & 3M & 801 & 118K & 2K & 405K & 9 & 3 & 481 & 1 & T.O & 3 & 25.34 & 3 & 2.94\\
Sor & 3M & 301 & 2K & 3 & 719K & 0 & 0 & 0 & 0 & 15.89 & 0 & 44.12 & 0 & 0.61\\
LinkedList & 3M & 801 & 290K & 802 & 176K & 9 & 3 & 10K & 3 & 4.7K & 3 & 48.02 & 3 & 2.06\\
HashMap & 3M & 801 & 555K & 802 & 169K & 1 & 3 & 10K & 3 & 4.4K & 2 & 504.36 & 2 & 1.65\\
WeakHashMap & 3M & 801 & 540K & 802 & 169K & 1 & 3 & 10K & - & T.O & 2 & 499.68 & 2 & 1.70\\
Swing & 4M & 8 & 31K & 739 & 2M & 0 & 0 & 0 & - & F & 0 & 0.72 & 0 & 0.88\\
Vector & 4M & 3 & 15 & 4 & 800K & 1 & 1 & 1B & - & T.O & 1 & 1.52 & 1 & 1.90\\
LinkedHashMap & 4M & 801 & 617K & 802 & 169K & 1 & 3 & 10K & 2 & 40.74 & 2 & 492.87 & 2 & 1.69\\
montecarlo & 8M & 3 & 850K & 3 & 26 & 0 & 0 & 0 & 0 & 2.6K & 0 & 1.81 & 0 & 0.79\\
TreeMap & 9M & 801 & 493K & 802 & 169K & 1 & 3 & 10K & 2 & 105.45 & 2 & 480.11 & 2 & 1.92\\
hsqldb & 20M & 46 & 945K & 403 & 419K & 0 & 0 & 0 & - & F & - & - & 0 & 2.38\\
sunflow & 21M & 16 & 2M & 12 & 1K & 0 & 0 & 0 & - & F & 0 & 8.35 & 0 & 1.62\\
jspider & 22M & 11 & 5M & 15 & 10K & 0 & 0 & 0 & - & F & 0 & 8.49 & 0 & 1.95\\
tradesoap & 42M & 236 & 3M & 6K & 245K & 2 & 1 & 4 & - & F & 0 & 108.16 & 0 & 7.06\\
tradebeans & 42M & 236 & 3M & 6K & 245K & 2 & 1 & 4 & - & F & 0 & 116.23 & 0 & 7.26\\
eclipse & 64M & 15 & 10M & 5K & 377K & 9 & 5 & 280 & - & F & 0 & 26.67 & 0 & 9.90\\
TestPerf & 80M & 50 & 599 & 9 & 197K & 0 & 0 & 0 & 0 & 795.04 & 0 & 47.56 & 0 & 4.30\\
Groovy2 & 120M & 13 & 13M & 10K & 69K & 0 & 0 & 0 & 0 & 1.7K & 0 & 38.06 & 0 & 8.92\\
Tsp & 200M & 6 & 24K & 3 & 882 & 0 & 0 & 0 & 0 & 7.6K & 0 & 72.62 & 0 & 12.70\\
lusearch & 203M & 7 & 3M & 98 & 273K & 0 & 0 & 0 & 0 & 1.3K & 0 & 75.88 & 0 & 14.44\\
biojava & 221M & 6 & 121K & 79 & 16K & 0 & 0 & 0 & - & F & 0 & 63.79 & 0 & 12.65\\
graphchi & 241M & 20 & 25M & 61 & 1K & 0 & 0 & 0 & - & F & 0 & 102.05 & 0 & 25.25\\
\hline\hline\textbf{Totals} & \textbf{1B} & \textbf{7K} & \textbf{70M} & \textbf{37K} & \textbf{8M} & \textbf{256} & \textbf{93} & \textbf{1B} & \textbf{35} & \textbf{>18h} & \textbf{40} & \textbf{2801} & \textbf{40} & \textbf{135}\\\hline
\end{tabular}
}
\end{table}

\Paragraph{Evaluation}
\cref{tab:expr-results} presents our results.
A bug identifies a unique tuple of source
code locations corresponding to events participating in the deadlock.
%Columns 2-6 present the characteristics of our benchmark traces.
Trace lengths vary vastly from $39$ to about $241$M, while the number of threads ranges from $3$ to about $800$,
which are representative features of real-world settings.
\texttt{Hsqldb} contains critical sections that are not well nested, 
and \seqc was not able to handle this benchmark;
our algorithm does not have such a restriction.
%We were unable to run \dirk on certain benchmarks due to \dirk's technical issues, which are marked with F.


\vspace{-0.1cm}
\SubParagraph{\underline{Abstract vs Concrete Patterns}}
Columns 7-9 present statistics on the 
abstract lock graph $\lkevgraph{\tr}$ of each trace $\tr$.
Many traces have a large number of concrete deadlock patterns 
but much fewer abstract deadlock patterns;
a single abstract deadlock pattern can 
comprise up to an order of $10^4$ more concrete patterns (Column $8$ v/s Column $9$).
Unlike all prior sound techniques, 
our algorithms analyze 
abstract deadlock patterns, instead of concrete ones. 
We thus expect our algorithms to be much more scalable in practice.


\SubParagraph{\underline{Deadlock-detection capability}}
In total, both \seqc and $\SyncPDOffline$ reported 40 deadlocks.
\seqc misses a deadlock of size $5$ in \texttt{DiningPhil},
which is detected by $\SyncPDOffline$,
and $\SyncPDOffline$ misses a deadlock in \texttt{jigsaw} which is detected by \seqc.
As $\SyncPDOffline$ is complete for sync-preserving deadlocks, we conclude that there are no more such deadlocks in our dataset.
Overall, $\SyncPDOffline$ and \seqc miss only three deadlocks reported by \dirk. 
On closer inspection, we found that these deadlocks are not witnessed by correct reorderings, and require reasoning about event values.
On the other hand, \dirk struggles to analyze even moderately-sized benchmarks and times out in $3$ of them. %(timeout is 3h).
This results in \dirk failing to report 5 deadlocks after $9$ hours, all of which are reported by $\SyncPDOffline$ in under a minute.
Similar conclusions were recently made in~\cite{Cai2021}.  
Overall, our results strongly indicate that the notion of sync-preservation characterizes most deadlocks that other tools are able to predict.


\SubParagraph{\underline{Unsoundness of \dirk}}
In our evaluation, we discovered that the soundness guarantee 
underlying \dirk~\cite{Kalhauge2018} is broken, resulting in it reporting false positives.
% Although \dirk is portrayed as sound, we have found two independent sources of unsoundness 
% resulting in it reporting false positives.
First, its constraint formulation~\cite{Kalhauge2018} 
does not rule out deadlock patterns when acquire events in the pattern hold common locks, 
in which case mutual exclusion disallows such a pattern to be a real predictable deadlock.
Second, \dirk also models conditional statements, allowing it to reason about witnesses beyond correct reorderings.
While this relaxation allows \dirk to predict additional deadlocks in \texttt{Transfer}, \texttt{Deadlock} and \texttt{HashMap}, 
its formalization is not precise and its implementation is erroneous.
We elaborate these aspects further in\begin{pldi}~\cite{arxiv}. \end{pldi}\begin{arxiv}~\appref{unsound-dirk}.\end{arxiv}


%We noticed that \dirk reports false positives in certain cases, which contradicts its theoretical soundness guarantee.
%We provide two such cases in~\appref{unsound-dirk}.
%The first case is a modified version of \texttt{Transfer}.
%Here, \dirk falsely reports a deadlock because it inadequately models conditional statements, which are used to allow more trace reorderings that can expose a deadlock.
%This relaxation enables \dirk to predict deadlocks in the benchmarks \texttt{Transfer} and \texttt{Deadlock}, which are missed by $\SyncPDOffline$ and \seqc. 
%However, this relaxation is not formalized and its implementation is erroneous.
%In the second case, \dirk falsely reports a deadlock due to missing that a common lock protects an otherwise deadlock pattern.

\SubParagraph{\underline{Running time}}
Our experimental results indicate that \dirk, backed by SMT solving, 
is the least efficient technique in terms of running time ---
it takes considerably longer or times out on large benchmark instances.
$\SyncPDOffline$ analyzed the entire set of traces $\sim\!\!\!21\times$ faster than \seqc.
On the most demanding benchmarks, such as 
HashMap and TreeMap, $\SyncPDOffline$ is more than $200\times$ faster than \seqc.
Although \seqc employs a polynomial-time algorithm for deadlock prediction, 
and thus significantly faster than the SMT-based \dirk,
the large polynomial complexity in its running time hinders scalability on 
execution traces coming from benchmarks that are more representative of realistic workloads.
In contrast, the linear time guarantees of $\SyncPDOffline$ are realized in practice, 
allowing it to scale on even the most challenging inputs.
More importantly, the improved performance comes while preserving essentially the same precision.


%\begin{figure}[!h]
\def\scatterscale{0.192}
\def\scatterwidth{0.23\textwidth}

\centering
\begin{subfigure}[b]{\scatterwidth}
\includegraphics[scale=\scatterscale]{figures/LinkedListDeadlockTest.pdf}
\label{subfig:sca-one-lock-clique}
\caption{LinkedList}
\end{subfigure}
\begin{subfigure}[b]{\scatterwidth}
\includegraphics[scale=\scatterscale]{figures/StackDeadlockTest.pdf}
\caption{Stack}
\label{subfig:sca-skewed}
\end{subfigure}


\caption{
Scalability experiments.
}
\label{fig:scalability}
\end{figure}


%\Andreas{We might revisit/remove the following}

%\textcolor{red}{check the numbers again.}
\SubParagraph{\underline{False negatives}}
Our benchmark set contains $93$ abstract deadlock patterns, $40$ of which are confirmed sync-preserving deadlocks.
We inspected the remaining $53$ abstract patterns to see if any of them are predictable deadlocks
missed by our sync-preserving criterion, independently of the compared tools.
$48$ of these $53$ patterns are in fact not predictable deadlocks ---
for every such pattern $D$, 
the set $S_D$ of events in the downward-closure of $\prev{}(D)$ with respect to $\tho{}$ and $\rf{}$,
already contains an event from $D$, disallowing any correct reordering
(sync-preserving or not) in which $D$ can be enabled.
Of the remaining, $4$ deadlock patterns obey the following scheme:
there are two acquire events $\acq_1, \acq_2$ participating in the deadlock pattern, 
each $\acq_i$ is preceded by a critical section on a lock that appears in 
$\lheld{}(\acq_{3-i})$, again disallowing a correct reordering that witnesses the pattern.
Thus, \emph{only one} predictable deadlock is not sync-preserving in our whole dataset.
This analysis supports that the notion of sync-preservation is not overly conservative in practice.
%\hunkar{I think here we should clarify that this analysis is not overly conservative as far as predictable deadlocks go.}

%\hunkar{
The above analysis concerns false negatives wrt. predictable deadlocks.
Some deadlocks are beyond the common notion of predictability we have adopted here, as they can only be exposed by reasoning about event values and control-flow dependencies, a problem that is $\NP$-hard even for 3 threads~\cite{Gibbons1997}.
We noticed $3$ such deadlocks in our dataset, found by \dirk,
though, as mentioned above, \dirk's reasoning for capturing such deadlocks is unsound in practice.
%}

%\hunkar{Maybe stress again that this reasoning about event values is non-trivial as it relies on heavyweight SMT solving. Also, maybe have the section "Unsoundness of Dirk" after this one and say that next we show that it is also tricky to implement in practice.}
%sync-preserving deadlocks are likely to be permissive and not lead to a high false negative rate.
%We conduct an analyses based on our benchmark set and characterize potential false negatives of our technique.
%Recall the conditions imposed on correct reorderings (see \secref{prelim}).
%One of the main conditions imposed by the definition of a correct reordering is that
%every read event reads from the same write as in the original trace. 
%We investigated the effect of this condition on the potential false negatives.
%Our benchmark set contains $93$ abstract deadlock patterns and our technique is able to find $42$ deadlocks.
%The remaining $51$ deadlock patterns constitute potential false negatives.
%If we only impose the above condition, then $46$ of the deadlock patterns cannot be realized to a real deadlock.
%The additional conditions that are specific to sync-preserving deadlocks (e.g., the order of critical sections on the same lock cannot be reversed) prevents the remaining $5$ deadlock patterns from being realizable.
%We remark that the definition of correct reorderings adopted in this work originate from the standard model that is widely used in this domain~\cite{Smaragdakis12,serbanuta2013,Koushik05}.
%Hence, this analyses further supports that given this standard program model, the additional restrictions introduced with the
%sync-preserving deadlocks are likely to be permissive and not lead to a high false negative rate.


\subsection{Online Experiments}
\seclabel{online-expr}

\Paragraph{Experimental setup}
%Dynamic analyses can incur high runtime overheads.
The objective of our second set of experiments is to evaluate 
the performance 
of our proposed algorithms in an \emph{online} setting.
For this, we implemented our $\SyncPDOnline$ algorithm inside the
framework of \dlfuzzer~\cite{Joshi2009} following closely the pseudocode in \algoref{online}. 
This framework instruments a concurrent program so that it can
perform analysis on-the-fly while executing it.
If a deadlock occurs during execution, it is reported and the execution halts.
However, if a deadlock is predicted in an alternate interleaving, 
then this deadlock is reported and the execution continues to search further deadlocks.
We used the same dataset as in \secref{offline-expr}, 
after discarding some benchmarks that could not be instrumented by \dlfuzzer.

To the best of our knowledge, all prior deadlock prediction techniques work offline.
For this reason, we only compared our online tool with the randomized 
scheduling technique of~\cite{Joshi2009} already
implemented inside the same \dlfuzzer framework.	
At a high level, this random scheduling technique works as follows.
Initially, it
(i)~executes the input program with a random scheduler, 
(ii)~constructs a \emph{lock dependency relation}, and 
(iii)~runs a cycle detection algorithm to discover deadlock patterns. 
For each deadlock pattern thus found, it spawns new executions that attempt 
to realize it as an actual deadlock.
To increase the likelihood of hitting the deadlock,
\dlfuzzer biases the random scheduler by pausing threads at specific locations.
%(e.g., before acquiring a certain lock).

The second, confirmation phase of~\cite{Joshi2009}
acts as a best-effort proxy for sound deadlock prediction.
On the other hand, $\SyncPDOnline$ is already sound and predictive, and thus does not require
additional confirmation runs, making it more efficient.
Towards effective prediction, we also implemented a simple bias to the scheduler.
If a thread $t$ attempts to write on a shared variable $x$ while holding a lock, then 
our procedure randomly decides to pause this operation for a short duration.
This effectively explores race conditions in different orders.
Overall, implementing $\SyncPDOnline$ inside \dlfuzzer provided the added advantage of supplementing a powerful prediction technique with a biased randomized scheduler.
%As an added advantage of implementing our algorithms in the
% framework, we are able to achieve
%the complementary objective of evaluating how well prediction supplements
%the effectiveness of an otherwise naive
%controlled concurrency testing
%technique like randomized scheduling.
To our knowledge, our work is the first to effectively 
combine these two orthogonal techniques.
We also remark that such a bias is of no benefit to \dlfuzzer itself
since it does not employ any predictive reasoning.

% a randomized testing procedure, even though the latter is rather simple.

For this experiment, we run \dlfuzzer on each benchmark, and for each deadlock pattern found in the initial execution, 
we let it spawn $3$ new executions trying to realize the deadlock, 
as per standard (\href{https://github.com/ksen007/calfuzzer}{https://github.com/ksen007/calfuzzer}).
We repeated this process $50$ times and recorded the total time taken.
Then, we allocated the same time for $\SyncPDOnline$ to repeatedly execute the same program and perform deadlock prediction.
We measured all deadlocks found by each technique.
%We also noticed that \dlfuzzer fails to work properly if the given input program deadlocks in the initial run.
%This results in \dlfuzzer itself to go into deadlock without producing any deadlock reports.  
%Hence, we made a modest modification on \dlfuzzer allowing it to work properly and report deadlocks in such cases.
%We used this modified version of \dlfuzzer in our evaluation.





%!TEX root = ../main.tex

\begin{table*}
\caption{
Performance comparison of $\SyncPDOnline$ ($\syncpdshort$) and $\dlfuzzer$ ($\dlfshort$).
%%Column 1 denotes the name of the benchmark.
Columns 2-3 show the total number of bug reports. 
Columns 4-6 show the total number of unique bugs found by each tool, and their union. 
Columns 7-12 show the hit rate on each bug.
Columns 13-16 show the runtime overhead of the tools.
%We refer to $\dlfuzzer$ as $\dlfshort$, to $\SyncPDOnline$ as $\syncpdshort$, and the instrumentation phases using the suffix $\mathsf{\tt -I}$.
%Performance comparison of $\SyncPDOnline$ ($\syncpdshort$) and $\dlfuzzer$ ($\dlfshort$).
%Column 1 denotes the name of the benchmark.
%Columns 2-3 show the total number of bug reports. 
%Columns 4-6 show the total number of unique bugs found by each tool, and their union. 
%Columns 7-12 show the hit rate on each bug.
%Columns 13-17 show the runtime overhead of the tools.
%We use the suffix $\mathsf{\tt -I}$ for the instrumentation phases.
\label{tab:expr-dlf-results}
}
\vspace{-0.15cm}
\setlength\tabcolsep{3pt}
\renewcommand{\arraystretch}{0.9}
\small
\centering
\scalebox{1}{
\begin{tabular}{|r|c|c|c|c|c|c|c|c|c|c|c|c|c|c|c|c|c|c|c|c|c|c|c|c|c|c|c|}
\hline
1 & 2 & 3 & 4 & 5 & 6 & 7 & 8 & 9 & 10 & 11 & 12 & 13 & 14 & 15 & 16 \\
\hline
\multirow{2}{*}{\textbf{Benchmark}}& 
\multicolumn{2}{c|}{ {\textsf{Bug Hits}}} & 
\multicolumn{3}{c|}{ {\textsf{Unique Bugs}}} 
& \multicolumn{2}{c|}{ \textsf{Bug 1}} &
\multicolumn{2}{c|}{ \textsf{Bug 2}} &
\multicolumn{2}{c|}{ \textsf{Bug 3}} & \multicolumn{4}{c|}{ {\textsf{Runtime Overhead}}}  \\
\cline{7-16}
\cline{2-6}
& \textsf{$\syncpdshort$} & \textsf{$\dlfshort$} & \textsf{$\syncpdshort$} & \textsf{$\dlfshort$}  & All 
& \textsf{\textsf{$\syncpdshort$}} 
& \textbf{$\dlfshort$} 
& {\textbf{$\syncpdshort$}} 
&{\textbf{$\dlfshort$}} 
& {\textbf{$\syncpdshort$}}
&{\textbf{$\dlfshort$}} & {\textbf{$\syncpdshortinstr$}}  & {\textbf{$\syncpdshort$}}  & 
{\textbf{$\dlfshortinstr$}}  &
 {\textbf{$\dlfshort$}} \\
\hline
%\multirow{2}{*}{\textbf{Benchmark}}& \multirow{2}{*}{$\mathcal{N}$} & \multirow{2}{*}{$\mathcal{T}$} & \multirow{2}{*}{$\mathcal{M}$}
%& \multirow{2}{*}{$\mathcal{L}$} &
%\multirow{2}{*}{$\NumAcquires+\mathcal{R}$} &\multicolumn{2}{c||}{ {\seqc}} & \multicolumn{2}{c||}{ {\textsf{$\SyncPDOffline$}}} & \multicolumn{2}{c|}{ {\textsf{$\SyncPDOnline$}}} \\
%\cline{7-12}
%& & & & & &  \textbf{$\#$Deadlocks} & { \textbf{Time (s)}} &{ \textbf{$\#$Deadlocks}} & { \textbf{Time (s)}} &{ \textbf{$\#$Deadlocks}} &{ \textbf{Time (s)}} \\
\hline
Deadlock & 50 & 50 & 1 & 1 & 1 & 50 & 50 & - & - & - & - & 2$\times$ & 3$\times$ & 2$\times$ & 4$\times$\\
Picklock & 227 & 97 & 2 & 1 & 2 & 226 & 97 & 1 & 0 & - & - & 2$\times$ & 2$\times$ & 2$\times$ & 5$\times$\\
Bensalem & 355 & 32 & 2 & 1 & 2 & 8 & 0 & 347 & 32 & - & - & 2$\times$ & 2$\times$ & 2$\times$ & 6$\times$\\
Transfer & 54 & 50 & 1 & 1 & 1 & 54 & 50 & - & - & - & - & 2$\times$ & 2$\times$ & 1$\times$ & 4$\times$\\
Test-Dimmunix & 702 & 0 & 2 & 0 & 2 & 351 & 0 & 351 & 0 & - & - & 2$\times$ & 2$\times$ & 2$\times$ & 4$\times$\\
StringBuffer & 153 & 131 & 2 & 2 & 2 & 128 & 118 & 25 & 13 & - & - & - & - & - & -\\
Test-Calfuzzer & 177 & 44 & 1 & 1 & 1 & 177 & 44 & - & - & - & - & 2$\times$ & 2$\times$ & 2$\times$ & 4$\times$\\
DiningPhil & 162 & 100 & 1 & 1 & 1 & 162 & 100 & - & - & - & - & - & - & - & -\\
HashTable & 169 & 120 & 2 & 2 & 2 & 82 & 21 & 87 & 99 & - & - & - & - & - & -\\
Account & 19 & 188 & 1 & 1 & 1 & 19 & 188 & - & - & - & - & 2$\times$ & 8$\times$ & 2$\times$ & 16$\times$\\
Log4j2 & 290 & 100 & 2 & 1 & 2 & 145 & 100 & 145 & 0 & - & - & - & - & - & -\\
Dbcp1 & 265 & 138 & 2 & 2 & 2 & 264 & 61 & 1 & 77 & - & - & - & - & - & -\\
Dbcp2 & 129 & 126 & 2 & 2 & 2 & 86 & 99 & 43 & 27 & - & - & - & - & - & -\\
RayTracer & 0 & 0 & 0 & 0 & 0 & - & - & - & - & - & - & 122$\times$ & 124$\times$ & 109$\times$ & 111$\times$\\
Tsp & 0 & 0 & 0 & 0 & 0 & - & - & - & - & - & - & 47$\times$ & 60$\times$ & 37$\times$ & 40$\times$\\
jigsaw & 1189 & 1 & 1 & 1 & 2 & 1189 & 0 & 0 & 1 & - & - & - & - & - & -\\
elevator & 0 & 0 & 0 & 0 & 0 & - & - & - & - & - & - & 2$\times$ & 2$\times$ & 2$\times$ & 2$\times$\\
JDBCMySQL-1 & 349 & 117 & 2 & 3 & 3 & 1 & 21 & 0 & 4 & 348 & 92 & 3$\times$ & 4$\times$ & 2$\times$ & 13$\times$\\
JDBCMySQL-2 & 559 & 73 & 1 & 1 & 1 & 559 & 73 & - & - & - & - & 2$\times$ & 4$\times$ & 2$\times$ & 18$\times$\\
JDBCMySQL-3 & 560 & 224 & 1 & 1 & 1 & 560 & 224 & - & - & - & - & 2$\times$ & 5$\times$ & 2$\times$ & 24$\times$\\
JDBCMySQL-4 & 1717 & 101 & 3 & 1 & 3 & 95 & 0 & 834 & 0 & 788 & 101 & 3$\times$ & 5$\times$ & 2$\times$ & 31$\times$\\
hedc & 0 & 0 & 0 & 0 & 0 & - & - & - & - & - & - & 2$\times$ & 2$\times$ & 1$\times$ & 2$\times$\\
cache4j & 0 & 0 & 0 & 0 & 0 & - & - & - & - & - & - & 2$\times$ & 2$\times$ & 2$\times$ & 2$\times$\\
lusearch & 0 & 0 & 0 & 0 & 0 & - & - & - & - & - & - & 16$\times$ & 17$\times$ & 13$\times$ & 16$\times$\\
ArrayList & 47 & 45 & 3 & 3 & 3 & 20 & 22 & 3 & 5 & 24 & 18 & 50$\times$ & 69$\times$ & 18$\times$ & 79$\times$\\
Stack & 44 & 27 & 3 & 3 & 3 & 18 & 13 & 8 & 4 & 18 & 10 & 69$\times$ & 91$\times$ & 64$\times$ & 86$\times$\\
IdentityHashMap & 68 & 62 & 2 & 2 & 2 & 13 & 47 & 55 & 15 & - & - & 4$\times$ & 8$\times$ & 3$\times$ & 10$\times$\\
LinkedList & 48 & 26 & 3 & 2 & 3 & 21 & 17 & 7 & 0 & 20 & 9 & 16$\times$ & 28$\times$ & 14$\times$ & 32$\times$\\
Swing & 0 & 0 & 0 & 0 & 0 & - & - & - & - & - & - & 5$\times$ & 6$\times$ & 4$\times$ & 6$\times$\\
Sor & 0 & 0 & 0 & 0 & 0 & - & - & - & - & - & - & 2$\times$ & 7$\times$ & 2$\times$ & 2$\times$\\
HashMap & 46 & 44 & 2 & 2 & 2 & 18 & 11 & 28 & 33 & - & - & 7$\times$ & 11$\times$ & 4$\times$ & 8$\times$\\
Vector & 126 & 50 & 1 & 1 & 1 & 126 & 50 & - & - & - & - & 2$\times$ & 2$\times$ & 2$\times$ & 3$\times$\\
LinkedHashMap & 57 & 43 & 2 & 2 & 2 & 22 & 10 & 35 & 33 & - & - & 10$\times$ & 10$\times$ & 4$\times$ & 8$\times$\\
WeakHashMap & 29 & 40 & 2 & 2 & 2 & 6 & 11 & 23 & 29 & - & - & 7$\times$ & 12$\times$ & 4$\times$ & 8$\times$\\
montecarlo & 0 & 0 & 0 & 0 & 0 & - & - & - & - & - & - & 16$\times$ & 100$\times$ & 13$\times$ & 126$\times$\\
TreeMap & 42 & 47 & 2 & 2 & 2 & 16 & 15 & 26 & 32 & - & - & 9$\times$ & 12$\times$ & 5$\times$ & 9$\times$\\
eclipse & 0 & 0 & 0 & 0 & 0 & - & - & - & - & - & - & 2$\times$ & 2$\times$ & 2$\times$ & 2$\times$\\
TestPerf & 0 & 0 & 0 & 0 & 0 & - & - & - & - & - & - & 2$\times$ & 2$\times$ & 2$\times$ & 2$\times$\\
\hline\hline\textbf{Total} & \textbf{7633} & \textbf{2076} & \textbf{49} & \textbf{42} & \textbf{51} & - & - & - & - & - & - & - & - & - & - \\ 
\hline
\end{tabular}
}
\end{table*}


\Paragraph{Evaluation}
\cref{tab:expr-dlf-results} presents our experimental results.
%A bug identifies a unique tuple of source code locations corresponding to events
%participating in the deadlock.
Columns $2$-$3$ of the table display the total number of bug hits,
which is the total number of times a bug was predicted by $\SyncPDOnline$ in the entire duration,
or was confirmed in any trial of \dlfuzzer.
Columns $4$-$6$ display the unique bugs (i.e., unique tuples of source code locations leading to a deadlock) 
found by the techniques.
The employed techniques are able to find a maximum of $3$ unique bugs for each benchmark
in our benchmark set. 
Respectively, columns $7$-$12$ display the detailed information on the number 
of times a particular bug was found by each technique.
Runtime overheads are displayed in the columns $13$-$16$, with $\mathsf{\tt -I}$ denoting the instrumentation phase only.


\SubParagraph{\underline{Deadlock-detection capability}}
\dlfuzzer had $2076$ bug reports in total, and it found $42$ unique bugs.
In contrast, $\SyncPDOnline$ flagged $7633$ bug reports, corresponding to $49$ unique bugs.
In more detail, \dlfuzzer missed $9$ bugs reported by \SyncPDOnline whereas 
$\SyncPDOnline$ missed $2$ bugs reported by \dlfuzzer.
Also, \SyncPDOnline significantly outperformed \dlfuzzer in total number of bugs hits.
Our experiments again support that the notion of sync-preservation  captures most deadlocks that occur in practice, to the extent that other state-of-the-art techniques can capture.
%\hunkar{
A further observation is that in the offline experiments, \SyncPDOffline  was not able to find deadlocks in \texttt{Transfer} and \texttt{Deadlock}. 
However, the random scheduling procedure allowed \SyncPDOnline 
to navigate to executions from which deadlocks can be predicted.
This demonstrates the potential of combining predictive dynamic
techniques with controlled concurrency testing.
%; a direction we find promising to be pursued further by the community.


\SubParagraph{\underline{Runtime overhead}}
We have also measured the runtime overhead of both \SyncPDOnline and \dlfuzzer,
both as incurred by instrumentation, as well as by the deadlock analysis.
The latter is the time taken by \algoref{online} for the case of $\SyncPDOnline$,
and the overhead introduced due to the new executions in the second confirmation phase for the case of \dlfuzzer.
Our results show that the instrumentation overhead of \SyncPDOnline is, in fact, comparable to that of \dlfuzzer, though somewhat larger. 
This is expected, as \SyncPDOnline needs to also instrument memory access events, while \dlfuzzer only instruments lock events, but at the same time surprising because the number of memory access events
is typically much larger than the number of lock events.
On the other hand, the analysis overhead is often larger for \dlfuzzer, 
even though it reports fewer bugs.
It was not possible to measure the runtime overhead in certain benchmarks as 
either they were always deadlocking or the computation was running indefinitely.




\section{Discussion}
\label{sec:discussion}

\section{Discussion}
\label{sec:discussion}

We have constructed and analyzed mathematical models of dynamic
task allocation in a multi-robot system. The models are general
and can be easily extended to other systems in which robots use a
history of local observations of the environment as a basis for
making decisions about future actions. These models are based on
theory of stochastic processes. In order to study a robot's
behavior, we do not need to know its exact trajectory or the
trajectories of other robots; instead, we derive a probabilistic
model that governs how a robot's behavior changes in time. In some
simple cases these models can be solved analytically. However,
stochastic models are usually too complex for exact analytic
treatment. Thus, in the scenario described in
\secref{sec:pucksonly} in which only observations of tasks are
made, though the individual model is tractable, the stochastic
model of the collective behavior is not. Instead, we use averaging
and approximation techniques to quantitatively study the dynamics
of the collective behavior. Such models, therefore, do not
describe the robots' behavior in a single experiment, but rather
the behavior that has been averaged over many experimental or
simulations runs. Fortunately, results of experiments and
simulations are usually presented as an average over many runs;
therefore, mathematical models of average collective behavior can
be used to describe experimental results. In fact, the stochastic
model produces excellent agreement with experimental results under
all experimental conditions and without using any adjustable
parameters.


Phenomenological models are more straightforward to construct and
analyze than exact stochastic models --- in fact, they can be
easily constructed from details of the individual robot
controller~\cite{Lerman04sab}. The ease of use comes at a price,
namely, the number of simplifying assumptions that were made in
order to produce a mathematically tractable model. First, we
assume that the robots are functioning in a dilute limit, where
they are sufficiently separated that their actions are largely
independent of one another. Second, we assume that the transition
rates can be represented by aggregate quantities that are
spatially uniform and independent of the details of the individual
robot's actions or history. We also assume the system is
homogeneous, with modeled robots characterized by a set of
parameters, each of them representing the mean value of some real
robot feature: mean speed, mean duration for performing a certain
maneuver, and so on. Real robot systems are heterogeneous: even if
the robots are executing the same controller, there will always be
variations due to inherent differences in hardware. We do not
consider parameter distributions in our models as would be
necessary to describe such heterogeneous systems. Finally,
phenomenological models more reliably describe systems where
fluctuations (deviations from the mean behavior) can be neglected,
as happens in large systems or when many experimental runs are
aggregated. However, even if phenomenological models don't agree
with experiments exactly, as we saw in \secref{sec:results2}, they
can still reliably predict most behaviors of interest even in
not-so-large systems. They are, therefore, a useful tool for
modeling and analyzing multi-robot systems.


% \subsubsection*{Reproducibility Statement}

% Our supplementary material contains the source code we used to produce all results in this work, including the code used to train the agents, our implementation of HetGPS, and the code used to produce all plots in the paper (see Appendix \ref{app:code_details}). The \texttt{readme} contains detailed instructions on how to use this code. All mathematical claims made in this work are fully proven in the Appendix, and our assumptions are described in detail in Section \ref{sec:problemsetting} (``Problem Setting'').  Appendices \ref{app:adverseinitializations} and \ref{app:stability_of_bilevel_optimization} addresses potential questions readers may have regarding the stability of the bilevel optimization process used in HetGPS and similar environment design algorithms in the literature. Finally, Appendix \ref{appendix:limitations} addresses the assumptions and scope of our work, and outlines some remaining open questions.


% It is important that the work published in ICLR is reproducible. Authors are strongly encouraged to include a paragraph-long Reproducibility Statement at the end of the main text (before references) to discuss the efforts that have been made to ensure reproducibility. This paragraph should not itself describe details needed for reproducing the results, but rather reference the parts of the main paper, appendix, and supplemental materials that will help with reproducibility. For example, for novel models or algorithms, a link to an anonymous downloadable source code can be submitted as supplementary materials; for theoretical results, clear explanations of any assumptions and a complete proof of the claims can be included in the appendix; for any datasets used in the experiments, a complete description of the data processing steps can be provided in the supplementary materials. Each of the above are examples of things that can be referenced in the reproducibility statement. This optional reproducibility statement is not part of the main text and therefore will not count toward the page limit. 

% \subsubsection*{Author Contributions}
% If you'd like to, you may include  a section for author contributions as is done
% in many journals. This is optional and at the discretion of the authors.

% \subsubsection*{Acknowledgments}
% This work is supported by European Research Council (ERC) Project 949940 (gAIa) and ARL DCIST CRA W911NF-17-2-0181. We gratefully acknowledge their support.



\bibliographystyle{iclr2026_conference}
\bibliography{bibliography}  % Without the .bib extension

\newpage
\appendix

\section{Computational Resources Used}
\label{appendix:computeused}
For the realization of this work, we have employed computational resources that have gone towards: experiment design, prototyping, and running final experiment results.
Simulation and training are both run on GPUs, no CPU compute has been used. Results have been stored on the WANDB cloud service.
We estimate:
\begin{itemize}
    \item  300 compute hours on an NVIDIA GeForce RTX 2080 Ti GPU. 
    \item  500 compute hours on an NVIDIA L40S GPU.
\end{itemize}

Simply reproucing our results using the available code will take considerably less compute hours (around a day).

\section{Code and Data Availability}
\label{appendix:code_availability}

We attach the code in the supplementary materials.
The code contains instructions on how to reproduce the experiments in the paper and dedicated YAML files containing the hyperparameters for each experiment presented. The YAML files are structured according to the HYDRA~\citep{yadan2019hydra} framework which allows smooth reproduction as well as systematic and standardized configuration. We further attach all scripts to reproduce the plots in the paper from the experiment results.

\section{Implementation Details}
\label{app:code_details}
For all experiments, we use the MAPPO MARL algorithm~\citep{yu2022surprising}.
Environments are implemented in the multi-agent environment simulator VMAS~\citep{bettini2022vmas}, and trained using TorchRL~\citep{boutorchrl}. Both the actor and critic are two-layer MLPs with 256 neurons per layer and Tanh activation. Further details, such as hyperparameter choices, are available in the attached code and YAML configuration files.

Experiments were run on a single Nvidia L40 GPU. In the HetGPS experiments (\autoref{sec:HetGPS}), a standard MARL training iteration (60,000 frames) takes approx. 15s; including the environment backpropagation increases this to approximately 20s.

\section{Use of LLMs}
\label{appendix:useofllms}
We used LLMs (ChatGPT 4o and Gemini 2.5 Pro) to improve some parts of the writing, e.g., make wording suggestions. We verified and take responsibility for all LLM-related outputs in this work.

\section{Colonel Blotto \& Level-Based Foraging}
\label{appendix:examples_of_marl_environments}

We describe how two well-known settings from the literature fit into our theoretical framework, and check what our theoretical results say about their  \heterogeneitygap. 

\subsection{Team Colonel Blotto (fixed adversary)}
The \textit{Colonel Blotto game} is a well-known allocation game studied in both game theory and MARL \citep{Roberson2006,noel2022reinforcementcolonelblotto}. It is used to model election strategies and other resource-based competitions. In the team variant with fixed adversary,  \(N\) friendly colonels (agents) each distribute a (fixed and equal)  budget of troops \(r_{ij}\!\ge 0, \sum_{j=1}^M r_{ij} = 1\) across \(M\) battlefields \(j\in\{1,\dots,M\}\) (our tasks).  
A fixed adversary selects a \emph{stochastic} opposing allocation
strategy, i.e.\ a distribution \(\pi_{\mathrm{adv}}\) over vectors
\(a=(a_1,\dots,a_M)\) which is fixed throughout
training and evaluation.  
Let \(s_j=\sum_{i=1}^{N} r_{ij}\) denote the team force committed to
battlefield \(j\). Our agents win against the adversary if the troops they allocate to a given field surpass the troops allocated by the adversary. The
expected value secured on battlefield \(j\) is therefore 

\[
T_j(\mathbf a_j)\;=\;v_j\,
\mathbb E_{a\sim\pi_{\mathrm{adv}}}\!\bigl[\mathbf 1\![\,s_j>a_j\,]\bigr]
   \;=\;v_j\,\Pr_{a\sim\pi_{\mathrm{adv}}}\!\bigl[s_j>a_j\bigr],
\]

where $1[x > y]$ denotes the indicator function. This is a \textbf{thresholded-sum} that remains symmetric and
coordinate-wise non-decreasing in every agent’s contribution \(r_{ij}\).
Aggregating across battlefields with a value-weighted sum yields the team
reward

\[
R(\mathbf A)\;=\;\sum_{j=1}^{M} T_j(\mathbf a_j)
           \;=\;\underbrace{\sum_{j=1}^{M}}_{U}\;
               T_j\!\Bigl(\underbrace{\sum_{i=1}^{N}}_{\inneragg}
                          r_{ij}\Bigr),
\]

so the game fits the double-aggregation
structure \(R(\mathbf A)=\outeragg_{j}\inneragg_{i}r_{ij}\) assumed in
our analysis.  

\underline{\HeterogeneityGap:} This is a continuous allocation game, and the inner aggregator $T_j$ is an indicator function over the sum of troop allocations to battlefield $j$. This function is Schur-concave (and Schur-convex at the same time!). Hence, by \autoref{thm:heterogeneity-gap-schurconcave}, heterogeneous colonel teams, where each colonel has a distinct troop allocation strategy, have no advantage over homogeneous teams, where all colonels employ the same allocation strategy: $\Delta R = 0$. This makes sense, as it makes no difference whether two different colonels allocate $x/2$ troops to a battlefield, or one colonel allocates $x$ troops to the battlefield.

Our analysis also tells us what happens when we change $T_j$: this provides insights for generalizations of the Colonel Blotto game. For example, maybe the troops of different colonels don't cooperate as well with each other, such that two colonels allocating $x/2$ troops to a battlefield results in a lower $T_j$-value than a single colonel allocating $x$ troops. In this case, $T_j$ becomes strictly Schur-convex, and \autoref{thm:heterogeneity-gap-schurconvex} tells us that $\Delta R > 0$ as long as the optimal allocation is non-trivial. Hence, heterogeneous teams are advantaged.

\subsection{Level-Based Foraging} 
The well-known \textit{level-based foraging} (LBF) benchmark, based on the  knapsack problem \citep{garey1990guidenpcompleteknapsack}, is a deceptively challenging, embodied MARL environment, where \(N\) agents are placed on a grid with \(M\) food items, and are tasked with collecting them. Each item \(j\) has an integer level \(L_j\) that must be met or exceeded by the combined skills of the agents standing on that cell before it can be collected  \citep{papoudakis2021benchmarking_multilevelforaging}.  
Let agent \(i\)’s skill be \(e_i\).  At a given step the binary variable  

\[
r_{ij}\;\in\;\{0,e_i\},
\qquad 
\text{with}\; \sum_{j=1}^{M} r_{ij}\le e_i ,
\]

denotes whether \(i\) contributes its skill to item \(j\).  In our setting, we assume all agents are equally skilled, so $e_i = 1\;\forall i$. Collecting these variables thus yields an allocation matrix \(\mathbf A=[r_{ij}] \in \{0,e_i\}^{N\times M}\), which again matches our framework.

\paragraph{Inner aggregator.}  
A food item is harvested if the summed skill on its cell reaches the threshold, so  

\[
T_j(\mathbf a_j)\;=\;L_j\;
\mathbf 1\!\bigl[\textstyle\sum_{i=1}^{N} r_{ij}\;\ge\;L_j\bigr],
\qquad 
\mathbf a_j=(r_{1j},\dots,r_{Nj})^\top .
\]

This \textbf{threshold–sum} is symmetric and monotone, depending only on the
sum of its arguments and therefore simultaneously Schur-convex and
Schur-concave.

\paragraph{Outer aggregator.}  
The stepwise team reward is the sum of harvested item
values,

\[
R(\mathbf A)
\;=\;
\sum_{j=1}^{M} T_j\!\Bigl(\sum_{i=1}^{N} r_{ij}\Bigr)
\;=\;
\underbrace{\sum_{j=1}^{M}}_{U}\;
      T_j\!\bigl(\underbrace{\sum_{i=1}^{N}}_{\inneragg} r_{ij}\bigr)
\;=\;
\outeragg_{j=1}^{M}\inneragg_{i=1}^{N} r_{ij},
\]

so LBF also conforms to the
double-aggregation form \(R(\mathbf A)=\outeragg_{j}\inneragg_{i}r_{ij}\).

\paragraph{MARL Environment Reward.}  In the level-foraging environment, items that are picked up either disappear; replace themselves with different items; or replace themselves with the same item (possibly at a different cell). In all of these cases we can represent the cumulative reward as \(\sum_{t= 0}^T\gamma^{t}\,R_t\bigl(\mathbf{A}_{t}\bigr)\) for some sequence $(R_t)_{t=1,\ldots T}$ of rewards adhering to the above reward structure.


\underline{\HeterogeneityGap:} We analyze the heterogeneity gap of a specific stepwise reward $R$. 

Because this is an embodied environment where each agent can either stand on an item (\(r_{ij}=1\)) or not
(\(r_{ij}=0\)), effort allocations are \emph{discrete}.  Our continuous
curvature test therefore does not apply directly, but the discrete
analysis in \autoref{fig:deltaR-vs-softmax} (left panel) does.

The table in \autoref{fig:deltaR-vs-softmax} tells us something about the case where all items have level \(L_j=1\). In this case, since we assumed \(e_i=1\) for all agents, the inner aggregator reduces to
\[
T_j(\mathbf a_j)=\mathbf 1\!\Bigl[\textstyle\sum_{i=1}^{N} r_{ij}\ge 1\Bigr]
            =\max_i r_{ij},
\]
while the outer aggregator is an unnormalized sum, which becomes the \textit{mean} when divided by $M$. Hence \(R(\mathbf A)=\sum_j\max_i r_{ij}\), which, up to the constant
\(1/M\), is exactly the case
\(U=\text{mean},\,T=\text{max}\) of
\autoref{fig:deltaR-vs-softmax}.  That table shows
\[
\frac{1}{M} \Delta R \;=\;\frac{\min\{M,N\}-1}{M},
\]
so the heterogeneity gap is \emph{strictly positive} whenever the team
could in principle cover more than one item (\(\min\{M,N\}>1\)).
Intuitively, a homogeneous team can only collect one item per step
(all agents flock to the same cell), whereas heterogeneous agents may
spread out and capture up to \(\min\{M,N\}\) items simultaneously.

This analysis can be extended to the case where all items have the same level \(L>1\) and \(L\mid N\) by grouping agents into \(\tilde N:=N/L\) \emph{agent teams}, each bundle contributing exactly \(L\) units of skill. This yields
\[
\frac{1}{ML} \Delta R \;=\;\frac{\min\{M,\tilde N\}-1}{M}.
\]
(We omit the formal analysis, which is not difficult). Thus, if the team can form at least two such bundles
(\(\tilde N>1\)), heterogeneity is again advantageous. If it cannot, then $\Delta R = 0$, and there is no advantage to heterogeneity. 

When the levels \(\{L_j\}\) differ, an exact closed form is harder, but in general we expect $\Delta R > 0$ whenever there is some combination of items that the heterogeneous team can collect, which in total is worth more than the largest single item that can be collected if all $N$ agents stand on its cell.

In LBF, therefore, our theory suggests that behavioral diversity is often advantageous. Note that (unlike the Colonel Blotto game) since LBF is an embodied, time-extended MARL environment, this analysis does not \textit{formally guarantee} an advantage to RL-based heterogeneous agent teams: rather, it identifies that there are  effort allocation strategies that will give these teams an advantage over homogeneous teams. The agents must still \textit{learn} and be able to execute these strategies to gain this advantage (e.g., they must learn how to move to attain the desired allocations).

\section{Sum-Form Aggregators}
\label{appendix:sumform_aggregators}

Many useful reward functions are  \textit{sum-form aggregators}:

% If our aggregators can be expressed simply as \textit{sums of functions}, we can extend our analysis to conventional convexity and concavity by carefully examining how individual task-level reward functions and the outer aggregator influence the \heterogeneitygap. Concretely, let us define:

\begin{definition}[Sum-Form Aggregator]
\label{def:sum_task_agg}
A task-level aggregator $f:  \mathbb{R}^N \to \mathbb{R}$ for task $j$ is a \textbf{sum-form aggregator} if it can be written as:
\(
f(\mathbf{x}_j) 
\;=\;
\sum_{i=1}^{N} g(x_{j}),
\)
where $g_{j}:\mathbb{R}\to\mathbb{R}$ is differentiable. We say $f$ is (strictly) \emph{convex or concave} if $g$  is (strictly) convex or concave, respectively.
\end{definition}


\autoref{tab:param-agg-extended} contains examples. When our aggregators have this form, Schur-convexity (concavity) is determined by whether $g$ is convex (concave)--a simple computational test.   This is because of the following \textit{known} connection between sum-form aggregators and Schur-convexity/concavity: 



\begin{lemma}[Schur Properties of Sum-Form Aggregators \citep{peajcariaac1992convex}]
\label{lemma:schur_sum_form}
Given sum-form task-level aggregator $f(\mathbf{x}) = \sum_{i=1}^{N} g(x_i)$, the following holds: \textbf{\emph{(i)}} if $g$ is (strictly) convex, then $f$ is (strictly) Schur-convex; and \textbf{\emph{(ii)}}  if $g$ is (strictly) concave, then $f$ is (strictly) Schur-concave.
\end{lemma}

This lemma simplifies checking the conditions of our \heterogeneitygap results. For example, the following corollary can be used to establish $\Delta R > 0$ for many of the aggregators in \autoref{tab:param-agg-extended}:

\begin{corollary}[Convex-Concave Positive \HeterogeneityGap]
\label{prop:concavity_forces_multi_task}
Let $N,M \ge 2$. Let $g:[0,1]\to\mathbb{R}_{\ge 0}$ be a non-negative strictly convex function satisfying $g(0)=0$, and let $h:\mathbb{R}_{\ge 0}\to\mathbb{R}$ be a strictly concave, increasing function satisfying $h(0)=0$. If each task-level aggregator is a strictly convex sum-form aggregator  
\(
  T_j(\mathbf{a}_j)
  \;=\;
  \sum_{i=1}^N g\bigl(r_{ij}\bigr)
\), and the outer aggregator is a strictly concave sum-form aggregator $U(\mathbf{y}) = \sum_{j=1}^M h(y_j)$, then 
\(
  \Delta R > 0
\).

%, and in the optimal allocation, each agent puts effort $1$ on a different task.
\end{corollary}

\begin{proof}[Proof of \autoref{prop:concavity_forces_multi_task}]
We will apply Theorem~\ref{thm:heterogeneity-gap-schurconvex} by verifying its conditions:

First, by Lemma~\ref{lemma:schur_sum_form}, since $g$ is strictly convex, the (identical) task-level aggregators $T_j(\mathbf{x}) = \sum_{i=1}^{N} g(x_i)$ are strictly Schur-convex, satisfying condition (i) of Theorem~\ref{thm:heterogeneity-gap-schurconvex}.

Second, the outer aggregator $U(y_1,\ldots,y_M)$ is strictly increasing at every coordinate by definition, satisfying condition (ii).

Hence, the conditions of \autoref{thm:heterogeneity-gap-schurconvex} apply. To establish $\Delta R > 0$, it remains to verify that the optimal allocation is non-trivial: it distributes effort across at least two tasks. In any  admissible \emph{homogeneous} solution, each of the $N$ agents chooses the same effort‐distribution $(c_1,\dots,c_M)$ on tasks, with $\sum_j c_j=1$.  
Then task $j$’s reward is $T_j=N\,g(c_j)$, so
\(
  R(\mathbf{A})
  \;=\;
  \sum_{j=1}^M h\!\bigl(N\,g(c_j)\bigr).
\)
The trivial, \emph{all-agent single‐task allocation} uses $(c_j=1,c_{k\neq j}=0)$.  Its reward is therefore 
\(
  R_{\text{corner}}
  =
  h\!\bigl(N\,g(1)\bigr)
  + 
  \sum_{k\neq j} h\bigl(N\,g(0)\bigr)
  =
  h\!\bigl(N\,g(1)\bigr)
\)
since $g(0)=0$ and $h(0)=0$.

Strict concavity of $h$ implies that $h\!\bigl(N\,g(1)\bigr) < N \cdot h(g(1))$. Hence, agents can attain a better reward by allocating effort $1$ to $N$ different tasks rather than a single task. This shows that  the best solution \emph{must} use at least two nonzero $c_j$, completing the proof. 
\end{proof}

\section{Formal Analysis}
\label{appendix:formal_analysis}
\subsection{Proof of \autoref{thm:heterogeneity-gap-schurconvex}}
\begin{proof}[Proof of \autoref{thm:heterogeneity-gap-schurconvex}] Let $\mathbf{A}_{\mathrm{hom}}$ be an optimal homogeneous allocation (i.e., $R(A_{hom}) = R_{hom}$), whose $i$th row is the vector 
\[
  \mathbf{c} \;=\; (c_1,\dots,c_M) \quad \text{with} \quad \sum_{j=1}^M c_j = 1.
\]
Then each column $j$ of $\mathbf{A}_{\mathrm{hom}}$ is the uniform vector 
\(
  \mathbf{u}_j \;=\; (c_j,c_j,\dots,c_j)^\top \;\in \mathbb{R}^N.
\)
Hence the task‐level reward is $T_j(\mathbf{u}_j)$, and the overall reward is
\[
R\bigl(\mathbf{A}_{\mathrm{hom}}\bigr) = U\!\bigl(T_1(\mathbf{u}_1),\dots,T_M(\mathbf{u}_M)\bigr).
\]

Because $\sum_j c_j=1$, there is at least one task $j$ with $c_j > 0$. We construct a heterogeneous allocation $A_{het}$ such that each column $x_j$ in $A_{het}$ has the same sum as the corresponding column in $A_{hom}$. 

The total effort allocated to a task $j$ can be expressed as $\lfloor N c_j \rfloor + f_k$, where  $0 \leq f_j < 1$. First, we assign $\lfloor N c_j \rfloor$ agents to allocate effort $1$ to task $j$, for every task $j$. These agents are all distinct. This leaves us with $\sum_j f_j~=~N~-~\sum_j \lfloor N c_j \rfloor$ agents that have not allocated any effort yet. Let $i$ be the first of those agents. We have agent $i$ allocate $f_1$ effort to task $1$, $f_2$ effort to task $2$, and so on, until we arrive at a task $k$ such that $f_1 + \ldots + f_k = 1 + s$, for some $s > 0$. We have $i$ allocate $f_k - s$ to this task $k$. Then, we move to agent $i+1$, and allocate the remaining fractional efforts in the same manner (and in particular, allocating $s$ effort to task $k$), until agent $i+1$ overflows. Then we move to agent $i+2$, and so on. This ensures that we have allocated $N$ effort in total across the agents, and that every agent's effort allocation sums exactly to $1$, so is feasible.

Let $x_j$ be the $j$th column of $A_{het}$. We note the following fact: any non-uniform vector whose sum is $Nc_j$ majorizes the uniform vector $u_j$. Hence, $T_j(x_j) \geq T_j(u_j)$, with equality only if $x_j = u_j$.  This means that if $A_{het} \neq A_{hom}$, then 

\[
  R\bigl(\mathbf{A}_{\mathrm{het}}\bigr)
  \;=\;
  U\!\bigl(T_1(\mathbf{x}_1),\dots,T_M(\mathbf{x}_M)\bigr)
  \;>\;
  U\!\bigl(T_1(\mathbf{u}_1),\dots,T_M(\mathbf{u}_M)\bigr)
  \;=\;
  R\bigl(\mathbf{A}_{\mathrm{hom}}\bigr).
\]

We note that $A_{hom} = A_{het}$ only if $A_{hom}$ is a trivial allocation, as $A_{het}$ contains at least one agent allocating effort $1$ to some task, and $A_{hom}$'s agents only allocate fractional efforts, if it is non-trivial. Otherwise, since $R(A_{hom}) = R_{hom}$, the above inequality implies \(\Delta R = R_{\mathrm{het}} - R_{\mathrm{hom}} > 0 \). This completes the proof.  
\end{proof}

\subsection{Proof of \autoref{thm:heterogeneity-gap-schurconcave}}

\begin{proof}[Proof of \autoref{thm:heterogeneity-gap-schurconcave}]
Let $\mathbf{A}$ be an arbitrary feasible allocation, and let $\mathbf{A}_{\mathrm{hom}}$ be a \emph{homogeneous} allocation with the same column sums.  Concretely, for each column $j$, define
\[
  s_j
  \;=\;
  \sum_{i=1}^N r_{ij}
  \quad\text{and}\quad
  \mathbf{u}_j
  \;=\;
  \bigl(\tfrac{s_j}{N},\tfrac{s_j}{N},\dots,\tfrac{s_j}{N}\bigr)^\top,
\]
so $\mathbf{u}_j$ is the \emph{uniform} distribution of total mass $s_j$ across $N$ agents.  
Then construct
\[
  \mathbf{A}_{\mathrm{hom}}
  \;=\;
  \begin{pmatrix}
  \tfrac{s_1}{N} & \cdots & \tfrac{s_M}{N} \\
  \vdots & \ddots & \vdots\\
  \tfrac{s_1}{N} & \cdots & \tfrac{s_M}{N}
  \end{pmatrix},
\]
which is clearly \emph{homogeneous} (each row is the same), and respects each column sum $s_j$. Since $\sum_j s_j = N$, each row sums to $1$, hence the allocation is feasible.  By Schur‐concavity of $T_j$, for each column $j$ we have
\[
  \mathbf{a}_j \;\succ\; \mathbf{u}_j
  \quad\Longrightarrow\quad
  T_j(\mathbf{a}_j) 
  \;\leq\;
  T_j(\mathbf{u}_j),
\]
unless $\mathbf{a}_j$ is $\mathbf{u}_j$.  
In other words, \emph{any} deviation from the uniform vector with the same sum \(\sum_{i=1}^N a_{ji} = s_j\) will not increase $T_j(\mathbf{a}_j)$ under Schur‐concavity. Hence for each column $j$ of $\mathbf{A}$, 
\(
  T_j(\mathbf{a}_j) 
  \;\le\;
  T_j(\mathbf{u}_j),
\).
Since $U$ is non-decreasing in each coordinate, 
\[
  R\bigl(\mathbf{A}\bigr)
  \;=\;
  U\!\bigl(T_1(\mathbf{a}_1),\dots,T_M(\mathbf{a}_M)\bigr)
  \;\le\;
  U\!\bigl(T_1(\mathbf{u}_1),\dots,T_M(\mathbf{u}_M)\bigr)
  \;=\;
  R\bigl(\mathbf{A}_{\mathrm{hom}}\bigr).
\]
 This implies 
\(
  \Delta R 
  \;=\;
  0.
\)
\end{proof}

\subsection{Proof of \autoref{thm:no-gap-schurconvex-outer-detailed}}


\begin{proof}[Proof of \autoref{thm:no-gap-schurconvex-outer-detailed}]

By hypothesis, the components of the task score vector 
\[
  \textbf{T}(\textbf{A}) = \Bigl(T_1(\mathbf{a}_1),\,T_2(\mathbf{a}_2),\,\dots,\,T_M(\mathbf{a}_M)\Bigr)
\]
always  sum to $C$. By strict Schur-convexity, the maximum value of $U$ over such vectors is attained precisely at an extreme point of the $C$-simplex, i.e.\ at some permutation of $(C,0,\ldots,0)$. Hence, we seek to find an allocation of efforts, $\mathbf{A}_{\mathrm{corner}}$, that causes $\textbf{T}(\textbf{A})$ to equal this vector.

Let each agent $i$ invest \emph{all} of its effort into task~1. This is the trivial allocation.  Then the first column of $\mathbf{A}_{\mathrm{corner}}$ is
\((1,1,\dots,1)^\top\),
and all other columns $\mathbf{a}_j$ are zero.  Since task scores sum to $C$, we get 
\(
  T_1\bigl(\mathbf{a}_1\bigr)
  =
  C,
  \quad
  T_j\bigl(\mathbf{a}_j\bigr)
  =
  0
  \;\;\text{for }j\neq 1.
\)
By assumption (2), we infer that the vector of task-level scores is indeed $(C,0,\dots,0)$.  

Notice that \emph{each row of $A_{corner}$ is the same} $(1,0,\dots,0)$, making $\mathbf{A}_{\mathrm{corner}}$ a \emph{homogeneous} allocation. Hence, we attained the maximum possible reward $R(\textbf{A})$ through a homogeneous allocation, implying $\Delta R = 0$. 
\end{proof}

\subsection{Proof of \autoref{thm:gap_NeqM_softmax_hetgap}}

Before proving the statement, let's write the expressions for homogeneous and heterogeneous optima. For each task \( j \), we defined
\[
T_j(\mathbf{A})
=
\sum_{i=1}^N 
\frac{\exp\bigl(t \cdot r_{i j}\bigr)}{\sum_{\ell=1}^N \exp\bigl(t \cdot r_{\ell j}\bigr)}
\; r_{i j},
\]

while defining the outer aggregator to be 
\[
U\bigl(T_1(\mathbf{a}_1) , \ldots T_M(\mathbf{a}_m) )
=
\sum_{j=1}^M 
\frac{\exp\bigl(\tau \cdot T_j(\mathbf{A})\bigr)}
     {\sum_{\ell=1}^M \exp\bigl(\tau \cdot T_\ell(\mathbf{A})\bigr)}
\; T_j(\mathbf{A}),
\]

where \(t, \tau \in \mathbb{R}\) are temperature parameters. In the \textbf{homogeneous setting}, where all agents share the same allocation \(\mathbf{c} = (c_1,\dots,c_M)\), we therefore have
  \(
  T_j(\mathbf{A})
  =
  \sum_{i=1}^N 
    \frac{\exp\bigl(t\,c_j\bigr)}{\sum_{\ell=1}^N \exp\bigl(t\,c_j\bigr)}
  \; c_j
  =
  c_j.
  \)
  Thus,
  \[
  R_{\mathrm{hom}}
  =
  \max_{\mathbf{c} \,\in\, \Delta^{M-1}}
  \quad
  \sum_{j=1}^M 
  \frac{\exp\bigl(\tau\,c_j\bigr)}{\sum_{\ell=1}^M \exp\bigl(\tau\,c_\ell\bigr)}
  \; c_j
  \]

where $\Delta^{M-1}$ is the simplex of all admissible allocations. 

  In the general \textbf{heterogeneous setting}, each row \((r_{i1},\dots,r_{iM})\) can be
  different. Then
  \[
  T_j(\mathbf{A})
  \;=\;
  \sum_{i=1}^N 
  \frac{\exp\bigl(t\,r_{i j}\bigr)}{\sum_{\ell=1}^N \exp\bigl(t\,r_{\ell j}\bigr)} \;
  r_{ij},
  \]
  and we choose \(\mathbf{A}\in(\Delta^{M-1})^N\) to maximize
  \[
  R_{\mathrm{het}} 
  \;=\;
  \max_{\mathbf{A}} 
  \sum_{j=1}^M 
  \frac{\exp\bigl(\tau \,T_j(\mathbf{A})\bigr)}{\sum_{k=1}^M \exp\bigl(\tau\,T_k(\mathbf{A})\bigr)}
  \; T_j(\mathbf{A}).
  \]

Keeping these expressions in mind, we proceed with the proof of \autoref{thm:gap_NeqM_softmax_hetgap}.

\underline{Reminder:} assuming $N = M \geq 2$, we want to prove $\Delta R(t,\tau;N)=0$ when $t \leq 0$, and 
\[
\boxed{%
\Delta R(t,\tau;N)\geq
\begin{cases}
\sigma(t,N)-\dfrac1N, & t>0,\;\;\tau\le 0,\\[10pt]
\max\!\bigl\{\sigma(t,N)-\sigma(\tau,N),\,0\bigr\}, & t>0,\;\;\tau\ge 0.
\end{cases}}
\] otherwise, where 
\(\sigma(t,N)\;:=\;\frac{e^{t}}{e^{t}+N-1}.
\)

%--------------------------------------------------------
%  Proof of Theorem 4.4  (heterogeneity gap, soft-max / soft-max, N=M)
%--------------------------------------------------------

\begin{proof}[Proof of \autoref{thm:gap_NeqM_softmax_hetgap}]

When $t\le0$, $T_j$ is Schur–concave, so $\Delta R = 0$ by \autoref{thm:heterogeneity-gap-schurconcave}. We assume $t > 0$ for the rest of the proof.

\textit{Homogeneous optimum.}
If every row of $\mathbf A$ equals the same allocation
$\mathbf c\in\Delta^{N-1}$, then $T_j(\mathbf A)=c_j$. 
$U$ is Schur–concave for $\tau\le0$, and Schur–convex for $\tau\ge0$, hence it is maximized by the uniform distribution in the former case, and by a $1$-hot vector in the latter case, yielding:

\[
  R_{\mathrm{hom}}
  \;=\;
  \max_{\mathbf c\in\Delta^{N-1}}U(\mathbf c)
  \;=\;
  \begin{cases}
     \dfrac1N,          & \tau\le0,\\[6pt]
     \sigma(\tau,N),    & \tau>0.
  \tag{H}
  \end{cases}
\]

\textit{Lower bound on $R_{het}$.}
The \emph{trivial} allocation, where every agent works on the same task, produces 
$R_{\mathrm{trivial}}=\sigma(\tau,N)$.
The \emph{spread} allocation, where agent $i$ works exclusively on task $i$,
makes each column ``one-hot''; this gives
$T_j=\sigma(t,N)$ for all $j$, and plugging this into $U$, we get 
$R_{\mathrm{spread}}=\sigma(t,N)$.
Consequently

\[
  R_{\mathrm{het}}\;\ge\;\max\{\sigma(t,N),\,\sigma(\tau,N)\}.
\tag{L}
\]

Combining (H) and (L) gives the desired lower bound.
\end{proof}

\section{Deriving the $\{\min, \textnormal{mean}, \max\}$ heterogeneity gains in the \autoref{fig:deltaR-vs-softmax} table}

We derive these heterogeneity gain case-by-case. \autoref{tab:continuous-minmeanmax} summarizes the derivation for continuous allocations ($r_{ij} \in [0,1]$), and  \autoref{tab:discrete-minmeanmax} does the same for discrete effort allocations ($r_{ij} \in \{0,1\}$).

\begin{table}[h!]
\centering
% Control vertical spacing between table rows:
\renewcommand{\arraystretch}{1.1}
% Control horizontal spacing between columns:
\setlength{\tabcolsep}{4pt}
\scriptsize
\caption{All nine extreme cases of inner/outer aggregators belonging to the set  $\{\min, \text{mean}, \max\}$. 
In each cell, we show the best possible outcome for Heterogeneous vs.\ Homogeneous 
allocations and the resulting \(\Delta R\).}
\label{tab:continuous-minmeanmax}
\begin{tabular}{l|p{3.8cm}p{3.8cm}p{3.8cm}}

 & $T = \min$ & $T = \text{mean}$ & $T = \max$ \\
\hline
\multicolumn{1}{c|}{$U = \min$} 
& \textbf{Inner:} $T_j=\min_i r_{ij}.$

\textbf{Best $R_{het}$, $R_{hom}$:}
All must have $r_{ij}\ge x$ to push $\min_i r_{ij}\!=\!x$,
so $x\le 1/M.$

$\implies T_j=1/M.$

\textbf{Outer:} $\min_j T_j=1/M \implies R=1/M.$

\textbf{Gap: }0.
& \textbf{Inner:} $T_j=\tfrac{1}{N}\sum_i r_{ij}$ (avg over $i$).

\textbf{Maximize $\min_j T_j$:}
Both $R_{het}$, $R_{hom}$ must make $T_j$ all equal (for best min),
so $T_j=1/M.$

\textbf{Outer:} $\min_j T_j=1/M \implies R=1/M.$

\textbf{Gap: }0.
& \textbf{Inner:} $T_j=\max_i r_{ij}.$

\textbf{Outer:} picks $\min_j T_j$.

$R_{het}$: $\min_j T_j=1 \implies R=1.$

$R_{hom}$: $\min_j T_j=1/M \implies R=1/M.$

\textbf{Gap: }$1 - \tfrac{1}{M} = \tfrac{M-1}{M}.$
\\
\hline

\multicolumn{1}{c|}{$U = $ mean} 
& \textbf{Inner:} $T_j = \min_i r_{ij} = 1/M.$

\textbf{Outer:} simple avg $\tfrac{1}{M}\sum_j T_j.$  

Since $\sum_j T_j = M \cdot (1/M) = 1 \implies R=1/M.$

Both $R_{het}$, $R_{hom}$ same $\implies \Delta R=0.$

\textbf{Gap: }0.

& \textbf{Inner:} $T_j = \tfrac{1}{N}\sum_i r_{ij}.$  

Then $\sum_j T_j=1.$

\textbf{Outer:} avg $\!=\!\tfrac{1}{M}\sum_j T_j.$

Hence $R = \tfrac{1}{M}\cdot 1 = \tfrac{1}{M}.$

Same for $R_{het}$, $R_{hom}$.

\textbf{Gap: }0.
& \textbf{Inner:} $T_j = \max_i r_{ij}.$  

% $R_{het}$: $T_j=1$ for all $j.$  

% $R_{hom}$: one $T_j=1$, rest 0.

\textbf{Outer:} avg $\!=\!\tfrac{1}{M}\sum_j T_j.$  

$R_{het}$: sum $= M \implies R=1.$

$R_{hom}$: sum $=1 \implies R=1/M.$

\textbf{Gap: }$1 - \tfrac{1}{M} = \tfrac{M-1}{M}.$
\\
\hline

\multicolumn{1}{c|}{$U = \max$} 
& \textbf{Inner:} $T_j = \min_i r_{ij}$ can be made $1$ for one task.

\textbf{Outer:} picks $\max_j T_j = 1 \implies R=1$

Same for $R_{het}$, $R_{hom}$.

\textbf{Gap: }0.
& \textbf{Inner:} $T_j =$ avg over $i.$

\textbf{Outer:} picks $\max_j T_j.$  

Both $R_{het}$, $R_{hom}$ can put all effort into one task 
to get $T_j=1,$ so $R=1.$

\textbf{Gap: }0.
& \textbf{Inner:} $T_j = \max_i r_{ij}.$

\textbf{Outer:} picks $\max_j T_j.$  

Both $R_{het}$, $R_{hom}$ can achieve $\max_j=1 \implies R=1.$

\textbf{Gap: }0.
\\
\hline
\end{tabular}


\end{table}

\begin{table}[h!]
\centering
\renewcommand{\arraystretch}{1.1}
\setlength{\tabcolsep}{4pt}
\scriptsize
\caption{
A “9 extreme cases” table for \emph{discrete, one-task-per-agent} allocations. }
\label{tab:discrete-minmeanmax}
\begin{tabular}{l|p{4.1cm}p{4.1cm}p{4.1cm}}
\hline
& $\min$ & mean & $\max$\\
\hline

\multicolumn{1}{c|}{$\min$} 
& 
\textbf{Inner:} 
\[
T_j \!\to\!
\begin{cases}
1, & \text{if all agents pick $j$},\\
0, & \text{otherwise}.
\end{cases}
\]

\textbf{Outer:} 
 $\min_j T_j$. 
To get $R > 0$, must have $T_j > 0$ for \emph{every} $j$ 
(i.e.\ all agents pick \emph{all} tasks, impossible).

Hence 
$R_{\mathrm{het}}=R_{\mathrm{hom}}=0$ typically,
$\Delta R=0.$

&
\textbf{Inner:}
$T_j = \frac{|\mathcal{I}_j|}{N}$

\textbf{Outer:} 
 $\min_j T_j$.

$R_{het} = \lfloor N / M \rfloor / N$. $R_{hom} = 0$. $\Delta R = \lfloor N / M \rfloor / N$.

% All tasks $j$ must have $|\mathcal{I}_j|/N$ as large as possible equally, or else the min is smaller.  
% But each agent chooses exactly one task, so $|\mathcal{I}_j|$ sums to $N$.  
% Maximizing the min across $j$ is best if distribution is \emph{as uniform as possible}.  

% Both $R_{\mathrm{het}},R_{\mathrm{hom}}$ can do “all $|\mathcal{I}_j|$ equal if $N$ is multiple of $M$,” 
% so $\min_j T_j = \frac{N/M}{N} = 1/M.$  
% Hence $R=1/M$, 
% $\Delta R=0.$

&
\textbf{Inner:}
\[
T_j \!\to\!
\begin{cases}
1, & \text{if at least 1 agent picks $j$},\\
0, & \text{if no agent picks $j$}.
\end{cases}
\]

\textbf{Outer:} 
$\min_j T_j.$

- \emph{Heterogeneous} can choose $s$ distinct tasks. 
  If want $\min_j=1$, must pick \emph{all} $M$ tasks. 
  That requires $N\ge M$. 
  Then $R=1.$ 
- \emph{Homogeneous} covers only 1 task $\implies \min_j=0$ for $M>1 \implies R=0.$

$\Delta R=1$ if $N\ge M$, else $0.$
\\
\hline

\multicolumn{1}{c|}{mean} 
& 
\textbf{Inner:}
$T_j=1$ only if all pick $j$, else 0.

Summation $\sum_j T_j$ is number of tasks chosen by \emph{all} agents.  
Usually 0 or 1.

\textbf{Outer:}
Average across $j$.  
$R = \frac{1}{M}\sum_j T_j.$

$\implies R=1/M,$ $\Delta R=0.$
&
\textbf{Inner:}
$T_j = \tfrac{|\mathcal{I}_j|}{N}$.

\textbf{Outer:}
Average across tasks: 
$
R=\frac{1}{M}\sum_{j=1}^M \frac{|\mathcal{I}_j|}{N}
= \frac{1}{M}.
$
No matter how agents are distributed, $\sum_{j=1}^M |\mathcal{I}_j|=N$.  
Hence $R_{\mathrm{het}}=R_{\mathrm{hom}}=\tfrac{1}{M},$ 
$\Delta R=0.$

&
\textbf{Inner:}
$T_j=1$ if chosen by at least 1 agent, else 0.

\textbf{Outer:}
Average across $j$: $\tfrac{1}{M}\sum_j T_j.$  
This is $\tfrac{1}{M}\cdot\bigl(\#\text{ of tasks chosen}\bigr).$

- \emph{Heterogeneous} can pick up to $\min(M,N)$ tasks, so $R= \frac{\min(M,N)}{M}.$  
- \emph{Homogeneous} covers exactly 1 task $\implies R=1/M.$

$\Delta R=\frac{\min(M,N)-1}{M}.$
\\
\hline

\multicolumn{1}{c|}{$\max$} 
& 
\textbf{Inner:} 
$T_j=1$ only if all pick $j$, else 0.

\textbf{Outer:}
 $\max_j T_j.$

$\Delta R=0.$

&
\textbf{Inner:}
$T_j=|\mathcal{I}_j|/N.$

\textbf{Outer ($\tau\to +\infty$):}
$\max_j T_j$. 
We can place \emph{all} agents on one task, get $T_j=1.$ Then $R=1.$  
Same for homogeneous or heterogeneous.  
$\Delta R=0.$

&
\textbf{Inner:}
$T_j=1$ if at least 1 picks $j$, else 0.

\textbf{Outer:}
$\max_j T_j=1$ if any agent picks $j$.  
Even a single task yields $R=1.$  
So $R_{\mathrm{hom}}=R_{\mathrm{het}}=1,$ 
$\Delta R=0.$
\\
\hline
\end{tabular}
\end{table}
\label{sec:theory_softmax}



\section{Parametrized Families of Aggregators}
\label{appendix:parametrized_aggregator_table}
The Table in this section illustrates several families of \textit{generalized aggregators} that the analysis in this paper applies to. The scalar $t$ parametrizes each family of aggregators, continuously shifting the aggregators from Schur-concave to Schur-convex. 

\begin{table}[!h]
\centering
\caption{Illustrative families of parametric (and one nonparametric) aggregators $f_t(\mathbf{x})$. 
Changing the real parameter $t$ can switch between Schur-convex and Schur-concave behaviors (on nonnegative inputs), 
or control how strongly the aggregator favors ``peaked'' vs.\ ``uniform'' distributions. 
As $t\to\pm\infty$ or $t\to 0$, many reduce to well-known extremes such as $\max$, $\min$, or the arithmetic mean. }
\label{tab:param-agg-extended}
\resizebox{1.0\textwidth}{!}{%
\begin{tabular}{@{}l p{7cm} p{6cm}@{}}
\toprule
\textbf{Name} 
& 
\textbf{Definition} 
& 
\textbf{Schur Property \& Limits}\\
\midrule

\textbf{Power-Sum} 
& 
\[
f_{t}(\mathbf{x})
~=~
\sum_{i=1}^N \bigl(x_i\bigr)^{t},
\quad
x_i \ge 0,
\quad
t > 0 
\]
& 
\begin{itemize}
  \item Strictly \emph{Schur-convex} for \(t > 1\).
  \item Strictly \emph{Schur-concave} for \(0 < t < 1\).
  \item At \(t=1\), it is linear (both Schur-convex and Schur-concave).
  \item Undefined at \(t \le 0\) if any \(x_i=0\), though one can extend with limits.
\end{itemize}
\\
% \textbf{Central-Moment} 
% & 
% \[
% \mathrm{CentralMoment}_t(\mathbf{x})
% ~=~
% \frac{1}{N}\sum_{i=1}^N 
% \bigl(x_i - \bar{x}\bigr)^{t}, t > 0
% \]
% & 
% \begin{itemize}
%   \item Strictly \emph{Schur-convex} for \(t>1\) (it rewards “spread” away from the mean).
%   \item At \(t=2\), it equals the (sample) variance up to a constant factor.
% \end{itemize}
% \\

\textbf{Power-Mean} 
& 
\[
M_{t}(\mathbf{x})
~=~
\biggl(\frac{1}{N}\sum_{i=1}^N (x_i)^{t}\biggr)^{\!\!1/t},
\quad
x_i \ge 0,\;
t \neq 0
\]
& 
\begin{itemize}
  \item Strictly \emph{Schur-convex} for \(t > 1\).
  \item Strictly \emph{Schur-concave} for \(0 < t < 1\).
  \item Reduces to arithmetic mean at \(t=1\).
  \item As \(t \to \infty\), converges to \(\max_i x_i\); 
    as \(t \to -\infty\), converges to \(\min_i x_i\).
\end{itemize}
\\

\textbf{Log-Sum-Exp (LSE)} 
& 
\[
\mathrm{LSE}_t(\mathbf{x})
~=~
\frac{1}{t}\ln\!\Bigl(\sum_{i=1}^N e^{\,t\,x_i}\Bigr),
\quad
t \neq 0
\]
& 
\begin{itemize}
  \item Strictly \emph{Schur-convex} for \(t > 0\).
  \item Strictly \emph{Schur-concave} for \(t < 0\).
  \item As \(t \to \infty\), approaches \(\max_i x_i\);
    as \(t \to -\infty\), approaches \(\min_i x_i\).
\end{itemize}
\\

\textbf{Softmax Aggregator} 
& 
\[
\mathrm{Softmax}_t(\mathbf{x})
~=~
\sum_{i=1}^N 
\frac{e^{\,t\,x_i}}{\sum_{j=1}^N e^{\,t\,x_j}}
\,x_i,
\quad
t \in \mathbb{R}
\]
& 
\begin{itemize}
  \item Strictly \emph{Schur-convex} for \(t > 0\).
  \item Strictly \emph{Schur-concave} for \(t < 0\).
  \item As \(t \to \infty\), converges to \(\max_i x_i\);
    as \(t \to -\infty\), converges to \(\min_i x_i\).
  \item At \(t=0\), each weight is \(\tfrac{1}{N}\), so 
    \(\mathrm{Softmax}_0(\mathbf{x}) = \tfrac{1}{N}\sum_i x_i\).
\end{itemize}
\\

\bottomrule
\end{tabular}
}% end of resizebox
\end{table}



\section{Additional Results in the Multi-Agent Multi-Task Matrix Game}
\label{app:matrix_game}

We report further details and results on the \heterogeneitygaps obtained in the multi-agent multi-task matrix game.

\subsection{Game formulation}
In \autoref{tab:matrix_game_formulation} we provide an example of the pay-off matrix in this game for $N=M=3$.

\begin{table}[!h]
    \centering
     \caption{Example of a Multi-Agent Multi-Task matrix game for $N=M=3$. Agents choose their actions $A=(r_{ij})$ and receive the global reward $R(\mathbf{A}) = \outeragg_{j=1}^M \inneragg_{i=1}^N r_{ij}$.}
    \label{tab:matrix_game_formulation}
\begin{tabular}{llccc} 
\toprule 
& & \multicolumn{3}{c}{Tasks} \\ 
& & {1} & {2} & {3} \\ 
\midrule 
\multirow{3}{*}{\hfill\rotatebox[origin=c]{90}{Agents}\hfill} 
& 1 & $r_{11}$ & $r_{12}$ & $r_{13}$ \\ 
& 2 & $r_{21}$ & $r_{22}$ & $r_{23}$ \\ 
& 3 & $r_{31}$ & $r_{32}$ & $r_{33}$ \\ 
\bottomrule 
\end{tabular}
\end{table}



\subsection{$N=M=2$}

We train with $N=2,M=2$ for 100 training iterations (each consisting of 60,000 frames). We report the results for the \textbf{continuous} case in \autoref{tab:matrix_game_cont} and for the \textbf{discrete} case in \autoref{tab:matrix_game_disc}. The evolution of the \heterogeneitygaps over training is shown in \autoref{fig:matrix_game}.


\begin{table}[!h]
    \centering
     \caption{\Heterogeneitygap $\Delta R \in \R_{0\leqslant x \leqslant 1}$ of the \textbf{continuous} matrix game with $N=M=2$. The results match the theoretical analysis in the Table of \autoref{fig:deltaR-vs-softmax}. We report mean and standard deviation after 6 million training frames over 9 different random seeds.}
    \label{tab:matrix_game_cont}
\begin{tabular}{llccc} 
\toprule 
& & \multicolumn{3}{c}{$T$} \\ 
& & {Min} & {Mean} & {Max} \\ 
\midrule 
\multirow{3}{*}{\hfill\rotatebox[origin=c]{0}{$U$}\hfill} 
& Min & $-0.002 \pm 0.002$ & $0.000 \pm 0.003$ & $\boldsymbol{0.504} \pm 0.007$ \\ 
& Mean & $-0.002 \pm 0.002$ & $0.000 \pm 0.000$ & $\boldsymbol{0.496} \pm 0.001$ \\ 
& Max & $-0.003 \pm 0.002$ & $-0.001 \pm 0.001$ & $0.003 \pm 0.001$ \\ 
\bottomrule 
\end{tabular}
\end{table}

\begin{table}[!h]
    \centering
       \caption{\Heterogeneitygap $\Delta R \in \R_{0\leqslant x \leqslant 1}$ of the \textbf{discrete} matrix game with $N=M=2$. The results match the theoretical analysis in \autoref{fig:deltaR-vs-softmax}. We report mean and standard deviation after 6 million training frames over 9 different random seeds.}
    \label{tab:matrix_game_disc}
\begin{tabular}{llccc} \toprule & & \multicolumn{3}{c}{$T$} \\ & & {Min} & {Mean} & {Max} \\ \midrule \multirow{3}{*}{\hfill\rotatebox[origin=c]{0}{$U$}\hfill} & Min & $0.0 \pm 0.0$ & $\boldsymbol{0.5} \pm 0.0$ & $\boldsymbol{1} \pm 0.0$ \\ & Mean & $0.0 \pm 0.0$ & $0.0 \pm 0.0$ & $\boldsymbol{0.5} \pm 0.0$ \\ & Max & $0.0 \pm 0.0$ & $0.0 \pm 0.0$ & $0.0 \pm 0.0$ \\ \bottomrule \end{tabular}
\end{table}

\begin{figure}[!ht] 
    \centering  % Centers the figure
    \includegraphics[width=\textwidth]{images/matrix_games_2_agents_graytext.pdf}  % Adjust width as necessary, replace 'example-image' with your image file path
    \caption{\Heterogeneitygap for the discrete and continuous matrix games with $N=M=2$ over training iterations. We report mean and standard deviation after 6 million training frames over 9 different random seeds. The final results match the theoretical predictions in \autoref{fig:deltaR-vs-softmax}.}  % Add your figure caption
    \label{fig:matrix_game}  % Add a label for referencing the figure in text
\end{figure}

\subsection{$N=M=4$}

In the case $N = M = 4$, the evolution of the \heterogeneitygaps during training is shown in \autoref{fig:matrix_game_4_agents}.
We further report the final obtained gains for the \textbf{continuous} case in \autoref{tab:matrix_game_cont_4} and for the \textbf{discrete} case in \autoref{tab:matrix_game_disc_4}. 


\begin{table}[!ht]
    \centering
     \caption{\Heterogeneitygap $\Delta R \in \R_{0\leqslant x \leqslant 1}$ of the \textbf{continuous} matrix game with $N=M=4$. The results match the theoretical analysis in \autoref{fig:deltaR-vs-softmax}. We report mean and standard deviation after 12 million training frames over 9 different random seeds.}
    \label{tab:matrix_game_cont_4}
\begin{tabular}{llccc} 
\toprule 
& & \multicolumn{3}{c}{$T$} \\ 
& & {Min} & {Mean} & {Max} \\ 
\midrule 
\multirow{3}{*}{\hfill\rotatebox[origin=c]{0}{$U$}\hfill} 
& Min & $-0.003 \pm 0.002$ & $0.000 \pm 0.001$ & $\boldsymbol{0.690} \pm 0.026$ \\ 
& Mean & $-0.002 \pm 0.000$ & $0.000 \pm 0.000$ & $\boldsymbol{0.722} \pm 0.002$ \\ 
& Max & $-0.037 \pm 0.023$ & $-0.009 \pm 0.005$ & $0.029 \pm 0.006$ \\ 
\bottomrule 
\end{tabular}
\end{table}

\begin{table}[!h]
    \centering
       \caption{\Heterogeneitygap $\Delta R \in \R_{0\leqslant x \leqslant 1}$ of the \textbf{discrete} matrix game with $N=M=4$. The results match the theoretical analysis in the Table of \autoref{fig:deltaR-vs-softmax}. We report mean and standard deviation after 12 million training frames over 9 different random seeds.}
    \label{tab:matrix_game_disc_4}
\begin{tabular}{llccc} \toprule & & \multicolumn{3}{c}{$T$} \\ & & {Min} & {Mean} & {Max} \\ \midrule \multirow{3}{*}{\hfill\rotatebox[origin=c]{0}{$U$}\hfill} & 
Min & $0.0 \pm  0.0$ & $\boldsymbol{0.25} \pm 0.0$ & $\boldsymbol{1.0} \pm 0.0$ \\ & 
Mean & $0.0 \pm 0.0$ & $0.0 \pm 0.0$ & $\boldsymbol{0.75} \pm 0.0$ \\ & 
Max & $0.0 \pm 0.0$ & $0.0 \pm 0.0$ & $0.0 \pm 0.0$ \\ \bottomrule \end{tabular}
\end{table}

\subsection{$N=M=8$}

To test scalability, we further report results for the discrete matrix game with $N=M=8$ in \autoref{tab:matrix_game_disc_8}. As discussed in \autoref{sec:experiments}, due to the larger agent scale and action dimensionality (8 agents and 8 tasks), for some reward structures the empirical heterogeneity gain is negative, since the neurally heterogeneous agents we train require more time to discover the optimal policy, hence lag behind their homogeneous counterparts under a fixed training budget. Increasing the number of training steps steers the negative $\Delta R$ values to $0$. This nuance aside, what is important is that the results still follow our theory exactly in terms of which reward structures yield a \textit{positive} heterogeneity gain. For such reward structures, our theory also precisely predicts the numerical value of $\Delta R$ (\autoref{fig:deltaR-vs-softmax}) . 

\begin{table}[!h]
    \centering
       \caption{\Heterogeneitygap $\Delta R \in \R_{0\leqslant x \leqslant 1}$ of the \textbf{discrete} matrix game with $N=M=8$. We report mean and standard deviation after 12 million training frames over 8 different random seeds.}
    \label{tab:matrix_game_disc_8}
\begin{tabular}{llccc} \toprule & & \multicolumn{3}{c}{$T$} \\ & & {Min} & {Mean} & {Max} \\ \midrule \multirow{3}{*}{\hfill\rotatebox[origin=c]{0}{$U$}\hfill} & 
Min & $0.0 \pm  0.0$ & $\boldsymbol{0.125} \pm 0.0$ & $\boldsymbol{1.0} \pm 0.0$ \\ & 
Mean & $-0.094 \pm 0.068 $ & $0.0 \pm 0.0$ & $\boldsymbol{0.875}\pm 0.0$ \\ & 
Max & $-0.75 \pm 0.5$ & $0.0 \pm 0.0$ & $0.0 \pm 0.0$ \\ \bottomrule \end{tabular}
\end{table}



\section{\textsc{\capturetheflag}}
\label{app:ctf}

In {\capturetheflag}, agents need to navigate to goals. Each agent observes the relative position to the goals, and agent actions are continuous 2D forces that determine their direction of motion. 
The entries $r^t_{ij}$ of matrix $\mathbf{A}^t$ at time $t$ represent the local reward of agent $i$ towards goal $j$,  computed as $r^t_{ij} =\left (1-d^t_{ij}/\sum_{j=1}^M d^t_{ij}\right )/(M-1),$
where $d^t_{ij}$ is the distance between agent $i$ and goal $j$. This makes it so that $ \sum_{j=1}^M r^t_{ij} = 1$ and $r^t_{ij}\geqslant0$. At each step, the agents receive the global reward $R(\mathbf{A}^t)$, with aggregators $U,T\in\{\min,\mathrm{mean},\max\}$.

Our results, shown in Fig. \ref{fig:ctf_gap}, show that  our curvature theory reliably predicts when there is a heterogeneity gain (\autoref{fig:deltaR-vs-softmax}): it is positive \emph{only} for the concave–convex pairs $U=\min,\,T=\max$ and $U=\mathrm{mean},\,T=\max$.  The heterogeneity gain is smaller in the latter case because learning dynamics matter: with $U=\min,\,T=\max$ the best homogeneous policy is unique (every agent must steer to the midpoint between the two goals) so homogeneous learners seldom find it, leaving room for heterogeneous policies to excel (see \autoref{app:ctf}). By contrast, $U=\mathrm{mean},\,T=\max$ admits a continuum of good homogeneous policies, which homogeneous teams execute more easily.  For $U=\max,\,T=\min$ and $U=\max,\,T=\mathrm{mean}$, the theoretical $\Delta R$ is $0$, yet the empirical  heterogeneity gap is negative (\autoref{fig:ctf_gap}). This occurs because the reward peaks only when all agents coordinate on the same goal. Neurally heterogeneous teams learn this uniform behavior slower than homogeneous teams, so they underperform within the fixed training budget. Additional training would  close this gap to $\Delta R = 0$.  

In \autoref{fig:ctf_traj} we juxtapose two representative  $N = M = 2$ roll-outs of the \textsc{\capturetheflag} environment for \emph{homogeneous} teams (top row) and \emph{heterogeneous} teams (bottom row) when~$U~=~\min,\,T=\max$. Consistent with the discussion in \autoref{sec:experiments}, homogeneous agents steer to the geometric midpoint between the two goals, producing almost overlapping paths--this is suboptimal, as they cannot cover both goals. On the other hand,  heterogeneous agents exaggerate their differences, taking sharply diverging trajectories and ensuring one goal each.

\begin{figure}[!h]
  \centering
  \begin{subfigure}[t]{0.45\textwidth}
    \centering
    \includegraphics[width=\linewidth]{images/min_max_hom_1.png}
    \caption{Homogeneous run 1}
  \end{subfigure}\hfill
  \begin{subfigure}[t]{0.45\textwidth}
    \centering
    \includegraphics[width=\linewidth]{images/min_max_hom_2.png}
    \caption{Homogeneous run 2}
  \end{subfigure}\\[0.8em]
  \begin{subfigure}[t]{0.45\textwidth}
    \centering
    \includegraphics[width=\linewidth]{images/min_max_het_2.png}
    \caption{Heterogeneous run 1}
  \end{subfigure}
  \begin{subfigure}[t]{0.45\textwidth}
    \centering
    \includegraphics[width=\linewidth]{images/min_max_het_1.png}
    \caption{Heterogeneous run 2}
  \end{subfigure}\hfill
  \caption{\textbf{Behaviour under the concave–convex aggregator $U=\min,\,T=\max$.}
  Each dot is an agent position; line segments indicate instantaneous velocity;
  green squares mark goal locations.
  Homogeneous policies collapse to a single ``mid-point'' route, while heterogeneous
  policies split and follow distinct paths to cover both goals. Note how the heterogeneous agents \textit{exaggerate} the difference in their trajectories, rather than head directly to the goal: this is an outcome of the reward structure, which encourages maximal diversity.}
  \label{fig:ctf_traj}
\end{figure}


% In \autoref{fig:ctf} we show the rendered scenario. Videos of learnt policies for the various settings are available in the attached supplementary material.


% \begin{figure}[!h] 
%     \centering  % Centers the figure
%     \includegraphics[width=0.5\textwidth]{images/ctf.png}  % Adjust width as necessary, replace 'example-image' with your image file path
%     \caption{\textsc{\capturetheflag} scenario. Two agents (spawned at random positions on the left-hand side of a workspace) need to navigate to two goals (spawned at random positions on the left-hand side).
% Each agent observes the relative position to both goals. The agent actions are 2D forces that determine their direction of motion }  % Add your figure caption
%     \label{fig:ctf}  % Add a label for referencing the figure in text
% \end{figure}


\section{2v2 Tag Experiments}
\label{app:tag}


\begin{figure}[ht]
    \centering
    \setlength{\tabcolsep}{2pt}
    \renewcommand{\arraystretch}{0}

    \begin{tabular}{ccccc}
        % Seed 0000
        \begin{adjustbox}{valign=t}
        \includegraphics[width=0.22\textwidth]{images/seed_0000_homogeneous.png}
        \end{adjustbox} &
        \begin{adjustbox}{valign=t}
        \includegraphics[width=0.22\textwidth]{images/seed_0001_homogeneous.png}
        \end{adjustbox} &
        \begin{adjustbox}{valign=t}
        \includegraphics[width=0.22\textwidth]{images/seed_0002_homogeneous.png}
        \end{adjustbox} &
        \begin{adjustbox}{valign=t}
        \includegraphics[width=0.22\textwidth]{images/seed_0003_homogeneous.png}
        \end{adjustbox} &
        \begin{adjustbox}{valign=t}
        % \includegraphics[width=0.22\textwidth]{images/seed_0004_homogeneous.png}
        \end{adjustbox} \\

        % Seed 0000 heterogeneous row
        \begin{adjustbox}{valign=t}
        \includegraphics[width=0.22\textwidth]{images/seed_0000_heterogeneous.png}
        \end{adjustbox} &
        \begin{adjustbox}{valign=t}
        \includegraphics[width=0.22\textwidth]{images/seed_0001_heterogeneous.png}
        \end{adjustbox} &
        \begin{adjustbox}{valign=t}
        \includegraphics[width=0.22\textwidth]{images/seed_0002_heterogeneous.png}
        \end{adjustbox} &
        \begin{adjustbox}{valign=t}
        \includegraphics[width=0.22\textwidth]{images/seed_0003_heterogeneous.png}
        \end{adjustbox} &
        \begin{adjustbox}{valign=t}
        % \includegraphics[width=0.22\textwidth]{images/seed_0004_heterogeneous.png}
        \end{adjustbox}
    \end{tabular}

    \caption{Comparison of homogeneous (top row) and heterogeneous (bottom row) 2v2 tag policies for chaser agents, trained with the reward structure $U=\min, T=\max$ across different initializations. Every column shows the trajectory of the homogeneous (top) and heterogeneous (bottom) policies. (Note that trajectories here are smoothened; agents don't go over obstacles in actual execution). The heterogeneous policies prioritize capturing both agents, whereas the homogeneous policies focus on just one. In the $U=\min, T=\max$ setting, this gives heterogeneous agents greater reward, hence $\Delta R > 0$. Please find more visualizations on \href{\websiteurltag}{our website}.}
    \label{fig:tag_hom_vs_het}
\end{figure}


The goal of our tag experiment is to showcase that our theoretical results, which predict the value of $\Delta R$ based on the curvature of the aggregators, hold for discrete, sparse rewards. Specifically, our results for discrete efforts in  \autoref{fig:deltaR-vs-softmax} predict that only $(U,T) = (\min,\max), (\min, \text{mean}), (\text{mean}, \max)$ will have positive heterogeneity gain, with $(\min,\max)$ maximizing the gain. We show in \autoref{fig:tag_gains} that this holds in 2v2 tag, despite the fact that this is a challenging, embodied, long-horizon, whereas our formal results are for instantaneous allocation games\footnote{The gain for $(\text{mean},\max)$ is small compared to the other two aggregator combinations, but still positive at $\Delta R \approx 0.37$. This is also significantly higher than aggregator combinations for which we predict $\Delta R$ is not positive, the largest of which attained $\Delta R < 0.01$.}. Note that this is a highly interpretable result: a $(\min,\max)$ means that agents are only awarded when \textit{both} escapers are caught, incentivizing heterogeneous strategies where chaser agents split their behaviour so that the chasing efforts $r^t_{ij}$ are equally distributed between both escapers. \autoref{fig:tag_hom_vs_het} visualizes the trajectories learned by agents trained under this $(\min,\max)$  reward structure, showing distinct emergent pursuit strategies emerging depending on whether the agents are neurally heterogeneous or neurally homogeneous. 
\section{Football Experiments}
\label{app:football}

\begin{table}[t]
\centering
\caption{Football heterogeneity gains across different reward formulations. Results obtained after 500 training iterations of 240k frames each (6 seeds). Opponent speed annealed from 0\% to 100\%.}
\label{tab:reward_football}
\begin{tabular}{@{}lccp{0.35\textwidth}@{}}
\toprule
Reward & $\Delta R$ & Theory $\Delta R>0$? & Reward meaning \\ \midrule
$U=\min,\; T=\max$ & $\boldsymbol{1.76} \pm 0.72$  & Yes & One agent should attend the ball, the other the opponent; reward capped by the less-covered task. \\
$U=\text{mean},\; T=\max$ & $\boldsymbol{1.18} \pm 0.11$ & Yes & Similar to $(\min,\max)$, but reward is dictated by average task performance. \\
$U=\text{mean},\; T=\text{mean}$ & $0.009 \pm 0.075$ & No & Agents should attend both the opponent and the ball. \\
$U=\min,\; T=\min$ & $-0.08 \pm 0.73$ & No & At least one agent should attend at least the opponent or the ball. \\ \bottomrule
\end{tabular}
\end{table}

In some environments, the reward structure might not entirely follow the double-generalized-aggregator structure we study in this work, but at least some part of the reward function might obey this structure. In our study of the VMAS football scenario \citep{bettini2022vmas}, we ask what happens when this is the case.

Football is a complex, embodied, long-horizon scenario that  requires the agents to learn low-level dribbling skills as well as high-level strategy purely from a shared cooperative reward. The VMAS scenario uses reward shaping to enable agents to learn such behaviors. We \textit{add} a reward structure $U(T(r_{11}^t,r_{21}^t),T(r_{12}^t,r_{22}^t))$ on top of this and ask how this affects heterogeneity. 

In our experimental scenario, two learning agents spawn at midfield. A ball is located between them and the goal to the right; a heuristic defender spawns to their left and chases the ball. Agents receive a global reward that increases when the ball moves toward the goal and the defender stays away from it. Additionally, we reuse the reward structure from our Multi‑Goal‑Capture to define rewards for two tasks: tackling the ball, and tackling the opponent.

The effort at time $t$ is:
$$r_{ij}^t=(1-\frac{d_{ij}^t}{\sum_j d_{ij}^t})/d_{ij}^t,$$
where $d_{ij}$ is distance of agent $i$ to ball or opponent). The global reward given to all agents is then computed as: $$R^t=U(T(r_{11}^t,r_{21}^t),T(r_{12}^t,r_{22}^t))+\beta[(d_{ball,goal}^{t-1}-d_{ball,goal}^t)-(d_{opp,ball}^{t-1}-d_{opp,ball}^t)],$$
where $\beta$ weighs the global football reward. 

Since this reward structure does not follow our theory entirely, we ask whether, when $U,T$ are, respectively, strictly Schur-concave and strictly Schur-convex, we should expect $\Delta R > 0$ as in our other scenarios. We test this for $U=\min, T=\max$ and $U=\text{mean},T=\max$. To control for the possibility that football is heterogeneous ``by default'', we also test the aggregator combinations $U=\text{mean}, T=\text{mean}$ and $U=\min,T=\min$ as controls. 

We report heterogeneity gains after training homogeneous and heterogeneous policies in \autoref{tab:reward_football}.
This shows that our curvature test predicts the heterogeneity gain of different reward structures, despite only being a component in the overall reward structure. This insight is important, as it indicates our theoretical insights (the curvature test) may extend beyond environments that strictly follow our task allocation setting.

The resulting policies are reported in \autoref{fig:football} with videos \href{\websiteurlfootball}{here}.

\begin{figure}
    \centering
    \includegraphics[width=1.0\linewidth]{images/football.png}
    \caption{\textbf{Left:} Results of training a heterogeneous policy on VMAS Football where agents are trained with $U=\min, T=\max$ aggregators. Our learning agents are drawn in blue; the heuristic opponent in red; and the ball in black. The learning agents split their efforts, tackling both the ball and the opponent. \textbf{Right:} Results of training a homogeneous policy. Agents are unable to split their efforts, so either they both tackle the ball, or both tackle the opponent. This results in lower reward, hence $\Delta R > 0$. These policies are visualized on \href{\websiteurlfootball}{our website}.}
    \label{fig:football}
\end{figure}

\section{Observability-Heterogeneity Trade-Off}
\label{app:observehettradeoff}

% \begin{figure}[t] 
%     \centering  % Centers the figure
%     \includegraphics[width=\textwidth]{images/ctf_embodied_lidar.pdf} 
%     \caption{\Heterogeneitygap for {\capturetheflag} over training iterations as a function of the LIDAR sensor radius.
%     The plot shows that the gap decreases as the agent observability increases.
%     We report mean and standard deviation after 30 million training frames over 4 different random seeds.}  % Add your figure caption
%     \label{fig:ctf_gap_lidar} 
% \end{figure}

\begin{figure}[t]
  \centering
   \centering
     \includegraphics[width=0.6\textwidth]{images/ctf_embodied_lidar.pdf} 
    \caption{Gain w.r.t. observability when $U=\min,T=\max$.
    }  % Add your figure caption
    \label{fig:ctf_gap_lidar} 
  \caption{\Heterogeneitygap for {\capturetheflag} throughout training when the agents' observation range is gradually increased from $0$ to $0.35$ over 4 random seeds (4 random seeds suffice as this phenomenon is established in the literature \citep{bettini2023hetgppo}, and we only wish to show its emergence in the context of our work.)}
\end{figure}

In this Appendix, we crystallize the relationship between environment observability and empirical \heterogeneitygaps. It is well known that neurally homogeneous agents (i.e., sharing the same parameters) can achieve behavioral heterogeneity by conditioning their actions on diverse input contexts (behavioral typing). 
%It is well known that \textit{neurally} homogeneous agents (i.e., sharing the same neural network) can emulate diverse behavior by conditioning on the input context (behavioral typing)~\citep{leibo2019malthusian}.
This can be achieved by naively appending the agent index to its observation~\citep{gupta2017cooperative} or by providing relevant observations that allows the agents to infer their role~\citep{bettini2023hetgppo}.
Behavioral typing is impossible in matrix games, as these games are observationless.
However, it is possible in more complex games, such as our {\capturetheflag} scenario.
We augment agents in the positive gain scenario ($U=\min,T=\max$) with a range sensor, providing proximity readings for other agents within a radius.
In \autoref{fig:ctf_gap_lidar}, we show that the \heterogeneitygap decreases as the agent visibility increases (higher sensing radius).
This is because, with a higher range, homogeneous agents can sense each other and coordinate to pursue different goals.
This result highlights the tight interdependence between the \heterogeneitygap and agents' observations. 
% We note that, even if homogeneous agents may be able to narrow the gap by leveraging high observability, this makes them less resilient and more brittle to noise and adversarial attacks due to their increased reliance on the input context~\citep{bettini2023hetgppo}. 

\section{Parametrized Dec-POMDP}
\label{app:pdecpomdp}
A Parametrized Decentralized Partially Observable Markov Decision Process (PDec-POMDP) is defined as a tuple
$$\left \langle \mathcal{N}, \mathcal{S}, \left \{ \mathcal{O}_i \right \}_{i \in \mathcal{N}}, \left \{ \sigma_i^\theta \right \}_{i \in \mathcal{N}},  \left \{ \mathcal{A}_i \right \}_{i \in \mathcal{N}}, \mathcal{R}^\theta , \mathcal{T}^\theta, \gamma, s_0^\theta \right \rangle_\theta,$$
where $\mathcal{N} = \{1,\ldots, n\}$ denotes the set of agents,
$\mathcal{S}$ is the state space, and,
$\left \{ \mathcal{O}_i \right \}_{i \in \mathcal{N}}$ and
$\left \{ \mathcal{A}_i \right \}_{i \in \mathcal{N}}$
are the observation and action spaces, with $\mathcal{O}_i \subseteq \mathcal{S}, \; \forall i \in \mathcal{N}$. 
Further, $\left \{ \sigma_i^\theta \right \}_{i \in \mathcal{N}}$ 
and
$\mathcal{R}^\theta $
are the agent observation and reward functions, such that
$\sigma_i^\theta : \mathcal{S} \mapsto \mathcal{O}_i$, and,
$\mathcal{R}^\theta: \mathcal{S} \times \left \{ \mathcal{A}_i \right \}_{i \in \mathcal{N}} \mapsto \R$.
$\mathcal{T}^\theta$ is the stochastic state transition model, defined as $\mathcal{T}^\theta : \mathcal{S} \times \left \{ \mathcal{A}_i \right \}_{i \in \mathcal{N}}   \mapsto  \Delta\mathcal{S}$, which outputs the probability $\mathcal{T}^\theta(s^t, \left \{ a^t_i \right \}_{i \in \mathcal{N}},s^{t+1})$ of transitioning to state $s^{t+1} \in \mathcal{S}$ given the current state $s^t \in \mathcal{S}$ and actions $\left \{ a^t_i \right \}_{i \in \mathcal{N}}$, with $a^t_i \in \mathcal{A}_i$. $\gamma$ is the discount factor.
Finally, $s_0^\theta \in \mathcal{S}$ is a the initial environment state.
A PDec-POMDP represents a set of traditional Dec-POMDPs~\citep{oliehoek2016concise}, where the observation function, the transition function, the reward function, and the initial state are conditioned on parameters $\theta$. This formalism is similar to the concepts of Underspecified POMDP~\citep{dennis2020emergent} and contextual MDP~\citep{modi2018markov}.

Agents are equipped with (possibly stochastic) policies $\pi_{i}(a_i|o_i)$, which compute an action given a local observation. 
Their objective is to maximize the discounted return:
$$
G^\theta(\mathbf{\pi}) = \mathbb{E}_\mathbf{\pi}\left [\sum_{t=0}^T \gamma^t\mathcal{R}^\theta\left (s^t,\mathbf{a}^t \right ) \middle \vert s^{t+1} \sim \mathcal{T}^\theta(s^t,\mathbf{a}^t), a_i^t \sim \pi_i(o_i^t), o_i^t = \sigma^\theta_i(s_t)  \right ],
$$
where $\mathbf{\pi}, \mathbf{a}$ are the vectors of all agents' policies and actions. 
$G^\theta(\mathbf{\pi})$ represents the expected sum of discounted rewards starting in state $s_0^\theta$ and following policy $\mathbf{\pi}$ in a PDec-POMDP parametrized by $\theta$.

\section{Stability of Bilevel Optimization in HetGPS}

\label{app:stability_of_bilevel_optimization}

The Heterogeneity Gain Parameter Search (HetGPS) algorithm employs a bilevel optimization framework to simultaneously optimize environment parameters and agent policies. This appendix discusses the structure of this optimization problem, its convergence properties, practical stability, and alternatives for non-differentiable environments.

\subsection{HetGPS as a Stackelberg Game}
HetGPS can be formalized as a Stackelberg game, a hierarchical optimization problem involving a leader and followers~\citep{simaan1973stackelberg}. In our setting:
\begin{enumerate}
    \item The \textbf{Leader} is the environment designer (the outer loop of HetGPS), which aims to maximize the heterogeneity gain $HetGain^{\theta}$ by adjusting the environment parameters $\theta$.
    \item The \textbf{Followers} are the homogeneous and heterogeneous multi-agent teams (the inner loop), which aim to maximize their respective returns $G^{\theta}(\pi)$ by optimizing their policies $\pi_{het}$ and $\pi_{hom}$ within the environment defined by $\theta$.
\end{enumerate}
The leader's objective function (the heterogeneity gain) depends on the optimized policies of the followers, which, in turn, depend on the parameters $\theta$ set by the leader. Formally, the objective is:
\begin{equation}
    \max_{\theta} \left[ G^{\theta}(\pi^*_{\text{het}}(\theta)) - G^{\theta}(\pi^*_{\text{hom}}(\theta)) \right]
\end{equation}
where $\pi^*(\theta)$ represents the optimized policies for a given environment configuration $\theta$.

\subsection{Convergence and Stability}

Generally speaking, multi-agent reinforcement learning is a concurrent optimization process that faces non-stationarity as agents constantly adapt to one another's evolving policies~\citep{BasarOverview}. HetGPS extends this challenge as agents must also adapt to a changing environment. Consequently, formal convergence guarantees to a *global* optimum remain an open question with regards to HetGPS in particular, but also MARL algorithms in general. However, recent theoretical work in environment co-design has established conditions under which convergence of bilevel optimization processes similar to HetGPS to *local* optima can be guaranteed, such as requiring sufficient smoothness of the environment dynamics and policy updates~\citep{gao2024codesign}. 

Despite the theoretical complexities inherent in multi-agent learning and bilevel optimization, as shown in \autoref{sec:experiments} and \autoref{app:adverseinitializations}, HetGPS demonstrates strong empirical stability even under adversarial initializations. This stability is expected, as it mirrors the practical success observed in related co-design and automated curriculum learning literature~\citep{dennis2020emergent, gao2024codesign}.

\subsection{Advantage of Differentiable Simulation}

In our experiments, HetGPS increased training time by roughly 25\% compared to training agents in an environment with a fixed reward structure. Hence, it is highly efficient and does not impose much overhead. A key strength contributing to the efficiency of HetGPS is its use of differentiable simulation (e.g., VMAS~\citep{bettini2022vmas}). By leveraging backpropagation through the entire rollout, HetGPS computes the exact gradient $\nabla_{\theta}HetGain^{\theta}$. This approach is  more  sample-efficient than alternative methods that treat the environment design as a separate RL problem (e.g., PAIRED~\citep{dennis2020emergent} or Designer-RL~\citep{gao2024codesign, amir2025recodereinforcementlearningbaseddynamic}). Such methods rely on high-variance policy gradient estimates for the outer loop and often struggle with exploration inefficiency~\citep{parker-holder2021that, jiang2021replayguided, xu2022accelerated}. By utilizing exact gradients, HetGPS mitigates these issues.

\subsection{Handling Non-Differentiable Environments}
A requirement for the implementation of HetGPS presented in Alg. 1 is access to a differentiable simulator. When the environment involves non-smooth physics or black-box components, direct backpropagation is infeasible.

In such cases, the environment optimization step (Line 6 of \autoref{alg:HetGPS}) can be replaced with the gradient-free methods mentioned above, such as PAIRED \citep{dennis2020emergent}, the bilevel method from \citep{ gao2024codesign}, or evolutionary strategies~\citep{stanley2019designing}. While these methods have empirically been shown to be stable and robust in other co-design settings, and may enable the extension of HetGPS to non-differentiable settings, they typically require more samples and may exhibit more noise compared to the direct backpropagation approach utilized in this work.
\section{HetGPS Under Adversarial Initial Conditions}
\label{app:adverseinitializations}

To evaluate the robustness of HetGPS to initialization, we repeated the Softmax experiment in Multi-Goal-Capture (Fig. 5a) with adverse initialization. We initialized the outer aggregator $U$ with $\tau=5$ (making it convex) and the inner aggregator $T$ with $\tau=-5$ (making it concave), which is the opposite of the concave-convex configuration predicted by theory to maximize heterogeneity gain.

As shown in Table~\ref{tab:adverse_init}, HetGPS successfully overcomes the adverse initialization and converges towards the theoretically optimal parameters (large positive $\tau$ for T, large negative $\tau$ for U).

\begin{table}[h]
\centering
\caption{Convergence of HetGPS parameters ($\tau$) in the Softmax Multi-Goal-Capture experiment starting from adverse initialization ($\tau_T=-5, \tau_U=5$). Mean and standard deviation reported over 3 seeds.}
\label{tab:adverse_init}
\begin{tabular}{@{}lcccc@{}}
\toprule
\textbf{Frames (M)} & \textbf{0} & \textbf{50} & \textbf{75} & \textbf{100} \\
\midrule
$\tau$ of T (Inner Agg.) & $-5.0\pm0.0$ & $13.32\pm2.15$ & $18.88\pm3.96$ & $22.95\pm5.71$ \\
$\tau$ of U (Outer Agg.) & $5.0\pm0.0$ & $-10.26\pm1.70$ & $-14.59\pm2.24$ & $-17.16\pm2.43$ \\
\bottomrule
\end{tabular}
\end{table}
\section{Limitations and Open Questions}
\label{appendix:limitations}

We list a number of limitations, open questions, and possible extensions.

\subsection{Theoretical scope}
\begin{itemize}[leftmargin=1.5em]

\item \textbf{Beyond task-allocation RL domains.}  
      The benchmark domains we study  and the additional settings covered in \autoref{appendix:examples_of_marl_environments} all fit into our abstract task-allocation
      framework: we can interpret agents' state, such as goal proximity in \capturetheflag, or whether they captured an escaping agent in tag, abstractly as ``efforts'' $r_{ij}$ and represent the reward in terms of such efforts.  This is what enables us to make predictions about these environments. Although our framework is quite general, and accommodates environments that one might not traditionally view as ``task allocation'' (such as football and tag), several notable multi-agent RL domains, e.g., multi-robot manipulation, might not be representable within this  framework. Our heterogeneity analysis does not directly apply to these settings, and extending our results to them is important for getting a complete picture of the benefits of heterogeneity.
\end{itemize}

\subsection{Algorithmic assumptions}
\begin{itemize}[leftmargin=1.5em]
    \item \textbf{Differentiable simulation.}  
          HetGPS requires $\nabla_\theta G^\theta(\boldsymbol\pi)$, hence a simulator that is
          end-to-end differentiable and tractable to back-propagate through.  We assume differentiability mainly for considerations of training efficiency. However, many realistic environments still rely on non-smooth physics or black-box generators, requiring us to modify HetGPS for these settings. We note that there are good, established methods for learning environment parameters in non-differentiable settings (at the cost of efficiency/increased noise). We discuss these in detail in \autoref{app:stability_of_bilevel_optimization}. However, we did not test such variants of HetGPS, and leave these extensions to future work. 
\end{itemize}

\subsection{Open questions}
\begin{enumerate}[leftmargin=2em,label=\roman*.]
    \item \textbf{What is the connection between the transition function and heterogeneity?}  
          Our analysis is reward-centric: the curvature criterion reasons only about the team reward.  
          In a Dec-POMDP, however, heterogeneity can be beneficial purely because agents are constrained by \emph{state transitions}.  When do state transition dynamics benefit heterogeneity?

    \item \textbf{Learning dynamics vs.\ reward structure.}  
          The theory predicts whether a given reward structure \textit{enables} an advantage to heterogeneous teams,  not whether a
          particular learning algorithm will learn in response to it. This is connected to the difference between \textit{neural} and \textit{behavioral} heterogeneity that we emphasize throughout the paper. Our experiments suggest, empirically, that neurally heterogeneous agents will, in practice, learn to exploit heterogeneous reward structures (i.e., be behaviorally heterogeneous); but can a formal link be established between our reward structure insights and what reward the learning dynamics converge to in practice?
\end{enumerate}

Tackling these challenges would sharpen our understanding of \emph{when} and \emph{how}
diversity should be engineered in cooperative multi-agent learning.



% \newpage
% \section*{NeurIPS Paper Checklist}

% % %%% BEGIN INSTRUCTIONS %%%
% % The checklist is designed to encourage best practices for responsible machine learning research, addressing issues of reproducibility, transparency, research ethics, and societal impact. Do not remove the checklist: {\bf The papers not including the checklist will be desk rejected.} The checklist should follow the references and follow the (optional) supplemental material.  The checklist does NOT count towards the page
% % limit. 

% % Please read the checklist guidelines carefully for information on how to answer these questions. For each question in the checklist:
% % \begin{itemize}
% %     \item You should answer \answerYes{}, \answerNo{}, or \answerNA{}.
% %     \item \answerNA{} means either that the question is Not Applicable for that particular paper or the relevant information is Not Available.
% %     \item Please provide a short (1–2 sentence) justification right after your answer (even for NA). 
% %    % \item {\bf The papers not including the checklist will be desk rejected.}
% % \end{itemize}

% % {\bf The checklist answers are an integral part of your paper submission.} They are visible to the reviewers, area chairs, senior area chairs, and ethics reviewers. You will be asked to also include it (after eventual revisions) with the final version of your paper, and its final version will be publisHetGPS with the paper.

% % The reviewers of your paper will be asked to use the checklist as one of the factors in their evaluation. While "\answerYes{}" is generally preferable to "\answerNo{}", it is perfectly acceptable to answer "\answerNo{}" provided a proper justification is given (e.g., "error bars are not reported because it would be too computationally expensive" or "we were unable to find the license for the dataset we used"). In general, answering "\answerNo{}" or "\answerNA{}" is not grounds for rejection. While the questions are phrased in a binary way, we acknowledge that the true answer is often more nuanced, so please just use your best judgment and write a justification to elaborate. All supporting evidence can appear either in the main paper or the supplemental material, provided in appendix. If you answer \answerYes{} to a question, in the justification please point to the section(s) where related material for the question can be found.

% % IMPORTANT, please:
% % \begin{itemize}
% %     \item {\bf Delete this instruction block, but keep the section heading ``NeurIPS Paper Checklist"},
% %     \item  {\bf Keep the checklist subsection headings, questions/answers and guidelines below.}
% %     \item {\bf Do not modify the questions and only use the provided macros for your answers}.
% % \end{itemize} 
 

% % %%% END INSTRUCTIONS %%%


% \begin{enumerate}

% \item {\bf Claims}
%     \item[] Question: Do the main claims made in the abstract and introduction accurately reflect the paper's contributions and scope?
%     \item[] Answer: \answerYes{} % Replace by \answerYes{}, \answerNo{}, or \answerNA{}.
%     \item[] Justification: Sections \autoref{sec:intro} and \autoref{sec:problemsetting} explain the scope. Each claim made in the introduction is provided alongside a reference to the relevant section in the paper.
%     \item[] Guidelines:
%     \begin{itemize}
%         \item The answer NA means that the abstract and introduction do not include the claims made in the paper.
%         \item The abstract and/or introduction should clearly state the claims made, including the contributions made in the paper and important assumptions and limitations. A No or NA answer to this question will not be perceived well by the reviewers. 
%         \item The claims made should match theoretical and experimental results, and reflect how much the results can be expected to generalize to other settings. 
%         \item It is fine to include aspirational goals as motivation as long as it is clear that these goals are not attained by the paper. 
%     \end{itemize}

% \item {\bf Limitations}
%     \item[] Question: Does the paper discuss the limitations of the work performed by the authors?
%     \item[] Answer: \answerYes{} % Replace by \answerYes{}, \answerNo{}, or \answerNA{}.
%     \item[] Justification: \autoref{sec:intro} and \autoref{sec:problemsetting} bound the scope of this work. The Discussion brings up open questions, and 
%     \autoref{appendix:limitations} extensively discusses limitations, and more open questions.
%     \item[] Guidelines:
%     \begin{itemize}
%         \item The answer NA means that the paper has no limitation while the answer No means that the paper has limitations, but those are not discussed in the paper. 
%         \item The authors are encouraged to create a separate "Limitations" section in their paper.
%         \item The paper should point out any strong assumptions and how robust the results are to violations of these assumptions (e.g., independence assumptions, noiseless settings, model well-specification, asymptotic approximations only holding locally). The authors should reflect on how these assumptions might be violated in practice and what the implications would be.
%         \item The authors should reflect on the scope of the claims made, e.g., if the approach was only tested on a few datasets or with a few runs. In general, empirical results often depend on implicit assumptions, which should be articulated.
%         \item The authors should reflect on the factors that influence the performance of the approach. For example, a facial recognition algorithm may perform poorly when image resolution is low or images are taken in low lighting. Or a speech-to-text system might not be used reliably to provide closed captions for online lectures because it fails to handle technical jargon.
%         \item The authors should discuss the computational efficiency of the proposed algorithms and how they scale with dataset size.
%         \item If applicable, the authors should discuss possible limitations of their approach to address problems of privacy and fairness.
%         \item While the authors might fear that complete honesty about limitations might be used by reviewers as grounds for rejection, a worse outcome might be that reviewers discover limitations that aren't acknowledged in the paper. The authors should use their best judgment and recognize that individual actions in favor of transparency play an important role in developing norms that preserve the integrity of the community. Reviewers will be specifically instructed to not penalize honesty concerning limitations.
%     \end{itemize}

% \item {\bf Theory assumptions and proofs}
%     \item[] Question: For each theoretical result, does the paper provide the full set of assumptions and a complete (and correct) proof?
%     \item[] Answer: \answerYes{} % Replace by \answerYes{}, \answerNo{}, or \answerNA{}.
%     \item[] Justification: Please find all proofs in the Appendix.
%     \item[] Guidelines:
%     \begin{itemize}
%         \item The answer NA means that the paper does not include theoretical results. 
%         \item All the theorems, formulas, and proofs in the paper should be numbered and cross-referenced.
%         \item All assumptions should be clearly stated or referenced in the statement of any theorems.
%         \item The proofs can either appear in the main paper or the supplemental material, but if they appear in the supplemental material, the authors are encouraged to provide a short proof sketch to provide intuition. 
%         \item Inversely, any informal proof provided in the core of the paper should be complemented by formal proofs provided in appendix or supplemental material.
%         \item Theorems and Lemmas that the proof relies upon should be properly referenced. 
%     \end{itemize}

%     \item {\bf Experimental result reproducibility}
%     \item[] Question: Does the paper fully disclose all the information needed to reproduce the main experimental results of the paper to the extent that it affects the main claims and/or conclusions of the paper (regardless of whether the code and data are provided or not)?
%     \item[] Answer: \answerYes{} % Replace by \answerYes{}, \answerNo{}, or \answerNA{}.
%     \item[] Justification: The code is provided in the supplementary material alongside YAML configuration files that enable one to re-run our experiments exactly (\autoref{appendix:code_availability}), while all technical derivations are in the paper body or in the Appendix.
%     \item[] Guidelines:
%     \begin{itemize}
%         \item The answer NA means that the paper does not include experiments.
%         \item If the paper includes experiments, a No answer to this question will not be perceived well by the reviewers: Making the paper reproducible is important, regardless of whether the code and data are provided or not.
%         \item If the contribution is a dataset and/or model, the authors should describe the steps taken to make their results reproducible or verifiable. 
%         \item Depending on the contribution, reproducibility can be accomplisHetGPS in various ways. For example, if the contribution is a novel architecture, describing the architecture fully might suffice, or if the contribution is a specific model and empirical evaluation, it may be necessary to either make it possible for others to replicate the model with the same dataset, or provide access to the model. In general. releasing code and data is often one good way to accomplish this, but reproducibility can also be provided via detailed instructions for how to replicate the results, access to a hosted model (e.g., in the case of a large language model), releasing of a model checkpoint, or other means that are appropriate to the research performed.
%         \item While NeurIPS does not require releasing code, the conference does require all submissions to provide some reasonable avenue for reproducibility, which may depend on the nature of the contribution. For example
%         \begin{enumerate}
%             \item If the contribution is primarily a new algorithm, the paper should make it clear how to reproduce that algorithm.
%             \item If the contribution is primarily a new model architecture, the paper should describe the architecture clearly and fully.
%             \item If the contribution is a new model (e.g., a large language model), then there should either be a way to access this model for reproducing the results or a way to reproduce the model (e.g., with an open-source dataset or instructions for how to construct the dataset).
%             \item We recognize that reproducibility may be tricky in some cases, in which case authors are welcome to describe the particular way they provide for reproducibility. In the case of closed-source models, it may be that access to the model is limited in some way (e.g., to registered users), but it should be possible for other researchers to have some path to reproducing or verifying the results.
%         \end{enumerate}
%     \end{itemize}


% \item {\bf Open access to data and code}
%     \item[] Question: Does the paper provide open access to the data and code, with sufficient instructions to faithfully reproduce the main experimental results, as described in supplemental material?
%     \item[] Answer: \answerYes{} % Replace by \answerYes{}, \answerNo{}, or \answerNA{}.
%     \item[] Justification: The code for generating all our results can be found in the supplementary materials.
%     \item[] Guidelines:
%     \begin{itemize}
%         \item The answer NA means that paper does not include experiments requiring code.
%         \item Please see the NeurIPS code and data submission guidelines (\url{https://nips.cc/public/guides/CodeSubmissionPolicy}) for more details.
%         \item While we encourage the release of code and data, we understand that this might not be possible, so “No” is an acceptable answer. Papers cannot be rejected simply for not including code, unless this is central to the contribution (e.g., for a new open-source benchmark).
%         \item The instructions should contain the exact command and environment needed to run to reproduce the results. See the NeurIPS code and data submission guidelines (\url{https://nips.cc/public/guides/CodeSubmissionPolicy}) for more details.
%         \item The authors should provide instructions on data access and preparation, including how to access the raw data, preprocessed data, intermediate data, and generated data, etc.
%         \item The authors should provide scripts to reproduce all experimental results for the new proposed method and baselines. If only a subset of experiments are reproducible, they should state which ones are omitted from the script and why.
%         \item At submission time, to preserve anonymity, the authors should release anonymized versions (if applicable).
%         \item Providing as much information as possible in supplemental material (appended to the paper) is recommended, but including URLs to data and code is permitted.
%     \end{itemize}


% \item {\bf Experimental setting/details}
%     \item[] Question: Does the paper specify all the training and test details (e.g., data splits, hyperparameters, how they were chosen, type of optimizer, etc.) necessary to understand the results?
%     \item[] Answer: \answerYes{} % Replace by \answerYes{}, \answerNo{}, or \answerNA{}.
%     \item[] Justification: The code contains instructions on how to reproduce the experiments in the paper and dedicated YAML files containing the hyperparameters for each experiment presented. The YAML files are structured according to the HYDRA~\citep{yadan2019hydra} framework which allows smooth reproduction as well as systematic and standardized configuration. We further attach all scripts to reproduce the plots in the paper from the experiment results.
%     \item[] Guidelines:
%     \begin{itemize}
%         \item The answer NA means that the paper does not include experiments.
%         \item The experimental setting should be presented in the core of the paper to a level of detail that is necessary to appreciate the results and make sense of them.
%         \item The full details can be provided either with the code, in appendix, or as supplemental material.
%     \end{itemize}

% \item {\bf Experiment statistical significance}
%     \item[] Question: Does the paper report error bars suitably and correctly defined or other appropriate information about the statistical significance of the experiments?
%     \item[] Answer:\answerYes{} % Replace by \answerYes{}, \answerNo{}, or \answerNA{}.
%     \item[] Justification: For every result presented, we report mean and standard deviation bars over various random seeds.
%     \item[] Guidelines:
%     \begin{itemize}
%         \item The answer NA means that the paper does not include experiments.
%         \item The authors should answer "Yes" if the results are accompanied by error bars, confidence intervals, or statistical significance tests, at least for the experiments that support the main claims of the paper.
%         \item The factors of variability that the error bars are capturing should be clearly stated (for example, train/test split, initialization, random drawing of some parameter, or overall run with given experimental conditions).
%         \item The method for calculating the error bars should be explained (closed form formula, call to a library function, bootstrap, etc.)
%         \item The assumptions made should be given (e.g., Normally distributed errors).
%         \item It should be clear whether the error bar is the standard deviation or the standard error of the mean.
%         \item It is OK to report 1-sigma error bars, but one should state it. The authors should preferably report a 2-sigma error bar than state that they have a 96\% CI, if the hypothesis of Normality of errors is not verified.
%         \item For asymmetric distributions, the authors should be careful not to show in tables or figures symmetric error bars that would yield results that are out of range (e.g. negative error rates).
%         \item If error bars are reported in tables or plots, The authors should explain in the text how they were calculated and reference the corresponding figures or tables in the text.
%     \end{itemize}

% \item {\bf Experiments compute resources}
%     \item[] Question: For each experiment, does the paper provide sufficient information on the computer resources (type of compute workers, memory, time of execution) needed to reproduce the experiments?
%     \item[] Answer: \answerYes{}
%     \item[] Justification: We provide details on the computational resources used in the supplementary material (Appendix \ref{appendix:computeused}).
%     \item[] Guidelines:
%     \begin{itemize}
%         \item The answer NA means that the paper does not include experiments.
%         \item The paper should indicate the type of compute workers CPU or GPU, internal cluster, or cloud provider, including relevant memory and storage.
%         \item The paper should provide the amount of compute required for each of the individual experimental runs as well as estimate the total compute. 
%         \item The paper should disclose whether the full research project required more compute than the experiments reported in the paper (e.g., preliminary or failed experiments that didn't make it into the paper). 
%     \end{itemize}
    
% \item {\bf Code of ethics}
%     \item[] Question: Does the research conducted in the paper conform, in every respect, with the NeurIPS Code of Ethics \url{https://neurips.cc/public/EthicsGuidelines}?
%     \item[] Answer: \answerYes{} % Replace by \answerYes{}, \answerNo{}, or \answerNA{}.
%     \item[] Justification: The paper is theoretical, does not use sensitive data, and does not violate anything in the Code of Ethics.
%     \item[] Guidelines:
%     \begin{itemize}
%         \item The answer NA means that the authors have not reviewed the NeurIPS Code of Ethics.
%         \item If the authors answer No, they should explain the special circumstances that require a deviation from the Code of Ethics.
%         \item The authors should make sure to preserve anonymity (e.g., if there is a special consideration due to laws or regulations in their jurisdiction).
%     \end{itemize}


% \item {\bf Broader impacts}
%     \item[] Question: Does the paper discuss both potential positive societal impacts and negative societal impacts of the work performed?
%     \item[] Answer: \answerNA{} % Replace by \answerYes{}, \answerNo{}, or \answerNA{}.
%     \item[] Justification: Given the topic of the paper (heterogeneity in multi-agent learning), this question is likely not applicable.
%     \item[] Guidelines:
%     \begin{itemize}
%         \item The answer NA means that there is no societal impact of the work performed.
%         \item If the authors answer NA or No, they should explain why their work has no societal impact or why the paper does not address societal impact.
%         \item Examples of negative societal impacts include potential malicious or unintended uses (e.g., disinformation, generating fake profiles, surveillance), fairness considerations (e.g., deployment of technologies that could make decisions that unfairly impact specific groups), privacy considerations, and security considerations.
%         \item The conference expects that many papers will be foundational research and not tied to particular applications, let alone deployments. However, if there is a direct path to any negative applications, the authors should point it out. For example, it is legitimate to point out that an improvement in the quality of generative models could be used to generate deepfakes for disinformation. On the other hand, it is not needed to point out that a generic algorithm for optimizing neural networks could enable people to train models that generate Deepfakes faster.
%         \item The authors should consider possible harms that could arise when the technology is being used as intended and functioning correctly, harms that could arise when the technology is being used as intended but gives incorrect results, and harms following from (intentional or unintentional) misuse of the technology.
%         \item If there are negative societal impacts, the authors could also discuss possible mitigation strategies (e.g., gated release of models, providing defenses in addition to attacks, mechanisms for monitoring misuse, mechanisms to monitor how a system learns from feedback over time, improving the efficiency and accessibility of ML).
%     \end{itemize}
    
% \item {\bf Safeguards}
%     \item[] Question: Does the paper describe safeguards that have been put in place for responsible release of data or models that have a high risk for misuse (e.g., pretrained language models, image generators, or scraped datasets)?
%     \item[] Answer: \answerNA{} % Replace by \answerYes{}, \answerNo{}, or \answerNA{}.
%     \item[] Justification: We present no data or models at risk of misuse.
%     \item[] Guidelines:
%     \begin{itemize}
%         \item The answer NA means that the paper poses no such risks.
%         \item Released models that have a high risk for misuse or dual-use should be released with necessary safeguards to allow for controlled use of the model, for example by requiring that users adhere to usage guidelines or restrictions to access the model or implementing safety filters. 
%         \item Datasets that have been scraped from the Internet could pose safety risks. The authors should describe how they avoided releasing unsafe images.
%         \item We recognize that providing effective safeguards is challenging, and many papers do not require this, but we encourage authors to take this into account and make a best faith effort.
%     \end{itemize}

% \item {\bf Licenses for existing assets}
%     \item[] Question: Are the creators or original owners of assets (e.g., code, data, models), used in the paper, properly credited and are the license and terms of use explicitly mentioned and properly respected?
%     \item[] Answer: \answerYes{} % Replace by \answerYes{}, \answerNo{}, or \answerNA{}.
%     \item[] Justification: The paper uses the open-source VMAS and TorchRL packages, which are credited.
%     \item[] Guidelines:
%     \begin{itemize}
%         \item The answer NA means that the paper does not use existing assets.
%         \item The authors should cite the original paper that produced the code package or dataset.
%         \item The authors should state which version of the asset is used and, if possible, include a URL.
%         \item The name of the license (e.g., CC-BY 4.0) should be included for each asset.
%         \item For scraped data from a particular source (e.g., website), the copyright and terms of service of that source should be provided.
%         \item If assets are released, the license, copyright information, and terms of use in the package should be provided. For popular datasets, \url{paperswithcode.com/datasets} has curated licenses for some datasets. Their licensing guide can help determine the license of a dataset.
%         \item For existing datasets that are re-packaged, both the original license and the license of the derived asset (if it has changed) should be provided.
%         \item If this information is not available online, the authors are encouraged to reach out to the asset's creators.
%     \end{itemize}

% \item {\bf New assets}
%     \item[] Question: Are new assets introduced in the paper well documented and is the documentation provided alongside the assets?
%     \item[] Answer: \answerNA{} % Replace by \answerYes{}, \answerNo{}, or \answerNA{}.
%     \item[] Justification: We do not release new assets, just the paper and code.
%     \item[] Guidelines:
%     \begin{itemize}
%         \item The answer NA means that the paper does not release new assets.
%         \item Researchers should communicate the details of the dataset/code/model as part of their submissions via structured templates. This includes details about training, license, limitations, etc. 
%         \item The paper should discuss whether and how consent was obtained from people whose asset is used.
%         \item At submission time, remember to anonymize your assets (if applicable). You can either create an anonymized URL or include an anonymized zip file.
%     \end{itemize}

% \item {\bf Crowdsourcing and research with human subjects}
%     \item[] Question: For crowdsourcing experiments and research with human subjects, does the paper include the full text of instructions given to participants and screenshots, if applicable, as well as details about compensation (if any)? 
%     \item[] Answer: \answerNA{} % Replace by \answerYes{}, \answerNo{}, or \answerNA{}.
%     \item[] Justification: Paper does not involve crowdsourcing or human subjects.
%     \item[] Guidelines:
%     \begin{itemize}
%         \item The answer NA means that the paper does not involve crowdsourcing nor research with human subjects.
%         \item Including this information in the supplemental material is fine, but if the main contribution of the paper involves human subjects, then as much detail as possible should be included in the main paper. 
%         \item According to the NeurIPS Code of Ethics, workers involved in data collection, curation, or other labor should be paid at least the minimum wage in the country of the data collector. 
%     \end{itemize}

% \item {\bf Institutional review board (IRB) approvals or equivalent for research with human subjects}
%     \item[] Question: Does the paper describe potential risks incurred by study participants, whether such risks were disclosed to the subjects, and whether Institutional Review Board (IRB) approvals (or an equivalent approval/review based on the requirements of your country or institution) were obtained?
%     \item[] Answer: \answerNA{} % Replace by \answerYes{}, \answerNo{}, or \answerNA{}.
%     \item[] Justification: Paper does not involve crowdsourcing nor research with human subjects.
%     \item[] Guidelines:
%     \begin{itemize}
%         \item The answer NA means that the paper does not involve crowdsourcing nor research with human subjects.
%         \item Depending on the country in which research is conducted, IRB approval (or equivalent) may be required for any human subjects research. If you obtained IRB approval, you should clearly state this in the paper. 
%         \item We recognize that the procedures for this may vary significantly between institutions and locations, and we expect authors to adhere to the NeurIPS Code of Ethics and the guidelines for their institution. 
%         \item For initial submissions, do not include any information that would break anonymity (if applicable), such as the institution conducting the review.
%     \end{itemize}

% \item {\bf Declaration of LLM usage}
%     \item[] Question: Does the paper describe the usage of LLMs if it is an important, original, or non-standard component of the core methods in this research? Note that if the LLM is used only for writing, editing, or formatting purposes and does not impact the core methodology, scientific rigorousness, or originality of the research, declaration is not required.
%     %this research? 
%     \item[] Answer: \answerNA{} % Replace by \answerYes{}, \answerNo{}, or \answerNA{}.
%     \item[] Justification: LLMs were used for editing/formatting suggestions but were not involved in the methodology.
%     \item[] Guidelines:
%     \begin{itemize}
%         \item The answer NA means that the core method development in this research does not involve LLMs as any important, original, or non-standard components.
%         \item Please refer to our LLM policy (\url{https://neurips.cc/Conferences/2025/LLM}) for what should or should not be described.
%     \end{itemize}

% \end{enumerate}


\end{document}

