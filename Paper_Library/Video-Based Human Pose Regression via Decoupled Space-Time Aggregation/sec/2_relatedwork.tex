\section{Related Work}
\label{sec:related_work}

\textbf{Heatmap-based Human Pose Estimation.}
%\textbf{Image-based approaches.} 
Since the introduction of likelihood heatmaps to represent human joint positions~\cite{FirstHeatmap_NIPS2014}, heatmap-based methods have become predominant in the field of 2D human pose estimation~\cite{HRNet_CVPR2019,HRFormer_NIPS2021,Transpose_ICCV2021,UDPPose_cvpr2020,Vitpose_NIPS2022}, owing to their superior performance. To perform multi-person human pose estiamtion, the top-down approaches initially identify person bounding boxes and subsequently conduct single-person pose estimation within the cropped regions~\cite{maskrcnn_ICCV2017,HRNet_CVPR2019,Vitpose_NIPS2022}. Conversely, bottom-up methods commence by detecting identity-free keypoints for all individuals and then cluster these keypoints into distinct persons~\cite{pifpaf_CVPR2019,hrhrnet_cvpr2020,SWAHR_CVPR2021,CenterAttention_iccv2021}. Recently, the heatmaps or CNN features from adjacent frames have been utilized to extract the temporal dependencies of human poses, thereby enhancing the performance of multi-person human pose estimation in video sequences~\cite{PoseWarper_NIPS2019,DCPose_CVPR2021,FAMIPose_CVPR2022}. Despite its effectiveness, the heatmap representation inherently suffers from several drawbacks, such as quantization errors and the high computational and storage demands associated with maintaining high-resolution heatmaps.

\noindent \textbf{Regression-based Human Pose Estimation.} 
Regression-based methods forgo the use of intermediate heatmaps, opting instead to map the input directly to the output joint coordinates~\cite{FirstRegression_CVPR2014}. This approach is flexible and efficient for a wide range of human pose estimation tasks and real-time applications, especially on edge devices. Despite their efficiency, regression-based methods have traditionally lagged behind heatmap-based methods in accuracy within the realm of human pose estimation, leading to less focus on their development~\cite{FirstRegression_CVPR2014,PRTR_CVPR2021,SPM_ICCV2019,PointSetAnchor_ECCV2020}. Recently, advancements such as RLE~\cite{RLE_ICCV2021} and Poseur~\cite{Poseur_ECCV2022} have significantly propelled regression-based approaches, elevating their performance to a level comparable with heatmap-based methods. However, these regression-based methods are designed exclusively for static images. When these image-based methods are directly applied to video sequences, they tend to yield suboptimal predictions due to their inability to capture temporal dependencies between frames. As a result, such models struggle with challenges inherent to video inputs, such as motion blur, defocusing, and pose occlusions, which are common in dynamic scenes. 

In this work, we present for the first time a regression-based approach for multi-person human pose estimation in video sequences, outperforming or is on par with state-of-the-art heatmap-based methods for video sequences.


