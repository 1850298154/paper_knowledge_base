\section{Introduction}
Time-optimal point-to-point motion planning, which involves transitioning the system from its current state to a desired terminal state in the shortest time while continuously satisfying stage constraints, finds wide applications in various fields such as robot manipulation, cranes, and autonomous navigation.
This problem can be approached in a two-level manner, comprising high-level geometric path planning and lower-level path following considering system dynamics (\cite{Verscheure09}). 
Alternatively, it can be directly solved in the system's state space, known as the direct approach (\cite{8264024}).
The direct approach commonly formulates the time-optimal motion planning problem as a receding horizon Optimal Control Problem (OCP) and is solved either offline or repeatedly in a Nonlinear Model Predictive Control (NMPC) implementation (\cite{Zhao04}).

In this paper, our focus is on direct approaches. 
One approach proposed in \cite{7331052} involves scaling the continuous-time system with a temporal factor before discretizing it.
The OCP formulated in this time scaling approach has a variable time grid with a fixed horizon length.
The temporal factor, treated as an additional decision variable, is minimized in the objective to achieve time-optimality.
% A work implementing this time scaling approach in the context of truck-trailer autonomous mobile robot parking maneuver planning is demonstrated in \cite{BOS20234877}.
When applied with a relatively small horizon length, the computational complexity remains manageable.
Yet, the time grid is typically coarse when the time needed to reach the terminal state is long, leading to a correspondingly coarse motion trajectory.
In accordance with the discrete-time control system, the resulting motion trajectory must undergo interpolation to align appropriately. 
Regrettably, this refinement process may give rise to infeasibility concerns attributed to interpolation errors.
An alternative approach proposed by \cite{8264024} chooses to use a fixed time grid, i.e., discretizing the system with the control sampling time.
This approach, referred to as the exponential weighting approach, opts for a fixed but much larger horizon length and minimizes the L1-norm of the deviation from the desired terminal state, weighted by exponentially increasing weights, to achieve time-optimality.
A limitation of the exponential weighting approach arises when the system transitions to a distant terminal state. 
The considerable horizon length poses computational challenges for real-time implementation and introduces numerical ill-conditions due to the exponentially increasing weights.

We propose a two-stage approach to formulate the time-optimal OCP. 
This approach involves a first stage with a fixed time grid corresponding to the control sampling time and a second stage with a variable time grid derived from the time-scaled system.
It leverages the advantages of both aforementioned approaches by formulating the time-optimal OCP with a fixed and low number of control steps for computational manageability and preempting interpolation errors in the first stage.
Solving the OCP formulated through this two-stage approach using the classical NMPC implementation scheme — specifically, solving the OCP within a single control sampling time and applying the first optimal control from stage 1 to the system — can still be challenging, particularly in scenarios characterized by complex system models and stage constraints.
To ensure complete convergence in every NMPC iteration, we employ the ASAP-MPC update strategy, an asynchronous NMPC implementation scheme (\cite{ASAP-MPC}) that is designed to handle the fluctuating computational delays.

\textit{Paper structure:} In Section \ref{Sec_II}, we introduce the time-optimal point-to-point motion planning problem and discuss two approaches to formulate it: the time scaling approach and the exponential weighing approach.
In Section \ref{Sec_III}, we state the proposed two-stage approach and subsequently present a scheme that integrates this two-stage approach with the ASAP-MPC update strategy to handle fluctuating computation delays.
In Section \ref{Sec_IV}, we compare the three approaches and showcase the presented NMPC scheme through numerical examples of autonomous navigation while avoiding collisions with obstacles.
Section \ref{Sec_V} concludes the paper.

\textit{Notation:} The set of positive integer numbers is denoted by $\mathbb{N}_+$.
The L1-norm of a variable $s$ is denoted by $\|s\|_1$.
The sequence of a variable $s$ is denoted by 
$\{s\}_{n=0}^N:=s_0,s_1,...,s_N$. 
