In this section, we first establish a tighter makespan lower bound as a function of the number of escorts. Then, we proceed to the more involved efforts of deriving tighter makespan upper bounds again as a function of the available number of escorts. The new and tighter lower and upper bounds are summarized in the table below. We further provide an exact constant for all upper bounds as a more precise characterization. In all cases, our upper and lower bounds match asymptotically, eliminating the gaps left by previous studies on \gstp. All upper bounds come with low-polynomial-time algorithms for computing the actual plan, which is clear from the corresponding proofs. 

\begin{table}[h!]
\begin{small}
\begin{tabularx}{\columnwidth}{| c | Y |}
\hline
$k$, the number of escorts  &  Makespan \textbf{lower} bound \\ \hline
$k < \min(m_1, m_2)$ & exp. $\Omega(\dfrac{m_1 m_2}{k})$ \\ \hline
$k \ge \min(m_1, m_2)$ & h.p. $\Omega(\max(m_1, m_2))$ \\ \hline  \hline
$k$, the number of escorts  &  Makespan \textbf{upper} bound \\ \hline
$k = 1$ & $(81 + o(1))m_1m_2$ \\ \hline
$k = 2$ & $(18 + o(1))m_1m_2$ \\ \hline
$2 < k < \min(m_1, m_2)$ & $(22 + o(1))\dfrac{m_1m_2}{\lfloor k/2\rfloor}$ \\ \hline
%$\min(m_1, m_2) \le k < m_1 + m_2 - 1 $ & \textcolor{red}{???} \\ \hline
% k $\min(m_1, m_2) \leq  < m_1 + m_2 - 1$ & $34\max(m_1, m_2)$ \\ \hline
$k \geq m_1 + m_2 - 1$ & $34\max(m_1, m_2)$ \\ \hline
\end{tabularx}
\end{small}
    \caption{Our matching makespan lower and upper bounds.}
    \label{tab:bounds}
\end{table}
For convenience, instead of viewing \gstp through batched tile movements, we focus on the movement of the escorts, which encodes tile motion more concisely.
%
A straight contiguous train of tiles moving in a single step may be equivalently viewed as a \emph{jump} of an escort. 
%
Since the new escort position must remain in the same row or column, we call the jump a \emph{row jump} or \emph{column jump}, respectively.
%
In addition, we use \emph{rectangular shift} or \emph{r-shift}, as a fundamental motion primitive in which the escort cycles through the four corners of a rectangle, thus shifting all boundary elements by one tile in the opposite direction. 
%
We call the rectangular shift \emph{cwr-shift} (resp., \emph{ccwr-shift}) if the escort  traverses the corners in the counterclockwise (resp., clockwise) direction. 

\subsection{Tighter Makespan Lower Bounds}
Using escort jumps instead of tile moves lets us see immediately that a single time step can only change the sum of the Manhattan distances by $k \max(m_1, m_2)$, where $k$ is the number of escorts. The observation readily leads to a tighter makespan lower bound than the previously established $\Omega(m_1+m_2)$ (or $\Omega(\max(m_1,m_2))$).

\begin{lemma}
%[Expected Makespan Lower Bound for \gstp]
    The expected minimum makespan for \gstp on $m_1 \times m_2$ grids with $k$ escorts is $\Omega(\frac{m_1 m_2}{k})$.
\end{lemma}

\begin{proof}
Consider the sum of Manhattan distances $S$ of each tile's start and goal positions.
%
Over all possible start and goal configurations, each tile is expected to have a Manhattan distance of $\Omega(m_1 + m_2) = \Omega(\max(m_1, m_2))$ \cite{santalo2004integral}. 
%
Because there are $m_1m_2 - k$ tiles, we have $S = \Omega((m_1m_2 - k)\max(m_1, m_2))$, in expectation.
%
Because each of the $k$ escorts can jump a distance of $\max(m_1, m_2)$ within the same row/column, altering the Manhattan distance contribution by $1$ for each tile in its jump path, in a single time step, $S$ can only change by at most $k\max(m_1, m_2)$.
%
Thus, at least $\frac{S}{k\max(m_1, m_2)}= \Omega(\frac{m_1 m_2}{k})$ steps are needed to solve the instance in expectation. %\textcolor{red}{Should this be made more rigorous? Computations of sum of Manhattan distances overall are not too difficult (last meeting - results exist and could be referenced). Also, can we get a high probability lower bound from random variables??? This seems reasonable$\ldots$}
\end{proof}
Combining with the previously known lower bounds yields a tighter makespan lower bound of $\Omega(\frac{m_1 m_2}{k})$ for $k \leq \min(m_1, m_2)$ and $\Omega(\max(m_1, m_2))$ for $k \geq \min(m_1, m_2)$, which match our developed upper bounds, established next. 

\subsection{Tighter Makespan Upper Bounds: Outline}
We leverage RTA (Sec.~\ref{sec:rta}) to establish tighter makespan upper bounds for \gstp. Each round of row/column shuffles in RTA can be executed in parallel, potentially leading to a significantly reduced makespan. However, the shuffles do not readily translate to feasible sliding-tile motion; performing in-place permutation of tiles in a single row/column is impossible. 
%
To enable the application of RTA, instead working with one grid row/column, we simulate row/column shuffles by grouping multiple rows or columns together. 
%
Therefore, at a high level, we derive better upper bounds by:
\begin{itemize}
    \item Applying RTA to obtain three batches of row or column shuffles (see, e.g., Fig.~\ref{fig:rta}) with escorts treated as labeled tiles. Each batch of shuffles will be executed to completion (via simulations according to rules of \gstp) before the next batch is started. 
    \item In a given batch of row/column shuffles, adjacent rows/columns will be grouped together (e.g., two or three rows per group), on which tile-sliding motions will be planned to realize the desired shuffles. 
\end{itemize}
%

Performing efficient tile-sliding motions with the \cfc constraint is key to establishing tighter upper bounds. We first describe subroutines for solving \gstp with $1$ or $2$ escorts on $3\times m$ and $2\times m$ grids. These subroutines will then be used to solve general \gstp instances. 

%
%While we spend significant effort pushing lower the constant factors for the makespan upper bounds, many details will be omitted due to limited space. We focus instead on ensuring the proofs are convincingly correct. Detailed arguments for the constant factors will be included in an extended version of the manuscript. 

\subsection{Upper Bounds for 2-3 Rows with 1-2 Escorts}
Our \gstp algorithms will build on subroutines for sorting multiple rows. We first prove such a routine on $3 \times m$ grids. 

\begin{lemma}\label{l:3m}%[Linear $3\times m$ Sort]
Feasible \gstp instances with a single escort on a $3 \times m$ grid can be solved in $120m$ steps.
\end{lemma}
\begin{proof}
We give a procedure that sorts the right $\frac{1}{3}$ of the $3 \times m$ grid in $O(m)$ steps. A recursive application of the procedure then yields an overall $O(m +\frac{2}{3}m +\frac{4}{9}m + \ldots) = O(m)$ makespan. 

To start, we move the escort to the bottom left corner for both the start and goal configurations, which takes $4$ steps. These will be the new start/goal configurations. 
%
From here, for tiles on a $3\times m$ grid, let $B$ denote the set of tiles corresponding to the $\lfloor \frac{m-2}{3} \rfloor$ rightmost columns in the goal configuration. We refer to these tiles as $B$ tiles and the rest as $W$ tiles. 
%
We will treat the boundary cells as a \emph{circular highway} moving clockwise and the inner middle line as a \emph{workspace} to move $B$ tiles to their destination. As an example, the $B$ (resp., $W$) tiles are shown in dark gray (resp., light gray) in Fig.~\ref{fig:3m}(b)-(g). The algorithm operates in three stages: (1) move $B$ tiles to the highway, (2) arrange $B$ tiles properly in the workspace, and (3) move $B$ tiles to goals.

\begin{figure}[h]
    \centering
    \begin{overpic}
    [width=\columnwidth]{figs/3m-w.pdf}    %{figs/MOGSTP Outline assembled.png}
    \put(43.5, 61){{\small (a)}}
    \put(93.5, 61){{\small (b)}}
    \put(43.5, 43){{\small (c)}}
    \put(93.5, 43){{\small (d)}}
    \put(43.5, 24.5){{\small (e)}}
    \put(93.5, 24.5){{\small (f)}}
    \put(43.5, 7){{\small (g)}}
    \put(93.5, 7){{\small (h)}}
    \end{overpic}
    \caption{Sorting right $\frac{1}{3}$ on a $3 \times 8$ grid with one escort. (a) and (h) are the start and goal configurations. (b)$\to$(c): A cwr-shift inserts $B$ tile $8$ to the circular highway. (c)$\to$(d)$\to\ldots\to$(e): A series of r-shifts orders $B$ tiles in the workspace. (e)$\to$(g): Additional r-shifts move $B$ tiles to goals.}
    \label{fig:3m}
\end{figure}

%
To execute the first stage, if a $B$ tile in the workspace has a $W$ tile above it, then execute a cwr-shift to insert the leftmost such $B$ tile into the highway to not affect tiles to the right (Fig.~\ref{fig:3m}(b)-(c)). 
%
Otherwise, apply \emph{adjustment} cwr-shifts to the circular highway until a $B$ tile in the workspace has a $W$ tile above it.
%Note that once a white tile appears over a $B$ tile in the workspace, it will never be shifted away through a circular highway rotation, since insertions are done left to right.
Because there are $m-2$ $B$ tiles, at most $m-2$ adjustments are needed to move a $W$ tile over each $B$ tile, and so the total number of steps for this stage is at most $4[(m-2) + (m-2)] = 8m - 16$.
% I think amortization refers to average case, which doesn't seem to apply here
% Amortization usually refers to local unevenness but total being the same 
% Ah, I see, okay

The second stage uses the same operation to insert the $B$ tiles into the workspace. The difference is that $B$ tiles are now being inserted in the exact spot in the workspace corresponding to the desired permutation.
%
Through the process, a tile in $B$ never makes a full lap around the circular highway. Therefore, at most $2m+1$ adjustments are needed, with at most $m-2$ $B$ tile insertions, taking at most $4[(2m+1) + (m-2)] = 12m - 4$ steps.

In the third stage, apply r-shifts to move $B$ tiles to their goals as shown in Fig.~\ref{fig:3m}(e)-(g), taking $4[m - \lfloor \frac{m-2}{3} \rfloor]$ steps.

Now, approximately the right third of the grid has been solved in $4[6m - 5 - \lfloor \frac{m-2}{3} \rfloor]$ steps; we recurse in the same manner for $m \geq 5$ and solve the base case of $m = 4$ in 53 times steps \cite{korf2008linear} by treating the problem as a normal $(n^2 - 1)$-puzzle instance.
%
Through careful counting, we can conclude that $120m$ steps are always sufficient. 
\end{proof}

The $120m$ makespan can be significantly reduced with more careful analysis, which we omit due to limited space. The important takeaway is Lemma~\ref{l:3m} shows \gstp on $3\times m$ grids can be solved in $O(m)$ steps, sufficient for establishing the upper bounds in our claimed contribution. In what follows, we describe related results needed to get the constant factors stated in Table~\ref{tab:bounds} omitting the proofs. 

If we have two escorts, we can cycle them on opposite corners of their respective r-shifts to allow two cwr-shifts to happen simultaneously, leading to the following. 
\begin{corollary}
\gstp instances with two escorts on a $3 \times m$ grid can be solved in $60m$ steps.
\end{corollary}

With significant additional efforts but following a similar line of reasoning, we can establish on $2\times m$ grids that
\begin{lemma}\label{l:2m1}%[Linear $3\times m$ Sort]
Feasible \gstp instances with a single escort on a $2 \times m$ grid can be solved in $58m$ steps.
\end{lemma}

While simulating two row or column permutations at once can be useful in solving \gstp faster, the limited amount of space may prevent us from doing so. Instead, simulating the permutation of one row or column will be much more useful.

\begin{corollary}\label{c:2m11}%[Linear $2 \times m$ Half-Sort]
Given a single escort, a $2 \times m$ grid can be permuted to fill one of its rows arbitrarily in $27m$ steps.
\end{corollary}

With two escorts, we get significantly faster algorithms.
\begin{lemma}\label{l:2m2}
\gstp instances with two escorts on a $2 \times m$ grid can be solved in $10m - 13$ steps.
\end{lemma}

\begin{corollary}\label{c:2m21}
Given two escorts on the left of the top row of a $2 \times m$ grid, the bottom row can be arbitrarily permuted in $6m-1$ time steps, maintaining the position of the escorts.
\end{corollary}

Corollaries~\ref{c:2m11} and~\ref{c:2m21} will be instrumental in parallelizing row and column permutations necessitated by the RTA Shuffles without wasting additional steps in permuting the other row. %While shuffling two rows at once with the previous lemma would be ideal, issues arise when trying to permute two rows at once since escorts must replace two of the tiles in those rows.

\subsection{Tighter Makespan Upper Bounds for \gstp}
We are now ready to tackle solving full \gstp instances. 
For \gstp, we will only examine the case in which grid dimensions are at least $2$; the problem is otherwise trivial. 

\begin{theorem}%[Single Escort MRPP with CFC Makespan]
Feasible single-escort \gstp instances can be solved in $81m_1 m_2 + 6m_1 + 9m_2 - 3$ steps.
\end{theorem}
\begin{proof}
First, move the escort to the top left for start/goal configurations to get new start/goal configurations. 
%
Then, RTA is applied in a row-column-row fashion to yield three batches of row/column shuffles. Each batch requires sorting $m_1$ or $m_2$ rows or columns. In the $4 \times 4$ grid shown in Fig.~\ref{fig:gstp1}(a), a batch of row shuffles must permute each of the four rows highlighted in different colors. We are done if we can successfully perform each batch of shuffles. 

To execute a batch of shuffles, e.g., performing the four row shuffles on the $4 \times 4$ grid shown in Fig.~\ref{fig:gstp1}(a), we move the escort to the top left of the bottom two rows and apply Corollary~\ref{c:2m11} sort the last row. The procedure is repeated with the escort moved one row above, until there are only two top rows, at which point Lemma~\ref{l:2m1} is invoked to arrange the two rows simultaneously. The top two rows may not be solved exactly because not all $(N^2-1)$-puzzles are solvable, but the issue will resolve on its own if the \gstp instance is solvable. Other shuffles are executed similarly.
\begin{figure}[h]
    \centering
    \begin{subfigure}{0.24\columnwidth}
        \includegraphics[width=\columnwidth]{figs/Slide1.png}
        \caption{}
    \end{subfigure}
    \begin{subfigure}{0.24\columnwidth}
        \includegraphics[width=\columnwidth]{figs/Slide2.png}
        \caption{}
    \end{subfigure}
    \begin{subfigure}{0.24\columnwidth}
        \includegraphics[width=\columnwidth]{figs/Slide3.png}
        \caption{}
    \end{subfigure}
    \begin{subfigure}{0.24\columnwidth}
        \includegraphics[width=\columnwidth]{figs/Slide4.png}
        \caption{}
    \end{subfigure}
    
    \caption{Illustrating performing a batch of row shuffles on a $4 \times 4$ grid with a single escort. 
    %
    (a). The (updated) start configuration, in which each row must be permuted. (b). To prepare for running Corollary~\ref{c:2m11}, the escort is moved to the top left of the last two rows. (c). After applying Corollary~\ref{c:2m11} to sort the last row, the escort is shifted above for the next application. (d) The top two rows will be sorted using Lemma~\ref{l:2m1}. Note that the top (resp., left) two rows (resp., columns) may not be fully solvable in the first two batches of shuffles, which is fine for the next set of column shuffles.}
    \label{fig:gstp1}
\end{figure}

Counting all steps, the total number is at most $81m_1 m_2 + 6m_1 + 9m_2 - 3$.
%The full procedure includes $2(m_1 - 2)$ half sorts of a $2 \times m_2$ grid, 2 full sorts of a $2 \times m_2$ grid, $m_2 - 2$ half sorts of a $m_1 \times 2$ grid, and a full sort of a $m_1 \times 2$ grid.
%Based on Corollary~\ref{c:2m11} and Lemma~\ref{l:2m1}, this takes $54m_2(m_1 - 2) + 116m_2 + 27m_1(m_2 - 2) + 58m_1 = 81m_1 m_2 + 8m_2 + 4m_1$ steps.
%Additionally, $4$ steps are needed to relocate the escort to the top left corner, $2(m_1 - 1)$ steps to move the escort between row permutations, and $(m_2 - 1)$ steps needed to move the escort between column permutations. The total number of steps is at most $81m_1 m_2 + 6m_1 + 9m_2 - 3$.
\end{proof}
%It takes $O(m_1^2 m_2)$ time to run the Rubik table algorithm, $O(m_1^2 m_2)$ time to compute the linear $2 \times m_1$ sorts and $O(m_2^2 m_1)$ time to compute the linear $2 \times m_2$ sorts, but we must specify all $m_1 m_2 - 1$ paths in the routing procedure, resulting in an $O(m_1^2 m_2^2)$ computation time.

\begin{theorem}%[Two Escort MRPP with CFC Makespan]
A two-escort \gstp instance can be solved in $18m_1 m_2 - 4m_1 - 5m_2 - 29$ steps.
\end{theorem}
\begin{proof}[Proof Sketch]
The proof is similar to the single escort case; with two escorts, we invoke Lemma~\ref{l:2m2} and Corollary~\ref{c:2m21} to speed up the process. The entire instance can be solved in $2[(m_1 - 2)(6m_2 - 1) + 10m_2 - 13] + [(m_2 - 2)(6m_1 - 1) + 10m_1 - 13] + 4 = 18m_1 m_2 - 4m_1 - 5m_2 - 29$ steps.
\end{proof}
%The computation takes $O(m_1^2 m_2^2)$ time as in the single escort case, bottlenecked by specifying the overall routing.

\begin{theorem}
A \gstp instance containing $2 \leq k < \min(m_1, m_2)$ escorts, where $k$ is even, can be solved with a makespan less than $\frac{44m_1 m_2}{k} + m_1(5 - \frac{24}{k}) + 15m_2 - 29$.
\end{theorem}
\begin{proof}[Proof Sketch]
The main strategy is distributing the escorts across the rows/columns to introduce parallelism in solving a batch of row/column shuffles. For example, given $k=2\ell$ escorts, to solve a batch of $m_1$ row shuffles, we can distribute two escorts per $\frac{m_1}{\ell}$ rows. For each such $\frac{m_1}{\ell}$ rows, we invoke Lemma~\ref{l:2m2} and Corollary~\ref{c:2m21} to solve them, in parallel.  
%
This allows the entire batch of row shuffles to be completed in $O(\frac{m_1}{\ell})O(m_2)=O(\frac{m_1m_1}{\ell})=O(\frac{m_1m_1}{k})$ steps. 
%
Tallying over the three phases, the total number of steps required is bounded by
$\frac{44m_1 m_2}{k} + m_1(5 - \frac{24}{k}) + 15m_2 - 29$.
\begin{comment} 
We can compute the makespan of the described algorithm as follows:
completing the first set of row permutations takes $(\lceil \frac{2 m_1}{k} \rceil - 2)(6m_2 - 1) + 10m_2 - 13$ steps for the $2 \times m_2$ sorts, with $1 + \lceil \frac{2m_1}{k} \rceil - 1$ steps needed to move the escorts between the rows.
Similarly, the set of column permutations takes $(\lceil \frac{2m_2}{k} \rceil - 2)(6m_1 - 1) + 10m_1 - 13$ steps for the $m_1 \times 2$ sorts, with $1 + \lceil \frac{2m_1}{k} \rceil - 1$ needed to move the escorts between the columns.
The last set of row permutations takes $\lceil \frac{2m_1}{k} \rceil (10 m_2 - 13)$ time steps to perform the row permutations and $\lceil \frac{2m_1}{k} \rceil + 1$ steps to move the escorts between the rows, moving them to the top row afterwards.
Thus, rounding upwards, the overall makespan is less than $[4+m_1 + m_2] + [\frac{4m_1}{k} + \frac{2m_2}{k} + 4] + [\frac{44m_1 m_2}{k} + m_1(4 - \frac{28}{k}) + m_2(14 - \frac{2}{k}) - 37] = \frac{44m_1 m_2}{k} + m_1(5 - \frac{24}{k}) + 15m_2 - 29$.
\end{comment}
\end{proof}
Note that if $k$ is odd, we can simply ignore one escort.
It is also clear the results continue to apply for $k \min(m_1, m_2)$ by ignoring extra escorts, but we can get additional speedups when $k \geq m_1 + m_2 - 1$ due to enough room to use \ref{l:2m2} straightforwardly by having escorts along the top row and left column.
Compiling everything so far yields the claimed bounds given in Table.~\ref{tab:bounds}.
%Computing the instance depends primarily on determining the routing as well as the Rubik table algorithm, and so this can be done in $O(m_1^2 m_2 + \frac{m_1^2 m_2^2}{k})$ time, or more simply, $O(\max(m_1, m_2)^2 \min(m_1, m_2))$ time.

