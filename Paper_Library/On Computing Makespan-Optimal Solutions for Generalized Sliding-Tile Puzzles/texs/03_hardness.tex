We proceed in this section to establish the NP-completeness of \mogstp on square grids, which will show  

\begin{theorem}\label{t:nph}
\mogstp is NP-complete, with or without an enclosing grid. 
\end{theorem}

First, we sketch the proof to provide key ideas behind the reduction of hardness. Then, detailed constructions of the required gadgets and the full instance construction follow. 

\subsection{Proof Outline}
We prove via a reduction from \ttfsat \cite{RATNER1990111} defined in Sec.~\ref{sec:224}. 
%
Our reduction constructs an \mogstp instance to force a flow of literal tiles from variable gadgets to clause gadgets in matching pairs, forming a truth side of literals and a false side of literals (realized through a \emph{gadget train}, see Fig.~\ref{fig:nph-sketch}(a) for a sketch and explanation). For each variable $x_i$, $1\le i\le n$, there are four sliding tiles labeled $x_i^1,x_i^2,\bar x_i^1,\bar x_i^2$ that correspond to the four literals for $x_i$, the first pair positive and the second pair negative.
When the context is clear, we simply say \emph{literals} instead of \emph{literal tiles}. A variable gadget (see Fig.~\ref{fig:nph-sketch}(b) and Fig.~\ref{fig:vg}) is constructed that forces the pair of unnegated literals (e.g., $x_i^1$ and $x_i^2$, ``+'' tiles in the figure) to only exit together from one side of the gadget (e.g., left) while forcing the pair of negated literal (e.g., $\bar x_i^1$ and $\bar x_i^2$, , ``-'' tiles in the figure) to exit together from the opposite side, each passing through limited openings of the \emph{rails} that flank the train and move in the opposite direction.
After all $4n$ literals exit from the $n$ variable gadgets, there are $2n$ each on the left and right side of the rails. These literals are then routed into clause gadgets (see Fig.~\ref{fig:nph-sketch}(c) and Fig.~\ref{fig:cg}), each allowing at most two literals to enter from each side.
The overall \mogstp instance is constructed such that if the \ttfsat is satisfiable, then in the \mogstp, the $2n$ literal tiles that move to the left side of the train can be chosen to be the true literals in the given truth assignment, and so all literal tiles can then be readily routed to the clause gadgets. Similarly, in the other direction of the reduction, because exactly $n$ pairs of literal tiles must be on the left side in a makespan-optimal solution, the corresponding $2n$ literals can be set to positive to satisfy the \ttfsat instance. 
\begin{figure}[h]
    \centering
    \begin{overpic}
    [width=\columnwidth,tics=5]{figs/nph-sketch.pdf}
    %{figs/MOGSTP Outline assembled.png}
    \put(7.8, -3){{\small (a)}}
    \put(33, -3){{\small (b)}}
    \put(62.5, -3){{\small (c)}}
    \put(88.5, -3){{\small (d)}}
    \end{overpic}
    \caption{Pieces of \mogstp. (a) Sketch of the train-like \mogstp instance split into two halves. The upward-moving gadget train is surrounded by two (red) rails of tiles that move strictly downwards, with a few gaps (not shown here, see Fig.~\ref{fig:rails}) to allow tiles to exit/enter.  
    The train, from top to bottom, contains a front padding car $P_f$, variable cars $x_1, \ldots, x_n$, a security car $S$, a middle padding car $P_m$, clause cars $c_1, \ldots, c_n$, and a rear padding car $P_r$. (b) A variable gadget (center $10\times 3$ portion) is constructed to force unnegated (``+'')  and negated (``-'') literal tiles to exit from different sides. The exited titles will be outside the rails. (c) A clause gadget is constructed to allow at most two literals to come in from each side of the rails. (d) The security car where the upper four purple tiles will exit to block variable exits on the rails (see Fig.~\ref{fig:rails}). The lower two light purple blocks are goals for two tiles initially on the rails (Fig.~\ref{fig:rails}).} 
    \label{fig:nph-sketch}
\end{figure}
%- Explain the proof at a high level, briefly explaining the reductions in both ways making it clear that the separation of variables positive/negative pairs is important. While doing this, use figures showing the overall constructed instance and the main phases (beginning, variable exiting/splitting, entering clauses)

\subsection{Gadgets}
Our gadgets consist of \emph{preset tiles} that move in a fixed direction throughout the solution routing process. The \emph{up} (resp., \emph{down}) tiles move one step up (resp., down) at each time step, which can be forced by setting their goals a distance upwards (resp. downwards) equal to the given makespan of the \mogstp instance.

\subsubsection{Rail (Gadget)} A \mogstp instance contains two symmetric rails (red strips on the two sides in Fig.~\ref{fig:nph-sketch}, with more details in Fig.~\ref{fig:rails}) consisting of down tiles with gaps, which are $3 \times 1$ blocks of escorts.
Each rail contains three gaps, separated into two groups: two lower gaps are designated as \emph{variable exits}, and a single upper gap functions as a \emph{clause entrance}.
A down tile separates the variable exits. The four purple tiles from the security car will enter the middle of these gaps and then move with the rails until the end. There are two (purple) tiles initially in the entrance gaps that will later enter the security car. These gaps will be explained in more detail. 

\begin{figure}[h]
    \centering
    \begin{overpic}
    [width=\columnwidth]{figs/rail.pdf}    %{figs/MOGSTP Outline assembled.png}
    \put(17, 7.5){{\small variable exit gaps }}
    \put(57, 7.5){{\small clause entrance gaps }}
    \end{overpic}
    \caption{Part of the rails, rotated $90$ degrees clockwise from Fig.~\ref{fig:nph-sketch}(a), showing the variable exit and clause entrance gaps.}
    \label{fig:rails}
\end{figure}

%Besides the downward-going rails, gadgets will take the form of \emph{cars} that move upwards. 
%which will be flanked by nearly-infinite downwards moving symmetric sliding doors built from $T_D$'s except with a few small openings, or contiguous blocks of 3 escorts, to allow for tiles from the train to enter and exit.
\subsubsection{Variable Car (Gadget)} 
A variable car ($x_1, \ldots, x_n$ blocks in Fig.~\ref{fig:nph-sketch}) is an upwards moving $10 \times 3$ block whose start configuration is shown in Fig.~\ref{fig:nph-sketch}(b). For the variable car corresponding to $x_i$, besides the (blue) up tiles as marked,  there are two unnegated literal tiles (the two ``+" tiles) corresponding to $x_i^1$ and $x_i^2$, and two negated literal tiles (the two ``-" tiles) corresponding to $\bar x_i^1$ and $\bar x_i^2$. These tiles must be moved to some clause cars to be introduced shortly. 
%
There are also (pink) \emph{single-delay} up tiles that must pause in place exactly once throughout the execution of the \mogstp instance. 
%
Additionally, there are eight obstacle tiles (the x tiles) whose goal configurations are three spots lower within the same variable car. These obstacle tiles help ensure that the pairs of positive and negative literals split up onto different sides.

\begin{figure}[h]
    \centering
    \includegraphics[width=\columnwidth]{figs/vg.pdf}
    \caption{Illustration of how (green) literal tiles may exit a variable gadget in pairs. The bottom left subfigure shows the four lower gaps on the rails, each a $3 \times 1$ block.}
    \label{fig:vg}
\end{figure}

\begin{lemma}
    As a variable car passes by the variable exits on the rails, the positive and negative literals can only exit to different sides of the rails.
\end{lemma}
\begin{proof}[Proof Sketch]
Only literal and obstacle tiles may move outside a variable car (the $10\times 3$ grid). It can be shown that obstacle tiles should not change columns.
Because of this, unnegated (resp., negated) literals can only exit from the $3$th (resp., $7$th) row. This forces the obstacle tiles to become asymmetric on the two sides of a variable car, resulting in the unnegated literals exiting from one side of the car and the negated literals exiting from the opposite side. One such exit sequence is illustrated in Fig.~\ref{fig:vg}. 
\end{proof}

\begin{comment}
\begin{proof}
Only literal and obstacle tiles may move outside a variable car (the $10\times 3$ grid). Because an obstacle tile only has three non-upward moves, it must immediately move back if it moves outside the car due to \cfc. If this is the case, the spot left as it moves outside the car must remain empty due to \cfc so it can move back immediately. This means the move does not facilitate any other tile movement and can be replaced by having the tile remain stationary for the two steps. An obstacle tile cannot move to the middle of the car because of the single-delay up tiles. Therefore, obstacle tiles do not change columns. 

Because obstacle tiles cannot change columns, unnegated (resp., negated) literals can only exit from the $3$th (resp., $7$th) row. This effectively forces the obstacle tiles to become asymmetric on the two sides of a variable car, resulting in the unnegatd literals exiting from one side of the car and the negated literals exiting from the opposite side. One such exit sequence is given in Fig.~\ref{fig:vg}; we omit additional details due to limited space. 
\end{proof}
\end{comment}
%First, note that if obstacle tiles ever moved away from the car, then they would have to move right back since they can only move down 3 spaces with respect to the car, 2 already being used to move horizontally.
%Then some tiles may follow behind it but must move right back due to the CFC.
%Let $B$ be the tile that follows last.
%    Then again by the CFC, no tile could have occupied its position in the middle time step, and so in a tile routing, we can instead have the tiles remain stationary for two time steps without impacting the solution routing.
%    Thus, the obstacle tiles can be assumed to remain inside the car.
    % In addition, obstacle tiles can be assumed to not move to the middle column.
    % If they do, then they can only occupy the 6th row, which can only be done by the two lower obstacle tiles to reach their destination in time.
    % However, the obstacle tiles must occupy different rows in the same order as the initial configuration due to the CFC, from which we can again assume that they stay in their side column
    
    %Now, note that negative literals must exit through the 7th row and that the positive literals must exit through the 3rd row:
    %if they didn't, then they would have to move to a side column to at least the 5th row, forcing all obstacle tile to move below, preventing any from occupying that spot in the final configuration.
    %Then since the openings in the sliding doors are 4 spaces apart, the positive tile must leave at least as early as the negative tile.

    %Suppose a positive and negative literal leave through the same side.
    %Note that when a positive literal leaves, it displaces at least 3 obstacle tiles downwards due to the CFC, preventing any negative literal from leaving on that side in a later time step.
    %Thus, the positive and negative literal must leave simultaneously, but this would cause an obstacle robot to move to the middle column, incapable of returning to its original side.
    %Thus, positive and negative literal tiles cannot leave through the same side.


\subsubsection{Clause Car (Gadget)}
As shown in Fig.~\ref{fig:nph-sketch}(c), the clause car is an upward-moving $10 \times 3$ subgrid entirely composed of up tiles and requires $4$ literal tiles corresponding to $c_i$ in the goal configuration. With two symmetric $3\times 1$ gaps on the rail, it is clear that at most two literals can enter from each side, as shown in Fig.~\ref{fig:cg}.
\begin{figure}[h]
    \centering
    \includegraphics[width=\columnwidth]{figs/cg.pdf}
    \caption{Green literal tiles entering a clause car gadget.}
    \label{fig:cg}
\end{figure}

\subsubsection{Security Car (Gadget)}
Shown in Fig.~\ref{fig:nph-sketch}(d), the security car is an upward-moving $10 \times 3$ subgrid whose frame consists of up tiles with four additional purple tiles at rows $3$ and $5$ that need to be injected into the middle of the four variable exits on the rails (see Fig.~\ref{fig:rails} for reference) as they pass by. This prevents these gaps from being used by clause gadgets. It will receive the two purple tiles that are initially inside the clause entrances (see Fig.~\ref{fig:rails}) to go to row $8$.

\subsection{Complete Specification and Reductions}
We construct the \mogstp instance as follows. Let $d = 24n + 38$; our upwards moving train is a $d \times 3$ block; from top to bottom, the three padding cards have $4$, $4n$, and $24$ rows of up tiles, respectively. The middle padding car allows the literal tiles to reorder before entering the clause cars. Initially, the bottoms of the rails are aligned with the bottom of the gadget train. 
%
The \emph{variable exit} openings occupy rows $d+2$ to $d+4$ and $d+6$ to $d+8$ from the bottom. The \emph{clause entrance} opening occupy rows $d+14n+12$ to $d+14n + 14$.
%
While a grid is not needed, we can select our grid $G$ to be of size $4d \times 4d$ with the construct positioned in the middle horizontally, with the bottom of the construct starting at the $(d+1)$th row from the bottom of $G$.
%
The makespan bound $K$ is set to $d$. The \mogstp instance is fully specified.

\begin{proof}[Proof of Thm.~\ref{t:nph}]
If the \ttfsat instance is satisfiable, we can select the positive (resp., negative) literals to exit to the left (resp., right) of the rails from the variable cars, when they pass by the literal exit gaps. Then, these literals can reorder and enter into the clause gadgets as required, reaching the target goal configuration with a makespan of $d$. 
% this seems too shortened and redundant with the previous proof outline, not giving any more insight; there is no mention of how the literal tiles sort, although it's not a key idea in the proof

Similarly, in the other direction, if the \mogstp instance has a solution with a makespan of $d$, then the up/down tiles must move uninterrupted. In this case, four literals must exit a variable car in pairs of the same truth value to different sides. Subsequently, these literals reorder and enter the clause cars as described. Therefore, we can pick literal tiles on one side of the rails, e.g., left, and make their corresponding literals positive, ensuring all clauses are true. This yields a satisfying assignment for the \ttfsat instance. 

\mogstp is in NP since the existence of a solution can be readily checked, and a feasible solution can be computed in polynomial time similar to how $(N^2-1)$-puzzles are solved. Thus, \mogstp is NP-complete.
\end{proof}

$\mogstp$ remains NP-hard when we specify that there are exactly $\lfloor |G|^\epsilon \rfloor$, $0< \epsilon < 1$ escorts (where $|G|$ is the number of cells of the grid) by blowing up the grid by a polynomial amount and filling the extra space with stationary tiles to achieve the desired number of escorts. Then note that \ttfsat is still simulated through the movement of the literal tiles around the preset tiles, and in addition, a solution routing can be constructed in the same manner from a truth assignment.

