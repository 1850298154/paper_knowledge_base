The $15$-puzzle \cite{loyd1959mathematical} is a sliding-tile puzzle in which fifteen interlocked square tiles, labeled $1$-$15$, and an empty escort square are located on a $4\times 4$ square game board (see Fig.~\ref{fig:15-gstp}). In each time step, a tile neighboring the escort may slide into it, leaving an empty square that becomes the new escort. The game's goal is to reconfigure the tiles to realize a row-major ordering of the labeled tiles. We study a natural generalization of the $15$-puzzle, in which the game board is an arbitrarily large rectangular grid with $1+$ escorts.
In addition, tiles can move synchronously in a given time step assuming no collision under uniform movement.
% A straight block of contiguous tiles may slide synchronously in the same direction, provided that the head of the block is slid into an escort.
We call this problem the \emph{generalized sliding-tile puzzle} or \gstp. 
% This doesn't seem to capture all the movement possible in the GSTP; maybe something along the lines of "tiles can move synchronously in a single step assuming no collision under uniform movement"
%general comment: this paragraph seems to be the same as the abstract - should there be something else? Seems a little redundant.
% It is fine to have redundancy between the abstract and the rest of the paper - this is actually common.

\begin{figure}
    \centering
    \begin{overpic}
    [width=0.60\columnwidth]{figs/15.pdf}
    %{figs/MOGSTP Outline assembled.png}
    %\put(11.3, 2){{\small (a)}}
    %\put(67.8, 2){{\small (b)}}
    \end{overpic}
    \caption{Start and goal configurations of a $15$-puzzle instance. In  \gstp, there can be $1+$ escorts and multiple tiles may move synchronously, e.g., tile $3$ and $9$ may move to the right in a single step in the left configuration.}
    \label{fig:15-gstp}
\end{figure}

\gstp provides a high-fidelity discretized model for multi-robot applications operating in grid-like environments, including the efficient coordination of a large number of robots in warehouses for order fulfillment \cite{wurman2008coordinating,mason2019developing}, motion planning in autonomous parking garages \cite{guo2023efficient}, and so on. 
%
A particularly important feature of \gstp is that, given two neighboring tiles sharing a side, one tile may only move in the direction toward the second tile if the second tile moves in the same direction. Otherwise, if the second tile moves in a perpendicular direction, a collision occurs, which we call the \emph{corner following constraint} or \cfc. 
%
Consideration of \cfc renders \gstp different from popular multi-agent/robot pathfinding (MAPF) problems \cite{stern2019multi} in which a classical formulation allows the second tile to move in a direction perpendicular to the moving direction of the first tile. Ignoring \cfc significantly reduces the steps required to solve a tile reconfiguration problem, making computing optimal solutions less challenging, but is less accurate in modeling many real-world applications.  

Given the strong connections between \gstp and today's grid-based multi-robot applications seeking ever more optimal solutions, we must have a firm grasp on the fundamental optimality structure of \gstp. Towards achieving such an understanding, this work studies the induced optimality structure in computing makespan-optimal solutions for \gstp, and brings forth the following main contributions: 
\begin{itemize}
    \item We establish that computing makespan-optimal solutions for \gstp is NP-complete with or without an enclosing grid. The problem remains NP-complete when there are $\lfloor |G|^\epsilon \rfloor$ escorts, where $|G|$ is the grid size (i.e., the total number of grid cells) and $0 < \epsilon < 1$ is a constant. 
    %
    \item We establish tighter makespan lower bounds for \gstp for all possible numbers of escorts. On an $m_1 \times m_2$ grid with $k$ escorts, in expectation, solving \gstp requires $\Omega(\frac{m_1 m_2}{k})$ steps for $1 \le k < \min(m_1, m_2)$ and $\Omega(m_1 + m_2)$ steps for $k \ge \min(m_1, m_2)$. 
    \item We establish tighter makespan upper bounds for \gstp for all possible numbers of escorts that match the corresponding makespan lower bounds, asymptotically, thus closing the makespan optimality gap for \gstp. This leverages a key intermediate result showing that \gstp instances on $2\times m$ and $3 \times m$ grids can be solved in $O(m)$ steps. For all upper bounds, via careful analysis, we further provide a constant factor that is relatively low, considering \cfc's severe restrictions on tile movements.  
\end{itemize}

Some proofs are sketched or omitted; see supplementary materials for additional details. 

\section{Related Work}
Modern studies on MAPF and related problems originated from the investigation of the generalization of the $15$-puzzle \cite{loyd1959mathematical} to the ($N^2-1$)-puzzle, with work addressing both computational complexity \cite{RATNER1990111} and the computation of optimal solutions \cite{culberson1994efficiently,culberson1998pattern}. Gradually, graph-theoretic abstractions emerged that introduced non-grid-based environments and allowed more escorts (i.e., there can be more than one empty vertex on the underlying graph). 
%
Whereas such problems are solvable in polynomial time if only a feasible solution is desired \cite{kornhauser1984coordinating,auletta1999linear,yu2013linear}, computing optimal solutions are generally NP-hard \cite{wilson1974graph,goldreich2011finding,surynek2010optimization,yu2013structure,Demaine2019CMP}.
%
With the graph-based generalization, \cfc is generally not enforced as the geometric constraint lengthens a motion plan and complicates the reasoning. 
%

Due to its close relevance to a great many high-impact applications, e.g., game AI \cite{pottinger1999implementing}, warehouse automation \cite{wurman2008coordinating,mason2019developing}, great interests started to develop in quickly computing (near-)optimal solutions for MAPF \cite{silver2005cooperative}. With this development, a variant of the ($N^2-1$)-puzzle was introduced, which does not require the presence of escorts \cite{standley2010finding}. In other words, in the most well-studied MAPF formulation, any non-self-intersecting chain of agents may potentially move synchronously, one following another, in a single step. In \cite{standley2010finding}, a bi-level algorithmic solution framework, \emph{operator decomposition} (OD) $+$ \emph{independence detection} (ID), is built upon the general idea of \emph{decoupling} \cite{erdmann1987multiple}, which treats each agent individually as if other agents do not exist and handles agent-agent interactions on demand. 
%
A super-majority of modern MAPF methods have generally adopted a bi-level decoupling search approach. Representative work along this line includes \emph{increasing cost-tree search} (ICTS) \cite{sharon2013increasing}, 
\emph{conflict-based search} (CBS) and variants \cite{sharon2015conflict,barer2014suboptimal,li2021eecbs}, \emph{priority inheritance with backtracking} (PIBT) \cite{okumura2022priority}, and most recently, \emph{lazy constraints addition search
for MAPF} (LaCAM) \cite{okumura2023lacam}. Besides search-driven methods, reduction-based approaches have also been proposed \cite{surynek2012towards,erdem2013general,yu2016optimal}. 

In contrast, MAPF formulations similar to \gstp, i.e., considering \cfc, have received relatively muted attention. On the side of computational complexity, besides the hardness result of the $(N^2-1)$-puzzle \cite{RATNER1990111} and a recent followup \cite{DEMAINE201880}, it has been shown that computing total distance-optimal solutions with \cfc is NP-complete in environments with specially crafted obstacles \cite{geft2022refined}. We note that sliding-tile puzzles can easily become PSPACE-hard in non-grid-based settings \cite{hopcroft1984complexity}, even for unlabeled tiles \cite{solovey2015hardness}. While of practical importance, hardness for computing optimal solutions for \gstp in obstacle-free settings has not been established. 
%
On the side of computational efforts in addressing \gstp, \cfc has been studied partially as part of $k$-robustness \cite{atzmon2018robust}. A recent SoCG competition has been held \cite{fekete2022computing} that addresses exactly the \gstp problem but with a focus on computing solutions for a set of benchmark problems. A variation of \gstp was studied in \cite{guo2023efficient} targeting autonomous parking garage applications. These computational studies largely leave unanswered fundamental questions on \gstp, including computational complexity and optimality bounds.  

%Grid Multi-Robot Path Planning (MRPP) has been extensively studied due to its numerous applications.
%In this problem, the objective is to route the robots from an initial configuration to a final configuration along collision free paths efficiently, usually with the goal of minimizing the overall routing time called the makespan.
%Recently, a specific variant prohibiting a robot from following another one around a corner, called the corner-following constraint (CFC), has been studied through the design of automated garages \cite{guo2023efficient}, but it is just as important for space efficient robotic systems, such as in automated grocery packing warehouses.
%We denote this problem as $\mrppcfc$.
%
%In this paper, we make significant progress towards understanding the hardness of $\mrppcfc$ and developing more efficient algorithms for it, bridging the gap between the upper and lower bounds when there are a small number of escorts.

%\subsection{Previous Work}
%$\mrppcfc$ has been previously studied and shown to have a tight upper and lower bound when there are $\Theta(m_1 m_2)$ escorts in an $m_1 \times m_2$ grid \cite{guo2023efficient}.
%However, there is a gap between the upper and lower bounds when there are $\Theta(m_1 + m_2)$ or $\Theta(1)$ escorts.
%This is in constrast to grid MRPP, which is known to have constant factor algorithms \cite{guo2022sub15, Demaine2019CMP}.
%Not much is known about the intractability of $\mrppcfc$ under the makespan or total arrival time objectives, nor anything about it's colored variants.
%On the other hand, grid MRPP has already been shown to be NP-hard \cite{Demaine2019CMP, gupta2019parameterized} including many of its colored versions \cite{Demaine2019CMP, geft2023finegrained, solovey2015hardness}.
%Under the total distance objective, it is NP-hard \cite{geft2022refined, RATNER1990111}, and when the grid can have holes, it is NP-hard under the makespan and total arrival time objectives \cite{banfi2017holes}.
%When there is only a single escort, the problem becomes $\snpuz$, and it is unknown whether the problem is intractable, including its other variants $\bsnpuz$ and $\psnpuz$.
%However, $\npuz$ is known to be NP-hard \cite{RATNER1990111, DEMAINE201880}.
%demaine2019coordinated
%fekete2022computing
