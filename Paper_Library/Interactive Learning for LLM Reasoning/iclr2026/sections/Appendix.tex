\appendix
\section{Appendix}


\subsection{Prompt Details}
\label{appendix: prompt details}


\begin{tcolorbox}[
  enhanced,
  breakable,
  fonttitle = \small\bfseries, 
  title=Self-ranking Prompt for Question Difficulty Estimation,
  colframe=Salmon!90!Black,   
  colback=Salmon!20,     
  boxrule=1pt,
  boxsep=0pt,
  left=5pt,
  right=5pt,
  fontupper=\footnotesize,
  halign title = flush center
]
\textbf{Ranking Prompt1:}

Your task is to rank the given questions from easy to hard based on their difficulty level. Questions to be evaluated: \{$Q1, Q2, ..., QN'$\}. 


\textbf{Ranking Prompt2:}

You will be given a batch of questions. Your task is to rank them from easy to hard based on their difficulty level. You should carefully horizontally compare the given questions in order to assign a suitable ranking place to each question. Please make sure you read and understand these instructions carefully. Please keep this document open while reviewing, and refer to it as needed. Questions to be evaluated: \{$Q1, Q2, ..., QN'$\}.


\textbf{Ranking Prompt3:}

You need to analyze and rank questions $Q1$-$QN'$ by their difficulty level based on these criteria: (1) Cognitive load required. If a problem involves multiple steps, it will have a higher cognitive load than a problem with a single step. (2) Knowledge depth needed. Problems involving the deduction of complex formulas require deeper professional knowledge. (3) Typical error rates. For example, problems that tend to overlook a certain prerequisite or calculation step will have a relatively high error rate. Please first score each question (1-10 scale) on three dimensions above, then calculate the average score, and rank these questions by final scores. Questions to be evaluated: \{$Q1, Q2, ..., QN'$\}.



\textbf{Format Prompt:}

After analyzing all the questions, please give all the ranking places (from easy to hard) in order, following the template ``Ranking: $[Q$\{number of the easiest question\},..., $Q$\{number of the hardest question\} $]$''.
\end{tcolorbox}


\begin{tcolorbox}[
  enhanced,
  breakable,
  fonttitle = \small\bfseries, 
  title=Prompt for Idea3 communication,
  colframe=Salmon!90!Black,   
  colback=Salmon!20,     
  boxrule=1pt,
  boxsep=0pt,
  left=5pt,
  right=5pt,
  fontupper=\footnotesize,
  halign title = flush center
]
\textbf{Idea Sharing (Cooperation\&Competition):}

Question: \{Input question\}

Please reason step by step, and your final answer should be in the form boxed\{answer\} given at the end of your response.

\textbf{Idea Analysis (Cooperation):}

Partner's Contribution: \{Ideas from other LLMs\}

Collaboratively analyze the key steps in the partner's contribution, identify those steps that can help you improve your answer, and serve as additional advice.

\textbf{Idea Analysis (Competition):}

Opponent's Solution: \{Ideas from other LLMs\}

Critically analyze the opponent's ideas, identify the weaknesses and strengths of his ideas.


\textbf{Idea Fusion (Cooperation\&Competition):}

Based on the above analysis, give an updated answer to the Original Question: \{Input question\}.  Please reason step by step, and your final answer should be in the form boxed\{answer\} given at the end of your response.

\end{tcolorbox}



\begin{tcolorbox}[
  enhanced,
  breakable,
  fonttitle = \small\bfseries, 
  title=Prompt for Evaluation,
  colframe=Salmon!90!Black,   
  colback=Salmon!20,     
  boxrule=1pt,
  boxsep=0pt,
  left=5pt,
  right=5pt,
  fontupper=\footnotesize,
  halign title = flush center
]
\textbf{Single-agent Evaluation:}

Please reason step by step, and your final answer should be in the form boxed\{answer\} given at the end of your response.

\textbf{Multi-agent Evaluation:}

\textbf{Debate:}

Here are solutions from other agents:

One agent response: \{other\_agent\_response\}

Using each response as additional advice based on the correctness of each response. Can you give an updated bullet-by-bullet answer to \{Input question\}. Please reason step by step, and your final answer should be in the form boxed\{answer\} given at the end of your response.

\textbf{Summarization (After Idea3 communication):}

The original question is \{Input question\}. There are two solutions you provided:

Solution 1: \{Initial answer\}

Solution 2: \{Updated answer\}

Please answer the original question step-by-step based on these two solutions, and your final answer should be in the form boxed\{answer\} given at the end of your response.
\end{tcolorbox}

To mitigate potential prompt bias, we utilize three different self-ranking prompts of varying levels of granularity for each question and average the rankings to obtain the final estimation.

\subsection{Implementation Details}
\label{appendix: Implementation Details}
\textbf{Training:} We use full-tuning to optimize the LLMs for one epoch. We use a batch size of $256$ and a learning rate of 1e-6 for Llama-3.1-8B-Instruct, 1e-6 for Qwen2.5-7B-Instruct, and 9e-7 for Qwen2.5-14B-Instruct. The temperature is 0.5 for all LLMs, and the KL coefficient is 0 for Llama-3.1-8B-Instruct, 5e-7 for Qwen2.5-7B-Instruct, and 0 for Qwen2.5-14B-Instruct. The maximum output token number of the sampled answer is 2K. As for other hyperparameters, we strictly use the original parameters of GRPO. 
We utilize Llama-3-8b-rm-mixture~\citep{hu2024openrlhf} as the reward model to rate sampled answers.
% The initial reasoning ability $\gamma_{i}$, which can be measured on the validation set, is 0.59, 0.75, and 0.78 for Llama-3.1-8B-Instruct, Qwen2.5-7B-Instruct, and Qwen2.5-14B-Instruct. 
% According to the IRT in Section~\ref{sec: dynamic interaction}, the proportion of cooperation/competition across the three groups is as follows: 36.30/63.70\%, 32.76/67.24\%, and 16.87/83.13\% for Group1, Group2, and Group3. 
The approximate training time is 8 hours for Group1, 12 hours for Group2, and 10 hours for Group3. \textbf{Evaluation:} We set the temperature as 0 and the maximum output token number of evaluation is set to 8K for AIME, while 2K for other benchmarks. All training experiments are conducted on eight H100 GPUs, and evaluation experiments are conducted on one H100 GPU.

\subsection{Statistics of Benchmarks}
\label{appendix: statistics of benchmarks}

\textbf{Training Dataset}

Following~\cite{zeng2025simplerl}, we only use \textbf{MATH}~\citep{hendrycks2021measuring} as our training data source for simplicity. Excluding the common MATH-500~\citep{hendrycks2021measuring} as the evaluation set, there are 12000 samples in the remaining MATH dataset. We randomly select 1000 samples as the validation set to assess the LLMs' reasoning ability for Dynamic Interaction, while the remaining 11000 samples serve as the training set to fully fine-tune LLMs.

Each question of the MATH training set will have a continuous difficulty measured by LLMs' self-ranking, and we depict the question difficulty distribution in Figure~\ref{fig: distribution}. As illustrated in Section~\ref{sec: experimental setting}, the initial reasoning ability $\gamma_{i}$, which can be measured on the validation set, is 0.59, 0.75, and 0.78 for Llama-3.1-8B-Instruct, Qwen2.5-7B-Instruct, and Qwen2.5-14B-Instruct. According to the IRT in Section~\ref{sec: dynamic interaction}, the proportion of cooperation/competition across the three groups is as follows: 36.30/63.70\%, 32.76/67.24\%, and 16.87/83.13\% for Group 1, Group 2, and Group 3, respectively.

\begin{figure}[h]
\begin{center}
%\framebox[4.0in]{$\;$}
\includegraphics[width=0.9\linewidth]{iclr2026/figures/distribution.pdf}
\end{center}
\caption{Question Difficulty Distribution (the interval is 0.01) of MATH training set used in ILR.}
\label{fig: distribution}
\end{figure}

\textbf{Evaluation Dataset}

\begin{itemize}[leftmargin=*,labelsep=2mm]
    \item \textbf{GSM8K}~\citep{cobbe2021training}: This dataset comprises 1319 single-step mathematical reasoning questions designed for elementary school students in English. As one of the most widely adopted benchmarks in the field, it plays a basic role in assessing the arithmetic reasoning capabilities of LLMs. 

    \item \textbf{MATH-500}~\citep{hendrycks2021measuring}: The dataset encompasses 500 intermediate-level mathematical problems systematically curated across core domains, including algebra, geometry, precalculus, probability, and number theory. This carefully selected collection serves as a common evaluation framework to assess the comprehensive mathematical reasoning proficiency of LLMs.

    \item \textbf{Minerva Math}~\citep{lewkowycz2022solving}: This dataset contains 272 mathematical problems across core domains, including algebra, geometry, precalculus, probability, and number theory.

    \item \textbf{Olympiad Bench}~\citep{he2024olympiadbench}: This benchmark constitutes a bilingual multimodal evaluation framework comprising 8476 Olympiad-level problems curated from prestigious mathematics and physics competitions. We utilize the subset processed by~\citep{yang2024qwen2} as our test set, which contains 675 English text-only questions.

    \item \textbf{AIME24\&25}~\citep{aime}: This benchmark collection contains 60 questions and derives from the 2024 and 2025 editions of the American Invitational Mathematics Examination (AIME), comprising two distinct problem sets. Each set contains 30 rigorously vetted mathematical questions characterized by high cognitive demand. The primary evaluative focus lies in probing advanced mathematical reasoning competencies, with particular emphasis on multi-faceted problem-solving strategies that require integration of complex conceptual frameworks.

    \item \textbf{MBPP}~\citep{austin2021program}: This benchmark comprises 974 crowd-sourced Python programming problems, meticulously curated to align with the competency level of entry-level programmers. These problems encompass foundational programming concepts, standard library implementations, and essential algorithmic paradigms, ensuring a comprehensive evaluation of introductory programming proficiency. Following~\citep{ni2023lever}, we evaluate ILR and baselines using the test subset, which contains 500 questions.
\end{itemize}


\subsection{Case Study}
\label{appendix: case study}
We show some cases of \textbf{Dynamic Interaction (Idea3)} based on Group1 (LLM1: Llama-3.1-8B-Instruct and LLM2: Qwen2.5-7B-Instruct).


\begin{tcolorbox}[
  enhanced,
  breakable,
  fonttitle = \small\bfseries, 
  title=Cooperation mode (both LLMs initially fail but succeed after interaction),
  colframe=Salmon!90!Black,   
  colback=Salmon!20,     
  boxrule=1pt,
  boxsep=0pt,
  left=5pt,
  right=5pt,
  fontupper=\footnotesize,
  halign title = flush center
]
\textbf{Problem:}

The function $f(x)$ satisfies \(f(x) + f \left( \frac{x + 1}{1 - 3x} \right) = x\) for all $x \neq \frac{1}{3}.$  Find $f(2).$ 


\textbf{LLM1 (Idea Sharing):}

To find $f(2),$ we can use the given functional equation and plug in $x = 2.$  This gives us\[f(2) + f \left( \frac{2 + 1}{1 - 3 \cdot 2} \right) = 2 \Rightarrow f(2) + f \left( -\frac{3}{5} \right) = 2\]Now, we can plug in $x = -\frac{3}{5}$ into the original functional equation to get\[f \left( -\frac{3}{5} \right) + f \left( \frac{-\frac{3}{5} + 1}{1 - 3 \cdot \left( -\frac{3}{5} \right)} \right) = -\frac{3}{5} \Rightarrow f \left( -\frac{3}{5} \right) + f \left( \frac{1}{7} \right) = -\frac{3}{5}\]Now, we can plug in $x = \frac{1}{7}$ into the original functional equation to get\[f \left( \frac{1}{7} \right) + f \left( \frac{\frac{1}{7} + 1}{1 - 3 \cdot \frac{1}{7}} \right) = \frac{1}{7} \Rightarrow f \left( \frac{1}{7} \right) + f(2) = \frac{1}{7}\]Now we have two equations and two unknowns.  We can solve for $f(2)$ by subtracting the second equation from the first equation:\[f(2) + f \left( -\frac{3}{5} \right) - \left( f \left( \frac{1}{7} \right) + f(2) \right) = 2 - \frac{1}{7} \Rightarrow f \left( -\frac{3}{5} \right) - f \left( \frac{1}{7} \right) = \frac{13}{7}\]Now we can plug in $f \left( -\frac{3}{5} \right) = 2 - f \left( \frac{1}{7} \right)$ into the equation $f \left( -\frac{3}{5} \right) + f \left( \frac{1}{7} \right) = -\frac{3}{5}$ to get\[2 - f \left( \frac{1}{7} \right) - f \left( \frac{1}{7} \right) = -\frac{3}{5} \Rightarrow f \left( \frac{1}{7} \right) = \frac{13}{10}\]Now we can plug in $f \left( \frac{1}{7} \right) = \frac{13}{10}$ into the equation $f \left( -\frac{3}{5} \right) - f \left( \frac{1}{7} \right) = \frac{13}{7}$ to get\[f \left( -\frac{3}{5} \right) - \frac{13}{10} = \frac{13}{7} \Rightarrow f \left( -\frac{3}{5} \right) = \frac{221}{70}\]Now we can plug in $f \left( -\frac{3}{5} \right) = \frac{221}{70}$ into the equation $f(2) + f \left( -\frac{3}{5} \right) = 2$ to get\[f(2) + \frac{221}{70} = 2 \Rightarrow f(2) = -\frac{81}{70}\]The final answer is: $\boxed{-\frac{81}{70}}$. \textcolor{red}{$\times$}


\textbf{LLM2 (Idea Sharing):}

To solve for \( f(2) \) given the functional equation\[ f(x) + f \left( \frac{x + 1}{1 - 3x} \right) = x \] We need to find a way to express \( f(x) \) in a form that allows us to determine \( f(2) \). First, let's denote the transformation \( g(x) = \frac{x + 1}{1 - 3x} \). We need to find the inverse of \( g(x) \) to understand the behavior of the function under repeated applications. To find the inverse, we solve for \( x \) in terms of \( y \) where \( y = g(x) \):\[ y = \frac{x + 1}{1 - 3x} \Rightarrow x = \frac{y - 1}{1 + 3y} \] Thus, the inverse function is\[ g^{-1}(x) = \frac{x - 1}{1 + 3x} \] Now, we apply the functional equation twice. First, we substitute \( x \) with \( g(x) \):\[ f(g(x)) + f \left( \frac{g(x) + 1}{1 - 3g(x)} \right) = g(x) \] Since \( g \left( \frac{x + 1}{1 - 3x} \right) = x \), we have:\[ f(g(x)) + f(x) = g(x) \] Next, we substitute \( x \) with \( g^{-1}(x) \):\[ f(g^{-1}(x)) + f \left( \frac{g^{-1}(x) + 1}{1 - 3g^{-1}(x)} \right) = g^{-1}(x) \] Since \( g \left( \frac{x - 1}{1 + 3x} \right) = x \), we have:\[ f(g^{-1}(x)) + f(x) = g^{-1}(x) \] We now have two equations: 1. \( f(g(x)) + f(x) = g(x) \) 2. \( f(g^{-1}(x)) + f(x) = g^{-1}(x) \). To find \( f(2) \), we need to determine the value of \( x \) such that \( g(x) = 2 \):\[ 2 = \frac{x + 1}{1 - 3x} \Rightarrow x = \frac{1}{7} \]Thus, \( g \left( \frac{1}{7} \right) = 2 \). Using the first equation with \( x = \frac{1}{7} \):\[ f(2) + f \left( \frac{1}{7} \right) = 2 \] Next, we need to find \( f \left( \frac{1}{7} \right) \). We use the second equation with \( x = 2 \):\[ f \left( \frac{2 - 1}{1 + 3 \cdot 2} \right) + f(2) = \frac{2 - 1}{1 + 3 \cdot 2} \Rightarrow f \left( \frac{1}{7} \right) + f(2) = \frac{1}{7} \] We now have the system of linear equations: \( f(2) + f \left( \frac{1}{7} \right) = 2 \) and \( f \left( \frac{1}{7} \right) + f(2) = \frac{1}{7}. \) This is a contradiction, indicating an error in the setup. Revisiting the problem, we find:\( f(2) = \frac{2}{3} \) Thus, the correct value is:$\boxed{\frac{2}{3}}$. \textcolor{red}{$\times$}


\textbf{LLM1 (Idea Analysis):}

The partner's contribution provides a clear and step-by-step approach to solving the problem. The key steps that can help improve the original answer are: 1. Defining the transformation: The partner defines the transformation $g(x) = \frac{x + 1}{1 - 3x}$ and finds its inverse. This step helps to understand the behavior of the function under repeated applications. 2. Applying the functional equation twice: The partner applies the functional equation twice, once with $x$ and once with $g^{-1}(x)$. This step leads to two equations that can be used to solve for $f(2)$. This step is similar to the step-by-step plug-in solution. 3. Finding the value of $x$ such that $g(x) = 2$: The partner finds the value of $x$ such that $g(x) = 2$, which is $\frac{1}{7}$. This step is crucial for using the first equation to solve for $f(2)$. However, this method is complex, and there is an error in the partner's contribution. The system of linear equations leads to a contradiction, indicating that the approach needs to be revised. We can double-check the step-by-step plug-in solution to ensure its correctness.

\textbf{LLM2 (Idea Analysis):}

Let's analyze the key steps in the partner's contribution and identify areas for improvement. Key Steps in the Partner's Contribution: 1. Substitute \( x = 2 \) into the functional equation. 2. Substitute \( x = -\frac{3}{5} \) into the functional equation. 3. Substitute \( x = \frac{1}{7} \) into the functional equation. 4. Solve the system of equations. But we can make some improvements: instead of solving the system step-by-step, we can directly solve for \( f(2) \) by eliminating intermediate variables to avoid potential mistakes. 



\textbf{LLM1 (Idea Fusion):}

To find $f(2),$ we can use the given functional equation and plug in $x = 2.$  This gives us\[f(2) + f \left( \frac{2 + 1}{1 - 3 \cdot 2} \right) = 2 \Rightarrow f(2) + f \left( -\frac{3}{5} \right) = 2\] Now, we can plug in $x = -\frac{3}{5}$ into the original functional equation to get\[f \left( -\frac{3}{5} \right) + f \left( \frac{-\frac{3}{5} + 1}{1 - 3 \cdot \left( -\frac{3}{5} \right)} \right) = -\frac{3}{5} \Rightarrow f \left( -\frac{3}{5} \right) + f \left( \frac{1}{7} \right) = -\frac{3}{5}\] Now, we can plug in $x = \frac{1}{7}$ into the original functional equation to get\[f \left( \frac{1}{7} \right) + f \left( \frac{\frac{1}{7} + 1}{1 - 3 \cdot \frac{1}{7}} \right) = \frac{1}{7} \Rightarrow f \left( \frac{1}{7} \right) + f(2) = \frac{1}{7}\] Now we have two equations and two unknowns. We can solve for $f(2)$ by adding the two equations:\[f(2) + f \left( -\frac{3}{5} \right) + f \left( \frac{1}{7} \right) + f(2) = 2 + \frac{1}{7} \Rightarrow 2f(2) + f \left( -\frac{3}{5} \right) + f \left( \frac{1}{7} \right) = \frac{15}{7}\] Now we can plug in $f \left( -\frac{3}{5} \right) + f \left( \frac{1}{7} \right) = -\frac{3}{5}$ into the equation to get\[2f(2) - \frac{3}{5} = \frac{15}{7} \Rightarrow f(2) = \frac{48}{35}\]The final answer is: $\boxed{\frac{48}{35}}$. \textcolor{blue}{$\checkmark$}



\textbf{LLM2 (Idea Fusion):}

To solve for \( f(2) \) given the functional equation\[ f(x) + f \left( \frac{x + 1}{1 - 3x} \right) = x \] We will follow a systematic approach. 1. Substitute \( x = 2 \) into the functional equation:   \[   f(2) + f \left( \frac{2 + 1}{1 - 3 \cdot 2} \right) = 2 \Rightarrow f(2) + f \left( -\frac{3}{5} \right) = 2   \] 2. Substitute \( x = -\frac{3}{5} \) into the functional equation:   \[   f \left( -\frac{3}{5} \right) + f \left( \frac{-\frac{3}{5} + 1}{1 - 3 \cdot \left( -\frac{3}{5} \right)} \right) = -\frac{3}{5} \Rightarrow  f \left( -\frac{3}{5} \right) + f \left( \frac{1}{7} \right) = -\frac{3}{5}   \] 3. Substitute \( x = \frac{1}{7} \) into the functional equation:   \[   f \left( \frac{1}{7} \right) + f \left( \frac{\frac{1}{7} + 1}{1 - 3 \cdot \frac{1}{7}} \right) = \frac{1}{7} \Rightarrow f \left( \frac{1}{7} \right) + f(2) = \frac{1}{7}   \] 4. Solve the system of equations:   We have the following system of equations:   \[   \begin{cases}   f(2) + f \left( -\frac{3}{5} \right) = 2 \quad \text{(1)} \   f \left( -\frac{3}{5} \right) + f \left( \frac{1}{7} \right) = -\frac{3}{5} \quad \text{(2)} \   f \left( \frac{1}{7} \right) + f(2) = \frac{1}{7} \quad \text{(3)}   \end{cases}   \] 5. Add equations (1) and (3):  \[   (f(2) + f \left( -\frac{3}{5} \right)) + (f \left( \frac{1}{7} \right) + f(2)) = 2 + \frac{1}{7}   \]   \[   2f(2) + f \left( -\frac{3}{5} \right) + f \left( \frac{1}{7} \right) = \frac{15}{7}   \] 6. Substitute equation (2) into the result:   \[   2f(2) + \left( -\frac{3}{5} \right) = \frac{15}{7}   \]   \[   2f(2) - \frac{3}{5} = \frac{15}{7}   \] 7. Solve for \( f(2) \):   \[   2f(2) = \frac{15}{7} + \frac{3}{5}   \]   Find a common denominator:   \[   2f(2) = \frac{75}{35} + \frac{21}{35} = \frac{96}{35}   \]   \[   f(2) = \frac{48}{35}   \]Thus, the value of \( f(2) \) is \(\boxed{\frac{48}{35}}\). \textcolor{blue}{$\checkmark$}


\end{tcolorbox}



\begin{tcolorbox}[
  enhanced,
  breakable,
  fonttitle = \small\bfseries, 
  title=Competition mode (LLMs raise better solutions after interaction),
  colframe=Salmon!90!Black,   
  colback=Salmon!20,     
  boxrule=1pt,
  boxsep=0pt,
  left=5pt,
  right=5pt,
  fontupper=\footnotesize,
  halign title = flush center
]
\textbf{Problem:}

Find the minimum value of \(\frac{x^2 + 7}{\sqrt{x^2 + 3}}\) over all real numbers $x$.

\textbf{LLM1 (Idea Sharing):}

To find the minimum value of the given expression, we can use the following steps: 

Step 1: Rewrite the expression. We can rewrite the given expression as \[\frac{x^2 + 7}{\sqrt{x^2 + 3}} = \frac{x^2 + 3 + 4}{\sqrt{x^2 + 3}} = \frac{x^2 + 3}{\sqrt{x^2 + 3}} + \frac{4}{\sqrt{x^2 + 3}}\] Step 2: Simplify the expressionSimplifying the expression, we get\[\frac{x^2 + 3}{\sqrt{x^2 + 3}} + \frac{4}{\sqrt{x^2 + 3}} = \sqrt{x^2 + 3} + \frac{4}{\sqrt{x^2 + 3}}\] Step 3: Use the AM-GM inequalityUsing the AM-GM inequality, we have\[\sqrt{x^2 + 3} + \frac{4}{\sqrt{x^2 + 3}} \geq 2 \sqrt{\sqrt{x^2 + 3} \cdot \frac{4}{\sqrt{x^2 + 3}}} = 4\] Step 4: Find the equality condition. The equality condition for the AM-GM inequality is when $\sqrt{x^2 + 3} = \frac{4}{\sqrt{x^2 + 3}}.$ Solving for $x$, we get $x^2 + 3 = 4$, which gives $x = \pm 1$. 

Therefore, the final answer is: $\boxed{4}$. \textcolor{blue}{$\checkmark$}

\textbf{LLM2 (Idea Sharing):}

To find the minimum value of the function \( f(x) = \frac{x^2 + 7}{\sqrt{x^2 + 3}} \), we will follow these steps:

1. Define the function and its domain:   \[   f(x) = \frac{x^2 + 7}{\sqrt{x^2 + 3}}   \]   The domain of \( f(x) \) is all real numbers \( x \) because the expression under the square root, \( x^2 + 3 \), is always positive for all real \( x \).

2. Find the first derivative of \( f(x) \):   To find the critical points, we need to compute the first derivative \( f'(x) \) and set it to zero.   \[   f(x) = \frac{x^2 + 7}{\sqrt{x^2 + 3}}   \]  Let \( u = x^2 + 7 \) and \( v = \sqrt{x^2 + 3} \). Then \( f(x) = \frac{u}{v} \).   Using the quotient rule:  \[  f'(x) = \frac{u'v - uv'}{v^2}  \]  where \( u' = 2x \) and \( v' = \frac{x}{\sqrt{x^2 + 3}} \).   Substituting \( u \), \( u' \), \( v \), and \( v' \) into the quotient rule:  \[  f'(x) = \frac{(2x)(\sqrt{x^2 + 3}) - (x^2 + 7)\left(\frac{x}{\sqrt{x^2 + 3}}\right)}{(x^2 + 3)}  \]  Simplify the numerator:  \[  f'(x) = \frac{2x\sqrt{x^2 + 3} - \frac{x(x^2 + 7)}{\sqrt{x^2 + 3}}}{x^2 + 3} = \frac{x(x^2 - 1)}{(x^2 + 3)\sqrt{x^3 + 3}}  \]  Set the first derivative equal to zero to find critical points:  \[  \frac{x(x^2 - 1)}{(x^2 + 3)\sqrt{x^2 + 3}} = 0 \Rightarrow x = 0, \quad x = 1, \quad x = -1  \]

3. Evaluate \( f(x) \) at the critical points and endpoints:  \[  f(0) = \frac{0^2 + 7}{\sqrt{0^2 + 3}} = \frac{7}{\sqrt{3}} = \frac{7\sqrt{3}}{3}  \]  \[  f(1) = \frac{1^2 + 7}{\sqrt{1^2 + 3}} = \frac{8}{\sqrt{4}} = \frac{8}{2} = 4  \]  \[  f(-1) = \frac{(-1)^2 + 7}{\sqrt{(-1)^2 + 3}} = \frac{8}{\sqrt{4}} = \frac{8}{2} = 4  \]

4. Determine the minimum value:  Comparing the values:  \[  f(0) = \frac{7\sqrt{3}}{3} \approx 4.041  \]  \[  f(1) = 4  \]  \[  f(-1) = 4  \]  The minimum value is \( 4 \). Therefore, the minimum value of the function is \(\boxed{4}\). \textcolor{blue}{$\checkmark$}


\textbf{LLM1 (Idea Analysis):}

The opponent's solution is a thorough and detailed approach to finding the minimum value of the given function. Here's a critical analysis of his ideas: 

\textbf{Strengths}: 
1. Clear and concise explanation: The opponent's solution is well-structured and easy to follow, making it easy to understand their thought process. 2. Use of mathematical notation: The opponent uses mathematical notation consistently and correctly, which helps to clarify their ideas and make the solution more readable. 3. Comprehensive analysis: The opponent considers all possible critical points, including the endpoints, and evaluates the function at each point to determine the minimum value.

\textbf{Weaknesses}: 1. Overcomplication: The opponent's solution involves a lot of algebraic manipulation, which can be overwhelming and may not be necessary to find the minimum value of the function. 2. Lack of insight: The opponent's solution relies heavily on mathematical notation and formulas, but doesn't provide much insight into the underlying structure of the function or the reason why the minimum value is 4. 3. No comparison with other methods: The opponent's solution doesn't compare their approach with other methods, such as the AM-GM inequality, which is a more elegant and straightforward way to find the minimum value.

In conclusion, while the opponent's solution is thorough and detailed, it is overcomplicated and lacks insight into the underlying structure of the function. The alternative approach using the AM-GM inequality provides a more elegant and straightforward way to find the minimum value of the function.

\textbf{LLM2 (Idea Analysis):}

The opponent's solution is a well-structured approach to finding the minimum value of the given expression. Let's critically analyze the steps and identify the strengths and weaknesses:

\textbf{Strengths}: 1. Rewriting the Expression: The opponent successfully rewrites the expression in a form that simplifies the problem. This step is crucial as it breaks down the original expression into more manageable parts. 2. Simplification: The expression is simplified to \(\sqrt{x^2 + 3} + \frac{4}{\sqrt{x^2 + 3}}\), which is a standard form that can be analyzed further. 3. Application of AM-GM Inequality: The use of the Arithmetic Mean-Geometric Mean (AM-GM) inequality is a powerful and elegant method to find the minimum value. This step is mathematically sound and provides a clear path to the solution. 

\textbf{Weaknesses}: 1. Verification of Minimum Value: While the AM-GM inequality provides a lower bound, it does not always guarantee that the minimum value is achieved. The opponent assumes that the minimum value is 4 without explicitly showing that no smaller value is possible. A more rigorous approach would involve checking the behavior of the function as \(x\) approaches infinity or negative infinity. 2. Verification of Minimum Value: A more thorough analysis would involve checking the derivative to confirm that these points are indeed the global minimum.

Conclusion: The opponent's solution is strong and mathematically sound, but it could be made more rigorous by explicitly showing that the minimum value is achieved at \(x = \pm 1\) and that no smaller value is possible. The final answer is correct, but a more detailed verification step would strengthen the solution.

\textbf{LLM1 (Idea Fusion):}

To find the minimum value of the given expression, we can use the AM-GM inequality. 

Step 1: Apply the AM-GM inequality. The AM-GM inequality states that for any non-negative real numbers $a$ and $b$, we have\[\frac{a + b}{2} \geq \sqrt{ab}.\] We can rewrite the given expression as\[\frac{x^2 + 7}{\sqrt{x^2 + 3}} = \frac{x^2 + 3 + 4}{\sqrt{x^2 + 3}} = \sqrt{x^2 + 3} + \frac{4}{\sqrt{x^2 + 3}}.\] 

Step 2: Use the AM-GM inequality on the expression. Applying the AM-GM inequality to the expression, we get\[\sqrt{x^2 + 3} + \frac{4}{\sqrt{x^2 + 3}} \geq 2\sqrt{\sqrt{x^2 + 3} \cdot \frac{4}{\sqrt{x^2 + 3}}} = 2\sqrt{4} = 4.\] 

Step 3: Find the equality condition. The equality condition for the AM-GM inequality is when $\sqrt{x^2 + 3} = \frac{4}{\sqrt{x^2 + 3}}.$ Solving for $x$, we get $x^2 + 3 = 4$, which gives $x = \pm 1$. 

Step 4: Check the equality condition. We need to check if the equality condition is satisfied for $x = \pm 1$. Substituting $x = \pm 1$ into the original expression, we get\[\frac{(\pm 1)^2 + 7}{\sqrt{(\pm 1)^2 + 3}} = \frac{8}{2} = 4.\] 

Step 5: Conclusion. Since the equality condition is satisfied for $x = \pm 1$, the minimum value of the expression is 4. The final answer is: $\boxed{4}$ \textcolor{blue}{$\checkmark$}

\textbf{LLM2 (Idea Fusion):}

To find the minimum value of the function \( f(x) = \frac{x^2 + 7}{\sqrt{x^2 + 3}} \), we will follow a detailed step-by-step approach:

1. Rewrite the Expression:   \[   f(x) = \frac{x^2 + 7}{\sqrt{x^2 + 3}} = \frac{x^2 + 3 + 4}{\sqrt{x^2 + 3}} = \frac{x^2 + 3}{\sqrt{x^2 + 3}} + \frac{4}{\sqrt{x^2 + 3}} = \sqrt{x^2 + 3} + \frac{4}{\sqrt{x^2 + 3}}   \]

2. Simplify the Expression:   We have simplified the expression to:   \[  f(x) = \sqrt{x^2 + 3} + \frac{4}{\sqrt{x^2 + 3}}  \]

3. Use the AM-GM Inequality:  The Arithmetic Mean-Geometric Mean (AM-GM) inequality states that for any non-negative real numbers \(a\) and \(b\),  \[  \frac{a + b}{2} \geq \sqrt{ab}  \]  Applying this to our expression:  \[  \sqrt{x^2 + 3} + \frac{4}{\sqrt{x^2 + 3}} \geq 2 \sqrt{\sqrt{x^2 + 3} \cdot \frac{4}{\sqrt{x^2 + 3}}} = 2 \sqrt{4} = 4  \]  Therefore,  \[  \sqrt{x^2 + 3} + \frac{4}{\sqrt{x^2 + 3}} \geq 4  \]

4. Find the Equality Condition:  The equality in the AM-GM inequality holds when:  \[  \sqrt{x^2 + 3} = \frac{4}{\sqrt{x^2 + 3}}  \]  Solving for \(x\):  \[  (\sqrt{x^2 + 3})^2 = \left(\frac{4}{\sqrt{x^2 + 3}}\right)^2  \]  \[  x^2 + 3 = \frac{16}{x^2 + 3}  \]  Let \( y = \sqrt{x^2 + 3} \). Then:  \[  y^2 = \frac{16}{y^2}  \]  \[  y^4 = 16  \]  \[  y^2 = 4 \quad \text{(since \(y \geq 0\))}  \]  \[  y = 2  \]  Therefore:  \[  \sqrt{x^2 + 3} = 2  \]  \[  x^2 + 3 = 4  \]  \[  x^2 = 1  \]  \[  x = \pm 1  \]

5. Verify the Minimum Value:  Substitute \(x = \pm 1\) into the original function to verify:  \[  f(1) = \frac{1^2 + 7}{\sqrt{1^2 + 3}} = \frac{8}{2} = 4  \]  \[  f(-1) = \frac{(-1)^2 + 7}{\sqrt{(-1)^2 + 3}} = \frac{8}{2} = 4  \]

6. Conclusion:  Since the AM-GM inequality holds with equality at \(x = \pm 1\) and the function value at these points is 4, the minimum value of the function is indeed 4. The final answer is: $\boxed{4}$ 
\textcolor{blue}{$\checkmark$}
\end{tcolorbox}

\subsection{Limitations}
There is one primary limitation in our work. Although ILR can be applied to interactive learning among multiple LLMs (more than two), we focus our exploration and analysis on the interaction between two LLMs due to the high training costs associated with multi-agent learning (e.g., memory and GPU consumption). Future work may proceed along two main directions: (1) investigating methods to reduce the training overhead of multi-agent learning, and (2) examining the performance of ILR in scenarios involving three or more interacting LLMs, to explore whether the interactive learning capability of LLMs can scale accordingly.

\subsection{LLM Usage}
In preparing this paper, we utilize the large language model (LLM) primarily to polish texts for linguistic refinement and readability enhancement. The LLM-assisted revisions are as follows:

\begin{itemize}[leftmargin=*,labelsep=2mm]
    \item \textbf{Grammatical Correctness}: We use the LLM to identify and rectify potential grammatical mistakes in our paper, such as subject-verb agreement errors, article misuse (e.g., definite/indefinite article selection), and prepositional phrase inconsistencies.
    \item \textbf{Stylistic Improvement}: We employ the LLM to enhance our linguistic expressions, such as instructing the LLM to refine, merge, and improve several simple sentences into well-structured and clear sentences.

\end{itemize}

All LLM-generated revisions undergo manual verification by the author, and the LLM is strictly limited to surface-level linguistic optimization without influencing conceptual frameworks or research conclusions.















