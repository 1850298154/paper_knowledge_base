\documentclass[11pt]{article}

\usepackage[top=1.0in, bottom=1.0in, left=0.95in, right=0.95in]{geometry}

\setlength{\parindent}{0pt}
\setlength{\parskip}{5pt}

\usepackage{nicefrac}
\usepackage[utf8]{inputenc} % allow utf-8 input
\usepackage[T1]{fontenc}    % use 8-bit T1 fonts
\usepackage[hidelinks]{hyperref}       % hyperlinks
\usepackage{url}            % simple URL typesetting
\usepackage{booktabs}       % professional-quality tables
\usepackage{amsfonts}       % blackboard math symbols
\usepackage{microtype}      % microtypography
\usepackage{xcolor}         % colors
%%%%%%%%%%%% My added commands %%%%%%%%%%%%
\usepackage{amsmath}
\let\openbox\undefined
\usepackage{amsthm}
\usepackage{graphicx}
\usepackage[font=small]{caption}
\usepackage{subcaption}
\usepackage{bm}
\usepackage[ruled,noend]{algorithm2e}
\usepackage{algpseudocode} 
\usepackage{varwidth}
\usepackage{placeins}
\usepackage{mathtools}      % For enhanced math tools
\usepackage{booktabs}       % For tables
\usepackage{verbatim}
\usepackage{enumitem}
\setlist{itemsep=3pt}
\usepackage{soul}
\usepackage{scalerel}
\usepackage{cleveref}

% \usepackage[numbers,sort&compress]{natbib}
\usepackage[numbers,sort]{natbib}
\usepackage{etoolbox}
\setlength{\bibsep}{5pt}

\DeclareMathOperator*{\argmin}{argmin} 
\DeclareMathOperator*{\argmax}{argmax} 
\DeclareMathOperator{\E}{\mathbb{E}}
\newcommand{\lmb}{\eta}
\newcommand{\lmbt}{\eta_t}
\newcommand{\minProb}{\Delta}
\newcommand{\bfpar}[1]{%
    \vspace{3pt}%
    \noindent%
    \textbf{#1:}%
}
\newcommand{\empar}[1]{%
    \vspace{3pt}%
    \noindent%
    \emph{#1:}%
}

\newcommand{\ind}[1]{\mathbf{1}_{\{#1\}}}

%% Comments %%
\usepackage{xcolor}
\definecolor{darkgreen}{rgb}{0,0.5,0}

% \usepackage{color-edits}
\usepackage[suppress]{color-edits}
\addauthor{yk}{darkgreen}
\addauthor{de}{purple}
\addauthor{et}{blue}
%%%%%%%%%%%%%%%%%%%%%%%%%%

%%%%%%%%%%%%%%%%%%%%%%%%%%%%%%%%%%
%%%%% Theorem formats %%%%%%%%%%%%
\newcommand{\ignore}[1]{}
\newtheorem{theorem}{Theorem}[section]
\newtheorem*{theorem*}{Theorem} %unnumbered theorem
\newtheorem{lemma}[theorem]{Lemma}
% \newtheorem{cor}{Corollary}
\newtheorem{corollary}[theorem]{Corollary}
\newtheorem{proposition}[theorem]{Proposition}
\newtheorem{observation}[theorem]{Observation}
\newtheorem{conjecture}[theorem]{Conjecture}
\newtheorem{claim}{Claim}
\newtheorem{definition}[theorem]{Definition}
\newtheorem{assumption}[theorem]{Assumption}
\newtheorem{example}[theorem]{Example}
% \newtheorem*{remark}{Remark}
\newtheorem*{example*}{Example}

%%%%%%%%%%%%%%%%%%%%%%%%%%%%%%%%%%%%%%%%%%%

\title{\vspace{-8pt}
Markets with Heterogeneous Agents:\\Dynamics and Survival of Bayesian vs. No-Regret Learners\thanks{This is an extended version of a paper originally published at EC 2025; DOI: \href{https://doi.org/10.1145/3736252.3742595}{10.1145/3736252.3742595}.} }


\author{%
  David Easley \\
  Cornell University \\
  \texttt{dae3@cornell.edu} 
  \and
  Yoav Kolumbus \\
  Cornell University \\
  \texttt{yoav.kolumbus@cornell.edu} 
  \and
  \'Eva Tardos \\
  Cornell University \\
  \texttt{eva.tardos@cornell.edu} \\
}
\date{}

\begin{document}
% \sloppy
\maketitle

% Abstract. Note that this must come before \maketitle.
\begin{abstract}

\input{abstract-MS}

\end{abstract}



\maketitle


%%%%%%%%%%%%%%%%%%%%%%%%%%%%%%%%%%%%%%%%%%%%%%%%%%%%%%%%%%%%%%%%%%%%%%

% Text of your paper here
%%%%%%%%%%%%%%%%%%%%%%%%%%%
\section{Introduction}\label{sec:intro}

% \section{Introduction}

One of the most fundamental problems in combinatorial optimization is the traveling salesperson problem (TSP), formalized as early as 1832 (c.f. \cite[Ch 1]{ABCC07}).
In an instance of  TSP we are given a set of $n$ cities $V$ along with their pairwise symmetric distances, $c:V\times V \to\R_{\geq 0}$. The goal is to find a Hamiltonian cycle of minimum cost. In the metric TSP problem, which we study here, the distances satisfy the triangle inequality. Therefore, the problem is equivalent to finding a closed Eulerian connected walk of minimum cost.%\footnote{Given such an Eulerian cycle, we can use the triangle inequality to shortcut vertices visited more than once to get a Hamiltonian cycle.}

It is NP-hard to approximate TSP within a factor of $\frac{123}{122}$ \cite{KLS15}.  An algorithm of Christofides-Serdyukov~\cite{Chr76,Ser78} from four decades ago gives a $\frac32$-approximation for TSP.
Over the years there have been numerous attempts to improve the Christofides-Serdyukov algorithm and exciting progress has been made for various special cases of metric TSP, e.g., \cite{OSS11,MS11,Muc12,SV12,HNR21, KKO20, HN19, GLLM21}.
 Recently, ~\cite{KKO21} gave the first improvement for the general case by demonstrating that the so-called ``max entropy" algorithm of \cite{OSS11} gives a randomized $\frac{3}{2}-\epsilon$ approximation for some $\epsilon > 10^{-36}$.% (see \cite{VS20} for a historical note about TSP)

%After a long line of work %~\cite{Wol80,SW90,BP91,Goe95,CV00,GLS05,BM10,BC11,SWV12, HNR17,HN19, KKO20a} 
	%the best known approximation algorithm for the general case of the problem is $\frac{3}{2}-\epsilon$ for some $\epsilon > 10^{-36}$ due to ~\cite{KKO21}, a result that built upon the work of the third author, Saberi, and Singh ~\cite{OSS11}. 
	The method introduced in \cite{KKO21} exploits the optimum solution to the following linear programming relaxation of metric TSP studied by \cite{DFJ59,HK70,BG93}, also known as the subtour elimination LP:
\begin{equation}\label{eq:tsplp}
\begin{aligned}
	\min \quad& \sum_{u,v} x_{\{u,v\}} c(u,v)& \\
	\text{s.t.,} \quad &  \sum_{u} x_{\{u,v\}} = 2&\forall v\in V,\\
	& \sum_{u\in S, v\notin S} x_{\{u,v\}}\geq 2,&\forall S \subsetneq V, S\not= \emptyset\\
	& x_{\{u,v\}}\geq 0 &\forall u,v\in V.
\end{aligned}	
\end{equation} 
	
	 However, ~\cite{KKO21} did not show that the integrality gap of the subtour elimination polytope is bounded below $\frac{3}{2}$, and therefore did not make progress towards the ``4/3 conjecture" which posits that the integrality gap of LP \eqref{eq:tsplp} is $\frac{4}{3}$. In this work we remedy this discrepancy by proving the following theorem, improving upon the bound of $\frac{3}{2}$ from Wolsey~\cite{Wol80} in 1980:

\begin{theorem}\label{thm:main}
	Let $x$ be a solution to LP \eqref{eq:tsplp} for a TSP instance. For some absolute constant $\epsilon > 10^{-36}$, the \hyperlink{tar:alg}{max entropy algorithm} outputs a TSP tour with expected cost at most $\frac{3}{2}-\epsilon$ times the cost of $x$. Therefore the integrality gap of the subtour elimination LP is at most $\frac{3}{2} - \epsilon$. 
\end{theorem} 

To prove \cref{thm:main}, we amend Section 4 of \cite{KKO21} but keep the remainder of the analysis essentially the same. Unlike \cite{KKO21}, this argument now preserves the integrality gap by avoiding the use of the optimum solution in bounding the cost of the matching. See \cref{sec:overview} for a discussion of our new approach.
%We note that the analysis in this paper is not specialized to the max entropy algorithm (although we rely on many results from \cite{KKO21} to obtain \cref{thm:main} itself); instead, it is valid for any algorithm which samples a spanning tree from the support of a solution to LP \eqref{eq:tsplp} and then adds the minimum cost matching on the odd degree vertices of the tree.  
%Instead, we use the polygon representation of near minimum cuts \cite{Ben95,BG08} to bound  the cost of the matching (see the following section for an overview of our new findings). %An added benefit of avoiding the use of OPT in the analysis is  %We remark this makes the analysis constructive 
%We remark that this allows future analyses to explicitly compute and possibly utilize the relevant laminar family of near minimum cuts (whereas previously one needed to know OPT to find the laminar family used in the analysis in \cite{KKO21}).
%In particular, we show that to get a bound better than $\frac{3}{2}$ for this class of algorithm it is (essentially) sufficient to handle the case in which the near minimum cuts of $x$ are a laminar family.

\subsection{Other Consequences}
\paragraph{Path TSP} In recent exciting work, Traub, Vygen, Zenklusen \cite{TVZ20} showed that an $\alpha$-approximation algorithm for metric TSP can be used as a black box to get a $\alpha(1+\eps)$ approximation algorithm for Path TSP. This together with \cite{KKO21} implies that there is a $3/2-\eps$ approximation algorithm for Path TSP (for $\eps>10^{-36}$). On the other hand, it is a folklore result that the integrality gap of the natural LP relaxation of Path TSP is at least $3/2$.  Therefore, a consequence of the above theorem is that although the best possible approximation factors of the two problem are the same (up to polynomial reductions), the natural LP relaxation of metric TSP has a strictly smaller integrality gap.


\paragraph{2-ECSM} In the 2-edge-connected multi-subgraph problem, or 2-ECSM for short, we are given a weighted graph $G$ and we want to find a minimum cost 2-edge-connected spanning subgraph, where an edge can be chosen multiple times.
The classical Christofides-Serdyukov algorithm gives a 3/2-approximation for 2-ECSM and despite significant attempts \cite{CR98,BFS16,SV14,BCCGISW20} improved algorithms were designed only for special cases of the problem.
Since in \cite{BG93} it is shown that LP \eqref{eq:tsplp} is a valid relaxation for 2-ECSM, we obtain:

\begin{corollary}	
For some absolute constant $\epsilon > 10^{-36}$ the \hyperlink{tar:alg}{max entropy algorithm} is a randomized $\frac{3}{2}-\epsilon$ approximation for the 2-edge-connected multi-subgraph problem.
\end{corollary}
%Beyond these theorems, we believe the analysis in this paper will open new avenues to improve the arguments in ~\cite{KKO21}. The analysis in that work is by nature non-constructive because it uses information about the optimal solution. Here we remove this weakness and could in principle construct the proposed fractional matching in polynomial time. Although of course this has no practical benefit since the algorithm always finds the minimum cost matching, this may allow future works to manipulate the algorithm to better serve the analysis.

%We analyze the max-entropy rounding algorithm introduced in \cite{OSS11} and slightly modified in \cite{KKO20, KKO21}. 

%In other words, we design a feasible vector for the $O$-join polytope to ``satisfy'' all near min cuts ``crossed on both  sides'' 


%Whereas Section 4 of ~\cite{KKO21} only deals with the near minimum cuts of $x$ (where $x$ is a solution to LP \eqref{eq:tsplp}) which lie along the optimal Hamiltonian cycle, we deal with all near minimum cuts of $x$ using the so-called polygon representation of near minimum cuts ~\cite{Ben97,BG08}. %The results give new intuition for the structure of cuts that are within $\frac{6}{5}$ or less of the edge connectivity of the graph.

 %: we show that we can incur a cost of $O(\eta^2) \cdot c(x)$ to ensure that the set of cuts with $x(\delta(S)) \le 2+\eta$ is a laminar family.


\subsection{New techniques and contributions}\label{sub:newtechniques}

This paper can be seen as a case study on how to reason about and deal with {\em near} minimum cuts. One can deduce from the classical cactus representation of a graph $G$ \cite{DKL76} (i) the structure of {\em all} min cuts of $G$ and (ii) the structure of the edges of $G$ in the sense that every edge $\{u,v\}$ maps to a unique {\em path} in the cactus between the images of $u$ and $v$. Furthermore, such a path intersects every cycle of the cactus on at most one cactus edge. The theory has found many application from designing fast algorithms
\cite{Kar00,KP09} to the analysis of approximation algorithms for TSP \cite{KKO20} and connectivity augmentation \cite{BGJ20,CTZ21}.

Two decades later, the theory of min cuts was extended to near min cuts in works of Bencz\'ur and Goemans \cite{Ben95, BG08} where they introduced the polygon representation which represents all cuts of a graph with at most $\frac{6}{5}k$ edges, where $k$ is its edge connectivity. Although these works completely classify the structure of all near min cuts of a given graph $G$, they do not characterize the structure of the \textit{edges} of $G$ with respect to these cuts, which can be important in applications (for example, in many of the recent applications of min cuts,
 one also needs to exploit the structure of the edges in relation to the cactus).
The structure on the edges turns out to be highly relevant in this work as well, and as a byproduct of our analysis we make progress towards classifying the way in which the edges of $G$ relate to the structure of the polygon representation.
 
 % and (to some extent) a classification of the set of edges of $G$ with respect to the polygon representation of Bencz\'ur and Goemans.
 
  %i
 %s to give a better understanding of the structure of edges of $G$ with respect to its near min cuts.

  %One can partition the edges of $G$ into sets $F_1\dots,F_m$ such that the set of edges in every min cut $(S,\overline{S})$ of $G$ is the union of edges in a pair $F_i,F_j$ for $i\ neq j$.
%\Nathan{Shayan can add something} For example...

For motivation, consider a generic family of network design problems in which we want to construct a network such that every pair $u,v$ of vertices has connectivity at least $c_{u,v}$. A natural approach is to write an LP relaxation to find a (minimum cost) vector $x: E \to \R_{\ge 0}$ such that for every cut $S$ separating $u$ and $v$, $x(\delta(S))\geq c_{u,v}$. We can round this LP using independent rounding or a dependent rounding scheme such as sampling from max entropy distributions. Using classical concentration bounds one can show that if $x(\delta(S))\gg c_{u,v}$ then with high probability the rounded solution has at least $c_{u,v}$ edges across this cut. So the main challenge is to ``fix'' near tight cuts, i.e., cuts where $x(\delta(S))\approx c_{u,v}$.  For an explicit instantiation of this scheme see \cite{KKOZ22}. A better understanding of the global structure of the family of near tight cuts has the potential to significantly simplify or even improve the approximation factor of such rounding algorithms. A classical technique to design algorithms for such network design problems is to apply uncrossing to extreme point solutions of the LP. One can view our contribution as an approximate uncrossing technique that deals with all near tight cuts (instead of just tight cuts) as we explain next.
%Next, we explain how our main theorem can be used to give global structure for near tight cuts in the case that $c_{u,v}=2$ for all $u,v$ and we contrast it with the classical uncrossing technique which only deals with tight/min cuts. 


\paragraph{An Approximate Uncrossing Technique.} A fundamental technique in the field of approximation algorithms is the uncrossing technique\footnote{See e.g. \cite{LRS11} for a number of applications of this technique.} of Jain \cite{Jai01}. Given a graph $G=(V,E)$,  a weight vector $x:E\to\R_{\geq 0}$, and a  function $f:V\to\R$, suppose that $x(\delta(S))\geq f(S)$ for all $S\subseteq V$. Let $\cN$ be the family of sets $S$ such that $x(\delta(S)) = f(S)$, i.e., the family of {\em tight} sets with respect to $f$. The uncrossing technique says that if $f$ is (weakly) supermodular then we can refine $\cN$ to a laminar family of sets, $\cH$, such that if all sets of $\cH$ are tight, then all sets of $\cN$ are tight as well. For a concrete example, suppose $f$ is a constant function, say $f(S)=2$ for all $\emptyset\subsetneq S\subsetneq V$. Then, sets of $\cH$ can be constructed using the cactus representation \cite{DKL76} of cuts in $\cN$. The significance of this method is that if $x$ is a basic feasible solution to a LP with constraints $x(\delta(S))\geq f(S)$ for all $S$, one can use this machinery to argue that the support of $x$ has size $O(|V|)$.

Informally, we prove the following, which 
can be seen as  an {\em approximate uncrossing technique}: 
\begin{theorem}[Informal]\label{thm:uncrossing}Suppose we have a vector $x:E\to\R_{\geq 0}$ such that $x(\delta(S))\geq f(S)$ for all $S$; define $\cN$ to be sets $S$ where $x(\delta(S))\leq f(S)(1+\eps)$ for some fixed $\eps>0$. If $f(.)$ is constant, say $f(S)=2$ for all $S$, then there is a set $\cN^*\subseteq \cN$ and a collection of edge sets $F_1,\dots,F_m\subseteq E$ such that the following hold:
\begin{itemize}
	\item $|\cN^*|= O(|V|)$, $m= O(|V|)$.
	\item $x(F_i)\geq 1-\eps/2$ for all $1\leq i\leq m$.
	\item Every edge $e$ is in at most $O(1)$ of the $F_i$'s.
	\item For every set $S\in \cN\smallsetminus \cN^*$ there exists $1\leq i<j\leq m$ such that $F_i\cap F_j=\emptyset$ and $F_i\cup F_j\subseteq \delta(S)$ and for every $S\in \cN^*$, there exists $1\leq i\leq m$ such that $F_i\subseteq \delta(S)$. 
\end{itemize}
\end{theorem}
In words, although we cannot simply refine $\cN$ to a linear number of sets, we can refine the edges in cuts of $\cN$ to a linear number of sets $F_1,\dots, F_m$ such  that we can essentially capture the edges of $\delta(S)$ for any $S\in \cN\smallsetminus \cN^*$ by a pair of disjoint $F_i$'s. We give a slightly weaker condition for cuts in $\cN^*$; namely we only capture half of their edges by $F_i$'s.

\begin{example}For a simple example of the above theorem, suppose $\eps=0$, i.e. $\cN$ is the set of min cuts of a graph $G$. Furthermore, suppose that every proper  cut in $\cN$ is \hyperlink{tar:crossing}{crossed} (recall that $S$ is proper if $1<|S|<|V|-1$) and that $\cN$ has at least one proper cut. 
Then, one can use an uncrossing technique, namely that if $A,B\in \cN$ then $A\cap B\in \cN$, to prove that $G$ must be cycle, namely we can order vertices of $G$, $v_0,\dots,v_{n-1}$ such that $x_{\{v_i,v_{i+1\text{ mod n}}\}}=1$.
In such a case we let $\cN^*=\emptyset$ and $F_i=E(v_i,v_{i+1\text{ mod }n})$.
%partition $V$ into sets $a_0,\dots,a_{m-1}$ such that 
%Let $\C$ be a connected component of crossing cuts of $\cN$, namely, for any pair of sets $A,B\in \C$ there is a path of crossing cuts all from $\C$ that goes from $A$ to $B$.
% and further suppose that $\cN$ can be represented by a cycle $C$ in the sense every min cut of $\cN$ corresponds to a min cut of $C$ and vice versa. Here we assume $a_0,\dots,a_{m-1}$ are the nodes of $C$ where each $a_i$ is identified with a disjoint set of vertices where $V=\uplus_{i=1}^m a_i$. In such a case, we can simply let $\cN^*=\emptyset$ and $F_i=E(a_i,a_{i+1\text{ mod }m})$. 
\label{eg:cycle}\end{example}

\begin{example}\label{eg:laminar}
For a second example, suppose again $\eps=0$ and $\cN$ is the set of mincuts of a graph $G$ where $\cN$ forms a laminar family (no two cuts cross). It turns out that we cannot decompose edges of cuts of $\cN$ into a linear sized collection of sets where every edge appears only a constant number of times. The main reason is that some edges may appear in an unbounded number of cuts. In this case we let $\cN^*=\cN$ and for every $A\in \cN$ (with immediate parent $B\in \cN$ in the laminar family) we add a set $F_A=\delta(A)\smallsetminus \delta(B)$  to our collection.  It is straightforward to show, using the structure of min cuts, that $x(F_A)\geq 1$; furthermore, since the size of a laminar family is linear in $V$, this gives a valid decomposition in the sense of above theorem.
\end{example}
Lastly, if $\eps=0$ and $\cN$ is the set of min cuts of an arbitrary graph, one can represent all min cuts of $\cN$ by a cactus \cite{DKL76} which can be seen as a tree of cycles. In such a case, one can use a construction similar to \cref{eg:cycle} for each cycle where instead of a vertex $v_i$ we have a set $a_i \subseteq V$ and one similar to \cref{eg:laminar} for the tree part of the cactus. For a concrete application of such a decomposition of min cuts see \cite{KKO20}.
%More generally, if $\cN$ corresponds to the set of min cuts of an arbitrary graph, the cuts of $\cN$ can be represented by a {\em cactus graph}. In such a case we add one $F_i$ for every edge of a cycle of the cactus. 


%and further for simplicity assume that there is a single connected component of crossing cuts in $\cN$, namely we can go from any $A$ to $B$ for $A,B\in\cN$ simply following crossing cuts of $\cN$. Then, one can represent cuts in $\cN$ by the set of min cuts of a cycle, namely we can contract vertices of $G$ 

%For a concrete application , suppose we need at least two edges in every set in $\cN^*$, say in a network optimization problem. Then, if we make sure that we have at least one edge in each $F_i$, all typical cuts constraints, $\cN\smallsetminus \cN^*$,  are satisfied, so we  reduce the problem to cuts in $\cN^*$. 


One of the main challenges in dealing with near min cuts relative to min cuts is that if $x(\delta(A)),x(\delta(B))\leq 2+\eps$ then $x(\delta(A\cap B))\leq 2+2\eps$. Therefore, if $\eps=0$, then min cuts are closed under intersection, set difference and union, but this is no longer true when $\eps>0$. So, to employ the classical uncrossing machinery one should be very careful to "uncross" only a constant number of times (independent of $\eps$) to make sure that every cut remains within $2+O(\eps)$. This is the main reason that the polygon representation of near min cuts (see below) is more sophisticated, e.g., we can no longer argue $x(E(a_i, a_{i+1}))\approx 1$, see \cref{fig:nearmincutbadexample}.

Although we don't study it here, we believe it may be worthwhile to find generalizations of \cref{thm:uncrossing} which hold for any (weakly) supermodular function.% That could be helpful in many questions based on the network optimization framework of Jain \cite{Jai01}.

\begin{remark} 
 We do not explicitly prove \cref{thm:uncrossing} in this extended abstract, as it is not used to prove \cref{thm:main}. However it can be deduced from arguments in \cref{sec:twoside} and \cref{app:oneside}. 
%In \cref{sec:overview} we discuss the main ideas of the proof of \cref{thm:uncrossing}. Here, let us explain the main challenge: In principal one might try to simply extend the above decomposition for the case $\eps=0$. The main challenge is that if $x(\delta(A)),x(\delta(B))\leq 2+\eps$ then $x(\delta(A\cap B))\leq 2+2\eps$. Therefore, if $\eps=0$, then min cuts are closed under intersection, set difference and union, but this is no longer true when $\eps>0$. So, to employ the classical uncrossing machinery one should be very careful to "uncross" only a constant number of times (independent of $\eps$) to make sure that every cut remains within $2+O(\eps)$. This is the main reason that the polygon representation of near min cuts (see below) is more sophisticated, e.g., we can no longer argue $x(E(a_i, a_{i+1}))\approx 1$, see \cref{fig:nearmincutbadexample}.
\end{remark}





\paragraph{Extensions to the Polygon Representation} To obtain our uncrossing framework we prove new properties of the polygon representation.
Given a graph $G=(V,E)$, let $k$ be the edge-connectivity of $G$, i.e. the number of edges in a minimum cut of $G$. For $\eps>0$, consider the set of $(1+\eps)$-near minimum cuts of $G$: cuts $(S,\overline{S})$ where $|E(S,\overline{S})| < (1+\eps)k$.
Bencz\'ur \cite{Ben95} and Bencz\'ur, Goemans \cite{BG08} proved that if $\eps \le 1/5$ then the near minimum cuts of $G$ admit a {\em polygon representation}. Namely, every connected component $\cC$ of \hyperlink{tar:crossing}{crossing} $(1+\eps)$ near min cuts can be represented by the diagonals of a convex polygon. In this polygon, the vertices of $G$ are partitioned into sets called \textit{atoms}, and every atom is mapped to a cell of this polygon defined by the diagonals and the boundary of the polygon itself (see \cref{sec:polyrep} for more details). 

The polygon representation can be seen as a generalization of the well-known cactus representation \cite{DKL76} of minimum cuts where a cycle of the cactus is replaced by a convex polygon. Unlike a cycle, some vertices/atoms map to the interior of the polygon, which are called ``inside'' atoms. The inside atoms at first look like a mystery and one can ask many questions about them such as how many can exist and what structures they can exhibit.



 Here, we explain two lemmas we proved which might find further applications beyond TSP in the future. 
%Our results give new intuition and understanding about the structure of polygon representations. These guide our analysis of the integrality gap of the subtour LP.
 %For example, one of our new observations is a 
 First, we give a necessary condition for a cell of a polygon to contain an inside atom:
\begin{lemma}[Informal, see \cref{thm:halfplanes}]
	Consider a polygon $P$ for a connected component $\C$ of a family of $1+\eps$ near min cuts for $\eps \le 1/5$ (where representing diagonals correspond to cuts in $\C$). Any cell of $P$ that has an inside atom must have at least $\Omega(1/\eps)$ many sides. 
\end{lemma}
This can be seen as a generalization of \cite[Lem 22]{BG08} to the case in which the cell is allowed to be adjacent to vertices of the polygon $P$.

Now, we explain our second extension: it follows from the cactus representation of minimum cuts that for a graph $G$ and a min cut $S$ one can partition the set of all min cuts that cross $S$ into two groups ${\cal A}=\{A_1,\dots,A_k\}$ and ${\cal B}=\{B_1,\dots,B_l\}$ for some $k,l\geq 0$ such that $S\cap A_1\subseteq S\cap A_2 \subseteq \dots S\cap A_k$ and, similarly, $S\cap B_1\subseteq \dots\subseteq S\cap B_l$. We prove a generalization of this fact for near min cuts:
\begin{lemma}[Informal, see \cref{lem:crosschain}]
Consider the set of $1+\eps$ near min cuts of a graph $G$ for $\eps\leq 1/10$; for any such near min cut $S$, one can partition the $1+\eps$ near min cuts crossing $S$ into two groups ${\cal A}=\{A_1,\dots,A_k\}$ and ${\cal B}=\{B_1,\dots,B_l\}$ such that $S\cap A_1 \subseteq S\cap A_2\subseteq \dots \subseteq S\cap A_k$ and similarly for cuts in ${\cal B}$.
\end{lemma}

\subsection{Outline of rest of paper} After reviewing preliminaries in \cref{sec:prelims}, we give a high-level overview of our proof technique in \cref{sec:overview}. The main new technical contributions of this paper are in \cref{sec:polyrep} and  \cref{sec:twoside}. The remaining content of the paper essentially follows from ~\cite{KKO21}. %Therefore, the reader may want to skip \cref{sec:proof-of-main}. 



\input{intro_MS}
%%%%%%%%%%%%%%%%%%%%%%%%%%%
\subsection{Summary of Main Results}\label{sec:results-overview}
To the best of our knowledge, this work is the first to analyze the dynamics of no-regret learners in competition with agents using other learning paradigms; to study the conditions under which no-regret learners survive in investment settings (in contrast to only bounding their regret level); and to establish a connection between the frameworks of regret minimization and Bayesian learning in asset markets. 
Our analysis yields a clear characterization of the relationship between regret rates, wealth shares, and long-term survival in such competitive environments. 

Beyond the theoretical contribution, our results provide practical insights for selecting learning strategies in market-facing applications. 
We show that competition imposes a high-stakes tradeoff between faster and more robust learning. 
The right approach depends on the extent to which campaign managers can access and trust information about the environment and the strategies of competitors. Bayesian methods, as suggested by economic theory, are powerful when the model class is accurate and known, and the update implementation is precise, but risky otherwise. No-regret learning methods, by contrast, are useful and robust when such knowledge is limited. Our hybrid approach, adding regularization to Bayesian updates, combines advantages of both, highlighting regularization as a practical and effective tool for balancing this tradeoff.

\vspace{8pt}
\noindent
\textbf{Main Theorems:} Before turning to related work, we conclude this part of the introduction with an informal summary of our main theorems. 

\empar{Regret and Survival}  
Theorem~\ref{thm:survival-by-finite-regret-gap} characterizes the regret condition for survival in a competitive asset market: an agent survives if and only if its regret remains bounded by an additive constant relative to every competitor at all times.  
Remarkably, if two players both have similar regret asymptotics with respect to the best strategy in hindsight (e.g., both have $O(\log T)$ regret) but with different coefficients, the agent with the larger constant almost surely vanishes from the market.  

\empar{Regret of the Optimal Investment Strategy}  
Regret with respect to the best strategy in hindsight is a strong benchmark; as we discuss in Section~\ref{sec:compare}, due to stochasticity in market dynamics, even the best implementable strategy (i.e., one that does not depend on future outcomes, but uses the true probabilities of states) has positive regret. In Theorem~\ref{thm:regret-of-q}, in Section~\ref{sec:no-regret}, we calculate this expected regret level, showing that it is constant and depends only on the number of assets in the market.
As a corollary of this result and Theorem \ref{thm:survival-by-finite-regret-gap}, any agent who survives in competition with an agent who invests knowing the true distribution of states must have regret bounded by a constant at all times. 

\empar{Survival against Perfect Bayesians}  
Theorem~\ref{thm:regret-of-a-Bayesian} shows that perfect Bayesian learners with a finite support have constant expected regret.  
Consequently, a learning agent survives in competition with perfect Bayesians if and only if it has constant regret as well.  
The proof uses Proposition~\ref{thm:Bayesians-suvive-against-q} about Bayesian learners together with the results described above.  

\empar{Bayesians with Inaccurate Priors}  
Theorem~\ref{thm:wrong-Bayesians-vanish} shows that a Bayesian agent with the correct state distribution not within its support suffers linear regret, and, as a result, vanishes from the market in competition with any no-regret learner.  
This theorem is followed by an analysis of the expected survival times of Bayesians with inaccurate priors, as a function of the size of the errors in the models considered in their priors and the regret rate of the best competitor.  

\empar{Bayesians with Inaccurate Updates}  
In Section~\ref{sec:noisy-Bayesian}, we analyze a scenario where a Bayesian learner performs ``trembling hand'' updates, making small zero-mean errors in the weight given to the current prior and the new data at each step.  
Theorem~\ref{thm:wrong-update-Bayesians-vanish} demonstrates that even such tiny errors break Bayesian learning, causing the learner to fail to converge and incur linear regret.  
As a result, any no-regret (or other) learner that converges to the correct model at any rate outperforms such an inaccurate Bayesian, driving it out of the market.  

\empar{Robust Bayesian Updates}  
In Section~\ref{sec:robust-Bayes-update}, we study a regularized version of Bayesian updating that offers greater robustness to distribution shifts.  
Proposition~\ref{thm:dist-shift-linear-regret-proposition} shows that standard Bayesian learning can incur linear regret under distribution shifts. Theorem~\ref{thm:robust-Bayesian} provides regret bounds for the method we propose: regret is constant when there are no distribution shifts and logarithmic in $T$ with shifts.

\vspace{8pt}
\noindent
\textbf{Roadmap:}
The paper is structured as follows. After a review of related work,   
Section~\ref{sec:model} presents the market model and preliminary analysis, discussing relative wealths and the role of relative entropy in characterizing survival under stationary investment rules.  
Section~\ref{sec:compare} reviews known results from no-regret learning and analyzes the relation between regret and wealth shares, showing how regret can be used to compare learners.  
Section~\ref{sec:Bayesians} restates in the language of our market setting several known analyses of Bayesian learning that are important for our comparison of Bayesians, no-regret learners, and imperfect Bayesians.   
Section~\ref{sec:no-regret} analyzes the regret of a player who knows the correct stochastic model, leading to a characterization of survival conditions in competition with a perfect Bayesian.  
Section~\ref{sec:imperfect-Bayesians} studies imperfect Bayesians with small mistakes in their priors or update process.  
Section~\ref{sec:robust-Bayes-update} presents a method for regularizing Bayesian updates to better handle distribution shifts, and analyzes its regret guarantees.   
Finally, Section~\ref{sec:simulations} presents simulation results for several scenarios of competition between no-regret learners, perfect Bayesian, and imperfect Bayesians, to complement our analysis and provide further intuition. Formal proofs are deferred to Appendix~\ref{sec:appendix-proofs}.

%%%%%%%%%%%%%%%%%%%%%%%%%%%
\subsection{Related Work}\label{sec:related-work}

\section{Related Work}
\label{sec:related}


One of the earlier works on MAP is the PGP framework by \cite{Durfee91,Decker92}. Recently, the MAP problem has started to receive an increasing amount of attention. Most of these recent research works consider agents separately for planning, and have been concentrated on how to explore the structure of agent interactions to reduce the search space, as well as solving the problem in a distributed fashion. \cite{Nissim2010} provide a search method by compiling MAP into a constraint satisfaction problem (CSP), and then using a distributed CSP framework to solve it. The MAP formulation is based on an extension of the STRIPS language called MA-STRIPS \cite{brafman2008}. In MA-STRIPS, actions are categorized into public and private actions. Public actions can influence other agents while private actions cannot. In this way, it is shown by \cite{brafman2008} that the search complexity of MAP is exponential in the tree-width of the agent interaction graph. Due to the poor performance of DisCSP based approaches, \cite{Nissim2012} apply the $A^*$ search algorithm in a distributed manner, which represents one of the state-of-art MAP solvers. \cite{torreno2012} propose a POP-based distributed planning framework for MAP, which uses a cooperative refinement planning technique that can handle planning with any level of coupling between the agents. Each agent at any step proposes a refinement step to improve the current group plan. Their approach does not assume complete information. A similar paradigm is taken by \cite{kvarnstrom11}. An iterative best-response planning and plan improvement technique using standard SAP algorithms is provided by \cite{jonsson2011}, which considers the previous singe agent plans as constraints to be satisfied while the following agents perform planning.


Given a problem, all of these MAP approaches solve it using the given set of agents, without first asking whether multiple agents are really required, let alone what is the minimum number of agents required. Answers to these questions not only separate MAP from SAP in a fundamental way, but also have real world applications when the team compositions can dynamically change. In this paper, we analyze these questions using the SAS$^+$ formalism \cite{backstrom96} with {\em causal graph} \cite{Knoblock94,helmert06}, which is often discussed in the context of factored planning \cite{bacchus93,amir03,brafman06,brafman2013}. The causal graph captures the interaction between different variables; intuitively, it can also capture the interactions between agents since agents affect each other through these variables. In fact, \cite{brafman2013} mention the causal graph's relation to the agent interaction graph when each variable is associated with a single agent.






























%%%%%%%%%%%%%%%%%%%%%%%%%%%
\section{Model and Preliminaries}\label{sec:model}
\section{Dynamic Task Allocation Mechanism\label{sec:task-allocation}}

The dynamic task allocation scenario we study considers a world
populated with tasks of $T$ different types and robots that are
equally capable of performing each task but can only be assigned
to one type at any given time. For example, the tasks could be
targets of different priority that have to be tracked, different
types of explosives that need to be located, etc. Additionally, a
robot cannot be idle --- each robot is always performing a task at
any given time. We introduce the notion of a robot state as a
shorthand for the type of task the robot is assigned to service. A
robot may switch its state according to its control policy when it
determines it is appropriate to do so. However, needlessly
switching tasks is to be avoided, since in physical robot systems,
this can involve complex physical movement that requires time to
perform.


The purpose of task allocation is to assign robots to tasks in a
way that will enhance the performance of the system, which
typically means reducing the overall execution time. Thus, if all
tasks take an equal amount of time to complete, in the best
allocation, the fraction of robots in state $i$ will be equal to
the fraction of tasks of type $i$. In general, however, the
desired allocation could take other forms ---  for example, it
could be related to the relative reward or cost of completing each
task type --- without change to our approach. In the dynamic task
allocation scenario, the number of tasks and the number of
available robots are allowed to change over time, for example, by
adding new tasks, deploying new robots, or removing malfunctioning
robots.


The challenge faced by the designer is to devise a mechanism that
will lead to a desired task allocation in a distributed MRS even
as the environment changes. The challenge is made even more
difficult by the fact that robots have limited sensing
capabilities, do not directly communicate with other robots, and
therefore, cannot acquire global information about the state of
the world, the initial or current number of tasks (total or by
type), or the initial or current number of robots (total or by
assigned type). Instead, robots can sample the world (assumed to
be finite) --- for example, by moving around and making local
observations of the environment. We assume that robots are able to
observe tasks and discriminate their types. They may also be able
to observe and discriminate the task states of other robots.

One way to give the robot an ability to respond to environmental
changes (including actions of other robots) is to give a robot an
internal state where it can store its knowledge of the environment
as captured by its observations~\cite{Jones03iros,Lerman03aamas}.
The observations are stored in a rolling history window of finite
length, with new observations replacing the oldest ones. The robot
consults these observations periodically and updates its task
state according to some transition function specified by the
designer. In an earlier work we
showed~\cite{Jones03iros,Lerman03iros} that this simple dynamic
task allocation mechanism leads to the desired task allocation in
a multi-foraging scenario.


In the following sections we present a mathematical model of
dynamic task allocation and study the role that transition
function and the number of observations (history length) play in
the performance of a multi-foraging MRS. In
\secref{sec:pucksonly}, we present a model of a simple scenario in
which robots base their decisions to change state solely on
observations of tasks in the environment. We study the simplest
form of the transition function, in which the probability to
change state to some type is proportional to the fraction of
existing tasks of that type. In \secref{sec:results1} we compare
theoretical predictions with no adjustable parameters to
experimental data and find excellent agreement. In
\secref{sec:phenomenological} we examine the more complex scenario
where the robots base their decisions to change task state on the
observations of both existing task types and task states of other
robots. In \secref{sec:results2} we study the consequences of the
choice of the transition function and history length on the system
behavior and find good agreement with the experimental data.


%%%%%%%%%%%%%%%%%%%%%%%%%%%
\section{Comparing Learners}\label{sec:compare}
\input{comparing-learners}

%%%%%%%%%%%%%%%%%%%%%%%%%%%
\section{Bayesian Learners}\label{sec:Bayesians}
\input{Bayesian-learning}

%%%%%%%%%%%%%%%%%%%%%%%%%%%
\section{The Regret of Bayesian Learners}\label{sec:no-regret}

In this section we further analyze the relationship between an agent's  regret  and long-term survival in market dynamics, and use this analysis to derive the regret level of Bayesian learners. This relationship depends on the competition. In particular, we saw in Theorem \ref{thm:survival-by-finite-regret-gap} that the learners who survive (have a positive expected wealth level in the limit) are only those who maintain a constant regret gap relative to the best competitor in the market. 

As mentioned in Section \ref{sec:compare}, using the state distribution $q$ as the investment rule (if $q$ were known)  maximizes the expected growth rate among constant strategies. However, even this strategy incurs some regret, as it generally differs from the best strategy in hindsight. We now ask:  what is the regret level of using the true state distribution $q$ as the investment rule? This regret level represents the best attainable expected regret. Thus, by Theorem \ref{thm:survival-by-finite-regret-gap}, evaluating this regret level would serve as the benchmark for determining which regret levels ensure survival (with high probability, see Definition \ref{def:survive-and-vanish}) against any competitors who do not have information about future state realizations. 
To set the ground, let us first consider the following definition:

\begin{definition}
Fix an arbitrary history of state realizations $s_1,\dots,s_T$ and denote the empirical distribution of this history by $\hat{q}$, where $\hat{q}_s = \frac{1}{T}\sum_{t=1}^T \ind{s_t=s}$. A {\em ``magic agent,''} indexed also by $\hat{q}$, is an agent playing the (eventual) empirical distribution in every step. That is,  $\alpha^{\hat{q}}_{st} = \hat{q}_s$ for all $t \in [T]$.     
\end{definition}

Note that this agent uses information regarding future realizations of random states, which is not available to real agents in a stochastic environment. Next, we derive the regret level of using the correct distribution of states $q$ as the investment strategy.
The proof is given in Appendix \ref{sec:appendix-proofs}; it uses our analysis from Section~\ref{sec:compare} and a result from information theory on the relative entropy between the underlying distribution $q$ and the empirical realization of states.  

\begin{theorem}\label{thm:regret-of-q}
    An agent using the state distribution $q$ as the investment strategy for all $t$ has a constant expected regret.
    The expected regret, denoted $R$, depends only on the number of states $S$. 
\end{theorem}
We can now also state the following corollary.
\begin{corollary}\label{cor:q-survives-against-zero-regret}
An agent indexed by $q$ using the correct distribution $q$ survives when competing against a magic agent. This is since by Theorem \ref{thm:regret-of-q}, the strategy $q$ yields constant expected regret $R$, and so the regret difference compared to $\hat{q}$---which  by definition yields zero regret---is constant as well. By Theorem \ref{thm:survival-by-finite-regret-gap}, this implies that agent $q$ survives.
\end{corollary}

Our analyses above and in the preceding sections lead to the following characterization of the regret of Bayesian learners and the survival conditions for other learning agents competing with them in the market. The proof is given in Appendix \ref{sec:appendix-proofs}.

\begin{theorem}\label{thm:regret-of-a-Bayesian}
    The following holds for any market parameters:
    \begin{enumerate}
        \item Bayesian learners with any finite-support prior that includes the correct state distribution have expected regret bounded by a constant at all times.
        \item A learning agent survives in competition with Bayesians if and only if it has constant regret.
    \end{enumerate}
\end{theorem}

The next lemma provides an expression for the regret of a player as a function of the entropy relative to strategy $q$, using our notation $R$ from Theorem \ref{thm:regret-of-q} for the expected regret of strategy $q$. This expression will be useful in the following Section on imperfect Bayesians.

\begin{lemma}\label{thm:expected-regret-lemma}
    The expected regret of a constant strategy sequence (that is, $\alpha^n_{1:T}$ such that $\alpha^n_t =  \alpha^n$ for all $t$) is given by 
    $
    \E[R^T(\alpha^n_{1:T})] = T \cdot I_q(\alpha^n) + R.
    $ 
\end{lemma}


%%%%%%%%%%%%%%%%%%%%%%%%%%%
\section{Imperfect Bayesians}\label{sec:imperfect-Bayesians}
In the preceding sections, we saw that a perfect Bayesian is optimal in the sense that it survives almost surely in the market and drives out any player with regret increasing over time. In this section, we explore a scenario where a Bayesian learner suffers small errors, either in its set of prior distributions or in its update rule. We find that, while perfect Bayesians are optimal, Bayesian learning is also very fragile; for example, it is key that the correct probability distribution is one of the models they consider. Specifically, we show that an imperfect Bayesian would eventually vanish from the market, either against the state distribution $q$ or in competition with any player with sub-linear regret. This holds even if the Bayesian's errors are very small and have zero mean.
 
\subsection{Bayesian Learning with Inaccurate Priors}\label{sec:noisy-priors-Bayesian}

As mentioned in Section \ref{sec:Bayesians}, a Bayesian learner with $q$ not within the support of its prior converges to using the best strategy within its prior (i.e., the one closest to $q$ in relative entropy), denoted here by $q'$. 
This convergence property leads to the following result. 
\begin{theorem}\label{thm:wrong-Bayesians-vanish}
    A Bayesian agent with the state distribution $q$ not in its support incurs regret 
    linear in $T$, and vanishes from the market in competition against any no-regret learner.   
\end{theorem}
Note that this contrasts with the case of Theorem \ref{thm:regret-of-a-Bayesian}, where a perfect Bayesian dominates the market in competition with any learners with regret increasing in time.
The proof is in Appendix \ref{sec:appendix-proofs}.

\vspace{5pt}
\noindent
{\em Remark:}
The possible strategies that the Bayesian can play span the entire convex hull of the prior. 
Bayesians with a finite prior can (and do) play convex combinations of the prior during the dynamic, but still, they always converge to the vertices---even when a more profitable strategy lies in the interior. 
In this sense, this convergence property of Bayesians is both a strength (if they have an accurate prior) and a weakness (if they do not). 
\vspace{5pt}

\bfpar{Typical Survival Time}
A question that arises now is what happens when errors are small.  
In the following analysis, we consider ``$\epsilon$-inaccurate Bayesians'' who have in their prior a close approximation of $q$, but still with some small error. Specifically, the total variation distance between the best strategy in the prior and the state distribution $q$ satisfies $TV(q', q) = \epsilon$ for some $\epsilon > 0$. 

On the one hand, we know that such an agent asymptotically fails to survive against a regret-minimizing agent for any error $\epsilon$. On the other hand, when $\epsilon$ is small, the inaccurate Bayesian will initially converge very quickly to a distribution close to the correct state distribution, and may capture a significant share of the market value during the early stages. 

Our goal now is to analyze the survival time during which the inaccurate Bayesian may still maintain a significant market share, and to understand how this survival time is related to the level of error $\epsilon$. This, of course, also depends on the best competitor and how fast they converge.

To estimate this relation and provide an upper bound on the survival time of an inaccurate Bayesian learner, we consider a player using $q'$ with $TV(q', q) = \epsilon$ (having already converged to this distribution), competing against a second player playing a dynamic strategy with expected regret level increasing as $f(t)$ that is sub-linear in $t$.

First, we would like to express the first player's error in terms of entropy (i.e., KL-divergence with $q$). We can bound the entropy using Pinsker's inequality~\citep{pinsker1964information} for a lower bound and its inverse for an upper bound, were the latter holds for finite distributions with full support, as in our case. We have (recall that $\Delta = \min_s q_s$),
\[
2\epsilon^2 \leq 
% D_{KL}(q || q') 
I_q(q')
\leq \frac{2}{\Delta} \epsilon^2.
\]
Using Lemma \ref{thm:expected-regret-lemma}, this can be translated into the regret of the inaccurate agent playing strategy $q'$ where $R$ is the regret of an investor using the true state distribution $q$ (see Theorem \ref{thm:regret-of-q}): 
\[
% D_{KL}(q \| q') 
I_q(q')
= \frac{1}{T} \big(\mathbb{E}[R^{T}(q'_{1:T})] - R \big).
\]

To maximize the survival time, we pick the distribution $q'$ to be the one with the smallest regret, which is when $I_q(q')$ is the smallest possible: $I_q(q') = 2\epsilon^2$. So we have\footnote{If the inaccurate agent uses the worst $q'$ with error $\epsilon$, the regret calculation is similar, but with $\epsilon$ rescaled by a factor of $1/\sqrt{\Delta}$, using the inverse of Pinsker's inequality: 
$2 \epsilon^2 \tau/\Delta  + R$. The smallest-regret  $q'$ is used for the bound.} 
$\mathbb{E}[R^{T}(q'_{1:T})] = 2\epsilon^2 T + R$. 
% 
By Equation (\ref{eq:logshare-as-regret}), the regret difference between two players is equal to the log of their wealth ratios  plus a constant. By comparing regret levels between the agents, we obtain a bound on the typical survival time $\tau$ up to which the inaccurate player holds, in expectation, more than half the market value; beyond this point, their expected share is less and continues decaying to zero:\footnote{An alternative question one may ask is: how accurate does my prior need to be to allow retaining a share of the market for $T$ steps against the competitor? For this, one can invert the equation and get: $\epsilon = \sqrt{(f(T) - R)/T}$.}
\[
f(\tau) = 2\epsilon^2 \tau + R.
\]

\vspace{5pt}
\noindent
{\em Remark:} Note that $f(t)$ is the competitor's actual regret. Regret bounds for known learning algorithms (e.g., bandit algorithms such as UCB (see, e.g.,   \cite{slivkins2019introduction}) or gradient-descent and second-order methods like those described in \cite{hazan2016introduction}) are typically worst-case bounds. Estimating the expected regret (or the most likely one) in game dynamics in general, or specifically in our investment scenario, is an interesting open problem.
\vspace{5pt}

For the special cases that the competitor has constant regret (for example, an accurate Bayesian), or the competitor has a $\log T$ or $\sqrt{T}$ regret level, we get the following results. 
\begin{observation}\label{obs:survival-time-gainst-constant-regret}
    The expected survival time of a Bayesian learner with $\epsilon$-inaccurate prior against any player with constant regret (e.g., a perfect Bayesian) is $O\big(\frac{1}{\epsilon^2}\big)$. 
\end{observation}

For a competitor with logarithmic regret, we get an equation of the form $\tau = c_1 \cdot \exp(c_2 \epsilon^2\tau)$. For small errors $\epsilon$, by expanding the exponent, we have the following bound:
\begin{observation}\label{obs:survival-time-gainst-log-regret}
    The expected survival time of a Bayesian learner with $\epsilon$-inaccurate prior against any player with logarithmic regret (e.g., a no-regret convex optimizer) is $O\big(\frac{1}{\epsilon^3}\big)$.   
\end{observation}

Finally, in competition against a learner with regret $f(T) = \sqrt{T}$ the survival time is longer:
\begin{observation}\label{obs:survival-time-gainst-sqrt-regret}
    The expected survival time of a Bayesian learner with $\epsilon$-inaccurate prior against any player with $\sqrt{T}$ regret is $O\big(\frac{1}{\epsilon^4}\big)$. 
    % For $c$ is $\tau \leq \frac{c-R^q}{2 \epsilon^2}$.  
\end{observation}

The interpretation of this analysis naturally depends on the time scale of interest. When $T$ is large, which is our main focus (e.g., when trade occurs at high frequency or if investments are long-term), only long-term survival matters. In this case, a Bayesian with an inaccurate prior will eventually vanish in competition with any learner who converges to the truth (e.g., a regret minimizer). However, when transient dynamics are also relevant, it is possible that a Bayesian with a prior that is inaccurate but relatively close to the true distribution may retain a significant market share for some time before ultimately vanishing. See Section~\ref{sec:simulations} for an example.


\subsection{Bayesian Learning with Noisy Updates}\label{sec:noisy-Bayesian}
Next, we consider a different type of imperfect Bayesian learner that does have $q$ in its prior, but in every step performs slightly inaccurate ``trembling hand'' updates. Also here, we demonstrate that Bayesian learning is fragile, even when the correct distribution lies in the support. To model this, we define a noisy Bayesian learner as one who at each step, either slightly over-weights the current observation or slightly over-weights its current prior, such that, in expectation, both the data and the prior receive the correct weights in every step (i.e., the errors in weight have zero mean).

For concreteness, consider the following scenario of a learner attempting to learn a distribution of states. Suppose that there are two states, $s_t = 0$ with a fixed probability $q \in (0,1)$, and $s_t = 1$ otherwise. The learner considers two models: $\theta_a = q$, which is the correct model, and $\theta_b \neq q$, with $\theta_b < 1$. The log-likelihood is given by
\begin{align}
    L(s_t) = 
    (1 - s_t) \log\Big(\frac{\theta_a}{\theta_b}\Big) + 
    s_t \log\Big(\frac{1 - \theta_a}{1 - \theta_b}\Big).
\end{align}

Now suppose that in every step $t$ the Bayesian learner performs ``$\lmb$-noisy updates'' where it over- or under-weights the data compared to the prior with a small excess weight $\lmbt$, where $\lmbt$ has $0$ mean. Specifically, for a parameter $\lmb > 0$, $\lmbt= \lmb$ with probability $1/2$ and $\eta_t = -\lmb$ otherwise. We find that even a tiny zero-mean error has a significant impact, essentially breaking the learning process.

\begin{theorem}\label{thm:wrong-update-Bayesians-vanish}
    For any $\lmb > 0$, the Bayesian learner with $\lmb$-noisy updates does not converge, and therefore incurs regret linear in $T$. 
\end{theorem}

The idea of the proof is to show that, with high probability, the learner has a systematic drift toward over-weighing new observations, despite the symmetry of the errors around zero. As a result, the learner’s beliefs fail to converge and continue to fluctuate in response to recent random events, leading to linear regret. The full proof appears in Appendix~\ref{sec:appendix-proofs}.


%%%%%%%%%%%%%%%%%%%%%%%%%%%
\section{Robust Bayesian Updates}\label{sec:robust-Bayes-update}
In the above analyses, we saw that the fast convergence of Bayesian learners is both a strength and a weakness. On the one hand, when a Bayesian converges to a correct model, it achieves constant regret and survives even against a fully informed competitor. On the other hand, if it converges to an incorrect model, it suffers linear regret (relative to the best fixed strategy in hindsight) and vanishes from the market. This strong convergence becomes especially problematic when the data-generating process undergoes distribution shifts. 

In this section, we propose a simple method that naturally retains the fast convergence of Bayesian learning at early stages while regularizing its update to ensure robustness under distribution shifts. The method provides provable regret guarantees and improves adaptability. We consider a setting in which the data-generating distribution may shift over time. We first show that, while standard Bayesian learners may eventually adapt, their regret against the true sequence of distributions after a distribution shift can be linear in the total time $T$.

\begin{proposition}\label{thm:dist-shift-linear-regret-proposition}
    Consider a market in which the data-generating distribution $q$ may shift over the course of $T$ steps among a finite set $Q$ of distributions. 
    A Bayesian learner with a finite prior that assigns positive probability to each model in $Q$ may incur regret linear in $T$ compared to a benchmark that knows the correct sequence of distributions in advance, even if there is only a single distribution shift during the $T$ steps. 
\end{proposition}

The proof is in Appendix~\ref{sec:appendix-proofs}. 
The Bayesian’s fast convergence is a feature if the Bayesian puts positive prior probability on a fixed, correct model of the data-generating process and for asymptotic results this all that matters; the prior is otherwise irrelevant. But if the data generating process shifts over time the Bayesian’s posterior at the (unknown) time of a shift becomes the prior. So intermediate-term properties matter. In the intermediate term, a Bayesian who has nearly converged to a past model learns slowly, as the log odds of the new model to the past one are extreme and the expected shift in these log odds is the fixed relative entropy between the two models. With repeated shifts, the Bayesian thus spends a large amount of time near models that are incorrect and accumulates regret that can be linear. 

In contrast, there are no-regret learning algorithms that are known to handle distribution shifts more gracefully \citep{hazan2009efficient,kozat2007universal}. In particular, for the portfolio selection problem we consider, such algorithms can achieve regret that is logarithmic in $T$ and linear in the number $n$ of distribution shifts (see also \cite{hazan2016introduction}, Sections 10.3 and 10.4). However, as we have seen, logarithmic regret is not sufficient for survival when competing against agents who invest according to the true distribution, or even against Bayesian learners. 

The main observation of this section is that, while the exponential convergence that Bayesians achieve is sufficient to obtain constant regret---and thus ensures long-run survival against fully informed competitors---it is not necessary. We propose a simple and natural method that combines advantages of both approaches, making a step towards a best-of-both-worlds learning strategy. It guarantees constant regret in stationary settings, like standard Bayesian learning, thus outperforming no-regret learners in such cases. At the same time, it guarantees logarithmic regret under distribution shifts, surviving against no-regret learners in settings where adaptability is essential. 

Technically, our method stays close to Bayesian learning. 
We use a regularized version of the Bayesian update rule that initially behaves like standard Bayesian learning, but slows convergence just enough to keep alternative hypotheses alive. This enables fast recovery after shifts. 

\vspace{8pt}
\noindent
\textbf{Robust Bayesian Update:}
The \emph{Robust Bayesian Update} process we propose uses a simple and efficient modification of standard Bayesian learning. Initially, the learner sets uniform prior weights $\lambda^1_0, \dots, \lambda^K_0$. At each time $t \geq 0$, the learner uses the current weights $\lambda_t$ as its investment strategy. Then, after observing the outcome, it updates beliefs with an added regularization term $\epsilon_t = t^{-2}$:
% 
\begin{equation}\label{eq:robust-Bayes-update}
\lambda_{t} \rightarrow \text{ (Bayes update) } \rightarrow \tilde{\lambda}_{t+1} \rightarrow \lambda_{t+1} = \frac{\tilde{\lambda}_{t+1} + \epsilon_t}{1 + K \epsilon_t}.    
\end{equation}
where $K$ is the number of hypotheses.  
That is, using the observed outcome at time~$t$, the learner applies the standard Bayesian update to obtain tentative weights~$\tilde{\lambda}_{t+1}$. It then adds regularization and normalizes to get weights $\lambda_{t+1}$ in the simplex.   
This simple process satisfies the following guarantee:

\begin{theorem}\label{thm:robust-Bayesian}
 Let $Q$ be the set of models in the learner's support. Suppose an adversary selects a sequence of intervals $\big(\{T_i, q_i\}\big)_{i=1}^n$, where $q_i \in Q$ and $T_i$ is the duration of the $i$-th interval during which $q_i$ is the data-generating distribution. Let $T = \sum_{i=1}^n T_i$ denote the total time. Then the regret of the Robust Bayesian Update, relative to a benchmark of fixed investment strategies in each interval, knowing the intervals in advance, is $O(n \log T)$ with high probability. In particular, when $n$ is constant, the regret is logarithmic in $T$. 

 Moreover, if there are no distribution shifts (i.e., when $n=1$), the regret remains bounded by a constant at all times with high probability.
\end{theorem} 
\vspace{3pt}

The intuition behind the proof is that when the weight of an incorrect model is large, it decays exponentially fast, as in the case of a standard Bayesian learner. When it is small---approaching the scale of the regularization term---it no longer contributes significantly to regret. Thus, the regret under a stationary process remains bounded. After a distribution shift, all model weights are of at least the size of the regularization level, which prevents suppressing the alternatives at an exponential rate and keeps them ``alive.'' This is then followed by exponential convergence, leading to regret that grows only logarithmically in $T$. The proof of the theorem is in Appendix~\ref{sec:appendix-proofs}.


%%%%%%%%%%%%%%%%%%%%%%%%%%%
\section{Simulations}\label{sec:simulations}
\input{simulations}

%%%%%%%%%%%%%%%%%%%%%%%%%%%


\section{Conclusion}\label{sec:Concln}
\section{Conclusions}
In this paper, we set out to address the problem of multi-tasking robots in multi-robot tasks. 
%A fundamental limitation of existing multi-robot systems was addressed by the removal of a restrictive assumption that was often made--robots are single-tasking.
%Our method allowed coalitions to overlap thus enabling multi-tasking robots. 
We observed that the key underlying challenge was to reason about the physical constraints that could be synergistically satisfied.
%which directly affected the feasibility of multi-tasking.
To address the challenge, we developed our method based on the information invariant theory and modeled constraints as information instances. 
%This allowed us to reason about the relationships between constraints by reasoning about those between information requirements. 
Thereby, a formal and general framework to achieve multi-tasking robots was developed. 
We showed that our algorithm was sound and complete under our problem settings. 
%Our method was integrated with a simple greedy heuristic for task allocation.
Simulation  results  were  provided  to  show  the  effectiveness  of  our approach under resource-constrained situations and in handling challenging situations. % in a multi-UAV simulator. 

% The idea of multi-tasking is attractive in many ways. 
% Humans are living in multi-tasking environments--at any point of time, 
% we are optimizing for more than one task. 
% Multi-task often leads to more efficient task performance since it allows us to exploit task synergies. 
% The work presented in this paper takes us one step forward in realizing multi-tasking robots. 
% In particular, we started looking at the feasibility of multi-tasking. 
% There are many potential directions to pursue along this direction. First, several limitations are present in the current approach. 
% For example, although our method guarantees that there exists a physical configuration that satisfies all the constraints, it does not explicitly take the environmental influence into account. For example, a narrow corridor may prevent a robot formation from passing through, even though all the constraints for the formation do not introduce any conflicts. In this sense, our work should better be characterized as establishing a necessary condition for multi-tasking. Also, our method is mainly focused on the ``{\it planning}'' phase and hence does not address how the robots reach the desired configuration and maintain the constraints. These issues are assumed to be handled by the execution layer.

% More generally, the question of how to execute the tasks with overlapping coalitions is not addressed in this work. 
% As we already discussed, executing individual tasks with non-overlapping coalitions is straightforward but task synergies impose additional requirements on the task execution: how should the robots that are assigned multiple tasks execute them? Should they consider them in a prioritized strategy~\cite{van2005prioritized}? Or should they combine the different tasks in a way that is similar to motor schemas~\cite{arkin2}. 
% Communication requirements for maintaining the constraints must also be taken into account. How should the robots optimize their communication to improve the task performance? 

% The stringency of the physical constraints is another interesting question. It may be desirable to relax the constraints in certain situations (e.g., due to environmental influences). In such cases, it may be important to consider the problem where the constraints are least violated~\cite{kim2012revision}, or specify task constraints in different ways to increase the diversity of the configurations~\cite{srivastava2007domain} so as to make it robust to different environments. 

\section*{Acknowledgments} 
We thank the anonymous referees at EC 2025 for useful feedback on an earlier version of the paper. 
\'Eva Tardos was supported in part by 
AFOSR grant FA9550-23-1-0410,  AFOSR grant FA9550-231-0068, and ONR MURI grant N000142412742.


\appendix
\section*{APPENDIX}
\section{Proofs}\label{sec:appendix-proofs}
%% The Appendices part is started with the command \appendix;
%% appendix sections are then done as normal sections
\appendix
% \onecolumn
\subsection{Prompt Template}
\label{app1}

Figs. \ref{appendix:prompt1}-\ref{appendix:prompt3} (due to the page length, we break it into three parts) show the prompt design for the information extraction in the context of construction project scheduling. 

\begin{figure}[t]
    \vspace{-4.5cm}
    \centering
    \begin{tcolorbox}[colback=gray!10!white, colframe=gray!50!gray, halign=left, boxrule=0.5pt, left=1mm, right=1mm, top=1mm, bottom=1mm]
    \fontsize{8pt}{8pt}\selectfont
    SYSTEM PROMPT: You are a project management assistant specializing in construction scheduling analysis. Your task is to analyze text descriptions of project changes and extract structured information about task relation changes in a construction project.
    \vspace{8pt}
    
    CONTEXT\\
    The project involves the following tasks and their relationships: \\
    \vspace{3pt}
    Task ID \textbar{} Predecessor \textbar{} Duration \textbar{} Description \textbar{} Robot Type
    \begin{itemize}
    \item T1 \textbar{} - \textbar{} 0.25 \textbar{} Move Electrical Conduit \textbar{} R1
    \item T2 \textbar{} - \textbar{} 0.25 \textbar{} Move Window Frame \textbar{} R1
    \item T3 \textbar{} - \textbar{} 0.25 \textbar{} Move Window \textbar{} R1
    \item T4 \textbar{} - \textbar{} 0.25 \textbar{} Move Duct Structural Materials \textbar{} R1
    \item T5 \textbar{} - \textbar{} 0.25 \textbar{} Move Duct \textbar{} R1
    \item T6 \textbar{} - \textbar{} 0.5 \textbar{} Drill Wall \textbar{} R4 or R2
    \item T7 \textbar{} T1, T6 \textbar{} 1 \textbar{} Install Electrical Conduit \textbar{} R5 or R2
    \item T8 \textbar{} T2 \textbar{} 1 \textbar{} Install Window Frame \textbar{} R4 or R2
    \item T9 \textbar{} T3, T8 \textbar{} 0.5 \textbar{} Install Window \textbar{} R3
    \item T10 \textbar{} T4 \textbar{} 2 \textbar{} Duct Structural Framing \textbar{} R4 or R2
    \item T11 \textbar{} T5, T10 \textbar{} 2 \textbar{} Install HVAC Duct \textbar{} R4 or R2
    \item T12 \textbar{} T7 \textbar{} 2 \textbar{} Install Wiring \textbar{} R5 or R2
    \item T13 \textbar{} T12 \textbar{} 1 \textbar{} Wall Painting \textbar{} R6
    \item T14 \textbar{} - \textbar{} 0.5 \textbar{} Construction Site Inspection \textbar{} R7
    \end{itemize}
    \vspace{8pt}
    
    The robot capabilities are listed below: \\
    \vspace{3pt}
    Robot ID \textbar{} Capabilities
    \begin{itemize}
    \item R1: Cargo container
    \item R2: High-payload, Precise parallel gripper, Normal parallel gripper
    \item R3: High-payload, Suction-based gripper
    \item R4: High-payload, Normal parallel gripper
    \item R5: Precise parallel gripper
    \item R6: Sprayer
    \item R7: Camera, IAQ sensors
    \end{itemize}
    \vspace{8pt}
    
    CONSTRAINT TYPES:
    \begin{enumerate}
    \item Task Dependency Adjustments
      \begin{itemize}
        \item Format: [task\_id, successor, +/-]
        \item task\_id: the target task
        \item successor: the successors of the target task
        \item +/-: ``+'' indicates a newly added successor, ``-'' means the dependency has been removed
      \end{itemize}
    \item Task Duration Variations
      \begin{itemize}
        \item Format: [task\_id, new\_duration]
        \item task\_id: the target task
        \item new\_duration: the new duration of the target task in hours
      \end{itemize}
    \item Task Starting Time Changes
      \begin{itemize}
        \item Format: [task\_id, start\_time\_change]
        \item task\_id: the target task
        \item start\_time\_change: the changes in start time of the target task (e.g., +2 means delayed by 2 hours; -2 means ahead by 2 hours)
      \end{itemize}
    \item Number of Robot Variations
      \begin{itemize}
        \item Format: [robot\_type\_id, robot\_number\_change]
        \item robot\_type\_id: the type of robot (e.g., R1, R2, R3, etc.)
        \item robot\_number\_change: the number changes of the robot (e.g., +1 means one more robot; -1 means one less robot)
      \end{itemize}
    \item Task Conflict Constraints
      \begin{itemize}
          \item Format: [task\_id1, task\_id2]
          \item task\_id1: the first task in the conflict
          \item task\_id2: the second task in the conflict
      \end{itemize}
    \end{enumerate}
    \end{tcolorbox}
    \caption{Prompt design for construction project scheduling - Part 1.}
    \label{appendix:prompt1}
\end{figure}


\begin{figure}[t]
    \centering
    \begin{tcolorbox}[colback=gray!10!white, colframe=gray!50!gray, halign=left, boxrule=0.5pt, left=1mm, right=1mm, top=1mm, bottom=1mm]
    \fontsize{8pt}{8pt}\selectfont
    STEP-BY-STEP INSTRUCTIONS:
    \begin{enumerate}
    \item Read through the entire description to understand the context.
    \item For each change mentioned in the description:
       \begin{itemize}
       \item[a.] Identify which task (T1-T14) or robot type (R1-R7) is being affected based on CONTEXT. 
        \begin{itemize}
          \item Be careful to distinguish between similar tasks, for example:
          \begin{itemize}
          \item T2 (Move Window Frame) vs. T3 (Move Window) vs. T8 (Install Window Frame) vs. T9 (Install Window) - These are different tasks.
          \item If text mentions ``window installation'', specifically, it refers to T9 (Install Window), not T3 or T8
          \item If text mentions "window frame installation," it refers to T8 (Install Window Frame), not T2
          \end{itemize}
        \item Be careful to distinguish between similar robots, for example:
          \begin{itemize}
          \item R2, R3, R4, and R5 are different robots. 
          \item Only R2 combines both high-payload and precise parallel gripper capabilities.
          \item If text only mentions ``high-payload and normal parallel gripper'', it refers to R4 not R2.
          \end{itemize}
       \end{itemize}
       \item[b.] Determine which constraint type (1-5) applies to the change based on CONSTRAINT TYPES.
       \item[c.] Extract the specific parameters needed for that constraint type.
       \item[d.] Format the parameters according to the required format for that constraint type.
       \end{itemize}
    \item Compile all identified changes into the JSON output format:
       \begin{itemize}
       \item[a.] Create a JSON object with a ``changes'' array.
       \item[b.] For each change, add an object with ``constraint\_type'' and ``parameters'' fields.
       \item[c.] Ensure numerical values (like durations and time changes) are formatted as numbers, not strings.
       \item[d.] Ensure task IDs, successors, and robot types are formatted as strings.
       \item[e.] For time-related values:
          \begin{itemize}
          \item Simplify all numerical values to their simplest form (e.g., 1.5 not 1.50, 2 not 2.0)
          \item Convert minutes to hours (e.g., 30 minutes = 0.5 hours, 45 minutes = 0.75 hours)
          \item Please be aware that if you identify the constraint as 3, the time change should be associated with ``+'' or ``-''. 
          \end{itemize}
       \item[g.] Please be aware that if you identify the constraint as 4, the robot change should be associated with ``+'' or ``-''.
       \end{itemize}
    \item Double-check your result to ensure all changes mentioned in the description have been captured.
       \begin{itemize}
       \item[a.] Please ensure that your output follows the required format; e.g., for constraint 1, the output should be [task\_id, successor, +/-] (do NOT nest successors in additional brackets) and the task\_id should be the predecessor of the successor. 
       \item[b.] Please ensure that if you identify the constraint as 1, you correctly identify the target task and the successor of the target task and put them in the right order [task\_id, successor, +/-].
       \item[c.] Please ensure that if you identify the constraint as 3, the time change should be associated with ``+'' or ``-''. 
       \item[d.] Please ensure that if you identify the constraint as 4, the robot change should be associated with ``+'' or ``-''. 
       \item[e.] Please ensure that the task description corresponds to the task\_id in the CONTEXT.
       \end{itemize}
    \end{enumerate}
    \end{tcolorbox}
    \caption{Prompt design for construction project scheduling - Part 2.}
    \label{appendix:prompt2}
\end{figure}


\begin{figure}[t]
    \vspace{-2cm}
    \centering
    \begin{tcolorbox}[colback=gray!10!white, colframe=gray!50!gray, halign=left, boxrule=0.5pt, left=1mm, right=1mm, top=1mm, bottom=1mm]
    \fontsize{8pt}{8pt}\selectfont
EXAMPLES:\\
    Example 1:\\
    Input: ``Due to how things are unfolding on-site, it's understood that the drilling machine is not functioning, so the wall will be drilled manually. The task is expected to take two hours, and in light of recent discussions, after coordinating with field staff, it seems that the original worker assigned to install the HVAC duct is no longer available; however, we have secured another worker who can arrive in 150 minutes.''\\
    Output:
    \begin{verbatim}
{"changes": [
{"constraint_type": 2, "parameters": [T6, 2]},
{"constraint_type": 3, "parameters": [T11, +2.5]}
]}
    \end{verbatim}

    Example 2:\\
    Input: ``Recent developments suggest that wall painting takes 1.5 hours instead of 1 hour due to the need for multiple coats, and in light of recent adjustments, a revised understanding across teams indicates that a specialist required for electrical conduit installation calls in sick, preventing work from starting for 2 hours., followed by further refinements as recent developments suggest that wall painting takes 1.5 hours instead of 1 hour due to the need for multiple coats.''\\
    Output:
    \begin{verbatim}
{"changes": [
{"constraint_type": 2, "parameters": [T13, 1.5]},
{"constraint_type": 3, "parameters": [T7, +2]},
{"constraint_type": 2, "parameters": [T13, 1.5]}
]}
    \end{verbatim}

    Example 3:\\
    Input: ``Task dependencies have shifted, and one of the robots capable of handling heavy loads and performing fine, precise tasks is currently out of service due to a mechanical failure. Additionally, in light of recent discussions and the evolving situation on-site, it appears that two robots with high-capacity arms and fine-movement grippers were not charged, and have now run out of power.''\\
    Output:
    \begin{verbatim}
{"changes": [
{"constraint_type": 4, "parameters": [R2, -1]},
{"constraint_type": 4, "parameters": [R2, -2]}
]}
    \end{verbatim}
    
    Now, analyze the following description and extract all task relation changes in the specified JSON format: \{description\}
    
    Please output your response in JSON format and do not output other things. 
    \begin{verbatim}
{"changes": [
{"constraint_type": <number>, 
 "parameters": [<value1>, <value2>, ...]},...
]}
    \end{verbatim}
    \end{tcolorbox}
    \caption{Prompt design for construction project scheduling - Part 3.}
    \label{appendix:prompt3}
\end{figure}


% \clearpage
\bibliographystyle{abbrvnat} % outcomment this and next line in Case 1
\bibliography{learning-in-markets} % if more than one, comma separated

%\bibliographystyle{informs2014} % outcomment this and next line in Case 1
%\bibliography{sample} % if more than one, comma separated

% CASE 2: BiBTeX used to generate mypaper.bbl (to be further fine tuned)
%\input{mypaper.bbl} % outcomment this line in Case 2

%If you don't use BiBTex, you can manually itemize references as shown below.

%\bibliographystyle{nonumber}

% \begin{thebibliography}{3}
% \providecommand{\natexlab}[1]{#1}
% \providecommand{\url}[1]{\texttt{#1}}
% \providecommand{\urlprefix}{URL }

% \bibitem[{Smith(2005)}]{smith2005}
% Smith J (2005) Optimal resource allocation in humanitarian logistics.
%   \emph{Journal of Operations Research} 30(2):123--135.
  
% \bibitem[{Jones(2010)}]{jones2010}
% Jones S (2010) Stochastic programming models for humanitarian logistics.
%   \emph{INFORMS Mathematics of Operations Research} 35(4):567--580.

% \bibitem[{Brown(2015)}]{brown2015}
% Brown D (2015) \emph{Introduction to Stochastic Programming} (Springer).

% \end{thebibliography}


%%%%%%%%%%%%%%%%%
\end{document}
%%%%%%%%%%%%%%%%%



