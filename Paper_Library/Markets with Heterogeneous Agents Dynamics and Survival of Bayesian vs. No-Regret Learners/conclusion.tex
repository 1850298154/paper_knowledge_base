As markets become increasingly populated by learning agents, understanding which learning approaches succeed in competitive environments becomes a foundational question. Our work brings together perspectives from economics and computer science by analyzing the survival and performance of Bayesian and no-regret learners in complete asset markets. We connect the notions of regret and survival in a competitive market, and show that regret minimization, while robust and widely applicable, does not guarantee survival: a Bayesian investor with correct support and proper updates can drive out no-regret learners, even if their regret is vanishing. Conversely, Bayesian learning is fragile---slight misspecification in priors or slight mistakes in belief updates may lead to linear regret and eventual extinction from the market.

These insights motivate the study of more robust ways to utilize Bayesian learning and combine advantages from both approaches. 
We proposed and analyzed a simple regularized variant of Bayesian updating that preserves fast convergence in stationary environments while ensuring adaptability to distribution shifts. The resulting method achieves constant regret in stable settings and only logarithmic regret when the environment changes, thereby dominating existing methods and offering a simple, practical step toward a best-of-both-worlds approach. Our findings highlight the speed of learning as a crucial determinant of survival and show that even constant factors in regret rates matter in competitive markets. We view the development of robust Bayesian learning methods, guided by principles from online learning theory, as a promising direction for future work. 

Another interesting direction for future work is to consider the impact of competing learners on aggregate outcomes such as equilibrium  prices. Both long run security prices, determined by who survives, and intermediate term security prices, determined by the evolution of investment rules and wealth shares, could be analyzed. Our focus has been on the survival of individual agents and they determine long run prices. But intermediate term prices also depend on how agents respond to the feedback they receive over time from markets and how wealth shares evolve between these agents. Both Bayesian and standard no-regret learning cause agents to invest more in assets that pay off and wealth share increases for agents who invest relatively more in these assets. So intermediate term security prices respond positively to past payoffs. These learning and wealth share evolution effects induce positive serial correlation---or momentum---in security prices.
