\section{Competitions}
In this section, we describe the Pommerman competitions. This includes both the upcoming NIPS 2018 event and the FFA competition that we already ran.

\subsection{FFA competition}
We ran a preliminary competition on June 3rd, 2018. We did not advertise this widely other than within our Discord social group (https://discord.gg/mtW7kp), nor did we have any prizes for it. Even so, we had a turnout of eight competitors who submitted working agents by the May 31st deadline.

The competition environment was the FFA variant \cite{pommermanffa} where four agents enter, all of whom are opponents. The top two agents were submitted by G{\"o}r{\"o}g Márton and a team led by Yichen Gong, with the latter being the strongest.

G{\"o}r{\"o}g's agent improved upon the repository's baseline agent through a number of edits. On the other hand, Yichen's agent was a redesign implementing a Finite State Machine Tree-Search approach \cite{zhou2018pommermanagent}. They respectively won 8 and 22 of their 35 matches (with a number of the remaining being ties).

\subsection{NIPS Competition}
The NIPS competition will be held live at NIPS 2018 and competitors are required to submit a team of two agents by November 21st, 2018. The featured environment will be the partially observable team variant without communication. Otherwise, we will be reusing the machinery that we developed to run the FFA competition.

% We are especially thankful for the generosity of our sponsors. These include Jane Street (janestreet.com), Facebook AI Research (research.fb.com/category/facebook-ai-research), Google Cloud (cloud.google.com), and NVidia Research (nvidia.com/en-us/research). Through their help, we are offering prizes as follows:
% \begin{itemize}
%     \item 1st Place: \$4k USD \& \$6k Google Cloud credit.
%     \item 2nd Place: \$2k USD \& \$4k Google Cloud credit.
%     \item 3rd Place: \$1k USD \& \$2k Google Cloud credit.
%     \item 4th Place: \$2k Google Cloud credit.
%     \item Each of the top two learning agents: NVidia Titan V GPU.
% \end{itemize}

% In addition, we are rewarding \$1k Google Cloud credit for outstanding community contributions.


\subsection{Submitting Agents}
We run the competitions using Docker and expect submissions to be accompanied by a Docker file that we can build on the game servers. For FFA competitions, this entails submitting a (possibly private) repository having one Docker file representing the agent. For team competitions, this means the submission should have two Docker files to represent the two agents. Instructions and an example for building Docker containers from trained agents can be found in our repository \cite{github}. 

The agents should follow the prescribed convention specified in our example code and expose an `act' endpoint that accepts the dictionary of observations. Because we are using Docker containers and http requests, we do not have any requirements for programming language or framework.

The expected response from the agent will be a single integer in [0, 5] representing which of the six actions that agent would like to take. In variants with messages, we also expect two more integers in [1, 8] representing the message. If an agent does not respond in an appropriate time limit for our competition constraints (100ms), then we will automatically issue them the Stop action and, if appropriate, have them send out the message (0, 0). This timeout is an aspect of the competition and not native to the game itself.