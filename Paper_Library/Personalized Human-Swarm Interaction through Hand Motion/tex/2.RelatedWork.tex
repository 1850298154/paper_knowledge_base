% \section{Related Work (this section is temporarily commented out)}
% \label{sec:relatedwork}
% \mm{Here we will write the state of the art and its current downsides.}
% \mm{@Ludovic, don't spend too much time on this. It would be of great help if you could
% \begin{itemize}
%     \item Include all of your reference
%     \item Write a very short sentence on the content of each paper
% \end{itemize}}
% \ld{All the references are here}
% \textcolor{grey}{
% \cite{cruse1987human} Redundance of human interfaces. Analysis of models of the human arm. The human arm is described as a redundant manipulator with some correlated degrees of freedom. The motion of the human arm to do a specific maneuvers can be explained using some hypotheses. It can be use to justify the fact that human body presents a lot of degrees of freedom and thus should be enough to control the swarm.\\
% \cite{stanton2012teleoperation} Use of an homologous body motion interface to control an humanoid robot. Uses neural network for the interface that learns from the user motion as we do. Benefits from the lack of robot modelisation by using machine learning.\\
% \cite{miehlbradt2018data} Data driven control of drones, gives an easy to learn interface based on body motion for the teleoperation of a drone\\
% \cite{katona2017hand} Article on Leap Motion integration to control mobile robotics. Non homologous interface but not intuitive and not on drones. It is based on a modelisation of the robot and on gesture recognition\\
% \cite{mellinger2013cooperative} Transport of objects using swarm of drone. It is based on dynamic equations and thus no teleoperation here. But it shows an example of the use of drones swarm.\\
% \cite{macchini2019personalized} Personalized mapping to the control of a single FPV fixed-wing drone. I think you know this one \\
% \cite{macchini2020hand} Hand interface with feedback and control of the swarm. You know this one too.\\
% \cite{weichert2013analysis} Accuracy of the Leap Motion. Analysis of the Leap motion performances. It gives measures of the Leap accuracy (=/= from the manufacturer previsions). You may not use it, I did use it in my report when introducing the Leap Motion\\
% \cite{jin2016multi} Multi Leap usage for teleoperation of a robotic arm. Adresses the occlusion problem by the use of two Leap Motions in different configurations. It shows an application of the Leap Motion\\
% \cite{gorisse2017first} Immersive environment for intuitive interface. Identification of the parameters of the environment making it more or less immersive and the actions in it more intuitive. May be used to discuss about 1st/3rd PV. I did not use it in my report\\
% \cite{casadio2012body} Body Machine interface. Overview of the basic concepts of BoMI.
% \cite{sarkar2016gesture} Use of the Leap motion for the control of a drone. Based on gesture recognition, it triggers different motions of the drone. It shows the control of a drone through hand motion and teleoperation using the leap. However, it is not intuitive nor personalized.\\
% \cite{sanger2000human} Principal component analysis. Analysis of movement patterns through superposition of principal component analysis\\
% }


