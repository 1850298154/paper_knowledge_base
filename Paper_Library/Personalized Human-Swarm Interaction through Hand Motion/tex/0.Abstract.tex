\begin{abstract}

The control of collective robotic systems, such as drone swarms, is often delegated to autonomous navigation algorithms due to their high dimensionality.
However, like other robotic entities, drone swarms can still benefit from being teleoperated by human operators, whose perception and decision-making capabilities are still out of the reach of autonomous systems. Drone swarm teleoperation is only at its dawn, and a standard human-swarm interface (HRI) is missing to date. In this study, we analyzed the spontaneous interaction strategies of naive users with a swarm of drones.  We implemented a machine-learning algorithm to define a personalized Body-Machine Interface (BoMI) based only on a short calibration procedure. During this procedure, the human operator is asked to move spontaneously as if they were in control of a simulated drone swarm.
% on a simple motion demonstration.
% In our work, the sensor coverage and the processing are adapted to the case of a robot swarm based on a user study.
We assessed that hands are the most commonly adopted body segment, and thus we chose a LEAP Motion controller to track them to let the users control the aerial drone swarm. This choice makes our interface portable since it does not rely on a centralized system for tracking the human body.
We validated our algorithm to define personalized HRIs for a set of participants in a realistic simulated environment, showing promising results in performance and user experience. 
Our method leaves unprecedented freedom to the user to choose between position and velocity control only based on their body motion preferences.
% Moreover, our method represents a significant simplification for the implementation of human-swarm interfaces, as it only relies on an individual calibration by demonstration.
\end{abstract}