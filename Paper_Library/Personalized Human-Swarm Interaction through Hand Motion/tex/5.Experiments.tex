\section{Experimental Results}\label{sec:experiments}


% \subsection{Teleoperation of the simulated swarm}

\begin{figure}[t]
\begin{center}
  \includegraphics[width=\columnwidth]{./protocol2}
\caption{
Experimental protocol for the qualification of the personalized HRI definition method. (A) Imitation task: the user's spontaneous interaction strategy is recorded and processed to implement a personalized interface. (B) Teleoperation task: the user controls the drone swarm through their personalized interface.
}
\label{f:protocol2}
\end{center}
\end{figure}
We recruited 10 participants to validate the effectiveness of the proposed method aimed at creating personalized HRIs for hand motion based swarm teleoperation.
The experiments consisted of 2 phases.
The first one was an imitation task, necessary for the user to show their preferred body motion to control the robot, and the second was the real teleoperation task (Fig. \ref{f:protocol2}).
In the first phase, we showed the same maneuvers described in Sec. \ref{sec:pilots} to the user through an HMD and asked them to move their hands accordingly (Fig. \ref{f:protocol2}A).
The teleoperation scenario consisted of a path composed of 4 gates (Fig. \ref{f:protocol2}B).
We instructed the participant to cross them in order from 1 to 4, performing the task as fast as possible while trying to avoid collisions.
The task was designed to require the robot swarm to be controlled in all its 4 DoF, as the gates are arranged in 3D on different altitudes and depth levels, and the spheric object inside gate 2 can be avoided only by expanding the swarm.
The participants were asked to steer the drone swarm across the path for a total of 10 times: 5 times through hand motion and 5 times using a standard remote controller (hereafter, conditions 'H' and 'R') to evaluate their performance prior and after training.
We pseudo-randomized the order of the interfaces to be used in order to compensate for the learning effects due to the user's increasing experience.

\begin{figure*}[t]
\begin{center}
\includegraphics[width=0.9\textwidth]{./res_alt}
	\caption{
Performance evaluation of groups using the remote controller (R) and the hand interface (H). (A) Total time needed to navigate along the path. While group R performed better before training, group H showed a higher learning rate. (B) Time needed to cross individual gates. Subjects showed significant task-dependent performance differences. (C) Number of occurred collisions during the navigation. Similarly to time performance, remote users performed better in the beginning, while hand interface users improved with training to reach comparable accuracy.}
	\label{f:res_alt}
\end{center}
\end{figure*}

\textbf{\textit{Remote controller users perform better initially, but hand interface users learn faster:}}
our results show that the teleoperation performance varies with both the used interface and the training for this task (Fig. \ref{f:res_alt}).
Group R outperformed group H in terms of time needed to navigate the whole path (Fig. \ref{f:res_alt}A). Similar results hold both in the first ($t^R_1 = 86.6 \pm 30.1s$, $t^H_1 = 120.1 \pm 44.5s$) and the last runs ($t^R_5 = 80.6 \pm 23.1s$, $t^H_5 = 105.1 \pm 30.1s$). However, group H showed a higher learning capability, reducing their time by $13.1\%$, in average, compared to the $7.0\%$ of group R.
Breaking down the path into the 4 inter-gate segments, however, we realized that the different maneuvers needed to steer the swarm through the gates were associated with different performance in our participants (Fig. \ref{f:res_alt}B).
Particularly, the time needed to cross gates 2 and 4 were similar for both groups, with a non-significant advantage for remote controller users.
Contrarily, group R performed significantly better in gate 3 before training ($t^R_1 = 22.4 \pm 6.1s $, $t^H_1 = 45.1 \pm 19.1s$, $p < 0.01$). The initial $100.1\%$ performance gap was reduced, with training, to $41.2\%$, non significant at a statistical level.
Finally, in crossing gate 1, group R performed closed to twice as fast both before ($t^R_1 = 5.46 \pm 2.71s $, $t^H_1 = 11.5 \pm 5.0s$, $p = 0.012$) and after training ($t^R_5 = 6.6 \pm 3.5s $, $t^H_5 = 13.1 \pm 5.3s$, $p = 0.011$).

We found that the use of different interfaces can affect the number of collisions during teleoperation (Fig. \ref{f:res_alt}C). 
Group R started with a lower number of collisions per run ($Coll^R_1 = 7$) than group H ($Coll^H_1 = 19$).
However, while Group R improved only marginally their performance ($Coll^R_5 = 5$), the hand interface users managed to reduce their collisions significantly with training ($Coll^H_5 = 8$).

\textbf{\textit{User preference is equally split between interfaces:}}
after the teleoperation task, we asked our participants to fill a subjective feedback survey (Tab. \ref{t:quest}).
The questionnaire consisted of 4 multiple choice questions and 1 final feedback open question.
We asked two different questions about the control of the position and the expansion, as the second is a peculiar DoF of swarms and cannot be controlled with a single agent.
The responses to the survey show that the users did not find any of the two interfaces clearly superior (Fig. \ref{f:quest}).
In particular, the results were identical for the expansion/contraction DoF: 5 participants preferred the remote and 5 preferred the hand interface.
We obtained similar results for the general preference: 3 participants preferred the remote, 3 preferred the hand interface, and 4 did not have a preference.
Finally, 3 participants preferred the hand interface to control the position of the swarm, and only 1 preferred the remote.
7 participants declared that the interface reflected perfectly their expected motion, 2 that it was adequately accurate, and only 1 that his motion was somehow reflected in the interface.

In the final open question, 5 subjects mentioned that they felt a faster improvement when using the hand interface with respect to the remote. 
4 subjects remarked that their prior experience in using a remote controller might be the reason for their higher performance in the initial trial.
4 participants stated that they found the motion-based interface more engaging than the standard solution (specifically: "funny", "attractive", "impressive").
Finally, 3 participants responded that the limited field of view of the sensor affected their performance during the task when using the hand interface.

\begin{table}[h]
\caption{Personal feedback questionnaire}
\normalsize
% \renewcommand{\arraystretch}{1.1} % Default value: 1
\begin{center}
\begin{tabular}{  c p{7cm} } 
%  \hline
 \textbf{ID} & \textbf{Question} \\ 
 \hline
%  \hline
 Q1 & Which interface did you prefer to control the position? \\ 
%   \hline
 Q2 & Which interface did you prefer to control the expansion? \\ 
%   \hline
 Q3 & Which interface did you prefer in general? \\ 
%   \hline
 Q4 & Did the interface match the one you imagined during the calibration phase? \\
%   \hline
 Q5 & Please give your personal feedback on the teleoperation experience \\
  \hline
\end{tabular}
\label{t:quest}
\end{center}
\end{table}

\begin{figure}[h]
\begin{center}
  \includegraphics[width=\columnwidth]{./quest}
\caption{
Survey results for questions 1-4. Despite the performance differences, users' preferences were split equally between the two interfaces, with a slight preference towards the hand interface to control the swarm position.
7/10 participants reported that the HRI was perfectly reflecting their desired behavior.
}
\label{f:quest}
\end{center}
\end{figure}


% \begin{itemize}
%     \item protocol
%     \item task description
%     \item time
%     \item collisions
%     \item surveys
% \end{itemize}
% For the final experiment, we asked 10 subjects to perform the calibration part as in the previous experiment and using the framework, we created a personalized mapping. Then a teleoperation task in simulation is done by each of the subjects with the personalized interface and with a controller.\\
% We evaluate two  metrics, the speed of the task and the precision of the teleoperation. To do so we plotted the mean time the subjects spent between two gates (see figure~\ref{f:res1}) and the number of gates collided during the task (see figure~\ref{f:res3}).\\

% We observe that improvements in speed over the runs are not significant. Indeed, both interfaces improve similarly and the personalized interface remains slower with the controller even after 5 runs. \grey{We can see that in particular, one gate (the third one) is difficult to cross for users of the personalized interface and the subjects did not improve concerning the time taken to go through this gate. However, we can also note that the fifth run with the personalized interface got similar results (except for the third gate) than the first run with the controller.}\\

% However, figure~\ref{f:res3} shows very promising results for our interface. We can see a clear improvement over the runs of the precision of the teleoperation with the personalized interface. The mean number of collided gates reduces over the runs with the personalized interface while with the controller this number do not improve at all. The personalized interface even beats the controller interface in terms of precision for the two last runs.\\

% Therefore, overall the subject improved more when using the personalized interface, which corresponds to the purpose of the intuitive interface.\\
% This experiment went with a survey allowing the users to comment and evaluate their performances on both interfaces. This survey reveals that people found both interfaces quite similar in term of efficiency for the control of the swarm of drones.\\

% \grey{Thus this experiment showed that this interface may be able to compete with existing interfaces}


