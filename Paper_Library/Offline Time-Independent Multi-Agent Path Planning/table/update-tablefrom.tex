{
  \newcommand{\f}[3]{\m{\left[(#1), (#2), (#3)\right]}}
  \begin{table}[t]
    \centering
    {
    \footnotesize
    \begin{tabular}{rrl}
      \toprule
      induction & key & new fragments
      \\ \midrule
      $\{\pi_1\}$
          & $u$ & \f{1}{1}{u,v} \\
          & $v$ & \f{1}{2}{v,w}
          %
      \\ \midrule
      %
      $\{\pi_1, \pi_2\}$
          & $u$ & \f{1,2}{1,1}{u,v,x} \\
          & $v$ & \f{2}{1}{v,x} \\
          & $x$ & \f{2}{2}{x,y}
          %
      \\ \midrule
      $\{\pi_{1}, \pi_{2}, \pi_{3}\}$
          & $u$ & \textcolor{blue}{\f{1,2,3}{1,1,2}{u,v,x,u}} \\
          & $v$ & \f{2,3}{1,2}{v,x,u} \\
          & $x$ & \f{3}{2}{x,u}, \f{3,1}{2,1}{x,u,v}\\
          & $z$ & \f{3}{1}{z,x}, \f{3,2}{1,2}{z,x,y}
      \\ \bottomrule
    \end{tabular}
    }
    \caption{
      Example of detecting potential cyclic deadlocks.
      We describe the update of \tablefrom for $\path{1} = (u, v, w), \path{2} = (v, x, y), \path{3} = (z, x, u)$.
      The table uses $[(\text{agents}), (\text{progress indexes}), (\text{path})]$ as a notation of fragment,  where ``path'' is a corresponding sequence of vertices of the fragment.
      The algorithm halts with a blue-colored fragment, a detected potential cyclic deadlock.
    }
    \label{table:update-tablefrom}
  \end{table}
}
