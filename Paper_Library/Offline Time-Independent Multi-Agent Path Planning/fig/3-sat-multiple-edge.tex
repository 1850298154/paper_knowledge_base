\begin{figure}[tb!]
  \centering
  \begin{tikzpicture}
    \newcommand{\edgesize}{1.6cm}
    \newcommand{\hedgex}{0.75}
    \newcommand{\hedgey}{0.6}
    \tikzset{
      sat-node/.style={vertex, minimum size=0.15cm, fill=black, anchor=center},
      start-node/.style={vertex, minimum size=0.45cm, text width=0.45cm, rectangle},
      goal-node/.style={vertex, minimum size=0.5cm},
      bold-line/.style={very thick},
      line-c3/.style={cyan,->},
    }
    \tiny
    %
    \node[sat-node](v1) at (0, 0) {};
    \node[sat-node](v2) at (3, 0) {};
    %
    \node[start-node](v3) at ($(v1) + (1.5,  0.8)$) {};
    \node[start-node](v4) at ($(v1) + (1.5,  0.0)$) {};
    \node[start-node](v5) at ($(v1) + (1.5, -0.8)$) {};
    %
    \node[goal-node](v6) at (v3) {};
    \node[goal-node](v7) at (v4) {};
    \node[goal-node](v8) at (v5) {};
    %
    \node[anchor=south east] at (v6.west) {$c^1_2$};
    \node[anchor=south west] at (v6.east) {$\hat{c}^1_2$};
    \node[anchor=south east] at (v4.west) {$c^2_2$};
    \node[anchor=south west] at (v4.east) {$\hat{c}^2_2$};
    \node[anchor=south east] at (v5.west) {$c^3_2$};
    \node[anchor=south west] at (v5.east) {$\hat{c}^3_2$};
    %
    \node[goal-node](v9) at ($(v2) + (-1.0, 2.0)$) {$\hat{c}^1_2$};
    \node[goal-node, right=0 cm of v9](v10) {$\hat{c}^2_2$};
    \node[goal-node, right=0 cm of v10](v11) {$\hat{c}^3_2$};
    %
    \draw[line,->] (v1) to[out=60,in=180] (v3);
    \draw[line-c3] (v1) to[out=0,in=180] (v4);
    \draw[line,->] (v1) to[out=-60,in=180] (v5);
    \draw[line,->] (v3) to[out=0,in=120] (v2);
    \draw[line,->] (v4) to[out=0,in=180] (v2);
    \draw[line,->] (v5) to[out=0,in=-120] (v2);
    %
    \draw[line,->] (v2) to[out=95,in=-60] (v9);
    \draw[line,->] (v2) to[out=92,in=-70] (v10);
    \draw[line,->] (v2) -- (v11);
  \end{tikzpicture}
  \caption{Example of \textit{clause constrainer} without multiple edges.
   Used in the proof of Lemma~\ref{lemma:deadlock-np-comp}.
  }
  \label{fig:3-sat-multiple-edges}
\end{figure}
