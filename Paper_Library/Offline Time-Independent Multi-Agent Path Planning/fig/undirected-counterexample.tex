\begin{figure}[t]
  \centering
  \begin{tikzpicture}
    \newcommand{\edgesize}{0.3cm}
    \newcommand{\edgesizehalf}{0.15cm}
    \newcommand{\edgesizethreetwo}{0.35cm}
    \newcommand{\longedgesize}{1.2cm}
    \tikzset{
      satnode/.style={vertex, minimum size=0.15cm, fill=black},
      start-node/.style={vertex, minimum size=0.33cm,rectangle},
      goal-node/.style={vertex, minimum size=0.33cm}
    }
    \tiny
    {
      \node[satnode, label=above:{\small $u$}](v1) at (0.0, 0.0) {};
      \node[satnode, right=\edgesize of v1](v2) {};
      \node[start-node, right=\edgesize of v2](v3) {$z_2$};
      \node[satnode, right=\edgesize of v3](v4) {};
      \node[satnode, right=\edgesize of v4,label=above:{\small $v$}](v5) {};
      \node[start-node, above=\edgesize of v2](v6) {$z_1$};
      \node[goal-node, below=\edgesize of v4](v7) {$z_1$};
      \node[goal-node, below=\edgesize of v7](v8) {$z_2$};
      %
      \foreach \u / \v in {v1/v2,v2/v3,v3/v4,v4/v5,v6/v2,v4/v7,v7/v8}
      \draw[line] (\u) -- (\v);
      \draw[line, densely dotted] (v1) -- ($(v1) + (-0.5, 0.0)$);
      \draw[line, densely dotted] (v5) -- ($(v5) + (0.5, 0.0)$);
      %
      \node[start-node, label=right:{start}](label-s) at (3.0, -0.4){};
      \node[goal-node, below=0.1cm of label-s, label=right:{goal}](label-g) {};
    }
    {
      \draw[line,->,orange]
      ($(v6)+(-0.1,-0.1)$) --
      ($(v2)+(-0.1, 0.1)$) --
      ($(v1)+(-0.3, 0.1)$) --
      ($(v1)+(-0.3, -1.8)$) --
      ($(v5)+(0.3, -1.8)$) --
      ($(v5)+(0.3, 0.05)$) --
      ($(v1)+(-0.25, 0.05)$) --
      ($(v1)+(-0.25, -1.75)$) --
      ($(v5)+(0.25, -1.75)$) --
      ($(v5)+(0.25, -0.05)$) --
      ($(v4)+(-0.05, -0.05)$) --
      ($(v7)+(-0.05, 0.2)$);
      %
      \draw[line,->,cyan]
      ($(v3)+(-0.05,-0.1)$) --
      ($(v1)+(-0.2, -0.1)$) --
      ($(v1)+(-0.2, -1.7)$) --
      ($(v5)+(0.2, -1.7)$) --
      ($(v5)+(0.2, -0.1)$) --
      ($(v4)+(0.05, -0.1)$) --
      ($(v8)+(0.05, 0.2)$);
    }
    % to beautify
    {
      \node[start-node, above=\edgesize of v2](v6) {$z_1$};
      \node[start-node, right=\edgesize of v2](v3) {$z_2$};
      \node[goal-node, below=\edgesize of v4](v7) {$z_1$};
    }
  \end{tikzpicture}
  \caption{
    Counterexample of \textit{oneway constrainer} without assmptions of simple paths.
    $z_2$ always arrives at its goal before the arrival of $z_1$ if those two agents follow colored lines.
  }
  \label{fig:undirected-counterexample}
\end{figure}
