{
  \newcommand{\edgesize}{0.2cm}
  \newcommand{\scaleratio}{1.0}
  \newcommand{\plotbody}{
    \node[agent1](v1) at (0.0, 0.0) {$i$};
    \node[vertex, right=\edgesize of v1](v2) {};
    \node[vertex, right=\edgesize of v2](v3) {};
    %
    \draw[line,->,color={rgb:red,0;green,1;blue,3},very thick] (0.22, 0.15) -- (2.6, 0.15);
    %
    \node[agent2, right=\edgesize of v3](v4) {$j$};
    \node[vertex, right=\edgesize of v4](v5) {};
    \node[vertex](v6) at (1.1, -0.7) {};
    \node[vertex, right=\edgesize of v6](v7) {};
    %
    \foreach \u / \v in {v1/v2,v2/v3,v3/v4,v4/v5,v7/v4,v3/v6,v3/v7}
    \draw[line] (\u) -- (\v);
  }
  \begin{figure}[tb!]
    \centering
    \begin{tabular}{cc}
    \scalebox{\scaleratio}{
      \begin{minipage}{0.45\linewidth}
        \centering
        \begin{tikzpicture}
          \plotbody
          \draw[line,->,color={rgb:red,3;green,1;blue,0},very thick]
          (2.0, 0.05) -- (1.4, 0.05) -- (1.0, -0.6);
        \end{tikzpicture}
      \end{minipage}
      }
      &
      \scalebox{\scaleratio}{
        \begin{minipage}{0.45\linewidth}
          \centering
          \begin{tikzpicture}
            \plotbody
            \draw[line,->,color={rgb:red,3;green,1;blue,0},very thick]
            (2.1, -0.25) -- (1.9, -0.65) -- (1.5, -0.1) -- (1.15, -0.7);
          \end{tikzpicture}
        \end{minipage}
        }
    \end{tabular}
    \caption{
      Example of OTIMAPP.
      A graph is depicted with black lines.
      Two agents ($i$, $j$) and their paths are colored.
      \textit{left}: Both agents stop progression permanently due to mutual exclusion (i.e., no collision) if $i$ moved two steps before $j$ moves.
      \textit{right}: As long as each agent follows a respective path, both agents eventually reach their last vertex;
      these paths constitute an OTIMAPP solution.
    }
    \label{fig:example}
  \end{figure}
}
