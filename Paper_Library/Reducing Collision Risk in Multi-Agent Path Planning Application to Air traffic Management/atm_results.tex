
Based on the individual aircraft MDP model, we build an MDP congestion game for the air traffic plan over France on July $3^{rd}$, $2017$. Between timestamps $39000$ and $41000$, $75$ planes left the Paris airports CDG and ORY to various destinations as shown in Figure~\ref{fig:time_discretization}. The collision risks $D^i$ and $G^i$~\eqref{eqn:DG_prob} can be interpreted as standard aircraft radial/vertical separation and longitudinal separation~\cite{brooker2006longitudinal}, respectively. In our simulations, only $D^i$ increases congestion costs.

\textbf{Interval-based collision risk computation}. Since each aircraft's time stamp is unique, we compute the congestion for time intervals. As shown in Figure~\ref{fig:time_discretization}, aircraft whose time stamp fall into the interval $\textstyle [t_k, t_k + \Delta t_{cong})$ will contribute to the congestion in time interval $k$.  
% We choose $\Delta t_{cong}$ to be the largest interval that ensures all consecutive timestamps on each aircraft's original flight plan fall into distinct intervals. 
% \begin{figure}[ht!]
%     \centering
    
%     \label{fig:flight_plan}
% \end{figure}
\begin{figure}[ht!]
    \centering
    \includegraphics[width=0.45\columnwidth]{figures/collision_reduction.pdf}
    \
    \includegraphics[width=0.45\columnwidth]{figures/potential_value.pdf}
    \caption{Left: Collision risk as a function of Frank-Wolfe algorithm iteration. Right: Congestion cost $\textstyle \sum_{i,t,s,a}kD_{tsa}^i(x)$~\eqref{eqn:DG_prob} as a function of Frank-Wolfe algorithm iteration. }
    \label{fig:FW_output}
\end{figure}

\textbf{Results and discussion}. We build the individual MDP and the interval-based congestion costs with the following user-defined parameter values: $\Delta t_{int} = 300$, $B = 3$, $\beta = 0.95$, $\alpha_f = 10$, $c_{tardy} = 2$, $\Delta t_{cong} = 19$, and $k = 10$. First, we verify that when solved without congestion cost $D^i$, all individual MDPs result in expected trajectories that match the original flight plan. The results are shown in Figure~\ref{fig:time_discretization}. We then define collision risk as $\sum_{a}x^i_{tsa}D^i_{tsa}(x)$, and found that for multiple flights, the maximum collision risk at any time was greater than $10\%$. The overall spread of collision risks for the original flight plan is shown on the  $x = 10^0$ line in Figure~\ref{fig:FW_output} left. We then augment individual costs with congestion cost $D^i$ and solve for the Nash equilibrium via the Frank Wolfe algorithm from~\cite[Alg.1]{li2022congestion}. The resulting collision risks and objective values are shown in Figure~\ref{fig:FW_output}. In the right figure, we see that the objective value decreases from $2200$ to $1900$ within the first $50$ iterations. Accompanying this, we observe that the maximum collision risks drops from $94\%$ to around $3\%$ within the first $10$ iterations of the Algorithm. Therefore, we conclude that our model was effective in reducing uncertainty-induced collision risks.
% Further work is needed to investigate its effect on flight arrival times. 
