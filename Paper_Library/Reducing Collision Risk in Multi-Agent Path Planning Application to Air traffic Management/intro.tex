\section{Introduction}
In robotics, aeronautics and warehouse logistics~\cite{ragi2013uav,trinh2020multi}, the operational dynamics are often inherently uncertain: delayed package arrivals may alter a warehouse's internal logistics, and quad-copters may be blown off its intended path by strong gusts of wind. 
% For autonomous vehicles, sampled-based and Markov decision process-based stochastic path planning methods allows for successful autonomous task completion despite the environmental uncertainty. 
In large-scale autonomy frameworks such as urban air mobility and automated warehouses, vehicles also experience \emph{congestion}---individual autonomous vehicles crowding the shared operational space.  Congestion can severely reduce system performance and requires inter-vehicle coordination to resolve. 
% When we can forecast congestion based on a priori models and coordinate between decision makers, game models that extend single-agent stochastic path planning can improve the efficiency of individual agents and space usage. 
In~\cite{li2022congestion}, a potential game solution is proposed.
% to minimize vehicle congestion while ensuring individual task completion. 
Using a heuristic to estimate work floor congestion,~\cite{li2022congestion} showed that multiple robots can share a work space with reduced collision risks.

Under uncertain operation dynamics, collision risks will always exist. While~\cite{li2022congestion} proves that individual task completion is optimal with respect to the congestion heuristic, no guarantee can be made on individual vehicle's collision risks. This prevents the adaptation of these path coordination games in safety-critical applications such air traffic management, in which government regulations require strict safety guarantees. To mitigate the lack of safety guarantee,  we consider an extension of the potential game model introduced in~\cite{li2022congestion,calderone2017markov} that directly minimizes the exact collision risk. By doing so, we can produce more rigorous guarantees on the individual vehicle's collision risks.

\textbf{Contributions}. Under the MDP congestion game framework~\cite{li2022congestion,calderone2017markov}, we propose a congestion model that directly weighs an autonomous vehicle's desire for task completion against its willingness to risk collisions. 
We show that this congestion cost has a potential function, so that the optimal multi-vehicle trajectory is the solution of a non-convex optimization problem with a multi-linear objective. 
% We show that at Nash equilibrium, each agent is optimally completing individual tasks given their opponents' strategies and their willingness to take collision risks. 
We develop an in-depth game model of air traffic management using historical flight plans from the French air space and show that the Frank-Wolfe gradient descent method can find locally optimal solutions where individual flight's collision risk drops both in occurrence and in likelihood. 

% \textbf{Motivation}. We are motivated by two existing bottlenecks in air traffic management: 1) disruptive weather events, and 2) air sector capacity constraints. These two bottleneck features are common across traffic management of aircraft and unmanned aerial vehicles, and result in inefficient usage of airspaces and undesirable delays for customers. Based on the upward trend of flight-based traveling as predicted by SESAR(CITE), their effects can only worsen in the foreseeable future. 


% Both disruptive weather events and sector capacity constraints are fundamentally related to congestion. During a disruptive weather event, a section of the airspace is rendered inoperable, and planes scheduled through such regions must be re-routed or diverted, creating congestion in neighboring airspaces. On the other hand, existing air traffic management infrastructure requires manual collision avoidance performed by air traffic controllers (ATCs).
% % Each ATC monitors a section of the airspace, denoted as an air sector, and routes aircraft that enter its sector accordingly to maximize safety of each aircraft. 
% However, since this task can only be performed by individual ATCs with limited focus capacity, airspaces have an upper limit of the number of aircraft they can safely contain within any time period. As a result, airspace congestion levels need to be minimized.

% \textbf{Problem Statement}. To resolve these bottlenecks, we propose a real-time re-routing algorithm for the tactile stage of en-route air traffic management. Our algorithm utilizes a game model that maximizes individual aircraft performance metrics and minimizes airspace congestion. 

% \textbf{Challenges}. In the development of our re-routing algorithm, we focus on addressing the following challenges inherent in air traffic management.
% \begin{enumerate}
%     \item\textbf{Unpredictability}: airspaces experience both environmental uncertainty and human-induced uncertainty. Although flights theoretically follow pre-determined schedules, weather events, airport delays, and aircraft malfunction all create uncertainty that result aircraft trajectories having stochastic dynamics. Furthermore, ATCs add a further layer of uncertainty in aircraft trajectory by manually regulating the airspace. In our model, we break down the total stochastic effects into \emph{local stochastic events} to optimally perform aircraft re-routing.
%     \item \textbf{Safety}: aircraft crashes result in significant human losses. We focus on two main safety constraints from a high-level system perspective: sector capacity and in trail separation. Sector capacity is one of our main motivations and correlates with the limited focus span of individual ATCs, while in trail separation refers to the ensuring that each aircraft is sufficiently separated from other aircraft. Although in trail separation is really the responsibility of ATCs, we note that it is also important to consider during routing to minimize the possiblility of in trail collision during execution.
%     \item \textbf{Scalability}: Since the global airspace is connectted, the air traffic management is ultimately a large-scale multi-agent trajectory planning problem spanning the globe. In our solution, we focus on developing scalable algorithm that efficiently generates globally optimal solutions. Specifically, we adopt a distributed algorithm that considers flight-specific subsets of the global airspace individually during trajectory generation, but combine resulting trajectories to compute and reduce congestion. 
% \end{enumerate}

% \tc{Emerald}{Include the flight schedule/pre-tactical/tactical plan}
