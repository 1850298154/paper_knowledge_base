\documentclass{article}

% \usepackage{corl_2022} % Use this for the initial submission.
\usepackage[final]{corl_2022} % Uncomment for the camera-ready ``final'' version.
% \usepackage[preprint]{corl_2022} % Uncomment for pre-prints (e.g., arxiv); This is like ``final'', but will remove the CORL footnote.
\usepackage{amsthm,amsmath,amsfonts, dsfont, hyperref,graphicx}
\usepackage[dvipsnames]{xcolor}
\newcommand{\N}{\mathbb{N}}
\newcommand{\Z}{\mathbb{Z}}
\newcommand{\R}{\mathbb{R}}
\newcommand{\any}{\text{ $\forall$ }}
\newcommand{\e}{\text{e}}
\newcommand{\E}{\mathcal{E}}
\newcommand\m[1]{\begin{bmatrix}#1\end{bmatrix}}

\newcommand{\X}{\mathcal{X}}
\newcommand{\U}{\mathcal{U}}
\newcommand{\D}{\mathcal{D}}
\renewcommand{\int}{\textbf{int }}
\newcommand{\B}{\mathbb{B}}
\newcommand{\F}{\mathcal{F}}
\newcommand{\T}{\mathcal{T}}
\newcommand{\V}{\mathcal{V}}
\renewcommand{\P}{\mathcal{P}}
\newcommand{\Q}{\mathcal{Q}}
\newcommand{\Y}{\mathcal{Y}}
\renewcommand{\S}{\mathcal{S}}
\newcommand{\M}{\mathcal{M}}

\renewcommand{\u}{\textbf{u}}
\newcommand{\x}{\textbf{x}}

\newcommand{\todo}[1]{\textcolor{red}{#1}}

\newtheorem{proposition}{Proposition}
\newtheorem{definition}{Definition}

\newcommand{\revision}[1]{\textcolor{black}{#1}}

\newcommand{\newrevision}[1]{\textcolor{black}{#1}}
\newcommand{\tc}[2]{\textcolor{#1}{#2}}
\title{Reducing Collision Risk in Multi-Agent Path Planning: Application to Air Traffic Management}
% The \author macro works with any number of authors. There are two
% commands used to separate the names and addresses of multiple
% authors: \And and \AND.
%
% Using \And between authors leaves it to LaTeX to determine where to
% break the lines. Using \AND forces a line break at that point. So,
% if LaTeX puts 3 of 4 authors names on the first line, and the last
% on the second line, try using \AND instead of \And before the third
% author name.
% NOTE: authors will be visible only in the camera-ready and preprint versions (i.e., when using the option 'final' or 'preprint'). 
% 	For the initial submission the authors will be anonymized.
\author{
  Sarah H.Q. Li\\
  Department of Aeronautics and Astronautics Engineering\\
  University of Washington, Seattle, United States\\
  \texttt{sarahli@uw.edu} \\
  %% examples of more authors
  \AND
  Avi Mittal \\
  Department of Aeronautics and Astronautics Engineering\\
  University of Washington, Seattle, United States \\
  \texttt{avim@uw.edu} \\
   \And
   Pierre-Loïc Garoche \\
   École Nationale de l'Aviation Civile \\
   Université de Toulouse, Toulouse, France \\
   \texttt{Pierre-Loic.Garoche@enac.fr} \\
   \And 
    A{\c{c}}{\i}kme{\c{s}}e, Beh{\c{c}}et\\
   Department of Aeronautics and Astronautics Engineering \\
   University of Washington, Seattle, United States \\
   \texttt{behcet@uw.edu} \\
  %% \And
  %% Coauthor \\
  %% Affiliation \\
  %% Address \\
  %% \texttt{email} \\
}
\begin{document}
\maketitle
%===============================================================================
\begin{abstract}
    To minimize collision risks in the multi-agent path planning problem with stochastic transition dynamics, we formulate a Markov decision process congestion game with a multi-linear congestion cost. Players within the game complete individual tasks while minimizing their own collision risks. We show that the set of Nash equilibria  coincides with the first order KKT points of a non-convex optimization problem. Our game is applied to a historical flight plan over France to reduce collision risks between commercial aircraft. 
\end{abstract}
% Two or three meaningful keywords should be added here
\keywords{Markov decision process, game theory, air traffic management} 
%===============================================================================
\section{Introduction}

One of the most fundamental problems in combinatorial optimization is the traveling salesperson problem (TSP), formalized as early as 1832 (c.f. \cite[Ch 1]{ABCC07}).
In an instance of  TSP we are given a set of $n$ cities $V$ along with their pairwise symmetric distances, $c:V\times V \to\R_{\geq 0}$. The goal is to find a Hamiltonian cycle of minimum cost. In the metric TSP problem, which we study here, the distances satisfy the triangle inequality. Therefore, the problem is equivalent to finding a closed Eulerian connected walk of minimum cost.%\footnote{Given such an Eulerian cycle, we can use the triangle inequality to shortcut vertices visited more than once to get a Hamiltonian cycle.}

It is NP-hard to approximate TSP within a factor of $\frac{123}{122}$ \cite{KLS15}.  An algorithm of Christofides-Serdyukov~\cite{Chr76,Ser78} from four decades ago gives a $\frac32$-approximation for TSP.
Over the years there have been numerous attempts to improve the Christofides-Serdyukov algorithm and exciting progress has been made for various special cases of metric TSP, e.g., \cite{OSS11,MS11,Muc12,SV12,HNR21, KKO20, HN19, GLLM21}.
 Recently, ~\cite{KKO21} gave the first improvement for the general case by demonstrating that the so-called ``max entropy" algorithm of \cite{OSS11} gives a randomized $\frac{3}{2}-\epsilon$ approximation for some $\epsilon > 10^{-36}$.% (see \cite{VS20} for a historical note about TSP)

%After a long line of work %~\cite{Wol80,SW90,BP91,Goe95,CV00,GLS05,BM10,BC11,SWV12, HNR17,HN19, KKO20a} 
	%the best known approximation algorithm for the general case of the problem is $\frac{3}{2}-\epsilon$ for some $\epsilon > 10^{-36}$ due to ~\cite{KKO21}, a result that built upon the work of the third author, Saberi, and Singh ~\cite{OSS11}. 
	The method introduced in \cite{KKO21} exploits the optimum solution to the following linear programming relaxation of metric TSP studied by \cite{DFJ59,HK70,BG93}, also known as the subtour elimination LP:
\begin{equation}\label{eq:tsplp}
\begin{aligned}
	\min \quad& \sum_{u,v} x_{\{u,v\}} c(u,v)& \\
	\text{s.t.,} \quad &  \sum_{u} x_{\{u,v\}} = 2&\forall v\in V,\\
	& \sum_{u\in S, v\notin S} x_{\{u,v\}}\geq 2,&\forall S \subsetneq V, S\not= \emptyset\\
	& x_{\{u,v\}}\geq 0 &\forall u,v\in V.
\end{aligned}	
\end{equation} 
	
	 However, ~\cite{KKO21} did not show that the integrality gap of the subtour elimination polytope is bounded below $\frac{3}{2}$, and therefore did not make progress towards the ``4/3 conjecture" which posits that the integrality gap of LP \eqref{eq:tsplp} is $\frac{4}{3}$. In this work we remedy this discrepancy by proving the following theorem, improving upon the bound of $\frac{3}{2}$ from Wolsey~\cite{Wol80} in 1980:

\begin{theorem}\label{thm:main}
	Let $x$ be a solution to LP \eqref{eq:tsplp} for a TSP instance. For some absolute constant $\epsilon > 10^{-36}$, the \hyperlink{tar:alg}{max entropy algorithm} outputs a TSP tour with expected cost at most $\frac{3}{2}-\epsilon$ times the cost of $x$. Therefore the integrality gap of the subtour elimination LP is at most $\frac{3}{2} - \epsilon$. 
\end{theorem} 

To prove \cref{thm:main}, we amend Section 4 of \cite{KKO21} but keep the remainder of the analysis essentially the same. Unlike \cite{KKO21}, this argument now preserves the integrality gap by avoiding the use of the optimum solution in bounding the cost of the matching. See \cref{sec:overview} for a discussion of our new approach.
%We note that the analysis in this paper is not specialized to the max entropy algorithm (although we rely on many results from \cite{KKO21} to obtain \cref{thm:main} itself); instead, it is valid for any algorithm which samples a spanning tree from the support of a solution to LP \eqref{eq:tsplp} and then adds the minimum cost matching on the odd degree vertices of the tree.  
%Instead, we use the polygon representation of near minimum cuts \cite{Ben95,BG08} to bound  the cost of the matching (see the following section for an overview of our new findings). %An added benefit of avoiding the use of OPT in the analysis is  %We remark this makes the analysis constructive 
%We remark that this allows future analyses to explicitly compute and possibly utilize the relevant laminar family of near minimum cuts (whereas previously one needed to know OPT to find the laminar family used in the analysis in \cite{KKO21}).
%In particular, we show that to get a bound better than $\frac{3}{2}$ for this class of algorithm it is (essentially) sufficient to handle the case in which the near minimum cuts of $x$ are a laminar family.

\subsection{Other Consequences}
\paragraph{Path TSP} In recent exciting work, Traub, Vygen, Zenklusen \cite{TVZ20} showed that an $\alpha$-approximation algorithm for metric TSP can be used as a black box to get a $\alpha(1+\eps)$ approximation algorithm for Path TSP. This together with \cite{KKO21} implies that there is a $3/2-\eps$ approximation algorithm for Path TSP (for $\eps>10^{-36}$). On the other hand, it is a folklore result that the integrality gap of the natural LP relaxation of Path TSP is at least $3/2$.  Therefore, a consequence of the above theorem is that although the best possible approximation factors of the two problem are the same (up to polynomial reductions), the natural LP relaxation of metric TSP has a strictly smaller integrality gap.


\paragraph{2-ECSM} In the 2-edge-connected multi-subgraph problem, or 2-ECSM for short, we are given a weighted graph $G$ and we want to find a minimum cost 2-edge-connected spanning subgraph, where an edge can be chosen multiple times.
The classical Christofides-Serdyukov algorithm gives a 3/2-approximation for 2-ECSM and despite significant attempts \cite{CR98,BFS16,SV14,BCCGISW20} improved algorithms were designed only for special cases of the problem.
Since in \cite{BG93} it is shown that LP \eqref{eq:tsplp} is a valid relaxation for 2-ECSM, we obtain:

\begin{corollary}	
For some absolute constant $\epsilon > 10^{-36}$ the \hyperlink{tar:alg}{max entropy algorithm} is a randomized $\frac{3}{2}-\epsilon$ approximation for the 2-edge-connected multi-subgraph problem.
\end{corollary}
%Beyond these theorems, we believe the analysis in this paper will open new avenues to improve the arguments in ~\cite{KKO21}. The analysis in that work is by nature non-constructive because it uses information about the optimal solution. Here we remove this weakness and could in principle construct the proposed fractional matching in polynomial time. Although of course this has no practical benefit since the algorithm always finds the minimum cost matching, this may allow future works to manipulate the algorithm to better serve the analysis.

%We analyze the max-entropy rounding algorithm introduced in \cite{OSS11} and slightly modified in \cite{KKO20, KKO21}. 

%In other words, we design a feasible vector for the $O$-join polytope to ``satisfy'' all near min cuts ``crossed on both  sides'' 


%Whereas Section 4 of ~\cite{KKO21} only deals with the near minimum cuts of $x$ (where $x$ is a solution to LP \eqref{eq:tsplp}) which lie along the optimal Hamiltonian cycle, we deal with all near minimum cuts of $x$ using the so-called polygon representation of near minimum cuts ~\cite{Ben97,BG08}. %The results give new intuition for the structure of cuts that are within $\frac{6}{5}$ or less of the edge connectivity of the graph.

 %: we show that we can incur a cost of $O(\eta^2) \cdot c(x)$ to ensure that the set of cuts with $x(\delta(S)) \le 2+\eta$ is a laminar family.


\subsection{New techniques and contributions}\label{sub:newtechniques}

This paper can be seen as a case study on how to reason about and deal with {\em near} minimum cuts. One can deduce from the classical cactus representation of a graph $G$ \cite{DKL76} (i) the structure of {\em all} min cuts of $G$ and (ii) the structure of the edges of $G$ in the sense that every edge $\{u,v\}$ maps to a unique {\em path} in the cactus between the images of $u$ and $v$. Furthermore, such a path intersects every cycle of the cactus on at most one cactus edge. The theory has found many application from designing fast algorithms
\cite{Kar00,KP09} to the analysis of approximation algorithms for TSP \cite{KKO20} and connectivity augmentation \cite{BGJ20,CTZ21}.

Two decades later, the theory of min cuts was extended to near min cuts in works of Bencz\'ur and Goemans \cite{Ben95, BG08} where they introduced the polygon representation which represents all cuts of a graph with at most $\frac{6}{5}k$ edges, where $k$ is its edge connectivity. Although these works completely classify the structure of all near min cuts of a given graph $G$, they do not characterize the structure of the \textit{edges} of $G$ with respect to these cuts, which can be important in applications (for example, in many of the recent applications of min cuts,
 one also needs to exploit the structure of the edges in relation to the cactus).
The structure on the edges turns out to be highly relevant in this work as well, and as a byproduct of our analysis we make progress towards classifying the way in which the edges of $G$ relate to the structure of the polygon representation.
 
 % and (to some extent) a classification of the set of edges of $G$ with respect to the polygon representation of Bencz\'ur and Goemans.
 
  %i
 %s to give a better understanding of the structure of edges of $G$ with respect to its near min cuts.

  %One can partition the edges of $G$ into sets $F_1\dots,F_m$ such that the set of edges in every min cut $(S,\overline{S})$ of $G$ is the union of edges in a pair $F_i,F_j$ for $i\ neq j$.
%\Nathan{Shayan can add something} For example...

For motivation, consider a generic family of network design problems in which we want to construct a network such that every pair $u,v$ of vertices has connectivity at least $c_{u,v}$. A natural approach is to write an LP relaxation to find a (minimum cost) vector $x: E \to \R_{\ge 0}$ such that for every cut $S$ separating $u$ and $v$, $x(\delta(S))\geq c_{u,v}$. We can round this LP using independent rounding or a dependent rounding scheme such as sampling from max entropy distributions. Using classical concentration bounds one can show that if $x(\delta(S))\gg c_{u,v}$ then with high probability the rounded solution has at least $c_{u,v}$ edges across this cut. So the main challenge is to ``fix'' near tight cuts, i.e., cuts where $x(\delta(S))\approx c_{u,v}$.  For an explicit instantiation of this scheme see \cite{KKOZ22}. A better understanding of the global structure of the family of near tight cuts has the potential to significantly simplify or even improve the approximation factor of such rounding algorithms. A classical technique to design algorithms for such network design problems is to apply uncrossing to extreme point solutions of the LP. One can view our contribution as an approximate uncrossing technique that deals with all near tight cuts (instead of just tight cuts) as we explain next.
%Next, we explain how our main theorem can be used to give global structure for near tight cuts in the case that $c_{u,v}=2$ for all $u,v$ and we contrast it with the classical uncrossing technique which only deals with tight/min cuts. 


\paragraph{An Approximate Uncrossing Technique.} A fundamental technique in the field of approximation algorithms is the uncrossing technique\footnote{See e.g. \cite{LRS11} for a number of applications of this technique.} of Jain \cite{Jai01}. Given a graph $G=(V,E)$,  a weight vector $x:E\to\R_{\geq 0}$, and a  function $f:V\to\R$, suppose that $x(\delta(S))\geq f(S)$ for all $S\subseteq V$. Let $\cN$ be the family of sets $S$ such that $x(\delta(S)) = f(S)$, i.e., the family of {\em tight} sets with respect to $f$. The uncrossing technique says that if $f$ is (weakly) supermodular then we can refine $\cN$ to a laminar family of sets, $\cH$, such that if all sets of $\cH$ are tight, then all sets of $\cN$ are tight as well. For a concrete example, suppose $f$ is a constant function, say $f(S)=2$ for all $\emptyset\subsetneq S\subsetneq V$. Then, sets of $\cH$ can be constructed using the cactus representation \cite{DKL76} of cuts in $\cN$. The significance of this method is that if $x$ is a basic feasible solution to a LP with constraints $x(\delta(S))\geq f(S)$ for all $S$, one can use this machinery to argue that the support of $x$ has size $O(|V|)$.

Informally, we prove the following, which 
can be seen as  an {\em approximate uncrossing technique}: 
\begin{theorem}[Informal]\label{thm:uncrossing}Suppose we have a vector $x:E\to\R_{\geq 0}$ such that $x(\delta(S))\geq f(S)$ for all $S$; define $\cN$ to be sets $S$ where $x(\delta(S))\leq f(S)(1+\eps)$ for some fixed $\eps>0$. If $f(.)$ is constant, say $f(S)=2$ for all $S$, then there is a set $\cN^*\subseteq \cN$ and a collection of edge sets $F_1,\dots,F_m\subseteq E$ such that the following hold:
\begin{itemize}
	\item $|\cN^*|= O(|V|)$, $m= O(|V|)$.
	\item $x(F_i)\geq 1-\eps/2$ for all $1\leq i\leq m$.
	\item Every edge $e$ is in at most $O(1)$ of the $F_i$'s.
	\item For every set $S\in \cN\smallsetminus \cN^*$ there exists $1\leq i<j\leq m$ such that $F_i\cap F_j=\emptyset$ and $F_i\cup F_j\subseteq \delta(S)$ and for every $S\in \cN^*$, there exists $1\leq i\leq m$ such that $F_i\subseteq \delta(S)$. 
\end{itemize}
\end{theorem}
In words, although we cannot simply refine $\cN$ to a linear number of sets, we can refine the edges in cuts of $\cN$ to a linear number of sets $F_1,\dots, F_m$ such  that we can essentially capture the edges of $\delta(S)$ for any $S\in \cN\smallsetminus \cN^*$ by a pair of disjoint $F_i$'s. We give a slightly weaker condition for cuts in $\cN^*$; namely we only capture half of their edges by $F_i$'s.

\begin{example}For a simple example of the above theorem, suppose $\eps=0$, i.e. $\cN$ is the set of min cuts of a graph $G$. Furthermore, suppose that every proper  cut in $\cN$ is \hyperlink{tar:crossing}{crossed} (recall that $S$ is proper if $1<|S|<|V|-1$) and that $\cN$ has at least one proper cut. 
Then, one can use an uncrossing technique, namely that if $A,B\in \cN$ then $A\cap B\in \cN$, to prove that $G$ must be cycle, namely we can order vertices of $G$, $v_0,\dots,v_{n-1}$ such that $x_{\{v_i,v_{i+1\text{ mod n}}\}}=1$.
In such a case we let $\cN^*=\emptyset$ and $F_i=E(v_i,v_{i+1\text{ mod }n})$.
%partition $V$ into sets $a_0,\dots,a_{m-1}$ such that 
%Let $\C$ be a connected component of crossing cuts of $\cN$, namely, for any pair of sets $A,B\in \C$ there is a path of crossing cuts all from $\C$ that goes from $A$ to $B$.
% and further suppose that $\cN$ can be represented by a cycle $C$ in the sense every min cut of $\cN$ corresponds to a min cut of $C$ and vice versa. Here we assume $a_0,\dots,a_{m-1}$ are the nodes of $C$ where each $a_i$ is identified with a disjoint set of vertices where $V=\uplus_{i=1}^m a_i$. In such a case, we can simply let $\cN^*=\emptyset$ and $F_i=E(a_i,a_{i+1\text{ mod }m})$. 
\label{eg:cycle}\end{example}

\begin{example}\label{eg:laminar}
For a second example, suppose again $\eps=0$ and $\cN$ is the set of mincuts of a graph $G$ where $\cN$ forms a laminar family (no two cuts cross). It turns out that we cannot decompose edges of cuts of $\cN$ into a linear sized collection of sets where every edge appears only a constant number of times. The main reason is that some edges may appear in an unbounded number of cuts. In this case we let $\cN^*=\cN$ and for every $A\in \cN$ (with immediate parent $B\in \cN$ in the laminar family) we add a set $F_A=\delta(A)\smallsetminus \delta(B)$  to our collection.  It is straightforward to show, using the structure of min cuts, that $x(F_A)\geq 1$; furthermore, since the size of a laminar family is linear in $V$, this gives a valid decomposition in the sense of above theorem.
\end{example}
Lastly, if $\eps=0$ and $\cN$ is the set of min cuts of an arbitrary graph, one can represent all min cuts of $\cN$ by a cactus \cite{DKL76} which can be seen as a tree of cycles. In such a case, one can use a construction similar to \cref{eg:cycle} for each cycle where instead of a vertex $v_i$ we have a set $a_i \subseteq V$ and one similar to \cref{eg:laminar} for the tree part of the cactus. For a concrete application of such a decomposition of min cuts see \cite{KKO20}.
%More generally, if $\cN$ corresponds to the set of min cuts of an arbitrary graph, the cuts of $\cN$ can be represented by a {\em cactus graph}. In such a case we add one $F_i$ for every edge of a cycle of the cactus. 


%and further for simplicity assume that there is a single connected component of crossing cuts in $\cN$, namely we can go from any $A$ to $B$ for $A,B\in\cN$ simply following crossing cuts of $\cN$. Then, one can represent cuts in $\cN$ by the set of min cuts of a cycle, namely we can contract vertices of $G$ 

%For a concrete application , suppose we need at least two edges in every set in $\cN^*$, say in a network optimization problem. Then, if we make sure that we have at least one edge in each $F_i$, all typical cuts constraints, $\cN\smallsetminus \cN^*$,  are satisfied, so we  reduce the problem to cuts in $\cN^*$. 


One of the main challenges in dealing with near min cuts relative to min cuts is that if $x(\delta(A)),x(\delta(B))\leq 2+\eps$ then $x(\delta(A\cap B))\leq 2+2\eps$. Therefore, if $\eps=0$, then min cuts are closed under intersection, set difference and union, but this is no longer true when $\eps>0$. So, to employ the classical uncrossing machinery one should be very careful to "uncross" only a constant number of times (independent of $\eps$) to make sure that every cut remains within $2+O(\eps)$. This is the main reason that the polygon representation of near min cuts (see below) is more sophisticated, e.g., we can no longer argue $x(E(a_i, a_{i+1}))\approx 1$, see \cref{fig:nearmincutbadexample}.

Although we don't study it here, we believe it may be worthwhile to find generalizations of \cref{thm:uncrossing} which hold for any (weakly) supermodular function.% That could be helpful in many questions based on the network optimization framework of Jain \cite{Jai01}.

\begin{remark} 
 We do not explicitly prove \cref{thm:uncrossing} in this extended abstract, as it is not used to prove \cref{thm:main}. However it can be deduced from arguments in \cref{sec:twoside} and \cref{app:oneside}. 
%In \cref{sec:overview} we discuss the main ideas of the proof of \cref{thm:uncrossing}. Here, let us explain the main challenge: In principal one might try to simply extend the above decomposition for the case $\eps=0$. The main challenge is that if $x(\delta(A)),x(\delta(B))\leq 2+\eps$ then $x(\delta(A\cap B))\leq 2+2\eps$. Therefore, if $\eps=0$, then min cuts are closed under intersection, set difference and union, but this is no longer true when $\eps>0$. So, to employ the classical uncrossing machinery one should be very careful to "uncross" only a constant number of times (independent of $\eps$) to make sure that every cut remains within $2+O(\eps)$. This is the main reason that the polygon representation of near min cuts (see below) is more sophisticated, e.g., we can no longer argue $x(E(a_i, a_{i+1}))\approx 1$, see \cref{fig:nearmincutbadexample}.
\end{remark}





\paragraph{Extensions to the Polygon Representation} To obtain our uncrossing framework we prove new properties of the polygon representation.
Given a graph $G=(V,E)$, let $k$ be the edge-connectivity of $G$, i.e. the number of edges in a minimum cut of $G$. For $\eps>0$, consider the set of $(1+\eps)$-near minimum cuts of $G$: cuts $(S,\overline{S})$ where $|E(S,\overline{S})| < (1+\eps)k$.
Bencz\'ur \cite{Ben95} and Bencz\'ur, Goemans \cite{BG08} proved that if $\eps \le 1/5$ then the near minimum cuts of $G$ admit a {\em polygon representation}. Namely, every connected component $\cC$ of \hyperlink{tar:crossing}{crossing} $(1+\eps)$ near min cuts can be represented by the diagonals of a convex polygon. In this polygon, the vertices of $G$ are partitioned into sets called \textit{atoms}, and every atom is mapped to a cell of this polygon defined by the diagonals and the boundary of the polygon itself (see \cref{sec:polyrep} for more details). 

The polygon representation can be seen as a generalization of the well-known cactus representation \cite{DKL76} of minimum cuts where a cycle of the cactus is replaced by a convex polygon. Unlike a cycle, some vertices/atoms map to the interior of the polygon, which are called ``inside'' atoms. The inside atoms at first look like a mystery and one can ask many questions about them such as how many can exist and what structures they can exhibit.



 Here, we explain two lemmas we proved which might find further applications beyond TSP in the future. 
%Our results give new intuition and understanding about the structure of polygon representations. These guide our analysis of the integrality gap of the subtour LP.
 %For example, one of our new observations is a 
 First, we give a necessary condition for a cell of a polygon to contain an inside atom:
\begin{lemma}[Informal, see \cref{thm:halfplanes}]
	Consider a polygon $P$ for a connected component $\C$ of a family of $1+\eps$ near min cuts for $\eps \le 1/5$ (where representing diagonals correspond to cuts in $\C$). Any cell of $P$ that has an inside atom must have at least $\Omega(1/\eps)$ many sides. 
\end{lemma}
This can be seen as a generalization of \cite[Lem 22]{BG08} to the case in which the cell is allowed to be adjacent to vertices of the polygon $P$.

Now, we explain our second extension: it follows from the cactus representation of minimum cuts that for a graph $G$ and a min cut $S$ one can partition the set of all min cuts that cross $S$ into two groups ${\cal A}=\{A_1,\dots,A_k\}$ and ${\cal B}=\{B_1,\dots,B_l\}$ for some $k,l\geq 0$ such that $S\cap A_1\subseteq S\cap A_2 \subseteq \dots S\cap A_k$ and, similarly, $S\cap B_1\subseteq \dots\subseteq S\cap B_l$. We prove a generalization of this fact for near min cuts:
\begin{lemma}[Informal, see \cref{lem:crosschain}]
Consider the set of $1+\eps$ near min cuts of a graph $G$ for $\eps\leq 1/10$; for any such near min cut $S$, one can partition the $1+\eps$ near min cuts crossing $S$ into two groups ${\cal A}=\{A_1,\dots,A_k\}$ and ${\cal B}=\{B_1,\dots,B_l\}$ such that $S\cap A_1 \subseteq S\cap A_2\subseteq \dots \subseteq S\cap A_k$ and similarly for cuts in ${\cal B}$.
\end{lemma}

\subsection{Outline of rest of paper} After reviewing preliminaries in \cref{sec:prelims}, we give a high-level overview of our proof technique in \cref{sec:overview}. The main new technical contributions of this paper are in \cref{sec:polyrep} and  \cref{sec:twoside}. The remaining content of the paper essentially follows from ~\cite{KKO21}. %Therefore, the reader may want to skip \cref{sec:proof-of-main}. 



%===============================================================================
\section{Markov decision process congestion game}
\textbf{Notation}. We use $\reals(\reals_+)$ to denote the real (non-negative real) numbers, $[N]$ to denote $\{1,\ldots N\}$, $\mc{T}$ to denote $\{0,\ldots,T\}$, and $\Delta_N = \{y \in \reals_+^N \ | \ \sum_{i} y_i = 1\}$ to denote the simplex in $\reals^N$. 
\subsection{Individual Markov decision process (MDP)}
The \emph{finite-horizon MDP} for player $i$ is given by $([S], [A], \mc{T}, P^i, C^i, p^i_0)$, where $[S]$ is the \textbf{finite set of states}, $[A] $ is the \textbf{finite set of actions}, $\mc{T}$ is the \textbf{finite time horizon}, $C^i \in \reals^{(T+1)SA}$ are the \textbf{state-action costs}, $P^i \in \reals_+^{TSSA}$ is the \textbf{transition dynamics}, and $p^i_0 \in \Delta_S$ is player's \textbf{initial state probability distribution}. Assume that each action $a \in [A]$ is admissible from each state $s \in [S]$. 

At time $t \in \mc{T}$ and state $s \in [S]$, player $i$ selects an action $a \in [A]$ and incurs a cost $C_{tsa} \in \reals$. At $t+1$, the player transitions to state $s'$ with probability $P_{ts'sa}\geq 0$. This is repeated for $t\in\mc{T}$. At $t=0$, player $i$'s probability of being in state $s$ is given by $p_{0s}$.

Player $i$'s \textbf{state-action distribution} is $x^i \in \reals_+^{(T+1)SA}$, where $x^i_{tsa}$ is player's joint probability of taking action $a$ at $(t,s) \in \mc{T}\times [S]$. The set of feasible MDP state-action distributions is given by 
\begin{equation}
\textstyle \mc{X}(P^i, p^i_0) = \big\{ z \in \reals_+^{(T+1)SA} \ | \ \sum_{a}z_{0sa} = p^i_{0s}, 
     \sum_{a}z_{(t+1)sa} = \sum_{a, s'}P^i_{tss'a}z_{ts'a}, \forall (t,s) \in \mc{T}\times[S]\big\}.
\end{equation}
% The standard \textbf{MDP policy} $\pi^i(t,s,a)$ can be derived from $x^i$ as $\textstyle  \pi(t,s,a) = $ $\textstyle \frac{x^i_{tsa}}{\sum_{a'}x^i_{tsa'}}$ if $\textstyle \sum_{a'}x^i_{tsa'} > 0$, and $\textstyle \pi(t,s,a) =1/A$ otherwise. 
% \textbf{Policy.} At time $t$ and state $s$, the player selects an action $a$ with probability $\pi(t,s,a)$,  where $\pi(t, s, \cdot) \in \Delta_A$; it outputs the decision maker's probability of taking action $a$, given a time $t$ and state $s$. Then $x_{tsa}$ becomes the player's joint probability at time $t$ of both being in state $s$ and taking action $a$ according to its policy:
% \[x_{tsa}  = p_{ts} \pi(t,s,a).\]
% The optimal policy for each decision maker is the one that minimizes the expected value of the cost-to-go. The policy returns a probability for each action instead of a single determined action because we wish to plan ahead for future actions, which must respond to the stochastic behavior of the dynamics and the other players.
% Each player's state-action distribution $x$ embeds its policy $\pi$, which can be explicitly derived as
% \begin{equation}
%     \pi(t,s,a) = \begin{cases}
%     \frac{x_{tsa}}{\sum_{a'}x_{tsa}} & \sum_{a'}x_{tsa} > 0 \\
%     0& \text{otherwise}
%     \end{cases}, \ \forall t, s,a \in [T]\times[S]\times[A].
% \end{equation}

% Then the probability that a player takes some state-action pair $(s,a)$ is the joint probability of the player both being in state $s$ and selecting action $a$ according to its policy, which yields the relationship \[x_{tsa}  = p_{ts} \pi(t,s,a).\]

Player $i$'s \textbf{Q-value function} $Q^i\in\reals^{(T+1)SA}$ is the expected incurred cost within the MDP~\cite[Chp.4.2.1]{puterman2014markov}. When player $i$ is at state $s$ and time $t$, $Q^i_{tsa}$ is the expected total cost player incurs from state $s$ and time $t$ if it first takes action $a$ and plays optimally thereafter.
\begin{equation}\label{eqn:q_value}
   \textstyle  Q^i_{Tsa} := C^i_{Tsa}, \ 
  \textstyle   Q^i_{(t-1)sa}:= C^i_{(t-1)sa} + \sum_{s'} P^i_{ts'sa}\underset{a'}{\min}\, Q^i_{t,s'a'}, \ 
\forall \  (t, s, a) \in [T]\times[S]\times[A]. %\underset{s'}{\sum}
\end{equation}
\subsection{Multi-player MDPs under collision risk-based congestion}
Inspired by autonomous vehicles sharing an operation space, we consider the scenario in which $N$ players each solve the MDP $([S],[A],\mc{T}, P^i, \ell^i, p^i_{0})$ for $i \in [N]$. Distinct from individual MDPs, the MDP costs $\ell^i: \reals^{N\times (T+1)SA} \mapsto \reals^{(T+1)SA}$ depend on all players' state-action distributions $(x^1, \ldots, x^N)$. We denote this joint state-action distribution as $x = (x^1, \ldots, x^N)$ and the resulting cost as $\ell^i(x)$. The players jointly solve an \textbf{MDP congestion game} under costs $(\ell^1,\ldots,\ell^N)$. 
% and the joint state-action distribution excluding player $i$ as $x^{-i}$, such that $x = (x^i, x^{-i})$. 

\textbf{Probabilistic collision risks}. A player's collision risk at $(t,s)$ and $(t,s,a)$ are the probabilities that at least one other player is in the same state and state-action, respectively. 
\begin{lemma}
 Under $x=(x^1,\ldots, x^N)$, player $i$'s probability of encountering at least one other player in $s$ and $(s,a)$ at time $t$ are respectively denoted by $D^i_{ts}(x)$ and $G^i_{tsa}(x)$ and computed as
 \begin{equation}
   \textstyle   D^i_{ts}(x)  = 1 - \prod_{j\neq i}(1 - \sum_{a'} x^j_{tsa'}), \ G^i_{tsa}(x) = 1  - \prod_{j\neq i}(1 -  x^j_{tsa}) \ \forall \ i, t, s, a \in [N]\times [T]\times [S]\times [A].\label{eqn:DG_prob}
 \end{equation}
\end{lemma}
\begin{proof}
 The probability of player $j$ taking state-action $(s, a)$ at time $t$ is $x^j_{tsa}$. The probability that player $j$ does \textit{not} take state-action $(s, a)$ at time $t$ is $1 - x^j_{tsa}$. The probability that \textit{none} of the players $j \neq i$ take state-action $(s, a)$ at time $t$ is $\textstyle \prod_{j \neq i} (1 - x^j_{tsa})$. The probability of \textit{at least one} other player $j \neq i$ taking state-action $(s, a)$ at time $t$ is given by $G^i_{tsa}(x)$ in~\eqref{eqn:DG_prob}. To derive $\textstyle D^i_{ts}(x)$~\eqref{eqn:DG_prob}, we apply similar arguments to the probability of player $j$ being in state $s$ at time $t$, given by $\textstyle \sum_a{x^j_{tsa}}$.
\end{proof}
% Both $D_{ts}$ and $G_{tsa}$ have non-trivial collision risk interpretations. 
As shown in Section~\ref{sec:flight_model}, $D^i$ and $G^i$ are flight separation constraints in air traffic management.

\textbf{Collision risk-based congestion}. We augment players' individual costs $C^i$ with $D^i(x)$ and $G^i(x)$. \begin{equation}\label{eqn:congestion_cost}
    \ell^i_{tsa}(x) = C^i_{tsa} + k\big(D^i_{ts}(x) + G^i_{tsa}(x)\big), \ \forall (t,s,a) \in \mc{T}\times[S]\times[A],
\end{equation}
where $k \in \reals_+$ is a user-defined parameter that signifies the players' willingness to risk collisions. Players are collision ignorant at $k=0$, and collision-averse at $k\rightarrow\infty$. Unique from~\cite{calderone2017markov,li2022congestion}, $\ell^i$~\eqref{eqn:congestion_cost} is independent of $x^i$; when player $i$'s opponents fix their strategies, player $i$ solves a standard MDP. 

When all players simultaneously achieve the minimum $Q^i(x)$, $x$ is a Nash equilibrium. 
\begin{definition}[Nash equilibrium]\label{def:NE}\cite{li2022congestion}
The state-action distribution $x$ is a Nash equilibrium if every player exclusively takes actions that minimize their $Q$-value function, $Q^i(x)$~\eqref{eqn:q_value}. 
% every player's policy is optimal with respect to their own MDP.
\begin{equation}\label{eqn:nash}
   \textstyle  x^i_{tsa}> 0 \Rightarrow a \in \argmin\{Q^i_{tsa'}(x)\ | \ a' \in [A]\}, \ \forall (i, t, s, a) \in [N] \times [T]\times [S]\times [A]. 
\end{equation}
\end{definition}
We consider solving for the Nash equilibrium using the potential game formulation given by
% Under congestion costs of the form~\eqref{eqn:congestion_cost}, the Nash equilibrium condition~\eqref{eqn:nash} is equivalent to the first order KKT conditions of the optimization problem given by 
\begin{equation}\label{eqn:potential_game}
\begin{aligned}
  \textstyle   \min_{x^1,\ldots, x^N} & F(x^1,\ldots, x^N) & \text{s.t.} \  x^i \in \mc{X}(P^i, p^i), \ \forall \ i \in [N],
\end{aligned}
\end{equation}
where the objective  $F:\reals_+^{N\times(T+1)\times S\times A}\mapsto \reals$ is defined as 
\begin{multline}~\label{eqn:game_potential}
   \textstyle  F(x^1, \ldots, x^N)  =\sum_{i,t,s,a} x^i_{tsa}C^i_{tsa} +  \sum_{t,s} k\Big(\sum_{i,a} 2x^i_{tsa} + \prod_{i \in [N]} (1 - \sum_{a}x^i_{tsa}) + \sum_{a} \prod_{i \in [N]}(1 -x^i_{tsa})\Big). 
\end{multline}
\begin{lemma}\label{lem:potential_gradient}
The joint state-action distribution $x$ satisfies~\eqref{eqn:potential_game}'s first order KKT conditions if and only if it corresponds to a Nash equilibria of the MDP congestion game with costs $(\ell^1,\ldots, \ell^N)$~\eqref{eqn:congestion_cost}. 
\end{lemma}
\begin{proof}
From~\cite[Thm.1.3]{calderone2017markov}, the first order KKT conditions of~\eqref{eqn:potential_game} are equivalent to the Nash equilibrium condition if $F$'s gradients satisfy $\textstyle \partial F/\partial x^i_{tsa} = \ell^i_{tsa}(x)$ for all $(i,t,s,a)\in[N]\times[T]\times[S]\times[A]$. We compute $\textstyle \partial F/\partial x^i_{tsa}$ via~\eqref{eqn:game_potential}. With respect to $x^i_{tsa}$, the gradient of $\sum_{i,t,s,a} x^i_{tsa}C^i_{tsa}$ is $C^i_{tsa}$, the gradient of $\textstyle  k\sum_{i,t,s,a} 2x^i_{tsa}$ is $2k$, the gradient of $\textstyle \sum_{t,s} k \prod_{i \in [N]} (1 - \sum_{a}x^i_{tsa})$ is $\textstyle -k\prod_{j\neq i} (1 - \sum_{a}x^j_{tsa})$, and the gradient of $\textstyle  \sum_{t,s,a} k \prod_{i \in [N]} (1 - x^i_{tsa})$ is $\textstyle -k\prod_{j\neq i}(1 - x^j_{tsa})$. Their sum recovers $\ell^i_{tsa}$~\eqref{eqn:congestion_cost} for all $\textstyle (i,t,s,a) \in [N]\times\mc{T}\times[S]\times[A]$. 
\end{proof}
\textbf{Non-convexity}. Distinct from~\cite{li2022congestion,li2019tolling}, congestion costs~\eqref{eqn:congestion_cost} results in a non-convex and multilinear optimization objective~\eqref{eqn:game_potential}. However, the proposed Frank-Wolfe solution ~\cite{li2022congestion, li2019tolling} for finding the Nash equilibria will still converge sublinearly~\cite{lacoste2016convergence}. We refer to~\cite[Alg.1]{li2022congestion} for an algorithm outline.
% \subsection{Player Heterogeneity, Equilibrium Stability, and Equilibrium Uniqueness}
% \tc{blue}{May take out.}Consider a single state, single time step, and two action MDP congestion game with two players shown in Figure~\ref{fig:stability_equilibrium}. On the left, both players have identical transition dynamics and identical MDP congestion costs. When player one selects state-action distribution $x^1 = [0, 1]$ and player two selects $x^2 = [1, 0]$, and assuming that $k^1 = k^2 = 1$, their congestion costs become $\ell^1(x^1, x^2)  = [1, -1]$, $\ell^2(x^1, x_2) = [0, 0]$. It is simple to verify against Definition~\ref{def:NE} that this is a Nash equilibrium. Because both players have identical roles within the game, an alternative Nash equilibrium is $x^1 = [1, 0]$ and $x^2 = [0, 1]$. 
% \begin{figure}[ht!]
%     \centering
%     \includegraphics[width=0.2\columnwidth, height=20mm]{figures/stability_equilibrium} 
%     \ 
%     \includegraphics[width=0.2\columnwidth, height=20mm]{figures/stability_equilibrium_b}
%     \caption{Illustration of two different MDP congestion games with two players. On the left, both players have homogeneous costs and transition dynamics. On the right, players have different costs.}
%     \label{fig:stability_equilibrium}
% \end{figure}
% Instead, suppose that the players have different congestion sensitivities: $k_1 = 2$ and $k_2 = 10$, in this case, under the joint state-action distribution, $x^1 = [0, 1], x^2 = [1, 0]$, each player achieves costs $\ell^1 = [1, -1]$ and $\ell^2 = [0, 9]$. Here while multiple equilibria the equilibria at $(x^1, x^2) = ([0, 1], [1,0])$ and $(x^1, x^2) = ([1,0], [0,1])$ are now both stable. 

% \begin{figure}
%     \centering
%     \includegraphics[width=\textwidth]{figures/mdp_description.png}
%     \caption{Example MDP figure}
%     \label{fig:mdp_desc}
% \end{figure}
\section{Stochastic Path Planning for Congested Air Traffic}\label{sec:flight_model}
Air traffic management operates under high operational uncertainty and strict collision risk requirements~\cite{shone2021applications}. Presently, air traffic authorities centrally plan deterministic trajectories and rely on human controllers to resolve local collision risks. 
% Due to the lack of global coordination, this local resolution method cannot minimize a flight's collision risk encountered per flight. 
% In real-time flight operations, runway delays, engine maintenance, and unexpected weather are all random events that cause flight paths to be highly uncertain.
We use the MDP congestion game model to embed the real-time operation uncertainty into path planning and find global collision risk-free trajectories. 
% At Nash equilibrium, each aircraft balances adherence to its designated trajectory with collision avoidance. 
% as much as possible given its sensitivity to collisions and the other flight trajectories. 
% In our model, each aircraft flies from an origin airport to a destination through a discretized 3D locations. The stochastic MDP transition dynamics capture operational uncertainties, and the MDP costs are selected to reduce collision risks without excessive delay or deviation from each aircraft's planned route. % is a planned flight flying from origin to destination airport. The state-action space models the 3D air space. The MDP transition dynamics maps each plane's deterministic actions to stochastic outcomes to reflect operational uncertainty. Finally, the MDP task completion costs are constructed based the plane's designated trajectory.
% Using the flight plan over France on July $10^{th}$, $2017$, we show that the designated flight plan carries non-trivial collision risks under operational uncertainty, and use the game model to compute flight plans with lower collision risks.

% \subsection{Individual Aircraft MDP}
\textbf{Individual aircraft MDP}. We use an MDP to model deterministic flight plans under operational uncertainty. 
% Our MDP model builds on the flight plan received by the aircraft.  
% Using discrete latitude-longitude points called \textbf{waypoints} and discrete altitudes called \textbf{flight levels}, the flight plan is a sequence of time-waypoint-flight level triplets.  
Aircraft $i$'s flight plan is $\textstyle \{(w^i_t, f^i_t) \ | t \in \mc{T}^i \}$, where $w^i_t$ are discrete waypoints used by the European Union Aviation Safety Agency (EASA), $f^i_t$ are discrete flight levels from 0 (sea level) to 450 ($45000$ feet) in increments of 50, and $\mc{T}^i$ are timestamps of the waypoints and flight-levels. 
% All flight plans share the same waypoints and flight levels but differ in timestamps. The aircraft-centric MDP is as follows:
\begin{figure}[ht!]
    \centering
    \includegraphics[width=0.24\columnwidth]{figures/aircraft_mdp_small.png}
    \
    \includegraphics[width=0.44\columnwidth]{figures/time_discretization.png}
    \
    \includegraphics[width=0.28\columnwidth]{figures/results/solved_no_congestion_costs_top_cropped.png}
    \caption{Left: Airspace state-action definitions. Center: Interval-based congestion computation. Right: Expected aircraft trajectories without congestion costs $D^i$~\eqref{eqn:DG_prob}. Colors correspond to time. }
    % \includegraphics[width=0.48\columnwidth]{figures/state_pruning.png}
    % \caption{Left: example figure for time discretization. Right: example figure for state pruning}
    \label{fig:time_discretization}
\end{figure}

\textbf{Time horizon}. Aircraft $i$'s time horizon is given by $\textstyle \mc{T}^i \cup \{t_L + b\Delta t_{int}\ | 0\leq b \leq B\}$, where $\mc{T}^i$ is from the flight plan, $t_L$ is the planned landing time, and $\Delta t_{int}, B\in \mathbb{N}$ are user-defined parameters.

\textbf{States}. Each state $(w, f) \in [S]$ consists of a waypoint $w$ and a flight level $f$, as shown in Figure \ref{fig:time_discretization}. 

\textbf{Actions}. At state $\textstyle(w,f)$, actions correspond to reaching one of $(w,f)$'s neighbors in the next time step. The set of neighbors is given by $\textstyle \mc{N}(w,f) = \big\{(w', f') \ | \ w' \in \mc{N}(w),  \ f' \in \{f-50, f, f+50\}, \ 0 \leq f'\leq 450 \big\}$, 
    % \begin{equation}\label{eqn:neighbor_definition}
    %     \mc{N}(w,f) = \Big\{(w', f') \ | \ w' \in \mc{N}(w),  \ f' \in \{f-50, f, f+50\}, \ 0 \leq f'\leq 450 \Big\},
    % \end{equation}
    where $\mc{N}(w)$ is the set of reachable waypoints from $w$.  Aircraft cannot loiter at $(w,f)$. The action of going to $(w', f')$ is $\textstyle a_{w',f'}$, such that $\textstyle [A_{w,f}] = \{a_{w',f'} \ | \ (w', f') \in \mc{N}(w,f)\}$.
    % the available actions are to traverse to any one of the neighboring states: $[A_{w,f}] = \{a_{w',f'} \ | \ (w', f') \in \mc{N}(w,f)\}$. Aircraft may maintain flight level but cannot loiter at the same waypoint.
    
\textbf{Transition Dynamics}. Under action $a_{w', f'}$ from $(w, f)$, an aircraft has $\textstyle \beta$ probability of reaching $(w', f')$ and $\textstyle 1 - \beta$ probability of diverting to another state in $\mc{N}(w,f)$. 
    % \begin{equation}
    %     P_{(w,f), a_{w', f'}, s'} = \begin{cases}
    %     0.95 & s' = (w', f') \\
    %     0.05 & s' = (w, f) \\
    %     0 & s' \in [S], s' \notin \{(w,f), (w', f')\}
    %     \end{cases}
    % \end{equation}
    
 \textbf{Cost}: Each state-action pair $(w, f, a_{w',f'})$ has a flight-dependent \textbf{deviation cost}, given by
    % determined by the flight plan. At time $t$, the state-action cost for player $i$ is then 
    \begin{equation}
       \textstyle   C^i_{t, w, f, a} = d(w^i_t, w) + \alpha_f | f - f^i_t| + L(t, w, f), \ \forall (w,f) \in [S], a \in [A_{w,f}],
    \end{equation}
    where $(w^i_t, f^i_t)$ is aircraft $i$'s planned location at $t$, $d(v, w)\in \mathbb{N}$ is the number of edges between $v$ and $w$, $\alpha_f \in \reals$ is user-defined parameter, and $L:\mc{T}^i\times[S]\mapsto \reals$ is a tardiness cost. 
    % We choose $\alpha_f > 1$ to reflect the energy intensive nature of changing flight levels. The tardiness cost is applied to all states past the landing time. 
    If the aircraft plans to land at $(w_T, f_T, T)$, then $\textstyle L(t, w, f) = 0$ if $\textstyle (w,f) = (w_T, f_T)$ or $\textstyle t \leq T$, else $\textstyle L(t, w, f) = c_{tardy}(t-T)$. 
    % In other words, the aircraft is penalized proportionally to its tardiness, but is not penalized for landing early.    
    The expected cost under the flight plan is zero and strictly positive otherwise. Therefore, aircraft are inclined to follow the flight plan in the absence of congestion. 
% This ensures that, even when the congestion cost is added, the trajectory of each aircraft stays close to the original flight plan.


Based on the individual aircraft MDP model, we build an MDP congestion game for the air traffic plan over France on July $3^{rd}$, $2017$. Between timestamps $39000$ and $41000$, $75$ planes left the Paris airports CDG and ORY to various destinations as shown in Figure~\ref{fig:time_discretization}. The collision risks $D^i$ and $G^i$~\eqref{eqn:DG_prob} can be interpreted as standard aircraft radial/vertical separation and longitudinal separation~\cite{brooker2006longitudinal}, respectively. In our simulations, only $D^i$ increases congestion costs.

\textbf{Interval-based collision risk computation}. Since each aircraft's time stamp is unique, we compute the congestion for time intervals. As shown in Figure~\ref{fig:time_discretization}, aircraft whose time stamp fall into the interval $\textstyle [t_k, t_k + \Delta t_{cong})$ will contribute to the congestion in time interval $k$.  
% We choose $\Delta t_{cong}$ to be the largest interval that ensures all consecutive timestamps on each aircraft's original flight plan fall into distinct intervals. 
% \begin{figure}[ht!]
%     \centering
    
%     \label{fig:flight_plan}
% \end{figure}
\begin{figure}[ht!]
    \centering
    \includegraphics[width=0.45\columnwidth]{figures/collision_reduction.pdf}
    \
    \includegraphics[width=0.45\columnwidth]{figures/potential_value.pdf}
    \caption{Left: Collision risk as a function of Frank-Wolfe algorithm iteration. Right: Congestion cost $\textstyle \sum_{i,t,s,a}kD_{tsa}^i(x)$~\eqref{eqn:DG_prob} as a function of Frank-Wolfe algorithm iteration. }
    \label{fig:FW_output}
\end{figure}

\textbf{Results and discussion}. We build the individual MDP and the interval-based congestion costs with the following user-defined parameter values: $\Delta t_{int} = 300$, $B = 3$, $\beta = 0.95$, $\alpha_f = 10$, $c_{tardy} = 2$, $\Delta t_{cong} = 19$, and $k = 10$. First, we verify that when solved without congestion cost $D^i$, all individual MDPs result in expected trajectories that match the original flight plan. The results are shown in Figure~\ref{fig:time_discretization}. We then define collision risk as $\sum_{a}x^i_{tsa}D^i_{tsa}(x)$, and found that for multiple flights, the maximum collision risk at any time was greater than $10\%$. The overall spread of collision risks for the original flight plan is shown on the  $x = 10^0$ line in Figure~\ref{fig:FW_output} left. We then augment individual costs with congestion cost $D^i$ and solve for the Nash equilibrium via the Frank Wolfe algorithm from~\cite[Alg.1]{li2022congestion}. The resulting collision risks and objective values are shown in Figure~\ref{fig:FW_output}. In the right figure, we see that the objective value decreases from $2200$ to $1900$ within the first $50$ iterations. Accompanying this, we observe that the maximum collision risks drops from $94\%$ to around $3\%$ within the first $10$ iterations of the Algorithm. Therefore, we conclude that our model was effective in reducing uncertainty-induced collision risks.
% Further work is needed to investigate its effect on flight arrival times. 


\section{Conclusion}
We derived an $N$-player MDP congestion game in which players solve MDPs that are coupled to the opponents through collision risk. We showed that its Nash equilibria are the KKT points of a potential minimization problem, and applied our model to collision risk reduction for commercial aircraft under operational uncertainty. Future work includes analyzing effect on flight delays.  
% Future work includes analyzing the effect on flight delays and incorporating additional aircraft separation constraints. 
%===============================================================================

\clearpage
% The acknowledgments are automatically included only in the final and preprint versions of the paper.
% \acknowledgments{If a paper is accepted, the final camera-ready version will (and probably should) include acknowledgments. All acknowledgments go at the end of the paper, including thanks to reviewers who gave useful comments, to colleagues who contributed to the ideas, and to funding agencies and corporate sponsors that provided financial support.}

%===============================================================================

% no \bibliographystyle is required, since the corl style is automatically used.
\bibliography{example}  % .bib

\end{document}
