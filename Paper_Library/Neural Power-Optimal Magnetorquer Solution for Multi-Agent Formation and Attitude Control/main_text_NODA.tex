\section{Introduction}
\IEEEPARstart{T}{he} magnetic field provides a sophisticated control system used in medicine, biomimetic robotics, and the aerospace field. Typically, they are generated by coils that wind conductive wires around magnetic materials or an air core. The current passing through these conductors generates a magnetic field. Its interaction with the surrounding magnets provides relative position and orientation information while generating electromagnetic forces and torques. In the medical field, magnetic resonance imaging is a well-known technique that uses magnetic fields to produce non-invasive images of internal structures. In biomedical robotics, these fields have been employed to steer miniature robots inside the human body, enabling targeted drug delivery and minimally invasive procedures \cite{ze2022soft}. The aerospace community originally utilized this actuation in the geomagnetic field as a satellite attitude actuator \cite{sakai2008design} and applied satellite state estimation \cite{%nurge2016satellite,
shibata2024relative}. 
% through the interaction between onboard magnetic torquers and Earth’s geomagnetic field.
%They have referred to the coils as magnetorquer (MTQ) and recently utilized MTQ in micro-gravity for satellite docking \cite{tajima2023study}, %separation \cite{Inamori_InOrb}, along with navigation %the vibration isolator \cite{Shibata}, debris remover \cite{}, and inter-satellite state estimation \cite{nurge2016satellite}.
\begin{figure}[!t]
\centering
%\begin{minipage}[b]{1\columnwidth}\subfloat[Learning overview of Neural power-Optimal Dipole Allocation (NODA).]{
\includegraphics[width=1\linewidth]{Figure/NODA.pdf}
%}\end{minipage}
\caption{Offline supervised learning of power-optimal dipole allocation model NODA for magnetically actuated multi-agent control.}%
\end{figure}

Multi-agent control methodologies using magnetorquers (MTQs) have been developed for the aerospace community. They
%, Electromagnetic Formation Flight (EMFF), 
use magnetic interactions to control the formation and attitudes of satellites. This actuation originally have underactuation for six degrees-of-freedom (DoF) control for $N$ systems \cite{takahashi2022kinematics,takahashi2025noda_mmh}. To deal this, previous work considered time-integrated current control to extend the controllability for 6-DoF swarm control. Their examples include a time-scheduled switching topology \cite{ramirez2010new} %ramirez2010new
and alternating current (AC) modulation \cite{takahashi2022kinematics}, which controls the amplitude and phase of the AC. The advantage of AC modulation also lies in the decoupling of magnetic interactions from different frequencies, including a geomagnetic field that enables decentralized control \cite{takahashi2025noda_mmh}. This actuation provides promising applications for proximity operations \cite{takahashi2025coil} and large distributed space structures \cite{shim2025feasibility,takahashi2025distance}.%They AC method, i.e., they can output arbitrary control force and torque for an arbitrary number of agents. %drives MTQ with the direct current (DC) %\cite{Fabacher,Kwon,Fan,Huang2,Ahsun,Elias,Schweighart,Ramirez-Riberos,Sakaguchi}  %\cite{abbasi2022decentralized,porter2014demonstration,Sakai,Ayyad,Zhang,Youngquist,nurge2016satellite,sunny2019single,Kaneda}.  %The AC method provides higher functionality than DC one: %\cite{Ahsun,Ayyad,Zhang,Kaneda,Sakai}.  %\cite{Schweighart,Fan,Ramirez-Riberos,Ayyad,Huang2} %, the evaluation function $\|\tau\|$ is optimized. %Spacecraft Formation Flying is a concept of controlling large-scale small satellites (0.1-2kg) in close range (0.1-2m) to form large space systems that are not feasible by monolithic space and ground assets (5m-1km). %mounted on small satellites MTQ femto \cite{Hu_Development}. and provides the proof of concepts of formation control by MTQ multiple experiments \cite{alvarez2021multi,hariri2018vision,porter2014demonstration,Sakai_Electromagnetic,sunny2019single,nurge2016satellite}. %位置を制御するformation control用としても使われている.EMFFの位置制御の実験らのリスト.

%for three or more satellites, 
Their drawback is that the solution of a computationally intensive optimization problem is required to derive the currents for coordinated magnetic-field interaction, which we refer to as ``dipole allocation.'' This problem contains inherent non-convex constraints \cite{takahashi2022kinematics}, rendering it an NP-hard non-convex quadratically constrained quadratic program (QCQP). As the problem dimensions increase, finding the global optimal solution becomes computationally expensive. Prior studies have explored several computational strategies. General optimization techniques 
%such as the homotopy (continuation) method \cite{morgan2009solving} 
have been applied to derive all optimal solutions \cite{schweighart2006electromagnetic} or track a viable solution using the solution from a previous step \cite{fabacher2017guidance,takahashi2022kinematics}. The closed-form Jacobian and Hessian expressions support efficient calculations as specialized methods for this system \cite{abbott2017computing}. For the two-satellite force control case, a unique global minimum power solution was analytically derived in a closed form \cite{abbasi2022decentralized}. However, space-qualified processors offer limited computational capabilities, rendering global solvers impractical. Local solvers provide no performance guarantees, and these locally optimal dipole moment solutions can be discontinuous because the solvers can jump between distinct local minima \cite{takahashi2022kinematics}. This can result in impulsive input and high-frequency disturbances. Moreover, recent applications \cite{takahashi2025distance,shim2025feasibility} assume simultaneous high-precision control of both the position and attitude without additional actuators and require repeated current computation. These require the optimal dipole solutions to be computed using a significantly reduced computational burden.

To this end, we derive a learning-based approximation algorithm called Neural power-Optimal Dipole Allocation (NODA) for real-time optimal dipole allocation. We also present a systematic multileader-based decentralized control strategy to improve scalability and reduce the sample region. The remainder of this paper is organized as follows: Section~\ref{Preliminaries} introduces the MTQ system for 6DoF swarm control. Section~\ref{NODA_subsection} derives a systematic decoupling approach focusing on a small group. In Section~\ref{Problem_Formulation}, we formulate the optimal dipole allocation problem with a proof of its control-induced disturbance reduction and baseline allocation. Section~\ref{convex_allocation} presents our main results, a continuous, optimal dipole solution derived by ODA that can be approximated by a multilayer perceptron (MLP) model using NODA. Numerical and experimental evaluations validate the effectiveness of the proposed methods and trained model in Section~\ref{Experimental_Validation} and Section~\ref{Conclusion} concludes paper.
%These centralized controls face scalability and reliability issues due to a high-end communication system, time delays, and computational loads as satellites increase. Distributed strategy generally lacks global superiority and encounters reliability challenges from nearby disturbances. Decentralized Control Strategy with Switching Topology \cite{takahashi2021simultaneous} Hence, control strategies for distributed systems using MTQ should consider continuous trajectory control based on distance constraints and low-current computation, while accounting for magnetic interactions. 
\section{Preliminaries}
\label{Preliminaries}
%: Orbit and Attitude Control and Electromagnetic System}
%This section summarizes the simultaneous control strategy for the electromagnetic force and torque of $ n$ agents.% and the learning-based functional approximation techniques. 
\subsection{Magnetic Interaction Approximation Model}
\label{EMFF}
%This subsection presents magnetic interaction approximation model and the AC modulation for 6-DoF control. 
We first introduces the dipole approximation of the magnetic field ${B}_k\left({\mu}_k, {r}_{j k}\right)$, ``far-field'' model \cite{schweighart2006electromagnetic}. %for large-distance conditions. %, which is accurate if the relative distance exceeds twice the diameter of the coil. 
We denote the $j$th magnetic moment as 
%$${\mu}_{j}(t)=N_t c(t) A {n}\in\mathbb{R}^{3}$$ 
$\mu_{j}=N_t Ac(1+\gamma
%\frac{\left(\mu_r-1\right)}{\left(1-N_d+\mu_r N_d\right)}
)\bm{n}$ 
%Because the ratio of length versus radius is so important it is assumed it can be kept constant in this case. This may not be a valid assumption depending on the difficulty of producing/obtaining the material. It may be necessary to accept a different thickness.
where $N_t$ is the number of coil turns, $c$ is the current
strength, $A$ is the area enclosed by the coil, $\bm{n}\in\mathbb{R}^3$ is the unit vector perpendicular to the coil plane, and $\gamma$ is the iron–core ratio
%$l$ is the length of the coil, $N_d={4(\ln({l}/{r})-1)}/({({l}/{r})^2-4 \ln({l}/{r})})$ 
\cite{%bellini2013magnetic,
takahashi2025noda_mmh}.
%\begin{equation*}
%\label{magnetic_filed}  
%\begin{aligned}
%B_k({\mu}_k, {r}_{j\leftarrow k})=\frac{\mu_0}{4 \pi d_{j\leftarrow k}^3}\left(3M_k\mathsf{e}_r -{\mu}_k\right)
%\end{aligned}
%\end{equation*}
This derives force ${f}_{j\leftarrow k}\left({\mu}_{j,k}, {r}_{j\leftarrow k}\right)=\nabla({\mu}_j \cdot {B}_k)$ and torque ${\tau}_{j\leftarrow k}\left({\mu}_{j,k}, {r}_{j\leftarrow k}\right)$ exerted on the $j$th agent by the $k$th one:
\begin{equation*}
%\label{basic_formulation}
\begin{aligned}
{f}_{j\leftarrow k}&=\frac{3 \mu_0/4\pi}{ d_{j\leftarrow k}^4}\left[\frac{{{\mu}_k \cdot {\mu}_j}-5 M_k M_j}{ d_{j\leftarrow k} }{{r}_{j\leftarrow k}}+M_k {\mu}_j+M_j{\mu}_k\right]\\
{\tau}_{j\leftarrow k}&={\mu}_j \times B_k=\frac{\mu_0/4 \pi}{d_{j\leftarrow k}^3}{\mu}_j \times \left[\frac{3M_k}{d_{j\leftarrow k}}{{r}_{j\leftarrow k}} -{\mu}_k\right],
%B({\mu}_k, {r}_{j k})=\frac{\mu_0}{4 \pi}\left(\frac{3 {r}_{j k}\left({\mu}_k \cdot {r}_{j k}\right)}{\left\|{r}_{j k}\right\|^5}-\frac{{\mu}_k}{\left\|{r}_{j k}\right\|^3}\right)
\end{aligned}
\end{equation*}
where $M_k=\frac{{\mu}_k \cdot {{r}_{j\leftarrow k}}}{d_{j\leftarrow k}}$ and $d_{j\leftarrow k}=\|{r}_{j\leftarrow k}\|$. We can also use exact model \cite{schweighart2006electromagnetic} or its learning-based approximation \cite{takahashi2025experimental,takahashi2025coil}.
%Note that this torque model is used often to realize cross-product attitude controller $\tau_{MTQ}=-[B_{e}]_{\times}d_{\mathrm{DC}}$ with geomagnetic field $B_{e}$. 
%where $M_k({\mu}_k,\mathsf{e}_r)={\mu}_k \cdot \mathsf{e}_r$ and $M_j({\mu}_j,\mathsf{e}_r)={\mu}_j \cdot \mathsf{e}_r$.%$d=\left\|{r}_{j k}\right\|$ and $\mathsf{e}_r={{r}_{j k}}/{d}$.
\subsection{Time-Integrated 6-DoF MTQ Control}% for 6-DoF Magnetorquer Control Under Nonholonomy}
\label{Kinematics_Control}
%\subsection{Alternating Current Magnetorquer Control and ``Kinematics'' for 6-DoF Control \cite{takahashi2022kinematics}}
%\label{EMFF}
We introduce the AC modulation for 6-DoF control. Let us assume the $j$th agent drives sinusoidal currents as follows:
\begin{equation}
\label{time-varying-dipole}
%\begin{aligned}
%{\mu}_{j}(t) ={\mu}_{\mathrm{DC}j}+{s}_j \sin (\omega_{j} t)+{c}_j \cos (\omega_{j} t)
{\mu}_{j(t)}=\mu^{\mathrm{amp}}_j \sin (\omega_{j} t+\bm\psi_j)={s}_j \sin (\omega_{j} t)+{c}_j \cos (\omega_{j} t),
%\end{aligned}
\end{equation}
%The total $2\omega_f$ disturbance of electromagnetic force and torque exerted on the $j$th satellite by the entire system is ${f,\tau_{j(2\omega_f)}}=\sum_{k=1}^{n}{f,\tau}_{2\omega_f}(t)$. 
where the amplitudes $\mu^{\mathrm{amp}}_{j,k}\in\mathbb{R}^3$, angular frequency $\omega_{j}\in\mathbb{R}$, phase $\theta\in\mathbb{R}^3$, and sine and cosine amplitudes ${s}_j,{c}_j\in\mathbb{R}^{3}$. We define the first-order averaged input from the neighbor as $u_{j}^{\mathrm{avg}}\triangleq\sum_{k\neq j} u_{j\leftarrow k}^{\mathrm{avg}}$, where $u_{j\leftarrow k}^{\mathrm{avg}}=[f_{j\leftarrow k}^{\mathrm{avg}};\tau_{j\leftarrow k}^{\mathrm{avg}}]$:
%$u_{j\leftarrow k}^{\mathrm{avg}}\triangleq \int_{T}{u}_{j\leftarrow k}({\mu}_{j,k(\tau)}, {r}_{j\rightarrow k})\frac{\mathrm{d}\tau}{T}$
\begin{equation}
\label{average_far_field_model}
\begin{aligned}
u_{j\leftarrow k}^{\mathrm{avg}}&
%=\begin{bmatrix}f_{j\leftarrow k}^{\mathrm{avg}}\\\tau_{j\leftarrow k}^{\mathrm{avg}}\end{bmatrix}
\triangleq \int_{T}\sum_{k\neq j}\begin{bmatrix}{f}_{j\leftarrow k}\left({\mu}_{k}(\tau),{\mu}_{j}(\tau), {r}_{j k}\right)\\{\tau}_{j\leftarrow k}\left({\mu}_{k}(\tau),{\mu}_{j}(\tau), {r}_{j k}\right)\end{bmatrix}\frac{\mathrm{d}\tau}{T}\\
&=\frac{1}{2}
\begin{bmatrix}    
{f}_{j\leftarrow k}({s_{j,k}},{r}_{j\leftarrow k})+{f}_{j\leftarrow k}({c_{j,k}},{r}_{j\leftarrow k})\\
{\tau}_{j\leftarrow k}({s_{j,k}}, {r}_{j\leftarrow k})+{\tau}_{j\leftarrow k}({c_{j,k}},{r}_{j\leftarrow k})
\end{bmatrix}
\end{aligned}
\end{equation}
where we use the fact that the agent does not interact with other frequency effects in the first-order averaged dynamics \cite{takahashi2022kinematics}, i.e., $\int {u}_{j\leftarrow k}\mathrm{d}t\approx 0$ for $\omega_k\neq \omega_j$. This modulation increases the number of variables and extends the controllability to realize 6-DoF control for an arbitrary number of agents \cite{takahashi2022kinematics}. We assume all $n$ agents have a circular triaxial MTQ and $m\in[1,n]$ agents are equipped with 3-axis RWs. %We use variables the command forces $f_c^{a}=[f_{c1}^i;\ldots;f_{cn}^i]\in\mathbb{R}^{3n}$, the command torques $\tau_c^{a}=[\tau_{c1}^{b_1};\ldots;\tau_{cn}^{b_n}]\in{R}^{3n}$, and the command forces $f_c^{a}=[\dot{h}_{c1}^{b_1};\ldots;\dot{h}_{cm}^{b_m}] \in{R}^{3m}$ $\xi^b=[\xi_1^{b_1}; \cdots;\xi_m^{b_m}]\in \mathbb{R}^{3 m}$
%$\mu^{\mathrm{amp}}_{N}\triangleq[\mu_{\mathrm{amp}1};\dots;\mu^{\mathrm{amp}}_{N}]\in\mathbb{R}^{3n}$ 
%without the trigonometric functions such as 
%${f}_{j\leftarrow k}(t)=f_{j\leftarrow k\mathrm{(avg)}}+{f}_{2\omega_f}(t)$ where $2\omega_f$ disturbance ${f}_{2\omega_f}(t)$ %={f}_{2\omega_f}({s_k},{c_j},{s_k},{c_j}, {r}_{jk},t)$ \cite{takahashi2022kinematics}, which is unique to the AC-based magnetorquers control.  
%The upper-bound of $\sup_{t\in[0,T)}\|{d}_{j\leftarrow k}^{2\omega}\|$ follows.
% 一般上限(常に成り立つ)
%Then, the conservative bound is $\sup_{t\in\mathbb{R}} f(t)\leq A_{12}^-+A_{12}^+$.
%kinematics control is generate the uncontinuousy and impulse control.
%\begin{remark}[Objective blending for power--disturbance trade-off]With the weight $W$ fixed (e.g.\ $W=W^\star$ or a fast approximate choice), the convex quadratic\[J(m)= m^\top\big(k_1 I+k_2 W\big)m\]balances power ($k_1$) and disturbance suppression ($k_2$). When $k_1=0$, minimizing $J$ directly reduces the certified bound$\sup_t\|d^{2\omega}\|\leq \frac{\mu_0}{8\pi} \tau (m^\top W m)^2$ with $\tau$ from the feasibility $M\preceq \tau(W\otimes W)$. \end{remark}
As its control does not change the angular momentum 
%Let us $r_c={\sum_{j} m_j r_j}/{\sum_{j} m_j}=r_c(0)$ and 
$\sum_{j=1}^n(m_j{r}_j\times\dot{r}_j+{I}_j \cdot {\omega}_j)+\sum_{j=1}^m {h}_j={L}\Leftrightarrow$
\begin{equation}
\label{angular_momentum_conservation}
%\begin{aligned}
%&\quad\sum_{j=1}^n\left(m_j\left({r}_j-{r}_1\right) \times \frac{\mathrm{d}{r}_j}{\mathrm{d} t}+{I}_j \cdot {\omega}_j\right)+\sum_{j=1}^m {h}_j={L}\\
%\Leftrightarrow &\quad
A_{(n,m)}\zeta \triangleq
\begin{bmatrix}
&m_1 [r_{1}^i]_\times, \cdots, m_n [r_{n}^i]_\times,\\
&C^{I/ B_1} J_1, \cdots, C^{I/ B_n} J_n,\\
&C^{I/B_1}, \cdots, C^{I/ B_{m}}
\end{bmatrix}
\begin{bmatrix}
\dot{r}^{a}\\
\omega^{b}\\
\xi^{b}
\end{bmatrix}
=0,
%\end{aligned}    
\end{equation}
where $A_{(n,m)}\in \mathbb{R}^{3 \times(6 n+3m)}$ and the states are $\zeta\in \mathbb{R}^{6 n+3 m}$. A previous study-derived commands $u_c\in\mathbb{R}^{6 n+3 m}$ for the submanifold stabilization of magnetically actuated swarms \cite{takahashi2022kinematics}.
\begin{theorem}[Kinematics Control \cite{takahashi2022kinematics}]
\label{theorem_experimental_controller}
%$S_{(n,m)}\in \mathbb{R}^{(6 n+3 m)\times (6 n+3 m-3)}$
Let $S_{(n,m)}$ be the tangent space of angular momentum conservation $S_{(n,m)}$, e.g.,
$$
S_{(n,m)}=
\begin{bmatrix}
E_{(21)}\\
-C^{B_2/I}A_{(n,m)[:,1:\mathrm{end-}3]}
\end{bmatrix}\in\text{Null Space}(A_{(n,m)})
$$
Then, the magnetic interaction can realize $u_c=[f_c^{a};\tau_c^{b};\dot{h}_{c}^{b}]=B_{(n,m)}^{-1}M_{(n,m)} S_{(n,m)} \mathfrak{u}_c$ where the arbitrary auxiliary commands are $\mathfrak{u}_c\in\mathbb{R}^{6 n+3 m-3}$, the inertia matrix is $M_{(n,m)}$, %$=\mathrm{Diag}([M_p,M_\alpha,E_{3m}])$,
and the input matrix is $B_{(n,m)}\in \mathbb{R}^{(6 n+3 m) \times(6 n+3 m)}$ in \cite{takahashi2022kinematics}.
%$$
\end{theorem}
%where $A_{(n,m)}$ is from (\ref{angular_momentum_conservation}) and $M, B_{(n,m)}$ is from Theorem~\ref{theorem_experimental_controller}. 
%\begin{proof}See the appendix of \cite{takahashi2022kinematics}.\end{proof} 
\section{Multileader-based Decentralized Control}
\label{NODA_subsection}
\label{Guidance_Strategy}
This section presents a multileader-based dipole allocation that decentralizes large-scale agent systems into smaller groups. We define $n$ agent nodes as $\mathcal{V}=\{1,\ldots,n\}$ and the edge set as $\mathcal{E} \subseteq \mathcal{V}\times \mathcal{V}$, where edge $(j, k)\in\mathcal{E}$ denotes that the $k$th node obtain some information from the $j$th node.

We define a directed multileader graph $\mathcal{G}^{\mathfrak{ml}}(\mathcal{V}, \mathcal{E}^{\mathfrak{ml}}=\bigcap_j \mathcal{E}_j^{\mathfrak{ml}})$ and assign specific agents to the leader subset $\mathcal{V}_l\in\mathcal{V}$ with a unique frequency $\omega_{k\in\mathcal{V}_l}$. Its edges $(j, k)\in\mathcal{E}_j^{\mathfrak{ml}}$ describe the hierarchical relationships of $j\in\mathcal{V}_l$ and $k\in\mathcal{V}$, i.e., 
%the $k$th agent following the $j$th one. %We label $j_{\in\mathcal{V}_l}$th leader satellite group as $\mathfrak{g}_{j}$: %$$\mathfrak{g}_{j}\triangleq\{j, k\in \mathcal{V}\  |\ (k, j)\in\mathcal{E}_j^{\mathfrak{ml}} \},\quad\mathcal{E}^{\mathfrak{ml}}\triangleq\mathcal{E}_1^{\mathfrak{ml}} \cap\ldots \cap \mathcal{E}_{n}^{\mathfrak{ml}}$$ %$\mathfrak{g}_{j}=j$ and 
%Note that $\mathcal{E}_{j}^{\mathfrak{ml}}=\{\}$ indicates $j$th leader agent does not have followers. of the $j$th leader 
the $k$th agents drive the $\omega_{j\in\mathcal{V}_l}$ current. This extends the dipole moment in (\ref{time-varying-dipole}) into the multi-frequency one:
%\begin{equation}
%\label{time-varying-dipole}
%\begin{aligned}
%{\mu}_{j}(t) ={\mu}_{\mathrm{DC}j}+{s}_j \sin (\omega_{j} t)+{c}_j \cos (\omega_{j} t)
%\end{aligned}
%\end{equation}
\begin{equation}
\label{multi_frequency_dipole_moment}
%\begin{aligned}
{\mu}_{k}(t) =%\boldsymbol{\mu}_{\mathrm{DC}j}+
\sum_{j\in\mathcal{V}_l}%\sum_{(j,k)\in\mathcal{E}_j}
({s_{k[j]}} \sin (\omega_{j} t)+{c_{k[j]}} \cos (\omega_{j} t)),\\
%&\vdots\\
%+{s_j} \sin \left(\omega_{fk} t\right)&+{c_j} \cos \left(\omega_{fk} t\right).
%\end{aligned}
\end{equation}
where ${s}_{k[j]},{c}_{k[j]}\in\mathbb{R}^{3}$ is the $k$th sine and cosine amplitude in the $j_{\in\mathcal{V}_l}$th leader group. One cycle $T$ and each $\omega_{j}$  should be derived to cancel out the coupling between different frequencies and hold the first-order approximation in (\ref{average_far_field_model}) as follows:%\cite{takahashi2022kinematics}
\begin{equation}
\label{AC_cycle}
\forall \ 1 \leq i<j \leq n,\quad
T\triangleq\frac{\mathrm{lcm}\left(\frac{2\pi k_{\mathrm{int}}}{2\omega_{i}}, \frac{2 \pi k_{\mathrm{int}}}{\left|\omega_{i}-\omega_{j}\right|}, \frac{2 \pi k_{\mathrm{int}}}{\omega_{i}+\omega_{j}}\right)}{k_{\mathrm{int}}}
\end{equation}
where $\mathrm{lcm}(\cdot)$ is the least common multiple and the coefficient $k_{\mathrm{int}}$ ensures the elements of $\mathrm{lcm}(\cdot)$ are integers. For a user-defined $T_{\mathrm{max}}$, the necessary conditions for $T\leq T_{\mathrm{max}}$ are
$$
\omega_{i}\geq \frac{\pi}{T_{\mathrm{max}}},\quad |\omega_{i}-\omega_{j}|\geq \frac{2\pi}{T_{\mathrm{max}}},\quad \omega_{i}+\omega_{j}\geq \frac{2\pi}{T_{\mathrm{max}}},
$$
such that the elements in (\ref{AC_cycle}) satisfy $T\leq T_{\mathrm{max}}$. %, and $i,j$ is arbitrary index that satisfies $\ 1 \leq i<j \leq n$.
\begin{remark}
One easy way to select $\omega_{i}$ for a given $T_{\mathrm{max}}$ is
$$
\omega_{i}=\frac{2\pi i}{T_{\mathrm{max}}}\quad \mathrm{s.t.}\quad T=\frac{ \mathrm{lcm}\left(\frac{ k_{\mathrm{int}}}{2i},\ \frac{k_{\mathrm{int}}}{|i-j|},\ \frac{ k_{\mathrm{int}}}{i+j}\right)} {k_{\mathrm{int}}/T_{\mathrm{max}}}=T_{\mathrm{max}}.
$$ 
%In this case, the current should be updated based on the maximum frequency $(2n+1)\times ({2\pi}/{T_{\mathrm{max}}})$ for an averaging. 
\end{remark}
\section{Problem Formulation of the Power-Optimal Dipole Allocation with Baseline Allocation}% for Approximation}
\label{Problem_Formulation}
%Strict Global Power-Optimal Dipole Allocation by Convex-Optimization for 6-DoF Magnetorquer Control
This section formulates the power-optimal dipole allocation problem and the baseline non-optimal one in Algorithm~\ref {alg:Inverse_based_dipole_allocation}. %Using the multileader-based decentralization introduced in the previous section, t
\subsection{Baseline: Decentralized Dipole Allocation}
Motivated by prior work \cite{abbasi2022decentralized}, we derive the baseline decentralized dipole allocation using an inverse matrix.
First, we simplify the time-averaged far-field magnetic interaction model as a bilinear polynomial formulation. For given the arbitrary vectors $v{^a},w{^a}\in\mathbb{R}^3$, we newly define the coordinates such that its $x$-axis aligns with $v{^a}$, i.e., $\mathsf{e}_x=\mathrm{nor}({v{^a}})$, its $y$-axis is orthogonal to $v{^a}$, and $w{^a}$ satisfies $\mathsf{e}_y=\mathrm{nor}(S({v{^a}})w{^a})$. This yields the following coordinate transformation matrix:% $\mathcal{C}_{({^a}v,{^a}w)}\in\mathbb{R}^{3\times 3}$ as
\begin{equation}
\label{new_frame}
%\begin{aligned}
\mathcal{C}{(v{^a},w{^a})}=
      \begin{bmatrix}
\mathsf{e}_x=\frac{v{^a}}{\|v{^a}\|},\mathsf{e}_y=
\frac{[{v{^a}}]_\times w{^a}}{\|[{v{^a}}]_\times w{^a}\|},\mathsf{e}_x\times \mathsf{e}_y
     \end{bmatrix}.%,\quad \mathsf{e}_y=\mathrm{nor}(S({v{^a}})w{^a})
%\end{aligned}
\end{equation}
%where $\mathsf{e}_y=\mathrm{nor}(S({v{^a}})w{^a})$. 
\begin{definition}
\label{C_LOS_new}
The line-of-sight frame $\{\mathcal{LOS}_{j\leftarrow k}\}$, fixed for the $k$th agent so that it can see the $j$th one, is defined by (\ref{new_frame}) such that its coordinate transformation matrix $C^{O/L_{j\leftarrow k}}\in\mathbb{R}^{3\times 3}$ is
\begin{equation}
%\begin{aligned}
%&C^{O/LOS}_{jk}=\begin{bmatrix}\mathsf{e}_x\triangleq\mathrm{nor}({{^o}r_{jk}}),\mathsf{e}_y,\mathsf{e}_x\times \mathsf{e}_y\end{bmatrix}\in\mathbb{R}^3\\
C^{A/L_{j\leftarrow k}}=\mathcal{C}(r_{j\leftarrow k}^a,S(f^a_{j})r^a_{j\leftarrow k}),
%\end{aligned}
\end{equation}
where $x$-axis in $\mathcal{LOS}$ is aligned with $r_{jk}$ and $f_{j}$ remains in the $x$-$y$ plane in $\mathcal{LOS}$, i.e., $r^{l_{j\leftarrow k}}_{j\leftarrow k}=[\|r^{a}_{j\leftarrow k}\|;0;0]$ and $f^{l_{j\leftarrow k}}_{j(3)}=0$. 
\end{definition}
%\begin{definition}
%\label{los_interaction_definition}
\noindent
The frame in Definition~\ref{C_LOS_new} yields the bilinear polynomial formulation in (\ref{average_far_field_model}) with the constant matrix 
\begin{equation}
\label{time_averaged_LOS}
%\begin{aligned}
u_{j\leftarrow k}^{a}=
\frac{1}{2}\frac{\mu_0}{4\pi}Q_{{j\leftarrow k}}
\left (
{s_{k}^{a}}
\otimes
{s_{j}^{a}}
+
{c_{k}^{a}}
\otimes
{c_{j}^{a}}
\right ),
%&=\frac{\mu_0}{8\pi}\mathrm{tr}[s_{j}^{a}({s_{k}^{a\top}}\mathcal{Q}_{j\leftarrow k[i]}^\top) +c_{j}^{a}({c_{k}^{a\top}} \mathcal{Q}_{j\leftarrow k[i]}^\top)]\\
%\end{aligned}
\end{equation}
%このaとbをlとfに変更する。
where %$u^{los}_{L\leftarrow F}=[{f^{los}_{L\leftarrow F}};{\tau^{los}_{L\leftarrow F}}]\in \mathbb{R}^{6}$, 
$d_{j\leftarrow k}=\|r_{j\leftarrow k}\|$ and $Q_{{j\leftarrow k}}\in \mathbb{R}^{6\times 9}$ is
\begin{equation*}
%\label{sec2::eq18}
%\left\{
\begin{aligned}
&Q_{{j\leftarrow k}}=(I_2\otimes C^{A/L_{j\leftarrow k}}) 
\begin{bmatrix}
\Psi_{f({d_{j\leftarrow k}})}\\
\Psi_{\tau({d_{j\leftarrow k}})}
\end{bmatrix}
(C^{L_{j\leftarrow k}/A}\otimes C^{L_{j\leftarrow k}/A})\\
&\left\{
\begin{aligned}
&\Psi_f({d_{j\leftarrow k}})=
%r_{\mathrm{sgn}}
\frac{1}{{d_{j\leftarrow k}^4}}
\begin{bmatrix}
%-6,0_3,3,0_3,3\\0,3,0,3,0_5\\0_2,3,0_3,3,0_2\\
-6&0&0&0&3&0&0&0&3\\
0&3&0&3&0&0&0&0&0\\
0&0&3&0&0&0&3&0&0
%\end{bmatrix}&\begin{bmatrix}
%0&3&0\\3&0&0\\0&0&0
%\end{bmatrix}&\begin{bmatrix}
%0&0&3\\0&0&0\\3&0&0
\end{bmatrix}
%d_{j\leftarrow k}\begin{bmatrix}
%0_5,1,0,-1,0\\
%0_2,2,0_3,1,0_2\\
%0,-2,0,-1,0_5\\
%0&0&0&0&0&1&0&-1&0\\
%0&0&2&0&0&0&1&0&0\\
%0&-2&0&-1&0&0&0&0&0
%\end{bmatrix}
%&\begin{bmatrix}
%\\0&0&0\\-d^2&0&0
%\end{bmatrix}&\begin{bmatrix}
%0&-d^2&0\\d^2&0&0\\0&0&0
%\end{bmatrix}
%\end{bmatrix}
\\
&%\frac{\Psi({d_{j\leftarrow k}})}{{1}/{d_{j\leftarrow k}^4}}
\Psi_\tau({d_{j\leftarrow k}})=
%\begin{bmatrix}
%r_{\mathrm{sgn}}
%\begin{bmatrix}
%-6,0_3,3,0_3,3\\0,3,0,3,0_5\\0_2,3,0_3,3,0_2\\
%-6&0&0&0&3&0&0&0&3\\
%0&3&0&3&0&0&0&0&0\\
%0&0&3&0&0&0&3&0&0\\
%\end{bmatrix}&\begin{bmatrix}
%0&3&0\\3&0&0\\0&0&0
%\end{bmatrix}&\begin{bmatrix}
%0&0&3\\0&0&0\\3&0&0
%\end{bmatrix}\\
\frac{1}{d_{j\leftarrow k}^3}
\begin{bmatrix}
%0_5,1,0,-1,0\\
%0_2,2,0_3,1,0_2\\
%0,-2,0,-1,0_5\\
0&0&0&0&0&1&0&-1&0\\
0&0&2&0&0&0&1&0&0\\
0&-2&0&-1&0&0&0&0&0
\end{bmatrix}
\end{aligned}
\right.
%&\begin{bmatrix}
%\\0&0&0\\-d^2&0&0
%\end{bmatrix}&\begin{bmatrix}
%0&-d^2&0\\d^2&0&0\\0&0&0
%\end{bmatrix}
%\end{bmatrix}
\end{aligned}
%\right.
\end{equation*}
where $\sigma_{\max}(\Psi_f)=\sqrt{54}/d_{j\leftarrow k}^4$ and $\sigma_{\max}(\Psi_\tau)=\sqrt{5}/d_{j\leftarrow k}^3$. 
%\end{definition}%$r_{\mathrm{sgn}}=\mathrm{sgn}(r_{j\leftarrow k}(1))=\frac{^ir_{j\leftarrow k}(1)}{d}$.
%\subsection{Baseline: Inverse-based Decentralized Dipole Allocation}
%\label{Inverse}
We allocate a dipole for two agents in a decentralized manner. In (\ref{time_averaged_LOS}), we randomly predetermine one dipole as a constant indexed by 0, e.g., $\{s_{F0},c_{F0}\in\mathbb{R}^{3}\mid\|s_{F0}\|=\|c_{F0}\|=1\}$. This derives another one using the inverse matrix calculation
%the nature of the bilinear polynomial:
\begin{equation}
\label{decentralized_allocation}
\left\{
\begin{aligned}
\begin{bmatrix}
{s_{L}^{los}}\\
{c_{L}^{los}}
\end{bmatrix}&=
\left\{
\frac{\mu_0}{8\pi}
Q_{L\leftarrow F}
\left(\begin{bmatrix}
s_{F0}^{los},c_{F0}^{los}\end{bmatrix} \otimes
   E_3\right)
\right\}^{-1}u_{L\leftarrow F}^{los}\\
\begin{bmatrix}
{s_{F}^{los}}\\
{c_{F}^{los}}
\end{bmatrix}&=
\left\{
\frac{\mu_0}{8\pi}
Q_{L\leftarrow F}
\left(E_3 \otimes
   \begin{bmatrix}
s_{L0}^{los},c_{L0}^{los}\end{bmatrix}\right)
\right\}^{-1}u_{L\leftarrow F}^{los}
\end{aligned}
\right..
\end{equation}
% that is compared with the power-optimal one in subsection~\ref{Power_Optimal_ODA_Baseline_Comparison}.% We assume the position vectors are from the center of mass position.% $r_c=\frac{\sum_{k\in\mathfrak{g}_{l}} m_k r_k}{\sum_{k\in\mathfrak{g}_{l}} m_k}$.
%W_{\mathrm{power}}$, %= R\|[s_L;s_F;c_L;c_F]\|^2/(2 \gamma_{\mu/c}^2)$, 
We make their norms uniform to minimize $\|\mu^{\mathrm{amp}}_{N}\|^2$:
$$
\{s_L,c_{L}\}=\frac{\{s_{L0},c_{L0}\}}{\sqrt{\|\{s_{L0},c_{L0}\}\|}},\ \{s_F,c_F\}=\frac{\{s_{F0},c_{F0}\}}{1/{\sqrt{\|\{s_{L0},c_{L0}\}\|}}}.
$$
%We summarize this randomized allocation in Algorithm~\ref {alg:Inverse_based_dipole_allocation}.
\begin{remark}%in (\ref{decentralized_allocation})]
We note the necessary conditions for the existence of the inverse in (\ref{decentralized_allocation}). Without loss of generality, we can set $\theta_{j(1)}^{los}=0$ 
%among the $\theta_{j}^{los}$ 
as $s_{F0} = [\mu^{\mathrm{amp}}_{F(1)};\mu^{\mathrm{amp}}_{F(2)}\cos\theta_{F(2)};\mu^{\mathrm{amp}}_{F(3)}\cos\theta_{F(3)}]$ and $c_{F0}=[0;\mu^{\mathrm{amp}}_{F(2)}\sin\theta_{F(2)};\mu^{\mathrm{amp}}_{F(3)}\sin \theta_{F(3)}]$ where $\{los\}$ are dropped in the remainder. A symbolic calculation verifies that $\mu^{\mathrm{amp}}_{F(1)}\mu^{\mathrm{amp}}_{F(2)}\mu^{\mathrm{amp}}_{F(3)}\sin(\theta_{F(2)}\mathrm{-}\theta_{F(3)})\neq 0$ guarantees the existence of an inverse. 
%\begin{equation}
%\label{time-varying-dipole}
%\begin{aligned}
%{\mu}_{j}(t) ={\mu}_{\mathrm{DC}j}+{s}_j \sin (\omega_{j} t)+{c}_j \cos (\omega_{j} t)
%{\mu}_{j}(t)&=\mu^{\mathrm{amp}}_j \sin (\omega_{j} t+\theta_j)\\&={s}_j \sin (\omega_{j} t)+{c}_j \cos (\omega_{j} t)    
%\end{aligned}
%\end{equation}
%The perpendicular vector $\mu_{F0\bot}$ of $\mu_F(t)$ based on (\ref{time-varying-dipole}) %is 
Its orthogonal vector $\mu_{F0\bot}$ satisfies 
$$
\mu_{F0\bot}/ \! /
({c}_{F0}\times {s}_{F0})=
\begin{bmatrix}
\mu^{\mathrm{amp}}_{j(2)} \mu^{\mathrm{amp}}_{j(3)} \sin(\theta_{j(2)} - \theta_{j(3)})\\
\mu^{\mathrm{amp}}_{j(3)} \mu^{\mathrm{amp}}_{j(1)}\sin(\theta_{j(3)} - \theta_{j(1)})\\
\mu^{\mathrm{amp}}_{j(1)} \mu^{\mathrm{amp}}_{j(2)}\sin(\theta_{j(1)} - \theta_{j(2)})
\end{bmatrix}\in\mathbb{R}^3.
$$
Then, dipoles with components only in the $l\in[1,3]$th axis, i.e., $\mu_{F\bot}=\delta_{il}\in\mathbb{R}^3$, which is Kronecker delta, are obtained by setting $\mu^{\mathrm{amp}}_{F(l)}=0$ and do not satisfy the above conditions.
\begin{algorithm}[tb!]
\begin{algorithmic}[1] 
\STATE  \textbf{Inputs: }trial number $n_{\mathrm{rand}}$, ${r}_{L\leftarrow F}\in\mathbb{R}^3,u_{L\leftarrow F}^{los}\in\mathbb{R}^{6}$
\STATE  \textbf{Outputs: }$[s_L;s_F;c_L;c_F]^{(i^*)}\in\mathbb{R}^{12}$, $W_{\mathrm{power}}^{(i^*)}\in\mathbb{R}$
%\STATE Calculate $u_{L\leftarrow F}^{los}$ and $Q_{j\leftarrow k}$ in Definition~\ref{los_interaction_definition}
\FOR{$i\in[1,n_{\mathrm{rand}}]$}
%\STATE Generate randomly $\{s_F,c_F\in\mathbb{R}^{3}\mid\|s_F\|=\|c_F\|=1\}$% based on Remark2 \ref{}
%\STATE Calculate $s_L,c_L$ in (\ref{decentralized_allocation}) and 
\STATE Derive randomly $[s_L;s_F;c_L;c_F]^{(i)}$ by (\ref{decentralized_allocation}) and $W_{\mathrm{power}}^{(i)}$% by Remark~\ref{Power_Normalization_Decentralized_Dipole_Allocation}
\ENDFOR
\STATE Choose $[s_L;s_F;c_L;c_F]^{(i^*)}$ satisfies $i^*=\argmin W_{\mathrm{power}}^{(i)}$
%\STATE Derive $\mu_j^a(t)=\mu^{\mathrm{amp}}_j^*\sin(\omega_{f}t+\theta_j^*+\theta_0)$
\end{algorithmic}
\caption{Inverse-based dipole allocation based on (\ref{decentralized_allocation}).}
\label{alg:Inverse_based_dipole_allocation}
\end{algorithm}
%\begin{remark}[Power Normalization for Baseline Allocation]
%\label{Power_Normalization_Decentralized_Dipole_Allocation}
%\end{remark}
%We show that the lack of an arbitrary component in LOS falls into the singularity of the inverse matrix.
%$$
%\mu_{j\bot}/ \! /({c}_j\times {s}_j)=\begin{bmatrix}
%\mu^{\mathrm{amp}}_{j(2)}\mu^{\mathrm{amp}}_{j(3)}\sin(\theta_{j(2)} - \theta_{j(3)})\\
%\mu^{\mathrm{amp}}_{j(3)}\mu^{\mathrm{amp}}_{j(1)}\sin(\theta_{j(3)} - \theta_{j(1)})\\
%\mu^{\mathrm{amp}}_{j(1)}\mu^{\mathrm{amp}}_{j(2)}\sin(\theta_{j(1)} - \theta_{j(2)})
%\end{bmatrix}.
%$$
%$$
%\mu_{F\bot}/ \! /
%\begin{bmatrix}
%\mu^{\mathrm{amp}}_{F(2)}\mu^{\mathrm{amp}}_{F(3)}\sin(\theta_{F(2)} - \theta_{F(3)})\\
%\mu^{\mathrm{amp}}_{F(3)}\mu^{\mathrm{amp}}_{F(1)}\sin\theta_{F(3)}\\
%-\mu^{\mathrm{amp}}_{F(1)}\mu^{\mathrm{amp}}_{F(2)}\sin\theta_{F(2)}
%\end{bmatrix}.
%$$
\end{remark}
\subsection{Power-Optimal Dipole Allocation and Lower Power Bound}
The baseline allocation in (\ref{time_averaged_LOS}) leads to a decentralized strategy that does not require intersatellite communication; however, these dipole moments are not power-optimal, and this motivated us to derive optimal power-consumption solutions, which are crucial for limited resources. %First, %we stack the vectors of the time-varying dipole moments in (\ref{time-varying-dipole}) as $\mu^{\mathrm{amp}}_N,\bm\psi_N,s_N,c_N\in\mathbb{R}^{3n}$.
\begin{comment}
\begin{equation}
\label{dipole_vectors}
\mu^{\mathrm{amp}}_N \triangleq 
    \begin{bmatrix}
    \mu^{\mathrm{amp}}_1\\
    \vdots\\
    \mu^{\mathrm{amp}}_n
    \end{bmatrix},\ 
\bm\psi_N \triangleq 
    \begin{bmatrix}
    \bm\psi_1\\
    \vdots\\
    \bm\psi_n
    \end{bmatrix},\     
s_N \triangleq
    \begin{bmatrix}
    s_1\\
    \vdots\\
    s_n
    \end{bmatrix},\ 
c_N \triangleq
    \begin{bmatrix}
    c_1\\
    \vdots\\
    c_n
    \end{bmatrix}
\end{equation}
\end{comment}
%where $\mu^{\mathrm{amp}}_N,\bm\psi_N\in\mathbb{R}^{3n}$.
First, we define the overall dipole vector for the $j_{\in\mathcal{V}_l}$th leader group as 
$$
{m}(t)={m}(\mu^{\mathrm{amp}}_{N},\bm\psi_N,t)=\begin{bmatrix}
    \sin(\omega_jt)s_N\\
    \cos(\omega_jt)c_N
\end{bmatrix}\in\mathbb{R}^{6n}
$$
where we stack the vectors of the time-varying dipole moments in (\ref{time-varying-dipole}) as $\mu^{\mathrm{amp}}_N,\bm\psi_N,s_N,c_N\in\mathbb{R}^{3n}$. For a given command 6-DoF control ${f}_{cj},{\tau}_{cj}$ by an arbitrary controller, the power-optimal dipole solutions $m^{*}$ satisfy the QCQP problem:
\begin{equation}
\label{basic_formulation}
%\begin{aligned}
\mathrm{min}%J_p(m)=
\int
%_{t\in[0,T]}
\frac{\|m(\tau)\|^2}{2}\frac{\mathrm{d}\tau}{T}\quad\mathrm{s.t.}\quad {u}_{cj}=u_{j}^{\mathrm{avg}}
%\left\{\begin{aligned}{f}_{cj}&={f}_{j(\mathrm{avg})},\\{\tau}_{cj}&={\tau}_{j(\mathrm{avg})},\end{aligned}\right.
\ \mathrm{for}\ j\in[1,n],
%\end{aligned}
\end{equation}
with an infinite number of solutions \cite{takahashi2022kinematics} and $6n$ equality constraints. 
%\begin{remark}[Lagrange Dual Problem of Power-Optimal Dipole Allocation]
%We derive the lower bound of the evaluation function in (\ref{basic_formulation}). 
For the problem in (\ref{basic_formulation}), the associated Lagrangian $L$ with the Lagrange multiplier vector ${\lambda}\in\mathbb{R}^{6(n-1)}$ is
\begin{equation*}
L(m,{\lambda})\triangleq \frac{1}{2}m^\top (I_2\otimes P_{{\lambda}}) m-
\frac{8\pi}{\mu_0}{\lambda}^\top
\begin{bmatrix}
u_{c1}^{a}; \ldots; 
%\vdots\\
u_{c(n-1)}^{a}
\end{bmatrix},
\end{equation*}
where $P_{{\lambda}}\triangleq I_{3n}+R_{{\lambda}}+R_{{\lambda}}^\top$ and $R_{{\lambda}}\in\mathbb{R}^{3n\times 3n}$ is the following non-symmetric matrix with $\mathrm{vec}({R}_{j\leftarrow k})\triangleq Q^{\top}_{j\leftarrow k}\lambda_j$:
\begin{equation*}
%\label{P_n}
%\begin{aligned}
R_{{\lambda}}\triangleq
\begin{bmatrix}
O_3&{R}_{1\leftarrow 2}&\cdots&{R}_{1\leftarrow (n-1)}&{R}_{1\leftarrow n}\\
{R}_{2\leftarrow 1}&O_3&\cdots
&{R}_{2\leftarrow (n-1)}&{R}_{2\leftarrow n}\\
\vdots&\vdots&\ddots
&\vdots&\vdots
\\
{R}_{(n-1)\leftarrow 1}&{R}_{(n-1)\leftarrow 2}&\cdots
&O_3&{R}_{{(n-1)}\leftarrow n}\\
O_3&O_3&\cdots&O_3&O_3
\end{bmatrix}
%\end{aligned}
\end{equation*}
Note that $P_{{\lambda}}$ is a positive definite matrix; otherwise, there exists an $m$ such that $L(m,{\lambda})\rightarrow-\infty$. Then, we obtain the lower bound as $J_{\mathrm{dual}}\leq \frac{1}{2}{\|m\|^2}$ using 
%Let ${\lambda}^*$ be defined as the optimum solution of 
the following Lagrange dual problem \cite{boyd2004convex}, which is convex and
efficiently solvable:
\begin{equation}
\label{opt2_N}
%\mathcal{OPT}_{\mathrm{DUAL}}\ 
\max
%_{{\lambda}\in\mathbb{R}^{6(N-1)}}
\ J_{\mathrm{dual}}=
%\frac{-{\lambda}^\top \hat{u}^{a}}{\mu_0/(8\pi)}
-\frac{8\pi}{\mu_0}{\lambda}^\top
\begin{bmatrix}
u_{c1}^{a};\ldots;
%\vdots\\
u_{c(n-1)}^{a}
\end{bmatrix}\ \mathrm{s.t.}\  P_{{\lambda}}\succeq 0.
\end{equation}
%and this defines optimal $P_{{\lambda}^*}$. 
%\end{remark}
%等式制約条件は双線形多項式でありそれ自体を満たす解は無限個あり得る.全体の問題はnon-convex最適問題でありlocal-optimumのパフォーマンスを保証することはできない.また,評価関数はstrictly convex functionだけど,拘束条件がnon-convex setなのでa strict global minimum, i.e., $f(x)< f(x^*)$ for $x\in\Omega$.したがって,計算量を犠牲にしてglobal-solverで解いたglobal最適解でMLPを作るときに,そのユニーク性が保証できずMLPの構築が不安定になる可能性がある.したがって,global最適解の学習用に特徴を絞りたい. 最終的にMLPで近似したい.関数はstrictly convex functionだけど制約条件がnon-convex setなので最適解は無限個あり得るからなので仮にglobal-solverで見つけた時機械学習用に特徴を絞りたい.a strict global minimum, i.e., $f(x)< f(x^*)$ for $x\in\Omega$ a uniqueness results from a non-strictly convex problemSDP緩和をすれば一意に決まるがdipoleの任意性はどうやって現れる?dipoleの任意性を消せばsdpで決まるとして,なんとduality gapは0でした!っていう結果になる?今までのXを使った話は緩和問題とは別! 前回は緩和問題を使わなくても解けるけど,今回は使う, 振幅が固定のときだけXが使える,なので双対問題もいらない,いる,%Some convex optimization problems may have multiple optimal solutions or have degenerate cases,uniquness!!!!SDP緩和で得た解の1) globalとの差と,2) そのuniquness(MLPにとって一番重要) N=2だと大丈夫.convex optimizationではuniqueに毎回なるとは限らない
\subsection{Control-Induced Disturbance Reduction via Optimization}
\label{control_induced_disturbance_reduction}
Before presenting our learning-based strategy for solving (\ref{basic_formulation}) in the next section, we prove that power-optimal dipole allocation also leads to a reduction in time-integrated control-induced disturbances. The rigorous formulation of the magnetic interaction includes  $u_{j\leftarrow k}^{\mathrm{avg}}$ in (\ref{time_averaged_LOS}) and sinusoidal disturbances ${d}_{j\leftarrow k(t)}^{2\omega}$ as ${u}_{j\leftarrow k(t)}=u_{j\leftarrow k}^{\mathrm{avg}}+{d}_{j\leftarrow k(t)}^{2\omega}$. We can %analytically 
derive the total sinusoidal disturbance from neighbor ${d}_{j(t)}^{2\omega}$ as
\begin{equation}
\label{sinusoidal_distrurbance_analytical}
\begin{aligned}
{d}_{j(t)}^{2\omega}&=\sum_{k\in\mathcal{N}_j}{d}_{j\leftarrow k(t)}^{2\omega}=
v_{(t)}^\top \begin{bmatrix}
{d}_{c}\\
{d}_{s}
\end{bmatrix},\ v_{(t)}\triangleq 
\begin{bmatrix}
\cos\left(2\omega t\right)\\
\sin\left(2\omega t\right)
\end{bmatrix}\\
%\cos\left(2\omega t\right){d}_{c}+\sin\left(2\omega t\right){d}_{s}\\
\begin{bmatrix}
{d}_{c}\\
{d}_{s}
\end{bmatrix}
&\triangleq\sum_{k\in\mathcal{N}_j}\frac{\mu_0}{8\pi}(I_2\otimes Q_{j\leftarrow k})
\begin{bmatrix}
c_k^b\otimes c_j^b-s_k^b\otimes s_j^b\\
c_k^b\otimes s_j^b+s_k^b\otimes c_j^b
\end{bmatrix}.
\end{aligned}
\end{equation}
where $\mathcal{N}_j= \{k\ |\ (j, k) \in \mathcal{E}\}$ is neighbor sets of $j$th agent. The next lemma states that the solution to the optimization in (\ref{basic_formulation}) reduces the upper bound of $\sup_{t\in[0,T)}\|{d}_{j(t)}^{2\omega}\|$.
%The upper-bound of $\sup_{t\in[0,T)}\|{d}_{j\leftarrow k}^{2\omega}\|$ follows.
% 一般上限(常に成り立つ)
%Then, the conservative bound is $\sup_{t\in\mathbb{R}} f(t)\leq A_{12}^-+A_{12}^+$.
\begin{lemma}
\label{quadratic_upper_bound}% for $2\omega$ disturbance]
$\|m\|^2$ linearly bounds $\sup_{t\in[0,T)}\|{d}_{j(t)}^{2\omega}\|$.
\end{lemma}
\begin{proof}
We derive the time-supremum $\|{d}_{j}^{2\omega}\|_2^2$ using (\ref{sinusoidal_distrurbance_analytical}) as
$$
\begin{aligned}
&\sup_{t}\|{d}_{j}^{2\omega}\|_2^2=\sup_{t}\ v_{(t)}^\top
%\begin{bmatrix}\|x\|^2 & x^\top y\\x^\top y & \|y\|^2\end{bmatrix}
\begin{bmatrix}
\|{d}_{c}\|^2&{d}_{c}^\top {d}_{s}\\
{d}_{c}^\top {d}_{s}&\|{d}_{s}\|^2
\end{bmatrix}
v_{(t)}\\
\leq&\frac{\left\|
\begin{bmatrix}
    {d}_{c}\\
    {d}_{s}
\end{bmatrix}\right\|_2^2+\sqrt{(\|{d}_{c}\|_2^2-\|{d}_{s}\|_2^2)^2+4({d}_{c}^\top {d}_{s})^2}}{2}
%\\&\leq\left\|\begin{bmatrix}x\\y\end{bmatrix}\right\|_2^2
\end{aligned},
%\end{equation}
$$  
where we use $\|v\|=1$.
%We first show $2\omega$-disturbance norm $\sup_{t\in[0,T)}\|{d}_{j\leftarrow k}^{2\omega}\|$ is linearly upper-bounded in $W_{\mathrm{power}}$.% for $m=[c_k^b;c_j^b;s_k^b;s_j^b]$.
We define $Z_0$ and $Z=Z^\top\succ 0$ such that $[c_k^b\otimes c_j^b-s_k^b\otimes s_j^b;c_k^b\otimes s_j^b+s_k^b\otimes c_j^b]=Z_0(m\otimes m)$ and $0 \preceq Z_0^\top (I_2\otimes Q_{j\leftarrow k})^\top (I_2\otimes Q_{j\leftarrow k})Z_0 \preceq  (Z\otimes Z)$. Then, $\lambda_{\max(\cdot)}\leq \mathrm{Trace}_{(\cdot)}$ and the Kronecker identity yield
$$
\sup_{t\in[0,T)}\|{d}_{j\leftarrow k}^{2\omega}\|_2^2\leq\left\|\begin{bmatrix}
    {d}_{c}\\
    {d}_{s}
\end{bmatrix}
\right\|_2^2\leq
%(m\otimes m)^\top (W\otimes W)(m\otimes m) = 
(\frac{\mu_0}{8\pi}m^\top Z m)^2.
$$
This indicates it is linearly bounded by $\|m\|^2$.
\end{proof}
%we can find the same results for $\omega_j\neq\omega_k$, i.e., 
\noindent
Minimizing $W_{\mathrm{power}}$ 
%for agents operating $\omega_j$
also leads to a reduction in the control-induced disturbance from an uncooperative neighbor using different frequencies, 
%$\omega_k$, 
although we omit this proof for simplicity.
%kinematics control is generate the uncontinuousy and impulse control.
%\begin{remark}[Objective blending for power--disturbance trade-off]With the weight $W$ fixed (e.g.\ $W=W^\star$ or a fast approximate choice), the convex quadratic\[J(m)= m^\top\big(k_1 I+k_2 W\big)m\]balances power ($k_1$) and disturbance suppression ($k_2$). When $k_1=0$, minimizing $J$ directly reduces the certified bound$\sup_t\|d^{2\omega}\|\leq \frac{\mu_0}{8\pi} \tau (m^\top W m)^2$ with $\tau$ from the feasibility $M\preceq \tau(W\otimes W)$. \end{remark}
\section{Neural Power-Optimal Dipole Allocation}% for Approximation}
\label{convex_allocation}
%Strict Global Power-Optimal Dipole Allocation by Convex-Optimization for 6-DoF Magnetorquer Control
This subsection presents the learning-based power-optimal dipole calculation model NODA for $n$ agents control. %\subsection{Multilayer Perceptron Model Approximation}
MLP represents the functional mapping from inputs $x$ into outputs such as an ($L$+1)-layer neural network $\mathcal{F}(x, {\bm\theta}) = W^{(L+1)}\phi(W^{(L)}(\cdots\phi(W^{(1)} x+b^{(1)})\cdots)+b^{(L)}))+b^{(L+1)}$, where the activation function $\phi(\cdot)$ and $\bm\theta$ include the weights $\bm\theta_w = W^{(1)},\ldots, W^{(L+1)}$ and bias $\bm\theta_b = b^{(1)},\ldots, b^{(L+1)}$.%, and are trained to minimize a loss function.
\subsection{Minimal Representation of the Dipole Solution}
\label{minimal_representation_dipole_solution}
%これは合成させると振幅と位相差を変数としていることになる。この位相差以外の任意の時刻からのスタートが解の無限化につながっている。それを無くせばただのnonconvex問題。そこにKTT条件。さらに,冗長な定式なので一つの評価関数に対して必ず無限個の解が存在する.それを示す.
Since we rely on the first-order averaged model in Section \ref{EMFF}, arbitrary ${m}(\mu^{\mathrm{amp}}_{N},\bm\psi_N+\psi_0,t)$ with free parameter $\psi_0$ and ${s}_{j},{c}_{j}$ for all $j,k\in[1,n]$ and $l,m\in\{1,2,3\}$, 
%We rewrite  ${s}_{j}\in\mathbb{R}^3$ and ${c}_{j}\in\mathbb{R}^3$ to express this redundant of the time-varying dipole $\mu_j(t)$: 
\begin{equation}
\label{new_definition_infinite}
\begin{aligned}
&{s}_{j(l)}(\mu^{\mathrm{amp}}_N,\bm\psi_N,\psi_0)=\mu^{\mathrm{amp}}_{j(l)}\cos(N_c\pi+\psi_0+\bm\psi_{j(l)})\\
&{c}_{j(l)}(\mu^{\mathrm{amp}}_N,\bm\psi_N,\psi_0)=\mu^{\mathrm{amp}}_{j(l)}\sin(N_s\pi+\psi_0+\bm\psi_{j(l)})
\end{aligned}
\end{equation}
yields the same results as those in (\ref{basic_formulation}). Next, we introduce a positive definite matrix variable $\mathfrak{X} \in\mathbb{\mathbb{P}}^{3n \times 3n}$ as follows:
$$
%\begin{aligned}
\mathfrak{X}\triangleq\int_T \frac{\|m(\tau)\|_2^2}{1/2}\frac{\mathrm{d}\tau}{T}%=s_N s_N^\top +c_N c_N^\top 
=
\begin{bmatrix}
    s_N&c_N
\end{bmatrix}
\begin{bmatrix}
    s_N&c_N
\end{bmatrix}^\top,
%\end{aligned}
$$
where the final equality denotes $\mathrm{rank}(\mathfrak{X})\leq 2$. Although this extends $s_N,c_N\in\mathbb{R}^{3n}$ to $\mathfrak{X} \in\mathbb{\mathbb{P}}^{3n \times 3n}$, a vecotr $\mathtt{x}(\mathfrak{X})\in\mathbb{R}^{9n}$% for a given $\mathfrak{X}$
\begin{equation}
\label{mathfrak_y_reconstructing}
\mathtt{x}(\mathfrak{X})= [\mathrm{Diag}(\mathfrak{X});\mathfrak{X}_{(:,1)};\mathfrak{X}_{(:,2)}]  
\end{equation}
recovers $3n$-dimensional information by reconstructing the trigonometric terms of each phase. 
\begin{lemma}[Reconstructing Phase $\bm\psi$ from $\mathfrak{X}$]
\label{lemma_reconstruction_phase_vector_fram_X}
For a given matrix $\mathfrak{X}\in\mathbb{R}^{3n\times 3n}$, we can derive $\mu^{\mathrm{amp}}_{N}\in\mathbb{R}^{3n}$ and define the phase difference matrix $\mathrm{COS}_{\mathfrak{X}}\in\mathbb{R}^{3n\times 3n}$ as
\begin{equation}
\label{mu_amp_N_COS_N}
\mu^{\mathrm{amp}}_{N}=\sqrt{\mathrm{Diag}(\mathfrak{X})},\quad \mathrm{COS}_{\mathfrak{X}}\triangleq\mathfrak{X} \oslash (\mu^{\mathrm{amp}}_{N}\mu^{\mathrm{amp}\top}_{N}),
\end{equation}
where $\oslash$ denotes element-wise (Hadamard) division. Then, 
%we obtain $s_N,\quad c_N\in\mathbb{R}^{3n}$ as
\begin{equation}
\label{reconstruction_c_s_N}
\begin{aligned}
\bm\psi_{1(1)}&\triangleq 0,\quad c_{N0}\triangleq c_{N}(\mu^{\mathrm{amp}}_N,\bm\psi_N,0)= \mu^{\mathrm{amp}}_{N}\otimes\mathrm{COS}_{\mathfrak{X}(:,1)}\\
s_{N0}&\triangleq s_N (\mu^{\mathrm{amp}}_N,\bm\psi_N,0)= \mu^{\mathrm{amp}}_{N}\otimes\left(\mathrm{sign}(\sin\bm\psi_{2})\sin\bm\psi_N\right),\\
\end{aligned}
\end{equation}
where $\otimes$ denotes the element-wise (Hadamard) product and 
\begin{equation}
\label{sin_psi_x}
\sin\bm\psi_{x}\triangleq\frac{\mathrm{COS}_{\mathfrak{X}(x,2)}-\mathrm{COS}_{\mathfrak{X}(x,1)}\mathrm{COS}_{\mathfrak{X}(2,1)}}{\sqrt{1-\mathrm{COS}_{\mathfrak{X}(2,1)}^2}}.
\end{equation}
\end{lemma}
%\begin{proof}See Appendix~\ref{proof_reconstruction_phase_vector_fram_X}.\end{proof}
\begin{proof}
For readability, we index the $3n$ components using a single index $x,y\in\{1,\dots,3n\}$ corresponding to $(j,l),(k,m)$: 
$$
\bm\psi_{x} \triangleq \bm\psi_{j(l)},\quad\bm\psi_{y} \triangleq \bm\psi_{k(m)},\quad
\mathfrak{X}_{(x,y)}\triangleq\mathfrak{X}_{(3(j-1)+l,3(k-1)+m)}.
$$
Note that $\mathfrak{X}$ does not explicitly contain the phases themselves but only their differences because
\begin{equation}
\label{no_phase_difference}
\mathfrak{X}_{(x,y)}=\mu^{\mathrm{amp}}_{x}\mu^{\mathrm{amp}}_{y}
\cos(\bm\psi_{x}-\bm\psi_{y})
\end{equation}
holds for all $x,y\in[1,3n]$. Then, we can derive $\mu^{\mathrm{amp}}_{N}\in\mathbb{R}^{3n}$ and define the phase-difference information matrix $\mathrm{COS}_{\mathfrak{X}}\in\mathbb{R}^{3n\times 3n}$ in (\ref{mu_amp_N_COS_N}). The definition in (\ref{new_definition_infinite}) allows us to assume $\bm\psi_{1}\triangleq 0$ and $\mathrm{sign}(\sin\bm\psi_{2})\triangleq 1$ without loss of generality. The given $\mathrm{COS}_{\mathfrak{X}}$ satisfies $\mathrm{COS}_{\mathfrak{X}(x,y)}=\cos(\bm\psi_{x}-\bm\psi_{y})=\cos\bm\psi_{x}\cos\bm\psi_{y}+\sin\bm\psi_{x}\sin\bm\psi_{y}$ based on (\ref{no_phase_difference}). Subsequently, applying $y=1$ yields $\cos\bm\psi_{x}
%=\cos(|\bm\psi_{x}-\bm\psi_{1}|)
=\mathrm{COS}_{\mathfrak{X}(x,1)}$, and this recovers $c_{N}$ in (\ref{reconstruction_c_s_N}). Moreover, (\ref{no_phase_difference}) yields $\mathrm{COS}_{\mathfrak{X}(x,2)}=\cos(\bm\psi_{x}-\bm\psi_{2})=\cos\bm\psi_{x}\cos\bm\psi_{2}+\sin\bm\psi_{x}\sin\bm\psi_{2}=\mathrm{COS}_{\mathfrak{X}(x,1)}\mathrm{COS}_{\mathfrak{X}(2,1)}+\sin\bm\psi_{x}\sqrt{1-\mathrm{COS}_{\mathfrak{X}(2,1)}^2}$. Since we have $\cos\bm\psi_{2}=\mathrm{COS}_{\mathfrak{X}(2,1)}$, applying $\sin\bm\psi_{2}=\sqrt{1-\cos^2\bm\psi_{2}}$ obtaines $\sin\bm\psi_{x}$ in (\ref{sin_psi_x}). To satisfy $\mathrm{sign}(\sin\bm\psi_{2})\triangleq 1$, we define $s_{N}$ in (\ref{reconstruction_c_s_N}).
\end{proof}
%We set
%$$
%\bm\psi_{1(1)}=0\Rightarrow
%\cos\theta_{k(m)}=Cos\mathfrak{D}\theta_{(1,3(k-1)+m)}
%$$
%\begin{equation}
%\label{eq_dipole}
%\begin{aligned}
%&\mathcal{EQ}_{\mu}:
%\theta^*\ \mathrm{s.t.}\left\{
%\begin{aligned}
%&\mathfrak{D}\theta_{(3(j-1)+l,3(k-1)+m)}=|\theta_{j(l)}-\theta_{k(m)}|\\
%&|\theta_{j(l)}|\leq \frac{\pi}{2}\ \mathrm{for}\ j,k\in[1,n],\ l,m\in[1,3]
%\end{aligned}
%\right.
%\end{aligned}
%\end{equation}
%本来であれば,ラグランジュの未定乗数法を使ってもnonconvexというか双線形方程式は双線形のままであり,何も簡略化されない.特にDCの場合は簡略化された後,残った多項式群は自由度的に解くことができず,その最適解を満たす解が存在しない.ACの場合は解けるけどそこで止まる.しかし,ACの平均化ではそこに隠れ制約があり,それを使うとなんと双線形をconvex setに変換できる.その結果,平均的だけどsuccessfullyに解ける.
%交流を入れると問題の次数的に全ての等式制約を満たす解がfeasibleになるだけでQCQPの解法に対して寄与はしない.いずれにせよ,rank(2)として残る.もしかしたらn=2のときに限って何かがうまく作用して常にrank(2)になるかも.でもnが3以上のときはQCQPのrank-constrained問題でしかないかも
%このformulationだとrank1制約をconvex制約に緩和できる.
%Dahrouj, H. and Yu, W., 2010. Coordinated beamforming for the multicell multi-antenna wireless system. IEEE transactions on wireless communications, 9(5), pp.1748-1759.
%Huang, Y. and Palomar, D.P., 2009. Rank-constrained separable semidefinite programming with applications to optimal beamforming. IEEE Transactions on Signal Processing, 58(2), pp.664-678.
\subsection{Continuous Power-Optimal Dipole Solutions}
We construct a continuous family of locally optimal solutions for the power-optimal allocation using sequential convex programming. This ensures smooth transitions between neighboring optimal configurations and is used to build the MLPs, as described in the next subsection. We consider an input $u^{a}=[u_{1}^a;\ldots;u_{n}^a]\in\mathbb{R}^{6n}$ for $n$ agents where $u_{j}^{a}=\frac{\mu_0}{8\pi}\sum_{k\neq j}u_{j\leftarrow k}^{a}$ by (\ref{time_averaged_LOS}). Then, the optimization problem for the power-optimal dipole solution of $n$ agents is
\begin{equation}
\label{opt1-N}
%\left\{
\begin{aligned}
\mathrm{min\ }&J(m)=\frac{1}{2}m^\top Wm=\frac{1}{2}(s_N^\top W s_N+c_N^\top W c_N)\\
\mathrm{s.t.\ }
%&\sum_{k\neq j}Q_{d_{jk}}^a ({s_{k}^a}\otimes {s_{j}^a}+{c_{k}^a}\otimes{c_{j}^a})=\frac{8\pi}{\mu_0}u_{j}^{a}\\
&\left\{
\begin{aligned}
&\ \forall i\in[1,6],\ j\in[1,n-1],\ \\
&\frac{u_{j(i)}^{a}}{\mu_0/8\pi}=\sum_{k\neq j}Q_{{j\leftarrow  k}(i,:)}^a({s_{k}^a}\otimes{s_{j}^a}+{c_{k}^a}\otimes{c_{j}^a})
%\mathrm{tr}\left[\left(s_{j}^{a}\hat{s}_j^{a\top} +c_{j}^{a}\hat{c}_j^{a\top} \right) {\mathcal{Q}}_{j[i]}^{\top}\right]
\\
\end{aligned}
\right.
\end{aligned},
%\right.
\end{equation}
where the weight matrix is $W$. Note that the $u_{j}^{a}$ are divided by $\mu_0/8\pi=5e^{-8}$ %Tm/A
to avoid excessive trial-and-error calculations. We define $\mathcal{Q}_{j\leftarrow k[i]}\in\mathbb{R}^{3\times 3}$ and ${\mathcal{Q}}_{j[i]}\in\mathbb{R}^{3\times 3n}$ that satisfy 
$$
%\begin{aligned}
\mathrm{vec}(\mathcal{Q}_{j\leftarrow k[i]})=Q^{\top}_{{j\leftarrow  k}(i,:)},\quad %\in\mathbb{R}^{9},\ 
{\mathcal{Q}}_{j[i]}\triangleq\left[\mathcal{Q}_{{j\leftarrow 1}[i]},\ldots,\mathcal{Q}_{{j\leftarrow n}[i]}\right],
%\end{aligned},
$$
%$$\begin{aligned}\hat{s}_{j}&\triangleq[s_{(1:3j-3)};s_{(3j+1:3n)}],\ \hat{c}_{j}=[c_{(1:3j-3)};c_{(3j+1:3n)}]\\{\mathcal{Q}}_{j[i]}&\triangleq\left[\mathcal{Q}_{{j\leftarrow 1}[i]},\ldots,\mathcal{Q}_{{j\leftarrow (j-1)}[i]},\mathcal{Q}_{{{j\leftarrow (j+1)}[i]}},\ldots,\mathcal{Q}_{{j\leftarrow n}[i]}\right].\end{aligned}$$
Applying $\mathfrak{X}$ in Section~\ref{minimal_representation_dipole_solution} and Roth's column lemma 
%\cite{roth1934direct} 
derives the trace constraints for $i_{\in[1,6]}$th component of $u_j^{a}$ in (\ref{opt1-N}):
\begin{equation*}
%\label{new_constraints}
\begin{aligned}
\frac{u_{j(i)}^{a}}{\mu_0/8\pi}
%&=\sum_{k\neq j}Q_{{j\leftarrow  k}(i,:)}^a({s_{k}^a}\otimes{s_{j}^a}+{c_{k}^a}\otimes{c_{j}^a})\\
%&=\sum_{k\neq j}\mathrm{tr}\left[\mathcal{Q}_{j\leftarrow k[i]}^\top(s_{j}^{a}{s_{k}^{a\top}}+c_{j}^{a}{c_{k}^{a\top}})\right]\\
&=\mathrm{tr}\left[s_{j}^{a}\sum_{k\neq j}\left(\mathcal{Q}_{j\leftarrow k[i]}{s_{k}^{a}}\right)^\top +c_{j}^{a}\sum_{k\neq j}\left( \mathcal{Q}_{j\leftarrow k[i]}{c_{k}^{a}}\right)^\top\right]\\
&=\mathrm{tr}\left[K_j \left(s_{N}^{a}(\hat{K}_j{s}_N^{a})^\top +c_{N}^{a}(\hat{K}_j{c}_N^{a})^\top \right) ({\mathcal{Q}}_{j[i]}\hat{K}_j^\top)^{\top}\right]\\
%&=\mathrm{tr}\left[\left(s_{N}^{a}{s}_N^{a\top}+c_{N}^{a}{c}_N^{a\top} \right) \hat{K}_j^\top \hat{K}_j{\mathcal{Q}}_{j[i]}^{\top}K_j\right]\\
%&=\mathrm{tr}\left[\mathfrak{X}\left(\hat{K}_j^\top \hat{K}_j{\mathcal{Q}}_{j[i]}^{\top}K_j\right)\right]\\
&=\mathrm{tr}\left[\mathfrak{X} \ {\mathcal{R}}_{j[i]}\right],\quad\ {\mathcal{R}}_{j[i]}\triangleq\hat{K}_j^\top \hat{K}_j{\mathcal{Q}}_{j[i]}^{\top}K_j\in\mathbb{R}^{3n\times 3n}
%&=\mathrm{tr}\left[\mathfrak{X}_{[3j-2:3j,[1:3j-3,3j+1:3n]]} \hat{\mathcal{Q}}_{j[i]}^{\top}\right]
\end{aligned}
%\right.
\end{equation*}
where $K_j\in\mathbb{R}^{3\times 3n}$ and $\hat{K}_j\in\mathbb{R}^{(3n-3)\times 3n}$ are
$$
\begin{aligned}
K_j &= 
\begin{bmatrix}
O_{3\times 3(j-1)}
&I_3&O_{3\times 3(n-j)}
\end{bmatrix}\\
\hat{K}_j &= 
\begin{bmatrix}
I_{3(j-1)}&O_{3(j-1)\times 3}&O_{3(j-1)\times 3(n-j)}\\
O_{3(n-j)\times 3(j-1)}
&O_{3(n-j)\times 3}&I_{3(n-j)}\\
\end{bmatrix}
\end{aligned}.
$$
We convexify the primal problem in (\ref{opt1-N}) by the standard (Shor) semidefinite program relaxation \cite{boyd2004convex} to derive the unique $\mathfrak{X}^*$ of the power-optimal solutions as follows:
\begin{equation}
\label{opt_final_0}
\begin{aligned}
&{\mathrm{ODA}_n^{(0)}}:\ \mathfrak{X}_0^*=\argmin_{\substack{\mathfrak{X}\in\mathbb{S}^{3n\times 3n}}}\mathrm{trace}(W\mathfrak{X})\\
&\mathrm{s.t.}\ 
%\left\{\begin{aligned}&
\frac{u_{j(i)}^{a}}{\mu_0/8\pi}=\mathrm{tr}\left[
\mathfrak{X}_0\ {\mathcal{R}}_{j[i]}
%\mathfrak{X}_{[3j-2:3j,[1:3j-3,3j+1:3n]]} \hat{\mathcal{Q}}_{j[i]}^{\top}
\right],\ \forall i_{\in[1,6]},\ \forall j_{\in[1,n-1]}%\right|\leq \varepsilon_0
%\end{aligned}\right.
\end{aligned}.
\end{equation}
As $\mathfrak{X}_0^*$ does not necessarily satisfy the rank conditions $\mathrm{rank}(\mathfrak{X}_0^*)\leq 2$ as mentioned in Section~\ref{minimal_representation_dipole_solution}, we reduce its rank using rank-constrained optimization \cite{sun2017rank}. We solve the sequential convex programming problem for iterations $k>0$:% as follows:
\begin{equation}
\label{opt_final}
\begin{aligned}
%\mathcal{OPT}_{\mathrm{ODA}}:\ 
&{\mathrm{ODA}_n^{(k)}}:\  \mathfrak{X}_{k}^*=\argmin_{\substack{\mathfrak{X}_k\in\mathbb{P}^{3n\times 3n}, e_{k}>0}}\mathrm{trace}(W\mathfrak{X}_k)+w_k e_k\\
&\mathrm{s.t.}\left\{
\begin{aligned}
&\frac{u_{j(i)}^{a}}{\mu_0/8\pi}=\mathrm{tr}\left[
\mathfrak{X}_{k}\ {\mathcal{R}}_{j[i]}
%\mathfrak{X}_{[3j-2:3j,[1:3j-3,3j+1:3n]]} \hat{\mathcal{Q}}_{j[i]}^{\top}
\right],\ \forall i_{\in[1,6]},\ \forall j_{\in[1,n-1]}\\
%&P_{{\lambda}^*}\mathfrak{X}=0\\
& V_{k-1}^{*\top} \mathfrak{X}_{k} V_{k-1}^*\preceq e_{k} I_{3n-2},\quad e_{k} \leq e_{k-1}^*,
%\varepsilon &\geq \left|\frac{8\pi}{\mu_0}u_{j}^{a}(i)+\mathrm{tr}\left[(\hat{P}_{j}^\sharp P_{j})^\top\hat{\mathcal{Q}}_{j}^{i\top}\mathfrak{X}_j\right]\right|\\
%0&\leq \mathfrak{X}_j(1,1),\mathfrak{X}_j(2,2),\mathfrak{X}_j(3,3)\\ 
%J_{dN}^*&=\frac{1}{2}\mathrm{tr}\left[\left(I_3 + (\hat{P}_{j}^\sharp P_{j})^\top\hat{P}_{j}^\sharp P_{j}\right)\mathfrak{X}_j\right]\\
%J_{dN}^*&=\frac{1}{2}\sum_j \mathrm{tr}\left[\mathfrak{X}_j\right]\\
\end{aligned}
\right.
\end{aligned}
\end{equation}
where $V_{k-1}^* \in \mathbb{R}^{3n \times (3n-2)}$ denotes the orthonormal eigenvectors of $\mathfrak{X}_{k-1}^*$ associated with its $(3n\mathrm{-}2)$ smallest eigenvalues.
%from the previous iteration. 
For matrices $\mathfrak{X}^*_k$ that have converged after $k$ iterations, Lemma~\ref{lemma_reconstruction_phase_vector_fram_X} converts $\mathfrak{X}_{k}^*$ into optimal dipole solutions $m_N^*\in\mathbb{R}^{6N}$. Our trials declared convergence when the third-smallest eigenvalue fell below a certain threshold and stabilized.
%with the submatrix of $P_{{\lambda}^*}$,  $\hat{P}_{j*}$:
%\begin{aligned}
%P_{j*}&=P_{{\lambda}^*}\ _{(:,\ 3j-2:3j)}\in\mathb{R}^{3n\times 3}\\
%\hat{P}_{j*}=\begin{bmatrix}P_{{\lambda}^*}\ _{(:,\ 1:3j-3)},&P_{{\lambda}^*}\ _{(:,\ 3j+1:3n)}  \end{bmatrix}\in\mathbb{R}^{3n\times 3n-3}.
%\hat{\mathcal{Q}}_{j}^i&=
%\begin{bmatrix}
%\mathcal{Q}_{d_{j1}}^i&\ldots&\mathcal{Q}_{d_{j(j-1)}}^i,&\mathcal{Q}_{d_{j(j+1)}}^i&\ldots&\mathcal{Q}_{d_{jN}}^i
%\end{bmatrix}.
%\end{aligned}
%where the non-symmetric matrix $\mathcal{Q}_{j\leftarrow k[i]}\in\mathbb{R}^{3\times 3}$ that satisfies $\mathrm{vec}(\mathcal{Q}_{j\leftarrow k[i]})=Q^a_{d_{jk}}(i,:)^\top$.
%\begin{remark}
%\label{conversion_X_into_sc}
%Without any loss of generality, there exists a $\delta_a^*\in\mathbb{R}^3$ corresponding to a positive vector $A^*\in\mathbb{R}_+^3$. Thus, we define $A^*$ based on the diagnal elements of $\mathfrak{X}_j$ and the relationship of trigonometric function concludes (\ref{optimal_s_a_s_L}) for associated $\delta_j^*$. We choose an arbitrary index $j$ and derives
\subsection{Learning-based Dipole Allocation Model}
\label{NODA_section}
This subsection summarizes the offline training method of the learning-based optimal dipole model NODA. To mitigate the curse of dimensionality for training, we chose the base frame as the line-of-sight frame $\{\mathcal{LOS}_{j\leftarrow k}\}$ of an arbitrary $k$th follower. The formulation in (\ref{time_averaged_LOS}) reduces $[r^a_{j\leftarrow k};u^{a}_{j}]\in\mathbb{R}^9$ to $[d_{j\leftarrow k};{f^{a}_{j(1,2)}};\tau^{a}_{j}]\in\mathbb{R}^6$.  Then, the NODA model takes $9(N\mathrm{-}1)\mathrm{-}3$ dimensional agent states of the specific $j_{\in\mathcal{V}_l}$th leader agents and yields the optimal $\mathtt{x}^*$ in (\ref{mathfrak_y_reconstructing}) as follows:
$$
%\begin{aligned}
%\mathfrak{X}_1^*,\ {\lambda}^*
\mathtt{x}^*
%s_{N0}^*,\ c_{N0}^*
=\mathfrak{M}([{r}^{l_{j\leftarrow k}},{f}^{l_{j\leftarrow k}},{\tau}^{l_{j\leftarrow k}}],{\bm\theta}) \quad\mathrm{s.t.}\quad\text{(\ref{opt1-N})},%\\
%\mathfrak{X}_j^*,\ {\lambda}^*&=\mathcal{M}_{\mathrm{KC-NODA}}(\hat{r}^{l},\hat{\dot{r}}^{l},\hat{\tau}^{l},{\theta})
%\end{aligned}
$$ 
to derive dipole solutions $s_{N0}^*,\ c_{N0}^*$ via Lemma~\ref{lemma_reconstruction_phase_vector_fram_X}. Although this NODA model $\mathfrak{M}$ is naturally controller-invariant, limiting the input norm via gain-tuning and guidance strategy reduces the sample region of ${f}$ and ${\tau}$ to deal with the curse-of-dimensionality problem. Moreover, the distance-dependent magnetic field property naturally restricts ${r}$ to a small region.
\begin{remark}
NODA is attitude-invariant when the far-field model is used. An exact model \cite{takahashi2025coil} includes the attitude information, thereby increasing the input dimensionality; otherwise, we decouple and compensate through controller design \cite{takahashi2025noda_mmh}. 
\end{remark}
\begin{remark}
\label{No_discontinuously_switching}
Discontinuously switching $m_{(t_-)}$ into $m_{(t_+)}$ at time $t$ induces an impulse input that may excite disturbances in the high-frequency band%or cause chattering
. The free parameter $\psi_0$ can address this problem such that $\psi_0^* =\argmin \|m_{(t_-)}-m_{(t_+,\psi_0)}\|$. %In the next subsection, we construct continuous gains using neural techniques to alleviate the additional chattering by discontinuously switching.
\end{remark}
%On the other hand, the output labels depend on the fourth power of position, and normalizing them can result in a significant variation in magnitude, including minimal values. To address this and improve the stability of the learning process, a logarithmic transformation is applied to compress the scale of the output labels.
%KC-NODAは異なる状態量でも入力制約でも同じ制御力が出てしまうのでlipchitz constantの制約が重要
%Learning an Approximate Model Predictive Controller With Guarantees
%Diagram of the proposed framework. We design an MPC πMPC with robustness to input disturbance d. The resulting feedback law is sampled offline and approximated (πapprox) via supervised learning. Hoeffding’s Inequality is used for validation and yields a bound on the error between approximate and original MPC in order to guarantee stability and constraint satisfaction. The result is an approximate MPC with statistical guarantee.
%\subsection{Overview of Guidance and Control Architecture for Electromagnetic Spacecraft Swarm}
%\begin{figure}[tb!]
%\centering
%\includegraphics[width=8truecm]{Figure/system_diagram_1008.pdf}
%\caption{System Diagram of Autonomous Relative Orbit Formation Architecture for Ultra-Close Nanoagent Swarm with Magnetic Torquer}
%\end{figure}
 %As shown in subsection~\ref{EMFF}, in first-order averaged dynamics. 
%Learning an Approximate Model Predictive Controller With Guarantees
%Diagram of the proposed framework. We design an MPC πMPC with robustness to input disturbance d. The resulting feedback law is sampled offline and approximated (πapprox) via supervised learning. Hoeffding’s Inequality is used for validation and yields a bound on the error between approximate and original MPC in order to guarantee stability and constraint satisfaction. The result is an approximate MPC with statistical guarantees.
%\subsection{Overview of Guidance and Control Architecture for Electromagnetic Spacecraft Swarm}
%\begin{figure}[tb!]
%\centering
%\includegraphics[width=8truecm]{Figure/system_diagram_1008.pdf}
%\caption{System Diagram of Autonomous Relative Orbit Formation Architecture for Ultra-Close Nanosatellite Swarm with Magnetic Torquer}
%\end{figure}
 %As shown in subsection~\ref{EMFF}, in first-order averaged dynamics. 
 \begin{algorithm}[tb!]
\begin{algorithmic}[1] 
\STATE  \textbf{Inputs: }position/command $\mathbb{r}_{N}\in\mathbb{R}^{3n},\mathbb{u}_{N}\in\mathbb{R}^{6n}$%, $j\in[1, n]$ 
\STATE  \textbf{Outputs: }
%$\mu^{\mathrm{amp}*}_{N},\theta_N^*\in\mathbb{R}^{3n}$ and 
$\mu_j^{a*}(t)\in\mathbb{R}^3$, $j\in[1, n]$
\STATE  Define the base frame $\{\mathcal{A}\}$ and derive $r_j^a$ and $u_j^a$
%\STATE 距離が直径以上とCLOSのエラー対策(NODA ganerationのコード), $ODA2$で制御力一成分だけでトルク0の場合は行列のランクは1にしないとエラーが起きる.どう判別?$eig(hatP_lambda)$の0の重解-1をその値に設定.多項式の個数から判断できる
\STATE  Solve dual problem in (\ref{opt2_N}) for $J_{\mathrm{dual}}$
\STATE  Solve ODA$_n$ in (\ref{opt_final_0}) and~(\ref{opt_final}) for $\mathfrak{X}^*$ and extract $\mathtt{x}$ %$\mathfrak{X}^*=\ \mathcal{OPT}_{\mathrm{ODA}}\{r_{j}^{a},u_{j}^{a},P_{{\lambda}^*}\}$% on (\ref{opt_final}) with $|\frac{8\pi}{\mu_0}u_{j[i]}^{a}-\mathrm{tr}\left[\mathfrak{X}{\mathcal{R}}_{j[i]}\right]|\leq \varepsilon_0$ for numerical safety.
%\STATE  Solve $\mu^{\mathrm{amp}*}_{N},\ \theta_N^*=\mathcal{EQ}_{\mu}\{\mathfrak{X}^*\}$.
\STATE Recover $s_{N0}^*,\ c_{N0}^*$ from $\mathfrak{X}^*$ by Lemma~\ref{lemma_reconstruction_phase_vector_fram_X}% with $\forall x,y, \mathrm{COS}_{\mathfrak{X}(x,y)}=\max(-1,\,\min(1,\,\mathrm{COS}_{\mathfrak{X}(x,y)}))$ and $\sin\bm\psi_{x}=\max(-1,\,\min(1,\,\sin\bm\psi_{x}))$ for numerical safety.
\STATE Choose $\psi_0$, e.g., Remark~\ref{No_discontinuously_switching} and derive $\mu_j^{a*}(t)$ in (\ref{new_definition_infinite})%$=\mu^{\mathrm{amp}*}_j\sin(\omega_{j}t+\bm\psi_j^*+\psi_0)$
\end{algorithmic}
\caption{Convex optimization-based power-optimal dipole allocation for 6-DoF MTQ control of $n$-agents.}
\label{alg:Convex_Optimization_based_Power_Optimal_Dipole_Allocation}
\end{algorithm}
\section{Validation of NODA Model}
\label{Experimental_Validation}
We validated our learning-based dipole allocation algorithm and multileader-based decentralization through numerical simulations and experiments on formation and attitude control. Our control goal was to achieve a regular triangle with a distance $r_d$ and their attitudes aligned with the others. 
%Ave Ave. $\int\frac{u_c-u_{\mathrm{avg}}}{u_c}$, Inte Inte$\int\frac{u_c-\int_T u(t)}{u_c}$, 
\subsection{Two-Dimensional Micro-Gravity Experiment Setup}
\label{experimental_setup_introduction}
We conduct numerical simulations replicating the ground-based environment \cite{takahashi2025noda_mmh} in Fig~\ref{fig:3MTQ_setup} as shown in Fig.~\ref{fig:3MTQ_experiment_concept}. This subsection briefly overviews the setup; a full description can be found in \cite{takahashi2025noda_mmh}. The testbed consists of a two-axis MTQ capable of generating horizontal-plane magnetic fields with one coil incorporating an iron core. The microcomputers achieve time synchronization via the GPS pulse-per-second signal with 0.1 ms accuracy. Then, we set the candidates of $\omega_{j}$ as $\omega_{j}=8\pi\times$\{1:5\} rad/s for the time-averaged model. As illustrated in Fig.~\ref{fig:3MTQ_setup}, MTQ assemblies were mounted on a linear air track and a single-axis air bearing. %We describe our experimental assumptions. 
%In preparation for the 6-DoF control experiment, 
We partially justified their mechanical constraints %of the linear track and air bearing using 
the following assumptions. %Note that the conservation of linear momentum is rational because of the immobility of MTQ $c$. 
%\begin{assumption}The MTQs on the linear air track have virtual 3-axis RWs that cancel their generated torques, compensating for their non-rotatability. Hence, the virtual states of their angular momentum $\xi_j^{b_j}=(h_j^{b_j}-L^{b_j}/2)$ are stabilized to zero to avoid excessive bias between them.%, we assume they have the circular triaxial MTQ, and the second and third agents are equipped with 3-axis RWs. Since we should avoid the excessive bias between their angular momentum, we control the bias states about their angular momentum $\xi_j^{b_j}=(h_j^{b_j}-L^{b_j}/2)$. \end{assumption}
\begin{assumption}
\label{6DoF_control_assumption}
Forces and torques are parallel to the movable directions of the linear air track and air bearing, respectively.
%The forces act along the linear air-track direction, while torques are restricted to out-of-plane motion.
%The forces is parallel to the movable directions of the linear air track and the torques are restricted to the out-of-plane direction.
\end{assumption}
\noindent
The base frame was defined as follows: the $y$-axis was along the linear track, and the $z$-axis was normal to the table. %, with the positive direction upward. 
The origin was set such that the $y$ axis penetrated the geometric center of MTQs $b\&c$, and the $x$ axis completed a right-handed orthogonal frame. The positions of the MTQs were defined as $p_{a}\triangleq[\frac{\sqrt{3}}{2}r_d;\frac{L_{\mathrm{air}}}{2};0]$ and $p_{b,c}\triangleq[0;y_{b,c};0]$ where $y_{b}<y_{c}$.
%, as shown in Fig.~\ref{fig:final}.
%Their desired constant values with a ``virtual RW'' were $y_{bd,cd}= (L_{\mathrm{air}}\mp r_d)/{2}$ and $\theta_{3d}=\xi_{1d}^{b_1}=\xi_{2d}^{b_2}=0$. 
%and their differential values %$\dot{y}_{1d,2d}$ and $\dot{\theta}_{3d}$ are set to 0. 
%The relative distance was measured using a time-of-flight sensor, and the AR marker information provided the angle of MTQ $a$. 
\begin{figure}[t]
\centering
\begin{minipage}[b]{0.49\columnwidth}
  \centering
  \begin{minipage}[b]{\linewidth}
    \centering
    \subfloat[Parameters of three coils.]{\includegraphics[width=0.9\linewidth]{Figure/3MTQ_experiment_concept.pdf}
    %\subcaption{(a) attitude}
    \label{fig:3MTQ_experiment_concept}}
  \end{minipage}
  \begin{minipage}[b]{\linewidth}
    \centering
    \subfloat[Numerical attitude control.]{\includegraphics[width=1\linewidth]{Figure/PDcontrol_results_attitudes.pdf}
    \label{Numerical_PDcontrol_results_attitudes_picture}}
  \end{minipage}
\end{minipage}
%\hfill
% 左半分:(a)
\begin{minipage}[b]{0.49\columnwidth}
  \centering
  \subfloat[Numerical position controls.]{\includegraphics[width=1\linewidth]{Figure/PDcontrol_results_positions.pdf}
  \label{Numerical_PDcontrol_results_positions_picture}}
\end{minipage}
\begin{minipage}[b]{0.49\columnwidth}
  \centering
  \begin{minipage}[b]{\linewidth}
    \centering
    \subfloat[Experiment setup \cite{takahashi2025noda_mmh}.]{\includegraphics[width=0.85\linewidth]
    {Figure/experimental_picture_final_3MTQ.pdf}
    %{Figure/NODA3_power_experiment_output.pdf}
    %\subcaption{(a) attitude}
    \label{fig:3MTQ_setup}
    }
  \end{minipage}
  \begin{minipage}[b]{\linewidth}
    \centering
    \subfloat[Decentralized attitude controls.]{\includegraphics[width=1\linewidth]{Figure/Inverse_attitude_experiment_output_4}
    \label{Experimental_PDcontrol_results_attitudes_picture}
    }
  \end{minipage}
\end{minipage}
%\hfill
% 左半分:(a)
\begin{minipage}[b]{0.49\columnwidth}
  \centering
  \subfloat[Decentralized position controls.]{\includegraphics[width=1\linewidth]{Figure/Inverse_position_experiment_output_4}
  \label{Experimental_PDcontrol_results_positions_picture}
  }
\end{minipage}
\caption{Numerical simulation results in Figs~\ref{Numerical_PDcontrol_results_attitudes_picture} and \ref{Numerical_PDcontrol_results_positions_picture} and experimental results using the linear air track and single-axis air bearing \cite{takahashi2025noda_mmh} in Figs~\ref{Experimental_PDcontrol_results_attitudes_picture} and \ref{Experimental_PDcontrol_results_positions_picture} applied the controller and decentralized allocation in subsection~\ref{controller_dipole_allocation_design}.
\label{Inverse_experiment_results}}
\end{figure}
\subsection{Kinematics Controller and Dipole Allocation Design}
\label{controller_dipole_allocation_design}
We designed a formation and attitude controller using Theorem~\ref{theorem_experimental_controller} and the multileader-based control introduced in Section~\ref{NODA_subsection}. For a decentralized allocation, we separated the three coils into three groups, and each pair of MTQs, $a$--$b$, $b$--$c$, and $c$--$a$, is controlled in a decentralized manner via different frequencies. Their adjacency matrices are
$$
\begin{aligned}
A_1=
\begin{bmatrix}
     0&1&0\\
     1&0&0\\
     0&0&0
\end{bmatrix},\ 
A_2=
\begin{bmatrix}
     0&0&0\\
     0&0&1\\
     0&1&0
\end{bmatrix},\ 
A_3=
\begin{bmatrix}
     0&0&1\\
     0&0&0\\
     1&0&0
\end{bmatrix}.
\end{aligned}
$$
%which define centralized adjacency matrix $A=\sum_{i=1}^3A_i$. 
and their Laplacian matrix $L_i\triangleq\mathrm{Diag}(A_i\bm{1})-A_i$ with the vector of all ones $\bm{1}$ derives auxiliary commands $\mathfrak{u}_{c[i]}^a \in\mathbb{R}^{24}$:
$$
\mathfrak{u}_{c[i]}^a=\begin{bmatrix}
    {f}_{c[i]}^{a}\\
    {\tau}_{c[i]}^{a}\\
    {\dot{h}}_{c1}^{b_1}]
\end{bmatrix}
=
%\begin{bmatrix}
-(L_i\otimes I_9)(K_{p}(x-x_d)+K_{d}(\dot{x}-\dot{x}_d)),
%-k_{\xi}\xi_l^{b_l}=-k_{\xi}(h_l^{b_l}- L^{b_l}_{\mathsf{g}}/2)
%\end{bmatrix}
$$
which includes $x=[p_N;\theta_N;\xi_1^{b_1};\xi_2^{b_2}]$. %\in\mathbb{R}^{24}$. 
%, position vectors $p_N$, and attitude parameters $\theta_N$. 
Theorem~\ref{theorem_experimental_controller} derives the commands for decentralized and centralized allocation:
$$
{u}_{c[i]}^a=B_{(3,2)}^{-1}M_{(3,2)} S_{(2,2)[i]} \mathfrak{u}_{c[i]}^a,\qquad {u}_c^a=\sum_{i\in\mathcal{V}_l} {u}_{c[i]}^a
$$
where $S_{(2,2)[i]}$ is the associated tangent space of angular momentum conservation for each pair in subsection~\ref{Kinematics_Control}.% introduced in Theorem~\ref{theorem_experimental_controller}. 

We experimentally verified this decentralized allocation for three-coil control. Each pair converts $\mathfrak{u}_{c[i]}^a$ to the dipoles based on (\ref{decentralized_allocation}) and drive the currents at $\omega_{j}=8\pi\times$\{1, 2, 3\} rad/s. The experimental result in Fig.~\ref{Inverse_experiment_results} shows that their states stabilize around the desired ones under the microgravity of the linear air track. This partially highlights the scalability extension of the multileader-based control introduced in Section~\ref{NODA_subsection}.
\subsection{Numerical Optimality Comparison of ODA and Baseline}
%\subsection{Controller with Centralized / Decentralized Allocation}
\label{Power_Optimal_ODA_Baseline_Comparison}
\begin{figure}[!tb]
\centering
% 右半分:(b) 上, (c) 下
 %\hspace*{-0.6cm} % ← 左にシフト(必要に応じて調整)
%\begin{minipage}[t]{1\linewidth}
%  \centering
\includegraphics[width=0.9\linewidth]{Figure/Allocation_results.pdf}
\caption{Numerical simulation results of energy consumption of the decentralized and centralized allocations with the lower bound 
%was derived from the dual formulation of ${\mathrm{ODA}_3}$ given 
given in (\ref{opt2_N}).\label{average_power_consumption}}
  % 図には共通キャプションだけで良ければここはキャプション不要
\end{figure}
%\end{minipage}\\
%\begin{minipage}[t]{\linewidth}
%  \centering\footnotesize
%\begin{table}[tb]
%\vspace{0.5cm}
\begin{table}[!tb]
\centering
\caption{Numerical performance comparisons: power consumption and control-induced disturbance reduction.}%, and calculation time.}
\label{fig:with_table_three_MTQ}
\renewcommand{\arraystretch}{1.3}
\label{numerical_performance_comparisons}
\centering
\begin{tabular}{c|c|c|c}
\hline 
Allocations&Inverse $\times 1e^4$ trials %&IR$_2$
&ODA$_2$&ODA$_3$\\%&NODA$_3$\\
%&$\times 1e^4$ trials&$\times 1e^4$&&\\\hline
\hline
Command&\multicolumn{3}{c}{$u_{c}=B_{(3,2)}^{-1}M_{(3,2)} S_{(3,2)} u_a$}\\\hline% on Theorem \ref{theorem_experimental_controller}}\\\hline
Grouping&\multicolumn{2}{c|}{Decentralized}&\multicolumn{1}{c}{Centralized}\\\hline
%Ave. [\%]&${0.092}$&-&0.095&0.69\\\hline
%Intef [\%]&${2.4}$&6.9&${2.6}$&${0.46}$\\%&${0.22}$\\\hline
Power (J)&790-1250
%&-
& 760&${250}$\\%&${220}$\\
\hline
$\Delta f$ (\%)&5.4
%&-
&5.15
&${0.40}$\\\hline%&${0.36}$\\\hline\hline
%Ave. [\%]&${0.092}$&-&0.095&0.69\\\hline%&0.6\\\hline
%Inte$\tau$ [\%]&0.9&$5.6e^3$&0.6&${1.9}$\\%&${0.70}$\\\hline
$\Delta \tau$ (\%)&5.1
%&-
&5.0&${0.53}$\\\hline%&${0.5}$\\\hline\hline
%Time [s] &${{4.2e^{-1}}/{1e^4}}$&2.5&1.0\\\hline%&${1e^{-2}}$\\\hline
%\bfseries Parameters & \bfseries Values&\bfseries Parameters & \bfseries Values\\
\end{tabular}
%\label{sec4:table1}
\end{table}
%\end{minipage}
%\caption{グラフ+表をひとつのフロートにまとめる例}
%\end{figure}
We numerically compare the power optimality of ODA$_n$ in (\ref{opt_final_0}) and~(\ref{opt_final}) with the baseline in (\ref{decentralized_allocation}) for the three coils formation and attitude control. We conducted 500 numerical simulations using the command inputs $u_{c}$ in subsection~\ref{controller_dipole_allocation_design}. %, and magnetic field interactions were not taken into account at this stage.
Figures~\ref{Numerical_PDcontrol_results_attitudes_picture} and \ref{Numerical_PDcontrol_results_positions_picture} show the obtained time-series data, and we convert them to associated dipole via three dipole allocations: the centralized allocation ${\mathrm{ODA}_3}$ and a two decentralized allocations (inverse method in Algorithm~\ref{alg:Inverse_based_dipole_allocation} with 1e$^4$ trials and ${\mathrm{ODA}_2}$). We define a power index as 
$W_{\mathrm{power}}$:
\begin{equation*}
%\label{lower_power_index}
W_{\mathrm{power}}\triangleq R\int_T c_i^2(\tau)\frac{\mathrm{d}\tau}{T}=R\frac{\|\mu^{\mathrm{amp}}_{N}\|^2}{2 \gamma_{\mu/c}^2}
\geq R\frac{J_{\mathrm{dual}}}{\gamma_{\mu/c}^2}
\end{equation*}
where $R\approx 2.1\Omega$ is the coil resistor \cite{takahashi2025noda_mmh}, $\gamma_{\mu/c}\approx 2.1$m$^2$ is the coil design ratio \cite{takahashi2025noda_mmh}, and the lower bound $J_{\mathrm{dual}}$ in (\ref{opt2_N}). Note that we memorize minimum values of the inverse method for each trials. We also derived the indexes $\Delta \{f,\tau\}\in\mathbb{R}$ to investigate time-integrated control-induced disturbances:% in Section~\ref{control_induced_disturbance_reduction}:
$$
\Delta \{f,\tau\}\triangleq \sup_k\sup_{j\in\mathcal{V}}\frac{\|\{f,\tau\}_{jc}-\sup_{t\in[t_{[k]},t_{[k]}+T]} \{f,\tau\}_{j(t)}\|_2}{\|\{f,\tau\}_{jc}\|_2}.
$$ 

The calculation results in Fig.~\ref{average_power_consumption} and Table~\ref{numerical_performance_comparisons} show the integration of $W_{\mathrm{power}}$ and $\Delta \{f,\tau\}$. The inverse approach showed large variations in energy consumption across $1e^4$ trials and converged to a performance close to that of the ${\mathrm{ODA}_2}$ result. Notably, the centralized ${\mathrm{ODA}_3}$ achieved significantly better power efficiency, requiring only approximately one-third to one-fifth of the power consumption compared to ODA$_2$ and inverse allocation with 10 trials. 
%Notably, the centralized ${\mathrm{ODA}_3}$ achieved significantly better power efficiency than the decentralized ${\mathrm{ODA}_2}$. 
This arises from magnetic field anisotropy, implying that certain directions are more effective for generating control inputs within a given power limit. In the two-dimensional setup, the three satellites cooperatively compensate for this anisotropy, thereby enabling the derivation of a more power-optimal current solution. We also confirm the reduction of $\Delta \{f,\tau\}$ by the power optimization in Table~\ref{numerical_performance_comparisons} as proved in Section~\ref{control_induced_disturbance_reduction}, which reduces steady-state errors. % in both position and attitude.
\subsection{Offline Supervised Learning for NODA Construction}
\label{Learning_based_NODA}
\begin{figure}[!t]
\centering
\begin{minipage}[b]{0.493\columnwidth}
  \centering
    \subfloat[Test loss for network of $N$ = 256.]{\includegraphics[width=1\linewidth]{Figure/depth_vs_loss_N256.pdf}
    %\label{experiment_picture}
    }
\end{minipage}
\begin{minipage}[b]{0.493\columnwidth}
  \centering
  \subfloat[Test loss for network of $N$ = 512.]{\includegraphics[width=1\linewidth]{Figure/depth_vs_loss_N512.pdf}
  %\label{fig:position}
  }
\end{minipage}
\caption{The trade-off between model size and performance of learning.\label{fig:tradeoff_study}}%
\end{figure}
% using the ground-based testbed introduced in Section~\ref{experimental_setup_introduction}. 
This subsection presents the numerical validation of the trained NODA model for Ten MTQs and the scalability extension by time-integrated control. We first train NODA $\mathfrak{M}$ for a three-coil formation and attitude control based on Section \ref{convex_allocation}. Assumption~\ref{6DoF_control_assumption} indicates ${f}^{a}$ has only $y$ components and ${\tau}^{b}$ has only $z$ components. Subsequently, along with linear and angular momentum conservation, we reduced the input data for our experiment to the following six variables:
$\chi=[y_1;y_2;{f}_{1y}^{a};{f}_{2y}^{a};{\tau}_{1z}^{b_1};{\tau}_{3z}^{b_3}]\in\mathbb{R}^{6}$ and the 15 dimentional label data $\mathtt{x}$, after removing redundant and zero components. This study collected 3.5 million training samples using Algorithm \ref{alg:Convex_Optimization_based_Power_Optimal_Dipole_Allocation} for the regions shown in Figs.~\ref{Numerical_PDcontrol_results_attitudes_picture} and \ref{Numerical_PDcontrol_results_positions_picture}. Our model $\mathfrak{M}(\chi,{\bm\theta})$ is composed of six ($\triangleq D$-1) hidden layers with 256 ($\triangleq N$) neurons and one hidden layer with 128 neurons, based on the trade-off between model size and test loss in Fig.~\ref{fig:tradeoff_study}. The activation function is an element-wise Leaky Rectified Linear Unit function.
%The parameters are optimized with the Adam algorithm~\cite{kingma2014adam} and The loss function is a smooth L1 objective $\mathcal{L}(\theta_w,\theta_b)=\frac{1}{6N_s}\sum_{j=1}^{N_s}\sum_{i=1}^{6}\rho_\delta\!\left(Y_i^{(j)}-\hat{Y}_i^{(j)}\right)$ where $\rho_\delta(\cdot)$ denotes the Huber loss. 
The batch sizes were $131{,}072$ 
%in the pre–training stage and $1{,}024$ during fine–tuning. 
and the learning rate followed a cosine–annealing schedule, decreasing from $10^{-3}$ to $10^{-6}$ during 5e$^3$ epoch.  %during pretraining and from $10^{-6}$ to $10^{-16}$ during fine–tuning. 
The training used a smooth L1-loss function that converged to a training loss of $2.66e^{-2}$ and a test loss of $3.59e^{-2}$. This result shows that the solutions derived in Algorithm \ref{alg:Convex_Optimization_based_Power_Optimal_Dipole_Allocation} successfully capture the features of an infinite number of optimal solutions to the non-convex problem in (\ref{basic_formulation}). 
%\subsection{Multileader Decentralization}
\subsection{Learning-based Power-Optimal 10 MTQs Control}
\label{Learning_based_NODA}
We apply the trained NODA model to Ten MTQs control and numerically demonstrate scalability via time-integrated control. We assume Ten MTQs are divided into five groups managed by leaders $[1,3,4,6,7]\in\mathcal{V}_l$ using AC frequencies $\omega_{1,3,4,6,7}=8\pi\times$\{1:5\} rad/s as shown in Fig.~\ref{10MTQ_experiment_picture}. Each leader decentrally converts local state and control vectors to the power-optimal dipoles by NODA model $\mathfrak{M}(\chi,{\bm\theta})$ and Lemma~\ref{lemma_reconstruction_phase_vector_fram_X}. For five cases using different initial conditions, their formation and attitude converge the desired states defined in subsection~\ref{controller_dipole_allocation_design}. We plot the control results of a total of 15 edges, the norm of the relative position, with the same case indicated by the same color in Fig.~\ref{15_edges_results}.
The power consumption was dramatically reduced one-third to one-ten for both the initial damping phase and steady-state 
%gravity compensation 
control, as reported in Fig.~\ref{power_consumption_NODA}. These results are consistent with those reported in subsection~\ref{Power_Optimal_ODA_Baseline_Comparison}.

The figures~\ref{NODA_experiment_results_duality_caluculation} present the duality gap \cite{boyd2004convex} of ODA$_3$ and the allocation time comparison obtained from time-series data in the 10-agent control.  As shown in Fig.~\ref{duality_gap_picture_ODA3}, the duality gap between ODA$_3$ and its dual problem in (\ref{opt2_N}) is nearly zero, and ODA$_3$ does not lose global optimality for our planar swarm control scenario. General verification is left for future work. As illustrated in Fig.~\ref{allocation_time_picture_NODA3}, the proposed NODA model reduces the computation time by about 1/50,000, comparable to the inverse approach. These results demonstrate that the proposed learning-based framework successfully achieves both optimality and computational efficiency. Combining our multileader-based control with NODA$_{n}$ achieves power-optimal current allocation for an arbitrary number of agents.
\begin{figure}[!tb]
\centering
% 右半分:(b) 上, (c) 下%
%\hspace*{-0.6cm}\begin{minipage}[b]{0.2175\columnwidth}\centering\subfloat[Testbed.]{\includegraphics[width=1\linewidth]{Figure/test_rotation_position.jpg}\label{fig:testbed_setup}}\end{minipage}
%\hspace*{0.00125\columnwidth}\begin{minipage}[b]{0.325\columnwidth}\centering\subfloat[Initial condition.]{\includegraphics[width=1\linewidth]{Figure/experimental_picture_initial_3MTQ.pdf}\label{fig:initial}}\end{minipage}
%\hspace*{0.00125\columnwidth}\begin{minipage}[b]{0.325\columnwidth}\centering\subfloat[Target condition.]{\includegraphics[width=1\linewidth]{Figure/experimental_picture_final_3MTQ.pdf}\label{fig:final}}\end{minipage}
\begin{minipage}[b]{0.8\columnwidth}
  \centering
\subfloat[Ten MTQs belong to five groups with five AC frequencies.]% $\omega_{1,3,4,6,7}$.]
{\includegraphics[width=1\linewidth]{Figure/10MTQ_experiment_concept.pdf}
\label{10MTQ_experiment_picture}
    %\subcaption{(a) attitude}
    %\label{fig:attitude}
    }
  \end{minipage}\\
  \begin{minipage}[b]{0.493\columnwidth}
    \centering
    \subfloat[Formation control command results.]{\includegraphics[width=1\linewidth]{Figure/edges_NODA3.pdf}
    \label{15_edges_results}
    }
  \end{minipage}
  \begin{minipage}[b]{0.493\columnwidth}
    \centering
    \subfloat[Energy
consumption results.]{\includegraphics[width=1\linewidth]{Figure/Power_Consumption.pdf}
    \label{power_consumption_NODA}
    }
  \end{minipage}
\caption{Multilearder-based decentralized control overview and numerical simulation results for 10 MTQs control using trained NODA.\label{NODA_experiment_results}}
%\begin{minipage}[t]{\linewidth}\vspace{0.3cm}\captionof{table}{{Experimental performance comparison.}}\renewcommand{\arraystretch}{1.3}\label{comparison_Inverse_NODA3}\centering\begin{tabular}{c|c|c}\hline Allocations&Inverse $\times$ 1 trial &NODA$_3$\\\hline Power [J]&800-1250&${250}$\\\hline$\lim_{t\rightarrow \infty}\|p-p_d\|$ [mm]&5.4&${0.40}$\\\hline$\lim_{t\rightarrow \infty}|\theta-\theta_d|$ [deg]&5.1&${0.53}$\\\hline\end{tabular}\end{minipage}
\end{figure}
\begin{figure}[t]
\centering
%\hspace{-0.05cm}
\begin{minipage}[b]{0.493\linewidth}
    \centering
    \subfloat[Duality gap comparison of ODA$_3$.]{\includegraphics[width=1\linewidth]{Figure/output_time_comparison_ODA_optimization.pdf}
    \label{duality_gap_picture_ODA3}
    }
  \end{minipage}
 % \hspace{0.05cm}
  \begin{minipage}[b]{0.493\linewidth}
    \centering
    \subfloat[Allocation time comparison.]{\includegraphics[width=1\linewidth]{Figure/output_time_comparison_NODA.pdf}
    \label{allocation_time_picture_NODA3}
    }
  \end{minipage}
%\end{minipage}
\caption{Duality gap of ODA$_3$ problem and the allocation time comparison for 10 MTQs control in subsection~\ref{Learning_based_NODA}. All computations were performed in MATLAB R2025a on a workstation with an AMD Ryzen Threadripper 7980X and 64 GB RAM (Windows 11 Pro, 64-bit).\label{NODA_experiment_results_duality_caluculation}}
\end{figure}
%These power-optimal dipole allocations reduce control-induced disturbances due to time-integrated control, as proved in Section~\ref{control_induced_disturbance_reduction}. This leads to an improvement in the steady-state error for both position and attitude, as summarized in Table~\ref{comparison_Inverse_NODA3}. 
%The appropriate number $n$ in NODA$_{n}$ is determined by the trade-off study between the curse-of-dimensionality issue during training and the power optimality.
%We collect approximately 10 million samples and train our MLP using supervised learning. $R^{[2]}=0.15. r = 0.74$, $R^{[3]}=0.30. r = 2.1$. To explore the impact of network depth, we test models with 0 to 5 hidden layers, each consisting of 1024 neurons and a fixed two-layer architecture with 256 and 128 neurons, respectively. Each layer follows Layer Normalization and GELU activation, allowing us to evaluate the trade-off between training stability and computational efficiency. The output layer maps directly to the target dimension. The loss function is the Smooth L1 loss. We used the Adam optimizer with an initial learning rate of $ 3e^{-3}$ and applied cosine annealing learning rate scheduling to improve convergence. The parameters were set as follows: $T_{\mathrm{max}} = 3000$ and a minimum learning rate of $ 1 \ times 10^ {-18}$. The model was trained for 3000 epochs with a batch size of 1024. The training process resulted in convergence with a test loss of $ 4.11 e^ {-3}$ and a training loss of $ 3.01 e^ {-3}$. This close agreement implies acceptable generalization performance. On the validation PC, the average value of the cyclic integral with $\delta\theta=1e^{-7}$ in (\ref{circulant_integration_term}) decreased from 121.2ms to 29.5ms over one million iterations. Furthermore, the standard deviation of the computation time was reduced from 176ms to 11.3ms, indicating more stable and consistent computation. To quantitatively evaluate the prediction accuracy of the proposed model, we employ four standard regression metrics: the coefficient of determination ($R^2$), mean absolute error (MAE), root mean squared error (RMSE), and maximum absolute error (MaxE). These metrics are defined as follows: $R^2 = 1 - \frac{\sum_{i=1}^{N} (y_i - \hat{y}_i)^2}{\sum_{i=1}^{N} (y_i - \bar{y})^2}$,\ $\mathrm{MAE}= \frac{1}{N} \sum_{i=1}^{N} |y_i - \hat{y}_i|$,\ $\mathrm{RMSE} = \sqrt{ \frac{1}{N} \sum_{i=1}^{N} (y_i - \hat{y}_i)^2 }$, $\mathrm{MaxE} = \max_{i=1,\dots,N} |y_i - \hat{y}_i|$ where $y^{(i)}$ and $\hat{y}^{(i)}$ denote the ground truth and predicted values, respectively, for the $i$th sample, $\bar{y}$ is the mean of the ground truth values, and $N$ is the total number of samples. The $R^2$ score represents how well the model explains the variance in the data, with $R^2 = 1$ indicating perfect prediction. MAE and RMSE evaluate the average prediction error, where RMSE penalizes larger errors more strongly. MaxE captures the most significant single deviation between prediction and ground truth. In our experiments, all models achieved $R^2$ scores greater than $0.97$, with MAE and RMSE values typically below $0.16$. MaxE remained within acceptable bounds in most cases, highlighting the model's robustness even in the presence of occasional outliers.
%Learning Rate is Batch Size&Epochs&Neurons&Layers
%Hyperparameters for Neural Optimal-power Dipole Allocation'' (NODA) and "Neural Kinematics Controller Solutions" models.
%r, I/O Dimensions, Learning Rate, Batch Size, Epochs, Neurons, Layers
%Note that $\mu_i(\tau)=\mu_{\mathrm{amp}}\sin(\omega t+\theta)$
%\subsection{Simulation Results of NODA$_3$ for 10 Magnetorquers}この章ではkzetaは0ではないが,その前の比較では全体の制御則を同じにするためにkzetaは0にした We finally show the experimental results using centralized allocation enabled by our NODA. NODA is implemented into 3, and it derives power-optimal current allocations in real-time to achieve the desired electromagnetic force and torque.
\section{Conclusion}
\label{Conclusion}
This paper introduces a learning-based current calculation framework for power-optimal control of magnetically actuated systems. Because power-efficient dipole computation incurs high computational costs, we present an optimal solution calculation using sequential convex optimization and approximate it with a data-driven model, NODA. To accommodate large robot groups, a systematic decoupling architecture is employed to partition the agents into smaller units. Experimental and Numerical results show the effectiveness of our decentralization strategy, and the trained model accurately reproduces the targeted magnetic interactions while reducing power consumption, control-induced disturbances, and computation time. The proposed framework offers a practical solution for generating current commands in resource-constrained robotic systems.
%\begin{comment}
%名前と所属は、原稿、付録、補足資料のどこにも含めないでください。
%人々や資金提供機関への謝辞は、原稿が受理された後にのみ記載してください。
\section*{Acknowledgments}
This work was supported by the JST's Next-Generation Challenging Research Program JPMJSP2106 and the JAXA Space Strategy Fund GRANT Number (JPJXSSF24MS09003), Japan. The authors would like to thank Associate Prof. Yoichi Tomioka at the University of Aizu and Hayate Tajima at The University of Tokyo for technical discussions, and Atsuki Ochi at Interstellar Technologies Inc. for assistance with controller gain tuning.
%\end{comment}
%\appendix
%\subsection{Proof of Lemma~\ref{lemma_reconstruction_phase_vector_fram_X}: Reconstructing Phase $\bm\psi$ from $\mathfrak{X}$}\label{proof_reconstruction_phase_vector_fram_X}