% systemoverview

%%\subsection{System Overview}
%%\label{sec:sysoverview}
%Our multi-MAV system does not consist of a central computational unit. Each of our MAVs is equipped with an on-board CPU and GPU to perform all computations. Although our architecture also does not depend on a centralized communication network, the field implementation is done through a central wifi access point. Each MAV runs its own instance of the following software modules (blue blocks in Fig.~\ref{fig:system}).
%\begin{itemize}
%	\item A low-level position and yaw controller (flight controller) moduSystem Overview and Problem Formulationle,
%	\item a self-localization module using on-board GPS, IMU and barometer,
%	\item a cooperative detection and tracking (CDT) module, which is the core contribution of this article and described in subsection~\ref{sec:cdt}, and
%	\item an MPC-based formation controller and obstacle avoidance module that allows us to maintain a perception-driven formation, described in subsection~\ref{sec:mpc}.
%\end{itemize}Figure~\ref{fig:system} details the flow of data among these modules. The data shared between any two MAVs consists of their self-pose estimates and the detection measurements of the tracked person. All the aforementioned software modules run on board in realtime. In the following sections, after introducing notations, we focus on the detailed description of our CDT module. Therein we describe how our proposed MCDT approach enables the MAV formation to track seemingly-small and far-away persons with high accuracy and without losing them from any MAV's FOV during tracking.
\subsection{Preliminaries}
We describe the proposed framework for a multi-robot system tracking a desired target. 
%However, it is important to note that the MPC based motion planner is agnostic to the type of robot (aerial or ground). 
For the concepts presented, we consider Micro Aerial Vehicles (MAVs) that hover at a pre-specified height $h_{gnd}$. Furthermore, we consider 2D target destination surface. However, the proposed approaches can be extended to any 3D surface. Let there be $K$ MAVs $R_1,..., R_K$ tracking a target $x_t^P$, typically a person $P$. Each MAV computes a desired destination position $\xproj_t^{R_k}$ in the vicinity of the target position. The pose of $k^{\text{th}}$ MAV in the world frame at time $t$ is given by $\xi_t^{R_k} = [ (\x_t^{R_k})^\top ~ (\Theta_t^{R_k})^\top] \in \mathbb{R}^6$. 
%$\x_t^{R_k}$ denotes the 3D position and $\Theta_t^{R_k}$ represents the 3 orientation angles. 
Let there be $M$ obstacles in the environment $O_1,...,O_M$. The $M$ obstacles include $R_k$'s neighboring MAVs and other obstacles in the environment.

%Our work is motivated by the broader application of motion capture of a target in an outdoor environment by multiple robots based on vision. The key requirements for an outdoor motion capture system are (i) to not lose track of the moving target, and (ii) to ensure that the robots avoid other robot agents and all obstacles (static and dynamic) in their vicinity. In order to tackle both these objectives, we formulate a formation control (FC) algorithm, as detailed in Algorithm \ref{Alg:fc}. The main steps have the following functionality, (i)  destination point computation depending on target movement, (ii) obstacle avoidance force generation, (iii)  decentralized quadratic model predictive control (DQMPC) based planner for way point generation, and (iv) a low-level position controller to track generated way-points.
The key requirements in a multi-robot target tracking scenario are, (i) to not lose track of the moving target, and (ii) to ensure that the robots avoid other robot agents and all obstacles (static and dynamic) in their vicinity. In order to address both these objectives in an integrated approach, we formulate a formation control (FC) algorithm, as detailed in Algorithm \ref{Alg:fc}. The main steps have the following functionality, (i)  destination point computation depending on target movement, (ii) obstacle avoidance force generation, (iii)  decentralized quadratic model predictive control (DQMPC) based planner for way point generation, and (iv) a low-level position controller.

To track the waypoints generated by the MPC based planner we use a geometric tracking controller.  The controller is based on the control law proposed in~\cite{lee2010geometric},  which has a proven global convergence, aggressive maneuvering controllability and excellent position tracking performance. Here, the rotational dynamics controller is developed directly on $SO(3)$ and thereby avoids any singularities that arise in local coordinates. Since the MAVs used in this work are under-actuated systems, the desired attitude generated by the outer-loop translational dynamics is controlled by means of the inner-loop torques.

%The following subsection describes DQMPC planner and local minima avoidance methodologies.

%\subsection{Low level controller: Waypoint tracking}\label{sec:geometrictracking}
%
%We use the control law proposed in~\cite{lee2010geometric},  which has a proven global convergence, aggressive maneuvering capability and excellent trajectory tracking performance.
%Here, the rotational dynamics controller is developed directly on $SO(3)$ and thereby avoids any singularities that arise in local coordinates.
%
%Considering the trajectory tracking task, at a given time step the tracking error in position and velocity are defined as  $\boldsymbol e_{ p} = \boldsymbol {p}_W - \boldsymbol p_a$ and $ \boldsymbol e_{ v} = \boldsymbol {\dot p}_W - \boldsymbol {\dot p}_a$ respectively.
%%
%The desired force for the translational dynamics is given as,
%\begin{align}\label{eq:poscontrol}
%\rho	=	&(m \ddot{\boldsymbol p}_a-\boldsymbol K_{d}\boldsymbol e_{ v} -\boldsymbol K_{p}\boldsymbol e_{p} -& \nonumber\\
%& \quad-\boldsymbol K_{i}\int_{t_0}^t \boldsymbol e_{\boldsymbol p}dt - mge_3) \cdot \boldsymbol{R}_B^We_3,&
%\end{align}
%where the diagonal positive definite gain matrices $\boldsymbol K_{d}$, $\boldsymbol K_{p}$, $\boldsymbol K_{i}$ define Hurwitz polynomials.
%The desired hovering thrust is realized by $f_z =  \rho\, e_3 $   and by aligning the body vertical axis along the direction of the $\rho$ defined as,
%
%{
%	\small{
%		\begin{align}
%		\vec{z}_{R_d} = \frac{
%			m \ddot{\boldsymbol p}_a - \boldsymbol K_{d}\boldsymbol e_{v} -\boldsymbol K_{p}\boldsymbol e_{p}  -\boldsymbol K_{i}\int_{t_0}^t \boldsymbol e_{p}dt - mge_3 }{\Vert m \ddot{\boldsymbol p}_a-\boldsymbol K_{d}\boldsymbol e_{v} -\boldsymbol K_{p}\boldsymbol e_{p}  -\boldsymbol K_{i}\int_{t_0}^t \boldsymbol e_{p}dt - mge_3 \Vert},
%		\end{align}
%	}\normalsize
%	
%	\noindent where $\vec{z}_{R_d}$ is the third column of the desired attitude rotation matrix $\boldsymbol{R}_{B_d}^W$ defined as $ \boldsymbol{R}_{B_d}^W= \begin{bmatrix}\vec{x}_{R_d}, \vec{y}_{R_d}, \vec{z}_{R_d} \end{bmatrix}\in SO(3)$. Since the quadrotor UAV is an underactuated system, the desired attitude generated by the outer-loop translational dynamics is controlled by means of the inner-loop torques, that are generated for controlling the rotational dynamics, to track a desired attitude rotation $\boldsymbol{R}_{B_d}^W$. 
%	The other two columns $\vec{x}_{R_d}$ and $\vec{y}_{R_d}$ of $\boldsymbol{R}_{B_d}^W$, which account for the remaining degrees of freedom, should be chosen such that their direction is orthogonal to $\vec{z}_{R_d}$ and minimize the yaw error.
%	Therefore%the heading direction is decided by them is defined as,
%	\begin{align}
%	\vec{x}_{R_d} = \vec{y}_{R_d} \times \vec{z}_{R_d},\qquad
%	\vec{y}_{R_d} = \frac{\vec{z}_{R_d} \times \vec{x}_{R_d}}{\Vert \vec{z}_{R_d} \times \vec{x}_{R_d} \Vert}.
%	\end{align}
%	
%	For the rotational dynamics, assuming that $\boldsymbol \omega_{B_d}=[ {\boldsymbol{R}^W_{B_d}}{}^T  {\dot{\boldsymbol{R}}^W_{B_d}}{}^T]_{\vee}$, where $[\cdot]_\vee$ represents the inverse (vee) operator from $so(3)$ $\to$ $\mathbb{R}^3$, the attitude tracking error $\boldsymbol e_R \in \mathbb{R}^3 $ is defined similarly to~\cite{lee2010geometric} as
%	\begin{equation}\label{eq:rot_error}
%	\boldsymbol e_{R}=\dfrac{1}{2}[{\boldsymbol{R}^W_{B_d}}^T  {\boldsymbol{R}^W_{B}}  - {\boldsymbol{R}^W_{B}}^T {\boldsymbol{R}^W_{B_d}}]_\vee,
%	\end{equation}
%	and the tracking error of the angular velocity $\boldsymbol e_\omega \in \mathbb{R}^3 $ is given by
%	%
%	\begin{equation}\label{eq:omega_error}
%	\boldsymbol e_{\omega}
%	=
%	{\boldsymbol \omega}_B -
%	{ {\boldsymbol{R}^W_{B}}^T {\boldsymbol{R}^W_{B_d}} {\boldsymbol \omega}_{B_d}}.
%	\end{equation}
%	%
%	In order to obtain an asymptotic convergence  to $\boldsymbol 0$ of the rotational error $\boldsymbol {e}_R$ one can choose the following controller
%	%
%	\begin{align}\label{eq:rotational_error}
%	{\boldsymbol \tau} &= 
%	- \boldsymbol K_{\omega} \boldsymbol {e}_\omega - \boldsymbol K_{r} \boldsymbol {e}_R
%	- \boldsymbol K_{ir}
%	\int_{t_0}^t \boldsymbol {e}_R
%	+\boldsymbol \omega_{B}\times \boldsymbol I_{B}\boldsymbol\omega_{B}
%	- & \nonumber\\
%	%
%	&  - \boldsymbol I_{B}
%	( \left[ {\boldsymbol \omega}_B \right]_\wedge
%	{ {\boldsymbol{R}^W_{B}}^T {\boldsymbol{R}^W_{B_d}} {\boldsymbol \omega}_{B_d}} -
%	{\boldsymbol{R}^W_{B}}^T {\boldsymbol{R}^W_{B_d}} 
%	\dot{\boldsymbol  \omega}_{B_d}),&
%	\end{align}%\red{CHECK SIGN OF $e_R$}
%	where the diagonal positive-definite gain matrices $\boldsymbol K_{\omega}$, $\boldsymbol K_{r}$, $\boldsymbol K_{ir}$ define Hurwitz polynomials and $\left[ {\boldsymbol \omega}_B \right]_\wedge$ is the skew symmetric matrix of ${\boldsymbol \omega}_B$.
%}
