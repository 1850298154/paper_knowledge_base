% introduction

The topic of multi-robot cooperative target tracking has been researched extensively in recent years \cite{aamir_pcmmc_ras_journal,dias2013cooperative,ahmad2017online,hausman2016cooperative,zhang2016cooperative}.  Target, here referes to a movable subject of interest in the environment for e.g., human, animal or other robot. Cooperative target tracking methods focus on improving the estimated pose of a tracked target while simultaneously enhancing the localization estimates of poorly localized robots, e.g., \cite{aamir_pcmmc_ras_journal}, by fusing the state estimate information acquired from team mate robots. The general modules involved in decentralized multi-robot target tracking are summarized in Fig. \ref{system}. Our work focuses on the modules related to obstacle avoidance (blue in Fig. \ref{system}). The other related modules (green in Fig. \ref{system}), such as target pose estimation, are assumed to be available (see \cite{price2018deep}). The developed obstacle avoidance module fits into any general cooperative target tracking framework as seen in Fig. \ref{system}. 

Robots involved in tracking a desired target must not collide with each other, and also with other entities (human, robot or environment). While addressing this problem, the state-of-art methodologies for obstacle avoidance in the context of cooperative target tracking have drawbacks. In \cite{nascimento2015nonlinear,nascimento2016multi} obstacle avoidance is imposed as part of the weighted MPC based optimization objective, thereby providing no guaranteed avoidance. In \cite{aamir_pcmmc_ras_journal} obstacle avoidance is a separate planning module, which modifies the generated optimization trajectory using potential fields. This leads to a sub-optimal trajectory and field local minima. 
%Providing gaurantees on obstacle avoidance is important, so as to ascertain safety of multi-robot target tracking systems. 

The goal of this work is to provide a holistic solution to the problem of obstacle avoidance, in the context of multi-robot target tracking in an environment with static and dynamic obstacles. Our solution is an asynchronous and decentralized, model-predictive control based convex optimization framework. Instead of directly using repulsive potential field functions to avoid obstacle, we convexify the potential field forces by replacing them as pre-computed external input forces in robot dynamics. As long as a feasible solution exists for the optimization program, obstacle avoidance is guaranteed. In our proposed solution we present three methods to resolve the field local minima issue. This facilitates convergence to a desired surface around the target. %Furthermore, the tangential band approach assures convergence to target surface because of its ability to overcome field local minima.
%In addition, the robots maintain a predefined distance with respect to the target while tracking it by means of forming a virtual surface around target. 
%The robots should always lie on any obstacle free point of this predefined 2D or 3D surface. 


%The main contributions of this work is the formulation of local motion planning problem as a fully convex optimization program in the context of multi-robot target tracking. The proposed algorithm is asynchronous, decentralized and scalable. We further propose three methods for field local minima avoidance. \todoGeneral{improve}
The main contributions of this work are,
\begin{itemize}
\item Fully convex optimization for local motion planning in the context of multi-robot target tracking
\item Handling non-convex constraints as pre-computed input forces in robot dynamics, to enforce convexity
\item Guaranteed static and dynamic obstacle avoidance
\item Asynchronous, decentralized and scalable algorithm
\item Methodologies for potential field local minima avoidance
%\item Multi-Robot target tracking without enforcing predefined trajectories or formation geometries.
\end{itemize}

Sec.~\ref{sec:sota} details the state-of-art methods related to obstacle avoidance, Sec.~\ref{sec:proposedappraoch} discusses the Decentralized Quadratic Model Predictive Controller along with the proposed methodologies to solve the field local minima and control deadlock problem,  Sec.~\ref{sec:results} elaborates on simulation results of different scenarios, and finally we discuss future work directions. 

\begin{figure}
\centering
\includegraphics[scale=0.3]{system_ssrr.pdf}
\caption{General modules involved in multi-robot target tracking. Our work focuses on the modules highlighted in blue}
\label{system}
\vspace{-1.5em}
\end{figure}


