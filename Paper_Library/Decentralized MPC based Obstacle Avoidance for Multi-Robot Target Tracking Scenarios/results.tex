In this section, we detail the experimental setup and the results of our DQMPC based approach along with  the field local minima resolving methods proposed in Sec.~\ref{sec:proposedappraoch} for obstacle avoidance and reaching the target surface.
\begin{figure}
	\centering
	\begin{subfigure}[t]{0.24\textwidth}        
		\includegraphics[scale=0.27]{Swivel_5.pdf}   
		\caption{}     
		\label{Swivel_5}
	\end{subfigure}
	\begin{subfigure}[t]{0.24\textwidth}        
		\includegraphics[scale=0.22]{Swivel_5_grad.pdf}
		\caption{}
		\vspace{0.1cm}
		\label{Swivel_5_grad}
	\end{subfigure} 
	\begin{subfigure}[t]{0.24\textwidth}
		\centering
		\includegraphics[scale=0.27]{Swivel_32.pdf}  
		\caption{}    
		\label{Swivel_32}
	\end{subfigure}
	\begin{subfigure}[t]{0.24\textwidth}
		\centering
		\includegraphics[scale=0.21]{Swivel_32_grad.pdf}
		\caption{}
		\vspace{0.1cm}
		\label{Swivel_32_grad}
	\end{subfigure}
	\begin{subfigure}[t]{0.24\textwidth}
		\centering
		\includegraphics[scale=0.27]{Swivel_14.pdf} 
		\caption{}   	
		\label{Swivel_14}
	\end{subfigure}
	\begin{subfigure}[t]{0.24\textwidth}
		\centering
		\includegraphics[scale=0.22]{Swivel_14_grad.pdf}
		\caption{}
		\label{Swivel_14_grad}
	\end{subfigure}      
    \caption{MAV trajectories and optimization gradients of swivelling destination method.}
    \label{Swivel_Result} 
    \vspace{-1em}
\end{figure}
%\begin{figure*}[t!]
%    \centering
%    \begin{subfigure}[t]{0.33\textwidth}        
%        \includegraphics[scale=0.38]{Ang_5.pdf}        
%        \label{Ang_5}
%    \end{subfigure}%
%    ~ 
%    \begin{subfigure}[t]{0.33\textwidth}
%        \centering
%        \includegraphics[scale=0.4]{Ang_32.pdf}       
%        \label{Ang_32}
%    \end{subfigure}%
%	\begin{subfigure}[t]{0.33\textwidth}
%    	\centering
%    	\includegraphics[scale=0.4]{Ang_14.pdf}    	
%    	\label{Ang_14}
%	\end{subfigure}  
%	~       
%    \begin{subfigure}[t]{0.33\textwidth}        
%        \includegraphics[scale=0.3]{Ang_5_grad.pdf}
%        \caption{Approach angle force with 5 Dynamic Obstacles}
%        \label{Ang_5_grad}
%    \end{subfigure}%
%    ~ 
%    \begin{subfigure}[t]{0.33\textwidth}
%        \centering
%        \includegraphics[scale=0.3]{Ang_32_grad.pdf}
%         \caption{Approach angle force with 2 Static Obstacles}
%        \label{Ang_32_grad}
%    \end{subfigure}%
%	\begin{subfigure}[t]{0.33\textwidth}
%    	\centering
%    	\includegraphics[scale=0.3]{Ang_14_grad.pdf}
%    	\caption{Approach angle force with U shaped static obstacle}
%    	\label{Ang_14_grad}
%	\end{subfigure}      
%    \caption{Trajectories and Optimization Gradients of the approach angle force using DQMPC}
%    \label{Ang_Result}    	
%\end{figure*}
\vspace{-0.5em}
\subsection{\label{sub:expsetup}Experimental Setup}
The algorithms  were simulated in a Gazebo+ROS integrated environment to emulate the real world physics and enable a decentralized implementation for validation of the proposed methods. The setup runs on a standalone Intel Xeon E5-2650 processor. The simulation environment consists of multiple hexarotor MAVs confined in a 3D world of $20 m\times 20 m\times 20 m$. All the experiments were conducted using multiple MAVs for 3 different task scenarios involving simultaneous target tracking and obstacle avoidance namely,
\begin{itemize}
\item Scenario I: 5 MAVs need to traverse from a starting surface to destination surface without collision. Each agent acts as a dynamic obstacle to every other MAV. 
\item Scenario II: 2 MAVs hover at certain height and act as static obstacles. The remaining 3 MAVs need to reach the desired surface avoiding static and dynamic obstacles.
\item Scenario III: 4 MAVs hover and form a U-shaped static obstacle. The remaining MAV must reach destination while avoiding field local minima and control deadlock.
\end{itemize}
It may be noted that in all the above scenarios, the target's position is drastically changed from  initial to final destination. This is done so as to create a more challenging target tracking scenario than simple target position transitions.
%In Scenario $1$ each MAV is a dynamic obstacle to every other MAV. Scenario $2$ tests the potency of the approach in the presence of both static and dynamic obstacles. Scenario $3$ investigates the control deadlock and local minima avoidance. 
Furthermore, the scalability and effectiveness of the proposed algorithms are verified by antipodal position swap of 8 MAVs within a surface. The MAVs perform 3D obstacle avoidance to reach their respective positions while ensuring that the surface center (target) is always in sight.

The convex optimization \eqref{DQMPC}-\eqref{last_DQMPC} is solved as a quadratic program using CVXGEN \cite{mattingley2012cvxgen}. The DQMPC operates at a rate of $100$ Hz. The state and velocity limits of each MAV $R_k$ are $[-20,-20,3]^{\top} \leq ~ \x_t^{R_k}(n) \leq ~ [20,20,10] ^{\top} $ in $m$ and $[-5,-5,-5]^{\top} \leq ~ \dot{\x}_t^{R_k}(n) \leq ~ [5,5,5]^{\top} $ in $m/s$ respectively, while the control limits are  $[-2,-2,-2]^{\top} \leq \ao_t^{R_k}(n) \leq [2,2,2]$ in $m/s^2$. 
The desired hovering height of each MAV is $h_{gnd} = 5 m$  and the yaw $\psi^{R_k}$ of each MAV is oriented towards the target. The horizon $N$ for the DQMPC and potential force computation is $15$ time steps each. It is important to mention that if no trajectory information is available for an obstacle or adversary $O_j$, the $n=0$ magnitude of potential field ($F_{rep}^{R_k,O_j}(0)$) is used for the entire horizon. $d_{min}= 0.5m$ and $d_{safe}= 3m$ for the potential field around obstacles. The destination surface is circular, with radius $d^{R_k} = 4 m ~ \forall k$,  around the target for all experiments. However, as stated earlier our approach can attain any desired 3D destination surface.

\subsection{DQMPC: Baseline Method}
Figure \ref{DQMPC_Result} showcases the multi-robot target tracking results for the three different scenarios (see Sec. \ref{sub:expsetup}) while applying the baseline DQMPC method. As clearly seen in Fig. \ref{DQMPC_Result}(a,c), the agents find obstacle free trajectories from starting to destination surface for the scenarios I and II. This is also indicated by the magnitude of gradient dropping close to $0$ after $40s$ and $30s$ respectively as seen in Fig. \ref{DQMPC_Result}(b,d). The rapidly varying gradient curve of Scenario I (Fig. \ref{DQMPC_Result}(b)) shows that, each agent's potential field pushes other agents to reach the destination surface\footnote{Note that the scale of the gradient plots are different. The magnitude of the gradient is just to suggest that the robot has reached the target surface.}. We observe that the MAVs converge to the destination (Fig. \ref{DQMPC_Result}(c)) by avoiding the obstacle fields. 

In the U-shaped static obstacle case (scenario III), the agent is stuck because of control deadlock and fails to reach the destination surface as seen in Fig. \ref{DQMPC_Result}(e). The periodic pattern of the gradient curve (Fig. \ref{DQMPC_Result}(f)) and  non-zero gradient magnitude makes the deadlock situation evident. Most of these scenarios can also be visualized in the attached video submission.
\begin{figure}
	\centering
	\begin{subfigure}[t]{0.24\textwidth}        
		\includegraphics[scale=0.27]{Ang_5.pdf}   
		\caption{}     
		\label{Ang_5}
	\end{subfigure}
	\begin{subfigure}[t]{0.24\textwidth}        
		\includegraphics[scale=0.22]{Ang_5_grad.pdf}
		\caption{}
		\vspace{0.1cm}
		\label{Ang_5_grad}
	\end{subfigure} 
	\begin{subfigure}[t]{0.24\textwidth}
		\centering
		\includegraphics[scale=0.27]{Ang_32.pdf}  
		\caption{}    
		\label{Ang_32}
	\end{subfigure}
	\begin{subfigure}[t]{0.24\textwidth}
		\centering
		\includegraphics[scale=0.21]{Ang_32_grad.pdf}
		\caption{}
		\vspace{0.1cm}
		\label{Ang_32_grad}
	\end{subfigure}
	\begin{subfigure}[t]{0.24\textwidth}
		\centering
		\includegraphics[scale=0.27]{Ang_14.pdf} 
		\caption{}   	
		\label{Ang_14}
	\end{subfigure}
	\begin{subfigure}[t]{0.24\textwidth}
		\centering
		\includegraphics[scale=0.22]{Ang_14_grad.pdf}
		\caption{}
		\label{Ang_14_grad}
	\end{subfigure}      
	\caption{MAV trajectories and optimization gradients of approach angle force.}
	\vspace{-1em}
	\label{Ang_Result} 
\end{figure}
%\begin{figure*}[t!]
%    \centering
%    \begin{subfigure}[t]{0.33\textwidth}        
%        \includegraphics[scale=0.38]{Tang_5.pdf}        
%    \end{subfigure}%
%    ~ 
%    \begin{subfigure}[t]{0.33\textwidth}
%        \centering
%        \includegraphics[scale=0.4]{Tang_32.pdf}       
%    \end{subfigure}%
%	\begin{subfigure}[t]{0.33\textwidth}
%    	\centering
%    	\includegraphics[scale=0.4]{Tang_14.pdf}    	
%	\end{subfigure}  
%	~       
%    \begin{subfigure}[t]{0.33\textwidth}        
%        \includegraphics[scale=0.3]{Tang_5_grad.pdf}
%        \caption{Tangential band force with 5 Dynamic Obstacles}
%        \label{Tang_5_grad}
%    \end{subfigure}%
%    ~ 
%    \begin{subfigure}[t]{0.33\textwidth}
%        \centering
%        \includegraphics[scale=0.3]{Tang_32_grad.pdf}
%         \caption{Tangential band force with 2 Static Obstacles}
%         
%    \end{subfigure}%
%	\begin{subfigure}[t]{0.33\textwidth}
%    	\centering
%    	\includegraphics[scale=0.3]{Tang_14_grad.pdf}
%    	\caption{Tangential band force with U shaped static obstacle}
%	\end{subfigure}      
%    \caption{Trajectories and Optimization Gradients of the tangential band force using DQMPC}
%    \label{Tang_14_grad}
%    \label{Tang_Result}    	
%\end{figure*}

\subsection{Swivelling Robot Destination Method}
In the swivelling destination method, we use $k_s = 0.05$ as the gain for gradient magnitude impact. Fig. \ref{Swivel_Result}(a) shows, the MAVs spreading themselves along the destination surface depending on their distance from target and therefore the gradient. This method has a good convergence time of about $12s$ and $15s$ (about 3 times faster than the baseline DQMPC) for the scenarios I and II as can be observed in  Fig. \ref{Swivel_Result}(b) and  Fig. \ref{Swivel_Result}(d) respectively. Here, the direction of swivel depends on the agents orientation w.r.t. to target and each agent takes a clockwise or anti-clockwise swivel based on its orientation as clearly seen in  Fig. \ref{Swivel_Result}(c). The positive and negative values of the gradient indicate the direction. It can be observed in  Fig. \ref{Swivel_Result}(c), that the agent at times crosses over the $d_{safe}$ because of the higher target attraction force but respects the $d_{min}$ where the repulsion force is infinite.
Despite the better convergence time, the MAV is stuck when encountered with a U-shaped static obstacle (Fig. \ref{Swivel_Result}(e)) because of a control deadlock. This can also be observed with the non-zero gradient in Fig. \ref{Swivel_Result}(f).

\subsection{Approach Angle Towards Target Method}
Figure \ref{Ang_Result} showcases the results of the approach angle method. Since the additional potential field force enforces that not more than one agent has the same approach angle, the MAVs spread themselves while approaching the destination surface. This is associated with fast convergence to target surface as observed in Fig. \ref{Ang_Result}(a,b,c,d), for the scenarios I and II. The direction of the approach angle depends on the orientation of the MAVs w.r.t. to the target and is also indicated by the positive and negative values of the gradient magnitude. From the many experiments conducted in environments having only dynamic obstacles or sparsely spaced static obstacles, we observed that this method has the smoothest transition to the destination surface. However, similar to the swivelling destination method, it also fails to overcome control deadlock in case of a U-shaped obstacle as observed in Fig. \ref{Ang_Result}(e,f).


% \setlength{\belowcaptionskip}{-5pt}

\begin{figure}
	\centering
	\begin{subfigure}[t]{0.24\textwidth}        
		\includegraphics[scale=0.27]{Tang_5.pdf}   
		\caption{}     
		\label{Tang_5}
	\end{subfigure}
	\begin{subfigure}[t]{0.24\textwidth}        
		\includegraphics[scale=0.22]{Tang_5_grad.pdf}
		\caption{}
		\vspace{0.1cm}
		\label{Tang_5_grad}
	\end{subfigure} 
	\begin{subfigure}[t]{0.24\textwidth}
		\centering
		\includegraphics[scale=0.27]{Tang_32.pdf}  
		\caption{}    
		\label{Tang_32}
	\end{subfigure}
	\begin{subfigure}[t]{0.24\textwidth}
		\centering
		\includegraphics[scale=0.21]{Tang_32_grad.pdf}
		\caption{}
		\vspace{0.1cm}
		\label{Tang_32_grad}
	\end{subfigure}
	\begin{subfigure}[t]{0.24\textwidth}
		\centering
		\includegraphics[scale=0.27]{Tang_14.pdf} 
		\caption{}   	
		\label{Tang_14}
	\end{subfigure}
	\begin{subfigure}[t]{0.24\textwidth}
		\centering
		\includegraphics[scale=0.22]{Tang_14_grad.pdf}
		\caption{}
		\label{Tang_14_grad}
	\end{subfigure}      
	\caption{MAV trajectories and optimization gradients of tangential band method.}
	\label{Tang_Result}
\end{figure}


\subsection{Tangential Band Method}
The MAVs using tangential band method reach the destination surface in all the scenarios of static and dynamic obstacles as seen in Fig.~\ref{Tang_Result}. This method facilitates convergence to the target, for complex static obstacles, because, by principle the MAV traverses within the band until it finds a feasible path towards the target surface. $k_{tang}=2$  was used in simulations. As seen in Fig.~\ref{Tang_Result}(e), as soon as the MAV reaches obstacle surface, the tangential force acts, pushing it in the anti-clockwise direction. Then the UAV travels within this band until it is finally pulled towards the destination surface. The same principle applies for dynamic obstacles scenarios (see Fig.~\ref{Tang_Result}(a,c)) as well. Fig.~\ref{Tang_Result}(f) shows that, since the MAV overcomes field local minima and control deadlock, the gradient reaches $0$ for the U-shaped obstacle with a convergence time of $12$ seconds.  

% \setlength{\belowcaptionskip}{-10pt}

\begin{figure}[h]
\centering
\includegraphics[scale=0.32]{timing.pdf}
\caption{Average convergence time comparison.}
\label{timing}
\vspace{-1em}
\end{figure}


\subsection{Convergence Time Comparison}
Figure \ref{timing} compares the average convergence time $T_{cvg}$ for each of the proposed methods for 3 trails, with approximately equal travel distances between starting and destination surface for the different scenarios mentioned in Sec.~\ref{sub:expsetup}. The $T_{cvg}$ for swivelling destination, approach angle and tangential band is approximately ($15s$), which is better compared to DQMPC in the scenarios of only dynamic (Scenario I) or simple static (Scenario II) obstacles. In the U-shaped obstacle (Scenario III), except tangential band the other methods get stuck in field local minima. It may be noted that all the method can be generalized for any shape of the destination surface. In summary, the tangential band method would be the most preferred choice when the type of obstacles in environment are unknown. All the mentioned results and additional experimental results can also be observed in the enclosed video file.

\subsection{Antipodal Movement}
In order to further emphasize the efficacy of tangential band and approach angle methods, we demonstrate  obstacle avoidance for a task of  intra-surface antipodal position swapping. Here 8 MAVs start on a circular surface with a radius of $8m$, at equal angular distance from each other w.r.t.\ the center (world frame origin $[0,0,0]^{\top}$). The MAVs must reach a point $180^{o}$ in the opposite direction while simultaneously maintaining their orientation towards the center. The plot in Tab. \ref{antipodal_approachangle_tangband} (a) shows the trajectories taken by MAVs in this task using the approach angle method. All the MAVs convergence to their antipodal positions in around $T_{cvg}=13s$. It may be noted that there is no trajectory specified and the MAVs compute their own optimal motion plans. To improve the convergence time in this scenario, a consistent clockwise direction is enforced, but this is not necessary and convergence is guaranteed either way. The video attachment visually showcases this antipodal movement.

Similarly the antipodal position swapping was carried out using the tangential band method with the same experimental criteria (see figure in Tab. \ref{antipodal_approachangle_tangband} (b)). A convergence time of around $T_{cvg}=11s$ was observed, further emphasizing the speed of obstacle avoidance and convergence of the proposed method.  

% \setlength{\belowcaptionskip}{-20pt}
\begin{table}[t]
\begin{tabular}{cc}
 \centering
\hspace*{-10pt} \includegraphics[scale=0.32]{Antipodal_angle.pdf} &  \hspace*{-28pt} \includegraphics[scale=0.32]{Antipodal_tang.pdf} \\
\scriptsize{(a) Approach angle method}  & \hspace*{-28pt} \scriptsize{(b) Tangential band method}
\end{tabular}
\caption{Robot trajectories during antipodal position swapping.}
\label{antipodal_approachangle_tangband}
\end{table}


% \begin{figure}
% \centering
% \includegraphics[scale=0.4]{Antipodal_angle.pdf}
% \caption{Antipodal position swapping using approach angle method}
% \label{antipodal_approachangle}
% \end{figure}
% \begin{figure}
% 	\centering
% 	\includegraphics[scale=0.4]{Antipodal_tang.pdf}
% 	\caption{Antipodal position swapping using tangential band method}
% 	\label{antipodal_tangband}
% 	\vspace{-1em}
% \end{figure}
% 








