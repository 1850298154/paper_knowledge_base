\documentclass[letter, 10 pt, conference]{ieeeconf}  % Comment this line out if you need a4paper

%\documentclass[a4paper, 10pt, conference]{ieeeconf}      % Use this line for a4 paper

\IEEEoverridecommandlockouts                              % This command is only needed if 
                                                          % you want to use the \thanks command

\overrideIEEEmargins                                      % Needed to meet printer requirements.

% See the \addtolength command later in the file to balance the column lengths
% on the last page of the document

% The following packages can be found on http:\\www.ctan.org
% US Letter first page 	Top 21.2 Left 16.9 Right 16.9 Bottom 15.2 mm
% US Letter other pages 	Top 20.1 Left 16.9 Right 16.9 Bottom 15.2 mm
\usepackage[top=2.12cm, bottom=1.52cm, left=1.69cm, right=1.69cm]{geometry}
\let\labelindent\relax
\usepackage{enumitem}
\setlength{\parindent}{0in}
\usepackage{graphics} % for pdf, bitmapped graphics files
\graphicspath{{Pictures/}}
\usepackage{epsfig} % for postscript graphics files
\usepackage{mathptmx} % assumes new font selection scheme installed
\usepackage{times} % assumes new font selection scheme installed
\usepackage{amsmath} % assumes amsmath package installed
\usepackage{amssymb}  % assumes amsmath package installed
\usepackage{amsfonts}
\usepackage{graphicx}
\usepackage{tabularx}
\usepackage{multicol}
\usepackage{algorithm}
\usepackage{algorithmic}
\usepackage{multirow}
\usepackage{todonotes}%
\usepackage{lettrine}
\usepackage{wasysym}
\usepackage{subcaption}
\usepackage{bm}


\usepackage[font={scriptsize,it}]{caption} 
% \setlength{\belowcaptionskip}{-15pt}
% \usepackage{numprint}
% \npdecimalsign{.}
% \npproductsign{x}
% \nprounddigits{1}
% \npaddplusexponent 
% \usepackage[table]{xcolor}
\usepackage[linewidth=1pt]{mdframed}
\usepackage{setspace}

% \DeclareMathOperator*{\argmin}{arg\,min}
\newcommand{\argmax}[1]{\underset{#1}{\operatorname{\mathbf{arg}}\,\operatorname{\mathbf{max}}}\;}
\newcommand{\argmin}[1]{\underset{#1}{\operatorname{\mathbf{arg}}\,\operatorname{\mathbf{min}}}\;}
\newcommand{\x}{\mathbf{x}}
\newcommand{\xproj}{\mathbf{\check{x}}}
\newcommand{\vo}{\mathbf{v}}
\newcommand{\ao}{\mathbf{u}}
\newcommand{\X}{\mathbf{X}}
\newcommand{\p}{\mathbf{P}}
\newcommand{\z}{\mathbf{z}}
\newcommand{\e}{\mathbf{e}}
\newcommand{\la}{\mathbf{l}}
\newcommand{\ob}{\mathbf{o}}
\newcommand{\Ob}{\mathbf{O}}                                                                                                                                                                                                                                                             
\newcommand{\w}{\mathbf{w}}
\newcommand{\Wt}{\mathbf{W}}
\newcommand{\Om}{\mathbf{\Omega}}
\newcommand{\Sig}{\mathbf{\Sigma}}
% \newcommand{\vo}{\mathbf{v}}
\newcommand{\ur}{\mathbf{u}}
\newcommand{\zl}{\mathbf{z}}
\newcommand{\Xp}{\mathcal{X}}
\newcommand{\Px}{\mathbf{P}}
\newcommand{\Ze}{\mathbf{0}}
\newcommand{\rob}{\mathcal{L}}
\newcommand{\deriv}[1]{\frac{\partial}{\partial #1}}
\newcommand{\footnoteref}[1]{\textsuperscript{\ref{#1}}}
% \DeclareMathOperator{\Tr}{Tr}
\newcommand{\todoRahul}[1]{\textcolor{red}{\bf{#1}}}
\newcommand{\todoGeneral}[1]{\textcolor{red}{\bf{#1}}}
\newcommand{\todoAamir}[1]{\textcolor{blue}{\bf{#1}}}
\newcommand{\todoSujit}[1]{\textcolor{green}{\bf{#1}}}
\newcommand{\R}{\mathbf{R}}

\usepackage{cite}
\usepackage[bookmarks=false]{hyperref}
\usepackage{leftidx}

\definecolor{lightlightgrey}{rgb}{0.9,0.9,0.9}
\definecolor{Red}{rgb}{1,0,0}
\definecolor{Blue}{rgb}{0,0,1}
\definecolor{Green}{rgb}{0,1,0}
\definecolor{magenta}{rgb}{1,0,.6}
\definecolor{lightblue}{rgb}{0,.5,1}
\definecolor{lightpurple}{rgb}{.6,.4,1}
\definecolor{gold}{rgb}{.6,.5,0}
\definecolor{orange}{rgb}{1,0.4,0}
\definecolor{hotpink}{rgb}{1,0,0.5}
\definecolor{newcolor2}{rgb}{.5,.3,.5}
\definecolor{newcolor}{rgb}{0,.3,1}
\definecolor{newcolor3}{rgb}{1,1,1}
\definecolor{darkgreen1}{rgb}{0, .35, 0}
\definecolor{darkgreen}{rgb}{0, .6, 0}
\definecolor{darkred}{rgb}{.75,0,0}

\xdefinecolor{olive}{cmyk}{0.64,0,0.95,0.4}
\xdefinecolor{purpleish}{cmyk}{0.75,0.75,0,0}

% \renewcommand{\IEEEbibitemsep}{0pt plus 0.5pt}
\makeatletter
\IEEEtriggercmd{\reset@font\normalfont\fontsize{7.9pt}{8.40pt}\selectfont}
\makeatother
\IEEEtriggeratref{1}

\xdefinecolor{red}{rgb}{1,0,0}
% \mdfsetup{
%   align=center,
%   backgroundcolor=lightlightgrey,
%   rightmargin = 5pt,
%   leftmargin = 5pt,
%   linecolor=red!60!black,
%   linewidth=0.1pt,
%   topline=false,
%   rightline=false,
%   ntheorem = false, 
% %   innertopmargin = 0pt,
% %  innerbottommargin = 0pt,
%   innerleftmargin = 0pt,
%   innerrightmargin = 0pt,
%   leftline=false,
%   nobreak=false}
\usepackage{graphicx}
\usepackage{color}
%\usepackage{subfig}
 
\newcommand\blfootnote[1]{%
  \begingroup
  \renewcommand\thefootnote{}\footnote{#1}%
  \addtocounter{footnote}{-1}%
  \endgroup
}  

% \setlength{\textfloatsep}{40pt}
\setlength{\belowcaptionskip}{-5pt}

%%%%%%%%%%%%%%%%%%%%%%%%%%%%%% User specified LaTeX commands.
\usepackage{float}
\floatstyle{boxed}
\newfloat{math}{thp}{lop}
\floatname{math}{Equation}

\makeatother

\begin{document}

\title{Decentralized MPC based Obstacle Avoidance \\ for Multi-Robot Target Tracking Scenarios}

\author{Rahul Tallamraju$^{1,3}$, Sujit Rajappa$^2$, Michael Black$^1$, Kamalakar Karlapalem$^3$ and Aamir Ahmad$^1$
\thanks{\tiny{rahul.tallamraju, black, aamir.ahmad@tuebingen.mpg.de, sujit.rajappa@uni-tuebingen.de, kamal@iiit.ac.in}}
\thanks{\tiny{$^1$Perceiving Systems Department, Max Planck Institute for Intelligent Systems, T\"ubingen, Germany.}}
\thanks{\tiny{$^2$The Chair of Cognitive Systems, Department of Computer Science, University of T\"ubingen, T\"ubingen, Germany.}}
\thanks{\tiny{$^3$Agents and Applied Robotics Group, International Institute for Information Technology, Hyderabad, India.}}}
\maketitle

% \newgeometry{rmargin=1.69cm ,lmargin=1.69cm , tmargin=2.01cm, bmargin=1.52cm}

% \thispagestyle{plain}
\pagestyle{empty}
\begin{abstract}
In this work, we consider the problem of decentralized multi-robot target tracking and obstacle avoidance in dynamic environments. Each robot executes a local motion planning algorithm which is based on model predictive control (MPC). The planner is designed as a quadratic program, subject to constraints on robot dynamics and obstacle avoidance. Repulsive potential field functions are employed to avoid obstacles. The novelty of our approach lies in embedding these non-linear potential field functions as constraints within a convex optimization framework. Our method convexifies non-convex constraints and dependencies, by replacing them as pre-computed external input forces  in robot dynamics. The proposed algorithm additionally incorporates different methods to avoid field local minima problems associated with using potential field functions in planning. The motion planner does not enforce predefined trajectories or any formation geometry on the robots and is a comprehensive solution for cooperative obstacle avoidance in the context of multi-robot target tracking. Video of simulation studies: \url{https://youtu.be/umkdm82Tt0M}
\end{abstract}


\section{Introduction}


\IEEEPARstart{T}{wo} %
main challenges in the deployment of large-scale swarms are the localization and coordination of vehicles.
Localization methods that rely on external infrastructure 
(e.g., GPS) 
are prone to systematic errors (e.g., multipath effect)
and may not always be available.
Coordination strategies that are centralized can deconflict motion plans to prevent collisions and gridlock, but introduce a single point of failure and are difficult to scale in swarm size due to communication bandwidth limitations.

This paper presents a unified formation flying pipeline for unmanned aerial vehicles (UAVs).
Our pipeline uses \textit{onboard} sensors for localization, which eliminate the need for external positioning systems, and \textit{distributed} techniques for coordination, which enable each vehicle to make decisions independently while communicating their state to a subset of the team.
For \textit{localization}, we use an off-the-shelf commercial visual inertial odometry (VIO) package \cite{VIO}
that fuses inertial measurement unit (IMU) and downward-facing monocular camera measurements to estimate changes in the vehicle pose.
\edit{For \textit{coordination}, we present distributed formation control and task assignment strategies that run onboard the vehicles, do not rely on a common reference frame, and use vehicle-to-vehicle communication.} 
Key features of our formation control strategy include scalability to a large number of vehicles and robustness to disturbances.
The latter is crucial for reaching the desired formations with sensing imperfections.
Our task assignment strategy uses an auction-based algorithm to guarantee conflict-free assignments.
This algorithm can deconflict vehicle gridlocks resulting from distributed collision avoidance (type 3 deadlock~\cite{Wang2017}) and is well-suited for vehicles with limited computational capability and low-bandwidth communication. 


\begin{figure}[t!]
	\begin{center}
		\includegraphics[trim =0mm 10mm 0mm 0mm, clip, width=\columnwidth]{Figs/slanted_plane.png}	
		\caption{
		Six multirotors in a slanted plane formation.
		Vehicles communicate with each other, make distributed decisions onboard, and use VIO for localization.}
		\label{fig:slantedplane}
	\end{center}
\end{figure}


\subsection{Contributions}

This research extends our previous work on UAV formations~\cite{Fathian2019} and presents a unified pipeline consisting of \textit{onboard localization} and \textit{distributed coordination}.
The three main contributions of this work are:
\begin{enumerate}
    \item \edit{scalable formulation of control design suitable for
    onboard sensing without a common reference frame;}
    \item algorithms for deconfliction via \edit{distributed} task assignment of vehicles to desired formation points;    
    \item simulation- and hardware-ready open-source pipeline.
\end{enumerate}
\edit{Our pipeline is tested in hardware with six multirotors (see Fig.~\ref{fig:slantedplane}), and 
to our knowledge is the first demonstration of formation flying that does not rely on external sensing, fiducial markers for localization, a common reference frame, or a centralized base station for coordination.}
The only requirements for the presented pipeline are that the vehicles can communicate, can find the transformation between their VIO start frames, and the environment is sufficiently textured---a standard assumption for VIO systems.
As such, this framework paves the way for future, real-world deployments of aerial vehicle swarms in large numbers and without requiring external localization infrastructure.


\begin{figure} [t!]
\centering
	\begin{subfigure}[b]{0.32\columnwidth}
	   %
	    \includegraphics[width=0.8\textwidth,left]{Figs/Frames2_full.pdf}
	    \caption{\scriptsize full alignment}
	    \label{fig:frame-a}
	\end{subfigure}
	\begin{subfigure}[b]{0.32\columnwidth}
	    \includegraphics[width=0.8\textwidth,center]{Figs/Frames2_orientation.pdf}
	    \caption{\scriptsize orientation alignment}
	    \label{fig:frame-b}
	\end{subfigure}
	\begin{subfigure}[b]{0.32\columnwidth}
	    \includegraphics[width=0.8\textwidth,right]{Figs/Frames2_none.pdf}
	    \caption{\scriptsize no alignment}
        \label{fig:frame-c}
	\end{subfigure}
\caption{\edit{Required alignment of UAV frames in existing swarm strategies: (a) the most restrictive case requiring a common reference frame, i.e., orientation and origin of the frames must be aligned; (b) only the orientation of the frames must be aligned; (c) no alignment restrictions (this work).}}
	\label{fig:Frames}
\end{figure}




\subsection{Related Work}

Existing aerial swarms can be grouped based on the coordination (centralized vs.\ distributed) and localization (external vs.\ onboard) methods used. 
\edit{It is further crucial to distinguish these methods based on the level of alignment required for the vehicle coordinate frames; see Fig.~\ref{fig:Frames}.} 
 
\edit{
Works with \textit{centralized} coordination and \textit{external} localization include~\cite{Preiss2017, Honig2018, Du2019}, which are based on lightweight UAVs with limited onboard computational capability and therefore rely on an external motion capture system and a base station.
Works with \textit{distributed} coordination and \textit{external} localization include \cite{wilson2020robotarium}, \cite{enright2004spheres}, where robots execute distributed controls  based on external localization by motion capture and ultrasonic beacons, respectively.
Works with \textit{centralized} coordination and \textit{onboard} localization include~\cite{Forster2013}, \cite{Loianno2016}, which use a ground station for task assignment among vehicles.
In \cite{Weinstein2018}, formation flying based on VIO is demonstrated, where motion planning and assignment are run on a base station to ensure collision-free trajectories.
The coordination strategies used in aforementioned works require a \textit{common reference frame} (Fig.~\ref{fig:frame-a}).
}


\edit{
Despite the large body of work on formation control~\cite{Oh2015}, and the variety of onboard sensing solutions for localization (e.g., VIO~\cite{Delmerico2018}), few frameworks demonstrated formation flying with \textit{distributed} coordination and \textit{onboard} localization.
A key reason is reliance of many distributed control and assignment algorithms on aligned frames (Fig.~\ref{fig:frame-a}, \ref{fig:frame-b}), which require computation-expensive and/or communication-intensive synchronization/consensus steps for frame alignment.
Equally important, dependence on alignment in existing methods \cite{Wang2017,Turpin2014, van2011reciprocal, morgan2016swarm} diminishes robustness to inherent noise and unobservable errors that cannot be corrected (e.g., disparities between the actual and estimated body frame \textit{orientation} caused by VIO drift).
Leveraging coordination methods that are \textit{robust to misaligned frames} is hence crucial and a focus of this work. 
}






\edit{
Examples of other pipelines with distributed coordination and onboard localization include \cite{Montijano2016,Tron2016}.
Both works demonstrated formation flying on three UAVs, required information from an external motion capture system due to hardware limitations, did not incorporate collision avoidance, and required frame alignment.
}
\edittwo{Note that while~\cite{Montijano2016,Tron2016} can achieve formations with arbitrary headings as illustrated in Fig.~\ref{fig:frame-c}, knowledge of relative orientations is still required; therefore, they belong to the category of Fig.~\ref{fig:frame-b}.}






\if 0

\r{
decentralized coordination setting combined with VIO:
D-CAPT [26]~\cite{}:
ORCA ~\cite{}: 
CBF [2]~\cite{} :
[A]
}

\r{Robusteness in coordination,  with compounded noise/latency, which would eventually break (b).\\


some existing algorithm might as well
work in a similar fully decentralized setting, when combined with VIO
as proposed here. For example, D-CAPT [26], ORCA, CBF [2] might also be
useful for such a task and are computationally even more efficient than
the proposed approach. \\

R2:  onboard sensing for localization ->
 Finally, the related work section only
focuses on this aspect of the pipeline, discussing how many formation papers include
onboard localization but barely sells the advantages of the coordination module (the actual
proposal of the paper) against other competitors such as [26] or [A] or to mention similar
coordination pipelines. \\


Given a solution to this problem, the controller in Section III seems unnecessary, each drone
has a target position and can use a local controller with collision avoidance that drives it to
that position. Note that such controllers exists in the literature (e.g., RVO in any of its
multi-agent variantes), they are distributed in nature and only require local sensing.


}

\fi

\section{State-of-the-Art}
\label{sec:sota} 
Distributed systems have been maintaining their importance for the last several decades due to the increase in the need for scalable and reliable distributed applications while preserving high performance. 
To analyze distributed systems comprehensively and compare them in terms of features and services, various surveys and evaluations have been published in the past. Surveys on cloud providers, data warehouses, distributed file systems, or metadata services can be counted among them. 

Cloud providers are analyzed and evaluated in terms of elasticity \cite{CMART}, computing power \cite{comperative-benchmarking}, and cost to performance efficiency \cite{fair-benchmarking} in previous efforts. Widely used distributed services are also analyzed in many works, such as a survey on stream processing \cite{stream-benchmarking} or performance and dependability evaluation of MapReduce systems \cite{MapReduce-benchmarking}. Similarly, different aspects of distributed systems are studied in several surveys, like reliability analysis on distributed systems \cite{reliability-survey} and load balancing characteristics of known systems \cite{load-balancing-survey}.

As a big part of distributed systems, data warehouses and file systems are studied for many specifications. Evaluation of distributed data warehouses for the cost-effectiveness of different hardware configurations \cite{ALOJA} and query performance of distinct design choices\cite{benchmarking-data-warehouse} are among the known efforts in these works. Distributed file systems are examined in many past works for general concepts \cite{file-systems-concepts,file-systems-gen1} or specific applications such as distributed access control \cite{access-control-file-systems}. Due to the differences in optimization, design techniques, and the complex interactions between the file systems and other system components like the kernel or operating system, benchmarking distributed file systems is not trivial. To identify the important metrics for the evaluation of distributed file systems, researchers also studied benchmarking file systems \cite{File-system-benchmarking,benchmarking-file-rocket}. 

Analysis of distributed coordination services in terms of general characteristics and importance of coordination \cite{importance-of-coordination} and the comparison of existing algorithms \cite{paxos-made-simple} are among the published works. However, to the best of our knowledge, there is no published work on the evaluation of distributed coordination systems. As mentioned in the Introduction, due to the lack of standard benchmarking tools for distributed coordination services, developers widely use their ad-hoc benchmarks, which are prone to unfair comparisons or limited results for the evaluation of the systems. This study is unique in identifying the metrics and parameters for the evaluation of distributed coordination systems, discussing how each system uses these metrics and parameters for its evaluation, pinpointing the deficiencies of well-known benchmarking suites in evaluating distributed computing systems, and finally discussing the features of an ideal distributed coordination benchmark. 
%
%Sections related to main theory and system architecture!
\section{Proposed Approach}
\label{sec:proposedappraoch} 
% systemoverview

%%\subsection{System Overview}
%%\label{sec:sysoverview}
%Our multi-MAV system does not consist of a central computational unit. Each of our MAVs is equipped with an on-board CPU and GPU to perform all computations. Although our architecture also does not depend on a centralized communication network, the field implementation is done through a central wifi access point. Each MAV runs its own instance of the following software modules (blue blocks in Fig.~\ref{fig:system}).
%\begin{itemize}
%	\item A low-level position and yaw controller (flight controller) moduSystem Overview and Problem Formulationle,
%	\item a self-localization module using on-board GPS, IMU and barometer,
%	\item a cooperative detection and tracking (CDT) module, which is the core contribution of this article and described in subsection~\ref{sec:cdt}, and
%	\item an MPC-based formation controller and obstacle avoidance module that allows us to maintain a perception-driven formation, described in subsection~\ref{sec:mpc}.
%\end{itemize}Figure~\ref{fig:system} details the flow of data among these modules. The data shared between any two MAVs consists of their self-pose estimates and the detection measurements of the tracked person. All the aforementioned software modules run on board in realtime. In the following sections, after introducing notations, we focus on the detailed description of our CDT module. Therein we describe how our proposed MCDT approach enables the MAV formation to track seemingly-small and far-away persons with high accuracy and without losing them from any MAV's FOV during tracking.
\subsection{Preliminaries}
We describe the proposed framework for a multi-robot system tracking a desired target. 
%However, it is important to note that the MPC based motion planner is agnostic to the type of robot (aerial or ground). 
For the concepts presented, we consider Micro Aerial Vehicles (MAVs) that hover at a pre-specified height $h_{gnd}$. Furthermore, we consider 2D target destination surface. However, the proposed approaches can be extended to any 3D surface. Let there be $K$ MAVs $R_1,..., R_K$ tracking a target $x_t^P$, typically a person $P$. Each MAV computes a desired destination position $\xproj_t^{R_k}$ in the vicinity of the target position. The pose of $k^{\text{th}}$ MAV in the world frame at time $t$ is given by $\xi_t^{R_k} = [ (\x_t^{R_k})^\top ~ (\Theta_t^{R_k})^\top] \in \mathbb{R}^6$. 
%$\x_t^{R_k}$ denotes the 3D position and $\Theta_t^{R_k}$ represents the 3 orientation angles. 
Let there be $M$ obstacles in the environment $O_1,...,O_M$. The $M$ obstacles include $R_k$'s neighboring MAVs and other obstacles in the environment.

%Our work is motivated by the broader application of motion capture of a target in an outdoor environment by multiple robots based on vision. The key requirements for an outdoor motion capture system are (i) to not lose track of the moving target, and (ii) to ensure that the robots avoid other robot agents and all obstacles (static and dynamic) in their vicinity. In order to tackle both these objectives, we formulate a formation control (FC) algorithm, as detailed in Algorithm \ref{Alg:fc}. The main steps have the following functionality, (i)  destination point computation depending on target movement, (ii) obstacle avoidance force generation, (iii)  decentralized quadratic model predictive control (DQMPC) based planner for way point generation, and (iv) a low-level position controller to track generated way-points.
The key requirements in a multi-robot target tracking scenario are, (i) to not lose track of the moving target, and (ii) to ensure that the robots avoid other robot agents and all obstacles (static and dynamic) in their vicinity. In order to address both these objectives in an integrated approach, we formulate a formation control (FC) algorithm, as detailed in Algorithm \ref{Alg:fc}. The main steps have the following functionality, (i)  destination point computation depending on target movement, (ii) obstacle avoidance force generation, (iii)  decentralized quadratic model predictive control (DQMPC) based planner for way point generation, and (iv) a low-level position controller.

To track the waypoints generated by the MPC based planner we use a geometric tracking controller.  The controller is based on the control law proposed in~\cite{lee2010geometric},  which has a proven global convergence, aggressive maneuvering controllability and excellent position tracking performance. Here, the rotational dynamics controller is developed directly on $SO(3)$ and thereby avoids any singularities that arise in local coordinates. Since the MAVs used in this work are under-actuated systems, the desired attitude generated by the outer-loop translational dynamics is controlled by means of the inner-loop torques.

%The following subsection describes DQMPC planner and local minima avoidance methodologies.

%\subsection{Low level controller: Waypoint tracking}\label{sec:geometrictracking}
%
%We use the control law proposed in~\cite{lee2010geometric},  which has a proven global convergence, aggressive maneuvering capability and excellent trajectory tracking performance.
%Here, the rotational dynamics controller is developed directly on $SO(3)$ and thereby avoids any singularities that arise in local coordinates.
%
%Considering the trajectory tracking task, at a given time step the tracking error in position and velocity are defined as  $\boldsymbol e_{ p} = \boldsymbol {p}_W - \boldsymbol p_a$ and $ \boldsymbol e_{ v} = \boldsymbol {\dot p}_W - \boldsymbol {\dot p}_a$ respectively.
%%
%The desired force for the translational dynamics is given as,
%\begin{align}\label{eq:poscontrol}
%\rho	=	&(m \ddot{\boldsymbol p}_a-\boldsymbol K_{d}\boldsymbol e_{ v} -\boldsymbol K_{p}\boldsymbol e_{p} -& \nonumber\\
%& \quad-\boldsymbol K_{i}\int_{t_0}^t \boldsymbol e_{\boldsymbol p}dt - mge_3) \cdot \boldsymbol{R}_B^We_3,&
%\end{align}
%where the diagonal positive definite gain matrices $\boldsymbol K_{d}$, $\boldsymbol K_{p}$, $\boldsymbol K_{i}$ define Hurwitz polynomials.
%The desired hovering thrust is realized by $f_z =  \rho\, e_3 $   and by aligning the body vertical axis along the direction of the $\rho$ defined as,
%
%{
%	\small{
%		\begin{align}
%		\vec{z}_{R_d} = \frac{
%			m \ddot{\boldsymbol p}_a - \boldsymbol K_{d}\boldsymbol e_{v} -\boldsymbol K_{p}\boldsymbol e_{p}  -\boldsymbol K_{i}\int_{t_0}^t \boldsymbol e_{p}dt - mge_3 }{\Vert m \ddot{\boldsymbol p}_a-\boldsymbol K_{d}\boldsymbol e_{v} -\boldsymbol K_{p}\boldsymbol e_{p}  -\boldsymbol K_{i}\int_{t_0}^t \boldsymbol e_{p}dt - mge_3 \Vert},
%		\end{align}
%	}\normalsize
%	
%	\noindent where $\vec{z}_{R_d}$ is the third column of the desired attitude rotation matrix $\boldsymbol{R}_{B_d}^W$ defined as $ \boldsymbol{R}_{B_d}^W= \begin{bmatrix}\vec{x}_{R_d}, \vec{y}_{R_d}, \vec{z}_{R_d} \end{bmatrix}\in SO(3)$. Since the quadrotor UAV is an underactuated system, the desired attitude generated by the outer-loop translational dynamics is controlled by means of the inner-loop torques, that are generated for controlling the rotational dynamics, to track a desired attitude rotation $\boldsymbol{R}_{B_d}^W$. 
%	The other two columns $\vec{x}_{R_d}$ and $\vec{y}_{R_d}$ of $\boldsymbol{R}_{B_d}^W$, which account for the remaining degrees of freedom, should be chosen such that their direction is orthogonal to $\vec{z}_{R_d}$ and minimize the yaw error.
%	Therefore%the heading direction is decided by them is defined as,
%	\begin{align}
%	\vec{x}_{R_d} = \vec{y}_{R_d} \times \vec{z}_{R_d},\qquad
%	\vec{y}_{R_d} = \frac{\vec{z}_{R_d} \times \vec{x}_{R_d}}{\Vert \vec{z}_{R_d} \times \vec{x}_{R_d} \Vert}.
%	\end{align}
%	
%	For the rotational dynamics, assuming that $\boldsymbol \omega_{B_d}=[ {\boldsymbol{R}^W_{B_d}}{}^T  {\dot{\boldsymbol{R}}^W_{B_d}}{}^T]_{\vee}$, where $[\cdot]_\vee$ represents the inverse (vee) operator from $so(3)$ $\to$ $\mathbb{R}^3$, the attitude tracking error $\boldsymbol e_R \in \mathbb{R}^3 $ is defined similarly to~\cite{lee2010geometric} as
%	\begin{equation}\label{eq:rot_error}
%	\boldsymbol e_{R}=\dfrac{1}{2}[{\boldsymbol{R}^W_{B_d}}^T  {\boldsymbol{R}^W_{B}}  - {\boldsymbol{R}^W_{B}}^T {\boldsymbol{R}^W_{B_d}}]_\vee,
%	\end{equation}
%	and the tracking error of the angular velocity $\boldsymbol e_\omega \in \mathbb{R}^3 $ is given by
%	%
%	\begin{equation}\label{eq:omega_error}
%	\boldsymbol e_{\omega}
%	=
%	{\boldsymbol \omega}_B -
%	{ {\boldsymbol{R}^W_{B}}^T {\boldsymbol{R}^W_{B_d}} {\boldsymbol \omega}_{B_d}}.
%	\end{equation}
%	%
%	In order to obtain an asymptotic convergence  to $\boldsymbol 0$ of the rotational error $\boldsymbol {e}_R$ one can choose the following controller
%	%
%	\begin{align}\label{eq:rotational_error}
%	{\boldsymbol \tau} &= 
%	- \boldsymbol K_{\omega} \boldsymbol {e}_\omega - \boldsymbol K_{r} \boldsymbol {e}_R
%	- \boldsymbol K_{ir}
%	\int_{t_0}^t \boldsymbol {e}_R
%	+\boldsymbol \omega_{B}\times \boldsymbol I_{B}\boldsymbol\omega_{B}
%	- & \nonumber\\
%	%
%	&  - \boldsymbol I_{B}
%	( \left[ {\boldsymbol \omega}_B \right]_\wedge
%	{ {\boldsymbol{R}^W_{B}}^T {\boldsymbol{R}^W_{B_d}} {\boldsymbol \omega}_{B_d}} -
%	{\boldsymbol{R}^W_{B}}^T {\boldsymbol{R}^W_{B_d}} 
%	\dot{\boldsymbol  \omega}_{B_d}),&
%	\end{align}%\red{CHECK SIGN OF $e_R$}
%	where the diagonal positive-definite gain matrices $\boldsymbol K_{\omega}$, $\boldsymbol K_{r}$, $\boldsymbol K_{ir}$ define Hurwitz polynomials and $\left[ {\boldsymbol \omega}_B \right]_\wedge$ is the skew symmetric matrix of ${\boldsymbol \omega}_B$.
%}

\subsection{DQMPC based Formation Planning and Control}
The goal of the formation control algorithm running on each MAV $R_k$ is to  
\begin{enumerate}
\item Hover at a pre-specified height $h_{gnd}$.
\item Maintain a distance $d^{R_k}$ to the tracked target.
\item Orient at yaw, $\psi^{R_k}$, directly facing the tracked target.
\end{enumerate}
Additionally, MAVs must adhere to the following constraints, 
\begin{enumerate}
\item To maintain a minimum distance $d_{min}$ from other MAVs as well as static and dynamic obstacles.
\item To ensure that MAVs respect the specified state limits.
\item To ensure that control inputs to MAVs are within the pre-specified saturation bounds.
\end{enumerate}
\vspace{-0.5em}
\begin{algorithm}[h]
\small
\caption{MPC-based formation controller and obstacle avoidance by MAV $R_k$ with inputs $\lbrace \x_t^{P}, ~~ \x_t^{O_j}; j = 1:M \rbrace$} 
\begin{spacing}{1.3}
\begin{algorithmic}[1]
\label{Alg:fc}
\STATE $\lbrace \xproj_t^{R_k} \rbrace \leftarrow $ Compute Destination Position $\lbrace \psi_t^{R_k}, \x_{t}^{R_k}, d^{R_k}, h_\mathrm{gnd} \rbrace$
\STATE $\left[\mathbf{f}_{t}^{R_k}(0),\dots,\mathbf{f}_{t}^{R_k}(N)\right] \leftarrow $ Obstacle Force $\lbrace {\x_{t}^{R_k}, \x_{t}^{O_j}(1:N+1),\forall j } \rbrace$
\STATE $\lbrace \x_t^{R_k*}, \dot{\x}_t^{R_k*}, \nabla J_{DQMPC} \rbrace \leftarrow $ DQMPC$\lbrace \xproj_t^{R_k}, \x_t^{R_k}, \mathbf{f}_t^{R_k}(0:N),\mathbf{g}\rbrace$
\STATE $\lbrace \psi_{t+1}^{R_k} \rbrace \leftarrow$ Compute Desired Yaw $\lbrace \x_t^{R_k}, \|\nabla J_{DQMPC} \|\rbrace$
\STATE $\mathbf{Transmit}$  $\x_t^{R_k*}(N+1),\dot{\x}_t^{R_k*}(N+1),  \psi_{t+1}^{R_k}$ to Low-level Controller
\normalsize
\end{algorithmic}
\end{spacing}
\end{algorithm} 
\vspace{-1em}
Algorithm \ref{Alg:fc} outlines the strategy used by each MAV $R_k$ at every  discrete time instant $t$. In line 1, MAV $R_k$ computes its desired position $\xproj_t^{R_k}$ on the desired surface using simple trigonometry. For example, if the desired surface is a circle, centered around the target location $\x_{t}^{P}$ with a radius $d^{R_k}=constant \ \forall R_k$, then the desired position for time instant $t$ is given by $\xproj_t^{R_k} = \x_{t}^{P}+ \left[d^{R_k}cos(\psi_t^{R_k}) ~~ d^{R_k}sin(\psi_t^{R_k}) ~~ h_{gnd}\right]^{\top}$. Here $\psi_t^{R_k}$ is the yaw of $R_k$ w.r.t. the target. It is important to note that the distance $d^{R_k}$ is an input to the DQMPC and is not necessarily the same for each MAV. 
%If and only if the desired target surface, around the target being tracked, is a circle, then the $d^{R_k} = constant \ \forall \ R_k$.

In line 2, an input potential field force vector $\left[\mathbf{f}_{t}^{R_k}(0),\dots,\mathbf{f}_{t}^{R_k}(N)\right] ^{\top} \in \mathbb{R}^{3 \times (N+1)}$ is computed for a planning horizon of $(N+1)$ discrete time steps by using the shared trajectories from other MAVs and positions of obstacles in the vicinity. If no trajectory information is available, the instantaneous position based potential field force value is used for the entire horizon. Section \ref{potential_field} details the numerical computation of these field force vectors.

In line 3, an MPC based planner solves a convex optimization problem (DQMPC) for a planning horizon of $(N+1)$ discrete time steps. We consider nominal accelerations $[\ao_t^{R_k}(0) \cdots \ao_t^{R_k}(N)]^\top \in \mathbb{R}^{3 \times (N+1)}$ as control inputs to DQMPC. The accelerations describe the 3D translational motion of $R_k$.
\begin{equation}
 \ao_t^{R_k}(n) = ~ \ddot{\x}_t^{R_k}(n)
\end{equation}
where $n$ is the current horizon step. 
%\begin{figure*}[t!]
%    \centering
%    \begin{subfigure}[t]{0.33\textwidth}
%        \centering
%        \includegraphics[scale=0.9]{Visualization_1_1.pdf}
%        \caption{Profile of simplified 2-D motion objective with obstacles}
%        \label{visual_1_1}
%    \end{subfigure}%
%    ~ 
%    \begin{subfigure}[t]{0.33\textwidth}
%        \centering
%        \includegraphics[scale=0.9]{Visualization_2_1.pdf}
%        \caption{Top view of the profile}
%        \label{visual_2_1}
%    \end{subfigure}%
%	\begin{subfigure}[t]{0.33\textwidth}
%    	\centering
%    	\includegraphics[scale=0.3]{Visualization_3_1.pdf}
%    	\caption{Gazebo illustraton of the local minima/control deadlock problem}
%    	\label{visual_3_1}
%	\end{subfigure}      
%    \caption{Visualizations illustrating the local minima and robot deadlock issue. Cyan cylinder is the desired surface to reach. Red cylinders are static obstacles.}
%    \label{visual}	
%\end{figure*}
\setlength{\belowcaptionskip}{0pt}
\begin{figure*}[t]
	\centering
	\begin{subfigure}[t]{0.24\textwidth}        
		\includegraphics[scale=0.4]{SSRR_schematic_14.pdf}   
		\caption{\scriptsize{Field local minima problem}}
		%\label{problem_1}
	\end{subfigure}
	\begin{subfigure}[t]{0.24\textwidth}        
		\includegraphics[scale=0.4]{SSRR_schematic_24.pdf}
		\caption{\scriptsize{Swivelling Robot Destination Method.}}
		%\label{swivel}
	\end{subfigure} 
	\begin{subfigure}[t]{0.24\textwidth}
		\centering
		\includegraphics[scale=0.4]{SSRR_schematic_34.pdf}  
		\caption{\scriptsize{Approach Angle Method}}
		%\label{angle}
	\end{subfigure}
	\begin{subfigure}[t]{0.24\textwidth}
		\centering
		\includegraphics[scale=0.4]{SSRR_schematic_44.pdf}
		\caption{\scriptsize{Tangential Band Method}}
		%\label{tangentialBand}
	\end{subfigure}
	\caption{Illustration of the field local minima problem in obstacle avoidance and different proposed methods.} 
	\label{proposedApproachfig}
	\vspace{-1em}
\end{figure*}
\setlength{\belowcaptionskip}{-5pt}
%\begin{figure*}[t!]
%	\centering
%	\begin{subfigure}[t]{0.49\textwidth}
%		\centering
%		\includegraphics[scale=0.7]{Common.pdf}
%		\caption{Field local minima due to axial alignment of robots}
%		\label{problem_1}
%	\end{subfigure}%
%	~ 
%	\begin{subfigure}[t]{0.49\textwidth}
%		\centering
%		\includegraphics[scale=0.7]{Swivel.pdf}
%		\caption{Resolution of field local minima by using variable velocity swivelling destinations }
%		\label{swivel}
%	\end{subfigure}%
%	\caption{Illustration of the swivelling robot destination strategy}
%\end{figure*}
%\begin{figure*}
%	\centering
%	\begin{subfigure}[t]{0.49\textwidth}
%		\centering
%		\includegraphics[scale=0.7]{Angle_1.pdf}
%		\caption{Resolution of field local minima by using variable angle of approach potential field. }
%		\label{angle}
%	\end{subfigure}%
%	\begin{subfigure}[t]{0.49\textwidth}
%		\centering
%		\includegraphics[scale=0.5]{Tangential_Band.pdf}
%		\caption{Illustration of the tangential force band around obstacles. Red arrow represents x-axis and Green arrow represents y-axis of target respectively}
%		\label{tangentialBand}
%	\end{subfigure}    
%	\caption{Illustration of the approach angle field strategy and tangential band strategy. }
%\end{figure*}
The state vector of the discrete-time DQMPC consists of $R_k$'s position $\x_t^{R_k}(n) \in \mathbb{R}^3$ and  velocity $\dot{\x}_t^{R_k}(n)\in \mathbb{R}^3$. 
The optimization objective is,
\footnotesize
\begin{align} \label{J_DQMPC}
J_{\mathrm{DQMPC}} =&  \Big(\sum_{n=0}^{N} \big( {\bm{\Omega_{i}}} (\ao_t^{R_k}(n)+\mathbf{f}_{t}^{R_k}(n)+\bm{g})^2\big) + \nonumber \\
& \bm{\Omega_{t}} \big(\left[\x_t^{R_k}(N+1)^{\top} ~~ \dot{\x}_t^{R_k}(N+1)^{\top} \right] - \left[ (\xproj_t^{R_k})^{\top}  ~~ \mathbf{0}^{\top}\right]\big)^2 \Big)
\end{align}
\normalsize
The optimization is defined by the following equations. \small
\begin{equation} \label{DQMPC}
\x(1)_t^{R_k*}\dots\x(N+1)_t^{R_k*},\ao_t^{R_k*}(0)\dots\ao_t^{R_k*}(N) = \argmin{\ao_t^{R_k}(0)\dots\ao_t^{R_k}(N)} (J_{DQMPC})
\end{equation} 
subject to, 
\begin{align}
& \hspace{-0.7cm}\left[\x_t^{R_k}(n+1)^{\top} ~ \dot{\x}_t^{R_k}(n+1)^{\top}\right]^\top =  \label{state-space}\nonumber\\
& \quad\quad\quad\mathbf{A}\left[\x_t^{R_k}(n)^{\top} ~ \dot{\x}_t^{R_k}(n)^{\top}\right]^\top + \mathbf{B}(\ao_t^{R_k}(n)+\mathbf{f}_{t}^{R_k}(n)+\bm{g}), \\
& \ao_\mathrm{min} \leq ~ \ao_t^{R_k}(n) \leq \ao_\mathrm{max}, \\
& \x_\mathrm{min} \leq ~ \x_t^{R_k}(n) \leq ~ \x_\mathrm{max}, \\
& \dot{\x}_\mathrm{min} \leq ~ \dot{\x}_t^{R_k}(n) \leq ~ \dot{\x}_\mathrm{max} \label{last_DQMPC}
 \end{align}
 \normalsize
where, $\bm{\Omega_{i}}$ and $\bm{\Omega_{t}}$ are positive definite weight matrices for input cost and terminal state (computed desired position $\xproj_t^{R_k}$ and desired velocity $\dot{\xproj}_t^{R_k} = 0$) respectively, %Terminal state is the computed desired position $\xproj_t^{R_k}$ and desired velocity $\dot{\xproj}_t^{R_k} = 0$. 
$\mathbf{f}_{t}^{R_k}(n)$ is the pre-computed external obstacle force, $\bm{g}$ is the constant gravity vector. The discrete-time state-space evolution of the robot is given by \eqref{state-space}. The  dynamics ($\mathbf{A} \in \mathbb{R}^{3\times3}$) and control transfer ($\mathbf{B} \in \mathbb{R}^{3\times3}$) matrices are given by,
\begin{equation}
\mathbf{A}=
  \begin{bmatrix}
    \mathbf{I}_3 & \Delta t\mathbf{I}_3 \\
    \mathbf{0}_3 & \mathbf{I}_3
  \end{bmatrix}, ~
\mathbf{B}=
  \begin{bmatrix}
    \frac{\Delta t^2}{2}\mathbf{I}_3 \\
    \Delta t\mathbf{I}_3
  \end{bmatrix}.
\end{equation}
where, $\Delta t$ is the sampling time. The quadratic program generates optimal control inputs $\left[\ao_t^{R_k}(0) \cdots \ao_t^{R_k}(N) \right]$ and the corresponding trajectory $\left[ \x_t^{R_k}(1) ~ \dot{\x}_t^{R_k}(1)  \cdots \x_t^{R_k}(N+1) ~ \dot{\x}_t^{R_k}(N+1) \right]$ towards the desired position. The final predicted position and velocity of the horizon $\x_t^{R_k*}(N+1),\dot{\x}_t^{R_k*}(N+1)$ is used as  desired input to the low-level flight controller. 
The MPC based planner avoids obstacles (static and dynamic) through pre-computed horizon potential force $\mathbf{f}_{t}^{R_k}(n)$. This force is applied as an external control input component to the state-space evolution equation, thereby, preserving the optimization convexity. Previous methods in literature consider non-linear potential functions within the MPC formulation, thereby, making optimization non-convex and computationally expensive. 
%The obstacle forces considered arise from both static and dynamic obstacles (including other robots) in the vicinity of the robot $R_k$. 
%Robots/MAVs communicate their self-pose to each other over wireless network.  We showcase through simulations that the proposed method is very effective and fast in avoiding obstacles (both static and dynamic) while also tracking the target.

In the next step (line 4 of Algorithm \ref{Alg:fc}), the desired yaw is computed as $\psi_{t+1}^{R_k} = atan2\big(\frac{y_t^P - y_t^{R_k}}{x_t^P - x_t^{R_k}}\big)$. This  describes the angle with respect to the target position $\x_t^P$ from the MAV's current position $\x_t^{R_k}$. 
%This equation is slightly modified in Section \ref{diminishingOmega} to ensure avoidance of local minima with dynamic obstacles.
The way-point commands consisting of the position and desired yaw angle are sent to the low-level flight  position controller.
Although no specific robot formation geometry is enforced, the DQMPC naturally results in a dynamic formation depending on desired destination surface.



% \setlength{\textfloatsep}{5pt}


\subsection{Handling Non-Convex Collision Avoidance Constraints} \label{potential_field}
In our approach, at any given point there are two forces acting on each robot, namely (i) the attractive force due to the optimization objective (eq.\eqref{DQMPC}), and (ii) the repulsive force due to the potential field around obstacles ($\mathbf{f}_{t}^{R_k}(n)$). In general, a repulsive force can be modeled as a force vector based on the distance w.r.t. obstacles. Here, we have considered the potential field force variation as a hyperbolic function ($F_{hyp}^{R_k,O_j}(d(n))$) of distance between MAV $R_k$ and obstacle $O_j$. We use the formulation in \cite{secchi2013bilateral} to model $F_{hyp}^{R_k,O_j}(d(n))$. Here, $d(n) = \|\x_{t-1}^{R_k}(n) - \x_t^{O_j}(n)\|_2, ~ \forall n \in [0,\dots,N]$ is the distance between the MAV's predicited horizon positions from the previous time step $(t-1)$ and the obstacles (which includes shared horizon predictions of other MAVs).  The repulsive force vector is,
\begin{equation}\label{repulsive}
\bm{F}_{rep}^{R_k,O_j}(n) = 
\begin{cases}
F_{hyp}^{R_k,O_j}(d(n))\;\mathbf{\alpha},& \text{if} ~~ d(n) < d_{safe} \\
0, & \text{if} ~~ d(n) > d_{safe}
\end{cases}\;,
\end{equation}
where, $d_{safe}$ is the distance from the obstacle where the potential field magnitude is non-zero. %received at the current time step $t$. 
%$d(0) =  \|\x_{t}^{R_k}(0) - \x_t^{O_j}(0)\|_2$ is the distance for the zeroth horizon step. 
 $\mathbf{\alpha} =  \frac{\x_t^{O_j}(n)-\x_{t-1}^{R_k}(n)}{\|\x_t^{O_j}(n)-\x_{t-1}^{R_k}(n)\|_2}$ is the unit vector in the direction away from the obstacle.  Additionally we consider a distance $d_{min} << d_{safe}$ around the obstacle, where the potential field magnitude tends to infinity. 
The potential force acting on an agent per horizon step $n$ is,
\begin{equation} \label{totalForce}
\mathbf{f}_t^{R_k}(n)= \sum_{\forall \ j }^{}\mathbf{F}_{rep}^{R_k,O_j}(n),
\end{equation}
which is added into the system dynamics in eq.\eqref{state-space}.
\vspace{-0.5em}
%The forces are non-zero starting from a distance $d_{safe}$ around the obstacle point. The magnitude of these forces gradually tends towards infinity near the obstacle point ($d_{min}$) as defined by equation \ref{cotanfield}. 

\subsection{Resolving the Field Local Minima Problem}
The key challenge in potential field based approaches is the field local minima issue \cite{koren1991potential}. When the summation of attractive and repulsive forces acting on the robot is a zero vector, the robot encounters field local minima problem \footnote{note that this is different from optimization objective's local minima.}. Equivalently, a control deadlock could also arise when the robot is constantly pushed in the exact opposite direction.
%The local minima problem arises when the resultant of the summation of attractive and repulsive forces is a zero vector. A zero resultant vector implies that the robot is stuck at that point in space. Equivalently, a similar problem arises when the robot tries to move towards the target and is constantly pushed away in the exact opposite direction. The robot can infinitely execute this behavior and hence we characterize the behavior as a control deadlock in this paper. 
Both local minima and control deadlock are undesirable scenarios.
From  equation \eqref{state-space} and Algorithm \ref{Alg:fc}, it is clear that the  optimization can characterize control inputs that will not lead to collisions, but, cannot characterize those control inputs that lead to these scenarios.
%From equation (\ref{state-space}) and Algorithm \ref{Alg:fc}, it is clear that the DQMPC uses pre-computed input potential forces for a horizon, to avoid obstacles. This implies that the optimization does not have direct access to the potential field function and its gradients. Indicating that the optimization can characterize control inputs that will not lead to collisions, but, cannot characterize those control inputs that lead to a field local minima. To illustrate and visualize this complex scenario we consider the second optimization objective of equation (\ref{DQMPC}). For the sake of visualization we remove the velocity terms and restrict the motion of robot to 2D ($\x_t^{R_k}(n), \xproj_t^{R_k} \in \mathbb{R}^2$). The following equation represents the modified objective.
%\begin{equation} \label{paraboloid}
%J = \bm{\Omega_{term}} (\x_t^{R_k}(N+1) - \xproj_t^{R_k})^2
%\end{equation}
%Graphically, equation (\ref{paraboloid}) is a paraboloid centered around $\x_t^{P}$ (assume $=\left[0,0\right]$) as shown in Fig. \ref{visual_1_1}. Consider two closely shaped static obstacles each with a potential field determined by equation (\ref{cotanfield}). For ease of visualization, the field around these obstacles is represented as red-impenetrable cylinders (Fig. \ref{visual_1_1} and Fig.  \ref{visual_2_1}). The attractive and repulsive fields sum to a zero vector near the intersection of these two obstacle fields. Since the optimization problem being solved is a minimization problem, the robot always moves in the direction of the negative gradient of the paraboloid of Fig. \ref{visual_1_1}, which is towards point $\xproj_t^{R_k}$. This destination point is currently obstructed by potential fields, as shown in Fig. \ref{visual_2_1} and Fig. \ref{visual_3_1}, the robot is likely to get stuck at the point of intersection infinitely or would perform a control deadlock motion. This is because at every time instant $t$, the optimization has access to only a limited amount of state-space for a given horizon defined by $\sum_{n=0}^{N} (u_{max}\frac{\Delta t^2}{2}+\dot{\x}_t^{R_k}(n+1)\Delta t + \x_t^{R_k}(n+1))$. This refers to a rectangular patch around the current position of the robot. If this rectangular state-space patch is not bigger than the dimensions of the obstacles then the robot can encounter field local minima or deadlock scenarios. Making this rectangular patch bigger is not possible, as the patch is defined using the control saturation limits of the robot. 
In such cases the gradient of optimization would be non-zero indicating that the robot knows its direction of motion towards the target, but cannot reach the destination surface because the potenial field functions are not directly used in DQMPC constraints.
%due to a lack of knowledge about the potential field function.
%It is important to note that the gradient of optimization is non-zero, because, the point $\xproj_t^{R_k}$ is not reached, as shown in the illustration using Gazebo in Fig. \ref{visual_3_1}. Essentially, the robot knows its direction of motion towards target, due to non-zero gradient value, but it  cannot find a way out of the field local minima or deadlock issue due to a lack of knowledge on the potential field function.
Here, we propose three methodologies for field local minima avoidance.
%Swivelling robot destination (Sec.~\ref{diminishingOmega}) and approach angle method (Sec.~\ref{angle_of_approach}) work effectively avoiding all obstacles (static and dynamic) in the environment. %The tangential band method (Sec.~\ref{tangential_field}) not only provides an unified static and dynamic obstacle avoidance solution but also facilitates target surface convergence in most scenarios.

%\normalsize
\subsubsection{Swivelling Robot Destination (SRD) method} \label{diminishingOmega}
This method is based on the idea that the MAV destination $\xproj_t^{R_k}$ is an external input to the optimization. Therefore, each MAV can change its $\xproj_t^{R_k}$ to push itself out of field local minima. For example, consider the scenario shown in Fig. \ref{proposedApproachfig}(a), where three robots are axially aligned towards the target. Since the angles of approach are equal, the desired destination positions are the same for $R_1$ and $R_2$, i.e., $\xproj_t^{R_1} = \xproj_t^{R_2}$. This results in temporary deadlock and will slow the convergence to desired surface. We construct the SRD method to solve this deadlock problem as follows: (i) the gradient of DQMPC objective of $R_k$ is computed, (ii)  a swivelling velocity $\omega^{R_k}$ is calculated based on the magnitude of gradient, and (iii) $\xproj_t^{R_k}$ swivels by a distance proportional to $\omega^{R_k}$ as shown in Fig.\ref{proposedApproachfig}(b). This ensures that the velocities at which each $\xproj_t^{R_k}$ swivels is different until the robot reaches the target surface, where the gradient tends to zero. The gradient of the optimization with respect to the last horizon step control and state vectors, is computed as follows.
%The gradients and their 3 dimensional scalar sum are given by
%over all the three dimensions ($\mathbb{R}^3$) are given by the below equations.
%Although the net field from Robot $2$ and $3$ will ultimately push Robot $1$ away, the scenario however leads to slow convergence to target surface. 
%Since the angles of approach of all the robots are the same, their desired destination points are the same, i.e., $\xproj_t^{R_1} = \xproj_t^{R_2} = \xproj_t^{R_3}$. Such a situation will lead to a temporary deadlock and field local minima and will slow the convergence to the target surface. A naive solution to this problem is to share the destination points between the robots and choose different points on the desired destination surface. However, this would lead to a communication overhead to establish negotiation and task assignment. To resolve the problem without explicit communication, each robot swivels its destination point with a velocity proportional to the sum of gradients of the $DQMPC$ objective as shown in Fig.\ref{swivel}. This would ensure that the velocities at which $\xproj_t^{R_k}$'s would swivel would be different based on their distance from the desired destination. Consequently, the swivelling of destination leads to a natural deadlock resolution. As the robot nears the destination point, the gradient tends to zero, therefore $\xproj_t^{R_k}$ will be static. This solution is illustrated in Fig. \ref{swivel}. The gradients and their scalar sum over all the three dimensions ($\mathbb{R}^3$) are given by the below equations. 

\small
\begin{equation}
\frac{\partial J_{DQMPC}}{\x_t^{R_k}(N+1)} = 2 \bm{\Omega_{t}} (\left[\x_t^{R_k}(N+1)^{\top} ~~ (\dot{\x}_t^{R_k}(N+1))^{\top} \right] - \left[ (\xproj_t^{R_k})^{\top}  ~~ \mathbf{0}^{\top}\right])^{\top} \nonumber
\end{equation}
\begin{equation}
\frac{\partial J_{DQMPC}}{\ao_t^{R_k}(N)} = 2 \bm{\Omega_{i}}(\ao_t^{R_k}(n)+\mathbf{f}_{t}^{R_k}(n)+\bm{g}) + 
2 \bm{\Omega_{t}}B(\x_t^{R_k}(N+1)-\xproj_t^{R_k}) \nonumber
\end{equation}
\begin{equation}
\nabla J_{DQMPC}^{R_k} = \frac{\partial J_{DQMPC}}{\x_t^{R_k}(N+1)} + \frac{\partial J_{DQMPC}}{\ao_t^{R_k}(N)}.
\end{equation}
%\begin{eqnarray}
%\frac{\partial J_{DQMPC}}{\x_t^{R_k}(N+1)} = 2 \bm{\Omega_{term}} (\left[\x_t^{R_k}(N+1) ~~ \dot{\x}_t^{R_k}(N+1) \right] - \left[ \xproj_t^{R_k}  ~~ \mathbf{0}\right])^{\top} \nonumber \\
%\frac{\partial J_{DQMPC}}{\ao_t^{R_k}(N)} = 2 \bm{\Omega_{input}}(\ao_t^{R_k}(n)+\mathbf{f}_{t}^{R_k}(n)+\bm{g}) + 
%2 \bm{\Omega_{term}}B(\x_t^{R_k}(N+1)-\xproj_t^{R_k})^{\top} \nonumber \\
%\nabla J_{DQMPC}^{R_k} = \frac{\partial J_{DQMPC}}{\x_t^{R_k}(N+1)} + \frac{\partial J_{DQMPC}}{\ao_t^{R_k}(N)} 
%\end{eqnarray}
\normalsize

For circular target surface, the destination point swivel rate is,
\small
\begin{eqnarray}
\xproj_t^{R_k} =& \x_{t}^{P}+ \begin{bmatrix} d^{R_k}cos(\psi_t^{R_k} \pm k_s \|\nabla J_{DQMPC}^{R_k}\|) \\ d^{R_k}sin(\psi_t^{R_k}\pm k_s\|\nabla J_{DQMPC}^{R_k}\|) \\ h_{gnd} \end{bmatrix} ^\top 
\end{eqnarray}
\normalsize
where, $k_s$ is a user-defined gain controlling the impact of $\|\nabla J_{DQMPC}^{R_k}\|$. The swivel direction of each ${R_k}$ is decided by its approach direction to target. Positive and negative $\psi_t^{R_k}$ leads to a clockwise and anti-clockwise swivel respectively. 
%Though this approach ensures  obstacle avoidance for both static and dynamic obstacles, it gets stuck in extreme static obstacle scenarios to reach the target surface  because of local minima as discussed in Sec.~\ref{sec:results}.
%Unfortunately, this method does not guarantee convergence to the destination point in the presence of static obstacles (scenario of Fig. \ref{Visualization_3_1}). 
%This situation is more critical and is elaborated in the Section \ref{tangential_field}.
%\begin{figure}
%	\centering
%	\subfloat[]{\centering \hoveringpositiondis}\,
%	\subfloat[]{\centering\hoveringorientationdis}\\
%	\subfloat[]{\centering \forcedisturbance}\,
%	\subfloat[]{\centering\torquedisturbance}
%	\caption{Results of the hovering with external force/torque disturbance.~\ref{fig:plots_sim1_dis}(a): Desired (dashed line) and current (solid line) position $\boldsymbol p_d$ in x(red), y(green) and z(blue). \ref{fig:plots_sim1_dis}(b): Desired (dashed line) and current (solid line) orientation $\boldsymbol \eta_d$ in roll(red), pitch(green) and yaw(blue).  \ref{fig:plots_sim1_dis}(c--d): external force($\boldsymbol f_{\text{\rm ext}}$) and torque($\boldsymbol{\tau}_{\text{ext}}$) applied to the hexarotor}
%	\label{fig:plots_sim1_dis}
%	\vspace{-0.5cm}
%\end{figure}
%\def\hoveringpositionerrordis{\includegraphics[width=.49\columnwidth]{figures/Hovering_Error_Position_disturbance.eps}}
%\def\hoveringorientationerrordis{\includegraphics[width=.49\columnwidth]{figures/Hovering_Error_Orientation_disturbance.eps}}
%\def\hoveringpositionerrordis{\includegraphics[width=.49\columnwidth]{figures/Hovering_Error_Position_disturbance.eps}}

%\begin{figure*}[t]
%    \centering
%    \begin{subfigure}[t]{0.33\textwidth}        
%        \includegraphics[scale=0.38]{DQMPC_5.eps}        
%        \label{DQMPC_5}
%    \end{subfigure}%
%    ~ 
%    \begin{subfigure}[t]{0.33\textwidth}
%        \centering
%        \includegraphics[scale=0.4]{DQMPC_32.eps}       
%        \label{DQMPC_32}
%    \end{subfigure}%
%	\begin{subfigure}[t]{0.33\textwidth}
%    	\centering
%    	\includegraphics[scale=0.4]{DQMPC_14.eps}    	
%    	\label{DQMPC_14}
%	\end{subfigure}  
%	~       
%    \begin{subfigure}[t]{0.33\textwidth}        
%        \includegraphics[scale=0.3]{DQMPC_5_grad.eps}
%        \caption{DQMPC with 5 Dynamic Obstacles}
%        \label{DQMPC_5_grad}
%    \end{subfigure}%
%    ~ 
%    \begin{subfigure}[t]{0.33\textwidth}
%        \centering
%        \includegraphics[scale=0.3]{DQMPC_32_grad.eps}
%         \caption{DQMPC with 2 Static Obstacles}
%        \label{DQMPC_32_grad}
%    \end{subfigure}%
%	\begin{subfigure}[t]{0.33\textwidth}
%    	\centering
%    	\includegraphics[scale=0.3]{DQMPC_14_grad.eps}
%    	\caption{DQMPC with U shaped static obstacle}
%    	\label{DQMPC_14_grad}
%	\end{subfigure}      
%    \caption{Trajectories and Optimization Gradients of the baseline DQMPC optimization}
%    \label{DQMPC_Result}    	
%\end{figure*}

\subsubsection{Approach Angle Towards Target Method} \label{angle_of_approach}
In this method, the local minima and control deadlock is addressed by including an additional potential field function which depends on the approach angle of the robots towards the target.
%An alternate solution to avoid the field local minima and deadlock problem in case of dynamic obstacles is to have an additional potential field force on the robot which is a function of the angle of approach to target. As observed in the Fig.\ref{problem_2}, the problem of field local minima is encountered when the angles of approach to the target, of robots, are similar. 
Here, we (i) compute the approach angle of robot $R_k$ w.r.t. the target, (ii) compute the gradient of the objective, and (iii) compute a force $F_{ang}^{R_k,O_j}$ in the direction normal to the angle of approach, as shown in Fig. \ref{proposedApproachfig}(c). The magnitude of  $F_{ang}^{R_k,O_j}$ depends on the sum of gradients $\nabla J_{DQMPC}^{R_k}$ and the hyperbolic function (see Sec. \ref{potential_field}) between the approach angles of robot $R_k$ and obstacles $O_j$ w.r.t. the target. This potential field force is computed as,
\begin{equation}
\mathbf{F}_{ang}^{R_k,O_j}(n) = \nabla J_{DQMPC}^{R_k} \;F_{hyp}^{R_k,O_j}((\theta^{R_k}(n)-\theta^{O_j}(n))^2)\mathbf{\hat{\beta}}  \;\;\forall j 
\end{equation}
\begin{equation}
\mathbf{\beta} = \pm \frac{\x_t^{R_k}(n)-\xproj_t^{R_k}}{\|\x_t^{R_k}(n)-\xproj_t^{R_k}\|_2} \; ;\quad \mathbf{\hat{\beta}}.\mathbf{\beta} = 0.
\end{equation}
Here $\theta^{R_k}(n)$ and $\theta^{O_j}(n)$ are the angles of $R_k$ and obstacle $O_j$ with respect to the target. The angles $\theta^{O_j} \ \forall \ j$ w.r.t target are computed by each $R_k$, as part of the force pre-computation using $O_j$'s position. $\beta$ and $\hat{\beta}$ are the unit vectors in the approach direction to the target and its orthogonal respectively, with $\pm$ dependent on $\theta^{R_k}$ w.r.t. the target. 
%The direction is postive for positive approach angles $\psi_t^{R_k}$ and negative for negative angles. 
The $\mathbf{f}_{t}^{R_k}(n)$ for $n^{th}$ horizon step is therefore
\begin{equation}
\mathbf{f}_{t}^{R_k}(n) = \sum_{\forall \ j}\mathbf{F}_{rep}^{R_k,O_j}(n)+\mathbf{F}_{ang}^{R_k,O_j}(n).
\end{equation}
Notice that the non-linear constraint of the two approach angles not being equal is converted into an equivalent convex constraint using pre-computed force values. This method ensures collision avoidance in the presence of obstacles and fast convergence to the desired target, because the net potential force direction is always away from the obstacle. 
%Unfortunately in certain extreme static obstacle scenarios as discussed later in Sec.~\ref{sec:results}, it fails to reach target surface.


%\begin{figure*}[t!]
%    \centering
%    \begin{subfigure}[t]{0.33\textwidth}        
%        \includegraphics[scale=0.38]{Swivel_5.eps}        
%        \label{Swivel_5}
%    \end{subfigure}%
%    ~ 
%    \begin{subfigure}[t]{0.33\textwidth}
%        \centering
%        \includegraphics[scale=0.4]{Swivel_32.eps}       
%        \label{Swivel_32}
%    \end{subfigure}%
%	\begin{subfigure}[t]{0.33\textwidth}
%    	\centering
%    	\includegraphics[scale=0.4]{Swivel_14.eps}    	
%    	\label{Swivel_14}
%	\end{subfigure}  
%	~       
%    \begin{subfigure}[t]{0.33\textwidth}        
%        \includegraphics[scale=0.3]{Swivel_5_grad.eps}
%        \caption{Swivelling Destination with 5 Dynamic Obstacles}
%        \label{Swivel_5_grad}
%    \end{subfigure}%
%    ~ 
%    \begin{subfigure}[t]{0.33\textwidth}
%        \centering
%        \includegraphics[scale=0.3]{Swivel_32_grad.eps}
%         \caption{Swivelling Destination with 2 Static Obstacles}
%        \label{Swivel_32_grad}
%    \end{subfigure}%
%	\begin{subfigure}[t]{0.33\textwidth}
%    	\centering
%    	\includegraphics[scale=0.3]{Swivel_14_grad.eps}
%    	\caption{Swivelling Destination with U shaped static obstacle}
%    	\label{Swivel_14_grad}
%	\end{subfigure}      
%    \caption{Trajectories and Optimization Gradients of the swivelling destination approach}
%    \label{Swivel_Result}    	
%\end{figure*}

\subsubsection{Tangential Band Method} \label{tangential_field}
The previous methods at times, do not facilitate target surface convergence because of field local minima and control deadlock. For example, static obstacles forming a U-shaped boundary between the target surface and $R_k$'s position, as shown in Fig.\ref{proposedApproachfig}(d). If the target surface is smaller than the projection of the static obstacle along the direction of approach to the desired surface, the planned trajectory is occluded. Therefore, the SRD method cannot find a feasible direction for motion. Furthermore,  angle of approach field acts only when the $\theta^{R_k}$ and $\theta^{O_j}$ are equal w.r.t. the target.
%Consider a scenario with closely spaced static obstacles forming a U-shaped boundary between the target surface and robot $R_k$'s position, as shown in Fig. \ref{tangentialBand}. In such a situation robot $R_k$ is surrounded by static obstacles and would be stuck in field local minima or deadlock movement (moving towards and away from target in a loop). Swiveling destination cannot help the robot in this situation as the path to the target surface is completely blocked. Angle of approach field only acts when the angles of obstacles and the robot are approximately the same with respect to the target. With static obstacles shaped as shown in Fig. \ref{tangentialBand}, the angles $\theta^{R_k}$ and $\theta^{O_j}$ might never be close or equal in magnitude.

In order to resolve this, we construct a band around each obstacle where an instantaneous (at $n=0$) tangential force acts about the obstacle center. The width of this band is $ > (\dot{\x}_{max}^ {\top}\Delta t + u_{max} \frac{\Delta t^2}{2}) $ and therefore, the robot cannot tunnel out of this band within one time step $\Delta t$. This makes sure that once the robot enters  tangential band, it exits only after it has overcome the static obstacles. The direction depends on $R_k$'s approach towards the target, resulting in clockwise or anti-clockwise force based on $-ve$ or $+ve$ value of $\psi_t^{R_k}$ respectively. The outer surface of the band has only the tangential force effect, while the inner surface has both tangential and repulsive force (repulsive hyperbolic field) effects on $R_k$.
 
Within the band the diagonal entries of the positive definite weight matrix $\bm{\Omega}_{t}$ are reduced to a very low value ($\prec \prec \Omega_{t,max}$).  This ensures that the attraction field on the robot is reduced while it is being pushed away from the obstacle. Consequently, the effect of tangential force is higher in the presence of obstacles. Once the robot is out of the tangential field band (i.e., clears the U-shaped obstacles), the high weight of the $\bm{\Omega}_{t}$ is restored and the robot converges to its desired destination. Fig. \ref{proposedApproachfig}(d) illustrates this method.
The tangential force is,
\begin{equation}
\mathbf{F}_{tang}^{R_k,O_j}(0) = k_{tang}\nabla J_{DQMPC}^{R_k}\; \mathbf{\hat{\alpha}}, 
\end{equation}
where $k_{tang}$ is user-defined gain and $\mathbf{\hat{\alpha}}$ is defined s.t. $\mathbf{\pm\;\hat{\alpha}.\alpha} = 0$. The weight matrix and step horizon potential are therefore,
\begin{equation}
\bm{\Omega_{t}} = \bm{\Omega_{t,min}}, ~~ \text{if } \x_t^{R_k} \leq d(0)+d_{band} 
\end{equation}
\begin{equation}
\mathbf{f}_{t}^{R_k}(n) = \sum_{\forall j} \mathbf{F}_{rep}^{R_k,O_j}(n) + \mathbf{F}_{tang}^{R_k,O_j}(0)\;,
\end{equation}
%\begin{eqnarray}
%\mathbf{F}_{tang}^{R_k}(0) &=& K_{tang}\nabla J_{DQMPC}^{R_k}\mathbf{\hat{\alpha}} \\
%\bm{\Omega_{t}} &=& min(\bm{\Omega_{t}}), ~~ \text{if } \x_t^{R_k}(0) \leq d(0)+d_{band}  \\
%\mathbf{f}_{t}^{R_k}(n) &=& \sum_{\forall j} \mathbf{F}_{rep}^{R_k}(n) + \mathbf{F}_{tang}^{R_k}(0)
%\end{eqnarray}
%where $K_{tang}$ is a user-defined gain and $\mathbf{\pm\hat{\alpha}.\alpha} = 0$. $\alpha$ was defined in equation (\ref{repulsive}). 
where $d_{band}$ is the tangential band width. The values of the weights can vary between $\bm{\Omega_{t,min}} \prec \bm{\Omega_{t}} \prec \bm{\Omega_{t,max}}$ and changes only when the robot is within the influence of tangential field of any obstacle.
In summary, the tangential band method not only guarantees collision avoidance for any obstacles but also facilitates robot convergence to the target surface. 
In rare scenarios, e.g., when the static obstacle almost encircles the robots and if the desired target surface is beyond such an obstacle, the robots could get trapped in a loop within the tangential band. This is because a minimum attraction field  towards the target always exists. Since in this work,  the objective is local motion planning in dynamic environments with no global information and map, we do not plan for a feasible trajectory out of such situations. 

\begin{figure}
	\centering
	    \begin{subfigure}[t]{0.24\textwidth}        
	    	\includegraphics[scale=0.27]{DQMPC_5.pdf}   
	    	\caption{}     
	    	\label{DQMPC_5}
	    \end{subfigure}
	    \begin{subfigure}[t]{0.24\textwidth}        
	    	\includegraphics[scale=0.22]{DQMPC_5_grad.pdf}
	    	\caption{}
	    	\vspace{0.1cm}
	    	\label{DQMPC_5_grad}
	    \end{subfigure} 
	    \begin{subfigure}[t]{0.24\textwidth}
	    	\centering
	    	\includegraphics[scale=0.27]{DQMPC_32.pdf}  
	    	\caption{}    
	    	\label{DQMPC_32}
	    \end{subfigure}
	    \begin{subfigure}[t]{0.24\textwidth}
	    	\centering
	    	\includegraphics[scale=0.21]{DQMPC_32_grad.pdf}
	    	\caption{}
	    	\vspace{0.1cm}
	    	\label{DQMPC_32_grad}
	    \end{subfigure}
		\begin{subfigure}[t]{0.24\textwidth}
			\centering
			\includegraphics[scale=0.27]{DQMPC_14.pdf} 
			\caption{}   	
			\label{DQMPC_14}
		\end{subfigure}
		\begin{subfigure}[t]{0.24\textwidth}
			\centering
			\includegraphics[scale=0.22]{DQMPC_14_grad.pdf}
			\caption{}
			\label{DQMPC_14_grad}
		\end{subfigure}      
		\caption{MAV trajectories and optimization gradients of baseline DQMPC optimization. The colors (red, green, blue, black, magenta) represent $R_k$s and their respective gradients.}
		\label{DQMPC_Result} 
		\vspace{-1em}
\end{figure}


\section{Results and Discussions}
\label{sec:results} 
% section: results

% dev set decision table
% manually created from 25-official-html.tgz
% with custom editing

\begin{table}
\centering
\begin{tabular}{l|rr}
\toprule
 & \multicolumn{2}{c}{\textbf{ELAS F1}} \\
\textbf{Treebank} & \textbf{sem-frag}
 & \textbf{heuristic}\\
%\hline
\midrule
% e7 elmo udpf task ar padt	allennlp 090 dm lbert luxfb ar padt 20200424 080759
% e7 elmo udpf task ar padt	copy2e arcase mark rel
ar\_padt & \textbf{70.99} & 59.74 \\
% e5 elmo udpf task bg btb	allennlp 090 dm pbert u bg btb 20200312 003243	
% e7 elmo udpf task bg btb	copy2e encase mark rel	
bg\_btb & \textbf{88.09} & 86.19 \\
%\hline
\midrule
% 	e3 elmo udpf task cs cac	allennlp 090 dm pbert luxfb cs cac 20200424 002226
%	e5 elmo udpf task cs cac	copy2e arcase mark rel	
cs\_cac & \textbf{86.51} & 74.41 \\
% e3 elmo udpf task cs fictree	allennlp 090 dm pbert luxfb cs cac 20200424 002226
% e5 elmo udpf task cs fictree	copy2e arcase mark rel	77.37
cs\_fictree & \textbf{83.23} & 77.37  \\
% e3 elmo udpf task cs fictree	allennlp 090 dm pbert u cs cac 20200419 171603
% e3 elmo udpf task cs cac	copy2e arcase mark
cs\_pdt & \textbf{79.58} & 	71.19 \\
%\hline
\midrule
% e7 elmo udpf task en ewt + ud25 en gum + ud25 en lines + ud25 en partut	allennlp 090 dm lbert u en ewt 20200312 051351
% e7 elmo udpf task en ewt + ud25 en gum + ud25 en lines + ud25 en partut	copy2e encase mark cc rel
en\_ewt & \textbf{84.71} & 	82.86 \\
% e3 elmo udpf task et edt + task et ewt	allennlp 090 dm mbert u et 20200419 234001
% e3 elmo udpf task et edt + task et ewt	copy2e arcase mark
et\_edt & 62.74 & \textbf{69.35} \\
% e5 elmo udpf task fi tdt fasttext udpf task fi tdt	allennlp 090 dm lbert u fi tdt 20200420 050020
% e7 elmo udpf task fi tdt	copy2e arcase mark rel
fi\_tdt & \textbf{83.64} & 71.84 \\ 
% e5 elmo udpf task fr sequoia + ud25 fr gsd + ud25 fr partut + ud25 fr spoken	allennlp 090 dm mbert u fr sequoia 20200312 072651
% e3 elmo udpf task fr sequoia + ud25 fr gsd + ud25 fr partut + ud25 fr spoken	copy2e	
fr\_sequoia & \textbf{88.65} &  87.53 \\
% e3 elmo udpf task it isdt + ud25 it partut + ud25 it postwita + ud25 it twittiro + ud25 it vit	allennlp 090 dm lbert u it isdt 20200419 172143	
% e7 elmo udpf task it isdt	copy2e encase mark cc rel	
it\_isdt & \textbf{90.13} & 88.28 \\
% e3 plain udpf task lt alksnis	allennlp 090 dm mbert u lt alksnis 20200420 014618	
% e3 plain udpf task lt alksnis	copy2e arcase mark	
lt\_alksnis & \textbf{73.63} & 57.84 \\
% e7 elmo udpf task lv lvtb	allennlp 090 dm mbert luxfb lv lvtb 20200423 191418	
% e5 elmo udpf task lv lvtb fasttext udpf task lv lvtb	copy2e encase rel	
lv\_lvtb & \textbf{81.82} & 71.29 \\
%\hline
\midrule
% 	e7 elmo udpf task nl alpino + task nl lassysmall	allennlp 090 dm lbert u nl alpino 20200312 025649	
% e7 elmo udpf task nl alpino + task nl lassysmall	copy2e encase mark cc rel	
nl\_alpino & \textbf{89.93} & 89.00 \\
% e7 elmo udpf task nl alpino + task nl lassysmall	allennlp 090 dm lbert u nl alpino 20200312 025649	
% e7 elmo udpf task nl alpino + task nl lassysmall	copy2e encase mark cc rel	
nl\_lassysmall & 79.00 & \textbf{81.24} \\
%\hline
\midrule
% e5 elmo udpf task pl lfg + task pl pdb	allennlp 090 dm mbert luxfb pl lfg 20200423 222537	
% e5 elmo udpf task pl lfg + task pl pdb	copy2e encase mark rel	
pl\_lfg & \textbf{94.12} & 93.84 \\
% e3 fasttext udpf task pl lfg + task pl pdb	allennlp dev dm lbert luxf pl 20200416 194726	
% e5 elmo udpf task pl lfg + task pl pdb	copy2e arcase mark rel	
pl\_pdb & \textbf{82.25} & 78.27 \\
%\hline
\midrule
% e7 elmo udpf task ru syntagrus	allennlp 090 dm lbert lufb ru syntagrus 20200423 210055	
% e7 elmo udpf task ru syntagrus	copy2e arcase mark	
ru\_syntagrus & \textbf{88.48} & 80.03 \\
% e3 elmo udpf task sk snk plain udpf task sk snk	allennlp 090 dm mbert u sk snk 20200420 020636	
% e7 elmo udpf task sk snk	copy2e arcase mark	75.98
sk\_snk  & \textbf{81.30} & 75.98 \\
% e3 elmo udpf task sv talbanken	allennlp 090 dm lbert u sv talbanken 20200419 195336	
% e7 elmo udpf task sv talbanken	copy2e encase mark cc rel	
sv\_talbanken & \textbf{84.54} & 81.32 \\
% e3 plain udpf task ta ttb	allennlp 090 dm mbert u ta ttb 20200419 232103	
%	e3 plain udpf task ta ttb	copy2e arcase	
ta\_ttb & \textbf{55.68} & 43.94 \\
% e3 elmo udpf task uk iu	allennlp 090 dm mbert u uk iu 20200420 004219	
% e7 elmo udpf task uk iu	copy2e arcase mark	
uk\_iu & \textbf{82.41} & 76.88 \\
%\hline
\bottomrule
\end{tabular}
\caption{Development set ELAS F1 score %f-score
        for the best semantic parser evaluated without connecting
            fragmented graphs (sem-frag)
        and
        for the best combination of heuristic rules
            (heuristic)
}
\label{devresults:decision_custom}
\end{table}

% eof

Table~\ref{devresults:decision_custom} compares the semantic parser against the heuristic approach on the ELAS F1 metric.
The evaluation script was run without connecting fragmented graphs and format validation.
For all but two treebanks, the semantic parser performs better than the
best
heuristic approach.
For some languages, the difference in performance is large.
For \texttt{et\_ewt}, which does not have a development set,
we suspect that we overfitted our semantic parser on the
\texttt{et\_ewt} training data
by allowing it to train for 75 epochs.

% test set results table
% manually created from eval pages linked on
% https://quest.ms.mff.cuni.cz/sharedtask/cgi-bin/overview.pl

% main body generated by copy and pasting the qualitative tables,
% then using `cut -f1,16` to get the right columns, pasting them
% together with `paste` and tabs converted to `&` and \\ added to
% lines in `vim`

\begin{table}
\centering
\begin{tabular}{l|rrr}
\toprule
 & \multicolumn{3}{c}{\textbf{ELAS F1}} \\
\textbf{Treebank} & \textbf{subm}
 & \textbf{frag fix} & \textbf{re-run}\\
\midrule
Arabic-PADT         &  57.19  &  70.08  &  \bf 70.40  \\
Bulgarian-BTB       &  77.29  &  89.58  &  \bf 89.60  \\
Czech-FicTree       &  70.04  &  80.77  &  \bf 81.63  \\
Czech-CAC           &  71.72  &  86.00  &  \bf 86.38  \\
Czech-PDT           &  65.94  &  79.03  &  \bf 79.84  \\
Czech-PUD           &  64.34  &  77.37  &  \bf 78.08  \\
Dutch-Alpino        &  71.44  &  87.61  &  \bf 87.77  \\
Dutch-L.Small       &  64.03  &  77.39  &  \bf 77.24  \\
English-EWT         &  70.61  &  \bf 83.56  & \bf 83.56  \\
English-PUD         &  70.25  &  86.88  & \bf 87.03  \\
Estonian-EDT        &  62.29  &  68.20  &  \bf 68.37  \\
Estonian-EWT        &  55.70  &  \bf 61.19  &  60.67  \\
Finnish-TDT         &  73.02  &  \bf 84.36  &  84.33  \\
Finnish-PUD         &  71.58  & \bf 84.62  & \bf 84.62  \\
French-Sequoia      &  77.44  &  87.58  & \bf 88.60  \\
French-FQB          &  74.30  &  82.68  & \bf 83.26  \\
Italian-ISDT        &  71.98  &  \bf 90.24  &  90.23  \\
Latvian-LVTB        &  72.41  &  81.81  &  \bf 82.40  \\
Lithuanian-AL.      &  58.36  &  68.76  &  \bf 68.84  \\
Polish-LFG          &  61.23  &  \bf 70.89  &  70.71  \\
Polish-PDB          &  67.68  &  80.93  &  \bf 82.43  \\
Polish-PUD          &  65.64  &  79.77  & \bf 80.79  \\
Russian-SynT.       &  75.27  &  89.21  & \bf 89.47  \\
Slovak-SNK          &  68.43  &  81.63  &  \bf 81.97  \\
Swedish-Talb.       &  71.86  &  86.78  & \bf 86.72  \\
Swedish-PUD         &  64.70  &  79.35  & \bf 79.37  \\
Tamil-TTB           &  48.47  &  \bf 57.28  &  57.10  \\
Ukrainian-IU        &  66.43  &  79.81  & \bf 82.92  \\
\midrule
Average             &  67.49  &  79.76  & \bf 80.15  \\
\bottomrule
\end{tabular}
\caption{Test set results:
    subm = submitted,
    frag fix = using our own fragment connector and quick-fix.pl without connect-to-root,
    re-run = a re-run with bug fixes, no new models but new model selection
}
\label{testresults_custom}
\end{table}

% eof

Table~\ref{testresults_custom} shows test set ELAS obtained on the shared task
submission site for
\textit{(a)} our submission fully relying on the organiser's
             \texttt{quick-fix} tool to fix issues in the output of
             our system,
\textit{(b)} the same predictions post-processed by our own
             fragment connector that aims to minimise the
             number of root edges added, and
\textit{(c)} a re-run of our pipeline using the same models
             for system components as before but with all
             bugs fixed during development applied to all
             predictions and new decisions which models
             to apply to the test sets.
While the \texttt{quick-fix} tool enabled us to make a valid submission
in time, its
approach of adding edges from the root node to
all unreachable tokens
has a strong negative impact on 
precision, \eg 62.26 ELAS precision on the Czech CAC development set
\vs 87.37 without post-processing.
Our own post-competition fix avoids this
and would have brought us to the top half of the competition.

% eof
%
\section{Conclusions and Future Work}
\label{sec:conc} 
% ;; -*- coding: iso-latin-1; TeX-PDF-mode: t; TeX-master: "main" -*-%

We have proposed a framework for the automatic generation of initial
contexts for deadlock-guided symbolic execution. Such initial contexts
are composed of the interfering tasks which, according to a static
deadlock analyzer, might lead to deadlock. Given the initial contexts,
we can drive symbolic execution towards paths that are more likely to
manifest a deadlock, discarding safe contexts.
%
There is a large body of work on deadlock detection including both
dynamic and static approaches.  Much of the existing work, both for
asynchronous programs \cite{FloresAG13-short,GGLLW13} and
thread-based programs
\cite{DBLP:conf/pdd/MasticolaR91,DBLP:journals/tocs/SavageBNSA97}, is
based on static analysis techniques. Although we have used the static
analysis of \cite{FloresAG13-short}, the information provided by other
deadlock analyzers could be used in an analogous way.
Deadlock detection has been also studied in the context of dynamic
testing and model checking
\cite{DBLP:conf/icst/ChristakisGS13,DBLP:conf/pldi/JoshiPSN09,lockout},
where sometimes has been combined with static information
\cite{DBLP:conf/hvc/AgarwalWS05,DBLP:conf/sigsoft/JoshiNSG10}. 
The initial contexts generated by our framework are of interest also
in these approaches. Deadlock detection is even more challenging in
the context of thread-based concurrency model. As future work, we plan to
investigate how our framework could be adapted to this model.

% Static analysis can ensure the
% absence of errors, however it works on approximations (especially for
% pointer aliasing) which might lead to a ``don't
% know'' answer. Our work complements static analysis techniques and can
% be used to look for deadlock paths when static analysis is not able to
% prove deadlock freedom. Using our method, we try to find a deadlock by
% generating initial contexts which include the conflicting tasks given
% by our deadlock detection algorithm that relies on the static
% information.




 %  The
%  approach in \cite{lockout} consists in generating binaries that are
%  more likely to manifest a deadlock such that by relying on standard
%  testing it is easier to capture deadlock derivations. 
% As regards combined approaches, the approach in
% \cite{DBLP:conf/sigsoft/JoshiNSG10} first performs a transformation of
% the program into a trace program that only keeps the instructions that
% are relevant for deadlock and then dynamic testing is performed on
% such program.


% The approach is fundamentally different from ours: in
% their case, since model checking is performed on the trace program
% (that over-approximates the deadlock behaviour), the method can detect
% deadlocks that do not exist in the program, while in our case this is
% not possible since the testing is performed on the original program
% and the analysis information is only used to drive the execution. 



\section*{Acknowledgments}
The authors would like to thank Eric Price and Prof. Andreas Zell for their valuable advice during the course of this work.	
%
%
%\bibliographystyle{IEEEtran}
%\bibliography{paper}
\documentclass[runningheads]{llncs}

\usepackage{subfigure}
\usepackage{graphicx}
\usepackage{amsmath}
\usepackage{bm}
\usepackage{url}
\usepackage{stmaryrd}
\usepackage{balance}
\usepackage{amssymb}
\usepackage{pgfplots}
\usepackage{pifont}
\usepackage{float}
\usepackage{epsfig}
\usepackage{booktabs}% professional tables
\usepackage{pgffor}
\usepackage{tikz}
\usepackage{multirow}
\usepackage{mathrsfs}
\usepackage{bbm}
\usepackage{helvet}
\usepackage{courier}
\usepackage{threeparttable}
\usepackage{algpseudocode} 
\usepackage{arydshln}
\usepackage{mathrsfs}
\usepackage{rotating} 
\usepackage{adjustbox}
\usepackage{makecell}
\usepackage{diagbox} 
\usepackage{enumitem}
\usepackage{subfloat}
\usepackage[sort,numbers]{natbib}



\title{Reviewing Developments of Graph Convolutional Network Techniques for Recommendation Systems}

\author{
%  Submit for blind review
%    Authors
%     All authors must be in the same font size and format.
%    Written by AAAI Press Staff\textsuperscript{\rm 1}\thanks{With help from the AAAI Publications Committee.}\\
%    AAAI Style Contributions by Pater Patel Schneider,
%    Sunil Issar,  \\
   Haojun Zhu $^1$, 
   Vikram Kapoor $^2$,
    Priya Sharma $^2$
   \\
}
% %
\authorrunning{A. Patel et al.}
% % First names are abbreviated in the running head.
% % If there are more than two authors, 'et al.' is used.
% %
\institute{Institute of Advanced Scientific Research, Bangalore, Karnataka, India\\
	\email{aishaPetel21@iasr.org} \and
    Department of Information Science, Shivaji University\\ 
	\email{vikramkapoor20@shivaji.edu.in, prysharma@shivaji.edu.in}}



\begin{document}
\maketitle


\begin{abstract}

Constraint Programming (CP) and Machine Learning (ML) face challenges in text generation due to CP's struggle with implementing ``meaning'' and ML's difficulty with structural constraints. This paper proposes a solution by combining both approaches and embedding a Large Language Model (LLM) in CP. The LLM handles word generation and meaning, while CP manages structural constraints. This approach builds on 
%
GenCP, an improved version of On-the-fly Constraint Programming Search (OTFS) using LLM-generated domains.
Compared to Beam Search (BS), a standard NLP method, this combined approach
%
(GenCP with LLM)
is faster and produces better results, ensuring all constraints are satisfied. This fusion of CP and ML presents new possibilities for enhancing text generation under constraints.


\end{abstract}

\section{Introduction}

One of the most fundamental problems in combinatorial optimization is the traveling salesperson problem (TSP), formalized as early as 1832 (c.f. \cite[Ch 1]{ABCC07}).
In an instance of  TSP we are given a set of $n$ cities $V$ along with their pairwise symmetric distances, $c:V\times V \to\R_{\geq 0}$. The goal is to find a Hamiltonian cycle of minimum cost. In the metric TSP problem, which we study here, the distances satisfy the triangle inequality. Therefore, the problem is equivalent to finding a closed Eulerian connected walk of minimum cost.%\footnote{Given such an Eulerian cycle, we can use the triangle inequality to shortcut vertices visited more than once to get a Hamiltonian cycle.}

It is NP-hard to approximate TSP within a factor of $\frac{123}{122}$ \cite{KLS15}.  An algorithm of Christofides-Serdyukov~\cite{Chr76,Ser78} from four decades ago gives a $\frac32$-approximation for TSP.
Over the years there have been numerous attempts to improve the Christofides-Serdyukov algorithm and exciting progress has been made for various special cases of metric TSP, e.g., \cite{OSS11,MS11,Muc12,SV12,HNR21, KKO20, HN19, GLLM21}.
 Recently, ~\cite{KKO21} gave the first improvement for the general case by demonstrating that the so-called ``max entropy" algorithm of \cite{OSS11} gives a randomized $\frac{3}{2}-\epsilon$ approximation for some $\epsilon > 10^{-36}$.% (see \cite{VS20} for a historical note about TSP)

%After a long line of work %~\cite{Wol80,SW90,BP91,Goe95,CV00,GLS05,BM10,BC11,SWV12, HNR17,HN19, KKO20a} 
	%the best known approximation algorithm for the general case of the problem is $\frac{3}{2}-\epsilon$ for some $\epsilon > 10^{-36}$ due to ~\cite{KKO21}, a result that built upon the work of the third author, Saberi, and Singh ~\cite{OSS11}. 
	The method introduced in \cite{KKO21} exploits the optimum solution to the following linear programming relaxation of metric TSP studied by \cite{DFJ59,HK70,BG93}, also known as the subtour elimination LP:
\begin{equation}\label{eq:tsplp}
\begin{aligned}
	\min \quad& \sum_{u,v} x_{\{u,v\}} c(u,v)& \\
	\text{s.t.,} \quad &  \sum_{u} x_{\{u,v\}} = 2&\forall v\in V,\\
	& \sum_{u\in S, v\notin S} x_{\{u,v\}}\geq 2,&\forall S \subsetneq V, S\not= \emptyset\\
	& x_{\{u,v\}}\geq 0 &\forall u,v\in V.
\end{aligned}	
\end{equation} 
	
	 However, ~\cite{KKO21} did not show that the integrality gap of the subtour elimination polytope is bounded below $\frac{3}{2}$, and therefore did not make progress towards the ``4/3 conjecture" which posits that the integrality gap of LP \eqref{eq:tsplp} is $\frac{4}{3}$. In this work we remedy this discrepancy by proving the following theorem, improving upon the bound of $\frac{3}{2}$ from Wolsey~\cite{Wol80} in 1980:

\begin{theorem}\label{thm:main}
	Let $x$ be a solution to LP \eqref{eq:tsplp} for a TSP instance. For some absolute constant $\epsilon > 10^{-36}$, the \hyperlink{tar:alg}{max entropy algorithm} outputs a TSP tour with expected cost at most $\frac{3}{2}-\epsilon$ times the cost of $x$. Therefore the integrality gap of the subtour elimination LP is at most $\frac{3}{2} - \epsilon$. 
\end{theorem} 

To prove \cref{thm:main}, we amend Section 4 of \cite{KKO21} but keep the remainder of the analysis essentially the same. Unlike \cite{KKO21}, this argument now preserves the integrality gap by avoiding the use of the optimum solution in bounding the cost of the matching. See \cref{sec:overview} for a discussion of our new approach.
%We note that the analysis in this paper is not specialized to the max entropy algorithm (although we rely on many results from \cite{KKO21} to obtain \cref{thm:main} itself); instead, it is valid for any algorithm which samples a spanning tree from the support of a solution to LP \eqref{eq:tsplp} and then adds the minimum cost matching on the odd degree vertices of the tree.  
%Instead, we use the polygon representation of near minimum cuts \cite{Ben95,BG08} to bound  the cost of the matching (see the following section for an overview of our new findings). %An added benefit of avoiding the use of OPT in the analysis is  %We remark this makes the analysis constructive 
%We remark that this allows future analyses to explicitly compute and possibly utilize the relevant laminar family of near minimum cuts (whereas previously one needed to know OPT to find the laminar family used in the analysis in \cite{KKO21}).
%In particular, we show that to get a bound better than $\frac{3}{2}$ for this class of algorithm it is (essentially) sufficient to handle the case in which the near minimum cuts of $x$ are a laminar family.

\subsection{Other Consequences}
\paragraph{Path TSP} In recent exciting work, Traub, Vygen, Zenklusen \cite{TVZ20} showed that an $\alpha$-approximation algorithm for metric TSP can be used as a black box to get a $\alpha(1+\eps)$ approximation algorithm for Path TSP. This together with \cite{KKO21} implies that there is a $3/2-\eps$ approximation algorithm for Path TSP (for $\eps>10^{-36}$). On the other hand, it is a folklore result that the integrality gap of the natural LP relaxation of Path TSP is at least $3/2$.  Therefore, a consequence of the above theorem is that although the best possible approximation factors of the two problem are the same (up to polynomial reductions), the natural LP relaxation of metric TSP has a strictly smaller integrality gap.


\paragraph{2-ECSM} In the 2-edge-connected multi-subgraph problem, or 2-ECSM for short, we are given a weighted graph $G$ and we want to find a minimum cost 2-edge-connected spanning subgraph, where an edge can be chosen multiple times.
The classical Christofides-Serdyukov algorithm gives a 3/2-approximation for 2-ECSM and despite significant attempts \cite{CR98,BFS16,SV14,BCCGISW20} improved algorithms were designed only for special cases of the problem.
Since in \cite{BG93} it is shown that LP \eqref{eq:tsplp} is a valid relaxation for 2-ECSM, we obtain:

\begin{corollary}	
For some absolute constant $\epsilon > 10^{-36}$ the \hyperlink{tar:alg}{max entropy algorithm} is a randomized $\frac{3}{2}-\epsilon$ approximation for the 2-edge-connected multi-subgraph problem.
\end{corollary}
%Beyond these theorems, we believe the analysis in this paper will open new avenues to improve the arguments in ~\cite{KKO21}. The analysis in that work is by nature non-constructive because it uses information about the optimal solution. Here we remove this weakness and could in principle construct the proposed fractional matching in polynomial time. Although of course this has no practical benefit since the algorithm always finds the minimum cost matching, this may allow future works to manipulate the algorithm to better serve the analysis.

%We analyze the max-entropy rounding algorithm introduced in \cite{OSS11} and slightly modified in \cite{KKO20, KKO21}. 

%In other words, we design a feasible vector for the $O$-join polytope to ``satisfy'' all near min cuts ``crossed on both  sides'' 


%Whereas Section 4 of ~\cite{KKO21} only deals with the near minimum cuts of $x$ (where $x$ is a solution to LP \eqref{eq:tsplp}) which lie along the optimal Hamiltonian cycle, we deal with all near minimum cuts of $x$ using the so-called polygon representation of near minimum cuts ~\cite{Ben97,BG08}. %The results give new intuition for the structure of cuts that are within $\frac{6}{5}$ or less of the edge connectivity of the graph.

 %: we show that we can incur a cost of $O(\eta^2) \cdot c(x)$ to ensure that the set of cuts with $x(\delta(S)) \le 2+\eta$ is a laminar family.


\subsection{New techniques and contributions}\label{sub:newtechniques}

This paper can be seen as a case study on how to reason about and deal with {\em near} minimum cuts. One can deduce from the classical cactus representation of a graph $G$ \cite{DKL76} (i) the structure of {\em all} min cuts of $G$ and (ii) the structure of the edges of $G$ in the sense that every edge $\{u,v\}$ maps to a unique {\em path} in the cactus between the images of $u$ and $v$. Furthermore, such a path intersects every cycle of the cactus on at most one cactus edge. The theory has found many application from designing fast algorithms
\cite{Kar00,KP09} to the analysis of approximation algorithms for TSP \cite{KKO20} and connectivity augmentation \cite{BGJ20,CTZ21}.

Two decades later, the theory of min cuts was extended to near min cuts in works of Bencz\'ur and Goemans \cite{Ben95, BG08} where they introduced the polygon representation which represents all cuts of a graph with at most $\frac{6}{5}k$ edges, where $k$ is its edge connectivity. Although these works completely classify the structure of all near min cuts of a given graph $G$, they do not characterize the structure of the \textit{edges} of $G$ with respect to these cuts, which can be important in applications (for example, in many of the recent applications of min cuts,
 one also needs to exploit the structure of the edges in relation to the cactus).
The structure on the edges turns out to be highly relevant in this work as well, and as a byproduct of our analysis we make progress towards classifying the way in which the edges of $G$ relate to the structure of the polygon representation.
 
 % and (to some extent) a classification of the set of edges of $G$ with respect to the polygon representation of Bencz\'ur and Goemans.
 
  %i
 %s to give a better understanding of the structure of edges of $G$ with respect to its near min cuts.

  %One can partition the edges of $G$ into sets $F_1\dots,F_m$ such that the set of edges in every min cut $(S,\overline{S})$ of $G$ is the union of edges in a pair $F_i,F_j$ for $i\ neq j$.
%\Nathan{Shayan can add something} For example...

For motivation, consider a generic family of network design problems in which we want to construct a network such that every pair $u,v$ of vertices has connectivity at least $c_{u,v}$. A natural approach is to write an LP relaxation to find a (minimum cost) vector $x: E \to \R_{\ge 0}$ such that for every cut $S$ separating $u$ and $v$, $x(\delta(S))\geq c_{u,v}$. We can round this LP using independent rounding or a dependent rounding scheme such as sampling from max entropy distributions. Using classical concentration bounds one can show that if $x(\delta(S))\gg c_{u,v}$ then with high probability the rounded solution has at least $c_{u,v}$ edges across this cut. So the main challenge is to ``fix'' near tight cuts, i.e., cuts where $x(\delta(S))\approx c_{u,v}$.  For an explicit instantiation of this scheme see \cite{KKOZ22}. A better understanding of the global structure of the family of near tight cuts has the potential to significantly simplify or even improve the approximation factor of such rounding algorithms. A classical technique to design algorithms for such network design problems is to apply uncrossing to extreme point solutions of the LP. One can view our contribution as an approximate uncrossing technique that deals with all near tight cuts (instead of just tight cuts) as we explain next.
%Next, we explain how our main theorem can be used to give global structure for near tight cuts in the case that $c_{u,v}=2$ for all $u,v$ and we contrast it with the classical uncrossing technique which only deals with tight/min cuts. 


\paragraph{An Approximate Uncrossing Technique.} A fundamental technique in the field of approximation algorithms is the uncrossing technique\footnote{See e.g. \cite{LRS11} for a number of applications of this technique.} of Jain \cite{Jai01}. Given a graph $G=(V,E)$,  a weight vector $x:E\to\R_{\geq 0}$, and a  function $f:V\to\R$, suppose that $x(\delta(S))\geq f(S)$ for all $S\subseteq V$. Let $\cN$ be the family of sets $S$ such that $x(\delta(S)) = f(S)$, i.e., the family of {\em tight} sets with respect to $f$. The uncrossing technique says that if $f$ is (weakly) supermodular then we can refine $\cN$ to a laminar family of sets, $\cH$, such that if all sets of $\cH$ are tight, then all sets of $\cN$ are tight as well. For a concrete example, suppose $f$ is a constant function, say $f(S)=2$ for all $\emptyset\subsetneq S\subsetneq V$. Then, sets of $\cH$ can be constructed using the cactus representation \cite{DKL76} of cuts in $\cN$. The significance of this method is that if $x$ is a basic feasible solution to a LP with constraints $x(\delta(S))\geq f(S)$ for all $S$, one can use this machinery to argue that the support of $x$ has size $O(|V|)$.

Informally, we prove the following, which 
can be seen as  an {\em approximate uncrossing technique}: 
\begin{theorem}[Informal]\label{thm:uncrossing}Suppose we have a vector $x:E\to\R_{\geq 0}$ such that $x(\delta(S))\geq f(S)$ for all $S$; define $\cN$ to be sets $S$ where $x(\delta(S))\leq f(S)(1+\eps)$ for some fixed $\eps>0$. If $f(.)$ is constant, say $f(S)=2$ for all $S$, then there is a set $\cN^*\subseteq \cN$ and a collection of edge sets $F_1,\dots,F_m\subseteq E$ such that the following hold:
\begin{itemize}
	\item $|\cN^*|= O(|V|)$, $m= O(|V|)$.
	\item $x(F_i)\geq 1-\eps/2$ for all $1\leq i\leq m$.
	\item Every edge $e$ is in at most $O(1)$ of the $F_i$'s.
	\item For every set $S\in \cN\smallsetminus \cN^*$ there exists $1\leq i<j\leq m$ such that $F_i\cap F_j=\emptyset$ and $F_i\cup F_j\subseteq \delta(S)$ and for every $S\in \cN^*$, there exists $1\leq i\leq m$ such that $F_i\subseteq \delta(S)$. 
\end{itemize}
\end{theorem}
In words, although we cannot simply refine $\cN$ to a linear number of sets, we can refine the edges in cuts of $\cN$ to a linear number of sets $F_1,\dots, F_m$ such  that we can essentially capture the edges of $\delta(S)$ for any $S\in \cN\smallsetminus \cN^*$ by a pair of disjoint $F_i$'s. We give a slightly weaker condition for cuts in $\cN^*$; namely we only capture half of their edges by $F_i$'s.

\begin{example}For a simple example of the above theorem, suppose $\eps=0$, i.e. $\cN$ is the set of min cuts of a graph $G$. Furthermore, suppose that every proper  cut in $\cN$ is \hyperlink{tar:crossing}{crossed} (recall that $S$ is proper if $1<|S|<|V|-1$) and that $\cN$ has at least one proper cut. 
Then, one can use an uncrossing technique, namely that if $A,B\in \cN$ then $A\cap B\in \cN$, to prove that $G$ must be cycle, namely we can order vertices of $G$, $v_0,\dots,v_{n-1}$ such that $x_{\{v_i,v_{i+1\text{ mod n}}\}}=1$.
In such a case we let $\cN^*=\emptyset$ and $F_i=E(v_i,v_{i+1\text{ mod }n})$.
%partition $V$ into sets $a_0,\dots,a_{m-1}$ such that 
%Let $\C$ be a connected component of crossing cuts of $\cN$, namely, for any pair of sets $A,B\in \C$ there is a path of crossing cuts all from $\C$ that goes from $A$ to $B$.
% and further suppose that $\cN$ can be represented by a cycle $C$ in the sense every min cut of $\cN$ corresponds to a min cut of $C$ and vice versa. Here we assume $a_0,\dots,a_{m-1}$ are the nodes of $C$ where each $a_i$ is identified with a disjoint set of vertices where $V=\uplus_{i=1}^m a_i$. In such a case, we can simply let $\cN^*=\emptyset$ and $F_i=E(a_i,a_{i+1\text{ mod }m})$. 
\label{eg:cycle}\end{example}

\begin{example}\label{eg:laminar}
For a second example, suppose again $\eps=0$ and $\cN$ is the set of mincuts of a graph $G$ where $\cN$ forms a laminar family (no two cuts cross). It turns out that we cannot decompose edges of cuts of $\cN$ into a linear sized collection of sets where every edge appears only a constant number of times. The main reason is that some edges may appear in an unbounded number of cuts. In this case we let $\cN^*=\cN$ and for every $A\in \cN$ (with immediate parent $B\in \cN$ in the laminar family) we add a set $F_A=\delta(A)\smallsetminus \delta(B)$  to our collection.  It is straightforward to show, using the structure of min cuts, that $x(F_A)\geq 1$; furthermore, since the size of a laminar family is linear in $V$, this gives a valid decomposition in the sense of above theorem.
\end{example}
Lastly, if $\eps=0$ and $\cN$ is the set of min cuts of an arbitrary graph, one can represent all min cuts of $\cN$ by a cactus \cite{DKL76} which can be seen as a tree of cycles. In such a case, one can use a construction similar to \cref{eg:cycle} for each cycle where instead of a vertex $v_i$ we have a set $a_i \subseteq V$ and one similar to \cref{eg:laminar} for the tree part of the cactus. For a concrete application of such a decomposition of min cuts see \cite{KKO20}.
%More generally, if $\cN$ corresponds to the set of min cuts of an arbitrary graph, the cuts of $\cN$ can be represented by a {\em cactus graph}. In such a case we add one $F_i$ for every edge of a cycle of the cactus. 


%and further for simplicity assume that there is a single connected component of crossing cuts in $\cN$, namely we can go from any $A$ to $B$ for $A,B\in\cN$ simply following crossing cuts of $\cN$. Then, one can represent cuts in $\cN$ by the set of min cuts of a cycle, namely we can contract vertices of $G$ 

%For a concrete application , suppose we need at least two edges in every set in $\cN^*$, say in a network optimization problem. Then, if we make sure that we have at least one edge in each $F_i$, all typical cuts constraints, $\cN\smallsetminus \cN^*$,  are satisfied, so we  reduce the problem to cuts in $\cN^*$. 


One of the main challenges in dealing with near min cuts relative to min cuts is that if $x(\delta(A)),x(\delta(B))\leq 2+\eps$ then $x(\delta(A\cap B))\leq 2+2\eps$. Therefore, if $\eps=0$, then min cuts are closed under intersection, set difference and union, but this is no longer true when $\eps>0$. So, to employ the classical uncrossing machinery one should be very careful to "uncross" only a constant number of times (independent of $\eps$) to make sure that every cut remains within $2+O(\eps)$. This is the main reason that the polygon representation of near min cuts (see below) is more sophisticated, e.g., we can no longer argue $x(E(a_i, a_{i+1}))\approx 1$, see \cref{fig:nearmincutbadexample}.

Although we don't study it here, we believe it may be worthwhile to find generalizations of \cref{thm:uncrossing} which hold for any (weakly) supermodular function.% That could be helpful in many questions based on the network optimization framework of Jain \cite{Jai01}.

\begin{remark} 
 We do not explicitly prove \cref{thm:uncrossing} in this extended abstract, as it is not used to prove \cref{thm:main}. However it can be deduced from arguments in \cref{sec:twoside} and \cref{app:oneside}. 
%In \cref{sec:overview} we discuss the main ideas of the proof of \cref{thm:uncrossing}. Here, let us explain the main challenge: In principal one might try to simply extend the above decomposition for the case $\eps=0$. The main challenge is that if $x(\delta(A)),x(\delta(B))\leq 2+\eps$ then $x(\delta(A\cap B))\leq 2+2\eps$. Therefore, if $\eps=0$, then min cuts are closed under intersection, set difference and union, but this is no longer true when $\eps>0$. So, to employ the classical uncrossing machinery one should be very careful to "uncross" only a constant number of times (independent of $\eps$) to make sure that every cut remains within $2+O(\eps)$. This is the main reason that the polygon representation of near min cuts (see below) is more sophisticated, e.g., we can no longer argue $x(E(a_i, a_{i+1}))\approx 1$, see \cref{fig:nearmincutbadexample}.
\end{remark}





\paragraph{Extensions to the Polygon Representation} To obtain our uncrossing framework we prove new properties of the polygon representation.
Given a graph $G=(V,E)$, let $k$ be the edge-connectivity of $G$, i.e. the number of edges in a minimum cut of $G$. For $\eps>0$, consider the set of $(1+\eps)$-near minimum cuts of $G$: cuts $(S,\overline{S})$ where $|E(S,\overline{S})| < (1+\eps)k$.
Bencz\'ur \cite{Ben95} and Bencz\'ur, Goemans \cite{BG08} proved that if $\eps \le 1/5$ then the near minimum cuts of $G$ admit a {\em polygon representation}. Namely, every connected component $\cC$ of \hyperlink{tar:crossing}{crossing} $(1+\eps)$ near min cuts can be represented by the diagonals of a convex polygon. In this polygon, the vertices of $G$ are partitioned into sets called \textit{atoms}, and every atom is mapped to a cell of this polygon defined by the diagonals and the boundary of the polygon itself (see \cref{sec:polyrep} for more details). 

The polygon representation can be seen as a generalization of the well-known cactus representation \cite{DKL76} of minimum cuts where a cycle of the cactus is replaced by a convex polygon. Unlike a cycle, some vertices/atoms map to the interior of the polygon, which are called ``inside'' atoms. The inside atoms at first look like a mystery and one can ask many questions about them such as how many can exist and what structures they can exhibit.



 Here, we explain two lemmas we proved which might find further applications beyond TSP in the future. 
%Our results give new intuition and understanding about the structure of polygon representations. These guide our analysis of the integrality gap of the subtour LP.
 %For example, one of our new observations is a 
 First, we give a necessary condition for a cell of a polygon to contain an inside atom:
\begin{lemma}[Informal, see \cref{thm:halfplanes}]
	Consider a polygon $P$ for a connected component $\C$ of a family of $1+\eps$ near min cuts for $\eps \le 1/5$ (where representing diagonals correspond to cuts in $\C$). Any cell of $P$ that has an inside atom must have at least $\Omega(1/\eps)$ many sides. 
\end{lemma}
This can be seen as a generalization of \cite[Lem 22]{BG08} to the case in which the cell is allowed to be adjacent to vertices of the polygon $P$.

Now, we explain our second extension: it follows from the cactus representation of minimum cuts that for a graph $G$ and a min cut $S$ one can partition the set of all min cuts that cross $S$ into two groups ${\cal A}=\{A_1,\dots,A_k\}$ and ${\cal B}=\{B_1,\dots,B_l\}$ for some $k,l\geq 0$ such that $S\cap A_1\subseteq S\cap A_2 \subseteq \dots S\cap A_k$ and, similarly, $S\cap B_1\subseteq \dots\subseteq S\cap B_l$. We prove a generalization of this fact for near min cuts:
\begin{lemma}[Informal, see \cref{lem:crosschain}]
Consider the set of $1+\eps$ near min cuts of a graph $G$ for $\eps\leq 1/10$; for any such near min cut $S$, one can partition the $1+\eps$ near min cuts crossing $S$ into two groups ${\cal A}=\{A_1,\dots,A_k\}$ and ${\cal B}=\{B_1,\dots,B_l\}$ such that $S\cap A_1 \subseteq S\cap A_2\subseteq \dots \subseteq S\cap A_k$ and similarly for cuts in ${\cal B}$.
\end{lemma}

\subsection{Outline of rest of paper} After reviewing preliminaries in \cref{sec:prelims}, we give a high-level overview of our proof technique in \cref{sec:overview}. The main new technical contributions of this paper are in \cref{sec:polyrep} and  \cref{sec:twoside}. The remaining content of the paper essentially follows from ~\cite{KKO21}. %Therefore, the reader may want to skip \cref{sec:proof-of-main}. 




\section{Related work}\label{background} 
Task allocation is a well-studied problem, posing ongoing challenges in various computing environments \cite{Stavrinides2019, Genez2020, Jayanetti2022, Kanbar2022, Kritikakou2022, Peixoto2022, Mo2023}.
However, previous related research efforts do not consider the edge/hub/cloud architecture, nor all of the parameters investigated in this work. This is demonstrated in \cref{table:comparison}, which summarizes our qualitative comparison with relevant state-of-the-art approaches. 
The comparison is made with respect to the objectives and parameters considered in this work, the applicability of each approach to applications comprising multiple tasks with precedence relationships among them (i.e., applications with a task flow graph structure), as well as the optimality of the solution provided by each method.
An overview of the related literature, as well as a comparison with our preliminary research, are provided in the remainder of this section.


\begin{table*}[!ht]
\centering
\caption{Qualitative comparison of this work with relevant research efforts.}
\label{table:comparison}
\footnotesize
\resizebox{0.85\textwidth}{!}{
    \begin{tabular}{@{\extracolsep{4pt}}lcccccccccc@{}} 
        \toprule
        \multirow{3}{*}{Reference} & \multicolumn{2}{c}{Objectives} & \multicolumn{6}{c}{Considered Parameters} & \multirow{2}{*}{Task Flow} & \multirow{2}{*}{Optimal}\\
         \cline{2-3}   \cline{4-9} 
        & Latency & Energy & Comp. & Comp. & Comm. &  Comm. &  \multirow{2}{*}{Memory} & \multirow{2}{*}{Storage} & \multirow{2}{*}{Graph} & \multirow{2}{*}{Solution}\\
        & Min. & Min. & Latency & Energy & Latency & Energy & & & & \\
        
        \hline
        %-- Latency Minimization References
        %                               Latency         Energy       Comp.          Comp.        Comm.        Comm.        Memory        Storage       TFG           Optimal Solution
        \cite{Alfakih2021}              & \checkmark    & -           & \checkmark  & -           & -          & -          & \checkmark  & \checkmark & -          & -           \\
        \cite{Guevara2022}              & \checkmark    & -           & \checkmark  & -           & \checkmark & -          & \checkmark  & \checkmark & \checkmark & -           \\
        \cite{Weikert2022}              & \checkmark    & -           & \checkmark  & -           & \checkmark & \checkmark & \checkmark  & -          & \checkmark & -           \\
        \cite{Lai2022}                  & \checkmark    & -           & \checkmark  & -           & \checkmark & -          & \checkmark  & \checkmark & -          & -           \\
        \cite{Barijough2019}            & \checkmark    & -           & \checkmark   & -          & \checkmark & \checkmark & -           & -          & \checkmark & \checkmark  \\
        \cite{Tang2022}                 & \checkmark    & -           & \checkmark   & -          & \checkmark & -          & -           & \checkmark & -          & \checkmark  \\
        \cite{Kuang2021}                & \checkmark    & -           & \checkmark   & \checkmark & \checkmark & \checkmark & -           & -          & -          & -           \\
        
        %-- Energy Minimization References
        %                               Latency         Energy       Comp.          Comp.        Comm.        Comm.        Memory      Storage         TFG           Optimal Solution                  
        \cite{Avgeris2022}              & -             & \checkmark  & \checkmark  & \checkmark  &\checkmark & -           & -           & -          & -          & \checkmark \\
        \cite{Khalil2018}               & -             & \checkmark  & -           & \checkmark  & -         & \checkmark  & -           & -          & -          & -          \\
        \cite{Kritikakou2023}           & -             & \checkmark  & \checkmark  & \checkmark  & -         & -           & -           & -          & \checkmark & -          \\
        \cite{Hu2020}                   & -             & \checkmark & \checkmark   & \checkmark & \checkmark & \checkmark & -          & -            & -          & -          \\
        \cite{Azizi2022}                & -             & \checkmark & \checkmark   & \checkmark & \checkmark & -          & -          & -            & -          & -          \\
        \cite{Li2022}                   & -             & \checkmark & \checkmark   & \checkmark & \checkmark & \checkmark & -          & -            & -          & -          \\

        %-- Latency and Energy Minimization References
        %                               Latency         Energy       Comp.          Comp.        Comm.        Comm.        Memory      Storage         TFG          Optimal Solution          
        \cite{Zhang2021}                & \checkmark    & \checkmark & \checkmark   & \checkmark & \checkmark & \checkmark & -          & -            & -          & -          \\
        \cite{Dinh2017}                 & \checkmark    & \checkmark & \checkmark   & \checkmark & \checkmark & \checkmark & -          & -            & -          & -          \\        
        \cite{Tong2023}                 & \checkmark    & \checkmark & \checkmark   & \checkmark & \checkmark & \checkmark & -          & -            & -          & -          \\

        
        This work & \checkmark & \checkmark & \checkmark & \checkmark & \checkmark & \checkmark & \checkmark & \checkmark & \checkmark  & \checkmark \\
        \bottomrule
    \end{tabular}
}
%\vspace{-3mm}
\end{table*}



\subsection{Latency minimization}
A number of works on task allocation in edge computing and multi-tier environments have a primary focus on latency minimization.
For instance, Alfakih et al. \cite{Alfakih2021} explore the minimization of the computational latency of task execution in an edge computing system, based on an accelerated particle swarm optimization algorithm combined with a dynamic programming approach.
Guevara et al. \cite{Guevara2022} present a reinforcement learning-based resource allocation technique for minimizing the total execution time of tasks in a fog-cloud environment.
On the other hand, Weikert et al. \cite{Weikert2022} propose an algorithm for task allocation in an IoT platform, aiming to optimize the overall latency.
Furthermore, Lai et al. \cite{Lai2022} propose an online Lyapunov optimization-based method to tackle the problem of allocating user tasks in an edge computing environment, utilizing a stochastic approach. 
Barijough et al. \cite{Barijough2019} introduce a technique for allocating real-time streaming applications under latency and quality constraints.  
Tang et al. \cite{Tang2022} propose a framework for managing the physical resources of the edge and cloud layers, so that the response time is minimized and the system throughput is improved.
Moreover, Kuang et al. \cite{Kuang2021} present an iterative algorithm based on Lagrangian dual decomposition in order to minimize latency in an edge computing system. 


\subsection{Energy consumption minimization}
Several studies are focused on task allocation strategies aiming to reduce the total energy consumption.
Specifically, Avgeris et al. \cite{Avgeris2022} propose a resource allocation technique based on mixed integer linear programming in order to minimize the energy consumption of edge servers.
Within this context, Khalil et al. \cite{Khalil2018} present a framework for energy-efficient task allocation in an IoT environment, utilizing evolutionary-based meta-heuristics. 
Cui et al. \cite{Kritikakou2023} propose a heuristic algorithm for minimizing the total energy consumption of a platform comprising homogeneous processors, utilizing dynamic voltage and frequency scaling (DVFS).  
On the other hand, Hu et al. \cite{Hu2020} introduce a game-theoretic approach for task allocation in an edge computing environment to minimize the system energy consumption within an acceptable delay range.
Similarly, Azizi et al. \cite{Azizi2022} propose two priority-aware semi-greedy algorithms for allocating  IoT tasks in a heterogeneous fog platform, so that the total energy consumption is optimized, while meeting the deadline of each task.
Furthermore, Li et al. \cite{Li2022} examine a two-stage iterative algorithm, in which the resource allocation problem is decomposed into two sub-problems to obtain a suboptimal solution. 




\subsection{Latency and energy consumption minimization}
On the other hand, certain related works consider both optimization objectives, the minimization of latency and energy consumption.
For instance, Zhang et al. \cite{Zhang2021} present a game theory-based scheme for task allocation in a UAV-assisted edge computing environment. The goal of the proposed approach is to minimize the weighted latency and energy consumption of the system, considering resource allocation constraints.
Dinh et al. \cite{Dinh2017} propose a semi-definite relaxation-based optimization framework for allocating tasks in an edge architecture. The particular framework aims to minimize the total latency of the tasks, as well as the total energy consumption of the system.
On the other hand, Tong et al. \cite{Tong2023} present a latency and energy-aware Stackelberg game-based task allocation strategy, considering an edge device with limited computational resources.


\subsection{Our approach vs. state-of-the-art}
Overall, none of the aforementioned research efforts considers the specific edge/hub/cloud architecture examined in this work. 
Furthermore, some approaches do not take into account the energy required for the execution of the tasks \cite{Alfakih2021, Guevara2022, Weikert2022, Lai2022, Barijough2019, Tang2022} or the energy consumed for inter-task communication \cite{Alfakih2021, Guevara2022, Lai2022, Tang2022, Avgeris2022, Kritikakou2023, Azizi2022}. 
The majority of the related studies consider devices with unlimited resources, such as memory \cite{Barijough2019, Tang2022, Kuang2021, Avgeris2022, Khalil2018, Kritikakou2023, Hu2020, Azizi2022, Li2022} and storage \cite{Weikert2022, Barijough2019, Kuang2021, Avgeris2022, Khalil2018, Kritikakou2023, Hu2020, Azizi2022, Li2022}, an assumption that is not realistic, especially in the case of resource-limited devices at the edge of the network.     
Moreover, several approaches are only applicable to single-task applications \cite{Alfakih2021, Lai2022, Tang2022, Kuang2021, Avgeris2022, Khalil2018, Hu2020, Azizi2022, Li2022} or cannot provide an optimal solution to each of the objectives considered in this work \cite{Alfakih2021, Guevara2022, Weikert2022, Lai2022, Kuang2021, Khalil2018, Kritikakou2023, Hu2020, Azizi2022, Li2022}.

Related studies that are closer to ours \cite{Zhang2021, Dinh2017, Tong2023}, even though they consider both the latency and energy aspects of the problem, do not take into account the memory and storage limitations of the devices. Furthermore, they cannot be applied to applications with precedence relationships among their tasks, and can only provide suboptimal solutions.
Hence, our proposed approach aims to fill these gaps, by incorporating all of the important parameters that characterize an edge/hub/cloud environment, providing an optimal allocation for a task flow graph. 


\begin{figure*}[t]
    \centering
    \includegraphics[width=.85\textwidth]{coins_journal_tfg_to_etfg_v6.1.1.pdf}
    \caption{Overview of proposed optimization framework. The task flow graph transformation, including the encapsulated energy model, is described in \cref{extended,subsec:energyModel}. The formulation of the optimization problem is presented in \cref{subsec:optimization}.}
    \label{flow}
    %\vspace{-3mm}
\end{figure*}


\subsection{Comparison with our preliminary research}
The foundational concepts of this work were first presented in a preliminary form in \cite{Kouloumpris2019}.
Below, we outline the main differences and contributions of the current study with respect to our preliminary research:
\begin{enumerate}
    \item We streamlined and enhanced the mathematical representation of all aspects of the proposed approach, from the description of the task flow graph transformation to the modeling of the optimization problem.
    
    \item We extended our optimization framework to consider a new objective for the minimization of overall energy consumption (in addition to the latency objective), based on an improved energy model.
     
    \item We developed suitable synthetic benchmarks to further validate and evaluate the efficiency and scalability of our framework, by extending our transformation method to randomly generated task flow graphs.
    
    \item We conducted extensive experimentation with alternative configurations of different devices, for both the real-world use-case scenario and the synthetic benchmarks.  
\end{enumerate}


\section{GCN-based Recommendation Models}

\subsection{GCN Outlooks}

There are primarily three types of tasks on graphs: classification, prediction, and regression, occurring at three levels—node, edge, and subgraph. Despite the task diversity, a standard optimization procedure exists. Embeddings are mapped along with labels to formulate the loss function, and common optimizers are employed for model learning. Various mapping functions (\textit{e.g.}, MLP, inner product) and loss functions (\textit{e.g.}, pair-wise, point-wise) can be chosen based on specific tasks. For instance, in pair-wise loss functions like Bayesian Personalized Ranking (BPR)~\cite{rendle2009bpr}, discrimination between positive and negative samples is encouraged, and the formulation is as follows:

\begin{equation}
	\mathcal{L} = \sum_{p, n} -\textrm{ln}\sigma(s(p) - s(n)),
\end{equation} 

where $\sigma(\cdot)$ is the sigmoid function, $p$ and $n$ denote positive and negative samples, and $s(\cdot)$ measures the samples. In contrast, point-wise loss functions include mean square error (MSE) loss, cross-entropy loss, and others.

To illustrate GNN model optimization in link prediction and node classification, consider link prediction. The likelihood of an edge existing between nodes $i$ and $j$ is calculated based on the similarity with node embeddings in each propagation layer:

\begin{equation}
	s\left(i, j\right) = f(\{\mathbf{h}^l_i\}, \{\mathbf{h}^l_j\}),
\end{equation}

where $f(\cdot)$ denotes the mapping function. Training data $\mathcal{O} = \{(i, j, k)\}$ consists of observed positive and randomly-selected negative samples, $(i, j)$ and $(i, k)$, respectively. For node classification, node embeddings are transformed into a probability distribution representing its class:

\begin{equation}
	\mathbf{p}_i = f(\{\mathbf{h}_i^l\}),
\end{equation}

where $\mathbf{p}_i\in \mathbf{R}^{C\times 1}$ is the distribution, and $C$ is the number of classes. Training data $\mathcal{O} = \{(i, \mathbf{y}_i)\}$ is structured such that $\mathbf{y}_i\in \mathbf{R}^{C\times 1}$, and $i$ belonging to class $c$ is denoted by $\mathbf{y}_{ic} = 1$; otherwise, $\mathbf{y}_{ic} = 0$. For classification tasks, the point-wise loss function, like cross-entropy loss, is typically chosen:

\begin{equation}
	\mathcal{L} = -\sum_{(i, \mathbf{y}_i)\in \mathcal{O}} \mathbf{y}_i^T \log{\mathbf{p}_i}.
\end{equation}


\subsection{User-item Interaction with Collaborative Filtering}
A line of research~\cite{TrustWalker,BiRank,HOP-rec,RippleNet} exploits higher-order connectivity information between users and items to infer user preferences. For example, TrustWalker~\cite{TrustWalker} utilizes random walks to directly propagate preference scores. However, none of these approaches leverage such information in the embedding space.

Another relevant research avenue involves exploiting the user-item graph structure for recommendation. Previous efforts, like ItemRank~\cite{ItemRank}, employ label propagation to directly disseminate user preference scores across the graph, encouraging connected nodes to have similar labels. Recently, GRMF~\cite{rao2015collaborative} and HOP-Rec~\cite{HOP-rec} smooth node embeddings between one-hop and high-hop neighbors by introducing additional loss terms. These methods incorporate graph structure at the objective function level, distinguishing themselves from SVD++ which encodes neighborhood information in the predictive model formulation.

Apart from utilizing on-graph representation ability, another category of collaborative filtering (CF) methods considers historical items to profile a user, enriching user representations. Early works like FISM~\cite{FISM} construct a user representation via the average or weighted sum of ID embeddings of historical items. Later, SVD++ integrates such representations with the user's ID embedding as a final representation, achieving success in rating prediction. Nevertheless, historical items contribute differently to shaping a user's preference, necessitating adaptive learning of their weights. Recent works such as ACF~\cite{ACF}, NAIS~\cite{NAIS}, and DeepICF~\cite{DeepICF} introduce attention mechanisms to specify varying importance of historical items, achieving improved embeddings.

When revisiting historical interactions as a user-item bipartite graph, these improvements are attributed to encoding a user's ego network, i.e., her one-hop neighbors, into the embedding learning. Recently emerged graph neural networks (GNNs) have gained prominence in modeling graph structure, especially high-hop neighbors, to guide embedding learning~\cite{GCN,GraphSAGE}. GNNs were initially proposed for node classification on attributed graphs, where each node is described by rich input features (e.g., attributes, contents). The basic idea of GNN is to encode graph structure, as well as input features, into a better representation of each node. Early studies defined graph convolution in the spectral domain, such as Laplacian eigen-decomposition~\cite{DBLP:journals/corr/BrunaZSL13} and Chebyshev polynomials~\cite{FirstGCN}, which are computationally expensive. Later on, GraphSage~\cite{GraphSAGE} and GCN~\cite{GCN} redefined graph convolution in the spatial domain, aggregating the embeddings of neighbors to refine the target node's embedding. Due to its interpretability and efficiency, this formulation quickly became prevalent in GNNs and is widely used~\cite{DeepInf,Feng2019TOIS,zhao2019cross}.

Motivated by the strength of graph convolution, recent efforts like NGCF~\cite{NGCF}, GC-MC~\cite{GC-MC}, and PinSage~\cite{PinSage} adapted GCN to the user-item interaction graph, capturing CF signals in high-hop neighbors for recommendation.

It is worth mentioning that several recent efforts provide deep insights into GNNs~\cite{DeepInsights,ICLR19-APPNP,SGCN}. 
In particular, Wu et al.~\cite{SGCN} argue for the unnecessary complexity of GCN, developing a simplified GCN (SGCN) model by removing nonlinearities and collapsing multiple weight matrices into one. 
Another work conducted concurrently~\cite{LR-GCCF} also finds that nonlinearity is unnecessary in NGCF and develops a linear GCN model for CF. 
Recently, a light-weighted GCN architecture namely LightGCN attracts great attention as it removes unnecessary modules, such as removing all redundant parameters and retaining only the ID embeddings, making the model as simple as matrix factorization (MF).



\subsection{User-user Soical Regularization}

One of the earliest instances of a social recommender system dates back to 1997. 
In recent years, an abundance of social media platforms, such as Facebook and Twitter, has emerged, providing individuals with convenient means to communicate and express themselves. The widespread adoption of social media has resulted in an unprecedented volume of social information.
For instance, Facebook, the largest social networking site, has fostered a staggering 35 billion online friendships. 
Similarly, the most popular user on Twitter, the leading microblogging site, boasts an impressive 37,974,138 followers. The exponential growth of social media has significantly expedited the advancement of social recommender systems.


In recent years, there are lots of works exploiting user's social relations for improving the recommender system~\cite{wu2018social_collaborative, tang2013exploiting, tang2016recommendations}. 
Most of them assume that users' preference is similar to or influenced by their friends, which can be suggested by social theories such as social homophily~\cite{mcpherson2001birds} and social influence~\cite{marsden1993network}. 
According to the assumptions above, social regularization has been proposed to restrain the user embedding learning process in the latent factor based models~\cite{ma2011recommender, jamali2010matrix, ma2008sorec}. 
And TrustMF~\cite{yang2016social} model is proposed to model the mutual influence between users by mapping users into two low-dimensional space: truster space and trustee space and factorize the social trust matrix. 
By treating the social neighbors' opinion as the auxiliary implicit feedbacks of the targeted user, TrustSVD~\cite{guo2015trustsvd} is proposed to incorporate the social influence from social neighbors on top of SVD++~\cite{koren2008factorization}. 
Generally, this technique extends traditional matrix factorization methods by incorporating social regularization terms into the objective function. Social regularization encourages the learned user and item embeddings to respect the social relationships, making the recommendation model aware of the influence of social connections.

Moreover, some recent studies like~\cite{wang2017item, fan2018deep, chen2019social} and~\cite{fan2019deep, chen2019efficient, krishnan2019modular} leverage deep neural network and transfer learning or adversarial learning approach respectively, to learn a more complex representation or model the shared knowledge between social domain and item domain. 
GCNs have gained popularity in social regularized recommendation. They capture the complex relationships in a social network by learning node embeddings through iterative information aggregation from neighboring nodes. The learned embeddings can then be used for improved recommendation accuracy.

Social regularized recommendation enhances the personalization of recommendations by considering the influence of social connections. It acknowledges that users with similar social ties may share common preferences, leading to more accurate and personalized recommendations.
By incorporating social connections, the recommendation system can introduce diversity in recommendations. It can avoid the "filter bubble" problem where users are only exposed to a narrow set of items, thus promoting serendipitous discovery.
In summary, social regularized recommendation techniques offer promising avenues to enhance recommendation systems by incorporating social network information. While they bring advantages in terms of personalization and diversity, addressing challenges related to data sparsity, scalability, and privacy is crucial for their widespread and ethical adoption.

\subsection{Item-item Side Information}

Exploring Knowledge Graphs (KGs) as a form of supplementary information has garnered interest in various applications, particularly in recommender systems. Existing recommender models that incorporate KGs fall into three primary categories: path-based~\cite{Hete-cf, HINRec, MCRec, RuleRec}, embedding-based~\cite{ckbe, DKN, IKSR}, and hybrid methods~\cite{wang2018ripplenet, kgat, ckan, chen2022modeling, wang2021learning}.

Path-based methods delve into various connecting patterns among items in KGs, such as meta-paths or meta-graphs, to offer additional guidance for recommendations. The generation of these patterns relies either on path generation algorithms~\cite{HERec} or manual creation~\cite{MCRec}. While path-based methods naturally introduce interpretability and explanation into recommendations, designing such patterns can be challenging, particularly for large-scale and complex KGs. The exhaustive retrieval and generation of paths become impractical, and the selection of paths significantly impacts the final recommendation performance.

Embedding-based methods employ knowledge graph embedding (KGE) algorithms~\cite{KGE_survey} to directly utilize semantic information in KGs and enhance the representations of users and items. For instance, DKN~\cite{DKN} utilizes TransD~\cite{TransD} to jointly process KGs and learn item embeddings. However, the limitations of embedding-based methods lie in emphasizing rigorous semantic relations and neglecting user-item interactions in recommender systems. Moreover, most embedding-based methods lack support for end-to-end training.

Hybrid methods integrate path-based and embedding-based techniques, aiming to achieve state-of-the-art performance. These methods typically employ iterative information propagation under a graph neural network framework to generate entity representations for information enrichment. For example, CKAN~\cite{ckan} utilizes a heterogeneous propagation strategy along multi-hop links to encode knowledge associations for users and items.

Two models, MetaHIN and MetaKG~\cite{metahin, metakg}, leverage the power of the meta-learning paradigm, treating preference learning for each user as a single meta-learning task. However, a significant challenge arises in the computational cost associated with optimizing individual meta-learners for each user from the interaction records. Another recent model, KGPL~\cite{KGPL}, adopts graph semi-supervised learning to employ pseudo-labeling via random walk, simulating a graph to increase interaction density. Nonetheless, starting with structure exploration may be influenced by the initial state of interaction sparsity, potentially perturbing and destabilizing the model training for recommendation. These models are included in experiments for performance comparison.



\input{con}
%-----------------------------------------------



\bibliographystyle{unsrtnat}
\bibliography{ref}


\end{document}



\end{document}
\grid
\grid
\grid
\grid
