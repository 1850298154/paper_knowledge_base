% state of art

The main goal in this work is to develop a decentralized multi-robot target tracker and collision free motion planner in obstacle (static and dynamic) prone environments. The multi-agent obstacle avoidance problem has gained a lot of attention in recent years. Single agent obstacle avoidance, motion planning and control is well studied \cite{hoy2015algorithms,LIU2017317}. However, the multi-agent obstacle avoidance problem is more complex due to motion planning dependencies between different agents, and the poor computational scalability associated with the non-linear nature of these dependencies. 
In general, collision free trajectory generation for multi-agents can be classified into, (i) reactive and, (ii) optimization based approaches. 
Many reactive approaches are based on the concept of velocity obstacle (VO) \cite{fiorini1998motion}, whereas, optimization based approaches avoid obstacles by embedding collision constraints (like VO) within cost function or as hard constraints in optimization. 
%Among the recently proposed optimization approaches for multi-robot collision avoidance
Recently, a mixed integer quadratic program (MIQP) in the form of a centralized non-linear model predictive control (NMPC)\cite{fukushima2013model} has been proposed for dynamic obstacle avoidance, where feedback linearization is coupled with a variant of the branch and bound algorithm. 
However, this approach suffers with agent scale-up, since increase in binary variables of MIQP has an associated exponential complexity. 
In general, centralized optimization approaches \cite{raghunathan2004dynamic,kushleyev2013towards} are not computationally scalable with increase in number of agents. 
In \cite{derenick2007convex}, a decentralized convex optimization for multi-robot collision-free formation planning through static obstacles is studied.  This approach involves, triangular tessellation of the configuration space to convexify static obstacle avoidance constraints. Tessellated regions are used as nodes in a graph and the paths between the cells are determined to guarantee obstacle avoidance.
A decentralized NMPC \cite{shim2003decentralized} has been proposed for pursuit evasion and static obstacle avoidance with multiple aerial vehicles. Here the optimization is constrained by non-linear robot dynamics, resulting in non-convexity and thereby affecting the real-time NMPC performance. Additionally, a potential field function is also used as part of a weighted objective function for obstacle avoidance. A similar decentralized NMPC has been proposed for the task of multiple UAVs formation flight control \cite{chao2011collision}.  

Sequential convex programming (SCP) has been applied to solve the problems of multi-robot collision-free trajectory generation \cite{augugliaro2012generation}, trajectory optimization and target assignment \cite{morgan2015swarm} and formation payload transport \cite{alonso2015multi}. These methodologies principally approximate and convexify the non-convex obstacle avoidance constraints, and iteratively solve the resulting convex optimization problem until feasibility is attained. 
Due to this approximation, the obtained solutions are fast within a given time-horizon, albeit sub-optimal. 
SCPs have been very effective in generating real-time local motion plans with non-convex constraints. Recent work in multi-agent obstacle avoidance \cite{alonso2015collision} builds on the concept of reciprocal velocity obstacle \cite{van2011reciprocal}, where a local motion planner is proposed to characterize and optimally choose velocities that do not lead to a collision. 
The approach in \cite{alonso2015collision} convexifies the velocity obstacle (VO) constraint to guarantee local obstacle avoidance. 

In summary, due to non-linear dynamics constraints or obstacle avoidance dependencies, most multi-robot obstacle avoidance techniques are either, (i) centralized, (ii) non-convex, or (iii) locally optimal. Furthermore, some approaches only explore the solution space partially due to constraint linearization \cite{alonso2015collision}. Additionally, current NMPC target tracking approaches using potential fields do not provide guarantees on obstacle avoidance and are linked with the field local-minima problem. Recent reinforcement learning solutions \cite{chen2017decentralized, long2017deep} require large number of training scenarios to determine a policy and also do not guarantee obstacle avoidance.

Our work generates collision-free motion plans for each agent using a convex model-predicitive quadratic program in a decentralized manner. 
This approach guarantees obstacle avoidance and facilitates global convergence to a target surface. 
To the best of our knowledge, the method of using tangential potential field functions \cite{chang2003collision} to generate different reactive swarming behaviors including obstacle avoidance, is  most similar to our approach. However in \cite{chang2003collision}, the field local minima is persistent in the swarming behaviors. 

Unlike previous NMPC based target tracking approaches, which use potential fields in the objective, here  we use potential field forces as constraints in optimization. 
Specifically, the non-linear potential field functions are not directly used as constraints in optimization. Instead, potential field forces are pre-computed for a horizon using the horizon motion trajectory of neighboring agents and obstacles in the vicinity.  A feasible solution of the optimization program guarantees obstacle avoidance.  The pre-computed values are applied as external control input forces in the optimization process thereby preserving the overall convexity. 
%The following section discusses the system overview and problem statement in detail.


