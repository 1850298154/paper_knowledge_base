Heterogeneous Bayesian DDF enables a system of robots to cooperatively share information in a scalable way, such that every robot only reasons about its variables of interest. These can include locally observed variables or any other subset of the global set of variables. By reducing their local set and communicating only common variables of interest, robots can significantly reduce their communication and computation costs.

This paper rigorously develops the theory for conservative filtering in decentralized multi-robot dynamic applications. The analysis and methods are developed in a Bayesian framework using factor graphs, representing each robot's local pdf. When the notion of conservativeness can be defined, e.g., for Gaussian distributions, a practical algorithm is developed and it is shown to yield a conservative estimate for a multi-robot multi-target tracking application. The methods developed are based on only the local knowledge a robot has regarding the structure of the pdf. Thus, robots can independently reason about their local tasks without the need for some global graph knowledge, and opportunistically fuse data for cooperative robots to obtain conservative state estimates. 

The suggested conservative filtering algorithm bridges the gap found in \cite{dagan_exact_2021}, and allows, together with the FG-DDF framework \cite{dagan_factor_2021} to solve different robotic problems, such as target tracking \cite{dagan_exact_2021} and terrain height mapping \cite{schoenberg_distributed_2009}.

Future research will explore non-linear problems, hardware implementation and cyclic network topologies.  







