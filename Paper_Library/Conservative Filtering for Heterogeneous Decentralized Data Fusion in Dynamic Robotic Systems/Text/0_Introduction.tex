Consider a team of robots operating in an unknown environment and collaborating on a global joint inference task such as building a map of an unknown environment \cite{schoenberg_distributed_2009}, tracking a set of targets \cite{dagan_heterogeneous_2020} or cooperatively localizing themselves \cite{loefgren_scalable_2019, li_cooperative_2012, zhu_cooperative_2019}. To enable a scalable operation of such teams, a common practice is to distribute the global joint inference task into smaller, overlapping tasks, and allow robots to fuse only data relevant to their local inference task. Here, for example, relevant data might be an overlapping part of the map; subset of targets or a robot's immediate neighboring vehicles, respectively. Assuming each robot is an independent entity that needs to send and locally fuse messages, the examples above describe different instances of \emph{heterogeneous} fusion \cite{dagan_heterogeneous_2020}.     

There are several approaches for heterogeneous multi-robot cooperation, such as consensus \cite{ sun_scaling_2018}, or by using a server/cloud computing \cite{kia_server-assisted_2018, schmuck_ccm-slam_2019}. This paper explores a peer-to-peer Bayesian heterogeneous decentralized data fusion (DDF) approach, as it has advantages in robustness and flexibility. Briefly, in DDF robots communicate their current posterior probability distribution function (pdf) based on their locally available data \cite{chong_distributed_1983}.   

In previous work \cite{dagan_heterogeneous_2020}, \cite{dagan_exact_2021} we define peer-to-peer Bayesian heterogeneous DDF as the process of fusing two non-equal, but overlapping, pdfs.\footnote{To be precise, we differentiate between the more general problem of forming a posterior pdf over a set of random variables, which might describe a random state vector, and the Bayesian point estimate of those variables, given for example by finding the minimum squared error (MMSE) of that pdf.} One of the main challenges of Bayesian DDF and more acutely of heterogeneous DDF is accounting for dependencies in the data gathered by the different robots. Incorrect treatment of these dependencies might lead to `double counting' data more than once, which results in an overconfident estimate. For the case of homogeneous DDF, i.e. where the communicated and posterior pdfs describe the same set of random variables, it is common to either (i) explicitly track dependencies, which can be done by keeping a pedigree of the incoming data \cite{martin_distributed_2005} or by adding a channel filter (CF) \cite{grime_data_1994} for example; or (ii) implicitly account for dependencies by discounting data using covariance intersection (CI) \cite{julier_non-divergent_1997} or with the geometric mean density (GMD) \cite{bailey_conservative_2012}, for example. Notice that these methods are not application-specific and used to solve a variety of problems. On the other hand, in the case of heterogeneous fusion, solutions to account for the data dependencies have so far been rather application-specific. 

In \cite{cunningham_ddf-sam_2013} a multi-robot SLAM problem is solved, where dynamic robots share static variables (the landmarks describing the unknown map) and the solution is given for a smoothing problem, i.e. robot's local position variables are augmented. Thus this algorithm does not solve the filtering or fixed-lag smoothing problem, which are relevant for target tracking and cooperative localization applications for example. In cooperative localization (CL), robots take relative measurements to their neighboring robots, which directly couples the robots' states. To the best of our knowledge, in current approaches, robots either explicitly keep track of the dependencies of all \cite{luft_recursive_2016} or groups \cite{li_cooperative_2012} of vehicles in the network, or implicitly account for dependencies with CI to approximate the joint coupled covariance by its block diagonal elements and then marginalize out the other robot \cite{kia_cooperative_2016, zhu_cooperative_2019}. While these approaches work for cooperative localization, they might not adapt well for other cooperative applications, e.g., target tracking or SLAM. Thus there is a need to look at heterogeneous fusion for robotics as a wider, non-application specific problem, and develop the necessary analysis tools and algorithms. For that reason, previous work \cite{dagan_exact_2021} aimed at gaining insight into the heterogeneous fusion problem. 

In \cite{dagan_heterogeneous_2020, dagan_exact_2021} it is shown that in order for the fused posterior to be conservative (doesn't underestimate the uncertainty) over all the robot's variables, conditional independence has to be maintained between non-mutual variables, given the common variables, e.g., in multi-robot SLAM, non-mutual position variables are independent given common landmark variables. An algorithm, named the \emph{Heterogeneous State Channel Filter} (HS-CF) was developed, and shown to be conservative for static problems and for dynamic problems when a smoothing approach was applied. However, for a filtering scenario, the algorithm was slightly overconfident (non-conservative). It was hypothesized that marginalization of past variables (states) in the filtering stage resulted dependencies between non-mutual variables.     

The goal of this work is to understand the nature of the  dependencies between non-mutual variables resulting from filtering in heterogeneous DDF applications. To explicitly track and account for dependencies between common variables, the CF framework is used, which requires the communication graph between robots to be an undirected a-cyclic graph, so data can flow only in one route (i.e., can not `circle' back). 
Factor graphs, representing the local robot's pdf are analysed to gain insight into the structure of the problem and the cause of dependencies between non-mutual variables. While this paper is motivated by a search for rigorous understanding of the heterogeneous fusion problem, to develop new theory and thus focus less on a specific application, the presented approach can be readily applied to different robotic applications. 

The major contributions of this paper are twofold: theoretical and practical. From a theoretical point of view, factor graph analysis i) reveals `hidden variable dependency dynamics' for Bayesian heterogeneous DDF and ii) sheds light on the interplay between groups of common variables, showing that preserving conditional independence structure through the filtering stage is key to helping ensure conservativeness. From a practical point of view, i) an algorithm for conservative filtering that can be integrated with other fusion algorithms is developed and ii) multi-robot target tracking simulation verifies that the presented approach overcomes previous difficulties for dynamic variables.

The rest of the paper is organized as follows: Sec. \ref{Sec:background} gives the necessary background on heterogeneous DDF problems and factor graphs; Sec. \ref{Sec:prob_statement} defines the problem of conservative filtering in the context of heterogeneous DDF and Sec. \ref{Sec:cons_filtering} details the analysis, presents the suggested solution and algorithm. Simulation results, discussion and conclusions are given in Sec. \ref{Sec:sim} and Sec. \ref{Sec:conclusions}. 
