\section{Conclusions}
\label{sec:conclusion}

We studied the application of novelty search to the evolution of controllers for swarms of robots. The study was based on two distinct swarm robotics tasks: (i)~an aggregation task, and (ii)~a resource sharing task. The aggregation task was non-deceptive, as fitness-based evolution consistently managed to find high fitness solutions. Nevertheless, novelty search could achieve a similar performance in terms of fitness scores. We also showed how novelty search found several alternative and successful solutions to the task. Our analysis was based on Kohonen self-organising maps that allowed for the visualisation of the degree of exploration conducted in different regions of the behaviour space. 

The resource sharing task was a deceptive setup in which fitness-based evolution often got stuck in local maxima. Novelty search was unaffected by deception and displayed a significantly better performance than fitness-based evolution. In both tasks, novelty search was distinctively able to bootstrap the evolutionary process, it could consistently find behaviours with high fitness scores early in the evolutionary process, and it was able to find successful solutions with lower neural network complexity than the solutions evolved by fitness-based evolution.

To the best of our knowledge, our study is the first in which novelty search has been applied to evolutionary swarm robotics. Since behaviour characterisations are domain-dependent and a fundamental component in novelty search, we studied two different approaches to the design of characterisations: one based on the spatial inter-robot relationships sampled at regular intervals, and one based on two to four quantities that summarise the swarm behaviour throughout an entire experiment. None of the characterisations depends on the swarm size and they are thus scalable. In our experiments, we combined different behaviour characterisations and found that such combinations were only effective when the dimensions in the characterisation were directly related to the task. The opportunistic nature of artificial evolution will cause the search to first focus on the behaviour dimensions that are easier to explore. If such dimensions are not related with the task, the search will spend considerable effort in unfruitful regions of the behaviour space, reducing the effectiveness of novelty search.

We discovered that novelty search may not always find high-fitness solutions, especially when the behaviour space has dimensions that are unrelated or only weakly related to the objective. To overcome this issue, we studied two variants of novelty search, PMCNS and linear scalarization of novelty and fitness scores, in which novelty search operates in concert with fitness-based evolution. Our experimental results showed that PMCNS and linear scalarization are effective in guiding the exploration towards behavioural regions of higher fitness, without compromising the capacity of novelty search to find a broad diversity of solutions. The diversity of evolved solutions was not negatively affected in PMCNS and linear scalarization, when compared to pure novelty search.

Overall, our study shows that novelty search can be successfully applied to swarm robotics systems. Novelty search has several advantages over fitness-based evolution. One of the prominent advantages is that novelty search often produces a broad diversity of successful behaviours based on self-organisation. In the swarm robotics domain, diversity and self-organisation are particularly important because of the difficulties of designing such behaviours by hand. Moreover, novelty search is unaffected by deception, less prone to bootstrapping issues, and can evolve solutions with less complex neural networks. Novelty search is therefore a promising alternative for the artificial evolution of controllers for swarm robotics systems. 

\subsection{Future work}

Our results showed that care must be taken in the definition of macroscopic behaviour characterisations, especially when combining distinct behaviour features. It is necessary to ensure that (i)~each component of the characterisation is relevant to the task, and (ii)~that the components do not differ too much in terms of how easy or hard it is for novelty search to explore the corresponding dimensions of the behaviour space. It may be difficult to guarantee that macroscopic behaviour characterisations always meet these criteria, and in ongoing work we are therefore studying how individual weights can be assigned to components and modified during evolution, either manually or automatically. Such weights could ensure that particularly important dimensions are thoroughly explored.

While pure novelty search performed reasonably well in our tasks, our results show that the inclusion of a fitness component can further increase the performance of novelty search. As such, in future work we are going to elaborate on strategies for combining novelty and fitness. For instance, we are studying if the use of dynamic weights in linear scalarization can bring significant advantages. Such approach could allow the search to focus more on exploitation or more on exploration at different stages of the evolutionary process.

An alternative that we are currently studying is \emph{generic novelty measures} for swarm robotics. \citet{doncieux10} proposed a behaviour characterisation based on the mapping between the sensor input and the actuator output. We are studying how such characterisations could be defined in the collective robotics domain, and if they represent a viable alternative to domain-dependent novelty measures. Such generic measures are typically used in combination with fitness-based evolution, since the behaviour spaces created by generic measures tend to be vast and only weakly related with the objective.
