\section{Evolution of Resource Sharing Behaviours with Novelty Search}
\label{sec:sharing}

In this section, we study the application of novelty search to a more complex task in which a swarm of robots share a single resource. The swarm must coordinate in order to allow each member periodical access to a single battery charging station. In the task, the robots should first find the charging station, and then effectively share the station to ensure the survival of all the robots in the swarm. The charging station can only hold one robot at the time.

The problem of autonomous charging and resource conflict management is widely studied in the literature. \citet{cao97} identify resource conflicts as one of the fundamental challenges in the design of cooperative behaviours in multi-robot systems. Resource conflicts arise when a single indivisible resource (in our task, the charging station) is requested by multiple robots at the same time. The problem of sharing an energy charging station in particular is addressed in \citep{munoz02,robichaud02,kernbach11}.

\subsection{Experimental Setup}
\label{sec:energy_setup}

We use a swarm composed of 5 homogeneous robots. The robots are identical to the ones used in the aggregation experiments (see Section~\ref{sec:aggregation_setup}), except that they are equipped with additional sensors. Each robot has (i)~8 IR sensors evenly distributed around its chassis for the detection of obstacles (walls or other robots) up to a range of 10\,cm; (ii)~8 sensors dedicated to the detection of other robots up to a range of 25\,cm; (iii)~a ring of 8 sensors for the detection of the charging station up to a range of 1\,m; (iv)~a binary sensor that indicates whether or not the robot is currently being recharged; and (v)~a proprioceptive sensor that reads the current energy level of the robot. The sensors (i), (ii) and (iii) return the distance of the object that is being sensed. The experimental setup and the sensor ranges are depicted on Figure~\ref{fig:energy_setup}. The environment is a 3\,m by 3\,m square arena bounded by walls.

\begin{SCfigure}[1.3][b]
\centering 
	\includegraphics[width=0.35\textwidth]{pic/setup_energy.pdf}
\caption{The resource sharing task experimental setup. The grey circle in the centre is the charging station. The black filled circles are the robots (starting positions vary in each simulation). The solid circle around the top left robot represents the range of the obstacles sensor, the dashed circle represents the range of the robot sensor, and the fine dashed circle represents the range of the charging station sensor.}
\label{fig:energy_setup}
\end{SCfigure}

Each robot starts with full energy (1000 units), and the energy consumption increases linearly with the speed of the motors: a robot spends 5 units per second when motors are off, and 10 units of energy per second when both motors operate at their maximum speed. The charging station has the same diameter as a robot, and to recharge, the robots must remain static inside the charging station. The charging station is located in the centre of the arena, and charges a robot at a rate of 100 units of energy per second. Similarly to the aggregation experiments, each controller is evaluated in 10 simulations with random starting positions for the robots. Each simulation lasts for 2500 simulation steps. The highest scoring individual of each generation was post-evaluated in 100 simulations.

\subsection{Configuration of the Evolutionary Algorithms}
\label{sec:energy_functions}

We used the same parameter values for novelty search and the NEAT algorithm as in the aggregation experiments (see Section~\ref{sec:aggregation_algsetup}). However, we continued each evolutionary run until the 400th generation, since the resource sharing task proved to be more challenging than the aggregation task.

The fitness function $F_{s}$ used to evaluate the controllers is a linear combination of the number of robots alive at the end of the simulation (henceforth referred to as \emph{survivors}) and the average energy of the robots throughout the entire simulation:
\begin{equation}
F_{s} = 0.9\cdot\frac{|a_{T}|}{N}+0.1\cdot \sum_{t=1}^{T}\sum_{i=1}^{N}\frac{e_{i,t}}{TNe_{max}}\enspace ,
\end{equation}
where $|a_{T}|$ is the number of survivors, $T$ is the length of the simulation, $N$ is the number of robots in the swarm, $e_{i,t}$ is the energy of the robot $i$ at time $t$, and $e_{max}$ is the energy capacity of a robot. The second term of $F_s$ concerning the average energy was included to differentiate solutions where the same number of robots survive. Without the second term, there would be no fitness gradient, since it is unlikely that the initial population contains solutions where at least one robot survives until the end.\footnote{We empirically determined the probability of randomly generating a solution where at least one robot survives until the end (in any of the 10 trials) to be approximately 1\%.}

We experimented with behaviour characterisations of a different nature in the resource sharing experiments, compared to the characterisations used in the aggregation experiments. While in the aggregation task, we used spatial relationships between the robots sampled every 5\,s during the simulation, in the resource sharing experiments, we use only quantities that characterise a simulation as a whole. We evaluated two behaviour characterisations:

\begin{description}
\item[$\mathbf{b_{simple}}$:] The first characterisation is closely related to the fitness function, and it is composed of two values (normalised to the interval $[0,1]$): (i)~the number of robots that survive till the end of the simulation; and (ii)~the average energy of all alive robots throughout the simulation. $\mathbf{b_{simple}}$ is given by:
\begin{equation} \label{eq:bsimple_energy}
\mathbf{b_{simple}} = \left ( \frac{|a_{T}|}{N} \;,\; \sum_{t=1}^{A} \sum_{i \in a_{t}}^{ }\frac{e_{i,t}}{A\cdot |a_{t}| \cdot e_{max}} \right ) \enspace ,
\end{equation}
where $A$ is the number of time steps in which there was at least one robot alive and $a_{t}$ is the set of robots alive at time $t$.

\item[$\mathbf{b_{extra}}$:] The second behaviour characterisation is an extension of $\mathbf{b_{simple}}$. Two more features, which are not directly related to the fitness function, were added to the characterisation: (i)~the average speed of the alive robots throughout the simulation; and (ii)~the average distance of the alive robots to the charging station. The movement of a robot in a given instant is determined by the average wheel speed at that instant. The two additional features are also normalised to the interval $[0,1]$. $\mathbf{b_{extra}}$ is defined as:
\begin{equation} \label{eq:bextra_energy}
\mathbf{b_{extra}} = \left (\mathbf{b_{simple}} \;,\; \sum_{t=1}^{A} \sum_{i \in a_{t}}^{ }\frac{s_{i,t}}{A\cdot |a_{t}| \cdot s_{max}} \;,\; \sum_{t=1}^{A} \sum_{i \in a_{t}}^{ }\frac{d_{i,t}}{A\cdot |a_{t}| \cdot d_{max}} \right ) \enspace ,
\end{equation}
where $s_{i,t}$ and $d_{i,t}$ are the speed of the robot $i$ and its distance to the charging station, respectively, at time $t$. $s_{max}$ is the maximum speed of a robot, and $d_{max}$ is half the length of the diagonal of the arena.
\end{description}

\subsection{Performance Comparison}

Figure~\ref{fig:energy_fitness} depicts the highest fitness scores achieved by novelty search, fitness-based evolution, and random evolution. Random evolution only achieves very low fitness scores ($<0.2$). In fitness-based evolution, the distribution of the highest fitness scores achieved is characteristically wide. Fitness-based evolution achieved close to the maximum fitness score in 10/30 of the evolutionary runs, but failed to evolve any viable solution (where at least one robot consistently survives) in 10/30 of the runs. Bootstrapping proved difficult in fitness-based evolution, as most runs got stuck in low regions of the fitness landscape for a large number of generations.

\begin{figure}[b]
	\centering
	\includegraphics[width=1\textwidth]{pic/energy_performance.pdf}
\caption{Highest fitness scores achieved in the resource sharing task with novelty search with $\mathbf{b_{simple}}$ and $\mathbf{b_{extra}}$ (\emph{NS-simple}, \emph{NS-extra}), fitness-based evolution (\emph{Fit}), and random evolution (\emph{Random}). Left: average fitness value of the highest scoring individual found so far at each generation. The values are averaged over 30 independent evolutionary runs for each method. Right: box-plots of the highest fitness score found in each evolutionary run, for each method. The whiskers extend to the lowest and the highest data point within 1.5 times the interquartile range. Outliers are indicated by circles. The maximum fitness score in practice is about 0.98, which corresponds to all robots surviving until the end of the experiment, while maintaining high levels of energy.}
\label{fig:energy_fitness}
\end{figure}

Novelty search, on the other hand, did not get stuck in local maxima. The result indicates that novelty search was unaffected by deception, and was capable of bootstrapping the evolutionary process. Novelty search with $\mathbf{b_{simple}}$ consistently achieved high fitness scores, with 4--5 robots surviving in the best solutions. The fitness scores achieved with $\mathbf{b_{simple}}$ are significantly higher than those achieved by fitness-based evolution (Mann--Whitney U test, $p$-value $<$ 0.01). With $\mathbf{b_{extra}}$, novelty search failed to match the performance of $\mathbf{b_{simple}}$, and the fitness scores achieved are significantly lower ($p$-value $<$ 0.01). Within 400 generations, \emph{NS-simple} could consistently achieve fitness scores close to the maximum value. However, it should be noted that \emph{NS-extra} and \emph{Fit} do not appear stable at that point, and if more generations were allowed, they might still improve. The differences between the behaviour characterisations and the evolved behaviours are discussed below.

\subsection{Behavioural Diversity}

Through visual inspection of the solutions that achieved the highest fitness scores (above 0.95), we found that all evolutionary methods produced solutions that display similar behaviours. In the successful behaviours, the robots always start by searching for the charging station. Depending on the solution, the robots move in straight lines, in large circles, or in spirals, until the station has been located. The second part of the successful behaviours concerns the coordination of access to the charging station. We observed three different coordination behaviours (see Figure~\ref{fig:behaviours_energy}):

\begin{description}
\item[\textbf{Charge and go away:}] If a robot in the charging station detects another robot approaching, it leaves the station to let the approaching robot recharge. The leaving robot performs what resembles a random walk in the arena and eventually returns to the station to recharge.
\item[\textbf{Charge and surround:}] If a robot in the charging station detects another robot approaching, it leaves the station to let the approaching robot recharge.  The leaving robot begins to circle the charging station, and approaches the station again when its energy level is below a solution-specific threshold.
\item[\textbf{Charge and wait:}] Once a robot is in the charging station, it continues to occupy the station until its energy level is above a solution-specific threshold. When the robot leaves, it moves only a short distance away from the station. Then, the robot remains almost static until its energy level is below a solution-specific threshold, at which point the robot tries to recharge again.
\end{description}

\begin{figure}
\centering 
	\includegraphics[width=1\textwidth]{pic/behaviours_rs_new.pdf}
\caption{The patterns of behaviour corresponding to the highest scoring solutions found by novelty search and fitness-based evolution in the resource sharing task. Each line represents the trajectory of the robot throughout the simulation. The circles indicate initial positions and squares indicate final positions. Videos of the behaviours are available as online supplemental material.}
\label{fig:behaviours_energy}
\end{figure}

\subsubsection{Fitness-based evolution}

As mentioned above, fitness-based evolution did not consistently evolve solutions with high fitness scores (see Figure~\ref{fig:energy_fitness}). In fact, of the 30 runs conducted, 10 runs never evolved solutions with a fitness score much higher than the initial randomly generated population. We analysed the controllers evolved in the runs that only achieved low fitness scores and identified two behaviour patterns:
\begin{itemize}
\item The robots move slowly in small circles. When one of the robots detects the charging station, it moves towards the station, and occupies it till the end of the simulation. As a result, the rest of the swarm dies.
\item All the robots remain almost static from the beginning of the simulation until they run out of energy.
\end{itemize}
Both behaviours represent local maxima in the fitness landscape. In the first case, evolution converges to solutions where only one robot survives at the expense of the rest of the swarm. In the second case, the evolutionary process starts to converge to controllers that reduce the wheel speed to conserve energy. Conserving energy causes the robots to survive longer, thus slightly increasing the fitness score of the controller. However, reducing the wheel speed also decreases the chance that a robot will find the charging station. Once an evolutionary process starts to converge to a local maximum based on energy conservation, it can take many generations to escape from that maximum, or evolution may not escape at all.

\subsubsection{Simple behaviour characterisation}

The highest scoring solutions evolved by novelty search follow the same behaviour patterns as the highest scoring solutions evolved by fitness-based evolution. However, an analysis of the explored behaviour space reveals that novelty search could in fact evolve a greater diversity of solutions (see Figure~\ref{fig:space_simple}). The greater diversity comes from variations of the same behaviour patterns. For instance, if the coordination behaviours use different energy thresholds to trigger entering and leaving the charging station, it will result in different average levels of energy. In such cases, the behaviour patterns may appear similar through visual inspection, but they have distinct behaviour characterisations.

\begin{figure}
\centering
\includegraphics[width=\textwidth]{pic/simple_density}
\caption{Behaviour space exploration with fitness-based evolution (\emph{Fit}) and novelty search with $\mathbf{b_{simple}}$ (\emph{NS-simple}), in all evolutionary runs. The $x$-axis is the average energy level of the robots still alive, the $y$-axis is the number of survivors. Each individual is mapped according to its characterisation. Darker zones indicate that there were more individuals evolved with the behaviour of that zone.}
\label{fig:space_simple}
\end{figure}

\subsubsection{Extra behaviour characterisation}

We adapted the visualisation technique based on Kohonen maps to the 4-dimensional behaviour space created by the $\mathbf{b_{extra}}$ characterisation to analyse the degree of exploration of different behaviour regions. The results in Figure~\ref{fig:extra_som} show how novelty search with the $\mathbf{b_{extra}}$ characterisation explored the behaviour space --- a notable variety of combinations of average energy, movement, and distance to the charging station. However, the behaviour dimension related to the number of surviving robots was the least explored dimension of the behaviour space. In almost all the explored regions, the number of surviving robots was either zero or one. Only a single behaviour region included behaviours in which the whole swarm often survives (highlighted in Figure~\ref{fig:extra_som}), and that region was one of the least explored.

\begin{SCfigure}[1.2]
\centering
\includegraphics[width=0.5\textwidth]{pic/som_nov_extra.pdf}
\caption{Kohonen map representing the explored behaviour space in novelty search with $\mathbf{b_{extra}}$. Each circle represents a behaviour pattern, depicted by the 4 slices of different colour. Each slice represents one component of the behaviour characterisation vector -- the bigger the slice, the higher the value of that component. The darker the background of a circle, the more individuals were evolved with the corresponding behaviour.}
\label{fig:extra_som}
\end{SCfigure}

The relatively low performance of novelty search is caused by the significantly larger behaviour space and the lack of correlation between the behaviour features. Novelty search can freely explore dimensions such as average movement and distance to charging station without any robots surviving. Furthermore, both of these dimensions are intuitively easier to explore than the dimension concerning the number of surviving robots. As such, the opportunistic nature of evolution causes the search to focus less on the surviving robots dimension, often inhibiting the evolution from finding successful solutions to the task. This is in contrast with the correlated features used in the $\mathbf{b_{cmcl}}$ characterisation from the aggregation experiments, in which novelty search performed well despite the enlarged behaviour space (see Section~\ref{sec:combined_aggregation}).

The two-dimensional $\mathbf{b_{simple}}$ characterisation restricts novelty search to explore the two dimensions directly related to the fitness function. Similar solutions in which robots move at different speeds are, for instance, conflated when the $\mathbf{b_{simple}}$ characterisation is used, whereas they are considered different (and potentially novel) when the $\mathbf{b_{extra}}$ characterisation is used. As a consequence, novelty search with the $\mathbf{b_{simple}}$ characterisation focused exploration on the dimensions directly related to the fitness function and had significantly more success in finding solutions with high fitness scores. On the other hand, conflation can be prejudicial to the diversity of solutions that is evolved. For instance, analysing the best solutions evolved with each characterisation, we verified that the behaviour pattern  \emph{Charge and wait} was less common with $\mathbf{b_{simple}}$ when compared to $\mathbf{b_{extra}}$. This behaviour is intuitively one of the most interesting solutions, since it takes advantage of the variable energy spending to extend the life time of the robots. However, it was conflated with $\mathbf{b_{simple}}$, since this characterisation does not consider the speed of the robots.

Our results suggest that caution must be displayed when behaviour characterisations are defined. Dimensions that novelty search may opportunistically explore at the cost of other dimensions that are more crucial for the task should not be included. Alternatively, a number of methods have been proposed that aim to guide exploration in novelty search towards high fitness solutions. In Section~\ref{sec:combination}, we study the application of two such methods to the resource sharing task.

\subsection{Neural Network Complexity}

The experiments with the aggregation task showed that novelty search found successful solutions with less complex neural networks than fitness-based evolution. To determine if the same is true for the resource sharing task, we analysed the complexity of the solutions evolved for this task. Table~\ref{tab:complexity_rs} shows a comparison between the complexity of the solutions found by novelty search with $\mathbf{b_{simple}}$ and fitness-based evolution. The results show that there was a considerable difference between the complexity of the solutions evolved by the two methods. At fitness levels above 0.2, the simpler solutions evolved by novelty search were of significantly lower complexity than the solutions evolved by fitness-based evolution (Mann--Whitney U test, $p$-value $<$ 0.05). Note that the complexity of the networks in the initial populations is 107 (29 neurons and 78 links).

\begin{table}
\caption{Comparison of the least complex solutions evolved by fitness-based evolution (Fit) and novelty search with $\mathbf{b_{simple}}$ (NS). The \emph{Complexity} columns lists the network complexity (sum of number of neurons and number of connections) of the least complex individual with a fitness score above a certain \emph{Fitness level}. The \emph{Generation} column lists the average generation in which the least complex individual was generated. The values are averages of 30 evolutionary runs for each method.}
\centering
\begin{tabular}{r r r r r r r}
\toprule
\multicolumn{ 1}{c}{} & \multicolumn{ 2}{c}{Generation} & \multicolumn{ 2}{c}{Complexity} \\ \cmidrule{ 2- 5}
\multicolumn{ 1}{r}{Fitness level} & \multicolumn{1}{r}{Fit} & \multicolumn{1}{r}{NS} & \multicolumn{1}{r}{Fit} & \multicolumn{1}{r}{NS} \\ \midrule
0.20 &  65 & 21  &  114.90 &  110.50 \\
0.40 &  150 & 69  & 127.30 & 118.40 \\
0.60 &  192 & 135 &  134.20 & 125.90 \\
0.80 & 213 & 192 & 136.90 & 132.40 \\
0.90 &  268 & 239 & 145.70 & 133.90 \\
\bottomrule
\end{tabular}
\label{tab:complexity_rs}
\end{table}
