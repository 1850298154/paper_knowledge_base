%%%%%%%%%%%%%%%%%%%%%%% file template.tex %%%%%%%%%%%%%%%%%%%%%%%%%
%
% This is a general template file for the LaTeX package SVJour3
% for Springer journals.          Springer Heidelberg 2010/09/16
%
% Copy it to a new file with a new name and use it as the basis
% for your article. Delete % signs as needed.
%
% This template includes a few options for different layouts and
% content for various journals. Please consult a previous issue of
% your journal as needed.
%
%%%%%%%%%%%%%%%%%%%%%%%%%%%%%%%%%%%%%%%%%%%%%%%%%%%%%%%%%%%%%%%%%%%
%
%
\RequirePackage{fix-cm}
%
%\documentclass{svjour3}                     % onecolumn (standard format)
\documentclass[smallcondensed]{svjour3}     % onecolumn (ditto)
%\documentclass[smallextended]{svjour3}       % onecolumn (second format)
%\documentclass[twocolumn]{svjour3}          % twocolumn
%
\smartqed  % flush right qed marks, e.g. at end of proof
%
\usepackage{graphicx}
\usepackage{verbatim}
\usepackage{afterpage}
%\usepackage[textwidth=55pt,textsize=small]{todonotes}
\usepackage{url}
%\usepackage{hyperref}
\usepackage{amsmath}
%\usepackage{xfrac}
\usepackage{enumerate}
%\usepackage{verbatim}
\usepackage[round]{natbib}
\usepackage{inputenc}
\usepackage{sidecap}
\usepackage{booktabs}
\usepackage{todonotes}
\usepackage{xfrac}
\usepackage{placeins}

%\usepackage{mathptmx}      % use Times fonts if available on your TeX system

\journalname{Swarm Intelligence}
%
\begin{document}
\sloppy

\title{Evolution of Swarm Robotics Systems with \\Novelty Search}

\author{Jorge Gomes \and Paulo Urbano \and \\Anders Lyhne Christensen}

\institute{J. Gomes \at
              LabMAg, FCUL \& Instituto de Telecomunica\c{c}\~oes, Lisbon, Portugal \\
              \email{jgomes@di.fc.ul.pt}
           \and
           P. Urbano \at
           	  LabMAg, FCUL, Lisbon, Portugal \\
              \email{pub@di.fc.ul.pt}
           \and
           A. L. Christensen \at
              Instituto Universit\'ario de Lisboa (ISCTE) \& Instituto de Telecomunica\c{c}\~oes, Lisbon, Portugal\\
           	  \email{anders.christensen@iscte.pt}
}

\date{Received: 2012 / Accepted: April 2013}

\maketitle

\begin{abstract}

Novelty search is a recent artificial evolution technique that challenges traditional evolutionary approaches. In novelty search, solutions are rewarded based on their novelty, rather than their quality with respect to a predefined objective. The lack of a predefined objective precludes premature convergence caused by a deceptive fitness function. In this paper, we apply novelty search combined with NEAT to the evolution of neural controllers for homogeneous swarms of robots. Our empirical study is conducted in simulation, and we use a common swarm robotics task -- aggregation, and a more challenging task -- sharing of an energy recharging station. Our results show that novelty search is unaffected by deception, is notably effective in bootstrapping the evolution, can find solutions with lower complexity than fitness-based evolution, and can find a broad diversity of solutions for the same task. Even in non-deceptive setups, novelty search achieves solution qualities similar to those obtained in traditional 
fitness-based evolution. Our study also encompasses variants of novelty search that work in concert with fitness-based evolution to combine the exploratory character of novelty search with the exploitatory character of objective-based evolution. We show that these variants can further improve the performance of novelty search. Overall, our study shows that novelty search is a promising alternative for the evolution of controllers for robotic swarms.

\keywords{Evolutionary robotics \and neuroevolution \and swarm robotics \and novelty search \and NEAT \and behavioural diversity \and deception}
\end{abstract}

% Content

\section{Introduction}

One of the most fundamental problems in combinatorial optimization is the traveling salesperson problem (TSP), formalized as early as 1832 (c.f. \cite[Ch 1]{ABCC07}).
In an instance of  TSP we are given a set of $n$ cities $V$ along with their pairwise symmetric distances, $c:V\times V \to\R_{\geq 0}$. The goal is to find a Hamiltonian cycle of minimum cost. In the metric TSP problem, which we study here, the distances satisfy the triangle inequality. Therefore, the problem is equivalent to finding a closed Eulerian connected walk of minimum cost.%\footnote{Given such an Eulerian cycle, we can use the triangle inequality to shortcut vertices visited more than once to get a Hamiltonian cycle.}

It is NP-hard to approximate TSP within a factor of $\frac{123}{122}$ \cite{KLS15}.  An algorithm of Christofides-Serdyukov~\cite{Chr76,Ser78} from four decades ago gives a $\frac32$-approximation for TSP.
Over the years there have been numerous attempts to improve the Christofides-Serdyukov algorithm and exciting progress has been made for various special cases of metric TSP, e.g., \cite{OSS11,MS11,Muc12,SV12,HNR21, KKO20, HN19, GLLM21}.
 Recently, ~\cite{KKO21} gave the first improvement for the general case by demonstrating that the so-called ``max entropy" algorithm of \cite{OSS11} gives a randomized $\frac{3}{2}-\epsilon$ approximation for some $\epsilon > 10^{-36}$.% (see \cite{VS20} for a historical note about TSP)

%After a long line of work %~\cite{Wol80,SW90,BP91,Goe95,CV00,GLS05,BM10,BC11,SWV12, HNR17,HN19, KKO20a} 
	%the best known approximation algorithm for the general case of the problem is $\frac{3}{2}-\epsilon$ for some $\epsilon > 10^{-36}$ due to ~\cite{KKO21}, a result that built upon the work of the third author, Saberi, and Singh ~\cite{OSS11}. 
	The method introduced in \cite{KKO21} exploits the optimum solution to the following linear programming relaxation of metric TSP studied by \cite{DFJ59,HK70,BG93}, also known as the subtour elimination LP:
\begin{equation}\label{eq:tsplp}
\begin{aligned}
	\min \quad& \sum_{u,v} x_{\{u,v\}} c(u,v)& \\
	\text{s.t.,} \quad &  \sum_{u} x_{\{u,v\}} = 2&\forall v\in V,\\
	& \sum_{u\in S, v\notin S} x_{\{u,v\}}\geq 2,&\forall S \subsetneq V, S\not= \emptyset\\
	& x_{\{u,v\}}\geq 0 &\forall u,v\in V.
\end{aligned}	
\end{equation} 
	
	 However, ~\cite{KKO21} did not show that the integrality gap of the subtour elimination polytope is bounded below $\frac{3}{2}$, and therefore did not make progress towards the ``4/3 conjecture" which posits that the integrality gap of LP \eqref{eq:tsplp} is $\frac{4}{3}$. In this work we remedy this discrepancy by proving the following theorem, improving upon the bound of $\frac{3}{2}$ from Wolsey~\cite{Wol80} in 1980:

\begin{theorem}\label{thm:main}
	Let $x$ be a solution to LP \eqref{eq:tsplp} for a TSP instance. For some absolute constant $\epsilon > 10^{-36}$, the \hyperlink{tar:alg}{max entropy algorithm} outputs a TSP tour with expected cost at most $\frac{3}{2}-\epsilon$ times the cost of $x$. Therefore the integrality gap of the subtour elimination LP is at most $\frac{3}{2} - \epsilon$. 
\end{theorem} 

To prove \cref{thm:main}, we amend Section 4 of \cite{KKO21} but keep the remainder of the analysis essentially the same. Unlike \cite{KKO21}, this argument now preserves the integrality gap by avoiding the use of the optimum solution in bounding the cost of the matching. See \cref{sec:overview} for a discussion of our new approach.
%We note that the analysis in this paper is not specialized to the max entropy algorithm (although we rely on many results from \cite{KKO21} to obtain \cref{thm:main} itself); instead, it is valid for any algorithm which samples a spanning tree from the support of a solution to LP \eqref{eq:tsplp} and then adds the minimum cost matching on the odd degree vertices of the tree.  
%Instead, we use the polygon representation of near minimum cuts \cite{Ben95,BG08} to bound  the cost of the matching (see the following section for an overview of our new findings). %An added benefit of avoiding the use of OPT in the analysis is  %We remark this makes the analysis constructive 
%We remark that this allows future analyses to explicitly compute and possibly utilize the relevant laminar family of near minimum cuts (whereas previously one needed to know OPT to find the laminar family used in the analysis in \cite{KKO21}).
%In particular, we show that to get a bound better than $\frac{3}{2}$ for this class of algorithm it is (essentially) sufficient to handle the case in which the near minimum cuts of $x$ are a laminar family.

\subsection{Other Consequences}
\paragraph{Path TSP} In recent exciting work, Traub, Vygen, Zenklusen \cite{TVZ20} showed that an $\alpha$-approximation algorithm for metric TSP can be used as a black box to get a $\alpha(1+\eps)$ approximation algorithm for Path TSP. This together with \cite{KKO21} implies that there is a $3/2-\eps$ approximation algorithm for Path TSP (for $\eps>10^{-36}$). On the other hand, it is a folklore result that the integrality gap of the natural LP relaxation of Path TSP is at least $3/2$.  Therefore, a consequence of the above theorem is that although the best possible approximation factors of the two problem are the same (up to polynomial reductions), the natural LP relaxation of metric TSP has a strictly smaller integrality gap.


\paragraph{2-ECSM} In the 2-edge-connected multi-subgraph problem, or 2-ECSM for short, we are given a weighted graph $G$ and we want to find a minimum cost 2-edge-connected spanning subgraph, where an edge can be chosen multiple times.
The classical Christofides-Serdyukov algorithm gives a 3/2-approximation for 2-ECSM and despite significant attempts \cite{CR98,BFS16,SV14,BCCGISW20} improved algorithms were designed only for special cases of the problem.
Since in \cite{BG93} it is shown that LP \eqref{eq:tsplp} is a valid relaxation for 2-ECSM, we obtain:

\begin{corollary}	
For some absolute constant $\epsilon > 10^{-36}$ the \hyperlink{tar:alg}{max entropy algorithm} is a randomized $\frac{3}{2}-\epsilon$ approximation for the 2-edge-connected multi-subgraph problem.
\end{corollary}
%Beyond these theorems, we believe the analysis in this paper will open new avenues to improve the arguments in ~\cite{KKO21}. The analysis in that work is by nature non-constructive because it uses information about the optimal solution. Here we remove this weakness and could in principle construct the proposed fractional matching in polynomial time. Although of course this has no practical benefit since the algorithm always finds the minimum cost matching, this may allow future works to manipulate the algorithm to better serve the analysis.

%We analyze the max-entropy rounding algorithm introduced in \cite{OSS11} and slightly modified in \cite{KKO20, KKO21}. 

%In other words, we design a feasible vector for the $O$-join polytope to ``satisfy'' all near min cuts ``crossed on both  sides'' 


%Whereas Section 4 of ~\cite{KKO21} only deals with the near minimum cuts of $x$ (where $x$ is a solution to LP \eqref{eq:tsplp}) which lie along the optimal Hamiltonian cycle, we deal with all near minimum cuts of $x$ using the so-called polygon representation of near minimum cuts ~\cite{Ben97,BG08}. %The results give new intuition for the structure of cuts that are within $\frac{6}{5}$ or less of the edge connectivity of the graph.

 %: we show that we can incur a cost of $O(\eta^2) \cdot c(x)$ to ensure that the set of cuts with $x(\delta(S)) \le 2+\eta$ is a laminar family.


\subsection{New techniques and contributions}\label{sub:newtechniques}

This paper can be seen as a case study on how to reason about and deal with {\em near} minimum cuts. One can deduce from the classical cactus representation of a graph $G$ \cite{DKL76} (i) the structure of {\em all} min cuts of $G$ and (ii) the structure of the edges of $G$ in the sense that every edge $\{u,v\}$ maps to a unique {\em path} in the cactus between the images of $u$ and $v$. Furthermore, such a path intersects every cycle of the cactus on at most one cactus edge. The theory has found many application from designing fast algorithms
\cite{Kar00,KP09} to the analysis of approximation algorithms for TSP \cite{KKO20} and connectivity augmentation \cite{BGJ20,CTZ21}.

Two decades later, the theory of min cuts was extended to near min cuts in works of Bencz\'ur and Goemans \cite{Ben95, BG08} where they introduced the polygon representation which represents all cuts of a graph with at most $\frac{6}{5}k$ edges, where $k$ is its edge connectivity. Although these works completely classify the structure of all near min cuts of a given graph $G$, they do not characterize the structure of the \textit{edges} of $G$ with respect to these cuts, which can be important in applications (for example, in many of the recent applications of min cuts,
 one also needs to exploit the structure of the edges in relation to the cactus).
The structure on the edges turns out to be highly relevant in this work as well, and as a byproduct of our analysis we make progress towards classifying the way in which the edges of $G$ relate to the structure of the polygon representation.
 
 % and (to some extent) a classification of the set of edges of $G$ with respect to the polygon representation of Bencz\'ur and Goemans.
 
  %i
 %s to give a better understanding of the structure of edges of $G$ with respect to its near min cuts.

  %One can partition the edges of $G$ into sets $F_1\dots,F_m$ such that the set of edges in every min cut $(S,\overline{S})$ of $G$ is the union of edges in a pair $F_i,F_j$ for $i\ neq j$.
%\Nathan{Shayan can add something} For example...

For motivation, consider a generic family of network design problems in which we want to construct a network such that every pair $u,v$ of vertices has connectivity at least $c_{u,v}$. A natural approach is to write an LP relaxation to find a (minimum cost) vector $x: E \to \R_{\ge 0}$ such that for every cut $S$ separating $u$ and $v$, $x(\delta(S))\geq c_{u,v}$. We can round this LP using independent rounding or a dependent rounding scheme such as sampling from max entropy distributions. Using classical concentration bounds one can show that if $x(\delta(S))\gg c_{u,v}$ then with high probability the rounded solution has at least $c_{u,v}$ edges across this cut. So the main challenge is to ``fix'' near tight cuts, i.e., cuts where $x(\delta(S))\approx c_{u,v}$.  For an explicit instantiation of this scheme see \cite{KKOZ22}. A better understanding of the global structure of the family of near tight cuts has the potential to significantly simplify or even improve the approximation factor of such rounding algorithms. A classical technique to design algorithms for such network design problems is to apply uncrossing to extreme point solutions of the LP. One can view our contribution as an approximate uncrossing technique that deals with all near tight cuts (instead of just tight cuts) as we explain next.
%Next, we explain how our main theorem can be used to give global structure for near tight cuts in the case that $c_{u,v}=2$ for all $u,v$ and we contrast it with the classical uncrossing technique which only deals with tight/min cuts. 


\paragraph{An Approximate Uncrossing Technique.} A fundamental technique in the field of approximation algorithms is the uncrossing technique\footnote{See e.g. \cite{LRS11} for a number of applications of this technique.} of Jain \cite{Jai01}. Given a graph $G=(V,E)$,  a weight vector $x:E\to\R_{\geq 0}$, and a  function $f:V\to\R$, suppose that $x(\delta(S))\geq f(S)$ for all $S\subseteq V$. Let $\cN$ be the family of sets $S$ such that $x(\delta(S)) = f(S)$, i.e., the family of {\em tight} sets with respect to $f$. The uncrossing technique says that if $f$ is (weakly) supermodular then we can refine $\cN$ to a laminar family of sets, $\cH$, such that if all sets of $\cH$ are tight, then all sets of $\cN$ are tight as well. For a concrete example, suppose $f$ is a constant function, say $f(S)=2$ for all $\emptyset\subsetneq S\subsetneq V$. Then, sets of $\cH$ can be constructed using the cactus representation \cite{DKL76} of cuts in $\cN$. The significance of this method is that if $x$ is a basic feasible solution to a LP with constraints $x(\delta(S))\geq f(S)$ for all $S$, one can use this machinery to argue that the support of $x$ has size $O(|V|)$.

Informally, we prove the following, which 
can be seen as  an {\em approximate uncrossing technique}: 
\begin{theorem}[Informal]\label{thm:uncrossing}Suppose we have a vector $x:E\to\R_{\geq 0}$ such that $x(\delta(S))\geq f(S)$ for all $S$; define $\cN$ to be sets $S$ where $x(\delta(S))\leq f(S)(1+\eps)$ for some fixed $\eps>0$. If $f(.)$ is constant, say $f(S)=2$ for all $S$, then there is a set $\cN^*\subseteq \cN$ and a collection of edge sets $F_1,\dots,F_m\subseteq E$ such that the following hold:
\begin{itemize}
	\item $|\cN^*|= O(|V|)$, $m= O(|V|)$.
	\item $x(F_i)\geq 1-\eps/2$ for all $1\leq i\leq m$.
	\item Every edge $e$ is in at most $O(1)$ of the $F_i$'s.
	\item For every set $S\in \cN\smallsetminus \cN^*$ there exists $1\leq i<j\leq m$ such that $F_i\cap F_j=\emptyset$ and $F_i\cup F_j\subseteq \delta(S)$ and for every $S\in \cN^*$, there exists $1\leq i\leq m$ such that $F_i\subseteq \delta(S)$. 
\end{itemize}
\end{theorem}
In words, although we cannot simply refine $\cN$ to a linear number of sets, we can refine the edges in cuts of $\cN$ to a linear number of sets $F_1,\dots, F_m$ such  that we can essentially capture the edges of $\delta(S)$ for any $S\in \cN\smallsetminus \cN^*$ by a pair of disjoint $F_i$'s. We give a slightly weaker condition for cuts in $\cN^*$; namely we only capture half of their edges by $F_i$'s.

\begin{example}For a simple example of the above theorem, suppose $\eps=0$, i.e. $\cN$ is the set of min cuts of a graph $G$. Furthermore, suppose that every proper  cut in $\cN$ is \hyperlink{tar:crossing}{crossed} (recall that $S$ is proper if $1<|S|<|V|-1$) and that $\cN$ has at least one proper cut. 
Then, one can use an uncrossing technique, namely that if $A,B\in \cN$ then $A\cap B\in \cN$, to prove that $G$ must be cycle, namely we can order vertices of $G$, $v_0,\dots,v_{n-1}$ such that $x_{\{v_i,v_{i+1\text{ mod n}}\}}=1$.
In such a case we let $\cN^*=\emptyset$ and $F_i=E(v_i,v_{i+1\text{ mod }n})$.
%partition $V$ into sets $a_0,\dots,a_{m-1}$ such that 
%Let $\C$ be a connected component of crossing cuts of $\cN$, namely, for any pair of sets $A,B\in \C$ there is a path of crossing cuts all from $\C$ that goes from $A$ to $B$.
% and further suppose that $\cN$ can be represented by a cycle $C$ in the sense every min cut of $\cN$ corresponds to a min cut of $C$ and vice versa. Here we assume $a_0,\dots,a_{m-1}$ are the nodes of $C$ where each $a_i$ is identified with a disjoint set of vertices where $V=\uplus_{i=1}^m a_i$. In such a case, we can simply let $\cN^*=\emptyset$ and $F_i=E(a_i,a_{i+1\text{ mod }m})$. 
\label{eg:cycle}\end{example}

\begin{example}\label{eg:laminar}
For a second example, suppose again $\eps=0$ and $\cN$ is the set of mincuts of a graph $G$ where $\cN$ forms a laminar family (no two cuts cross). It turns out that we cannot decompose edges of cuts of $\cN$ into a linear sized collection of sets where every edge appears only a constant number of times. The main reason is that some edges may appear in an unbounded number of cuts. In this case we let $\cN^*=\cN$ and for every $A\in \cN$ (with immediate parent $B\in \cN$ in the laminar family) we add a set $F_A=\delta(A)\smallsetminus \delta(B)$  to our collection.  It is straightforward to show, using the structure of min cuts, that $x(F_A)\geq 1$; furthermore, since the size of a laminar family is linear in $V$, this gives a valid decomposition in the sense of above theorem.
\end{example}
Lastly, if $\eps=0$ and $\cN$ is the set of min cuts of an arbitrary graph, one can represent all min cuts of $\cN$ by a cactus \cite{DKL76} which can be seen as a tree of cycles. In such a case, one can use a construction similar to \cref{eg:cycle} for each cycle where instead of a vertex $v_i$ we have a set $a_i \subseteq V$ and one similar to \cref{eg:laminar} for the tree part of the cactus. For a concrete application of such a decomposition of min cuts see \cite{KKO20}.
%More generally, if $\cN$ corresponds to the set of min cuts of an arbitrary graph, the cuts of $\cN$ can be represented by a {\em cactus graph}. In such a case we add one $F_i$ for every edge of a cycle of the cactus. 


%and further for simplicity assume that there is a single connected component of crossing cuts in $\cN$, namely we can go from any $A$ to $B$ for $A,B\in\cN$ simply following crossing cuts of $\cN$. Then, one can represent cuts in $\cN$ by the set of min cuts of a cycle, namely we can contract vertices of $G$ 

%For a concrete application , suppose we need at least two edges in every set in $\cN^*$, say in a network optimization problem. Then, if we make sure that we have at least one edge in each $F_i$, all typical cuts constraints, $\cN\smallsetminus \cN^*$,  are satisfied, so we  reduce the problem to cuts in $\cN^*$. 


One of the main challenges in dealing with near min cuts relative to min cuts is that if $x(\delta(A)),x(\delta(B))\leq 2+\eps$ then $x(\delta(A\cap B))\leq 2+2\eps$. Therefore, if $\eps=0$, then min cuts are closed under intersection, set difference and union, but this is no longer true when $\eps>0$. So, to employ the classical uncrossing machinery one should be very careful to "uncross" only a constant number of times (independent of $\eps$) to make sure that every cut remains within $2+O(\eps)$. This is the main reason that the polygon representation of near min cuts (see below) is more sophisticated, e.g., we can no longer argue $x(E(a_i, a_{i+1}))\approx 1$, see \cref{fig:nearmincutbadexample}.

Although we don't study it here, we believe it may be worthwhile to find generalizations of \cref{thm:uncrossing} which hold for any (weakly) supermodular function.% That could be helpful in many questions based on the network optimization framework of Jain \cite{Jai01}.

\begin{remark} 
 We do not explicitly prove \cref{thm:uncrossing} in this extended abstract, as it is not used to prove \cref{thm:main}. However it can be deduced from arguments in \cref{sec:twoside} and \cref{app:oneside}. 
%In \cref{sec:overview} we discuss the main ideas of the proof of \cref{thm:uncrossing}. Here, let us explain the main challenge: In principal one might try to simply extend the above decomposition for the case $\eps=0$. The main challenge is that if $x(\delta(A)),x(\delta(B))\leq 2+\eps$ then $x(\delta(A\cap B))\leq 2+2\eps$. Therefore, if $\eps=0$, then min cuts are closed under intersection, set difference and union, but this is no longer true when $\eps>0$. So, to employ the classical uncrossing machinery one should be very careful to "uncross" only a constant number of times (independent of $\eps$) to make sure that every cut remains within $2+O(\eps)$. This is the main reason that the polygon representation of near min cuts (see below) is more sophisticated, e.g., we can no longer argue $x(E(a_i, a_{i+1}))\approx 1$, see \cref{fig:nearmincutbadexample}.
\end{remark}





\paragraph{Extensions to the Polygon Representation} To obtain our uncrossing framework we prove new properties of the polygon representation.
Given a graph $G=(V,E)$, let $k$ be the edge-connectivity of $G$, i.e. the number of edges in a minimum cut of $G$. For $\eps>0$, consider the set of $(1+\eps)$-near minimum cuts of $G$: cuts $(S,\overline{S})$ where $|E(S,\overline{S})| < (1+\eps)k$.
Bencz\'ur \cite{Ben95} and Bencz\'ur, Goemans \cite{BG08} proved that if $\eps \le 1/5$ then the near minimum cuts of $G$ admit a {\em polygon representation}. Namely, every connected component $\cC$ of \hyperlink{tar:crossing}{crossing} $(1+\eps)$ near min cuts can be represented by the diagonals of a convex polygon. In this polygon, the vertices of $G$ are partitioned into sets called \textit{atoms}, and every atom is mapped to a cell of this polygon defined by the diagonals and the boundary of the polygon itself (see \cref{sec:polyrep} for more details). 

The polygon representation can be seen as a generalization of the well-known cactus representation \cite{DKL76} of minimum cuts where a cycle of the cactus is replaced by a convex polygon. Unlike a cycle, some vertices/atoms map to the interior of the polygon, which are called ``inside'' atoms. The inside atoms at first look like a mystery and one can ask many questions about them such as how many can exist and what structures they can exhibit.



 Here, we explain two lemmas we proved which might find further applications beyond TSP in the future. 
%Our results give new intuition and understanding about the structure of polygon representations. These guide our analysis of the integrality gap of the subtour LP.
 %For example, one of our new observations is a 
 First, we give a necessary condition for a cell of a polygon to contain an inside atom:
\begin{lemma}[Informal, see \cref{thm:halfplanes}]
	Consider a polygon $P$ for a connected component $\C$ of a family of $1+\eps$ near min cuts for $\eps \le 1/5$ (where representing diagonals correspond to cuts in $\C$). Any cell of $P$ that has an inside atom must have at least $\Omega(1/\eps)$ many sides. 
\end{lemma}
This can be seen as a generalization of \cite[Lem 22]{BG08} to the case in which the cell is allowed to be adjacent to vertices of the polygon $P$.

Now, we explain our second extension: it follows from the cactus representation of minimum cuts that for a graph $G$ and a min cut $S$ one can partition the set of all min cuts that cross $S$ into two groups ${\cal A}=\{A_1,\dots,A_k\}$ and ${\cal B}=\{B_1,\dots,B_l\}$ for some $k,l\geq 0$ such that $S\cap A_1\subseteq S\cap A_2 \subseteq \dots S\cap A_k$ and, similarly, $S\cap B_1\subseteq \dots\subseteq S\cap B_l$. We prove a generalization of this fact for near min cuts:
\begin{lemma}[Informal, see \cref{lem:crosschain}]
Consider the set of $1+\eps$ near min cuts of a graph $G$ for $\eps\leq 1/10$; for any such near min cut $S$, one can partition the $1+\eps$ near min cuts crossing $S$ into two groups ${\cal A}=\{A_1,\dots,A_k\}$ and ${\cal B}=\{B_1,\dots,B_l\}$ such that $S\cap A_1 \subseteq S\cap A_2\subseteq \dots \subseteq S\cap A_k$ and similarly for cuts in ${\cal B}$.
\end{lemma}

\subsection{Outline of rest of paper} After reviewing preliminaries in \cref{sec:prelims}, we give a high-level overview of our proof technique in \cref{sec:overview}. The main new technical contributions of this paper are in \cref{sec:polyrep} and  \cref{sec:twoside}. The remaining content of the paper essentially follows from ~\cite{KKO21}. %Therefore, the reader may want to skip \cref{sec:proof-of-main}. 



\section{Evolution of Aggregation Behaviours with Novelty Search}
\label{sec:aggregation}

% section summary
In this section, we apply novelty search to the problem of swarm aggregation --- a commonly studied task in swarm robotics. In the aggregation task, a dispersed robot swarm must form a single cluster. We conduct three sets of experiments with novelty search, each with a distinct behaviour characterisation. We compare the performance of novelty search to the performance of traditional fitness-based evolution. We also include results from evolutionary runs with NEAT in which random fitness scores are assigned to individuals. Random evolution serves as a baseline for performance comparisons.

% aggregation related work
Aggregation is a task of fundamental importance in many biological systems. It is the basis for the emergence of various forms of cooperation, and can be a considered a prerequisite for the accomplishment of many collective tasks~\citep{trianni03}. Several works describe the evolution of aggregation behaviours for swarms of robots. Commonly, the parameters of neural networks with fixed topologies are optimised by fitness-based evolutionary algorithms. \citet{baldassarre03} successfully evolved controllers for a swarm of robots to aggregate and move towards a light source in a clustered formation. Three classes of behaviours were evolved, but each evolutionary run always converged to only one of the classes. \citet{trianni03} studied the evolution of a swarm of simple robots to perform aggregation in a square arena. Two different behaviours were evolved: \emph{static clustering} which leads to the formation of compact and stable clusters, and \emph{dynamic clustering} which leads to loose but moving clusters. \citet{sahin05b} used a similar experimental setup as~\citep{trianni03}, and studied how some of the parameters of the evolutionary algorithm affected the performance and the scalability of the evolved behaviours.

In the previous studies discussed above, the robots used directional sound sensing. Directional sensing allowed the robots to follow gradients towards groups of other robots emitting sound signals. The use of sound and directional microphones makes the aggregation task sufficiently easy for controllers based on reactive neural networks without any hidden neurons to solve the task~\citep{trianni03, baldassarre03, sahin05b}. In our work, the aggregation task is more challenging: the robots do not use sound, the range of the sensors is significantly lower than in previous studies, and the arena is larger. These modifications increase the difficulty of the task and may require radically different strategies for aggregation~\citep{soysal07}, since robots can only sense one another when they are close.


\subsection{Experimental Setup}
\label{sec:aggregation_setup}

Our experimental framework is based on the Simbad 3d Robot Simulator~\citep{hugues06} for the robotic simulations, and on NEAT4J\footnote{NeuroEvolution for Augmenting Topologies for Java -- \url{http://neat4j.sourceforge.net}} for the implementation of NEAT. Simbad 3D simulates kinematics and implements simple collision handling. The environment is a 3\,m by 3\,m square arena bounded by walls. The swarm is homogeneous and composed of 7 robots. The robots are modelled based on the e-puck educational robot~\citep{mondada09}, but do not strictly follow its specification. Each robot is circular with a diameter of 8\,cm and can move at speeds of up to 12\,cm/s. Regarding sensors, each robot is equipped with 8 IR sensors evenly distributed around its chassis for the detection of obstacles (walls or other robots) within a range of 10\,cm, and 8 sensors dedicated to the detection of other robots within a range of 25\,cm. Both types of sensors return the distance of the object that is being sensed, or the maximum value if nothing is sensed. An additional sensor (count sensor) returns the percentage of nearby robots (within a radius of 25\,cm), relative to the desired cluster size (the total swarm size in our experiments). The simulated sensors are not based on any specific hardware. Nevertheless, the obstacle sensors could be implemented with active IR sensors, while the robot sensors and the robot count sensor could be implemented with short-range communication~\citep{correll07,gutierrez2008}. The inputs of the neural network controller are the normalised readings from the sensors mentioned above. The controller has three outputs: one to control the speed of each motor, and one dedicated to completely stopping the robot if its activation is above 0.5.

We evaluate each controller 10 times. In each simulation, we vary the initial position and orientation of each robot. The initial positions are randomised in such a way that the robots are placed at least 50\,cm from one another, which ensures that the robots always are reasonably well distributed at the beginning of the simulation. Each simulation lasts for 2500 simulation steps, which corresponds to 250\,s of simulated time. The best individual of each generation was post-evaluated in 100 simulations, in order to obtain a more accurate fitness estimate.

\subsection{Configuration of the Evolutionary Algorithms}
\label{sec:aggregation_algsetup}

Fitness-based evolution and random evolution use the default NEAT implementation provided by the NEAT4J library. In random evolution, random fitness scores are assigned to each individual. Novelty search was implemented over NEAT, following the description and parameters in~\citep{stan11a}. We used a $k$ value of 15 nearest neighbours and individuals are stochastically added to the archive with a probability of 2\%, as suggested in \citep{stan10b}. The parameters for NEAT were the same in all experiments: recurrent links are allowed, crossover rate -- 25\%, mutation rate -- 10\%, population size -- 200, and each evolutionary process was conducted for 250 generations. The remaining parameters were assigned their default value according to the NEAT4J implementation.

The fitness function is based on the average distance to the centre of mass~(also used in \citep{trianni03}). The fitness $F_{a}$ of a simulation with $T$ time steps and $N$ robots is defined as:
\begin{equation}
F_{a}=\sum_{i=1}^{N}\frac{1-dist(\mathbf{R}_T,\mathbf{r}_{i,T})}{N} \enspace ,
\end{equation}
where $\mathbf{R}_T$ is the centre of mass at the end of the simulation, and $\mathbf{r}_{i,T}$ is the position of robot $i$ at the same instant. The distance values are normalised to $[0,1]$. The fitness scores obtained in each of the 10 simulations are combined to a single value using the harmonic mean, as advocated in~\citep{sahin05b}.

The behaviour characterisations we use in novelty search are based on spatial inter-robot relationships, measured at regular intervals of 5\,s throughout the simulation. We devised three characterisations:

\begin{description}
\item[$\mathbf{b_{cm}}$:] The average distance to centre of mass of the swarm is sampled throughout the simulation. Considering a simulation with $N$ robots and $\tau$ temporal samples, the behaviour characterisation $\mathbf{b_{cm}}$ is given by:
\begin{equation}
\mathbf{b_{cm}}=\frac{1}{N} \left [ \sum_{i=1}^{N}dist(\mathbf{R}_1,\mathbf{r}_{i,1}),\cdots , \sum_{i=1}^{N}dist(\mathbf{R}_\tau,\mathbf{r}_{i,\tau}) \right ] \enspace .
\label{eq:cm}
\end{equation}

\item[$\mathbf{b_{cl}}$:] The number of robot clusters is sampled at regular intervals throughout the simulation, inspired by the metric used in \citep{sahin05b}. Two robots belong to the same cluster if the distance between them is less than their sensor range (25\,cm). The behaviour characterisation $\mathbf{b_{cl}}$ is given by:
\begin{equation}
\mathbf{b_{cl}}=\frac{1}{N}\left [clustersCount(1), \cdots , clustersCount(\tau) \right ] \enspace .
\label{eq:cl}
\end{equation}

\item[$\mathbf{b_{cmcl}}$:] The two characterisations $\mathbf{b_{cm}}$ and $\mathbf{b_{cl}}$ are concatenated to form a single characterisation. Both $\mathbf{b_{cm}}$ and $\mathbf{b_{cl}}$ have the same length and each element of the characterisation vectors ranges from 0 to 1. Thus both components of the behaviour characterisation approximately have the same contribution to the novelty metric. The new characterisation $\mathbf{b_{cmcl}}$ is given by:
\begin{equation}
\mathbf{b_{cmcl}}=(\mathbf{b_{cm}},\mathbf{b_{cl}}) \enspace .
\label{eq:comb}
\end{equation}
\end{description}

We computed the spatial inter-robot relationships at every 5\,s and the simulation lasted for 250\,s. This resulted in behaviour characterisation vectors of length 50 for $\mathbf{b_{cm}}$ and $\mathbf{b_{cl}}$, and vectors of length 100 for $\mathbf{b_{cmcl}}$. As 10 simulations are conducted to evaluate each controller, the corresponding final behaviour characterisation vector is the element-wise mean of the vectors obtained in all 10 simulations. In order to establish a basis for comparison, all the controllers evolved by novelty search and by random evolution were also scored by the fitness function $F_{a}$. It is important to note that the fitness scores did not have any influence on the evolutionary process in the novelty search experiments.

\subsection{Performance Comparison}

The fitness trajectories for novelty search, fitness-based evolution, and random evolution, are depicted in Figure~\ref{fig:ccm_graph}(left). The data points plotted are the averages of the highest fitness score found so far from the start of the evolutionary run and until the current generation.\footnote{The values that are plotted are fitness scores measured by the fitness function $F_{a}$, and not the actual scores used for selection in novelty search or in random evolution. Note that even random evolution has an ascending fitness trajectory because the data points plotted are the highest score achieved so far in the evolutionary process. The trajectory for random evolution can thus increase from time to time when an individual, by chance, scores higher than any previously evaluated individual.} Note that since novelty search does not follow a fitness gradient, the best solutions are not necessarily found in the last generation. As such, it is necessary to save the interesting solutions (for instance, the best one found so far) throughout the evolutionary process. We continued the evolutionary process beyond the 150th generation for all experiments, but there was no significant change in the fitness scores after that point. Although the lowest possible fitness score is $\approx 0.05$, the highest fitness of an initial random population is on average $\approx 0.55$. The relatively high fitness of an initial random population is explained by the fact that stochastically moving robots tend be significantly closer to one another than in the worst case scenario where robots are located in opposite corners of the arena.

\begin{figure}
	\centering
	\includegraphics[width=1\textwidth]{pic/aggregation_performance.pdf}
\caption{Highest fitness scores achieved in the aggregation task with fitness-based evolution (\emph{Fit}), random evolution (\emph{Random}), and novelty search with the three behaviour characterisations (\emph{NS-cmcl}, \emph{NS-cm}, \emph{NS-cl}). Left: average fitness value of the highest scoring individual found so far at each generation. The values are averaged over 30 independent evolutionary runs for each method. Right: box-plots of the highest fitness score found in each evolutionary run, for each method. The whiskers extend to the lowest and the highest data point within 1.5 times the interquartile range. Outliers are indicated by circles.}
\label{fig:ccm_graph}
\end{figure}

Figure~\ref{fig:ccm_graph}(right) shows box-plots of the highest fitness score for each method in 30 evolutionary runs. The results show that novelty search could consistently find relatively high scoring solutions, with all the behaviour characterisations. There was no significant differences between the highest fitness scores achieved with fitness-based evolution and with novelty search with $\mathbf{b_{cmcl}}$ (Mann--Whitney U test, $p$-value $<$ 0.05). With $\mathbf{b_{cl}}$, the highest fitness scores were significantly lower than the highest scores found by novelty search with $\mathbf{b_{cmcl}}$ and fitness-based evolution. The capacity of novelty search to bootstrap the evolutionary process should be noted. From around generation 5 to generation 40, all novelty search variants achieve fitness scores significantly higher than fitness-based evolution ($p$-value $<$ 0.01).

Previous studies have shown that novelty search can perform better than fitness-based evolution in deceptive tasks, but fails to match the performance of fitness-based evolution when the task is non-deceptive \citep{stan11a,mouret11}. Our results reveal that in the aggregation task, the fitness-function is not deceptive, as fitness-based evolution typically converges to the most effective strategy, achieving high fitness scores (except in a single run, see Figure~\ref{fig:ccm_graph}). Still, novelty search managed to achieve fitness scores that are comparable to the scores achieved in fitness-based evolution. 

\subsection{Behavioural Diversity}

The analysis of the explored behaviour space in novelty search and in fitness-based evolution allows for a better understanding of the evolutionary dynamics. Since each behaviour characterisation vector has either 50 or 100 dimensions, we applied a dimensionality reduction method to facilitate visualisation of the explored regions of the behaviour space. We used a Kohonen self-organising map \citep{kohonen90}. Kohonen maps are neural networks trained using unsupervised learning to produce a two-dimensional discretisation of the input space of the training samples, while preserving the topological relations. We trained a Kohonen map with the behaviour vectors found both by novelty search and by fitness-based evolution, and then mapped each individual to the region whose vector is more similar to the individual's behaviour vector.

\subsubsection{Centre of mass behaviour characterisation} \label{sec:agg_cm}

\afterpage{
\begin{figure}
\centering 
	\includegraphics[width=1\textwidth]{pic/aggregation_som_cm.pdf}
\caption{Kohonen maps representing the explored behaviour space in fitness-based evolution (Fit) and in novelty search with $\mathbf{b_{cm}}$ (NS-cm). Each circle is a neuron corresponding to the vector depicted by the embedded plot (the average distance to the centre of mass over time). Each behaviour vector is mapped to the neuron with the most similar vector. The darker the background of a neuron is, the more behaviours were mapped to it. The regions associated with higher fitness scores are indicated with a bold circle (regions \emph{c} to \emph{g}).}
\label{fig:som_cm}
\end{figure}

\begin{table}
\caption{The number of individuals mapped to each behaviour region \emph{a-g} (see Figure~\ref{fig:som_cm}), relative to the total number of mapped individuals.}
\centering
\begin{tabular}{r r r}
\toprule
Behaviour region & Fit exploration (\%) & NS-cm exploration (\%) \\
\midrule
a & 18.17 & 3.80 \\
b & 22.18 & 6.39 \\
c & 10.27 & 1.82 \\
d & 4.57 & 1.54 \\
e & 1.23 & 2.75 \\
f & 0.15 & 1.90 \\
g & 0.14 & 3.50 \\
\bottomrule
\end{tabular}
\label{tab:som_cm}
\end{table}
}

The Kohonen maps corresponding to the explored behaviour space in fitness-based evolution and novelty search with $\mathbf{b_{cm}}$ can be seen in Figure~\ref{fig:som_cm}. The results show that fitness-based evolution focused the search on only a subset of the behaviour regions. In contrast, novelty search explored the behaviour space much more uniformly. If we consider the behaviour regions associated with higher fitness scores (regions \emph{c} to \emph{g}), important differences become apparent: fitness-based evolution avoids behaviour regions where the average distance to the centre of mass rises beyond the initial value, such as the regions \emph{f} and \emph{g} (see Table~\ref{tab:som_cm}). Instead, the evolutionary process is much more focused on behaviours that lead to a monotonic decrease in the average distance to the centre of mass. This bias is introduced by the fitness function, as it favours a low average distance to centre of mass. Evolving solutions where the average distance rises beyond the initial value might require going against the fitness gradient. As such, these types of solutions are impeded in the fitness-based evolutionary process. Novelty search, on the other hand, is not subject to a static evolutionary pressure, and can therefore explore and discover a wider range of solutions to the task.

The analysis of the behaviour patterns (see plots inside the neurons in Figure~\ref{fig:som_cm}) reveals an interesting point: the regions explored by novelty search are characterised by vectors that differ much less than uniformly sampled vectors of length 50 would. This happens because the elements of a given behaviour vector are inherently correlated. Robots cannot, for instance, travel instantly to any location in the environment, and the difference between consecutive samples of the average distance to the centre of mass is therefore limited by the speed of the robots. As such, the reachable behaviour regions constitute only an (often small) subset of the total behaviour space. This explains why novelty search can consistently find successful solutions to the task, even in a high-dimensional behaviour space.

The results also show that fitness-based evolution spends a considerable amount of time in behaviour regions where there are no aggregation dynamics at all (regions \emph{a} and \emph{b}). These regions correspond to the initial best solutions, where robots do not move or move randomly in the arena, in order to maintain an average distance to the centre of mass that is at least as low as the average distance at the start of the simulation. This class of behaviours constitutes a local maximum. Novelty search does not spend as much time in such behaviour regions (see Table~\ref{tab:som_cm}), as the search moves towards regions with novel behaviours. Since the initial average distance to the centre of mass is situated in the middle of the spectrum, the novel behaviours can be aggregation behaviours as well as dispersion behaviours. Typically, each novelty search evolutionary run explores both types of behaviour simultaneously, maintaining a healthy diversity in the population. This phenomenon can offer an explanation for the relatively good performance of novelty search in the earlier stages of evolution: novelty search does not get stuck in the early local maximum, and as such can explore and discover better solutions faster.

To confirm the behavioural diversity evolved by each method, we resorted to the visual inspection of the best solutions. In fitness-based evolution, the highest scoring individuals tended to follow the same behaviour pattern (Figure~\ref{fig:behaviours_cm},~\textbf{fit}):

\begin{description}
\item[\bf fit:] The robots explore the environment in large circles, and form static clusters when they encounter one another. If a cluster is small, the robots abandon it after a while and recommence the circular exploration.
\end{description}

The \textbf{fit} behaviour pattern was also often found by novelty search. However, novelty search commonly evolved a different type of solutions in which walls are exploited to achieve aggregation. Examples of behaviour patterns that use the walls are described below and depicted in Figure~\ref{fig:behaviours_cm}~(\textbf{cm1} and \textbf{cm2}).

\begin{description}
\item[\bf cm1:] The robots go towards the walls while avoiding any other robots encountered. When they reach a wall, they follow the wall for a while and then depart with a certain circular trajectory that causes them to pass through the centre of the arena. Eventually, the robots end up forming a loose cluster close to the centre.
\item[\bf cm2:] The robots move in straight lines until they encounter a wall, and then, depending on the approach angle, they either remain static for a while or start to follow the wall. When two or more robots encounter one another, they stop and form a cluster near the wall.
\end{description}

\begin{figure}
\centering 
	\includegraphics[width=1\textwidth]{pic/behaviours_cm_new.pdf}
\caption{The typical best solution evolved by fitness-based evolution (fit), and two examples of solutions found by novelty search with $\mathbf{b_{cm}}$ (cm1 and cm2). Each line represents the trajectory of a single robot throughout the simulation. The circles depict the initial positions of the robots and the squares depict their final positions. Videos of the behaviours are available as online supplemental material.}
\label{fig:behaviours_cm}
\end{figure}

Visual inspection of the behaviours confirms that fitness-based evolution did not explore some classes of solutions, and  in particular, solutions where the robots navigate near the walls. If all the robots initially move towards one of the walls surrounding the arena, they will often end up far apart. Consequently, the centre of mass of the robots will often be close to the centre of the arena, far from the robots, which explains the initial high average distance to the centre of mass (regions \emph{e, f, g} in Figure~\ref{fig:som_cm}). Learning to navigate near the walls potentially requires the evolution of many solutions with very low fitness scores. Avoiding navigation close to walls, on the other hand, results in higher fitness scores, because robots will, by chance, be closer to one another, and therefore to the centre of mass. Given the opportunistic nature of fitness-based evolution, the stepping stone of first navigating along walls to later achieve aggregation, is thus unlikely to be found. In fact, we have been unable to find reports in the literature on the evolution of aggregation behaviours that exploit walls. Such behaviours have, however, proven successful in biological systems, such as in self-organised aggregation of cockroaches \citep{jeanson05}.

\subsubsection{Number of clusters behaviour characterisation}

The difference between the fitness scores of the solutions achieved with $\mathbf{b_{cm}}$ and $\mathbf{b_{cl}}$ was not significant. Nevertheless, we found that the different characterisations affected the evolved behaviours. Many of the best solutions evolved with $\mathbf{b_{cl}}$ were similar to the solutions evolved by fitness-based evolution, namely the formation and disbandment of small clusters. However, new behaviour patterns were also found. The distinctive behaviours evolved by novelty search with $\mathbf{b_{cl}}$ were focused on the exploitation of inter-robot relations and of flocking in particular. For example, the following two behaviour patterns were identified (see Figure~\ref{fig:behaviours_cl}):

\begin{description}
\item[\bf cl1:] The robots navigate in circles, and when two robots meet, one starts to follow the other, which leads to flocking with circular trajectories. Eventually, a single moving file is formed.
\item[\bf cl2:] Similar to (a), but when a robot cluster reaches a reasonable size, the cluster becomes static.
\end{description}

\begin{figure}
\centering 
	\includegraphics[width=0.66\textwidth]{pic/behaviours_cl_new.pdf}
\caption{Examples of distinctive behaviours evolved by novelty search with $\mathbf{b_{cl}}$. Each line represents the trajectory of a single robot throughout the simulation. The circles depict the initial positions of the robots and the squares depict their final positions. Videos of the behaviours are available as online supplemental material.}
\label{fig:behaviours_cl}
\end{figure}

Overall, novelty search with $\mathbf{b_{cl}}$ focused on different classes of behaviours than novelty search with $\mathbf{b_{cm}}$. One of the main reasons for the difference in the evolved behaviours is conflation. Conflation occurs when individuals with distinct observable behaviours have very similar behaviour characterisation vectors~\citep{stan11a}. The consequence is that an individual with a distinct observable behaviour might not be considered novel by the novelty measure, and may thus disappear from the population. Conflation can represent both an advantage in terms of efficiency because it reduces the size of the search space, and a disadvantage when it inhibits the discovery of successful solutions or important stepping stones.

Two examples of behaviours that can be conflated are shown in Figure~\ref{fig:conflation}. For the $\mathbf{b_{cm}}$ characterisation, the degree of clustering of the robots is irrelevant; while for the $\mathbf{b_{cl}}$ characterisation, the distance between robots (and clusters) is irrelevant. The impact of conflation can be seen in the evolved behaviours: with the $\mathbf{b_{cm}}$ characterisation, there were more behaviours that exploited walls, because navigating near them has a great impact on the novelty measure; while with $\mathbf{b_{cl}}$, the behaviours focused on the interactions between the robots and clusters, including following one another and disbanding clusters.

\begin{figure}[h]
\centering 
	\includegraphics[width=0.8\textwidth]{pic/conflation.pdf}
\caption{An illustration of conflation for the centre of mass behaviour characterisation $\mathbf{b_{cm}}$ (left) and for the number of clusters behaviour characterisation $\mathbf{b_{cl}}$ (right). In both cases, the differences would not be reflected in the respective behaviour characterisations.}
\label{fig:conflation}
\end{figure}

\subsubsection{Combined behaviour characterisation}
\label{sec:combined_aggregation}
%
Novelty search with the composed behaviour characterisation $\mathbf{b_{cmcl}}$ achieved on average marginally higher fitness scores than $\mathbf{b_{cm}}$ and $\mathbf{b_{cl}}$ (Mann--Whitney U test, $p$-value $<$ 0.05), see Figure~\ref{fig:ccm_graph}(right). The composed behaviour characterisation considers both the average distance to centre of mass and the number of clusters formed. As such, different behaviours are less likely to be conflated, which can potentially lead to better solutions.

It should, however, be noted that $\mathbf{b_{cm}}$ and $\mathbf{b_{cl}}$ are closely related with one another. It is not possible, for instance, to have a low average distance to the centre of mass and at the same time a large number of clusters. A large number of clusters implies that the swarm is scattered, and as such, the robots will have a high average distance to the centre of mass. Combining $\mathbf{b_{cm}}$ and $\mathbf{b_{cl}}$ reduces conflation to some extent, but since the two characterisations are related to one another, no additional effort is needed to explore the larger behaviour space. This explains why the behaviour space with more dimensions ($\mathbf{b_{cmcl}}$) did not have a negative impact in the effectiveness of novelty search in this case.

\subsection{Neural Network Complexity}

Previous work has shown that an advantage of novelty search, when used together with NEAT, is its ability to evolve solutions with lower genomic complexity \citep{stan11a}. To determine if such advantage holds in the aggregation task, we analysed the complexity (sum of the number of neurons and number of connections) of the solutions found by fitness-based evolution and by novelty search. The comparison is established by analysing the average complexity (across the multiple evolutionary runs) of the least complex solution with a fitness score above a certain threshold. The results are shown in Table~\ref{tab:complexity}. We only show the results for the $\mathbf{b_{cmcl}}$ characterisation because the results for the $\mathbf{b_{cm}}$ and $\mathbf{b_{cl}}$ characterisations are similar.

\begin{table}[b]
\caption{Comparison of the least complex solutions evolved by fitness-based evolution (Fit) and novelty search with $\mathbf{b_{cmcl}}$ (NS). The \emph{Complexity} columns lists the network complexity (sum of the number of neurons and the number of connections) of the least complex individual with a fitness score above a certain \emph{Fitness level}. The \emph{Generation} column lists the average generation in which the least complex individual was generated. The values are averages of 30 evolutionary runs for each method.}
\centering
\begin{tabular}{r r r r r r r}
\toprule
\multicolumn{ 1}{c}{} & \multicolumn{ 2}{c}{Generation} & \multicolumn{ 2}{c}{Complexity} \\ \cmidrule{ 2- 5}
\multicolumn{ 1}{r}{Fitness level} & \multicolumn{1}{r}{Fit} & \multicolumn{1}{r}{NS} & \multicolumn{1}{r}{Fit} & \multicolumn{1}{r}{NS} \\ \midrule
0.60 &  9 & 5  & 74.20 & 71.33 \\
0.65 &  18 & 7  & 75.77 & 71.67 \\
0.70 &  28 & 14 & 77.10 & 72.40 \\
0.75 &  42 & 22 & 79.10 & 74.13 \\
0.80 &  55 & 41 & 80.90 & 77.33 \\
0.85 &  76 & 63 & 83.17 & 81.70 \\
\bottomrule
\end{tabular}
\label{tab:complexity}
\end{table}

The results show that on average, for the same fitness levels, novelty search finds individuals with significantly less complex neural networks (Mann--Whitney U test, $p$-value $<$ 0.05). The difference is especially pronounced in the lower fitness levels. Note that the networks in the initial populations (without any hidden neurons) have a complexity of 71 (20 neurons and 51 links). The difference in the network complexity can be ascribed to the convergent nature of fitness-based evolution. If the best controllers in the earlier stages of evolution have more complex neural networks, fitness-based evolution starts to converge to such complex structures. As novelty search does not converge, and has tendency to explore simple solutions before moving on to more complex ones \citep{stan11a}, it is capable of finding solutions with a lower network complexity.

\section{Evolution of Resource Sharing Behaviours with Novelty Search}
\label{sec:sharing}

In this section, we study the application of novelty search to a more complex task in which a swarm of robots share a single resource. The swarm must coordinate in order to allow each member periodical access to a single battery charging station. In the task, the robots should first find the charging station, and then effectively share the station to ensure the survival of all the robots in the swarm. The charging station can only hold one robot at the time.

The problem of autonomous charging and resource conflict management is widely studied in the literature. \citet{cao97} identify resource conflicts as one of the fundamental challenges in the design of cooperative behaviours in multi-robot systems. Resource conflicts arise when a single indivisible resource (in our task, the charging station) is requested by multiple robots at the same time. The problem of sharing an energy charging station in particular is addressed in \citep{munoz02,robichaud02,kernbach11}.

\subsection{Experimental Setup}
\label{sec:energy_setup}

We use a swarm composed of 5 homogeneous robots. The robots are identical to the ones used in the aggregation experiments (see Section~\ref{sec:aggregation_setup}), except that they are equipped with additional sensors. Each robot has (i)~8 IR sensors evenly distributed around its chassis for the detection of obstacles (walls or other robots) up to a range of 10\,cm; (ii)~8 sensors dedicated to the detection of other robots up to a range of 25\,cm; (iii)~a ring of 8 sensors for the detection of the charging station up to a range of 1\,m; (iv)~a binary sensor that indicates whether or not the robot is currently being recharged; and (v)~a proprioceptive sensor that reads the current energy level of the robot. The sensors (i), (ii) and (iii) return the distance of the object that is being sensed. The experimental setup and the sensor ranges are depicted on Figure~\ref{fig:energy_setup}. The environment is a 3\,m by 3\,m square arena bounded by walls.

\begin{SCfigure}[1.3][b]
\centering 
	\includegraphics[width=0.35\textwidth]{pic/setup_energy.pdf}
\caption{The resource sharing task experimental setup. The grey circle in the centre is the charging station. The black filled circles are the robots (starting positions vary in each simulation). The solid circle around the top left robot represents the range of the obstacles sensor, the dashed circle represents the range of the robot sensor, and the fine dashed circle represents the range of the charging station sensor.}
\label{fig:energy_setup}
\end{SCfigure}

Each robot starts with full energy (1000 units), and the energy consumption increases linearly with the speed of the motors: a robot spends 5 units per second when motors are off, and 10 units of energy per second when both motors operate at their maximum speed. The charging station has the same diameter as a robot, and to recharge, the robots must remain static inside the charging station. The charging station is located in the centre of the arena, and charges a robot at a rate of 100 units of energy per second. Similarly to the aggregation experiments, each controller is evaluated in 10 simulations with random starting positions for the robots. Each simulation lasts for 2500 simulation steps. The highest scoring individual of each generation was post-evaluated in 100 simulations.

\subsection{Configuration of the Evolutionary Algorithms}
\label{sec:energy_functions}

We used the same parameter values for novelty search and the NEAT algorithm as in the aggregation experiments (see Section~\ref{sec:aggregation_algsetup}). However, we continued each evolutionary run until the 400th generation, since the resource sharing task proved to be more challenging than the aggregation task.

The fitness function $F_{s}$ used to evaluate the controllers is a linear combination of the number of robots alive at the end of the simulation (henceforth referred to as \emph{survivors}) and the average energy of the robots throughout the entire simulation:
\begin{equation}
F_{s} = 0.9\cdot\frac{|a_{T}|}{N}+0.1\cdot \sum_{t=1}^{T}\sum_{i=1}^{N}\frac{e_{i,t}}{TNe_{max}}\enspace ,
\end{equation}
where $|a_{T}|$ is the number of survivors, $T$ is the length of the simulation, $N$ is the number of robots in the swarm, $e_{i,t}$ is the energy of the robot $i$ at time $t$, and $e_{max}$ is the energy capacity of a robot. The second term of $F_s$ concerning the average energy was included to differentiate solutions where the same number of robots survive. Without the second term, there would be no fitness gradient, since it is unlikely that the initial population contains solutions where at least one robot survives until the end.\footnote{We empirically determined the probability of randomly generating a solution where at least one robot survives until the end (in any of the 10 trials) to be approximately 1\%.}

We experimented with behaviour characterisations of a different nature in the resource sharing experiments, compared to the characterisations used in the aggregation experiments. While in the aggregation task, we used spatial relationships between the robots sampled every 5\,s during the simulation, in the resource sharing experiments, we use only quantities that characterise a simulation as a whole. We evaluated two behaviour characterisations:

\begin{description}
\item[$\mathbf{b_{simple}}$:] The first characterisation is closely related to the fitness function, and it is composed of two values (normalised to the interval $[0,1]$): (i)~the number of robots that survive till the end of the simulation; and (ii)~the average energy of all alive robots throughout the simulation. $\mathbf{b_{simple}}$ is given by:
\begin{equation} \label{eq:bsimple_energy}
\mathbf{b_{simple}} = \left ( \frac{|a_{T}|}{N} \;,\; \sum_{t=1}^{A} \sum_{i \in a_{t}}^{ }\frac{e_{i,t}}{A\cdot |a_{t}| \cdot e_{max}} \right ) \enspace ,
\end{equation}
where $A$ is the number of time steps in which there was at least one robot alive and $a_{t}$ is the set of robots alive at time $t$.

\item[$\mathbf{b_{extra}}$:] The second behaviour characterisation is an extension of $\mathbf{b_{simple}}$. Two more features, which are not directly related to the fitness function, were added to the characterisation: (i)~the average speed of the alive robots throughout the simulation; and (ii)~the average distance of the alive robots to the charging station. The movement of a robot in a given instant is determined by the average wheel speed at that instant. The two additional features are also normalised to the interval $[0,1]$. $\mathbf{b_{extra}}$ is defined as:
\begin{equation} \label{eq:bextra_energy}
\mathbf{b_{extra}} = \left (\mathbf{b_{simple}} \;,\; \sum_{t=1}^{A} \sum_{i \in a_{t}}^{ }\frac{s_{i,t}}{A\cdot |a_{t}| \cdot s_{max}} \;,\; \sum_{t=1}^{A} \sum_{i \in a_{t}}^{ }\frac{d_{i,t}}{A\cdot |a_{t}| \cdot d_{max}} \right ) \enspace ,
\end{equation}
where $s_{i,t}$ and $d_{i,t}$ are the speed of the robot $i$ and its distance to the charging station, respectively, at time $t$. $s_{max}$ is the maximum speed of a robot, and $d_{max}$ is half the length of the diagonal of the arena.
\end{description}

\subsection{Performance Comparison}

Figure~\ref{fig:energy_fitness} depicts the highest fitness scores achieved by novelty search, fitness-based evolution, and random evolution. Random evolution only achieves very low fitness scores ($<0.2$). In fitness-based evolution, the distribution of the highest fitness scores achieved is characteristically wide. Fitness-based evolution achieved close to the maximum fitness score in 10/30 of the evolutionary runs, but failed to evolve any viable solution (where at least one robot consistently survives) in 10/30 of the runs. Bootstrapping proved difficult in fitness-based evolution, as most runs got stuck in low regions of the fitness landscape for a large number of generations.

\begin{figure}[b]
	\centering
	\includegraphics[width=1\textwidth]{pic/energy_performance.pdf}
\caption{Highest fitness scores achieved in the resource sharing task with novelty search with $\mathbf{b_{simple}}$ and $\mathbf{b_{extra}}$ (\emph{NS-simple}, \emph{NS-extra}), fitness-based evolution (\emph{Fit}), and random evolution (\emph{Random}). Left: average fitness value of the highest scoring individual found so far at each generation. The values are averaged over 30 independent evolutionary runs for each method. Right: box-plots of the highest fitness score found in each evolutionary run, for each method. The whiskers extend to the lowest and the highest data point within 1.5 times the interquartile range. Outliers are indicated by circles. The maximum fitness score in practice is about 0.98, which corresponds to all robots surviving until the end of the experiment, while maintaining high levels of energy.}
\label{fig:energy_fitness}
\end{figure}

Novelty search, on the other hand, did not get stuck in local maxima. The result indicates that novelty search was unaffected by deception, and was capable of bootstrapping the evolutionary process. Novelty search with $\mathbf{b_{simple}}$ consistently achieved high fitness scores, with 4--5 robots surviving in the best solutions. The fitness scores achieved with $\mathbf{b_{simple}}$ are significantly higher than those achieved by fitness-based evolution (Mann--Whitney U test, $p$-value $<$ 0.01). With $\mathbf{b_{extra}}$, novelty search failed to match the performance of $\mathbf{b_{simple}}$, and the fitness scores achieved are significantly lower ($p$-value $<$ 0.01). Within 400 generations, \emph{NS-simple} could consistently achieve fitness scores close to the maximum value. However, it should be noted that \emph{NS-extra} and \emph{Fit} do not appear stable at that point, and if more generations were allowed, they might still improve. The differences between the behaviour characterisations and the evolved behaviours are discussed below.

\subsection{Behavioural Diversity}

Through visual inspection of the solutions that achieved the highest fitness scores (above 0.95), we found that all evolutionary methods produced solutions that display similar behaviours. In the successful behaviours, the robots always start by searching for the charging station. Depending on the solution, the robots move in straight lines, in large circles, or in spirals, until the station has been located. The second part of the successful behaviours concerns the coordination of access to the charging station. We observed three different coordination behaviours (see Figure~\ref{fig:behaviours_energy}):

\begin{description}
\item[\textbf{Charge and go away:}] If a robot in the charging station detects another robot approaching, it leaves the station to let the approaching robot recharge. The leaving robot performs what resembles a random walk in the arena and eventually returns to the station to recharge.
\item[\textbf{Charge and surround:}] If a robot in the charging station detects another robot approaching, it leaves the station to let the approaching robot recharge.  The leaving robot begins to circle the charging station, and approaches the station again when its energy level is below a solution-specific threshold.
\item[\textbf{Charge and wait:}] Once a robot is in the charging station, it continues to occupy the station until its energy level is above a solution-specific threshold. When the robot leaves, it moves only a short distance away from the station. Then, the robot remains almost static until its energy level is below a solution-specific threshold, at which point the robot tries to recharge again.
\end{description}

\begin{figure}
\centering 
	\includegraphics[width=1\textwidth]{pic/behaviours_rs_new.pdf}
\caption{The patterns of behaviour corresponding to the highest scoring solutions found by novelty search and fitness-based evolution in the resource sharing task. Each line represents the trajectory of the robot throughout the simulation. The circles indicate initial positions and squares indicate final positions. Videos of the behaviours are available as online supplemental material.}
\label{fig:behaviours_energy}
\end{figure}

\subsubsection{Fitness-based evolution}

As mentioned above, fitness-based evolution did not consistently evolve solutions with high fitness scores (see Figure~\ref{fig:energy_fitness}). In fact, of the 30 runs conducted, 10 runs never evolved solutions with a fitness score much higher than the initial randomly generated population. We analysed the controllers evolved in the runs that only achieved low fitness scores and identified two behaviour patterns:
\begin{itemize}
\item The robots move slowly in small circles. When one of the robots detects the charging station, it moves towards the station, and occupies it till the end of the simulation. As a result, the rest of the swarm dies.
\item All the robots remain almost static from the beginning of the simulation until they run out of energy.
\end{itemize}
Both behaviours represent local maxima in the fitness landscape. In the first case, evolution converges to solutions where only one robot survives at the expense of the rest of the swarm. In the second case, the evolutionary process starts to converge to controllers that reduce the wheel speed to conserve energy. Conserving energy causes the robots to survive longer, thus slightly increasing the fitness score of the controller. However, reducing the wheel speed also decreases the chance that a robot will find the charging station. Once an evolutionary process starts to converge to a local maximum based on energy conservation, it can take many generations to escape from that maximum, or evolution may not escape at all.

\subsubsection{Simple behaviour characterisation}

The highest scoring solutions evolved by novelty search follow the same behaviour patterns as the highest scoring solutions evolved by fitness-based evolution. However, an analysis of the explored behaviour space reveals that novelty search could in fact evolve a greater diversity of solutions (see Figure~\ref{fig:space_simple}). The greater diversity comes from variations of the same behaviour patterns. For instance, if the coordination behaviours use different energy thresholds to trigger entering and leaving the charging station, it will result in different average levels of energy. In such cases, the behaviour patterns may appear similar through visual inspection, but they have distinct behaviour characterisations.

\begin{figure}
\centering
\includegraphics[width=\textwidth]{pic/simple_density}
\caption{Behaviour space exploration with fitness-based evolution (\emph{Fit}) and novelty search with $\mathbf{b_{simple}}$ (\emph{NS-simple}), in all evolutionary runs. The $x$-axis is the average energy level of the robots still alive, the $y$-axis is the number of survivors. Each individual is mapped according to its characterisation. Darker zones indicate that there were more individuals evolved with the behaviour of that zone.}
\label{fig:space_simple}
\end{figure}

\subsubsection{Extra behaviour characterisation}

We adapted the visualisation technique based on Kohonen maps to the 4-dimensional behaviour space created by the $\mathbf{b_{extra}}$ characterisation to analyse the degree of exploration of different behaviour regions. The results in Figure~\ref{fig:extra_som} show how novelty search with the $\mathbf{b_{extra}}$ characterisation explored the behaviour space --- a notable variety of combinations of average energy, movement, and distance to the charging station. However, the behaviour dimension related to the number of surviving robots was the least explored dimension of the behaviour space. In almost all the explored regions, the number of surviving robots was either zero or one. Only a single behaviour region included behaviours in which the whole swarm often survives (highlighted in Figure~\ref{fig:extra_som}), and that region was one of the least explored.

\begin{SCfigure}[1.2]
\centering
\includegraphics[width=0.5\textwidth]{pic/som_nov_extra.pdf}
\caption{Kohonen map representing the explored behaviour space in novelty search with $\mathbf{b_{extra}}$. Each circle represents a behaviour pattern, depicted by the 4 slices of different colour. Each slice represents one component of the behaviour characterisation vector -- the bigger the slice, the higher the value of that component. The darker the background of a circle, the more individuals were evolved with the corresponding behaviour.}
\label{fig:extra_som}
\end{SCfigure}

The relatively low performance of novelty search is caused by the significantly larger behaviour space and the lack of correlation between the behaviour features. Novelty search can freely explore dimensions such as average movement and distance to charging station without any robots surviving. Furthermore, both of these dimensions are intuitively easier to explore than the dimension concerning the number of surviving robots. As such, the opportunistic nature of evolution causes the search to focus less on the surviving robots dimension, often inhibiting the evolution from finding successful solutions to the task. This is in contrast with the correlated features used in the $\mathbf{b_{cmcl}}$ characterisation from the aggregation experiments, in which novelty search performed well despite the enlarged behaviour space (see Section~\ref{sec:combined_aggregation}).

The two-dimensional $\mathbf{b_{simple}}$ characterisation restricts novelty search to explore the two dimensions directly related to the fitness function. Similar solutions in which robots move at different speeds are, for instance, conflated when the $\mathbf{b_{simple}}$ characterisation is used, whereas they are considered different (and potentially novel) when the $\mathbf{b_{extra}}$ characterisation is used. As a consequence, novelty search with the $\mathbf{b_{simple}}$ characterisation focused exploration on the dimensions directly related to the fitness function and had significantly more success in finding solutions with high fitness scores. On the other hand, conflation can be prejudicial to the diversity of solutions that is evolved. For instance, analysing the best solutions evolved with each characterisation, we verified that the behaviour pattern  \emph{Charge and wait} was less common with $\mathbf{b_{simple}}$ when compared to $\mathbf{b_{extra}}$. This behaviour is intuitively one of the most interesting solutions, since it takes advantage of the variable energy spending to extend the life time of the robots. However, it was conflated with $\mathbf{b_{simple}}$, since this characterisation does not consider the speed of the robots.

Our results suggest that caution must be displayed when behaviour characterisations are defined. Dimensions that novelty search may opportunistically explore at the cost of other dimensions that are more crucial for the task should not be included. Alternatively, a number of methods have been proposed that aim to guide exploration in novelty search towards high fitness solutions. In Section~\ref{sec:combination}, we study the application of two such methods to the resource sharing task.

\subsection{Neural Network Complexity}

The experiments with the aggregation task showed that novelty search found successful solutions with less complex neural networks than fitness-based evolution. To determine if the same is true for the resource sharing task, we analysed the complexity of the solutions evolved for this task. Table~\ref{tab:complexity_rs} shows a comparison between the complexity of the solutions found by novelty search with $\mathbf{b_{simple}}$ and fitness-based evolution. The results show that there was a considerable difference between the complexity of the solutions evolved by the two methods. At fitness levels above 0.2, the simpler solutions evolved by novelty search were of significantly lower complexity than the solutions evolved by fitness-based evolution (Mann--Whitney U test, $p$-value $<$ 0.05). Note that the complexity of the networks in the initial populations is 107 (29 neurons and 78 links).

\begin{table}
\caption{Comparison of the least complex solutions evolved by fitness-based evolution (Fit) and novelty search with $\mathbf{b_{simple}}$ (NS). The \emph{Complexity} columns lists the network complexity (sum of number of neurons and number of connections) of the least complex individual with a fitness score above a certain \emph{Fitness level}. The \emph{Generation} column lists the average generation in which the least complex individual was generated. The values are averages of 30 evolutionary runs for each method.}
\centering
\begin{tabular}{r r r r r r r}
\toprule
\multicolumn{ 1}{c}{} & \multicolumn{ 2}{c}{Generation} & \multicolumn{ 2}{c}{Complexity} \\ \cmidrule{ 2- 5}
\multicolumn{ 1}{r}{Fitness level} & \multicolumn{1}{r}{Fit} & \multicolumn{1}{r}{NS} & \multicolumn{1}{r}{Fit} & \multicolumn{1}{r}{NS} \\ \midrule
0.20 &  65 & 21  &  114.90 &  110.50 \\
0.40 &  150 & 69  & 127.30 & 118.40 \\
0.60 &  192 & 135 &  134.20 & 125.90 \\
0.80 & 213 & 192 & 136.90 & 132.40 \\
0.90 &  268 & 239 & 145.70 & 133.90 \\
\bottomrule
\end{tabular}
\label{tab:complexity_rs}
\end{table}

\section{Combining Novelty and Fitness}
\label{sec:combination}

The results of the resource sharing experiments showed that although novelty search finds a broad diversity of behaviours, regions with high-fitness behaviours may never be explored. This issue has also been reported in other studies \citep{stan10a,cuccu11b}, and a common solution is to combine novelty search with fitness-based evolution (see Section~\ref{sec:ns_variants}). The combination is promising, because while novelty search promotes exploration of the behaviour space, fitness-based evolution exploits existing high fitness solutions. In this section, we study the application of \emph{progressive minimal criteria novelty search} (PMCNS) \citep{gomes12} and \emph{linear scalarization} \citep{cuccu11b} to the evolution of solutions for the resource sharing task. We compare the results obtained with PMCNS and with linear scalarization to the results obtained with pure novelty search and with fitness-based evolution.

\subsection{Progressive Minimal Criteria Novelty Search}

PMCNS~\citep{gomes12} is an extension of \emph{minimal criteria novelty search} (MCNS)~\citep{stan10a}. In MCNS, the exploration of the behaviour space is restricted by domain-dependent minimal criteria that evolved individuals must meet in order to be selected for reproduction. The objective of PMCNS is to take advantage of the restrictions on behaviour space exploration provided by MCNS, but without having to define fixed, domain-dependent criteria \emph{a priori}. In PMCNS, a dynamic fitness threshold is used as the minimal criterion: individuals with a fitness score above the threshold meet the criterion. The fitness threshold is progressively increased during the evolutionary process. The idea behind the increasing fitness criterion is to progressively restrict the search space to regions of the behaviour space with higher fitness scores, and thereby avoid that novelty search spends much or all effort on regions of the behaviour space with novel, but low fitness behaviours.

The minimal criterion starts at the theoretical minimum of the fitness score (typically zero), so all controllers initially meet the criterion. In each generation $g$, the new criterion $mc_{g}$ is based on the fitness score $v_{g}$ of the $P$-th percentile of the individuals in the current population, i.e., the fitness score below which $P$ percent of the individuals fall. The parameter $P$ controls the exigency of the minimal criterion (0 -- all individuals meet the criterion, 1 -- only the individual with the highest fitness meets the criterion). Only increases in the minimal criterion are allowed, and in order to smoothen the increase, the minimal criterion from the previous generation is used to determine the criterion for the current generation (Eq.~\ref{eq:pmcns1}). The score used for selection of each individual $i$ in the population is then calculated according to Eq.~\ref{eq:pmcns2}.

\begin{equation}
mc_{g} = mc_{g-1} + \mathrm{max}(0, (v_{g} - mc_{g-1}) \cdot S)
\label{eq:pmcns1}
\end{equation}

\begin{equation}
\mathrm{score}(i) =	
\begin{cases}
	nov_{i} & \text{if $f\!it_{i} \geq mc_{g}$}
	\\
	0 & \text{otherwise}
\end{cases}
\label{eq:pmcns2}
\end{equation}

The variables $nov_{i}$ and $f\!it_{i}$ are the novelty and fitness score of the individual $i$, respectively. The smoothening parameter $S$ controls the speed of the adaptation of the minimal criterion.

% the following is not true: (0 -- no changes at all, 1 -- the value from the previous generation is not considered).

\subsection{Linear Scalarization of Novelty and Fitness Scores}

\citet{cuccu11b} proposed a linear scalarization of novelty and fitness score, as an approach to sustain diversity while improving the performance of traditional fitness-based evolution. Each individual $i$ is evaluated to obtain both fitness score, $f\!it(i)$, and novelty score, $nov(i)$, which after being normalised (Eq.~\ref{eq:blendnorm}) are combined according to Eq.~\ref{eq:blend}.

\begin{equation}
\overline{f\!it}(i)=\frac{f\!it(i)-f\!it_{min}}{f\!it_{max}-f\!it_{min}}, \quad \overline{nov}(i)=\frac{nov(i)-nov_{min}}{nov_{max}-nov_{min}} \enspace ,
\label{eq:blendnorm}
\end{equation}

\begin{equation}
score(i)=(1-\rho)\cdot \overline{f\!it}(i)+\rho \cdot \overline{nov}(i) \enspace .
\label{eq:blend}
\end{equation}

The parameter $\rho$ controls the relative weight of fitness and novelty, and must be specified by the experimenter (usually through trial and error). $f\!it_{min}$ and $nov_{min}$ are respectively the lowest fitness and novelty scores in the current population, and $f\!it_{max}$ and $nov_{max}$ are the corresponding highest scores.

\subsection{Experimental Setup}

We used the experimental setup of the resource sharing task (see Section~\ref{sec:energy_setup}) for this set of experiments. The fitness function and the behaviour characterisations are also the same as we used previous experiments (see Section~\ref{sec:energy_functions}).

For PMCNS, we chose a percentile value of $P=0.50$ (values of 0.25, 0.50 and 0.75 were tested), which corresponds the median value of the fitness scores in the population, and a smoothening parameter of $S=0.25$, which corresponds to relatively slow increases in the minimal criterion value. For linear scalarization, the parameter $\rho$ was set to $\rho=0.75$ (values of 0.25, 0.50 and 0.75 were tested), which means that the score of each individual is composed of 75\% of the novelty score and 25\% of the fitness score. A discussion of the parameter values for linear scalarization and PMCNS can be found in \citep{cuccu11b} and in \citep{gomes12}, respectively.

\subsection{Results}

We ran experiments with both behaviour characterisations $\mathbf{b_{simple}}$ and $\mathbf{b_{extra}}$ to evaluate the performance of PMCNS and linear scalarization in behaviour spaces of different dimensionality. The resulting fitness trajectories are shown in Figure~\ref{fig:mix_fitness_trajectories}.

\begin{figure}[b]
\centering
\includegraphics[width=\textwidth]{pic/mix_performance.pdf}
\caption{Highest fitness scores obtained in the resource sharing task with fitness-based evolution (\emph{Fit}), and novelty-based evolutionary techniques with $\mathbf{b_{simple}}$ (\emph{PMCNS-s}, \emph{Scalarization-s}, \emph{NS-s}) and $\mathbf{b_{extra}}$ (\emph{PMCNS-e}, \emph{Scalarization-e}, \emph{NS-e}). Top: average fitness value of the highest scoring individual found so far at each generation. The values are averaged over 30 independent evolutionary runs for each method. Bottom: box-plots of the highest fitness score found in each evolutionary run, for each method. The whiskers extend to the lowest and the highest data point within 1.5 times the interquartile range. Outliers are indicated by circles.}
\label{fig:mix_fitness_trajectories}
\end{figure}

The results show that both PMCNS and linear scalarization are more effective than pure novelty search when using $\mathbf{b_{extra}}$ (Mann--Whitney U test, $p$-value~$<$~0.01), and achieve similar fitness scores when using $\mathbf{b_{simple}}$. With the $\mathbf{b_{extra}}$ characterisation, pure novelty search fails to reach high fitness scores, as evolution tends to focus the exploration on behaviour dimensions that are not directly relevant for solving the task. The results suggest that PMCNS and linear scalarization can overcome this issue, and that the inclusion of a fitness component in the selection criteria helps to guide the evolutionary process towards solutions with high fitness. It should also be noted that PMCNS and linear scalarization do not appear to be affected by the deceptiveness of the fitness function since they could consistently achieve high fitness scores.

% space exploration ns/blend/pmcns (simple)
\begin{figure}[b]
\centering
\includegraphics[width=\textwidth]{pic/mix_simple_exploration}
\caption{Behaviour space exploration for each variant of novelty search, with the $\mathbf{b_{simple}}$ characterisation, in all evolutionary runs. The $x$-axis is the average energy level of the robots still alive, the $y$-axis is the number of survivors. Each individual is mapped according to its behaviour. Darker zones indicate that there were more individuals evolved with the behaviour of that zone.}
\label{fig:space_exploration_simple}
\end{figure}

% space exploration ns/blend/pmcns (extra) 
\begin{figure}[b]
\centering
\includegraphics[width=\textwidth]{pic/mix_extra_exploration}
\caption{Kohonen maps representing the explored behaviour space with each variant of novelty search, with the $\mathbf{b_{extra}}$ characterisation. Each circle represents a behaviour pattern, depicted by the 4 slices of different colour. Each slice represents one component of the behaviour characterisation -- the bigger the slice, the greater the value of that component. The darker the background of a circle is, the more individuals were evolved with the corresponding behaviour. The behaviour patterns with higher fitness scores are indicated in the upper right corner of each map.}
\label{fig:space_exploration_extra}
\end{figure}

% behaviour space comparison ns vs mixes
Figure~\ref{fig:space_exploration_simple} and Figure~\ref{fig:space_exploration_extra} depict the behaviour space exploration with $\mathbf{b_{simple}}$ and $\mathbf{b_{extra}}$, respectively. An analysis of the behaviour space exploration shows that PMCNS and linear scalarization have a greater focus on regions associated with high fitness scores compared to pure novelty search. The coverage of the behaviour space was not negatively affected in PMCNS and linear scalarization. In fact, the two methods found the same broad range of behaviours as pure novelty search. The difference between pure novelty search and novelty search combined with fitness-based search is the amount of exploration done in each behaviour region: PMCNS and linear scalarization focused less on the low fitness behaviours (few or no robots surviving till the end of the simulation) and more on the high-fitness behaviours (high number of surviving robots). As a result, the PMCNS and linear scalarization could achieve solutions with significantly higher fitness scores than the best solutions evolved with pure novelty search.

\section{Conclusions}
In this paper, we set out to address the problem of multi-tasking robots in multi-robot tasks. 
%A fundamental limitation of existing multi-robot systems was addressed by the removal of a restrictive assumption that was often made--robots are single-tasking.
%Our method allowed coalitions to overlap thus enabling multi-tasking robots. 
We observed that the key underlying challenge was to reason about the physical constraints that could be synergistically satisfied.
%which directly affected the feasibility of multi-tasking.
To address the challenge, we developed our method based on the information invariant theory and modeled constraints as information instances. 
%This allowed us to reason about the relationships between constraints by reasoning about those between information requirements. 
Thereby, a formal and general framework to achieve multi-tasking robots was developed. 
We showed that our algorithm was sound and complete under our problem settings. 
%Our method was integrated with a simple greedy heuristic for task allocation.
Simulation  results  were  provided  to  show  the  effectiveness  of  our approach under resource-constrained situations and in handling challenging situations. % in a multi-UAV simulator. 

% The idea of multi-tasking is attractive in many ways. 
% Humans are living in multi-tasking environments--at any point of time, 
% we are optimizing for more than one task. 
% Multi-task often leads to more efficient task performance since it allows us to exploit task synergies. 
% The work presented in this paper takes us one step forward in realizing multi-tasking robots. 
% In particular, we started looking at the feasibility of multi-tasking. 
% There are many potential directions to pursue along this direction. First, several limitations are present in the current approach. 
% For example, although our method guarantees that there exists a physical configuration that satisfies all the constraints, it does not explicitly take the environmental influence into account. For example, a narrow corridor may prevent a robot formation from passing through, even though all the constraints for the formation do not introduce any conflicts. In this sense, our work should better be characterized as establishing a necessary condition for multi-tasking. Also, our method is mainly focused on the ``{\it planning}'' phase and hence does not address how the robots reach the desired configuration and maintain the constraints. These issues are assumed to be handled by the execution layer.

% More generally, the question of how to execute the tasks with overlapping coalitions is not addressed in this work. 
% As we already discussed, executing individual tasks with non-overlapping coalitions is straightforward but task synergies impose additional requirements on the task execution: how should the robots that are assigned multiple tasks execute them? Should they consider them in a prioritized strategy~\cite{van2005prioritized}? Or should they combine the different tasks in a way that is similar to motor schemas~\cite{arkin2}. 
% Communication requirements for maintaining the constraints must also be taken into account. How should the robots optimize their communication to improve the task performance? 

% The stringency of the physical constraints is another interesting question. It may be desirable to relax the constraints in certain situations (e.g., due to environmental influences). In such cases, it may be important to consider the problem where the constraints are least violated~\cite{kim2012revision}, or specify task constraints in different ways to increase the diversity of the configurations~\cite{srivastava2007domain} so as to make it robust to different environments. 

\section*{Acknowledgements}

This research has been supported by Funda\c{c}\~{a}o para a Ci\^{e}ncia e a Tecnologia (FCT) grants PTDC/EEACRO/104658/2008, PEst-OE/EEI/LA0008/2011, and SFRH/BD/89095/2012. 

\section*{Supplemental Material}

Videos of the behaviours are available as online supplemental material and at the following URL: \url{http://home.iscte-iul.pt/~alcen/si2013}. The source code of the software used in the experiments can be found at: \url{https://github.com/jorgemcgomes/evosimbad}.


% Biblio
\bibliographystyle{abbrvnat}
\bibliography{journal}

\end{document}
