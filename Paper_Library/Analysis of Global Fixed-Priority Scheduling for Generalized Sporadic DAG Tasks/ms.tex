\documentclass[sigconf]{acmart}

% Remove the "ACM reference format" paragraph at the beginning
\settopmatter{printacmref=false}
\renewcommand\footnotetextcopyrightpermission[1]{} % removes footnote with conference information in first column
\pagestyle{plain} % removes running headers

\usepackage{booktabs} % For formal tables
\usepackage{amsmath}
\usepackage{amsthm}
\usepackage{amssymb}
\usepackage{graphicx}
\usepackage{algorithm}
\usepackage{algorithmicx}
\usepackage{algpseudocode}
\usepackage{subcaption}
\usepackage{url}
\usepackage{verbatim}

\setcopyright{none}
\acmDOI{}
\acmISBN{}

\theoremstyle{theorem}
\newtheorem{constraint}{Constraint}


\begin{document}
\title{Analysis of Global Fixed-Priority Scheduling for Generalized Sporadic DAG Tasks}

\author{Son Dinh, Christopher Gill, Kunal Agrawal}
\affiliation{%
  \institution{Washington University in Saint Louis, \\
  Department of Computer Science and Engineering}
}
\email{sonndinh, cdgill, kunal@wustl.edu}


\begin{abstract}
We consider global fixed-priority (G-FP) scheduling of parallel tasks, 
in which each task is represented as a directed acyclic graph (DAG). 
We summarize and highlight limitations of the state-of-the-art analyses for G-FP and 
propose a novel technique for bounding interfering workload, which can be applied directly to generalized DAG tasks. 
Our technique works by constructing optimization problems for which 
the optimal solution values serve as safe and tight upper bounds for interfering workloads. 
Using the proposed workload bounding technique, we derive a response-time analysis 
and show that it improves upon state-of-the-art analysis techniques for G-FP scheduling. 
\end{abstract}

\maketitle



\IEEEPARstart{T}{wo} %
main challenges in the deployment of large-scale swarms are the localization and coordination of vehicles.
Localization methods that rely on external infrastructure 
(e.g., GPS) 
are prone to systematic errors (e.g., multipath effect)
and may not always be available.
Coordination strategies that are centralized can deconflict motion plans to prevent collisions and gridlock, but introduce a single point of failure and are difficult to scale in swarm size due to communication bandwidth limitations.

This paper presents a unified formation flying pipeline for unmanned aerial vehicles (UAVs).
Our pipeline uses \textit{onboard} sensors for localization, which eliminate the need for external positioning systems, and \textit{distributed} techniques for coordination, which enable each vehicle to make decisions independently while communicating their state to a subset of the team.
For \textit{localization}, we use an off-the-shelf commercial visual inertial odometry (VIO) package \cite{VIO}
that fuses inertial measurement unit (IMU) and downward-facing monocular camera measurements to estimate changes in the vehicle pose.
\edit{For \textit{coordination}, we present distributed formation control and task assignment strategies that run onboard the vehicles, do not rely on a common reference frame, and use vehicle-to-vehicle communication.} 
Key features of our formation control strategy include scalability to a large number of vehicles and robustness to disturbances.
The latter is crucial for reaching the desired formations with sensing imperfections.
Our task assignment strategy uses an auction-based algorithm to guarantee conflict-free assignments.
This algorithm can deconflict vehicle gridlocks resulting from distributed collision avoidance (type 3 deadlock~\cite{Wang2017}) and is well-suited for vehicles with limited computational capability and low-bandwidth communication. 


\begin{figure}[t!]
	\begin{center}
		\includegraphics[trim =0mm 10mm 0mm 0mm, clip, width=\columnwidth]{Figs/slanted_plane.png}	
		\caption{
		Six multirotors in a slanted plane formation.
		Vehicles communicate with each other, make distributed decisions onboard, and use VIO for localization.}
		\label{fig:slantedplane}
	\end{center}
\end{figure}


\subsection{Contributions}

This research extends our previous work on UAV formations~\cite{Fathian2019} and presents a unified pipeline consisting of \textit{onboard localization} and \textit{distributed coordination}.
The three main contributions of this work are:
\begin{enumerate}
    \item \edit{scalable formulation of control design suitable for
    onboard sensing without a common reference frame;}
    \item algorithms for deconfliction via \edit{distributed} task assignment of vehicles to desired formation points;    
    \item simulation- and hardware-ready open-source pipeline.
\end{enumerate}
\edit{Our pipeline is tested in hardware with six multirotors (see Fig.~\ref{fig:slantedplane}), and 
to our knowledge is the first demonstration of formation flying that does not rely on external sensing, fiducial markers for localization, a common reference frame, or a centralized base station for coordination.}
The only requirements for the presented pipeline are that the vehicles can communicate, can find the transformation between their VIO start frames, and the environment is sufficiently textured---a standard assumption for VIO systems.
As such, this framework paves the way for future, real-world deployments of aerial vehicle swarms in large numbers and without requiring external localization infrastructure.


\begin{figure} [t!]
\centering
	\begin{subfigure}[b]{0.32\columnwidth}
	   %
	    \includegraphics[width=0.8\textwidth,left]{Figs/Frames2_full.pdf}
	    \caption{\scriptsize full alignment}
	    \label{fig:frame-a}
	\end{subfigure}
	\begin{subfigure}[b]{0.32\columnwidth}
	    \includegraphics[width=0.8\textwidth,center]{Figs/Frames2_orientation.pdf}
	    \caption{\scriptsize orientation alignment}
	    \label{fig:frame-b}
	\end{subfigure}
	\begin{subfigure}[b]{0.32\columnwidth}
	    \includegraphics[width=0.8\textwidth,right]{Figs/Frames2_none.pdf}
	    \caption{\scriptsize no alignment}
        \label{fig:frame-c}
	\end{subfigure}
\caption{\edit{Required alignment of UAV frames in existing swarm strategies: (a) the most restrictive case requiring a common reference frame, i.e., orientation and origin of the frames must be aligned; (b) only the orientation of the frames must be aligned; (c) no alignment restrictions (this work).}}
	\label{fig:Frames}
\end{figure}




\subsection{Related Work}

Existing aerial swarms can be grouped based on the coordination (centralized vs.\ distributed) and localization (external vs.\ onboard) methods used. 
\edit{It is further crucial to distinguish these methods based on the level of alignment required for the vehicle coordinate frames; see Fig.~\ref{fig:Frames}.} 
 
\edit{
Works with \textit{centralized} coordination and \textit{external} localization include~\cite{Preiss2017, Honig2018, Du2019}, which are based on lightweight UAVs with limited onboard computational capability and therefore rely on an external motion capture system and a base station.
Works with \textit{distributed} coordination and \textit{external} localization include \cite{wilson2020robotarium}, \cite{enright2004spheres}, where robots execute distributed controls  based on external localization by motion capture and ultrasonic beacons, respectively.
Works with \textit{centralized} coordination and \textit{onboard} localization include~\cite{Forster2013}, \cite{Loianno2016}, which use a ground station for task assignment among vehicles.
In \cite{Weinstein2018}, formation flying based on VIO is demonstrated, where motion planning and assignment are run on a base station to ensure collision-free trajectories.
The coordination strategies used in aforementioned works require a \textit{common reference frame} (Fig.~\ref{fig:frame-a}).
}


\edit{
Despite the large body of work on formation control~\cite{Oh2015}, and the variety of onboard sensing solutions for localization (e.g., VIO~\cite{Delmerico2018}), few frameworks demonstrated formation flying with \textit{distributed} coordination and \textit{onboard} localization.
A key reason is reliance of many distributed control and assignment algorithms on aligned frames (Fig.~\ref{fig:frame-a}, \ref{fig:frame-b}), which require computation-expensive and/or communication-intensive synchronization/consensus steps for frame alignment.
Equally important, dependence on alignment in existing methods \cite{Wang2017,Turpin2014, van2011reciprocal, morgan2016swarm} diminishes robustness to inherent noise and unobservable errors that cannot be corrected (e.g., disparities between the actual and estimated body frame \textit{orientation} caused by VIO drift).
Leveraging coordination methods that are \textit{robust to misaligned frames} is hence crucial and a focus of this work. 
}






\edit{
Examples of other pipelines with distributed coordination and onboard localization include \cite{Montijano2016,Tron2016}.
Both works demonstrated formation flying on three UAVs, required information from an external motion capture system due to hardware limitations, did not incorporate collision avoidance, and required frame alignment.
}
\edittwo{Note that while~\cite{Montijano2016,Tron2016} can achieve formations with arbitrary headings as illustrated in Fig.~\ref{fig:frame-c}, knowledge of relative orientations is still required; therefore, they belong to the category of Fig.~\ref{fig:frame-b}.}






\if 0

\r{
decentralized coordination setting combined with VIO:
D-CAPT [26]~\cite{}:
ORCA ~\cite{}: 
CBF [2]~\cite{} :
[A]
}

\r{Robusteness in coordination,  with compounded noise/latency, which would eventually break (b).\\


some existing algorithm might as well
work in a similar fully decentralized setting, when combined with VIO
as proposed here. For example, D-CAPT [26], ORCA, CBF [2] might also be
useful for such a task and are computationally even more efficient than
the proposed approach. \\

R2:  onboard sensing for localization ->
 Finally, the related work section only
focuses on this aspect of the pipeline, discussing how many formation papers include
onboard localization but barely sells the advantages of the coordination module (the actual
proposal of the paper) against other competitors such as [26] or [A] or to mention similar
coordination pipelines. \\


Given a solution to this problem, the controller in Section III seems unnecessary, each drone
has a target position and can use a local controller with collision avoidance that drives it to
that position. Note that such controllers exists in the literature (e.g., RVO in any of its
multi-agent variantes), they are distributed in nature and only require local sensing.


}

\fi
\section{Other Related Work}
\label{sec:related}
There has been a renewed interest in integrated task and motion planning algorithms. Most research in this direction has been focused on deterministic environments~\citep{cambon09_asymov,plaku10_sampling,dornhege12_semantic,kaelbling11_hierarchical,garrett15_ffrob,dantam16_incremental}. \cite{kaelbling13_hpnPOMDP} consider  a partially observable formulation of the problem. Their approach utilizes regression modules on belief fluents to develop a regression-based solution algorithm. \cite{sucan12_tmp_mdp} use an explicit multigraph to represent the plan or policy for which motion planning refinements are desired.  \cite{hadfield15_modular} address problems where the high-level formulation is deterministic and the low-level is determinized using most likely observations. In contrast, our approach employs abstraction to bridge MDP solvers and motion planners to solve problems where the high-level model is stochastic. In addition, the transitions in our MDP formulation depend on properties of the refined motion planning trajectories (e.g., battery usage). 

Principles of abstraction in MDPs have been well studied~\citep{hostetler14_state,bai16_markovian,li06_abstractMDP,singh95_abstractRL}. However, these directions of work assume that the full, unabstracted MDP can be efficiently expressed as a discrete MDP. \cite{marecki06_cmdp} consider continuous time MDPs with finite sets of states and actions. In contrast, our focus is on MDPs with high-dimensional, uncountable state and action spaces. Recent work on deep reinforcement learning  (e.g., \citep{hausknecht16_iclr,mnih15_drl}) presents  approaches for using deep neural networks in conjunction  with reinforcement learning to solve MDPs with continuous state spaces. We believe that these approaches can be used in a complementary fashion with our proposed approach. They could be used to learn maneuvers spanning shorter-time horizons, while our approach could be used to efficiently abstract their representations and to use them as actions or macros in longer-horizon tasks. 

Efforts towards improved representation languages are orthogonal to our contributions~\citep{fox02_pddl+}. The fundamental computational complexity results indicating growth in complexity with increasing sizes of state spaces, branching factors, and time horizons remain true regardless of the solution approach taken. It is unlikely that a uniformly precise model, a simulator at the level of precision of individual atoms, or even circuit diagrams of every component used by the agent will help it solve the kind of complex tasks on which humans would appreciate assistance. On the other hand, not using any model at all would result in dangerous agents that would not be able to safely evaluate the possible outcomes of their actions. Our results show that these divides can be bridged using hierarchical modeling and solution approaches that simplify the representational requirements and offer computational advantages that could make autonomous robots feasible in the real world. 


\section{Dynamic Task Allocation Mechanism\label{sec:task-allocation}}

The dynamic task allocation scenario we study considers a world
populated with tasks of $T$ different types and robots that are
equally capable of performing each task but can only be assigned
to one type at any given time. For example, the tasks could be
targets of different priority that have to be tracked, different
types of explosives that need to be located, etc. Additionally, a
robot cannot be idle --- each robot is always performing a task at
any given time. We introduce the notion of a robot state as a
shorthand for the type of task the robot is assigned to service. A
robot may switch its state according to its control policy when it
determines it is appropriate to do so. However, needlessly
switching tasks is to be avoided, since in physical robot systems,
this can involve complex physical movement that requires time to
perform.


The purpose of task allocation is to assign robots to tasks in a
way that will enhance the performance of the system, which
typically means reducing the overall execution time. Thus, if all
tasks take an equal amount of time to complete, in the best
allocation, the fraction of robots in state $i$ will be equal to
the fraction of tasks of type $i$. In general, however, the
desired allocation could take other forms ---  for example, it
could be related to the relative reward or cost of completing each
task type --- without change to our approach. In the dynamic task
allocation scenario, the number of tasks and the number of
available robots are allowed to change over time, for example, by
adding new tasks, deploying new robots, or removing malfunctioning
robots.


The challenge faced by the designer is to devise a mechanism that
will lead to a desired task allocation in a distributed MRS even
as the environment changes. The challenge is made even more
difficult by the fact that robots have limited sensing
capabilities, do not directly communicate with other robots, and
therefore, cannot acquire global information about the state of
the world, the initial or current number of tasks (total or by
type), or the initial or current number of robots (total or by
assigned type). Instead, robots can sample the world (assumed to
be finite) --- for example, by moving around and making local
observations of the environment. We assume that robots are able to
observe tasks and discriminate their types. They may also be able
to observe and discriminate the task states of other robots.

One way to give the robot an ability to respond to environmental
changes (including actions of other robots) is to give a robot an
internal state where it can store its knowledge of the environment
as captured by its observations~\cite{Jones03iros,Lerman03aamas}.
The observations are stored in a rolling history window of finite
length, with new observations replacing the oldest ones. The robot
consults these observations periodically and updates its task
state according to some transition function specified by the
designer. In an earlier work we
showed~\cite{Jones03iros,Lerman03iros} that this simple dynamic
task allocation mechanism leads to the desired task allocation in
a multi-foraging scenario.


In the following sections we present a mathematical model of
dynamic task allocation and study the role that transition
function and the number of observations (history length) play in
the performance of a multi-foraging MRS. In
\secref{sec:pucksonly}, we present a model of a simple scenario in
which robots base their decisions to change state solely on
observations of tasks in the environment. We study the simplest
form of the transition function, in which the probability to
change state to some type is proportional to the fraction of
existing tasks of that type. In \secref{sec:results1} we compare
theoretical predictions with no adjustable parameters to
experimental data and find excellent agreement. In
\secref{sec:phenomenological} we examine the more complex scenario
where the robots base their decisions to change task state on the
observations of both existing task types and task states of other
robots. In \secref{sec:results2} we study the consequences of the
choice of the transition function and history length on the system
behavior and find good agreement with the experimental data.

\section{Related work}\label{background} 
Task allocation is a well-studied problem, posing ongoing challenges in various computing environments \cite{Stavrinides2019, Genez2020, Jayanetti2022, Kanbar2022, Kritikakou2022, Peixoto2022, Mo2023}.
However, previous related research efforts do not consider the edge/hub/cloud architecture, nor all of the parameters investigated in this work. This is demonstrated in \cref{table:comparison}, which summarizes our qualitative comparison with relevant state-of-the-art approaches. 
The comparison is made with respect to the objectives and parameters considered in this work, the applicability of each approach to applications comprising multiple tasks with precedence relationships among them (i.e., applications with a task flow graph structure), as well as the optimality of the solution provided by each method.
An overview of the related literature, as well as a comparison with our preliminary research, are provided in the remainder of this section.


\begin{table*}[!ht]
\centering
\caption{Qualitative comparison of this work with relevant research efforts.}
\label{table:comparison}
\footnotesize
\resizebox{0.85\textwidth}{!}{
    \begin{tabular}{@{\extracolsep{4pt}}lcccccccccc@{}} 
        \toprule
        \multirow{3}{*}{Reference} & \multicolumn{2}{c}{Objectives} & \multicolumn{6}{c}{Considered Parameters} & \multirow{2}{*}{Task Flow} & \multirow{2}{*}{Optimal}\\
         \cline{2-3}   \cline{4-9} 
        & Latency & Energy & Comp. & Comp. & Comm. &  Comm. &  \multirow{2}{*}{Memory} & \multirow{2}{*}{Storage} & \multirow{2}{*}{Graph} & \multirow{2}{*}{Solution}\\
        & Min. & Min. & Latency & Energy & Latency & Energy & & & & \\
        
        \hline
        %-- Latency Minimization References
        %                               Latency         Energy       Comp.          Comp.        Comm.        Comm.        Memory        Storage       TFG           Optimal Solution
        \cite{Alfakih2021}              & \checkmark    & -           & \checkmark  & -           & -          & -          & \checkmark  & \checkmark & -          & -           \\
        \cite{Guevara2022}              & \checkmark    & -           & \checkmark  & -           & \checkmark & -          & \checkmark  & \checkmark & \checkmark & -           \\
        \cite{Weikert2022}              & \checkmark    & -           & \checkmark  & -           & \checkmark & \checkmark & \checkmark  & -          & \checkmark & -           \\
        \cite{Lai2022}                  & \checkmark    & -           & \checkmark  & -           & \checkmark & -          & \checkmark  & \checkmark & -          & -           \\
        \cite{Barijough2019}            & \checkmark    & -           & \checkmark   & -          & \checkmark & \checkmark & -           & -          & \checkmark & \checkmark  \\
        \cite{Tang2022}                 & \checkmark    & -           & \checkmark   & -          & \checkmark & -          & -           & \checkmark & -          & \checkmark  \\
        \cite{Kuang2021}                & \checkmark    & -           & \checkmark   & \checkmark & \checkmark & \checkmark & -           & -          & -          & -           \\
        
        %-- Energy Minimization References
        %                               Latency         Energy       Comp.          Comp.        Comm.        Comm.        Memory      Storage         TFG           Optimal Solution                  
        \cite{Avgeris2022}              & -             & \checkmark  & \checkmark  & \checkmark  &\checkmark & -           & -           & -          & -          & \checkmark \\
        \cite{Khalil2018}               & -             & \checkmark  & -           & \checkmark  & -         & \checkmark  & -           & -          & -          & -          \\
        \cite{Kritikakou2023}           & -             & \checkmark  & \checkmark  & \checkmark  & -         & -           & -           & -          & \checkmark & -          \\
        \cite{Hu2020}                   & -             & \checkmark & \checkmark   & \checkmark & \checkmark & \checkmark & -          & -            & -          & -          \\
        \cite{Azizi2022}                & -             & \checkmark & \checkmark   & \checkmark & \checkmark & -          & -          & -            & -          & -          \\
        \cite{Li2022}                   & -             & \checkmark & \checkmark   & \checkmark & \checkmark & \checkmark & -          & -            & -          & -          \\

        %-- Latency and Energy Minimization References
        %                               Latency         Energy       Comp.          Comp.        Comm.        Comm.        Memory      Storage         TFG          Optimal Solution          
        \cite{Zhang2021}                & \checkmark    & \checkmark & \checkmark   & \checkmark & \checkmark & \checkmark & -          & -            & -          & -          \\
        \cite{Dinh2017}                 & \checkmark    & \checkmark & \checkmark   & \checkmark & \checkmark & \checkmark & -          & -            & -          & -          \\        
        \cite{Tong2023}                 & \checkmark    & \checkmark & \checkmark   & \checkmark & \checkmark & \checkmark & -          & -            & -          & -          \\

        
        This work & \checkmark & \checkmark & \checkmark & \checkmark & \checkmark & \checkmark & \checkmark & \checkmark & \checkmark  & \checkmark \\
        \bottomrule
    \end{tabular}
}
%\vspace{-3mm}
\end{table*}



\subsection{Latency minimization}
A number of works on task allocation in edge computing and multi-tier environments have a primary focus on latency minimization.
For instance, Alfakih et al. \cite{Alfakih2021} explore the minimization of the computational latency of task execution in an edge computing system, based on an accelerated particle swarm optimization algorithm combined with a dynamic programming approach.
Guevara et al. \cite{Guevara2022} present a reinforcement learning-based resource allocation technique for minimizing the total execution time of tasks in a fog-cloud environment.
On the other hand, Weikert et al. \cite{Weikert2022} propose an algorithm for task allocation in an IoT platform, aiming to optimize the overall latency.
Furthermore, Lai et al. \cite{Lai2022} propose an online Lyapunov optimization-based method to tackle the problem of allocating user tasks in an edge computing environment, utilizing a stochastic approach. 
Barijough et al. \cite{Barijough2019} introduce a technique for allocating real-time streaming applications under latency and quality constraints.  
Tang et al. \cite{Tang2022} propose a framework for managing the physical resources of the edge and cloud layers, so that the response time is minimized and the system throughput is improved.
Moreover, Kuang et al. \cite{Kuang2021} present an iterative algorithm based on Lagrangian dual decomposition in order to minimize latency in an edge computing system. 


\subsection{Energy consumption minimization}
Several studies are focused on task allocation strategies aiming to reduce the total energy consumption.
Specifically, Avgeris et al. \cite{Avgeris2022} propose a resource allocation technique based on mixed integer linear programming in order to minimize the energy consumption of edge servers.
Within this context, Khalil et al. \cite{Khalil2018} present a framework for energy-efficient task allocation in an IoT environment, utilizing evolutionary-based meta-heuristics. 
Cui et al. \cite{Kritikakou2023} propose a heuristic algorithm for minimizing the total energy consumption of a platform comprising homogeneous processors, utilizing dynamic voltage and frequency scaling (DVFS).  
On the other hand, Hu et al. \cite{Hu2020} introduce a game-theoretic approach for task allocation in an edge computing environment to minimize the system energy consumption within an acceptable delay range.
Similarly, Azizi et al. \cite{Azizi2022} propose two priority-aware semi-greedy algorithms for allocating  IoT tasks in a heterogeneous fog platform, so that the total energy consumption is optimized, while meeting the deadline of each task.
Furthermore, Li et al. \cite{Li2022} examine a two-stage iterative algorithm, in which the resource allocation problem is decomposed into two sub-problems to obtain a suboptimal solution. 




\subsection{Latency and energy consumption minimization}
On the other hand, certain related works consider both optimization objectives, the minimization of latency and energy consumption.
For instance, Zhang et al. \cite{Zhang2021} present a game theory-based scheme for task allocation in a UAV-assisted edge computing environment. The goal of the proposed approach is to minimize the weighted latency and energy consumption of the system, considering resource allocation constraints.
Dinh et al. \cite{Dinh2017} propose a semi-definite relaxation-based optimization framework for allocating tasks in an edge architecture. The particular framework aims to minimize the total latency of the tasks, as well as the total energy consumption of the system.
On the other hand, Tong et al. \cite{Tong2023} present a latency and energy-aware Stackelberg game-based task allocation strategy, considering an edge device with limited computational resources.


\subsection{Our approach vs. state-of-the-art}
Overall, none of the aforementioned research efforts considers the specific edge/hub/cloud architecture examined in this work. 
Furthermore, some approaches do not take into account the energy required for the execution of the tasks \cite{Alfakih2021, Guevara2022, Weikert2022, Lai2022, Barijough2019, Tang2022} or the energy consumed for inter-task communication \cite{Alfakih2021, Guevara2022, Lai2022, Tang2022, Avgeris2022, Kritikakou2023, Azizi2022}. 
The majority of the related studies consider devices with unlimited resources, such as memory \cite{Barijough2019, Tang2022, Kuang2021, Avgeris2022, Khalil2018, Kritikakou2023, Hu2020, Azizi2022, Li2022} and storage \cite{Weikert2022, Barijough2019, Kuang2021, Avgeris2022, Khalil2018, Kritikakou2023, Hu2020, Azizi2022, Li2022}, an assumption that is not realistic, especially in the case of resource-limited devices at the edge of the network.     
Moreover, several approaches are only applicable to single-task applications \cite{Alfakih2021, Lai2022, Tang2022, Kuang2021, Avgeris2022, Khalil2018, Hu2020, Azizi2022, Li2022} or cannot provide an optimal solution to each of the objectives considered in this work \cite{Alfakih2021, Guevara2022, Weikert2022, Lai2022, Kuang2021, Khalil2018, Kritikakou2023, Hu2020, Azizi2022, Li2022}.

Related studies that are closer to ours \cite{Zhang2021, Dinh2017, Tong2023}, even though they consider both the latency and energy aspects of the problem, do not take into account the memory and storage limitations of the devices. Furthermore, they cannot be applied to applications with precedence relationships among their tasks, and can only provide suboptimal solutions.
Hence, our proposed approach aims to fill these gaps, by incorporating all of the important parameters that characterize an edge/hub/cloud environment, providing an optimal allocation for a task flow graph. 


\begin{figure*}[t]
    \centering
    \includegraphics[width=.85\textwidth]{coins_journal_tfg_to_etfg_v6.1.1.pdf}
    \caption{Overview of proposed optimization framework. The task flow graph transformation, including the encapsulated energy model, is described in \cref{extended,subsec:energyModel}. The formulation of the optimization problem is presented in \cref{subsec:optimization}.}
    \label{flow}
    %\vspace{-3mm}
\end{figure*}


\subsection{Comparison with our preliminary research}
The foundational concepts of this work were first presented in a preliminary form in \cite{Kouloumpris2019}.
Below, we outline the main differences and contributions of the current study with respect to our preliminary research:
\begin{enumerate}
    \item We streamlined and enhanced the mathematical representation of all aspects of the proposed approach, from the description of the task flow graph transformation to the modeling of the optimization problem.
    
    \item We extended our optimization framework to consider a new objective for the minimization of overall energy consumption (in addition to the latency objective), based on an improved energy model.
     
    \item We developed suitable synthetic benchmarks to further validate and evaluate the efficiency and scalability of our framework, by extending our transformation method to randomly generated task flow graphs.
    
    \item We conducted extensive experimentation with alternative configurations of different devices, for both the real-world use-case scenario and the synthetic benchmarks.  
\end{enumerate}

\section{The State-of-the-Art Analysis for G-FP}
\label{sec:sota}

\begin{figure}[h!]
\centering
\includegraphics[width=\linewidth]{figures/melani.pdf}
\caption{Workload generated by an interfering task $\tau_i$ in Melani et al.~\cite{melani2015response}.}
\label{fig:melani}
\end{figure}

Melani et al.~\cite{melani2015response} proposed a response-time analysis for G-FP scheduling 
of conditional DAG tasks that may contain conditional vertices, for modeling conditional constructs 
such as \texttt{if-then-else} statements. 
They bounded the interfering workload by assuming that jobs of the interfering task execute perfectly 
in parallel on all $m$ processors. Their bound for the interfering workload is computed as follows. 
$$W_i(\Delta) = \Big\lfloor\frac{\Delta+R_i-C_i/m}{T_i} \Big\rfloor C_i + 
\min\big\{ C_i, m((\Delta+R_i-C_i/m)\mod T_i) \big\}.$$
Figure~\ref{fig:melani} illustrates the workload computation for an interfering task $\tau_i$ given 
in~\cite{melani2015response}. As shown in this figure, both carry-in and carry-out jobs are assumed to 
execute with perfect parallelism upon $m$ processors. Thus their workload contributions in the considered 
window are maximized. This assumption simplifies the workload computation as it ignores the 
internal DAG structures of the interfering tasks. However, assuming that DAG tasks have such 
abundant parallelism is likely unrealistic and thus makes the analysis pessimistic. 

Fonseca et al.~\cite{fonseca2017improved} later considered a task model similar to the one in this paper 
and proposed a method to improve the bounds for carry-in and carry-out 
workloads by explicitly considering the DAGs. The carry-in workload was bounded using a 
\emph{hypothetical schedule} for the carry-in job, in which the carry-in job can use as many processors 
as it needs to fully exploit its parallelism. They proved that the carry-in workload of the hypothetical 
schedule is maximized when: (i) the hypothetical schedule's completion time is aligned with the worst-case 
completion time of the interfering task, (ii) every subtask in the hypothetical schedule starts executing 
as soon as all of its predecessors finish, and (iii) every subtask in the hypothetical 
schedule executes for its full WCET. Figure~\ref{fig:workload} shows the hypothetical schedule 
of the carry-in job for the task in Figure~\ref{fig:example_task}. In this paper, we adopt their method for 
computing carry-in workload. In particular, the carry-in workload of task $\tau_i$ with a carry-in window of 
length $\Delta_i^{CI}$, i.e., from the start of the problem window to the completion time of the carry-in job 
(see Figure~\ref{fig:workload}), is computed as follows.
\begin{equation}
\label{eqn:carryin}
W_i^{CI}(\Delta_i^{CI}) = \sum_{v_{i, k}\in V_i}\max\big\{ C_{i, k} - \max(L_i - S_{i, k} - \Delta_i^{CI}, 0), 0 \big\}.
\end{equation}
In Equation~\ref{eqn:carryin}, $S_{i, k}$ is the start time of subtask $v_{i, k}$ in the hypothetical schedule for 
the carry-in job described above. It can be computed by taking a longest path among all paths from source 
subtasks to $v_{i, k}$ and adding up the WCETs of the subtasks along that path excluding $v_{i, k}$ itself. 

\begin{figure}[t!]
\centering
\begin{subfigure}{0.7\linewidth}
  \centering
  \includegraphics[width=\linewidth]{figures/nfjdag_conflict.pdf}
  \caption{A non-nested-fork-join DAG task.}
  \label{fig:nfjdag_conflict}
\end{subfigure}
\hfill
\begin{subfigure}{.7\linewidth}
  \centering
  \includegraphics[width=\linewidth]{figures/nfjdag.pdf}
  \caption{A nested fork-join DAG task.}
  \label{fig:nfjdag}
\end{subfigure}
\caption{Example for a general DAG task and a nested fork-join DAG task.}
\label{fig:nfjdagtask}
\end{figure}

For the carry-out workload,~\cite{fonseca2017improved} considered a subset of generalized DAG tasks, 
namely nested fork-join DAG (NFJ-DAG) tasks. A NFJ-DAG is constructed recursively from smaller NFJ-DAGs 
using two operations: \emph{series composition} and \emph{parallel composition}. Figure~\ref{fig:nfjdag} 
shows an example NFJ-DAG task. Figure~\ref{fig:nfjdag_conflict} shows a similar DAG with one more 
edge $(v_{i, 7}, v_{i, 8})$. The DAG in Figure~\ref{fig:nfjdag_conflict} is not a NFJ-DAG due to a single 
cross edge $(v_{i, 7}, v_{i, 8})$. To deal with a non NFJ-DAG,~\cite{fonseca2017improved} first transforms 
the original DAG to a NFJ-DAG by removing the conflicting edges, such as $(v_{i, 7}, v_{i, 8})$ in 
Figure~\ref{fig:nfjdagtask}. Then they compute the upper-bound for the carry-out workload using the 
obtained NFJ-DAG. The computed bound is proved to be an upper-bound for the carry-out workload. 
We note that the transformation removes some precedence constraints from the original DAG, and thus 
the resulting NFJ-DAG may have higher parallelism than the original DAG. Hence, computing the carry-out 
workload of a generalized DAG task via its transformed NFG-DAG may be pessimistic, especially for 
a complex DAG, as the transformation may remove many edges from the original DAG. 


In this paper, we propose a new technique to directly compute an upper-bound for the carry-out workload of 
generalized DAG task. The high level idea is to frame the problem of finding the bound as 
an optimization problem, which can be solved effectively by solvers such as 
the CPLEX~\cite{cplex}, Gurobi~\cite{gurobi}, or SCIP~\cite{scip}. 
The solution of the optimization problem then serves as a safe and tight upper-bound for the 
carry-out workload. In the next section we present our method in detail. 







\section{Bound for Carry-Out Workload}
\label{sec:carryout}

\begin{figure}[h]
\centering
\begin{subfigure}{0.5\linewidth}
  \centering
  \includegraphics[width=\linewidth]{figures/carryout_wcet.pdf}
  \caption{Carry-out workload when all subtasks execute for WCETs.}
  \label{fig:carryout_wcet}
\end{subfigure}
\hfill
\begin{subfigure}{0.4\linewidth}
  \centering
  \includegraphics[width=\linewidth]{figures/carryout_nonwcet.pdf}
  \caption{Carry-out workload when subtasks may execute less than WCETs.}
  \label{fig:carryout_nonwcet}
\end{subfigure}

\caption{An illustration of generating the maximum carry-out workload.}
\label{fig:carryout}
\end{figure}


In this section we propose a method to bound the carry-out workload that can be 
generated by a job of task $\tau_i$ by constructing an \textbf{integer linear program} (ILP) 
for which the optimal solution value is an upper-bound of the carry-out workload. 

Consider a carry-out job of task $\tau_i$, which is scheduled with an unrestricted number of processors, 
meaning that it can use as many processors as it requires to fully exploit its parallelism. 
Each subtask of the carry-out job executes as soon as it is ready, i.e., immediately after all of its predecessors 
have finished. 
We label such a schedule for the carry-out job $\mathcal{SCHE}^{CO}(\tau_i)$. 
We prove in the following lemma that the workload generated by $\mathcal{SCHE}^{CO}(\tau_i)$ is 
an upper-bound for the carry-out workload.
\begin{lemma}
\label{lem:asap}
For specific values of the execution times for the subtasks of $\tau_i$, workload generated by 
$\mathcal{SCHE}^{CO}(\tau_i)$ in a carry-out window of length $\Delta_i^{CO}$ is an upper-bound 
for the carry-out workload generated by $\tau_i$ with the given subtasks's execution times. 
\end{lemma}
\begin{proof}
We prove by contradiction. Consider a schedule $\mathcal{SCHE}^*$ for the carry-out job in which 
subtasks execute for the same lengths as in $\mathcal{SCHE}^{CO}(\tau_i)$. Suppose 
subtask $v_{i, k}$ is the first subtask in time order that produces more workload in $\mathcal{SCHE}^*$ 
than it does in $\mathcal{SCHE}^{CO}(\tau_i)$. This means $v_{i, k}$ must have started executing earlier in 
$\mathcal{SCHE}^*$ than it have in $\mathcal{SCHE}^{CO}(\tau_i)$. Hence, $v_{i, k}$ must have 
started its execution before all of its predecessors have finished in $\mathcal{SCHE}^*$. This is 
impossible and the lemma follows. 
\end{proof}

Unlike the carry-in workload, the carry-out workload generated when all subtasks execute for their full WCETs 
is not guaranteed to be the maximum. 
Consider an interfering task $\tau_i$ shown in Figure~\ref{fig:example_task} and a carry-out window of length 
3 time units. If all subtasks of the carry-out job of $\tau_i$ execute for their WCETs, the carry-out workload 
would be 4 time units, as shown in Figure~\ref{fig:carryout_wcet}. 
However, if subtask $v_{i, 1}$ finishes immediately, i.e., 
executes for 0 time units, the carry-out workload would be 7 time units, as shown in 
Figure~\ref{fig:carryout_nonwcet}. From Lemma~\ref{lem:asap} and the discussion above, to compute an 
upper-bound for carry-out workload we must consider all possible execution times of the subtasks and 
subtasks must execute as soon as they are ready. 

For each subtask $v_{i, a}$ of the carry-out job of an interfering task $\tau_i$, we define two non-negative integer 
variables $X_{i, a}\geq 0 $ and $W_{i, a}\geq 0$. $X_{i, a}$ represents the actual execution time of subtask 
$v_{i, a}$ in the carry-out job and $W_{i, a}$ denotes the contribution of subtask $v_{i, a}$ to 
the carry-out workload. Let $\Delta^{CO}$ be an integer constant denoting the length of the 
carry-out window. Then the carry-out workload is the sum of the contributions of all subtasks in 
$\mathcal{SCHE}^{CO}(\tau_i)$, which is upper-bounded by the maximum of the following 
\emph{optimization objective function}: 
\begin{equation}
\label{eqn:objective}
obj(\tau_i, \Delta^{CO}) \triangleq \sum\limits_{v_{i, a}\in V_i} W_{i, a}.
\end{equation}

The optimal value for the above objective function gives the actual maximum workload generated by the 
carry-out job with unrestricted number of processors. 
We now construct a set of constraints on the contribution of each subtask in 
$\mathcal{SCHE}^{CO}(\tau_i)$ to the carry-out workload. 
From the definitions of $X_{i, a}$ and $W_{i, a}$, we have the following bounds for them. 
\begin{constraint}
\label{con:exetime}
For any interfering task $\tau_i$: \\
$$\forall v_{i, a}\in V_i: 0\leq X_{i, a}\leq C_{i, a}.$$
\end{constraint}

\begin{constraint}
\label{con:workload}
For any interfering task $\tau_i$: \\
$$\forall v_{i, a}\in V_i: 0\leq W_{i, a}\leq X_{i, a}.$$
\end{constraint}

These two constraints come from the fact that the actual execution time of subtask $v_{i, a}$ 
cannot exceed its WCET, and each subtask can contribute at most its whole execution time to 
the carry-out workload. Let $S_{i, a}$ be the starting time of $v_{i, a}$ in $\mathcal{SCHE}^{CO}(\tau_i)$ 
assuming that the carry-out job starts at time instant 0. For simplicity of exposition, we assume 
that the DAG $G_i$ has exactly one source vertex and one sink vertex. If this is not the case, we 
can always add a couple of dummy vertices, $v_{i, source}$ and $v_{i, sink}$, with zero WCETs for 
source and sink vertices, respectively. Then we add edges from $v_{i, source}$ to all vertices with 
no predecessors in the original DAG $G_i$, and edges from all vertices with no successors in $G_i$ 
to $v_{i, sink}$. Without loss of generality, we assume that $v_{i, 1}$ and $v_{i, n_i}$ are the source 
vertex and sink vertex of $G_i$, respectively. 
Let $\sigma^p_{i, a}$ denote a path from the source 
$v_{i, 1}$ to $v_{i, a}$: $\sigma^p_{i, a}\triangleq (v_{i, j_1}, ..., v_{i, j_p})$, where $j_1 = 1$, 
$j_p = a$, and $(v_{i, j_x}, v_{i, j_{x+1}})$ is an edge in $G_i$ $\forall 1\leq x < p$. 
Let $\mathcal{P}(v_{i, a})$ denote the set of all paths from $v_{i, 1}$ to $v_{i, a}$ in $G_i$: 
$\mathcal{P}(v_{i, a})\triangleq \{ \sigma^p_{i, a} \}$. 
$\mathcal{P}(v_{i, a})$ for all subtasks can be constructed by a graph traversal algorithm. 
For instance, a simple modification of depth-first search would accomplish this. 


For a particular path $\sigma^p_{i, a}$, the sum of execution times of all subtasks in this path, 
excluding $v_{i, a}$ is called the \emph{distance} to $v_{i, a}$ with respect to this path. 
We let $D^p_{i, a}$ be a variable denoting the distance to $v_{i, a}$ in path $\sigma^p_{i, a}$. 
We impose the following two straightforward constraints on $D^p_{i, a}$ based on its definition. 
\begin{constraint} 
\label{con:dist_smaller}
For any interfering task $\tau_i$: \\
$$\forall v_{i, a}\in V_i, \forall\sigma^p_{i, a}\in \mathcal{P}(v_{i, a}): 
D^p_{i, a} \leq \sum\limits_{v_{i, j_x}\in\{\sigma^p_{i, a}\setminus v_{i, a}\}} X_{i, j_x}.$$ 
\end{constraint}

\begin{constraint}
\label{con:dist_larger}
For any interfering task $\tau_i$: \\
$$\forall v_{i, a}\in V_i, \forall\sigma^p_{i, a}\in \mathcal{P}(v_{i, a}): 
D^p_{i, a} \geq \sum\limits_{v_{i, j_x}\in\{\sigma^p_{i, a}\setminus v_{i, a}\}} X_{i, j_x}.$$
\end{constraint}


In the schedule $\mathcal{SCHE}^{CO}(\tau_i)$, the starting time $S_{i, a}$ of a subtask $v_{i, a}$ 
cannot be smaller than the distance to $v_{i, a}$ in any path $\sigma^p_{i, a}$. We prove this 
as follows. 
\begin{lemma}
\label{lem:starting_time1}
In the schedule $\mathcal{SCHE}^{CO}(\tau_i)$ of any interfering task $\tau_i$:\\
$$\forall v_{i, a}\in V_i, \forall \sigma^p_{i, a}\in \mathcal{P}(v_{i, a}): 
S_{i, a} \geq D^p_{i, a}.$$
\end{lemma}
\begin{proof}
We prove by contradiction. Let $\sigma^{p*}_{i, a}$ be a path so that the starting time 
$S_{i, a}$ is smaller than $D^{p*}_{i, a}$. Subtask $v_{i, a}$ must be ready to start execution, meaning 
all of its predecessors must finish, at time $S_{i, a}$. Since 
$S_{i, a} < D^{p*}_{i, a}$, there must be a subtask $v_{i, j_x}\in\{\sigma^{p*}_{i, a}\setminus v_{i, a}\}$ 
executing (and thus not finished) at time $S_{i, a}$. Then $v_{i, a}$ cannot be ready at time 
$S_{i, a}$ since it depends on $v_{i, j_x}$. This contradicts the assumption that $v_{i, a}$ is ready 
at $S_{i, a}$ and the lemma follows. 
\end{proof}

In fact, in the schedule $\mathcal{SCHE}^{CO}(\tau_i)$ the starting time $S_{i, a}$ of $v_{i, a}$ 
is equal to the longest distance among all paths to it. 
\begin{lemma}
\label{lem:starting_time2}
In the schedule $\mathcal{SCHE}^{CO}(\tau_i)$ of any interfering task $\tau_i$:\\
$$\forall v_{i, a}\in V_i: 
S_{i, a} = \max\limits_{\sigma^p_{i, a}\in\mathcal{P}(v_{i, a})} D^p_{i, a}.$$ 
\end{lemma}
\begin{proof}
Consider a path $\sigma^{p*}_{i, a}$ constructed as follows. First we take a last-completing 
predecessor of $v_{i, a}$, say $v_{i, j_x}$. Since $v_{i, a}$ executes as soon as it is ready, 
it executes immediately after $v_{i, j_x}$ finishes. We recursively trace back through the last-completing 
predecessors in that way until we reach the source vertex $v_{i, 1}$. Path $\sigma^{p*}_{i, a}$ is then 
constructed by chaining the last-completing predecessors together with $v_{i, a}$. 
We note that any subtask $v_{i, j_x}$ in $\sigma^{p*}_{i, a}$ executes as soon as its immediately 
preceding subtask finishes, since no other predecessors of $v_{i, j_x}$ finish later than it does. 
Therefore, $S_{i, a} = D^{p*}_{i, a}$. From Lemma~\ref{lem:starting_time1}, $\sigma^{p*}_{i, a}$ 
must have the longest distance to $v_{i, a}$ among all paths in $\mathcal{P}(v_{i, a})$. 
Thus the lemma follows. 
\end{proof}


Based on Lemmas~\ref{lem:starting_time1} and~\ref{lem:starting_time2}, we have the following constraint 
for the starting time of $v_{i, a}$. 
\begin{constraint}
\label{con:starttime}
For any interfering task $\tau_i$:\\
$$\forall v_{i, a}\in V_i, \forall\sigma^p_{i, a}\in\mathcal{P}(v_{i, a}): 
S_{i, a} \geq D^p_{i, a}.$$
\end{constraint}
\begin{proof}
We prove that this constraint requires that $S_{i, a}$ of every subtask $v_{i, a}$ for which 
$\max_{\sigma^p_{i, a}\in\mathcal{P}(v_{i, a})} D^p_{i, a} < \Delta^{CO}$ 
satisfies Lemma~\ref{lem:starting_time2}, that is 
$S_{i, a} = \max_{\sigma^p_{i, a}\in\mathcal{P}(v_{i, a})} D^p_{i, a}$. 
(Recall that $\Delta^{CO}$ is a constant denoting the carry-out window's length.) 
In other words, we prove that 
it requires that every subtask $v_{i, a}$, which would start executing within the carry-out window in an 
unrestricted-processor schedule $\mathcal{SCHE}^{CO}(\tau_i)$, gets exactly the same starting 
time from the solution to the optimization problem. 
Let $\mathcal{Q}_i$ denote the collection of such subtasks 
--- the ones that would start executing within the carry-out window in $\mathcal{SCHE}^{CO}(\tau_i)$. 

Let $\pi^*$ be the solution to the optimization problem and $S^*_{i, a}$ be the corresponding 
value for the starting time of any subtask $v_{i, a}\in\mathcal{Q}_i$ in the solution $\pi^*$. Obviously 
$S^*_{i, a} \geq \max_{\sigma^p_{i, a}\in\mathcal{P}(v_{i, a})} D^p_{i, a}$ for any $v_{i, a}$ 
since any solution to the optimization problem satisfies this constraint. 
If $S^*_{i, a} = \max_{\sigma^p_{i, a}\in\mathcal{P}(v_{i, a})} D^p_{i, a}$ for any 
$v_{i, a}\in\mathcal{Q}_i$, then we are done. Suppose instead that 
$S^*_{i, a} = \max_{\sigma^p_{i, a}\in\mathcal{P}(v_{i, a})} D^p_{i, a} + \epsilon_{i, a}$, 
$\epsilon_{i, a} > 0$ for some $v_{i, a}\in\mathcal{Q}_i$. Let $\mathcal{Q'}_i$ denote 
the set of such subtasks. 
We construct a solution $\pi'$ to the optimization problem from $\pi^*$ as follows. 
Consider a first subtask $v_{i, a}\in\mathcal{Q'}_i$ in time. We reduce its starting time 
by $\epsilon_{i, a}$: $S'_{i, a} = S^*_{i, a} - \epsilon_{i, a}$. Since $v_{i, a}$ is the first 
delayed subtask, doing this does not violate the precedence constraints for other subtasks. 
We iteratively perform that operation for other subtasks in $\mathcal{Q'}_i$ in increasing time order. 
The solution $\pi'$ constructed in this way yields a larger carry-out workload since more workload 
from individual subtasks can fit in the carry-out window. Therefore $\pi'$ is a better solution, which 
contradicts the assumption that $\pi^*$ is an optimal solution. 
\end{proof}

The workload contributed by a subtask $v_{i, a}$ is: \\
$W_{i, a} = \min\big\{ \max\{ \Delta^{CO} - S_{i, a}, 0 \}, X_{i, a} \big\}$. 
The second part of the outer minimization has been taken care of by Constraint~\ref{con:workload}. 
We now construct constraints to impose the first part of the minimization. 
Let $M_{i, a}$ be an integer variable representing the expression $\max\{ \Delta^{CO} - S_{i, a}, 0 \}$. 
Let $A_{i, a}$ be a binary variable which takes value either 0 or 1. 
We have the following constraints. 
\begin{constraint}
\label{con:workload2}
For any interfering task $\tau_i$:\\
$$ \forall v_{i, a}\in V_i: W_{i, a} \leq M_{i, a}.$$
\end{constraint}

\begin{constraint}
\label{con:max1}
For any interfering task $\tau_i$:\\
$$ \forall v_{i, a}\in V_i: M_{i, a}\geq 0.$$
\end{constraint}

\begin{constraint}
\label{con:max2}
For any interfering task $\tau_i$:\\
$$ \forall v_{i, a}\in V_i: M_{i, a}\leq (\Delta^{CO} - S_{i, a}) A_{i, a}.$$
\end{constraint}

Constraints~\ref{con:max1} and~\ref{con:max2} bound the value for $M_{i, a}$ and 
Constraint~\ref{con:workload2} enforces another upper bound for the workload $W_{i, a}$. 
If $\Delta^{CO} < S_{i, a}$, $A_{i, a}$ can only be 0 in order to satisfy both 
Contraints~\ref{con:max1} and~\ref{con:max2}. If $\Delta^{CO} = S_{i, a}$, the value 
of $A_{i, a}$ does not matter. In both cases, these three constraints together with 
Constraint~\ref{con:workload} bound $W_{i, a}$ to zero contribution of $v_{i, a}$ to the carry-out 
workload. If $\Delta^{CO} > S_{i, a}$, the maximizing process enforces that $A_{i, a}$ takes 
value 1. Therefore in any case Constraints~\ref{con:workload},~\ref{con:workload2},~\ref{con:max1}, 
and~\ref{con:max2} enforce a correct value for the workload contribution $W_{i, a}$ of $v_{i, a}$. 


We have constructed an ILP with a quadratic constraint (Constraint~\ref{con:max2}) 
for each $v_{i, a}$, for which the optimal solution value is an upper bound for the carry-out workload. The 
carry-out workload of $\tau_i$ in a carry-out window of length $\Delta^{CO}$ can also 
be upper-bounded by the following straightforward lemma. 
\begin{lemma}
\label{lem:carryout_bound}
The carry-out workload of an interfering task $\tau_i$ scheduled by G-FP in a carry-out 
window of length $\Delta^{CO}$ is upper-bounded by $m\Delta^{CO}$.
\end{lemma}

Lemma~\ref{lem:carryout_bound} follows directly from the fact that the carry-out job 
can execute at most on all $m$ processors of the system during the carry-out window. 
Since the carry-out workload of $\tau_i$ is upper-bounded by both the maximum value 
returned for the optimization problem and Lemma~\ref{lem:carryout_bound}, 
it is upper-bounded by the minimum of the two quantities. 
\begin{theorem}
\label{thm:carryout}
The carry-out workload of an interfering task $\tau_i$ scheduled by G-FP in a carry-out 
window of length $\Delta^{CO}$ is upper-bounded by: 
$\min\Big\{ \mathcal{OBJ}, m\Delta^{CO} \Big\}$, where 
$\mathcal{OBJ}$ is the maximum value returned for the maximization problem (Equation~\ref{eqn:objective}).
\end{theorem}

As discussed in Section~\ref{sec:sota}, the technique proposed by Fonseca et al.~\cite{fonseca2017improved} 
can be applied directly for NFJ-DAGs but not for general DAGs. For a general DAG, the procedure to transform 
the general DAG to an NFJ-DAG will likely inflate the carry-out workload bound as it removes some precedence 
constraints between subtasks and enables a higher parallelism (and thus a greater interfering workload) 
for the carry-out job. In contrast, our method 
directly bounds the carry-out workload for any DAG and the optimal value obtained is the actual 
maximum carry-out workload. Hence, our method theoretically yields better schedulability 
than~\cite{fonseca2017improved}'s for general DAGs. 
The cost of our method is higher time complexity for computing carry-out 
workload due to the hardness of the ILP problem. However, it can be implemented and works effectively 
with modern optimization solvers, as we show in our experiments (Section~\ref{sec:evaluation}).

















\section{Response-Time Analysis}
\label{sec:rta_schedtest}

\begin{algorithm}[ht]
\caption{Response-Time Analysis}
\label{algo:sched_test}
\begin{algorithmic}[1]

\Procedure{SchedulabilityTest}{$\tau$}
%\Comment{\parbox[t]{0.5\linewidth}{Without loss of generality, assuming tasks are sorted in decreasing order
%of priority}}
\Comment{Without loss of generality, assuming tasks are sorted in decreasing order of priority}
%\item[]
\For{Each $\tau_k\in\tau$} \label{alg:ln:initstart}
	\Comment{\parbox[t]{.5\linewidth}{Initialize the values for response-time bounds}}
	\State $R_k^{ub} \leftarrow L_k + \frac{1}{m}(C_k - L_k)$
	\If{Any $R_k^{ub} > D_k$}
		\State Return \emph{Unschedulable}
	\EndIf
\EndFor \label{alg:ln:initend}
\item[]

\For{$\tau_k$ from $\tau_2$ to $\tau_n$}
	\State Calculate $R_k^{ub}$ in Theorem~\ref{thm:responsetime_bound} \label{alg:ln:bound}
	\If{$R_k^{ub} > D_k$} \label{alg:ln:checkstart}
		\State Return \emph{Unschedulable}
	\EndIf \label{alg:ln:checkend}
\EndFor

\State Return \emph{Schedulable} \label{alg:ln:ok}
\EndProcedure

\end{algorithmic}
\end{algorithm}


From the above calculations for the bounds of intra-task interference and inter-task interference 
on $\tau_k$, we have the following theorem for the response-time bound of $\tau_k$.
\begin{theorem}
\label{thm:responsetime_bound}
A constrained-deadline task $\tau_k$ scheduled by a global fixed-priority algorithm has response-time 
upper-bounded by the smallest integer $R_k^{ub}$ that satisfies the following fixed-point iteration: \\
$$R_k^{ub} \leftarrow L_k + \frac{1}{m}(C_k - L_k) + \frac{1}{m}\sum\limits_{\tau_i\in hp(\tau_k)} W_i(R_k^{ub}).$$
\end{theorem}
\begin{proof}
This follows from Equation~\ref{eqn:resptime}, Lemma~\ref{lem:intra_interference} and the fact that the 
inter-task interference of $\tau_i$ on $\tau_k$ is bounded by the workload generated by $\tau_i$ 
(Equation~\ref{eqn:workload_relation}). 
\end{proof}


In Theorem~\ref{thm:responsetime_bound}, $W_i(R_k^{ub})$ is computed using Equation~\ref{eqn:max_workload} 
for all carry-in and carry-out windows that satisfy Equation~\ref{eqn:ci_co_length}. 
For specific carry-in and carry-out window lengths, the carry-in workload is bounded using 
Equation~\ref{eqn:carryin} and the carry-out workload is bounded as discussed in Section~\ref{sec:carryout}. 
The lengths for carry-in window $\Delta_i^{CI}$ and carry-out window 
$\Delta_i^{CO}$ are varied as follows. 
Let $\Gamma$ denote the right-hand side of Equation~\ref{eqn:ci_co_length}. 
First $\Delta_i^{CI}$ takes its largest value: $\Delta_i^{CI}\leftarrow \min\{\Gamma, L_i\}$, and $\Delta_i^{CO}$ takes 
the remaining sum: $\Delta_i^{CO}\leftarrow \min\{\Gamma-\Delta_i^{CI}, L_i\}$. Then in each subsequent step, 
$\Delta_i^{CI}$ is decreased and $\Delta_i^{CO}$ is increased until $\Delta_i^{CO}$ takes its largest value and 
$\Delta_i^{CI}$ takes the remaining value. 
We note that if at the first step both $\Delta_i^{CI}$ and $\Delta_i^{CO}$ are greater than or equal to 
$L_i$, the carry-in workload and carry-out workload are bounded by $\min(C_i, m\Delta_i^{CI})$ and 
 $\min(C_i, m\Delta_i^{CO})$, respectively. 
Similarly, if the sum of $\Delta_i^{CI}$ and $\Delta_i^{CO}$ is 0 in Equation~\ref{eqn:ci_co_length}, both the 
carry-in workload and the carry-out workload are 0. 
We also note that for the highest priority task, there is no interference from any other task, and thus its 
response-time bound can be computed simply by: $R_k^{ub} \leftarrow\big(L_k + \frac{1}{m}(C_k - L_k)\big)$. 

Using the above response-time bound, we derive a schedulability test, shown in Algorithm~\ref{algo:sched_test}. 
First we initialize the response-times for the tasks to be $\big(L_k + \frac{C_k - L_k}{m}\big)$ for all tasks $\tau_k$. 
If for any task, the initial response-time is larger than its relative deadline, then the task set is deemed 
unschedulable (lines~\ref{alg:ln:initstart}-\ref{alg:ln:initend}). 
Otherwise, we repeatedly compute the response-time bound for each task in descending order of 
priority using the fixed-point iteration in Theorem~\ref{thm:responsetime_bound} (line~\ref{alg:ln:bound}). 
After the computation for each task finishes, we check whether the response-time bound is larger than its deadline. 
If it is, then the task set is deemed unschedulable (lines~\ref{alg:ln:checkstart}-\ref{alg:ln:checkend}). 
Otherwise, the task set is deemed schedulable after all tasks have been checked (line~\ref{alg:ln:ok}). 

As expected for response-time analysis, for each task $\tau_i$ the number of iterations in the fixed-point equation 
(Theorem~\ref{thm:responsetime_bound}) is pseudo-polynomial in the task's deadline $D_i$ 
(line~\ref{alg:ln:bound}). In each iteration of the fixed-point equation and for each interfering task, 
we consider all combinations of carry-in and carry-out window lengths that satisfy Equation~\ref{eqn:ci_co_length} 
to compute the maximum interfering workload. There are $\mathcal{O}(L_i)$ such combinations, and thus the 
ILP for the carry-out workload is solved $\mathcal{O}(L_i)$ times. 
The maximum workload over all combinations of carry-in and carry-out window lengths gives an upper-bound 
for the interfering workload generated by the given interfering task. 















\section{Implementation and Evaluation}

We have implemented ownership semantics with omitted set and deadlock
detection in Java. We give a brief discussion of some of the practical
considerations in the design of this implementation. We then present
the results of a performance evaluation on a set of benchmark
programs.

\subsection{Objected-Oriented Promise Movement}

Introducing an explicit conception of ownership is minimally
disruptive. It is already the case that every promise is fulfilled by
at most one task, since two sets cause a runtime error. We only ask
that the programmer identify this task by leveraging the existing
structure of \kwasync directives.
%
However, for large, complex synchronization patterns that rely on many
promises, it can become tedious for a programmer to specify all the
relevant promises, one by one.

\lstChannel

In our Java implementation, an object-oriented approach can reduce the
burden of identifying which promises should be moved to new tasks.
%
In our Java implementation of these language features, classes
containing many promises may implement a \textsf{PromiseCollection}
interface so that moving a composite object to a new task is
equivalent to moving each of its constituent promises.
%
A channel class is shown in \cref{lst:channel}, illustrating that
complex and versatile primitives can be built on top of promises with
the aid of \textsf{PromiseCollection}.
%
This class behaves like a promise that can be used repeatedly, where
the $n$th \textsf{recv} operation obtains the value from the $n$th
\textsf{send} operation.
%
This behavior depends on dynamically allocated promises, and the
responsibility for the sending end of the channel is associated not to
the ownership of a single promise, but to the ownership of different
promises at different times. It is abstraction-breaking to ask the
channel user to manually specify which promise to move to a new task
in order to effectively move the sending end of the channel.
%
Instead, we give the impression that the channel object itself is
movable like a promise (line~\ref{ln:channel:b}), since it is a
\textsf{PromiseCollection}, and the implementation of \kwasync relies on
the \textsf{getPromises} method (line~\ref{ln:channel:a}) to
determine which promises should be moved.

\subsection{Exception Handling}

In an implementation of \cref{alg:owners}, some care must go into an
exception handling mechanism.
%
What code is capable of and responsible for recovering from the failed
assertion in line~\ref{ln:owners:async:E}?
%
And what happens if a task terminates early, with unfulfilled
promises, because of an exception?

Observe that line~\ref{ln:owners:async:E} occurs within an
asynchronous task after the user-supplied code for that task has
completed.
%
One solution is to add a parameter to \textsc{Async} so that the user
can supply a post-termination exception handler, which accepts the
list of unfulfilled promises, $t'.\fldowned$, as input.
%
Indeed, the fix for the AWS omitted set bug included such a mechanism
(not shown in \cref{lst:amazon})~\cite{AWSBugFixed}.
%
Alternatively, the runtime could automatically fulfill every
unfulfilled promise upon an assertion failure in
line~\ref{ln:owners:async:E}.
%
Some APIs, including in C++ and Java, provide an exceptional variant
of the completion mechanism for
promises~\cite{Cpp17,JavaCompletableFuture}.
%
In our implementation, we use this mechanism to propagate an exception
through the promises that were left unfulfilled.

Finally, observe that the correctness of \cref{alg:owners} only
depends on knowing when a task's $\fldowned$ list is empty. Therefore,
the $\fldowned$ list could be correctly replaced with a counter, which
would at least reduce the memory footprint of ownership tracking, if
not also the execution time of maintaining a list. However, doing so
would mean that an assertion failure in line~\ref{ln:owners:async:E}
could not indicate \emph{which} promises went unfulfilled. Therefore,
the implementation we evaluate uses an actual list.

\subsection{Benchmarks}

We evaluate the execution time and memory usage overheads introduced
by our promise deadlock detector on nine task-parallel programs. The
overheads are measured relative to the original, unverified baseline
versions.

\begin{enumerate}
\item Conway~\cite{ConwayBench} parallelizes a 2D cellular automaton
  by dividing the grid into chunks. We adapted the code from C to
  Java, using our \textsf{Channel} class (\cref{lst:channel}) in place
  of MPI primitives used by worker tasks to exchange chunk borders
  with their neighbors.

\item Heat~\cite{HeatBench} simulates diffusion on a one-dimensional
  surface, with 50 tasks operating on chunks of 40,000 cells for 5000
  iterations. Neighboring tasks again use \textsf{Channel} in place of
  MPI primitives.

\item QSort sorts 1M integers using a parallelized divide-and-conquer
  recursion; the partition phase is not parallelized. This is a
  standard technique for parallelizing Quicksort~\cite{QuicksortAlg}
  and has been previously implemented using the Habanero-Java
  Library~\cite{HJlib}. We implemented the finish construct, which
  awaits task termination using promises.

\item Randomized distributes 5000 promises over 2535 tasks spawned in
  a tree with branching factor of 3. Each task awaits a random promise
  with probability 0.8 before performing some work, fulfilling its own
  promises, and awaiting all its child tasks. We chose a random seed
  that does not construct a deadlock.

\item Sieve counts the primes below 100,000 with a pipeline of tasks,
  each filtering out the multiples of an earlier prime. A similar
  program is found in prior work~\cite{Ng16}.

\item SmithWaterman (adapted from HClib~\cite{hclib}; also used in
  prior work \cite{TJ,KJ}) aligns DNA sequences having 18,000--20,000
  bases. Each task operates on a $25 \times 25$ tile.

\item Strassen (such a program is found in the Cilk, BOTS, and KASTORS
  suites~\cite{Cilk,BOTS,Kastors}) multiplies sparse $128 \times 128$
  matrices containing around 8000 values. Divide-and-conquer recursion
  issues asynchronous addition and multiplication tasks, up to depth
  5.

\item StreamCluster (from PARSEC~\cite{Parsec}) computes a streaming
  $k$-means clustering of 102,400 points in 128 dimensions, using 8
  worker tasks at a time. We replaced the OpenMP barriers with
  promises in an all-to-all dependence pattern.

\item StreamCluster2 reduces synchronization in StreamCluster by
  replacing some of the all-to-all patterns with all-to-one when it is
  correct to do so. We also correct a data race in the original
  implementation.
\end{enumerate}

All benchmarks were run on a Linux machine with a 16-core AMD Opteron
processor under the OpenJDK 11 VM with a 1 GB memory limit.
%
A thread pool schedules asynchronous tasks by spawning a new thread
for a new task when all existing threads are in use. This execution
strategy is necessary in general for promises because there is no
\emph{a priori} bound on the number of tasks that can block
simultaneously.
%
We measured both execution time and, in a separate run, average memory
usage by sampling every 10 ms.
%
Each measurement is averaged over thirty runs within the same VM
instance, after five discarded warm-up runs; this is a standard
technique to mitigate the variability of JVM overheads, including JIT
compilation~\cite{Georges07}.

\tabResults

\begin{figure}
    \includegraphics[width=\columnwidth]{time-plot.pdf}
    \caption{Execution times for each benchmark
      showing the mean with a 95\% confidence interval (red).}
    \label{fig:time}
    \Description{A plot of the baseline and verified execution times
      for each benchmark. The Sieve, SmithWaterman, and StreamCluster
      benchmarks have noticeable overheads.}
\end{figure}

\Cref{tab:results} gives the unverified baseline measurements for each
program and the overhead factors introduced by the verifiers.
%
The table also gives the geometric mean of overheads across all
benchmarks. There is an overall factor of \geomeanTime in execution
time and \geomeanMem in memory usage.
%
The total number of tasks in the program and the average rates of
promise get and set actions per millisecond (with respect to the
baseline execution time) are also reported.
%
\Cref{fig:time} represents the execution times of each benchmark,
showing the 95\% confidence interval.
%
The low overheads indicate that our deadlock detection algorithm does
not introduce serialization bottlenecks.

The overall execution time overheads are within 1.1$\times$ for each
of Conway, Heat, QSort, Randomized, SmithWaterman, Strassen, and
StreamCluster2. The same is true of the memory overheads for this
subset of benchmarks, excepting SmithWaterman. In many cases, the
verified run narrowly out-performs the baseline, which can be
attributed to perturbations in scheduling and garbage collection.

It is worth noting that the execution overhead for Sieve is in excess
of 2$\times$. Sieve has the single highest rate of get operations by
an order of magnitude (over 37,000, compared to SmithWaterman's
536). The Sieve program requires almost 9594 tasks to be live
simultaneously, each waiting on the next, with the potential to form
very long dependence chains for \cref{alg:detector} to traverse.

We can also remark on the 1.4$\times$ memory overhead in
SmithWaterman. Unlike Conway, Heat, Sieve, and both of the
StreamCluster benchmarks, in which most promises are allocated by the
same task that fulfills them, SmithWaterman (and Randomized) allocates
all promises in the root task and moves them later. In maintaining the
$\fldowned$ lists in \cref{alg:owners}, one can make trade-offs
between speed and space. Our implementation favors speed, so instead
of literally removing a promise $p$ from $t.\fldowned$ in
lines~\ref{ln:owners:async:Y} and~\ref{ln:owners:set:C}, we simply
rely on the fact that $p.\fldowner \ne t$ anymore to detect that $p$
should no longer be counted in line~\ref{ln:owners:async:E}.

For comparison with deadlock verification in other settings, the Armus
tool~\cite{Armus} can identify barrier deadlocks as soon as they
occur, with execution overheads of up to 1.5$\times$ on Java
benchmarks.
%
Our benchmark results represent an acceptable performance overhead
when one desires runtime-identifiable deadlocks and omitted sets with
attributable blame.

\section{Conclusions}
In this paper, we set out to address the problem of multi-tasking robots in multi-robot tasks. 
%A fundamental limitation of existing multi-robot systems was addressed by the removal of a restrictive assumption that was often made--robots are single-tasking.
%Our method allowed coalitions to overlap thus enabling multi-tasking robots. 
We observed that the key underlying challenge was to reason about the physical constraints that could be synergistically satisfied.
%which directly affected the feasibility of multi-tasking.
To address the challenge, we developed our method based on the information invariant theory and modeled constraints as information instances. 
%This allowed us to reason about the relationships between constraints by reasoning about those between information requirements. 
Thereby, a formal and general framework to achieve multi-tasking robots was developed. 
We showed that our algorithm was sound and complete under our problem settings. 
%Our method was integrated with a simple greedy heuristic for task allocation.
Simulation  results  were  provided  to  show  the  effectiveness  of  our approach under resource-constrained situations and in handling challenging situations. % in a multi-UAV simulator. 

% The idea of multi-tasking is attractive in many ways. 
% Humans are living in multi-tasking environments--at any point of time, 
% we are optimizing for more than one task. 
% Multi-task often leads to more efficient task performance since it allows us to exploit task synergies. 
% The work presented in this paper takes us one step forward in realizing multi-tasking robots. 
% In particular, we started looking at the feasibility of multi-tasking. 
% There are many potential directions to pursue along this direction. First, several limitations are present in the current approach. 
% For example, although our method guarantees that there exists a physical configuration that satisfies all the constraints, it does not explicitly take the environmental influence into account. For example, a narrow corridor may prevent a robot formation from passing through, even though all the constraints for the formation do not introduce any conflicts. In this sense, our work should better be characterized as establishing a necessary condition for multi-tasking. Also, our method is mainly focused on the ``{\it planning}'' phase and hence does not address how the robots reach the desired configuration and maintain the constraints. These issues are assumed to be handled by the execution layer.

% More generally, the question of how to execute the tasks with overlapping coalitions is not addressed in this work. 
% As we already discussed, executing individual tasks with non-overlapping coalitions is straightforward but task synergies impose additional requirements on the task execution: how should the robots that are assigned multiple tasks execute them? Should they consider them in a prioritized strategy~\cite{van2005prioritized}? Or should they combine the different tasks in a way that is similar to motor schemas~\cite{arkin2}. 
% Communication requirements for maintaining the constraints must also be taken into account. How should the robots optimize their communication to improve the task performance? 

% The stringency of the physical constraints is another interesting question. It may be desirable to relax the constraints in certain situations (e.g., due to environmental influences). In such cases, it may be important to consider the problem where the constraints are least violated~\cite{kim2012revision}, or specify task constraints in different ways to increase the diversity of the configurations~\cite{srivastava2007domain} so as to make it robust to different environments. 

\bibliographystyle{ACM-Reference-Format}
\bibliography{ms}

\end{document}
