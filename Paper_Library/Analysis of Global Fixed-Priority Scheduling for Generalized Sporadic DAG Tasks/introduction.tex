\section{Introduction}
\label{sec:introduction}

With the prevalence of multiprocessor platforms and parallel programming languages and runtime systems 
such as OpenMP~\cite{openmp}, Cilk Plus~\cite{frigo1998implementation, cilkplus}, 
and Intel's Threading Building Blocks~\cite{tbb}, the demand for computer programs to be 
able to exploit the parallelism offered by modern hardware is inevitable. In recent 
years, the real-time systems research community has worked to address this trend for 
real-time applications that require parallel execution to satisfy their deadlines, such as real-time hybrid 
simulation of structures~\cite{ferry2014real}, and autonomous vehicles~\cite{kim2013parallel}. 


Much effort has been made to develop analysis techniques and schedulability tests for scheduling 
parallel real-time tasks under scheduling algorithms such as Global Earliest Deadline First (G-EDF), 
and Global Deadline Monotonic (G-DM). However, schedulability analysis for parallel tasks is inherently 
more complex than for conventional sequential tasks. 
This is because \emph{intra-task parallelism} is allowed within individual tasks, which enables each individual 
task to execute simultaneously upon multiple processors. The parallelism of each task can also 
vary as it is executing, as it depends on the precedence constraints imposed on the task. 
Consequently, this raises questions of how to account for inter-task interference caused by other 
tasks on a task and intra-task interference among threads of the task itself. 


In this paper, we consider task systems that consist of parallel tasks scheduled under Global Fixed-Priority 
(G-FP), in which each task is represented by a Directed Acyclic Graph (DAG). 
Our analysis is based on the concepts of \emph{critical interference} and 
\emph{critical chain}~\cite{chwa2013global, melani2015response, chwa2017global}, 
which allow the analysis to focus on a special 
chain of sequential segments of each task, and hence enable us to use techniques similar to the ones developed for 
sequential tasks~\cite{baker2003multiprocessor, bertogna2005improved, bertogna2007response, 
bertogna2009schedulability}. 


The contributions of this paper are as follows:
\begin{itemize}
\item We summarize the state-of-the-art analyses for G-FP and highlight their limitations, specifically 
for the calculation of interference of carry-in jobs and carry-out jobs. 
\item We propose a new technique for computing upper-bounds on carry-out workloads, 
by transforming the problem into an optimization problem that can be solved by modern optimization solvers. 
\item We present a response-time analysis, using the workload bound computed with the new technique. 
Experimental results for randomly generated DAG tasks confirm that our technique dominates existing 
analyses for G-FP. 
\end{itemize}

The rest of this paper is organized as follows. In Sections~\ref{sec:related} and~\ref{sec:model} 
we discuss related work and present the task model we consider in this paper. 
Section~\ref{sec:background} reviews the concepts of \emph{critical interference} and 
\emph{critical chain} and discusses a general framework to bound response-time. 
Section~\ref{sec:sota} summarizes the most recent analyses of G-FP, and also highlights limitations of 
those analyses. In Section~\ref{sec:carryout} we propose a new technique to bound carry-out workload. 
A response-time analysis and a discussion of the complexity of our method are given in 
Section~\ref{sec:rta_schedtest}. Section~\ref{sec:evaluation} presents the evaluation of our method for 
randomly generated DAG tasks. We conclude our work in Section~\ref{sec:conclusion}. 






















