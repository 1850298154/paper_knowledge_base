\section{Bound for Carry-Out Workload}
\label{sec:carryout}

\begin{figure}[h]
\centering
\begin{subfigure}{0.5\linewidth}
  \centering
  \includegraphics[width=\linewidth]{figures/carryout_wcet.pdf}
  \caption{Carry-out workload when all subtasks execute for WCETs.}
  \label{fig:carryout_wcet}
\end{subfigure}
\hfill
\begin{subfigure}{0.4\linewidth}
  \centering
  \includegraphics[width=\linewidth]{figures/carryout_nonwcet.pdf}
  \caption{Carry-out workload when subtasks may execute less than WCETs.}
  \label{fig:carryout_nonwcet}
\end{subfigure}

\caption{An illustration of generating the maximum carry-out workload.}
\label{fig:carryout}
\end{figure}


In this section we propose a method to bound the carry-out workload that can be 
generated by a job of task $\tau_i$ by constructing an \textbf{integer linear program} (ILP) 
for which the optimal solution value is an upper-bound of the carry-out workload. 

Consider a carry-out job of task $\tau_i$, which is scheduled with an unrestricted number of processors, 
meaning that it can use as many processors as it requires to fully exploit its parallelism. 
Each subtask of the carry-out job executes as soon as it is ready, i.e., immediately after all of its predecessors 
have finished. 
We label such a schedule for the carry-out job $\mathcal{SCHE}^{CO}(\tau_i)$. 
We prove in the following lemma that the workload generated by $\mathcal{SCHE}^{CO}(\tau_i)$ is 
an upper-bound for the carry-out workload.
\begin{lemma}
\label{lem:asap}
For specific values of the execution times for the subtasks of $\tau_i$, workload generated by 
$\mathcal{SCHE}^{CO}(\tau_i)$ in a carry-out window of length $\Delta_i^{CO}$ is an upper-bound 
for the carry-out workload generated by $\tau_i$ with the given subtasks's execution times. 
\end{lemma}
\begin{proof}
We prove by contradiction. Consider a schedule $\mathcal{SCHE}^*$ for the carry-out job in which 
subtasks execute for the same lengths as in $\mathcal{SCHE}^{CO}(\tau_i)$. Suppose 
subtask $v_{i, k}$ is the first subtask in time order that produces more workload in $\mathcal{SCHE}^*$ 
than it does in $\mathcal{SCHE}^{CO}(\tau_i)$. This means $v_{i, k}$ must have started executing earlier in 
$\mathcal{SCHE}^*$ than it have in $\mathcal{SCHE}^{CO}(\tau_i)$. Hence, $v_{i, k}$ must have 
started its execution before all of its predecessors have finished in $\mathcal{SCHE}^*$. This is 
impossible and the lemma follows. 
\end{proof}

Unlike the carry-in workload, the carry-out workload generated when all subtasks execute for their full WCETs 
is not guaranteed to be the maximum. 
Consider an interfering task $\tau_i$ shown in Figure~\ref{fig:example_task} and a carry-out window of length 
3 time units. If all subtasks of the carry-out job of $\tau_i$ execute for their WCETs, the carry-out workload 
would be 4 time units, as shown in Figure~\ref{fig:carryout_wcet}. 
However, if subtask $v_{i, 1}$ finishes immediately, i.e., 
executes for 0 time units, the carry-out workload would be 7 time units, as shown in 
Figure~\ref{fig:carryout_nonwcet}. From Lemma~\ref{lem:asap} and the discussion above, to compute an 
upper-bound for carry-out workload we must consider all possible execution times of the subtasks and 
subtasks must execute as soon as they are ready. 

For each subtask $v_{i, a}$ of the carry-out job of an interfering task $\tau_i$, we define two non-negative integer 
variables $X_{i, a}\geq 0 $ and $W_{i, a}\geq 0$. $X_{i, a}$ represents the actual execution time of subtask 
$v_{i, a}$ in the carry-out job and $W_{i, a}$ denotes the contribution of subtask $v_{i, a}$ to 
the carry-out workload. Let $\Delta^{CO}$ be an integer constant denoting the length of the 
carry-out window. Then the carry-out workload is the sum of the contributions of all subtasks in 
$\mathcal{SCHE}^{CO}(\tau_i)$, which is upper-bounded by the maximum of the following 
\emph{optimization objective function}: 
\begin{equation}
\label{eqn:objective}
obj(\tau_i, \Delta^{CO}) \triangleq \sum\limits_{v_{i, a}\in V_i} W_{i, a}.
\end{equation}

The optimal value for the above objective function gives the actual maximum workload generated by the 
carry-out job with unrestricted number of processors. 
We now construct a set of constraints on the contribution of each subtask in 
$\mathcal{SCHE}^{CO}(\tau_i)$ to the carry-out workload. 
From the definitions of $X_{i, a}$ and $W_{i, a}$, we have the following bounds for them. 
\begin{constraint}
\label{con:exetime}
For any interfering task $\tau_i$: \\
$$\forall v_{i, a}\in V_i: 0\leq X_{i, a}\leq C_{i, a}.$$
\end{constraint}

\begin{constraint}
\label{con:workload}
For any interfering task $\tau_i$: \\
$$\forall v_{i, a}\in V_i: 0\leq W_{i, a}\leq X_{i, a}.$$
\end{constraint}

These two constraints come from the fact that the actual execution time of subtask $v_{i, a}$ 
cannot exceed its WCET, and each subtask can contribute at most its whole execution time to 
the carry-out workload. Let $S_{i, a}$ be the starting time of $v_{i, a}$ in $\mathcal{SCHE}^{CO}(\tau_i)$ 
assuming that the carry-out job starts at time instant 0. For simplicity of exposition, we assume 
that the DAG $G_i$ has exactly one source vertex and one sink vertex. If this is not the case, we 
can always add a couple of dummy vertices, $v_{i, source}$ and $v_{i, sink}$, with zero WCETs for 
source and sink vertices, respectively. Then we add edges from $v_{i, source}$ to all vertices with 
no predecessors in the original DAG $G_i$, and edges from all vertices with no successors in $G_i$ 
to $v_{i, sink}$. Without loss of generality, we assume that $v_{i, 1}$ and $v_{i, n_i}$ are the source 
vertex and sink vertex of $G_i$, respectively. 
Let $\sigma^p_{i, a}$ denote a path from the source 
$v_{i, 1}$ to $v_{i, a}$: $\sigma^p_{i, a}\triangleq (v_{i, j_1}, ..., v_{i, j_p})$, where $j_1 = 1$, 
$j_p = a$, and $(v_{i, j_x}, v_{i, j_{x+1}})$ is an edge in $G_i$ $\forall 1\leq x < p$. 
Let $\mathcal{P}(v_{i, a})$ denote the set of all paths from $v_{i, 1}$ to $v_{i, a}$ in $G_i$: 
$\mathcal{P}(v_{i, a})\triangleq \{ \sigma^p_{i, a} \}$. 
$\mathcal{P}(v_{i, a})$ for all subtasks can be constructed by a graph traversal algorithm. 
For instance, a simple modification of depth-first search would accomplish this. 


For a particular path $\sigma^p_{i, a}$, the sum of execution times of all subtasks in this path, 
excluding $v_{i, a}$ is called the \emph{distance} to $v_{i, a}$ with respect to this path. 
We let $D^p_{i, a}$ be a variable denoting the distance to $v_{i, a}$ in path $\sigma^p_{i, a}$. 
We impose the following two straightforward constraints on $D^p_{i, a}$ based on its definition. 
\begin{constraint} 
\label{con:dist_smaller}
For any interfering task $\tau_i$: \\
$$\forall v_{i, a}\in V_i, \forall\sigma^p_{i, a}\in \mathcal{P}(v_{i, a}): 
D^p_{i, a} \leq \sum\limits_{v_{i, j_x}\in\{\sigma^p_{i, a}\setminus v_{i, a}\}} X_{i, j_x}.$$ 
\end{constraint}

\begin{constraint}
\label{con:dist_larger}
For any interfering task $\tau_i$: \\
$$\forall v_{i, a}\in V_i, \forall\sigma^p_{i, a}\in \mathcal{P}(v_{i, a}): 
D^p_{i, a} \geq \sum\limits_{v_{i, j_x}\in\{\sigma^p_{i, a}\setminus v_{i, a}\}} X_{i, j_x}.$$
\end{constraint}


In the schedule $\mathcal{SCHE}^{CO}(\tau_i)$, the starting time $S_{i, a}$ of a subtask $v_{i, a}$ 
cannot be smaller than the distance to $v_{i, a}$ in any path $\sigma^p_{i, a}$. We prove this 
as follows. 
\begin{lemma}
\label{lem:starting_time1}
In the schedule $\mathcal{SCHE}^{CO}(\tau_i)$ of any interfering task $\tau_i$:\\
$$\forall v_{i, a}\in V_i, \forall \sigma^p_{i, a}\in \mathcal{P}(v_{i, a}): 
S_{i, a} \geq D^p_{i, a}.$$
\end{lemma}
\begin{proof}
We prove by contradiction. Let $\sigma^{p*}_{i, a}$ be a path so that the starting time 
$S_{i, a}$ is smaller than $D^{p*}_{i, a}$. Subtask $v_{i, a}$ must be ready to start execution, meaning 
all of its predecessors must finish, at time $S_{i, a}$. Since 
$S_{i, a} < D^{p*}_{i, a}$, there must be a subtask $v_{i, j_x}\in\{\sigma^{p*}_{i, a}\setminus v_{i, a}\}$ 
executing (and thus not finished) at time $S_{i, a}$. Then $v_{i, a}$ cannot be ready at time 
$S_{i, a}$ since it depends on $v_{i, j_x}$. This contradicts the assumption that $v_{i, a}$ is ready 
at $S_{i, a}$ and the lemma follows. 
\end{proof}

In fact, in the schedule $\mathcal{SCHE}^{CO}(\tau_i)$ the starting time $S_{i, a}$ of $v_{i, a}$ 
is equal to the longest distance among all paths to it. 
\begin{lemma}
\label{lem:starting_time2}
In the schedule $\mathcal{SCHE}^{CO}(\tau_i)$ of any interfering task $\tau_i$:\\
$$\forall v_{i, a}\in V_i: 
S_{i, a} = \max\limits_{\sigma^p_{i, a}\in\mathcal{P}(v_{i, a})} D^p_{i, a}.$$ 
\end{lemma}
\begin{proof}
Consider a path $\sigma^{p*}_{i, a}$ constructed as follows. First we take a last-completing 
predecessor of $v_{i, a}$, say $v_{i, j_x}$. Since $v_{i, a}$ executes as soon as it is ready, 
it executes immediately after $v_{i, j_x}$ finishes. We recursively trace back through the last-completing 
predecessors in that way until we reach the source vertex $v_{i, 1}$. Path $\sigma^{p*}_{i, a}$ is then 
constructed by chaining the last-completing predecessors together with $v_{i, a}$. 
We note that any subtask $v_{i, j_x}$ in $\sigma^{p*}_{i, a}$ executes as soon as its immediately 
preceding subtask finishes, since no other predecessors of $v_{i, j_x}$ finish later than it does. 
Therefore, $S_{i, a} = D^{p*}_{i, a}$. From Lemma~\ref{lem:starting_time1}, $\sigma^{p*}_{i, a}$ 
must have the longest distance to $v_{i, a}$ among all paths in $\mathcal{P}(v_{i, a})$. 
Thus the lemma follows. 
\end{proof}


Based on Lemmas~\ref{lem:starting_time1} and~\ref{lem:starting_time2}, we have the following constraint 
for the starting time of $v_{i, a}$. 
\begin{constraint}
\label{con:starttime}
For any interfering task $\tau_i$:\\
$$\forall v_{i, a}\in V_i, \forall\sigma^p_{i, a}\in\mathcal{P}(v_{i, a}): 
S_{i, a} \geq D^p_{i, a}.$$
\end{constraint}
\begin{proof}
We prove that this constraint requires that $S_{i, a}$ of every subtask $v_{i, a}$ for which 
$\max_{\sigma^p_{i, a}\in\mathcal{P}(v_{i, a})} D^p_{i, a} < \Delta^{CO}$ 
satisfies Lemma~\ref{lem:starting_time2}, that is 
$S_{i, a} = \max_{\sigma^p_{i, a}\in\mathcal{P}(v_{i, a})} D^p_{i, a}$. 
(Recall that $\Delta^{CO}$ is a constant denoting the carry-out window's length.) 
In other words, we prove that 
it requires that every subtask $v_{i, a}$, which would start executing within the carry-out window in an 
unrestricted-processor schedule $\mathcal{SCHE}^{CO}(\tau_i)$, gets exactly the same starting 
time from the solution to the optimization problem. 
Let $\mathcal{Q}_i$ denote the collection of such subtasks 
--- the ones that would start executing within the carry-out window in $\mathcal{SCHE}^{CO}(\tau_i)$. 

Let $\pi^*$ be the solution to the optimization problem and $S^*_{i, a}$ be the corresponding 
value for the starting time of any subtask $v_{i, a}\in\mathcal{Q}_i$ in the solution $\pi^*$. Obviously 
$S^*_{i, a} \geq \max_{\sigma^p_{i, a}\in\mathcal{P}(v_{i, a})} D^p_{i, a}$ for any $v_{i, a}$ 
since any solution to the optimization problem satisfies this constraint. 
If $S^*_{i, a} = \max_{\sigma^p_{i, a}\in\mathcal{P}(v_{i, a})} D^p_{i, a}$ for any 
$v_{i, a}\in\mathcal{Q}_i$, then we are done. Suppose instead that 
$S^*_{i, a} = \max_{\sigma^p_{i, a}\in\mathcal{P}(v_{i, a})} D^p_{i, a} + \epsilon_{i, a}$, 
$\epsilon_{i, a} > 0$ for some $v_{i, a}\in\mathcal{Q}_i$. Let $\mathcal{Q'}_i$ denote 
the set of such subtasks. 
We construct a solution $\pi'$ to the optimization problem from $\pi^*$ as follows. 
Consider a first subtask $v_{i, a}\in\mathcal{Q'}_i$ in time. We reduce its starting time 
by $\epsilon_{i, a}$: $S'_{i, a} = S^*_{i, a} - \epsilon_{i, a}$. Since $v_{i, a}$ is the first 
delayed subtask, doing this does not violate the precedence constraints for other subtasks. 
We iteratively perform that operation for other subtasks in $\mathcal{Q'}_i$ in increasing time order. 
The solution $\pi'$ constructed in this way yields a larger carry-out workload since more workload 
from individual subtasks can fit in the carry-out window. Therefore $\pi'$ is a better solution, which 
contradicts the assumption that $\pi^*$ is an optimal solution. 
\end{proof}

The workload contributed by a subtask $v_{i, a}$ is: \\
$W_{i, a} = \min\big\{ \max\{ \Delta^{CO} - S_{i, a}, 0 \}, X_{i, a} \big\}$. 
The second part of the outer minimization has been taken care of by Constraint~\ref{con:workload}. 
We now construct constraints to impose the first part of the minimization. 
Let $M_{i, a}$ be an integer variable representing the expression $\max\{ \Delta^{CO} - S_{i, a}, 0 \}$. 
Let $A_{i, a}$ be a binary variable which takes value either 0 or 1. 
We have the following constraints. 
\begin{constraint}
\label{con:workload2}
For any interfering task $\tau_i$:\\
$$ \forall v_{i, a}\in V_i: W_{i, a} \leq M_{i, a}.$$
\end{constraint}

\begin{constraint}
\label{con:max1}
For any interfering task $\tau_i$:\\
$$ \forall v_{i, a}\in V_i: M_{i, a}\geq 0.$$
\end{constraint}

\begin{constraint}
\label{con:max2}
For any interfering task $\tau_i$:\\
$$ \forall v_{i, a}\in V_i: M_{i, a}\leq (\Delta^{CO} - S_{i, a}) A_{i, a}.$$
\end{constraint}

Constraints~\ref{con:max1} and~\ref{con:max2} bound the value for $M_{i, a}$ and 
Constraint~\ref{con:workload2} enforces another upper bound for the workload $W_{i, a}$. 
If $\Delta^{CO} < S_{i, a}$, $A_{i, a}$ can only be 0 in order to satisfy both 
Contraints~\ref{con:max1} and~\ref{con:max2}. If $\Delta^{CO} = S_{i, a}$, the value 
of $A_{i, a}$ does not matter. In both cases, these three constraints together with 
Constraint~\ref{con:workload} bound $W_{i, a}$ to zero contribution of $v_{i, a}$ to the carry-out 
workload. If $\Delta^{CO} > S_{i, a}$, the maximizing process enforces that $A_{i, a}$ takes 
value 1. Therefore in any case Constraints~\ref{con:workload},~\ref{con:workload2},~\ref{con:max1}, 
and~\ref{con:max2} enforce a correct value for the workload contribution $W_{i, a}$ of $v_{i, a}$. 


We have constructed an ILP with a quadratic constraint (Constraint~\ref{con:max2}) 
for each $v_{i, a}$, for which the optimal solution value is an upper bound for the carry-out workload. The 
carry-out workload of $\tau_i$ in a carry-out window of length $\Delta^{CO}$ can also 
be upper-bounded by the following straightforward lemma. 
\begin{lemma}
\label{lem:carryout_bound}
The carry-out workload of an interfering task $\tau_i$ scheduled by G-FP in a carry-out 
window of length $\Delta^{CO}$ is upper-bounded by $m\Delta^{CO}$.
\end{lemma}

Lemma~\ref{lem:carryout_bound} follows directly from the fact that the carry-out job 
can execute at most on all $m$ processors of the system during the carry-out window. 
Since the carry-out workload of $\tau_i$ is upper-bounded by both the maximum value 
returned for the optimization problem and Lemma~\ref{lem:carryout_bound}, 
it is upper-bounded by the minimum of the two quantities. 
\begin{theorem}
\label{thm:carryout}
The carry-out workload of an interfering task $\tau_i$ scheduled by G-FP in a carry-out 
window of length $\Delta^{CO}$ is upper-bounded by: 
$\min\Big\{ \mathcal{OBJ}, m\Delta^{CO} \Big\}$, where 
$\mathcal{OBJ}$ is the maximum value returned for the maximization problem (Equation~\ref{eqn:objective}).
\end{theorem}

As discussed in Section~\ref{sec:sota}, the technique proposed by Fonseca et al.~\cite{fonseca2017improved} 
can be applied directly for NFJ-DAGs but not for general DAGs. For a general DAG, the procedure to transform 
the general DAG to an NFJ-DAG will likely inflate the carry-out workload bound as it removes some precedence 
constraints between subtasks and enables a higher parallelism (and thus a greater interfering workload) 
for the carry-out job. In contrast, our method 
directly bounds the carry-out workload for any DAG and the optimal value obtained is the actual 
maximum carry-out workload. Hence, our method theoretically yields better schedulability 
than~\cite{fonseca2017improved}'s for general DAGs. 
The cost of our method is higher time complexity for computing carry-out 
workload due to the hardness of the ILP problem. However, it can be implemented and works effectively 
with modern optimization solvers, as we show in our experiments (Section~\ref{sec:evaluation}).
















