\section{Related Work}
\label{sec:related}

For the sequential task model, Bertogna et al.~\cite{bertogna2007response} proposed a 
response-time analysis that works for G-EDF and G-FP. They bound the interference of 
a task in a problem window by the worst-case workload it can generate 
in that window. The worst-case workload is then bounded by considering a 
worst-case release pattern of the interfering task. This technique was later extended 
by others to analyze parallel tasks, as is done in this work. Bertogna et al.~\cite{bertogna2009schedulability} 
proposed a sufficient slack-based schedulability test for G-EDF and G-FP in which 
the slack values for the tasks are used in an iterative algorithm to improve the schedulability gradually. 
Later, Guan et al.~\cite{guan2009new} proposed a new response-time analysis for both 
constrained-deadline and arbitrary-deadline tasks. 


Initially, simple parallel real-time task models were studied, such 
as the fork-join task model and the synchronous task model. Lakshmanan et al.~\cite{lakshmanan2010scheduling} 
presented a transformation algorithm to schedule fork-join tasks where all parallel segments of each task 
must have the same number of threads, which must be less than the number of processors. 
They also proved a resource augmentation bound of 3.42 for their algorithm. Saifullah et al.~\cite{saifullah2013multi} 
improved on that work by removing the restriction on the number of threads in parallel segments. 
They proposed a task decomposition algorithm and proved resource augmentation bounds for the 
algorithm under G-EDF and Partitioned Deadline Monotonic (P-DM) scheduling. 
Axer et al.~\cite{axer2013response} presented a response-time analysis for fork-join tasks under 
Partitioned Fixed-Priority (P-FP) scheduling. Chwa et al.~\cite{chwa2013global} developed an 
analysis for synchronous parallel tasks scheduled under G-EDF. They introduced the concept of 
critical interference and presented a sufficient test for G-EDF. Maia et al.~\cite{maia2014response} 
reused the concept of critical interference to introduce a response-time analysis for synchronous tasks 
scheduled under G-FP. 
A general parallel task model was presented by Baruah et al.~\cite{baruah2012generalized} in which 
each task is modeled as a Directed Acyclic Graph (DAG) and can have an arbitrary deadline. They presented 
a polynomial test and a pseudo-polynomial test for a DAG task scheduled with EDF and proved their 
speedup bounds. However, they only considered a single DAG task. Bonifaci et al.~\cite{bonifaci2013feasibility} 
later developed feasibility tests for task systems with multiple DAG tasks, scheduled under G-EDF and 
G-DM. 

Melani et al.~\cite{melani2015response} proposed a response-time analysis for conditional 
DAG tasks where each DAG can have conditional vertices. Their analysis 
utilizes the concepts of critical interference and critical chain, and works for both G-EDF and G-FP. However, 
the bounds for carry-in and carry-out workloads are likely to be overestimated since they ignore the 
internal structures of the tasks. Chwa et al.~\cite{chwa2017global} extended their work in~\cite{chwa2013global} for 
DAG tasks scheduled under G-EDF. They proposed a sufficient, workload-based schedulability test 
and improved it by exploiting slack values of the tasks. 
Fonseca et al.~\cite{fonseca2017improved} proposed a response-time 
analysis for sporadic DAG tasks scheduled under G-FP that improves upon the response-time analysis 
in~\cite{melani2015response}. They improve the upper bounds for interference by taking the DAGs of the 
tasks into consideration. In particular, by explicitly considering the DAGs the workloads generated by 
the carry-in and carry-out jobs can be reduced compared to the ones in~\cite{melani2015response}, 
and hence schedulability can be improved. The carry-in workload is bounded by considering a schedule 
for the carry-in job with unrestricted processors, in which subtasks execute as soon as they are ready and 
for their full WCETs. The carry-out workload is bounded for a less general type of DAG tasks, called 
nested fork-join DAGs. 
We discuss the state-of-the-art analyses for G-FP and differentiate our work in detail in Section~\ref{sec:sota}. 


















