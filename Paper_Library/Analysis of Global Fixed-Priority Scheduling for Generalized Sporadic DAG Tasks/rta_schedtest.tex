\section{Response-Time Analysis}
\label{sec:rta_schedtest}

\begin{algorithm}[ht]
\caption{Response-Time Analysis}
\label{algo:sched_test}
\begin{algorithmic}[1]

\Procedure{SchedulabilityTest}{$\tau$}
%\Comment{\parbox[t]{0.5\linewidth}{Without loss of generality, assuming tasks are sorted in decreasing order
%of priority}}
\Comment{Without loss of generality, assuming tasks are sorted in decreasing order of priority}
%\item[]
\For{Each $\tau_k\in\tau$} \label{alg:ln:initstart}
	\Comment{\parbox[t]{.5\linewidth}{Initialize the values for response-time bounds}}
	\State $R_k^{ub} \leftarrow L_k + \frac{1}{m}(C_k - L_k)$
	\If{Any $R_k^{ub} > D_k$}
		\State Return \emph{Unschedulable}
	\EndIf
\EndFor \label{alg:ln:initend}
\item[]

\For{$\tau_k$ from $\tau_2$ to $\tau_n$}
	\State Calculate $R_k^{ub}$ in Theorem~\ref{thm:responsetime_bound} \label{alg:ln:bound}
	\If{$R_k^{ub} > D_k$} \label{alg:ln:checkstart}
		\State Return \emph{Unschedulable}
	\EndIf \label{alg:ln:checkend}
\EndFor

\State Return \emph{Schedulable} \label{alg:ln:ok}
\EndProcedure

\end{algorithmic}
\end{algorithm}


From the above calculations for the bounds of intra-task interference and inter-task interference 
on $\tau_k$, we have the following theorem for the response-time bound of $\tau_k$.
\begin{theorem}
\label{thm:responsetime_bound}
A constrained-deadline task $\tau_k$ scheduled by a global fixed-priority algorithm has response-time 
upper-bounded by the smallest integer $R_k^{ub}$ that satisfies the following fixed-point iteration: \\
$$R_k^{ub} \leftarrow L_k + \frac{1}{m}(C_k - L_k) + \frac{1}{m}\sum\limits_{\tau_i\in hp(\tau_k)} W_i(R_k^{ub}).$$
\end{theorem}
\begin{proof}
This follows from Equation~\ref{eqn:resptime}, Lemma~\ref{lem:intra_interference} and the fact that the 
inter-task interference of $\tau_i$ on $\tau_k$ is bounded by the workload generated by $\tau_i$ 
(Equation~\ref{eqn:workload_relation}). 
\end{proof}


In Theorem~\ref{thm:responsetime_bound}, $W_i(R_k^{ub})$ is computed using Equation~\ref{eqn:max_workload} 
for all carry-in and carry-out windows that satisfy Equation~\ref{eqn:ci_co_length}. 
For specific carry-in and carry-out window lengths, the carry-in workload is bounded using 
Equation~\ref{eqn:carryin} and the carry-out workload is bounded as discussed in Section~\ref{sec:carryout}. 
The lengths for carry-in window $\Delta_i^{CI}$ and carry-out window 
$\Delta_i^{CO}$ are varied as follows. 
Let $\Gamma$ denote the right-hand side of Equation~\ref{eqn:ci_co_length}. 
First $\Delta_i^{CI}$ takes its largest value: $\Delta_i^{CI}\leftarrow \min\{\Gamma, L_i\}$, and $\Delta_i^{CO}$ takes 
the remaining sum: $\Delta_i^{CO}\leftarrow \min\{\Gamma-\Delta_i^{CI}, L_i\}$. Then in each subsequent step, 
$\Delta_i^{CI}$ is decreased and $\Delta_i^{CO}$ is increased until $\Delta_i^{CO}$ takes its largest value and 
$\Delta_i^{CI}$ takes the remaining value. 
We note that if at the first step both $\Delta_i^{CI}$ and $\Delta_i^{CO}$ are greater than or equal to 
$L_i$, the carry-in workload and carry-out workload are bounded by $\min(C_i, m\Delta_i^{CI})$ and 
 $\min(C_i, m\Delta_i^{CO})$, respectively. 
Similarly, if the sum of $\Delta_i^{CI}$ and $\Delta_i^{CO}$ is 0 in Equation~\ref{eqn:ci_co_length}, both the 
carry-in workload and the carry-out workload are 0. 
We also note that for the highest priority task, there is no interference from any other task, and thus its 
response-time bound can be computed simply by: $R_k^{ub} \leftarrow\big(L_k + \frac{1}{m}(C_k - L_k)\big)$. 

Using the above response-time bound, we derive a schedulability test, shown in Algorithm~\ref{algo:sched_test}. 
First we initialize the response-times for the tasks to be $\big(L_k + \frac{C_k - L_k}{m}\big)$ for all tasks $\tau_k$. 
If for any task, the initial response-time is larger than its relative deadline, then the task set is deemed 
unschedulable (lines~\ref{alg:ln:initstart}-\ref{alg:ln:initend}). 
Otherwise, we repeatedly compute the response-time bound for each task in descending order of 
priority using the fixed-point iteration in Theorem~\ref{thm:responsetime_bound} (line~\ref{alg:ln:bound}). 
After the computation for each task finishes, we check whether the response-time bound is larger than its deadline. 
If it is, then the task set is deemed unschedulable (lines~\ref{alg:ln:checkstart}-\ref{alg:ln:checkend}). 
Otherwise, the task set is deemed schedulable after all tasks have been checked (line~\ref{alg:ln:ok}). 

As expected for response-time analysis, for each task $\tau_i$ the number of iterations in the fixed-point equation 
(Theorem~\ref{thm:responsetime_bound}) is pseudo-polynomial in the task's deadline $D_i$ 
(line~\ref{alg:ln:bound}). In each iteration of the fixed-point equation and for each interfering task, 
we consider all combinations of carry-in and carry-out window lengths that satisfy Equation~\ref{eqn:ci_co_length} 
to compute the maximum interfering workload. There are $\mathcal{O}(L_i)$ such combinations, and thus the 
ILP for the carry-out workload is solved $\mathcal{O}(L_i)$ times. 
The maximum workload over all combinations of carry-in and carry-out window lengths gives an upper-bound 
for the interfering workload generated by the given interfering task. 













