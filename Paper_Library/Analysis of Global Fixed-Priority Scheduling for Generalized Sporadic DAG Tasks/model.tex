\section{System Model}
\label{sec:model}

\begin{figure}[ht]
\centering
\includegraphics[width=\linewidth]{figures/example_task.pdf}
\caption{An example DAG task.}
\label{fig:example_task}
\end{figure}

We consider a set $\tau$ of $n$ real-time parallel tasks, $\tau = \{\tau_1, \tau_2, ..., \tau_n\}$, 
scheduled preemptively by a global fixed-priority scheduling algorithm upon $m$ identical processors. 
Each task $\tau_i$ is a recurrent, sporadic 
process which may release an infinite sequence of jobs and is modeled by $\tau_i = \{G_i, D_i, T_i\}$, 
where $D_i$ denotes its relative deadline and $T_i$ denotes the minimum inter-arrival time of 
two consecutive jobs of $\tau_i$. We assume that all tasks have constrained deadlines, i.e., 
$D_i\leq T_i, \forall i\in [1, n]$. 
Each task $\tau_i$ is represented as a Directed Acyclic Graph (DAG) $G_i=(V_i, E_i)$, 
where $V_i=\{v_{i, 1}, v_{i, 2}, ..., v_{i, n_i}\}$ is the set of vertices of the DAG $G_i$ and 
$E_i\subseteq (V_i\times V_i)$ is the set of directed edges of $G_i$. 
In this paper, we also use \emph{subtasks} and \emph{nodes} to refer to the vertices of the tasks. 
Each subtask $v_{i, a}$ of $G_i$ represents a section of instructions that can only be run sequentially. 
A subtask $v_{i, a}$ is called a \emph{predecessor} of $v_{i, b}$ if there exists an edge from $v_{i, a}$ to 
$v_{i, b}$ in $G_i$, i.e., $(v_{i, a}, v_{i, b})\in G_i$. Subtask $v_{i, b}$ is then called a \emph{successor} of 
$v_{i, a}$. Each edge $(v_{i, a}, v_{i, b})$ represents a precedence constraint between the two subtasks. 
A subtask is \emph{ready} if all of its predecessors have finished. 
Whenever a task releases a job, all of its subtasks are released and have the 
same deadline as the job's deadline. We use $J_i$ to denote an arbitrary job of $\tau_i$ 
which has release time $r_i$ and absolute deadline $d_i$. 


Each subtask $v_{i, a}$ has a worst-case execution time (WCET), denoted by $C_{i, a}$. 
The sum of WCETs of all subtasks of $\tau_i$ is the worst-case execution time of 
the whole task, and is denoted by $C_i = \sum_{v_{i, a}\in V_i} C_{i, a}$. The WCET of 
a task is also called its \emph{work}. A sequence of subtasks $(v_{i, u_1}, v_{i, u_2}, ..., v_{i, u_t})$ 
of $\tau_i$, in which $(v_{i, u_j}, v_{i, u_{j+1}})\in E_i, \forall 1\leq j\leq t-1$, 
is called a \emph{chain} of $\tau_i$ and is denoted by $\lambda_i$. The length of a chain $\lambda_i$ 
is the sum of the WCETs of subtasks in $\lambda_i$ and is denoted by $len(\lambda_i)$, i.e., 
$len(\lambda_i) = \sum_{v_{i, u_j}\in\lambda_i} C_{i, u_j}$. A chain of $\tau_i$ which has 
the longest length is a \emph{critical path} of the task. 
The length of a critical path of a DAG is called its \emph{critical path length} or 
\emph{span}, and is denoted by $L_i$. 
Figure~\ref{fig:example_task} illustrates 
an example DAG task $\tau_i$ with 6 subtasks, whose work and span are 
$C_i = 13$ and $L_i = 8$, respectively. In this paper, we consider tasks that 
are scheduled using a preemptive, global fixed-priority algorithm where each task is assigned a 
fixed task-level priority. All subtasks of a task have the same priority as the task. 
Without loss of generality, we assume that tasks have distinct priorities, 
and $\tau_i$ has higher priority than $\tau_k$ if $i < k$. 



















