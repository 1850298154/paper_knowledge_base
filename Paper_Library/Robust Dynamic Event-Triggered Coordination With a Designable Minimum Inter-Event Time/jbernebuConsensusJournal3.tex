%\documentclass[letterpaper,10pt,conference,romanappendices]{ieeetran}
\documentclass[final,twocolumn,twoside]{IEEEtran}
%\documentclass[12pt,draftcls,onecolumn]{IEEEtran}
%\documentclass[final,twocolumn,twoside]{IEEEtran}

\usepackage{amsmath,amssymb,theorem}
\usepackage{graphicx,floatflt}
\usepackage{algorithm,algorithmic}
\usepackage{psfrag}
	
\usepackage{color,xspace}
\usepackage{subfigure}
%% \setlength\topmargin{0.25 in}

\pdfminorversion=4

\newtheorem{theorem}{Theorem}[section]
\newtheorem{lemma}[theorem]{Lemma}
\newtheorem{remarks}[theorem]{Remarks}
\newtheorem{remark}[theorem]{Remark}
\newtheorem{example}[theorem]{Example}	
\newtheorem{corollary}[theorem]{Corollary}
\newtheorem{proposition}[theorem]{Proposition}
\newtheorem{definition}[theorem]{Definition}
\newtheorem{problem}[theorem]{Problem}

\newcommand{\real}{{\mathbb{R}}}
\newcommand{\realpositive}{\mathbb{R}_{>0}}
\newcommand{\realnonnegative}{\mathbb{R}_{\ge 0}}
\newcommand{\integernonnegative}{\mathbb{Z}_{\ge 0}}
\newcommand{\integerpositive}{\mathbb{Z}_{> 0}}
\newcommand{\domain}{\mathcal{D}}
\newcommand{\opendomain}{\mathcal{S}_o}
\newcommand{\closeddomain}{\mathcal{S}_c}
\newcommand{\opencurve}{\gamma_{\texttt{opc}}}
\newcommand{\closedcurve}{\gamma_{\texttt{cpc}}}
\newcommand{\dist}{\operatorname{d}}
\newcommand{\GG}{{\mathcal{G}}}
\newcommand{\PP}{{\mathcal{P}}}
\newcommand{\EE}{{\mathcal{E}}}
\newcommand{\Ac}{{\mathcal{A}}}
\newcommand{\MM}{{\mathcal{M}}}
\newcommand{\NN}{{\mathcal{N}}}
\newcommand{\Lie}{{\mathcal{L}}}
\newcommand{\UU}{{\mathcal{U}}}
\newcommand{\KK}{{\mathcal{K}}}
\renewcommand{\natural}{{\mathbb{N}}}
\newcommand{\HH}{{\mathcal{H}}}
\newcommand{\WW}{{\mathcal{W}}}
\newcommand{\data}{{\mathcal{D}}}
\newcommand{\ragents}{{\mathcal{A}}}
\newcommand{\evol}{\operatorname{Evl}}
\newcommand{\bound}{\operatorname{bnd}}
\newcommand{\ebound}{\operatorname{ebnd}}
\newcommand{\zbound}{\operatorname{zbnd}}
\newcommand{\halfspace}{{H}}
\newcommand{\HHc}{{\mathcal{H}}_c}
\newcommand{\HHo}{{\mathcal{H}}_o}
\newcommand{\VV}{{\mathcal{V}}}
\newcommand{\TT}{{\mathcal{T}}}
\newcommand{\XX}{{\mathcal{X}}}
\newcommand{\CM}{{\text{CM}}}
\newcommand{\cntr}{\operatorname{cntr}}
\newcommand{\We}{{\text{We}}}
\newcommand{\cm}[1]{\operatorname{cntr}(#1)}
\newcommand{\cc}[1]{\operatorname{cc}(#1)}
\newcommand{\ccr}[1]{\operatorname{cr}(#1)}
\newcommand{\unit}[1]{\operatorname{unit}(#1)}
\newcommand{\tbb}{\operatorname{tbb}}
\newcommand{\loc}[1]{\operatorname{loc}(#1)}
\newcommand{\normal}{\operatorname{n}}
\newcommand{\norm}{\operatorname{norm}}
\newcommand{\mass}[1]{\operatorname{mass}(#1)}
\newcommand{\AreaWeight}{f}
\newcommand{\ones}[1]{\mathbf{1}_{#1}}
\newcommand{\zeros}[1]{\mathbb{0}_{#1}}
\newcommand{\spn}{\operatorname{span}}
\newcommand{\diag}[1]{\operatorname{diag}\left( #1\right)}
\newcommand{\trace}[1]{\operatorname{tr}( #1)}
\newcommand{\interior}[1]{\operatorname{int}\left( #1\right)}
\renewcommand{\tilde}{\widetilde}
\newcommand{\grad}{\nabla}
\newcommand{\graph}{G}
\newcommand{\vertices}{V}
\newcommand{\edges}{E}
\newcommand{\adjmat}{A}
\newcommand{\subscr}[2]{#1_{\textup{#2}}}
\newcommand{\rank}{\text{rank} \,}
\renewcommand{\div}{\text{div}}
\newcommand\reg{$^{\textrm{\tiny\textregistered}}$\xspace}
\newcommand{\Polygon}{\mathcal{P}}
\newcommand{\proj}{\operatorname{pr}}
\newcommand{\automaton}{\texttt{intersection-free gradient automaton}}
\newcommand{\vmax}{v_{\text{max}}}
\newcommand{\fmax}{f_{\text{max}}}
\newcommand{\unom}{u^{\text{nominal}}}
\newcommand{\uunknown}{u^{\text{unknown}}}
\newcommand{\timestep}{\Delta t}
\newcommand{\sleep}{t_\text{sleep}}
\newcommand{\goal}{p_\text{goal}}
\newcommand{\eps}{\varepsilon}
\renewcommand{\epsilon}{\varepsilon}
\newcommand{\diam}{\operatorname{diam}}
\newcommand{\argmin}{\operatorname{argmin}}
\newcommand{\argmax}{\operatorname{argmax}}
\renewcommand{\complement}{\operatorname{comp}}
\newcommand{\reachable}{\mathcal{R}}
\newcommand{\powerset}{\mathbb{P}^\text{cc}}
\newcommand{\powersetc}{\mathbb{P}^\text{c}}
\renewcommand{\time}[1]{t_{\text{#1}}}
\newcommand{\Norm}[1]{\|#1\|}
\newcommand{\Ave}[1]{\operatorname{Ave}(#1)}
\newcommand{\Tmax}{T_{\text{max},i}}

\newcommand{\algostep}[1]{{\small\texttt{#1:}}\xspace}
\newcommand{\CVT}{\operatorname{CVT}}

\newcommand{\Algo}{\operatorname{T}}
\newcommand{\CentroidPowerAlgo}{\operatorname{T_{quad-perf}}}
\newcommand{\WeberWeightedAlgo}{\operatorname{T_{linear-perf}}}

\newcommand{\outside}{\textup{out}}
\newcommand{\inside}{\textup{in}}
\newcommand{\Outside}{\textup{Outside}}
\newcommand{\Inside}{\textup{Inside}}
\newcommand{\versor}{\textup{unit}}
\newcommand{\until}[1]{\{1,\dots, #1\}}
\newcommand{\fromto}[2]{\langle #1,\dots, #2 \rangle}
\newcommand{\sgn}[1]{\operatorname{sgn}(#1)}
\newcommand{\map}[3]{#1: #2 \rightarrow #3}
\newcommand{\svmap}[3]{#1: #2 \rightrightarrows #3}
\newcommand{\setdef}[2]{\{#1 \; | \; #2\}}
\newcommand{\pder}[2]{\frac{\partial #1}{\partial #2}}
\newcommand{\ppder}[2]{\frac{\partial^2 #1}{\partial #2^2}}
\newcommand{\pcder}[3]{\frac{\partial^2 #1}{\partial #2 \partial #3}}
\newcommand{\oball}[2]{B(#1,#2)}
\newcommand{\cball}[2]{\overline{B}(#1,#2)}
\newcommand{\cballinfty}[2]{\overline{B}_\infty(#1,#2)}
\renewcommand{\dist}[2]{\operatorname{dist}(#1,#2)}
\newcommand{\inn}[2]{\langle #1,#2 \rangle}
\newcommand{\osegment}[2]{]#1,#2[}
\newcommand{\csegment}[2]{[#1,#2]}
\newcommand{\cosegment}[2]{[#1,#2[}
\newcommand{\ray}[2]{\texttt{ray}(#1,#2)}
\renewcommand{\wedge}[3]{\texttt{wedge}(#1,#2,#3)}
\newcommand{\TwoNorm}[1]{\|#1\|_2}
\renewcommand{\TwoNorm}[1]{\|#1\|}
\newcommand{\InfNorm}[1]{\|#1\|_\infty}
\newcommand{\ortho}[1]{#1^\perp}
\newcommand{\variance}[1]{\operatorname{var}(#1)}
\renewcommand{\hat}{\widehat}

\newcommand{\sensgraph}{\GG_\text{sens}}
\newcommand{\commgraph}{\GG_\text{comm}}
\renewcommand{\sensgraph}{\GG}
\renewcommand{\commgraph}{\GG}
\newcommand{\desgraph}{\GG_\text{des}}
\newcommand{\allowable}{A}
\newcommand{\umax}{u_\text{max}}
\newcommand{\uperiodic}{u^\text{per}}
\newcommand{\uevent}{u^\text{event}}
\newcommand{\uself}{u^\text{self}}
\newcommand{\uteam}{u^\text{team}}
\newcommand{\selfcontrol}{u^\text{self}}
\newcommand{\socialcontrol}{u^\text{team}}
\newcommand{\safecontrol}{u^\text{sf}}
\newcommand{\exactstates}[1]{x^{\NN(#1)}}
\newcommand{\availablestates}{x_{\NN}}
\newcommand{\estimatestates}{\hat{x}_{\NN}}
\newcommand{\promisetime}{T_\text{pr}}
\newcommand{\promisesets}{X_{\NN}}
\newcommand{\guaranteedsets}{\mathbf{X}_\NN}
\newcommand{\gset}{\mathbf{X}}
\newcommand{\rguaranteedset}{\mathbf{\hat{X}}}
\newcommand{\rpromiseset}{\mathbf{X}}
\newcommand{\socialcontroldistributed}{k^\text{team}}
\newcommand{\omegamax}{\bar{\omega}}
\newcommand{\delay}{\Delta}
\newcommand{\maxdelay}{\bar{\Delta}}
\newcommand{\maxnoise}{\bar{\omega}}
\newcommand{\si}{\mathcal{L}_iV^\text{sup}}
%\newcommand{\si}{\overline{\mathcal{L}_{f_i} V(\promisesets^i)}}
\newcommand{\sit}{\overline{\mathcal{L}_{f_i} V(\promisesets^i(t))}}
\newcommand{\sitstar}{\overline{\mathcal{L}_{f_i} V(\promisesets^i(t^*))}}
\newcommand{\sitstari}{\overline{\mathcal{L}_{f_i} V(\promisesets^i(t^*_i))}}
\newcommand{\sitzero}{\overline{\mathcal{L}_{f_i} V(\promisesets^i(t_0))}}
\newcommand{\sitprime}{\overline{\mathcal{L}_{f_i} V(\promisesets^i(t'))}}
\newcommand{\sitplusd}{\overline{\mathcal{L}_{f_i} V(\promisesets^i(t + T_d))}}
\newcommand{\zmax}{z_{\text{max}}}
\newcommand{\dwellself}{T_{\text{d,self}}}
\newcommand{\dwellevent}{T_{\text{d,event}}}
\newcommand{\socialtime}{\tau}
\newcommand{\hausdorff}{d_H}
\newcommand{\functiondist}{d_{\text{func}}}
\newcommand{\setdistance}{d}
\newcommand{\staterule}{R^\text{s}}
\newcommand{\controlrule}{R^\text{c}}
\newcommand{\staticrule}{R^\text{sb}}
\newcommand{\dynamicrule}{R^\text{db}}
\newcommand{\nullrule}{R^\text{s,n}}
\newcommand{\ruleset}{\mathfrak{R}}
\renewcommand{\loc}{\operatorname{loc}}
\newcommand{\boundary}{\partial}
\newcommand{\open}[1]{\mathring{#1}}
\newcommand{\sign}[1]{\operatorname{sgn}(#1)}

\newcommand{\algomap}{M}
\newcommand{\timeschedule}{\mathcal{T}}

\newcommand{\Nouti}{\NN_i^\text{out}}

\newcommand{\clovar}{\chi}
\newcommand{\reldis}{\hat{\xi}}
\newcommand{\sumvar}{\phi}
\newcommand{\solvar}{\varphi}


\newcommand{\oprocendsymbol}{\hbox{$\bullet$}}
\newcommand{\oprocend}{\relax\ifmmode\else\unskip\hfill\fi\oprocendsymbol}
\def\eqoprocend{\tag*{$\bullet$}}

% this wraps theorem titles (or proposition, etc)
\newcommand{\longthmtitle}[1]{\mbox{}\textup{\textbf{(#1)}}}

%% Enumerate environment
\renewcommand{\theenumi}{(\roman{enumi})}
\renewcommand{\labelenumi}{\theenumi}

\newcommand{\margin}[1]{\marginpar{\color{red}\tiny\ttfamily#1}}
\newcommand{\marginjb}[1]{\marginpar{\color{blue}\tiny\ttfamily#1}}
\newcommand{\todo}[1]{\par\noindent{\color{blue}
    \raggedright\textsc{#1}\par\marginpar{\Large $\star$}}}

% \renewcommand{\margin}[1]{}
% \renewcommand{\marginjc}[1]{}
% \renewcommand{\todo}[1]{}

\newcommand{\myclearpage}{\clearpage}
\renewcommand{\myclearpage}{}

%Algorithm names
\newcommand{\algoeventconsensus}{\textsc{event-triggered communication and control law}\xspace} 

\parskip = .5ex 
%\renewcommand{\baselinestretch}{1}

\begin{document}

\title{Robust Dynamic Event-Triggered Coordination \\ With a Designable Minimum Inter-Event Time}

\author{James Berneburg \qquad Cameron Nowzari \thanks{The authors are
    with the Department of Electrical and Computer Engineering,
    George Mason University, Fairfax, VA 22030, USA, {\tt\small
      \{jbernebu,cnowzari\}}@gmu.edu}}
\maketitle


\begin{abstract}
This paper revisits the classical multi-agent average consensus problem for which many different event-triggered control strategies have been proposed over the last decade. Many of the earliest versions of these works conclude asymptotic stability without proving that Zeno behavior, or deadlocks, do not occur along the trajectories of the system. More recent works
that resolve this issue either: (i) propose the use of a dwell-time that forces inter-event times to be lower-bounded away from zero but sacrifice asymptotic convergence in exchange for practical convergence (or convergence to a neighborhood); (ii) guarantee non-Zeno behaviors and asymptotic convergence but do not provide a positive minimum inter-event time guarantee; or (iii) are not fully distributed. 
Additionally, the overwhelming majority of these works provide no form of robustness analysis on the event-triggering strategy. More specifically, if arbitrarily small disturbances can remove the non-Zeno property then the theoretically correct algorithm may not actually be implementable. Instead, this work for the first time presents a fully distributed, robust, dynamic event-triggered algorithm, for general directed communication networks, for which a desired positive minimum inter-event time can be chosen by each agent in a distributed fashion. Simulations illustrate our results.
\end{abstract} 
%  This paper uses Lyapunov analysis to examine a dynamic distributed event-triggered mechanism for a system of multiple agents where communication is limited. This is done for the consensus problem with an undirected graph and agents with single integrator dynamics. It shows that stabilization is possible with a calculable nonzero local inter-event time for each agent, and it develops a triggering mechanism by which each agent uses the local states in order to use communication efficiently.
%\margin{since this is literally the journal version of the ACC paper, we do not have to write it as if the undirected one has been solved yet. Both the undirected and directed are contributions of this paper.}

\section{Introduction}

%\margin{
%Add more citations from Heemels, particularly Borgers and Heemels TAC 2014. Introduction should be re-written after that to make connections to their work and notions on event separation, robustness, positive MIETs, etc. 
%}
%\margin{Add more citations from me! Check the survey and borrow language and references from there. That article motivates this paper very well so you should bring in all that motivation in your own words. This journal should easily have over 40 references. Let's bring the conference paper up to at least 30 as well - we don't want to upset anyone }

Systems composed of individually controlled agents are increasingly common and a very active area of research. Such systems are designed for many different applications including the coordination of unmanned air vehicles, distributed reconfigurable sensor networks, and attitude alignment for satellites, etc; see~\cite{RosRmm2004} and~\cite{WrRwbEma2007} and their references. These are often intended to fulfill some coordinated task, but require distributed control to be scaled with large systems. In this case, communication limitations, such as wireless bandwidth, mean that agents cannot be assumed to have continuous access to others' states.
Therefore, many works have recently considered communication to be a limited resource, where individual agents must autonomously schedule when to take various actions, rather than doing so periodically or continuously.

A common solution to these types of problems comes in the form of event-triggered coordination, where actions occur at specific instances of time when some event condition is satisfied, such as when an error state~\cite{XyKlDvdKhj17} or a clock state~\cite{CdpRp2017} hits some threshold. A similar strategy is self-triggered control, where the controller uses state information to schedule events ahead of time. An introduction to these ideas for single-plant systems is found in~\cite{WpmhhKhjPt2012}. 

One potential problem in event-triggered coordination is the Zeno phenomenon, where the number of events triggered goes to infinity in a finite time period. This is problematic as it asks for solutions that cannot be realized by actual devices. A way to prevent this problem is to design triggering conditions that guarantee a positive minimum inter-event time (MIET) exists. Note that this is different from first designing an event-triggering condition, and then afterwards \emph{forcing} a minimum inter-event time (dwell-time), which has drawbacks as we discuss later. For example, the self-triggered strategy in~\cite{CdpPf2013} enforces a MIET and so only guarantees convergence to a set.

As noted in~\cite{WpmhhKhjPt2012}, the existence of a positive MIET is important to ensure that the event-triggering mechanism does not become unimplementable because it requires actions to be taken arbitrarily fast. This issue has been addressed recently for single-plant systems, e.g., in~\cite{AvpMmj2018}, where a general method for achieving stabilization based on Lyapunov functions is developed, and~\cite{RpPtDnAa2015}, where a general framework for event-triggering mechanisms is provided using the hybrid systems formalism of~\cite{RgRgsArt2012}. 
{Hybrid systems formulations seem especially useful in networked control systems because they conveniently describe systems with continuous-time dynamics and discrete-time memory updates via communication.} %conveniently describes systems with continuous time dynamics and discrete time memory.} 
However, this is still a major challenge for multi-agent systems with distributed information.

To address this, we turn to a simple but widely applicable canonical problem: multi-agent average consensus. Consensus problems are when multiple agents, each with its own dynamics and limited access to the other agents' states, are intended to be stabilized such that all the agents' states are equal. Applications include distributed computing, networks of sensors, flocking, rendezvous, attitude alignment, and synchronization of coupled oscillators; see~\cite{BlWlTc2011},~\cite{WrRwbEma2007} and~\cite{RosRmm2004} and their references. 

Event-triggered strategies for consensus problems have been studied quite extensively over the last decade, with some of the earliest works appearing in 2009~\cite{DvdEf2009,DvdKhj2009,EkXwNh2010}. %Additionally, self-triggered strategies for consensus, such as~\cite{CdpPf2013}, also exist. 
We refer to~\cite{CnEgJc2019} for a detailed survey on the history of this problem but summarize the relevant points next.
 
A seminal work on this topic is~\cite{DvdEfKhj2012}, which develops centralized event- and self-triggered strategies that lead to multi-agent consensus, and then modifies them to be distributed. Unfortunately, although the centralized event-triggered strategy is able to guarantee a positive MIET, the distributed strategies are unable to guarantee the prevention of Zeno solutions. 
Similarly, the results in~\cite{EgYcHyPaDc2013} are unable to exclude Zeno behavior. 

Some works have addressed this issue by considering a periodically sampled (or sampled-data) implementation to trivially address this issue, but this assumes perfect synchronization among the entire network which is neither practical or scalable~\cite{XmTc2013,XmLxYcsCnGjp2015,CnJc2016,AaAaAm2018}. 
More recent works have even considered asynchronous periodic implementations but
these require some sort of global knowledge to find periods that will work~\cite{FxTc2016,YlCnZtQl2017}. We are instead interested in designing event-triggering conditions that guarantee a positive MIET rather than forcing one artificially. 

More related to our work,~\cite{GssDvdKhj2013,BcZl2017,XyKlDvdKhj17} present distributed event-triggered strategies for the consensus problem that prevent Zeno solutions and ensure convergence to consensus. The first two include an explicit function of time in the trigger mechanism, while the third uses a dynamic triggering mechanism, by including a virtual state. %It is based on the dynamic event trigger of~\cite{Ag2015}, which, in turn, is augmenting the event trigger mechanism of~\cite{Pt2007} with a dynamic state. 
While these are a good start, unfortunately none of these can guarantee a positive MIET for the agents, which is our main goal. 

The distributed event-triggered strategy in~\cite{VsdMaWpmhh2017} is able to guarantee convergence to consensus with a positive MIET enforced; however, it requires global parameters in order to design each agent's controller, so it is not fully distributed. {Alternatively, and most similarly to the methods used in this work, the authors of~\cite{CdpRp2017} utilize a hybrid systems formulation to solve a closely related problem in which a different communication model is considered.} In their work they show the existence of a positive MIET for a fully distributed event-triggered strategy that guarantees asymptotic convergence with an event trigger that employs a dynamic virtual state; however, this work still requires a type of synchronization as agents need to trigger events in pairs. 
%Are these MIETs robust? If not we should mention it here or in the next paragraph.
%they are robust: both use a clock variable to trigger events. VsdMaWpmhh2017 includes an explicit function of time in the clock's dynamics, so its MIET does not depend on the state. CdpRp2017's MIET seems to depend on the state, but the state would have to go to infinity to make it 0.

%This work considers events to be triggered by multiple agents monitoring the same quantity at the same time, requiring a sort of synchronization. Instead, we assume agents alone are responsible for deciding when to broadcast information to their neighbors, which is known as an agent-triggered strategy. %On the other hand, the algorithm in~

Finally, another important consideration is the robustness of a MIET. {
We note here that we are primarily concerned with the robustness of the event-triggering strategy rather than robustness in terms of feedback stabilization.}
The authors of~\cite{DpbWpmhh2014} acutely point out that, even if an event-triggered controller may guarantee a positive MIET, it is possible that arbitrarily small disturbances remove this property which means it is still equally useless for implementing on physical systems. {We refer to~\cite{DpbWpmhh2014} which introduces a notion of strongly non-Zenoness for a more precise mathematical definition for this type of robustness.} Another potential issue for event-triggered strategies is robustness to imperfect event detection. More specifically, analysis on event-based solutions often rely on the very precise timings of actions in response to events. Consequently, we are also interested in designing a solution that is robust to small timing errors in determining when event conditions have been satisfied.

% %They show which classifications of event triggering mechanisms have MIETs which are robust to disturbances by defining such robustness.
%Informally, a system is said to have the robust global event-separation property if it maintains a positive MIET %for all relevant initial conditions 
%in the presence of a bounded disturbance. %, provided the bound is small enough. 
%%If a system lacks this property, it indicates that it may trigger an undesirably large amount of events in a short period of time.

{
\emph{Statement of contributions:} This problem revisits a simple single-integrator multi-agent average consensus problem originally conceived in~\cite{DvdEf2009,DvdKhj2009,EkXwNh2010}. By re-formulating the problem using hybrid systems, we are able to provide a novel algorithm with several key fundamental improvements over similar event-triggered strategies in the literature. The contributions of this paper are threefold. First, we provide the first known fully distributed solution that guarantees a positive MIET. 
Second, we develop a method to design the triggering functions such that each agent is able to independently prescribe their guaranteed MIET in a distributed way; which has important implications on implementability of the proposed algorithms in real applications. Finally, we investigate the robustness of the MIET against both imperfect event detection, for which we provide an alternative robust trigger, and additive state disturbances, discussing its effects.} Simulations illustrate our results. 


\section{Preliminaries}\label{se:preliminaries}

The Euclidean norm of a vector $v \in \real^n$ is denoted by $||v||$. An n-dimensional column vector with every entry equal to $1$ is denoted by $\mathbf{1}_n$, and an n-dimensional column vector with every entry equal to $0$ is denoted by $\mathbf{0}_n$. 
The minimum eigenvalue of a square matrix $A$ is given by $\operatorname{eigmin}(A)$ and its maximum eigenvalue is given by $\operatorname{eigmax}(A)$. 
%The cardinality of a finite set $\mathcal{N}$ is denoted by $|\mathcal{N}|$.
The distance of $x$ from the set $\mathcal{A}$, which is $\operatorname{min}_a ||x-a||$, where $a \in \mathcal{A}$, is denoted by $||x||_\mathcal{A}$.
Given a vector~$v \in \real^N$, we denote by $\operatorname{diag}(v)$ the $N\times N$ diagonal matrix with the entries of $v$ along its diagonal.

Young's inequality is
\begin{align}\label{eq:young}
xy \leq \dfrac{a}{2}x^2 + \dfrac{1}{2a}y^2,
\end{align}
for $a>0$ and $x,y\in \real$~\cite{GhhJelGp1952}.

%By $V^{-1}(c)$, where $V(x)$ maps $\real^n$ to $\real$ and $c\in \real$, we denote the set of points $\{s \in \real^n : V(s) = c\}$. Similarly,
By $V^{-1}(C)$ where $C \subset \real^m$, we denote the set of points $\{s \in \real^n : V(s) \in C\}$, for a function $V: \real^n \rightarrow \real^m$.
By $\real_{\geq 0}$ we denote the set of nonnegative real numbers, and by $\integernonnegative$ we denote the set of nonnegative integers.
%A class $\mathcal{K}_\infty$ function $\alpha(t): \real_{\geq 0} \rightarrow \real_{\geq 0}$ is an increasing function of t such that $\alpha(0) = 0$ and $\alpha$ is unbounded.
The closure of a set $U \in \real^n$ is denoted by $\overline{U}$. %A precompact $U$ set is one for which $\overline{U}$ is compact.
The domain of a mapping $f$ is denoted by $\operatorname{dom}f$ and its range is denoted by $\operatorname{range}f$.
%i don't like referring to this as a graph, because we already discuss a type of graph in this paper
%The graph of a (set valued, in general,) mapping $M : \real^n \rightrightarrows \real^m$ is $\{(x,y) : y \in M(x)\}$.


\paragraph*{Graph Theory}
%An unweighted graph $\mathcal{G} = (V, \mathcal{E}, A)$ has a set of vertices $V = \{1, 2, ..., N\}$, a set of edges $\mathcal{E} \subset V \times V$, and an adjacency matrix $A \in \real^{N\times N}$ with each entry $a_{ij} \in \{0, 1\}$, where $a_{ij}=1$ if $(i, j) \in \mathcal{E}$, and $a_{ij} = 0$ otherwise. The set of neighbors for vertex $i$ is $\mathcal{N}_i = \{j:(i,j) \in \mathcal{E}\}$.
%A connected graph is one for which there exists a path between any two vertices. A complete graph is one for which $(i,j) \in \mathcal{E}\ \forall \ (i,j)\in V\times V$. The Laplacian of an unweighted, undirected graph $\mathcal{G}$ is $L = \operatorname{diag}([|\mathcal{N}_1|,\ |\mathcal{N}_2|,\ ... ,\ |\mathcal{N}_N|]) - A$. 
%For an undirected graph, $(i,j) \in V$ implies $(j,i) \in V$, but, for a digraph (directed graph), edge $(i, j)$ is distinct from edge $(j, i)$. 
An unweighted graph $\mathcal{G} = (V, \mathcal{E}, A)$ has a set of vertices $V = \{1, 2, ..., N\}$, a set of edges $\mathcal{E} \subset V \times V$, and an adjacency matrix $A \in \real^{N\times N}$ with each entry $a_{ij} \in \{0, 1\}$, where $a_{ij}=1$ if $(i, j) \in \mathcal{E}$, and $a_{ij} = 0$ otherwise. For a digraph (directed graph), edge $(i, j)$ is distinct from edge $(j, i)$.
 A path between vertex $i$ and vertex $j$ is a finite sequence of edges $(i,k)$, $(k,l)$, $(l,m)$, $\dots$, $(n,j)$. A digraph is strongly connected if there exists a path between any two vertices. 
A weighted digraph is one where each edge $(i, j) \in \mathcal{E}$ has a weight $w_{ij} > 0$ to it. For an edge $(i, j)$, $j$ is an out neighbor of $i$ and $i$ is an in neighbor of $j$. The in-degree, $d_i^\text{in}$, for a vertex $i$ is the sum of all the weights for the edges that correspond to its in neighbors, and the out-degree, $d_i^\text{out}$, is the same for its out neighbors. A weight-balanced digraph is a digraph where $d_i^\text{in}=d_i^\text{out} = d_i$ for each vertex $i$. A weighted digraph has a weighted adjacency matrix $A$ where the $ij$th element is the weight for edge $(i,j)$. For a weight-balanced digraph, the degree matrix $D^\text{out}=D^\text{in}$ is a diagonal matrix with $d_i$ as the $i$th diagonal element, and the Laplacian is $L=D^\text{out}-A$.

\paragraph*{Hybrid Systems}
A hybrid system $\mathcal{H} = (C,f,D,G)$ is a tuple composed of a flow set $C\in \real^n$, where the system state $x \in \real^n$ continuously changes according to $\dot{x}=f(x)$, and a jump set $D \in \real^n$, where $x$ discretely jumps to $x^+ \in G(x)$, where $f$ maps $\real^n \rightarrow \real^n$ and $G : \real^n \rightrightarrows \real^n$ is set valued~\cite[Definition 2.2]{RgRgsArt2012}.  While $x \in C$, the system can flow continuously and while $x \in D$, the system can jump discontinuously.

A compact hybrid time domain is a subset $E_\text{compact} \subset \real \times \natural$ for which $E_\text{compact} = \cup_{j=0}^{J-1}([t_j, t_{j+1}],j)$, for a finite sequence of times $0 \leq t_0 \leq t_1 \leq ... \leq t_J$, and a hybrid time domain is a subset $E\subset \real \times \natural$ such that $\forall (T,J)\in E$, $E \cap ([0,T]\times \{0,1,...J\})$ is a compact hybrid time domain~\cite[Definition 2.3]{RgRgsArt2012}. The hybrid time domain is used to keep track of both the elapsed continuous time $t$ and the number of discontinuous jumps $j$. %not sure if i need to explicitly define what a solution is here. i don't need to reference a definition of it directly

%i've cut anything from here that is *only* relevant to the proof.
See the appendix for more definitions and results relating to hybrid systems.

\section{Problem Formulation}\label{se:statement}

\renewcommand{\arraystretch}{1}
\begin{table}
\begin{center}
\begin{tabular}{|ll|}
\hline
Definition & Domain \\
\hline
$q_i = \left[x_i, \hat{x}_i, \chi_i \right]^T$ & $\in \real^3$ \\ %& extended state \\
$v_i = (q_i, \{ \hat{x}_j \}_{j \in \NN_i^\text{out}})$ & $\in \real^3 \times \real^{|\NN_i^\text{out}|}$ \\%& local information \\
$e_i = x_i - \hat{x}_i$ & $\in \real$ \\%& state error \\
$\hat{z}_i = (L\hat{x})_i = \sum_{j \in \Nouti} w_{ij}(\hat{x}_i - \hat{x}_j)$ & $\in \real$ \\
$\hat{\phi}_i = \sum_{j \in \Nouti} w_{ij}(\hat{x}_i - \hat{x}_j)^2$ & $\in \realnonnegative$ \\
\hline
\end{tabular}
\end{center}
\caption{Agent~$i$ model definitions.}\label{tab:notation}
\end{table}


%\begin{table}[htb]
%\framebox[.9\linewidth]{\parbox{.85\linewidth}{
%\begin{center}

%Global terms\\
%\vspace{1ex}
%\begin{tabular}{c c}
%\hline\\
%$\hat{z} = L\hat{x}$ & control input of the entire system\\
%$\bar{x} \triangleq \dfrac{1}{N}\sum_{i=1}^{N}x_i(0)$ & average initial state \vspace{1ex}\\
%\hline\\
%\end{tabular}
%
%For agent~$i$\\
%\vspace{1ex}
%\begin{tabular}{c}
%\hline\\
%%$x_i$ & physical state\\
%%$\hat{x}_i$ & last broadcast state\\
%%$\chi_i$ & dynamic virtual state\\
%$q_i = \left[x_i, \hat{x}_i, \chi_i \right]^T$\\% & extended state\\
%$v_i(t) = (q_i(t), \{ \hat{x}_j(t) \}_{j \in \NN_i^\text{out}})$\\% & available information\\
%%$1/\tau_i$ & max communication rate\\
%%$t^i_\ell$, $\ell \in \mathbb{Z}_{> 0}$ & $\ell$th event time\\
%$e_i = x_i - \hat{x}_i$\\% & state error\\
%$\hat{z}_i = (L\hat{x})_i = \sum_{j \in \Nouti} w_{ij}(\hat{x}_i - \hat{x}_j)$\\
%$\hat{\phi}_i = \sum_{j \in \Nouti} w_{ij}(\hat{x}_i - \hat{x}_j)^2$\vspace{1ex}\\
%%$\gamma_i(v_i) = \dot{\chi}_i$ & designed dynamics of $\chi_i$\\
%%$T_i$ & minimum inter-event time\\
%\hline\\
%\end{tabular}
%\end{center}
%\caption{Important notation}\label{tab:notation}
%}}
%\end{table}
We begin by stating a long-standing version of the event-triggered consensus problem. We then show why existing solutions to it are not pragmatic and how we reformulate the problem to obtain solutions that can be implemented on physical platforms. Please see Table~\ref{tab:notation} for a summary of notation used in the following.

Consider a group of $N$ agents whose communication topology is described by a directed, weight-balanced, and strongly connected graph $\mathcal{G}$ with edges $\mathcal{E}$ and Laplacian matrix $L$. Each agent is able to receive information from its out neighbors and send information to its in neighbors, and each weight of the graph is a gain applied to the information sent from one agent to another. 

The state of each agent~$i$ at time~$t \geq 0$ is given by~$x_i(t)$ with single-integrator dynamics
\begin{align}\label{eq:dynamics}
\dot{x}_i(t) = u_i(t),
\end{align}
where $u_i$ is the input for agent~$i$. It is well known that the input
\begin{align}\label{eq:idealcontrol}
u_i(t) = -\sum_{j \in \Nouti} w_{ij} (x_i(t) - x_j(t) )
\end{align}
drives all agent states to the average of the initial conditions~\cite{RosRmm2004}, which is defined as
\begin{align*}
\bar{x} \triangleq \dfrac{1}{N}\sum_{i=1}^{N}x_i(0).
\end{align*}
Note that under the control law~\eqref{eq:idealcontrol}, the average~$\bar{x}(t)$ is an invariant quantity. Defining $x = [x_1, x_2, \dots, x_N ]^T$ and $u = [u_1,u_2, \dots, u_N]^T$ as the vectors containing all the state and input information about the network of agents, respectively, we can describe all inputs together by
\begin{align*}
u(t) = -Lx(t). 
\end{align*}
However, in order to implement this control law, each agent must have continuous access to the state of each of its out neighbors. Instead, we assume that each agent~$i$ can only measure its own state~$x_i$ and must receive neighboring state information through wireless communication. We consider event-triggered communication and control where each agent only broadcasts its state to its neighbors at discrete instances of time. More formally, letting $\{t_\ell^i \}_{\ell \in \integernonnegative} \subset \realnonnegative$ be the sequence of times at which agent~$i$ broadcasts its state to its in neighbors~$j \in \NN_i^\text{in}$, the agents instead implement the control law
\begin{align}\label{eq:input}
u(t) &= -\hat{z} \triangleq -L\hat{x}(t) ,
\end{align}
where $\hat{x} = [ \hat{x}_1 , \dots ,\hat{x}_N ]^T$ is the vector of the last broadcast state of each agent. More specifically,
given the sequence of broadcast times~$\{t_\ell^i\}_{\ell \in \integernonnegative}$ for agent~$i$, we have 
\begin{align}\label{eq:communication}
\hat{x}_i(t) &= x_i(t^i_\ell) \quad \text{for} \quad t \in [t^i_\ell,t^i_{\ell+1}).
\end{align}
Note that the input~\eqref{eq:input} still ensures that the average of all agent states is an invariant quantity because $\dot{\bar{x}} = \dfrac{1}{N}\mathbf{1}_N^T\dot{x} = \dfrac{1}{N}\mathbf{1}_N^T(-L\hat{x})=0$, which follows from the weight-balanced property of the graph.

At any given time~$t \geq 0$, we define
\begin{align}\label{eq:localmemory}
v_i(t) \triangleq (x_i(t),\hat{x}_i(t),\{\hat{x}_j(t)\}_{j \in \Nouti})
\end{align}
as all the dynamic variables locally available to agent~$i$. The problem of interest, formalized below, is then to obtain a triggering condition based on this information such that the sequence of broadcasting times~$\{t_\ell^i\}_{\ell \in \integernonnegative}$, for each agent~$i$, guarantees that the system eventually reaches the average consensus state. 

\begin{problem}\longthmtitle{Distributed Event-Triggered Consensus}\label{pr:mainoriginal}
Given the directed, weight-balanced, and strongly connected graph $\mathcal{G}$ with dynamics~\eqref{eq:dynamics} and input~\eqref{eq:input}, find a triggering condition for each agent~$i$, which depends only on locally available information $v_i$, such that
$x_i \rightarrow \bar{x}$ for all~$i \in \until{N}$. 
\end{problem}
%~\cite{DvdEf2009,DvdKhj2009,EkXwNh2010}

This problem was first formulated in~\cite{DvdEf2009,DvdKhj2009} in 2009. Since then, there have been many works dedicated to this problem in both the undirected~\cite{XmTc2013,GssDvdKhj2013,DvdEfKhj2012,EgYcHyPaDc2013,XyKlDvdKhj17,JbCn2019} and directed~\cite{XmLxYcsCnGjp2015,CnJc2016,PxCnZt2018} cases. %Unfortunately, we are not aware of a single solution to this problem that can actually be implemented on physical platforms. 
%More specifically, although the above referenced papers provide theoretical solutions to the problem, trying to actually implement these solutions immediately reveals some prohibitive problems. 
Unfortunately, although the above referenced papers provide theoretical solutions to this problem, we are not aware of a single solution which can be implemented on physical systems when considering some practical concerns which are described next. For details on the history of this problem and its many theoretical solutions we refer the interested reader to~\cite{CnEgJc2019}, but we summarize the main points here. 

The earliest solutions to this problem did not adequately investigate the Zeno phenomenon which invalidates their correctness~\cite{DvdEfKhj2012,EgYcHyPaDc2013}. In particular, these solutions to Problem~\ref{pr:mainoriginal} did not rule out the possibility of Zeno behavior meaning that it was possible for a sequence of broadcasting times~$\{t_\ell\}_{\ell \in \integernonnegative}$ to converge to some finite time~$t_{\ell} \rightarrow T > 0$. This is clearly troublesome since all theoretical analysis then falls apart after~$t > T$, invalidating the asymptotic convergence results. 
More recently, the community has acknowledged the importance of ruling out Zeno behavior to guarantee that all the sequences of times~$t_\ell^i \rightarrow \infty$ as~$\ell \rightarrow \infty$ for all~$i \in \until{N}$. While enforcing this additional constraint on the sequences of broadcasting times guarantees that \emph{theoretically} the solutions will converge to the average consensus state, there are still some important practical issues that must be considered.

Even if it can be guaranteed that Zeno behaviors do not occur and the inter-event times are strictly positive for all agents~$i$, 
\begin{align*}
t_{\ell+1}^i - t_{\ell}^i > 0 ,
\end{align*}
unfortunately this is still not enough to guarantee that the solution can be realized by physical devices. This is because although the inter-event times are technically positive, they can become arbitrarily small to the point that no physical hardware exists that can keep up with the speed of actions required by the event-triggered algorithm. The solutions in~\cite{GssDvdKhj2013},~\cite{BcZl2017},~\cite{XyKlDvdKhj17}, and~\cite{PxCnZt2018} have this problem. 
This is inherently different from guaranteeing a strictly positive MIET~$\tau$, where $t_{\ell+1}^i - t_\ell^i \geq \tau > 0$, which is the focus of our work here. 

Specifically, we consider the case where each agent~$i$ has some maximum rate $\frac{1}{\tau_i}$ at which it can take actions (e.g., broadcasting information, computing control inputs). That is, each agent~$i$ cannot broadcast twice in succession in less than~$\tau_i$ seconds. In other words, each agent~$i \in \until{N}$ is limited by hardware in terms of how fast they are able to take actions,
\begin{align*}
t_{\ell+1}^i - t_\ell^i \geq \tau_i,
\end{align*}
for all~$\ell \in \integernonnegative$.
 Note that there are also solutions that guarantee a MIET, but make other sacrifices to do so. For example, the solution in~\cite{GssDvdKhj2013} is able to guarantee a MIET under certain conditions, but convergence is only to a neighborhood of consensus. 
Additionally, the algorithms in~\cite{VsdMaWpmhh2017,AaAaAm2018} are able to enforce a MIET, but only by using global parameters of the system to design the algorithm, which is impractical in cases where the parameters may change or are otherwise difficult to measure. 
The algorithm in~\cite{CdpRp2017} is fully distributed and has a positive MIET, but it still requires pair of agents to trigger events at the same time, which necessitates synchronization. 
With all this in mind, we reformulate Problem~\ref{pr:mainoriginal} such that solutions to the problem can be implemented on physical platforms given that each agent~$i$ is capable of processing actions at a frequency of up to~$\frac{1}{\tau_i}$.


%Note that this problem has not been solved with these restrictions before. The algorithms in~\cite{DvdEfKhj2012} and~\cite{EgYcHyPaDc2013} cannot exclude Zeno behavior, so they cannot guarantee convergence in all cases. The ones in~\cite{XmTc2013} and~\cite{CnJc2016} require synchronization between agents to prevent Zeno behavior, which is impractical for agents which do not share a common clock. The algorithm in~\cite{XyKlDvdKhj17} is able to prevent Zeno behavior without those requirements, but it is still unable to guarantee a positive MIET, which makes it unimplementable on real systems which cannot respond arbitrarily fast. The algorithm in~\cite{VsdMaWpmhh2017} is able to enforce a MIET, but only by using global parameters of the system to design the algorithm, which is impractical in cases where the parameters may change or otherwise be difficult to measure.

\begin{problem}\longthmtitle{Distributed Event Triggered Consensus with Designable MIET}\label{pr:main}
Given the directed, weight-balanced, and strongly connected graph $\mathcal{G}$ with dynamics~\eqref{eq:dynamics}, input~\eqref{eq:input}, and the minimum periods~$(\tau_1, \dots, \tau_N)$ for each agent, find a triggering
condition for each agent $i$, which depends only on local information $v_i$, such that
$x_i \rightarrow \bar{x}$ and 
\begin{align}\label{eq:MIETcondi}
\min_{\ell \in \integernonnegative} t^i_{\ell + 1} - t^i_\ell \geq \tau_i ,
\end{align}
for all~$i \in \until{N}$.
\end{problem}

{To the best of our knowledge, Problem~\ref{pr:main} has not yet been fully solved. Rather than being able to guarantee 
a strictly positive minimum inter-event time, similar works often settle for only ruling out Zeno executions meaning communication between agents may still need to occur arbitrarily fast in order for the convergence results to hold. Conversely, other works more simply force a MIET through the use of a dwell-time at the cost of losing the exact asymptotic convergence guarantee in exchange for practical consensus. The works~\cite{GssDvdKhj2013,BcZl2017,XyKlDvdKhj17} only preclude Zeno behavior but cannot guarantee that~\eqref{eq:MIETcondi} holds even for arbitrarily small minimum periods~$\tau_i$. The algorithm proposed in~\cite{VsdMaWpmhh2017} comes close but requires global system information. The methodology used in~\cite{CdpRp2017} is similar to ours here but ultimately solves a different problem. Thus, we provide the first complete solution to Problem~\ref{pr:main}.} For now we consider no state disturbances and perfect event detection but will relax these assumptions in Section~\ref{se:robustness} where we study the robustness of the {event-triggering strategy} with respect to various forms of disturbances. 

\iffalse{\color{blue}
\subsection{Motivating Example}
We present this as more practical example to motivate the problem statement. 

Consider a network of $N = 50$ {(i want this to be pretty large scale, but i dunno how to define the graph?)} smart sensors distributed over an area. The goal is to obtain the average temperature over the area, in a timely manner. Communication is limited due to power constraints, so a distributed solution is needed. Therefore, the sensors are constrained to communicate only by broadcasting to their out-neighbors on a digraph. Each agent $i$'s state $x_i$ is its estimate of the average temperature, and each sensor uses single integrator dynamics to update this estimate as new information is received. We know that, with proper choice of communication times, the input given by~\eqref{eq:input} will accomplish this, and there are many works that present event-triggered communication strategies to do so.

However, we consider senors with finite communication speeds $1/\tau_i$ for each agent $i$, such that each agent $i$ can only broadcast a message to its out neighbors once in every time window of length $\tau_i$, which these existing algorithms cannot account for. The heterogeneous communication periods are $\tau_i = 0.01$ for the newest sensors (agents ), $\tau_i = 0.05$ for older sensors (agents ), and $\tau_i = 0.5$ for one especially out-of-date sensor.

Additionally, we wish the algorithm to behave in the presence of disturbances {(i still must quantify robustness somehow)}.

To the best of our knowledge, no existing work can solve this problem, but by solving Problem~\ref{pr:main}, we will have an algorithm that can do so.
} \fi

\subsection{Hybrid Systems Formulation}
In order to solve Problem~\ref{pr:main}, we first reformulate it using hybrid systems tools; similar to~\cite{CdpRp2017}. {We refer to the original state~$x$ as the `physical' state that represents the actual state that we wish to control. Separately, we maintain a set of `cyber' or virtual states corresponding to the internal memory of each agent. Given the communication model described by~\eqref{eq:communication}, it seems natural to keep track of the last broadcast state~$\hat{x}_i$ for each agent as one of the cyber states. Additionally, we introduce an extra virtual state $\clovar_i$ for each agent~$i\in\until{N}$ to introduce dynamics into our triggering strategy and collect these components in the vector~$\clovar = [\clovar_1, \clovar_2, \dots, \clovar_N]^T$. Note that the internal variable~$\clovar_i$ is only available to agent~$i$. In this work we consider scalar internal variables~$\clovar_i \in \real$ but note that more sophisticated controllers or even learning-based controllers could be captures by increasing the complexity of the internal variables. The hybrid systems formulation will aid us here in properly modeling both the continuous-time dynamical system with discrete-time memory and control updates.}

{We now define the extended state vector for agent~$i$ as
\begin{align*}
q_i(t) = \left[ \begin{array}{c}
x_i(t)\\
\hat{x}_i(t)\\
\clovar_i(t)
\end{array} \right],
\end{align*} 
and the extended state (capturing both `physical' and `cyber' states) of the entire system is $q = [q_1^T, q_2^T, \dots, q_N^T]^T \in \real^{3N}$. 

With a slight abuse of notation, we now redefine the local information~\eqref{eq:localmemory} available to agent~$i$ at any given time as its own extended state~$q_i$ and the last broadcast states of its out-neighbors~$\{\hat{x}_j\}_{j \in \NN_i^\text{out}}$,
%
%Letting~$\clovar_i$ be an internal dynamic variable only available to agent~$i$, with a slight abuse of notation we redefine the information locally available to agent~$i$ by augmenting it with the additional variable~$\clovar_i$, 
\begin{align}\label{eq:localmemory2}
v_i(t) \triangleq (q_i(t), \{ \hat{x}_j(t) \}_{j \in \NN_i^\text{out}}) . 
\end{align}
\
}

In this work, the goal is to use the internal variable~$\clovar_i$ to determine exactly when agent~$i$ should broadcast its current state to its neighbors. To achieve this we let the state take values~$\clovar_i \geq 0$ and prescribe an event-trigger whenever~$\clovar_i = 0$. The exact dynamics of~$\clovar_i$ will be designed later in Section~\ref{se:design} to guarantee a solution to Problem~\ref{pr:main}.
%
%The exact dynamics of each $\clovar_i$ will be designed later as part of our algorithm. {The intuitive idea of the virtual state~$\clovar_i \geq 0$ is a measure of the beneficial contribution of an agent toward consensus since its most recent broadcast. %It will increase after a broadcast, as $\hat{x}_i$ will be close to $x_i$, but it may eventually decrease as the information in $\hat{x}_i$ becomes outdated. 
%This will be made more mathematically precise later, and this variable $\chi_i$ will drive the timing of events. We define events as occurring when this beneficial contribution since the last broadcast reaches zero and $\hat{x}_i \neq x$ for each agent $i$; the latter condition ensuring that each agent only broadcasts when it has new information to share. 
%%Additionally, with the benefit of hindsight, we allow for a secondary trigger function $h_i(v_i)$ and its purpose for specific circumstances will become clear later in the paper. There is one more additional concern; while the dynamics of $\chi_i$ will be designed to ensure consensus while exhibiting a positive minimum inter-event time, external disturbances may disrupt this positive minimum inter-event time property. Therefore, we constrain our algorithm for each agent $i$ to only check for events once the minimum inter-event time $T_i$ has passed. 
%Thus, the triggering condition is 
%\begin{align}
%\clovar_i = 0 \text{ and } \hat{x}_i \neq x_i . 
%\end{align} 
More specifically, at any given time~$t \geq t_\ell^i$, the next triggering time~$t_{\ell+1}^i$ is given by
\begin{align}\label{eq:trigFunc}
t^i_{\ell+1} = \inf \{t \geq t^i_{\ell}: \clovar_i(t) = 0 \text{ and }\hat{x}_i \neq x_i \}, 
\end{align}
for all~$\ell \in \integernonnegative$, for each agent $i$. 

%{
%Note that we want the dynamics of~$\clovar_i$ to be determined by only the information~$v_i$ that is available to agent~$i$. Thus, let $\gamma_i(v_i) \triangleq \dot{\clovar}_i$ be the function describing the dynamics of~$\clovar_i$ that is to be designed in Section~\ref{se:design}. 
%}
%
%
%{\color{blue}With the the role of our clock-like variable defined, we are ready to rewrite the system dynamics as a hybrid system, as in~\cite{CdpRp2017}. 
%Note that hybrid systems simply provides a convenient formalism for the problem, but reformulating the  event-triggered system in this way does not change how it functions, and the results could have be achieved through more conventional analysis. %, works such as~\cite{RpPtDnAa2015,CdpRp2017,VsdMaWpmhh2017}.. 
With the role of the new internal variable~$\clovar_i$ established (although its dynamics will be designed later), we can now formalize our hybrid system
\begin{align}\label{eq:system} 
\mathcal{H} = (C,f,D,G).
\end{align}
We refer to the appendix and~\cite{RgRgsArt2012} for formal hybrid systems concepts and assume the reader is familiar with this formalism.

\noindent \textbf{Flow set} (between event-triggering times):
The flow set~$C$ for the entire system is then given by
\begin{align}\label{eq:flowset}
C = \{q \in \real^{3N}: \clovar_i \geq 0 \text{ for all } i \in \until{N} \}.
\end{align}
While the system state~$q \in C$, the system flows according to~$f$
\begin{align}\label{eq:flowdynamics}
\dot{q} = f(q) = \left[ \begin{array}{c}
f_1(v_1)\\
\vdots\\
f_N(v_N)\\
\end{array} \right] \quad \text{for } q \in C,
\end{align}
with the individual extended states evolving according to
\begin{align}
\dot{q}_i = f_i(v_i) \triangleq \left[ \begin{array}{c}
-\hat{z}_i\\
0\\
\gamma_i(v_i)
\end{array} \right],
\end{align}
where $(L\hat{x})_i$ denotes the $i$th element of the column vector $L\hat{x}$ and~$\gamma_i$ is the function to be designed. 

The first row is exactly~$\dot{x}_i~=~u_i$ as defined in~\eqref{eq:input}, the second row says the last broadcast state is not changing between event-triggers~$\dot{\hat{x}_i}~=~0$, and the last row is the dynamics of the internal variable~$\dot{\clovar_i}~=~\gamma_i(v_i)$. Recall that~$v_i$ defined in~\eqref{eq:localmemory2} only contains information available to agent~$i$ and thus the dynamics of the internal variable~$\clovar_i~=~\gamma_i(v_i)$ must be a function only of~$v_i$.


%where $\gamma(q) = [\gamma_1(w_1),\dots, \gamma_N(w_N)]^T$ denotes the vector of functions to be designed.
%Let $\tilde{\clovar}_i \in [0,\overline{\clovar}_i)$ be the highest value of $\clovar_i$ at which agent $i$ can broadcast; the existence of such a value will be necessary to ensure a MIET exists for each agent. 

\noindent \textbf{Jump set} (at event-triggering times):
The jump set~$D$ for the entire system is then given by
\begin{align}\label{eq:jumpset}
D = \cup_{i=1}^N \{q \in \real^{3N}: \clovar_i \leq 0 \}.
\end{align}
Although in a general hybrid system there may be no notion of distributed information, since in this work the jumps correspond to \emph{some} agent triggering an event, we formally define
\begin{align}\label{eq:jumpsetlocal}
D_i \triangleq \{q \in \real^{3N}: \clovar_i \leq 0 \}
\end{align}
as the subset of~$D$ corresponding to when specifically agent~$i$ is responsible for the jump. 
%Unfortunately this raises a minor complication in the use of hybrid systems tools applied to the multi-agent system of interest here.....
{For~$q \in D_i$, we consider the following local jump map
\begin{align*}
g_i(q) = \left[ \begin{array}{c} q_1^+ \\ \vdots \\ q_i^+ \\ \vdots \\ q_{N}^+ \end{array} \right] \triangleq \left[ \begin{array}{c} q_1 \\ \vdots \\ \left( \begin{array}{c} x_i \\ x_i \\ \clovar_i\end{array} \right) \\ \vdots \\ q_{N} \end{array} \right].
\end{align*}
More specifically, letting~$t_\ell^i$ be the time at which agent~$i$ triggers its~$\ell$th event~$q(t_\ell^i) \in D_i$, this map leaves the physical state and the dynamic variable unchanged~$x_i^+ = x_i,\chi_i^+ = \chi_i$, and updates its ``last broadcast state'' to its current state~$\hat{x}_i^+ = x_i$.} Note also that this leaves all other agents' states unchanged~$q_j^+ = q_j$ for all~$j \neq i$.

%However, note that $g(q)$ is only piecewise continuous because, for a single agent $k$, its next state will either be $q_k^+$ or $q_k$, depending on the value of $q$. The graph of $q$ will be open on one side of this discontinuity. Therefore, $g$ is not outer semicontinuous by Lemma~\ref{th:OSCcondi} and the function fails the second condition of~\ref{th:nomWellPosed}.

Now, since multiple agents may trigger events at once, the jump map must be described by a set-valued map~$G : \real^{3N} \rightrightarrows \real^{3N}$~\cite{RgsJjbbNvdwWpmhh2014,CdpRp2017}, where
\begin{align}\label{eq:jumpDyn}
G(q) \in \{\dots, g_i(q), \dots\},
\end{align}
for all $i$ such that $q \in D_i$. Note that this construction of the jump map ensures that it is outer-semicontinuous, which is a requirement for some hybrid systems results.

%This allows multiple jumps to occur consecutively, until all agents which meet the trigger conditions have broadcast, one at a time. 
%The order in which this occurs has no physical meaning, because they all still occur at the same physical time $t$. This note is especially relevant when two (or more) agents can trigger simultaneously, i.e.,~$q \in D_i, D_j$ for some~$i \neq j$. 

%Note that the jump set overlaps the flow set when $\clovar_i \geq 0$ for $i = 1, 2, \dots, N$ and $\exists\ j$ such that $\clovar_j \leq \tilde{\clovar}_j$, indicating that the system can either jump or continue to flow in this region. Thus, this model applies to any system where each agent $i$ triggers an event while $\clovar_i \in [0, \tilde{\clovar}_i]$. 
Now, we reformulate Problem~\ref{pr:main} in a more structured manner by using the hybrid system~\eqref{eq:system}. More specifically, by using this formulation we have formalized the objective of finding a local triggering strategy to designing the function~$\gamma_i$ that depends only on the local information~$v_i$ defined in~\eqref{eq:localmemory2}.
%note that AaAaAm2018 has a built in MIET, because they build sampling into their event-triggering mechanism. FxTc2016 is edge triggered, so not exactly the same problem as us.

\begin{problem}\longthmtitle{Distributed Event-Triggered Consensus with Designable MIET}\label{pr:reform}
Given the directed, weight-balanced, and strongly connected graph $\mathcal{G}$ with dynamics~\eqref{eq:dynamics}, input~\eqref{eq:input}, and the minimum periods~$(\tau_1, \dots, \tau_N)$ for each agent, find the dynamics of the clock-like variable, $\gamma_i(v_i)$, such that $x_i \rightarrow \bar{x}$ and 
\begin{align*}
\min_{\ell \in \integernonnegative} t^i_{\ell + 1} - t^i_\ell \geq \tau_i ,
\end{align*}
for all~$i \in \until{N}$.
\end{problem}

\section{Dynamic Event-Triggered Algorithm Design}\label{se:design}
In order to solve Problem~\ref{pr:reform}, we perform a Lyapunov analysis to design~$\gamma \triangleq \dot{\clovar}$, the dynamics of $\clovar$.
Inspired by~\cite{CdpRp2017}, we use a Lyapunov function with two components: $V_P$ represents the physical aspects of the system, while $V_C$ represents the cyber aspects, related to communication and error. 
%Additionally, we consider two different physical Lyapunov functions, $V_{P1}$ and $V_{P2}$. 
For convenience, let $e \triangleq x - \hat{x}$ denote the vector containing the error for each agent's state, which is the difference between the actual state and the last broadcast state. We begin by considering 
%Inspired by~\cite{PxCnZt2018} for the second physical Lyapunov function, we let
{
\begin{align*}
V_{P}(q) &= (x - \bar{x})^T(x - \bar{x}) = ||x - \bar{x}||^2\\
V_C(q) &= \sum_{i=1}^N \clovar_i.
\end{align*}}
%where we define $\mathcal{X} = \operatorname{diag}(\clovar)$.  %Note that $V_{P2} \geq 0$ because $L$ is positive semi-definite, and 
Note $V_C \geq 0$ because $\clovar_i \geq 0$ for all $i \in \until{N}$. We then consider the Lyapunov function
\begin{align}\label{eq:LyapFunc}
V(q) = V_{P}(q) + V_C(q) .
\end{align}
%\begin{align*}
%V_2(q) = V_{P2}(q) + V_C(q).
%\end{align*}
Note that $V(q) \geq 0$ is continuously differentiable for all $q \in \real^{3N}$. {Moreover,%~$V_2(q) = 0$ only when the agents have reached consensus and each agent's error or clock is equal to 0, and
~$V(q)=0$ when all agents have reached their target state and each clock-like variable~$\clovar_i$ is equal to 0.} %We define 
%\begin{align}\label{eq:finalset}
%\mathcal{A} & \triangleq  \{ q \in \real^{3N} : V(q) = 0 \} \\ \notag
%\end{align}%&= \{q \in \real^{3N} : V_C(q) =0 \text{ and }  ||x - \bar{x}||^2 = 0\}
%as this target set we want our hybrid system to reach. 

Now we will examine the evolution of $V$ along the trajectories of our algorithm to see under what conditions it is nonincreasing, and design~$\gamma$ accordingly. 
{In order to do so, we will have to split $\dot{V}$ into components $\dot{V}_i$ such that $\dot{V} = \sum_{i=1}^N \dot{V}_i$ and each $\dot{V}_i$ depends only on the local information~$v_i$ available to agent~$i$. Choosing $V$ properly to ensure that $\dot{V}$ can be split like this is essential to designing $\gamma_i$ and doing so is nontrivial.}
%After performing this analysis for $V_1$, we will present the results, $\gamma_2 \triangleq \dot{\clovar}$, for $V_2$, which can be found using the same analysis for the second Lyapunov function but has been omitted for space.
%Then, the dynamics of the clock, $\gamma_1$ for $V_1$ and $\gamma_2$ for $V_2$, will be used to ensure that these conditions are always met, resulting in one algorithm for each Lyapunov function.% The Lyapunov functions and the resulting analysis is different for the undirected and the directed graphs (because L is different for those graphs), so we consider them one at a time. Let $\gamma_u$ be the clock dynamics for the undirected graph, and $\gamma_d$ be the clock dynamics for the directed graph.
%Consider the physical Lyapunov function
%\begin{align*}
%V_{P1} = \dfrac{1}{2}(x - \bar{x})^T(x - \bar{x}).
%\end{align*}
{
Recalling our system flow dynamics~\eqref{eq:flowdynamics}, we write
\begin{align*}
\dot{V}_P &= -2(x - \bar{x})^T\hat{z}\\
\dot{V}_C &= \sum_{i=1}^N \gamma_i.
\end{align*}
Because the graph is weight-balanced, $\bar{x}^TL = \mathbf{0}_N^T$. Therefore, $\dot{V}_{P} = -2x^T\hat{z} = -2\hat{x}^T\hat{z} -2 e^T\hat{z}$ and
\begin{align*}
\dot{V} = \dot{V}_{P} + \dot{V}_C = -2\hat{x}^T\hat{z} - 2e^T\hat{z} + \sum_{i=1}^N \gamma_i.
\end{align*}
Expanding this out and using notation defined in Table~\ref{tab:notation} yields
\begin{align}\label{eq:V1dot}
\dot{V} &= \sum_{i=1}^N \left( -\sum_{j \in \Nouti} w_{ij}(\hat{x}_i - \hat{x}_j)^2 - 2e_i(\hat{z}_i  + \gamma_{i} \right) \\
\dot{V} &= \sum_{i=1}^N \dot{V}_i = \sum_{i=1}^N \left( -\hat{\sumvar}_i - 2e_i\hat{z}_i + \gamma_{i} \right) \nonumber.
\end{align}}

{
We are now interested in designing~$\gamma_i$ for each agent~$i \in \until{N}$ such that~$\dot{V}(q) \leq 0$. 
Therefore, we choose
\begin{align}\label{eq:clockdyn1}
\gamma_i = \sigma_i\hat{\phi}_i + 2e_i\hat{z}_i,
\end{align}
where $\sigma_i \in (0,1)$ is a design parameter. Note that whenever an agent~$i$ triggers an event the error~$e_i$ is immediately set to 0, and since~$\dot{\clovar}_i = \gamma_i \geq 0$ at these times we ensure that~$\clovar_i \geq 0$ at all times~$t \geq 0$. We can now write the derivative of the Lyapunov function as
\begin{align}\label{eq:vdot}
\dot{V} = \sum_{i=1}^N -(1-\sigma_i)\hat{\phi}_i \leq 0.
\end{align}}
{This choice of the clock-like dynamics~$\gamma_{i}$ is continuous in $q$ for constant $\hat{x}$ }{and ensures that~$\dot{V} \leq 0$.}


\subsection*{Algorithm Synthesis}
With the dynamics~$\gamma_i$ of the clock-like variable~$\clovar_i$ defined for each agent~$i \in \until{n}$, we can now summarize all the components of our synthesized distributed dynamic event-triggered coordination algorithm and formally describe it from the viewpoint of a single agent.

The control input at any given time~$t \geq 0$ is 
\begin{align*}
u_i(t) = -\hat{z}(t)_i = -\sum_{j \in \Nouti}w_{ij} (\hat{x}_i(t)-\hat{x}_j(t)).
\end{align*}
%where~$\hat{x}(t)$ is updated according to~\eqref{eq:input}. 
{The sequence of event times~$\{t_\ell^i\}_{\ell \in \integernonnegative}$ at which agent~$i$ broadcasts its state to neighbors is given by each time
the clock-like variable reaches zero when that agent's error is nonzero, i.e., 
\begin{align*}
t_\ell^i = \inf \{t \geq t_{\ell - 1}^i : \clovar_i(t) = 0 \text{ and } e_i \neq 0\}.
\end{align*}
}
The algorithm is formally presented in Table~\ref{tab:algorithm}.

\begin{table}[htb]
  \centering
  \framebox[.9\linewidth]{\parbox{.85\linewidth}{%
    \parbox{\linewidth}{Initialization; at time~$t = 0$ each agent~$i \in \until{N}$ performs:}
      \vspace*{-2.5ex}
      \begin{algorithmic}[1]
      \STATE Initialize~$\hat{x}_i = x_i$
      \STATE Initialize~$\clovar_i = 0$
      \end{algorithmic}
      \parbox{\linewidth}{At all times $t$ each agent $i \in \until{N}$
        performs:}
      \vspace*{-2.5ex}
      \begin{algorithmic}[1]
        \IF{
          $\clovar_i = 0$ and $e_i \neq 0$}  \label{algo:1}
        \STATE set $~\hat{x}_i = x_i$ (broadcast state information to neighbors)
        \STATE set $u_i = -\sum_{j \in \Nouti} w_{ij}(\hat{x}_i - \hat{x}_j)$ (update control signal)
        \ELSE
        \STATE propagate~$\clovar_i$ according to its dynamics~$\gamma_i$ in~\eqref{eq:clockdyn1}
        \ENDIF
        \IF{
          new information $\hat{x}_k$ is received from some neighbor(s)~$k \in \Nouti$} 
        %\IF{agent $i$ has broadcast its state at any time $t' \in (t - \epsilon_i, t)$} 
        %\STATE broadcast state information $x_i(t)$ 
        %\ENDIF
        \STATE update control signal~$u_i = -\sum_{j \in \Nouti} w_{ij}(\hat{x}_i - \hat{x}_j)$
        \ENDIF
      \end{algorithmic}}}
  \caption{Distributed Dynamic Event-Triggered Coordination Algorithm.}\label{tab:algorithm}
\end{table}
%\underline{\clovar}_i
% \textbf{and} $\hat{x}_i(t) - x_i(t) \neq 0$

%\vspace*{-2ex}
\section{Main Results}
%\vspace*{-2ex}
Here we present the main results of the paper by discussing the properties of our algorithm. We begin by finding the guaranteed positive minimum inter-event time (MIET) for each agent and showing how it can be tuned individually. 

\begin{theorem}[Positive MIET]\label{th:ldtResult}
Given the hybrid system $\mathcal{H}$, if each agent~$i$ implements the distributed dynamic event-triggered coordination algorithm presented in Table~\ref{tab:algorithm} {with~$\sigma_i \in (0,1)$,
%$\tilde{\clovar}_i \in [0, \overline{\clovar}_i)$, and $c_i < \operatorname{min} \dfrac{2}{d_j}$ for $j \in \mathcal{N}_i^\text{in}$,
 then the inter-event times for agent~$i$ are lower-bounded by
%\begin{enumerate}
%\item
\begin{align}\label{eq:MIET1}
T_i \triangleq \frac{\sigma_i}{d_i} > 0.
\end{align}
%(\tilde{\clovar}_i + 1)
%for Algorithm 1,
%where $d_i$ is the degree of agent $i$.
}
  %, and $W_i$ is a column vector of the weights $w_{ij}$ for $j \in \Nouti$.
%\item
%\begin{align}\label{eq:MIET2}
%T_{2,i} \triangleq & \sqrt{\dfrac{(2-C_i)c_i}{(1 + \epsilon_i)d_i}} \nonumber\\
%& \cdot \left(\operatorname{arctan}\left[\sqrt{\dfrac{c_i}{(2-C_i)(1 + \epsilon_i)d_i}}(\overline{\clovar}_i + d_i) \right] \right. \nonumber \\
%& \left. - \operatorname{arctan}\left[ \sqrt{\dfrac{c_i}{(2-C_i)(1 + \epsilon_i)d_i}}(\tilde{\clovar}_i + d_i) \right] \right) \nonumber\\
%& > 0,
%\end{align}
%for Algorithm 2.
%\end{enumerate}
That is,
\begin{align*}
t_{\ell+1}^i - t_\ell^i \geq T_i
\end{align*}
for all~$i \in \until{N}$ and~$\ell \in \integernonnegative$.% for Algorithm $k$, with $k \in \{1,2\}$.
\end{theorem}
\begin{IEEEproof}
See the appendix.
\end{IEEEproof}

%Note that the weights of the graph effect the MIETs for both algorithms through $\lambda_{min,i}$, $\lambda_{max,i}$, $C_i$, $c_i$, and $d_i$. To quantify this, . Note that this $K$ acts as a gain for the input for each agent, so it will change the speed at which each agent moves and the amount of time before convergence. 

%\begin{remark}[Remark name?]\label{re:MIET}
%{\rm
%We can upper bound the MIETs which can be achieved by this algorithm. For an undirected graph, setting $\tilde{\clovar}_i = 0$ and $\sigma_i = 1$, and taking the limit as  $\overline{\clovar}_i \rightarrow \infty$ and $\beta_i \rightarrow 1$, we have
%\begin{align}\label{eq:MIETmax}
%\mathcal{T}_{u,max,i} = \dfrac{1}{|\mathcal{N}_i|}.
%\end{align}
%For a directed graph, the upper bound on the MIET for agent $i$ can be found by setting $\tilde{\clovar}_i = 0$ and $\sigma_i = 1$, and taking the limit as $\overline{\clovar}_i \rightarrow \infty$, $\beta_i \rightarrow 1$, $c_i \rightarrow \dfrac{2}{D_i} \triangleq \operatorname{min} \dfrac{2}{d_j}$ for $j \in \mathcal{N}_i^{in}$, and $C_i \rightarrow 0$. The result is
%\begin{align}\label{eq:MIETmaxd}
%\mathcal{T}_{d,max,i} = \sqrt{\dfrac{4}{d_iD_i}}\left(\dfrac{\pi}{2} - \operatorname{arctan} \sqrt{\dfrac{d}{D_i}} \right) .
%\end{align}
%However, note that this result is weaker than in the undirected case, because it cannot be achieved for all agents. For agent $i$ to have a long MIET, $c_i$ should be chosen high, but $C_i$, which is directly proportional to each $c_j$ for $j \in \mathcal{N}_i^{in}$, should be chosen low. This implies that choosing $c_i$ high increases the MIET for agent $i$ but decreases it for its out neighbors.
%} \oprocend
%\end{remark}

%\begin{remark}

%This algorithm can recover an asynchronous periodic strategy with $\sigma = 1$, with each agent $i$ sampling faster than $\dfrac{1}{|\mathcal{N}_i|}$, in the case of the undirected graph. This is an improvement over existing synchronous periodic strategies, which allow synchronous periodic updates at a rate of $ \operatorname{min}\left( \dfrac{1}{|\mathcal{N}_i|} \right) $, as shown in~\cite{RosJafRmm2007}.

%\end{remark}

%Therefore, $\forall\ T_i < T_{max\ i}$, $\exists\ \sigma_i \in (0,1)$ and $\beta_i > 1$ and $\overline{\clovar}_i > 0$ s. t. $T_i = T_{desired\ i}$. This provides an upper bound on the minimum inter-event time which can be achieved for an agent for given $\sigma_i$, $\beta_i$, and $|\mathcal{N}_i|$.
%this has been fixed, but is this the appropriate place for it?

%we'll jam both results together into one theorem and then itemize it. we'll also move the proof of the first part to the appendix
%the appendix should be split into multiple sections: the proof, and the extra hybrid stuff

%Let
%\begin{align}\label{setU}
%U(\bar{x}(0))\subset \real^{3N} \triangleq \{q \in \real^{3N} : \dfrac{1}{N} \sum_{i=1}^N x_i = \bar{x}(0)\}.
%\end{align}
%Let 
%\begin{align}\label{setAprime}
%\{q \in \real^{3N} : ||e|| = 0\text{ and } ||z||=0\}.
%\end{align}

%{\color{green}
%Discussion here on how the result Theorem~\ref{th:ldtResult} is our main contribution. unlike existing works, given a maximum operating speed of hardware we can design an algorithm that guarantees convergence because we designed our algorithm to guarantee a minimum inter-event time. this is in contrast with works that force a minimum dwell time and then can't guarantee asymptotic convergence all the way to the desired state or only conclude no Zeno behavior. 

%Also could describe in more detail the parameter~$\beta_i$. When we set this to 0 we recover an asynchronous but periodic implementation that guarantees convergence. You can then also show that this result is more conservative than what is already known by citing papers that find the the upper-bound on the periods that make the system converge. 

\begin{remark}[Design Trade off]\label{re:tradeoff}
{\rm
The design parameter $\sigma_i$ represents the trade off between larger inter-event times and faster convergence speeds. Larger $\sigma_i$ makes the magnitude of $\dot{V}$ smaller~\eqref{eq:vdot}. However, it is a coefficient of the nonnegative term in $\chi_i$'s dynamics~\eqref{eq:clockdyn1}, so larger $\sigma_i$ means longer inter-event times, and increasing it increases the MIET~\eqref{eq:MIET1}. 
Additionally, note that the MIETs can be guaranteed up to a maximum of {$T_{i,\text{max}} \triangleq \dfrac{1}{d_i}$ for each agent $i$, because $\sigma_i \in (0,1)$.} 
}
\end{remark}

Next, we present our main convergence result. To the best of our knowledge, this is the first work to design a fully distributed event-triggered communication and control algorithm that guarantees asymptotic convergence to the average consensus state with a lower bound on the agent-specific MIET that can be chosen by the designer.
%If this is not large enough, $\tau_i \geq T_{i,\text{max}}$, {then the graph itself must be adjusted so that agent $i$'s degree is smaller.}%then a global gain can be adjusted as explained later in Remark~\ref{rm:gainK}.

%Other works require global information or edge triggered strategies for similar results, or are only able to guarantee non-Zeno behavior without a local minimum inter-event time, or can only guarantee convergence to a neighborhood of consensus for an enforced dwell time, for examples.%cites
%probably expand this into multiple sentences and explicitly note what ours does that theirs do not do
%This is an improvement over existing synchronous periodic strategies, which allow synchronous periodic updates at a rate of $ \operatorname{min}(\dfrac{1}{|\mathcal{N}_i|)} $, as shown in~\cite{RosJafRmm2007}.

\begin{theorem}[Global Asymptotic Convergence]\label{th:mainResult}
Given the hybrid system~$\mathcal{H}$, if each agent~$i$ implements the distributed dynamic event-triggered coordination algorithm presented in Table~\ref{tab:algorithm} with agent $i$ triggering events when %$\clovar_i \in [0, \tilde{\clovar}_i]$
$\clovar_i = 0$ {and $\hat{x}_i \neq x_i$ with~$\sigma_i \in (0,1)$, then all trajectories of the system are guaranteed to asymptotically converge to the set
\begin{align*}
\{q : \hat{\phi}_i = 0\text{ } \forall\text{ } i\}.
\end{align*}}
\end{theorem}
\begin{IEEEproof}
See the appendix.
\end{IEEEproof}

{
\begin{remark}[Convergence]\label{rm:conv}
{\rm
From Theorem~\ref{th:mainResult}, the algorithm presented in Table~\ref{tab:algorithm} does not entirely solve Problem~\ref{pr:reform}, because agents only converge to $\{q : \hat{\phi}_i = 0\text{ } \forall\text{ } i\}$, and, in order for $x_i \rightarrow \bar{x}$ for each agent~$i$, we also require  and $e_i = 0$, $\forall i$. 
Therefore, if the algorithm is modified with an additional trigger which guarantees that each agent $i$ will always broadcast again, eventually, when $e_i \neq 0$, then full convergence is guaranteed. 
More formally, for $\ell \in \mathbb{Z}_{\geq 0}$, if we guarantee that $t^i_{\ell+1}<\infty$ when $\exists t \in [t^i_\ell,t^i_{\ell+1})$ such that $e(t) \neq 0$, then $x_i(t) \rightarrow \bar{x}$ as~$t \rightarrow \infty$, for all~$i \in \until{N}$. 

A very simple way to do so is to set a maximum time $T_\text{max}^i \geq T_i$ between events for each agent $i$, so that, if it triggers an event at time $t_\ell^i$, the latest another event can trigger is $t_\ell^i + T_i$.
}
\end{remark}
}

\subsection*{Minimum Inter-Event Time Design}
{As noted in Remark~\ref{re:tradeoff}, the design parameter $\sigma_i$ can be used to choose the desired MIET for agent $i$, up to a maximum of $T_{i,\text{max}}$.} According to Problem~\ref{pr:main}, we must be able to guarantee that the lower-bound on the MIET~$T_i$ as provided in Theorem~\ref{th:ldtResult} is greater than or equal to the prescribed~$\tau_i$. 

%From Theorem~\ref{th:ldtResult} we see that {\color{red}the guaranteed MIET is a linearly increasing function of $\sigma_i$ from Theorem~\ref{th:ldtResult}. Therefore, since $\sigma \in (0,1)$, we can write
% \begin{align}\label{eq:Tmax1}
%T_i < \frac{1}{d_i} \triangleq T_{i,\text{max}}
%\end{align}
%as an upper bound on MIETs achievable by choosing this parameter.} 
Consequently, if~$\tau_i < T_{i,\text{max}}$ for all~$i \in \until{N}$, then it is easy to see how~\eqref{eq:MIET1} in Theorem~\ref{th:ldtResult} can directly be used to choose the design {parameter} appropriately for each agent. 
{{In the case that there exists some agent(s)~$j$ such that~$\tau_j \geq T_{j,\text{max}}=\frac{1}{d_j}$,
 then the graph must be redesigned so that $d_j$ is lower for each agent $j$. Note that there exist distributed methods of choosing these gains for an existing strongly connected digraph so that it will be weight-balanced~\cite{BgJc2012,AirTcCnh2014}. However, further analysis here is beyond the scope of this paper.}
 
%One final note is that choosing a higher MIET $T_i$ with a higher $\sigma_i$ will result in slower convergence, as seen in \eqref{eq:vdot}, so there is a trade-off present here.}
% we must actually redefine the original controller~\eqref{eq:input} by introducing a scaling factor to essentially slow down the convergence of the entire network to accommodate the slow agents~$i$ with large~$\tau_i$. This is discussed in Remark~\ref{rm:gainK} below. 

%\begin{remark}[Arbitrary MIET Selection]\label{rm:gainK}
%{\rm
%The preceding discussion seems to imply that there is a maximum achievable MIET for each agent. However, this is because we have not yet considered that the weights of the graph might be adjusted in the control input~\eqref{eq:input}. 
%Note that there exist distributed methods of choosing these gains for an existing strongly connected digraph so that it will be weight-balanced~\cite{BgJc2012,AirTcCnh2014}. 
%However, to quantify this in a simple manner, let us consider the graph defined by the Laplacian $L' =KL$, where $K > 0$ is a gain applied to each agent's input, so that $L'$ is another weight-balanced Laplacian for the same set of vertices and set of edges. In this case,~\eqref{eq:Tmax1} % and~\eqref{eq:Tmax2}
% becomes
%\begin{align*}
%T_{i,\text{max}}' \triangleq \dfrac{w_{\text{min},i}}{K\mathbf{w}_i}.
%\end{align*}
%and
%\begin{align*}
%T_{2,i,\text{max}}' \triangleq & \dfrac{1}{K} \sqrt{\dfrac{(2 - C_i)c_i}{d_i}} \\
%& \cdot \left(\dfrac{\pi}{2} - \text{arctan}\ \left[K\sqrt{\dfrac{c_i}{d_i(2 - C_i)}}(\tilde{\clovar}_i + Kd_i)\right]\right),
%\end{align*}
%respectively.
% This indicates that, if the maximum achievable MIET is not high enough for some agent~$i$, e.g., $T_{i,\text{max}} < \tau_i$, then the gain $K$ can be chosen smaller than 1, allowing for the selection of arbitrarily large MIETs. However, this must be done with care, because the same gain $K$ must be applied to the input of each agent, to preserve the weight-balanced property of the graph, and because this will ultimately result in slower convergence.
 
%{\color{green} there's a comment on this remark which i dunno how to address}
%} \oprocend
%\end{remark}

\section{Robustness}\label{se:robustness}
{A problem with many event triggered algorithms is a lack of robustness
guarantees in the triggering strategies. In particular, we consider the effects of two different types of disturbances that are often problematic for event-triggered control systems, discussing how our algorithm can be implemented to preserve the MIET in the presence of additive state disturbances and providing a modification that makes it robust against imperfect event detection.} 

\subsection*{Robustness Against State Disturbances}
We first analyze the robustness of our MIET against state disturbances. As noted in~\cite{DpbWpmhh2014}, simply guaranteeing a positive MIET may not be practical if the existence of arbitrarily small disturbances can remove this property, resulting again in solutions that might require the agents to take actions faster than physically possible in an attempt to still ensure convergence. Therefore, it is desirable for our algorithm to exhibit robust global event-separation as defined in~\cite{DpbWpmhh2014}, which means that the algorithm can still guarantee a positive MIET for all initial conditions even in the presence of state disturbances, which is referred to as a robust MIET.

{
More formally, we desire the MIET given in~\eqref{eq:MIET1} to hold, even in the presences of arbitrary disturbances. Instead of the deterministic dynamics~\eqref{eq:dynamics}, consider
\begin{align}\label{eq:distributedynamics}
\dot{x}_i(t) = u_i(t) + w_i(t),
\end{align}
where~$w_i(t)$ is an arbitrary, unknown, additive state disturbance applied to each agent's state. However, ensuring that the MIET holds in these circumstances depends on the specific implementation, as discussed in the following remark.

\begin{remark}[Trigger Robustness]\label{rm:robTrig}
{\rm
Note that, following an event at time $t_\ell^i$, each agent $i$ does not even need to check the trigger condition~\eqref{eq:trigFunc} until time $t_\ell^i+T_i$, because, by Theorem~\ref{th:ldtResult}, it cannot be satisfied before that time, in the absence of any disturbance. Therefore, in implementation, each agent can wait $T_i$ seconds after triggering an event before triggering a new one. This will have no effect on the performance of the algorithm in the absence of disturbances, but it will ensure that the MIET is observed in the presence of disturbances of the form given in~\eqref{eq:distributedynamics}. 
Additionally, see the definition of the hybrid system $\mathcal{H}'$ in~\eqref{eq:modSys} for how this can be modeled in theory.
} \oprocend
\end{remark}
%\margin{I think this should be highlighted more to completely demonstrate that we are not concerned with the instability problems the disturbance might cause but rather just the instability in triggering, and standard robustness is beyond the scope of this paper and can be studied using more standard tools there.}
Note that this does not guarantee convergence all the way to consensus in the presence of disturbances, simply that the positive MIET will be preserved.} 
Analyzing the actual convergence properties in any formal sense is beyond the scope of this work. 
%The conditions of convergence would, of course, depend on the properties of the disturbance~$w_i(t)$, but it is easy to show that for zero-mean, slowly changing disturbances all agent states will asymptotically converge to their initial average in expectation. 
Instead, we consider the following simple example to show how the algorithm may handle a disturbance. 
%that the algorithm doesn't completely fall apart in the presence of noise. 
If each $w_i$ is an independent and identical Gaussian process with zero mean and variance $\sigma^2$, then dynamics of the average position, $\dot{\bar{x}}$, will be a random variable with $E[\dot{\bar{x}}] = 0$ and $\text{var}(\dot{\bar{x}}) = \sigma^2/N$. This indicates that $\bar{x}$ is a Wiener process, which has a Gaussian distribution with a mean equal to the initial average and a variance of $t\frac{\sigma^2}{N}$.  {We demonstrate the effects of such a disturbance in Section~\ref{se:simulations}.}

\subsection*{Robustness Against Imperfect Event Detection}
In addition to robustness against state disturbances, another important source of uncertainty that cannot be overlooked in event-triggered control systems
%\margin{would really like some good citation(s) here on the danger of imperfect event detection -- not for arxiv but for TAC}
is imperfect event detection. Event-triggered controllers are generally designed and analyzed assuming very precise timing of different actions is possible while continuously monitoring the event conditions. This is not only impractical but problematic if not accounted for in the event-triggered control design. 

The algorithm presented in Table~\ref{tab:algorithm} is no longer guaranteed to converge if there are delays in the triggering times. More specifically, let~$t_{\ell+1}^{i^*}$ be the time at which the event condition~\eqref{eq:trigFunc} is actually satisfied, but the condition is not actually detected until the actual triggering time~$t_{\ell+1}^i = t_{\ell+1}^{i^*} + \delta t^i_\ell$, where~$\delta t^i_\ell \in [0,\Delta_i]$ is a random delay in the detection of the event with an upper-bound of~$\Delta_i > 0$. We refer to this as triggering the $\ell$th event, for agent~$i$, $\delta t^i_\ell$ seconds late. An example where this would be the case is if the trigger condition for agent~$i$ is checked periodically, with frequency $1/\Delta_i$.

Fortunately, our algorithm is still guaranteed to converge as long as agent~$i$ triggers before~$\chi_i$ reaches~$0$. By utilizing the knowledge of the maximum delay~$\Delta_i$,  we can modify the local jump set~\eqref{eq:jumpsetlocal} that defines when agent~$i$ should trigger an event such that we are guaranteeing that all events are detected by the time~$\chi_i$ reaches 0 when accounting for the delay. Let
\begin{align*}
\tilde{D}_i = \{q \in \mathbb{R}^{3N} : h_i(v_i) \in [0, \Delta_i]\},
\end{align*}
where
\begin{align*}
h_i(v_i) = \left\lbrace
\begin{matrix}
\frac{\sigma_i}{d_i}\left( 1 - \frac{e_i^2}{\chi_i+e_i^{2}}\right), & \text{for } (\chi_i,e_i) \neq (0,0)\\
\frac{\sigma_i}{d_i}, & \text{otherwise}
\end{matrix}
  \right. .
\end{align*}

%To help formalize this into a modified triggering function, we present the following lemma. 
As will be shown in the proof, this provides a time window with a length of $\Delta_i$ seconds starting from when $h_i(v_i) = \Delta_i$ during which agent~$i$ can trigger an event and still guarantee $\chi_i \geq 0$. Therefore, we define a robust event trigger condition
\begin{align}\label{eq:modTrigger}
h_i(v_i) \leq \Delta_i .
\end{align}
%Note that with the bound~\eqref{eq:eventDelayBound}, this trigger guarantees that the event will be processed before $\chi_i$ becomes negative, that is, $\chi_i(t^{i*}_\ell + \delta t_{\ell}^i) \geq 0$. 
Note that the MIET with this new event-trigger condition is now $\frac{\sigma_i}{d_i} - \Delta_i$. 
}
This is formalized next.

{
\begin{theorem}[Robust Convergence with MIET]\label{th:robMIET}
Given the hybrid system $\mathcal{H}$, if each agent~$i$ implements the distributed dynamic event-triggered coordination algorithm presented in Table~\ref{tab:algorithm} and each event is triggered at most $\Delta_i$ seconds after~\eqref{eq:modTrigger} is satisfied, with $\sigma \in (0,1)$ and $\Delta_i \in [0, \frac{\sigma_i}{d_i})$, then all trajectories of the system are guaranteed to asymptotically converge to the set
\begin{align*}
\{q : \hat{\phi}_i = 0\text{ } \forall\text{ } i\},
\end{align*}
for all~$i \in \until{N}$. 
Additionally, the inter-event times for agent~$i$ are lower-bounded by
\begin{align}\label{eq:robMIET1}
\tilde{T}_{i} \triangleq & \frac{\sigma_i}{d_i} - \Delta_i > 0.
\end{align}
%That is,
%\begin{align*}
%t_{\ell+1}^i - t_\ell^i \geq \tilde{T}_i
%\end{align*}
%for all~$i \in \until{N}$ and~$\ell \in \integernonnegative$. %Moreover,~$\tilde{T}_i$ is a robust positive MIET for disturbances of the form~\eqref{eq:distributedynamics}.
% for Algorithm $k$, with $k \in \{1,2\}$.
\end{theorem}}
\begin{IEEEproof}
%The proof is the same as the proof for Theorem~\ref{th:ldtResult} and can be found in the appendix.
See the appendix.
\end{IEEEproof}


\begin{remark}[Trigger Robustness]\label{rm:robust}
{\rm
The implication of Theorem~\ref{th:robMIET} is the following. 
Intuitively, rather than agent~$i$ waiting for the condition~\eqref{eq:trigFunc} to be satisfied exactly and respond immediately, it simply begins triggering an event {when condition~\eqref{eq:trigFunc} could be satisfied soon, and as long as the event can be detected and fully responded to before then, the algorithm will work as intended.} However, note that this imposes a trade off because triggering earlier will result in a shorter guaranteed MIET, as shown in the result of Theorem~\ref{th:robMIET}.
} \oprocend
\end{remark}

%As a simple motivating example, consider a sampled-data implementation of the problem considered in this paper. Until now, we have simply assumed that each agent measures its own state continuously for simplicity, but, in practice, this must be done at discrete time instances, which can result in events being triggered imperfectly. More specifically, if the sampling rate for agent~$i$ is $1/\Delta_i$, then events may occur at most $\Delta_i$ seconds late. By Theorem~\ref{th:robMIET}, the triggering condition~\eqref{eq:modTrigger} can still guarantee convergence in this case, as long as $\Delta < \frac{\sigma_i}{d_i}$.
%{\color{red}
%\begin{corollary}[Sampled State Implementation]
%Given the hybrid system~$\mathcal{H}$, if each agent~$i$ implements the distributed dynamic event-triggered coordination algorithm presented in Table~\ref{tab:algorithm} except that each agent $i$ triggers events when $h_i \in [-\delta t^i, 0]$, with agent $i$ sampling its own state $x_i$ with a maximum time between samples of $T_{s,i}$ and with~$\sigma_i \in (0,1)$, then all trajectories of the system are guaranteed to asymptotically converge to the set $\{q : \hat{\phi}_i = 0\text{ } \forall\text{ } i\}$, if the following condition holds
%\begin{enumerate}
%\item
%\begin{align}\label{eq:minTimeToZero1}
%T_{s,i} \leq \delta t^i.
%\end{align}
%\end{corollary}
%}
%\begin{IEEEproof}

%{\color{red}
%For each agent~$i$, this trigger with the sampling period of $T_{s,i}$ ensures that events occur at most $T_{s,i}$ seconds after $h_i = 0$. By Lemma~\ref{lem:timeToEvent}, we can conclude that events happen when $\chi_i \geq 0$. By similar reasoning to the proof of Theorem~\ref{th:robMIET}, we can guarantee the existence of a MIET for each agent~$i$ of $\tilde{T}_i$. This allows us to apply Lemma~\ref{lem:genConverge} from the appendix, which guarantees asymptotic convergence of the system to the set~$\{q : \hat{\phi}_i = 0\text{ } \forall\text{ } i\}$.
%}
%\end{IEEEproof}

\section{Simulations}\label{se:simulations}
{
To demonstrate our distributed event-triggered control strategy, we perform various simulations using~$N = 5$ agents and a directed graph whose Laplacian is given by 
\begin{align*}
L = \left[ \begin{array}{ccccc}
 2 & -1 &  0 &  0 & -1\\
 0 &  2 &  0 &  0 & -2\\
-2 &  0 &  2 &  0 &  0\\
 0 & -1 & -2 &  3 &  0\\
 0 &  0 &  0 &  -3 & 3
\end{array} \right].
\end{align*}

Additionally, we consider that agents have a minimum operating period of $[\tau_1,\tau_2,\tau_3,\tau_4,\tau_5]=[0.4,0.25,0.25,0.1,0.2]$. Because the agents have out degrees of $[d_1,d_2,d_3,d_4,d_5]=[2,2,2,3,3]$, we use~\eqref{eq:MIET1} and choose $[\sigma_1,\sigma_2,\sigma_3,\sigma_4,\sigma_5] = [0.9,0.4,0.4,0.3,0.6]$, so that the guaranteed MIETs for the agents are $[T_1,T_2,T_3,T_4,T_5] = [0.45,0.2,0.2,0.1,0.2]$.

All simulations use the same initial conditions of $\hat{x} = x = [-1, 0, 2, 1, 2]^T$ in order to explore the effects of the different design parameters. 
%Each simulation was run until $\sum_{i=1}^N \left(x(t) - \bar{x}\right)^2 \leq 10^{-4}$. %$t_{\text{end}}$, where $t_{\text{end}} = \text{min}\ \{t>0 | \sum_{i=1}^N \left(x(t) - \bar{x}\right)^2 \leq 10^{-4}\}$.

%The first simulations were run with these nominal parameters. %for each algorithm.
The top plot in Figure \ref{fig:TrajectGraphs} shows the main results. Figure \ref{fig:TrajectGraphs} (a) shows the positions of the agents over time, demonstrating that they converge to initial average position, indicated by the dashed line, and the bottom plot shows the evolution of the Lyapunov function, which can be seen to be nonincreasing, although the physical portion and the cyber portion are allowed to increase individually. 
The top plot in Figure \ref{fig:TrajectGraphs} (b) shows the evolution of the clock-like state, $\clovar_5$, for agent $5$, and the bottom plot shows when each agent triggers an event, demonstrating the asynchronous, aperiodic nature of event triggering.  
%The number of events, cost, observed MIET, and calculated lower bound on the MIET were $40$, $1.6216$, $0.2552$ and $0.0897$, respectively. %, for Algorithm 1, and 1.1602, 90, 0.0648, and 0.0321, respectively, for Algorithm 2.
Figure~\ref{fig:TrajectGraphs} (c) shows the inter-event times for all agents, with the horizontal lines indicating the theoretical lower bounds. For each agent $i$, the minimum inter-event time $T_i$ can be seen to be respected, and the bound appears to be tight, as expected from the theoretical analysis.

Figure \ref{fig:varyParms} (a) shows the effect of applying an additive white Gaussian noise disturbance, i.e. $\dot{x} = u + w$, where each element of $w$ is an independent and identically distributed Gaussian process, with zero mean and a variance of $\sigma_w^2 = 0.1$. To ensure that the MIET is respected in the presence of noise, the algorithm is implemented in a self-triggered fashion. That is, instead of measuring $e_i$ to propagate the dynamics~\eqref{eq:clockdyn1}, $e_i$ is approximated assuming no noise. This simulation suggests that the expected value of each agent's state is the current average position, although that average can now change with time. Figure \ref{fig:varyParms} (b) shows that the minimum inter-event times are still respected with this implementation.
%In this case, the calculated lower bound on the MIET was $0.0897$, which is less than the observed MIET of $0.1824$, showing that the bound on the MIET was preserved, as expected.


Next, to show the effect of the design parameter $\sigma_i$ on the algorithm's performance, we set $\sigma_i = \sigma$ for each agent~$i$ and varied it.
Figure \ref{fig:varyParms} (c) shows the results on 2 statistics: the average communication rate $r_\text{com}$, which is the number of events divided by the simulation length, and the cost $\mathcal{C}$, defined as follows. Considering each agent's difference from the average as an output, similar to~\cite{SdYgNm2018}, we adopt the square of the $\mathcal{H}_2$-norm of the system as a cost performance metric 
\begin{align}\label{eq:cost}
\mathcal{C} \triangleq \int_{t=0}^{t=T_\text{max}}\sum_{i=1}^N \left(x(t) - \bar{x}\right)^2,
\end{align}
where $T_\text{max} = \infty$. However, for simplicity in simulations, we use the rough approximation of $T_\text{max}=20$ being the simulation length. The choice of parameter $\sigma$ can be seen to be a trade off between cost (speed of convergence) and communication rate. Higher values of $\sigma$ result in a higher cost, but also requires less communication and results in a higher MIET by~\eqref{eq:MIET1}.}

%\begin{figure*}
%\subfigure[]{\includegraphics[width=.5\linewidth]{journal2pos.eps}} \hfill
%\subfigure[]{\includegraphics[width=.5\linewidth]{journal2lyap.eps}} \hfill
%\caption{This figure shows plots of the simulation results of the dynamic event-triggered algorithm with~$K=1$, %$\theta = 0.65$, 
%$\overline{\clovar} = 1$, and $\epsilon = 0$, showing (a) the trajectories of the agents, with each row of stars showing the event times for each agent and the dashed line representing the average,  and (b) the evolution of the Lyapunov function.% for both algorithms.
%}\label{fig:nomGraphs}
%\end{figure*}

%\begin{figure*}
%\subfigure[]{\includegraphics[width=.5\linewidth]{journal2parmEtab.eps}} \hfill
%\subfigure[]{\includegraphics[width=.5\linewidth]{journal2parmEpsi.eps}} \hfill
%\subfigure[]{\includegraphics[width=.5\linewidth]{journal2parmK.eps}} \hfill
%\subfigure[]{\includegraphics[width=.5\linewidth]{journal2parmc.eps}} \hfill
%\caption{This figure shows plots of the statistics for varying one design parameter at a time while keeping the others fixed at nominal values, showing (a) the results of changing $\overline{\clovar}$, (b) the results of changing $\epsilon$, and (c) the results of changing $K$.%, and (d) the results of changing each $c_i$ through parameter $\theta$.
%}\label{fig:varyParms}
%\end{figure*}


\begin{figure*}
\subfigure[]{\includegraphics[width=.3\linewidth]{plots/trajectLyap}
\put(-77,-3){{\small $t$}}%
\put(-150,87){{\small $x$}}%
\put(-150,29){{\small $V$}}
} \hfill
\subfigure[]{\includegraphics[width=.3\linewidth]{plots/events}
\put(-77,-3){{\small $t$}}%
\put(-160,87){{\small $\chi_5$}}%
\put(-168,29){{\small Events}}
} \hfill
\subfigure[]{\includegraphics[width=.3\linewidth]{plots/interEventTimes}
\put(-77,-3){{\small $t$}}%
%\put(-170,68){{\small Inter-}}%
%\put(-170,58){{\small event}}%
%\put(-170,48){{\small times}}%
\put(-175,61){{\tiny $t^i_{\ell+1}-t_\ell^i$}}%
} \hfill
\caption{{Plots of the simulation results of the dynamic event-triggered algorithm showing (a) the trajectories of the agents (top), with the dashed line representing the average, and the evolution of the whole Lyapunov function $V$ (bottom) as well as the physical component $V_P$ and the cyber component $V_C$; (b) the clock-like state variable~$\clovar_5$ for agent~$5$ (top) and rows of stars indicating the event times of all agents (bottom); and (c) the inter-event times~$t^i_{\ell+1}-t_\ell^i$ of each agent~$i$, with the lower bounds as computed by~\eqref{eq:MIET1} marked by the lines.}
% for both algorithms.
}\label{fig:TrajectGraphs}
\end{figure*}

\begin{figure*}
\subfigure[]{\includegraphics[width=.3\linewidth]{plots/trajectNoise}
\put(-77,-3){{\small $t$}}%
\put(-153,58){{\small $x$}}%
} \hfill
\subfigure[]{\includegraphics[width=.3\linewidth]{plots/noiseMIET}
\put(-77,-3){{\small $t$}}%
%\put(-170,68){{\small Inter-}}%
\put(-175,61){{\tiny $t^i_{\ell+1}-t_\ell^i$}}%
%\put(-170,48){{\small  times}}%
} \hfill
\subfigure[]{
\includegraphics[width=0.3\linewidth]{plots/varySigma}
\put(-77,-3){{\small $\sigma$}}%
\put(-161,87){{\small $r_\text{com}$}}%
\put(-150,29){{\small $\mathcal{C}$}}
}%
\caption{{ Simulation results with additive state disturbances showing (a) the trajectories of the agents subjected to zero-mean additive white Gaussian noise with a variance of $0.1$ (the dashed blue line indicates the expected value of the average position ($\bar{x}(0)$), and the dotted red lines show the variance over time ($\bar{x}(0)\pm t\frac{\sigma_w^2}{N}$); (b) the inter-event times~$t^i_{\ell+1}-t_\ell^i$ of each agent~$i$, with the lower bounds as computed by~\eqref{eq:MIET1} marked by the lines; and (c) the communication rate (top) and~$\mathcal{H}_2$-norm cost~\eqref{eq:cost} (bottom) as we vary the design parameter~$\sigma$.}}\label{fig:varyParms}
\end{figure*}

%For the simulation shown in Figure~\ref{fig:midAlpha}, we use the nominal values~$\overline{\clovar}=500$, $\beta = 1.1$, and $\sigma = 0.8$, which satisfies the conditions of Theorem~\ref{th:mainResult}. Figure~\ref{fig:midAlpha}(a) shows the cumulative number of events triggered for each agent over time, demonstrating the asynchronous and aperiodic broadcasting schedule of the agents. Figure~\ref{fig:midAlpha}(b) shows the Lyapunov function $V$ for the system $\mathcal{H}$, demonstrating that it is a decreasing function of time. Figure~\ref{fig:midAlpha}(c) shows the trajectories of the agents over time, demonstrating that they do approach consensus to the average of the initial conditions~$\bar{x} = 0.8$, validating our convergence result. The value for the MIET is computed using~\eqref{eq:MIET} to be $T_i \leq 0.1804$ for all agents, the amount of time before the system reaches convergence (defined, for these simulations, as when $V_1 < 0.0001$) is $7.6651$ seconds, and a total of 56 events are triggered before convergence. 

%The dynamic event-triggered mechanism is then simulated for varying design parameters. While leaving two out of three design parameters at the nominal values, we vary~$\sigma$ between 0.1 and 0.9, $\overline{\clovar}$ between 10 and 500, and~$\beta$ between 1.1 and 15.1. 
%In Figure~\ref{fig:parameters}, this means that normalized values of $0$ and $1$ correspond to the minimums and maximums of these these intervals, respectively.
%Also note that the precise shape of the curves in Figures~\ref{fig:parameters}(a) and (b) are dependent on the initial conditions. We show simulations for one specific initial condition to show the effects of varying the design parameters only. 

%Depending on the application of the network, these design parameters can be used to tune the performance of the system to the application. For instance, in some applications wireless bandwidth may be no issue, so choosing a
%high~$\beta$

%This indicates that strategies which result in shorter inter-event times and trigger more events do not necessarily lead to faster convergence, because they are not sampling at opportune times, as can be seen at low values of $\overline{\clovar}$ and high values of $\beta$. The value of $\sigma$, on the other hand, can be used to achieve faster convergence, because $\dfrac{1}{\sigma}$ acts as a gain for the maximum clock speed while $(1 - \sigma)$ acts as a gain for the state dependent term $\dfrac{\hat{z}_i}{e_i}$ in the clock dynamics, as seen in~\eqref{eq:gammabound}. This means that decreasing $\sigma$ both makes the clock's maximum speed faster, while also increasing the effect of the state dependent term, allowing it to slow down more opportunistically. This helps ensure that broadcasts occur only when necessary.
%The total number of events which were triggered is $112$, the minimum inter-event time for an agent is $0.3962$, and the calculated lower bound on the inter-event times for a single agent in this case is $0.1827$, which this is seen to satisfy. The average inter-event time for an agent is $0.6216$, $340\%$ of the calculated minimum.

%The periodic sampling was simulated with a rate of 10 events per time unit for each agent.
%\begin{figure}\label{fig:periodicPos}
%\includegraphics[width=\linewidth]{periodicPos}
%\caption{This shows the movement of the agents over time for a periodic strategy, showing the rate of convergence.}
%\end{figure}

%Figure~\ref{fig:periodicPos} shows the movement of the agents over time, demonstrating similar performance to the event-triggered algorithm in the long run. The total number of events which were triggered is 745.

%In this scenario, the periodic strategy produces many more broadcasts for each agent.

%To show that the design parameters can be used to affect the number of events and the rate of convergence, the event-triggered strategy was run with another set of design parameters. These were chosen for high performance, to trigger more events and achieve faster convergence.
%To show how the design parameter $\sigma_i$ affects the performance, the simulation was run with $\sigma = 0.1$ (the second case) and $\sigma = 0.99$ (the third case).

%In the second case, the total number of events which were triggered is $127$, the minimum inter-event time for an agent is $0.3962$, and the average inter-event time for an agent is $0.6216$, which is $275\%$ of the calculated minimum of $0.2261$. In Figure~\ref{fig:highAlpha}, the performance of the Lyapunov function can be seen to be similar to the first case.

%In the third case, the total number of events which were triggered is $305$, the minimum inter-event time for an agent is $0.0693$, and the average inter-event time for an agent is $0.2409$, which is $1057\%$ of the calculated minimum of $0.02284$. In Figure~\ref{fig:lowAlpha}, the performance of the Lyapunov function can be seen to be faster and smoother than in the other cases.

%This shows the effect of different values of the design parameter $\sigma$. The moderate value ($\sigma = 0.8$) generated the fewest events, the high value ($\sigma = 0.99$) had the highest minimum inter-event time but shorter ones on average, and the low value ($\sigma = 0.1$) had the lowest minimum inter-event time but the largest relative increase from the calculated minimum to the actual average. This shows how higher $\sigma$ increases the minimum inter-event time for each agent while decreasing the effect of the rest of the system state on the clock speed, and these two properties should be balanced for the desired performance. Additionally, the strategies which generate more events have performance closer to the case where each agent has continuous access to its neighbors' states.

\section{Conclusions}\label{se:conclusions}
This paper has used the multi-agent average consensus problem to present a dynamic agent-focused event-triggered mechanism which ensures stabilization and prevents Zeno solutions by allowing for a chosen minimum inter-event time for each agent. The algorithm is fully distributed in that it not only requires no global parameters, but the correctness of the algorithm can also be guaranteed by each agent individually. That is, no global conditions (besides connectivity of the graph) need to even be checked to ensure the overall system asymptotically converges. Additionally, it provides robustness against missed event times, guaranteeing convergence as long as events are triggered within a certain window of time. 

While this work has presented an algorithm that distributed agents can implement to guarantee asymptotic convergence, further research is needed to study the transient properties or our proposed and related algorithms. {We plan to examine this algorithm to see if it can guarantee exponential convergence to consensus, and, in particular, how the secondary trigger discussed in Remark~\ref{rm:conv} should be designed for good performance. }%More specifically, our algorithm is guaranteed to asymptotically converge but provides no guarantees on the benefits with respect to traditional implementation methods in terms of different performance metrics, such as amount of communication. We plan to rigorously quantify these types of trade-offs in future works. 

{\scriptsize
\begin{thebibliography}{10}

\bibitem{RosRmm2004}
R.~Olfati-Saber and R.~M. Murray, ``Consensus problems in networks of agents
  with switching topology and time-delays,'' {\em IEEE Transactions on
  Automatic Control}, vol.~49, no.~9, pp.~1520--1533, 2004.

\bibitem{WrRwbEma2007}
W.~Ren, R.~W. Beard, and E.~M. Atkins, ``Information consensus in multivehicle
  cooperative control,'' {\em IEEE Control Systems Magazine}, vol.~27, no.~2,
  pp.~71--82, 2007.

\bibitem{XyKlDvdKhj17}
X.~Yi, K.~Liu, D.~V. Dimarogonas, and K.~H. Johansson, ``Distributed dynamic
  event-triggered control for multi-agent systems,'' in {\em IEEE 56th Annual
  Conference on Decision and Control}, (Melbourne, VIC, Australia),
  pp.~6683--6688, 2017.

\bibitem{CdpRp2017}
C.~{De Persis} and R.~Postoyan, ``A {Lyapunov} redesign of coordination
  algorithms for cyber-physical systems,'' {\em IEEE Transactions on Automatic
  Control}, vol.~62, no.~2, pp.~808--823, 2017.

\bibitem{WpmhhKhjPt2012}
W.~P. M.~H. Heemels, K.~H. Johansson, and P.~Tabuada, ``An introduction to
  event-triggered and self-triggered control,'' in {\em IEEE Conference on
  Decision and Control}, (Maui, Hawaii, USA), pp.~3270--3285, 2012.

\bibitem{CdpPf2013}
C.~{De Persis} and P.~Frasca, ``Robust self-triggered coordination with ternary
  controllers,'' {\em IEEE Transactions on Automatic Control}, vol.~58, no.~12,
  pp.~3024--3038, 2013.

\bibitem{AvpMmj2018}
A.~V. Proskurnikov and M.~Mazo, Jr., ``Lyapunov design for event-triggered
  exponential stabilization,'' in {\em HSCC '18: 21st International Conference
  on Hybrid Systems: Computation and Control}, (Porto, Portugal), pp.~111--119,
  2018.

\bibitem{RpPtDnAa2015}
R.~Postoyan, P.~Tabuada, D.~Nesic, and A.~Anta, ``A framework for the
  event-triggered stabilization of nonlinear systems,'' {\em IEEE Transactions
  on Automatic Control}, vol.~60, no.~4, pp.~982--996, 2015.

\bibitem{RgRgsArt2012}
R.~Goebel, R.~G. Sanfelice, and A.~R. Teel, {\em Hybrid Dynamical Systems}.
\newblock 41 William Street, Princeton, New Jersey 08540: Princeton University
  Press, 2012.

\bibitem{BlWlTc2011}
B.~Liu, W.~Lu, and T.~Chen, ``Consensus in networks of multiagents with
  switching topologies modeled as adapted stochastic processes,'' {\em SIAM
  Journal on Control and Optimization}, vol.~49, no.~1, pp.~227--253, 2011.

\bibitem{DvdEf2009}
D.~V. Dimarogonas and E.~Frazzoli, ``Distributed event-triggered control
  strategies for multi-agent systems,'' in {\em 47th Annual Allerton
  Conference}, (Monticello, IL), pp.~906--910, 2009.

\bibitem{DvdKhj2009}
D.~V. Dimarogonas and K.~H. Johansson, ``Event-triggered control for
  multi-agent systems,'' in {\em 48th IEEE Conference on Decision and Control},
  (Shanghai, P.R., China), pp.~7131--7136, 2009.

\bibitem{EkXwNh2010}
E.~Kharisov, X.~Wang, and N.~Hovakimyan, ``Distributed event-triggered
  consensus algorithm for uncertain multi-agent systems,'' in {\em AIAA
  Guidance, Navigation, and Control Conference}, (Toronto, ON, Canada),
  pp.~1--15, 2010.

\bibitem{CnEgJc2019}
C.~Nowzari, E.~Garcia, and J.~Cort\'es, ``Event-triggered communication and
  control of networked systems for multi-agent consensus,'' {\em Automatica},
  vol.~105, pp.~1 -- 27, 2019.

\bibitem{DvdEfKhj2012}
D.~V. Dimarogonas, E.~Frazzoli, and K.~H. Johansson, ``Distributed
  event-triggered control for multi-agent systems,'' {\em IEEE Transactions on
  Automatic Control}, vol.~57, no.~5, pp.~1291 -- 1297, 2012.

\bibitem{EgYcHyPaDc2013}
E.~Garcia, Y.~Cao, H.~Yu, P.~Antsaklis, and D.~Casbeer, ``Decentralized
  event-triggered cooperative control with limited communication,'' {\em
  International Journal of Control}, vol.~86, no.~9, pp.~1479--1488, 2013.

\bibitem{XmTc2013}
X.~Meng and T.~Chen, ``Event based agreement protocols for multi-agent
  networks,'' {\em Automatica}, no.~49, p.~2125–2132, 2013.

\bibitem{XmLxYcsCnGjp2015}
X.~Meng, L.~Xie, Y.~C. Soh, C.~Nowzari, and G.~J. Pappas, ``Periodic
  event-triggered average consensus over directed graphs,'' (Osaka, Japan),
  pp.~4151--4156, Dec. 2015.

\bibitem{CnJc2016}
C.~Nowzari and J.~Cort\'es, ``Distributed event-triggered coordination for
  average consensus on weight-balanced digraphs,'' {\em Automatica}, no.~68,
  p.~237 – 244, 2016.

\bibitem{AaAaAm2018}
A.~Amini, A.~Asif, and A.~Mohammadi, ``{CEASE}: A collaborative event-triggered
  average-consensus sampled-data framework with performance guarantees for
  multi-agent systems,'' {\em IEEE Transactions on Signal Processing}, vol.~66,
  no.~23, pp.~6096 -- 6109, 2018.

\bibitem{FxTc2016}
F.~Xiao and T.~Chen, ``Sampled-data consensus in multi-agent systems with
  asynchronous hybrid event-time driven interactions,'' {\em Systems and
  Control Letters}, vol.~89, pp.~24 -- 34, 2016.

\bibitem{YlCnZtQl2017}
Y.~Liu, C.~Nowzari, Z.~Tian, and Q.~Ling, ``Asynchronous periodic
  event-triggered coordination of multi-agent systems,'' in {\em IEEE
  Conference on Decision and Control}, (Melbourne, Australia), pp.~6696--6701,
  December 2017.

\bibitem{GssDvdKhj2013}
G.~S. Seyboth, D.~V. Dimarogonas, and K.~H. Johansson, ``Event-based
  broadcasting for multi-agent average consensus,'' {\em Automatica}, no.~49,
  p.~245 – 252, 2013.

\bibitem{BcZl2017}
B.~Cheng and Z.~Li, ``Consensus of linear multi-agent systems via fully
  distributed event-triggered protocols,'' in {\em Proceedings of the 36th
  Chinese Control Conference}, (Dalian, China), pp.~8607--8612, 2017.

\bibitem{VsdMaWpmhh2017}
V.~S. Dolk, M.~Abdelrahim, and W.~P. M.~H. Heemels, ``Event-triggered consensus
  seeking under non-uniform time-varying delays,'' {\em IFAC-PapersOnLine},
  vol.~50, no.~1, pp.~10096--10101, 2017.

\bibitem{DpbWpmhh2014}
D.~P. Borgers and W.~P. M.~H. Heemels, ``Event-separation properties of
  event-triggered control systems,'' {\em IEEE Transactions on Automatic
  Control}, vol.~59, no.~10, pp.~2644 -- 2656, 2014.

\bibitem{GhhJelGp1952}
G.~H. Hardy, J.~E. Littlewood, and G.~Polya, {\em Inequalities}.
\newblock Cambridge, UK: Cambridge University Press, 1952.

\bibitem{JbCn2019}
J.~Berneburg and C.~Nowzari, ``Distributed dynamic event-triggered coordination
  with a designable minimum inter-event time,'' in {\em American Control
  Conference}, (Philadelphia, PA), pp.~1424 -- 1429, 2019.

\bibitem{PxCnZt2018}
P.~Xu, C.~Nowzari, and Z.~Tian, ``A class of event-triggered coordination
  algorithms for multi-agent systems on weight-balanced digraphs,'' in {\em
  American Control Conference}, (Milwaukee, WI), pp.~5988--5993, June 2018.

\bibitem{RgsJjbbNvdwWpmhh2014}
R.~G. Sanfelice, J.~J.~B. Biemond, N.~van~de Wouw, and W.~P. M.~H. Heemels,
  ``An embedding approach for the design of state-feedback tracking controllers
  for references with jumps,'' {\em International Journal of Robust and
  Nonlinear Control}, vol.~24, no.~11, pp.~1585--1904, 2014.

\bibitem{BgJc2012}
B.~Gharesifard and J.~Cort\'es, ``Distributed strategies for generating
  weight-balanced and doubly stochastic digraphs,'' {\em European Journal of
  Control}, vol.~18, no.~6, p.~539 – 557, 2012.

\bibitem{AirTcCnh2014}
A.~I. Rikos, T.~Charalambous, and C.~N. Hadjicostis, ``Distributed weight
  balancing over digraphs,'' {\em IEEE Transactions on Control of Network
  Systems}, vol.~1, no.~2, p.~190 – 201, 2014.

\bibitem{SdYgNm2018}
S.~Dezfulian, Y.~Ghaedsharaf, and N.~Motee, ``On performance of time-delay
  linear consensus networks with directed interconnection topologies,'' in {\em
  American Control Conference}, (Milwaukee, USA), pp.~4177--4182, June 2018.

\bibitem{FllDlvVls2012}
F.~L. Lewis, D.~L. Vrabie, and V.~L. Syrmos, {\em Optimal Control}.
\newblock Hoboken, New Jersey: John Wiley \& Sons, Inc., 2012.

\end{thebibliography}

}
%\margin{refernece for Gharesifard/Cortes 2012 has a typo in Cortes' name}
%\margin{reference for Ping Xu and my paper is not correct - that was published already. check cameron.bib in the shared dropbox}

\appendix
{The necessary results from hybrid systems for the rigorous proof are presented first, then a modified hybrid system which can account for different event triggers is presented, and then a general convergence result is given in Lemma~\ref{lem:genConverge}. Finally, the proofs of Theorems~\ref{th:ldtResult},~\ref{th:mainResult}, Lemma~\ref{lem:timeToEvent}, and~\ref{th:robMIET} follow.}

\subsection{Hybrid Systems Results}
\begin{lemma}[Lemma 5.10 in~\cite{RgRgsArt2012}]\label{th:OSCcondi}
A set-valued mapping $M : \real^n \rightrightarrows \real^m$ is outer semicontinuous if and only if the graph of $M$ is closed.%this is slightly abridged; book has a more general form in the lemma, too
\end{lemma}
\begin{definition}[Definition 2.4 in~\cite{RgRgsArt2012}]
A function $\solvar$: $E \rightarrow \real^n$ is a hybrid arc if $E$ is a hybrid time domain and if for each $j \in \natural$, the function $t \rightarrow \solvar(t,j)$ is locally absolutely continuous on the interval $I^j = \{t: (t,j) \in E\}$.
\end{definition}

\begin{definition}[Definition 2.6 in~\cite{RgRgsArt2012}]
A hybrid arc $\solvar$ is a solution to the hybrid system $(C,F,D,G)$ if $\solvar \in \overline{C} \cup D$, and
\begin{enumerate}
\item for all $j \in \natural$ such that $I^j \triangleq \{t: (t,j) \in \operatorname{dom}\solvar\}$ has a nonempty interior
\begin{align*}
\solvar(t,j) \in C\ \forall\ t \in \operatorname{int} I^j
\end{align*}
\begin{align*}
\dot{\solvar}(t,j) = F\left(\solvar(t,j)\right) \text{ for almost all } t \in I^j
\end{align*}
\item for all $(t,j) \in \operatorname{dom} \solvar$ such that $(t,j+1)\in \operatorname{dom}\solvar$
\begin{align*}
\solvar(t,j) \in D
\end{align*}
\begin{align*}
\solvar(t,j+1) \in G\left(\solvar(t,j)\right).
\end{align*}
\end{enumerate}
\end{definition}

\begin{definition}[Definition 2.7 in~\cite{RgRgsArt2012}]
A solution $\solvar$ is maximal if there does not exist another solution $\solvar '$ to $\mathcal{H}$ such that $\operatorname{dom}\solvar$ is a strict subset of $\operatorname{dom} \solvar '$ and $\solvar(t,j) = \solvar '(t,j)\ \forall\ (t,j)\in \operatorname{dom}\solvar$.
\end{definition}

%A complete solution has a hybrid time domain that includes either an infinite amount of time or an infinite number of jumps~\cite[Definition 2.5]{RgRgsArt2012}.

%i'm omitting this definition for now, and giving a brief explanation for nominally well-posed systems instead. We never actually use the definition in this paper.

%I believe this definition was only used for the nominally well-posed definition
%\begin{definition}[Definition 5.21 in~\cite{RgRgsArt2012}]
%A sequence $\{\solvar_i\}^\infty_{i=1}$ of hybrid arcs $\solvar_i : \operatorname{dom}%\solvar_i \rightarrow \real^n$ converges graphically if the sequence of sets $\{gph\}^\infty_{i=1}$ converges in the sense of set convergence. The graphical limit of a graphically convergent sequence $\{\solvar_i\}^\infty_{i=1}$ is the mapping $M: \real^2 \rightrightarrows \real^n$ such that $\operatorname{gph} M = \operatorname{lim}_{i \rightarrow \infty}\operatorname{gph}\solvar_i$.
%\end{definition}

%We never need to use this definition.
%\begin{definition}[Definition 6.2 in~\cite{RgRgsArt2012}]\label{def:nominallyWellPosed}
%A hybrid system $\mathcal{H}$ is called nominally well-posed if the following property holds: for every graphically convergent sequence $\{\solvar_i\}^\infty_{i=1}$ of solutions to $\mathcal{H}$ with $\operatorname{lim}_{i \rightarrow \infty}\solvar_i(0,0) = q$ for some $q \in \real^n$,
%\begin{enumerate}
%\item if the sequence $\{\solvar_i\}_{i=1}$ is locally eventually bounded then the sequence $\{\operatorname{length)}(\solvar_i)\}^\infty_{i=1}$ is either convergent or properly divergent to $\infty$ and
%\begin{align*}
%\solvar = \operatorname{gph-lim}_{i \rightarrow \infty}\solvar_i
%\end{align*}
%is a solution to $\mathcal{H}$ with $\solvar(0,0) = q$ and $\operatorname{length} = \operatorname{lim}_{i \rightarrow \infty}$%\operatorname{length}(\solvar_i)$;
%\item if the sequence $\{\solvar_i\}^\infty_{i=1}$ is not locally eventually bounded then there exists a number $m \in (0,\infty)$ for which there exist $(t_i,j_i \in \operatorname{dom}\solvar_i$ for $i = 1,2, \dots,$ such that $\operatorname{lim}_{i\rightarrow \infty}|\solvar_i(t_i,j_i)| = \infty$ and
%\begin{align*}
%\solvar = (\operatorname{gph-lim}\solvar_i)|_{t+j>m}
%\end{align*}
%is a maximal solution to $\mathcal{H}$ with $\operatorname{length}(\solvar)=m$ and
%\begin{align*}
%\operatorname{lim}_{t\rightarrow \operatorname{sup}_t\solvar}|\solvar(t, \operatorname{sup}_j \operatorname{dom}\solvar)|=\infty.
%\end{align*}
%\end{enumerate}
%of hybrid arcs $\solvar_i : \operatorname{dom}\solvar_i \rightarrow \real^n$ for which there exists a compact set $K \in \real^n$ such that $\solvar_i(0,0) \in K$ for all $\i \in \natural$ there exists a subsequence $\{\solvar_{i_k}$...
%\end{definition}
%Nominally well-posed systems are a class of hybrid systems whose properties can be necessary for some invariance principles to hold. The nominally well-posed quality can be verified by the following theorem.

%\vspace*{-2ex}
\begin{theorem}[Theorem 6.8 in~\cite{RgRgsArt2012}]\label{th:nomWellPosed}
If a hybrid system $\mathcal{H}$ satisfies the following assumption, then it is nominally well-posed.
$C$ and $D$ are closed subsets of $\real^n$;

\begin{enumerate}
\item $F:\real^n \rightarrow \real^n$ is outer semicontinuous and locally bounded relative to $C$, $C$ is a subset of the domain of $F$, and $F(q)$ is convex for every $q \in C$;
\item $G:\real^n \rightrightarrows \real^n$ is outer semicontinuous and locally bounded relative to $D$ and $D$ is a subset of the domain of $G$.
\end{enumerate}
\end{theorem}
%\vspace*{-2ex}
%\vspace*{-2ex}
A solution $\solvar$ is complete if $\text{dom}\ \solvar$ is unbounded.
\begin{definition}[Weak Invariance~\cite{RgRgsArt2012}]\label{def:weakInvariance}
{\rm
Given a hybrid system $\mathcal{H}$, a set $S \subset \real^n$ is said to be

\begin{itemize}
\item \emph{weakly forward invariant} if, for every $q \in S$, $\exists$ a complete solution $\solvar$ to $\mathcal{H}$ with initial condition $q$ whose range is a subset of $S$,

\item \emph{weakly backward invariant} if for every $q \in S$ and every $\tau > 0$, there exists at least one maximal solution $\solvar$ to $\mathcal{H}$ with initial condition in $S$ such that for some $(t^*,j^*)$ in the domain of $\solvar$, $t^* + j^*>\tau$, it is the case that $\solvar(t^*,j^*)=q$ and $\solvar(t,j) \in S\ \forall\ (t,j)$ in the domain of $\solvar$ with $t+j\leq t^* + j^*$,

\item and \emph{weakly invariant} if it is both weakly forward invariant and weakly backward invariant.
\end{itemize}
}
\end{definition}
%\vspace*{-2ex}
For a solution $\solvar$ to a hybrid system $\mathcal{H}$, $t(j)$ denotes the least time $t$ such that $(t,j)$ is in its domain and $j(t)$ denotes the least index $j$ such that $(t,j)$ is in its domain~\cite{RgRgsArt2012}.
Given $V:\real^n \rightarrow \real$, any functions $u_C, u_D : \real^n \rightarrow [-\infty, \infty]$, and a set $U \subset \real^n$, it is said that the growth of $V$ along solutions to $\mathcal{H}$ is bounded by $u_C, u_D$ on $U$ if for any solution $\solvar$ to $\mathcal{H}$ with its range in $U$,
\begin{align}\label{eq:boundedGrowth}
V\left(\solvar(\overline{t}, \overline{j})\right) - V\left(\solvar(\underline{t}, \underline{j})\right) & \leq \int_{\underline{t}}^{\overline{t}} u_C\left(\solvar \left(s,j(s)\right)\right)ds \nonumber \\
& + \sum_{j=\underline{j}+1}^{\overline{j}}u_D\left(\solvar \left(t(j),j\right)\right)
\end{align}
for all $(\underline{t}, \underline{j}), (\overline{t}, \overline{j})$ in the domain of $\solvar$ such that $(\underline{t}, \underline{j}) \prec (\overline{t}, \overline{j})$~\cite{RgRgsArt2012}.
Intuitively, $u_C$ acts as a bound for the growth of $V$ during system flow, while $u_D$ acts as a bound for its growth during system jumps.

\begin{theorem}\longthmtitle{Invariance Principle for Hybrid Systems~\cite{RgRgsArt2012}}\label{th:invariance}
Consider a continuous function $V :\real^n \rightarrow \real$, any functions $u_C, u_D : \real^n \rightarrow [-\infty, \infty]$, and a set $U \subset \real^n$ such that $u_C(q), u_D(q) \leq 0$ $\forall\ q \in U$ and such that the growth of $V$ along solutions to $\mathcal{H}$ is bounded by $u_C, u_D$ on $U$. Let a complete, bounded solution to $\mathcal{H}$, $\solvar^*$, be such that the closure of its range $\in U$. Then, for some $r \in V(U)$, $\solvar^*$ approaches the nonempty set that is the largest weakly invariant subset of
\begin{align}\label{eq:invarianceSet}
\mathcal{S} \triangleq V^{-1}(r) \cap U \cap \left[\overline{u_C^{-1}(0)} \cup \left(u_D^{-1}(0) \cap G\left(u_D^{-1}(0)\right)\right)\right].
\end{align}
\end{theorem}

\subsection{Proofs}

{
\emph{More General Hybrid System Formulation}

To aid in several proofs, we define another hybrid system which will account for the different event-triggers discussed in this paper. This is a more generalized version of~\eqref{eq:system}, with the jump set extended, using a new timer state, to account for earlier events. We begin by extending the state with a timer variable $\mathcal{T}_i$ for each agent~$i$, so that, with a slight abuse of notation, we have
\begin{align*}
q_i = \left[ \begin{array}{c}
x_i\\
\hat{x}_i\\
\clovar_i\\
\mathcal{T}_i
\end{array} \right],
\end{align*}
$q = [q_1^T,\hdots,q_n^T]^T$, and $\dot{q} = f'(q)$, where $f'(q) = [f_1^{'T},\hdots,f_n^{'T}]^T$ and
\begin{align*}
\dot{q}_i = f_i'(q) \triangleq \left[ \begin{array}{c}
-(L\hat{x})_i\\
0\\
\gamma_i(v_i)\\
1
\end{array} \right] \quad \text{for } q \in C',
\end{align*}
where $C' = \cap _{i=1}^N C'_i$ and
\begin{align*}
C'_i = \{q \in \real^{4N}: \clovar_i \geq 0 \}.
\end{align*}
The extended jump set is now given by~$D' = \cup_{i=1}^N D'_i$, where
{
\begin{align}\label{eq:jumpset}
D'_i =  \{q \in \real^{4N}: \mathcal{T}_i \geq T_i \},
\end{align}}
for any $T_i \in (0,\frac{\sigma_i}{d_i}]$. The new local jump map, for $q \in D_i'$, is
\begin{align*}
g_i'(q) = \left[ \begin{array}{c} q_1^+ \\ \vdots \\ q_i^+ \\ \vdots \\ q_{N}^+ \end{array} \right] \triangleq \left[ \begin{array}{c} q_1 \\ \vdots \\ \left( \begin{array}{c} x_i \\ x_i \\ \clovar_i \\ 0\end{array} \right) \\ \vdots \\ q_{N} \end{array} \right].
\end{align*}
The jump map is described by a set-valued map~$G' : \real^{4N} \rightrightarrows \real^{4N}$%~\cite{RgsJjbbNvdwWpmhh2014,CdpRp2017}
, where
\begin{align}\label{eq:modJumpDyn}
G'(q) \in \{ g_1(q), \dots, g_N(q) \}.
\end{align}
Informally, this defines the timer  $\mathcal{T}_i$ to count up between agent $i$'s events, and it is set to $0$ at each of agent $i$'s events. This means that $\mathcal{T}_i$ measures the time since agent $i$'s most recent event. The extended jump set now allows events to occur as soon as an inter-event time $T_i$ (which can be shorter than the time given in~\eqref{eq:MIET1}, but not longer) has passed, but the system does not have to jump until it is about to exit the flow set when $\exists$ $i:\chi_i = 0$. Finally, we define
\begin{align}\label{eq:modSys}
\mathcal{H}' = (C',f',D',G').
\end{align}

We use this general system~\eqref{eq:modSys} to give the following lemma, which will be used to prove the main results.
\begin{lemma}[General Asymptotic Convergence]\label{lem:genConverge}
Given the hybrid system~$\mathcal{H}'$, for any $q(0) \in \mathbb{R}^{4N}$ such that $x(0) \in \mathbb{R}^{N}$, $\hat{x}(0) = x(0)$, $\chi(0) = \mathbf{0}_N$, and $\mathcal{T} = \mathbf{0}_N$, the system state is guaranteed to asymptotically converge to the set
\begin{align*}
\mathcal{B} \triangleq \{q \in \mathbb{R}^{4N} : \hat{\phi}_i = 0\text{ } \forall\text{ } i\}.
\end{align*}
\end{lemma}}
\begin{IEEEproof}

{Ultimately we wish to apply Theorem~\ref{th:invariance} to our hybrid system. 

Intuitively, we are interested in showing that~$V$ is nonincreasing along the trajectories of our system so that we can apply an invariance principle to show convergence. 
While the state is flowing~($q \in C$), we have already shown that the clock defined by~\eqref{eq:clockdyn1} ensures that~$\dot{V} < 0$ $\forall q \notin \mathcal{B}$. 
When the system jumps~($q \in D$), we have
\begin{align*}
 V \left( g(q) \right) - V(q) = 0,
\end{align*}
for $q \in D$, because $V$ does not depend on $\hat{x}$ or $\mathcal{T}$.

This means that while the system is flowing but not in the target state~($q \in C \setminus \mathcal{B}$), we have~$\dot{V} < 0$. When the state jumps~($q \in D$), the value of~$V$ remains unchanged.} Combining this with the fact that $\bar{x}$ is constant and with the positive MIET result %Theorem~\ref{th:ldtResult} 
to ensure that~$t \rightarrow \infty$ without exhibiting Zeno behavior guarantees {that~$q(t) \rightarrow \mathcal{B}$.} {This is the intuitive argument, which we will formalize next.} 

{
More formally, we are interested in showing that~$V$ is nonincreasing along the trajectories of our system, then characterizing the largest invariant subset where it is zero as~$\mathcal{B}$.

We first show that~$\mathcal{H'}$ is nominally well-posed using~\cite[Theorem~6.8]{RgRgsArt2012}. 
The flow set $C$ and the jump set $D$ are both closed subsets of $\real^{4N}$, $f'(q)$ and $G'(q)$ are outer semicontinuous, and both are locally bounded so all conditions of~\cite[Theorem~6.8]{RgRgsArt2012} are satisfied and the hybrid system $\mathcal{H'}$ is nominally well-posed. 

Now we establish bounds for $V$'s rate of change. 
Let
\begin{align}\label{eq:ud}
u_D = \begin{cases}
 0 & \text{for } q\in D\\
 -\infty & \text{otherwise}\\
\end{cases}.
\end{align}
and
\begin{align}\label{eq:uc}
u_C = \begin{cases}
\sum_{i=1}^N -(1 -\sigma_i)\hat{\phi}_i  & \text{for } q\in C\\
 -\infty & \text{otherwise}\\
\end{cases}.
\end{align}
The function $u_D$ acts as an upper bound on the rate of change of $V(q)$ for each jump (because $V$ does not jump) while $u_C$ acts as an upper bound on the rate of change during flow. %Note that $u_D$ and $u_C$ have no meaning outside the jump set and flow set, respectively. 
Note that we have shown that the growth of $V(q)$ along any solution is bounded by $u_C$, $u_D$.

Let $U(\bar{x}(0))\subset \real^{4N} \triangleq \{q \in \real^{4N} : \dfrac{1}{N} \sum_{i=1}^N x_i = \bar{x}(0)\}$. Note that, because the average position of the agents remains constant along all solutions, $U$ is invariant and any solution $\solvar$ such that $\solvar(0) \in U(\bar{x}(0))$ remains in $U(\bar{x}(0))$ for as long as it is defined.

Therefore, by Theorem~\ref{th:invariance}, every complete, bounded solution $\solvar$ such that $\solvar(0) \in U(\bar{x}(0))$ approaches the largest weakly invariant subset of
\begin{align}\label{eq:invarianceSet}
\mathcal{S} \triangleq V^{-1}(r) \cap \left[\overline{u_C^{-1}(0)} \cup \left(u_D^{-1}(0) \cap G\left(u_D^{-1}(0)\right)\right)\right],
\end{align}
for some $r \in V(U(\bar{x}(0)))$. 
The weakly invariant subset of $\overline{u_C^{-1}(0)}$ is $\mathcal{B}$.

As an aside, note that, if we have an addition trigger that guarantees that agent~$i$ will trigger an event and cause a jump in finite time if $e_i \neq 0$, then this set is instead $\{q : \hat{\phi}_i = e_i = 0\text{ } \forall\text{ } i\}$. 

%Note that $u_C = 0$ only if $\hat{z} = \mathbf{0}_N$ for $q \in C$, but this set is only weakly invariant if $e = \mathbf{0}_N$, because otherwise $\dot{\clovar}_i < 0$ for some $i$ and another event will occur to change $\hat{z}$.
%Note also $\mathcal{A} = V^{-1}(0)$, $\dot{x} = u = 0$ in $\mathcal{A}$, and $G(q) = q$ for all $q \in \mathcal{A}$. Therefore, all solutions $\solvar$ with $\solvar(t,j) \in \mathcal{A}$ for $(t,j) \in \operatorname{dom}\solvar$ remain in $\mathcal{A}$ $\forall\ (t^*, j^*) \in \operatorname{dom}\solvar \text{ s. t. } t^* + j^* > t+j$, which means that~$\mathcal{A}$ is weakly invariant.
%It will now be shown that the largest weakly invariant subset of $\mathcal{S}$ cannot include any points outside of $\mathcal{A}$.
%Note that our algorithm deals with a specific subset of $G(q)$, $G'(q) \triangleq \{G_i\ \forall \ i \text{ s. t. } \clovar_i \leq 0\}$, which more closely describes the algorithm in Table~\ref{tab:algorithm}. %Although $G$ was defined for simplicity, it did not rule out the possibility of each agent $i$ triggering an event while $\clovar_i > 0$ if there is another agent $j$ such that $\clovar_i \leq 0$, which cannot occur for the algorithm presented in Table~\ref{tab:algorithm}.
%$G'(q)$ is outer semicontinuous and the system $\mathcal{H}' = (C,F,D,G')$ is nominally well-posed by the same arguments as before.
The points in $G(u_D^{-1}(0)) \cap u_D^{-1}(0)$ are points inside the jump set which can be reached by jumping. Note that the timer variables $\mathcal{T}_i$ preclude degenerate cases where the same agent broadcasts more than once at a single time instant, and so the system must leave the jump set after a finite number of jumps. %Therefore, any maximal solution will leave the jump set when all agents meeting the trigger condition have broadcast. 
Therefore, the largest weakly invariant subset of $\mathcal{S}$ cannot include any points outside $\mathcal{B}$, and so all complete, bounded solutions to $\mathcal{H}'$ starting in $U(\bar{x}(0))$ converge to $\mathcal{B}$ by Theorem~\ref{th:invariance}.

Because $V$ is radially unbounded with respect to $x$ and $\chi$ and $\dot{V} \leq 0$, every solution must be bounded, and, because $C' \cup D' = \real^{4N}$, every solution is complete. Therefore every solution to $\mathcal{H}'$ such that $x(0) \in \real^N$, $\hat{x}(0) = x(0)$, and $\clovar_i \geq 0 $, for $i = 1, 2, \dots , N$ is complete and bounded and converges to $\mathcal{B}$.
}
\end{IEEEproof}



%\emph{Proof of Lemma~\ref{th:ContClock}}

%To show that the chosen dynamics of the clock given by~\eqref{eq:clockdyn1} is continuous for constant $\hat{x}$, we must show that $\operatorname{lim}_{e_i \rightarrow 0} \min \{ \overline{\gamma}_{i} , 0 \} - \epsilon_i = -\epsilon$. %Note that $\hat{x}_i$ and, therefore, $\chi_i$ are both piecewise constant in $t$ for $i = 1,2,\dots ,N$.

%First, if $\hat{\sumvar}_i = 0$, then $\hat{z}_i = 0$. Therefore,
%\begin{align*}
%\overline{\gamma}_i = 0 ,
%\end{align*}
%for all $e_i$, and so
%\begin{align*}
%\operatorname{lim}_{e_i \rightarrow 0} \min \{ \overline{\gamma}_{i} , 0 \} - \epsilon_i = -\epsilon_i ,
%\end{align*}
%for $\hat{\sumvar}_i = 0$. 
% Otherwise, letting $\sumvar_i = 1/e_i$,
%\begin{align*}
%\text{lim}_{e_i \rightarrow 0^+}\ \overline{\gamma}_i - \epsilon_i = \text{lim}_{\sumvar_i \rightarrow + \infty}\ & \reldis_i^T\mathcal{W}_i\reldis_i\sumvar_i^2\\
% & + 2(\clovar_i + 1)W_i^T\reldis_i\sumvar_i - \epsilon_i\\
%& = + \infty .
%\end{align*}
%Otherwise, noting that $\overline{\gamma}_i$ is a concave up parabola in terms of $\frac{1}{e_i}$, we have
%\begin{align*}
%& \text{lim}_{e_i \rightarrow 0^+}\ \overline{\gamma}_i - \epsilon_i\\
%= & \text{lim}_{e_i \rightarrow 0^+}\ \hat{\sumvar}_i\frac{1}{e_i^2} + 2(\clovar_i + 1)\hat{z}_i\frac{1}{e_i} - \epsilon_i = + \infty .
%\end{align*}
%Similarly,
%\begin{align*}
%& \text{lim}_{e_i \rightarrow 0^-}\ \overline{\gamma}_i - \epsilon_i\\
%= & \text{lim}_{e_i \rightarrow 0^-}\ \hat{\sumvar}_i\frac{1}{e_i^2} + 2(\clovar_i + 1)\hat{z}_i\frac{1}{e_i} - \epsilon_i = + \infty .
%\end{align*}
%\begin{align*}
%\text{lim}_{e_i \rightarrow 0^-}\ \overline{\gamma}_i = \text{lim}_{\sumvar_i \rightarrow - \infty}\ & \reldis_i^T\mathcal{W}_i\reldis_i\sumvar_i^2\\
% & + 2(\clovar_i + 1)W_i^T\reldis_i\sumvar_i\\
%& = + \infty .
%\end{align*}
%Therefore, for $\hat{\sumvar}_i \neq 0$, we also have
%\begin{align*}
%\operatorname{lim}_{e_i \rightarrow 0} \min \{ %\overline{\gamma}_{i} , 0 \} - \epsilon_i =  - \epsilon_i .
%\end{align*}
%$\blacksquare$\\

\emph{Proof of Theorem~\ref{th:ldtResult}}

{
To determine the minimum inter-event time for agent $i$, we write the relevant states as their own local dynamical system, and, considering unknowns as inputs to this local system, we apply optimal control to see how quickly the system can be driven to the next event state. The relevant states are the auxiliary variable $\chi_i$ 
%with dynamics
%\begin{align*}
%\dot{\chi}_i = \sigma_i \left(\sum_{j\in\mathcal{N}_i}w_{ij}(\hat{x}_i - \hat{x}_j)^2\right) + 2e_i\left(\sum_{j\in\mathcal{N}_i}w_{ij}(\hat{x}_i - \hat{x}_j)\right),
%\end{align*}
and the error $e_i$. %with dynamics
%\begin{align*}
%\dot{e}_i = -\sum_{j\in\mathcal{N}_i}w_{ij}(\hat{x}_i - \hat{x}_j).
%\end{align*}
Since there are no guarantees on the last broadcast states of our neighbors, $\hat{x}_j$ for $j \in \mathcal{N}^\text{out}_i$, we consider the directed distance to each out neighbor as an input to this system. We define our local nonlinear system as
\begin{align}\label{eq:stateDyn}
\dot{\zeta} = f(\zeta,\mu) \triangleq \left[ \begin{matrix}
\sigma_i \mu^T\mathcal{W}\mu + 2\zeta_2W\mu\\
-W\mu\\
\end{matrix}\right],
\end{align}
where $\zeta_1 = \chi_i$, $\zeta_2 = e_i$, $\mu$ is a column vector such that $\mu_j = \hat{x}_i - \hat{x}_j$, $W$ is a row vector such that $W_j = w_{ij}$, and $\mathcal{W} = \text{diag }(W)$. 

%Note that, in practice, $u$ must be piecewise constant, because it only changes discretely when new information is received, but we drop this assumption and consider a general $u$.
To determine the MIET, we want to see how short the time between events can be. Therefore, we look to minimum time optimal control. Our performance index is
\begin{align}\label{eq:perfIndex}
J = \int^{T_i}_01dt,
\end{align}
where $T_i$ is the minimum time we wish to find. 
We assume that the agent has just triggered an event and communicated, so the initial condition is
\begin{align}\label{eq:initCondi}
\zeta(0) = \left[ \begin{matrix}
0\\
0
\end{matrix}\right],
\end{align}
because $\chi_i = \zeta_1 = 0$ is required to trigger the broadcast which sets $e_i = \zeta_2 = 0$. We wish the terminal condition to be when  the condition for a triggering an event is first reached again, that is, when $\zeta_1 = 0$ and $\zeta_2 \neq 0$. However, this set is open, so we cannot use it to perform optimization. Instead, we assume that the final value of $\zeta_2$ is known a priori, so that $\zeta_2(T_i) = e_T$. We can then investigate the effects of different choices of $e_T$. Therefore, the terminal condition is
\begin{align}\label{eq:termCondi}
\zeta(T_i)= \left[ \begin{matrix}
0\\
e_T
\end{matrix}\right].
\end{align}
The Hamiltonian is
\begin{align*}
H &= 1 + \lambda^Tf(\zeta,\mu)\\
&= 1 + \lambda_1(\sigma \mu^T\mathcal{W}\mu + 2\zeta_2W\mu) - \lambda_2W\mu,
\end{align*}
where $\lambda \in \real^2$ is the costate. 
%To derive the costate equation and the stationarity condition, we must find the Jacobian of the Hamiltonian with respect to $z$ and to $u$~\cite{FllDlvVls2012}:
%\begin{align*}
%\frac{\partial H}{\partial z} &= \lambda^T\left[\begin{matrix}
%0 & 2Bu\\
%0 & 0
%\end{matrix}\right]\\
%\frac{\partial H}{\partial u} &= \lambda^T\left[\begin{matrix}
%2\sigma\mathcal{B}u + 2z_2B^T\\
%-B
%\end{matrix}\right].
%\end{align*}
The costate equation~(\cite{FllDlvVls2012}) is
\begin{align*}
-\dot{\lambda} &= \frac{\partial H}{\partial \zeta}^T = \left[\begin{matrix}
0\\
2\lambda_1W\mu
\end{matrix}\right].
\end{align*}
%This indicates that $\lambda_1$ is constant. 
The stationarity condition~\cite{FllDlvVls2012} is
\begin{align*}
0 &= \frac{\partial H}{\partial \mu} = (2\sigma\mathcal{W}\mu + 2\zeta_2W^T)\lambda_1 - W^T\lambda_2.
\end{align*}

Therefore, the relevant solutions for the state, costate, and input can be found to be
\begin{align}\label{eq:optimalInput}
\zeta_2(t) &= -\frac{W\mathcal{W}^{-1}W^T\lambda_2(0)}{2\lambda_1\sigma}t\nonumber\\
\lambda_1 &= \lambda_1(0)\nonumber\\
\lambda_2(t) &= \lambda_2(0) - \frac{W\mathcal{W}^{-1}W^T\lambda_2(0)}{\sigma}t\nonumber\\
\mu &= -\mathbf{1}_{|\mathcal{N}_i|}\frac{e_T}{d_i T_i},
\end{align}
where $\mathcal{W}$ is invertible because it is a diagonal matrix with positive entries on its diagonal. Note that $\mathcal{W}^{-1}W^T = \mathbf{1}_{|\mathcal{N}_i|}$ and so $W\mathcal{W}^{-1}W^T = \sum_{j\in \mathcal{N}_i}w_{ij} = d_i$. Note that this $\mu$ will be useful in other proofs.

However, $T_i$ is still unknown, so we apply the boundary conditions to the Hamiltonian. Note that $H(T_i) = 0$, because this is a minimum time problem with a fixed final state~\cite{FllDlvVls2012}. Additionally, $\dot{H} = 0$, because $H$ is not an explicit function of time~\cite{FllDlvVls2012}. This indicates that $H(0) = 0$, so we can solve for $\lambda(0)$. This allows us to solve for $\lambda(T_i)$ in terms of $T_i$, and we can finally solve $H(T_i)=0$ for $T_i$. This yields
\begin{align}\label{eq:proofmiet}
T_i = \frac{\sigma}{W\mathcal{W}^{-1}W^T} = \frac{\sigma_i}{d_i}.
\end{align}
Note that $T_i$ does not depend on the choice of $e_T$, so we can conclude that the minimum inter-event time is given by~\eqref{eq:proofmiet}.

\hfill $\blacksquare$
}

\emph{Proof of Theorem~\ref{th:mainResult}}

{This is a straightforward application of the general convergence result Lemma~\ref{lem:genConverge}, because Theorem~\ref{th:ldtResult} guarantees a positive minimum inter-event time so the algorithm is described by hybrid system $\mathcal{H}'$ from~\eqref{eq:modSys}.}\hfill $\blacksquare$

\begin{lemma}[Time until next event]\label{lem:timeToEvent}
For any time $t \in [t^i_\ell,t^i_{\ell+1})$, $t^i_{\ell+1}$ is next time that agent $i$ would trigger an event under~\eqref{eq:trigFunc}. That is,
\begin{align*}
t^i_{\ell+1} = \inf \{t' \geq t : \chi_i(t')=0 \text{ and } e_i(t') \neq 0\}.
\end{align*}
Under the hybrid system~\eqref{eq:system} with $\gamma_i$ defined in~\eqref{eq:clockdyn1}, the remaining time until $t^i_{\ell+1}$ is lower bounded as follows
\begin{align*}
t^i_{\ell+1} - t \geq \left\lbrace
\begin{matrix}
\frac{\sigma_i}{d_i}\left( 1 - \frac{e_i^2}{\chi_i+e_i^{2}}\right), & \text{for } (\chi_i,e_i) \neq (0,0)\\
\frac{\sigma_i}{d_i}, & \text{otherwise}
\end{matrix}
  \right. ,
\end{align*}
assuming $\chi_i \geq 0$.
\end{lemma}

\emph{Proof:}

{
This proof relies on examining each agent as a local system, as defined in~\eqref{eq:stateDyn}. We must find the minimum time to reach a point such that $\chi_i = 0$ and $e_i > 0$ from any initial point with $\chi_i(0) \geq 0$ and $e_i(0) \in \mathbb{R}$. 
We must consider three cases.
\begin{itemize}
\item \textbf{Case 1: $\chi_i(0) = 0$ and $e_i(0) = 0$}

In this case, we can use Theorem~\ref{th:ldtResult} directly, and the minimum time is $T_i$.

\item \textbf{Case 2: $\chi_i(0) > 0$ and $e_i(0) = 0$}

Note that in the proof of Theorem~\ref{th:ldtResult}, the dynamics of $\zeta$ does not depend on $\chi_i$. Therefore, with a change of coordinates $\chi_i' = \chi_i - \chi_i(0)$ we can use the same reasoning to show that the minimum time for $\chi_i$ to reach a point such that $\chi_i(t) = \chi_i(0)$ and $e_i > 0$ is $T_i$. Because $\dot{\chi}_i(0) \geq 0$ if $e_i(0) = 0$, and $\dot{\chi}_i  > 0$ if $\dot{e}_i \neq 0$, we must reach such a point before we reach a point where $\chi_i = 0$ and $e_i \neq 0$. Therefore, the minimum time is lower bounded by $T_i$.

\item \textbf{Case 3: $\chi_i > 0$ and $e_i \neq 0$}

First, we show that all points in this set are reachable using the optimal input $\mu$ from~\eqref{eq:optimalInput}. Letting $k = -\frac{e_T}{d_i T_i}$, we can write the input as
\begin{align*}
\mu = k\mathbf{1}_{|\mathcal{N}_i|},
\end{align*}
for $e_T \neq 0$. For any $k \neq 0$, then, the state will be driven along an optimal trajectory that reaches $\chi_i(T_i),e_i(T_i)= (0,e_T)$, and the value of $e_T$ will depend on $k$. Solving the differential equations with $\mu$ defined, we have
\begin{align*}
\chi_i(t) &= -(d_ik)^2 t^2 + \sigma d_ik^2 t\\
e(t) &= -d_ikt.
\end{align*}
Assuming we wish to reach the point $(\chi^*,e^*)$, we choose 
%must check if there exists a $k \in \mathbb{R}$ and a $t\geq 0$ that satisfies the equations
%\begin{align*}
%\chi^* &= -(dk)^2 t^2 + \sigma dk^2 t\\
%e^* &= -dkt.
%\end{align*}
%We can use the second equation to eliminate $k$:
%\begin{align*}
$k = \frac{e^*}{-dt}.$
%\end{align*}
This indicates that any value of $e^*\neq 0$ is reachable. Now, plugging that back into the first equation yields
\begin{align}\label{eq:chiAndTime}
\chi^* &= -e^{*2} + \frac{\sigma e^{*2}}{dt}.
\end{align}
This will indicate which values of $\chi^*$ can be reached for a given $e^*$ from the origin. Since $e^* \neq 0$, then $\lim_{t \rightarrow 0^+} \chi^* = + \infty$ and $\lim_{t \rightarrow T_i} \chi^* = e^{*2} + \frac{\sigma e^{*2}}{d\frac{\sigma}{d}}=0$. Therefore, with the proper choice of $t \in (0,T_i]$ and $k \neq 0$, our trajectory can reach any $\chi^* \geq 0$ and $e^* \neq 0$.

%To summarize the results here, we know now that optimal trajectories can pass through any point $(\chi, e)$ in the set $\{\mathbb{R}^2 | \chi \geq 0\text{ and } e \neq 0\} \cup (0,0)$.
By the principle of optimality, the optimal input from any point in that set is still $\mu = k\mathbf{1}_{|\mathcal{N}_i|}$, and the value of $k$ will depend on the specific point.
%This simplifies things greatly for determining the minimum time until the next event. 
%We know the optimal time to go from the origin to any other point on the line $\chi = 0$ is $T=\frac{\sigma}{d}$, so if we can determine the time it would take to reach a point using the optimal input $u = k\mathbf{1}$, we know how much time is left if we continue to use that input.

Returning to~\eqref{eq:chiAndTime}, we solve for the time we would have reached that point on an optimal trajectory
\begin{align}\label{eq:minTime}
t &= \frac{\sigma_i e_i^2}{d_i(\chi_i+e_i^{2})}.
\end{align}
%This implies that, for any given $(\chi,e)$ such that $e \neq 0$, the remaining time until the next event is lower bounded by
Finally, this indicates that the minimum time is
\begin{align*}
&T_i - \frac{\sigma_i e_i^2}{d_i(\chi_i+e_i^{2})}\\
= &\frac{\sigma_i}{d_i}\left( 1 - \frac{e_i^2}{\chi_i+e_i^{2}}\right).
\end{align*}

\end{itemize}

In summary, the time to reach any point such that $\chi_i = 0$ and $e_i > 0$ from any initial point with $\chi_i(0) \geq 0$ and $e_i(0) \in \mathbb{R}$ is lower bounded by
\begin{align*}
\left\lbrace
\begin{matrix}
\frac{\sigma_i}{d_i}\left( 1 - \frac{e_i^2}{\chi_i+e_i^{2}}\right), & \text{for } (\chi_i,e_i) \neq (0,0)\\
\frac{\sigma_i}{d_i}, & \text{otherwise}
\end{matrix}
  \right. .
\end{align*}

}\hfill $\blacksquare$

\emph{Proof of Theorem~\ref{th:robMIET}}

{
The lower bound on the inter-event times $\tilde{T}_i$ follows from the analysis in the proof of Lemma~\ref{lem:timeToEvent}, using the local system for agent~$i$~\eqref{eq:stateDyn}. In order for  $h_i \in [-\delta t^i, 0]$ to be satisfied  with $ \delta t^i< T_i$, we must have $e_i \neq 0$ and $\chi \geq 0$. Intuitively, the trigger condition allows events to be triggered $\delta t^i$ seconds early, so the MIET is reduced by that much. 

More formally, from the analysis in the proof of Lemma~\ref{lem:timeToEvent}, we know the time it would take to reach that point along an optimal trajectory~\eqref{eq:minTime}. Since these trajectories are optimal in a minimum time sense, there can be no faster way to reach that point from $(\chi_i,e_i)=(0,0)$, where the previous event occurred. 

For $h_i \in [-\delta t^i, 0]$, we have
\begin{align*}
\frac{\sigma_i}{d_i}\left( 1 - \frac{e_i^2}{\chi_i+e_i^{2}}\right) \in [0, \delta t^i],
\end{align*}
and so 
\begin{align*}
 \frac{\sigma_i}{d_i}\frac{e_i^2}{\chi_i+e_i^{2}} \in \left[\frac{\sigma_i}{d_i} - \delta t^i, \frac{\sigma_i}{d_i} \right].
\end{align*}
Therefore, for agent~$i$, using~\eqref{eq:minTime}, the minimum time to reach a point where the trigger condition is satisfied is
\begin{align*}
\frac{\sigma_i}{d_i} - \delta t^i.
\end{align*}

This is the MIET given in Theorem~\ref{th:robMIET}.

Now that a positive minimum inter-event time has been established, the algorithm in Theorem~\ref{th:robMIET} is known to be described by hybrid system $\mathcal{H}'$, and we can apply our general convergence result Lemma~\ref{lem:genConverge} to guarantee convergence to the set~$\{q : \hat{\phi}_i = 0\text{ } \forall\text{ } i\}$.
}\hfill $\blacksquare$

%{\color{red}

%\vspace*{-6ex}
%\begin{biography}[{\includegraphics[width=1in,height=1.25in,clip,keepaspectratio]{figures/photo-CN}}]
%  {Cameron Nowzari} received his B.S. in Mechanical Engineering from
%  the University of California, Santa Barbara in June 2009. He received
%  his M.S. and Ph.D. in Engineering Sciences from the University
%  of California, San Diego in December 2010 and September 2013,
%  respectively. He is currently working as a Postdoctoral Research
%  Associate at the University of Pennsylvania.
%  He was a finalist for the Best Student Paper Award at the 2011 American
%  Control Conference and received the 2012 O. Hugo Schuck Best Paper
%  Award in the Theory category. His current research interests include dynamical systems and
%  control, sensor networks, distributed coordination algorithms,
%  robotics, applied computational geometry, event- and self-triggered
%  control, Markov processes, and spreading processes.
%\end{biography}
%
%\vspace*{-7ex}


%Bios

\begin{floatingfigure}[l]{1.25in}
\includegraphics[width=1in,height=1.25in,clip,keepaspectratio]{plots/CNbio.jpg}
\end{floatingfigure}
Cameron Nowzari received the Ph.D. in Mechanical Engineering from the University of California, San Diego in September 2013. He then held a postdoctoral position with the Electrical and Systems Engineering Department at the University of Pennsylvania until 2016. He is currently an Assistant Professor with the Electrical and Computer Engineering Department at George Mason University, in Fairfax, Virginia. He has received several awards including the American Automatic Control Council's O. Hugo Schuck Best Paper Award, the IEEE Control Systems Magazine Outstanding Paper Award, and the International Conference on Data Mining Best Paper Award. His current research interests include dynamical systems and control, distributed coordination algorithms, robotics, event- and self-triggered control, Markov processes, network science, spreading processes on networks, and the Internet of Things.

\hspace{-50ex}
%\hspace{-25ex}
\begin{floatingfigure}[l]{1.25in}
\includegraphics[width=1.25in,height=1in,clip,keepaspectratio]{plots/JBbio2.jpg}
\end{floatingfigure}
James Berneburg received the BS in Electrical Engineering from George Mason University in 2017. He is currently pursuing a Ph.D. in electrical engineering from George Mason University. He was a finalist for the Best Student Paper Award at the 2019 American Control Conference. His research interests include multi-agent systems, event-triggered control, and nonlinear control.


 
%\begin{biography}[{\includegraphics[width=1in,height=1.25in,clip,keepaspectratio]{figures/photo-JC}}]
  %{Jorge Cort\'es} 
  %received the Licenciatura degree in mathematics from Universidad de
  %Zaragoza, Zaragoza, Spain, in 1997, and the Ph.D. degree in
  %engineering mathematics from Universidad Carlos III de Madrid,
  %Madrid, Spain, in 2001.  He held post-doctoral positions with the
  %University of Twente, Twente, The Netherlands, and the University of
  %Illinois at Urbana-Champaign, Urbana, IL, USA. He was an Assistant
  %Professor with the Department of Applied Mathematics and Statistics,
  %University of California, Santa Cruz, CA, USA, from 2004 to 2007. He
  %is currently a Professor in the Department of Mechanical and
  %Aerospace Engineering, University of California, San Diego, CA,
  %USA. He is the author of Geometric, Control and Numerical Aspects of
  %Nonholonomic Systems (Springer-Verlag, 2002) and co-author (together
  %with F. Bullo and S. Mart\'inez) of Distributed Control of Robotic
  %Networks (Princeton University Press, 2009). He is an IEEE Fellow
  %and an IEEE Control Systems Society Distinguished Lecturer.  His
  %current research interests include distributed control, networked
  %games, power networks, distributed optimization, spatial estimation,
  %and geometric mechanics.
%\end{biography}


\end{document}
