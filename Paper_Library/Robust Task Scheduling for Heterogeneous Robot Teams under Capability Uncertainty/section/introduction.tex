\section{Introduction}
\IEEEPARstart{T}{echnological} advances in sensing and control have enabled robotic applications in an ever-growing scope. On the other hand, the growing complexity and requirements of the applications soon exhaust the capability of a single robot: limited by design rules and actuator/sensor power, even the most competent robot is not able to handle all real-world tasks alone. This trend fosters the recent proliferation of multi-agent system applications in agriculture \cite{tokekar2016sensor}, warehouse management \cite{wurman2008coordinating}, construction \cite{werfel2014designing}, defense \cite{shishika2020cooperative}, exploration \cite{cai2021non, quann2017energy, quann2018ground, quann2019chance, quann2020power}, and surveillance \cite{schlotfeldt2018resilient, yu2019coverage, sung2020distributed}.
The advantages of replacing a large omnipotent robot with a team of smaller and less powerful robots include the robustness to agent failures, resilience of team configuration, and lower maintenance costs (a large robot with the same task capabilities is usually harder to design and costlier due to the system complexity) \cite{tan2013research}.

\begin{figure}[t]
    \centering
    \includegraphics[width=0.9\linewidth]{figure/model_overview.pdf}
    \caption{
    A graphical model for a problem with three agent species and two tasks. There are three types of capabilities: flying, perception, and transport people.
    The capability values are defined according to actual tasks. For instance, a quadcopter that can fly, explore an area at the rate of 1 m\(^2/\)s on average, and seat 0 people, has the expected capability values as \([1,1,0]^\transpose\).
    The rescue task requires the ability of perception and transporting people. The surveillance task requires both flying and perception capabilities.
    The capability to fly is noncumulative, which is defined in in Sec. \ref{sec:terminology} and equation \eqref{eqn:task_input}.
    The framework will consider time, energy, agent capabilities, and task constraints, and generate the schedules and routes for the teams.
    In this example, there are three agent species for selection and after the optimization, the rescue task is assigned to a team of two ambulances and a quadcopter, while the surveillance task is conducted by three quadcopters.
    }
    \label{fig:model_overview}
\end{figure}

\begin{figure*}[t]
    \centering
    \includegraphics[width=1\linewidth, trim=0 0 0 0, clip]{figure/main_flowchart.pdf}
    \caption{System architecture. Input: cost (energy) maps (an example is in Sec. \ref{sec:explore_experiment}), agent capabilities, and task requirements. Output: agent routes, schedules, and team formations. There are two major components. A routing, scheduling, and risk minimization model (Sec. \ref{sec:problem_model}) generates a network flow for each agent species. And then these networks are further decomposed into individual agent plans through a flow decomposition model (Sec. \ref{sec:flow_decompose}).}
    \label{fig:flow_chart}
\end{figure*}

The reason that a team of less powerful robots can achieve the same or more complex tasks with the above advantages is that the functional heterogeneity within the team, i.e., distinct sensor and actuator capabilities of the robots, can complement each other during a task.
In contrast to structural heterogeneity (e.g., maximum velocity or energy capacity), functional heterogeneity usually leads to fundamental differences between task capabilities (e.g., the ability to fly or transport people) \cite{sundar2017path}.
The concepts of agent and team capabilities and task requirements are described in Sec. \ref{sec:terminology}.

The fundamental problems \cite{korsah2013comprehensive} that arise when applying a functionally heterogeneous multi-agent system to a mission containing multiple complex tasks include: understanding how to decompose a task into elemental tasks (how), determining which agents should be assigned to a particular elemental task (who), and deriving a schedule that enables the heterogeneous team to successfully complete the task (when).
Therefore, considering the whole problem as task allocation; then, task allocation = task decomposition + assignment + scheduling.
The concepts of task decomposition, elemental tasks, and complex tasks are formally defined in Sec. \ref{sec:terminology}.
In the example in Fig. \ref{fig:model_overview}, the task decomposition problem considers how to decompose a task's required capabilities into agent roles such that after being assigned to the appropriate agents, the team capability surpasses the requirements. Task scheduling determines the routes such that the agents will arrive at a same task together while minimizing the energy and time usage.


A task is considered complex if there exist multiple decompositions and it is unknown which decomposition should be selected without simultaneously considering task assignment and schedule planning (i.e., the problem is coupled) \cite{korsah2013comprehensive}.


Among the few previous works that deal with functional heterogeneity in a complex task setting, most optimization models are designed for use with a specific application.
Therefore, there is a need for fundamental methods that provide a more systematic representation that can generalize within a larger scope of complex tasks.

An important aspect to consider when dealing with dynamic systems is the concept of uncertainty. Two strategies can be adopted to counter failures \cite{zhou2021multi}: 1) robust planning algorithms that consider the uncertainty pre-execution and generate a plan that is prepared to withstand the uncertain environment; 2) adaptive and reactive algorithms that recover the system from failure due to uncertainties in real-time.

In a task allocation setting, one source of the uncertainty is the inaccurate estimation of or the inherent variation within the task requirements and agent capabilities. We apply the robust planning strategy to generate more reliable plans in the pre-execution stage. A robust task allocation plan and schedule should secure task success under such uncertainty by providing redundant capabilities in the team.
Mathematically, a more robust task assignment plan means there is a higher probability that the number of capabilities in a team exceeds the number required by the task.
However, such a redundancy increases the time and energy cost, and a careful trade-off is required.
In Sec. \ref{sec:problem_model}, we will introduce how we encode this robustness in the optimization and determine a trade-off.

In this paper, we present a \textbf{Capability-based robust Task Assignment and Scheduling (\modelrisk{})} framework to optimize task decomposition, assignment, and scheduling simultaneously within a stochastic task model, which can generalize to multiple practical scenarios.
Consider a set of heterogeneous agents (can be robots/vehicles/humans) and complex tasks; the proposed framework will form agent teams, schedules, and routes to minimize energy and time costs for completing the task combined with the risk of non-completion.
We define this class of problems as the \textbf{heterogeneous teaming problem (HTP)}.
The system architecture is summarized in Fig.~\ref{fig:flow_chart} and will be discussed in the following sections.

Note that the system focuses on the optimization that distributes the capabilities to the tasks at a low cost.
The information about agent capabilities, task requirements, and spatial traveling costs are gathered and a routing and teaming plan is generated premission in a centralized planner. Individual plans are sent to the agents and do not change during the task execution.
Though the plans do not change, they are designed to withstand the uncertainty and variations in capability estimation and are less likely to fail. Future extensions will consider a recovery strategy and update the plans during the task execution. 
Communication between agents that enables information gathering and plan synchronizing is not the focus of this paper and will be considered in future work.

\subsection{Contributions}
In \cite{fu2020heterogeneous}, we dealt with a deterministic variation of such a problem, where we assumed exact information (instead of a distribution) of the agent capabilities and task requirements was known. In this work, we develop a generalizable framework for task assignment and scheduling that systematically represents heterogeneous and uncertain task requirements and agent capabilities. This paper provides the following contributions

\begin{enumerate}[label={\arabic*)}] %\setcounter{enumi}{-1}
  \item The development of a modeling framework that captures uncertainties within the task requirements and agent capabilities.  \label{item:first_contribution} 
  \item The derivation of a cost function that incorporates the concept of risk within the minimization.
  \item Reformulation of the heterogeneous teaming problem (HTP) to provide a more scalable algorithm that is solved by using a flow decomposition subproblem.
  \item The implementation of a capture-the-flag game simulation to compare the task assignment performance to a baseline algorithm.
\end{enumerate}

The problem size depends on the number of tasks, agent species, and agents per species. With the reformulation in this paper (compared to \cite{fu2020heterogeneous}), the routing of individual agents is decoupled and postponed to a flow decomposition subproblem which reduces the size of the optimization program for the HTP.
The HTP only considers the behavior of an entire agent species, and its size no longer depends on the number of agents per species. The complexity of the framework is still exponential, as we use mixed-integer programming to find the exact optimal solution \cite{karp1972reducibility}. But the reformulation suppresses the growth such that the framework can solve practical problems with up to 140 agents and 40 tasks.

The open-source code is available at: \\ \indent
{\small \href{https://brg.engin.umich.edu/publications/robust-task-scheduling/}{https://brg.engin.umich.edu/publications/robust-task-scheduling}}

\subsection{Outline}
The remainder of this paper is organized as follows. Sec.~\ref{sec:related_work} briefly introduces the related work of multi-agent task allocation and concludes with research gaps that this work investigates.
Sec. \ref{sec:problem_model} describes the general problem mathematically and presents a risk minimization framework.
Sec. \ref{sec:lshaped_algorithm} introduces an algorithm to solve the risk minimization problem and output a general routing plan described as network flows of agent movements.
Sec. \ref{sec:flow_decompose} proposes a network flow decomposition problem to generate routing plans and schedules for individual agents.
The two-step process in Sections \ref{sec:lshaped_algorithm} and \ref{sec:flow_decompose} improves the scalability of the framework compared to a solution method that outputs individual plans directly within one optimization.
Sec.~\ref{sec:experiments} discusses the experiments and results.
Finally, Sec.~\ref{sec:conclusion} concludes the paper and provides ideas for future work based on current limitations.


\subsection{Concepts and Terminology}\label{sec:terminology}

In this section, we provide definitions of multiple concepts that will be used throughout the paper. For some of the terminology, we refer to \cite{korsah2013comprehensive}, but simplify the description to provide only the core meaning.

\textbf{Agent capabilities.} A numeric vector that describes the agent's fitness to conduct a task.

\textbf{Team capabilities.} A set of fitness values aggregated from agent capabilities in the team.

\textbf{Cumulative capabilities.} A capability that sums the agent capabilities across an entire team to get the team capability.

\textbf{Noncumulative capabilities.} A capability that does not sum.
In this paper, noncumulative team capability is the minimum of the agent capabilities in the team.\footnote{In Fig. \ref{fig:model_overview}, the capability to fly is noncumulative: the team can fly only when all agents in the team can fly.}


\textbf{Task requirement.} A set of constraints that relate the task and agents. Usually, it is specified as a number of required agents, agent actions, or agent capabilities that have to be satisfied. In this paper, it is defined as a set of required team capabilities and will be described in Sec. \ref{sec:problem_model}.

\textbf{Task decomposition.} Representing a task as a set of elemental tasks (there can be constraints between these subtasks), such that completing the subtasks completes the original task.

\textbf{Task assignment.} Identifying which specific agent should handle an elemental task.

\textbf{Elemental task}: A task that is not decomposable and should be assigned to only one agent.

\textbf{Compound task.} A task that can be decomposed into a set of elemental tasks in only one fixed way.

\textbf{Complex task.} A task that can be decomposed into elemental tasks in more than one way, i.e., following different decompositions, the resulting sets will contain different numbers or types of elemental tasks.

