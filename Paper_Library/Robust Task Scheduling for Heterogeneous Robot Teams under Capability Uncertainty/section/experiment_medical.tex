\subsection{Robotic Services during a Pandemic}\label{sec:explore_experiment}

In this section, we demonstrate the generality and the task assignment and scheduling components of our framework by considering a more complex scenario inspired by the COVID-19 pandemic.
We use our previous work in \cite{fu2020heterogeneous} as a baseline.
The energy cost and probability of success will be used to evaluate the optimality of the generated plans. The optimality gap given limited time for the optimization will be used to evaluate the scalability of the framework.


\subsubsection{Experiment Setup and Model Description}

Consider pandemic robotic services in a city environment consisting of multiple delivery, disinfection, test, and treatment sub-tasks that require one or multiple agents to complete.
The task-required capabilities and agent capabilities are modeled as random distributions.
We choose to include 8 types of tasks and 7 agent species in our investigation, some of the agents needing a human operator.

Applying the \modelrisk{} model to describe the problem of the pandemic robotic services, we first define 9 capability types according to the chosen tasks and agents in TABLE \ref{tab:capability_type}.

The 7 agent species and their capabilities are defined in TABLE \ref{tab:medical_vehicle_capability}, according to real-world references \cite{barfoot2020making}. The columns are different capabilities. The agent capabilities are Gaussian distributed random variables in this example. The values in the table are expectations of the random distributions, and the standard deviations are 10\% of the expectations.

The 8 task types are described in TABLE \ref{tab:medical_task_description}. Their required capabilities are listed in TABLE \ref{tab:medical_task_requirement}.
The columns are different capabilities. A value in the table is the expected requirement for that specific capability, and the standard deviation is 10\% of the expectation. \(\gamma\) is a scaling coefficient. E.g., the \(\gamma\) in (row 1, column 3) means task type \(1\) requires the team's capability \(\alpha_3 \geq \gamma\). When \(\gamma\) is set larger, more agents get involved in the optimization and the problem space is larger. We do not scale noncumulative capabilities. For one task, the requirements on different capability types are imposed with `and' logic.
These tasks are distributed in a city. An example of 16 tasks (two tasks from each type) is shown in Fig. \ref{fig:medical_task_distribution}. The agents start from the base, which, in practice, can be a hospital.

We use the M3500 dataset \cite{olson2006fast} as the city's road map in this illustration.
We assume that we have a viral exposure level (a real number) at some locations and can infer a cost map of the virality level using Gaussian process \cite{williams2006gaussian, fu2020heterogeneous} based on the assumption that the neighborhood of a high-cost location is also high-cost.
For example, in the city map in Fig. \ref{fig:medical_task_distribution}, we randomly choose 6 contaminated and 6 proven-safe regions as samples and learn a viral exposure map. The blue and red regions have low and high cost, respectively, while the white regions have less information, high ambivalence, and are assigned a medium cost. Based on this, we compute a viral-exposure-based travel cost (A* path \cite{pohl1973avoidance}) between the tasks and regard it as the energy cost for the edges in the graph in Fig. \ref{fig:graphical_model}.

We choose the agents and tasks from their types in TABLEs \ref{tab:medical_vehicle_capability} and \ref{tab:medical_task_requirement} and compose 32 mission cases where the agent numbers, task numbers, and the coefficient \(\gamma\) are chosen from \{21, 70, 140\}, \{16, 24, 32, 40\}, and \{1, 3, 5, 10\}, respectively.
For each mission case, we generate 6 instances with randomly selected tasks locations, which result in different graphical models in Fig. \ref{fig:graphical_model}. For the test cases, task \(i\) and \({i+8}\) have the same requirements, but are at different locations.

Based on these cases, we evaluate the mission performance and computational cost of the three models in TABLE \ref{tab:three_models}:
\begin{itemize}
    \item \modelrisk{}: The risk minimization model
    \item \modeldet{}: The risk part, \(h_i\), is removed from the objective function \eqref{eqn:nonlin_penalty_objective}.
    \item \modelint{}: Solve the same problem as \modeldet{} without the flow decomposition. The size of the math problem is larger and harder to solve. This is our baseline from \cite{fu2020heterogeneous} to compare.
\end{itemize}

\newlength{\tasktabwid}
\setlength{\tasktabwid}{0.02\linewidth}

% Table generated by Excel2LaTeX from sheet 'medical'
\begin{table}[t]
  \centering
  \caption{Definitions of the capability types.}
  \label{tab:capability_type}%
    \begin{tabular}{ll}
    \toprule
    Cap & Definition \\
    \midrule
    \(1\)     & Fly (noncumulative) \\
    \(2\)     & Equipped with a freezer \\
    \(3\)     & Deliver materials \\
    \(4\)     & Conduct perception \\
    \(5\)     & Remove and collect harmful materials \\
    \(6\)     & Conduct viral test \\
    \(7\)     & Physically interact and conduct treatment \\
    \(8\)     & Spray disinfectant \\
    \(9\)     & Place signals and barricades to conduct quarantine enforcement \\
    \bottomrule
    \end{tabular}%
\end{table}%

% Table generated by Excel2LaTeX from sheet 'medical'
\begin{table}[t]
  \centering
  \caption{Stochastic agent capabilities.}
  \label{tab:medical_vehicle_capability}%
    \begin{tabular}{lp{\tasktabwid}p{\tasktabwid}p{\tasktabwid}p{\tasktabwid}p{\tasktabwid}p{\tasktabwid}p{\tasktabwid}p{\tasktabwid}p{1.2\tasktabwid}}
    \toprule
    Agent species & \(1\)     & \(2\)     & \(3\)     & \(4\)     & \(5\)     & \(6\)     & \(7\)     & \(8\)     & \(9\) \\
    \midrule
    \(1\): Quadcopter & 1     &       & 1     & 1     &       &       &       & 1     &   \\
    \(2\): Vehicle &       &       & 1     &       &       &       &       &       & 2 \\
    \(3\): Vehicle (freezer) &       & 1     & 1     &       &       &       &       &       &   \\
    \(4\): Vehicle (contaminants) &       &       &       &       & 1     &       &       & 1     &   \\
    \(5\): Guidance robot &       &       &       & 1     &       &       &       &       & 5 \\
    \(6\): Test robot &       &       &       &       &       & 1     &       &       &   \\
    \(7\): Treatment robot &       &       &       &       &       & 1     & 1     &       &   \\
    \bottomrule
    \end{tabular}%
\end{table}%

% Table generated by Excel2LaTeX from sheet 'medical'
\begin{table}[t]
  \centering
  \caption{Stochastic task requirements.}
  \label{tab:medical_task_requirement}%
    \begin{tabular}{l@{\hskip -0.005\linewidth} p{\tasktabwid}p{\tasktabwid}p{\tasktabwid}p{\tasktabwid}p{\tasktabwid}p{\tasktabwid}p{\tasktabwid}p{\tasktabwid}p{1.2\tasktabwid}}
    \toprule
    Task type & \(1\)     & \(2\)     & \(3\)     & \(4\)     & \(5\)     & \(6\)     & \(7\)     & \(8\)     & \(9\) \\
    \midrule
    \(1\): Goods delivery &       &       & \(\gamma\)     &       &       &       &       &       &  \\
    \(2\): Goods delivery (fly) & \(1\)     &       & \(\gamma\)     &       &       &       &       &       &  \\
    \(3\): Medical kits delivery &       & \(\gamma\)     & \(\gamma\)     &       &       &       &       &       &  \\
    \(4\): Contaminants removal &       &       &       &       & \(\gamma\)     &       &       & \(\gamma\)     &  \\
    \(5\): Open area disinfection &       &       &       & \(\gamma\)     & \(\gamma\)     &       &       & \(2\gamma\)     &  \\
    \(6\): Quarantine enforcement &       &       & \(\gamma\)     & \(\gamma\)     & \(\gamma\)     &       &       & \(\gamma\)     & \(10\gamma\) \\
    \(7\): Viral tests &       &       &       &       &       & \(\gamma\)     &       &       &  \\
    \(8\): Remote treatment &       &       &       &       &       & \(\gamma\)     & \(\gamma\)     &       &  \\
    \bottomrule
    \end{tabular}%
\end{table}%

\begin{table}[t]
  \centering
  \caption{Task descriptions.}
  \label{tab:medical_task_description}%
    \begin{tabular}{lp{0.55\linewidth}}
    \toprule
    Task type & Description \\
    \midrule
    \(1\): Goods delivery & Deliver goods. \\
    \(2\): Goods delivery (fly) & Deliver goods using quadcopters. \\
    \(3\): Medical kits delivery & The team should be equipped with freezers. \\
    \(4\): Contaminants removal &  Remove contaminated materials and then disinfect the location. \\
    \(5\): Open area disinfection &  Perceive a contaminated area to conduct contaminants removal and disinfection.\\
    \(6\): Quarantine enforcement & Perceive, carry materials to, disinfect a contaminated area, and then place signals and barricades to enforce a quarantine.\\
    \(7\): Viral tests & Conduct viral tests.\\
    \(8\): Remote treatment & Conduct viral tests and treatment.\\
    \bottomrule
    \end{tabular}%
\end{table}%

% Table generated by Excel2LaTeX from sheet 'medical'
\begin{table}[htb!]
  \centering
  \caption{Three teaming models. `Agent var' stands for `agent variables'.}
  \label{tab:three_models}%
    \begin{tabular}{llll}
    \toprule
    Model & Agent var & Objective & Variable number \\
    \midrule
    \modelint{} \cite{fu2020heterogeneous}  & Binary & Energy + Time & Large \\
    \modeldet{}   & Real & Energy + Time & Relatively small \\
    \modelrisk{}  & Real & Energy + Time + Risk & Relatively small \\
    \bottomrule
    \end{tabular}%
\end{table}%

\subsubsection{Computational Cost and Discussion}\label{sec:medical_experiment_computation}

A 120-second time limitation is added to the solvers of the models for all test cases.
For each mission case, we run the 6 test instances and if the solver can find a feasible solution within the time limit, we consider it successful. The success rate of each model is shown in Fig. \ref{fig:medical_success_rate}.
We show the detailed average performances for the mission sizes with success rates larger than 50\% in Fig. \ref{fig:medical_three_models}-\ref{fig:medical_risk_approx_gap}.
The optimality gaps after the rounding process in Sec. \ref{sec:flow_round_up} are given in Fig. \ref{fig:medical_ori_lin_gap}. 
The blue cells mean the solver's success rate for the specific mission size is less than 50\% for the time limit picked.
According to the two figures, \modeldet{} and \modelrisk{} can solve much larger teaming problems than \modelint{}. The largest problem that \modeldet{} solves in this example involves 140 agents and 40 tasks.

The increases in the optimality gaps due to the rounding process are shown in Fig. \ref{fig:medical_d_lin_gap}. For most of the cases, the rounding process introduces an increase in the optimality gap within 1\%. The leftmost cases have larger increases because the agent numbers are small, and rounding has a larger overall influence. Since \modelint{} involves no rounding process, the increased gap is 0 for all test cases.

In these test cases, the agent capabilities are Gaussian distributed. Therefore, given the estimation of the means and variances, we can calculate the probability of success and its mean as follows
\begin{align}
    P (\text{task } i \text{ succeeds}) =& \underset{\text{capability } a}{\prod} P(a \text{ satisfied}) \quad \forall i \in M \nonumber \\ % \label{eqn:task_prob_success} \\
    \text{mean } P(\text{success}) =& \left(\underset{i \in M}{\prod} P (\text{task } i \text{ succeeds}) \right)^{1/n_m}  \nonumber
\end{align}

The mean probabilities of success for the three models are shown in Fig. \ref{fig:medical_mean_prob}. For all cases, \modelint{} and \modeldet{} models roughly result in a probability of 0.25. This is because each task requires two capabilities on average. When the risk is not considered, these two models tend only to match the expected requirement, and the probability of matching a single capability is 0.5. 
Clearly, the risk minimization model increases the probability of success for the tasks.

In Fig. \ref{fig:medical_risk_vs_det}, we compare the result of \modelrisk{} to its deterministic version \modeldet{}. With the chosen \(\beta\)\footnote{\(\beta\): hyperparameter in the CVaR function \(\eta_\beta(\cdot)\) in Fig. \ref{fig:cvar_definition} and objective function \eqref{eqn:nonlin_cvar_objective_h_min} and \eqref{eqn:nonlin_cvar_objective_h_sum}.},
\(C_e\), and \(C_h\)\footnote{\(C_e\) and \(C_h\): the penalty coefficients on energy cost and task completion in equation \eqref{eqn:nonlin_penalty_objective}.}, for most of the cases, a \(\sim\)20\% increase in energy cost introduces a  \(\sim\)35\% increase in the mean probability of success.

The trade-off between energy cost and robustness in the objective function and can be tuned smoothly.
The trade-off gained by changing the \(C_e\), \(C_h\), and \(\beta\) is shown in Fig. \ref{fig:medical_prob_eng_tradeoff}, using the smallest test case as an example. The penalty coefficient has a major impact on the trade-off between energy and the probability of success, while \(\beta\) has a local impact on the trade-off. According to our investigation, choosing \(\beta = 0.8 - 0.97\) would mostly cover the Pareto set. This shows the advantage of using the CVaR as a risk metric: better Pareto optima are gained than optimizing the expectation (\(\beta \rightarrow 0\)).


\newcommand{\medicalfolder}{figure/medical_revise} % figure/medical1: rainbow color map, figure/medical: summer color map
\begin{figure}[hbt!]
	\centering
	\includegraphics[height=0.60\linewidth, trim=120 190 140 215, clip]{figure/medical_revise/success_rate.pdf}
	\caption{The success rate of the models for each mission test case (6 instances are tested for each case). The grayscale colors correspond to the value specified in the color bars on the right. }
	\label{fig:medical_success_rate}
\end{figure}

\begin{figure}[hbt!]
    \centering
	\subfloat[\label{fig:medical_ori_lin_gap}]{
    	\includegraphics[height=0.60\linewidth, trim=120 190 140 215, clip]{\medicalfolder/ori_lin_gap.pdf}}
	\subfloat[\label{fig:medical_d_lin_gap}]{
    	\includegraphics[height=0.60\linewidth, trim=210 190 140 215, clip]{\medicalfolder/d_lin_gap.pdf}}
    \hfill
	\caption{The optimality gap of the three models applied to the 32 test cases, defined as {(objective value - lower bound) / lower bound}. A smaller optimality gap indicates a better solution. The grayscale colors correspond to the value specified in the color bars on the right. The \textbf{blue cells} mean the solver's success rate for the specific mission case is \(<50\%\) for the time limit picked (120 seconds). (a) The optimality gap before the flow rounding process. The two white cells in the \modelrisk{} group are outliers whose values are around 1.0. (b) The increased optimality gap due to the rounding process.}
	\label{fig:medical_three_models}
\end{figure}

\begin{figure}[hbt!]
	\centering
	\includegraphics[height=0.60\linewidth, trim=120 190 140 215, clip]{\medicalfolder/mean_prob.pdf}
	\caption{The mean probability of success for the tasks using the three models.}
	\label{fig:medical_mean_prob}
\end{figure}

\begin{figure}[hbt!]
    \centering
    \subfloat[]{
    	\includegraphics[height=0.23\linewidth, trim=115 290 140 325,  clip]{\medicalfolder/compare_enegy32.pdf}} % 160 300 140 330
    \subfloat[]{
    	\includegraphics[height=0.23\linewidth, trim=115 290 140 325, clip]{\medicalfolder/compare_prob32.pdf}}
	\hfill
	\caption{Comparing the results of \modelrisk{} to \modeldet{}. The values (colors) in the girds are (a) increased energy (relative) and (b) increased mean probability (relative) by adding the risk as an objective. I.e., the values are (\modelrisk{} \(-\) \modeldet{}) / \modeldet{}.}
	\label{fig:medical_risk_vs_det}
\end{figure}

\begin{figure}[hbt!]
	\centering
	\includegraphics[width=0.52\linewidth, trim=120 280 140 320, clip]{\medicalfolder/approx_gap.pdf} % 120 300 140 320
	\caption{The relative approximation gap of the nonlinear CVaR through sampling and linear programming.}
	\label{fig:medical_risk_approx_gap}
\end{figure}

\begin{figure}[hbt!]
	\centering
	\includegraphics[width=0.8\linewidth, trim=75 240 100 250, clip]{\medicalfolder/prob_eng_tradeoff.pdf}
	\caption{Trade-off between energy cost and risk of task's non-completion for a case with \{16 tasks, 21 agents, and \(\gamma = 1\)\}. The upper-left corner is the direction of the Pareto set.}
	\label{fig:medical_prob_eng_tradeoff}
\end{figure}


For the sample average approximation of the CVaR, we use 500 samples to approximate the Gaussian distributions. Given the solution, we compare the nonlinear objectives to their sample approximations and find that the relative approximation gaps are smaller than \(1\%\) for most of the cases, as shown in Fig. \ref{fig:medical_risk_approx_gap}. This shows that the approximation quality is good using 500 samples.



\begin{figure*}[t]
% 	\centering
	\subfloat[\label{fig:medical_task_distribution}]{
    	\includegraphics[width=0.53\linewidth, trim=90 280 60 260, clip]{figure/medical_traj/medical_map.pdf}}
	\subfloat[]{
    	\includegraphics[width=0.4488\linewidth, trim=90 280 130 260, clip]{figure/medical_traj/medical_v1.pdf}}
	\hfill
	\subfloat[]{
    	\includegraphics[width=0.4488\linewidth, trim=90 280 130 260, clip]{figure/medical_traj/medical_v4.pdf}}
    \hspace{0.073\linewidth}
	\subfloat[]{
    	\includegraphics[width=0.4488\linewidth, trim=88 280 130 260, clip]{figure/medical_traj/medical_v5.pdf}}
	\caption{(a) Task distribution: the 16 tasks are distributed in a city \cite{olson2006fast} where the unit travel cost depends on viral exposure. Blue and red stand for low and high energy costs, respectively. (b)-(d) The planned routes from the \modelrisk{} model for species \(1\), \(4\), and \(5\). Different colored lines represent distinct agent individuals from the same species.}
	\label{fig:medical_traj}
\end{figure*}


\subsubsection{Mission Performance and Discussion}\label{sec:medical_experiment_performance}

This section uses a test case with 16 tasks, 21 agents, and \(\gamma = 1\) as an example to compare the solutions generated by the three models. 
In TABLE \ref{tab:medical_picked_result}, we list the performance metrics of 5 tasks whose success rate is increased by the \modelrisk{} model. All three models are solved to optimal within the time limit. The models presented in this paper end up with much fewer variables for the same problem. By comparing \modelrisk{} with \modeldet{}, we see that the risk minimization model reduces the CVaR and increases the probability of success for 5 tasks out of 16, with a small increase (5.3\%) in the overall energy cost.

The teams for the last 8 tasks are shown in TABLE \ref{tab:medical_picked_teams}.
In the table, \(v_i\) stands for one agent from species \(i\).
As expected, the \modelrisk{} model puts more agents in the team to reduce the CVaR. This task assignment results from simultaneously considering the energy cost. As an example, \modelrisk{} puts more agents in the team of tasks \(9\)-\({12}\), but not tasks \(1\)-\(4\), even if they are of the same task types. Because energy and time costs are jointly considered, and tasks \(1\)-\(4\) are far from the depot, adding more agents to ensure redundancy and robustness could potentially result in much higher costs. The risk minimization model also generates the routes and a consistent schedule. For example, one agent from species \(1\), \(4\), and \(5\) visit task \({13}\) at the same time as a team. The routes of species \(1\), \(4\), and \(5\) are shown in Fig. \ref{fig:medical_traj}. These routes minimize overall travel distances and avoid the high-cost red regions to lower energy costs.

In summary, according to the teams in TABLE \ref{tab:medical_picked_teams} and routes in Fig. \ref{fig:medical_traj}, the \modelrisk{} framework generates a consistent schedule for coordination, outputs routes that minimize energy costs, and assigns tasks to agents such that redundancy is preserved at low costs to ensure a higher probability of task completion. The computational evaluation in Sec. \ref{sec:medical_experiment_computation} (particularly, Fig. \ref{fig:medical_three_models}) shows that the frameworks \modelrisk{} and \modeldet{} scale to 140 agents and 40 tasks with low optimality gaps. The scalability in agent number is better than the task number. Furthermore, \modeldet{} still shows no optimality gap degeneration dealing with the largest test case we tested.

% Table generated by Excel2LaTeX from sheet 'medical2'
\begin{table}[t]
  \centering
  \caption{A comparison of the results of the three models.
  }
  \label{tab:medical_picked_result}%
    \begin{tabular}{llll}
    \toprule
    Item  & \modelint{}   & \modeldet{}   & \modelrisk{} \\
    \midrule
    Variables & 6194  & 4276  & 4384 \\
    Task CVaR & 5.41 \(\times 10^3\) & 5.41 \(\times 10^3\) & -1.03 \(\times 10^4\) \\
    Energy cost & 2.06 \(\times 10^5\) & 2.06 \(\times 10^5\) & 2.17 \(\times 10^5\) \\
    \(P(\text{task }  {9}) \)    & 0.5 & 0.5 & 1 \\
    \(P(\text{task }  {10})\)    & 0.25 & 0.25 & 0.5  \\
    \(P(\text{task }  {11})\)    & 0.25 & 0.25 & 1 \\
    \(P(\text{task }  {12})\)    & 0.25 & 0.25 & 0.5  \\
    \(P(\text{task }  {13})\)    & 0.125 & 0.125 & 0.25 \\
    \bottomrule
    \end{tabular}%
\end{table}%


% Table generated by Excel2LaTeX from sheet 'medical2'
\begin{table}[t]
  \centering
  \caption{Team configurations given by the three models.
  }
  \label{tab:medical_picked_teams}%
    \begin{tabular}{llll}
    \toprule
    Task  & \modelint{}   & \modeldet{}   & \modelrisk{} \\
    \midrule
    \(9\)  & \(v_3    \) & \(v_3    \) & \(v_1 \times 3, v_2 \times 3, v_3 \times 3 \) \\
    \({10}\)  & \(v_1    \) & \(v_1    \) & \(v_1 \times 2 \) \\
    \({11}\)  & \(v_3    \) & \(v_3    \) & \(v_1 \times 2, v_2 \times 3, v_3 \times 3 \) \\
    \({12}\)  & \(v_4    \) & \(v_4    \) & \(v_1 \times 2, v_4 \) \\
    \({13}\)  & \(v_1, v_4    \) & \(v_1, v_4    \) & \(v_1, v_4, v_5 \) \\
    \({14}\)  & \(v_2 \times 3, v_4, v_5\) & \(v_2 \times 3, v_4, v_5\) & \(v_2 \times 3, v_4, v_5\) \\
    \({15}\)  & \(v_7\) & \(v_7\) & \(v_7\) \\
    \({16}\)  & \(v_7\) & \(v_7\) & \(v_7\) \\
    \bottomrule
    \end{tabular}%
\end{table}%


