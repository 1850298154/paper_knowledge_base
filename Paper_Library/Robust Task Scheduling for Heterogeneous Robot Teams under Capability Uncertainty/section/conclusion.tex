\section{Conclusions and Future Work}\label{sec:conclusion}

This paper addresses a complex task allocation problem in the category of CD-[ST-MR-TA].
We propose a mixed-integer programming model that simultaneously optimizes the task decomposition, assignment, and scheduling. The uncertainty within the team's capability is considered through risk minimization, and a robust metric, conditional value at risk (CVaR), is minimized to ensure the robustness to task completion (enough capability are assigned). The framework contributes to a domain-independent representation for complex tasks and heterogeneous agent capabilities that can generalize to multi-agent applications where the major goals are satisfying task-required capabilities. A two-step solution method is described, and the whole framework is evaluated in two different practical test cases. Results show that the framework scales up to 140 agents and 40 tasks for the cases tested and solves the problems with low optimality gaps. Given the selected hyper-parameters, the resulting assignments and schedules provide a reasonable trade-off between energy, time, and the probability of success. The task assignment performance (apart from the scheduling) is also demonstrated through a comparison with the STRATA framework in the capture the flag case.

Future work will consider an extension of this work to a distributed framework, which could further improve the scalability of the system and the robustness to communication constraints and the loss of agents.
In addition, a probabilistic learning method that automatically estimates the parameters in the representation of task requirements and agent capabilities from current and previous task executions is an interesting future work. Such a learning method would enable the possibility of closing the loop of the task assignment and scheduling, and iteratively improving the performance.
For modeling choices, we will consider imposing a necessary and sufficient energy constraint in the \modelrisk{} model in Sec. \ref{sec:problem_model} while still preserving the scalability of the current framework.


