\appendices
\section{Non-trivialness of the Flow Decomposition}\label{sec:appendix_flow}

A rounding or splitting without solving an optimization can break flow constraints or result in suboptimal solutions.

The rounding process has to maintain the flow constraints: for all nodes, the incoming flow must equal the outgoing flow. A naive rounding might break the flow constraint. Take the node \(3\) in Fig. \ref{fig:round_case1} as an example: naively rounding up the flows on an edge will result in 2 incoming agents, but 3 outgoing agents.

The rounding selection process is not unique, even when meeting the flow constraints.
For instance, Fig. \ref{fig:round_case}b-c are both valid ways to round the flow in Fig. \ref{fig:round_case1}. However, the energy cost of the flow in Fig. \ref{fig:round_case3} is higher. Therefore, the optimal rounding process will need to maintain the flow constraint and introduce the lowest additional energy cost.

\begin{figure}[h!]
	\centering
	\subfloat[\label{fig:round_case1}]{
    	\includegraphics[width=0.8\linewidth, trim=0 0 0 0, clip]{figure/flow/round_case1.pdf}}
    \hfill
	\subfloat[\label{fig:round_case2}]{
    	\includegraphics[width=0.8\linewidth, trim=0 0 0 0, clip]{figure/flow/round_case2.pdf}}
    \hfill
	\subfloat[\label{fig:round_case3}]{
    	\includegraphics[width=0.8\linewidth, trim=0 0 0 0, clip]{figure/flow/round_case3.pdf}}
    \hfill
	\caption{Round the flow of agent species \(1\). The red number is the agent flow on an edge. The black number below an edge is the energy cost of the edge. (a) The original flow with real numbers. (b) Rounding choice 1. (c) Rounding choice 2.}
	\label{fig:round_case}
\end{figure}

After rounding the flow to integers, there are multiple choices to assign individual agent paths. We call this `splitting the flow into paths'. For example, the flow in Fig. \ref{fig:round_case2} can be split into the 3 agent routes either in Fig. \ref{fig:cover_case1} or Fig. \ref{fig:cover_case2}. Though the total energy costs are equivalent, the energy costs of the three paths in the two choices are \{20, 20, 20\} and \{20, 24, 16\}, respectively. The choice affects the behavior of an individual agent. Here, we prefer the former split, because the energy costs of individual agents are more evenly distributed and the maximum energy cost of an individual agent is smaller. We define the optimal flow split as the set of paths that minimizes the maximum individual energy cost.

\begin{figure}[h!]
	\centering
	\subfloat[\label{fig:cover_case1}]{
    	\includegraphics[width=0.8\linewidth, trim=0 0 0 0, clip]{figure/flow/cover_case1.pdf}}
    \hfill
	\subfloat[\label{fig:cover_case2}]{
    	\includegraphics[width=0.8\linewidth, trim=0 0 0 0, clip]{figure/flow/cover_case2.pdf}}
    \hfill
	\caption{Split the agent flow in Fig. \ref{fig:round_case2} into 3 paths (colored). The black number below an edge are the energy cost of the edge. There are two choices with different individual costs. (a) Choice 1. (b) Choice 2.}
	\label{fig:cover_case}
\end{figure}


\section{Proof of Claim}\label{sec:appendix_integer}

\textbf{Definition 1}: The incidence matrix for a directed graph is a matrix of \(n_\text{node}\) rows and \(n_\text{edge}\) columns: for each row, if an edge enters the node, the matrix value is 1; if an edge leaves the node, the value is -1; otherwise 0.

\textbf{Theorem 1}: The incidence matrix of any directed graph is a totally unimodular (TU) matrix \cite{schrijver2003combinatorial, heller1956extension}.

\textbf{Theorem 2}: If \(A_1\) is a TU matrix and \(E\) is an identity matrix, then \(A=[A_1 | E]\) will be a TU matrix \cite{schrijver2003combinatorial}.

\textbf{Theorem 3}: \(A\) is a TU matrix iff \(A^\transpose{}\) is a TU matrix (by definition of a TU matrix \cite{schrijver2003combinatorial}).

\textbf{Theorem 4}: If \(A\) is a TU matrix and \(\mathbf{b}\) is integer-valued, then each vertex of the polytope defined by \(A \mathbf{x} \leq \mathbf{b}\) is integer-valued \cite{schrijver2003combinatorial}.

\begin{figure}[t]
	\centering
	\includegraphics[width=1\linewidth]{figure/totally_unimodular_proof.pdf}
	\caption{Totally unimodularity of the rounding problem.}
	\label{fig:totally_unimodular_proof}
\end{figure}

\begin{proof}
\textbf{The solutions of the rounding problem from Simplex-based algorithms will be integer-valued.}

Now we write the constraints of the rounding problem, i.e., \eqref{eqn:round_bound}-\eqref{eqn:round_flow_constraint}, in the form of \(A \mathbf{x} \leq \mathbf{b}\). See Fig. \ref{fig:totally_unimodular_proof}, suppose the flow constraint \eqref{eqn:round_flow_constraint} is converted into \(A_1 \mathbf{x} \leq \mathbf{b}_1\), and the bound constraint \eqref{eqn:round_bound} is converted into \(A_2 \mathbf{x}_1 \leq \mathbf{b}_2\).

According to \textbf{Definition 1}, \(A_1\) is an incidence matrix of a directed graph. \(\mathbf{b}_1\) is a zero vector. According to the format of constraints \eqref{eqn:round_bound}, \(A_2\) is an identity matrix, and \(\mathbf{b}_2\) is an integer-valued vector. Therefore, \(\mathbf{b}\) is integer-valued.

According to \textbf{Theorem 1}, \(A_1\) is a TU matrix.

According to \textbf{Theorem 3}, \(A_1^\transpose{}\) is a TU matrix.

Because \(A_2^\transpose{}\) is an identity matrix, according to \textbf{Theorem 2}, \([A_1^\transpose{} | A_2^\transpose{}]\) is a TU matrix.

According to Fig. \ref{fig:totally_unimodular_proof} and \textbf{Theorem 3}, \(A = [A_1^\transpose{} | A_2^\transpose{}]^\transpose{}\) is a TU matrix.

Because matrix \(A\) is a TU matrix and \(\mathbf{b}\) is integer-valued, according to \textbf{Theorem 4}, all vertices of the polytope defined by \(A \mathbf{x} \leq \mathbf{b}\) are integer-valued.

If Simplex-based algorithms are applied to the optimization problem, the algorithms search through vertices of the polytope to find the optimal point.
Therefore, the optimal point must be a vertex of the polytope. Since we have proved that all vertices are integral, the solution to \(\mathbf{x}\) will be integral.
\end{proof}





