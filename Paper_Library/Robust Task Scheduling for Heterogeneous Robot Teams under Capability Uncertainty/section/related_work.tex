\section{Related Work}\label{sec:related_work}

According to the taxonomy in \cite{korsah2013comprehensive, gerkey2004formal, khamis2015multi, nunes2017taxonomy, rizk2019cooperative, prorok2021beyond}, our framework for the HTP deals with a task allocation problem in the category CD-[ST-MR-TA]: complex task dependencies (CD), single-task robot (ST), multi-robot tasks (MR), and time-extended assignment (TA). It is one of the most challenging task assignment categories and has been considered in few previous works in literature. The time-extended assignment considers scheduling, in contrast to instantaneous assignments (IA), which only make matches between tasks and robots.

While many previous works about multi-robot tasks are in the category [ST-MR-TA], most of their papers do not deal with complex tasks.
Their papers decompose a multi-robot task into elemental tasks that can be assigned to a subset of single agents.
If such a decomposition exists, the task decomposition problem is presolved and decoupled from the assignment and scheduling problem. Many previous works in vehicle routing \cite{sundar2017path, korsah2012xbots}, job shop scheduling \cite{ozguven2012mixed, ku2016mixed, moser2020flexible}, and robotic soccer games \cite{liemhetcharat2011modeling, liemhetcharat2012weighted} deal with such non-complex (or compound) tasks.
We refer interested readers to the survey paper \cite{korsah2013comprehensive} for more work with compound tasks.

There are systems dealing with complex tasks, but their models do not generalize and are applied to one specific problem. Examples include fire extinguishing and debris removal \cite{jones2011time, pujol2015efficient}, and a coverage problem of unmanned aerial/ground vehicles with recharging behaviors \cite{yu2019coverage}, where tasks are considered complex because multiple decomposition combinations are possible in the problem space.

Some frameworks for CD-[ST-MR-TA] created in previous work generalize to a type of problem, but most of them have scalability issues and are only applicable to a small group of robots.
For instance, Bayesian network representations consisting of tasks, observations, constraints, and action nodes can be used to represent a task with interdependent elemental subtasks \cite{song2010learning, ek2010task}.
Hierarchical tree networks (HTN) are used in \cite{zlot2003multirobot, zlot2003market, zlot2005complex, hayes2016autonomously} to represent tasks where leaf nodes are roles for assignment.
Planning domain definition language (PDDL) can be used to represent a task as a graph of states (nodes) and actions (edges) \cite{klee2015graph, nicolescu2003natural, ekvall2008robot, niekum2012learning, grollman2010incremental, hayes2015effective, galindo2008robot, aeronautiques1998pddl, torreno2017cooperative}.
Temporal planning can be considered an extension to the basic PDDL \cite{benton2012temporal, sapena2016parallel, carreno2020decentralised, crosby2014temporal, cashmore2018temporal}. In addition to a graph of predefined actions and state-space representation, temporal planning allows continuous temporal constraints and interdependence between subtasks to be encoded as predefined logical formulas.
However, these representations based on actions and states are usually applied to systems with less than five agents. The number of states in their trees/graphs could explode rapidly with respect to the number of agents and available actions. The underlying solver (e.g., graph search algorithms) would take a long time to explore a subset of the high-dimensional space. Such a scalability issue limits its application to larger multi-agent systems.

Uncertainty exists widely in the estimation of task requirements and agent capabilities. Some recent works capture such uncertainty explicitly in their models. However, the metrics they optimize do not necessarily result in a high-probability task requirement satisfaction (which is considered robust).
For example, the work of \cite{faruq2018simultaneous} models the agent action uncertainty in Markov decision processes (MDP) and optimizes the expected objective.
In \cite{liemhetcharat2011modeling, liemhetcharat2012weighted, ravichandar2020strata}, an agent's capabilities are represented as a Gaussian random vector where the uncertainty is captured in the distribution. The work of \cite{ravichandar2020strata} then penalizes the variance of the assigned agent capabilities to limit the uncertainty.
However, expectation or variance are not well-justified quantitative metrics in such an optimization problem
and can result in problematic solutions. For instance, if the required capability of a task is matched precisely, it is reasonable to limit the variances within a small threshold. However, if the team's capability surpasses the task requirement by a lot, larger variances are acceptable. As another example, optimizing an expected cost (average performance) can be problematic for safety-critical applications.
In \cite{rudolph2021desperate}, the authors directly maximize the probability that enough capabilities are assigned. However, probability is non-convex in the decision variables, and global optima are hard to obtain for large problem cases.


\begin{figure}[t]
    \centering
    \includegraphics[width=0.7\linewidth]{figure/cvar_definition.pdf}
    \caption{Graphical illustration of CVaR, defined as expected cost of the worst \(\beta\)-proportion of the cases. Denoted as \(\eta_\beta(\cdot)\), it is a function of the random distribution with \(\beta\) as a hyper-parameter.}
    \label{fig:cvar_definition}
\end{figure}

Our proposed model considers a generalizable framework for allocating agent capabilities for complex tasks under capability and requirements uncertainty, and it can be applied to systems consisting of hundreds of agents.
We represent the agent capabilities as a vector of random distributions (not necessarily Gaussian) and the task requirements within a function of the team capabilities (defined in Sec. \ref{sec:problem_model}).
The task requirement is verified once the aggregated capabilities of the team drive the binary function to one.
We then solve the task decomposition, assignment, and scheduling problem simultaneously, where we optimize the time, energy, and robustness of task allocation.
As mentioned in the introduction, a robust allocation should ensure the capability of the team exceeds the task requirement with a high probability.
We choose to minimize the conditional value at risk (CVaR) of (requirement \(-\) capability), which is consistent\footnote{According to \cite{sarykalin2008value},  maximizing the probability is equivalent to minimizing the Value at Risk, and the Conditional Value at Risk is a risk metric with additional math properties that facilitate optimization problems.}
with maximizing the probability \(P(\textnormal{capability} \geq \textnormal{requirement})\) \cite{sarykalin2008value}.
CVaR is a provably sensible measurement of the uncertainties in practical applications \cite{majumdar2020should}. It is widely accepted by the finance community and appears with a growing frequency in recent robotic applications for risk-aware single-robot control \cite{balasubramanian2020risk, bernhard2019addressing, lindemann2020control, hakobyan2019risk} and multi-robot coordination \cite{zhou2018approximation, zhou2021multi}. The definition of CVaR is illustrated in Fig. \ref{fig:cvar_definition}.

\begin{table}[t]
  \centering
  \caption{A comparison between STRATA and \modelrisk{}.}
  \label{tab:strata_vs_ours}
    \begin{tabular}{l|cc}
    \toprule
          & STRATA\cite{ravichandar2020strata} & \modelrisk{} (Ours) \\
    \midrule
    Agent capabilities & continuous/Gaussian & continuous/stochastic \\
    Capability types & cumulative & cumulative/noncumulative \\
    Task requirements & and   & and/or \\
    Scheduling & no    & yes \\
    Uncertainty control & limit variance & minimize CVaR \\
    Optimization type & continuous nonlinear & mixed-integer nonlinear \\
    \bottomrule
    \end{tabular}%
\end{table}%

Our problem can be considered a coalition formation with spatial and temporal constraints. Multiple models have been proposed in previous work \cite{koes2005heterogeneous, ramchurn2010coalition, vig2006multi}. The work of \cite{ramchurn2010coalition} has a similar application to ours, but it assumes that there exists a predefined utility function that outputs a value for a team configuration. While an abstract function generalizes the problem space, explicitly defining the function values for all team configurations could be hard in practice. We provide our task requirement and agent capability model to avoid such an explicit definition.
The work of \cite{neville2021interleaved} and its extension \cite{messing2022grstaps} use a similar requirement and capability model. However, they do not consider the uncertainty in their capability model. A graph search-based algorithm is applied in their task allocation process, limiting the scalability of the number of agents (the size of the graph).

STRATA \cite{ravichandar2020strata, prorok2017impact} shares many similarities with our proposed approach, \modelrisk, including stochastic agent capability vectors and task requirements on the team's capabilities. Therefore, we choose STRATA as our baseline algorithm in one of the experiments in Sec. \ref{sec:experiments}.
However, STRATA falls in the area of CD-[ST-MR-IA], where it assumes the tasks happen at the same time, which simplifies the scheduling problem.
Meanwhile, our task model, represented using requirement functions, is more expressive. TABLE \ref{tab:strata_vs_ours} provides a more detailed comparison between the two models.
The meaning of `and/or' in the table is defined in Sec. \ref{sec:problem_model}. Note that the definition of `noncumulative' in STRATA differs from ours. STRATA claims that it can deal with noncumulative capability types (such as the speed of an agent). However, it thresholds on a value and then treats the binary value after thresholding as a cumulative capability. For noncumulative types, we define the team's capability to be the minimum of all agents and require instead that all agents in the team meet the minimum task requirement. STRATA is not able to enforce such requirements for their noncumulative capabilities.
