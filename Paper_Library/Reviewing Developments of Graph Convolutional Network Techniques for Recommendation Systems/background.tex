\section{Backgrounds}

\subsection{Recommendation Systems}

Recommendation systems have undergone a remarkable evolution, transforming from early collaborative filtering and content-based filtering methods to the sophisticated algorithms that shape digital experiences today. 
Collaborative filtering, whether user-based or item-based, established itself by leveraging user interactions and similarities to predict preferences. User-based collaborative filtering identifies users with similar tastes, recommending items liked by these users to the target user. 
In contrast, item-based collaborative filtering recommends items based on the similarity between items themselves. 
These approaches laid the groundwork for hybrid methods, which amalgamate collaborative and content-based filtering to harness the strengths of both, effectively mitigating challenges like the cold start problem and data sparsity.

The collaborative filtering paradigm is built on the assumption that users with similar preferences in the past will continue to share similar tastes in the future. 
However, these methods face challenges when dealing with sparse and incomplete data, a common issue in real-world scenarios. As a response to these challenges, the field has seen a paradigm shift with the integration of advanced technologies, such as matrix factorization~\cite{koren2009mf} or factorization machine~\cite{rendle2010factorization} and deep learning, to enhance the accuracy and efficiency of recommendation systems.
Particularly in industrial applications, where adherence to real-world system engineering demands is essential, recommendation systems are consistently divided into three distinct stages: \textit{matching}, \textit{ranking}, and \textit{re-ranking}. 

In the initial "Matching" stage, the objective is to efficiently select candidate items from an extensive pool, often numbering in the millions or billions, significantly reducing the overall scale. Due to the large volume of data and strict latency constraints in online serving, the utilization of complex algorithms like very deep neural networks is impractical, necessitating the use of concise models. Real-world recommender systems frequently integrate multiple matching channels, each employing distinct models such as embedding matching, geographical matching, popularity matching, and social matching \cite{covington2016deep,kang2019candidate}.

Following the "Matching" stage, in the "Ranking" stage, candidate items from various channels are consolidated into a unified list and scored by a singular ranking model. This model ranks items based on scores, enabling the incorporation of more sophisticated algorithms and considerations of rich features such as user profiles and item attributes to enhance recommendation accuracy \cite{kang2018self,song2019autoint,lian2018xdeepfm,chen2022learning}. The primary challenge in this stage lies in designing models capable of capturing intricate feature interactions.

The "Re-ranking" stage is necessary after the "Ranking" stage to address additional criteria like freshness, diversity, and fairness \cite{pei2019personalized}. This stage may involve the removal or reordering of items to align with business needs. The primary concern is considering various relationships among the top-scored items, as the proximity of similar or substitutable items can lead to information redundancy, necessitating adjustments in the displayed order \cite{ai2018learning,zhuang2018globally}.

Recommendation settings encompass a diverse array of techniques and methodologies aimed at enhancing the precision and relevance of content suggestions in various domains. Among these, three key aspects stand out: \textit{user-item collaborative filtering}, \textit{user-user social regularization}, and \textit{item-item side information supplementation}.
Specifically, in user-item collaborative filtering, the primary focus is on leveraging the collective preferences and behaviors of users to make personalized recommendations~\cite{he2017neural,DBLP:conf/sigir/Wang0WFC19,sedhain2015autorec,guo2020fastif,kang2018self}. This method establishes connections between users and items, allowing the system to infer user preferences based on historical interactions. The collaborative filtering paradigm can be further categorized into user-based and item-based approaches. User-based collaborative filtering identifies users with similar preferences, recommending items liked by those with analogous tastes. Item-based collaborative filtering, on the other hand, recommends items based on their similarity to items previously favored by the user. These collaborative approaches are foundational to many recommendation systems, providing a robust framework for generating accurate and contextually relevant suggestions.

User-user social regularization~\cite{cialdini2004social,mcpherson2001birds} introduces a social dimension to the recommendation setting by incorporating social network information. 
Beyond individual preferences, this approach considers the relationships and interactions between users within a social network. 
Users are connected based on social ties, and the recommendation algorithm takes into account the preferences and activities of users within one's social circle. 
By integrating social regularization, the system aims to improve the accuracy of recommendations by considering the influence of friends or connections on an individual's preferences. 
This adds a layer of personalization that extends beyond direct user-item interactions.

As for item-item side information supplementation, using knowledge graphs into recommendation systems broadens the recommendation landscape by incorporating additional information about items beyond user interactions~\cite{DBLP:conf/sigir/YangHuang22,wang2019kgat,wang2019knowledge,chen2021attentive,ckan,kgat}. This supplementary information can include textual descriptions, categorical tags, or any other relevant metadata associated with items. By considering these side information attributes, the recommendation system gains a more comprehensive understanding of item characteristics. This enrichment facilitates a more nuanced matching of user preferences with item features, leading to improved recommendation accuracy. The incorporation of side information becomes particularly valuable in scenarios where the content of items plays a significant role in user choices.




\subsection{Graph Convolution Networks}

A notable advancement in recent years is the incorporation of graph convolutional networks (GCN)~\cite{kipf2017semi,velivckovic2017gat,berg2018gcmc,zhang2019heterogeneous,fout2017protein,feng2019hypergraph} into recommendation systems. 
GCNs excel in modeling complex relationships within graph-structured data, making them well-suited for capturing intricate user-item interactions. Motivated by the high-order connectivity, structural properties, and enhanced supervision signals offered by GCNs~\cite{zhou2020graph,wu2020comprehensive}., researchers have explored their application in recommendation scenarios. 
The utilization of GCNs introduces a more comprehensive understanding of user preferences, considering not only the intrinsic qualities of items but also the relationships and interactions within the user-item graph.

When dealing with regular Euclidean data such as images or texts, Convolutional Neural Networks (CNNs) prove highly effective in extracting localized features. However, their application to non-Euclidean data, like graphs, necessitates a degree of generalization to address situations where the objects of operation (such as pixels in images or nodes in graphs) lack a fixed size. In the context of Graph Representation Learning (GRL), the primary objective is to generate low-dimensional vectors representing graph nodes, edges, or subgraphs, capturing the intricate connection structures within graphs. For instance, pioneering works such as DeepWalk~\cite{perozzi2014deepwalk} employ the SkipGram~\cite{mikolov2013efficient} approach on randomly generated paths through graph walks, aiming to learn node representations. This technique forms the foundation for graph-based neural networks.

The synergy of CNNs and GRL has led to the development of various Graph Convolutional Networks (GCNs), designed to distill structural information and derive high-level representations from graph data. These networks integrate the strength of CNNs in extracting localized features with the graph-aware representation learning provided by GRL. By combining these methodologies, GCNs can effectively handle non-Euclidean data structures and capture the relational intricacies present in graphs. This integration has proven particularly valuable in tasks where understanding the structural dependencies within graph data is essential, such as social network analysis, molecular chemistry, or the topic of this paper recommendation systems~\cite{zhou2020graph,wu2020comprehensive,zhang20explainable,chen2023wsfe}.

A notable example illustrating this integration is the use of DeepWalk, where the learned node representations from random walks on graphs are further refined using CNNs. This fusion allows the model to discern complex patterns and dependencies within graph structures, enhancing its ability to generalize and make accurate predictions. As the field of graph-based neural networks continues to evolve, the amalgamation of CNNs and GRL stands as a potent approach for effectively processing and understanding non-Euclidean data, offering promising avenues for advancements in various application domains.

The success of GCN-based recommenders can be attributed to three key perspectives: \textit{structural data}, \textit{high-order connectivity}, and \textit{supervision signal}.

Data derived from online platforms manifests in diverse forms, including user-item interactions, user profiles, and item attributes. 
Traditional recommender systems often struggle to harness this multi-modal data efficiently, typically focusing on specific sources and thereby overlooking valuable information. 
GCNs provide a unified approach by representing all data as nodes and edges on a graph. 
This not only enables the utilization of various data forms but also empowers GCNs to generate high-quality embeddings for users, items, and other features, crucial for optimizing recommendation performance.

The accuracy of recommendations hinges on capturing the similarity between users and items, with this similarity reflected in the learned embedding space. Specifically, the embedding for a user should align with the embeddings of items the user has interacted with, as well as those interacted with by users with similar preferences (the collaborative filtering effect). 
Traditional approaches often fall short in explicitly capturing this effect, primarily considering only first-order connectivity based on directly connected items. 
The collaborative filtering effect is naturally expressed as multi-hop neighbors on the graph, seamlessly integrated into learned representations through embedding propagation and aggregation.

Supervision signals, typically sparse in collected data, pose a challenge for recommender systems. 
GCN-based models address this by leveraging semi-supervised signals during the representation learning process. For instance, in an E-Commerce platform, the target behavior like purchases may be sparse compared to other behaviors. 
GCN-based models effectively incorporate multiple non-target behaviors, such as searches and adding to carts, by encoding semi-supervised signals over the graph, leading to significant improvements in recommendation performance \cite{jin2020multi}. 

In summary, GCN-based recommenders have redefined the landscape of recommendation systems by capitalizing on the richness of structural data, embracing high-order connectivity, and strategically leveraging supervision signals. These advancements contribute to their remarkable performance across diverse recommendation scenarios and pave the way for more effective and personalized content suggestions.