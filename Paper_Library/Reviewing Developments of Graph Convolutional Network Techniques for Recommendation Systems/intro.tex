\section{Introduction}


recommendation systems play a pivotal role in today's digital landscape, facilitating personalized content delivery and enhancing user experiences across various platforms. 
These systems analyze user preferences and behavior to predict and recommend items, such as movies, products, or articles, that are likely to be of interest. With the rapid development of recent computer science and artificial intelligence techniques, recommendation systems recieve great attention in recent years~\cite{DBLP:conf/sigir/HeDeng20,DBLP:journals/computer/KorenBV09,DBLP:conf/sigir/Wang0WFC19,koren2009matrix}.


Traditional recommendation systems have relied on methods like collaborative filtering~\cite{rendle2009bpr,koren2009mf,rendle2010factorization} that leverages the preferences and behavior of users to generate recommendations. 
Collaborative filtering relies on user interactions and similarities between users to make predictions about their preferences for items.
One of the advantages of collaborative filtering is its ability to make recommendations without requiring explicit information about the items themselves. Instead, it relies on the collective behavior and preferences of users. However, collaborative filtering does face challenges, such as the cold start problem (difficulty recommending items for new users or items with limited interactions) and the sparsity of user-item interaction data.
In recent years, collaborative filtering techniques have been combined with other approaches, including advanced machine learning techniques such as matrix factorization (MF)~\cite{koren2009mf} or factorization machine~\cite{rendle2010factorization} and deep learning such as neural networks~\cite{he2017neural,guo2017deepfm,cheng2016wide,ying2018graph}, to enhance the accuracy and effectiveness of recommender systems. 


In recent years, there has been a paradigm shift with the emergence of graph neural networks (GNNs) as a state-of-the-art approach.
The integration of collaborative filtering with graph neural networks (GNNs)~\cite{hamilton2017graphsage,kipf2017semi} is an example of how cutting-edge technologies are being applied to address the complexities of recommendation scenarios, providing more personalized and accurate suggestions to users.
The motivation behind incorporating GNNs into recommendation systems lies in their ability to exploit the inherent graph structure of user-item interactions. 
GNNs leverage the relationships between users and items to provide more accurate and context-aware recommendations~\cite{wu2022graph}. 
This is particularly beneficial in scenarios where user preferences are influenced not only by the inherent qualities of items but also by the relationships and interactions within the user-item graph.

One of the key advantages of using GNNs in recommendation systems is their capacity to handle sparse and incomplete data~\cite{wu2022graph}. Traditional collaborative filtering methods may struggle when dealing with missing or incomplete user-item interactions. GNNs, on the other hand, can effectively propagate information through the graph, inferring potential connections and preferences even in situations where explicit interactions are limited.
Despite the promising advancements facilitated by GNNs, challenges persist in the areas of graph construction, embedding propagation/aggregation, model optimization, and computation efficiency. Addressing these challenges is crucial to unlocking the full potential of GNNs in recommendation systems. 
In this paper, we introduce these contents with careful discussions.