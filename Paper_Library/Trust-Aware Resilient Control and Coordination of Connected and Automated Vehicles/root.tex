 %%%%%%%%%%%%%%%%%%%%%%%%%%%%%%%%%%%%%%%%%%%%%%%%%%%%%%%%%%%%%%%%%%%%%%%%%%%%%%%%
%2345678901234567890123456789012345678901234567890123456789012345678901234567890j
%        1         2         3         4         5         6         7         8

\documentclass[letterpaper, 10 pt, conference]{ieeeconf}  % Comment this line out if you need a4paper
\usepackage{cite}
\usepackage{amssymb,amsfonts,amsmath}
\usepackage{array}
\usepackage{bm}
\usepackage{textcomp} 
\usepackage{xcolor}
\usepackage{csquotes}
\usepackage{color}
\usepackage{comment}
\usepackage{mdwmath}
\usepackage{centernot}
\usepackage{mdwtab}
\usepackage{ifpdf}
\usepackage{eqparbox}
\usepackage{bbold}
\usepackage[font=footnotesize]{subfig}
\usepackage{fixltx2e}
\usepackage{stfloats}
\DeclareMathOperator*{\argmin}{argmin}   
\DeclareMathOperator*{\argmax}{argmax}
\usepackage{hyperref}
\usepackage[pdftex]{graphicx}
\usepackage[linesnumbered,commentsnumbered,ruled,vlined]{algorithm2e}
\usepackage{url}
\usepackage{float}
\usepackage{lipsum}
\usepackage{algpseudocode}
\usepackage{listings}
\usepackage{multirow}
\newtheorem{assumption}{\rm\textbf{Assumption}}
\newtheorem{problem}{\textbf{Problem}}
\newtheorem{definition}{\rm\textbf{Definition}}
\newtheorem{theorem}{\rm\textbf{Theorem}}
\newtheorem{lemma}{\rm\textbf{Lemma}}
\newtheorem{corollary}{\rm\textbf{Corollary}}
\newtheorem{Proposition}{\rm\textbf{Proposition}}
\newtheorem{Remark}{\textbf{Remark}}
\newtheorem{remark}{\rm\textbf{Remark}}
\newcommand{\Break}{\State \textbf{break} }
%\documentclass[a4paper, 10pt, conference]{ieeeconf}      % Use this line for a4 paper

\IEEEoverridecommandlockouts                              % This command is only needed if 
                                                          % you want to use the \thanks command

\overrideIEEEmargins                                      % Needed to meet printer requirements.

%In case you encounter the following error:
%Error 1010 The PDF file may be corrupt (unable to open PDF file) OR
%Error 1000 An error occurred while parsing a contents stream. Unable to analyze the PDF file.
%This is a known problem with pdfLaTeX conversion filter. The file cannot be opened with acrobat reader
%Please use one of the alternatives below to circumvent this error by uncommenting one or the other
%\pdfobjcompresslevel=0
%\pdfminorversion=4

% See the \addtolength command later in the file to balance the column lengths
% on the last page of the document

% The following packages can be found on http:\\www.ctan.org
%\usepackage{graphics} % for pdf, bitmapped graphics files
%\usepackage{epsfig} % for postscript graphics files
%\usepackage{mathptmx} % assumes new font selection scheme installed
%\usepackage{times} % assumes new font selection scheme installed
%\usepackage{amsmath} % assumes amsmath package installed
%\usepackage{amssymb}  % assumes amsmath package installed

\newcommand{\li}[1]{{\color{magenta}{Li: #1}}}
\newcommand{\sabbir}[1]{{\color{blue}{Sabbir: #1}}}

\title{\LARGE \bf
Trust-Aware Resilient Control and Coordination\\ of Connected and Automated Vehicles %\li{My preference: Trust-Aware Resilient Control and Coordination of Connected and Automated Vehicles}
}


\author{H.M. Sabbir Ahmad$^{1}$, Ehsan Sabouni$^{1}$, Wei Xiao$^2$, Christos G. Cassandras$^1$ and Wenchao Li$^1$% <-this % stops a space
%\thanks{$^*$These authors contributed equally to this paper.}
\thanks{$^1$Division of Systems Engineering and Department of Electrical \& Computer Engineering, Boston University, Boston, MA, USA, \texttt{{\small \{sabbir92, esabouni, cgc, wenchao\}@bu.edu}}}
\thanks{$^2$Computer Science \& Artificial Intelligence Laboratory, Massachusetts Institute of Technology, Cambridge, MA, USA \texttt{{\small \{weixy@mit.edu\} }}}
\thanks{This work was supported in part by NSF under grants ECCS-1931600,
DMS-1664644, CNS-1645681, CNS-2149511, by AFOSR under grant FA9550-19-1-0158,
by ARPA-E under grant DE-AR0001282, by the MathWorks and by NPRP grant
(12S-0228-190177) from the Qatar National Research Fund, a member of
the Qatar Foundation (the statements made herein are solely the responsibility
of the authors).}
}%

\newif\ifarvix
\arvixfalse

\newif\ifread  
\readfalse

\newif\ifITSC
\ITSCtrue

\begin{document}
\lstset{language=C}
\maketitle
\thispagestyle{empty}
\pagestyle{empty}

%%%%%%%%%%%%%%%%%%%%%%%%%%%%%%%%%%%%%%%%%%%%%%%%%%%%%%%%%%%%%%%%%%%%%%%%%%%%%%%%
\begin{abstract}
We address the security of a network of Connected and Automated Vehicles (CAVs) cooperating to navigate through a conflict area.  Adversarial attacks such as Sybil attacks can cause safety violations resulting in collisions and traffic jams.  In addition, uncooperative (but not necessarily adversarial) CAVs can also induce similar adversarial effects on the traffic network.  We propose a decentralized resilient control and coordination scheme that mitigates the effects of adversarial attacks and uncooperative CAVs by utilizing a trust framework.  Our trust-aware scheme can guarantee safe collision free coordination and mitigate traffic jams. Simulation results validate the theoretical guarantee of our proposed scheme, and demonstrate that it can effectively mitigate adversarial effects across different traffic scenarios.
\end{abstract}

\section{INTRODUCTION}
%{\color{red}cut some citations, this is too much for a conference paper.} 
The rise of connected and automated vehicles (CAVs) and advancements in traffic infrastructure \cite{li2013survey} promise to offer solutions to transportation issues like accidents, congestion, energy consumption, and pollution \cite{Schrank20152015UM,kavalchuk2020performance}. To achieve these benefits, efficient traffic management is crucial, particularly at bottleneck locations such as intersections, roundabouts, and merging roadways \cite{VANDENBERG201643}.

Thus far, two approaches, centralized \cite{Liu_01} and decentralized \cite{Xiao_03}, have been proposed for controlling and coordinating CAVs at conflict points. %In decentralized methods, as opposed to centralized ones, each CAV is responsible for its own onboard computation with information from other vehicles limited to a set of neighbors \cite{7313484}. 
There has been extensive research on cybersecurity of CAVs summarized in \cite{Shukla_01,Sun_01,Pham_01}. The attacks can be categorized into in-vehicle network attacks and attacks on (V2V or V2X) communication networks \cite{Sun_01}. A significant amount of research has been done from a control point of view with the aim of designing smart and efficient coordination algorithms for real-world implementation. However, security for this next generation of CAV algorithms has received virtually no attention, with only \cite{Jarouf_01, Zhao_01} tackling security for merging roadways, and our previous work \cite{Ahmad_01} providing an extensive study of security threats to this class of algorithms for various conflict areas. 

There is literature that considers cyberattacks on connected vehicles and investigates their effects on intersections \cite{Huang_01,Chen_01} and freeway \cite{Reilly_01} control systems; however, the fundamental difference is that they do not consider the security of cooperative control of CAVs. One class of cooperative algorithms for autonomous vehicles whose security has been extensively studied \cite{Biroon_02,Boddupalli_01,Farivar_01} is Cooperative Adaptive Cruise Control (CACC).

%with \cite{Biroon_02} addressing Sybil/impersonation attacks, \cite{Boddupalli_01} investigated mutation, denial of delivery, masquerading and false data injection (FDI) attacks, \cite{Alipour_01} looked at signal jamming attacks, \cite{Merco_01} focusing on  replay attacks, and \cite{Farivar_01} looked at stealthy/covert attacks. 

%\sabbir{sybil attack and no cooperation, why cant be extended, Stephanie gil, provide digital signatures with timestamp to vehiclenodes within their coverage periodically or on demand, In addition,the independency and feasibility of our algorithm aremore robust than the existing solutions that rely oncollaboration of neighboring nodes., replay, fdi, jamming}

An idea that has been extensively applied to multi-agent systems is the notion of trust/reputation \cite{Chen_01, Cheng_02, Hu_01}. 
%using subjective logic, \cite{Hu_01,  } had adopted Beta reputation systems, and \cite{Josang_01} applying Dirichlet reputation systems. 
%However, these works do not address the security of the coordination algorithms for CAVs.
A novel CBF-based trust metric was introduced in \cite{Parwana_01} for multi-robot systems (MRS) for providing safe control against adversarial agents; however, it cannot be directly applied to our application. The authors in \cite{Garlichs_01} used a trust framework to address the security of CACC. Lastly, the authors in \cite{Shoukry_01} used a trust framework based on a macroscopic model of the network to tackle Sybil attacks for traffic intersections without analyzing the fidelity of the model and commenting about the classification accuracy of their proposed method. 
%no guarantees against false positives (FP) 
%(i.e. detecting all fake vehicles accurately and not detecting any real vehicle as fake).
%however, their proposed framework has several shortcomings mainly, (i) they provide no analysis on the fidelity of the model used for estimating traffic density to detect fake CAVs, and (ii) they provide no guarantees on not having any false positives (i.e. detecting all fake vehicles accurately and not detecting any real vehicle as fake).  


In this paper, we present distributed resilient control and coordination scheme for CAVs at conflict areas that is resilient to adversarial agents and uncooperative CAVs. We use Sybil attacks to validate our proposed scheme as they can be used to achieve both adversarial objectives. Sybil attacks can't be tackled using existing road infrastructure including namely sensors and cameras as they are placed sparsely in the network, and, their reliability degrades with age \cite{Shoukry_01}. %Due to cost and maintenance constraints, these legacy sensors and cameras can not instrument the whole traffic network, and hence they are placed many kilometers apart from each other. Moreover, due to aging, several of these sensors are not reliable (report very noisy measurements) and cannot be exclusively used to provide accurate estimates of traffic conditions to regulate the traffic network and achieve its highest performance. Reilly et al.in [26] reports a thorough study on the vulnerability of these legacy sensors. It is shown that these sensors have two main vulnerabilities namely (i) physical access to the sensor hardware (copper theft, wire clipping) and (ii) communication links between the sensor and the rest of the infrastructure. 
The key contributions of the paper are as follows. %{\color{red}Contributions are too long for each of them. Try to describe them in simple words.}: 

\begin{enumerate}
    \item  We propose trust-aware resilient control and coordination that guarantees safe coordination against adversarial attacks and uncooperative CAVs. It is important to add that, our proposed framework is agnostic to the specific implementation of the trust framework. 
    %to propose resilient coordination and control of CAVs in conflict areas against Sybil attacks, by identifying key adversarial objectives  (from our previous study in \cite{Ahmad_01}), which are: i. safety violation, and ii. traffic holdup at different roads in the network. Our proposed framework can guarantee safety and mitigate traffic holdups in the presence of such attacks.
    %we identify the various pitfalls such agents can introduce in the network of the attacker through Sybil attack and propose objective-specific resilience to tackle them through our cooperative control framework. We identified two key pitfalls which are: i. safety violation, and ii. disruption of normal traffic flow causing traffic buildup at different points in the network. We introduce multi-objective resilience through the traffic infrastructure and the control of individual agents in the network. By doing so, our proposed resilience scheme can guarantee safe coordination in the face of %Sybil/spoofing attacks. adversarial agents. In addition, our proposed framework can alleviate the effects of CAVs that may deliberately attempt to cause traffic holdups in the network. 
    %It is important to add that, our proposed framework is agnostic to the specific implementation of trust framework, provided the adopted framework can accurately encapsulate the normal behavior of the CAVs in the network based on our presented specifications, and accordingly adjust the trust scores of the CAVs.

    \item We provide resilient coordination using a \emph{robust event-driven scheduling scheme} that can successfully alleviate traffic holdups due to adversarial attacks and uncooperative CAVs. %{\color{red} This does not look like a contribution (At least I do not feel strong). You should frame it as what is new to the community.}

    \item We present simulation results that validate our proposed resilient control and coordination scheme guarantees safety; and our robust scheduling scheme besides mitigating traffic jams also improves the travel time and fuel economy of real cooperative CAVs in the presence of adversarial attacks and uncooperative CAVs. 
    
    %\item  We propose an attack detection and mitigation scheme against Sybil attacks based on the trust score of the CAVs, which can alleviate the effect of the attack by restoring normal coordination. Our proposed scheme guarantees safety against false positive (FP) cases, which may arise due to a poor choice (or, design) of trust framework.
    %we propose an attack detection and mitigation scheme against Sybil attacks based on the trust score of the CAVs in the network. As our framework is independent of the choice of the trust framework, hence certain adopted framework may result in false positive (FP) cases. Our proposed scheme mitigation scheme alleviates the effects of fake CAVs from the network, and guarantees safety to false positive CAVs. 
    %Our incorporated trust framework monitors the conformance of individual agents to the control and coordination rules whilst in the CZ
    %and accordingly adjusts their trust values which paves means to make the intersection secure against cyber attacks. In this work we look specifically at Sybil attack and propose means to detect and mitigate the effect of Sybil attack from the intersection.
\end{enumerate}
%Sybil/spoofing attacks by proposing attack detection, mitigation and

Our proposed scheme is computationally tractable, minimally invasive, and can be readily incorporated into the existing intelligent traffic infrastructure like intersections, roundabouts, merging roadways, etc. without extensive overhaul. %Finally, we present simulation results to illustrate the proposed merits of our work.
The paper is organized in seven sections. We present the background materials and the threat models in sections \ref{problem_formulation} and \ref{threat_model} respectively. In section \ref{trust_framework}, we present the trust framework for a cooperative network of CAVs in conflict areas. Our proposed resilient control and coordination scheme is presented in section \ref{resilient_control_formulation}. The results from our simulations have been included in section \ref{Results} which is followed by the conclusion in section \ref{conclusion}.


%\begin{table}[]
%\begin{tabular}{ll}
%Symbols               & Defintions                                  \\
%$L$                     & Length                                      \\
%$t$                     & Time                                        \\
%$S(t)$                & Set of CAV indices in the coordinator table \\
%$N(t)$                & Number of cars at time t                    \\
%$\boldsymbol{x}$                & States of the CAV                   \\
%$x$                     & Position                                    \\
%$v$                     & Velocity                                    \\
%$u$                     & Acceleration                                \\
%$r$                     & Vision radius                               \\
%$\theta$                & Vision angle                                \\
%$v_{\min}$              & Minimum velocity                            \\
%$v_{\max}$               & Maximum velocity                            \\
%$u_{\min}$               & Minimum Acceleration                        \\
%$u_{\max}$               & Maximum Acceleration                        \\
%$t^0_i$               & CAV $i$ arrival time                        \\
%$t^f_i$               & CAV $i$ exit time                           \\
%$i_p$                 & Preceding CAV for CAV $i$                   \\
%$\textit{z}_{i,j}$   & The distance between CAV $i$ and $j$          \\
%$\varphi$             &   Reaction time                                          \\ 
%$\Delta$                     & Minimum safe distance                                             \\
%$m$                      & The index of the merging point                                            \\
%$i_m$                     & Conflicting CAV for CAV $i$ at merging point $m$                                            \\
%$t_i^m$                      &  Arrival time at merging point $m$                                           \\
% $n_i$                     &  Number of merging point CAV %$i$ traverses                                          \\
%$\boldsymbol{s}$                & States of the fake CAV    \\    
%$S_x(t)$              & The indices of normal CAVs                                 \\    
%$S_{s}(t)$    & The indices of fake CAVs                                 \\   
%$\tau_i$              &  Trust values for CAV $i$                   \\     
%$S_{i}^{o}$             &   set of CAVs that can observe CAV $i$      \\     
%$S^{\hat{o}}_{i}$       &   set of CAVs that reports CAV $i$          \\
%$B_c$                      & set of behavioral specifications                                            %\\     
%$r_{i,j}$             & Positive evidence for CAV $i$ %corresponds to specification $j$   \\     
%$p_{i,j}$             & Negative evidence for CAV $i$ corresponds to specification $j$   \\    
%$R_i(t)$                      & Cumulative positive evidences                                             \\     
%$P_i(t)$                      & Cumulative negative evidences                                             \\     
%$\zeta_l$                      & lane priority of lane $l$                                             \\     
%$h_i$                      &  non-informative prior weight                                           \\     
%$B^a_{c}$                       &  Specification involved other agents                                          \\   
%$S^a_i$                       &  Agents involved in checks of CAV $i$                                         %\\   
%$\delta$                      &  Trust threshold          %                                 \\     
%$n$                    &    Number of constraints         %                                \\                          
%$b_q(.)$                      & Control barrier function for constraint $q$                                            \\     
%$L_f$                      &  Lie derivative with respect to $f$                                            \\     
%$\kappa $                     &  Extended class k function for CBF corresponds to constraint $q$                                         \\     
%$v_i^{ref}$           &       Reference velocity for CAV $i$   \\     
%$V(.)$                      &  CLF function                                           \\     
%$e_i$                      &   Soft Constraint for CLF of CAV $i$                                          \\     
%$\alpha$                      &  Time weight in optimization                                           \\     
%$\beta$                      &  Adjustable weight to penalize time                                          \\ 
%$u_i^{ref}$                      & Reference acceleration for CAV $i$                                            \\     
%$\lambda$                      &  Weight factor for relaxation variable\\     
%$dt$                      & sampling time                                            \\     
%$S_{i,\cal{M}}$                      & indices of all conflicting CAVs at merging point $j$                                             \\     
%$S_{i}^{M}$                      & indices of all conflicting CAVs at all MPs                                            \\     
%$S_{i}^p$                      & indices of all preceding CAVs for CAV $i$                                             \\     
%$\hat{x}_{i,j}$                      & Estimated position of CAV $j$ by CAV $i$ through vision                                   \\ 
%$\w_x$                                    & Noise on position                                            \\
%$\w_v$                                    & Noise on velocity                                            \\
%$\epsilon$                                &  bound noise                                           \\
%$\tau_{low}$                                     & Trust delta threshold                                            \\
%$ \Gamma$                                     & Trust threshold                                            \\
%$\gamma$                                     &   Time related discount factor                                          \\
%$\omega$                                     &  Percentage of new CAVs with low trust                                           \\  
%$S_R(t)$                                     & Indices of CAVs with decrescent trust value below thershold                                                           \\
%$R$                                          &  New indices of CAVs after trust based rescheduling                                                            \\
%$a_i$                                        &  Index of CAV $i$ in the queue table                                                         \\
%$v_{low}$                                             & Abnormally minimum velocity threshold                                                          \\
%$\Omega$                                             &  Minimum percentage for sequencing                                                         \\
 %                                            &                                                           \\
  %                                           &                                                           \\
%                                             &                                                           \\
%\end{tabular}
%\end{table}



\section{Background} 
\label{problem_formulation}
%{\color{red} This section is not about resilient control and coordination?} 
We present resilient control and coordination approach for secure coordination of CAVs in conflict areas, \textit{using the signal-free intersection presented in \cite{Xu_02} as an illustrative example}. Fig. \ref{fig:intersection} shows a typical intersection with multiple lanes. The Control Zone (CZ) is the area within the outer red circle. It contains eight entries labeled from $o_1$ to $o_8$ and lanes labeled from $l_1$ to $l_8$ %\li{mix of capitalized and non-capitalized symbols} 
each of length $L$ which is assumed to be the same here. Red dots show all the merging points (MPs)
%\li{acronyms introduced before?}
where potential collisions may occur. All the CAVs have the following possible planned trajectories when they enter the CZ: going straight, turning left from the leftmost lane, or turning right from the rightmost lane. 

%{\color{red}Any reason for excluding the objective function here?}

The vehicle dynamics for each CAV in the CZ take the following form:
\begin{equation}
\left[
\begin{array}
[c]{c}%
\dot{x}_{i}(t)\\
\dot{v}_{i}(t)
\end{array}
\right] =\left[
\begin{array}
[c]{c}%
v_{i}(t)\\
u_{i}(t)
\end{array}
\right], \label{VehicleDynamics}%
\end{equation}
where $x_{i}(t)$ is the distance from the origin at which CAV $i$ arrives, $v_{i}(t)$ and $u_{i}(t)$ denote the velocity and control input (acceleration/deceleration) of CAV $i$, respectively. We also consider that each CAV has a vision-based perception capability defined by a radius and angle tuple denoted as $(r,\theta)$, (where $r \in \mathbb{R}^+, \theta \in [0, 2\pi]$) 
%\geq \frac{v_{\max}^2}{2|u_{\min}|} + \varphi v_{\max} \text{ and } 
%$\theta \in [0,2\pi]$). %\li{why are we setting $r$ this way? what about $\theta$?},
Let $t_{i}^{0}$ and $t_{i}^{f}$ denote the time that CAV $i$ arrives at the origin and exits the CZ, respectively. The control is implemented in a \textit{decentralized manner} whereby each CAV $i$ determines a control policy to jointly minimize the travel time and energy consumption governed by the dynamics (\ref{VehicleDynamics}). Expressing energy through $\frac{1}{2}u_i^2(t)$ and normalizing travel time and energy, we use the weight $\alpha\in[0,1]$ to construct a convex combination as follows:

\begin{equation}\label{eqn:energyobja}
J_{i}(u_{i}(t),t_i^f):=%\underbrace
{\beta(t_{i}^{f}-t_{i}^{0})}%_{J_1,i}
+%\underbrace
{\int_{t_{i}^{0}}^{t_{i}^{f}}\frac{1}{2}u_{i}^{2}(t)dt}%_{J_2,i},
\end{equation} 
where $\beta:=\frac{\alpha\max\{u_{\max}^{2},u_{\min}^{2}\}}{2(1-\alpha)}$ is an adjustable weight to penalize
travel time relative to the energy cost of CAV $i$. 

%\noindent\textbf{FIFO Queuing and passing sequence}

%However, CAVs are allowed to change lanes in the CZ while they are in the "lane-changing zone",% In order to avoid potential collisions with newly arriving vehicles and to conform with the driving rules that vehicles are prohibited from changing lanes when they are very close to the intersection, these lane changes are only allowed in a "lane-changing zone", 
%i.e., an area between the two red lines shown in Fig. \ref{fig:intersection}. Due to lane-changing, apart from the fixed MPs in the intersection, some "floating" MPs may also appear in lane-changing zones which are not fixed in space. %Thus, there are two kinds of MPs: (i) the fixed MPs in the intersection and (ii) the floating MPs in the lane-changing zones.

%The lanes are labeled from $l_1$ to $l_8$ in a counterclockwise direction with corresponding origins $O_1$ to $O_8$. %The rightmost lanes in each direction only allow turning right or going straight, while the leftmost lanes only allow turning left or going straight. However, 
%Thus, some CAVs must change their lanes so as to execute a movement, e.g., left-turning CAV 6 in $l_8$ in Fig. \ref{fig:intersection}. Due to such lane-changing behavior, a new MP $M_{6,7}$ is generated since a conflict of CAV 6 with a CAV in $l_7$ may arise. Similarly, possible MPs may also appear in other lanes when vehicles perform lane-changing maneuvers, as the red dots $\left(M_{i, 1}, M_{i, 2}, \cdots\right.$, and $\left.M_{i, 8}\right)$ indicate in Fig. \ref{fig:intersection}. The coordinator (RSU) stores all individual CAV information and maintains the crossing sequence using FIFO queue.% However, other types of sequencing can also be used to solve this problem.

 %It is assumed that the motion trajectory of each CAV in the intersection is determined upon its entrance to the $\mathrm{CZ}$ (see grey lines in Fig. 1). Based on these trajectories, all MPs in the intersection are fixed, and can be easily determined. 
 
 %\li{start this paragraph with a sentence on the high-level control and coordination scheme, esp. clarifying what is centralized and what is decentralized; use paragraph headers to give the section more structures} 
A central Roadside unit (RSU) receives the state and control information $[x_i(t), v_i(t), u_i(t)]^T$ from CAVs through vehicle-to-infrastructure (V2X) communication and stores them in a table as shown in Fig. \ref{fig:intersection}. It is assumed that the coordinator knows the entry and exit lanes for each CAV upon their arrival and uses them to determine
%\li{trajectory can be a bit confusing here; perhaps qualify it as planned trajectory and then elaborate further on what it entails} 
the list of MPs from the set $\{M_1, \dots, M_{24}\}$ (shown in Fig. \ref{fig:intersection}) in its planned trajectory. It facilitates safe coordination by providing each CAV with relevant information about other CAVs in the network, that the CAV has to yield to while traveling through the CZ. It does so by assigning each CAV a unique index based on a passing sequence policy and, tabulates and stores the information of the CAVs according to the assigned indices as shown in Fig. \ref{fig:intersection}. Let $S(t)$ be the set of CAV indices in the coordinator queue table and $N(t) = |S(t)|$ be the total number of CAVs in the CZ at time $t$. %\li{say what $N(t)$ represents and how it relates to the FIFO ordering}. 
The default passing sequence is implemented using First In First Out (FIFO) policy which assigns $N(t)+1$ to a newly arrived CAV, and decrements the indices of all CAV with index greater than $i$ by 1, when CAV $i$ exits the CZ.

%Whilst in the CZ, each CAV has to stay safe to the CAV that is physically proceeding it immediately and from the CAV that will pass immediately before it at every MPs it will traverse through in the CZ. This is done by imposing constraints namely rear end constraint (for staying safe from physically preceding CAV) and merging constraint (for avoiding collision at the MPs). The coordinator identifies the constraints for every CAV based on crossing sequence policy used (e.g. FIFO queuing policy) by checking the queue table.  %In case of merging problem, the coordinator identifies the constraints for every CAV upon its arrival by checking the queue table. As the intersection contains multiple MPs hence the coordinator uses an extended queue table as shown in \cite{Xu_02} to identify all the constraints for a CAV upon its arrival by checking its origin and destination.

\begin{figure*}
\centering
\includegraphics[scale = 0.9]{Intersection_Figure.png}
\caption{The multi-lane intersection problem. Collisions may happen at the merging points. The table shows the order of the CAVs in the queue based on the FIFO sequencing scheme, trust-aware scheduling scheme, and lane-priority based scheduling scheme. %{\color{red} which two? make it clear}.
%The figure illustrates the proposed schemes using the example presented in this figure. 
%{\color{red}The caption may be too long for a figure.}\\
%\textbf{Resilience}: Since trust of CAV 3 is less than $ 1-\delta$, therefore the $i_p$ for CAV 4 $= \{2,3\}$
%{\color{red}I cannot follow the logic here. Why the trust is less than xx, then the $i_p$ is both 2 and 3? This does not depend on the value of $\delta$?}.\\ 
%{\color{red} It would be better to give these examples when talking about each of the terms in the paper instead of in the caption. Otherwise, it is very hard to follow.}
%\li{My preference: remove the table on the top-left. Use longer caption to explain the key points that we want to convey with this figure. The number of CAVs may still be too large for an illustrative example.}
%\sabbir{This will be modified shortly based on the our final contribution list, we are in the process of trimming some contributions from the paper}
%\textbf{Lane-priority based rescheduling}: As $v_8 \leq v_{low}$, lane-priority based rescheduling will result in CAV 14 and CAV 17 jumping ahead of CAV 8 in the queue (CAV 8 will turn left in the intersection, therefore, CAV 14 because of turning left from lane 3 and CAV 17 going straight from lane 3 may collide with CAV 8).\\
%\textbf{Trust based rescheduling}: At first CAV 5 is unconstrained but because it has low trust it will jump behind CAV 6 and CAV 7 which entered CZ after CAV 5.\\
%\textbf{Mitigation}: CAV 9 is a fake CAV which happens to be $i_p$ for CAV $10$. After mitigation gets triggered when $\tau_9 \leq 1- \delta$, CAV $10$ will relax its CBF and get closer to the CAV $9$ to possibly overtake it. Once, CAV $10$ gets close enough and its vision does not detect any car nearby it will overtake CAV 9. Same scenario happens for CAV $11$ and $CAV 12$ where all of them overtake CAV $9$ in the same way.\\
%\textbf{False Positive}: For CAV $5$, $\tau_5 \leq 1- \delta$ (i.e. false positive) therefore, mitigation will be triggered for it and CAV $6$ and CAV $7$ will as a result jump ahead of it since they are in different lanes but however, CAV $8$ will not overtake it since it can detect it through its vision, hence, it will stay behind CAV $5$ thus guaranteeing safety. CAV 6 and 7 (going straight) have common MPs as CAV 5 (turning left). Hence, during mitigation, they will jump ahead of CAV 5 as they are still located in the re-sequencing zone, and followingly CAV 5 will stay safe relative to them at the common MPs.
}
\label{fig:intersection}%
\end{figure*}

%When CAV $i \in S(t)$ is traveling in the roundabout, there are several important events whose times are used in our analysis: ( $i$ ) CAV $i$ enters the $\mathrm{CZ}$ at tme $t_i^0$,(ii) CAV $i$ arrives at MP $M_k$ at time $t_i^k, k \in\{1,2,3\}$, (iii) CAV $i$ leaves the $\mathrm{CZ}$ at time $t_i^f$. The coordinator uses these events to determine $i_p$ and $i_m$ for CAV $i$ to avoid any collision in CZ \cite{Xu_02}. %Compared to the single-lane merging or intersection control problems where the constraints are determined and fixed immediately when CAV $i$ enters the CZ, the main difficulty in a roundabout is that these constraints generally change after every event (defined earlier). In particular, for each CAV $i$ at time $t$ only the merging constraint related to the next MP ahead is considered. In other words, it is necessary to determine at most one $i_p$ and one $i_m$ at any time instant. There are two different ways proposed in \cite{Xu_02} for the problem: (i) Treat the system as a single CZ with three MPs with advance knowledge of each CAV’s sequence of MPs, or (ii) Decompose the roundabout into three separate merging problems corresponding to the three MPs, each with a separate CZ. The first approach requires the coordinator to maintain an extended queue table, whereas the second approach additionally requires local queue tables for each MP.

\subsection{\textbf{Constraints/Rules in the Control Zone}}
\label{rules}

The following section summarizes the rules that CAVs in the CZ must follow to navigate safely through the intersection.

\noindent{\bf Constraint 1} (Rear-End Safety Constraint): Let $i_{p}$ denote the index of the CAV which physically immediately precedes CAV $i$ in the CZ (if one is present). It is required that CAV $i$ conforms to the following constraint:
\begin{equation}
 x_{i_p}(t) - x_i(t) - \varphi v_{i}(t) - \Delta\geq 0,\text{ \ }\forall t\in[ t_{i}^{0},t_{i}^{f}] \label{Safety}%
\end{equation}
where $\varphi$ denotes the reaction time and $\Delta$ is a given minimum safe distance which depends on the length of these two CAVs.

\noindent{\bf Constraint 2} (Safe Merging Constraint): Every CAV $i$ should leave enough room for the CAV preceding it upon arriving at a MP, to avoid a lateral collision i.e.,
\begin{equation}
\label{SafeMerging}
x_{i_m}(t_{i}^{m}) - x_{i}(t_{i}^{m}) - \varphi v_{i}(t_{i}^{m}) - \Delta\geq 0,
\end{equation}
where $i_m$ is the index of the CAV that may collide with CAV $i$ at the merging points $m \in \mathcal{M}_i$ where $\mathcal{M}_i \subset \lbrace M_1,...,M_{24} \rbrace$, $\mathcal{M}_i$ is the set of MPs that CAV $i$ passes in the CZ, and $t_i^m$ is time of arrival of CAV $i$ at the MP. %{\color{red}explain $t_i^m$ as well.}

\noindent{\bf Constraint 3} (Vehicle Limitations): Finally, there are constraints on the speed and acceleration for each $i\in S(t)$:
\begin{equation}
\label{VehicleConstraints1}%
\begin{aligned} v_{\min} \leq v_i(t)\leq v_{\max}, \forall t\in[t_i^0,t_i^f]\end{aligned}
\end{equation}
\begin{equation}
\label{VehicleConstraints2}%
\begin{aligned} u_{{min}}\leq u_i(t)\leq u_{{max}}, \forall t\in[t_i^0,t_i^f] \end{aligned}
\end{equation}
where $v_{min} \geq 0$, $v_{max} > 0$ denote the minimum and maximum speed, and $u_{min} < 0$ and $u_{max}>0$ denote the minimum and maximum control respectively.

%\begin{remark}
%    The coordinator finds CAV $i_p$ and CAV $i_m$ for each CAV $i \in S(t)$ from their trajectory and communicates it to CAV $i$. The determination of CAV $i_m$ depends on the policy adopted for sequencing CAVs through the CZ. 
%\end{remark}

\section{Threat model}
\label{threat_model}
The adversarial effects of malicious attacks have been highlighted in our preliminary study in \cite{Ahmad_01}, namely, creating traffic jams across multiple roads due to the cooperative aspect of the control scheme, and in the worst case accidents, thus warrant making the control robust against these attacks.

\begin{definition}
    (Safe coordination) In our context, it is defined as the ability of the coordination and control framework to guarantee the satisfaction of \eqref{Safety} and \eqref{SafeMerging} for every CAV $i \in S(t) \ \forall t$ by conforming to \eqref{VehicleConstraints1} and \eqref{VehicleConstraints2}, to navigate through the CZ without any collision.
\end{definition}

\begin{definition}
    (Uncooperative vehicle) We define a CAV $i \in S(t)$ as \emph{uncooperative} if its free-flow speed is abnormally low in the CZ i.e. $v_i(t) \leq v_{low}$ (where $v_{low}$ is considered abnormally low for the CZ), thus worsening traffic throughput.
\end{definition}

\begin{definition}
    (Adversarial agent) An agent is called adversarial if it has one of the following objectives: (i) prevent \textit{safe coordination}, (ii) \emph{reduce traffic throughput}, by introducing \emph{cyber-attacks}. 
\end{definition}

Note that adversarial agents introduce attacks with malicious intent, whereas uncooperative CAVs are not malicious and may be going slow due to various reasons like faults, failures, and so on. 

\begin{assumption}
\label{accident_assumption}
    Adversarial agents do not collide with other CAVs, nor do they attempt to cause collisions between CAVs and themselves.
\end{assumption}

\subsection{Sybil Attack:}

A single malicious client (could be a CAV or attacker nearby the $\mathrm{CZ}$) may spoof one or multiple unique identities and register them in the coordinator queue table. We assume at any time $t$, there are two groups of CAVs in CZ: i. Normal CAVs and ii. fake CAVs. Let $S_x(t)$ and $S_s(t)$ be the set of the indices of normal and fake CAVs in the FIFO queue of the coordinator unit. Therefore at any time $t$, there are $N(t)=\left|S_x(t)\right|+\left|S_{s}(t)\right|$ CAVs which communicate their state and control information to the RSU. There can be one or more fake clients/CAVs in the $\mathrm{CZ}$ at any time $t$.

%and, $\left\{\cup_{i = 1}^{|S_x(t)|}\boldsymbol{x}_i(t)\right\}$ denote the set of normal CAV states where $\boldsymbol{x}_i\in\mathbb{R}^2$ includes the position and velocity of CAV $i$, and $\left\{\cup_{i = 1}^{|S_x(t)|}u_i(t)\right\}$ denote the set of control inputs for each of the normal CAVs where ${u}_i\in\mathbb{R}$ is the control input, that are communicated to the RSU by each CAV. Let, $S_{s}(t)$ is the set of their indices in the FIFO queue, $\left\{\cup_{i = 1}^{S_{s}(t)|}\boldsymbol{s}_i(t)\right\}$ is the set of the fake CAV states where $\boldsymbol{s_i} \in \mathbb{R}^2$ includes the position and velocity of fake CAV $i$, and $\left\{\cup_{i = 1}^{|S_x(t)|}\tilde{u}_i(t)\right\}$ is the set of the control inputs of the fake CAVs where $\tilde{u}_i \in \mathbb{R}$ is the acceleration input, that are communicated by the attacker at every time instant.  
A Sybil attack is one where the $S_{s} (t) \subset S(t)$ is a nonempty set that is located in the coordinator queue table, but unknown to the coordinator.
%\end{definition}
For example, Fig. \ref{fig:intersection} presents a scenario, where there are multiple fake CAVs with indices $S_{s}(t) = \{3,5\}.$
%{\color{red} Give an example corresponding to Fig. 1.}

\begin{assumption}
\label{max_count_assumption}
    There is a limit on the number of fake CAVs that an adversary can spoof during a Sybil attack due to resource and energy limitations.
\end{assumption}

%Let $[t^{i}_{s}, t^{i}_{e}]$ be the duration when attacker spoofs CAV $i \in S_{s}(t)$ in the network. 
%We assume three types of attacker models:
%We assume that the maximum number of fake CAV is %\begin{enumerate}
%\item Non-informed attacker: Attacker possesses no knowledge of the infrastructure and vehicle dynamics, and, hence the information sent by the attacker is modeled as follows:
%\begin{equation}
%    \left[\begin{array}{c}
%    \boldsymbol{s}_{i,k} \\
%    \tilde{u}_{i,k}
%    \end{array}\right] \in \mathbb{R}^3, \ \forall t_k \in [t^{i}_{s}, t^{i}_{e}], \ k \in [1,2,3 \dots]
%\end{equation}

%\item Infrastructure aware attacker: Has knowledge about the data packet structures, coordination requirements (constraints), and vehicle dynamics and limitations. Thus, the data sent by the adversary is modeled as follows:
%\begin{align}
%\label{inf_aware_adversary}
%&\left[\begin{array}{c}
% \boldsymbol{s}_{i,k} \\
% \tilde{u}_{i,k}
%\end{array}\right]=\left[\begin{array}{c}
%\left(f(\boldsymbol{s}_{i,k-%1})+g(\boldsymbol{s}_{i,k-1}) \tilde{u}_{i,k}\right) dt \\
%\Tilde{u}^{k}_{i}
%\end{array}\right] \nonumber \\
%&\forall t_k \in [t^{i}_{s}, t^{i}_{e}], \ k \in [1,2,3 \dots]
%\end{align}

%where $dt$ is the sampling period and $\Tilde{u}^{k}_{i} \in \mathbb{R}$ s.t. rear end \eqref{Safety}, merging constraint \eqref{SafeMerging} and vehicle limitations in \eqref{VehicleConstraints1} and \eqref{VehicleConstraints2} are satisfied. This attacker model is intended to model the situation when an adversary decides to be covert in pursuit of imparting damage to the network without getting detected by trivial checks.
%\item Strategic attacker: This model is intended to emulate the scenario when the adversary's aim is to cause havoc in the traffic network in the shortest time possible before getting detected. Hence, the data is generated using the same model as in (\ref{inf_aware_adversary}) except that the constraints in \eqref{Safety}, \eqref{SafeMerging}, \eqref{VehicleConstraints1} and \eqref{VehicleConstraints2} are not always satisfied.
%\end{enumerate}

%We consider two types of attacks namely:

%
%\begin{definition}
 %   (Single Point Attack): In this case the adversary uses a single point of attack i.e. using a single fake CAV to target the road network and achieve its adversarial objective.
%\end{definition}

%\begin{definition}
 %   (Multiple Point Attack):  During this case the adversary uses multiple attack points i.e. Spoofs  more than one CAV to adversely affect road network. The maximum number of fake agent $|S_{s}(t)| < \lvert S_x(t) \rvert \ \forall t$.
%\end{definition}

%\subsection{FDI attack}
%We consider that there are one, or more adversarial CAVs with index set $S_f(t) \in S(t)$ physically located in the CZ. The set of indices of normal CAVs is $S_x(t) =  S(t) /S_f(t).$ Then an FDI attack is one, where $S_f(t) \neq \emptyset$ and each CAV $i \in S_f(t)$ sends false state and control information to the coordinator during the attack.

\section{Trust framework}
\label{trust_framework}
In this section, we present our trust framework inspired from the ideas in \cite{Cheng_01,Cheng_02,Hu_01}. We consider that the central coordinator is trustworthy, and monitors, computes and stores the trust of every CAV $i \in S(t)$ in the network at every time $t$ denoted as $\tau_i(t) \in [0,1]$. The trust is determined based on identified behavioral specifications specific to the CAVs in the CZ, which are described below.
%These specifications correspond to normal behavior, and adherence or violation of these specifications provides positive or negative evidence used to update the trust of the CAVs. At first, we describe the behavioral specifications deemed pertinent for computing the trust of the CAVs in the CZ. 


%{\color{red} You may want to discuss uncertainties in behavioral checks. Or just briefly mention how you are going to deal with uncertainties.}
\noindent\textbf{Behavioral Specifications:}
 
\begin{enumerate}
    \item \label{vision_perception} \textbf{Co-observation consistency checks}:  %We include co-observation of CAVs in the CZ for forming another specification, due to the particular threat we consider in this work.
    Based on the reported position of the CAVs, for each CAV $i$ the coordinator identifies a set $S_{i}^{o}(t)$ of CAVs that CAV $i$ should be visible to at time $t$. Let $S_{i}^{\hat{o}}(t)$ be the set of CAVs which report estimated states of CAV $i$. Then the specification is $S_{i}^{o}(t) = S_{i}^{\hat{o}}(t)$. 
    
    \item \textbf{Initial condition checks}: The reported initial states particularly the position information of the CAVs has to be consistent. %The specification is: $x_i(t_i^0) \in \{o_1,\dots,o_8\} \ \forall i \in S(t)$.
    
    \item \textbf{Dynamic model checks}: 
    The physical model similar to \eqref{VehicleDynamics} is invariant %assuming that the CAVs are homogeneous, 
    and hence, the data communicated by each CAV has to always satisfy the underlying model. %We assume that the coordinator uses some well-known model-based estimation algorithm (e.g. Kalman Filter) for estimating the states of the CAVs in the CZ. Let the estimated state of CAV $i$ based on the communicated information be $\boldsymbol{x}^e_i(t)$. Then the specification is: $\lVert\boldsymbol{x}^e_i(t) - \boldsymbol{x}_i(t)\rVert_{\infty} \leq e_x$, where $e_x$ is an error threshold hyper-parameter.
    \item \textbf{Control zone rule checks:} The rules for safe coordination and the vehicle limitations presented in \ref{rules} are invariant and mandatory for every CAV in the CZ. Hence, the specification is, every CAV $i \in S(t) \ \forall t$ has to conform to all rules in \ref{rules} while in the CZ. %{\color{red}you can make it clear this is basically the constraints check defined in the problem setup.}
    %However, since (some of) the constraints \eqref{Safety} and \eqref{SafeMerging} for any CAV $i$ involve other CAVs hence the specification for these constraints are as follows:
    %\begin{equation}
    %     \mathbb{1}[\tau_{i_p(i_m)} \geq 1-\delta][z_{i,i_{p}(i_m)}(t) - \varphi v_{i}(t) - \Delta] \geq 0, \label{Safety_spec}%
    %\end{equation}
    %Equation \eqref{Safety_spec} specifies that the constraints \eqref{Safety} and \eqref{SafeMerging} for CAV $i$ are only evaluated if the other CAVs are trustworthy as violation of these will return negative evidence with large magnitude.
\end{enumerate}

Let $\mathcal{B}$ be the index set of the behavioral specifications in the order they are enumerated above. For each CAV $i \in S(t), \ \forall t \in [t_i^0,t_i^f]$ the coordinator assigns positive evidence $r_{i,j}(t)$ and negative evidence $p_{i,j}(t)$ for conformance and violation of every specification $j \in \mathcal{B}$ respectively (where $0 \leq r_{i,j}(t) \leq r_{max}, 0 \leq p_{i,j}(t) \leq p_{max}$), which it uses to update $\tau_i(t)$. We define $R_i(t)$ and $P_i(t)$ as cumulative positive and negative evidence for CAV $i$ at time $t$ discounted by trust of other CAVs (if the check involves another CAV, like in \eqref{Safety} and \eqref{SafeMerging}, as they can be untrustworthy). We also define a time discount factor $\gamma \in (0,1)$ as defined in \eqref{evidence}. In addition, we have a non-informative prior weight $h_i$ as in \cite{Cheng_01,Cheng_02}. Let the set of checks for every CAV involving another CAV(s) be denoted as $\mathcal{B}_{a} \subset \mathcal{B}$. The set of other CAVs involved in check $j \in \mathcal{B}_a$ when applied to CAV $i$, is denoted as $S_{i,j}(t) \subseteq S(t)/\{i\}$. %In the case of our considered example of the intersection, such checks are the \textit{control zone rule checks} in \eqref{Safety} and \eqref{SafeMerging} both of which involve another CAV. 
Then, the trust metric %{\color{red}$\tau_i$ is the trust metric? make it clear. Also move (8) to the first equation as it is the most important.} 
is updated as follows: 
%a 3-tuple $\{\tau_i(t),d_i(t),u_i(t)\}$ where 
%is the belief mass 

\begin{equation} \label{trust}
    \tau_i(t) = \frac{R_i(t)}{R_i(t) + P_i(t) + h_i} \ \ \forall i \in S(t)
\end{equation}

\begin{align} \label{evidence}
    R_i(t) = & \gamma R_i(t-1) + \sum_{j \in \mathcal{B} \backslash \mathcal{B}_{a}} r_{i,j}(t) + \sum_{j \in \mathcal{B}_{a}}\prod_{k\in S_{i,j}} \tau_k(t)r_{i,j}(t)  \nonumber \\
    P_i(t) = &\gamma P_i(t-1) + \underbrace{\sum_{j \in \mathcal{B} \backslash \mathcal{B}_{a}} p_{i,j}(t) + \sum_{j \in \mathcal{B}_{a}} \prod_{k\in S_{i,j}} \tau_k(t)p_{i,j}(t)}_{p_i(t)} \nonumber \\
    &\forall i \in S(t), \forall t \in [t_i^0,t_i^f]
\end{align}


Finally, we define a lower trust threshold $\delta \in (0, 1/2)$, and a higher trust threshold $1 - \delta$ for subsequent sections. It is important to emphasize that, in practice, the magnitude of negative evidence is different and significantly higher compared to the magnitude of positive evidence. This model of trust relationships considers the social aspect, where a single action can cause significant damage to a trust relationship, and recovery from such damage is challenging \cite{Garlichs_01}. 

\begin{remark}
    Our implementation is agnostic to the specific implementation of the trust framework and the ideas can be used for any framework provided that the trust metric can accurately encapsulate the behavioral specification of the network and distinguish between normal and anomalous behavior for every CAV in real-time.
\end{remark}

\section{Safe and Resilient Control Formulation}
\label{resilient_control_formulation}
We adopt a decentralized \emph{Optimal Control Problem} (OCP) controller for the CAVs that uses Control Barrier Functions (CBF). CBFs provide manifold benefits namely, i. their forward invariance property guarantees satisfaction of the constraints of the OCP, and ii. they transform the original constraints to linear constraints in terms of the control input which makes them computationally efficient, thus, attractive for real-time applications \cite{Xiao_03}.

\textbf{The OCBF Controller} \cite{Xiao_03}. Firstly, Control Barrier Functions (CBFs) that ensure the constraints (\ref{Safety}), (\ref{SafeMerging}), \eqref{VehicleConstraints1} and \eqref{VehicleConstraints2} are derived, subject to the vehicle dynamics in (\ref{VehicleDynamics}) by defining $f(\boldsymbol{x}_i(t))=[v_i(t),0]^T$ and $g(\boldsymbol{x}_i(t))=[0,1]^T$. Each of these constraints can be easily written in the form of $b_q(\boldsymbol{x}(t)) \geq 0$, $q \in \lbrace 1,...,n \rbrace$ where $n$ stands for the number of constraints only dependent on state variables and $\boldsymbol{x}(t)=[\boldsymbol{x}_1(t),\boldsymbol{x}_2(t),...,\boldsymbol{x}_{N(t)}(t)]$. The CBF method (details provided in \cite{Xiao_03, Xiao2019}) maps a constraint $b_q(\boldsymbol{x}(t)) \geq 0$ onto a new constraint which is \emph{linear} in the control input and takes the general form
\begin{equation} \label{cbf_condition}
L_fb_q(\boldsymbol{x}(t))+L_gb_q(\boldsymbol{x}(t))u_i(t)+\kappa_q( b_q(\boldsymbol{x}(t))) \geq 0.
\end{equation}
where. $\kappa_q$ is a class $\mathcal{K}$ function.

A Control Lyapunov Function (CLF) is used for velocity tracking with $v_{i}^{ref}(t)$ as the reference by setting $V(\boldsymbol{x}_i(t))=(v_i(t)-v_{i}^{ref}(t))^2$, rendering the following CLF constraint:
\begin{equation}\label{CLF_constraint}
L_fV(\boldsymbol{x}_i(t))+L_gV(\boldsymbol{x}_i(t))\boldsymbol{u}_i(t)+ c_3 V(\boldsymbol{x}_i(t))\leq e_i(t),
\end{equation}
where $e_i(t)$ makes this a soft constraint. \emph{Note that} the CBFs are used to enforce hard constraints mentioned in section \ref{rules}, whereas CLFs are used to enforce soft constraints. 
%The control goal is to determine a control law jointly minimizing the travel time and energy consumption subject to constraints \eqref{Safety}, \eqref{SafeMerging}, \eqref{VehicleConstraints1} and \eqref{VehicleConstraints2} for each $i \in S(t)$ governed by the dynamics (\ref{VehicleDynamics}). Expressing energy through $\frac{1}{2}u_i^2(t)$ and normalizing travel time and energy, we use the weight $\alpha\in[0,1]$ to construct a convex combination as follows: The solution is \emph{decentralized} in the sense that CAV $i$ requires information only from CAVs that are relevant to it due to one or both constraints in \eqref{Safety} and \eqref{SafeMerging}.

The OCBF problem corresponding to \eqref{eqn:energyobja} is formulated as:
\begin{equation}\label{QP-OCBF}\small
\min_{u_i(t),e_i(t)}J_i(u_i(t),e_i(t)):=\int_{t_i^0}^{t_i^f}\big[\frac{1}{2}(u_i(t)-u_{i}^{ref}(t))^2+\lambda e^2_i(t)\big]dt
\end{equation}
subject to vehicle dynamics (\ref{VehicleDynamics}), the CBF constraints \eqref{cbf_condition}, $\forall q=\{1,...,n\}$ and CLF constraint \eqref{CLF_constraint}. In this approach,(i) $u_i^{ref}$ is generated by solving the unconstrained optimal control problem in \eqref{eqn:energyobja}, (ii) the resulting $u_i^{ref}$ is optimally tracked such that constraints including the CBF constraints \eqref{cbf_condition} $\forall q=\{1,...,n\}$ are satisfied. We can solve this dynamic optimization problem by discretizing $[t_i^0,t_i^f]$ into intervals $[t_i^0,t_i^0+t_s],...,[t_i^0+kt_s,t_i^0+(k+1)t_s],...$ with equal length $t_s$ and solving (\ref{QP-OCBF}) over each time interval through solving a QP at each time step:
\begin{align} \label{QP}
\min_{u_{i,k},e_{i,k}}&[ \frac{1}{2}(u_{i,k}-u_i^{ref}(t_{i,k}))^2+\lambda e_{i,k}^{2}]
\end{align}
subject to the CBF constraints \eqref{cbf_condition}, $\forall q=\{1,...,n\}$, CLF constraint (\ref{CLF_constraint}) and dynamics \eqref{VehicleDynamics}, where all constraints are linear in the decision variables.

%\begin{definition}
%    (Resilience) It is the ability of the framework to guarantee \textit{safe coordination} and \textit{mitigate any traffic jam} introduced by adversarial agents and uncooperative CAVs.
%\end{definition}

\subsection{Resilient Control and Coordination Scheme}

We propose a resilient coordination and control scheme to mitigate the adversarial effects in terms of causing (i) collision and (ii) traffic congestion. Resilience is the ability of the framework to guarantee \textit{safe coordination} and \textit{mitigate any traffic jam} introduced by adversarial agents and uncooperative CAVs.

\subsubsection{\noindent \textbf{Resilience goal (collision avoidance)}}

\

\noindent {\textbf{Trust-based search:}}  The coordinator incorporates trust besides the default passing sequence policy (e.g. FIFO) to identify indices of CAVs, any CAV may conflict within the CZ based on \eqref{Safety} and \eqref{SafeMerging}. Under the default passing sequence, for every CAV $i \in S(t)$, %for every MP $\mathcal{M} \in \mathcal{M}_i$,
the coordinator has to identify indices of all CAVs which includes i. index of the CAV that immediately precedes CAV $i$ physically in its lane and ii. index of the CAV that will precede $i$ immediately at every $m \in \mathcal{M}_i$ in the intersection. For example, in Fig. \ref{fig:intersection}, $\mathcal{M}_6 = \{M_{11}, M_{13}, M_{16}, M_{20}\}$, and as per FIFO sequencing, $6_{M_{20}} = 5$, since CAV 5 is the CAV that will precede it. % in $M_{21}$, CAVs 5, 6 and 7 don't travel through $M_{21}$.  

The trust-based search process identifies all the CAVs that will precede $i$ until the first CAV whose trust value is greater than or equal to $1- \delta$ and forms a set $S_{i,m}(t) \subset S(t)$ containing all the CAV indices identified during the search process. It follows the same search process for every MP in $\mathcal{M}_i$ and also for \eqref{Safety}. Therefore, for each CAV $i$, the coordinator identifies $S_{i}^p(t) \subset S(t)$, and $S_{i}^{M}(t) = \bigcup_{m \in \mathcal{M}_i} S_{i,m}(t)$ (where $S_{i}^p(t)$ is the set for \eqref{Safety} and $S_{i}^{M}(t)$ correspond to the set of indices for every MP). The search process is formalized as follows:
\begin{align}
    &S_m(t) = \{i_+ \in S(t) |\ i_+ < i, m \in \mathcal{M}_i\}  \\
    &k_{min} = \text{min} \ \{k \in S_m(t)| \tau_k \geq 1- \delta\}  \label{searchindex}\\
    &\tilde{S}_{i,m}(t) = S_m(t,1) \ \ \ \ \ \ \ \label{default search} \\
    &S_{i,m}(t) = \cup_{k = 1}^{k_{min}} S_{m}(t,k)  \label{trust-based search} 
\end{align}

\noindent where $S_{i,m} (t,k)$ is the $k-th$ element of set $S_{i,m}(t)$. The set returned by the default search process is given in \eqref{default search}, and the trust based search returns the set in \eqref{trust-based search}. Note that, there are three scenarios possible from the search process: i. $k_{min} = \emptyset$ meaning there are no constraints for MP $m$, ii. $\tilde{S}_{i,m} = S_{i,m}$ when $k_{min} = S_m(t,1)$ meaning that the trust of the CAV immediately proceeding CAV $i$ at $m$ is greater than or equal to $1-\delta$, and iii. $k_{min} > S_m(t,1)$, hence $\tilde{S}_{i,m}(t) \subset S_{i,m}$ implying that the immediately preceding CAV has trust lower than $1-\delta$. For the example in Fig. \ref{fig:intersection}, notice $4_p = 3$. However, since $\tau_3 < 1-\delta$, the search process will continue and return $S_{4,p} = \{3,1\}$. Similarly, under the trust-based search scheme $6_{M_{20}} = \{5,4,3,2\}$ as CAVs 2, 3 4, and 5 have trust less than $1- \delta$.

%To describe the search process, let us consider a CAV $i \in S(t)$. For any merging point $j \in m$ (as defined before) at time $t$, the coordinator first identifies the index of the CAV that can conflict with $i$ based on FIFO passing sequence. Following that it checks the trust of that CAV, and if it is less than a preset threshold $1- \delta$, it moves up the queue table checking the trust of each CAV that will cross MP $j$ before CAV $i$, until it finds the first CAV that meets the required trust score. %The search can terminate in one of the two ways, namely: i. It moves all the way to the top of the table and can't find any CAV with trust greater than $1 - \delta$, ii. it finds a CAV with a trust score greater than, or equal to $1 - \delta$.In both cases, the coordinator 


The state and control information of the CAVs in $S_i^p(t) \cup S_{i}^{M}(t)$ are communicated to CAV $i$ at each $t$, and the corresponding CBF constraints for each CAVs are incorporated to the control in \eqref{QP}.  %It is important to note that the addition of the constraints by the \emph{trust-based search} method does not affect the computational tractability of the QP since the maximum number of constraints is bounded by $\lvert S(t) \rvert$, and QP complexity is linear in the number of constraints. %{\color{red}Briefly explain what Thm. 1 is about (i.e., the transition is not smooth).}

%\begin{remark}
%The addition of new constraints due to trust- based search in \eqref{QP} for any CAV $i \in S(t)$ at time $t'$ where $t' \in [t_{i}^{0},t_{i}^{f}]$ doesn't affect the feasibility of the problem \eqref{QP} at $t'$ necessary for guaranteeing their satisfaction $\forall \ t\geqslant t'$ using the forward invariant property of CBFs.
%\end{remark}

\begin{lemma}
\label{feasibility_lemma}
    The introduction of additional constraints due to \emph{trust-based search}, (including those due to default search process) in the control for any CAV $i \in S(t)$ in \eqref{QP} at time $t'$ where $t' \in [t_{i}^{0},t_{i}^{f}]$ does not affect the feasibility of the problem \eqref{QP} at $t'.$ 
\end{lemma}
\ifITSC
\begin{proof}
    As mentioned, for any CAV $i \in S(t)$, $S_{i,m}(t) \subset \tilde{S}_{i,m}(t)$ is the set of indices of the CAVs that $i$ needs to stay safe to at MP $m \in \mathcal{M}_i$ under trust based search scheme. Let the trust-based search adds an index of a CAV $i_- (< i)$ to $S_{i,m}(t)$. We define, $i_1 = S_{i,m}(t,1)$, $b_{i,i_1}(\boldsymbol{x}(t^{'})) = x_{i_1}(t') - x_{i}(t') - \varphi v_{i}(t') - \Delta$. Similarly, $b_{i_1,i_-}(\boldsymbol{x}(t^{'}))$ and $b_{i,i_-}(\boldsymbol{x}(t^{'}))$ can be defined.

    Notice that, $m \in \mathcal{M}_{i_-}$. Also notice, $i_- < i_1 < i$, since $i_-$ will cross MP $\mathcal{M}$ before $i_1$ which will cross before $i$. This implies, $b_{i_1,i_-}(\boldsymbol{x}(t^{'})) \geq 0 \text{ and }, b_{i,i_1}(\boldsymbol{x}(t^{'})) \geq 0$. Hence $b_{i,i_-}(\boldsymbol{x}(t^{'}))  = b_{i_1,i_-}(\boldsymbol{x}(t^{'})) + b_{i,i_1}(\boldsymbol{x}(t^{'})) \geq 0$, implying the constraint is initially feasible and $i$ is safe to $i^-$ at $t'$. Hence, the addition of a new CBF constraint due to \emph{trust-based search} corresponding to $i_-$ to the control of CAV $i$ (or any CAV) in \eqref{QP} doesn't affect the feasibility at time $t'$.
\end{proof}
\fi

\begin{remark}
    Lemma 1 is necessary for guaranteeing the satisfaction of the CBF constraints corresponding to the CAVs returned by \emph{trust-based search} process $\forall \ t\geqslant t'$ using the forward invariant property of CBFs \cite{Xiao_03}.
\end{remark}

\begin{theorem} Given $0 \leq r_{i,j}(t) \leq r_{max} \ \forall t, \ \forall i \in S(t), \ \forall j \in \mathcal{B},$ the introduction of trust-based search guarantees avoidance of collision by guaranteeing the satisfaction of \eqref{Safety} and \eqref{SafeMerging} that can be caused by adversarial agents.
\end{theorem}
 
\ifITSC
\begin{proof} Let, adversarial CAV $i \in S(t) /\{k\}$ attempts to induce an accident to CAV $k$ in the CZ at time $t$ through using one of the attacks in section \ref{threat_model}. Firstly, notice $k$ must be greater than $i$, else it is impossible to create an accident due to i. each CAV staying safe to all immediately preceding CAVs in their trajectory, and ii. assumption \eqref{accident_assumption}. At some time $t>t_i^0$, CAV $i$ has to violate its own constraint; else, if $i$ satisfies its own constraint, so will each CAV $i_+ \in S(t)$ ($i_+ = \{i_+ \in S(t) |\ i_+ >i, (\mathcal{M}_i \cap \mathcal{M}_{i_+}) \neq \emptyset\}$) queuing behind $i$ and so does CAV $k$.  Upon violation of a constraint by CAV $i$, two scenarios can occur. 

Case (i) $\tau_i(t) > 1-\delta$:  Given $r_{i, j}(t) < r_{max}$,
\begin{align}
&\sum_{j \in \mathcal{B} \backslash \mathcal{B}_{a}} r_{i,j}(t) + \sum_{j \in \mathcal{B}_{a}} \prod_{k\in S_{i,j}^a} \tau_k(t)r_{i,j}(t) \leq \left|\mathcal{B}\right| r_{max} \nonumber\\
& \therefore R_i(t) \leq\left|\mathcal{B}\right| r_{max}+\gamma R_i(t-1) \leq \frac{\left|\mathcal{B}\right| r_{max}}{1-\gamma} \nonumber\\
& \text{and}, p_{i, j}(t) \geqslant 0 \Rightarrow P_i(t) \geqslant 0 \nonumber
\end{align}
We need the trust $\tau_i < 1 - \delta$ immediately to trigger trust-based search. Hence we need to show given $\tau_i(t) > 1-\delta$, $\exists p_{i}(t+1)$ and $p_{i,j}(t+1)$ s.t. $\tau_i(t+1)<1-\delta$ i.e. $trust-based \ search$ is triggered in next iteration.

\begin{align}
& \tau_i(t+1)=\frac{R_i(t+1)}{R_i(t+1)+P_i(t+1)+h_i} < 1-\delta \nonumber\\
& \Rightarrow P_i(t+1)>\frac{R_i(t+1)}{1-\delta}-R_i(t+1) - h_i \nonumber\\
& \Rightarrow p_i(t+1)+\gamma P_i(t)>\frac{\delta}{1-\delta} \frac{|\mathcal{B}|r_{max}}{1-\gamma} \nonumber\\
& \Rightarrow p_i(t+1)>\frac{\delta}{1-\delta} \frac{|\mathcal{B}|r_{max}}{1-\gamma}\geq\frac{\delta}{1-\delta} \frac{|\mathcal{B}|r_{max}}{1-\gamma}-\gamma P_i(t) \nonumber
\end{align}

\noindent and, when $\tau_i(t+1) < 1-\delta$,  then, CAV $k$ will stay safe from $i$, and, as well to all other CAVs that will arrive before $i$ as well as all CAVs in $S_i^p(t) \cup S_i^M(t)$. We set the sampling time to be in the order of $ms$, combining this with Lemma \ref{feasibility_lemma} will guarantee safety for CAV $k$, thus preventing any collision.

Case (ii) $ \tau_i(t)<1- \delta$: The same argument in the preceding paragraph apply, and hence guarantee safety for CAV $k$. A similar argument can be extended to guarantee safety for every CAV $i_+ \in S(t)$ which completes the proof.
\end{proof}

\fi

\ifarvix 
\subsection{Resilient Coordination}
\begin{definition}
    (Resilience) We define resilience in our context as the ability of the control and coordination framework to guarantee \textit{safe coordination} and mitigate any traffic jam introduced by adversarial agents through attacks mentioned in section \ref{threat_model}.
\end{definition}
\fi

\ifITSC
\subsubsection{\noindent \textbf{Resilience goal (traffic jam avoidance)}} The goal is to avoid traffic buildup in the network due to uncooperative/malicious agents acting deliberately to create traffic congestion in the network.  

\textbf{Robust Scheduling:}
We propose a central, $event-driven$, $robust \ scheduling$ scheme that implements FIFO passing sequence for the CAVs in the CZ during normal operation; however, reschedule the CAVs in the presence of adversarial CAVs to prevent any traffic jam. We define a rescheduling zone in the CZ of length $L_1$ as shown in Fig. \ref{fig:intersection}. 
%To achieve this, we introduce a scheduling scheme implemented centrally by the roadside coordinator unit. The scheduling is done in an \textit{event-driven} manner. 
We first present the rescheduling schemes followed by the events resulting in CAV scheduling (rescheduling).

\noindent{\textbf{Trust-aware scheduling:}} %This objective of introducing trust-aware scheduling (rescheduling) is to make the coordination algorithm resilient to CAVs with low trust score.
Under this scheme, CAVs are indexed (sequenced) in descending order of their trust value, which is intended to encourage CAVs to act in a manner that earns them trust as quickly as possible upon arrival in the CZ. The algorithm is presented in Algorithm \ref{Trust_rescheduling_algorithm}.
%or, discourage them from taking actions that decrease their trust score. \textcolor{red}{explain with figure 1}     
\begin{algorithm} [b]
\SetKwInOut{Input}{Input}\SetKwInOut{Output}{Output}
\SetKwComment{comment}{\#}{}
\caption{Trust-aware rescheduling algorithm}
\label{Trust_rescheduling_algorithm}
\Input{$\tau_i(t), \tau_i(t-1) \ \forall i \in S(t),$  \ 
$\mathcal{A}$ = \text{allowable proportion of CAVs with low trust} 
}
\Output {New sequence}%{New indices of the CAVs following trust based scheduling}
\BlankLine
Set of CAVs with low trust $S_R(t) = \emptyset$ \\
\For {each CAV $i$ in Rescheduling zone}
{
    \If{${\tau}_i\left(t-1\right)- {\tau}_i\left({t}\right) \geq 0 \ \& \ {\tau}_i\left(t\right) \leq \delta$}
        {append $i$ to $S_R\left(t\right)$%=S_R\left(t\right) \cup i$
        }
}

\If{$\lvert S_R(t)\rvert \geq \mathcal{A} \times N(t)$} %and more than $\mathcal{T}_n \% \ \text{of CAVs in} \ S_R(t) \ \text{are different from} \ R$
{
    %solve problem in equation \eqref{trust_based_resequencing} to get new indices of CAVs in the re-sequencing zone. \\
    %$R \leftarrow S_R(t)$  
    Solve \eqref{trust_based_resequencing}
}
\end{algorithm}

The problem of the rescheduling (i.e. finding a passing sequence) based on the trust score of the CAVs is formulated as a Integer Linear Program (ILP) as in \eqref{trust_based_resequencing}. We define the index of the first CAV in the queue to re-sequence from as $k_{min} = \min S_R(t)$ (where $S_R(t)$ is defined in Algorithm \ref{Trust_rescheduling_algorithm}) and $S_+(k_{min}) = \{k_{min},\dots,N(t)\}$, as the set of indices of the CAVs to be rescheduled in $S(t)$. 

\begin{align}
    \argmax_{\left\{a_i \in S_+(k_{min})\right\}} &\sum_{i=k_{min}}^{N(t)} (1-{\tau}_i\left(t\right))a_i     \label{trust_based_resequencing} \\
    \textnormal{s.t.} \ \ &  a_j - a_k \geq \nu \ \ \forall j \in S_+(k_{min}), k \in S_{j}^p \label{rearend_ILP} \\
    &a_j \neq a_k \ \ j,k \in S_+(k_{min})  \label{unique_ILP}\\
    & \nu \geq 1  \label{feasibleset_ILP} 
\end{align}

where \eqref{rearend_ILP} corresponds to constraint \eqref{Safety}, $\{a_{k_{min}},\dots, a_{N(t)}\}$ are the new indices of the CAVs in $S_+(k_{min})$. For example in Fig. \ref{fig:intersection}, rescheduling moves CAV 3 (and immediately preceding CAV 4) down in the queue beneath the remaining CAVs in the CZ since $\tau_3$ is the lowest of all CAVs in the Rescheduling zone.  

\noindent\textbf{Lane-priority based rescheduling:} This idea is based on lane priority assignment where lanes are prioritized by observing the number of uncooperative CAVs in that lane. However, note that the presence of slow CAVs in the trajectory of a particular CAV $i$ (i.e. constraints of CAV $i$) can also cause it to go slower than $v_{low}$. Hence, we identify any CAV $i \in S(t)$ as uncooperative at time $t$, if $v_i(t) \leq v_{low}$ and $\nexists i_+ \in S_{i}^{M}$ s.t. $v_{i_+}(t) \leq v_{low}$; we group the slow moving CAVs at time $t$ for lane $l$ into the set $S_{l}^a(t)$ where $l \in \{l_1, \dots, l_8\}$. Following that we compute \emph{the priority of any lane} $l$ using the following equation.

\begin{equation}
\label{lane_priority}
    \zeta_l(t) = 1- \frac{S_{l}^a(t)}{\sum_{l \in [l_1,\dots,l_8]}S_{l}^a(t)+c}, \ \ c (\approx 0) \in \mathbb{R}^+
\end{equation}

We define $k_{min} = \min \{k | k \in S^a(t)\}$ and $S_+(k_{min}) = \{k_{min},\dots,N(t)\}$, where $S^a(t) = \cup_{l \in \{l_1, \dots, l_8\}}S_{l}^a(t)$. We also define a set $S_+^r(t)$ containing the indices of CAVs that are not physically following any slow moving CAV:

\begin{equation}
    S_+^r(t) = \{i \in S_+(k_{min})|i_p(t) \cap S^a(t) = \emptyset\} \nonumber
\end{equation}

Then, we define the following condition that triggers the re-sequencing event:

\begin{equation}
\label{lane_priority_resequencing_event}
    \frac{|S_+^r(t)|}{|S_+(k_{min})|} \geq \mathcal{A}_l, \ \ \mathcal{A}_l \in \mathbb{R}^+ \text{is a preset threshold}
\end{equation}

The re-sequencing is done by solving the following ILP that returns the new indices of the CAVs in $S_+(k_{min})$

\begin{align}
    &\argmax_{\left\{a_i \in S_+(k_{min})\right\}} \sum_{i \in S_a} (1-{\zeta_{i}^l}\left(t\right))a_i \nonumber \\
    & \ \ \ \ \ \ \ \ \ \ \  \ \text{s. t.} \eqref{rearend_ILP}, \eqref{unique_ILP} \text{ and } \eqref{feasibleset_ILP}.
    \label{lane_priority_resequencing}
    %&a_j - a_k \geq \nu \ \ \forall j \in S_+(k_{min}), k \in S_{j}^p  label{rear_lane_priority_resequencing}  \\
    %&a_j \neq a_k \ \ j,k \in S_+(k_{min}) \nonumber \\
    %&a_j-a_i \geq -\varepsilon+ N(t)y_{i,j}  \ \forall i \in [l,\dots, N(t)] \nonumber \\
    %&j \in [l, \dots, N(t)] \nonumber \\
    %&a_j-a_i \geq\varepsilon-(1-y_{i,j}) N(t) \ \forall i\in [l,\dots, N(t)] \nonumber \\
    %&j \in [l, \dots, N(t)] \nonumber \\
    %&\left\{a_{k_{min}}, \ldots, a_N(t)\right\} \in S_+(k_{min}),  \nu \geq 1  \label{lane_priority_resequencing}  \nonumber
\end{align}

where $\zeta_{i}^l(t)$ is the priority associated to the lane that CAV $i$ is physically located at time $t$ which can be found in \eqref{lane_priority}, and $\{a_{k_{min}},\dots, a_{N(t)}\}$ are the new indices of the CAVs in $S_+(k_{min})$. For example, in Fig. \ref{fig:intersection}, velocities of CAV 2 and 5 are $v_2 < v_{low}$ and $v_5 < v_{low}$ in lane $l_8$ and $l_5$ respectively. Hence, $S^{a}_{l_8} = \{2, 8, 9\}$ and $S^{a}_{l_5} = \{5, 6\}$. Therefore, the priorities of $l_8$ and $l_5$ become $0.4$ and $0.6$ respectively, while all other lane priority remains equal and high. This causes CAVs in $S^{a}_{l_8}$ to be moved to the very end of the queue followed by $S^{a}_{l_5}$, as seen in the table in Fig. \ref{fig:intersection}. 

%\begin{algorithm}[thb]
%\SetKwInOut{Input}{Input}
%\SetKwInOut{Output}{Output}
%\SetKwComment{comment}{\#}{}
%\Input{Velocity of each CAV in the re-sequencing zone, ($\Omega, v_{low}$) = (minimum percentage for sequencing, minimum velocity) for slow moving CAVs, $Priority$ of lanes at time $t-1$}
%\Output{Priorities for each lane, CAV sequencing based on lane priority}

%Number of slow moving cars in lane $k$ = 0 $\forall k \in [1,\dots,8]$
%Find the total number of CAVs explicitly constrained 

%\For{each lane $k$ in re-sequencing zone}
%{
%    $Priority$ of lane $k$ = $1-\frac{\text{Number of slow moving CAVs in lane } k}{\sum_{j = 0}^{8} \text{Number of slow moving CAVs in lane } j}$
%}
%$\Omega$ = sort()%sorted list of lane based on priority in descending order \\
%$S_{slow}$ = ordered set of indices of all slow moving CAVs in ascending order. \\
%$k_{slow}$ = ordered set of lanes based on priorities in ascending order. \\
%$j = \min S_{slow}$ \\
%$Sum$= $\sum_{k = 0}^{8} \text{Number of slow moving CAVs in lane} \ k$ \\
%\For {$i$ from $j+1$ to $N(t)$}
%{
%  \If {$i$ is constrained by one or more CAVs in $S_{slow}$ and lane of $i  \notin k_{slow}$}
%  {
%    $S_{reseq} = S_{seq} \cup i$
%  }
%}

%\If{$\frac{|S_{reseq}| * 100}{N(t) - j + 1} \geq \Omega$ }
%{
%    \For{each lane $k$ in $k_{slow}$} %lanes$$sorted(\textit{priority of lanes})$}
%    {
%        \If {priority of $k < 1$}
%        {
%            $k_{reseq} =$ Find the set of CAVs in lane $k$ with indices $\geq \ j$ \\
%            Place $k_{reseq}$ at the end of the queue. \\
         %   $k_{old} \leftarrow k$
%        }
%        \Else
%        {
%            \textbf{Break} for loop
%        }
%    }
%}

%\caption{Lane based rescheduling algorithm}
%\label{lane_priority_rescheduling}
%\end{algorithm}

\begin{lemma}
    The rescheduled sequence is guaranteed to be feasible if $ L - L_1 \geq \frac{v_{\max}^2}{2|u_{\min}|} + \Delta$, where $\Delta$ is defined as in \eqref{Safety} and \eqref{SafeMerging}.
\end{lemma}

\ifITSC
\begin{proof}The maximum velocity for any CAV $i$ is $v_{\max}$, and the maximum deceleration is $|{u_{\min}}|$. Thus the minimum distance required to come to full stop for any CAV $i$ is $\frac{v_{\max}^2}{2|u_{\min}|}$. Hence, to satisfy the constraint \eqref{Safety} and \eqref{SafeMerging}, the minimum distance between the merging point and the end of the re-sequencing zone has to be greater than or equal to $\frac{v_{\max}^2}{2|u_{\min}|} + \Delta$, which will guarantee the feasibility of rescheduled sequence.
\end{proof}
\fi

The list of $events$ that cause scheduling (rescheduling) CAVs in the CZ are enumerated below:

\begin{enumerate}
    \item \textbf{Arrival event:} This corresponds to a CAV that has just arrived at the CZ, and hence has to be added in the queue Table \ref{fig:intersection} and an index has to be assigned to it using the default sequencing scheme (FIFO).
    \item \textbf{Departure event:} An exiting CAV triggers this event, after which the row corresponding to that CAV is removed from the coordinator table and the indices of all CAVs are decreased by 1.%This event is triggered when a CAV exits the CZ, post which the row corresponding to that CAV is removed from the coordinator table (see Fig. \ref{fig:intersection}) and the indices of all CAVs are decreased by 1.
    \item \textbf{Reschedule event:} This corresponds to an event,  
    %the presence of adversarial (uncooperative) agents are detected in the CZ whether
    when the presence of uncooperative CAVs triggers the event as a result of the condition in \eqref{lane_priority_resequencing_event} , or, the presence of low trustworthy CAVs results in trust-based rescheduling as in algorithm \ref{Trust_rescheduling_algorithm}. 
    %or, due to the presence of more than a preset number slow moving CAVs triggering lane priority rescheduling as in algorithm \ref{lane_priority_rescheduling}; 
    %hence the CAV indices are rearranged with the aim of providing resilience against traffic jam.
\end{enumerate}

\begin{remark}
    Notice, the two robust rescheduling schemes are event-driven, and hence, can be simultaneously incorporated. 
\end{remark}

The default scheduling scheme in \cite{Xu_02} and the presented rescheduling schemes render a unique index list for the CAVs in the CZ based on which they cross the intersection in descending order of their indices. %Hence, by looking at the coordinator queue table as in Fig. \ref{fig:intersection}, it is possible to identify the list of CAVs that a CAV $i$ may conflict with at the merging points it will traverse through in the CZ. Details of this can be found in \cite{Xu_02}. 

\fi
%{\color{red} A natural question would be is it possible to combine the two resilient goals? how?}

\ifread
\subsection{\textbf{Attack Detection and Mitigation}}
%{\color{red} This section should go before the last section?}
The problem of detection involves the identification of fake CAVs accurately and mitigation can be defined as reestablishing the normal cooperation in the network close to what it would be in the ideal scenario without any attack. Resilience is necessary to ensure safe coordination until the attack is detected and in the presence of any false misidentification of fake CAVs.

\subsubsection{Determination of Fake CAVs} Initially, every CAV is considered untrustworthy (i.e. $\tau_i(t_i^0) = 0$). Upon arrival in CZ, the coordinator monitors the trust for each CAV and if it detects any CAV $i \in S(t)$ s.t. $\tau_i(t) \leq 1 - \delta$ and $\tau_i(t) \leq \tau_i(t-1)$, it initiates an observation window for that particular CAV of lenth $\eta$. If the trust for CAV $i$ is non-increasing and stays below the threshold of $1 - \delta$ during the observation window then the coordinator proceeds to the mitigation step.

\subsubsection{Robust Mitigation}
The most trivial strategy that can be adopted is to rescind cooperation with the fake CAVs; however it is essential to note that our presented framework can (although highly unlikely if our proposed framework in Section \ref{trust_framework}) output false positives. Therefore, we offer a \textit{soft} mitigation scheme -- a passive scheme that relies on the local sensory information of the CAVs. 
%This will become apparent in the remainder of the section. 

\noindent{\textbf{Local sensing}} The incorporation of local sensing to a CAV $i \in S(t)$ adds additional constraints to the control problem in \eqref{QP}, besides the constraints corresponding to $S_{i}^p(t)$ and $S_{i}^{M}(t)$ returned by \textit{trust-based search}. Every CAV $i \in S(t)$ is able to estimate the states of the every observed CAV $j$ within sensing range $S_{i}^{o}(t)$ (as defined before) at time $t$ denoted as $\boldsymbol{\hat{x}}_{i,j}(t)$ defined below:
%Note that every CAV $i$ has the knowledge of the type of constraints imposed by every subset of CAVs $S_{i}^p(t)$ and $S_{i}^{M}(t) \ \forall t$. However, it also has local sensing as mentioned in Section~\ref{vision_perception}, which is combined in order to determine additional constraints for the OCBF problem to provide added resilience as follows. 

\begin{equation}
\label{CAV_vision_estimate}
    \boldsymbol{\hat{x}}_{i,j} (t) = \boldsymbol{x}_{i,j} (t) + \boldsymbol{w}(t)
\end{equation}
where $\boldsymbol{w}(t)$
%$ = [w_x(t), w_v(t)]^T$ are random noise terms s.t. $\|\boldsymbol{w}\|_{\infty} \leq \epsilon$ where $\epsilon \in \mathbb{R}^+$, 
is random noise which signifies the uncertainty in the sensors and state estimation algorithms,
%in the position and velocity estimation.
Hence, CAV $i$ is able to estimate the state of the preceding CAV (if there is any and it is within sensing range) i.e. $S_{i}^{o} \cap S_{i}^p \neq \emptyset$, %denoted as $\boldsymbol{x}_{i,i_p}^{s}(t)$. Similarly, as CAV $i$ approaches the intersection, it is able to detect CAVs (if there are any) 
and in the vicinity of MPs in its own trajectory i.e. $S_{i}^{o} \cap S_{i}^{M} \neq \emptyset$ in particular the CAV (if there is any) that will precede $i$ immediately at its next MP should be visible to CAV $i$. 
%(i.e. belong to $S_{i}^{o}$) and estimate their states.


%and incorporate that information in particular CAV $i$ . Let $x_{i,im}^{s}(t)$ be the sensed position of the CAV (like before) at the nearest MP to the CAV respectively at time $t$. 

%In the absence of local sensing or, if the preceding vehicle at the next MP is out of the sensing range for a CAV $i$, then it uses the communicated position for solving the QP's. Hence the CBF constraints  (\ref{Safety}), (\ref{SafeMerging}), \eqref{VehicleConstraints1} and \eqref{VehicleConstraints2} and (\ref{VehicleConstraints1}) are derived, subject to the vehicle dynamics in (\ref{VehicleDynamics}) by defining $f(\boldsymbol{x}_i(t))=[v_i(t),0]^T$ and $g(\boldsymbol{x}_i(t))=[0,1]^T$. 

We incorporate CBF constraints corresponding to the indices of the CAVs in $S_{i}^{o} \cap (S_{i}^p \cup S_{i}^{M})$ using the state estimates in \eqref{QP} for CAV $i$. In the presence of noisy measurements (estimates) as in \eqref{CAV_vision_estimate}, the state vector $\boldsymbol{x}(t) = \boldsymbol{\hat{x}}(t) - \boldsymbol{w}(t)$ 
where $\boldsymbol{x}(t) = [x(t), v(t)]^T$. Hence, 
the corresponding CBF constraint in \eqref{cbf_condition} is as follows:
\begin{align} \label{noisy_cbf_condition}
&L_fb_q(\boldsymbol{\hat{x}}(t) - \boldsymbol{w}(t))+L_gb_q(\boldsymbol{\hat{x}}(t) - \boldsymbol{w}(t))u_i(t)+ \nonumber \\ 
&\kappa_q( b_q(\boldsymbol{\hat{x}}(t) - \boldsymbol{w}(t))) \geq 0.
\end{align}

\begin{lemma}
    Given a CBF $b_q(\boldsymbol{x}(t))$ associated with the set $\mathrm{C}:=\{\boldsymbol{x}\in \mathbb{R}^n:b_q(\boldsymbol{x})\geq 0\}$ and $\|\boldsymbol{w}\|_{\infty} \leq \epsilon$, any Lipschitz continuous controller $u(t)$ that satisfies (\ref{noisy_consv_cbf_condition}) renders the set $C$ forward invariant $\forall t \geq t_{0}$ for the control system \eqref{VehicleDynamics}.
\begin{align} \label{noisy_consv_cbf_condition}
&\min_{\{\boldsymbol{w}: \|\boldsymbol{w}\|_{\infty} \leq \epsilon\}} [L_fb_q(\boldsymbol{\hat{x}}(t) - \boldsymbol{w}(t))+L_gb_q(\boldsymbol{\hat{x}}(t) - \boldsymbol{w}(t))u_i(t)+  \nonumber \\
&\kappa_q( b_q(\boldsymbol{\hat{x}}(t) - \boldsymbol{w}(t)))] \geq 0 
\end{align}
\end{lemma}

%\noindent$Proof:$ The satisfaction of \eqref{noisy_consv_cbf_condition} guarantees satisfaction of the constraint \eqref{noisy_cbf_condition} (and \eqref{cbf_condition}) since it is a lower bound for \eqref{cbf_condition} which according to Theorem 1 in \cite{Xiao_03} makes the set $C$ forward invariant $\forall t \geq t_0$ with respect to \eqref{VehicleDynamics}.

%If $\|\boldsymbol{w}\|_{\infty} \leq \epsilon$ with probability $1-\zeta$ then we can acquire a probabilistic bound on the satisfaction %of the constraint \eqref{noisy_consv_cbf_condition}. Note that we can decrease $\zeta$ by increasing $\epsilon$, however it will make the controller $u(t)$ more conservative.

\ifarvix
\noindent At first, we present the following definitions.

\begin{definition}
    (Explicitly constrained agent) An agent $i$ is called explicitly constrained by an agent $j$ at time $t$ if it has a constraint directly involving states of agent $j$ at that time.
\end{definition}

\begin{definition}
    (Implicitly constrained agent) An agent $i$ is called implicitly constrained by an agent $j$ at time $t$ if there is any other agent $k$ in the environment constrained by $j$, which constrained agent $i$.
\end{definition}
\fi

%We make a few observations to motivate the idea of our presented mitigation scheme. Firstly, the re-sequencing zone facilitates rescheduling similar to the concept of trust based rescheduling. However, it is important to note that, in this scenario, every CAV apart from the fake CAV should have equal priority and the fake CAV should have the least priority. This is because we don't want to penalize the other CAVs apart from the fake CAV by changing their index in the passing sequence through rescheduling.Based on the observation we present a ILP in \eqref{resequencing_mitigation} similar to \eqref{lane_priority_resequencing} except the objective is to

We mitigate the effect of fake CAVs by \textit{un-constraining} the CAVs that are explicitly constrained by them (including the physically following CAVs if they are within their perception range and does not see any vehicle ahead) by solving the ILP in \eqref{resequencing_mitigation}. Let the set of the ordered indices of detected fake CAVs that we want to mitigate is denoted as $S_f(t)$. We define the index $k_{min}=\min S_f\left(t\right)$ as the index of the first (fake) CAV in the queue to re-sequence from and $S_+(k_{min})$ follows the same definition as before. Then the ILP is as follows:
\begin{align}
    \max_{i \in S_f(t)} &\sum_{i \ \in S_f(t)} {a_i}  \nonumber \\
    &a_j - a_k \geq \nu, \ \forall k \in \bar{S_f}(t) \cap S_{j}^p(t), \nonumber \\ 
    & \text{and } j \in S_+(k_{min})   \label{rear2_ILP} \\
    &a_j - a_k \geq \nu,  j \in S_+(k_{min}), k \in S_{j}^M(t)   \label{merging2_ILP} \\
    &a_j \neq a_k \ j,k \in S_+(k_{min}) \nonumber \\
    %&a_j-a_i \geq -\varepsilon+ N(t)y_{i,j}  \ \forall i \in [s,\dots, N(t)] \nonumber \\
    %&j \in [s, \dots, N(t)] \nonumber \\
    %&a_j-a_i \geq\varepsilon-(1-y_{i,j}) N(t) \ \forall i\in [s,\dots, N(t)] \nonumber \\
    %&j \in [s, \dots, N(t)] \nonumber \\
    &\left\{a_{k_{min}}, \ldots, a_N(t)\right\} \in S_+(k_{min}); \nu \geq 1 \label{resequencing_mitigation}
    %&k_{new}=a_k \ \forall k_{new} \in [k_{min},\dots, N(t)] 
\end{align}

where $\left\{a_{k_{min}}, \ldots, a_N(t)\right\}$ is the new indices of CAVs in $S_+(k_{min})$, \eqref{merging2_ILP} correspond to safe merging constraint in \eqref{SafeMerging}, and \eqref{rear2_ILP} corresponds to \eqref{Safety}. 

\ifarvix
Based on the above definitions we now outline the scenarios that are of importance to us and derive an approximate solution of the ILP for them.
\begin{enumerate}
    \item \textbf{No CAVs are constrained by  CAVs in $S_f(t)$}: In this case, the solution of \eqref{resequencing_mitigation} will reschedule the CAVs starting from index $k = \min S_f(t)$ in $S_f(t)$ by moving them at the end of the queue and move the remaining CAVs with original index $i \geq k \text{ and } i \notin S_f(t)$ ahead in the queue to fill their places in their current order. This process will be repeated for $\forall k \in S_f(t)$.
    \item \textbf{There are CAVs in $S_f(t)$ which physically precede another CAV in the CZ}: First, let us consider CAV $k \in S_f(t)$ and $j$ is the index of physically immediately following CAV, and let $S_j^c(t) \subseteq S(t)$ be the set of CAVs explicitly and implicitly constrained by $j$. At first, the CAVs with index between $k$ to $j-1$ are moved ahead in the queue by incrementing their index by 1, and, then we make $k \leftarrow j-1$ where $j-1 > k$. The reason for moving $k$ down the queue upto $j-1$ is because $k$ can be a real CAV which has been falsely identified as fake CAV. Finally, remove the $\{j-1,j\} \cup S_j^c(t)$ from the queue, rearrange the queue by incrementing the indices appropriately of the remaining CAVs in the queue and add $\{j-1,j\} \cup S_j^c(t)$ in the queue. Then, repeat the process for remaining CAVs in $S_f(t)$. 
    
    The final step is done to move the CAVs that are explicitly constrained (but in a different road), or implicitly constrained but not constrained by the immediately preceding CAV of CAV $k$ ahead of CAV $k \in S_f(t)$ in the queue.    \label{shuffle}
\end{enumerate}
    %intended to move those CAVs that are unconstrained by the $physically \ following$ CAVs $\forall k \in S_f(t)$. This is 
\fi 

Note that the rear-end constraints are excluded for the CAVs that are physically immediately following any CAV $k\in S_f(t)$ in \eqref{resequencing_mitigation} to allow CAVs that are physically immediately behind the CAVs in $S_f(t)$ to overtake them \textit{only if} they are not visible when within sensing range. This is necessary to guarantee safety for the FP cases which will be described shortly. To do so, upon completion of the rescheduling as per the relevant scenario described previously, we modify the CBF constraint in \eqref{noisy_cbf_condition} as follows. 
\begin{align} \label{noisy_modified_cbf_condition}
&L_fb_{k+1,q}(\boldsymbol{\hat{x}}(t) - \boldsymbol{w}(t))+L_gb_{k+1,q}(\boldsymbol{\hat{x}}(t) - \boldsymbol{w}(t))u_{k+1}(t) \nonumber \\ 
&+(\gamma+\rho)( b_{k+1,q}(\boldsymbol{\hat{x}}(t) - \boldsymbol{w}(t))) \geq 0. \nonumber \\
&\text{where} \ \rho \in \mathbb{R}_{+} \ if \ (k \in S_f(t) \ \land \ k \text{\ is not in sensing range})
\end{align}

\ifarvix Observe that, for CAV $k \in S_f(t)$, upon rescheduling, the index of its immediately following CAV will become $k+1$. \fi Once within sensing range of CAV $k+1$, if CAV $k$ is not visible, it changes its control in \eqref{QP} by removing the CBF constraint corresponding to CAV $i$ and sets $u_{ref} = u_{\max}$ to complete the overtake. The coordinator detects the overtake completion by checking the satisfaction of the inequality in \eqref{overtake_check}, upon which it completes the final step of problem in \eqref{resequencing_mitigation} by swapping the indices of CAV $k$ and $k+1$ with each other. This step is performed $\forall k \ \in S_f(t)$ and repeated by following the scenarios mentioned previously (i.e. solution of \eqref{resequencing_mitigation}) until all fake CAVs reach the end of the queue.

\begin{equation}
 \hat{x}_{i}(t) - \hat{x_{i_{p}}}(t) - \varphi \hat{v}_{i_{p}}(t) - \Delta\geq 0,\text{ \ }\forall t\in[ t_{i}^{0},t_{i}^{f}] \label{overtake_check}%
\end{equation}

However, the rescheduling may not be completed for all CAVs, for every CAV in $S_f(t)$ within the rescheduling zone. We assume that all CAVs can observe each other in the intersection. 
If a CAV approaches the intersection explicitly constrained by any CAV(s) in $S_f(t)$ and does not perceive that CAV through its local vision once within sensing range, it simply removes the CBF constraint corresponding to the CAV from the control in \eqref{QP}. \ifarvix For example, as illustrated in Fig. \ref{fig:intersection}, CAV 9 is a fake CAV, and it is $i_p$ for CAV 10. Once $\tau_9 < 1- \delta$ and the observation window expires, CAV 10 will relax its CBF to close its proximity to CAV 9 with the intent of overtaking it. When CAV 10 gets close enough, its vision won't detect CAV 9 since its a fake CAV, so it will initiate and attempt to overtake it. Upon completion, the coordinator will detect the overtaking event, following which it will swap the indices of CAV 9 and 10 with each other and then repeat the process for CAV 11 and 12 in the same way. Finally, CAV 5 is a real CAV that gets falsely identified as a fake CAV. Hence, the immediately following CAV 8 will approach it, however, it will not initiate an overtake due to detecting CAV 5 through its local vision. However, due to the mitigation process, CAV 6 and 7 (located on a different road than CAV 5) will jump ahead of CAV 5 in the queue. \fi   

%Note that, if there are multiple CAVs in $S_f(t)$ for which there are one or more CAVs which are constrained by them, and let $S_c(t) \subseteq S_f(t)$ be the ordered set of elements (in descending order) in $S_f(t)$. Then the process outlined in \ref{shuffle} has to be repeated for every element $k \in S_c(t)$. 

\begin{lemma}
    The proposed mitigation scheme guarantees safety for real CAVs even if they are falsely identified as fake CAV due to a Sybil attack.
\end{lemma}

\ifread
$Proof:$ In the rescheduling zone, any real CAV $i \in S(t)\backslash S_f(t)$ only overtakes a CAV $k \in S_f(t)$ if it doesn't observe $k$ through its local perception. Similarly, any CAV $i \in S(t)\backslash S_f(t)$ only ignores the CBF condition in its control and jumps ahead of a CAV in $S_f(t)$ in the intersection if it doesn't observe it through its local vision. This makes our proposed mitigation scheme soft (or, passive) and guarantees safety for false positive cases i.e. real CAVs which have been misidentified as fake CAVs. 
\fi

\ifarvix
The fake CAVs are removed from the coordinator in one of two ways, namely: (i) the attacker stops sending information about a fake CAV, and (ii) the fake CAV leaves the CZ. 
\fi
\fi

\section{Simulation Results}
\label{Results}
In this section, we present the results of our proposed resilient control and coordination scheme %along with the results for attack detection and mitigation 
for the threats mentioned in section \ref{threat_model}. We performed the simulations in Matlab and \textsc{ode45} to integrate the CAV dynamics. The value of $\delta$ was set to 0.1. The positive and negative evidence magnitudes for the tests in the order they are mentioned in section \ref{trust_framework} are: $r_i(t) = [0.6,0.6,0.6,0.6]^T$ and $p_i(t) = [1000,100,50,1]^T \ \forall i \in S(t)$ and $\forall t$. The intersection dimensions are: $L= 300\textnormal{m}$, $A = 30\textnormal{m}^2$; and, the remaining parameters are $\varphi = 1.8\textnormal{s}$,  $\Delta = 3.78 \textnormal{m}$, $\beta_1= 1, u_{\max} = 4.905 \textnormal{m/s}^2, u_{\min} = -5.886\textnormal{m/s}^2, v_{\max} = 108 \textnormal{km/h}, v_{\min} = 0 \textnormal{km/h}$. %v_{slow} = 5 \text{km/h}$.
Finally, we also adopted a realistic energy consumption model from \cite{Xu_02} to supplement the simple surrogate $L_2$-norm ($u^2$) model in our analysis.

%Whereas fake cars either could not gain any trust or they lose it over time after initial gain.In Fig. it can be seen that cooperative CAV (car 2 denoted by red color) start to gain trust upon arrival and able to attain it through the whole control zone.

\subsection{Resilient Control and Coordination}

\textbf{Resilience control}: Presented in Fig. \ref{fig:resilient_control_2} are results for the scenarios when a fake CAV attempts to violate safety constraints between real CAVs with the aim of creating an accident. The plot shows the value of the safety constraints that the fake CAV attempts to violate with and without our proposed resilient control scheme for \emph{safe coordination}. As can be seen, both rear collision and collision at merging point inside the intersection (which can cause traffic disruption and jam inside the intersection) is possible which is eliminated through our proposed safe and resilient control and coordination scheme, using \textit{trust-based search}. %Note that, although vision can prevent rear collisions, collisions at MPs are not preventable through vision alone as CAVs arrive from multiple directions in the intersection.

%\begin{figure}[htb]
%\begin{center}
% \includegraphics[scale = 0.35]{Final_RearEnd_graph.pdf}
%\caption{Comparison of rear end constraint value with and without the proposed resilient control scheme. 
%\color{red}The font is too small in the figure. \li{The lines should be thicker as well.} %
%}
%\label{fig:resilient_control_1}%
%\end{center}
%\end{figure}

%\begin{figure}[h]
%\begin{center}
% \includegraphics[scale = 0.35]{Final_graph_lateral_constraint.pdf}
%\caption{Comparison of lateral constraint value with and without the proposed resilient control scheme.}\label{fig:resilient_control_2}%
%\end{center}
%\end{figure}

\begin{figure}[h]
\begin{center}
 \includegraphics[scale = 0.45]{Constraint_val.png}
\caption{Comparison of rear-end and lateral constraint value given in \eqref{Safety} and \eqref{SafeMerging},  for a real CAV with respect to another real CAV, with and without the proposed resilient control scheme.% As can be seen, the constraints are violated (meaning become $< 0$) resulting in a collision that is eliminated by incorporating our \emph{trust-based search}.
} \label{fig:resilient_control_2}%
\end{center}
\end{figure}

%\begin{table}[h]
%\caption{Comparison of results with and without lane priority based rescheduling in the presence of unexpectedly slow moving CAVs in the CZ.}
%\label{tb:lane_priorirty_scheduling_result}
%\begin{tabular}{|l|l|l|l|}
%\hline
%                                                                            & Ave. time (s) & Ave. $\frac{1}{2}u_2^2$ & Ave. Fuel \\ \hline
%\begin{tabular}[c]{@{}l@{}}Without lane \\ priority scheduling\end{tabular} & 65.75            & 70.24           & 34.04          \\ \hline
%\begin{tabular}[c]{@{}l@{}}With lane \\ priority scheduling\end{tabular}    & 34.00            & 82.96           & 26.32          \\ \hline
%\end{tabular}
%\end{table}


%Another example for resilience control can be found in Fig. \eqref{fig:intersection} since $\tau_3(t) \leq 1-\delta$, therefore the $i_p$ for CAV 4 will not only be CAV $3$ but also CAV $2$ will be added as the redundant constraints $S_{4,p}= \{2,3\}$. It can be also seen from Fig. \eqref{fig:TrustValue} where CAV $3$ which is a strategic fake attacker loses its trust despite initial trust gain and starts to violate its rear end safety constraints by overtaking CAV $2$, which shown by blue and yellow color respectively in Fig.  \eqref{fig:Positions}. Nevertheless because of resilient control in place CAV $4$ (i.e. purple car in \eqref{fig:Positions}) stays safe with respect to CAV $2$. 

\ifITSC
\noindent\textbf{Lane-priority based rescheduling}: %To show how the lane-priority based rescheduling algorithm described in section ... works, the scenario is \eqref{fig:intersection} can be considered. Here we consider CAV $8$  with abnormally low velocity, $v_8 \leq v_{low}$. As a result to alleviate the effect of the slow cars the lane-priority based rescheduling takes place. Subsequently, CAV 14 and CAV 17 are constrainted by CAV $8$ based on FIFO (CAV 8 will turn left in the intersection, therefore, CAV 14 because of turning left from lane 3 and CAV 17 going straight from lane 3 may collide with CAV 8) will be put ahead of CAV $8$ in the queue table to avoid any slowing down in lane $3$. 
An extensive simulation with multiple slow CAVs was conducted to demonstrate the effectiveness and significance of our lane-priority based re-scheduling scheme with its results demonstrated in Fig. \ref{fig:lane_priority}. We introduced from 2 up to 8 uncooperative CAVs in the intersection across 3 arbitrarily chosen lanes during our simulation. As can be noticed, the cooperative nature of the algorithm can cause traffic holdups with the average travel times of CAVs from over $4$ mins. (270 secs precisely) upto around $5$ mins. However, with our proposed robust scheduling scheme based on lane priority, the average travel time was significantly reduced, and the maximum average travel time was a little over $1$ min (74 secs precisely) which was an improvement of over $3$ mins from the least average travel time without our rescheduling scheme.

%In this particular scenario, we had 3 abnormally slow moving CAVs in two lanes, (one in one of the eight lanes and two on another lane) while the traffic flow was normal in the other lanes. The arrival density was 400 veh/hr. As can be noticed, the average time and energy is improved with our proposed rescheduling scheme. However, the average acceleration metric deteriorates which is expected as the average travel time decreases which means vehicles are able to accelerate more i.e. there is less traffic density per unit length inside the CZ which is desired to avoid traffic jam.
%Finally, we present the result for our trust based re-sequencing scheme in Fig. \ref{fig:trust_resequencing}. The trust threshold was set to 0.2 and the threshold for percentage of low trusted vehicles allowed in the CZ prior to rescheduling was set to 30$\%$. The presented results successfully demonstrate the validity of our trust-aware rescheduling scheme which makes our coordination scheme resilient to low trusted vehicles by giving high priority to CAVs with high trust and rescheduling the traffic in a dedicated rescheduling zone. 

%\begin{figure}[h]
%\begin{center}
% \includegraphics[scale = 0.46]{TrustBasedRequensing_final.png}
%\caption{Illustration of trust-aware scheduling.} %
%\label{fig:trust_resequencing}%
%\end{center}
%\end{figure}  

\begin{figure}[h]
\begin{center}
 \includegraphics[scale = 0.4]{Lane_priority_plot.png}
\caption{Average time for of real CAVs for lane-based priority scheduling for various number of uncooperative CAVs.} %
\label{fig:lane_priority}%
\end{center}
\end{figure}

\noindent\textbf{Trust-aware rescheduling:} Finally, we present the results for our proposed trust-aware rescheduling scheme in Fig. \ref{fig:trust_aware_reschedule}. We introduced various percentages of fake CAVs ranging from 2\% to 15 \% fake CAVs through Sybil attack (using the model in section \ref{threat_model}). We used the various attacker models for the fake CAVs presented in \cite{Ahmad_01}. Our results demonstrate that the average travel times, energy, and fuel consumption of the real CAVs improve with the inclusion of our proposed rescheduling scheme. However, notice that the average energy eventually becomes identical, since a large proportion of spoofed CAVs cause the average travel times of the normal CAVs to increase, thus decreasing the average acceleration input (related to energy, \eqref{eqn:energyobja}). However, the average fuel consumption is improved with our proposed rescheduling scheme. Note that, eventually, as the percentage of fake CAVs approaches 100 \%, the curves for all three metrics will coincide, since all CAVs are fake.

\begin{figure*}[t]
\begin{center}
 \includegraphics[scale = 0.6]{Comparison_results.png}
\caption{The values of average travel time, average energy and average fuel consumption for real CAVs as a result of trust-aware rescheduling for different proportion of fake CAVs.} %
\label{fig:trust_aware_reschedule}%
\end{center}
\end{figure*}

\fi

\ifread
\subsection{Attack detection and mitigation}

We first present a plot illustrating the trust value for various types of attackers mentioned in section \ref{threat_model}. As can be seen, based on our selected behavior specification, the trust framework is able to detect all attacks based on our chosen value of $\delta$. We used the attacker model for the Sybil attack presented in our earlier work in \cite{Ahmad_01}. We consider that the adversary spoofs 4 fake CAVs whose trust values are illustrated in in Fig. \eqref{fig:TrustValue}. In this scenario, for CAV $3$ (shown in blue color) the adversary adopts a strategic attack model, for CAV $9$ (shown in pink) the adversary chooses a naive attacker model, for CAV $13$ (shown in light blue color in Fig. \eqref{fig:TrustValue} the adversary has the knowledge of underlying vehicle dynamics but not the rules mentioned in section \ref{rules}.  

%For example, CAV $3$ shown in blue color in Fig. \eqref{fig:TrustValue} is an strategic attack which initially gains trust and then attempts to create accidents. CAV $9$ shown in pink color in Fig. \eqref{fig:TrustValue} follows a naive attacker who sends false data which results in failing to gain enough trust to be trustworthy. CAV $13$ shown by light blue color in Fig. \eqref{fig:TrustValue} can be considered as an informed attacker which follows the dynamics but fails to follow any constraints. Finally, CAV $14$ shown by brown color in Fig. \eqref{fig:TrustValue} is another case of a naive attacker who shows up in the middle of the road and obviously fails to gain any trust. %In this case, CAV 6 is a non-informed attacker which randomly selects the initial position of the fake CAV, whereas CAV 1 and 4 are strategic attackers and they attempt to send false position information and violate constraint respectively to jeopardize the normal operation of the network as illustrated in the figure. 
%\\

\noindent\textbf{Mitigation}: From Fig. \eqref{fig:TrustValue} it can be seen that trust value of CAV 9 falls below the threshold $\tau_9 \leq 1- \delta$. %Therefore it can be considered as a naive attacker who sends false data which does not follow dynamics. 
which triggers mitigation, causing CAV $10$ which happens to be the trailing CAV $9$ immediately, relax its CBF and get closer to the CAV $9$ to possibly overtake it. Once, CAV $10$ gets close enough and its vision does not detect any car nearby it overtakes CAV 9 which can be seen from Fig. \eqref{fig:Positions}. The same scenario happens for CAV $11$ and CAV $12$ where all of them overtake CAV $9$ in the same way which also is illustrated in Fig. \eqref{fig:Positions}.\\
\textbf{False Positive}: %The main reason behind introducing soft mitigation instead of hard mitigation is detecting a FP case in which a real CAV falsely gets detected as the fake car and the mitigation gets triggered for it. 
In the given example in \eqref{fig:intersection}, we synthesized a scenario where a real CAV $5$ loses trust and eventually $\tau_5 \leq 1- \delta$ as can be seen in Fig. \eqref{fig:TrustValue} (in orange color). As a result, mitigation is triggered resulting in CAV $6$ and CAV $7$ (located in re-sequencing zone) jumping ahead of it since they are in different lanes, as can be corroborated from Fig. \eqref{fig:Positions}.  CAV 6 and 7 (going straight) have common MPs as CAV 5 (turning left). Hence, during mitigation, they will jump ahead of CAV 5, and followingly CAV 5 stays safe relative to them at the common MPs. However, CAV $8$ does not overtake it since it detects it through its vision, and stays behind CAV $5$ thus guaranteeing safety, \eqref{fig:Positions}.

%The results for our proposed detection and mitigation scheme has been illustrated in figure \ref{fig:mitigation}. 
%and \ref{fig:mitigation_2} respectively. 
%We consider a strategic attacker located in lane 8 (index = 7) who initially conforms to all behavioral specifications to gain trust before starting to violate the rules in the CZ which causes it to rapidly loose trust and trigger mitigation. This results in CAVs following it in same lane (lane 8) to overtake it; additionally CAVs located in other lane (lane 2) constrained by the fake CAV also jumps ahead of it in the queue as can be seen from the plot of constraint and position of the CAVs over time. Note that, although CAVs 8 and 9's position coincides in the plot however they are located in different lanes.    

%In the second scenario, we again consider a strategic adversary located in lane 8 who initially acquires trust of the coordinator before loosing it due to sending false data triggering mitigation.
 
The goal of having resilience and mitigation is to avoid accidents and minimize the effects of fake CAVs on the performance of the network (i.e. average travel time, average energy consumption). So, we compare the scenario as in Fig. \eqref{fig:intersection} without any attacks (i.e. only real CAVs) with the scenario with attacks but with and without our mitigation framework. Our empirical results in Table \ref{tb:mitigation_resilience} shows that we can alleviate the negative impact of fake CAVs on the network performance as well as guarantee safety for FP cases by reducing the average time expended by real CAVs in the intersection. 

\begin{table}
\caption{Comparison of results of the scenario without attacks with the scenario with attacks but with and without our mitigation and resilience framework.}
\label{tb:mitigation_resilience}
\begin{tabular}{|l|l|l|l|}
\hline
                                                                            & Ave. time (s) & Ave. $\frac{1}{2}u_2^2$ & Ave. Fuel  \\ \hline
\begin{tabular}[c]{@{}l@{}}Without attacks \end{tabular} & 29.60            & 8.86           & 16.66       \\ \hline
\begin{tabular}[c]{@{}l@{}}With attacks \\ with mitigation and resilience\end{tabular}    & 29.80            & 9.69           & 17.37      \\ \hline
\begin{tabular}[c]{@{}l@{}}With attacks \\ with resilience only \end{tabular}    & 32.51           & 9.07           & 15.07        \\ \hline
\end{tabular}
\end{table}



\begin{figure}[htb]
\begin{center}
 \includegraphics[scale = 0.5]{TrustValue.pdf}
\caption{Illustration of variation of trust values of fake CAVs.} %
\label{fig:TrustValue}%
\end{center}
\end{figure}  


\begin{figure}[htb]
\begin{center}
 \includegraphics[scale = 0.5]{Positions.pdf}
\caption{Illustration of variation of position of CAVs in the CZ.}
\label{fig:Positions}%
\end{center}
\end{figure}  
\fi

\section{Conclusion}
\label{conclusion}
We have presented a resilient coordination and control scheme by incorporating a trust framework that offers resilience against adversarial objectives that can be introduced by malicious attacks and uncooperative CAVs. Based on our previous study we identified two main adversarial objectives namely, (i) safety violation and (ii) creating traffic congestion in the network. We used Sybil attacks to validate and demonstrate the merit of our proposed scheme which guarantees \emph{safe coordination} and can \emph{mitigate traffic jam}. In addition, we demonstrated that our proposed robust scheduling scheme, mainly, lane-priority based rescheduling can successfully mitigate the effect of uncooperative CAVs and mitigate traffic holdups introduced by them due to the cooperative coordination scheme. %that may be deliberately driving slow to cause traffic holdup and uses lane priority to mitigate effects of such CAVs in the network. We also incorporated a trust framework to propose attack detection and mitigation scheme against Sybil attacks. %Our proposed mitigation scheme is soft (passive) and guarantees safe mitigation in the presence of false positive cases. 
Finally, we have presented results from computer simulation to validate and demonstrate the effectiveness of our proposed attack resilient control and coordination scheme for Sybil attacks, and uncooperative CAVs. 


\bibliographystyle{IEEEtran}
% Generated by IEEEtran.bst, version: 1.14 (2015/08/26)
\begin{thebibliography}{10}
\providecommand{\url}[1]{#1}
\csname url@samestyle\endcsname
\providecommand{\newblock}{\relax}
\providecommand{\bibinfo}[2]{#2}
\providecommand{\BIBentrySTDinterwordspacing}{\spaceskip=0pt\relax}
\providecommand{\BIBentryALTinterwordstretchfactor}{4}
\providecommand{\BIBentryALTinterwordspacing}{\spaceskip=\fontdimen2\font plus
\BIBentryALTinterwordstretchfactor\fontdimen3\font minus
  \fontdimen4\font\relax}
\providecommand{\BIBforeignlanguage}[2]{{%
\expandafter\ifx\csname l@#1\endcsname\relax
\typeout{** WARNING: IEEEtran.bst: No hyphenation pattern has been}%
\typeout{** loaded for the language `#1'. Using the pattern for}%
\typeout{** the default language instead.}%
\else
\language=\csname l@#1\endcsname
\fi
#2}}
\providecommand{\BIBdecl}{\relax}
\BIBdecl

\bibitem{li2013survey}
D.~W. L.~Li and D.~Yao, ``A survey of traffic control with vehicular
  communications,'' \emph{IEEE Trans. on Intelligent Transportation Systems},
  vol.~15, no.~1, pp. pp. 425--432, 2013.

\bibitem{Schrank20152015UM}
T.~L. D.~Schrank, B.~Eisele and J.~Bak, ``2015 urban mobility scorecard,''
  2015.

\bibitem{kavalchuk2020performance}
I.~Kavalchuk, A.~Kolbasov, K.~Karpukhin, A.~Terenchenko \emph{et~al.}, ``The
  performance assessment of low-cost air pollution sensor in city and the
  prospect of the autonomous vehicle for air pollution reduction,'' in
  \emph{IOP Conference Series: Materials Science and Engineering}, vol. 819,
  no.~1.\hskip 1em plus 0.5em minus 0.4em\relax IOP Publishing, 2020, p.
  012018.

\bibitem{VANDENBERG201643}
\BIBentryALTinterwordspacing
V.~A. {van den Berg} and E.~T. Verhoef, ``Autonomous cars and dynamic
  bottleneck congestion: The effects on capacity, value of time and preference
  heterogeneity,'' \emph{Transportation Research Part B: Methodological},
  vol.~94, pp. 43--60, 2016. [Online]. Available:
  \url{https://www.sciencedirect.com/science/article/pii/S0191261515300643}
\BIBentrySTDinterwordspacing

\bibitem{Liu_01}
J.~Liu, W.~Zhao, and C.~Xu, ``An efficient on-ramp merging strategy for
  connected and automated vehicles in multi-lane traffic,'' \emph{IEEE
  Transactions on Intelligent Transportation Systems}, vol.~23, no.~6, pp.
  5056--5067, 2022.

\bibitem{Xiao_03}
\BIBentryALTinterwordspacing
W.~Xiao, C.~G. Cassandras, and C.~A. Belta, ``Bridging the gap between optimal
  trajectory planning and safety-critical control with applications to
  autonomous vehicles,'' \emph{Automatica}, vol. 129, p. 109592, 2021.
  [Online]. Available:
  \url{https://www.sciencedirect.com/science/article/pii/S0005109821001126}
\BIBentrySTDinterwordspacing

\bibitem{Shukla_01}
R.~M. Shukla and S.~Sengupta, ``Analysis and detection of outliers due to data
  falsification attacks in vehicular traffic prediction application,'' in
  \emph{2018 9th IEEE Annual Ubiquitous Computing, Electronics \& Mobile
  Communication Conference}, 2018, pp. 688--694.

\bibitem{Sun_01}
X.~Sun, F.~R. Yu, and P.~Zhang, ``A survey on cyber-security of connected and
  autonomous vehicles (cavs),'' \emph{IEEE Transactions on Intelligent
  Transportation Systems}, vol.~23, no.~7, pp. 6240--6259, 2022.

\bibitem{Pham_01}
\BIBentryALTinterwordspacing
M.~Pham and K.~Xiong, ``A survey on security attacks and defense techniques for
  connected and autonomous vehicles,'' \emph{Computers \& Security}, vol. 109,
  p. 102269, 2021. [Online]. Available:
  \url{https://www.sciencedirect.com/science/article/pii/S0167404821000936}
\BIBentrySTDinterwordspacing

\bibitem{Jarouf_01}
A.~Jarouf, N.~Meskin, S.~Al-Kuwari, M.~Shakerpour, and C.~G. Cassanderas,
  ``Security analysis of merging control for connected and automated
  vehicles,'' in \emph{2022 IEEE Intelligent Vehicles Symposium (IV)}, 2022,
  pp. 1739--1744.

\bibitem{Zhao_01}
X.~Zhao, A.~Abdo, X.~Liao, M.~Barth, and G.~Wu, ``Evaluating cybersecurity
  risks of cooperative ramp merging in mixed traffic environments,'' \emph{IEEE
  Intelligent Transportation Systems Magazine}, pp. 2--15, 2022.

\bibitem{Ahmad_01}
\BIBentryALTinterwordspacing
H.~M.~S. Ahmad, N.~Meskin, and M.~Noorizadeh, \emph{Cyber-Attack Detection for
  a Crude Oil Distillation Column}.\hskip 1em plus 0.5em minus 0.4em\relax
  Cham: Springer International Publishing, 2022, pp. 323--346. [Online].
  Available: \url{https://doi.org/10.1007/978-3-030-97166-3_13}
\BIBentrySTDinterwordspacing

\bibitem{Huang_01}
S.~Huang, Y.~Feng, W.~Wong, Q.~A. Chen, Z.~Mao, and H.~Liu, ``Impact evaluation
  of falsified data attacks on connected vehicle based traffic signal control
  systems,'' 01 2021.

\bibitem{Chen_01}
Q.~A. Chen, Y.~Yin, Y.~Feng, Z.~Mao, and H.~Liu, ``Exposing congestion attack
  on emerging connected vehicle based traffic signal control,'' 01 2018.

\bibitem{Reilly_01}
\BIBentryALTinterwordspacing
J.~Reilly, S.~Martin, M.~Payer, and A.~M. Bayen, ``Creating complex congestion
  patterns via multi-objective optimal freeway traffic control with application
  to cyber-security,'' \emph{Transportation Research Part B: Methodological},
  vol.~91, pp. 366--382, 2016. [Online]. Available:
  \url{https://www.sciencedirect.com/science/article/pii/S0191261516303307}
\BIBentrySTDinterwordspacing

\bibitem{Biroon_02}
R.~A. Biroon, P.~Pisu, and Z.~Abdollahi, ``Real-time false data injection
  attack detection in connected vehicle systems with pde modeling,'' in
  \emph{2020 American Control Conference (ACC)}, 2020, pp. 3267--3272.

\bibitem{Boddupalli_01}
S.~Boddupalli, A.~S. Rao, and S.~Ray, ``Resilient cooperative adaptive cruise
  control for autonomous vehicles using machine learning,'' \emph{IEEE
  Transactions on Intelligent Transportation Systems}, pp. 1--18, 2022.

\bibitem{Farivar_01}
F.~Farivar, M.~Sayad~Haghighi, A.~Jolfaei, and S.~Wen, ``On the security of
  networked control systems in smart vehicle and its adaptive cruise control,''
  \emph{IEEE Transactions on Intelligent Transportation Systems}, vol.~22,
  no.~6, pp. 3824--3831, 2021.

\bibitem{Cheng_02}
M.~Cheng, J.~Zhang, S.~Nazarian, J.~Deshmukh, and P.~Bogdan, ``Trust-aware
  control for intelligent transportation systems,'' in \emph{2021 IEEE
  Intelligent Vehicles Symposium (IV)}, 2021, pp. 377--384.

\bibitem{Hu_01}
H.~Hu, R.~Lu, Z.~Zhang, and J.~Shao, ``Replace: a reliable trust-based platoon
  service recommendation scheme in vanet,'' \emph{IEEE Transactions on
  Vehicular Technology}, vol.~66, pp. 1--1, 01 2016.

\bibitem{Parwana_01}
H.~Parwana, A.~Mustafa, and D.~Panagou, ``Trust-based rate-tunable control
  barrier functions for non-cooperative multi-agent systems,'' in \emph{2022
  IEEE 61st Conference on Decision and Control (CDC)}, 2022, pp. 2222--2229.

\bibitem{Garlichs_01}
K.~Garlichs, A.~Willecke, M.~Wegner, and L.~C. Wolf, ``Trip: Misbehavior
  detection for dynamic platoons using trust,'' in \emph{2019 IEEE Intelligent
  Transportation Systems Conference (ITSC)}, 2019, pp. 455--460.

\bibitem{Shoukry_01}
Y.~Shoukry, S.~Mishra, Z.~Luo, and S.~Diggavi, ``Sybil attack resilient traffic
  networks: A physics-based trust propagation approach,'' in \emph{2018
  ACM/IEEE 9th International Conference on Cyber-Physical Systems (ICCPS)},
  2018, pp. 43--54.

\bibitem{Xu_02}
H.~Xu, W.~Xiao, C.~G. Cassandras, Y.~Zhang, and L.~Li, ``A general framework
  for decentralized safe optimal control of connected and automated vehicles in
  multi-lane signal-free intersections,'' \emph{IEEE Transactions on
  Intelligent Transportation Systems}, vol.~23, no.~10, pp. 17\,382--17\,396,
  2022.

\bibitem{Cheng_01}
M.~Cheng, C.~Yin, J.~Zhang, S.~Nazarian, J.~Deshmukh, and P.~Bogdan, ``A
  general trust framework for multi-agent systems,'' in \emph{Proceedings of
  the 20th International Conference on Autonomous Agents and MultiAgent
  Systems}, ser. AAMAS '21.\hskip 1em plus 0.5em minus 0.4em\relax Richland,
  SC: International Foundation for Autonomous Agents and Multiagent Systems,
  2021, p. 332–340.

\bibitem{Xiao2019}
W.~Xiao and C.~Belta, ``Control barrier functions for systems with high
  relative degree,'' in \emph{Proc. of 58th IEEE Conference on Decision and
  Control}, Nice, France, 2019, pp. 474--479.

\end{thebibliography}


\ifarvix
\section*{APPENDIX}
\textbf{Lemma 1:} The addition of new constraints due to trust-based search in Quadratic Program \eqref{QP} for any CAV $i \in S(t)$ at time $t'$ where $t' \in [t_{i}^{0},t_{i}^{f}]$ doesn't affect the feasibility of the problem \eqref{QP} at $t'$ (which is necessary for guaranteeing their satisfaction $\forall \ t\geqslant t'$ using the forward invariant property of CBFs).

$proof:$ As mentioned, for any CAV $i \in S(t)$, $S_{i,\cal{M}}(t) \subset S(t)$ is the set of indices for the CAVs that $i$ needs to stay safe to at MP $j \in m_i$. Let the trust-based search added an index corresponding to CAV $i_j (> i)$ to $S_{i,\cal{M}}$. Notice that this implies $j \in m_{i_j}$. Also notice, $i_j > S_{i,\cal{M}}(1)>1$ since, $i_j$ will cross MP $j$ before $S_{i,\cal{M}}(1)$ and $S_{i,\cal{M}}(1)$ is returned by default search mechanism and corresponds to the index of CAV that will immediately precede CAV $i$ at MP $j$. We define, $i_1 = S_{i,\cal{M}}(1)$; $b_{i,1}(x(t^{'})) = x_{i_m}(t) - x_{i,1}(t') - \varphi v_{i}(t') - \Delta$ and $b_{i,i_j}(x(t^{'})) = x_{i_m}(t) - x_{i_j}(t') - \varphi v_{i}(t') - \Delta$ are CBF constraints. Now, $b_{i,i_j}(x(t^{'})) -b_{i,1}(x(t^{'})) \geq 0$ and $b_{i,1}(x(t^{'})) \geq 0$, hence $b_{i,i_j}(x(t^{'})) \geq 0$. Hence, the addition of a new CBF constraint due to trust-based search corresponding to $i_j$ to the control of CAV $i$ in \eqref{QP} doesn't affect the feasibility at time $t'$.

\textbf{Theorem 1:} Given $0 \leq r_{i,j}(t) \leq r_{max} \ \forall t, \ \forall j \in \mathcal{B},$ the introduction of trust-based search guarantees safety preventing any collision due to adversarial agents.
 
\noindent$Proof:$ Let, adversarial CAV $i \in S(t) /\{k\}$ attempts to induce an accident to CAV $k$ in the CZ at time $t$ through using one of the attacks in section \ref{threat_model}. Firstly, notice that $k$ must be greater than $i$, else it is impossible to create an accident due to the indexing policy. At some time $t>t_i^0$, it has to violate its own constraint; else if, $i$ satisfies its own constraint, so will each CAV $i^+ \in S(t)$ (where $i^+>i$) queuing behind $i$ as long as the QP is feasible. Upon violation, two scenarios can occur. 

\noindent \underline{Case (i) $\tau_i(t) > 1-\delta$:}

Given $r_{i, j}(t) < r_{max}$,

\begin{align}
&\sum_{j \in \mathcal{B} \backslash \mathcal{B}_{a}} r_{i,j}(t) + \sum_{j \in \mathcal{B}_{a}} \prod_{k\in S_{i,\cal{M}}^a} \tau_k(t)r_{i,j}(t) \leq \left|\mathcal{B}\right| r_{max} \nonumber \\
& \therefore R_i(t) \leq\left|\mathcal{B}\right| r_{max}+\gamma R_i(t-1) \leq \frac{\left|\mathcal{B}\right| r_{max}}{1-\gamma} \nonumber \\
& \text{and}, p_{i, j}(t) \geqslant 0 \Rightarrow P_i(t) \geqslant 0 \nonumber
\end{align}

\noindent We need the trust $\tau_i < 1 - \delta$ immediately to trigger trust-based search. Hence we need to show given $\tau_i(t) > 1-\delta$, $\exists p_{i}(t+1)$ and $p_{i,j}(t+1)$ s.t. $\tau_i(t+1)<1-\delta$ i.e. $trust-based \ search$ is triggered in next iteration.

\begin{align}
& \tau_i(t+1)=\frac{R_i(t+1)}{R_i(t+1)+P_i(t+1)+h_i} < 1-\delta \nonumber \\
& \Rightarrow P_i(t+1)>\frac{R_i(t+1)}{1-\delta}-R_i(t+1) - h_i \nonumber \\
& \Rightarrow p_i(t+1)+\gamma P_i(t)>\frac{\delta}{1-\delta} \frac{|\mathcal{B}|r_{max}}{1-\gamma} \nonumber \\
& \Rightarrow p_i(t+1)>\frac{\delta}{1-\delta} \frac{|\mathcal{B}|r_{max}}{1-\gamma}\geq\frac{\delta}{1-\delta} \frac{|\mathcal{B}|r_{max}}{1-\gamma}-\gamma P_i(t)  \nonumber
\end{align}

\noindent and, when $\tau_i(t+1) < 1-\delta$,  then, CAV $k$ will stay safe from $i$, and, as well to all other CAVs that will arrive before $i$ till the first CAV with trust greater than $1- \delta$ (due to $trust-based \ search$); this will be applied for every MP CAV $k$ will travel through. This will guarantee safety for CAV $k$, thus preventing any collision.

\noindent \underline{Case (ii) $ \tau_i(t)<1- \delta$:} The same argument in the preceding paragraph apply, and hence guarantee safety for CAV $k$. A similar argument can be extended to guarantee safety for every CAV $i^+ \in S(t)$ which completes the proof.


\textbf{Lemma 2:} The rescheduled sequence is guaranteed to be feasible if $ L_1 - L_2 \geq \frac{v_{\max}^2}{2|u_{\min}|} + \Delta$, where $\Delta$ is defined as in \eqref{Safety} and \eqref{SafeMerging}.


\textit{Proof:} The maximum velocity for any CAV $i$ is $v_{\max}$, and the maximum deceleration is $|{u_{\min}}|$. Thus the minimum distance required to come to full stop for any CAV $i$ is $\frac{v_{\max}^2}{2|u_{\min}|}$. Hence, to satisfy the constraint \eqref{Safety} and \eqref{SafeMerging}, the minimum distance between the merging point and the end of the re-sequencing zone has to be greater than or equal to $\frac{v_{\max}^2}{2|u_{\min}|} + \Delta$, which will guarantee the feasibility of rescheduled sequence.

\textbf{Lemma 3:} The proposed mitigation scheme guarantees safety for real CAVs even if they are falsely identified as fake CAV due to a Sybil attack.

$Proof:$ Let us consider a CAV $i$ and CAV $k \in S_f(t)$ which is identified as FP. Notice, a collision is only possible, if $k$ is either physically preceding $i$ or will precede $i$ at any of the merging points $i$ goes through. CAV $k$ may be located in the rescheduling zone or in the intersection, each of which we consider below. Notice from Theorem 1, trust-based search guarantees safety between CAV $i$ and $k$, then when: 

i. CAV $k$ is in the re-scheduling zone: When $\tau_k \leq 1- \delta$, CAV $i$ only overtakes CAV $k$, if it doesn't observe CAV $k$ through its local perception. 

ii. CAV $k$ is in the re-scheduling zone: Similarly, when $\tau_k \leq 1- \delta$,  CAV $i$ won't ignore the CBF condition in its control and jump ahead of CAV $k$ in the intersection if it doesn't observe it through its local vision.

This makes our proposed mitigation scheme soft (or, passive) and guarantees safety for false positive cases i.e. real CAVs which have been misidentified as fake CAVs. 
\fi
%\section*{ACKNOWLEDGMENT}

\end{document}
