\section{\ours{} Implementation Details}

\subsection{Details of skill description \& clustering}
\label{sec:detail_skill_description}

\textbf{Skill Description.} To extract skill descriptions given the training dataset, we prompt GPT-4\footnote{gpt-4-32k} with its questions and answers. (The prompt is provided in \cref{fig:prompt_skill_abstraction})
Each extracted skill is written as a concise skill phrase (6–12 words), preserving the core visual or temporal reasoning concept. 
Here, we intentionally exclude audio-based cues (e.g., sound, speech, or music) in this process.
Specific object names (e.g., "TV", "sofa", "John") are replaced with generic terms, and vague terms (e.g., "reasoning", "analysis") are avoided to enhance clarity. 
We also provide the exact name of the skills in \cref{tab:skill_list}. 


\subsection{Details of skill-based CoT generation}
\label{sec:detail_cot_generation}
For skill-based CoT generation, we utilize Gemini-2.0 Flash with video input. As illustrated in \cref{fig:prompt_gemini}, we first prompt Gemini-2.0 to identify the relevant skills and generate corresponding sub-questions and answers. Then, we construct step-by-step reasoning based on this output. Finally, we use GPT-4 to filter and verify the reasoning by assessing its relevance to the ground-truth answers using \cref{fig:prompt_filtering} as a prompt.



\begin{table*}[ht]
\centering 
\small 
\renewcommand{\arraystretch}{1.3}
\resizebox{.7\textwidth}{!}{%
\begin{tabular}{c|l}
\hline
 & \multicolumn{1}{c}{Skill descriptions} \\ \hline
VSI-Bench & \begin{tabular}[c]{@{}l@{}}"Determine object location relative to a person's orientation.",\\     "Locate specific objects and identify their positions in the scene.",\\     "Identify the closest object to a reference point.",\\     "Recognize objects based on shape, color, and features.",\\     "Identify and count distinct objects based on location and appearance.",\\     "Assess spatial proximity of objects relative to a reference point.",\\     "Identify initial appearances of objects in the video timeline.",\\     "Estimate distance between two objects using visual cues.",\\     "Determine room boundaries using structural elements like walls and floors.",\\     "Identify sequential order of object appearances in a scene."\end{tabular} \\ \hline

ET-Bench & \begin{tabular}[c]{@{}l@{}}"Identify head orientation and gaze direction to infer focus.",\\    "Identify spatial proximity between people in the scene.",\\    "Identifying the timestamp of an action in a scene.",\\    "Detect object using shape, texture, and visual features.",\\    "Track individuals interacting with an object and their actions.",\\    "Identify a person performing an action involving an object.",\\    "Identifying the moment an action occurs in a scene.",\\    "Detect a person using body shape, face, and clothing.",\\    "Identify actions through body movements and postures.",\\    "Identify hand movement and physical interaction with an object."\end{tabular} \\  \hline

CinePile & \begin{tabular}[c]{@{}l@{}}"Identify thematic parallels between actions and overarching narrative themes.",\\     "Inferring emotional tone from facial expressions and actions.",\\     "Tracking emotional shifts through expressions and body language changes.",\\     "Identifying interpersonal conflict through observed actions and interactions.",\\     "Inferring symbolic meaning of an object in a scene.",\\     "Inferring emotional state from expressions and body language.",\\     "Track a person's movements and reactions to scene changes.",\\     "Identifying body language among the scene.",\\     "Identify event sequence to infer action context and significance.",\\     "Identifying the main person or subject in the scene."\end{tabular} \\ \hline
\end{tabular}%
}
\caption{\textbf{Detailed skill descriptions from three datasets.} }
\label{tab:skill_list}
\end{table*}




\subsection{Details of training}
\label{sec:detail_training}

\textbf{Training datasets.} 
Instead of using the full video instruction tuning dataset, we randomly sampled 10k and 2.1k examples from ET-Bench and CinePile, respectively. For VSI-Bench, which is intended solely for evaluation and does not provide a training set, we manually split the available data into training and test sets using a 7:3 ratio. We use 3k training dataset for VSI-Bench. 

\noindent\textbf{Hyperparameters.}
For training, we set the learning rate as 1e-5 and the batch size as $1$. For LoRA, we use rank $32$. 
We set 1 epoch for ET-Bench training and 3 epochs for the other two datasets. 
For other parameters, we use the default setup of LLaVA Video. 
We use 4 A6000 GPUs for training. 



\subsection{Prompts}
\label{sec:prompt}
In \cref{fig:prompt_skill_abstraction,fig:prompt_gemini,fig:prompt_merging,fig:prompt_filtering}, we attach prompts for skill-based CoT annotation. 
We also attach prompt to generate regular CoT in \cref{fig:prompt_regular_cot_generation}.



\begin{table*}[ht]
\centering 
\small 
\renewcommand{\arraystretch}{1.2}
\resizebox{.95\textwidth}{!}{
\begin{tabular}{l|cccccccc|c}
\toprule
 & Appr. Order & Rel. Dist. & Route Plan & Rel. Dir. & Abs. Dist. & Obj. Count & Obj. Size & Room Size & Avg \\
 \midrule
LLaVA-Video & 31.28 & 41.57 & 35.59 & 46.05 & 10.78 & 52.24 & 47.54 & 19.77 & 35.60 \\
LLaVA-Video (fine-tune) & 54.16 & 36.97 & 41.66 & 42.26 & 29.71 & 56.89 & 70.16 & 54.51 & 47.45 \\
Single-LoRA + Regular-CoT & 68.75 & 36.97 & 41.66 & 43.07 & 31.37 & 62.62 & 70.22 & 65.96 & 52.58 \\
Single-LoRA + Skill-CoT & 64.58 & 42.01 & 36.11 & 44.45 & 33.11 & 56.89 & 74.27 & 64.35 & 52.97 \\
Multi-LoRA + Regular-CoT & 56.25 & 34.45 & 41.66 & 39.53 & 12.24 & 43.00 & 55.11 & 23.87 & 38.27 \\ 
\midrule
\rowcolor{lightblue}
\ours{} (Ours) & 68.75 & 36.13 & 50.00 & 47.69 & 32.13 & 62.61 & 70.83 & 57.09 & 53.15 \\

\bottomrule

\end{tabular}
}
\caption{\textbf{Detailed VSI-Bench Results}.}
\label{tab:detail_vsi}
\end{table*}



\begin{table*}[ht]
\centering 
\small 
\renewcommand{\arraystretch}{1.2}
\resizebox{.95\textwidth}{!}{
\begin{tabular}{l|cccccccccccccc|c}
\toprule
 & RAR & EVC & RVQ & TVG & ERM & TAL & EVS & VHD & DVC (F1) & DVC (Sim) & SLC (F1) & SLC (Sim) & TEM & GVQ & Avg \\
 \midrule
LLaVA-Video & 41.6 & 38.8 & 56.6 & 8.2 & 1.8 & 14.0 & 14.8 & 28.2 & 20.1 & 10.0 & 11.5 & 8.1 & 15.7 & 1.5 & 19.3 \\
LLaVA-Video (fine-tune) & 44.8 & 34.6 & 58.2 & 9.4 & 2.2 & 12.0 & 9.2 & 29.7 & 28.6 & 14.8 & 10.1 & 10.2 & 18.6 & 2.1 & 20.3 \\
Single-LoRA + Regular-CoT & 43.6 & 37.8 & 58.6 & 10.7 & 1.9 & 13.7 & 7.5 & 33.8 & 14 & 12.2 & 6.5 & 10.6 & 12.4 & 3.1 & 19.0 \\
Single-LoRA + Skill-CoT & 43.2 & 40.8 & 56.8 & 10.0 & 1.7 & 15.2 & 7.1 & 30.2 & 17.3 & 16.5 & 10.0 & 10.6 & 11.2 & 1 & 19.4 \\
Multi-LoRA + Regular-CoT & 43.2 & 40.6 & 59.8 & 6.9 & 1.9 & 16.5 & 11.1 & 31.1 & 30.3 & 16.4 & 6.2 & 9.5 & 16.6 & 1.0 & 20.7 \\
\midrule
\rowcolor{lightblue}
\ours{} (Ours) & 49.0 & 41.2 & 59.4 & 15.8 & 2.4 & 16.4 & 8.4 & 35.5 & 27.0 & 15.0 & 11.9 & 10.0 & 16.5 & 2.4 & 22.2 \\
\bottomrule
\end{tabular}
}
\caption{\textbf{Detailed E.T-Bench Results}.}
\label{tab:detail_et}
\end{table*}

\begin{table*}[ht]
\centering 
\small 
\renewcommand{\arraystretch}{1.2}
\resizebox{.65\textwidth}{!}{
\begin{tabular}{l|ccccc|c}
\toprule
 & CRD & NPA & STA & TH & TEMP & Avg \\
 \midrule
LLaVA Video & 56.78 & 58.53 & 60.31 & 60.52 & 43.02 & 55.83 \\
LLaVA Video (fine-tune) & 57.74 & 58.32 & 60.44 & 61.05 & 43.90 & 56.29 \\
Single-LoRA + Regular CoT & 59.18 & 59.61 & 61.81 & 61.05 & 40.91 & 56.11 \\
Single-LoRA + Skill-CoT & 57.88 & 56.8 & 60.77 & 60.52 & 40.84 & 56.36 \\
Multi-LoRA + Regular CoT & 58.17 & 59.17 & 60.18 & 62.63 & 42.44 & 56.52 \\ 
\midrule
\rowcolor{lightblue}
\ours{} (Ours) & 60.00 & 59.61 & 61.44 & 63.15 & 45.20 & 57.88 \\
\bottomrule

\end{tabular}
}
\caption{\textbf{Detailed Cinepile Results.} We ablate the accuracies across the question categories: TEMP
- Temporal, CRD - Character and Relationship Dynamics, NPA - Narrative and Plot Analysis, STA - Setting and
Technical Analysis, TH - Thematic Exploration. 
}
\label{tab:detail_cine}
\end{table*}


\section{Additional Quantitative Results}
\label{sec:add_quant_results}

\subsection{Per-category performance}
\label{sec:per_category}
In \cref{tab:detail_vsi,tab:detail_cine,tab:detail_et}, we additionally report the per-category performance for each dataset. We also include ablation studies comparing regular CoT vs skill-based CoT, and single-LoRA vs multi-LoRA configurations. \ours{}, which combines skill-based CoT with multi-LoRA training, consistently outperforms across all datasets, showing particularly strong gains on reasoning-intensive tasks such as Route Planning in VSI-Bench and temporal understanding tasks in CinePile.

\subsection{Qwen2.5-VL backbone}
\label{sec:qwen}

\begin{table*}[ht]
\centering 
\small 
\renewcommand{\arraystretch}{1.2}
\resizebox{.95\textwidth}{!}{
\begin{tabular}{l|cccccccc|c}
\toprule
\textbf{Model} & \textbf{Appr. Order} & \textbf{Rel. Dist.} & \textbf{Route Plan} & \textbf{Rel. Dir.} & \textbf{Abs. Dist.} & \textbf{Obj. Count} & \textbf{Obj. Size} & \textbf{Room Size} & \textbf{Avg} \\
\midrule
Qwen2.5-VL & 10.41 & 35.29 & 36.11 & 37.80 & 20.43 & 23.98 & 54.22 & 33.06 & 31.41 \\
Qwen2.5-VL (fine-tune) & 18.75 & 34.45 & 36.11 & 38.65 & 21.88 & 31.74 & 62.55 & 47.74 & 36.48 \\
Single-LoRA + Regular-CoT & 20.83 & 35.29 & 36.11 & 39.92 & 21.73 & 34.56 & 63.33 & 48.06 & 37.48 \\
Single-LoRA + Skill-CoT & 27.08 & 35.61 & \textbf{41.66} & 40.12 & 21.57 & 35.33 & 65.77 & \textbf{48.58} & 39.47 \\
\midrule
\rowcolor{lightblue}
Ours (Multi-LoRA + Skill-CoT) & \textbf{31.28} & \textbf{37.89} & 36.50 & \textbf{43.67} & \textbf{28.12} & \textbf{37.48} & \textbf{67.31} & 43.79 & \textbf{40.76} \\
\bottomrule
\end{tabular}
}
\caption{\textbf{Qwen2.5-VL (7B) Results on VSI-Bench}. }
\label{tab:qwen}
\end{table*}

We further evaluate \ours{} on VSI-Bench using the Qwen2.5-VL (7B)~\cite{bai2025qwen2} backbone. As shown in \cref{tab:qwen}, \ours{} achieves the highest overall performance (40.76 avg), consistently surpassing both regular-CoT and single-LoRA baselines. These results highlight the robustness and effectiveness of \ours{} when applied to a different backbone architecture.

\begin{table*}[t]
\centering 
\small 
\renewcommand{\arraystretch}{1.2}
\resizebox{.70\textwidth}{!}{
\begin{tabular}{l|c|ccccc|c}
\toprule
\textbf{Setting} & \textbf{Training Data} & \textbf{CRD} & \textbf{NPA} & \textbf{STA} & \textbf{TH} & \textbf{TEMP} & \textbf{Avg} \\
\midrule
LLaVA-Video & - & 56.78 & 58.53 & 60.31 & 60.52 & 43.02 & 55.83 \\
LLaVA-Video (fine-tune) & CinePile & 57.74 & 58.32 & 60.44 & 61.05 & 43.90 & 56.29 \\
LLaVA-Video (fine-tune) & ET-Bench & 53.34 & 57.02 & 57.83 & 60.53 & 40.55 & 53.85 \\
Single-LoRA + \textbf{Regular-CoT} & ET-Bench & 56.23 & 60.47 & 58.09 & 63.68 & 38.66 & 55.43 \\
\midrule
\rowcolor{lightblue}
Multi-LoRA + \textbf{Skill-CoT (Ours)} & ET-Bench & 56.02 & 61.25 & 58.88 & 63.33 & 41.57 & \textbf{56.21} \\
\bottomrule
\end{tabular}
}
\caption{\textbf{Cross-dataset evaluation on CinePile}. Source: ET-Bench $\rightarrow$ Target: CinePile. Backbone: LLaVA-Video. 
}
\label{tab:cross}
\end{table*}

\subsection{Cross-dataset generalization}
\label{sec:cross_dataset}
We evaluate cross-domain generalization from ET-Bench (source) to CinePile (target). As shown in \cref{tab:cross}, \ours{} achieves the best average performance (56.21) among ET-Bench–trained variants, performing competitively with the CinePile fine-tuned model (56.29) and surpassing the zero-shot baseline (55.83). 
This highlights the effectiveness of skill-guided reasoning for transfer across domains.

\begin{figure*}[t]
    \centering
    {
    \includegraphics[width=\textwidth]{figures/selected_skills1.pdf}}
    \caption{
    \textbf{Skill selection results of VSI-Bench (1)}
    }
    \label{fig:selected_skills1}
\end{figure*}

\begin{figure*}[t]
    \centering
    {
    \includegraphics[width=\textwidth]{figures/selected_skills2.pdf}}
    \caption{
    \textbf{\textbf{Skill selection results of VSI-Bench (2)}}
    }
    \label{fig:selected_skills2}\vspace{-0.1in}
\end{figure*}


\section{Additional Qualitative Results}
\label{sec:add_qual_results}

\subsection{Skill descriptions over different datasets}

In \cref{fig:skill_description_tsne}, we visualize the skill descriptions for each dataset after performing skill extraction and clustering (\cref{sec:data_collection}). 
To create the visualization, we first obtain text embeddings using SentenceTransformer and compute $N^{\text{skills}}$ cluster centroids. 
We then apply t-SNE to reduce the dimensionality of the embeddings for visualization purposes. 
The results highlight that each domain-specific dataset emphasizes different skill sets, though certain skills are shared across datasets. For instance, the skill “\textit{Inferring emotional tone from facial expressions and actions}” from CinePile is distinct from “\textit{Estimating distance between two objects in the video timeline}” from VSI-Bench. However, general skills like “\textit{Identifying objects or people}” appear across multiple datasets.
A more detailed list of the extracted skills is provided in \cref{tab:skill_list}. 

\subsection{Selected skills over different video datasets}

In \cref{fig:selected_skills1,fig:selected_skills2}, we present statistics on the selected top 3 assigned skills for each task in VSI-Bench (presented in \cref{sec:data_collection}). As shown in the results, object identification skills are commonly used across tasks. However, each task also requires domain-specific skills. For instance, the Room Size Estimation task necessitates skills such as “\textit{Determining room boundaries using structural elements like walls and floors.}”


\subsection{Additional comparison with regular CoT}
In \cref{fig:additional_comparison_regular_cot_generation}, we provide additional comparison with regular CoT and ours. 


\section{License}
We list the license of the benchmark dataset and models we used. We use these existing artifacts consistently with their intended use. 
\begin{itemize}
    \item LLaVA-Video: \href{https://github.com/LLaVA-VL/LLaVA-NeXT/blob/main/LICENSE}{Apache License 2.0} 
    \item CinePile: \href{https://huggingface.co/datasets/tomg-group-umd/cinepile}{cc-by-nc-sa-4.0}
    \item VSI-Bench: \href{https://huggingface.co/datasets/choosealicense/licenses/blob/main/markdown/apache-2.0.md}{Apache License 2.0}
    \item ET-Bench: \href{https://huggingface.co/datasets/PolyU-ChenLab/ET-Instruct-164K}{cc-by-nc-sa-4.0}
    
\end{itemize}


\begin{figure*}[t]
    \centering
    {
    \includegraphics[width=\textwidth]{figures/qual_example3.png}}
    \caption{
    \textbf{\textbf{Additional comparison of CoT annotations: (a) regular CoT and (b) our skill-based CoT.}}
    }
    \label{fig:additional_comparison_regular_cot_generation}\vspace{-0.1in}
\end{figure*}



\begin{figure*}[t]
    \centering
    {
    \includegraphics[width=\textwidth]{figures/skill_description_by_datasets_tsne.pdf}}
    \caption{
    \textbf{Skill description from different domain datasets.} We visualize the skill descriptions for each dataset after performing skill extraction and clustering. (\cref{sec:data_collection})
    }
    \label{fig:skill_description_tsne}\vspace{-0.1in}
\end{figure*}


\begin{figure*}[t]
    \centering
    {
    \includegraphics[width=\textwidth]{figures/prompt_skill_abstraction.pdf}}
    \caption{
    \textbf{\textbf{Prompt for Skill Description}}
    }
    \label{fig:prompt_skill_abstraction}\vspace{-0.1in}
\end{figure*}

\begin{figure*}[t]
    \centering
    {
    \includegraphics[width=\textwidth]{figures/prompt_gemini.pdf}}
    \caption{
    \textbf{\textbf{Prompt for skill selection and sub-QA generation}}
    }
    \label{fig:prompt_gemini}\vspace{-0.1in}
\end{figure*}


\begin{figure*}[t]
    \centering
    {
    \includegraphics[width=\textwidth]{figures/prompt_merging.pdf}}
    \caption{
    \textbf{\textbf{Prompt for skill-based CoT generation}}
    }
    \label{fig:prompt_merging}\vspace{-0.1in}
\end{figure*}


\begin{figure*}[t]
    \centering
    {
    \includegraphics[width=\textwidth]{figures/prompt_verifiy.pdf}}
    \caption{
    \textbf{\textbf{Prompt for CoT filtering}}
    }
    \label{fig:prompt_filtering}\vspace{-0.1in}
\end{figure*}


\begin{figure*}[t]
    \centering
    {
    \includegraphics[width=\textwidth]{figures/prompt_normal_cot.pdf}}
    \caption{
    \textbf{\textbf{Prompt for regular CoT generation}}
    }
    \label{fig:prompt_regular_cot_generation}\vspace{-0.1in}
\end{figure*}
