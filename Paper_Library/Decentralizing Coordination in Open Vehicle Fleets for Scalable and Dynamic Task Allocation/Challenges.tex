\section{Challenges in open vehicle fleet coordination}\label{Challenges}
In this paper, we proposed new mathematical programming models of dynamic
versions of the following assignment problems well-known in combinatorial optimization and applicable in open vehicle fleets:
the assignment problem, bottleneck assignment problem, fair matching problem, dynamic minimum deviation assignment problem,  $\sum_{k}$- assignment problem, the semi-assignment problem, the assignment problem with side constraints,  and the assignment problem while recognizing agent   qualification.
%
The goal of the studied  problems   is finding an optimal (minimum cost or maximum profit) assignment to the (vehicle) agents of the   tasks that are known at the time of decision-making. These approaches do not take into account unknown tasks that may appear once when the current tasks are completed.


With the long term objective of decentralizing and democratizing shared mobility, we categorized solution approaches for static and dynamic task assignment problems applicable in open vehicle fleets  into centralized, distributed, and decentralized and discussed their main characteristics.
%
The presented distributed and decentralized task assignment methods are   applicable in distributed and decentralized open vehicle fleets, respectively. In case of decentralized fleets, the issues related with privacy, trust, and control intrinsic to centralized systems are gone.

We focused on homogeneous vehicle agents and tasks, i.e., each vehicle agent is able to complete each task with equal efficiency but varying cost or profit. In the real world that might not be the case since in open vehicle fleets, the vehicles tend to be heterogeneous. The proposed mathematical programs can easily be adapted to this case by varying the agent-task assignment cost/profit depending on the performance efficiency of an agent; in case of an agent inapt to perform a task, its agent-task assignment cost is assigned a very large value.

With fully decentralized scalable coordination of task allocation, there is no need to put limits to the size of the system. However, even though scalable task allocation and related coordination mechanisms are essential for efficiently managing large-scale open vehicle fleet systems, it should be noticed that, for real-world applications, they  need to be complemented with scalable and efficient solution approaches to other combinatorial optimization problems depending on the context, e.g., dial-a-ride problem, traveling salesperson problem, etc.

We dealt with the deterministic and dynamic assignment problem where real-time reassignment is beneficial since both the variables and parameters of the optimization problem are perfectly known at each period. However, when dealing with real-world stochastic environments with increased sensor noise, a too high frequency of task re-assignment may result in a churning effect in the assignment and may lead to increased human errors. Thus, a chosen coordination method must consider churning in this context to obtain good overall task allocation performance (see, e.g., \cite{alighanbari2008robust}).

A truly open vehicle fleet system should work also based on heterogeneous software agents produced by multiple producers. The agent software could be an open source and/or there may be multiple proprietary software companies working on a common open fleet coordination standard.
The Agreement Technologies (AT) paradigm \cite{791} identifies and relates various such technologies. It provides a sandbox of mechanisms to support coordination among (heterogeneous) autonomous software agents, which focuses on the concept of agreement between them. To this respect, AT-based systems not only support the interactions for reaching agreement in a coordinated manner (e.g. as part of a distributed or decentralized algorithm) but are also endowed with means to specify and govern the ``space'' of agreements that can be reached, as well as monitoring agreement execution. In particular, in truly open vehicle fleet systems where there may be a multitude of (possibly heterogeneous) software providers, semantic mismatches among vehicle agents need to be dealt with through the alignment of ontologies, so that vehicle agents can reach a common understanding on the elements of agreements.

Furthermore, (weak) constraints on agreement and agreement processes (often also called \emph{norms}) need to be defined and represented in a declarative manner, so autonomous agents can decide as to whether they will adopt them, determine as to how far they are applicable in a certain situation, dynamically generate priorities among conflicting norms depending on the context, etc. In addition, trust and reputation models are necessary for keeping track of whether the agreements reached, and their executions, respect the requirements put forward by norms and organisational constraints. So, norms and trust can be conceived as a priori and a posteriori approaches, respectively, to support the security in relation to the coordination process. How to find seamless and effective means of integrating the different distributed and decentralized algorithms outlined in this paper in such a framework is still an open issue that we will treat in our future work.


The presented distributed and decentralized coordination methods  for dynamic task assignment may be applied to semi-autonomous and autonomous vehicles and are  a necessary part of reaching full vehicle fleet autonomy. They may not fix the mobility concerns, but they will definitely improve them as they are directly related to giving a higher control both to an individual driver (or to an autonomous vehicle) and to a customer (rider).  Intrinsically, these methods aid in changing the hierarchical tree structure of the transportation networks to a more horizontal one. Indirect benefits of such coordination methods, among others, include  higher efficiency, smaller carbon footprints and less traffic jams.
In the long run, they will facilitate   more decentralized, autonomous, and transparent open vehicle fleets, but above all, they will further  the task allocation efficiency and fair rewards and benefits of vehicles, drivers, customers, and riders  proportional to  their participation  in large and open fleets.


