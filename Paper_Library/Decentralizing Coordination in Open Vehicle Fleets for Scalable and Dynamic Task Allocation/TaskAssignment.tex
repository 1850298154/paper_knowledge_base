\section{Task assignment models for open vehicle fleets}\label{taskAssignment}
Assignment problems (APs) are among the earliest optimization problems studied in the operations research field. They involve optimally matching the elements of two or more sets, where the
dimension of the problem refers to the number of sets to be matched \cite{pentico2007assignment}.
For example, in two-dimensional assignment problems,  given is a set of agents $A$ and a set of tasks $T$ and we have to match (assign) tasks to agents. Tasks are assumed atomic, i.e., each task cannot be decomposed into subtasks and it can be completed by a single vehicle.
%
In general, two-dimensional assignment problems can be solved in polynomial time, while  $d$-dimensional assignment problems, with $d > 2$, in general are NP-hard  (see, e.g., \cite{burkard1999linear}).

We distinguish between the static and dynamic assignment problems (see, e.g., \cite{spivey2004dynamic}). The former refer to the assignment of a  set of tasks to a set of agents in a given static environment in which the problem data does not change  during the planning horizon, while in the dynamic task assignment problems, both agents and tasks may appear and disappear dynamically over time.
In the open vehicle fleet setting, agents  can be in one of the following three states: \textit{idle}, \textit{assigned} without still having reached the customer, or \textit{assisting a customer}, and only idle and assigned agents that have still not reached their customers can be (re)assigned to unassisted tasks. In general,  agents are assumed renewable, i.e., after completing a task, an agent's state changes from \textit{assisting a customer} to  \textit{idle} and it becomes assignable again to customers (tasks) that have not been assisted yet. This  is a special case of a more general computationally complex dynamic vehicle routing problem (DVRP) in which, for each (vehicle) agent, we find  a minimum cost route that visits a  dynamically changing set of  tasks (customers) \cite{pillac2013review}.
%
Due to the high computational complexity, myopic algorithms are the most usual solution approaches for DVRP. For simplicity, we can assume that agents are nonrenewable, i.e., an agent can be assigned only to one task; if, after completing a task, it is still available for new task assignment, it appears as a new agent.

The static and deterministic AP is a computationally easy problem, which allows us (in theory) to find an optimal solution in close-to real-time  (in the nonrenewable agent case).
%
Dynamic AP can be solved by (suboptimal) myopic approaches that consider only the information available at the present time with no consideration for future events and possibly reassign tasks among idle and already assigned agents to improve the system's efficiency (see, e.g., \cite{billhardt2014dynamic,lujak2016distributed,billhardt2014dynamic2,lujak2013coordinating}).
%
However, in the case where tasks are not randomly appearing, this approach can be significantly improved by considering future developments.

\subsection{Static Task Assignment}
Based on the categorization of the  AP models presented in \cite{pentico2007assignment}, in this section, we consider  the classic assignment problem and its variations relevant in the open fleet vehicle task assignment considering self-interested and individually rational vehicle users whose tasks can be performed simultaneously: the classic linear assignment problem (LAP), assignment problem recognizing agent qualification (APRAQ), the bottleneck assignment problem (BAP), the fair matching problem (FMP), the minimum deviation assignment problem (MDAP), the $\sum_k$-assignment problem ($\sum_k$-AP), the semi-assignment problem (SAP), and the assignment problem with side constraints (APSC). In Figure \ref{Framework}, we give a framework for easier understanding of the characteristics of both the static and dynamic version of these problems.

For self-completeness of this article,  we bring in the following the descriptions of these problems. Considering that the number of publications concerning assignment problems is enormous, the
references in this section constitute only a very limited part of them.
%
For the details and other assignment problem variations the reader is referred to \cite{pentico2007assignment}.

\begin{figure}[ht]
\begin{center}
\includegraphics[width=0.8\columnwidth]{Pictures/Framework.png}
\caption{Static and dynamic task assignment problems in open vehicle fleets}
\label{Framework}
\end{center}
\end{figure}

\paragraph{Classic (linear) assignment problem (LAP)}
The static classic linear assignment problem involves two sets of the same size and consists of finding, in a weighted complete bipartite graph, a perfect matching in which the sum of weights of the matched edges is as low as possible, i.e., a \textit{minimum-weight perfect matching}. Perfect weighted matching implies that each node must be matched to some other node by minimizing the total cost of the arcs in the (perfect) matching.

The classic linear assignment problem (LAP) can be defined  as follows: Given a weighted complete bipartite graph $G = (A \cup T, E)$ with two vertex sets $A$ and $T$, with $n = |A| = |T|$, and an edge set $E = A \times T$, with edge weights $c_{ij}$ on edge $(i,j) \in E$,
find a minimum weight perfect matching of $G$, i.e., a perfect matching among vertices in $A$ and vertices in $T$ such that the sum of the costs of the matched edges is minimum. An edge $(i,j) \in E$ is matched if  two extreme vertices $i$ and $j$ are mutually matched, and a matching is perfect if every vertex $i$ of $A$ is matched (assigned) exactly to one vertex $j$ of $T$, and viceversa.
The LAP is equivalent to the weighted bipartite matching, since we may assume that the bipartite graph is always complete by letting the weights of the edges that are missing being sufficiently large.
%
If $|A| \neq |T|$,  we can add a number of dummy nodes to the set with lower cardinality and connect them by dummy arcs of zero cost to the other set. The number of dummy nodes should be  sufficient to balance the cardinalities of the two sets.

The LAP is equivalent to the maximum weighted bipartite matching (with edge weights $w_{ij} \geq 0$), since we may assume that the bipartite graph is always complete by letting the weights of the edges that are missing being sufficiently large. Furthermore, also in this case we can assume that the two vertex sets of the bipartite graph have the same size. At this point we can reformulate the problem as a minimization problem by considering costs $c_{ij} = W - w_{ij}$, where $W$ is larger than the maximum of the $w_{ij}$, and hence this problem corresponds to the LAP.

The LAP is a special case of the transportation problem assuming an equal  number of supplier agents and customer agents and each one with their unitary  supply and unitary demand, respectively. The transportation problem is one of special cases of the minimum cost flow problem together with, e.g., the shortest path problem and the max flow problem. While it is possible to solve this problem using the Simplex algorithm, specialized  algorithms   take advantage of its special network structure and are thus more efficient.

From the multi-agent systems' point of view, in the assignment problem, a number of agents need to be assigned to a number of tasks based on the given cost of agent-task assignment. In general, each agent can be assigned to any task. In  case an agent is not capable of performing a task, a given agent-task assignment cost is modelled as a very large number.  All tasks should be performed with the objective to minimize the   total cost of the assignment such that exactly one agent is assigned to each task and  exactly one task  to each agent.
%
The mathematical formulation of the problem is:
\begin{equation}\label{APobjective}
\min \sum_{i,j}c_{ij}x_{ij}
\end{equation}
subject to
\begin{equation}\label{tasks}
\sum_{i=1}^n x_{ij}=1,\;\;\; \forall\; j \in T,
\end{equation}
and
\begin{equation}\label{agents}
\sum_{j=1}^n x_{ij}=1, \; \; \; \forall\; i \in A,
\end{equation}
\begin{equation}\label{non-negat}
    x_{ij} \in \{0,1\}, \; \; \; \; \forall\; i \in A,\;j \in T.
\end{equation}

Constraints (\ref{tasks}) ensure that every
task is assigned to only one agent and constraints (\ref{agents}) ensure that every agent is assigned
to only one task.

The structure of the problem, i.e., the total unimodularity of the constraint matrix, makes the binary requirements on the variables unnecessary. In fact, in this case, it can be proven that the linear relaxation has always an optimal binary solution (see, e.g., \cite{lawler2001combinatorial,papadimitriou1982combinatorial}) and, therefore, the LAP is a linear programming (LP) problem.

\paragraph{The classic assignment problem recognizing agent qualification (APRAQ)}

Caron et al. in \cite{caron1999assignment} propose a mathematical model in which not every agent is qualified
to do every task, and the objective is utility maximization:
\begin{equation}
\max \sum_{i,j}p_{ij}x_{ij}
\end{equation}
subject to
\begin{equation}\label{qualifiedTasks}
\sum_{i\in A} q_{ij} x_{ij} \leq 1,\;\;\; \forall\;j \in T\;,
\end{equation}
and
\begin{equation}\label{qualifiedAgents}
\sum_{j\in T} q_{ij} x_{ij}\leq 1, \; \; \; \forall\; i \in A\; ,
\end{equation}
\begin{equation}
    x_{ij}\in \{ 0,1\}\;, \forall i \in A, \;j \in T,
\end{equation}
where parameter $q_{ij}$ = 1 if agent $i$ is qualified to perform task
$j$, 0 otherwise, parameter $p_{ij}$ is the utility of assigning agent
$i$ to task $j$ (with $p_{ij}$ = 0 if $q_{ij}$ = 0), and variable
$x_{ij}$ = 1 if agent $i$ is assigned to task $j$, 0 otherwise.
%
Constraints (\ref{qualifiedTasks}) ensure that no more than one qualified agent is assigned to any task, while  constraints (\ref{qualifiedAgents}) guarantee that each agent is assigned
to not more than one task.

The classic assignment problem does not consider fairness.  The solution of classic AP (\ref{APobjective})-(\ref{non-negat})  maximizes utilitarian social welfare (see, e.g., \cite{moulin2004fair}), but it may be unfair and unsatisfactory since there may be one or more agents with a much higher  task cost than the rest. This is why it is best applied to centralized open vehicle fleets with a single owner of the fleet's vehicles that is interested in the minimization of the overall cost of the fleet's operation costs but not in how they are distributed among the vehicles.

\paragraph{Bottleneck assignment problem (BAP)}

To resolve the issues with fairness and workload distribution, we may  minimize maximum cost among the individual agent-task assignments and thus maximize the system's egalitarian social welfare (see, e.g., \cite{burkard2009}). The mathematical program for the BAP is as follows:
Minimize $\max_{i,j}\{c_{ij}x_{ij}\}$ or minimize $\max_{i,j} \{c_{ij}| x_{ij}=1\}$
subject to constraints (\ref{tasks})--(\ref{non-negat}) and definitions of
 the LAP.

Note that here the integrality requirements cannot be relaxed.
Contrary to the classic AP model, the BAP  model pursues the objective of fairness among agents.
%
It is based on the optimization of the worst-off performance and provides a good solution when the minimum requirements of all agents should be  satisfied. However, only the most costly agent-task assignment influences the objective function, while the contribution of  the rest  of the agents is ignored. For this reason, this approach deteriorates the system efficiency and thus, the system's utilitarian social welfare.

\paragraph{The  fair matching problem (FMP)}

The fair matching problem  minimizes the difference between the maximum and minimum assignment values \cite{martello1984balanced}:

Minimize $\max_{i,j}\{c_{ij}|x_{ij}=1\}-\min_{i,j}\{c_{ij}|x_{ij}=1\}$
subject to the same constraints and definitions as in the classic AP.

This formulation of fairness is not unique. Sun and Yang (2003) in \cite{sun2003general} study the concept of fair and optimal allocations.
They define an allocation to be  fair and optimal if it is envy-free and the sum of compensations is maximized, subject to the compensation assigned to each object is less than or equal to the maximum compensation limit. They prove that fair and optimal allocations exist and demonstrate that the fair and optimal allocation mechanism achieves efficiency, fairness and strategy-proofness simultaneously. \cite{andersson2009general} demonstrates that it is also coalitionally strategy-proof, i.e., it is not possible for any agent or any coalition of agents to successfully manipulate the allocation rule.

\paragraph{The minimum deviation assignment problem (MDAP)}

The objective here is to minimize the difference between the maximum and average assignment costs:\\

\begin{equation}
Minimize \;\;\min\{n,m\}\times \max_{p,q}\{c_{pq}x_{pq}\}-\sum_{i=1}^n \sum_{j=1}^m c_{ij}x_{ij}
\end{equation}
or  to minimize the difference between the average  and minimum assignment profit:
\begin{equation}
Minimize \;\;\sum_{i=1}^n \sum_{j=1}^m p_{ij}x_{ij}-\min\{n,m\}\times \min_{s,t}\{p_{st}x_{st}\},
\end{equation}
subject to constraints (\ref{tasks})--(\ref{non-negat}). Here, $n$ is cardinality of agent set $A$, and $m$ of task set $T$, other  definitions are the same as in the LAP \cite{duin1991minimum,gupta1988minimum}.

\paragraph{The $\sum_k$-assignment problem ($\sum_k$-AP)}

Since there may be generally multiple different sets of assignments with the same minimum value for $\max\{c_{ij}x_{ij}\}$, the objective here is to find a set of assignments for
which the sum of the $k$ largest values is minimized. The BAP and LAP can be viewed as special cases of $\sum_k$-AP with $k=1$ and $k=n$, respectively.

A recent study on generic mixed integer problem with $\sum_k$ optimization is done by Filippi {\em et al.} \cite{filippi2019bridging}.



\paragraph{The semi-assignment problem (SAP)} This is the version of the assignment problem where every agent or task may not be unique. This results in a constraint matrix containing a number of rows or columns with equal  coefficients. Kennington and Wang in \cite{kennington1992shortest} show examples of such a problem in workforce  and project planning and scheduling as use case examples.
Here, constraints (\ref{tasks}) from the classic LAP are substituted by
\begin{equation}
\sum_{i=1}^m x_{ij}=d_j,\;\;\; \forall\;j,
\end{equation}
everything else  being the same as in the classic LAP for the situation in which there are $n$ agents and $m$ task categories. Here, $m \leq n$, and $d_j$ is the number of tasks in task group $j$ with $\sum_j d_j = n$.

Note that if also the agents are not unique and are clustered into agent groups, with $q_i$ agents in each group $i$, where $\sum_j d_j = \sum_i q_i$, the problem is equivalent to the transportation problem.

\paragraph{The assignment problem with side constraints (APSC)}
Classic assignment  problem can be solved by multiple centralized and efficient polynomial algorithms. However, by introducing side constraints, generally, this problem becomes NP-hard.
Side constraints may include budgetary limitations, degree of technical training of personnel, the rank of personnel, or time restrictions, that limit the assignment of agents to tasks.

Aggarval \cite{aggarwal1985lagrangean} introduces to the classical LAP problem an additional knapsack-type constraint
\begin{equation}\label{knapsackConstraint}
\sum_{i,j}r_{ij}x_{ij}\leq b,
\end{equation}
where  $r_{ij}$ is the amount of resource   used if agent $i$ is assigned to task $j$ and
$b$ is the amount of a resource   available. Adding constraint (\ref{knapsackConstraint}) to LAP results in a Resource Constrained Assignment Problem (RCAP), which is a knapsack problem under perfect matching over a bipartite network. Constraint
(\ref{knapsackConstraint}) deranges the unimodularity of the LAP set of constraints so that the optimal solution of the linear relaxation of the problem is no more always within the values $\{0,1\}$ and, hence, integrality constraints cannot be relaxed.
The resulting problem  belongs to the class of NP-complete problems for which no polynomially-bounded algorithm is likely to exist (see, e.g., \cite{aggarwal1985lagrangean}).

Mazzola and Neebe
\cite{mazzola1986resource} present a general model for the assignment problem with side constraints that generalizes the General Assignment Problem (GAP) (see, e.g., \cite{cattrysse1992survey}) and adds the following  constraints to either the classic LAP model or the classic LAP recognizing agent qualifications:
\begin{equation}\label{sideConstraints}
\sum_{i,j}r_{ijk}x_{ij}\leq b_k, \;\; \forall k,
\end{equation}

\noindent where $r_{ijk}$ is the amount of resource $k$ used if agent $i$ is assigned to task $j$ and $b_k$ is the amount of resource $k$ available.

By side constraints, we can model   drivers that belong to different seniority classes and customers that have different  priority levels. Seniority constraints impose for the solution to be such that no unassigned agent can be assigned to a task unless an assigned agent with the same or higher seniority becomes unassigned, while priority constraints specify that the solution must be such that no unassigned task can become assigned without a task with the same or higher priority becoming unassigned \cite{caron1999assignment}.

\subsection{Dynamic task assignment}
In this section, we propose extensions of the static assignment problem models presented previously   to the dynamic  versions in which new agents and tasks may enter the system in each time period and the costs or profits of agent-task assignment are updated in (close-to) real-time.
This problem is similar to the on-line bipartite matching problem, in which  tasks that appear in sequence should be assigned to the agents immediately as they appear.
Relating to the previously presented terminology of the static AP, a set of available (idle and assigned)  agents $A$ (that are not assisting any customer) is  known in the given weighted bipartite graph $G = (A \cup T, E)$. Tasks in $T$ (along with their incident edges) arrive online. Upon the arrival of a task $j \in T$, we must assign it  to one of agents $i \in A$ with an existing edge $(i,j) \in E$. At all times, the set of matched edges must form a (feasible) matching, i.e.,  each agent should be matched with at most one task and viceversa. In case of different cardinalities of the two sets, to balance the two, dummy elements are added to the set with lower cardinality.

We assume random arrivals of customer demands (tasks) over time. In open fleets, we also assume that agents (drivers and couriers)  either  become available randomly after assisting previous tasks (customers) or by entering and leaving the fleet based on personal interest, available time, and/or other individual constraints and preferences. Given are attribute parameters both  for agents and tasks that define their main characteristics in terms of the assignment.

We consider deterministic on-demand task allocation where the (re-)assignment of vehicles (agents) to  tasks is performed as soon as  a new vehicle or task enters the system. Close to real time reassignment is beneficial here since the parameters and variables of the assignment problem are perfectly known.

Spivez and Powell \cite{spivey2004dynamic} propose a Markov decision process model for the dynamic assignment problem.  In this paper, inspired by their work,  we propose mathematical programming models for the variations of the static task assignment described in the previous section while respecting agent-task taxonomy used previously in this paper.

The decisional variables in the dynamic AP receive a third index such that:
\begin{equation}
  x_{i j \tau}=
\begin{cases}
  1, & \text{if task } j \in T \text{ is assigned to agent } i \in A \text{  at period } \tau \in \mathcal{T}\\
  0, & \text{ otherwise.}
\end{cases}
\end{equation}

Moreover, we introduce two additional binary variables $\alpha_{\tau i}$ and $\beta_{\tau j}$, for all $i \in A$, $j \in T$ defined as follows.
\begin{equation}
  \alpha_{i\tau }=
\begin{cases}
1, & \text{if agent } i \in A \text{ is known and available for assignment in period } \tau \\
  0, & \text{otherwise.}
\end{cases}
\end{equation}

\begin{equation}
  \beta_{j\tau }=
\begin{cases}
  1, & \text{if task } j \in T \text{ is known and available for assignment in period } \tau \\
  0, & \text{otherwise.}
\end{cases}
\end{equation}

Let $\mathcal{T}$ be a set of consecutive time periods of the planning time horizon.
%
The mathematical formulation of the deterministic and dynamic  LAP problem  considering utility maximization is then given by:
\begin{equation}\label{DAP}
Z=\max \sum_{\tau \in \mathcal{T}}\sum_{i \in A}\sum_{j \in T} p_{i j \tau} x_{i j \tau}
\end{equation}

subject to:

\begin{equation}\label{agentsLimit}
\sum_{j \in T} x_{i j \tau} \leq \alpha_{i\tau }, \; \forall i, \tau
\end{equation}

\begin{equation}\label{tasksLimit}
\sum_{i \in A} x_{i j \tau} \leq \beta_{ j\tau } \;\forall j, \tau
\end{equation}

%
\begin{equation}\label{conservationOfAgents}
\alpha_{i,\tau+1} = \alpha_{i\tau }-\sum_{j \in T} x_{i j \tau}+\hat{A}_{ i,\tau+1}, \;\forall  i, \forall \tau \in \{1, \ldots, |\mathcal{T}|-1\}
%\sum_{j \in T} x_{i j \tau} - \alpha_{i\tau }\leq \hat{A}_{ i,\tau+1}, \;\forall  i, \; \forall \tau \in \{1, \ldots, |\mathcal{T}|-1\}
\end{equation}

\begin{equation}\label{conservationOfTasks}
\beta_{j,\tau+1} = \beta_{j\tau}-\sum_{i \in A} x_{i j \tau}+\hat{T}_{j,\tau+1}, \; \forall j, \forall \tau \in \{1, \ldots, |\mathcal{T}|-1\}
%\sum_{i \in A} x_{i j \tau} - \beta_{j\tau} \leq \hat{T}_{j,\tau+1}, \; \forall j, , \; \forall \tau \in \{1, \ldots, |\mathcal{T}|-1\}
\end{equation}

\begin{equation}\label{initialCondAgents}
\alpha_{i,1} =  \hat{A}_{ i,1}, \;\forall  i
\end{equation}

\begin{equation}\label{initialCondTasks}
\beta_{j,1} =  \hat{T}_{j,1}, \; \forall j
\end{equation}



\begin{equation}\label{nonNeg1}
x_{i j \tau}\in \{0,1\}, \forall i \in A, j \in T, \tau \in \mathcal{T}
\end{equation}

\begin{equation}\label{nonNeg2}
\alpha_{i \tau} \in \{0,1\},  \forall i \in A, \tau \in \mathcal{T}
\end{equation}

\begin{equation}\label{nonNeg}
\beta_{j \tau}\in \{0,1\}, \forall j \in T, \tau \in \mathcal{T},
\end{equation}

where $p_{ij\tau}$ is the utility of assigning agent $i$ to task $j$ at period $\tau$ (note that it may vary through time) and  $\hat{A}$ and $\hat{T}$ are given parameters such that:

\begin{equation}
  \hat{A}_{i\tau}=
\begin{cases}
1, & \text{if agent } i \in A \text{ enters into set $A$ (the fleet) in period } \tau \\
  0, & \text{otherwise.}
\end{cases}
\end{equation}

\begin{equation}
  \hat{T}_{j\tau}=
\begin{cases}
  1, & \text{if task } j \in T \text{ becomes known in period } \tau \\
  0, & \text{otherwise.}
\end{cases}
\end{equation}

Moreover, based on the assumption of nonrenewable agents and tasks, we assume that:  $\sum_{\tau \in \mathcal{T}}\hat{A}_{i\tau} \leq 1$ and $\sum_{\tau \in \mathcal{T}}\hat{T}_{j\tau } \leq 1$, i.e., every agent and task are unique and enter into the fleet and thus become available for assignment only once.
%

The aim is maximizing the total utilitarian social welfare over the planning time horizon, which is achieved by maximizing the  assignment utility (\ref{DAP}) over all agent-task assignments in all periods of the planning time horizon. Constraints (\ref{agentsLimit}) guarantee that each available agent at time period $\tau$ is assigned to at most one task while unavailable agents cannot be assigned to any task.
Constraints (\ref{tasksLimit}) ensure that at most one agent is assigned to any available task while no agent can be assigned to any unavailable task.

Constraints (\ref{conservationOfAgents}) and (\ref{conservationOfTasks})
represent the dynamics of dependant variables $\alpha_{\tau i}$ and  $\beta_{\tau j}$, assuming that both agents and tasks disappear from the system at the end of the period when they are assigned.
Furthermore, constraints (\ref{initialCondAgents}) and (\ref{initialCondTasks}) represent initial conditions of the problem, while the variable ranges are given by (\ref{nonNeg1})-(\ref{nonNeg}).

We can also consider cost minimization problem where we substitute (\ref{DAP}) with the following objective function
\begin{equation}\label{DAPMin}
Z=\min \sum_{\tau \in \mathcal{T}}\sum_{i \in A}\sum_{j \in T} c_{i j \tau} x_{i j \tau}
\end{equation}
subject to:
\begin{equation}\label{assignAllTasksorAgents}
\sum_{i \in A}\sum_{j \in T} \sum_{\tau \in \mathcal{T} }  x_{i j \tau} = n
\end{equation}
and   (\ref{agentsLimit})--(\ref{nonNeg}). Constraint
(\ref{assignAllTasksorAgents}) guarantees the assignment of all the tasks and/or agents in the planning time horizon, depending on the relative size of these two sets.

\paragraph{The dynamic classic assignment problem recognizing agent qualification}
Here,  the objective function is again the utility maximization (\ref{DAP}),
while constraints (\ref{agentsLimit}) and (\ref{tasksLimit}) are substituted by the following ones, everything else remaining the same as in the dynamic LAP:
\begin{equation}\label{DynamicqualifiedTasks}
\sum_{j\in T} q_{ij\tau} x_{ij\tau} \leq \alpha_{\tau i},\;\;\; \forall\;i, \tau
\end{equation}
and
\begin{equation}\label{DynamicqualifiedAgents}
\sum_{i\in A} q_{ij\tau} x_{ij\tau}\leq \beta_{\tau j}, \; \; \; \forall\; j, \tau
\end{equation}

where parameter $q_{ij\tau}$ = 1 if agent $i$ is qualified to perform task
$j$ at period $\tau$, 0 otherwise, parameter $p_{ij\tau}$ is the utility of assigning agent
$i$ to task $j$ at period $\tau$ (with $p_{ij\tau}$ = 0 if $q_{ij\tau}$ = 0), and variable
$x_{ij\tau}$ = 1 if agent $i$ is assigned to task $j$ at period $\tau$, 0 otherwise.
%
Constraints (\ref{DynamicqualifiedTasks}) guarantee that no more than one qualified agent is assigned to any task, while  constraints (\ref{DynamicqualifiedAgents}) ensure that each agent is assigned
to not more than one task.
Instead of the profit  maximization, here, we can introduce cost minimization by  substituting (\ref{DAP}) with (\ref{DAPMin}) and introducing (\ref{assignAllTasksorAgents}) into the constraint set.

\paragraph{The dynamic bottleneck assignment problem (DBAP)}
The objective function of the DBAP problem can be formulated   as follows: at each period $\tau \in \mathcal{T}$,
maximize $Z= \min_{i,j}\{p_{ij\tau}x_{ij\tau}\}$ or maximize $Z= \min_{i,j} \{p_{ij\tau}| x_{ij\tau}=1\}$.
 This maxmin problem can be  expressed by maximizing an additional variable $L$ that is a lower bound for each of the individual values $\{p_{ij\tau}|x_{ij\tau}=1\}$ as follows: $\max L$ subject to constraints $L \leq \sum_{j \in T} p_{ij\tau} x_{ij\tau}$ for all $i \in A_{\tau}$, $\tau \in \mathcal{T}$, and   (\ref{agentsLimit})--(\ref{nonNeg}) and definitions of
 the dynamic LAP.

\paragraph{The dynamic fair matching problem (DFMP)}
Here, at each period $\tau \in \mathcal{T}$,we minimize the objective function $\max_{i,j}\{c_{ij\tau}|x_{ij\tau}=1\}-\min_{i,j}\{c_{ij\tau}|x_{ij\tau}=1\}$ and
subject to constraints (\ref{agentsLimit})--(\ref{nonNeg}).
Similarly, we can minimize the difference between the maximum and minimum profit obtained among agents, i.e.,
$minimize \;\;(\max_{i,j}\{p_{ij\tau}|x_{ij\tau}=1\}-\min_{i,j}\{p_{ij\tau}|x_{ij\tau}=1\}$ and
subject to constraints (\ref{agentsLimit})--(\ref{nonNeg}).

\paragraph{The dynamic minimum deviation assignment problem (DMDAP)} At each period $\tau \in \mathcal{T}$, the objective function is as follows:
\begin{equation}
Minimize \;\; \min\{n,m\}\times \max_{p,q}\{c_{pq\tau}x_{pq}\}-\sum_{i\in A} \sum_{j\in T} c_{ij\tau}x_{ij\tau}
\end{equation}
or:
\begin{equation}
Minimize \;\;\sum_{i=1}^n \sum_{j=1}^m p_{ij}x_{ij}-\min\{n,m\}\times \min_{s,t}\{p_{st}x_{st}\},
\end{equation}
subject to constraints (\ref{agentsLimit})--(\ref{nonNeg}) and definitions of
 the minimum deviation assignment problem.

\paragraph{The dynamic $\sum_k$-assignment problem (D$\sum_k$-AP)}
Given parameter $k$,  objective function (\ref{DAP}) is modified to:
\begin{equation}\label{DSUMAP}
Z=\max \sum_{\tau \in \mathcal{T}}\sum_{i=1}^k\sum_{j \in T} p_{i j \tau} x_{i j \tau}
\end{equation}
subject to constraints (\ref{agentsLimit})--(\ref{nonNeg}) and definitions of
 the dynamic LAP.

\paragraph{The semi-assignment problem}
Here, constraints (\ref{tasksLimit}) from the dynamic LAP are substituted by
\begin{equation}
\sum_{i=1}^m x_{ij\tau}=d_{j} \beta_{j \tau},\;\;\; \forall\;j,\tau
\end{equation}
everything else  being the same as in the dynamic LAP for the situation in which there are $n$ agents and $m$ task categories, where $m \leq n$.

\paragraph{The assignment problem with side constraints}

Side constraints (\ref{sideConstraints}) here include also the time index:
\begin{equation}\label{dynamicsideConstraints}
\sum_{i,j}r_{ijk\tau}x_{ij\tau}\leq b_{k\tau} \;\; \forall k,\tau,
\end{equation}
\noindent where $r_{ijk\tau}$ is the amount of resource $k$ used if agent
$i$ is assigned to task $j$ at period $\tau$ and $b_{k\tau}$ is the amount of resource
$k$ available at period $\tau \in \mathcal{T}$. Constraints (\ref{dynamicsideConstraints}) are simply added to the formulation of the dynamic LAP.

\subsection{Bottomline}
To sum up,  in  Table \ref{tab:table-name}, we give the overview of the characteristics of the treated (static and dynamic) task assignment problems   related with: i) the kind of the social welfare they optimize (utilitarian, egalitarian, elitist, or a difference between them), ii) whether agents are qualified to perform only certain tasks or not, iii), including fairness or not, iv) whether the agents are considered homogeneous or not, v) time restrictions, vi) personal ranking, and vii) technical training.
\begin{table}\centering%[]
\begin{tabular}{l|ccccccc} %\label{TableComparisonLAP}%{|l|l|l|l|l|}
\toprule
    & Soc. & Agent  & Fairness & Unique & Time & Pers. & Tech. \\
     Model & welfare & qualif.  &  & ag./tasks & restr. & rank & train. \\\midrule
    LAP  & util. & no  & no & yes  & no & no &  no  \\
    APRAQ & util. & yes  & no & yes  & no & no & no   \\
    BAP  & egal. & no  & no & yes  & no & no  & no     \\
    FMP  & el. - eg. &no   & no & yes   & no & no & no \\\midrule
    MDAP  & el.- ut. & no  & no & yes  & no & no & no    \\
     & ut.- el. & no  & no & yes & no & no & no    \\
    $\sum_k$-AP  & egal. & no  & yes & no  & no & no & no     \\
    SAP & util. & no  & no & yes  & no & no  & no       \\ %\midrule
    APSC & util. & no  & no & no  & yes & yes & yes  \\\bottomrule
    % & 0.598  & -0.597 & 0.598  & 0.052 & 0.025 & 92.3    \\
\end{tabular}
\caption{\label{tab:table-name}Characteristics of the discussed task assignment models}
\end{table}

Note that once we introduce additional constraints to the classic assignment problem,  the resulting model is, generally, no more resolvable in polynomial time and is highly computationally expensive.
%
Additionally, \emph{we consider tasks and agents that may be  known both at some future time period   and at the first period of the planning time horizon}. Therefore, we can use this model to coordinate task allocation for  planned tasks and agents  that  schedule their appearance in advance for some future time period, but also for the tasks and agents that need to be allocated    on short-notice or immediately as they get known and enter the system. To this aim, we must use     highly computationally efficient close-to real-time  solution approaches and, generally, exact methods do not suffice for this purpose. Therefore, we are obliged to use heuristic-based approximations.

