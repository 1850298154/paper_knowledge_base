\subsection{Decentralized coordination approaches}
In contrast to centralized and distributed coordination approaches to task allocation where full knowledge of global information is assumed available to every relevant decision maker (central decision maker  or fleet coordinator (fleet owner) and (vehicle) bidder agents), in the decentralized task assignment approaches, there is no coordinator and each vehicle agent disposes only of its local (possibly incomplete and imperfect) information and finds its local assignment  based exclusively on this information and the communication with the rest of the agents and interaction with its environment.

In general, decentralized approaches
have several advantages, i.e., real-time property, robustness, and scalability. These characteristics are in general absent in centralized and distributed approaches that  outperform decentralized approaches in terms of efficiency especially for large-scale instances.
%
The decentralized decision-making   does not include any intermediary. In case of imperfect  communication, conflicts may occur. This is why
the related literature in decentralized multi-vehicle cooperative control is related with consensus, i.e., the agreement of all vehicles on some common features by negotiating with their local neighbors. General consensus issues are related with, e.g., positions, velocities, and attitudes.
%
In the following, we analyze  localized,  scalable, and decentralized heuristic algorithms for coordination of  deterministic and dynamic task assignment in open vehicle fleets. We concentrate on the approaches resulting    both in task assignment  feasibility and efficiency even though  these approaches usually have no quality of solution guarantees.

Decentralized task assignment approaches have been mostly developed in the multi-robot and Unmanned Aerial Vehicle (UAV) coordination domain. The most known ones are  sequential auction-based or consensus and negotiation-based algorithms (e.g., \cite{nunes2015multi,choi2009consensus,johnson2011asynchronous}).

One of the most known approaches for the decentralized task assignment in the coordination of a fleet of unmanned vehicles when all-to-all inter-vehicle communication  is not possible  is the Consensus-Based Auction Algorithm (CBAA) and its more general version that allows for the assignment of bundles of tasks to each agent called the Consensus-Based Bundle Algorithm (CBBA) \cite{choi2009consensus}.

The CBAA is a polynomial time market-based decentralized task selection agreement protocol running in two phases: in the first phase,  each vehicle places a bid on a task asynchronously with the rest of the fleet, and in the second, consensus phase, conflicting assignments are identified and resolved through local communication between neighboring agents within certain predefined rules to avoid task conflicts.
The agents use a consensus strategy to converge on the list of winning bids and use that list to determine the winner and associated winning scores. The list  accounts for inconsistent information among agents guaranteeing a conflict-free assignment for all.
%
This allows conflict resolution over all tasks that is robust  to inconsistencies in the situational awareness across the fleet and  the changes in the communication network topology. If the resulting scoring scheme satisfies a diminishing marginal gain  property (i.e., the value of a task does not increase as other tasks are assigned to the same agent before it), a feasible, conflict-free solution is guaranteed.

Provided that the scoring function abides by the principle of diminishing marginal gains, the CBBA has convergence guarantees. In a synchronized conflict resolution phase over a static communication network, it produces the same solution as the sequential greedy algorithm  sharing across the fleet the corresponding winning bid values and winning agent information.  Moreover, the convergence time is bounded from above and it does not depend on the inconsistency in the situational awareness over the agent set.

In \cite{choi2009consensus}, it is analytically shown that CBAA produces the same solution as some centralized sequential greedy procedures, and this solution guarantees 50$\%$ optimality.
%
Segui-Gasco et al. \cite{segui2015decentralised} propose a decentralized algorithm for multi-robot task allocation  with
a constant factor approximation of 63 $\%$
 for positive-valued monotone submodular utility functions, and of 37 $\%$ for   general positive-valued submodular utility functions.
Therefore, the authors improve the approximation guarantee of Choi et
al. \cite{choi2009consensus} for monotone positive-valued submodular utility functions from 50$\%$ to 37$\%$.

The CBBA has also been extended to consider coupled constraints \cite{choi2010decentralized,whitten2011decentralized}.
%
Choi et al. in \cite{choi2010decentralized} extended CBBA for heterogeneous task allocation to UAV agents with different qualifications and various cooperation constraints. The CBBA was extended with task decomposition and a scoring modification to allow for soft-constraints related with cooperation preferences and a decentralized task elimination protocol that ensures satisfaction of the hard-constraints related with cooperation requirements. The performance of the algorithms was analyzed in Monte-Carlo simulations in some randomly generated experiments.

The CBBA was also extended in \cite{whitten2011decentralized} to consider
the assignment of tasks with assignment constraints and also with different types of coupled and temporal constraints, where it was assumed   that assigned tasks are  executed in the order defined by their temporal precedence.

The Temporal Sequential Single-Item auction (TeSSI) algorithm \cite{nunes2015multi} allocates tasks with time windows to cooperative robot agents using a variant
of the sequential single-item auction algorithm.
Contrary to the CBBA algorithm that does not let the change of the
start time of the tasks once they are allocated, and thus reduces
the number of tasks that the algorithm allocates, the TeSSI algorithm overcomes this limitation by allowing tasks’ start times
to change, which results in higher allocation rates.

The main features of the TeSSI algorithm are a fast and systematic processing of temporal constraints and two bidding methods that optimize either completion time or a combination of completion time
and distance.
The main objective function used in the TeSSI algorithm is the makespan (the time the last task is finished)even though it is also combined with  total distance traveled.
Each robot maintains temporal consistency of its
allocated tasks using a simple temporal network.   The authors show that TeSSI outperforms a baseline greedy algorithm and the  CBBA through random experiments and related work datasets.

Ponda et al. in \cite{ponda2010decentralized} further extend the CBBA  to tasks with time windows and address re-planning in dynamic environments and consider agents with different capabilities. Agents obtain new plans  based on the changes in the environment considering new tasks while pruning older or irrelevant ones.

One of the drawbacks of the CBBA algorithm is that it relies on global synchronization mechanisms which  are hard to enforce in decentralized environments.
Johnson et al.  \cite{johnson2011asynchronous} proposed the asynchronous CBBA (ACBBA) for agents that communicate asynchronously. To allow for asynchrony in communication, the  ACBBA contains a set of local deconfliction rules that do not require access to the global information.
In ACBBA, agents locally replan their actions that, possibly, affect only a limited number of agents.

Johnson et al. \cite{johnson2013hybrid} propose a situational awareness algorithm for task assignment when agents   predict the bids of their neighbors, in order to obtain more informed decisions in a cooperative way.

To respond to the problem with local information consistency assumption that reduces optimization capabilities compared to  global information assumption approaches, Johnson et al.
\cite{johnson2017decentralized}  proposed a  Bid Warped Consensus-Based Bundle Algorithm that converges for all deterministic objective functions and has nontrivial performance guarantees for submodular and some non-submodular objective functions. They  analyse convergence and performance of the algorithm and show its efficiency compared with some other relevant local and global information approaches.

Another extension to the CBBA is provided by Binetti et al. \cite{binetti2013decentralized} that consider the decentralized surveillance problem by a team of robots. Tasks are assigned to each robot with the additional constraint that a subset of the tasks called critical tasks must be assigned.  The authors use the CBBA incorporating hard constraints in order to ensure that the critical tasks are not left unassigned.

In \cite{garcia2015cooperative}, Garcia and Casbeer present a robust task assignment algorithm that reduces communication between vehicles in uncertain environments.  Piece-wise optimal decentralized allocation of tasks is considered for a group of unmanned aerial vehicles. They present a framework for multi-agent cooperative decision making under communication constraints.  Each vehicle estimates the position of all other vehicles in order to assign tasks based on these estimates, and it also implements event-based broadcasting strategies that allow the multi-agent system representing the vehicle fleet to use communication resources more efficiently. The agents implement a simple decentralized auction scheme in order to resolve possible conflicts.

Cui et al. in \cite{cui2013game}  investigate game  theory-based  negotiation for task  allocation  in the multi-robot task assignment context.   Tasks are initially allocated using a Contract Net (see \cite{smith1980contract}) -based approach, after which,   a   negotiation approach  employing the utility functions to  select  the  negotiation  robot agents and construct the negotiation set is proposed.  Then, a  game  theory-based  negotiation  strategy   achieves the Pareto-optimal solution for the task reallocation. Extensive simulation results  demonstrate the efficiency of such a task assignment approach.

Yet another extension of the Consensus-Based Bundle Algorithm (CBBA) allowing for the fast allocation of new tasks without a full
reallocation of existing tasks is CBBA with Partial Replanning (CBBA-PR) \cite{buckman2019partial}. The algorithm enables the multi-agent system to trade-off between convergence time and increased coordination by resetting a portion of their
previous allocation at every round of bidding on tasks. By resetting the last tasks allocated
by each agent, the convergence of the MAS to a conflict-free solution is assured.
CBBA-PR can be further improved by reducing the team size involved in the replanning, further
reducing the communication burden of the team and runtime of CBBA-PR.

In \cite{sayyaadi2010distributed}, Sayyaadi and Moarref investigate a proportional task assignment problem in which it is desired for (robot) agents to have equal duty to capability ratios, i.e., the agents with more capability should perform more tasks. They address this problem as a combination of deployment and consensus problems in which agents should reach consensus over the value of their duty to capability ratios. They propose a distributed, asynchronous and scalable algorithm for this problem in continuous time domain.

Duran et al. in \cite{duran2017} study the problem of finding the list of solutions with strictly increasing cost for the Semi-Assignment Problem. Four different algorithms are described and compared. The results show that they find the exact list of solutions, and considerably reduce the computation times in comparison with the other exact approaches.

Spivey et al. in \cite{spivey2015} propose a distributed, flexible, and scalable control scheme that evenly allocates tasks.  Dynamic load balancing exploits feedback information about the status of tasks and vehicles with the objective to keep  a balanced task load and, thus, force cooperation in the solution of the randomized bottleneck task assignment problem.

In summary, most of the state-of-the-art decentralized and deterministic coordination approaches for task allocation are  heuristic algorithms developed for    multi-robot or UAV task allocation  scenarios that often include both operational and tactical constraints of a vehicle fleet and its environment. Even though their adaptation for the use in open vehicle fleets does not seem difficult, it remains an open challenge, especially if we consider  task allocation efficiency, the key performance indicator of commercial open fleets.

