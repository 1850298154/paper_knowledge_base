\section{Introduction}\label{Introduction}
Open collaborative vehicle fleets composed of autonomous self-interested system participants are ever more widespread. However, even though the drivers are autonomous and self-interested, the authority and the ownership of these systems today remains centralized in terms of management, control, and access. 
The trend seems to be an ever increasing access to mobility and last-mile services for the average person at the cost of relying on just a few (centralized) worldwide enterprises.
State-of-the-art algorithms for the allocation of tasks to vehicle fleets solve customer requests in very large fleets in almost near real-time, but they seem to be limited to centralized systems.
Centralization here can be a source of: failure  (a single bottleneck of the system), obsolete information due to significant computation delay while processing ever increasing quantity of data, privacy evasion, and mistrust if the interests of the enterprise mismatch  the users' interest.



%
Distributed Decision-Making (DDM)  obviously resolves the drawbacks of centralized systems.
The multitude of the connected smart devices of the vehicles' drivers and customers makes it possible to combine their potential and to coordinate fleets at a scale exceeding spatial and computational boundaries. This potential can be exploited for the benefit of the fleet system as a whole as well as for the interest of individual vehicle drivers and customers.


The decision-making authority in the DDM is distributed throughout a system, and the decisions are taken locally based on the local and shared global information and the interactions of an individual with the rest of the system and with the environment. Here, each fleet participant is modelled as an autonomous collaborative individually-rational software agent installed on a user's smart device. The agent has only a local vision of the fleet and it needs to cooperate with other agents in order to find the allocation of dynamically appearing tasks faced by the whole fleet.

The behaviour of the fleet as a whole is a result of inter-vehicle coordination. Distributed task allocation strongly contributes to the shift of knowledge and power from the individual (fleet owner) to the collective (vehicles composing the fleet). A desired behaviour of the fleet emerges from the identifiable interest of its participating vehicles, their beliefs, and collective actions and, as such, is a shift away from the hierarchical organizational paradigm (see, e.g., \cite{horling2004survey}). A major challenge  is the identification of a right decision maker for each part of the problem, timely exchange of relevant and up-to-date information among vehicle agents, and modelling of complex relations in such a multi-agent system. %%%???
A trade-off between the amount of computation and the quality of the solution is often necessary. Moreover, minimizing the overhead of communication required to converge to a desirable global solution is desirable.

\emph{Decentralized} coordination algorithms may be the means to obtain scalability for task allocation in the context of large-scale open fleets. Here, each self-concerned (vehicle, driver or courier) agent aims at achieving a desired local objective based on a limited local information and by communicating with the rest of the fleet and interacting with the environment.
Due to the limited local information, one of the drawbacks of decentralization is lack of control of the emerging fleet behaviour that cannot be predicted with certainty.
Moreover, to facilitate cooperation, assuming individually rational agents, we have to consider efficiency and fairness.
How to balance decentralization and centralization to improve system performance is much investigated but still not a completely-solved question.

\paragraph{Contribution} In this work, we present a survey on Multi-Agent System (MAS) coordination mechanisms for computationally complex dynamic  (one-on-one) task allocation problem (DTAP) and its variations for open vehicle fleet applications.
These problems may be modelled by a variety of deterministic and dynamic two-dimensional linear assignment problems, i.e., the problems regarding the assignment of two sets that may be referred to as ``agents'' and ``tasks'' with at most one task per agent and one agent per task, where the tasks appear dynamically and the task assignment is fully determined by the (cost, profit or revenue) parameter values and the initial conditions. We extend  mathematical models of the variations of the static task assignment problem  to their dynamic counterparts in open vehicle fleet scenarios considering, among others, self-interested and individually rational vehicle drivers, time restrictions, fairness, agent qualification and personal rank.

We identify some of the main scalable solution methods, i.e., coordination mechanisms, that can be put at work to solve these  problems. We investigate the theoretical scalability of these approaches and introduce a taxonomy to classify them in terms of the   level of inter-dependence in decision-making available to individual vehicles and customers during the coordination process (centralized, distributed, decentralized coordination). Our intention here is not to perform exhaustive search nor to identify the most scalable solution procedure. Contrarily, we  identify and mathematically model the  variations of the dynamic task assignment problem applicable to the studied fleet task allocation  contexts  and provide general scalability characteristics of their solution approaches. Our intention is to make it easier for a researcher to solve some variation of the task allocation problem in  large-scale open vehicle fleets by describing state-of-the-art solutions and their theoretical scalability results.

Even though some works exist that include reviews of the state of the art in multi-agent task allocation (see, e.g., \cite{chevaleyre2006issues,ahuja2017task,tang2010survey,jiang2012rich,jiang2019group}) and in vehicle fleet coordination (see, e.g., \cite{mariani2020coordination,billhardt2014dynamic,bielli2011trends}) or ridesharing optimization (see, e.g., \cite{agatz2012optimization,furuhata2013ridesharing}), none of them addresses one-on-one dynamic task assignment problems in open vehicle fleets.  In addition, a few approaches apply methods of multi-agent task allocation to the field of vehicle fleet coordination (see, e.g., \cite{billhardt2015towards}) but, to the best of our knowledge, there is no systematic survey combining both fields. 

The paper is organized as follows. In Section \ref{motivation}, we discuss some relevant concepts in the context of coordination for dynamic task allocation in open systems with the focus on distribution and decentralization of decision-making.
In Section \ref{taskAssignment}, we present mathematical models of various static and dynamic task assignment problems applicable in the open vehicle fleet context.
Centralized, distributed, and decentralized state-of-the-art  solution  methods and mechanisms  for the problems presented in Section \ref{taskAssignment} are discussed in Section \ref{scalabilityTheory}.
We conclude the paper emphasizing open issues and challenges for possible future research directions  in Section \ref{Challenges}. 