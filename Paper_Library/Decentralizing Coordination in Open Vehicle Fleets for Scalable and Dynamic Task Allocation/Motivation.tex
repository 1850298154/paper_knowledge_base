\section{Coordination in Open Vehicle Fleets}\label{motivation}
In this Section, we introduce some key concepts and characteristics of the target domains related to decentralizing coordination for scalable and dynamic task allocation.  The coordination problem arises   due to the distributed nature of the control exercised by the fleet's vehicles.

Generally, coordination may be defined as ``the process of organizing people or groups so that they work together properly and well''\footnote{\url{https://www.merriam-webster.com/dictionary/coordination}}.
%
By the coordination in   open vehicle fleets for task allocation, we refer to the organization and management of decision-making within the fleet with the aim to improve given key performance indicator(s) of a fleet's task allocation.

The topics of coordination and task allocation are the object of studies in multiple disciplines---e.g.,  operations research, economics, and computer science.
%
The corresponding definitions and related concepts may vary based on the specific discipline at hand.
%
In the so-called field of coordination models and languages, for instance, the focus is on the general-purpose abstractions (so-called \emph{coordination media}) that can be generally used to model and engineer the patterns of interaction between computational agents---with no specific reference to a particular application scenario or coordination problem.
%
In our survey, and in the following, we focus on the specific issues of dynamic task allocation and   distributed/decentralized coordination, with a particular emphasis on open vehicle fleets.


\subsection{Fleet coordination}
We consider  the context with cooperative vehicles in a large vehicle fleet, which functions as an organization that constrains the cooperation schemes within it.
%
The coordination problem here can be tackled from a bottom-up point of view, considering the emergence of global properties from the  inter-fleet direct vehicle to vehicle communication and fleet-environment interaction.


For simplicity and without loss of generality, we consider a two-dimensional space in which tasks may appear randomly at any location in space and time while the vehicles circulate through a transportation network within the space to reach them.
%
 Each vehicle can have three states: \emph{idle}, in which a vehicle is waiting for the assignment of a task, \emph{assigned} in which a vehicle is assigned to a task but has still not reached the task, and \emph{assisting} in which the vehicle has reached its assigned task and is assisting it. Only idle and assigned vehicles can be assigned or reassigned from one task to another. Once assigned, the vehicles start moving towards their assigned task. A task is considered completed once when it is reached and assisted by a vehicle.

Given a dynamically changing set (fleet) of idle and assigned vehicles, a dynamically changing set of  randomly appearing tasks, and a cost function of the assignment of each task to every idle and assigned vehicle (e.g., the  distance or time traveled or a given execution cost), the objective is to dynamically assign  these vehicles to tasks in a given time horizon reaching a globally minimum cost assignment considering that each task must be performed by exactly one vehicle.

Coordinating the vehicles in this respect requires that they find the globally best allocation in a distributed or decentralized way and resolve conflicts that violate local constraints. An efficient strategy in this context is a  dynamic   (re-)assignment of the vehicles in the fleet to the tasks as they appear. The vehicles require continuous communication and processing for task allocation. The coordination system must ensure a balanced use of shared resources, such as, e.g., Vehicle to Cloud (V2C) communication bandwidth and vehicle processing capacities.

V2C communication is limited in bandwidth and latency; so is the vehicle processing capacity.  Coordination strategies that ignore these communication and computation constraints may fail to find a fleet's action plan in close to real time and thus may be inapt for the application in   real-time fleets (see, e.g., \cite{zhu2018fog}).
%
These fleets require both autonomous and collaborative behaviors since vehicles have localized viewpoints, knowledge, and control and lack the overview of the global data integrated from various locations beyond their local capabilities. Such a  dynamic context requires for coordination-fault detection that indicates if the coordination exists within the fleet (see, e.g., \cite{lindner2009representation}). Once a coordination fault is detected, a coordination recovery process can begin in which cooperation can be rebuilt.


Vehicle fleets that rely on one-on-one vehicle-task assignment are, for example, rescue fleets (see, e.g., \cite{pujol2015efficient}), ride hailing and taxi service (see, e.g., \cite{billhardt2017coordinating}), ambulance assistance of urgent out-of-hospital patients (see, e.g., \cite{lujak2016distributed}),  and home-delivered restaurant hot meal services (see, e.g., \cite{ulmer2017restaurant}).
%
Ride hailing  and restaurant hot-meal delivery services are examples of open vehicle fleets that use online on-demand service platforms (see, e.g., \cite{taylor2018demand}) to allocate  in real-time customers and independent private  vehicle owners, drivers or couriers,  using their personal vehicles.
%
These platforms usually exploit sensor and GPS data to track the delivery process in real time \cite{dai2018information}.

Our focus is on the dynamic scenario with non-recurring  prearranged and  spontaneously requested single rider (customer), single driver  trips with at most one pickup and delivery  for each rider and driver.
%
Dynamically appearing riders (customers)  should be allocated to drivers in a one-on-one manner.
Before the allocation, in ride hailing, a customer chooses the driver based on the time of arrival and the price of the ride. In case of hot meal delivery, the system gives an estimated delivery time to the customer and assigns a courier that meets such an estimate.

\emph{Coordination} here is the key issue, including the stages of communication, resource allocation, and agreement.   The  allocation of the dynamically appearing customers over time needs to be performed in real-time and it fails if not completed within a specified deadline relative to an arrival of a customer; deadlines must always be met, regardless of the system load. Conventionally, the matching is based just on the rider's personal preferences and the nearby drivers' availabilities.
Reallocation of already-matched drivers to riders that are awaiting the service is not possible even if a more efficient matching exists.
%
At the end of each trip, every driver is available for a new rider allocation.

Speedy meal delivery services are constrained in geographic availability and timing. Usually, restaurants, riders, and  customers have access to the system through  an app. A customer detects his/her location and displays  restaurants that participate in the platform in the region of interest  and are open at the time.
Couriers participate in this open fleet context by delivering whenever they choose  and they may get paid on the individual delivery basis.
Once a customer requests a meal from a restaurant via his/her app, the corresponding delivery is assigned to a courier available nearby.
%
The courier picks up the delivery from the restaurant and delivers it to the customer.
%
After the delivery, a courier is available for new deliveries.

The allocation  of a courier to the customer is conventionally done based on the shortest arrival time to the restaurant (First-Come-First-Served strategy) and the availability of the courier; reallocation is not possible once the courier is allocated.
%
The  challenge here is to assign couriers to dynamically-appearing pickups and deliveries in order to maximize customer satisfaction (which can be measured in different ways, as explored in \cite{dai2018information}) without violating delivery times agreed at the time of the customer's hot meal request.

Task allocation problem in open vehicle fleets  considers both providers of transportation services (vehicle drivers) and their customers  and thus both of them may be considered active participants in the transportation process.
In the ride hailing scenario,   drivers are usually modelled as   agents and riders as tasks, while  in the hot meal delivery scenario, couriers are agents while meal deliveries are tasks.

Even though the ownership of most of open fleet systems today is centralized, not only customers, but also drivers with vehicles may appear dynamically and spontaneously in time and space influenced by a variety of   factors unknown in advance  such that it is reasonable to assume that they appear randomly.
%
In this \textit{dynamic task allocation} context, available vehicles are assigned to pending customers as they appear. Each agent and task is assumed to be characterized by a set of attributes that influences   the cost or profit resulting from an agent-task allocation. In this way, the \textit{task allocation} problem that  assigns tasks to agents in time is simplified to \textit{task assignment} problem focusing on the one agent - one task allocation at the time  (see, e.g., \cite{lujak2016distributed,billhardt2014dynamic2}).
Optimized and dynamic task (re-)assignment may considerably improve the performance of the fleet  while considering individual fairness and efficiency (see, e.g., \cite{billhardt2014dynamic2}). If dynamic courier (rider) reallocation is allowed, a substantial increase in efficiency may be observed, as in the case of ambulance allocation to out-of-hospital patients (see, e.g., \cite{billhardt2014dynamic,billhardt2014dynamic2,lujak2013coordinating}).



\subsection{Coordination models for open vehicle fleets}
Based on the ownership of the fleet, its structure, and the level of decentralizing coordination that we want to achieve in the fleet task allocation, we can design:
\begin{itemize}
  \item \textbf{a centralized coordination model}, where the task allocation problem   is solved in a single block by only one decision-maker (e.g., a single enterprise) having total control over and complete information about the vehicle fleet;
  \item \textbf{a distributed coordination model}, where the global task allocation problem is decomposed such that each customer is represented by an autonomous decision maker (agent) that may solve its own subproblem only with its own local decision variables and parameters. The allocation  of a limited number of vehicles (global constraints) is done through the interaction between competing customer agents and a vehicle fleet owner (a single autonomous agent) having available all the   fleet information.
   Customer agents that compete for the resources  are not willing to disclose their complete information but will share a part of it if it facilitates achieving their local objectives.
   %
      The vehicle fleet owner agent is responsible of achieving globally efficient resource allocation by interacting with  customer agents usually through an auction. The problem decomposition here is done  to gain computational efficiency since customer agents can compute their bids in parallel. However, the resource allocation decisions  are still made by a single decision maker (vehicle fleet owner) with the requirement on synchronous bidding of customer agents (see, e.g. \cite{zavlanos2008distributed,giordani2010distributed,giordani2013distributed});
  \item \textbf{a decentralized cooordination model}, which further decentralizes the distributed model by allowing for multiple resource owner (vehicle) agents, multiple competing customer agents requesting the transportation service, and  asynchrony in decision-making. Customer agents compete for fleet's vehicles held by multiple resource owners   while each customer and resource owner agent has access only to its local information  with  no global information available. Therefore, they must negotiate resource allocation by running localized algorithms while exchanging relevant (possibly obsolete) information. Localized algorithms make the achievement of a desired global objective easier through simple local interactions of agents with their environment and other agents, with no need for a central decision maker. The decisions specifying these interactions emerge from local information. Fairness in resource allocation here plays a major role.
    The same as in the distributed model, an  agent is not willing to disclose its complete information but will share a part of it if it facilitates achieving its local objective. Resource allocation here is achieved by the means of a decentralized protocol.
\end{itemize}

Generally speaking, coordination is distributed when complex behavior within a system does not emerge due to the control of the system owner, but through interactions and communication of individual agents operating on local information, while sharing globally relevant knowledge.
%
This form of control is typically known as \emph{distributed control}, that is, control where each agent is equally responsible for contributing to the global, complex behaviour by acting properly on local information.
%
Agents are implicitly aware of the interaction rules through mechanisms that are based on the agent's interaction with other agents and the environment.
%
The system behaviour is then an emergent property of distributed coordination mechanisms (algorithms) that act upon agents, rather than the result of a control mechanism of a centralised system owner.
%
In decentralized algorithms, no global clock is assumed, no agent has complete information about the systems’ state, every agent takes decisions based only on local information, and failure of one agent does not prevent the system to continue running. An example is BitCoin: Instead of one central server owned and operated by a single entity, Bitcoin’s ledger is distributed across the Globe making it impossible to shut down, break-in, or hack as there is no single central bottleneck of the system.

Let us notice the main difference between distributed and decentralized coordination models.
%
Distributed coordination relies on local and shared (global) parameters and variables.
%
Local parameters and variables are private, whereas shared and global parameters and variables need to be shared among two or more agents---even among all the agents of the system.
%
If we assume self-concerned agents, resource owner can manipulate these parameters and variables or deceive agents in communicating their values to influence the individual decision making of each one of them and thus obtain the behavior of the system the resource owner wants.
%
This can be prevented by ensuring individual agent access to non-obsolete and truthful information---using e.g.\ blockchain technology. Reaching a globally optimal solution with quality of solution guarantees is then possible, contrary to the decentralized coordination case.  In the latter case, due to  the lack of the global non-obsolete and truthful   information, quality of solution guarantees generally do not exist. In general, solution approaches for decentralized coordination concentrate on finding a feasible (admissible) solution without quality of solution guarantees. Contrary to the distributed case most often studied in the operations research field where the emphasis is on the method's optimality gap,  decentralized coordination methods are mostly approximate heuristics-based methods  without quality of solution guarantees but with proven completeness, soundness, and termination.
	
Open vehicle fleets are intrinsically distributed systems since they comprise a multitude of  geographically distributed and mutually communicating customers' and vehicle drivers'  apps.
%
Traditionally, distributed systems refer to systems consisting of sequential processes (each one with an independent thread of control, possibly located on geographically distributed processors) that coordinate their actions by exchanging messages to meet a common goal (see, e.g., \cite{ghosh2014distributed,tanenbaum2007distributed}).  The common goal in this context is an efficient and cost-effective transportation service of the vehicle fleet while considering individual rationality, preferences and constraints whether it is of drivers, riders, or hot meal delivery customers. Quality of solution guarantees play a crucial role of sustainable competitive advantage in any transportation network company.

Distributed open vehicle fleets exhibit some clear strong points over their centralized counterparts.
%
First of all, they are more robust than their centralized counterparts  because they can rely on their intrinsic built-in redundancy. They can operate at a larger scale and assist more customers at once since they are aggregating vehicle capacity and customer throughput across all their individual vehicle drivers.
%
However, distributed open vehicle fleets also have to deal with inter-vehicle communication and coordination overhead that can sometimes make them slower or more difficult to control than their centralized counterparts.
%
Applying trustless distributed systems that are meant to operate in an adversarial environment, such as Bitcoin, in open fleets entails an additional overhead.

