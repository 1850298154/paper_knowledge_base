\subsection{Multi-DOF Visuomotor Control}
\label{sec:multi_dof_visuomotor_control}

While ego-centric cameras are typically used when learning to navigate planar scenes from images \citep{jaderberg2016reinforcement, zhu2017target, gupta2017cognitive, parisotto2017neural}, \textit{static} scene-centric cameras are the de facto when learning multi-DOF controllers for robot manipulators \citep{levine2016end, james2017transferring, matas2018sim, james2019sim}. We consider the more challenging and less explored setup of learning multi-DOF visuomotor controllers from ego-centric cameras, and also from \textit{moving} scene-centric cameras. LSTMs are the de facto memory architecture in the RL literature \citep{jaderberg2016reinforcement, espeholt2018impala, kapturowski2018recurrent, mirowski2018learning, bruce2018learning}, making this a suitable baseline. NTMs represent another suitable baseline, which have outperformed LSTMs on visual navigation tasks \citep{wayne2018unsupervised}. Many other works exist which outperform LSTMs for planar navigation in 2D maze-like environments \citep{gupta2017cognitive, parisotto2017neural, henriques2018mapnet}, but the top-down representation means these methods are not readily applicable to our multi-DOF control tasks. LSTM and NTM are therefore selected as competitive baselines for comparison.


\subsubsection{Imitation Learning}

For our imitation learning experiments, we test the utility of the ESM module on two simulated visual reacher tasks, which we refer to as Drone Reacher (\textit{DR}) and Manipulator Reacher (\textit{MR}). Both are implemented using the CoppeliaSim robot simulator~\citep{rohmer2013v}, and its Python extension PyRep~\citep{james2019pyrep}. We implement DR ourselves, while MR is a modification of the reacher task in RLBench~\citep{james2019rlbench}. Both tasks consist of 3 targets placed randomly in a simulated arena, and colors are newly randomized for each episode. The targets consist of a cylinder, sphere, and "star", see Figure \ref{fig:reacher_tasks}.

\begin{wrapfigure}[14]{R}{0.5\textwidth}
    \centering
    \includegraphics[scale=0.1]{figures/reacher_diagrams.png}
      \caption{Visualization of (a) Drone Reacher and (b) Manipulator Reacher tasks.}
      \label{fig:reacher_tasks}
\end{wrapfigure}

In both tasks, the target locations remain fixed for the duration of an episode, and the agent must continually navigate to newly specified targets, reaching as many as possible in a fixed time frame of 100 steps. The targets are specified to the agent either as RGB color values or shape class id, depending on the experiment. The agent does not know in advance which target will next be specified, meaning a memory of all targets and their location in the scene must be maintained for the full duration of an episode. Both environments have a single body-fixed camera, as shown in Figure \ref{fig:reacher_tasks}, and also an external camera with freeform motion, which we use separately for different experiments.

For training, we generate an offline dataset of 100k 16-step sequences from random motions for both environments, and train the agents using imitation learning from known expert actions. Action spaces of joint velocities $\dot{q}\in\mathbb{R}^{7}$ and cartesian velocities $\dot{x}\in\mathbb{R}^{6}$ are used for \textit{MR} and \textit{DR} respectively. Expert translations move the end-effector or drone directly towards the target, and expert rotations rotate the egocentric camera towards the target via shortest rotation. Expert joint velocities are calculated for linear end-effector motion via the manipulator Jacobian. For all experiments, we compare to baselines of single-frame, dual-stacked LSTM with and without spatial auxiliary losses, and NTM. We also compare against a network trained on partial oracle omni-directional images, masked at unobserved pixels, which we refer to as Partial-Oracle-Omni (PO2), as well as random and expert policies. PO2 cannot see regions where the monocular camera has not looked, but it maintains a pixel-perfect memory of anywhere it has looked. Full details of the training setups are provided in Appendix \ref{app:imitation_learning}. The results for all experiments are presented in Table \ref{table:main_results}.


\begin{table}[h!]
\begin{center}
\resizebox{\columnwidth}{!}{%
 \begin{tabular}{|| c || c | c | c | c || c | c | c | c||} 
 \cline{2-9}
 \multicolumn{1}{c||}{} & \multicolumn{4}{c||}{Drone Reacher} & \multicolumn{4}{c||}{Manipulator Reacher} \\ 
 \cline{2-9}
 \multicolumn{1}{c||}{} & \multicolumn{2}{c|}{Ego Acq} & \multicolumn{2}{c||}{Freeform Acq} & \multicolumn{2}{c|}{Ego Acq} & \multicolumn{2}{c||}{Freeform Acq} \\ 
 \cline{2-9}
 \multicolumn{1}{c||}{} & Color & Shape & Color & Shape & Color & Shape & Color & Shape \\
 \hline
 \hline
 Mono & 0.6(0.7) & 0.9(1.7) & 2.4(5.0) & 0.5(1.6) & 1.8(1.5) & 1.6(1.1) & 0.1(0.2) & 0.1(0.2) \\
 \hline
 LSTM & 12.7(3.4) & 4.1(2.3) & 1.0(1.0) & 0.6(0.8) & 1.0(0.5) & 0.1(0.2) & 0.1(0.2) & 0.1(0.4) \\
 \hline
 LSTM Aux & 1.3(0.8) & 0.4(0.8) & 2.4(2.2) & 1.9(1.7) & 1.0(0.7) & 0.1(0.3) & 0.3(0.6) & 0.0(0.2) \\
 \hline
 NTM & 10.5(4.2) & 2.5(1.9) & 3.2(2.9) & 1.6(1.5) & 1.0(0.6) & 0.2(0.4) & 0.1(0.3) & 0.1(0.2) \\
 \hline
 ESMN-RGB & \textbf{20.6}(7.3) & 4.1(3.8) & \textbf{16.1}(12.7) & 1.1(2.6) & \textbf{11.4}(5.1) & 3.1(3.2) & \textbf{10.5}(5.7) & \textbf{0.9}(1.6) \\
 \hline
 ESMN & \textbf{20.8}(7.8) & \textbf{18.3}(6.4) & \textbf{16.6}(12.9) & \textbf{8.5}(11.2) & \textbf{11.7}(5.3) & \textbf{4.7}(4.0) & \textbf{11.0}(5.8) & \textbf{1.0}(1.2) \\
 \hline\hline
 Random & 0.1(0.2) & 0.1(0.2) & 0.1(0.2) & 0.1(0.2) & 0.1(0.2) & 0.1(0.2) & 0.1(0.2) & 0.1(0.2) \\
 \hline
 PO2 & 21.0(8.6) & 14.4(6.1) & 19.1(12.7) & 3.9(8.1) & 13.0(5.9) & 4.1(3.5) & 12.5(6.1) & 2.6(2.5) \\
 \hline
 Expert & 21.3(8.4) & 21.3(8.4) & 21.3(8.4) & 21.3(8.4) & 16.5(5.2) & 16.5(5.2) & 16.5(5.2) & 16.5(5.2) \\
 \hline
\end{tabular}%
}
\caption{Final policy performances on the various drone reacher (DR) and manipulator reacher (MR) tasks, from egocentric acquired (Ego Acq) or freeform acquired (Freeform Acq) cameras, with the network conditioned on either target color or shape. The values indicate the mean number of targets reached in the 100 time-step episode, and the standard deviation, when averaged over 256 runs. ESMN-RGB stores RGB features in memory, while ESMN stores learnt features.}
\label{table:main_results}
\end{center}
\end{table}

\textbf{Ego-Centric Observations:}
In this configuration we take observations from body-mounted cameras. We can see in Table \ref{table:main_results} that for both DR and MR, our module significantly outperforms other memory baselines, which do not explicitly incorporate geometric inductive bias. Clearly, the baselines have difficulty in optimally interpreting the stream of incremental pose measurements and depth. In contrast, ESM by design stores the encoded features in memory with meaningful indexing. The ESM structure ensures that the encoded features for each pixel are aligned with the associated relative polar translation, represented as an additional feature in memory. When fed to the post-ESM convolutions, action selection can then in principle be simplified to target feature matching, reading the associated relative translations, and then transforming to the required action space. A collection of short sequences of the features in memory for the various tasks are presented in Figure \ref{fig:trajectories}, with (a), (b) and (d) coming from egocentric observations. In all three cases we see the agent reach one target by the third frame, before re-orienting to reach the next.

We also observe that ESMN-RGB performs well when the network is conditioned on target color, but fails when conditioned on target shape id. This is to be expected, as the ability to discern shape from the memory is strongly influenced by the ESM resolution, quantization holes, and angular distortion. For example, the "star" shape in Figure \ref{fig:trajectories} (a) is not apparent until $t_5$. However, ESMN is able to succeed, and starts motion towards this star at $t_3$. The pre-ESM convolutional encoder enables ESMN to store useful encoded features in the ESM module from monocular images, within which the shape was discernible. Figure \ref{fig:trajectories} (a) shows the 3 most dominant ESM feature channels projected to RGB.

\begin{figure}[h!]
\centering
\includegraphics[width=\textwidth]{figures/trajectories.png}
  \caption{Sample trajectories through the memory for (a) ESMN on DR-Ego-Shape, and ESMN-RGB on (b) DR-Ego-Color, (c) DR-Freeform-Color, (d) MR-Ego-Color, (e) MR-Freeform-Color. The images each correspond to features in the full $90\times180$ memory at that particular timestep $t$.}
  \label{fig:trajectories}
\end{figure}

\textbf{Scene-Centric Observations:}
Here we explore the ability of ESM to generalize to unseen camera poses and motion, from cameras external to the agent. The poses of these cameras are randomized for each episode during training, and follow random freeform rotations, with a bias to face towards the centre of the scene, and linear translations. Again, we see that the baselines fail to learn successful policies, while ESM-augmented networks are able to solve the task, see Table \ref{table:main_results}. The memories in these tasks take on a different profile, as can be seen in Fig \ref{fig:trajectories} (c) and (e). While the memories from egocentric observations always contain information in the memory image centre, where the most recent monocular frame projects with high density, this is not the case for projections from arbitrarily positioned cameras which can move far from the agent, resulting in sparse projections into memory. The targets in Fig \ref{fig:trajectories} (e) are all represented by only 1 or 2 pixels. The large apparent area in memory is a result of the variance-based smoothing, where the low-variance colored target pixels are surrounded by high-variance unobserved pixels in the ego-sphere.

\begin{wrapfigure}[10]{R}{0.25\textwidth}
    \vspace{-11pt}
    \centering
    \includegraphics[width=1.0\linewidth]{figures/drone_obstacle_avoid.png}
      \caption{Avoidance task}
      \label{fig:dr_obs_avoid}
\end{wrapfigure}

\textbf{Obstacle Avoidance:} To further demonstrate the benefits of a local spatial geometric memory, we augment the standard drone reacher task with obstacles, see Figure \ref{fig:dr_obs_avoid}. Rather than learning the avoidance in the policy, we exploit the interpretable geometric structure in ESM, and instead augment the policy output with a local avoidance component. We then compare targets reached and collisions for different avoidance baselines, and test these avoidance strategies on random, ESMN-RGB and expert target reacher policies. We see that the ESM geometry enables superior avoidance over using the most recent depth frame alone. The obstacle avoidance results are presented in Table \ref{table:obstacle_avoidance_results}, and further details of the experiment are presented in Appendix \ref{app:obstacle_avoidance}.


\begin{table}[h!]
\begin{center}
\resizebox{\columnwidth}{!}{%
 \begin{tabular}{| c | c || c | c || c | c || c | c ||}
 \cline{3-8}
 \multicolumn{2}{c||}{} & \multicolumn{6}{c||}{Policy} \\
 \cline{3-8}
 \multicolumn{2}{c||}{} & \multicolumn{2}{c||}{Random} & \multicolumn{2}{c||}{ESMN-RGB} & \multicolumn{2}{c||}{Expert} \\ 
 \cline{3-8}
 \multicolumn{2}{c||}{} & Reached & Collisions & Reached & Collisions & Reached & Collisions \\
 \cline{3-8}
 \hline
 \hline
 \parbox[t]{2mm}{\multirow{4}{*}{\rotatebox[origin=c]{90}{Avoidance}}} &
 No Avoidance & 0.0(0.2) & 29.4(27.9) & 9.9(7.0) & 64.8(67.4) & 21.3(2.1) & 121.7(15.8) \\
 \cline{2-8}
 & Single Depth Frame &  0.1(0.2) & 9.6(11.8) & 11.1(4.3) & 10.8(13.0) & 15.3(2.0) & 7.7(6.9) \\
 \cline{2-8}
 & ESM Depth Map & 0.1(0.3) & \textbf{4.2}(7.1) & 9.6(4.5) & \textbf{2.7}(5.5) & 10.8(4.4) & \textbf{2.2}(3.8) \\
 \cline{2-8}\cline{2-8}
 & Ground Truth & 0.1(0.3) & 0.1(0.5) & 9.6(5.1) & 0.0(0.0) & 15.2(3.9) & 0.0(0.0) \\
 \hline
\end{tabular}%
}
\caption{Targets reached and number of collision for the obstacle avoidance drone task. Full details of this experiment are provided in Appendix \ref{app:obstacle_avoidance}.}
\label{table:obstacle_avoidance_results}
\end{center}
\end{table}

\textbf{Camera Generalization:}
We now explore the extent to which policies trained from egocentric observations can generalize to cameras moving freely in the scene, and vice-versa. The results of these transfer learning experiments are presented in Table \ref{table:transfer_learning}. Rows not labelled ``Transferred'' are taken directly from Table \ref{table:main_results}, and repeated for clarity. Example image trajectories for both egocentric and free-form observations are presented in Figure \ref{fig:ego_vs_scene_images}. The trained networks were not modified in any way, with no further training or fine-tuning applied before evaluation on the new image modality.

\begin{figure}[h!]
\centering
\includegraphics[width=\textwidth]{figures/ego_vs_scene_image_differences.png}
  \caption{Example image sequences from both egocentric (E) and freeform (F) cameras, on both reacher tasks. The images are all time-aligned, and correspond to the same agent motion in the scene.}
  \label{fig:ego_vs_scene_images}
\end{figure}

\begin{table}[h!]
\begin{center}
\resizebox{\columnwidth}{!}{%
\begin{tabular}{|| c || c | c | c | c || c | c | c | c||} 
\cline{2-9}
\multicolumn{1}{c||}{} & \multicolumn{4}{c||}{Drone Reacher} & \multicolumn{4}{c||}{Manipulator Reacher} \\ 
\cline{2-9}
\multicolumn{1}{c||}{} & \multicolumn{2}{c|}{Ego Acq} & \multicolumn{2}{c||}{Freeform Acq} & \multicolumn{2}{c|}{Ego Acq} & \multicolumn{2}{c||}{Freeform Acq} \\ 
\cline{2-9}
\multicolumn{1}{c||}{} & Color & Shape & Color & Shape & Color & Shape & Color & Shape \\
\hline
\hline
 LSTM & 12.7(3.4) & 4.1(2.3) & 1.0(1.0) & 0.6(0.8) & 1.0(0.5) & 0.1(0.2) & 0.1(0.2) & 0.1(0.4) \\
\hline
Transferred & 0.3(0.6) & 0.5(0.7) & 0.1(0.3) & 0.3(0.7) & 0.1(0.3) & 0.0(0.2) & 0.0(0.0) & 0.0(0.2) \\
\hline
\hline
 ESMN-RGB & 20.6(7.3) & 4.1(3.8) & 16.1(12.7) & 1.1(2.6) & 11.4(5.1) & 3.1(3.2) & 10.5(5.7) & 0.9(1.6) \\
\hline
Transferred & \textbf{20.3}(11.5) & \textbf{5.8}(4.7) & \textbf{15.3}(10.9) & 0.8(1.6) & \textbf{11.2}(5.6) & \textbf{3.9}(4.3) & \textbf{9.8}(3.8) & 1.1(1.3) \\
\hline
\hline
 ESMN & 20.8(7.8) & 18.3(6.4) & 16.6(12.9) & 8.5(11.2) & 11.7(5.3) & 4.7(4.0) & 11.0(5.8) & 1.0(1.2) \\
\hline
Transferred & \textbf{10.4}(5.5) & 0.7(0.9) & \textbf{8.9}(6.5) & 1.0(1.4) & \textbf{7.5}(4.9) & \textbf{6.3}(6.2) & \textbf{6.8}(4.7) & 0.6(1.0) \\
\hline
\end{tabular}%
}
\caption{Reacher performances both with and without camera transfer, using the same notation and setup as described in Table \ref{table:main_results}. Successful transfers are highlighted in bold.}
\label{table:transfer_learning}
\end{center}
\end{table}


\subsubsection{Reinforcement Learning}
\label{sec:image_to_action_RL}
Assuming expert actions in partially observable (PO) environments is inherently limited. It is not necessarily true that the best action always rotates the camera directly to the next target for example. In general, for finding optimal policies in PO environments, methods such as reinforcement learning (RL) must be used. We therefore train both ESM networks and all the baselines on a simpler variant of the MR-Ego-Color task via DQN \citep{mnih2015human}. The manipulator must reach red, blue and then yellow spherical targets from egocentric observations, after which the episode terminates. We refer to this variant as MR-Seq-Ego-Color, due to the sequential nature. The only other difference to MR is that MR-Seq uses $128\times128$ images as opposed to $32\times32$. The ESM-integrated networks again outperform all baselines, learning to reach all three targets, while the baseline policies all only succeed in reaching one. Full details of the RL setup and learning curves are given in Appendix \ref{app:reinforcement_learning}.