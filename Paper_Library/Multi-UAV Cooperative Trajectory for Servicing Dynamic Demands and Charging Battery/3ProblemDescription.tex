\section{System Model and Problem Description }\label{sec:problem-description}


\begin{figure}[!t]%[htbp]
\centerline{
\includegraphics[width = 0.95 \columnwidth]{images/system-model.pdf}
}
\caption{An illustration of the system model, where two UAVs cooperatively route and service $|S|=5$ locations or clusters with radii $r$ in the ($x, y ,z$) 3D space. Each location (e.g., location $s_1$ here) releases a series of user demands over time and each user demand requests to be serviced by a UAV within user's waiting time. The connected blue dashed and red dashed arrows show the first and second UAVs' trajectories, respectively.}
\label{fig:model}
\end{figure}


\iffalse
\begin{figure}[!t]%[htbp]
\centerline{
\includegraphics[width = 0.8 \columnwidth]{images/drone_model.pdf}
}
\caption{An illustration of the system model, where two UAVs cooperatively route and service $|S|=5$ locations on the 2D ground plane. Each location (e.g., location 1 here) releases a series of user demands in set $J_s$ with $s=1$ over time and each demand $j \in J_s$ requests to be serviced by a UAV within time window $[r_j, d_j)$. The connected blue (black) arrows show the first (second) UAV's routing path.}
\label{fig:model}
\end{figure}
\fi

As illustrated in Fig.~\ref{fig:model},
given a set $K$ of UAVs to service a set $J$ of dynamically arriving users over a set of potential locations (e.g., shopping malls and other hotspots) across the 3D space, we seek to design UAV-swarm's cooperative path planning among these locations to meet delay-sensitive users' demands as many as possible.
%
%We have a set $K$ of UAVs to provide delay-sensitive service to dynamically arriving  users in a set $S$ of potential locations (e.g., shopping malls and other hotspots). As illustrated in Fig.~\ref{fig:model}, the UAV-swarm aims to meet as many user demands as possible, by routing UAVs among different locations.
% we consider a set $S$ of spatial locations (i.e., hotpots such as shopping malls and parks) where
%
To successfully service a user demand, a UAV must reach the corresponding location  before the waiting deadline of the demand.
The objective is to determine cooperative path planning for the UAV-swarm so that the total number of successfully serviced demands is maximized.
In the following, we detail the system model for the dynamic demands, UAV service, and summarize the key notations in Table~\ref{tbl:notation}.

As in \cite{zeng2017energy,ProcIEEE2019}, we adopt the air-to-ground model where the UAVs are deployed to the positions above ground and the wireless communication channels between UAVs and users are dominated by LoS links. LoS links are expected for air-to-ground channels in many scenarios. Therefore, the channel power gain from the UAV to each user $k$ is modeled as the free-space path loss model, i.e., $g_k = \xi \bar{d_k}^{-2}$, where $\xi$ denotes the channel power gain at a reference distance. $\bar{d_k}$ is the link distance between the UAV $i$ and ground user $k$. Given a standard transmission power $P$, the signal-to-noise ration (SNR) at ground user $k$ is given by $\gamma_k = \frac{P g_k}{\sigma^2}$, where $\sigma^2$ denotes the noise power at each ground user. We say a ground user $k$ at any location can be serviced by a UAV if the SNR at user $k$ is no less than a threshold value $\gamma_{th}$, and the data rate is $\log (1+\gamma_{th})$. Thus, we can obtain each UAV's wireless coverage disk on the ground with range $r$ and flying altitude $H$ to serve users nearby to meet target SNR as follows.
\begin{equation}\label{eq:radii}
r = \sqrt{\frac{P \xi}{\gamma_{th} \sigma^2} - H^2}.
\end{equation}

Given the spatial distribution of the ground users and the coverage radii $r$, the ground users can be clustered into the location set $S$ by applying the classical clustering algorithm \cite{hochbaum1985best}.
As illustrated in Fig.~\ref{fig:model}, each cluster of users is a circular region with ground radius $r$,
correspondingly, the associated UAV location in 3D is $s = \langle x_s,y_s,z_s \rangle \in S$, thus we denote $J_{s}$ as the set of user demands in this cluster, which are dynamically released over time.
%
%within coverage radius $r$ of UAV $i$, the associated UAV location $s_i = \langle x_i,y_i,z_i \rangle \in S$, a set $J_{s_i}$ of user demands are dynamically released over time.
Our following algorithms are thus designed based on such wireless model and set $S$.

We assume that the UAVs have sufficient bandwidth resources so that all UAVs can be assigned orthogonal channels and for achieving interference-free \cite{8660516}. Note that in our designed algorithms, multiple UAVs servicing the same UAV location simultaneously is prohibited. In practice, the assigned channels for distant UAVs can be reused during servicing. Thus, the interference among UAVs can be ignored, and henceforth we focus our study on the UAV-swarm path planning.

%released \rv{within radius $r$} of

Each user demand $j \in J_s$ associated by UAV location $s \in S$ is characterized by its release time $r_j$ and waiting deadline $d_j$, where the time difference $d_j - r_j$ tells the user's maximum waiting time.
%\footnote{Without much loss of generality, we assume $r_j$ and $d_j$ are integers. In practice, one can discrete their values for engineering purpose.}.
For each demand, it takes a fixed amount of time $q$ to service this demand.
We denote $J = \cup_{s \in S} J_s$ as the set of demands at all the locations and denote $n=|J|$ as the total number of demands.
To locally service user demand $j \in J_s$, the UAV must reach location $s$ within time window $[r_j,d_j)$ and service at location $s$ for at least a consecutive time $q$.
%wireless communication metric (e.g., SINR, data rate, etc.,) ?
%.%and ensures  sufficient energy to finish servicing the demand.
%The UAV can service an user demand only if it is hovering and if it decides to service a demand, it must hovering at least for a consecutive time $q$ to finish the demand.
If a UAV just reaches the location at time point $t = d_j$, it cannot service demand $j$ but just misses the demand.


%
%At any time, an user demand is referred to as either  \emph{alive}, \emph{serviced} or \emph{missed}.  %to remove
%Initially, the demand is \emph{alive} after being released, indicating that the user is waiting to be serviced, then it becomes \emph{serviced} when a UAV finishes servicing the demand or  \emph{missed} when no UAV services the demand before the waiting deadline of the demand.

In our model, the UAV can simultaneously service multiple user demands at the same location, which holds for many applications such as edge computing and information broadcasting to users \cite{broadcasting2014}.

%Specifically, when the UAV starts to service some user demand at location $s$ at time $t$, any demand $j \in J_s$ satisfying $r_j \le t < d_j$ can be serviced by the UAV simultaneously.
To ensure reliable service,
the servicing demands can not be interrupted and should be finished before the UAV leaves the location.
%
Note that in our problem of optimizing multi-UAV trajectory planning,
the demand set $J_s$ is already determined as users need to register their demands beforehand and otherwise the demand will not be serviced.
The UAV path planning is computed based on the given $S, J_s, K$ in the offline.
Later in Section~\ref{sec:experiments}, we will show that our designed algorithms also provide good performances even if there is some unexpected error in estimating the demands.

%\footnote{Our analysis and results can be easily extended to the case that it takes a fixed number of time units to complete a service. Similar to \cite{wang2018traffic, xu2018uav, zeng2017energy}. An example of such service is disseminating/broadcasting latest news to users.}.

%Without much loss of generality, we assume the service time for each user demand is short, and as long as we have at least one UAV at location $s$ at time $t$, any demand $j \in J_s$ satisfying $r_j \le t < d_j$ can be serviced by the UAV \footnote{Our analysis and results can be easily extended to the case that it takes a fixed number of time units to complete a service. Similar to \cite{wang2018traffic, xu2018uav, zeng2017energy}, we assume that the UAV can service multiple demands at the same time. An example of such service is disseminating/broadcasting latest news to users.}.
%


%To simplify our analysis, we assume the charging station is a subset of user locations, i.e. $C \subset S$, and restrict that no user demand is released at any charging location.
%.%and charging stations
%
%We consider the users' ground locations in a general graph and
%
We model the spatial connectivity of the UAV locations $S$ by a distance matrix $a(\cdot,\cdot)$, where $a(s,s')$ indicates the pairwise travelling distance from location $s$ to $s'$ with  $s, s' \in S$.
In our model, we assume the UAV moves at constant speed $v$, and without loss of generality, we normalize it as $v = 1$.
%
In addition, we practically consider that the UAV's traveling distance matrix satisfies the triangle inequality, i.e., for any three different locations $s, s', s'' \in S$, it holds that
\begin{equation}\label{eq:triangle}
a(s,s') + a(s',s'') \ge a(s,s'').
\end{equation}

When a UAV is flying from location $s$ to $s'$, it cannot provide any service to users in the meantime, since UAV-enabled service coverage is small as compared to the distance between location $s$ and $s'$ \cite{8933037}.

%in our model since it is not close to either locations $s$ and  $s'$.

%All input values can be continuous values as our



Our problem can be described as an optimization problem to maximize the number of successfully serviced demands
%or the chance to hit any demand $j$ in its time window $[r_j, d_j)$
as follows.
%| \se{ j~|~[r_j,d_j) \cap T_{s} \not = \emptyset, j \in J_s} |
\begin{equation}\label{eq:obj}
\begin{array}{c}
    \max_{
    \se{T_{k,s}, ~\forall ~ \text{UAV} ~ k ~ \text{and~location}~ s}
    } \limits ~
    \sum_{s \in S} \limits \sum_{
    j \in J_s} \limits \text{serviced}(j, T_s) \\
    s.t. ~(T_{k,1},T_{k,2},\ldots,T_{k,|S|}) \longrightarrow \text{feasible routing}, ~  \forall k. \\
    \se{T_{s}  = \bigcup_{k=1}^K  T_{k,s}, s \in S}
     \longrightarrow \text{collision-free routing}. \\
\end{array}
\end{equation}
%
%&s.t. ~~~~~~T_{s}  = \bigcup_{k=1}^K ~ T_{k,s}, ~~~  \forall s \in S \\
%j \in J_s: [r_j,d_j) \cap T_{s} \not = \emptyset %wrong!
%
%\parbox{.7\textwidth}
\begin{equation}\label{eq:service}
\text{serviced}(j, T_s) =
    \begin{cases}
    1, &
    \text{if}~ \exists ~k, ~ \exists~ [t_1,t_2) \in T_{k,s}, \\
    & t_2 - q \in [r_j,d_j) \cap [t_1,t_2)
    \\
    0,& \text{otherwise}
    \end{cases}
\end{equation}
As a decision, the schedule set $T_{k,s}$  in the whole discrete time horizon $[1, T_{max}]$ contains all the time periods during which UAV $k$ is hovering at location $s \in S$ to service user demands there, and the time periods $(T_{k,1},T_{k,2},...,T_{k,|S|})$ are the result of a feasible routing of a single UAV $k$.
In particular, $T_s$ summarizes the overall servicing time periods at location $s$ by all UAVs.
To address the UAV collision issues, the path planning defined by $T_{k,s}$ is \emph{collision-free} if any two UAVs neither meet at any location nor meet during flying to its target location.
At time point $t \in T_{k,s}$, UAV $k$ can start to service a demand $j \in J_s$ at location $s$ only if this demand is released prior to time $t$ and not missed yet, i.e., $t \in [r_j,d_j)$.
As such, demand $j$ can be serviced if time period $[t_1,t_2) \in T_{k,s}$ of UAV $k$ is long enough to finish the demand, i.e., condition
$t_2 - q \in [r_j,d_j) \cap [t_1,t_2)$ in Eq.~\eqref{eq:service} indicates a service success.
%As such, demand $j$ can be serviced whenever at least one UAV reaches location $s$ during time periods $[r_j,d_j)$, that is, $ [r_j,d_j) \cap T_{s} \not = \emptyset$ in the objective to count as a service success.
Thus, the optimal UAV trajectory planning sends each UAV $k$ to service at each location $s \in S$ appropriately at time periods $T_{k,s}$ such that the total number of serviced demands is maximized.
From above formulation of problem in Eq.~\eqref{eq:obj},
each UAV needs to decide at any time point to visit which location, translating to a huge computational complexity for solving Eq.~\eqref{eq:obj}, which is $O( (T_{max})^{2 K H})$ with $H$ being the number of visits per UAV per location.
Actually, we can rigorously prove that this problem is NP-hard and the existing solutions (e.g., \cite{sugihara2011path, 6198334, xu2020optimized}) in the literature cannot apply, requiring us to innovate and propose new algorithms with provable performance bounds.
%[14], [17], [18]
% in $O(S^K^(T_max))$
%%Should it be O(S^K^(T_max)) ?? H is vague and hard to understand % not correct.
%we observe that if a UAV visits location $s$ for $H$ times, $T_{s}$ will contains up to $K\cdot H$ time periods, leading to complexity of  $O( (T_{max})^{2 K H})$ with $T_{max}$ being the maximum considered integral time point.
%In our model, the demand information $J_s$ is  pre-known, as we assume the users will register their requests at center stations and we only provide mobile service for their registered demands at spatial locations.
%%Moreover, the computation of the best UAV path planning are also performed at center station, prior to the deployment of UAVs.
% kai: [introduction/systm model: emphasize edge computing, service, cache; first paragraph of intro]
%Later in Section \ref{sec:energy} we will generalize our model to consider UAV battery charging at center stations.

\iffalse
\myparagraph{UAVs.}
We consider homogeneous UAVs and each UAV has a battery with energy capacity limit $E$.
%We denote $E_t$ as the remaining energy in battery for a particular UAV at time $t$. %define later
By varying power consumption, each UAV can travel under different flying speed, i.e.,
different flying modes.
There are $L$ flying modes by support, and the UAV travels at speed $v(l)$ with power consumption $p(l)$ under the $l$-th ($l \in \se{1,2,...,L}$) flying mode.
Specifically, when traveling from location $s$ to $s'$ ( $s,s' \in S \cup C$),
the travelling time is $m(s,s')/v(l)$ under  the $l$-th flying mode and meanwhile the UAV consumes $m(s,s')p(l)/v(l)$ amount of battery energy.
%
%The UAV takes $m(s,s')/v(l)$  amount of time to travel from location $s$ to $s'$  under the $l$-th flying mode ($l \in \se{1,2,...,L}$), and meanwhile consumes $m(s,s')p(l)/v(l)$ amount of energy.
%
We denote $l^{*} \in \se{1,2,...,L}$ as the most energy-efficient mode which consumes the least amount of energy when travelling one unit of distance.
%\cite{di2015energy}
%
Moreover, for each location $s\in S$, we denote $E_s^{*}$ as the minimum amount energy that is needed for the UAV to travel to the nearest charging station $C$ from location $s$, which is calculated as follows.

\begin{equation}
    E_s^{*} = \min_{s' \in C} m(s,s') p(l^*)/v(l*)
\end{equation}


Especially, the first flying mode is referred to as \emph{hovering}, i.e. $v(1) = 0$ and we denote $p_{hover} = p(1)$ as the power consumption for hovering
\footnote{We neglect the energy consumed by servicing user demands as these energy is often very small compared to the energy consumed by propeller.}.
%
In our model, each UAV can service an user demand only if it is in hovering mode.
Additionally, if the UAV starts to service a demand, it must hovering for at least  a consecutive time $q$ to finish the demand.

When UAV reaches a charging station, it can immediately start battery charging.
Time $t_c(E')$ denotes the amount of time to charge the battery from current energy $E'$ to full capacity (i.e., 100\%).
%
The alternative situation where the UAV battery is replaced can be regarded as a special case where $t_c(E') = t_c$ with $t_c$ being the battery replacement time.
%During charging, no demand can be serviced.
When battery charging process is completed, the UAV has to immediately return to the air to provide service.
%When the battery is fully charged (or replaced), the charging time is denoted by $t_c$.
%Moreover, the UAV can start to service a demand only if the remaining energy is sufficient enough to at least finish servicing the demand with power consumption $p_{hover}$, i.e. $E(t) \ge q \cdot p_{hover}$.

%\KK{need to think flexible charging time! not to consider}

\fi




\begin{table}[ht!]
\renewcommand{\arraystretch}{1.0}
\caption{Notations and Physical Meanings.}
\label{tbl:notation}
\centering %|p{9cm}|p{6cm}
\begin{tabular}{ l |  p{5.5cm}}
\hline
\hline
{\bf Math notation} & {\bf Physical Meaning}\\
\hline
%$K,S$ & the set of UAVs and Locations respectively \\
$K$ & the set of UAVs \\
\hline
$S$ & the set of user locations \\
\hline
$J, n$ & the set (resp. the number) of all user demands\\
\hline
$J_s$ & the set of user demands at location $s \in S$\\
\hline
$r_j, d_j$ & the release time and waiting deadline of user demand $j \in J$, respectively \\
\hline
$q$ & servicing time for each demand \\
\hline
$a(s,s')$ & the travelling distance from location $s$ to location $s'$\\
\hline
%{\bf Math notation} & {\bf Physical Meaning}\\
%\hline
$U(t)$ & the state of the UAV at time $t$ \\
\hline
$M(t)$ & the set of historical UAV states up to time $t$\\
\hline
$J(M(t))$ & the set of serviced demands up to time $t$ \\
\hline
$\pe{U(t), J(M(t))}$ & the
decision state at time $t$ \\
\hline
\hline
\end{tabular}
\end{table}




\begin{figure}[!t]%[htbp]
\centerline{
\includegraphics[keepaspectratio, width = 0.95 \columnwidth]{images/figcoop.pdf}
}
\caption{An illustrative example of two-UAV cooperative path planning (in red dashed line and blue solid line) to service three locations ($s_1, s_2, s_3$) placed on a line over 16 discrete time units.
All demands last for 4 units of time and are released at each discrete time point, i.e.,
$J_{s_1} = \se{[0,4), [1,5), [2,6), ...}$,
$J_{s_2} = \se{[0,4), [0,4), [1,5), [1,5), [2,6),[2,6), ...}$,
$J_{s_3} = \se{[0,4), [1,5),$ $[2,6), ...}$.
The arrows indicate the optimal path planning of the two UAVs, where the horizontal intervals at locations $s_1, s_3$ indicate 4 missed demands.
}
\label{fig-coop}
\end{figure}



%TODO: Difference when including energy:
%more decision for target location


%\KK{routing example needs to change}
%Noise : Modify solution

Seeking UAV-swarm cooperation is important for the optimal path planning for meeting dynamic demand patterns over space and time,
%it is necessary for the UAVs to cooperate in designing their routing jointly,
and we illustrate the cooperation advantage by a simple example in Fig.~\ref{fig-coop}, which motivates our problem formulation in next sections.
Here we have two UAVs to provide service at three locations along a road in set $S = \se{s_1,s_2,s_3}$.
The three locations along a road have 1D coordinates $\se{0,2,4}$, respectively, telling a UAV's travelling delay $a(s_1,s_2) = a(s_2,s_3) = 2$ if we normalize the UAV velocity as one unit per time.
%Once reaching a user location, the UAV can service the user fast with $q \rightarrow 0$ in this example.
At each discrete time $t \in \se{0,1,2,...}$, there is one user demand released at location $s_1$, one user demand released at $s_3$ and two user demands released at middle location $s_2$.
All demands have fixed service time $q = 1$ and last for $4$ time units, i.e., $d_j - r_j = 4, \forall j \in J$.


Fig.~\ref{fig-coop} shows how the two UAVs cooperate with each other in a periodic cycle of $8$ time units at  optimum.
%UAV 1 returns to location $s_1$ (and location $s_2$) in every 8 time units.
Specifically, in every $8$ time units, the two UAVs take turns to visit location $s_2$ with a time gap of $3$ time units
(without missing any demand there), and each of them stays at one of the two ends (location $s_1$ or $s_3$) for $3$ time units for each visit there.
During the first time period cycle $[0,8)$ for example, at location $s_1$ (resp. $s_3$) only two demands with release times $6$ and $7$  (resp. release times $2$ and $3$) are missed.
%, as shown in Fig.~\ref{fig-coop}
That is to say, totally  $87.5\%$ of demands are serviced.
Without cooperation, however, the two
%UAVs stay at their own locations will miss many more demands.
UAVs which stay at their own locations will miss much more demands.
For example, if the first UAV stays at location $s_1$ to meet all the demands there and the second UAV services the remaining two locations, the strategy of the second UAV that
travels between $s_2$ and $s_3$ back-and-forth staying one unit time at each
will miss $6$ demands ($4$ demands from $s_2$ and $2$ demands from $s_3$) in the period time cycle $[0,6)$, which does not service more demands than simply staying at location $s_2$ to service all demands there. As a result, at most $75\%$ of demands can be serviced.
%at most $16$ demands can be serviced by the second UAV,
%resulting in $75\%$ of successfully serviced demands.
Later in Lemma~\ref{lmm:nonshare}, we theoretically prove that the performance of UAV-swarm's partitioned path planning without overlapped locations can be arbitrarily poor.


%When the demand information are completely unknown in advance, no demand can be serviced in any algorithm in the worst case for our problem. One can imagine the case where a new demand always appears at the location where no UAV is nearby and disappears when some UAV arrives the location.
%offline
%In the following, we start with optimizing each UAV given its demands to service. The designed algorithm is the basis of our algorithm for routing multiple UAVs in Section~\ref{sec-multiple-UAVs}.
%lays the foundation of our algorithm for UAV-swarm cooperative routing in Section~\ref{sec-multiple-UAVs}.


%In introduction on page 1, Please cite the following paper:
%\cite{zhang2015data}
%Zhang, Yongmin, Shibo He, and Jiming Chen. "Data gathering optimization by dynamic sensing and routing in rechargeable sensor networks." IEEE/ACM Transactions on Networking 24.3 (2015): 1632-1646.



The advantage of UAV-swarm cooperation is not only to hit the dynamic demands, but also to cover each other to charge battery and enable sustainable service provisioning. Later in Section~\ref{sec:energy} we will generalize our model to consider UAV battery charging at ground stations. 