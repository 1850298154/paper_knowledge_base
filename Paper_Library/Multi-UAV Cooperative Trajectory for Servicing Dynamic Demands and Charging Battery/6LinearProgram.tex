







\section{Algorithms for UAV-Swarm Powered with Battery Charging Stations}\label{sec:energy}

In previous sections, we have successfully designed and analyzed cooperative path planning algorithms, where each UAV has a finite time in the sky to route and service due to the limited battery capacity. 
%under UAVs' limited battery capacity. 
In this section, to provide sustainable service provisioning, we further consider that each UAV can charge its battery by reaching an battery charging 
station. Accordingly, we jointly design UAVs' cooperative path planning over not only users' locations but also battery charging stations in the long run. 
The battery charging option introduces new decision choices into UAV path planning and makes it more challenging to design the cooperative path planning algorithm.
%
In the following, we first extend the system model with battery charging and show %Lemma~\ref{lmm:prnc-1}, Lemma~\ref{lmm:prnc-2} and 
%characterize the major difference of the optimal solution when battery charging is involved.
%We find that directly applying our proposed algorithms here will not give optimal solution since the 
that the dominance property of optimal path planning in Lemma~\ref{lmm:dominated} do not hold anymore. 
We then analyze new dominance property and refine our previous algorithms.
%with similar performance.
%The design becomes more challenging and multiple routing paths will be required for UAVs' spatial-temporal information of charging. 
%
%In this section, we generalize the problem into a more practical case  with the consideration of UAV battery  charging.

On top of the system model defined in Section~\ref{sec:problem-description}, 
we additionally consider a set $C$ of battery charging stations distributed spatially on the ground to power each UAV with fixed battery capacity $B$.
We assume each UAV has power consumption rate $p_0$ per unit time when hovering and $p_1$ when flying.
Without loss of generality, we assume $p_0 \le p_1$.
%with two flying mode \emph{hovering} and \emph{flying}, with power consumption rate $p_{0}$ and $p_{1}$ respectively. 
With remaining energy $B_t$ in battery, it takes time $t_c(B_t)$ for a UAV to fully charge the battery to capacity $B$.
%, where time $t_c(B_t)$ can be a constant if UAV battery is replaced instead of charged.
% We generalize the travelling distance matrix defined in Section~\ref{sec:problem-description} to include the charging stations $C$ with similar triangle inequality in Equation~\eqref{eq:triangle}.
%To simplified the algorithm design, we denote $B_s^{*}$ as the minimum amount of energy for travelling from  location $s \in S$ to the nearest charging station from $C$, which can be calculated as follows.
%\begin{equation}\label{eq:min-energy}
%    B_s^{*} = \min_{c \in C} \limits ~ p_1 \cdot a(s,c)
%\end{equation}
We require the path planning that each UAV starts at some charging station and finally lands at a charging station when finishing servicing the assigned demands.
%To simplify the analysis, we do not allow partially charging options, i.e., the UAV battery can only be charged to full capacity or not charged at all. Moreover, when the charging process completes, we allow the UAV to wait on the ground for some extra time for new demands to be released.
Yet we flexibly allow a UAV (after fully charged) to wait extra time on the ground charging station for new demands to be released.
%

Based on the system model in Section~\ref{sec:problem-description}, we update the UAV  routing constraint in Eq.~\eqref{eq:obj} in the following equation. With charging, for each UAV $k$, the set $T_{k,s}$ includes the time periods when the UAV stays at location $s$, where $s \in \se{1,2,...,|S|}$ indicates a service location, and $s \in \se{|S|+1,|S|+2,...,|S|+|C|}$ indicates a charging location.
The routing defined by $T_{k,s}$ is feasible if the battery level is non-negative at each time points of  $T_{k,s}$.
\begin{equation}
(T_{k,1},T_{k,2},\ldots,T_{k,|S|+|C|}) \longrightarrow \text{feasible routing with charging}
\end{equation}


With such model extension to include charging stations, we first refine our dominance property in  Lemma~\ref{lmm:dominated} to fit the new charging model.
%
%we first refine our lemmas~\ref{lmm:prnc-1} and \ref{lmm:prnc-2} to fit the new model. 
Our previous algorithms in Sections~\ref{sec:single-drone} and \ref{sec:multiple-drone} can be regarded as a special case here for only one battery cycle of UAV operation before using up energy storage $B$. 
%That is to say, they aim to service the maximum number of demands up to a fixed time $T$, which is the maximum time that a UAV can stay in the air by support of one battery cycle.
With the consideration of battery charging, however, the problem becomes complex since each UAV needs to decide when and where (not necessarily the nearest charging station) to charge the battery.
Considering its energy consumption, it is no longer optimal for each UAV to service the demands as early as possible, which violates the dominance property in Definition~\ref{def:dominanted}.
%This can be seen from the intuition that the UAV has to wait in the air with a longer time for the next demand to be released when it finishes servicing the previous demands earlier and hence consumes more energy during waiting.
%with the support of flexible waiting time on the ground 

This can be seen from an example that after charging, the UAV can indeed return back to some user location as early as possible or 
exactly at the time when a new demand is released there.
However, this may not be optimal since after servicing this demand, the UAV may have to consume more energy to hover in the air for a longer time for the next demand to be released. In this situation, a better strategy is just to wait on the ground for more time after charging such that this demand is serviced with the later released demands together.
%
%An example is that the UAV needs to hover for a longer time to wait for the new demands to be released if it finishes servicing the last demand earlier. 
%On the other hand, arriving at the user location late will miss the demands.
%
%
As a result, the UAV now needs to reach the user location within the 
overlapping
%``overlapping''
waiting time window of the user demands as much as possible (i.e., neither earlier nor later).
%As a result, the UAV now needs to arrive at the latest at the user location to just not miss the target demand there.
%%the routing algorithm now needs to send the UAV to the user location as much as possible within the waiting time window of the user demand.
%
Except for this, both lemmas \ref{lmm:prnc-1} and \ref{lmm:prnc-2} still work here for characterizing the UAV movement in the optimal solution, since any violation of these two lemmas will lead to unnecessary waste of time or energy.
Based on these observations, we still manage to characterize the decision states of the MDP problem.
%, as well as the times for making routing decisions (including charging decisions).

Inspired by the design idea of Algorithm~\ref{alg:one-general} in section~\ref{subsec:dp-transition}, we first propose a dimensionality-reduced dynamic programming based algorithm to obtain the optimal path planning of a single UAV with battery charging.
Similar to Algorithm~\ref{alg:greedy} in Section~\ref{subsec:iterative-greedy}, for the whole UAV-swarm problem with battery charging, we then 
%Then, we propose an integer linear programming (ILP) formulation to compute the optimal path planning for the UAV-swarm. The ILP formulation is based on the feasible path planning for a single UAV. In our previous dynamic programming approach, each feasible path planning of a single UAV corresponds to a path on the directed acyclic graph (DAG) of the state transition diagram, we use this DAG to formulate the path planning for all UAVs by integer linear programming approach. Though ILP can solve the problem optimally, it may consume unacceptable computational time. 
present a fast iterative greedy algorithm with low complexity and prove its theoretical performance.

%Algorithm~\ref{alg:greedy} to obtain the path planning for multiple UAVs with the same provable approximation ratio (new algorithm is shown in Algorithm~\ref{alg:greedy-charging}).

%In the following, we first generalize the dynamic programming algorithm to obtain the optimal path planning of a single UAV with battery charging.
%Based on this, we show that our proposed greedy algorithm (Algorithm~\ref{alg:greedy}) can also be applied to obtain the path planning for multiple UAVs, with the same provable approximation ratio.
%Additionally, we propose an integer linear programming (ILP) formulation to compute the optimal path planning for the UAV-swarm.
%The ILP formulation is based on the feasible path planning for a single UAV.
%In our previous dynamic programming approach, each feasible path planning of a single UAV corresponds to a  path on the directed acyclic graph (DAG) of the state transition diagram, we use this DAG to formulate the path planning for all UAVs by integer linear programming approach.

\subsection{Optimal path planning of a single UAV with battery charging}\label{subsec:energy-state}
In this subsection, we characterize the decision states for computing the optimal path planning of a single UAV powered by battery charging stations.
Recall that in Algorithm~\ref{alg:one-general} 
we use set $Q$ to label the UAV departure times at all the locations, where each tuple $\pe{s^{*},t^{*}} \in Q$ 
tells the last UAV departure time $t^{*}$ at location $s^{*}$.
%tells that time $t^{*}$ is the last time when the UAV leaves location $s^{*}$.
At time $t$, the decision state can be represented by a tuple $\pe{Q, B_t, i, s, t}$, indicating that the UAV at location $s$ just services an accumulated number $i$ of demands by time $t$ with remaining battery energy $B_t$ and latest departure times $Q$.
%at locations. 
Compared with the decision state in Section~\ref{sec:single-drone}, we only need to include the energy storage state $B_t$ in the decision state here.
%
However, we cannot directly adopt the variable $g(Q,i,s)$ from Algorithm~\ref{alg:one-general} to compute the optimal path planning because servicing demands at the earliest may not be the optimal strategy.
%path planning strategy
%abuse notation $g(\cdot)$ and
Instead, as the UAV prefers to finish more demands with more energy left in battery, decision state $\pe{Q, B_t, i, s, t}$ dominates $\pe{Q, B_t', i, s, t}$ if $B_t > B_t'$.
%when comparing different decision choices, 
Therefore, we redefine $g(Q,i,s,t)$ as the maximum energy that the UAV at location $s$ can keep in battery at time $t$ for servicing an accumulated number of $i$ demands under the constraints of departure times $Q$.
%
%In other words, 
Since the dominance property on time domain does not hold, now the computational complexity of variables $g(Q,i,s,t)$ relies on the value of time $t$. %We still make the restriction for the time of decisions such that a demand is finished by the UAV exactly at time $t$.

%In term of state transition, we need to consider the new charging decision when the UAV decides to leave the current location $s \in S$ to another location $s' \in S$. 
%That is to say, the UAV can choose to charge the battery at some ground station before flying to the target location $s'$.
In term of charging decisions, the UAV now can choose to charge the battery at some ground station $c \in C$ before flying from the current location $s \in S$ to some target location $s' \in S$.
%to the target location $s'$. we need to consider the new charging decision when the UAV decides to leave the current location $s \in S$ to another location $s' \in S$. 
We describe all possible subsequent decisions based on the current state.
%
%Suppose $g(Q,i,s,t) = B_t$ is computed, i.e., the UAV routing decision is made based on state $\pe{Q, B_t, i, s, t}$.
%a demand is finished at time $t$ at location $s$. 
%
%
%Suppose the UAV reaches decision state  $\pe{Q, B_t, i, s, t}$, similar as in Algorithm~\ref{alg:one-general},  we  restrict the state such that a demand is finished by the UAV exactly at time $t$ and regard time $t-q$ as the decision-making time stamps due to service time $q$.
%In the new algorithm, we enumerate all possible subsequent decisions after time $t-q$ and identify the corresponding 
In new decision state $\pe{Q', B_{t'}, i', s', t'}$, we disregard the choices such that the resulting battery energy $B_{t'}$ is not sufficient for the UAV to reach the nearest charging station after time $t'$, i.e., $B_{t'} < p_1 \cdot \min_{c \in C} a(s',c)$.
Similar to Algorithm~\ref{alg:one-general} in Section~\ref{subsec:dp-transition}, we consider time $t-q$ as a decision-making time stamp only if some demand will be serviced at time $t$ as it takes time $q$ to service a demand.
%Note that it takes time $q$ to service a demand, the UAV leaving from location $s$ at time $t$ decides to stay or leave at time $t-q$.
We separate the state transition analysis associated with the possible decisions at time $t-q$ into the following two cases.
\begin{itemize}
%\item If the UAV decides at time $t-q$ to stay at location $s$ to service more demands, the next demand release time $t^{*}$ can be calculated as in  Algorithm~\ref{alg:one-general}. Then, at time $t' = t^{*} + q$ the UAV will finish servicing this new demand with the remaining energy 
\item If the UAV decides to stay at location $s$ at time $t-q$ to service more demands, the earliest time for the next demand to be released at location $s$ is $t^{*} = \min \se{r_j ~|~ t - q < r_j, j \in J_s}$. The UAV will finish servicing this demand at time $t^{*} + q$. In this case, we have $Q' = Q, s' = s, t' = t^{*} + q$, $i' = i + \sum_{j \in J_{s'}: r_j = t^{*}} 1$ and the energy level $B_{t'}$ at time $t'$ is updated to:
\begin{equation}
    B_{t'} = B_t - p_0 (t' - t).
\end{equation}

% at location $s$
\item Otherwise, if the UAV decides to leave at time $t-q$, it does not service any new demand during time interval $(t-q,t]$. After time $t$, the UAV can fly to the target location $s' \in S$ with or without battery charging on the way, as discussed in the following two sub-cases. 
%We separate this decision into two sub-cases, depending on whether the UAV energy is charged or not.

\begin{itemize}
\item The UAV directly flies to new location $s' \in S$ without battery charging, i.e., it will reach location $s'$ at future time $t^{*} = t + a(s,s')$.
By Lemma \ref{lmm:prnc-2}, the UAV should at least service one demand at location $s'$ and let $t'- q$ (with $t' - q \ge t^{*}$) be the earliest time to  service a new demand.
Similar to Algorithm~\ref{alg:one-general}, 
time $t'$ can be calculated based on the previous departure times in $Q$ and the demand information at location $s'$.
For the remaining energy level at time $t'$, it is updated to:
\begin{equation}
B_{t'} = B_t - p_1 \cdot a(s,s') - p_0 (t'-t^{*}).
\end{equation}

\item The UAV decides to charge the battery before flying to location $s'$. 
In this process, the UAV flies to charging station $c \in C$ and possibly another charging station $c' \in C$ later to charge the battery to full capacity.
This is possible if it consumes a lot energy when flying to location $s'$ directly, and it is better to refill from another charging station $c'$ located around location $s'$. 
It is also possible that the UAV stays at charging station $s'$ after fully charged and return to sky at time $t^*$. 
In the new algorithm, we examine every feasible possibility of the charging process, i.e., values for $s',c,c',t*$.
At time $t^{*} + a(c',s')$, the UAV will arrive at the new location $s'$ to service a new demand.
The UAV's battery energy 
%at time $t' = t_5 + q$ can be computed as follows.
after servicing this demand at time $t' = t^* + a(c',s') + q$ is updated to: 

%Generally, this decision can be decomposed as follows. The UAV first flies directly to a charging station $c \in C$ to charge the battery to full capacity. Optionally, it then flies to charging station $c' \in C$ and again charges the battery to full capacity there. Optionally, it can wait on the ground for some time and return to the air at time $t^*$. Finally, the UAV flies to the target location $s' \in S$. The final step can be disregarded if no more demands at location $s'$ can be serviced when the UAV reaches there. In the above decomposed charging decisions, the two charging stations $c$ and $c'$ can be the same and they are regarded as the entry and the exit of the ground. Also, the waiting time can be $0$, in which case the UAV goes back to the air immediately after charging. In the new algorithm, we test every possibility of such decomposed decisions, i.e., values for $s',c,c',t*$ and further refine those choices that are applicable. Firstly, to ensure sustainable service, the UAV needs to make sure it has enough energy to reach the first charging station $c$. That is, $B_{t_1} \ge 0$ with $B_{t_1} = B_t - p_1 \cdot a(s,c)$ being the remaining battery energy when it arrives charging station $c$ at time $t_1 = t + a(s,c)$. After time $t_2 = t_1 + t_c(B_{t_1})$, the battery will be fully charged. We enumerate time $t^{*}$, which is the time when the UAV is back to the air at charging station $c'$. To achieve this, the UAV may fly to charging station $c'$ if $c \not = c'$ and arrive at time $t_3 = t_2 + a(c,c')$ with battery energy $B_{t_3} = B - p_1 \cdot a(c,c')$, and then charge the battery again to full capacity at time $t_4 = t_3 + t_c(B_{t_3})$. At charging station $c'$, the UAV might wait for some time on the ground for the demands to be released at user location $s'$ and we assume it returns back to the air at time $t^{*}$ with $t^{*} \ge t_4$. At time $t_5 = t^{*} + a(c',s')$, the UAV will arrive the new location $s'$ to service new demands. We assume the UAV can always immediately start to service a new demand when it arrives at location $s'$ (i.e., at time $t_5$), since it can choose to wait on the ground before flying to location $s'$ for an arbitrary long time. If it finds no demand at time $t_5$, this case is interpreted that no more demand will be released at location $s'$ and the UAV should not choose to wait at location $s'$ but choose to not fly to there at all. The battery energy at time $t' = t_5 + q$ can be computed as follows.

\begin{equation}\label{eq:double-charge}
B_{t'} = B - p_1 \cdot a(c', s') - p_0 \cdot q.
\end{equation}



%The reason to consider two charging stations $c,c'$ here is that from location $s$, the UAV might only has enough energy to reach the nearest charging station $c$, while there may be many demands going to be released at the user location $s'$ near the charging station $c'$. Directly flying from charging station $c$ to the user location $s'$ might already consume a lot of energy and lead to less time for servicing the demands there. Hence, it might be better for the UAV to directly fly to charging station $c'$ to charge the battery first and then fly to the nearby location $s'$ to service the demands.
\end{itemize}
\end{itemize}

\noindent
For both cases above, the UAV can finish  servicing a demand at future time $t'$ at location $s'$. We count the additional demands that can be started to service at time $t'-q$ and update parameter $i'$. In the revised procedure above, we successfully refine Algorithm~\ref{alg:one-general} for returning an optimal path planning solution for the single UAV powered by ground charging stations.

%and $T_w \le T$ be the maximum allowed waiting time slot number on the ground
\begin{proposition}
Let $T = \max_{j \in J} d_j - \min_{j \in J} r_j$ be the total time slot number, the variables $g(Q,i,s,t)$ in the revised MDP can be computed in complexity $O(T^2 \cdot n^{\alpha} \cdot |C|^2 \cdot |S|^{\alpha+1})$.
\end{proposition}
% T^2 -> T T_w,
%\begin{proof}
\noindent
\textit{Sketch of proof.} 
Similar to the computational complexity analysis of Algorithm~\ref{alg:one-general} in the proof of  Proposition~\ref{prop:time-one-drone}, 
after servicing this demand at time $t' = t^* + a(c',s') + q$, the computational complexity for 
encoding $g(Q,i,s,t)$ here is $O(Q)\cdot O(n |S| T)$ with $O(Q) = O( n^{\alpha-1} |S|^{\alpha-1})$ being the computational complexity for set $Q$.
It takes $O(T |S| \cdot |C|^2)$ operations to enumerate all possible decisions for joint routing and charging, especially for the last sub-case in Eq.~\eqref{eq:double-charge} above.
In that sub-case, enumerating all possible $s', c, c', t^{*}$ takes $O(|S|), O(|C|), O(|C|),O(T)$ operations, respectively.
Hence, the computational complexity for computing all
the variables $g(Q,i,s,t)$ is  $O(T^2 \cdot  n^{\alpha} \cdot |C|^2 \cdot |S|^{\alpha+1})$.
%$O(T^2  n^{\alpha} \cdot |C|^2 \cdot |S|^{\alpha+1})$.
%\end{proof}




\subsection{Cooperative path planning for the UAV-swarm with battery charging}\label{subsec:lp}
In this subsection, we focus on finding the path planning for the UAV-swarm.
%As a matter of fact, the approximation routing algorithm in Section~\ref{subsec:iterative-greedy} still works here since we can solve the path planning optimally for a single UAV.
%
As an extension of the single-UAV case in Section~\ref{subsec:energy-state}, the difficulty of the problem here increases substantially as the number of the UAVs increases. 
Despite of the problem difficulty, we successfully transform the problem to an integer linear program by creating novel directed acyclic graph (DAG) of the UAV-state transition diagram. 
Then we determine the optimal solution with detailed steps, though the overall complexity is still high. 
% given in Appendix~\ref{ap:ilp}
The detailed description of our ILP approach to compute the optimal UAV-swarm with battery charging can be found in Appendix \ref{ap:ilp}.


To further reduce the complexity, 
%It can be observed from Proposition~\ref{lmm-greedy-cover} and Proposition~\ref{lmm:energy-greedy} that locally optimizing each UAV iteratively (Algorithm~\ref{alg:greedy-charging}) will only give approximated solution.
%Therefore, a global optimization approach is essential to find the optimal solution. To tackle the challenge, 
we treat charging stations as special user locations (to go once out of energy) and focus on all the refined feasible path planning choices for each UAV. As the UAVs are identical, any feasible path planning for one UAV is also feasible for another UAV. Inspired by the iterative path planning idea for the UAV-swarm in Section~\ref{sec:multiple-drone} (i.e., Algorithm~\ref{alg:greedy}), we successfully present a fast iterative algorithm with theoretical performance guarantee for the new UAV-swarm problem with battery charging in Algorithm~\ref{alg:greedy-charging}.
%Moreover, we observe that each feasible path planning for a single UAV can be represented by a path on a directed acyclic graph.
%%When merging the path plannings for all UAVs, we focus on reducing the common serviced demands to achieve the best collaboration between UAVs.

\iffalse
In terms of partition-based path planning, we create a binary variable $y_{k,s}$ for each UAV $k \in K$ and each location $s \in S$, where $y_{k,s} = 1$ indicates that location $s$ can only be visited by UAV $k$.
Meanwhile, we add the following constraints into the formulation.

\begin{align} \label{eq:6}
\sum_{k \in K} y_{k,s} \le 1, &~ \forall s \in S
\end{align}

\begin{equation}
\begin{split} \label{eq:7}
  x_{k,e} -  y_{k,s} \le 0, &~ \forall k \in K, \forall s \in S, \\
  &~ \forall v \in V: s(v) = s, \forall e \in \text{IN}(v) \cup \text{OUT}(v)
\end{split}
\end{equation}

Equation~\eqref{eq:6} indicates that each location can only be visited by at most one drones, and Equation~\eqref{eq:7} indicates that if UAV $h$ is not allowed to visit location $s$, then any edge $e$ that contains a node which includes location $s$ cannot occur in the path of UAV $k$.

\fi




\begin{algorithm}[!tbh]\small
%\footnotesize
\caption{UAV-swarm algorithm with battery charging.}
\label{alg:greedy-charging}
\begin{algorithmic}[1]
\State $J_u \gets J$
\For{$k \in \se{1,2,...,|K|}$}
\State {\bf call} refined dynamic programming algorithm in Section~\ref{subsec:energy-state} 
to route UAV $k$ to demand set $J_u$
\State $J(k) \gets $ serviced demands by UAV $k$
\State $J_u \gets J_u \setminus J(k)$
\EndFor
\State {\bf Return} demands $\cup_{k \in K} J(k)$.
\end{algorithmic}
\end{algorithm}

Algorithm~\ref{alg:greedy-charging} iteratively finds the optimal path planning for each individual UAV 
by calling the refined single-UAV algorithm in Section~\ref{subsec:energy-state}.
As it only incurs one UAV at a time and hence can be solved efficiently by our proposed dimensionality-reduced dynamic programming approach. 
%
%show that our previously proposed greedy algorithm (Algorithm~\ref{alg:greedy}) can also be applied here to obtain the path planning for multiple UAVs, which is given in Algorithm~\ref{alg:greedy-charging}.
%
In Algorithm~\ref{alg:greedy-charging}, UAVs are dispatched to spatial locations and charging stations in a greedy manner, as a result, its computational complexity is just linear to the number of UAVs $|K|$, which is given by $O(|K| \cdot T^2 \cdot  n^{\alpha} \cdot |C|^2 \cdot |S|^{\alpha+1})$.
\begin{proposition}\label{lmm:energy-greedy}
%With UAV battery charging, if the optimal path planning of a single UAV can be solved, 
%Algorithm~\ref{alg:greedy} can still be applied to obtain an approximated path planning for multiple UAVs with the same approximation ratio as in Proposition~\ref{lmm-greedy-cover}.
Algorithm~\ref{alg:greedy-charging} achieves constant approximation ratio $1 - (1-1/|K|)^{|K|}$ in the worst case, which is greater than constant $1-1/e$.
\end{proposition}
%\begin{proof}
\textit{Sketch of proof.}
Algorithm~\ref{alg:greedy-charging} works based on the fact that the path planning of one UAV can also be feasibly applied to another UAV since UAVs are identical, i.e., they have the same flying speed and power consumption rate.
Each feasible path planning choice for a UAV can be mapped to a subset of serviced demands by the UAV.
As Algorithm~\ref{alg:greedy-charging} optimizes the path planning for each UAV $k$ individually, it at least uses the same routing strategy as one of the $|K|$ UAVs from the globally optimized solution.
Due to this iterative nature of Algorithm~\ref{alg:greedy-charging}, it can always guarantee that at least a portion of $\frac{1}{|K|}$ demands from the current set $J_u$ of demands are serviced.
As a result, applying mathematical induction, the approximation ratio bound $1 - (1-1/|K|)^{|K|}$ can be calculated at the worst case.
%As long as ... (\KK{TBD})
%The problem can still be related to the maximum k-coverage problem as the same demand is counted as one when it can be serviced by multiple UAVs.
%As a result, the analysis in Proposition~\ref{lmm-greedy-cover} also holds here.
%It is exactly identical to the maximum k-coverage problem except that in our problem it is impossible to enumerate all possible path plannings for a single UAV.
%This can be observed that the optimal path planning for a single UAV explores all possible path plannings.
%\end{proof}

%Later in experiments, though we implement a novel and optimal algorithm based on integer linear programming (ILP) for finding the path planning for UAV swarm with charging stations in Section \ref{ap:ilp}, it has high computational complexity and can hardly be used in practice for solving large scale UAV-swarm path planning problems.



This proposition tells that our proposed Algorithm~\ref{alg:greedy-charging} has desirable performance bound even in the worst case, while enjoying the low computational complexity. Later in Section~\ref{sec:experiments} we will use extensive simulations in Fig.~\ref{fig-experiment} to show that the performance of our algorithm is near-optimal.