%\clearpage
\appendices

\iffalse
\section{Missing Proof for Proposition~\ref{prop:time-one-drone} in Section~\ref{sec:single-drone}} \label{ap:prop-one-drone}

\setcounter{lemma}{10}
\begin{proposition}%\label{prop:time-one-drone}
Algorithm~\ref{alg:one-general} has low computational complexity $O(n^{\alpha} |S|^{\alpha+1})$, which is only polynomial to both the demand number $n$ and location number $|S|$.
\end{proposition}
%\textit{Proof of sketch: }

\begin{proof}
As Algorithm~\ref{alg:one-general} has three loops (Line~\ref{alg-line-roundi}, \ref{alg-line-loop2}, \ref{alg-line-loop3}), the overall computational complexity is $O(Q) \cdot O(n |S|^2)$, where $O(Q)$ is the space complexity for set $Q$ and $O(n |S|^2)$ is the computational complexity for computing each variable $g(Q,i,S)$. In the following, we show that $O(Q) = O(n ^{\alpha-1} |S| ^{\alpha-1})$.

%
Although there are arbitrarily many choices for set $Q$, we only focus on those which are necessary for computing the optimal path planning.
This is achieved by applying a trimming process on set $Q$ (Line~\ref{alg-line-trim}).
Specifically, given a partial path planning $\pe{Q,i,s,t}$, we apply a trimming process on set $Q$ as follows (Line~\ref{alg-line-trimcond}).

\begin{enumerate}[label = (\roman*)]
\item%(i)
We remove any tuple $\pe{s^{*},t^{*}}$ from $Q$ if $s^{*} = s$ {\bf or} there exists $\pe{s'',t''} \in Q$ such that $s^{*} = s'', t^{*} < t''$ {\bf or} $t^{*} - q + \max_{j \in J_{s^{*}}} (d_j-r_j) \le t + a(s,s^{*})$.
\item %(ii)
We replace each $\pe{s^{*},t^{*}} \in Q$ by $\pe{s^{*},t'}$, where $t' - q$ is the largest demand release time such that $t' \le t^{*}$, i.e. $t' - q = \max \se{r_j ~|~ r_j \le t^{*} - q, j \in J_{s^{*}}}$.
\end{enumerate}

Note that set $Q$ is only used to count the number of serviced demands, which happens only at Line \ref{alg-line-move-i}.
The above trimming process is designed to make sure that they will not affect the result in Line \ref{alg-line-move-i}.
As we mentioned earlier, if the UAV repeatedly visits a location $s^{*}$, we only label the latest departure time $t^{*}$.
This can be achieved by trimming condition $s^{*} = s$ and condition $s^{*} = s'', t^{*} < t''$ in step (i).
For condition $s^{*} = s$, there is no need to record the latest departure time at location $s^{*}$ since currently the UAV is already at location $s^{*}$.
For condition $s^{*} = s'', t^{*} < t''$, the latest departure time at location $s^{*}$ is $t''$ instead of $t^{*}$.
%
%Assume the UAV has visited some location $s^{*}$ at an earlier time $t^{*}$ (i.e., $\pe{s^{*},t^{*}} \in Q$) and now it is at location $s$ at time $t$, aiming to finish the future demands after time $t$.
%for feasible solution $\pe{Q, i,s,t}$,
%If the UAV visits the same location $s^{*}$ twice, we only take the one with the latest visit, this corresponds to condition $s^{*} = s$ and condition $s^{*} = s', t^{*} < t'$ in step (i).

Moreover, when the UAV arrives location $s^{*}$ in the near future, it has to differentiate whether the demand there has been previously serviced or not since we do not explicitly record the serviced demands in the algorithm.
If $t^{*} - q + \max_{j \in J_{s^*}} (d_j-r_j) \le t + a(s,s^{*})$, we can find that any (already serviced) demand $j$ released before time $t^{*} - q$ at location $s^*$ will be due before the UAV visits location $s^{*}$ at the earliest possible time $t + a(s,s^{*})$.
%, i.e., demand $j$ will not be hit by the UAV twice.
That is to say, there is no need for the UAV to differentiate the serviced demands there since they will be due before the UAV arrival.
%when the UAV arrives location $s^{*}$ in the near future, any previously serviced demand will be already due.
%any demand that can be serviced is not considered by the UAV before.
Therefore, we do not need to record the last departure time $t^{*}$ at
location $s^{*}$, i.e.,  $\pe{s^{*},t^{*}}$ can be removed from $Q$.
%
%

On the other hand, in step (ii), if $t^{*} - q$ is not a demand release time, no demand is released during $(t'-q,t^{*}-q]$ at location $s^{*}$, where $t' - q = \max \se{r_j ~|~ r_j \le t^{*} - q, j \in J_{s^{*}}}$. Hence, without affecting counting the number of serviced demands at  Line~\ref{alg-line-move-i} in Algorithm~\ref{alg:one-general}, we could replace $\pe{s^{*},t^{*}}$ by $\pe{s^{*},t'}$ and note that $t'-q$ is a demand release time.
%the number of possible value for time $t'$ here is bounded by $O(n)$.
After step (i), there are at most $\alpha-1$ elements in set $Q$ by definition of $\alpha$, and after step (ii) the time labels in set $Q$ are demand release times where each can be encoded in $O(n)$.
As a result, after applying the trimming process, the space complexity for set $Q$ is $O( n^{\alpha-1} |S|^{\alpha-1})$, which yields overall computational complexity $O(n^{\alpha} |S|^{\alpha+1})$.
This completes the proof.
\end{proof}

\fi




%Integer Linear Programming (ILP) Algorithm for optimally solving the problem with energy charging
%Optimal Integer Linear Programming (ILP) Algorithm with energy charging
\section{Optimal Integer Linear Programming (ILP) Algorithm with energy charging} \label{ap:ilp}

%\noindent
%Due to page limit, the description of our ILP approach to compute the optimal UAV-swarm with energy charging can be found in \cite{wang:2020}.


%\iffalse
When we relax UAVs' energy capacity limit for sustainable service provisioning, UAVs are allowed to travel to charging stations in the mean time and thus jointly design UAVs' path planning over users' locations and charging stations. Multiple path planning should take into account of UAVs' temporal-spatial information of charging and its design becomes more challenging. Recall that the dimensionality-reduced dynamic programming algorithm in previous sections finds the optimal path planning of a single UAV by extending the decision state at each critical time stamp until reaching the decision state where no more demands can be serviced by the UAV.

%To tackle the challenge, we focus on all the refined feasible path plannings for a single UAV.
%As the UAVs are identical, any feasible path planning for one UAV is also feasible for other UAVs.
%Moreover, we observe that each feasible path planning for a single UAV can be represented by a path on a directed acyclic graph.
%%When merging the path plannings for all UAVs, we focus on reducing the common serviced demands to achieve the best collaboration between UAVs.


Thus, we find a novel way to construct a directed acyclic graph (DAG) to represent the whole state transition diagram of the algorithm.
That is to say, each directed path on the DAG corresponds to a feasible path planning for a single UAV.
Therefore, solving the problem of $|K|$ UAVs is simplified to select $|K|$ paths on the DAG.
We then apply integer linear programming approach to select such multiple paths on the DAG globally to find the optimal solution.
Although the DAG contains all feasible path plannings, the number of nodes in DAG (i.e., the number of decision states) can be effectively reduced by our previous analysis of the dimensionality-reduced dynamic programming approach in Section~\ref{sec:single-drone}.

To be specific, We first describe the construction of the DAG $G = (V,E)$  representing the state transition diagram, in which any path from the root node to the leaf node indicates a feasible path planning of a single UAV.
%We exploit this DAG graph to solve the UAV-swarm via integer linear programming approach as follows.
%We use a directed acyclic graph $G = (V,E)$ to store all feasible path plannings of a single UAV, in which any path from the root node to the leaf node indicates a feasible path planning.
%with computed variable $g(Q,i,s,t) = B_t$
Specifically, each node $v \in V$ corresponds to a decision state $v = \pe{Q, B_{t}, i, s, t}$ as described in Section~\ref{subsec:energy-state} and each directed edge $e = (v,v') \in E$ connecting two nodes $v,v' \in V$ indicates a routing action that transits the decision state $v = \pe{Q, B_{t}, i, s, t}$ into the new decision state $v' = \pe{Q', B_{t'}, i', s', t'}$.
Such a graph can be generated based on our proposed dimensionality-reduced dynamic programming approach in Section~\ref{subsec:energy-state}.
Recall that the decision state $v$ will be disregarded if the battery energy $B_{t}$ is not sufficient enough for the UAV to reach the nearest charging station after time $t$, i.e., $B_{t} < p_1 \cdot \min_{c \in C} a(s,c)$.
In the following of this section, notation $e$ is only used to refer to a directed edge in the graph.

%
%\myparagraph{Integer Linear Programming Approach}
%Given the directed acyclic graph $G = (V,E)$, we aim to find the optimal solutions for multiple drones.
%As mentioned above, each node $v = (\bd{u},s,t)$ is defined by three parameters, pending tasks defined by $\bd{u}$, location $s \in S$ and time $t \in \mathcal{T}$.
Now we describe the integer linear programming approach to find the optimal solution for multiple UAVs, where we regard the solution of each UAV as an independent path planning, i.e., the same demand might be serviced by more than one UAV.
%
%For each location $s \in S$, for each node $v \in V$, we write $s(v)$ indicates the location that is contained in node $v$.
We denote ${root}$ as the entrance of the dimensionality-reduced dynamic programming algorithm, which is also the only root node of the graph.
%
For each node $v \in V$, let $\text{IN}(v)$ and $\text{OUT}(v)$ be the set of incoming edges and outgoing edges connected to node $v$, respectively.
%For each user demand $j \in J$, let $V(j) \subseteq V$ be the set containing any node $v$ in which demand $j$ is serviced according to the decision state $v = \pe{Q, B_{t}, i, s, t}$ up to time $t$, i.e. $j \in J(v)$, and let $E(j) = \bigcup_{v \in V(j)} \text{IN}(v) \cup \text{OUT}(v)$ be the set of edges that contains any node from $V(j)$.
Recall that each routing action will result in at least one new demand to be serviced, hence each transition (i.e., edge in the graph) corresponds to the additionally serviced demands by that routing action.
Then, for each user demand $j \in J$, we denote $E(j) \subset E$ as the set containing any edge $e = (v,v')$ such that demand $j$ is one of the additional demands serviced by the UAV at location $s'$ when transiting from decision state $v = \pe{Q, B_{t}, i, s, t}$ to  $v' = \pe{Q', B_{t'}, i', s', t'}$.
%, i.e., one of the demands represented by $i'-i$.
%
For each UAV $k \in K$ and each edge $e = (v,v') \in E$ from the graph, we create a binary decision variable $x_{k,e}$, where $x_{k,e} = 1$ indicates that the path that corresponds to the solution of UAV $k$ contains the edge $e$ in the graph, i.e., the UAV indeed encounters decision state $v$ and transits to decision state $v'$.
%the decision state $v'$ is indeed transited from state $v$.
For each UAV $k \in K$ and each demand $j \in J$, we create a binary decision variable $b_{k,j}$ where $b_{k,j} = 1$ indicates that demand $j$ is serviced by UAV $k$.
Moreover, we create binary variable $o_j$ for each demand $j \in J$ where $o_j = 1$ indicates that demand $j$ is serviced by at least one UAV.
The integer linear programming is formulated as follows.





\begin{align}
  \max   \sum_{j \in J} o_j \label{eq:obj} &~\\
  \sum_{e \in \text{OUT}(v)} x_{k,e} - \sum_{e \in \text{IN}(v)} x_{k,e}  \le 0, &~ \forall k \in K, v \in V\setminus\se{\text{root}} \label{eq:1}\\
  \sum_{e \in \text{OUT}(v)} x_{k,e} \le 1,  &~  \forall k \in K, v = {\text{root}} \label{eq:2}\\
  b_{k,j} - \sum_{ e \in  E(j)} x_{k,e} \le 0, &~ \forall k \in K,  \forall j \in J \label{eq:3}\\
  o_{j} - \sum_{k \in K} b_{k,j} \le 0, &~  \forall j \in J \label{eq:4} 
  \\
  x_{k,e}, b_{k,j}, o_j  \in \se{0,1}, &~ \forall k \in K, e \in E, j \in J \label{eq:5}
  %\\
  %\sum_{k \in K} \sum_{e \in \text{OUT}(v)} x_{k,e}   \le 1, &~ \forall v \in V\setminus\se{\text{root}} \label{eq:5}
\end{align}
In the above formulation,
Equation~\eqref{eq:obj} defines the objective function, which indicates the total number of serviced demands.
Equation~\eqref{eq:1} and \eqref{eq:2} guarantees that each UAV corresponds to a path starting with the \textit{root} node and ending with any node in the graph. 
Equation~\eqref{eq:3} ensures that if demand $j$ is serviced by UAV $k$, at least one edge from $E(j)$ is selected in the path.
Equation~\eqref{eq:4} ensures that demand $j$ is counted as serviced only if at least one UAV can service the demand.
%Equation~\eqref{eq:5} is an optional constraint which means that two UAVs can not share the same node, which is correct since each node $v = (Q,s,t)$ indicates that the UAV is at location $s$ at time $t$. If two UAVs share the same node, it means the two UAVs are at the same location at the same time, which should definitely be avoid.



After solving the above ILP, the path of UAV $k$ can be directly constructed by including the edge $e$ as long as $x_{k,e} = 1$.
Based on that, the path planning of the UAV $k$ can be constructed according to the decision state of the nodes in the path. In Section~\ref{sec:experiments}'s experiment, we use Gurobi as the ILP solver.
%\fi