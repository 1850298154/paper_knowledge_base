%mention muti-UAV DP, story.
%Mention. System model, expected demand time variance, demand release.
\section{Simulation  Experiments}\label{sec:experiments}



In this section, we implement our proposed  algorithms and conduct extensive simulations to evaluate their performances.
%as compared to the optimum where we implement the integer linear programming algorithm to obtain the optimal routing of UAV-swarm.
%
%\myparagraph{Experiment Setting.}
The data used in simulations are generated randomly to mimic the practical UAV service provisioning.
%and we sample $100$ instances to obtain average performances for each setting.
%we generate $60$
%Specifically, the demands are generated randomly over time span up to $50$ minutes over up to $30$ locations. For each demand, we first pick random values for each demand's duration in the discrete range of $[1,20]$ minutes and set random values for demand release times in the discrete range of $[1,50]$ minutes. 
Specifically, the demands are generated randomly over time span up to $40$ minutes. We first pick random values for each demand's duration in the discrete range of $[1,20]$ minutes and set random values for demand release times and deadlines in the discrete range of $[1,40]$ minutes. 
The UAV's service time for each demand is $2$ minutes. 
For user locations and charging stations, we generate integer value of 2D ground coordinates with uniform distribution in range $[0,10]$ km for both $x$-axis and $y$-axis, and  adopt \emph{Manhattan Distance} to measure their pairwise distance for the UAVs to fly through.
Each UAV's battery capacity is set to be $30$ units of energy, the energy consumption rates for hovering and flying are set to be $2$ and $3$ units per minute, respectively.
We consider $3$ charging stations and it takes $3$ minutes to fast charge or swap the battery.
The above setting applies to all the following experiments unless separately described.






\iffalse
\begin{figure*}[!h]
\subfloat[Cooperative trajectories at optimum with charging]{
\begin{minipage}[t]{0.45\linewidth}
\centering
\includegraphics[height = 60mm]
{images/figure-greedy-energy.pdf}
\label{Fig.trajA.energy}
\end{minipage}
}
\subfloat[Cooperative trajectories returned by our algorithm with charging]{
\begin{minipage}[t]{0.45\linewidth}
\centering
\includegraphics[height = 60mm]
{images/figure-opt-energy.pdf}
\label{Fig.trajB.energy}
\end{minipage}
}
\caption{
Three UAVs cooperative routing over the 2D ground plane including $10$ location nodes with $30$ demands (marked by time interval $[r_j, d_j]$ for demand $j \in J_s$ near each location node $s$). The optimal solution and Algorithm~\ref{alg:greedy}'s solution are shown in Figure~\protect\subref{Fig.trajA.energy} and \protect\subref{Fig.trajB.energy}, respectively. Different UAVs' trajectories are marked in different colors (black, blue, green), and the time interval along each trajectory tells the flight time interval of each UAV. The location nodes with double circles are the initial locations of UAVs.
}
\label{Fig.traj.energy}
\end{figure*}
\fi




\subsection{Performance evaluation of UAV-swarm algorithms without battery charging}





%without charging

%with/without charging
%example. then , data.
%similar pattern.
In this experiment, we present the results without battery charging.
Recall that our Algorithm~\ref{alg:one-general} already returns the optimal solution for a single-UAV case in Section~\ref{sec:single-drone}, and Algorithm~\ref{alg:greedy} extends Algorithm~\ref{alg:one-general} for $|K|$-many cooperative UAVs in Section~\ref{sec:multiple-drone}. We aim to verify the performance of Algorithm~\ref{alg:greedy}, and we implemented a conventional baseline approach where UAV locations are partitioned into clusters and one cluster is assigned to one UAV, as discussed in Section \ref{subsec:benchmark}. 
Specifically, we marge a pair of clusters/locations with the minimum marginal distance into one cluster until a number of $|K|$ clusters achieved \cite{grygorash2006minimum}. For each single UAV, we still apply our optimal Algorithm~\ref{alg:one-general} to obtain the path planning.


%We aim to verify the performance ratio of Algorithm~\ref{alg:greedy} versus the optimum. 
We first verify the performance of Algorithm~\ref{alg:greedy} versus the optimum, here we use average performance ratio metric \cite{stefas2020approximation},
where performance ratio is the ratio of the number of serviced demands by Algorithm~\ref{alg:greedy} to that of optimum.

%and present the performance ratio in Fig.~\ref{fig-experiment}, 
%The performance ratio is averaged by running 100 samples for each data point.
%obtained by ILP in Section~\ref{ap:ilp} in much more running time.
%
%\noindent
%
%
\begin{figure}[!ht]%
\centerline{
\includegraphics[keepaspectratio,width = 0.98\columnwidth]{images/figure_data.pdf}
}
\caption{Average performance ratios achieved by our Fast Iterative UAV-swarm Routing Algorithm versus the optimum under different values of the UAV number $|K|$ and the location number $|S|$.}
\label{fig-experiment}
\end{figure}
%
%
%and with small demand service time, i.e., $q \rightarrow 0$.
%We first present the cooperative UAV Swarm under Algorithm~\ref{alg:greedy} and the optimum for an example with $10$ locations $30$ demands in Fig.~\ref{Fig.traj}\protect\subref{Fig.greedy} and Fig.~\ref{Fig.traj}\protect\subref{Fig.opt}.
%There are some mild differences between these two solutions. In total, the optimum services $26$ demands and our algorithm services $24$ demands with performance ratio $24/26 > 90\%$.
%
%We study how the relative performance of Algorithm~\ref{alg:greedy} versus the optimum changes with the number of UAVs $|K|$ in Fig.~\ref{fig-experiment}.
In Fig.~\ref{fig-experiment}, as the performance of the algorithm is measured in average sense (average over 100 random inputs), it is expected to be greater than the approximation ratio in the worst case analysis as in Proposition \ref{lmm-greedy-cover}.
%In experiments, we vary the value of $|K|$ from 2 to 8, and sample $100$ instances for each value. Besides $|K|$, we vary the number of locations $|S|$ from $10$ to $20$ and $30$.
%The summarized performance results are shown in Fig.~\ref{fig-experiment}.
We can see from Fig.~\ref{fig-experiment} that the average performance ratio of Algorithm~\ref{alg:greedy} is overall high (above $96\%$), and it is very close to 1 (i.e., optimum) when the number of UAVs is small or large enough. When $|K|$ is small, UAVs are sparsely distributed among many user locations and the "separated" UAVs cover very different user locations at optimum. 
Thus, our iterative Algorithm~\ref{alg:greedy} which routes UAVs one by one in a greedy manner performs very close to the optimum. When $|K|$ is large, most user locations are covered by enough UAVs and our algorithm missing very few performs very close to the optimum as well. Furthermore, as the location number $|S|$ increases, the performance gap between the optimum and our algorithm increases as there are more joint routing possibilities among UAVs to consider.

\iffalse
\begin{figure}[!t]%
\centering
\begin{minipage}[t]{0.47\columnwidth}
\centerline{
\includegraphics[width = 0.97\columnwidth]{images/figure_data.pdf}
}
\caption{Performance ratios achieved by our Fast Iterative UAV-swarm Routing Algorithm versus the optimum under different values of the UAV number $|K|$ and the location number $|S|$.}
\label{fig-experiment}
\end{minipage}
\centering
\begin{minipage}[t]{0.47\columnwidth}
\centerline{
\includegraphics[width = 0.97\columnwidth]{images/figure_data_noise.pdf}
}
\caption{Performance ratio achieved by Algorithm~\ref{alg:greedy-charging} with expected demand arrivals versus the random demand arrivals under Gaussian distribution.}
\label{fig-noise-energy}
\end{minipage}
\end{figure}
\fi






\iffalse
%both figures are removed
\subfloat[Trajectories returned by Algorithm~\ref{alg:greedy}]{
\begin{minipage}[t]{0.235\columnwidth}
\centering
\includegraphics[keepaspectratio, width = 1.05\columnwidth]
{images/figure-greedy.pdf}
\label{Fig.greedy}
\end{minipage}
}
\subfloat[Trajectories at optimum]{
\begin{minipage}[t]{0.235\columnwidth}
\centering
\includegraphics[keepaspectratio, width = 1.05\columnwidth]
{images/figure-opt.pdf}
\label{Fig.opt}
\end{minipage}
}
\fi




\begin{figure}[!ht]%
\centerline{
\includegraphics[keepaspectratio,width = 0.98\columnwidth]{images/figure_data_serviced_percentage.pdf}
}
\caption{
Percentage of serviced demands versus the number of UAVs
for $|S|=100$ locations by Algorithm~\ref{alg:greedy} (solid lines) and by the conventional partitioned-based algorithm in Section \ref{subsec:benchmark} (dashed lines), respectively.
}
\label{fig-serviced-percentage}
%n=400,K=100,P=50,beta=5,q=1,[1,50); {'xyrange': 30, 'weight': False, 'prop': [0.0, 2]}
\end{figure}

We then verify the performance of Algorithm~\ref{alg:greedy} with the baseline approach mentioned above for large scale input of $|S|=100$ locations (which is hard to compute the optimum). The coordinates of user locations are randomly selected from $[0,30]$ km to support up to $n=400$ demands release.
Fig.~\ref{fig-serviced-percentage} shows how the percentage of serviced demands changes with the increase number of UAVs.
%We can see from the figure that Algorithm~\ref{alg:greedy} significantly outperforms the baseline approach, especially for large number of UAVs.
When the UAV number becomes large and sufficient to service all the demands, the serviced demands rate tends to converge. Besides, Algorithm~\ref{alg:greedy} significantly outperforms the conventional approach for various setting of UAV number $|K|$.


%\subsection{Impact of battery charging on UAV path planning}
\subsection{
Performance evaluation of UAV-swarm algorithms with battery charging} \label{subsec:exp-2}


\begin{figure}[t!]
\centering
\subfloat[UAVs' Trajectories returned by Algorithm~\ref{alg:greedy-charging}  with charging]{
\begin{minipage}[t]{0.98\columnwidth}
\centering
\includegraphics[keepaspectratio, width = 0.98\columnwidth]
{images/figure-greedy-energy.pdf}
\label{Fig.greedy.energy}
\end{minipage}
}
\newline
\centering
\subfloat[UAVs' Trajectories at optimum with charging]{
\begin{minipage}[t]{0.98\columnwidth}
\centering
\includegraphics[keepaspectratio, width = 0.98\columnwidth]
{images/figure-opt-energy.pdf}
\label{Fig.opt.energy}
\end{minipage}
}
\caption{
Three UAVs' cooperative path planning over 2D ground plane across $10$ locations $s_1$ - $s_{10}$ and battery charging stations $c_1$ - $c_{3}$, where the circle nodes indicate user locations and square nodes indicate charging stations.
%Fig.~\protect\subref{Fig.greedy} and \protect\subref{Fig.opt} correspond to the problem without charging and near-zero demand service time, and they represent the routing solutions returned by Algorithm~\ref{alg:greedy} and by the optimal algorithm, respectively.
Fig.~\ref{Fig.traj}\protect\subref{Fig.greedy.energy} and \protect\subref{Fig.opt.energy} show each UAV's trajectory (over different nodes) returned by Algorithm~\ref{alg:greedy-charging} and by the optimal algorithm, respectively.
%in Appendix~\ref{ap:ilp}
%(n,S,K) = (20, 10, 3)
%(minT, maxT) = (1, 40)
%(beta) = (10)
%(q,C,B) = (2, 3, 30)
%P = [2,3]
%The battery capacity is $30$,  service time $q=2$, charging time $t_c=2$, alg=15 vs. opt=17
%line shapes with
The three UAVs' trajectories are marked in different colors (black, purple, blue).
%, and the time interval along each trajectory tells the flight time interval of each UAV
Near each location node $s$, the demand $j \in J_s$ is depicted as time interval $[r_j, d_j)$ 
with arrivel time $r_j$ and deadline $d_j$, and for example location $s_3$ faces two demand windows $[4, 9)$ and $[35, 39)$ minutes.
Each directed edge connecting two nodes corresponds a routing decision for the corresponding UAV, where the label $[t_1,t_2]$ for each edge indicates that the UAV departs from one node at time $t_1$ and arrives at the other node at time $t_2$.
%The legends in the upper-right corner contain the sequence of locations that each UAV has visited in the path planning.
%Three UAVs cooperative routing over the 2D ground plane including $10$ location nodes with $30$ demands (marked by time interval $[r_j, d_j]$ for demand $j \in J_s$ near each location node $s$). The optimal solution and Algorithm~\ref{alg:greedy}'s solution are shown in Figure~\protect\subref{Fig.opt} and \protect\subref{Fig.greedy}, respectively.
%The location nodes with double circles are the initial locations of UAVs.
}
\label{Fig.traj}
\end{figure}

%in Appendix~\ref{ap:ilp}
In this subsection, we examine our low-complexity Algorithm~\ref{alg:greedy-charging}'s average performance and compare it to our optimal algorithm based on ILP as detailed in Appendix \ref{ap:ilp}.
%
%In previous section, we show the average performance ratios archived by our algorithms, which confirms our analytical theoretical bounds. Here, we study how battery charging affects UAV-swarm cooperation in the path planning. From extensively tested examples, we found that  Algorithm~\ref{alg:greedy-charging} can service as many demands as optimum with much high probability.
%in Appendix~\ref{ap:ilp}
Here, we present one typical example of three UAVs' cooperative trajectories with three charging stations, where the solutions returned by our Algorithm~\ref{alg:greedy-charging} and our optimal ILP algorithm are depicted in
Fig.~\ref{Fig.traj}\protect\subref{Fig.greedy.energy} and Fig.~\ref{Fig.traj}\protect\subref{Fig.opt.energy}, respectively.
%in the 2D ground plane
With $20$ user demands released over $10$ locations, the optimum services $17$ demands and our Algorithm~\ref{alg:greedy-charging} services $16$ demands with average performance ratio $94\%$ 
as compared to the optimum. Though the trajectories in these two sub-figures look somewhat different, in both the optimum and Algorithm~\ref{alg:greedy-charging}'s solution, we observe smart UAVs' cooperation by overlapping their trajectories in certain user locations.

%which is consistent with the results in
%We first observe that the UAV could travel directly from one charging station to another charging station.

Next, we study how battery charging affects UAV-swarm cooperation in the path planning. 
As shown in Fig.~\ref{Fig.traj}\protect\subref{Fig.greedy.energy}, UAV 3 flies from charging station $c_1$ to another charging station $c_2$, before reaching location $s_7$ to service the demands there.
%UAV 1 flies from charging station $c_1$ to another charging station $c_3$, before reaching location $s_2$ to service the demands there.
One may wonder why the UAV needs to charge again at charging station $c_2$ during the flight.
%Intuitively, charging twice compensates the energy consumption during the flight from $c_1$ to $s_2$, especially when the stopping point $c_3$ is close to $s_2$.  
Intuitively, charging twice compensates the energy consumption during the flight from $c_1$ to $s_7$, especially when the stopping point $c_2$ is close to target user location $s_7$.  
%
%This can be seen from the fact that directly flying from charging station $c_1$ to location $s_2$ takes longer time (one more units of time) and leads to less energy (three less units of energy) in battery when it reaches location $s_2$, hence less hovering time to service the demands there.
%%%Conversely, without exploiting such options, in the optimal solution in Fig.~\ref{Fig.traj}\protect\subref{Fig.opt.energy} the first demand at location $s_2$ is missed.
%%%Indeed, in that situation at most two demands at location $s_2$ can be serviced instead of three by our algorithm.
%Moreover, under the constraint of battery capacity, some user locations may not be serviced by the UAVs at all. This can be seen in the optimal solution in Fig.~\ref{Fig.traj}\protect\subref{Fig.opt.energy} that location $s_1$ is completely abandoned, while similar phenomenon occurs in Algorithm~\ref{alg:greedy-charging}.
%it is location $s_3$ in the greedy solution. Finally, UAV cooperation becomes more necessary due to the exigency of  charging. This can be seen in  Fig.~\ref{Fig.traj}\protect\subref{Fig.greedy.energy} that the UAV cooperation occurs at location $s_4$ where UAV 1 services the first two demands there in the very beginning, while the last demand there is serviced later by UAV 3. This is because waiting at the location for the demand to be released might not be efficient due to the exigency of battery charging. Such a cooperation phenomenon can also be seen  in the optimal path planning in Fig.~\ref{Fig.traj}\protect\subref{Fig.opt.energy} where both UAV 2 and UAV 3 take turns to service location $s_3$ and $s_{10}$, successfully servicing all demands there.
%
Moreover, even with the battery charging option, our optimal algorithm using ILP chooses not to service location $s_1$ which is far from the other nodes, and similarly our sub-optimal Algorithm~\ref{alg:greedy-charging} chooses not to service location $s_8$ at the corner.

%
%Overall, Algorithm~\ref{alg:greedy-charging} takes about 3.2 seconds to compute the path plannings of servicing 60 demands, compared to that of 437 seconds for the ILP algorithm, on the compute platform with Intel Core i5-3570 CPU.
%这里简单说下运算时间的差距有多大,可以按之前结果自己估计一下就好了,没必要重新做实验再算。
%While, the performance ratio of Algorithm~\ref{alg:greedy-charging} is much better than the theoretical bound in worst case, and it is close to 1 in practice (i.e., optimum). By this, our proposed Algorithm~\ref{alg:greedy-charging} is more suitable for solving the UAV-swarm path planning problem with battery charging in practice.




\subsection{Algorithm robustness to noisy demand information}



%demands are known.
%full knowledge each hotspot
One may wonder the performances of our proposed algorithms if the UAVs do not have precise information about the demands to service at each location.
This is possible in practice if the demand forecasting has some error. We choose Algorithm~\ref{alg:greedy-charging} as it fits the most general scenario with battery charging stations and conduct experiments to examine its robustness to noisy demand information.
We refine our Algorithm~\ref{alg:greedy-charging} 
to use the expected demand information such as mean arrival time as input to obtain the cooperative path planning solution, while the actual arrival of each demand may deviate from the mean with some noisy Gaussian distribution in the time domain.
%some particular variance.
%We show how this variance affects the performance of the algorithm.
We show how this noisy demand distribution affects the performance of the algorithm, by counting the finally serviced demands.


\begin{figure}[!ht]%
\centerline{
\includegraphics[keepaspectratio, width = 0.98\columnwidth]{images/figure_data_noise.pdf}
}
\caption{Average performance ratio between Algorithm~\ref{alg:greedy-charging} and the offline optimum (with precise information) versus the variance of the demand arrival information. 
}
\label{fig-noise-energy}
\end{figure}

%n=60, B = 40.
%The result is shown in  
Fig.~\ref{fig-noise-energy} presents the average performance ratio of Algorithm~\ref{alg:greedy-charging}, as compared to the offline optimum which knows precisely all the demand information. 
%
Naturally, more demands are missed given a larger variance of the demand information unknown to the UAV-swarm. We can see that as compared to the offline optimum, Algorithm~\ref{alg:greedy-charging} performs well given reasonable variance of noisy demand information, with average performance ratio above $75\%$ for various settings of energy capacity. Though sometimes a UAV may leave a location early to miss a delayed demand, due to the trajectory overlaps among the cooperative UAVs, this demand can still be serviced by another UAV flying to the same location later.





\iffalse
\begin{figure}[t]%[!h]
%\subfloat[Cooperative trajectories returned by Algorithm~\ref{alg:greedy}]{
\centering
\begin{minipage}[t]{0.7\linewidth}
\centering
\includegraphics[height = 60mm]
{images/figure_trajB.pdf}
\label{Fig.trajB}
\end{minipage}
%}
\caption[12pt]{
Trajectories returned by Algorithm~\ref{alg:greedy} where three UAVs cooperative routing over the 2D ground plane including $10$ location nodes with $30$ demands (marked by time interval $[r_j, d_j]$ for demand $j \in J_s$ near each location node $s$).
%The solution returned by Algorithm~\ref{alg:greedy} is shown in Figure~\protect\subref{Fig.trajB}.
Different UAVs' trajectories are marked in different colors (black, blue, green), and the time interval along each trajectory tells the flight time interval of each UAV. The location nodes with double circles are the initial locations of UAVs.
}
\label{Fig.traj}
\end{figure}
\fi

%n=60,K=30,P=8,beta=3,[1,30)


%On the other hand, when there are more locations, the performance of the algorithm becomes worse.
%This maybe the reason that with more locations,
