%\IEEEraisesectionheading{\section{Introduction}\label{sec:introduction}}

\IEEEraisesectionheading{\section{Introduction}\label{sec:introduction}}
\IEEEPARstart{U}{AV}
technology recently emerges as a promising solution to provide high-quality mobile services (e.g., edge computing, high-speed Internet access, and local caching) to ground users (e.g., \cite{wang20uav, asheralieva2019hierarchical, TGCN2021}). Unlike the conventional wireless communication systems, UAVs offer line-of-sight (LoS) wireless links with users \cite{saad2019vision}, which greatly improves the quality of service. Besides, due to their agility and mobility, UAVs can be deployed fast to achieve seamless wireless coverage and provide on-demand mobile services to the users under emergency conditions \cite{zhang2018fast, LiangTMC2020}. Despite of these advantages, a UAV has small service coverage due to limited antenna size and low transmit power, and needs to fly closely to service ground users \cite{wang2018traffic, xu2018uav, jiangchun2021proactive}. To service many dynamic users across different spatial locations sustainably, it is necessary for UAVs to intelligently cooperate with each other by taking into account their locations, trajectories, battery charging and the users' demands over time, and how to determine UAVs' cooperative path planning is an important question in many applications.

\subsection{Related works}\label{sec:related}
In the literature of wireless communications and networking, there are some studies of UAV deployment schemes to service ground users (e.g., \cite{pan2012cooperative, ProcIEEE2019}), given their distribution in a geographical area.
Zeng and Zhang \cite{zeng2017energy} consider a UAV-enabled base station to service multiple users for achieving maximum throughput per user, by jointly optimizing the transmit power and the UAV trajectory.
Without knowing users' spatial distribution, Xu et al. \cite{xu2018uav} study the UAV-user interaction for learning users' real locations before the single UAV's deployment.
To meet users' instantaneous traffic requests over different locations, Wang et al. \cite{wang2018traffic} study how a single UAV should dynamically adapt its location to user movements following predefined random processes.
Wang and Duan \cite{wang2018dynamic}\cite{zhewangTMC} further consider the energy allocation of a UAV for meeting dynamically arriving users' demands.
Yang et al. \cite{yang2019energy} studied energy efficient UAV path planning with energy harvesting.

{Jiang} and {Swindlehurst} \cite{Heading2012} present a heading algorithm to position an UAV for improving uplink communications for ground users. Xu et al. \cite{NFZ2020} consider various uncertainties (e.g., user location, wind speed, and no-fly zones) and jointly optimize an UAV's path planning and transmit beamforming vector for the overall energy consumption, which is a non-convex optimization problem.
In \cite{PowerTransfer2018}, UAV's trajectory is optimally designed in an UAV-enable wireless transfer system in order to maximize the total energy transfer. The cooperative trajectories of UAV-swarm are rarely exploited to improve the system performance, e.g., energy efficiency and service quality.
%The UAV with low delay is introduced in \cite{saad2019vision} and the energy efficient routing for UAV was studied in \cite{yang2019energy}.


%
In term of drone/vehicle routing in the broader literature, there are some studies on multiple vehicular data mules' joint path planning (\cite{sugihara2011path, zhang2015data}).  %somasundara2004mobile
%When the demand information are unknown in advance, Enright et al. studied the scheduling for UAVs in a dynamic environment \cite{enright2015uav}, in which targets (user demands) arrive at random and remain active until some UAV reaches the target’s location.
Zhang et al. \cite{zhang2015data} studied data sensing and transmission problem with energy consideration and proposed a two-stage data gathering approach for dynamic sensing and routing. As to path planning of UAVs, Sun et al. \cite{sun2019optimal} studied 3D trajectory design for UAV communication systems and  considered ground users' static demands only. These works do not explicitly study the UAV-to-UAV cooperation to meet users' dynamic demands spatially and temporally, and there is a lack of studies of such UAV-swarm cooperation for dynamic path planning.

%There are a few studies on UAV-swarm planning by using artificial intelligence tools %\cite{6198334, xu2020optimized}. In \cite{6198334}, the authors apply the genetic %algorithm and the particle swarm optimization algorithm to search UAV's optimal %trajectories in a complex 3D environment. In \cite{xu2020optimized}, Cheng et al. adopt %swarm intelligence algorithm to generate UAVs' path planning such that each UAV can reach %the mission area quickly and reduce the probability of being captured and destroyed. %However, such techniques suffer from high computational resources and tends to be trapped %in local optimum.








%In such works, travelling salesman problem (TSP) is formulated for each data mule to travel through multiple sensor nodes and collect sensor data once and for all, without looking at the dynamic demands over the time domain.
%However in general, no demand can be serviced in the worst case scenario and one can imagine the case where a new demand always appears at the location when no UAV is nearby and disappears when some UAV arrives the location.


%The solutions from previous works act like one-time deployment of a UAV where the UAV visits each location for at most once. The problem becomes difficult if the UAV can visit some locations for multiple times and there is a lack of research for routing UAVs repeatly over spatial locations.


We are aware that in the literature of artificial intelligence and robotics, there are some preliminary designs of cooperative vehicle trajectories to visit target locations once and for all (\cite{6198334, xu2020optimized, TNSE2021}).
% ([8] [9] [10] [11] ACTUALLY CAN REDUCE TO ONLY 2 OR 3 REFERENCES ).
%6198334, xu2020optimized, sugihara2011path, Zhao2005infocom
However, in practice the UAV-swarm may need to repeatedly visit the same set of user locations (e.g., shopping malls and other hotspots) over time according to users' activity patterns and waiting time deadline, and there is another temporal domain to optimize other than the spatial domain.
As the sustainable UAV servicing over many users is also limited by the UAV's energy capacity (\cite{zhang2020energy} \cite{yao2019qos}), we should allow UAVs to return to a charging station to recharge or change their batteries.
These problems are not addressed in previous studies by designing UAVs' cooperative path planning over users' locations and charging stations for achieving sustainable servicing.

\subsection{Main contributions }\label{sec:contr}
Given the aforementioned review, no prior optimization technique can be applied to this challenging problem, and we are interested in developing new algorithms for cooperative UAV-swarm path planning with provable performance bounds. When there are many spatial locations or user demands to service over time, our problem becomes complex and it is challenging to design  tractable algorithms with low computational complexity.

Our key novelty and main contributions in this paper are summarized as follows.



%[label={\large\textbullet}]

\begin{itemize}
\item \emph{Dynamic UAV-swarm cooperation to service users' spatial-temporal demands (Section~\ref{sec:problem-description}):} To our best knowledge, this is the first paper to design and analyze cooperative mutli-UAV trajectory planning algorithms for dynamically servicing many spatial demand locations over time. Unlike existing routing problems (e.g.,\cite{sugihara2011path, 6198334, xu2020optimized}), we consider the practical and challenging problem that the UAV-swarm need to repeatedly visit users' locations and charging stations  over a long time, and design low-complexity algorithms for guiding UAV-swarm cooperation with provable performance guarantee.
%
\item \emph{Optimal path planning for a single UAV (Section~\ref{sec:single-drone}):}
Regarding a single UAV's dynamic trajectory planning problem, we manage to substantially simplify the dynamic programming problem and propose a fast algorithm for returning the UAV's optimal path planning. The algorithm's complexity is low and regardless of the scale of time domain, only polynomial with respect to both the numbers of user locations and user demands.
%
\item \emph{Cooperative path planning of a large UAV-swarm to service many user demands (Section~\ref{sec:multiple-drone}):} When a large number $|K|$ of UAVs are cooperating to service many user locations, the simplified dynamic optimization problem becomes intractable and we alternatively present a fast iterative cooperation algorithm with provable approximation ratio $1-(1-\frac{1}{|K|})^{|K|}$ in the worst case, which arbitrarily approaches to constant guarantee $1-1/e$. Our approximation algorithm is proved to obviously outperform the traditional approach of partitioning UAVs to serve different user/location clusters separately.
%
\item \emph{Refined algorithm design for UAV-swarm by adding UAV charging stations} (Section~\ref{sec:energy}):
For achieving sustainable service provisioning, we jointly design UAVs' cooperative path planning over spatial-temporal users' demands and various charging stations. The problem becomes more challenging and we successfully transform it to an integer linear programming by creating novel directed acyclic graph (DAG) of the UAV-state transition diagram. To further lower the complexity we accordingly propose an iterative algorithm with constant approximation ratio.
\end{itemize} 